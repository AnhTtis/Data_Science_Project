\documentclass[reprint, double column, superscriptaddress,prl, showkeys]{revtex4-1}

\usepackage{float}
\usepackage{graphicx,epsfig}
\usepackage{amssymb}
\usepackage{amsmath}
\usepackage{bm}
\usepackage{braket}
\newcommand* {\frack}[2]{{\textstyle\frac{#1}{#2}}}
\usepackage{textcomp}
\usepackage{color}
\usepackage{pgffor}

\usepackage{soul}

\usepackage[bookmarksnumbered,bookmarksopen]{hyperref}
\hypersetup{
   colorlinks=true,       % false: boxed links; true: colored links
}

\usepackage{pdfpages}
\usepackage{BOONDOX-cal}

\makeatletter

\AtBeginDocument{\let\LS@rot\@undefined}
\makeatother
\def\supplementfilename{Supplemental.pdf}
\newif\ifarXiv
\arXivtrue

\usepackage{soul}

\begin{document}
\setcounter{page}{1}

\title[]{Highly-Anisotropic Even-Denominator Fractional Quantum Hall State in an Orbitally-Coupled Half-Filled Landau Level}
\author{Chengyu \surname{Wang}}
\author{A. \surname{Gupta}}
\author{Y. J. \surname{Chung}}
\author{L. N. \surname{Pfeiffer}}
\author{K. W. \surname{West}}
\author{K. W. \surname{Baldwin}}
\affiliation{Department of Electrical and Computer Engineering, Princeton University, Princeton, New Jersey 08544, USA}
\author{R. \surname{Winkler}}
\affiliation{Department of Physics, Northern Illinois University, DeKalb, Illinois 60115, USA}
\author{M. \surname{Shayegan}}
\affiliation{Department of Electrical and Computer Engineering, Princeton University, Princeton, New Jersey 08544, USA}
\date{\today}

\begin{abstract}
The even-denominator fractional quantum Hall states (FQHSs) in half-filled Landau levels are generally believed to host non-Abelian quasiparticles and be of potential use in topological quantum computing. Of particular interest is the competition and interplay between the even-denominator FQHSs and other ground states, such as anisotropic phases and composite fermion Fermi seas.  Here we report the observation of an even-denominator fractional quantum Hall state with highly-anisotropic in-plane transport coefficients at Landau level filling factor $\nu=3/2$. We observe this state in an ultra-high-quality GaAs two-dimensional \textit{hole} system when a large in-plane magnetic field is applied. By increasing the in-plane field, we observe a sharp transition from an isotropic composite fermion Fermi sea to an anisotropic even-denominator FQHS. Our data and calculations suggest that a unique feature of two-dimensional holes, namely the coupling between heavy-hole and light-hole states, combines different orbital components in the wavefunction of one Landau level, and leads to the emergence of a highly-anisotropic even-denominator fractional quantum Hall state. Our results demonstrate that the GaAs two-dimensional hole system is a unique platform for the exploration of exotic, many-body ground states.

\end{abstract}

\maketitle  
When a strong perpendicular magnetic field ($B_{\perp}$) is applied to a two-dimensional electron system (2DES) at low temperatures, the thermal and kinetic energies are quenched as the electrons occupy the quantized Landau levels (LLs), and various many-body ground states emerge as the Coulomb energy dominates. Half-filled LLs are particularly interesting as they host a plethora of emergent, exotic quantum phases, including compressible composite fermion (CF) Fermi seas in the lowest orbital energy ($N=0$) LLs, e.g., at Landau level filling factors $\nu=1/2$ and $3/2$ in GaAs 2DESs \cite{Jain.Book.2007, Halperin.Jain.Book.2020}, and even-denominator fractional quantum Hall states (FQHSs) in the first-excited ($N=1$) LLs, e.g., at $\nu=5/2$ and $7/2$ \cite{Willett.PRL.1987, Pan.PRL.1999}. The even-denominator FQHSs have been of considerable, continued attention because they are generally expected to be Pfaffian states with non-Abelian statistics \cite{Haldane.PRL.1988, Moore.NPB.1991, Morf.PRL.1998, Nayak.RMP.2008}. FQHSs have also been reported at $\nu=3/2$ in different 2DESs. It was first observed in a wide GaAs quantum well with a bilayer-like charge distribution when the Fermi level ($E_F$) lies in the $N=0$ LL \cite{Suen.PRL.1994}. More recently, $\nu=3/2$ FQHSs were reported in different materials, such as ZnO \cite{Falson.Nat.Phy.2015}, AlAs \cite{Hossain.PRL.2018}, WSe$_2$ \cite{Shi.Nat.Nanotech.2020}, and bilayer graphene \cite{ Zibrov.Nature.2017, Li.Science.2017, Huang.PRX.2022}, when $E_F$ is in an $N=1$ LL. In contrast to these incompressible FQHSs in the $N=0$ and $N=1$ LLs, anisotropic phases emerge in the $N\geq 2$ LLs as the ground states at half fillings \cite{Koulakov.PRL.1996, Fogler.PRB.1996, Moessner.PRB.1996, Fradkin.PRB.1999, Fradkin.PRL.2000, Fradkin.ARCMP.2010, Lilly.PRL.1999, Du.SSC.1999, Fu.PRL.2020}. These are generally believed to be stripe phases which, at finite temperatures and in the presence of ubiquitous disorder, turn into nematic phases.

\begin{figure*}[t]
  \begin{center}
    \psfig{file=Fig1_base, width=1\textwidth}
  \end{center}
  \caption{\label{base}
    Longitudinal resistances ($R_{xx}$ in black, and $R_{yy}$ in red) vs. perpendicular magnetic field $B_{\perp}$ at tilt angles: (a) $\theta=0^{\circ}$ and (b) $\theta=70.4^{\circ}$. (c) Enlarged version of (b). Inset in (a): The self-consistently-calculated hole charge distribution (red) and potential (black). We emphasize that our 2DHS is confined to a narrow quantum well and has a single-layer charge distribution. Inset in (b): A schematic of the experimental setup. The sample is mounted on a rotating stage to support \textit{in situ} tilt; $\theta$ is the angle between $B$ and $B_{\perp}$, and $R_{xx}$ and $R_{yy}$ denote resistances measured along and perpendicular to $B_{\parallel}$. }
  \label{fig:base}
\end{figure*}

As highlighted in Fig. 1, here we report a $B_{\parallel}$-induced, highly-anisotropic, even-denominator FQHS at $\nu=3/2$ in an ultra-high-mobility 2D \textit{hole} system (2DHS) confined to a GaAs quantum well \cite{Chung.PRM.2022, Footnote.3/4, Wang.PRL.2022, SM}. In a purely $B_{\perp}$, our 2DHS is nearly isotropic in the vicinity of $\nu=3/2$, and its in-plane longitudinal magneto-resistance traces ($R_{xx}$ and $R_{yy}$) exhibit smooth and shallow minima at $\nu=3/2$, flanked by the standard, odd-denominator FQHSs at $\nu=1+p/(2p\pm 1)$, such as $\nu=4/3, 7/5, 10/7, \ldots$, and $5/3, 8/5, 11/7, \ldots$ [Fig. 1(a)]. This is consistent with a compressible (Fermi sea) ground state of CFs at $\nu=3/2$, similar to what has been seen previously in very high-mobility GaAs 2DESs \cite{Jain.Book.2007, Halperin.Jain.Book.2020, Du.PRL.1995, Chung.NM.2021} and 2DHSs \cite{Liu.PRB.2016, Chung.PRM.2022}. When we tilt the sample in magnetic field to induce a large $B_{\parallel}$ component [Figs. 1(b,c)], our 2DHS breaks its rotational symmetry and becomes highly anisotropic near $\nu=3/2$, with $R_{xx}/R_{yy} \simeq 10^4$, and yet shows deep minima in both $R_{xx}$ and $R_{yy}$ at $\nu=3/2$ and $R_{xy} \simeq (2/3)(e^2/h)$ \cite{SM}. Note that $R_{xx}$ exceeds 10 k$\Omega$ near $\nu=3/2$ whereas $R_{yy}\lesssim 1$ $\Omega$. 

Figure 1(b,c) traces are truly intriguing. On the one hand, the deep minima in both $R_{xx}$ and $R_{yy}$ at $\nu=3/2$, and an $R_{xy}\simeq (2/3)(e^2/h)$ are consistent with a developing FQHS at this filling. On the other hand, the enormous transport anisotropy near $\nu=3/2$ is reminiscent of the $B_{\parallel}$-induced anisotropic states that are observed in GaAs 2DESs in the $N=1$ LL, e.g. at $\nu=5/2$, when the sample is tilted to a large angle $\theta$ \cite{Lilly.PRL.1999.tilt, Pan.PRL.1999.tilt, Xia.PRL.2010, Liu.PRB.2013, Friess.PRL.2014, Shi.PRB.2015}. Note, however, that the anisotropic states observed at $\nu=5/2$ at large $\theta$ replace the FQHS which is present at small $\theta$ \cite{Lilly.PRL.1999.tilt, Pan.PRL.1999.tilt, Xia.PRL.2010, Liu.PRB.2013, Friess.PRL.2014, Shi.PRB.2015}. In contrast, in our 2DHS, at $\nu=3/2$, an even-denominator FQHS with highly-anisotropic transport coefficients replaces a compressible (CF Fermi sea) ground state at small $\theta$ \cite{Samkharadze.NP.2016, Schreiber.NC.2018, Falson.SA.2018, Footnote.phases}. Also remarkable is the presence of numerous FQHSs at odd-denominator filings such as $4/3$, $6/5$, $9/5$, and $11/9$ in Fig. 1(c) at large $\theta$. These FQHSs are generally seen in 2DESs in an $N=0$ LL \cite{Pan.PRB.2008, Footnote.second.LL, Kumar.PRL.2010}. The even-denominator FQHS at $\nu=3/2$ and its anisotropic behavior, however, are telltale signs that $E_F$ lies in an $N>0$ LL. In this Letter we present the evolution of this very unusual, anisotropic FQHS at $\nu=3/2$ with temperature and $B_{\parallel}$.  Our data and calculations suggest that a unique feature of 2D holes, namely the coupling between heavy-hole (HH) and light-hole (LH) states, combines $N=0$ and $N>0$ components in the wavefunction of one LL, and leads to the emergence of the anisotropic FQHS at $\nu=3/2$.

The ultra-high-quality 2DHS studied here is confined to a $20$-nm-wide GaAs quantum well grown on a GaAs (001) substrate. The 2DHS has a density of $1.3\times 10^{11}$ cm$^{-2}$ and a low-temperature ($0.3$ K) record-high mobility of $5.8\times 10^{6}$ cm$^2$/Vs \cite{Chung.PRM.2022}. We performed our experiments on a $4\times 4$ mm$^2$ van der Pauw geometry sample, with alloyed In:Zn contacts at the four corners and side midpoints. The sample was then mounted on a rotating stage and cooled in a dilution refrigerator. We measured the longitudinal resistances, $R_{xx}$ and $R_{yy}$, along and perpendicular to $B_{\parallel}$ (see Fig. 1(b) inset), using the lock-in amplifier technique.

\begin{figure*} [t]
  \begin{center}
    \psfig{file=Fig2_Tdep, width=1\textwidth}
  \end{center}
  \caption{\label{Tdep} 
     Temperature dependence of $R_{xx}$ and $R_{yy}$. (a, b) $R_{xx}$ and $R_{yy}$ vs. $B_{\perp}$ traces near $\nu=3/2$ taken at $\theta \simeq 70^{\circ}$ and at different temperatures. (c) $R^{\nu=3/2}_{xx}$, $R^{\nu=3/2}_{yy}$, and $R^{bg}_{xx}$ vs. $T$. Inset shows an $R_{xx}$ vs. $B_{\perp}$ trace near $\nu=3/2$, showing how $R_{xx}^{bg}$ at $\nu=3/2$ is defined. (d, e) Arrhenius plots of $R^{\nu=3/2}_{xx}/R^{bg}_{xx}$ and $R_{yy}$ vs. $1/T$. The lines are linear fits to the data at intermediate temperatures, and their slopes give energy gaps $\simeq 0.5$ and $\simeq 2.1$ K for (d) and (e) respectively. 
}
  \label{fig:Tdep}
\end{figure*}

In Figs. 2(a, b), we show $R_{xx}$ and $R_{yy}$ vs. $B_{\perp}$ traces taken at $\theta \simeq 70^{\circ}$ at different temperatures. With decreasing temperature, $R_{xx}$ near $\nu=3/2$ grows whereas $R_{yy}$ becomes smaller and approaches zero. Figure 2(c) presents $R^{\nu=3/2}_{xx}$, $R^{\nu=3/2}_{yy}$, and the background resistance $R^{bg}_{xx}$ \cite{Footnote.Rbg} on the flanks of $\nu=3/2$ vs. $T$. The data show that the 2DHS is highly anisotropic at low temperatures but becomes nearly isotropic at $T\gtrsim 0.2$ K. To quantify the strength of the $\nu=3/2$ resistance minimum and extract an estimate for the FQHS energy gap, we make Arrhenius plots of $R_{yy}$ and the ratio $R^{\nu=3/2}_{xx}/R^{bg}_{xx}$ as a function of $1/T$ [Figs. 2(d, e)]. Note that, at $\nu=3/2$, $R_{xx}$ does not decrease as temperature is lowered, likely because of the strong rise in $R_{xx}$ on the flanks of $\nu=3/2$. We therefore plot $R^{\nu=3/2}_{xx} / R^{bg}_{xx}$ to obtain an estimate for the $\nu=3/2$ FQHS energy gap \cite{Mallett.PRB.1988}. The lines in Figs. 2(d, e) are linear fits to the data points at intermediate temperatures, and their slopes give an estimate of the $\nu=3/2$ FQHS gap, $\sim 1$ K.

Measuring the Hall resistance, $R_{xy}$, for a highly-anisotropic 2D system is very challenging and seldom reported \cite{Xia.Nat.Phys.2011, Hossain.PRL.2018}. Near $\nu=3/2$ our 2DHS becomes very resistive along the hard-axis direction ($R_{xx} \simeq 20$ k$\Omega$), which inevitably induces non-uniformities in the current distribution in the sample and can also cause severe $R_{xx}$ mixing to $R_{xy}$. The effect of $R_{xx}$ mixing can be minimized by averaging $R_{xy}$ taken for opposite polarities of $B$ \cite{Sajoto.PRL.1993, Goldman.PRL.1993}. We did such averaging and, although we do not see a quantized Hall plateau at $\nu=3/2$, we find that $R_{xy}$ approaches $(2/3)(h/e^2)$ to within $1\%$; see Supplemental Material (SM) \cite{SM} for details. The fact that $R_{xy}$ does not exhibit a well-developed plateau is consistent with the deep, but non-zero, $R_{xx}$ minimum.

\begin{figure}[t!]
  \begin{center}
    \psfig{file=Fig3_tilt, width=0.5\textwidth}
  \end{center}
  \caption{\label{tilt} 
   Tilt-angle dependence of $R_{xx}$ is shown vs. $\nu$ at $20$ mK. Each trace is vertically shifted by $0.4$ k$\Omega$ for clarity. The heights of some traces are divided by $10$, as marked on the left. Inset: An enlarged version of $R_{xx}$ vs. $B_{\perp}$ at $\theta=65.2^{\circ}$, $67.1^{\circ}$, $68.7^{\circ}$, and $70.3^{\circ}$.}
  \label{fig:tilt}
\end{figure}

We now turn to the evolution of the different phases at and near $\nu=3/2$ with tilt-angle $\theta$. Figure 3 shows $R_{xx}$ vs. $\nu$ traces at different $\theta$ \cite{Footnote.Config}. With increasing $\theta$, the FQHSs go through transitions, starting at the lowest $\nu$ (highest $B_{\perp}$). For example, the $R_{xx}$ minimum at $\nu=9/7$ turns into a maximum at $\theta=65.2^{\circ}$ and disappears at higher $\theta$. At larger $\theta$, $R_{xx}$ rises very abruptly above a $B_{\perp}$ which depends on $\theta$ (see Fig. 3 inset), and attains very high values in a range of fields near $\nu=3/2$ (main Fig. 3). The ground state at $\nu=3/2$ exhibits an abrupt transition from a CF Fermi sea at $\theta \leq 65.2^{\circ}$ to a highly-anisotropic FQHS when $\theta \geq 68.7^{\circ}$. The strongest $\nu=3/2$ FQHS is seen at $\theta \simeq 70^{\circ}$. With further increase in $\theta$, $R_{xx}$ remains large, and the $\nu=3/2$ $R_{xx}$ minimum gradually gets weaker, while all the FQHSs between $\nu=4/3$ and 5/3 disappear. We also observe qualitatively similar transitions, including the sudden appearance of a $\nu=3/2$ FQHS, in another GaAs 2DHS with a higher density ($\simeq 2.2\times10^{11}$ cm$^{-2}$); however, the transitions occur at smaller $\theta$ ($\simeq 60^{\circ}$ instead of $\simeq 69^{\circ}$); see SM \cite{SM, Footnote.crossing}. 

The evolution of the FQHSs near $\nu=3/2$ presented in Fig. 3 is consistent with a crossing of the 2DHS LLs induced by increasing $B_{\parallel}$. Such a crossing has been previously reported \cite{Liu.PRB.2016, Zhang.PRB.2017, Footnote.differences}, and can be understood based on the energy ($E$) vs. $B_{\perp}$ LL diagram shown in Fig. 4. This LL diagram is from calculations performed for our 2DHS in a purely $B_{\perp}$ using the multiband envelope function approximation (EFA) and the $8 \times 8$ Kane Hamiltonian \cite{Winkler.Book.2003}. Because of the strong HH-LH coupling, the LLs of GaAs 2DHSs are highly nonlinear and show numerous crossings \cite{Winkler.Book.2003}. There are two crossings between the top two LLs, one between the $\boldsymbol \alpha$ level (in blue) and $\boldsymbol \beta$ level (in green) at very low $B_{\perp}$ ($\simeq 1$ T), and another between the $\boldsymbol \beta$ level and $\boldsymbol \gamma$ level (in red) at higher $B_{\perp}$ ($\simeq 18$ T). These two crossings are by themselves very interesting and lead to many exotic interaction phenomena \cite{Liu.PRB.2014, Liu.PRB.2015, Liu.PRL.2016, Ma.PRL.2022} near $\nu=1/2$ and 1. 

The relevant crossing in our study is between the $\boldsymbol \alpha$ and $\boldsymbol \gamma$ levels \cite{Liu.PRB.2016}. Note that in Fig. 4 this crossing occurs at $\nu < 3/2$. While we cannot perform calculations for a 2DHS in tilted fields because of computational challenges, we expect that the addition of $B_{\parallel}$ would increase the effective Zeeman energy and the separation between the $\boldsymbol \alpha$ and $\boldsymbol \beta$ levels. This implies that the crossing between $\boldsymbol \alpha$ and $\boldsymbol \gamma$ would move to higher fillings, eventually reaching $\nu=3/2$ and even higher $\nu$. Our data support this scenario. At $\theta=0^{\circ}$, $E_F$ at $\nu=3/2$ lies in the $\boldsymbol \alpha$ level (Fig. 4). This is essentially an $N=0$ LL, consistent with our observation of a compressible CF Fermi sea ground state at $\nu=3/2$, flanked by the standard, odd-denominator FQHSs at $\nu=1+p/(2p\pm 1)$ [Figs. 1(a) and 3]. As we increase $\theta$ and, therefore, $B_{\parallel}$, the $B_{\perp}$ position of the crossing moves toward lower $B_{\perp}$. At sufficiently large $\theta$, $E_F$ at $\nu=3/2$ moves to the other side of the crossing, and lies in the $\boldsymbol \gamma$ level. This is manifested by an extremely abrupt rise of $R_{xx}$ at $\nu=3/2$ as a function of $B_{\parallel}$ (Fig. 4 inset). We elaborate on the above crossing in the SM \cite{SM} based on our measured strengths of the different FQHSs near $\nu=3/2$, and show that it occurs in a very narrow range of $B_{\parallel}$ near 8 T (see Fig. S5 of SM \cite{SM}).

Next we discuss how HH-LH coupling in 2DHSs combines $N=0$ and $N>0$ components in the wavefunction of one LL, which can explain the unusual behavior observed in our experiments.
%
In the EFA, LLs in 2D systems can
be described using two sets of dimensionless ladder operators $a,
a^\dagger$ and $b, b^\dagger$, corresponding to two harmonic
oscillators \cite{Suzuki.PRB.1974, MacDonald.1995,
Winkler.Book.2003}, see also SM \cite{SM}.  The single-particle
Hamiltonian $H$ only depends on the $a$ oscillators; the $a$
oscillators thus define the energy spectrum of $H$ as a function of
$B_\perp$. The $b$ operators commute with $H$, and represent an oscillator with frequency zero that describes the degeneracy of the LLs.

In the EFA applied to the $s$-like electrons in the conduction band
of GaAs, the wavefunctions have two spinor components corresponding
to a spin $s = \pm 1/2$.  However, ignoring the weak spin-orbit
coupling in electron systems, the spin is decoupled from the orbital
motion, and eigenstates of $H$ have only one nonzero spinor
component $s$.  Therefore, the quantum number $N = 0, 1, 2, \ldots$
that labels the eigenstates $\ket{N}$ of the $a$ oscillator can also
label the eigenstates of $H$ with energies $E_N = \hbar \omega_c (N
+ \frac{1}{2}) \pm \Delta_\mathrm{Z}$ and Zeeman energy
$\Delta_\mathrm{Z}$.

The $p$-like holes in the valence band require four spinor
components representing $s= \pm 3/2$ (HH) and $\pm 1/2$ (LH) \cite{Footnote.spinor}. The hole eigenstates of $H$ are
generally a linear combination of HH and LH states.  However, to a
good approximation, the eigenstates of $H$ remain eigenstates of
total angular momentum $j = l + s$ (axial approximation
\cite{Suzuki.PRB.1974, Winkler.Book.2003}), where $l=xp_y - yp_x = \hbar \left(a^\dagger a - b^\dagger b\right)$ is the orbital angular momentum.  The eigenstates of $H$ become
\begin{equation}
  \label{eq:ax-state}
  \psi_\mathcal{N} = \sum_s \ket{N = \mathcal{N} - s - \frack{3}{2}}
  \, \xi^\mathcal{N}_s u_s
\end{equation}
with $\ket{N} = 0$ when $N < 0$.  Here, $u_s$ are the base spinors
with spin $s$ and $\xi^\mathcal{N}_s$ gives the relative weights of
the HH and LH spinors.  We ignore in Eq.\ (\ref{eq:ax-state}) the
$b$ oscillators that are not relevant for the present discussion.
$\mathcal{N} = 0, 1, 2, \ldots$ labels the energies of the LLs.  In
contrast to 2DESs, $N$ is not a good quantum number for hole LLs
because of the HH-LH mixing that is parameterized via the
coefficient $\xi^\mathcal{N}_s$ \cite{Footnote.2DES.mixing}. 

Note that in Fig. 4 the $\boldsymbol \alpha$ level is a pure HH LL
($s=-3/2$, $\mathcal{N}=0$) containing only $N=0$, consistent with
Eq.\ (\ref{eq:ax-state}).  The $\boldsymbol \gamma$ level with
$\mathcal{N}=3$, on the other hand, is strongly affected by HH-LH
coupling, which makes this level a mixture of different oscillator
states $\ket{N}$ associated with the different spinor components
representing the HH and LH parts of its wavefunction.  The
decomposition of the $\boldsymbol \gamma$ level yields approximately
49\% $s=+3/2$ with $N=0$, 45\% $s=+1/2$ with $N=1$, and 6\% $s=-3/2$
with $N=3$. This can explain why the traces we observe at and near
$\nu=3/2$ at large angles, e.g., at $\theta \simeq 70^{\circ}$, are
intriguingly unique as they exhibit features normally seen in an
\textit{electron} system in LLs with different $N$: $N=0$ (numerous
odd-denominator FQHSs at 6/5, 11/9, etc.), $N=1$ (an
even-denominator FQHS), and $N>1$ LLs (states with extremely
anisotropic transport coefficients). It is also worth emphasizing
that, while the anisotropic $\nu=3/2$ FQHS in our sample abruptly
appears after the LL crossing at $B_{\parallel} \simeq 8$ T, it
persists well past the crossing, even when $B_{\parallel} \simeq 15$
T.

\begin{figure}[t!]
  \begin{center}
    \psfig{file=Fig4_LL, width=0.5\textwidth}
  \end{center}
  \caption{\label{LL} 
   Calculated LL diagram at $\theta=0^{\circ}$. The red dash-dotted line traces the Fermi energy. The position of $E_F$ at $\nu=3/2$ is marked with a star. Inset: $R_{xx}$ at $\nu=3/2$ vs. $B_{\parallel}$.
   }
  \label{fig:LL}
\end{figure}

We emphasize that the above calculations are performed in purely perpendicular magnetic fields and should be interpreted cautiously when tilted fields are involved. In large $B_{\parallel}$, the HH-LH coupling may be different. It is unlikely that the reduced symmetry in tilted fields would reduce the amount of HH-LH coupling experienced by the $\boldsymbol \gamma$ level. It is indeed more likely that the eigenstates in a tilted field would become yet more complicated linear combinations of different oscillator states $\ket{N}$. Note that such combinations are unique in GaAs 2DHSs, and cannot be achieved in GaAs 2DESs. Our results presented here demonstrate the power of applying $B_{||}$ as a knob to engineer the properties of 2D hole LLs and create exotic, new ground states. They also highlight the need for calculations of 2DHS LLs in titled magnetic fields; such calculations, while computationally very challenging, would be very helpful in understanding the physics of the observed ground states.

\begin{acknowledgments}

We acknowledge support by the National Science Foundation (NSF) (Grants No. DMR 2104771 and No. ECCS 1906253) for measurements, the U.S. Department of Energy (DOE) Basic Energy Sciences (Grant No. DEFG02-00-ER45841) for sample characterization, and the Eric and Wendy Schmidt Transformative Technology Fund and the Gordon and Betty Moore Foundation's EPiQS Initiative (Grant No. GBMF9615 to L. N. P.) for sample fabrication. Our measurements were partly performed at the National High Magnetic Field Laboratory (NHMFL), which is supported by the NSF Cooperative Agreement No. DMR 1644779, by the State of Florida, and by the DOE. This research is funded in part by QuantEmX grant from Institute for Complex Adaptive Matter and the Gordon and Betty Moore Foundation through Grant No. GBMF9616 to C. W., A. G., and M. S. We thank L. Jiao and T. Murphy at NHMFL for technical assistance, and Jainendra K. Jain and Ajit C. Balram for illuminating discussions.

\end{acknowledgments}

\begin{thebibliography}{99}

\bibitem{Jain.Book.2007} J. K. Jain, \textit{Composite Fermions}, (Cambridge Univ. Press, 2007).

\bibitem{Halperin.Jain.Book.2020} \textit{Fractional Quantum Hall Effects: New Developments}, edited by B. I. Halperin and J. K. Jain (World Scientific, Singapore, 2020).

\bibitem{Willett.PRL.1987} R. Willett, J. P. Eisenstein, H. L. St\"ormer, D. C. Tsui, A. C. Gossard, and J. H. English, Observation of an even-denominator quantum number in the fractional quantum Hall effect, Phys. Rev. Lett. {\bf 59}, 1776 (1987).

\bibitem{Pan.PRL.1999} W. Pan, J.-S. Xia, V. Shvarts, D. E. Adams, H. L. Stormer, D. C. Tsui, L. N. Pfeiffer, K. W. Baldwin, and K. W. West, Exact Quantization of the Even-Denominator Fractional Quantum Hall State at $\nu=5/2$ Landau Level Filling Factor, Phys. Rev. Lett. {\bf 83}, 3530 (1999).

\bibitem{Haldane.PRL.1988} F. D. M. Haldane and E. H. Rezayi, Spin-singlet wave function for the half-integral quantum Hall effect, Phys. Rev. Lett. {\bf 60}, 956 (1988).

\bibitem{Moore.NPB.1991} G. Moore and N. Read, Nonabelions in the fractional quantum Hall effect, Nucl. Phy. B {\bf 360(2-3)}, 362 (1991).

\bibitem{Morf.PRL.1998} R. H. Morf, Transition from Quantum Hall to Compressible States in the Second Landau Level: New Light on the $\nu=5/2$ Enigma, Phys. Rev. Lett. {\bf 80}, 1505 (1998).

\bibitem{Nayak.RMP.2008} Chetan Nayak, Steven H. Simon, Ady Stern, Michael Freedman, and Sankar Das Sarma, Non-Abelian anyons and topological quantum computation, Rev. Mod. Phys. {\bf 80}, 1083 (2008).

\bibitem{Suen.PRL.1994} Y. W. Suen, H. C. Manoharan, X. Ying, M. B. Santos, and M. Shayegan, Origin of the $\nu=1/2$ fractional quantum Hall state in wide single quantum wells, Phys. Rev. Lett. {\bf 72}, 3405 (1994).

\bibitem{Falson.Nat.Phy.2015} J. Falson, D. Maryenko, B. Friess, D. Zhang, Y. Kozuka, A. Tsukazaki, J. H. Smet, and M. Kawasaki, Even-denominator fractional quantum Hall physics in ZnO, Nat. Phys. {\bf 11}, 347 (2015).

\bibitem{Hossain.PRL.2018} Md. Shafayat Hossain, Meng K. Ma, Y. J. Chung, L. N. Pfeiffer, K. W. West, K. W. Baldwin, and M. Shayegan, Unconventional Anisotropic Even-Denominator Fractional Quantum Hall State in a System with Mass Anisotropy, Phys. Rev. Lett. {\bf 121}, 256601 (2018).

\bibitem{Shi.Nat.Nanotech.2020} Qianhui Shi, En-Min Shih, Martin V. Gustafsson, Daniel A. Rhodes, Bumho Kim, Kenji Watanabe, Takashi Taniguchi, Zlatko Papi\'c, James Hone and Cory R. Dean, Odd- and even-denominator fractional quantum Hall states in monolayer WSe$_2$, Nat. Nanotechnol. {\bf 15}, 569 (2020).

\bibitem{Zibrov.Nature.2017} A. A. Zibrov, C. Kometter, H. Zhou, E. M. spanton, T. Taniguchi, K. Watanabe, M. P. Zaletel, and A. F. Young, Tunable interacting composite fermion phases in a half-filled bilayer-graphene Landau level, Nature (London) {\bf 549}, 360 (2017).

\bibitem{Li.Science.2017} J. I. A. Li, C. Tan, S. Chen, Y. Zeng, T. Taniguchi, K. Watanabe, J. Hone, and C. R. Dean, Even-denominator fractional quantum Hall states in bilayer graphene, Science {\bf 358}, 648 (2017).

\bibitem{Huang.PRX.2022} Ke Huang, Hailong Fu, Danielle Reifsnyder Hickey, Nasim Alem, Xi Lin, Kenji Watanabe, Takashi Taniguchi, and Jun Zhu, Valley Isospin Controlled Fractional Quantum Hall States in Bilayer Graphene, Phys. Rev. X {\bf 12}, 031019 (2022).

\bibitem{Koulakov.PRL.1996} A. A. Koulakov, M. M. Fogler, and B. I. Shklovskii, Charge Density Wave in Two-Dimensional Electron Liquid in Weak Magnetic Field, Phys. Rev. Lett. {\bf 76}, 499 (1996).

\bibitem{Fogler.PRB.1996} M. M. Fogler, A. A. Koulakov, and B. I. Shklovskii, Ground state of a two-dimensional electron liquid in a weak magnetic field, Phys. Rev. B {\bf 54}, 1853 (1996).

\bibitem{Moessner.PRB.1996} R. Moessner and J. T. Chalker, Exact results for interacting electrons in high Landau levels, Phys. Rev. B {\bf 54}, 5006 (1996).

\bibitem{Fradkin.PRB.1999} Eduardo Fradkin and Steven A. Kivelson, Liquid-crystal phases of quantum Hall systems, Phys. Rev. B {\bf 59}, 8065 (1999).

\bibitem{Fradkin.PRL.2000} Eduardo Fradkin, Steven A. Kivelson, Efstratios Manousakis, and Kwangsik Nho, Nematic Phase of the Two-Dimensional Electron Gas in a Magnetic Field, Phys. Rev. Lett. {\bf 84}, 1982 (2000).

\bibitem{Fradkin.ARCMP.2010} E. Fradkin, S. A. Kivelson, M. J. Lawler, J. P. Eisenstein, and A. P. Mackenzie, Nematic Fermi Fluids in Condensed Matter Physics, Annu. Rev. Condens. Matter Phys. {\bf 1}, 153 (2010).

\bibitem{Lilly.PRL.1999} M. P. Lilly, K. B. Cooper, J. P. Eisenstein, L. N. Pfeiffer, and K. W. West, Evidence for an Anisotropic State of Two-Dimensional Electrons in High Landau Levels, Phys. Rev. Lett. {\bf 82}, 394 (1999).

\bibitem{Du.SSC.1999} R. R. Du, D. C. Tsui, H. L. Stormer, L. N. Pfeiffer, K. W. Baldwin, and K. W. West, Strongly anisotropic transport in higher two-dimensional Landau levels, Solid State Commun. {\bf 109}, 389 (1999).

\bibitem{Fu.PRL.2020} X. Fu, Q. Shi, M. A. Zudov, G. C. Gardner, J. D. Watson, M. J. Manfra, K. W. Baldwin, L. N. Pfeiffer, K. W. West, Anomalous Nematic States in High Half-Filled Landau Levels, Phys. Rev. Lett. {\bf 124}, 067601 (2020).

\bibitem{Chung.PRM.2022} Yoon Jang Chung, C. Wang, S. K. Singh, A. Gupta, K. W. Baldwin, K. W. West, R. Winkler, M. Shayegan, and L. N. Pfeiffer, Record-quality GaAs two-dimensional hole systems, Phys. Rev. Mat. {\bf 6}, 034005 (2022).

\bibitem{Wang.PRL.2022} Chengyu Wang, A. Gupta, S. K. Singh, Y. J. Chung, K. W. Baldwin, K. W. West, L. N. Pfeiffer, R. Winkler, and M. Shayegan, Even-denominator fractional quantum Hall effect at filling factor $\nu=3/4$, Phys. Rev. Lett. {\bf 129}, 156801 (2022).

\bibitem{Footnote.3/4} The same 2DHS also exhibits a FQHS at the even-denominator filling $\nu=3/4$; see Ref. \cite{Wang.PRL.2022} and also the Supplemental Material \cite{SM}.

\bibitem{SM} See Supplemental Material for additional data and analysis, which includes Refs. [29-44].

\bibitem{Shayegan.PhysicaE.2000} M. Shayegan, H. C. Manoharan, S. J. Papadakis, E. P. DePoortere, Anisotropic transport of two-dimensional holes in high Landau levels, Physica E (Amsterdam) {\bf 6}, 40 (2000).

\bibitem{Manfra.PRL.2007} M. J. Manfra, R. de Picciotto, Z. Jiang, S. H. Simon, L. N. Pfeiffer, K. W. West, and A. M. Sergent, Impact of Spin-Orbit Coupling on Quantum Hall Nematic Phases, Phys. Rev. Lett. {\bf 98}, 206804 (2007).

\bibitem{Koduvayur.PRL.2011} S. P. Koduvayur, Y. Lyanda-Geller, S. Khlebnikov, G. Csathy, M. J. Manfra, L. N. Pfeiffer, K. W. West, and L. P. Rokhinson, Effect of Strain on Stripe Phases in the Quantum Hall Regime, Phys. Rev. Lett. {\bf 106}, 016804 (2011).

\bibitem{Croxall.PRB.2019} A. F. Croxall, F. Sfigakis, J. Waldie, I. Farrer, and D. A. Ritchie, Orientation of hole quantum Hall nematic phases in an out-of-plane electric field, Phys. Rev. B {\bf 99}, 195420 (2019).

\bibitem{Mueed.PRL.2015} M. A. Mueed, D. Kamburov, M. Shayegan, L. N. Pfeiffer, K. W. West, K. W. Baldwin, and R. Winkler, Splitting of the Fermi Contour of Quasi-2D Electrons in Parallel Magnetic Fields, Phys. Rev. Lett. {\bf 114}, 236404 (2015).

\bibitem{Chung.NM.2021} Yoon Jang Chung, K. A. Villegas Rosales, K. W. Baldwin, P. T. Madathil, K. W. West, M. Shayegan and L. N. Pfeiffer, Ultra-high-quality two-dimensional electron systems, Nat. Mater. {\bf 20}, 632-637 (2021).

\bibitem{Shi.PRB.2015} X. Shi, W. Pan, K. W. Baldwin, K. W. West, L. N. Pfeiffer, and D. C. Tsui, Impact of the modulation doping layer on the $\nu=5/2$ anisotropy, Phys. Rev. B {\bf 91}, 125308 (2015).

\bibitem{Suzuki.PRB.1974} K. Suzuki and J. C. Hensel, Quantum resonances in the valence bands of germanium. I. Theoretical considerations, Phys. Rev. B {\bf 9}, 4184 (1974).

\bibitem{Ekenberg.PRB.1985} U. Ekenberg and M. Altarelli, Subbands and Landau levels in the 2D hole gas at the GaAs-Al$_x$Ga$_{1-x}$As interface,
Phys. Rev. B {\bf 32}, 3712 (1985).

\bibitem{Mulligan.PRB.2011} Michael Mulligan, Chetan Nayak, and Shamit Kachru, Effective field theory of fractional quantized Hall nematics, Phys. Rev. B {\bf 84}, 195124 (2011).

\bibitem{Regnualt.PRB.2017} N. Regnault, J. Maciejko, S. A. Kivelson, and S. L. Sondhi, Evidence of a fractional quantum Hall nematic phase in a microscopic model, Phys. Rev. B {\bf 96}, 035150 (2017).

\bibitem{Yang.PRR.2020} Bo Yang, Microscopic theory for nematic fractional quantum Hall effect, Phys. Rev. Research {\bf 2}, 033362 (2020).

\bibitem{Wan.PRB.2016} X. Wan and K. Yang, Striped quantum Hall state in a half-filled Landau level, Phys. Rev. B {\bf 93}, 201303(R) (2016).

\bibitem{Santos.PRX.2019} Luiz H. Santos, Yuxuan Wang, and Eduardo Fradkin, Pair-Density-Wave Order and Paired Fractional Quantum Hall Fluids, Phys. Rev. X {\bf 9}, 021047 (2019).

\bibitem{Fradkin.RMP.2015} Eduardo Fradkin, Steven A. Kivelson, and John M. Tranquada, Colloquium: Theory of intertwined orders in high temperature superconductors, Rev. Mod. Phys. {\bf 87}, 457 (2015).

\bibitem{Agterberg.ARCMP.2020}  D. F. Agterberg, J. C. S. Davis, S. D. Edkins, E. Fradkin, D. Harlingen, S. A. Kivelson, P. A. Lee, L. Radzihovsky, J. M. Tranquada, and Y. Wang, The Physics of Pair-Density Waves: Cuprate Superconductors and Beyonds, Annu. Rev. Condens. Matter Phys. {\bf 11}, 231 (2020).

\bibitem{Du.PRL.1995} R. R. Du, A. S. Yeh, H. L. Stormer, D. C. Tsui, L. N. Pfeiffer, and K. W. West, Fractional Quantum Hall Effect around $\nu=3/2$: Composite Fermions with a Spin, Phys. Rev. Lett. {\bf 75}, 3926 (1995).

\bibitem{Chung.NM.2021} Yoon Jang Chung, K. A. Villegas Rosales, K. W. Baldwin, P. T. Madathil, K. W. West, M. Shayegan and L. N. Pfeiffer, Ultra-high-quality two-dimensional electron systems, Nat. Mat. {\bf 20}, 632 (2021).

\bibitem{Liu.PRB.2016} Yang Liu, M. A. Mueed, Md. Shafayat Hossain, S. Hasdermir, L. N. Pfeiffer, K. W. West, K. W. Baldwin, and M. Shayegan, Morphing of two-dimensional hole systems at $\nu=3/2$ in parallel magnetic fields: Compressible, stripe, and fractional quantum Hall phases, Phys. Rev. B {\bf 94}, 155312 (2016)

\bibitem{Lilly.PRL.1999.tilt} M. P. Lilly, K. B. Cooper, J. P. Eisenstein, L. N. Pfeiffer, and K. W. West, Anisotropic States of Two-Dimensional Electron Systems in High Landau Levels: Effect of an In-Plane Magnetic Field, Phys. Rev. Lett. {\bf 83}, 824 (1999).

\bibitem{Pan.PRL.1999.tilt} W. Pan, R. R. Du, H. L. Stormer, D. C. Tsui, L. N. Pfeiffer, K. W. Baldwin, and K. W. West, Strongly Anisotropic Electronic Transport at Landau Level Filling Factor $\nu=9/2$ and $\nu=5/2$ under a Tilted Magnetic Field, Phys. Rev. Lett. {\bf 83}, 820 (1999).

\bibitem{Xia.PRL.2010} Jing Xia, Vaclav Cvicek, J. P. Eisenstein, L. N. Pfeiffer, and K. W. West, Tilt-Induced Anisotropic to Isotropic Phase Transition at $\nu=5/2$, Phys. Rev. Lett. {\bf 105}, 176807 (2010).

\bibitem{Liu.PRB.2013} Yang Liu, S. Hasdemir, M. Shayegan, L. N. Pfeiffer, K. W. West, and K. W. Baldwin, Evidence for a $\nu=5/2$ fractional quantum Hall nematic state in parallel magnetic fields, Phys. Rev. B {\bf 88}, 035307 (2013).

\bibitem{Friess.PRL.2014} Benedikt Friess, Vladimir Umansky, Lars Tiemann, Klaus von Klitzing, and Jurgen H. Smet, Probing the Microscopic Structure of the Stripe Phase at Filling Factor $5/2$, Phys. Rev. Lett. {\bf 113}, 076803 (2014).

\bibitem{Shi.PRB.2015} X. Shi, W. Pan, K. W. Baldwin, K. W. West, L. N. Pfeiffer, and D. C. Tsui, Impact of the modulation doping layer on the $\nu=5/2$ anisotropy, Phys. Rev. B {\bf 91}, 125308 (2015).

\bibitem{Footnote.phases} Transitions between an even-denominator FQHS and a compressible anisotropic phase at $\nu=5/2$ have also been reported in GaAs 2DESs under large hydrostatic pressures \cite{Samkharadze.NP.2016, Schreiber.NC.2018}, and in ZnO 2DESs in tilted fields \cite{Falson.SA.2018}. No coexisting FQHSs and anisotropic phases were observed in these cases.

\bibitem{Samkharadze.NP.2016} N. Samkharadze, K. A. Schreiber, G. C. Gardner, M. J. Manfra, E. Fradkin, and G. A. Cs\'athy, Observation of a transition from a topologically ordered to a spontaneously broken symmetry phase, Nat. Phys. {\bf 12}, 191 (2016).

\bibitem{Schreiber.NC.2018} K. A. Schreiber, N. Samkharadze, G. C. Gardner, Y. Lyanda-Geller, M. J. Manfra, L. N. Pfeiffer, K. W. West, and G. A. Cs\'athy, Electron-electron interactions and the paired-to-nematic quantum phase transition in the second Landau level, Nat. Commun. {\bf 9}, 2400 (2018).

\bibitem{Falson.SA.2018} J. Falson, D. Tabrea, D. Zhang, I. Sodemann, Y. Kozuka, A. Tsukazaki, M. Kawasaki, K. von Klitzing, and J. H. Smet, A cascade of phase transitions in an orbitally mixed half-filled Landau level, Sci. Adv. {\bf 4}, eaat8742 (2018).

\bibitem{Pan.PRB.2008} W. Pan, J. S. Xia, H. L. Stormer, D. C. Tsui, C. Vicente, E. D. Adams, N. S. Sullivan, L. N. Pfeiffer, K. W. Baldwin, and K. W. West, Experimental studies of the fractional quantum Hall effect in the first excited Landau level, Phys. Rev. B {\bf 77}, 075307 (2008).

\bibitem{Footnote.second.LL}  In ultra-high-mobility 2DESs, equivalent FQHSs have been reported in the second LL, e. g., at $\nu=2+1/3$, $2+1/5$, $2+4/5$, and $2+7/9$; see Ref. \cite{Kumar.PRL.2010}.

\bibitem{Kumar.PRL.2010} A. Kumar, G. A. Cs\'athy, M. J. Manfra, L. N. Pfeiffer, and K. W. West, Nonconventional Odd-Denominator Fractional Quantum Hall States in the Second Landau Level, Phys. Rev. Lett. {\bf 105}, 246808 (2010).

\bibitem{Footnote.Rbg} We define background resistance  near $\nu=3/2$ ($R^{bg}_{xx}$) by connecting a straight line between the two $R_{xx}$ peaks on the two sides of $\nu=3/2$ and picking the resistance value at the field position of $\nu=3/2$.

\bibitem{Mallett.PRB.1988} J. R. Mallett, R. G. Clark, R. J. Nicholas, R. Willett, J. J. Harris, and C. T. Foxon, Experimental studies of the $\nu=1/5$ hierarchy in the fractional quantum Hall effect, Phys. Rev. B {\bf 38}, 2200(R) (1988).

\bibitem{Xia.Nat.Phys.2011} Jing Xia, J. P. Eisenstein, L. N. Pfeiffer, and K. W. West, Evidence for a fractionally quantized Hall state with anisotropic longitudinal transport, Nat. Phys. {\bf 7}, 845 (2011).

\bibitem{Sajoto.PRL.1993} T. Sajoto, Y. P. Li, L. W. Engel, D. C. Tsui, and M. Shayegan, Hall resistance of the reentrant insulating phase around the 1/5 fractional quantum Hall liquid, Phys. Rev. Lett. {\bf 70}, 2321 (1993).

\bibitem{Goldman.PRL.1993} V. J. Goldman, J. K. Wang, Bo Su, and M. Shayegan, Universality of the Hall effect in a magnetic-field-localized two-dimensional electron system, Phys. Rev. Lett. {\bf 70}, 647 (1993).

\bibitem{Footnote.Config} In Fig. 3 $R_{xx}$ is measured with a different contact configuration compared to $R_{xx}$ in Figs. 1 and 2.

\bibitem{Footnote.crossing} In Ref. \cite{Liu.PRB.2016}, a transition from a CF Fermi sea to an anisotropic phase near $\nu=3/2$ is also seen in wide-well GaAs 2DHSs at a smaller $\theta$ ($\simeq 65^{\circ}$ instead of $\simeq 69^{\circ}$). This is consistent with our expectation that the LL crossing highlighted in Fig. 4 occurs at a smaller $\theta$ in a 2DHS confined to a wider quantum well.

\bibitem{Zhang.PRB.2017} Po Zhang, Ruiyuan Liu, Rui-Rui Du, L. N. Pfeiffer, and K. W. West, Composite fermion states around the two-dimensional hole Landau level filling factor 3/2 in tilted magnetic fields, Phys. Rev. B {\bf 95}, 155316 (2017).

\bibitem{Footnote.differences} In Ref. \cite{Liu.PRB.2016},  at large $\theta$, an anisotropic phase with \textit{no FQHE features} was observed at $\nu=3/2$. We emphasize that the 2DHSs studied in Ref. \cite{Liu.PRB.2016} are confined to \textit{wide} quantum wells (30 and 35 nm). The absence of the $\nu=3/2$ FQHS in these samples is likely due to the bilayer charge distribution and lower sample quality. See SM \cite{SM} Section III for a detailed discussion.

\bibitem{Winkler.Book.2003} R. Winkler,
\textit{Spin-Orbit Coupling Effects in Two-Dimensional Electron and Hole Systems}, (Springer, Berlin, 2003).

\bibitem{Liu.PRB.2014} Yang Liu, S. Hasdemir, D. Kamburov, A. L. Graninger, M. Shayegan, L. N. Pfeiffer, K. W. West, and K. W. Baldwin, Even-denominator fractional quantum Hall effect at a Landau level crossing, Phys. Rev. B {\bf 89}, 165313 (2014).

\bibitem{Liu.PRB.2015} Yang Liu, S. Hasdemir, M. Shayegan, L. N. Pfeiffer, K. W. West, and K. W. Baldwin, Unusual Landau level pinning and correlated $\nu=1$ quantum Hall effect in hole systems confined to wide GaAs quantum wells, Phys. Rev. B {\bf 92}, 195156 (2015).

\bibitem{Liu.PRL.2016} Yang Liu, S. Hasdemir, L. N. Pfeiffer, K. W. West, K. W. Baldwin, and M. Shayegan, Observation of an Anisotropic Wigner Crystal, Phys. Rev. Lett. {\bf 117}, 106802 (2016).

\bibitem{Ma.PRL.2022} Meng K. Ma, Chengyu Wang, Y. J. Chung, L. N. Pfeiffer, K. W. West, K. W. Baldwin, R. Winkler, and M. Shayegan, Robust Quantum Hall Ferromagnetism near a Gate-Tuned $\nu=1$ Landau Level Crossing, Phys. Rev. Lett. {\bf 129}, 196801 (2022).

\bibitem{Suzuki.PRB.1974} K. Suzuki and J. C. Hensel, Quantum
 resonances in the valence bands of germanium. I. Theoretical
 considerations, Phys. Rev. B {\bf 9}, 4184 (1974).

\bibitem{MacDonald.1995} A. H. MacDonald, Introduction to the
 Physics of the Quantum Hall Regime, in \textit{Mesoscopic Quantum
 Physics}, edited by E. Akkermans \emph{et al.} (Elsevier,
 Amsterdam, 1995), p. 659.

\bibitem{Footnote.spinor} For conceptual clarity, we ignore that the more accurate Hamiltonian underlying our numerical calculations uses 8 spinor components \cite{Winkler.Book.2003}.

\bibitem{Footnote.2DES.mixing} In 2DESs, a typically
small mixing between the $s=+1/2$ and $-1/2$ states similar to
Eq.\ (\ref{eq:ax-state}) is induced, e.g., by a Rashba term.  A
Dresselhaus term breaks the axial symmetry; it mixes all $\ket{N}$
\cite{Winkler.Book.2003}.

\end{thebibliography}

\foreach \x in {1,...,15}
{
\clearpage
\includepdf[pages={\x,{}}]{Supplemental.pdf}
}

\end{document}
