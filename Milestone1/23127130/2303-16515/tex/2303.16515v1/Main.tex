\documentclass[12pt]{article}

\usepackage[utf8]{inputenc}
\usepackage{url, color, epsfig, amsmath, amsthm, amssymb}
\usepackage{hyperref}
\usepackage{enumerate}
\usepackage{graphicx,wrapfig,lipsum}
\usepackage{svg}
\usepackage{thmtools}
\usepackage{subcaption}
\usepackage{float}
\usepackage{algorithm,algorithmic}
\usepackage[colorinlistoftodos]{todonotes}
\usepackage{apxproof}
\newcommand{\hide}[1]{}
\newcommand{\D}{\mathcal{D}}
\newcommand{\A}{\mathcal{A}}
\newcommand{\AD}{\mathcal{A}(\mathcal{D})}
\newcommand{\C}{\mathcal{C}}
\newcommand{\R}{\mathbf{r}}
\newcommand{\B}{\mathbf{b}}
\newcommand{\LL}{\mathcal{L}}
\newcommand{\OO}{\mathcal{O}}
\newcommand{\loc}{\textsc{Local}}
\newcommand{\VB}{\textsc{VB}}
\newcommand{\opt}{\textsc{Opt}}
\newcommand{\T}{\mathcal{T}}
\newcommand{\p}{\mathcal{P}}
\newcommand{\eps}{\mathcal \epsilon}
\newcommand{\s}{\mathcal{S}}
\newcommand{\RR}{\mathbb{R}}
\newcommand{\depth}{\textsc{depth}}
\newtheorem{theorem}{Theorem}
\newtheorem{corollary}[theorem]{Corollary}
\newtheorem{lemma}[theorem]{Lemma}
\newtheorem{claim}[theorem]{Claim}
\newtheorem{definition}[theorem]{Definition}
\newtheorem{observation}[theorem]{Observation}
\newtheorem{proposition}[theorem]{Proposition}
\newtheorem{remark}[theorem]{Remark}


\newtheoremstyle{case}{}{}{}{}{}{:}{ }{}
\theoremstyle{case}
\newtheorem{case}{Case}

\DeclareMathOperator{\Ex}{E}
\bibliographystyle{plainurl}

\title{On Hypergraph Supports}
\author{Rajiv Raman\footnote{Part of this work was done when the first author was at LIMOS, Université Clermont Auvergne, and was partially supported by the French government research program “Investissements d’Avenir” through the IDEX-ISITE initiative 16-IDEX-0001 (CAP 20-25).} \\ IIIT-Delhi, India. \\ rajiv@iiitd.ac.in\and Karamjeet Singh \\ IIIT-Delhi, India. \\ karamjeets@iiitd.ac.in}

\setlength {\marginparwidth }{2cm}
\begin{document}
 
\maketitle              
\begin{abstract}
Let $\mathcal{H}=(X,\mathcal{E})$ be a hypergraph. A support is a graph $Q$ on $X$ such that for each
$E\in\mathcal{E}$, the subgraph of $Q$ induced on the elements in $E$ is connected. 
In this paper, we consider hypergraphs
defined on a host graph. Given a graph $G=(V,E)$, with $c:V\to\{\R,\B\}$, and a collection
of connected subgraphs $\mathcal{H}$ of $G$, a primal support is a graph $Q$ on $\B(V)$
such that for each $H\in \mathcal{H}$, the induced subgraph $Q[\B(H)]$ on vertices $\B(H)=H\cap c^{-1}(\B)$
is connected. A \emph{dual support} is a graph $Q^*$ on $\mathcal{H}$ s.t.
for each $v\in X$, the induced subgraph $Q^*[\mathcal{H}_v]$ is connected, where $\mathcal{H}_v=\{H\in\mathcal{H}: v\in H\}$.
We present sufficient conditions on the host graph and hyperedges so that
the resulting support comes from a restricted family.

We primarily study two classes of graphs: $(1)$ If the host graph has genus $g$
and the hypergraphs satisfy a topological condition of being \emph{cross-free}, 
then there is a primal and a dual support of genus at most $g$.
$(2)$ If the host graph has treewidth $t$ and the hyperedges satisfy a combinatorial condition of being \emph{non-piercing}, 
then there exist 
primal and dual supports of treewidth $O(2^t)$.
We show that this exponential blow-up is sometimes necessary. As an intermediate case, we also study the case when the host graph is outerplanar.
Finally, we show applications of our results to packing and covering, and coloring problems on geometric hypergraphs.

\end{abstract}

\section{Introduction}
A hypergraph $(X,\mathcal{E})$ is defined by a set $X$ of elements and a collection $\mathcal{E}$ of subsets of $X$. 
In this paper, we study the notion of a \emph{support} for a hypergraph.
A support is a graph $Q$ on $X$ s.t. $\forall\; E\in\mathcal{E}$, the subgraph induced by $E$ in $Q$,  $Q[E]$ is connected.
The notion of a support was introduced by Voloshina and Feinberg \cite{voloshina1984planarity} in the context of VLSI circuits.
Since then, this notion has found wide applicability in several areas, such as visualizing hypergraphs 
\cite{bereg2015colored,bereg2011red,brandes2010blocks,brandes2012path,buchin2011planar,havet2022overlaying,hurtado2018colored},
in the design of networks \cite{anceaume2006semantic,baldoni2007tera,baldoni2007efficient,chand2005semantic,hosoda2012approximability,korach2003clustering,onus2011minimum}, and 
similar notions have been used in the analysis of local search algorithms for geometric problems \cite{Mustafa17,BasuRoy2018,Cohen-AddadM15,krohn2014guarding,mustafa2010improved,RR18}.

Any hypergraph clearly has a support - A complete graph on $X$ is a support.
The problem becomes interesting if we introduce a global constraint on the graph that is in \emph{tension}
with the \emph{local} connectivity requirement for each hyperedge.
In particular, we are interested in the restrictions on the hypergraph that guarantees
the existence of a support from a \emph{sparse} family of graphs.

We can broadly classify sparse graphs into two categories - those that are \emph{easily decomposable},
for example, graphs with sublinear sized balanced separators\footnote{A graph has a sublinear sized balanced separator if there are constants $\epsilon >0$, and $c>0$ and a set $S$ of size $O(|V|^{1-\epsilon})$
such that $G\setminus S$ contains two disconnected components $A$ and $B$ such that $|A|,|B|\le c|V|$.
}, and graphs that do not admit small separators, such as \emph{expanders}
\cite{holberg1992decomposition}. Examples of the former are planar graphs \cite{lipton1979separator}, graphs of bounded genus \cite{gilbert1984separator}, graphs excluding a minor \cite{alon1990separator}, and more generally, graphs with polynomially bounded $t$-shallow minors \cite{nevsetvril2012sparsity}. 
The fact that a family of graphs is \emph{easily decomposable} has been exploited to develop faster algorithms, or algorithms with better approximation factors than in general graphs. Some results that use this paradigm are \cite{AW13,DBLP:conf/focs/Arora97,Frederickson87,DBLP:journals/cpc/FoxP14,lipton1980applications}. In a similar vein, one would like to develop a theory of \emph{easily decomposable} hypergraphs, and exploit this notion algorithmically. The existence of a sparse support for a hypergraph is such a notion.

In this paper, motivated by problems of packing and covering we study hypergraphs defined on a sparse \emph{host graph}. We show sufficient conditions under which the hypergraph defined on this host graph has a sparse support, and present applications of this result in the analysis of several packing, covering and coloring problems.


\section{Related Work}
The notion of a planar analogue of a hypergraph was first suggested by Zykov \cite{zykov}, who defined a hypergraph to be planar if there is a plane graph on the elements of the hypergraph s.t. for each hyperedge, there
is a bounded face of the embedding containing only the elements of this hyperedge. Equivalently, a hypergraph is \emph{Zykov-planar} iff its incidence bipartite graph is planar. 
Voloshina and Feinberg \cite{voloshina1984planarity} introduced the notion of hypergraph planarity that is now called \emph{a planar support}
 in the context of planarizing VLSI circuits (see  the monograph by Feinberg et al., \cite{feinberg2012vlsi} and
references therein). 
Johnson and Pollak \cite{Johnson1987HypergraphPA} showed that the problem of deciding if a hypergraph has a planar support is NP-hard.

Since then, several authors have studied the question of deciding if a hypergraph admits a support from a restricted family of graphs.
Tarjan and Yannakakis \cite{tarjan1984simple} showed that we can decide in linear time if a hypergraph admits a tree support.
Buchin et al. \cite{buchin2011planar} showed linear time algorithms to decide if a hypergraph admits a support that is a path or a cycle, 
and a polynomial time algorithm to decide if a hypergraph admits a support that is a tree with bounded degree.
Further, the authors sharpen the result of Johnson and Pollak \cite{Johnson1987HypergraphPA} by showing that deciding if a hypergraph admits a support that is a 2-outerplanar\footnote{A graph is 2-outerplanar if the graph can be embedded in the plane such that the vertices are on two concentric circles and removing all vertices of outer face results in an outerplanar graph.} graph is NP-hard.
The notion of constructing a support with the fewest number of edges, or with minimum maximum degree have also been studied in \cite{baldoni2007tera,onus2011minimum}.

Another motivation for studying the existence of sparse supports, and the main motivation for our work comes from
the design and analysis of algorithms for  geometric hypergraphs. 
Chan and Har-Peled \cite{ChanH12}, and  Mustafa and Ray \cite{mustafa2010improved} used the existence of a planar graph with a \emph{simple locality property} to show that for pseudodisks (and more generally, simply connected k-admissible regions\footnote{A set of connected bounded regions $\mathcal{R}$ in the plane, each of whose boundary is a simple jordan curve is $k$-admissible (for even $k$) if for any $R,R'\in\mathcal{R}$, $R\setminus R'$ and $R'\setminus R$ are connected, their boundaries are in general position, and 
intersect each other at most $k$ times. If $k=2$, the regions are called pseudodisks.}), a local search algorithm yields a PTAS for the Independent Set problem, and the Hitting Set problem, respectively. Their analyses rely on the existence of a graph that
satisfies the \emph{simple locality property} and is from a class that admits sublinear separators.
In such analysis, the main difficulty of showing whether local search yields a PTAS lies in showing the existence of an appropriate graph.
The results of \cite{ChanH12,mustafa2010improved} were extended by Basu-Roy et al. \cite{BasuRoy2018} who constructed the appropriate graphs corresponding to the Set Cover, Dominating Set problem, and the Capacitated Packing problem.
Raman and Ray \cite{RR18}  generalized this further and gave a unified construction which implied a planar support for general non-piercing regions in the plane (not necessarily simply connected)\footnote{This includes disks, pseudodisks, unit height-axis parallel rectangles, halfspaces, etc.}. In \cite{DBLP:journals/dcg/RamanR22}, it was shown that 
LP-rounding can be combined with local search to obtain $(2+\epsilon)$-approximations for the problem of hitting set and set cover with capacities for hypergraphs obtained from
points and pseudodisks. 
Besides packing and covering problems, the existence of a planar support  implies that the hypergraph is 4-colorable, a quesetion considered by Keller and Smorodinsky \cite{KellerS18} and Keszegh\cite{Keszegh20}. Further hypergraph coloring questions were studied by P{\'{a}}lv{\"{o}}lgyi and Kezegh \cite{DBLP:journals/jocg/KeszeghP19} and
Ackerman et al. \cite{ackerman2020coloring}. Ackerman et al. studied $ABAB$-free hypergraphs
in \cite{ackerman2020coloring} and showed that these hypergraphs admit a 
representation with stabbed pseudodisks and points in the plane. Their results are closely related to our results and we discuss this connection
in Section \ref{sec:applications}.

While there has been some progress on approximation algorithms for geometric problems, the existing techniques are rather weak in that they can handle basic questions in the plane, and the resulting algorithms are far from practical.
The two main motivations of this work are to extend the techniques of \cite{RR18} to higher dimensional surfaces, and to build techniques for proving the existence of supports, an area that does is lacking in techniques currently.
Cabello and Gajser \cite{CabelloG14} showed that Independent Set and Vertex Cover admit a PTAS on graphs excluding a fixed minor, and
 Aschner et al.,\cite{DBLP:conf/walcom/AschnerKMY13} study some packing and covering problems involving geometric non-planar graphs. In both these cases however, the authors do not construct a support, and the existence of a graph
 satisfying the desired locality properties is straightforward. 

\section{Preliminaries}
A graph $G=(V,E)$ and a collection of subgraphs $\mathcal{H}$ of $G$ naturally defines a hypergraph 
$(V(G),\{V(H): H\in\mathcal{H}\})$, where $V(H) = \{v\in V: v\in H\}$. We call the tuple $(G,\mathcal{H})$ a \emph{graph system},
since we will use the pair to define other hypergraphs. 
We primarily study two classes of graphs: graphs that are 2-cell embedded in an orientable surface of genus $g$ (i.e.,
each face of the embedding is homeomorphic to a disk), henceforth called \emph{embedded graphs}. We allow multi-edges and self-loops. 
The second class of graphs we study are graphs of bounded treewidth. As an intermediate case, we also study the setting where $G$ is an outerplanar graph. 

For a vertex $v\in V(G)$, we use $N_G(v)$ to denote the neighbours of $v$ in $G$ 
(or just $N(v)$ if $G$ is clear from context), and we use $N_G[v]$ (or $N[v]$) to denote $N(v)\cup\{v\}$. 
We use $e\sim v$ to denote an edge $e$ incident on vertex $v$. For $S \subseteq V$, we use $G[S]$ to denote the subgraph of $G$ induced on $S$.
We use $\mathcal{H}_v = \{H\in\mathcal{H}: v\in H\}$ 
to we let $\depth(v)=|\mathcal{H}_v|$.

Let $c:V\to\{\R,\B\}$ be an arbitrary coloring of the vertices of $G$ with two colors. 
Let $\B(V)$ and $\R(V)$ denote respectively $c^{-1}(\B)$ and $c^{-1}(\R)$. We also refer to the vertices in $\B(V)$ as \emph{blue vertices} and the vertices in $\R(V)$ as \emph{red vertices}.

A \emph{primal support} for $(G,\mathcal{H})$ is a graph $Q$ on $\B(V)$ s.t. $\forall\; H\in\mathcal{H}$, $Q[\B(H)]$ is connected\footnote{Note that we cannot simply project each $H$ on $\B(V)$ as the resulting subgraphs may not be connected in $G.$}. 
The system $(G,\mathcal{H})$ also defines a \emph{dual hypergraph} $(\mathcal{H}, \{\mathcal{H}_v\}_{v\in V(G)})$. A \emph{dual support} is a graph $Q^*$ on $\mathcal{H}$ s.t. $\forall\; v\in V(G)$, $Q^*[\mathcal{H}_v]$
is connected\footnote{ 
To make the definition symmetric, we could have considered a coloring $c:\mathcal{H}\to\{\R,\B\}$, and 
required that $Q^*[\mathcal{H}^{\B_v}]$ be connected only for each $v\in V$, where $\mathcal{H}^{\B_v} = \{H\in\mathcal{H}: v\in H \mbox{ and } c(H) =\B\}$. However, this problem reduces to constructing a dual support restricted to the hypergraphs $\mathcal{H}^{\B} = \{H\in\mathcal{H}: c(H)=\B\}$. Therefore, in the dual setting, it is sufficient to study the uncolored version of the problem.}.
When constructing the dual support, we can assume without loss of generality that there are no \emph{containments}, i.e., $\forall\; H,H'\in\mathcal{H},\; H\setminus H'\neq\emptyset$ and $H'\setminus H\neq\emptyset$.
\begin{proposition}
\label{prop:nocontainment}
For a system $(G,\mathcal{H})$ if $\mathcal{H}'\subseteq\mathcal{H}$ is a maximal subset of $\mathcal{H}$ such that $\forall\; H, H'\in\mathcal{H}'$, $H\setminus H'\neq\emptyset$ and $H'\setminus H\neq\emptyset$,
then from a dual support $Q'$ on $\mathcal{H}'$, we can obtain a dual support $Q^*$ on $\mathcal{H}$ s.t. $Q^*$ has the same genus as $Q'$, and $Q^*$ has the same treewidth as $Q'$.
\end{proposition}
\begin{proof} 
Consider a partial order $(\mathcal{H},\prec)$, where $H\prec H'$ if
$H\subseteq H'$. If all elements in $\mathcal{H}$ are pairwise incomparable, we are done. 
Otherwise, let $H\subseteq H'$ for some $H,H'\in\mathcal{H}$. We can assume wlog that $H'$ is the immediate successor of $H$ wrt $\prec$. Let $\mathcal{H}_1=\mathcal{H}\setminus \{H\}$. Then $\mathcal{H}'$ is also maximal for $\mathcal{H}_1$. By induction, the system $(G,\mathcal{H}_1)$ has
a dual support $Q_1^*$ embedded in $\Sigma$ s.t. $Q_1^*$ is obtained from a dual support $Q'$ of $\mathcal{H}'$ and that $tw(Q_1^*)=tw(Q')$.

We claim that the graph $Q^*$ obtained from $Q_1^*$ by adding a vertex for $H$ and attaching it to
an arbitrary subgraph that is an immediate successor of $H$, i.e., \emph{a parent} of $H$ in $\prec$ via a single edge yields a support for $(G,\mathcal{H})$. 
For each vertex $v\in V(G)$, the induced subgraph $Q_1^*[\mathcal{H}_{1_v}]$, on the subgraphs
in $\mathcal{H}_1$ containing $v$, is connected. Consider the parent $H'$ of $H$ in the partial order. For any vertex $v\in V(G)$ such that $v\in H$, it follows
that $v\in H'$. Thus, adding the edge $\{H, H'\}$ guarantees that $Q^*[\mathcal{H}_v]$ is
connected for each $v\in V$. Clearly, $Q^*$ is also embeddable on $\Sigma$. A similar argument implies that $Q^*$ has the same treewidth as $Q'$.
\end{proof}

Let $\mathcal{H}$ and $\mathcal{K}$ be two sets of connected subgraphs of a graph $G$, and let $(G,\mathcal{H},\mathcal{K})$ denote the \emph{intersection hypergraph}
$(\mathcal{H},\{\mathcal{H}_K\}_{K\in\mathcal{K}})$, where 
$\mathcal{H}_K = \{H\in\mathcal{H}: K\cap H \neq\emptyset\}$.
A support for $(G,\mathcal{H},\mathcal{K})$ is then a graph $\tilde{Q}=(\mathcal{H}, F)$ s.t. 
$\forall K\in\mathcal{K}$, $\tilde{Q}[\mathcal{H}_{K}]$ is connected. $\tilde{Q}$ is called the \emph{intersection support}.
The notion of an intersection hypergraph generalizes both the primal and dual hypergraphs defined above
- taking the vertices of $G$ as the singleton sets of $\mathcal{H}$ for the primal, 
and taking the vertices of $G$ as the singleton sets of $\mathcal{K}$ for the dual, respectively. 

Our goal is to consider restrictions of hypergraphs so that the support is guaranteed to be from a restricted family of graphs. 
To that end, we introduce a notion of cross-free hypergraphs and non-piercing hypergraphs.

\begin{definition}[Reduced graph]
Let $(G,\mathcal{H})$ be a graph system with $G$, an embedded graph. 
For any two subgraphs $H, H'\in\mathcal{H}$, the \emph{reduced graph} $R_G(H,H')$ (or just $R(H, H')$ if $G$ is clear from context) is the graph obtained by contracting all edges, both
of whose end-points lie in $H\cap H'$, where multi-edges and self-loops are retained.
\end{definition}

Note that if $G$ can be embedded in a surface $\Sigma$, then so can be $R_G(H,H')$.


\begin{definition}[Cross-free at $v$]
A graph system $(G,\mathcal{H})$ with $G$ embedded in an oriented surface, is cross-free at a vertex
$v\in V(G)$ if for any
two subgraphs $H, H'\in\mathcal{H}_v$, the following holds: Let $\tilde{v}$ be the image of $v$ in the reduced graph $R(H, H')$. Then, there are no 4 edges 
$e_i=\{\tilde{v},v_i\}$ in $R(H,H')$, $i=1,\ldots, 4$ incident to $\tilde{v}$ in cyclic order around $\tilde{v}$,
s.t.  $v_1, v_3\in H\setminus H'$, and $v_2, v_4\in H'\setminus H$. 
\end{definition}

If there exist $H, H'\in \mathcal{H}$ s.t. $H$ and $H'$ are not cross-free at $v$, we say that $H$ and $H'$ are \emph{crossing} at $v$.
$(G,\mathcal{H})$ is \emph{cross-free} if there exists an embedding of $G$ s.t. $(G,\mathcal{H})$ is cross-free at each $v\in V(G)$. 
Note that for cross-free system, we implicitly assume
$\mathcal{H}$ is a collection of connected subgraphs.

An intersection hypergraph $(G,\mathcal{H},\mathcal{K})$
is cross-free
if both $(G,\mathcal{H})$ and $(G,\mathcal{K})$ are cross-free.
Note that we can have $H\in\mathcal{H}$, $K\in\mathcal{K}$ that are crossing.
Finally, we use the term $(G,\mathcal{H})$ is a cross-free system of genus $g$ or the term
$(G,\mathcal{H},\mathcal{K})$ is an intersection system of genus $g$ to mean that the host
graph $G$ has genus $g$.

We need one more definition, namely that of $abab$-free hypergraphs before we move on to non-piercing subgraphs. 
An equivalent notion, namely $ABAB$-free hypergraphs was studied by Ackerman et al. 
\cite{ackerman2020coloring}, where the elements
of the hypergraph are placed in a linear order instead of a cyclic order.

\begin{definition}[$abab$-free]
\label{defn:abab}
A hypergraph $(X,\mathcal{H})$ is said to be $abab$-free if there is a cycle $C$ on $X$ s.t.
there are no four vertices $x_1, x_2, x_3, x_4$ in cyclic order around $C$ s.t. $x_1,x_3\in H\setminus H'$,
and $x_2,x_4\in H'\setminus H$.
\end{definition}

\begin{definition}[Non-piercing]
\label{defn:nonpiercing}
A graph system $(G,\mathcal{H})$ with $\mathcal{H}$ a collection of subgraphs of $G$
is non-piercing if each $H\in\mathcal{H}$ is connected and for any two 
subgraphs $H, H'\in\mathcal{H}$, $H\setminus H'$ induces a connected
subgraph of $H$, and $H'\setminus H$ induces a connected subgraph of $H'$.
\end{definition}

We say $(G,\mathcal{H})$ is a  non-piercing system of treewidth $t$ to mean it is a non-piercing system
defined on a host graph $G$ whose treewidth is $t$.


Note that non-piercing is a purely combinatorial notion, 
and unlike the cross-free property above, it does not require an embedding of the graph. 
If $\exists\; H, H'\in\mathcal{H}$ s.t. either the indued subgraph $H\setminus H'$ of $H$ or the induced subrgaph $H'\setminus H$ of $H'$ is not connected, then 
$H$ and $H'$ are \emph{piercing}. Otherwise, $(G,\mathcal{H})$ is a non-piercing system.

Given a graph $G=(V,E)$, a tree-decomposition of $G$ is a pair $(T,\mathcal{B})$, 
where $T$ is a tree and $\mathcal{B}$ is a collection of \emph{bags} - subgraphs of $G$ indexed by the nodes of $T$\footnote{Throughout the paper, we use the term \emph{node} to refer
to the elements of $V(T)$ and \emph{vertices} to refer to the elements of $V(G)$.}, that satisfies the following properties:
$(i)$ For each edge, $\{u,v\}\in E(G)$ there is a bag $B\in\mathcal{B}$ that contains both $u$ and $v$. $(ii)$ For each vertex $v\in v(G)$, the set of bags containing $v$ induce
a connected sub-tree of $T$. 

The \emph{width} of a tree-decomposition is defined as $\max_{x\in V(T)} |B_x|-1$. The treewidth of $G$ is the minimum width of a tree-decomposition of $G$, denoted by $tw(G)$.
See \cite{bodlaender1998partial} for additional properties of a tree-decomposition. 

An outerplanar graph is a graph that can be embedded in the plane such that all vertices lie on the outer face. It is a well-known
fact that an outerplanar graph has treewidth at most 2 and that an embedding of an outerplanar graph can be obtained in polynomial time.
If $(G,\mathcal{H})$ is a (cross-free/non-piercing) system where $G$ is an outerplanar graph, 
we call it an \emph{outerplanar (cross-free/non-piercing)} system.
\section{Contribution}
The existence of a support from a family with sublinear sized 
balanced separators
has been at the heart of the analysis of PTASes for several packing and covering problems with geometric hypergraphs
starting with the work of Chan and Har-Peled \cite{ChanH12} and Mustafa and Ray \cite{mustafa2010improved}. Raman and Ray in \cite{RR18}
unified earlier constructions of a support for non-piercing regions in the plane.

A natural question is whether we can extend these results from the plane to surfaces of higher genus or general graphs. 
The main contribution of our paper is a generalization of the results of \cite{RR18}. 
We study two settings: When the host graph has bounded genus, 
and when the host graph has bounded treewidth.
While the results for bounded genus graphs roughly follow the proof outline of \cite{RR18}, 
several new ideas are required for the proofs to go through. In particular, it turns out that
for graphs of bounded genus, the non-piercing condition is insufficient for the existence of sparse supports. 
We introduce the notion of \emph{cross-free} subgraphs and show that if $(G,\mathcal{H})$ is a cross-free
system and $G$ has genus $g$,  then the primal, dual and intersection supports have genus at most $g$. 
Dealing with the cross-free condition on graphs is more challenging than in the geometric case. 
In particular, we show:
\begin{enumerate}
    \item If $(G,\mathcal{H})$ is a cross-free system of genus $g$, there exist primal and dual supports of genus at most $g$.
    \item If $(G,\mathcal{H},\mathcal{K})$ is a cross-free intersection system of genus $g$, there exists an intersection support of genus at most $g$.
\end{enumerate}


Next, we study outerplanar graphs. Here, we show that there is a subtle difference between the primal and
dual settings:
\begin{enumerate}
    \item Let $(G,\mathcal{H})$ be a cross-free outerplanar system. Then, there exists an outerplanar primal
    support on $V(G)$.
    \item Let $(G,\mathcal{H})$ be a \emph{non-piercing} outerplanar system. 
    Then, there exists an outerplanar dual support $Q^*$ on $\mathcal{H}$.
    In this case, we show that the cross-free condition is insufficient.
\end{enumerate}


Finally, we consider the case where the host graphs have bounded treewidth. Here, we show the following:
\begin{enumerate}
    \item If $(G,\mathcal{H})$ is non-piercing system of treewidth $t$, it has both primal and dual supports of
treewidth $O(2^{t})$.
\item There exist non-piercing systems of treewidth $t$ s.t. any primal or dual support has treewidth $\Omega(2^t)$.
\end{enumerate} 

The rest of the paper is organized as follows. In Section \ref{sec:nonpiercingimpliescrossfree}, we contrast non-piercing graph
systems with cross-free set systems. In Sections \ref{sec:boundedgenus} and \ref{sec:constructiongenus}, we present our results for cross-free systems of bounded genus.
In Section \ref{sec:outerplanar} we present our results for outerplanar host graphs.
In Section \ref{sec:treewidth}, we present results for non-piercing systems of bounded treewidth. We describe some applications
in Section \ref{sec:applications}, and conclude in Section \ref{sec:conclusion} with open questions.


\section{Non-piercing and Cross-free systems}
\label{sec:nonpiercingimpliescrossfree}
The cross-free condition implies the non-piercing condition in the plane, but they are incomparable in higher genus surfaces.
We start with the following result that shows that if a system is non-piercing in the plane, it is cross-free.

\begin{theorem}
\label{thm:planenp}
Let $(G,\mathcal{H})$ be a planar non-piercing system
Then, $(G,\mathcal{H})$ is cross-free.
\end{theorem}

\begin{proof}
Consider an embedding of $G$ in the plane. Abusing notation, let $G$ also denote the embedding
of $G$ in the plane.
If $(G,\mathcal{H})$ is crossing, there are two subgraphs 
$H, H'\in\mathcal{H}$ and a vertex $x$ in $R_{G}(H, H')$ that lies in $H\cap H'$ and four neighbors
$x_1, \ldots, x_4$ in cyclic order around $x$ s.t. $x_1, x_3\in H\setminus H'$ and
$x_2, x_4\in H'\setminus H$. It cannot be that both $x_1=x_3$, and $x_2=x_4$ 
without violating planarity. 
So assume without loss of generality that $x_2\neq x_4$.

Since $H$ is connected, there is an $x_1$-$x_3$ path in $R_G(H, H')$ that lies in $H$. Let
$P$ be any $x_1$-$x_3$ path lying in $H$. Similarly, let $P'$ be any $x_2$-$x_4$ path that lies in $H'$.
Observe that $P\cup\{x_1,x\}\cup\{x,x_3\}$
induces a jordan curve with $x_2$ and $x_4$ on either side of it. 
Thus, $P'$ and $P$ cross at a vertex that lies in $H\cap H'$, 
which implies $H'\setminus H$ is not connected 
and thus, $\mathcal{H}$ is piercing.
\end{proof}


Note that the reverse implication does not hold. It is easy to construct examples of graph systems in the plane that are cross-free, but are piercing. 
Consider the graph system consisting of a graph $K_{1,4}$ embedded in the plane,
with central vertex $v$, and spokes $\{a,b,c,d\}$ in cyclic order.
Let $H$ and $H'$ be two subgraphs where $H$ is the graph induced on $\{v,a,b\}$ and $H'$ is the graph induced on $\{v,c,d\}$. Then, $H$ and $H'$ are cross-free, but neither $H\setminus H'$ nor $H'\setminus H$ is connected.

The proof of Theorem \ref{thm:planenp} relies on the Jordan curve theorem, and hence
the corresponding statement does not hold for surfaces of higher genus. For example, let $G$ be the
torus grid graph $T_{n,n} = C_n \Box C_n$\cite{weisstein2016torus}. The subgraphs are the $n$ non-contractible cycles perpendicular to the hole, 
and the $n$ non-contractible cycles parallel to the hole. Each vertex of the graph belongs to exactly
two subgraphs forcing them to be adjacent in the dual support, and thus the dual support is
$K_{n,n}$ which is not embeddable on the torus for large enough $n$.
However, we show in Theorem \ref{thm:intsupport} 
that non-crossing is a sufficient condition
for a system $(G,\mathcal{H})$ on a graphs of genus $g$ to have a (primal/dual/intersection) support of genus at most $g$.

\section{Bounded Genus graphs}
\label{sec:boundedgenus}
In this section we consider the setting where the host graph has bounded genus. 
We start in Section \ref{sec:vertexbypassing} where we define the Vertex Bypassing operation that we require to obtain the primal, dual and
intersection support.

\subsection{Vertex Bypassing}
\label{sec:vertexbypassing}
Vertex Bypassing (\VB$(v)$\footnote{Short for Vertex Bypassing}) takes a cross-free system $(G,\mathcal{H})$ as input and \emph{simplifies} the system around a vertex $v$
of $G$.
Since $(G,\mathcal{H})$ is cross-free, we assume that we are given a cross-free embedding of $G$ on a surface of genus $g$. \VB$(v)$ is defined as follows: 

\begin{definition}[\VB$(v)$]
\label{defn:vertexbypassing}
Let $G$ be embedded in an oriented surface $\Sigma$.
Let $N(v)=(v_1,\ldots, v_k, v_1)$ be the cyclic order of vertices around $v$.
\begin{enumerate}
\item Subdivide each edge $\{v,v_i\}$ by a vertex $u_i$. By connecting consecutive vertices
$u_i, u_{i+1}$ (with indices taken~$\mod k$) by a simple arc not intersecting the edges of $G$, 
construct a cycle $C=(u_1,\ldots, u_k, u_1)$
s.t. the resulting graph $G''$ remains embedded in $\Sigma$. Remove $v$.
\label{step:one}
\item $\forall\; H\in \mathcal{H}_v$ s.t. $\{v, v_i\}\in H$, let $H'$ denote the
subgraph of $G''$ induced by $(H\setminus \{v\})\cup \{u_i\}$. 
Let $\mathcal{H}'_v =\{H': H\in\mathcal{H}_v\}$.
Let $\mathcal{H}' = (\mathcal{H}\setminus\mathcal{H}_v)\cup\mathcal{H}'_v$ (Note that the subgraphs in $\mathcal{H}'_v$ may not be connected).\label{step:two}
\item Add a set $D$ of non-intersecting\footnote{We use the term non-intersecting to mean internally non-intersecting}  chords in $C$ so that $\forall\; H\in \mathcal{H}'$,
$H$ induces a connected subgraph in $C\cup D$, and $\mathcal{H}'$ remains cross-free. 
\label{step:three}
\end{enumerate}
Let $(G',\mathcal{H}')$ be the resulting system.
\end{definition}

It is easy to see that the graph $G'$ obtained from $G$ 
is also embedded in $\Sigma$ as each operation preserves
the embedding. We show in Lemmas \ref{lem:keylem} and Lemma \ref{lem:remcrossfree} 
that we can implement Step \ref{step:three} of \VB$(.)$.

At the end of Step \ref{step:one}, since we remove vertex $v$, the subgraphs in $\mathcal{H}'_v$ in $G''$ 
may be disconnected.
Each $H\in\mathcal{H}'_v$, $H\cap C$ is a set of one or more disjoint \emph{runs} (a consecutive sequence) of vertices of $C$. 
Since $\mathcal{H}$ is cross-free, the induced sub-hypergraph $(V(C),\{H\cap C: H\in\mathcal{H}'_v\})$ is $abab$-free.

We now show that we can add a set $D$ of non-intersecting chords in $C$ 
s.t. $\forall\; H\in\mathcal{H}'_v$, $H\cap (C\cup D)$ induces a connected subgraph of $G'$.
By construction, the interior of $C$ is homeomorphic to an open disk (after removing $v$),
and therefore the vertices of $C$ can be seen arranged on a circle
with $u_1, \ldots, u_k, u_1$ in the clockwise order. 
For a pair of vertices $u_i, u_j$ with $i < j$ (indices~$\mod k$) we use $[u_i, u_j]$ to denote the arc from $u_i$ to $u_j$
in the direction of $C$. Thus, $[u_i,u_j]$ consists of the vertices $u_i, u_{i+1},\ldots, u_{j-1}, u_j$,
and arc
$[u_j, u_i]$ consists of the vertices $u_j, u_{j+1},\ldots, u_{i-1}, u_i$. 
We also use $(u_i, u_j)$ to denote the 
open arc that does not include the end-vertices $u_i$ and $u_j$. 
Similarly, we use $(u_i, u_j]$ to denote the half-open arc that does not 
include $u_i$, but includes $u_j$. 

The addition of a chord $d=\{u_i, u_j\}$ divides $C$ into two open arcs - $(u_i,u_j)$ and $(u_j, u_i)$.
The chord $d$ \emph{blocks} a subgraph $H\in\mathcal{H}'_v$ 
if both open arcs contain a run of $H$, and neither end-point of $d$ is contained in $H$.
Such a chord $d$ is called a \emph{blocking chord}. 
If $d$ does not block any subgraph in $\mathcal{H'}_v$, it is called a \emph{non-blocking} chord. 
We show in Lemma \ref{lem:keylem2} that there always exists a non-blocking chord $d$ 
that connects two disjoint runs of some subgraph $H\in\mathcal{H}'_v$. 
This is sufficient to guarantee the existence of a set $D$ of non-blocking, non-intersecting chords s.t.
for each $H\in \mathcal{H}'_v$, $H$ induces a connected subgraph of $G'$ in $C\cup D$, as we can recursively add non-blocking chords 
to the two smaller cycles created by adding $d$ to $C$ until all subgraphs are connected. 
We show this in Lemma \ref{lem:nblock}.

\begin{lemma}
\label{lem:keylem2}
Let $C$ be a cycle and let $\mathcal{K}$ be a collection of $abab$-free subgraphs on $C$.  
Then, for some $K\in\mathcal{K}$, there exists a non-blocking chord joining two disjoint runs of $K$.
\end{lemma}
\begin{proof}
Assume wlog that each subgraph $K\in\mathcal{K}$ induces at least two runs in $C$,
and no two subgraphs contain the same subset of vertices of $C$.
Define a partial order $\prec_C$ on $\mathcal{K}$, where for $K, K'\in\mathcal{K}$, 
$K\prec_C K'$ iff $K\cap C\subset K'\cap C$. 
Let $K_0\in \mathcal{K}$ be a minimal subgraph with respect to the order $\prec_C$.

Let $K_0^0, \ldots, K_0^q$ denote the runs of $K_0$. We let $A$ denote the run $K_0^0$ and
let $B = \cup_{i=1}^q K_0^i$. Wlog, we assume $C$ is oriented such that 
$A$ lies in the lower semi-circle of $C$ and that $B$ lies in the upper semi-circle of $C$, where the runs $K_0^1,\ldots, K_0^q$ appear in counter-clockwise order. Let $a_0,\ldots, a_r$ denote the
vertices of $A$ in clockwise order and let $b_0,\ldots, b_s$ denote the vertices of $B$ in 
counter-clockwise order. See Figure \ref{fig:circle}.

We show that there is a chord $d$ from a vertex in $A$ to a vertex in $B$ that
is non-blocking. In order to do so, we start with the chord $d_0 = a_0b_0$, and construct
a sequence of chords until we either find a non-blocking chord, or we end up with the chord $d_k = a_rb_s$,
which will turn out to be non-blocking. 
Having constructed chords $d_0,\ldots, d_{i-1}$, where $d_{i-1} = a_{\ell}b_{j}$, $d_i$ 
will be either the chord $a_{\ell}b_{j'}$ or
$a_{\ell'}b_{j}$, where $j'>j$ and $\ell' > \ell$. 

Next, we describe the construction of the chords. Each chord $d$ we construct satisfies the following invariant:
If $K$ is a subgraph blocked by a chord $d=a_{\ell}b_{j}$, then 
\begin{enumerate}
\item[$(i)$] $K$ is contained in $K_0$ in the arc $(a_{\ell},b_{j})$, and
\item[$(ii)$]
there is a vertex $k\in K\setminus K_0$ in the arc $(b_{j}, a_{\ell})$.
\end{enumerate}

Let $d_0$ denote the chord $a_0b_0$. If $d_0$ is non-blocking, we are done. Otherwise,
if $K\in \mathcal{K}$ is blocked by $d_0$, there is a vertex $k\in K$ that lies in $(b_0,a_0)$. Since
$(b_0,a_0)$ does not contain a vertex of $K_0$, this implies $k\in K\setminus K_0$, and hence $d_0$
satisfies condition $(ii)$ of the invariant.
Since we assumed the subgraphs $\mathcal{K}$ are $abab$-free, 
this implies that any vertex of $K$ in arc $(a_0,b_0)$ is contained in $K_0$. 
This ensures that condition $(i)$ of the invariant is satisfied by $d_0$.

Having constructed $d_0=a_0b_0,\ldots, d_{i-1}=a_{\ell}b_j$, each of which satisfy conditions $(i)$ and $(ii)$
of the invariant, 
we construct $d_i$ as follows: We simultaneously scan the vertices of $B$ in counter-clockwise order from $b_j$, 
and the vertices of $A$ in clockwise order from $a_{\ell}$ 
until we find the
first vertex $x$ that belongs to a subgraph blocked by $d_{i-1}$. Let $K_i$ denote this subgraph.
If $x=b_{j'}\in B$, we set
$d_i = a_{\ell}b_{j'}$. Otherwise, $x=a_{\ell'}\in A$, and we set $d_i = a_{\ell'}b_j$. Assume without loss
of generality that $d_i = a_{\ell}b_{j'}$ (the other case is similar).


\begin{figure}[ht!]

\centering
\begin{subfigure}{0.4\textwidth}
 \includegraphics[scale=.9]{circle}\vspace{0.5cm}
\caption{Ordering the vertices of $K$ in sets $A$ and $B$.
Here, $A=\{a_0,\ldots,a_r\}$ and $B=\{b_0,\ldots,b_s\}$.}
\label{fig:circle}
\end{subfigure}\hspace{1cm}
\begin{subfigure}{0.5\textwidth}
\includegraphics[scale=.9]{nextchord}\vspace{0.41cm}
\caption{Adding chords between $A$ and $B$.
Crossing among $u,b_{j'},v$ and $w$ if $K\setminus K_0\neq \emptyset$ in $(a_\ell,b_{j'}).$\vspace{0.5cm}}
\label{fig:nextchord}
 \end{subfigure}\vspace{0.5cm}
 \caption{All red vertices are in $K_0$ and no black vertex is in $K_0$.}
 \end{figure}

If $d_i$ is a non-blocking chord, we are done. Otherwise, let $K$ denote a subgraph blocked by $d_i$. Then,
both the arcs $(a_{\ell}, b_{j'})$ and $(b_{j'},a_{\ell})$ contain a run of $K$.
We now show that $d_i$ satisfies the invariant. Most of the work will go in showing that $d_i$ satisfies condition
$(i)$ of the invariant. We show this by contradiction - If $d_i$ does not satisfy invariant $(i)$, we will
exhibit a pair of subgraphs violating the $abab$-free property.

Suppose $d_i$ does not satisfy condition $(i)$ of the invariant, 
there is a vertex $u\in K\setminus K_0$ 
that lies in $(a_{\ell}, b_{j'})$. 
Since $d_{i-1}$ satisfies both the conditions of the invariant, the subgraph
$K_i$ blocked by $d_{i-1}$ is contained in $K_0$ in $(a_{\ell}, b_j)$. Since $(a_{\ell}, b_{j'})\subset (a_{\ell}, b_j)$, it implies $u\not\in K_i$, and thus $u\in K\setminus K_i$.
By construction of $d_i$, the vertex $b_{j'}\in K_i$, and since $d_i$ blocks $K$, $b_{j'}\not\in K$. 
Thus, $b_{j'}\in K_i\setminus K$.

Now, we claim that $K$ is not blocked by $d_{i-1}$. To see this, since $d_{i-1}$ satisfies condition $(i)$ of
the invariant, for any subgraph $K'$ blocked by $d_{i-1}$, we have that $K'\subseteq K_0$ in $(a_{\ell},b_j)$.
Since $(a_{\ell}, b_{j'})\subset (a_{\ell}, b_{j})$, combined with the facts that
$u\in K\setminus K_0$ and that $u$ lies in $(a_{\ell}, b_{j'})$ implies that $K$ is not
blocked by $d_{i-1}$. Since $K$ is blocked by $d_i$, it follows that there is a vertex $v$ of $K$ in the
arc $(b_{j'},b_j]$. 
However, no vertex in this arc lies in $K_i$, since $b_{j'}$ was the first vertex encountered
that was contained in a subgraph blocked by $d_{i-1}$ when traversing the vertices of $B$ in counter-clockwise order
from $b_j$.
Therefore, $v\in K\setminus K_i$. 

Finally, since $d_{i-1}$ satisfies condition $(ii)$ of the invariant, it implies
that there is a vertex $w\in K_i\setminus K_0$ that lies in $(b_j, a_{\ell})$. 
However, since $u\in K\setminus K_0$, and $u$ lies in $(a_{\ell}, b_{j'})$, $K\subseteq K_0$ in
$(b_{j'}, a_{\ell})$ since the arrangement is $abab$-free. 
However, since $(b_{j}, a_{\ell})\subset (b_{j'}, a_{\ell})$, this implies 
$w\not\in K$. Therefore, $w\in K_i\setminus K$. See Figure \ref{fig:nextchord}.


From the above arguments it follows that the subgraphs $K$ and $K_i$ are not $abab$-free, 
as witnessed by the sequence
of vertices $u, b_{j'}, v$ and $w$, a contradiction.
Thus, $d_i$ satisfies condition $(i)$ of the invariant. 
The fact that $d_i$ satisfies condition $(ii)$ of the invariant follows from the fact that $K_0$ is minimal.
Otherwise, $K\subseteq K_0$ in $(a_{\ell}, b_{j'})$ and in $(b_{j'}, a_{\ell})$, and therefore 
$K\subset K_0$.

Since the set of chords is finite, the sequence of chords constructed either ends in a 
non-blocking chord, or we end up with the chord $d=a_rb_s$. We claim
that $d$ must be a non-blocking chord. Suppose $d$ blocks a subgraph $K$. Then, $(a_r, b_s)$ contains
a vertex $u\in K\cap K_0$, as $d$ satisfies invariant $(i)$ and $(ii)$. However, $(a_r, b_s)$ 
does not contain a vertex in $K_0$. 
Therefore, $d$ must be a non-blocking chord.
\end{proof}
\begin{lemma}
\label{lem:nblock}
Let $C$ be a cycle and let $\mathcal{K}$ be an $abab$-free collection of subgraphs of $C$.
Then, we can add a set $D$ of non-intersecting chords to $C$ s.t. each $K\in\mathcal{K}$ 
induces a connected subgraph of $C\cup D$.
\end{lemma}
\begin{proof}
Let $n_K$ denote the number of disjoint runs of $K$ on $C$. Let 
\begin{align*}
cost(C,\mathcal{K}) = \sum_{K\in\mathcal{K}} (n_K-1) 
\end{align*}
If $cost(C,\mathcal{K})=0$, then every subgraph $K\in\mathcal{K}$ consists of one run, 
and therefore $C\cap K$ is connected for each $K\in\mathcal{K}$, and
$D=\emptyset$ suffices. 

Suppose the lemma holds for all $(C',\mathcal{K}')$ with $cost(C',\mathcal{K}')<N$. Given an instance with cost $(C,\mathcal{K}) = N$, by Lemma \ref{lem:keylem2}, 
there is a non-blocking chord $d=\{x,y\}$ joining two disjoint runs of some subgraph $K\in\mathcal{K}$.

The chord $d=\{x,y\}$ divides the cycle $C$ into two arcs, $[x,y]$, and $[y,x]$. 
We construct two disjoint sub-problems on the cycles $C_{\ell}$ and $C_r$ obtained from $C$, where $C_{\ell}$ is obtained by adding the edge
$\{x,y\}$ to the arc $[y,x]$, and $C_r$ is obtained by adding the edge $\{x,y\}$ to the arc $[x,y]$.
The subgraphs in
$C_{\ell}$ and $C_r$ are respectively those induced by $\mathcal{K}$, namely 
$\mathcal{K}_{\ell} =\{K\cap C_{\ell}: K\in\mathcal{K}\}$, 
and $\mathcal{K}_r = \{K\cap C_r:K\in\mathcal{K}\}$. 
Note that $\mathcal{K}_{\ell}$ and $\mathcal{K}_r$ 
are $abab$-free on $C_{\ell}$ and $C_r$, respectively. 
Let $n^{\ell}_K$ and $n^r_{K}$ denote respectively, the
number of runs of $K$ in $C_{\ell}$ and in $C_r$. Since $n^{\ell}_K < n_K$ and $n^r_K < n_K$,
it follows that $cost(C_r,\mathcal{K}_r)<cost(C,\mathcal{K})$ and $cost(C_{\ell},\mathcal{K}_{\ell}) < cost(C,\mathcal{K})$.

Hence, by the inductive hypothesis on $C_{\ell},\mathcal{K}_{\ell}$ and $C_{r},\mathcal{K}_r$ respectively, there exists a set of non-intersecting
chords $D_{\ell}$ s.t. each $K\in\mathcal{K}_{\ell}$ induces a connected subgraph of $C_{\ell}\cup D_{\ell}$. Similarly, there exists a set of non-intersecting chords $D_r$ 
s.t. for each $K\in\mathcal{K}_r$, induces a connected subgraph of $C_{r}\cup D_{r}$.
It follows that $D=D_{\ell}\cup D_r\cup d$ is a set of non-crossing chords such that $K\in\mathcal{K}$ induces  
a connected subgraph of $(C\cup D)$.
\end{proof}


\begin{lemma}
\label{lem:keylem}
Let $(G,\mathcal{H})$ be a cross-free system with $G$ an embedded graph. Suppose we apply \VB$(v)$ to vertex $v\in V(G)$. Then, each subgraph $H$ in $(G',\mathcal{H}')$ is connected.
\end{lemma}
\begin{proof}
Let $C$ be the cycle added on the subdividing vertices around vertex $v$. Since $(G,\mathcal{H})$ is cross-free, the subgraphs $\{H\cap C: H\in\mathcal{H}'_v\}$ 
satisfy the $abab$-free property on $C$. Therefore, by Lemma \ref{lem:nblock}, there is a collection $D$ of non-intersecting chords in $D$ s.t. 
each subgraph in $\mathcal{H}'_v$ induces a connected subgraph of $C\cup D$. Hence, each subgraph $H\in \mathcal{H}'$ is a connected subgraph of $G'$ since each $H\in\mathcal{H}_v$ is modified only in the vertices of subdivision.
\end{proof}


\begin{lemma}
\label{lem:remcrossfree}
Let $(G,\mathcal{H})$ be a cross-free system. Let $(G',\mathcal{H}')$ be the system obtained after applying vertex bypassing at a vertex
$v\in V(G)$. Then, $(G',\mathcal{H}')$ is cross-free.
\end{lemma}

\begin{proof}
By Lemma \ref{lem:keylem}, each subgraph $H'\in (G',\mathcal{H}')$ is connected. We will show that
the system is cross-free.
Let $N(v)=\{v_1,\ldots, v_k\}$. Let 
$S(v)=\{u_1,\ldots, u_k\}$ where $u_i$ subdivides edge $\{v,v_i\}$.

Assume $(G',\mathcal{H}')$ is crossing. Then, there are two subgraphs $H_1$ and $H_2$ in $\mathcal{H}$ 
that correspond respectively, to subgraphs $H'_1$ and $H'_2$ in $\mathcal{H}'$, and a crossing vertex $x_{C'}$ 
of $R_{G'}(H'_1, H'_2)$. Here, $C'$ is the connected component of $G'[H'_1\cap H'_2]$ that is contracted
to $x_{C'}$ in $R_{G'}(H'_1, H'_2)$. Let $x_1, x_2, x_3, x_4$ be the four neighbors encountered in cyclic
order around $x_{C'}$ s.t. $x_1, x_3\in H'_1\setminus H'_2$ and $x_3, x_4\in H'_2\setminus H'_1$.

Suppose $C'\cap N(v)=\emptyset$. Since $G'$ is obtained from $G$ by modifying only $N(v)$, it implies
$(G,\mathcal{H})$ is crossing. 
Hence, we can assume that $C'\cap N(v)\neq\emptyset$.
Let  $v_1\in C'$. 
Since $\mathcal{H}_{e}\subseteq \mathcal{H}_{v_1}\cap\mathcal{H}_v$ in $(G,\mathcal{H})$, 
where $e=\{v,v_1\}$,
it follows that if $u_1\not\in C'$, then $v\not\in H_1\cap H_2$. Therefore, there does not exist vertices $u$ and $u'$ in $S(v)$ s.t. $u\in H'_1$ and $u'\in H'_2$. 
Since $x_{C'}$ is a crossing vertex, replacing each occurrence of a vertex
in $\{x_1,\ldots, x_4\}\cap S(v)$ by $v$ we obtain a crossing vertex in $R_{G}(H_1, H_2)$ implying
$(G,\mathcal{H})$ is crossing, a contradiction. 

Finally, consider the case that $v\in H_1\cap H_2$. There are again two cases to consider. If no
neighbor of $v$ is contained in $H_1\cap H_2$. Then, the component of $G[H_1\cap H_2]$ containing $v$
is a singleton. This component is removed in $G'[H'_1\cap H'_2]$ since $v$ is removed. So $v\in H_1\cap H_2$,
and there is a vertex, say $v_1\in N(v)$ s.t. $v_1\in C'$. 
Since $H'_1$ and $H'_2$ are crossing at $x_{C'}$, if $N(x_{C'})\cap S(v)=\emptyset$, then if $C$ is the component containing $v_1$ in $G[H_1\cap H_2]$,
and $x_C$ is the vertex in $R_G(H_1, H_2)$ obtained by contracting $C$, then $H_1$ and $H_2$ are crossing at $v$. Otherwise, if
$S(v)\cap N(x_{C'})\neq\emptyset$, then replacing each occurrence of $u_i$ in $N(x_{C'})$ by 
$v_i$ yields a crossing vertex in $R_{G}(H_1, H_2)$, a contradiction.
\end{proof}
\section{Construction of Supports}
\label{sec:constructiongenus}
In this section, we show that for cross-free systems on a graph of genus $g$, there exist primal, dual
and intersection supports of genus at most $g$. While the existence of an intersection support implies the
existence of the primal and dual supports, we use the dual support and techniques from the primal support
in order to construct the intersection support, and hence present these first.

We obtain a polynomial time algorithm
to construct a primal support when an embedding of the graph system $(G,\mathcal{H})$ is given. However, we are
unable to prove a similar result for a dual support or intersection support, and this is an intriguing open question.


\subsection{Primal Support}
\label{sec:primalsupportgenus}
In this section, we show that for a cross-free system $(G,\mathcal{H})$ with $c:V\to\{\R,\B\}$, where $G$ has
genus $g$, there is a primal support $Q$ of genus at most $g$, and that $Q$ can be constructed in polynomial 
time in $|G|$ and $|\mathcal{H}|$ if a cross-free embedding of $G$ is given.

\begin{theorem}
\label{thm:primalsupport}
Let $(G,\mathcal{H})$ be a cross-free system of genus $g$, 
with $c:V(G)\to\{\R,\B\}$. Then, there exists a primal support $Q$ of genus at most $g$ on $\B(V)$.
Further, if a cross-free embedding of $G$ is given, $Q$ can be constructed in time polynomial in $|G|$ and $|\mathcal{H}|$.
\end{theorem}

\begin{proof}
If $\R(V)=\emptyset$, $G$ itself is a support. Since $(G,\mathcal{H})$ is cross-free, there exists a cross-free 
embedding of $G$ in a surface of genus $g$. Consider such an embedding of $(G,\mathcal{H})$. We abuse notation slightly,
and use $G$ to also refer to an embedding of $G$.
Let $v$ be a  maximal vertex in $\R(V)$, i.e., for each $v\in \R(V)$ s.t. $e\sim v$, $\depth(e)<\depth(v)$. We apply \VB$(v)$.
For $i=1,\ldots, \deg(v)$, let $u_i$ be the vertices added by \VB$(v)$ subdividing
edges incident on $v$. Set $c(u_i)=\R$, $\forall\;i=1,\ldots, \deg(v)$.
Since $v$ was maximal, it follows that 
$\depth(u_i)<\depth(v)$, for all $i=1,\ldots, k$.
Further, no $u_i$ is maximal since $\depth(u_i)\le \depth(v_i)$, where $u_i$ subdivides the edge $\{v,v_i\}$, and they
remain non-maximal when we bypass other maximal vertices in $\R(V)$. To see this, note that 
if $v_i$ is maximal, the
vertex $u'_i$ subdividing $\{v_i, u_i\}$ satisfies $\mathcal{H}_{u'_i}=\mathcal{H}_{u_i}$.
Thus, by Lemmas \ref{lem:nblock} and \ref{lem:keylem}, the resulting system
is cross-free and we have one fewer maximal red vertex in the resulting system. We repeatedly apply vertex bypassing to
each maximal vertex in $\R(V)$ until we obtain the system $(G',\mathcal{H}')$ where no vertex in $\R(V(G'))$ is maximal.

Since no vertex in $\R(V(G'))$ is maximal in $(G',\mathcal{H}')$, it follows that for each $v\in \R(V(G'))$
there is an edge $e=\{v,u\}$, $e\sim v$ s.t. 
$\mathcal{H}_e = \mathcal{H}_v$, which implies
$\mathcal{H}_v\subseteq\mathcal{H}_u$. For each $v\in\R(V(G'))$, choose such a neighbor $u$ of $v$ arbitrarily and
orient the edge $\{v,u\}$ from $v$ to $u$.

Let $D$ denote the directed subgraph on $V(G')$ induced by these directed arcs. Observe that
in $D$, each vertex has out-degree at most 1, and for an arc $(v,u)$ in $D$, we have $\mathcal{H}_v\subseteq\mathcal{H}_u$,
which implies $D$ is acyclic. Thus, D is a forest. By construction, $D$ satisfies the following properties: 
In each tree $T$ of D, $(i)$ the arcs of $T$ are directed towards the
root of $T$, $(ii)$ all vertices in $T$ except the root are red, and $(iii)$ the root of $T$ is blue.
Further, for each arc $(v,u)$ of $D$, we have $\mathcal{H}_v\subseteq\mathcal{H}_u$.
Thus, contracting each forest to its root yields the desired support $Q$.

Finally, to see that $Q$ can be constructed in polynomial time, note that there are at most $|V(G)|$ maximal vertices
in $\R(V)$. If a cross-free embedding of $G$ is given, then for a maximal vertex $v\in\R(V(G))$, \VB$(v)$ 
can be implemented in polynomial time by Lemma \ref{lem:keylem}, and 
moreover, we obtain a cross-free embedding of the resulting graph system. 
Further, no new vertex added during \VB$(v)$ becomes maximal on applying further vertex bypassing operations. 
Thus, while the size of the graph grows with each vertex bypassing operation, we apply
the vertex bypassing operation at most $|V(G)|$ times. Finally, constructing $D$ and contracting the arcs in $D$ to obtain $Q$ can clearly be done in polynomial time.
\end{proof}

\subsection{Dual Support}
\label{sec:dualsupportgenus}
In this section, we show that for a cross-free system on a graph of genus $g$, there is a dual support of genus at most $g$.
Unlike the primal case however, we are unable to prove that the algorithm implied in the
proof of Theorem \ref{thm:dualembedded} runs in polynomial time, even if a cross-free
embedding of the system is given.

\begin{theorem}
\label{thm:dualembedded}
Let $(G,\mathcal{H})$ be a cross-free system of genus $g$. Then, there exists a dual support $Q^*$ on $\mathcal{H}$
of genus at most $g$.
\end{theorem}

\begin{proof}
We assume by Proposition \ref{prop:nocontainment} that there are no containments in $\mathcal{H}$. Since $(G,\mathcal{H})$ is cross-free,
there exists a cross-free embedding of $G$ in a surface of genus $g$. Consider such an embedding of $(G,\mathcal{H})$. We abuse notation
slightly and also refer to the embedded graph by $G$.
We prove by induction on $(d, n_d)$, where $d$ is the maximum depth of a vertex in $G$ and $n_d$ is the number of vertices of depth $d$. 
We need the following definition: An edge $\{u,v\}\in E(G)$ is said to be a \emph{special edge} if $\mathcal{H}_u\cap\mathcal{H}_v=\emptyset$.
Let $Spl_{\mathcal{H}}(E)$ be the set of special edges in $E(G)$. 
We need a stronger inductive
hypothesis, namely given $(G,\mathcal{H})$ there is a support with the \emph{special edge property}, i.e.,
there is a dual support $Q^*$ on $\mathcal{H}$ s.t. for each $e=\{u,v\}\in Spl_{\mathcal{H}}(E)$, there is an edge between some $H\in\mathcal{H}_u$ and $H'\in\mathcal{H}_v$ in $Q^*$.

If $d = 1$, i.e., the maximum depth of a vertex is 1, then, no two subgraphs $H, H'\in\mathcal{H}$ share a vertex of $G$. 
Therefore, contracting each $H\in\mathcal{H}$ to a vertex, 
we obtain a support $Q^*$ of genus $g$ satisfying the special edge property, and we are done.

So suppose $d > 1$. Let $v$ be a vertex of maximum depth in $G$. Then, we claim $v$ must be a maximal vertex, 
i.e., $\depth(v)>\depth(e)$ $\forall\; e\sim v$. If not,
there is an edge $e=\{u,v\} \in E_v$ such that $\mathcal{H}_e = \mathcal{H}_v$. Since $v$ has maximum depth, the other end-point of $e$, namely
$u$ satisfies $\mathcal{H}_u=\mathcal{H}_v$. Hence, contracting $e$, we obtain a system $(G/e, \mathcal{H})$  
with smaller $n_d$. It is easy to see that $(G/e, \mathcal{H})$ is cross-free. Thus, by the inductive hypothesis there
is a support $Q^*$ for $(G/e, \mathcal{H})$ since we assumed there are no containments in $\mathcal{H}$.
It is easy to check that $Q^*$ is a support for $(G,\mathcal{H})$.   

Therefore, we can assume that for each $e\sim v$, $\mathcal{H}_e\subset \mathcal{H}_v$. Now, we apply \VB$(v)$
to obtain the system $(G',\mathcal{H}')$. By Lemmas \ref{lem:keylem} and \ref{lem:remcrossfree}, $(G',\mathcal{H}')$ is cross-free, and 
$(d',n'_d) \prec (d,n_d)$, where $d'$ and $n'_d$ are respectively, the
depth of a maximum depth vertex, and the number of vertices of maximum depth in $(G',\mathcal{H}')$.
By the inductive hypothesis, there is a support $Q_1^*$ for $(G',\mathcal{H}')$ satisfying the special edge property. 
We show that $Q_1^*$ is also a support for $(G,\mathcal{H})$. 
For each $u\neq v\in V(G)$, it follows from the inductive hypothesis that the support property is satisfied.
For $i=1, \ldots, u_{\deg(v)}$, let $u_i$ be the vertex subdividing the edges $\{v,v_i\}$ in $G$ added in \VB$(v)$ and
let $C$ denote the cycle on $u_1,\ldots, u_{\deg(v)}$ added in \VB$(v)$.

Since we assumed $\mathcal{H}$ as no containments, we have $\cup_{i=1}^{\deg(v)} \mathcal{H}'_{u_i} = \mathcal{H}_v$. 
If none of the edges of $C$ are in $Spl_{\mathcal{H}'}(E)$, then $\mathcal{H}_v$ is connected since adjacent vertices of $C$ 
share at least one subgraph. On the other hand, if an edge $e=\{u_i, u_{i+1}\}$ of $C$ is in
$Spl_{\mathcal{H}'}(E)$, by the inductive hypothesis, at least one subgraph of $\mathcal{H}'_{u_i}$ and one subgraph of $\mathcal{H}'_{u_{i+1}}$ is connected. It follows that $\mathcal{H}_v$ is connected and thus $Q^*=Q^*_1$ is the desired support.
\end{proof}


\subsection{Intersection Support}
\label{sec:intsupportgenus}

We now show that the intersection system $(G,\mathcal{H},\mathcal{K})$ has an intersection support of genus at most $g$
if $G$ has genus $g$. The proof builds on the construction of primal and dual supports, but is more involved. 

Given the system $(G,\mathcal{H},\mathcal{K})$, we define the \emph{intersection graph} $G_K$ for each $K\in\mathcal{K}$ as
the graph with vertex set $\mathcal{H}_K$, i.e., the subgraphs in $\mathcal{H}$, intersecting $K$, and an edge between two subgraphs $H,H'\in\mathcal{H}_K$ if and only if
they share a vertex in $K$. 
We start with the special case where $G_K$ is connected for each $K\in \mathcal{K}$.

\begin{lemma}
\label{lem:conn}
Let $(G,\mathcal{H},\mathcal{K})$ be a cross-free intersection system. 
If $\forall\; K\in\mathcal{K}$, the intersection graph $G_K=(\mathcal{H}_K,E_K)$ is connected, then
a dual support for $(G,\mathcal{H})$ is an intersection support for $(G,\mathcal{H},\mathcal{K})$.
\end{lemma}
\begin{proof}
Let $Q^*$ denote a dual support for $(G,\mathcal{H})$. 
Consider an arbitrary subgraph $K\in\mathcal{K}$. Since $G_K$ is connected, for any two 
subgraphs $H, H'\in \mathcal{H}_K$ there is a path $P= H=H_0, H_1, H_2,\ldots, H_k = H'$ in $G_K$ from $H$ to $H'$.
By definition of $G_K$, for $i=0,\ldots, k-1$, there is a vertex $v\in K\cap H_i\cap H_{i+1}$.
Since $Q^*$ is a dual support for $(G,\mathcal{H})$, there is a path from $H_i$ to $H_{i+1}$ in $Q^*$ that consists of
only the subgraphs containing $v$, and hence in $\mathcal{H}_K$.
The concatenation of these paths yields a path from 
$H$ to $H'$ in $Q^*$. Since $H$ and $H'$ were arbitrary subgraphs from $\mathcal{H}_K$ for any $K\in\mathcal{K}$, it follows that
$Q^*$ is a support for $(G,\mathcal{H},\mathcal{K})$.
\end{proof}
 
If the intersection graph
$G_K$ is not connected for some $K\in\mathcal{K}$, we 
modify the arrangement and augment $\mathcal{H}$ with \emph{dummy} subgraphs $\mathcal{F}$
such that the intersection graph $G_K$ with subgraphs $\mathcal{H}\cup\mathcal{F}$ is connected. 
We then construct
a support $Q'$ on $\mathcal{H}\cup\mathcal{F}$, and then carefully remove the vertices of $\mathcal{F}$ to obtain
a graph on $\mathcal{H}$ that is a support for the original intersection system. But before we can add dummy
subgraphs, we need to simplify the arrangement. 
We show how this is done in Lemma \ref{lem:removeMaximal}. 
In Lemma \ref{lem:dummy}, we show how we can add dummy subgraphs to satisfy the conditions of Lemma \ref{lem:conn},
and then we show how to construct a support for the original system.

A vertex of $G$ that is contained only in subgraphs in $\mathcal{K}$ but not in any subgraph in $\mathcal{H}$ is called a $\mathcal{K}$-vertex.
Recall that a vertex $v$ is maximal if 
$\mathcal{K}_e \subset \mathcal{K}_v $ for all ${e}\sim v$.
In the following, we repeatedly apply vertex bypassing to a maximal $\mathcal{K}$-vertex of
maximum depth. Note that a maximum depth $\mathcal{K}$-vertex need not be maximal in this case.

\begin{lemma}
\label{lem:removeMaximal}
Let $(G,\mathcal{H},\mathcal{K})$ be a cross-free intersection system of genus $g$. 
Then, we can modify the arrangement 
to a cross-free arrangement $(G',\mathcal{H},\mathcal{K}')$
so that $G'$ has genus $g$, no $\mathcal{K}$-vertex of $G'$ is maximal, and 
a support $Q'$ for $(G',\mathcal{H}, \mathcal{K}')$ is a support for 
$(G,\mathcal{H}, \mathcal{K})$.
\end{lemma}
\begin{proof}
We assume a cross-free embedding of $(G,\mathcal{H})$ is given, and with a slight abuse of notation use $G$ to also refer
to the embedded graph.
Let $d$ denote the maximum depth of a maximal $\mathcal{K}$-vertex, and
let $n_d$ denote the number of maximal $\mathcal{K}$-vertices of depth $d$.
We repeatedly choose a maximal $\mathcal{K}$-vertex $v$ of maximum depth and apply \VB$(v)$. The operation
\VB$(v)$ modifies the graph $G$ and the subgraphs in $\mathcal{K}$, but does not modify any subgraph in
$\mathcal{H}$, as $v$ is a $\mathcal{K}$-vertex. 
Let $(G',\mathcal{H},\mathcal{K}')$ denote
the new arrangement. Let $d'$ and $n'_{d'}$ denote respectively, the maximum depth of a maximal $\mathcal{K}$-vertex,
and the number of maximal $\mathcal{K}$-vertices in $(G',\mathcal{H},\mathcal{K}')$ of depth $d'$. 
It follows that $(d', n'_{d'})$ is lexicographically 
smaller than $(d, n_d)$. Since the newly added vertices and edges are not contained in any subgraph $H\in\mathcal{H}$,
it follows that $\mathcal{H}$ remains cross-free in $G'$.
Further, the fact that $\mathcal{K}'$ remains cross-free follows from Lemmas \ref{lem:keylem} and \ref{lem:remcrossfree}. Thus, 
$(G',\mathcal{H}, \mathcal{K}')$ remains cross-free.
Since the underlying hypergraph $(\mathcal{H},\{\mathcal{H}_K\}_{K\in\mathcal{K}})$ is 
not modified, a support $Q'$ for $(G',\mathcal{H},\mathcal{K}')$ is also a support for $(G,\mathcal{H},\mathcal{K})$.
\end{proof}

\begin{lemma}
\label{lem:dummy}
Let $(G,\mathcal{H},\mathcal{K})$ be a cross-free intersection system of genus $g$ s.t. no $\mathcal{K}$-vertex
is maximal.
Then, we can add a collection of \emph{dummy subgraphs} $\mathcal{F}$ s.t. $\mathcal{H}\cup\mathcal{F}$
remains cross-free, and a support $\tilde{Q}$ for $(G,\mathcal{H},\mathcal{K})$ can be obtained from 
the dual support $Q^*$ for the system $(G,\mathcal{H}\cup\mathcal{F})$.
\end{lemma}
\begin{proof}
Given the system $(G,\mathcal{H},\mathcal{K})$ such that no $\mathcal{K}$-vertex is maximal,
we first describe how we modify the arrangement and add dummy subgraphs to obtain the arrangement
$(G',\mathcal{H}',\mathcal{K}')$. As before, we assume a cross-free embedding of $G$ is given,
i.e., an embedding where both $\mathcal{H}$ and $\mathcal{K}$ are simultaneously cross-free.
While constructing the new arrangement, we also  
construct an auxiliary graph $D$ on the vertices of $G'$ that will be helpful
in constructing a support for $(G,\mathcal{H}, \mathcal{K})$. 

An edge $e\in E(G)$ is a $\mathcal{K}$-edge if $\mathcal{K}_e\neq\emptyset$ and
$\mathcal{H}_e=\emptyset$. For each $\mathcal{K}$-edge $e=\{u,v\}$, we can assume
that either $u$ or $v$ is a $\mathcal{K}$-vertex. Otherwise, we can subdivide
$e$ with a vertex $v_e$.
Since $\mathcal{K}_{v_e}=\mathcal{K}_e\subseteq\mathcal{K}_u\cap\mathcal{K}_v$,
it follows that $v_e$ is not a maximal $\mathcal{K}$-vertex.
Thus, in the following,
we assume that for each $\mathcal{K}$-edge, one of its end-points is a $\mathcal{K}$-vertex.

We first subdivide each $\mathcal{K}$-edge $e=\{u,v\}$ with a vertex $v_e$. Note that 
either $u$ or $v$ is a $\mathcal{K}$-vertex. Let $G'$ denote the resulting graph.
Observe that for each $\mathcal{K}$-vertex $v$, each edge incident on $v$ is subdivided.
For each $\mathcal{K}$-vertex $v$, let $N(v)$ denote its neighbors in $G$,
and let $S(v)$ be the set of vertices subdividing edges incident on $v$. 
Next, we add a dummy subgraph $F_v$ corresponding to each $\mathcal{K}$-vertex $v$ in $G$
spanning $\{v\}\cup S(v)$.
Finally, for each edge $\{v, v'\}$ incident on $v$ in $G$, we extend the subgraphs in $\mathcal{H}_{v'}$
to contain the vertex subdividing the edge $\{v,v'\}$. Let $\mathcal{F}$ denote the dummy subgraphs added and
let $\mathcal{H}''$ denote the modified subgraphs in $\mathcal{H}$, and let
$\mathcal{H}'=\mathcal{H}''\cup\mathcal{F}$.

By construction, there is no $\mathcal{K}$-vertex and no $\mathcal{K}$-edge in $(G,\mathcal{H}',\mathcal{K})$.
Hence, for each $K\in\mathcal{K}$ the graph $G_K$ is connected. 
Therefore, by Theorem \ref{thm:dualembedded}, there is a dual support $Q^*$ on $\mathcal{H}'$ 
for $(G',\mathcal{H}')$. By Lemma \ref{lem:conn}, $Q^*$ is also a support for
$(G',\mathcal{H}',\mathcal{K})$. 

We obtain a support for $(G,\mathcal{H},\mathcal{K})$ from $Q^*$ as follows:
We maintain a directed graph $D$ on $\mathcal{H}'$.
Since we assumed that no $\mathcal{K}$-vertex in $G$ is maximal, each $\mathcal{K}$-vertex $v$
is adjacent to a \emph{full edge} $e=\{v,v'\} \sim v$, i.e., $\mathcal{K}_e=\mathcal{K}_v$. 
Let $u_e$ be the vertex subdividing $e$ in $G'$. 
Since $Q^*$ is a dual support for $(G',\mathcal{H}')$, the subgraphs containing $u_e$ is connected.
Since $\mathcal{H}'_e \subseteq\mathcal{H}'_{v'}$, $F_v$ is adjacent to a subgraph $H$
in $\mathcal{H}'$ containing $v'$. We add an arc in $D$ from $F_v$ to $H$.

By construction, each dummy subgraph in $D$ has out-degree 1. If $(F_v, F_{v'})$ is an
arc in $D$, then $\mathcal{K}_v\subseteq\mathcal{K}_{v'}$. Thus, $D$ is acyclic, and hence
a forest s.t. each tree in the forest is rooted at a subgraph of $\mathcal{H}''$, and
all other vertices correspond to dummy subgraphs. 
We obtain a support $\tilde{Q}$ on
$\mathcal{H}$ by contracting the subgraphs corresponding to each tree in $D$ to its root.
We claim that $\tilde{Q}$ is the desired support. Let $F$ be a dummy subgraph. For any $K\in\mathcal{K}$, if $F$ intersects $K$ 
and $(F,F')$ is an arc in $D$, then $K$ intersects $F'$ and thus the
root of the tree containing $F$ intersects
$K$.
Hence, $\tilde{Q}$ is the desired support.
\end{proof}

We are now ready to prove the main theorem of this section.

\begin{theorem}
\label{thm:intsupport}
Let $(G,\mathcal{H},\mathcal{K})$ be an intersection system of genus $g$. 
Then, there exists an intersection support $\tilde{Q}$ on $\mathcal{H}$ of genus at most $g$.
\end{theorem}
\begin{proof}
If for each $K\in\mathcal{K}$, the intersection graph $G_K=(\mathcal{H}_K, E_K)$ is connected, from Lemma \ref{lem:conn},
we obtain a support of genus at most $g$. 
Otherwise, by Lemma \ref{lem:removeMaximal}, we obtain a cross-free system $(G',\mathcal{H},\mathcal{K}')$ 
such that no $\mathcal{K}$-vertex is maximal, 
and a support for $(G',\mathcal{H},\mathcal{K}')$ is a support for the original system. 
Finally, by Lemma \ref{lem:dummy}, we obtain
a support for $(G',\mathcal{H},\mathcal{K}')$, and thus a support for $(G,\mathcal{H},\mathcal{K})$, and the resulting support has
genus at most $g$.
\end{proof}

\section{Outerplanar Graphs}
\label{sec:outerplanar}
In this section, consider the case when $G$ is outerplanar. We assume 
$G$ is embedded in the plane in the natural manner with $C$ denoting the outer face.

\begin{theorem}
\label{thm:primalOuter}
Let $(G,\mathcal{H})$ be an outerplanar cross-free system, with $c:V\to\{\R,\B\}$.
Then, there is a support $Q$ on $\B(V)$ i.e. $Q[\B(H)]$ is connected for each 
$H\in\mathcal{H}$. If a cross-free embedding is given, then an outerplanar support can be computed
in time polynomial in $|V(G)|, |\mathcal{H}|$.
\end{theorem}
\begin{proof}
If $\R(V)=\emptyset$, $G$ itself is the desired support. 
Otherwise, let $C'$ be a cycle on $\B(V)$ in the same order as in the outer face of $G$. 
Wlog, let each $H\in\mathcal{H}$ induce a disjoint collection of runs on $C'$. It is easy to see that the collection of induced subgraphs $\{H\cap C'\}_{H\in\mathcal{H}}$ is \emph{abab-free} on $C'$.
By Lemma \ref{lem:nblock}, there is a collection of non-intersecting chords $D$ connecting all
the runs of $H$ for each subgraph $H\in\mathcal{H}$. Then $Q=C'\cup D$ is 
the desired support.

The proof of Lemma \ref{lem:keylem2}, gives a polynomial time algorithm to add a non-blocking diagonal.
For a fixed subgraph $H\in\mathcal{H}$, we try adding one of at most $\binom{n}{2}$ diagonals,
where $n=|V(G)|$. For each
choice, to check that it is non-blocking, we take $(|\mathcal{H}|)$ time to check if the given diagonal
blocks a subgraph. Hence, we find a non-blocking diagonal in $O(n^2|\mathcal{H}|)$. A maximal outerplanar
graph has $n-3$ diagonals, and therefore the running time is $O(n^3|\mathcal{H}|)$.
\end{proof}

We show an example of a cross-free system $(G,\mathcal{H})$ that does not admit an outerplanar dual support. 
Let $G$ be an asteroidal triple graph with vertex set $\{1,2,\ldots 6\}$. Let $\mathcal{H}$ be a family of cross-free subgraphs induced by the vertex sets $\{1,2,3\},\{3,4,5\},\{5,6,1\}$ and $\{2,4,6\}$,
as shown in Figure \ref{fig:asteroidal}. 
The support for the dual hypergraph is $K_4$ which is not outerplanar. 
A natural question is the following: If $(G,\mathcal{H})$ is a non-piercing system, and $G$ is a tree, is there
a support that is a tree? We show that the answer to this question is negative in both the primal and dual settings.
For the primal setting, consider the graph $K_{1,3}$ with $v$ being the central vertex colored red, 
with spokes $v_1,v_2, v_3$ colored blue. We put three subgraphs $H_1, H_2, H_3$, where $H_i = \{v_i, v, v_{i+1}\mod 3\}$.
It is easy to see that the primal support is a triangle. The same example without colors on the vertices shows that
the dual support is also a triangle.

We now show that if $\mathcal{H}$ 
is non-piercing and $G$ is an outerplanar graph, then $(G,\mathcal{H})$ admits an outerplanar dual support.
We start with the following definition:

\begin{definition}[$axax$-free]
Let $(C,\mathcal{H})$ be a graph system where $C$ is a cycle and $\mathcal{H}$ is a collection
of (not necessarily connected) subgraphs of $C$. Then, $(C,\mathcal{H})$ is $axax$-fee if for
any two subgraphs $H, H'\in\mathcal{H}$, 
there are no four vertices $a_1,x_1, a_2, x_2$ in cyclic order 
around $C$ s.t.
$a_1, a_2\in H\setminus H'$ and $x_1, x_2\in H'$.
\end{definition}

\begin{lemma}
\label{lem:nblockop}
Let $(G,\mathcal{H})$ be a non-piercing outerplanar graph system.
Then, $(C,\mathcal{H})$ is $axax$-free, where $C$ is the cycle defining the outer face of $G$.
\end{lemma}
\begin{proof}
Since $H'$ is connected, $G$ contains a path between $x_1$ and $x_2$ that lies in $H'$.
Consider any path $P'$ in $H'$ between $x_1$ and $x_2$. Since $a_1, a_2\not\in H'$, 
$P'$ does not contain $a_1$ or $a_2$. Since $H$ is connected, there is a path
between $a_1$ and $a_2$ that lies in $H$.
By the Jordan curve theorem, any path $P$ in $H$ between $a_1$ and $a_2$ intersects
any path $P'$ in $H'$ between $x_1$ and $x_2$ as shown in Figure \ref{fig:axax}. 
Since $G$ is outerplanar, $P$ and $P'$ intersect at a vertex of $G$.
But, this implies that $P$ does not lie in $H\setminus H'$, or equivalently, that $H\setminus H'$ is
not connected, violating the assumption that the subgraphs are non-piercing.
\end{proof}

\begin{corollary}
\label{cor:axax}
Let $(G,\mathcal{H})$ be a non-piercing outerplanar system. Then, 
for any $H\in\mathcal{H}$, any chord $d$ whose end-points are in $H$ is non-blocking. 
\end{corollary}
\begin{proof}
This follows directly from Lemma \ref{lem:nblockop}.
\end{proof}

\begin{figure}[ht!]

    \centering
    \begin{subfigure}{0.4\textwidth}
    \vspace{0.31cm}
        \includegraphics[scale=.8]{asteroidal}\vspace{0.5cm}
\caption{Cross-free Asteroidal with the
 only dual support $K_4$.\vspace{1.4cm}}        \label{fig:asteroidal}
 \end{subfigure}\hspace{1cm}
    \begin{subfigure}{0.4\textwidth}\vspace{1.5cm}
        \includegraphics[scale=.85]{outerplanardual}\vspace{0.25cm}
        \caption{Crossing caused by $a_1,x_1,a_2,x_2$ sequence in outerplanar graph. Here, $a$ and $x$ are some vertices of $H\setminus H'$ and $H'$ respectively.}
        \label{fig:axax}
    \end{subfigure}
     \caption{}

\end{figure}

 Now we can obtain the result for the dual support.
 
\begin{theorem}
\label{thm:outerplanardual}
Let $(G,\mathcal{H})$ be a non-piercing outerplanar system.
Then, there is an outerplanar dual support $Q^*$ on $\mathcal{H}$. 
Further, an outerplanar support can be computed in time polynomial in $|\mathcal{H}|$ and $|V(G)|$.
\end{theorem}
\begin{proof}  
By Proposition \ref{prop:nocontainment}, we can assume there is no containment in $\mathcal{H}$.
By Lemma \ref{lem:nblockop}, $(C,\mathcal{H})$ is $axax$-free.
For $H\in\mathcal{H}$, let $n_H$ denote the number of runs of $H$
on $C$, and let
$N(C,\mathcal{H})=\sum_{H\in\mathcal{H}} (n_H-1)$. 
We prove by induction on $N(C,\mathcal{H})$ that if $(C,\mathcal{H})$ is $axax$-free, there is an outerplanar
support on $\mathcal{H}$. 
 
If $N(C,\mathcal{H})=0$, each subgraph consists of a single run. We claim that  
a cycle on $\mathcal{H}$ yields a support:
Let $v_1,\ldots, v_n$ be the cyclic order of vertices
on $C$. Traversing $C$ in clockwise order, we obtain a cyclic order 
on the subgraphs ordered on the last vertex of their run. 
Let $Q^*$ be the cycle on $\mathcal{H}$ in this order.
Since we assumed there is no containment in $\mathcal{H}$,
for any $v\in V(C)$, the subgraphs in $\mathcal{H}_v$ appear consecutively in $Q^*$, and
thus induce a connected subgraph of $Q^*$. Hence, $Q^*$ is a support. 

Suppose for any cycle $C'$ and subgraphs $\mathcal{H}'$ such that $(C',\mathcal{H}')$ is $axax$-free
and $N(C',\mathcal{H}')<N$,
there is an outerplanar support on $\mathcal{H}'$.
Consider $(C,\mathcal{H})$ with $N(C,\mathcal{H}) = N$. 
For $H\in\mathcal{H}$ with $n_H > 1$, a chord $d$ is a \emph{good chord}
if it connects the last vertex of a run of $H$ with the first vertex of the next run of $H$
along $C$.
Its length $\ell(d)$ is
the number of vertices along $C$ between its end-points. 
A good chord of minimum length, denoted $d_H$ is the
\emph{critical chord} of $H$. 

Let $H = \arg\min_{H\in\mathcal{H}} \ell(d_H)$, breaking ties arbitrarily.
Let $d_H=\{u_1, u_2\}$. $d_H$ partitions $C$ into two open arcs $\alpha_1=(u_1, u_2)$,
and $\alpha_2=(u_2, u_1)$. Since $d_H$ is a critical chord of $H$, 
either $\alpha_1\cap H=\emptyset$, or $\alpha_2\cap H=\emptyset$. Assume the former. 
We obtain two induced sub-problems
on $C_R=\alpha_1\cup d_H$ and $C_L=\alpha_2\cup d_H$.
By Corollary \ref{cor:axax}, $d$ is non-blocking. A subgraph $H'\in\mathcal{H}$ s.t.
there is a vertex $v\in \alpha_1\cap H'$ is said to 
\emph{appear} in $C_R$. It follows from Lemma \ref{lem:nblockop} and Corollary \ref{cor:axax} that
if $H'$ appears in $C_R$, then $(H'\cap\alpha_2)\subseteq (H\cap\alpha_2)$ and if $(H'\cap\alpha_2)\ne\emptyset$, then $H'$ contains $u_1$ or $u_2$.
Thus, in the sub-problem induced on $C_L$, we can remove any $H'$ that appears in $C_R$.
Since $d_H$ joins two disjoint runs of $H$, $N(C_L, \mathcal{H}_{C_L}) < N$, where
$\mathcal{H}_{C_L}$ are the subgraphs $H\cap C_L$ for $H\in\mathcal{H}$ with containments removed. By the inductive
hypothesis, there is a support $Q_L$ on $\mathcal{H}_{C_L}$.

Now, consider the induced problem on $C_R$. By the minimality of $d_H$, each subgraph
contributes at most one run to the outer face $C_R$. By the base case of the induction
hypothesis, there is a support $Q'$ on $\mathcal{H}_{C_R}$ that is a cycle, where
$\mathcal{H}_{C_R}$ are the subgraphs $H\cap C_R$ for $H\in\mathcal{H}$.
Since the original system did not have any containments, it follows
that each subgraph in $\mathcal{H}\setminus H$ is in $\mathcal{H}_{C_R}$ or 
$\mathcal{H}_{C_L}$.

We obtain a graph $Q_R$ from the support $Q'$ of $\mathcal{H}_{C_R}$ by adding a chord from $H$ to each $H'\in \mathcal{H}_{C_R}$.
By construction, $\mathcal{H}_{C_R}\cap\mathcal{H}_{C_L}=\{H\}$. We obtain the desired
support $Q^*$ by identifying $H$ in $Q_L$ and $Q_R$. It follows that $Q^*$ is outerplanar.
Let $v$ be a vertex in $C_L$ that is contained in a subgraph
$H'$ that appears in $C_R$. Since $(H'\cap\alpha_2)\subseteq(H\cap\alpha_2)$, it follows
from the induction hypothesis and the fact that $H$ and $H'$ are adjacent in $Q_R$ that
$Q^*[\mathcal{H}_v]$ is connected. Finally, by Lemma \ref{lem:nblockop},
for any subgraph $H'\in\mathcal{H}\setminus H$ having a vertex in $\alpha_1$ and $\alpha_2$, contains $u_1$ or $u_2$, 
and thus is adjacent to $H$ in $Q_R$. The theorem follows.

Finding a subgraph with a critical chord of minimal length can be done in $O(|V(G)||\mathcal{H}|)$ time.
Since the two sub-problems are smaller, the overall running time is upper bounded by $O(|V(G)|^2|\mathcal{H}|)$.
\end{proof}

\section{Graphs of bounded treewidth}
\label{sec:treewidth}
In this section, we show that if $(G,\mathcal{H})$ is a non-piercing system, then there exist both a primal and dual support of treewidth $O(2^{tw(G)})$. Further, the supports can be computed in polynomial time if $G$ has bounded treewidth.


\subsection{Basic Tools for bounded treewidth graphs}
Let $G$ be a graph of treewidth $t$ and $\mathcal{H}$ be a collection of connected non-piercing subgraphs of $G$.
Throughout this section, we use $(T,\mathcal{B})$ to denote a tree decomposition, where
we assume without loss of generality that $T$ is a binary tree rooted
at a node $\rho$.
 
Let $CC(G)$ denote the \emph{chordal completion}
of $G$, i.e., for each bag $B\in\mathcal{B}$, we add edges between non-adjacent vertices such that each bag induces a complete subgraph. 
It is well-known that a chordal completion does not increase the treewidth of the underlying graph. 
It is easy to check that the subgraphs $\mathcal{H}$ remain non-piercing in $CC(G)$ if they were non-piercing in $G$. 
Further, in both the primal and dual settings, 
a support for the subgraphs defined on $CC(G)$ is also a support for the subgraphs defined on $G$.
Therefore, we assume without loss of generality in this section that $G$ is a chordal graph of treewidth $t$.

We use the following notation in this section: for a node $u$ of $T$,
let $T_u$ denote the sub-tree rooted at $u$, and let $G_u$ denote the subgraph induced by the bags associated with nodes in $T_u$ and
let $G'_u$ denote the graph induced by the bags in $T\setminus T_u$.
Let $A_{uv}=B_u\cap B_v$ denote the adhesion set between the bag $B_u$ at $u$ and the bag $B_v$ at its parent $v$
in $T$. $A_{uv}$ is a separator of $G$, and $G\setminus A_{uv}$ yields two disjoint induced subgraphs: $G_u\setminus A_{uv}$ and $G'_u\setminus A_{uv}$. 
Let $\mathcal{H}_{A_{uv}}=\{H\in\mathcal{H}: H\cap A_{uv}\neq\emptyset\}$. 

\begin{lemma}
\label{lem:sepnp}
Let $(G,\mathcal{H})$ be a non-piercing system with tree-decomposition $(T,\mathcal{B})$ of $G$. Then,
for any adhesion set $A_{uv}=B_u\cap B_v$, $(G_u,\mathcal{H}_u)$ and 
$(G'_u,\mathcal{H}'_u)$ are non-piercing systems, where 
$\mathcal{H}_u = \{H\cap G_u\neq\emptyset: H\in\mathcal{H}\}$ 
and $\mathcal{H}_{u'} = \{H\cap G'_u\neq\emptyset: H\in\mathcal{H}\}$.
\end{lemma}
\begin{proof}
We show that $\mathcal{H}_u$ is a collection of non-piercing subgraphs. An identical argument shows that $\mathcal{H}'_u$ is also a collection of non-piercing subgraphs. 
Let $H'_1$ and $H'_2$ be two arbitrary subgraphs in $\mathcal{H}_u$ corresponding respectively, to subgraphs $H_1$ and $H_2$ in $\mathcal{H}$. Since $\mathcal{H}$
is a non-piercing family, 
it follows that $H_1\setminus H_2$ and $H_2\setminus H_1$ are connected subgraphs of $G$. 
If $H_1\setminus H_2$ does
not intersect $A_{uv}$, then $H_1\setminus H_2$, and therefore $H'_1\setminus H'_2$ lies entirely in either $G_u$ or
$G'_u$ and is connected since $H_1\setminus H_2$ is connected. 

Otherwise, $H'_1\setminus H'_2$ intersects $A_{uv}$. By assumption, $G$ is a chordal graph, and
hence $A_{uv}$ is a complete subgraph. It follows that $H'_1\setminus H'_2$ is connected.
\end{proof}

We next show the following lemma that will be crucial for the construction of both the primal support and the dual support. For two sets $A$ and $B$ on the same ground set, we say that $A$ and $B$ properly intersect if $A\setminus B\neq\emptyset$ and $B\setminus A\neq\emptyset$.

\begin{lemma}
\label{lem:npsep}
Let $(G,\mathcal{H})$ be a non-piercing system with $(T,\mathcal{B})$,
a tree-decomposition of $G$.
Let $A_{uv}=B_u\cap B_v$ be an adhesion set corresponding to edge $e=(u,v)\in E(T)$ and 
let $H, H'\in\mathcal{H}_{A_{uv}}$. Suppose $(H\cap G_u)\subset (H'\cap G_u)$ and $H\cap A_{uv} = H'\cap A_{uv}$, then $(H'\cap G'_u)\subseteq (H\cap G'_u)$. Suppose
$H\cap G_u$ and $H'\cap G_u$ properly intersect, and $H\cap A_{uv} = H'\cap A_{uv}$, then $H\cap G'_u = H'\cap G'_u$. If
 $H\cap G_u$ and $H'\cap G_u$ properly intersect and $(H\cap A_{uv}) \subset (H'\cap A_{uv})$, then $(H\cap G'_u) \subseteq (H'\cap G'_u)$.
\end{lemma}
\begin{proof}
Consider $(H\cap G_u)\subset (H'\cap G_u)$ and let $x\in (H'\setminus H)\cap G_u$.
If $\exists y\in (H'\setminus H)\cap G'_u$ then $H'\setminus H$ has $x$ and $y$ in two different components since $H\cap A_{uv} = H'\cap A_{uv}$ will form a separator in $H'$. This contradicts the fact that $H$ and $H'$ are non-piercing.

Secondly, let $h\in (H\setminus H')\cap G_u$ and $h'\in(H'\setminus H)\cap G_u$ so that $H\cap G_u$ and $H'\cap G_u$ intersect properly.
If $\exists\; y\in (H\setminus H')\cap G'_u$, then $h$ and $y$ are not connected in $H\setminus H'$ since $H\cap A_{uv} = H'\cap A_{uv}$. Hence, $(H\cap G'_u)\subseteq (H'\cap G'_u)$. 
A symmetric argument shows $(H'\cap G'_u)\subseteq (H\cap G'_u)$.

Finally, $(H\setminus H')\cap G_u\ne\emptyset\ne(H'\setminus H)\cap G_u$ since $H\cap G_u$ and $H'\cap G_u$ intersect properly.
Given that $(H\cap A_{uv}) \subset (H'\cap A_{uv})$. Now, if $(H\setminus H')\cap G'_u\ne\emptyset$, then $H'\cap A_{uv}$ forms a separator for $H$; the result follows.
\end{proof}



\subsection{Primal Support}
\label{sec:primal}
Let $(G,\mathcal{H})$ be a non-piercing system, and let 
$c:V(G)\to\{\R,\B\}$. We show that there is a primal support $Q$ on $\B(V)$ of 
treewidth $O(2^{tw(G)})$. The proof is algorithmic and yields a polynomial time algorithm if $tw(G)$ is bounded.

Suppose the tree-decomposition $(T,\mathcal{B})$ of $G$ enjoys the additional property that for each adhesion set $A$, and
each subgraph $H\in\mathcal{H}_A$, we have that $H\cap \B(A)\neq\emptyset$, i.e., for each adhesion set $A$ and each subgraph $H$ intersecting $A$,
$H$ intersects $A$ in a blue vertex. Then, $(T,\mathcal{B})$ is said to be an \emph{easy} tree-decomposition. 
If $(T,\mathcal{B})$ is an easy tree-decomposition, it is straightforward to obtain the desired support $Q$, and in fact $tw(Q)\le tw(G)$. 

\begin{lemma}
\label{lem:easytd}
Let $(G,\mathcal{H})$ be a non-piercing system with $c:V(G)\to\{\R,\B\}$. Let $(T,\mathcal{B})$ be an easy tree-decomposition of
width $t$. Then, there is a support $Q$ on $\B(V)$ of treewidth at most $t$.
\end{lemma}
\begin{proof}
Given $(T,\mathcal{B})$, we obtain a tree-decomposition of $Q$ on $\B(V)$ by removing vertices of $\R(V)$ from
each bag $B\in\mathcal{B}$. To see that $Q$ is a support, consider a subgraph $H\in\mathcal{H}$. Note that
$H\cap \B(B)$ is connected as each $B\in\mathcal{B}$ induces a complete graph and $(T,\mathcal{B})$ is an easy tree decomposition.
\end{proof}

If $(T,\mathcal{B})$ is not an easy tree-decomposition, we modify it to obtain an easy
tree-decomposition of width $O(2^t)$, where $t$ is the width of the tree-decomposition $(T,\mathcal{B})$. 
We then obtain a support by applying Lemma \ref{lem:easytd}.


Let $T$ be rooted at $\rho$. The algorithm to modify $(T,\mathcal{B})$ into an easy tree-decomposition 
consists of two phases: In the first phase, we go bottom-up adding carefully chosen vertices of $\B(V)$
to the bags such that the resulting structure is a legal tree-decomposition. At the end of the first phase, for each adhesion set $A$, only a subset of the subgraphs
intersecting $A$ do so at a blue vertex. In the second phase, we go top-down from $\rho$, again adding carefully chosen vertices of $\B(V)$. At the end of the second phase,
we end up with an easy tree-decomposition of width at most $3\cdot 2^t$. 

Let $e=(u,v)$ be an edge in $T$ where $v$ is a parent of $u$. 
Consider a non-empty set $S\subseteq A_e$ such that
$S\cap \B(V)=\emptyset$. We define $\mathcal{H}'_S = \{H\in\mathcal{H}_A: H\cap A = S,\; H\cap \B(G_u)\neq\emptyset \; \mbox{ and }\; H\cap \B(G'_u)\neq\emptyset\}$. We want to 
add vertices in $\B(V)$ to $A$ so that the subgraphs in $\mathcal{H}'_S$ intersect $A$ at a blue vertex.
In the rest of this section, we make the following
assumptions: when we consider an adhesion set $A_{uv}$ corresponding to edge $(u,v)$ of $T$, we assume that $v$ is the parent of $u$. 
When we consider subsets $S$ of an 
adhesion set $A$, we implicitly assume that $\mathcal{H}'_S\neq\emptyset$. Further, we use $\mathcal{M}_S\subseteq\mathcal{H}'_S$ to denote
the set of minimal subgraphs in the containment order $\preceq_{G_u}$ defined on $\{H\cap G_u: H\in \mathcal{H}'_S\}$, i.e., for $H, H'\in\mathcal{H}'_S$, 
$H\preceq_{G_u} H' \Leftrightarrow (H\cap G_u)\subseteq (H'\cap G_u)$.  We use $(\preceq_{G_u}, \mathcal{H}'_S)$ to denote this partial order.

For a tree-decomposition $(T,\mathcal{B})$ of $(G,\mathcal{H})$, 
we say that a tree-decomposition $(T,\mathcal{B}'')$ for $(G,\mathcal{H})$ satisfies the \emph{bottom-up property with respect to} $(T,\mathcal{B})$ at an adhesion set $A_{uv}$ if $\forall S\subseteq A_{uv}$, 
$\exists H\in\mathcal{M}_S$ s.t. $H\cap \B(A''_{uv})\neq\emptyset$, 
where $A''_{uv}$ is the adhesion set in $(T,\mathcal{B}'')$ corresponding to $A_{uv}$. 
$(T,\mathcal{B}'')$ satisfies the bottom-up property with respect to $(T,\mathcal{B})$ if it
satisfies the bottom-up property at each
adhesion set in $(T,\mathcal{B})$.

\begin{lemma}
\label{lem:bottomup}
Let $(T,\mathcal{B})$ be a tree-decomposition of width $t$ of a non-piercing system $(G,\mathcal{H})$. 
Then, there is a tree-decomposition $(T,\mathcal{B}'')$ of width at most $2\cdot 2^t$ 
that satisfies the bottom-up property with respect to $(T,\mathcal{B})$.
\end{lemma}
\begin{proof}
To construct $(T,\mathcal{B}'')$ we proceed bottom-up from the leaves of $T$.
If an adhesion set $A_{uv}=B_u\cap B_v$ satisfies the following condition:
\begin{itemize}
\item[$(*)$] For each $S\subseteq A_{uv}$, $\exists\; H\in\mathcal{M}_S$ s.t. $H\cap \B(B_u)\neq\emptyset$
\end{itemize}

Then adding $\beta\in H\cap \B(B_u)$ to $B_v$ for each $S\subseteq A_{uv}$ ensures that  the bottom-up property is satisfied in the resulting adhesion set $A''_{uv}$.
We say
$B_v$ satisfies the bottom-up property if each adhesion set $A_{wv}$ satisfies the bottom-up property where $w$ is a child of $v$ in $T$.

By definition of $\mathcal{H}'_S$, condition $(*)$ is satisfied for each adhesion set $A_{uv}$ where
$u$ is a leaf of $T$. For a node $u$ with children $x$ and $y$, and parent $v$, we claim that if 
$A_{xu}$ and $A_{yu}$ satisfy the bottom-up property, then the adhesion set $B''_u\cap B_v$ satisfies the property $(*)$. 
This is sufficient to prove the lemma as we can process the adhesion sets bottom-up. 

Thus, suppose the bottom-up property is satisfied at $A_{xu}$ and $A_{yu}$. 
Let $S\subseteq B''_u\cap B_v$.
If there is a subgraph $H$ in $\mathcal{M}_S$ s.t. $H\cap A_{xu}=\emptyset$ and
$H\cap A_{yu}=\emptyset$, then, $\B(H)\cap B''_u\neq\emptyset$ by definition of $\mathcal{H}'_S$.
So, we can assume that $H\cap A_{xu}\neq\emptyset$ or $H\cap A_{yu}\neq\emptyset$ for each 
$H\in\mathcal{M}_S$.
Assume wlog the former holds for some $H\in\mathcal{M}_S$. 
Then, $\exists H\in\mathcal{M}_S$ s.t.
$H\cap A_{xu}=S'$. Since $A''_{xu}$ satisfies the bottom-up property, 
there is a subgraph $H'\in\mathcal{M}_{S'}$ s.t. $H'\cap A_{xu}=S'$ and
$H'\cap \B(A''_{xu})\neq\emptyset$. 
If $H'\preceq_{G_x} H$ in the partial order $(\preceq_{G_x}, \mathcal{H}'_{S'})$, then, 
$\B(H)\cap B''_u\neq\emptyset$. Otherwise, 
if $H'$ and $H$ are incomparable in $(\preceq_{G_x}, \mathcal{H}'_{S'})$, then by Lemma \ref{lem:npsep} $H'\cap G'_x=H\cap G'_x$ since $H\cap A_{xu}=S'=H'\cap A_{xu}$. Therefore, $H'\in\mathcal{M}_S$, and
$H'\cap \B(B''_u)\neq\emptyset$, since $x$ is a child of $u$.
\end{proof}

\begin{lemma}
\label{lem:topdown}
Let $(G,\mathcal{H})$ be a non-piercing system with a tree-decomposition $(T,\mathcal{B})$ of width $t$ that is not an easy tree-decomposition.
Then, there exists an easy tree-decomposition
$(T,\mathcal{B}')$ of width at most $3\cdot 2^t$.
\end{lemma}
\begin{proof}
By Lemma \ref{lem:bottomup}, we obtain a tree-decomposition $(T,\mathcal{B}'')$ of width at most 
$2\cdot 2^t$ that satisfies the bottom-up property with respect to $(T,\mathcal{B})$. 

For an adhesion set $A_{uv}$ in $(T,\mathcal{B})$ and $S\subseteq A_{uv}$, $S$ is said to be satisfied if for all $H\in\mathcal{H}'_S$, $H\cap \B(A''_{uv})\neq\emptyset$, 
where $A''_{uv}$ is the adhesion set corresponding to $A_{uv}$ in $(T,\mathcal{B}'')$.
An adhesion set $A_{uv}$ in $(T,\mathcal{B})$ is said to be \emph{satisfied} if $S$ is satisfied
for all $S\subseteq A_{uv}$. 
We say $A_{uv}$ is \emph{nearly satisfied}
if for all edges $e$ closer to the root $\rho$ of $T$ than
$uv$, $A_e$ is satisfied. 

We claim that 
if $A_{uv}$ is nearly satisfied, then for each $S\subseteq A_{uv}$ and each $H\in\mathcal{H}'_S$,
$H\cap \B(B''_v)\neq\emptyset$. Further, it is sufficient to pick one vertex in $\B(B''_v)$ and add it to 
$B''_u$ to ensure that all subgraphs in $\mathcal{H}'_S$ intersect $A''_{uv}$ in a blue vertex.

Suppose there is a subgraph $H\in\mathcal{H}'_S$ such that
$H\cap \B(A''_{uv})=\emptyset$. Since $(T,\mathcal{B}'')$ satisfies the bottom-up property, 
there is a subgraph $H'\in\mathcal{M}_S$ s.t. $H'\cap \B(A''_{uv})\neq\emptyset$. Since 
$H'$ is a minimal subgraph in $\mathcal{H}'_S$, it follows
that $H$ and $H'$ are incomparable in $(\mathcal{H}'_S, \preceq_{G_u})$. 

Suppose $A_{uv}$ is nearly satisfied and $w\neq u$ is the other child of $v$.
Further, by definition of $\mathcal{H}'_S$, $H$ contains a blue vertex in
$G'_u$. 
Therefore, $H$ contains a vertex in the adhesion set $A_{vx}$ 
where $x$ is the parent of $v$ in $T$,
or $H$ intersects $A_{wv}$ or $H$ intersects $B''_v$ only. In the third case, $H\cap \B(B''_v)\neq\emptyset$.
In the first case, $H\in\mathcal{H}'_{S'}$
for some $S'\subseteq A_{vx}$. Since $A_{uv}$ is nearly satisfied, $H\cap \B(A''_{vx})\neq\emptyset$ and
therefore $H\cap \B(B''_v)\neq\emptyset$.

So suppose $H\cap A''_{vx}=\emptyset$, and $H\cap A_{wv}=S'$. Since $(T,\mathcal{B}'')$ 
satisfies the bottom-up property, there is a subgraph 
$H''\in\mathcal{M}_{S'}$ s.t. 
$H''\cap \B(A''_{wv})\neq\emptyset$ and
therefore $H''\cap \B(B''_{v})\neq\emptyset$. 
If $H\cap \B(B''_{v})=\emptyset$, it must be that $H$ and $H''$ are incomparable in $G_w$.
By Lemma \ref{lem:npsep}, therefore, $H\cap G'_w= H''\cap G'_w$. 
Since $H$ and $H'$ are incomparable
in $G_u$, it follows again by Lemma \ref{lem:npsep} that $H\cap G'_u = H'\cap G'_u$. 
This implies $H'\cap G_w$ and $H''\cap G_w$ are also incomparable.
But, it is impossible for $H''\cap G_u$ to be equal to $H\cap G_u$ and $H'\cap G_u$ which are incomparable.
Therefore, $H\cap \B(B''_{v})\neq\emptyset$ and since the argument above holds for any subgraph
in $\mathcal{H}'_S$ incomparable with $H'$, it follows that there is a single vertex in $\B(B''_v)$
that is contained in all subgraphs in $\mathcal{H}'_S$ not comparable with $H'$.
Adding such a vertex $\beta\in H\cap \B(B''_{u})$ to $B''_u$ ensures that $S$ is satisfied. Repeating this 
process for each $S\subseteq A_{uv}$ ensures that $A_{uv}$ is satisfied.

$A_{u\rho}$ is clearly nearly satisfied, where $\rho$ is the root of $T$.
By the argument above, we can add a single blue node from $B''_{\rho}$ to $B''_u$ for each
$S\subseteq A_{u\rho}$, where $u$ is a child of $\rho$. This ensures that $A''_{yu}$ is nearly
satisfied for each child $y$ of $u$. Repeating this process top-down till the leaves of $T$ ensures
that the resulting tree-decomposition $(T,\mathcal{B}')$ is easy. For each subset of an adhesion set $A_{uv}$
we add at most one blue vertex from $B''_{v}$ to $B''_u$. Therefore the treewidth of $(T,\mathcal{B}')$ is at most $3\cdot 2^t$.
\end{proof}


\begin{theorem}
Let $(G,\mathcal{H})$ be a non-piercing system. Let $c:V(G)\to\{\R,\B\}$.
Then, there is a support $Q$ on $\B(V)$ s.t. $tw(Q)\le 3\cdot 2^{tw(G)}$. Further, $Q$ can be computed in time polynomial in $|G|, |H|$ if $G$ has bounded treewidth.
\end{theorem}
\begin{proof}
If $(T,\mathcal{B})$ is an easy tree-decomposition, then by Lemma \ref{lem:easytd}, we obtain a support $Q=(\B(V), F)$ of treewidth at most $tw(G)\le 3\cdot 2^{tw(G)}$. 
Otherwise, we apply Lemma \ref{lem:topdown} to obtain an easy tree-decomposition $(T,\mathcal{B}')$ of 
width at most $3\cdot 2^t$.
Applying Lemma \ref{lem:easytd} to $(T,\mathcal{B}')$ yields a support $Q=(\B(V), F)$ of
the treewidth at most $3\cdot 2^t$ for $(G,\mathcal{H})$.

If $G$ has treewidth bounded above by $t$, then an optimal tree-decomposition of $G$ can be computed in
$O(2^{t}poly(n))$ time \cite{korhonen2022single}. Now, Lemma \ref{lem:easytd}, Lemma \ref{lem:bottomup} and Lemma \ref{lem:topdown} suggest a natural two phase-algorithm:
Going bottom-up in $T$ and for each adhesion set $A$ and each $S\subseteq A$, adding a blue subgraph
corresponding to a minimal subgraph to a bag, and then doing a similar operation top-down.
Therefore, the time required to process an adhesion set is $O(2^t poly(\mathcal|H|))$. Thus, 
the overall running time is $O(poly(|G|,|\mathcal{H}|)2^t)$, which is polynomial for bounded $t$.
\end{proof}



\subsection{Dual Support}
\label{sec:dual}

Let $(G,\mathcal{H})$ be a non-piercing system. We show in this
section that the system admits a dual support $Q^*$ s.t. $tw(Q^*)\le 4.c\dot 2^{tw(G)}$. Further, if $tw(G)$ is bounded above by a constant, then 
$Q^*$ can be computed in time polynomial in $|V(G)|,|\mathcal{H}|$.
We start with a special case where it is easy to construct a support and then 
show how the general case can be reduced to this simple case. 
For a graph system $(G,\mathcal{H})$, where $\mathcal{H}$ is a collection of (possibly piercing) induced subgraphs of $G$, if a tree-decomposition $(T,\mathcal{B})$ of $G$ 
is such that for each bag
$B\in \mathcal{B}$, $|\mathcal{H}\cap B|\le k$, then, we call the system $(G,\mathcal{H})$ $k$-sparse.

\begin{lemma}
\label{lem:easydual}
Let $(G,\mathcal{H})$ be a (possibly piercing) system where $tw(G)=t$ and each $H\in\mathcal{H}$ induces a connected induced subgraph of $G$. Let $(T,\mathcal{B})$ be a nice tree-decomposition of $G$ that is $k$-sparse. Then 
there is a dual support $Q^*=(\mathcal{H},F)$ of treewidth at most $k$.
\end{lemma}
\begin{proof}
We obtain a tree-decomposition $(T',\mathcal{B}')$ of $Q^*=(\mathcal{H},F)$ as follows: the tree $T'$ is isomorphic to $T$.
Corresponding to each bag $B\in\mathcal{B}$, we construct a bag $B'\in\mathcal{B}'$ such that
$B'$ consists of a vertex $v_H$ for each subgraph $H\in\mathcal{H}$ such that $H\cap B\neq\emptyset$. 
Since each subgraph is connected $G$, each vertex $v_H$ corresponding to a subgraph $H\in\mathcal{H}$ lies in a connected subset of bags of $T'$. 
Further, since $|B\cap\mathcal{H}| \le k$, it follows that the resulting tree-decomposition has width at most $k$. Finally, we claim that adding an edge between each pair of vertices $u_H$, and $v_{H'}$
such that $u_H$ and $v_{H'}$ lie in the same bag results in a dual support for $(G,\mathcal{H})$. 
To see this, consider a vertex $v\in V(G)$ and a bag $B\in\mathcal{B}$ containing $v$. The bag $B'\in\mathcal{B}'$ corresponding to $B$ contains the subgraphs $\mathcal{H}_v$ by construction. Adding edges between all pairs of subgraphs in $B'$ ensures that $\mathcal{H}_v$ is connected. The result follows.
\end{proof}

To obtain a dual support, we sparsify the input graph so that it satisfies the conditions of Lemma \ref{lem:easydual} and such that a support for the original system
can be obtained from the support for the new system. We proceed bottom-up in the tree-decomposition of $G$, and for each adhesion set $A_{uv}=B_u\cap B_v$,
and for each $S\subseteq A_{uv}$, we choose a minimal subgraph in $\mathcal{H}_S$ and use it to 
\emph{push out} a collection of subgraphs. 
This sparsification will ensure that at the end, there are at most $O(2^{t})$ distinct subgraphs
(subgraphs $H$ and $H'$ are distinct if $(H\cap V(G))\neq (H'\cap V(G))$) intersecting each bag as there are 
at most $2^t$ distinct subsets intersecting each adhesion set. In the following, for
an adhesion set $A$ and $S\subseteq A$, we let $\mathcal{H}'_S=\{H\in\mathcal{H}: H\cap A = S\}$. 

\begin{lemma}
\label{lem:mvi}
Let $(G,\mathcal{H})$ be a non-piercing system with tree-decomposition $(T,\mathcal{B})$ of width $t$.
Then, we can obtain a system $(G,\mathcal{H}')$ of (possibly piercing) connected induced subgraphs 
such that each bag of $(T,\mathcal{B})$ intersects at most $4\cdot 2^t$ distinct subgraphs.
\end{lemma}
\begin{proof}
We process the adhesion sets bottom-up from the leaves of $T$. Let $A_{uv}$ be an adhesion set with
$v$ the parent of $u$ in $T$. Having processed the adhesion sets below $A_{uv}$, we do the following
at $A_{uv}$: For each $S\subseteq A_{uv}$, choose a subgraph $H\in\mathcal{H}'_S$ 
that is \emph{minimal} in $(\mathcal{H}'_S, \preceq_{G_u})$. For each subgraph
$H'\in \mathcal{H}_{A_{uv}}$ s.t. $(H'\cap G_u)\setminus (H\cap G_u)\neq\emptyset$ and
$(H'\cap A_{uv})\subseteq S$, replace $H'$ by $H''=(H'\cap G_u)\setminus (H\cap G_u)$. We say that
$H'$ is \emph{pushed out} by $H$. The collection $\mathcal{H}''$ of subgraphs is obtained by 
replacing each subgraph
in $\mathcal{H}$ by its pushed out copy. Observe that $\mathcal{H}''$ may contain identical subgraphs 
even if $\mathcal{H}$ did not. Let $unique(\mathcal{H}'')$ denote the subgraphs obtained by 
keeping a unique copy of each set of identical subgraphs.

We claim that at the end of this process, each subgraph is pushed out at most once,
each subgraph in $\mathcal{H}''$ is a connected induced subgraph of $G$, 
and that the resulting
system $(G,unique(\mathcal{H}''))$ is $4\cdot 2^t$ sparse. 

For the first part, observe that since a subgraph
$H'$ is connected, it belongs to a connected subset of bags of $T$. Once $H'$ is pushed out at an
adhesion set $A_{uv}$, $H'$ does not intersect any adhesion set in $T'_u$. Thus, each subgraph
is pushed out at most once since we process the adhesion sets bottom-up.

The second part follows from the fact that the system $(G,\mathcal{H})$ is non-piercing, and therefore
$H''=(H'\cap G_u)\setminus (H\cap G_u)$ is a connected induced subgraph of $G_u$. The fact that $\mathcal{H}''$ consists
of connected induced subgraphs of $G$ follows from the fact that each subgraph is pushed out at most once.

For the third part, Once we have pushed out subgraphs at an adhesion set $A$, observe that for each
adhesion set $A$ and each $S\subseteq A$, 
there is at most one subgraph $H\in unique(\mathcal{H}'')$ s.t. $H\cap A=S$.
Since $T$ is a binary tree, each bag intersects at most 3 adhesion sets and each adhesion set is
intersected by 
at most $2^t$ subgraphs in $unique(\mathcal{H}'')$. Therefore,
there are at most $3\cdot 2^t$ subgraphs intersecting a bag $B\in\mathcal{B}$, and
an additional at most $2^t$ distinct subgraphs intersecting $B$ at vertices of $B$ not contained in any adhesion set intersecting $B$. The result follows.
\end{proof}

\begin{theorem}
Let  $(G,\mathcal{H})$ be a non-piercing system. There is a dual support $Q^*$ on $\mathcal{H}$
s.t. $tw(Q^*)\le 4\cdot 2^t$ where $t$ is the treewidth of $G$. Further, $Q^*$ can be
computed in time polynomial in $|G|, |\mathcal{H}|$ if $G$ has bounded treewidth.
\end{theorem}
\begin{proof}
If $(T,\mathcal{B})$ is at most $4\cdot 2^t$-sparse, we obtain a support $Q^*$ by Lemma \ref{lem:easydual}.
Otherwise, we apply Lemma \ref{lem:mvi} to obtain a system $(G, unique(\mathcal{H}''))$ s.t. 
$(G,unique(\mathcal{H}'')$ is $4\cdot 2^t$ sparse. Now, by Lemma \ref{lem:easydual}, 
we obtain a support $Q$ for $(G, unique(\mathcal{H}''))$ of width at most $4\cdot 2^t$.

To obtain a support $Q^*$ for $(G,\mathcal{H})$, for each $H\in unique(\mathcal{H}'')$, 
we add a new vertex $v_{H'}$ for each $H'\in\mathcal{H}''$ identical to $H$ and add the edges 
$\{v_{H}',v_{H}\}$ to $Q$. Since this operation does not increase the treewidth, $tw(Q^*)\le 4\cdot 2^t$.

We show that
$Q^*$ is a support for $(G,\mathcal{H})$. If $H'$ was pushed out by $H$, it follows that
$(H'\cap G'_u)\subseteq (H\cap G'_u)$ by Lemma \ref{lem:npsep}. Further, since $H'$ is connected, 
there is an edge $e=\{u,v\}$ in $G$ s.t. $u\in H'\setminus H$ and $v\in H\cap H'$. 
Hence, there is a bag $B\in\mathcal{B}$ containing $e$. By the way we construct a
support in Lemma \ref{lem:easydual}, it follows that $H$ and $H'$ are connected. 
Let $v\in V(G)$. The subgraphs in $unique(\mathcal{H}'')$ containing $v$ induce a connected
subgraph of $Q^*$. If $H'\ni v$ was pushed out, there is a subgraph $H\in unique(\mathcal{H}'')$
that contains $v$. By the argument above, it follows that there is a path from $H'$ to $H$ in $Q^*$
containing only subgraphs in $\mathcal{H}_v$.


If $G$ has treewidth $t$, bounded above by a constant, then by the result of Korhonen \cite{korhonen2022single},
a tree-decomposition of $G$ of width $t$ can be computed in time $O(2^t poly(n))$.
Lemma \ref{lem:easydual} and Lemma \ref{lem:mvi} suggest a natural bottom-up algorithm. The algorithm 
works by iterating over all subsets of each adhesion set and pushes out a subset of subgraphs. 
It is easy to see that the time taken to process an adhesion set is $O(2^t poly(|\mathcal{H}|)$, 
and the overall algorithm runs in $O(poly(|G|,|\mathcal{H}|2^t)$, which is polynomial for bounded $t$.
\end{proof}
One may wonder if the non-piercing condition is necessary to obtain a support of bounded treewidth. The following examples
show that this is indeed the case. For the primal, consider a star $K_{1,n}$ with the leaves colored blue, and the central vertex
red. Consider a collection of induced subgraphs defined by all pairs of leaves. It is easy to see that the support is a complete graph $K_n$ on the blue vertices.
For the dual, consider a star $K_{1, \binom{n}{2}}$. Each leaf of a star is labelled by a unique pair of $\{1,\ldots, n\}$.
There are $n$ subgraphs. The subgraph $i$ contains the central vertex and the leaves that contain the label $i$. The subgraphs
are piercing, and the dual support is $K_{n}$. 

\section{Lower Bounds}
In this section, we show that there exist graphs of treewidth $t$ whose (primal or dual) support requires treewidth $\Omega(2^t)$.

\begin{theorem}
\label{thm:primallb}
For any $\epsilon > 0$, there exists a graph $G=(V,E)$ with $c:V\to\{\R,\B\}$, and a collection of connected non-piercing induced subgraphs $\mathcal{H}$ such that 
any primal support $Q=(\B(V),F)$ of $(G,\mathcal{H})$ has treewidth $\Omega(2^{t/8(1+\epsilon)})$.
\end{theorem}
\begin{proof}
For any $t$,  let $N=2^t$. We construct the following graph $G(V,E)$: We start with an $N\times N$ grid of blue points $b_{i,j}$ for $i,j=1,\ldots, N$.
We construct four sets $U,D,L,R$ of $2n$ red points each, where $n = c\log N$, for $c>0$ chosen s.t. $\binom{2n}{n}\ge N$.
Since $\binom{2n}{n}\ge 2^{n}/(n+1)$, for any $\epsilon > 0$, choosing $c=1+\epsilon$ satisfies the condition for $N$ sufficiently large.

\begin{figure}[ht!]
    \centering
    \begin{subfigure}{0.4\textwidth}
            \includegraphics[scale=.57]{primaltw}
        \caption{$H_{3,2}$ and $H'_{3,2}$ are shown on $5\times 5$ gird.}
        \label{fig:primaltw}
    \end{subfigure}\hspace{2cm}
    \begin{subfigure}{0.4\textwidth}
        \includegraphics[scale=.48]{primaltwnp}\vspace{0.5cm}
        \caption{The subgraphs $H_{ij}$ and $H_{k\ell}$ are shown non-piercing.}
        \label{fig:primaltwnp}
    \end{subfigure}
    \caption{Construction of primal hypergraph with support containing a grid minor.}
\end{figure}

Let $U_1,\ldots, U_{N-1}$ and $L_1,\ldots, L_{N-1}$ denote $N-1$ distinct subsets of $U$ and $L$, respectively, each of size $n$. Similarly, let $D_1,\ldots, D_N$ and $R_1,\ldots, R_N$ denote $N$ distinct 
subsets of $D$ and $L$, respectively, each of size $n$.
Since we assume that $\binom{2n}{n}\ge N$, this can indeed be done.

For each pair $b_{i,j}, b_{i,j+1}$ for $i=1,\ldots, N$ and $j=1,\ldots, N-1$,
we make $b_{i,j}$ and $b_{i,j+1}$ adjacent to all vertices in the set $U_j$ and all vertices
in the set $R_i$.
Next, for each pair $b_{i,j}, b_{i+1,j}$ for $i=1,\ldots, N-1$, $j=1,\ldots, N$ we 
make $b_{i,j}$ and $b_{i+1,j}$ adjacent to all vertices in the set $D_j$ and the set $L_i$.
This completes the construction of the graph (see Figure \ref{fig:primaltw}).

Since there are no edges between any pair of blue vertices or between any pair of red vertices, 
it follows that $G$ is bipartite.
Further, $|r(V)| = 8c\log N$ and $|b(V)| = N^2$. Thus, $tw(G)\le 8c\log N$.

Now we add a collection of non-piercing subgraphs to $G$ that will force the treewidth of the support to be $\Omega(N)$.
For each pair $b_{i,j},b_{i,j+1}$, $i=1,\ldots, N, j=1,\ldots, N-1$ we add the subgraph $H_{ij}$ induced by the vertices 
$b_{i,j}\cup b_{i,j+1}\cup U_j \cup R_i$.
Similarly, for each pair $b_{i,j},b_{i+1,j}$, $i=1,\ldots, N-1$ and $j=1,\ldots, N$ we add
a subgraph $H'_{ij}$ induced on the vertices $\{b_{i,j}\}\cup\{b_{i+1,j}\}\cup D_j\cup L_i$.
Let 
$\mathcal{H}=\{H_{ij}:i=1,\ldots, N, j=1,\ldots,N-1\}\cup\{H'_{ij}:i=1,\ldots, N-1,j=1,\ldots, N\}$.

We claim that $\mathcal{H}$ is a non-piercing collection of connected induced subgraphs of $G$. By definition, each subgraph in
$\mathcal{H}$ is a connected induced subgraph of $G$. It only remains to show that the subgraphs are non-piercing. Consider two
subgraphs $H_{ij}$ and $H_{k\ell}$ in $\mathcal{H}$ (as shown in Figure \ref{fig:primaltwnp}). $H_{ij}$ is the graph induced on the vertices $b_{i,j}\cup b_{i,j+1}\cup U_j\cup R_i$ and
$H_{k\ell}$ is the graph induced on the vertices $b_{k,\ell}\cup b_{k,\ell+1}\cup U_{\ell}\cup R_{k}$.
Thus, $H_{ij}\setminus H_{k\ell}$ consists
of the graph induced on the vertices $b_{i,j}\cup b_{i,j+1}\cup (U_j\setminus U_{\ell})\cup (R_i\setminus R_{k})$. If $j\neq\ell$,  $U_j\setminus U_\ell$ is non-empty,
and if $ i\neq k$,  $R_i\setminus R_{k}$ is non-empty. Since $b_{i,j}$ and $b_{i,j+1}$ are adjacent to each vertex in $U_j$ and $R_i$, 
it follows that $H_{ij}\setminus H_{k\ell}$ is connected. A symmetric argument shows that $H_{k\ell}\setminus H_{ij}$ is connected. A similar argument
shows that subgraphs $H_{ij}$ and $H'_{k\ell}$ are non-piercing for any choice of $i,j,k$ and $\ell$. 

Each subgraph in $\mathcal{H}$ consists of exactly two blue vertices in consecutive rows or two blue vertices in consecutive columns. Therefore, any support 
$Q(\B(G),F)$
for the system $(G,\mathcal{H})$ must have an $N\times N$ grid as a subgraph. Therefore, $tw(Q)\ge N \geq 2^{tw(G)/8c}$.
\end{proof}

\begin{theorem}
\label{thm:dualtwlb}
For any $\epsilon > 0$ there exists a graph $G=(V,E)$ and a collection of connected non-piercing induced subgraphs $\mathcal{H}$ such that
any dual support $Q=(\mathcal{H},F)$ has treewidth $\Omega(2^{tw(G)})$.
\end{theorem}
\begin{proof}
Our construction of $G$ is similar to the construction for the primal support in Theorem \ref{thm:primallb}.
We start with a grid of $(2N+1)\times (2N+1)$ points $b_{ij}$, $i,j=1,\ldots, 2N+1$.
At each point $b_{2i, 2j+1}, i=1,\ldots, N, j=0,\ldots, N$ add a vertex $g_{2i,2j+1}$ (See Figure \ref{fig:dualtw}). Similarly, at each point
$b_{2i+1,2j}, i=0,\ldots, N, j=1,\ldots, N$ add a vertex $g_{2i+1,2j}$. Let $K$ denote this set of vertices added.
Let $A$ and $B$ be two sets of vertices of size $2n$ each, where $n=c\log N$ such that $\binom{2n}{n}\ge N$. 
Let $A_1,\ldots, A_N$ be distinct subsets of $A$ of size $n$ each, and let $B_1,\ldots, B_N$ be distinct subsets of $B$ of size $n$ each.
Each point $g_{2i,2j+1}$ is adjacent to each vertex in $A_i,B_j$ and $B_{j+1}$ where $B_0$ and $B_{N+1}$ are empty sets.
Similarly, each vertex in $g_{2i+1, 2j}$ is adjacent to each vertex
in $A_i,A_{i+1}$ and $B_j$ where $A_0$ and $A_{N+1}$ are empty sets. This completes the construction of the graph. $G$ is a bipartite graph with bipartition $K$ and $A\cup B$, 
as the vertices in $K$ are pairwise non-adjacent, and so are the vertices in $A\cup B$. Further, $|K| = N^2$ and $ |A\cup B| = 4n$. 
Thus, $tw(G)\le 4n = 4c\log N$.

\begin{figure}[ht!]

    \centering
    \begin{subfigure}{0.5\textwidth}
    \vspace{0.31cm}
        \includegraphics[scale=.55]{dualtw}\vspace{0.5cm}
        \caption{$H_{11}$ (blue) and $H_{22}$ (orange) are shown on a $5\times 5$ gird.}
        \label{fig:dualtw}
        \end{subfigure}\hspace{0.5cm}
    \begin{subfigure}{0.4\textwidth}\vspace{1.3cm}
        \includegraphics[scale=.47]{dualtwnp}\vspace{1cm}
        \caption{The subgraphs $H_{ij}$ and $H_{k\ell}$ are shown non-piercing.}
        \label{fig:dualtwnp}
    \end{subfigure}
    \caption{Construction of dual hypergraph with support containing a grid as a subgraph.}
\end{figure}
For each point $b_{2i,2j}, i,j=1,\ldots, N$ we construct a subgraph $H_{ij}$ induced on the vertices $g_{2i-1, 2j}, g_{2i+1,2j}, g_{2i,2j-1},g_{2i,2j+1}\cup A_i\cup B_j$.
It is easy to see that the subgraphs $\mathcal{H} = \cup_{i,j\in\{1,\ldots, N\}} H_{ij}$ are non-piercing: For $H_{ij}$ and $H_{k\ell}$,
it follows that $H_{ij}\setminus H_{k\ell}$ contains as a subset, the vertices in  $(A_i\setminus A_k)\cup(B_j\setminus B_{\ell})$ as shown in Figure \ref{fig:dualtwnp}. 
Since $H_{ij}$ and
$H_{k\ell}$ differ in at least one index, at least one of the sets $A_i\setminus A_k$, or $B_j\setminus B_{\ell}$ are non-empty, and the vertices in $K\cap (H_{ij}\setminus H_{k\ell})$ are adjacent to all vertices in $(A_i\setminus A_k)\cup(B_j\setminus B_{\ell})$. By the construction of $\mathcal{H}$, the set $K\cap (H_{ij}\setminus H_{k\ell})$ contains at least one vertex. Hence $H_{ij}\setminus H_{k\ell}$ is connected.

By construction, the vertex $g_{2i,2j+1}$ is contained only in the subgraphs $H_{ij}$ and $H_{i,j+1}$. This pair of subgraphs must be adjacent in any dual support $Q$.
Similarly, the vertex $g_{2i+1,2j}$ is contained only in subgraphs $H_{ij}$ and $H_{i+1,j}$ and this pair of subgraphs should also be adjacent in $Q$.
Therefore, $Q$ contains an $N\times N$ grid as an induced subgraph. It follows that $tw(Q)=\Omega(N) = \Omega(2^{tw(G)/4c})$. 
For any $\epsilon > 0$, setting $c = 1+\epsilon$, there is an $N$ large enough so that $\binom{2n}{n}\ge N$. Therefore, $tw(Q)=\Omega(2^{tw(G)/4(1+\epsilon)})$.
\end{proof}

\section{Applications}
\label{sec:applications}
In this section, we describe some applications of the existence of supports.
Raman and Ray \cite{RR18} showed that for an intersection hypergraph defined on a
set of \emph{non-piercing regions}\footnote{A set of path connected regions $\mathcal{R}$ in the plane is non-piercing if for any $A, B\in\mathcal{R}$,
$A\setminus B$ and $B\setminus A$ are path connected.}
in the plane, there is a planar support (See \cite{RR18} for
precise definitions), which implies a support for both the primal and dual settings for the
hypergraph defined by points and non-piercing regions in the plane.

Since graphs of genus $g$ admit separators of 
size $O(\sqrt{gn})$ \cite{gilbert1984separator}, all the algorithmic consequence of \cite{RR18}
generalize to cross-free systems on bounded genus graphs.
We highlight three results that follow as a consequence of Theorem \ref{thm:intsupport}.

\begin{theorem}
\label{thm:ptascrossfree}
Let $(G,\mathcal{H})$ be a cross-free system of genus $g$, then there exists 
\begin{enumerate}
    \item a PTAS for the Dominating Set problem, which is a problem to find $\mathcal{H}'\subseteq\mathcal{H}$ of minimum 
    cardinality s.t. for each $H\in\mathcal{H}$, either $H\in\mathcal{H}'$ or $H\cap H'\neq\emptyset$ for
    some $H'\in\mathcal{H}'$.
\item a PTAS for the problem of packing points when each subgraph $H\in\mathcal{H}$ has capacity $D_H$ 
bounded by a constant, i.e.,
find $V'\subseteq V$ of maximum cardinality s.t. $|H\cap V'|\le D_H$ for each $H\in\mathcal{H}$.
\item a PTAS for the problem of packing subgraphs when each vertex $v\in V$ has capacity $D_v$ bounded by a constant, i.e.,
find $\mathcal{H}'\subseteq\mathcal{H}$ of maximum cardinality s.t. $|\{H\in\mathcal{H}': H\ni v\}|\le D_v$ for each $v\in V$.
\end{enumerate}
\end{theorem}

We believe that the cross-free condition is essential to obtain PTASes for packing
and covering problems when the host graph has bounded genus.
Chan and Grant \cite{DBLP:journals/comgeo/ChanG14}
proved that for a hypergraph defined by a set of horizontal and vertical slabs 
in the plane and a set of points $P$, the Hitting Set problem and
the Set Cover problems are  APX-hard. A simple modification of their result implies the following.

\begin{theorem}
\label{thm:apxhard}
There exist crossing non-piercing systems $(G,\mathcal{H})$ with $G$ embedded in the torus such that the
hitting set or set cover problems are APX-hard.
\end{theorem}

Keller and Smorodinsky \cite{KellerS18} showed that the intersection hypergraph of disks in the plane can be colored
with 4 colors, and this was generalized by Keszegh \cite{Keszegh20} for pseudodisks, which was 
further generalized in \cite{RR18} to show that the intersection hypergraph of non-piercing regions is 4-colorable. 
As a consequence of Theorem \ref{thm:intsupport}, 
we obtain the following.

\begin{theorem}
\label{thm:colorhypergraph}
Let $(G,\mathcal{H},\mathcal{K})$ be a cross-free intersection system where $G=(V,E)$ is embedded in an orientable 
surface of genus $g$. Then, $\mathcal{H}$ can be colored with at most $\frac{7 + \sqrt{1+24g}}{2}$ colors such that 
no hyperedge ${\mathcal{H}_K}$ is monochromatic. 
\end{theorem}
\begin{proof}
By Theorem \ref{thm:intsupport}, $(G,\mathcal{H},\mathcal{K})$ has a support $\tilde{Q}$ of genus at most $g$.
Now, $\chi(\tilde{Q})\le \frac{7 + \sqrt{1+24g}}{2}$ \cite{diestel2005graph}. Since $\tilde{Q}$ is a support,
for each $K\in\mathcal{K}$, there is an edge between some two subgraphs $H,H'\in {\mathcal{H}_K}$. Therefore,
no hyperedge ${\mathcal{H}_K}$ is monochromatic.
\end{proof}

Ackerman et al. \cite{ackerman2020coloring} considered a notion of $ABAB$-free hypergraphs, which is defined as follows: a 
hypergraph $(X,\mathcal{S})$ is $ABAB$-free if there exists a  linear ordering $x_1<\ldots< x_n$ of $X$ such that for any pair of subgraphs
$A,B\in\mathcal{S}$, there are no four elements $x_i < x_j < x_k < x_{\ell}$ s.t. $x_i,x_k\in A\setminus B$ and $x_j,x_{\ell}\in B\setminus A$.
The notion of $ABAB$-free hypergraphs is equivalent to the notion of $abab$-free hypergraphs (See Defn. \ref{defn:abab}).
Indeed, if there exists a linear ordering $x_1<\ldots<x_n$ that is $ABAB$-free, then the cyclic order $x_1<\ldots<x_n<x_1$ is $abab$-free, 
and similarly, if $x_1<\ldots, x_n<x_1$ is a cyclic order that is $abab$-free, then $x_1<\ldots<x_n$ is an $ABAB$-free linear order.

The authors show that $ABAB$-free hypergraphs are equivalent to 
hypergraphs with a 
\emph{stabbed pseudo-disk representation}, i.e.,
each $S\in\mathcal{S}$ is mapped to a closed and bounded region $D_S$ containing the origin whose boundary is a simple Jordan curve,
each $x\in X$ is mapped to a point $p_x$ in $\mathbb{R}^2$ s.t. $p_x\in D_S$ iff $x\in S$.
The regions $\mathcal{D}=\{D_S: S\in\mathcal{S}\}$
form a stabbed pseudodisk arrangement, i.e., the boundaries of any two of them are either disjoint or intersect exactly twice.
Let $(P,\mathcal{D})$ denote the embedding of the hypergraph where $P=\{p_x:x\in X\}$.

The authors show that to any stabbed pseudodisk arrangement $\mathcal{D}$ and a set $P$ of points,
we can add additional pseudodisks $\mathcal{D}'$ s.t. $(i)$ each $D'\in\mathcal{D}'$ contains exactly 2 points of $P$, 
$(ii)$ $\mathcal{D}\cup\mathcal{D}'$ is a pseudodisk arrangement, and
$(iii)$ Each $D\in\mathcal{D}$ s.t. $|D\cap P|\ge 3$ contains a pseudodisk $D'\in\mathcal{D}'$. The graph on $P$ whose edges are defined
by $\mathcal{D}'$ is called the \emph{delaunay graph} of the arrangement. 


Our result, namely Theorem \ref{thm:primalOuter} in Section \ref{sec:outerplanar} is stronger. A delaunay graph ensures that for each
pseudodisk $D\in\mathcal{D}$, the induced subgraph on the elements in $D$ is non-empty, while a support implies that the induced subgraph
of the support on the elements in $D$ is connected.
At the outset, it seems like the results of Ackerman et al. \cite{ackerman2020coloring},
especially the proof of Lemma $2.1$ can be used to prove Lemma \ref{lem:keylem}. However, there is a subtle difference between the two.
The authors show that there is a 2-element hyperedge, or equivalently a non-blocking diagonal can be added between two elements of a hyperedge, but in the vertex bypassing operation we require this diagonal to be between two disjoint runs of a hyperedge which is a more stringent condition. 

The authors show that for a stabbed pseudodisk arrangement, the delaunay graph as constructed above is outerplanar, and hence
is 3-colorable. This implies that $ABAB$-free hypergraphs, and thus hypergraphs induced by stabbed pseudodisks can be colored with
3 colors so that no hypergedge with $\ge 2$ elements is mono-chromatic. 
This result also follows directly from Theorem \ref{thm:primalOuter}. 

Ackerman et al., \cite{ackerman2020coloring} ask if we can 3 color the elements of the dual of an
$ABAB$-free hypergraph s.t. no hyperedge of the dual with at least two elements
is monochromatic. In Figure \ref{fig:dualDColor} we show that this is not true - even if the regions are defined by unit disks in the plane. 
The dual hypergraph contains four elements corresponding to the four unit disks, and six hyperedges corresponding to the six points. Each hyperedge defined by a point is a pair of disks containing that point. The intersection of the disks is non-empty, and therefore the hypergraph is $ABAB$-free.
Each point is of depth 2, and therefore the delaunay graph is $K_4$, which is not 3-colorable. However, by Theorem \ref{thm:outerplanardual}, it follows that if a hypergraph $(X,\mathcal{S})$ admits a representation 
as non-piercing subgraphs on a host outerplanar graph, then the hyperedges can be 3-colored so that no point
is monochromatic.

Consider the following natural extension of the result of Ackerman et al. \cite{ackerman2020coloring}: Given a collection of stabbed non-piercing regions in the plane, does there exist a coloring of the points with 3
colors s.t. no region is monochromatic?
We answer this question again in the negative by giving a counter-example (see Figure \ref{fig:primalNPColor}).
It is easy to check that in this case again, the delaunay graph is $K_4$ and therefore the arrangement is not $3$-colorable.
The reason why hypergraphs defined by stabbed pseudodisks is 3 colorable, but stabbed non-piercing regions is not is
the following: Let $\overrightarrow{H}$ denote the directed graph obtained from the dual arrangement graph
where each edge is directed from a cell to its adjacent cells of lower depth. If $(P,\mathcal{D})$ is
a stabbed pseudodisk arrangement, then we can show that every cell is reachable from $o$, where $o$ is the
cell in the intersection of all pseudodisks (marked by $\times$ in Figure \ref{fig:primalNPColor}). This is not true for example, in the graph $\overrightarrow{H}$
corresponding to the arrangement in Figure \ref{fig:primalNPColor}. In particular, the cell containing
$d$ is not reachable in $\overrightarrow{H}$ from the cell $o$ in the intersection of all the regions.

\begin{figure}[ht!]
\centering
\begin{subfigure}{0.4\textwidth}
 \includegraphics[scale=.6]{dualDisk}\vspace{1cm}
\caption{Dual: Every point $a,b,\ldots,f$ is contained in two disks.}
\label{fig:dualDColor}
\end{subfigure}\hspace{1cm}
\begin{subfigure}{0.5\textwidth}\vspace{0.5cm}
\includegraphics[scale=.70]{primalNPColor}\vspace{0.41cm}
\caption{Primal: Every region contains two points.}

\label{fig:primalNPColor}
 \end{subfigure}\vspace{0.5cm}
 \caption{Stabbed hypergraphs of disks (dual) and non-piercing regions (primal) requiring four colors.}
 \label{coloring}
 \end{figure}



\section{Conclusion}
\label{sec:conclusion}
In this paper, we studied the problem of construction of supports for systems $(G,\mathcal{H})$ defined on a host graph
$G$. We primarily studied two settings, namely when $G$ has bounded genus, and when $G$ has bounded treewidth. We showed that
if $G$ has bounded genus, then the cross-free property is sufficient to obtain a support of genus at most that of $G$.
If $G$ has bounded treewidth, we showed that the non-piercing condition on $\mathcal{H}$ is a sufficient condition
to obtain a support of bounded treewidth in the primal and dual settings. However, an exponential blow-up of the
treewidth of the support is sometimes necessary. Along the way, we also studied the setting of outerplanar graphs.
There are several intriguing open questions and research directions
and we mention a few: We do not know if the algorithm to construct a dual or intersection support 
in the bounded-genus case runs in polynomial time. Settling this will be a very interesting question. 
A broader line of research is to obtain necessary and sufficient conditions for a hypergraph to have a \emph{sparse support} - 
where sparsity could be a graph with sublinear-sized separators or even just a graph with a linear number of edges.

\bibliography{ref}

\end{document}
    