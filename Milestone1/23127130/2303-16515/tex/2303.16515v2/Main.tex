\documentclass[12pt]{article}
\usepackage[utf8]{inputenc}
\usepackage{url, color, epsfig, amsmath, amsthm, amssymb}
\usepackage{hyperref}
\usepackage{enumerate}
\usepackage{graphicx,wrapfig,lipsum}
\usepackage{svg}
\usepackage{thmtools}
\usepackage{subcaption}
\usepackage{float}
\usepackage{algorithm,algorithmic}
\usepackage[colorinlistoftodos]{todonotes}
\usepackage{apxproof}
\usepackage{thmtools}
\usepackage{thm-restate}
\usepackage{svg}

\usepackage{hyperref}
\usepackage{cleveref}

\newcommand{\hide}[1]{}
\newcommand{\D}{\mathcal{D}}
\newcommand{\A}{\mathcal{A}}
\newcommand{\AD}{\mathcal{A}(\mathcal{D})}
\newcommand{\C}{\mathcal{C}}
\newcommand{\R}{\mathbf{r}}
\newcommand{\B}{\mathbf{b}}
\newcommand{\LL}{\mathcal{L}}
\newcommand{\OO}{\mathcal{O}}
\newcommand{\loc}{\textsc{Local}}
\newcommand{\VB}{\textsc{VB}}
\newcommand{\opt}{\textsc{Opt}}
\newcommand{\T}{\mathcal{T}}
\newcommand{\p}{\mathcal{P}}
\newcommand{\eps}{\mathcal \epsilon}
\newcommand{\s}{\mathcal{S}}
\newcommand{\RR}{\mathbb{R}}
\newcommand{\depth}{\textsc{depth}}
\newcommand{\Spl}{\textrm{Spl}}
\newcommand{\cost}{\textrm{cost}}
\newcommand{\unique}{\textrm{unique}}
\newtheorem{theorem}{Theorem}
\newtheorem{corollary}[theorem]{Corollary}
\newtheorem{lemma}[theorem]{Lemma}
\newtheorem{claim}[theorem]{Claim}
\newtheorem{definition}[theorem]{Definition}
\newtheorem{observation}[theorem]{Observation}
\newtheorem{proposition}[theorem]{Proposition}
\newtheorem{remark}[theorem]{Remark}


\declaretheorem[name=Lemma,numberlike=lemma]{lem}


\newtheoremstyle{case}{}{}{}{}{}{:}{ }{}
\theoremstyle{case}
\newtheorem{case}{Case}

\DeclareMathOperator{\Ex}{E}
\bibliographystyle{plainurl}

\title{On Hypergraph Supports}
\author{Rajiv Raman\footnote{Part of this work was done when the first author was at LIMOS, Université Clermont Auvergne, and was partially supported by the French government research program “Investissements d’Avenir” through the IDEX-ISITE initiative 16-IDEX-0001 (CAP 20-25).} \\ IIIT-Delhi, India. \\ rajiv@iiitd.ac.in\and Karamjeet Singh \\ IIIT-Delhi, India. \\ karamjeets@iiitd.ac.in}

\setlength {\marginparwidth }{2cm}
\begin{document}
 
\maketitle              
\begin{abstract}
Let $(X,\mathcal{E})$ be a hypergraph. A support is a graph $Q$ on $X$ such that for each
$E\in\mathcal{E}$, the subgraph of $Q$ induced on the elements in $E$ is connected. 
In this paper, we consider hypergraphs
defined on a host graph.
Given a host graph $G=(V,E)$, with $c:V\to\{\R,\B\}$, and a collection
of connected subgraphs $\mathcal{H}$ of $G$, a primal support is a graph $Q$ on $\B(V)$
such that for each $H\in \mathcal{H}$, the induced subgraph $Q[\B(H)]$ on vertices $\B(H)=H\cap c^{-1}(\B)$
is connected. A \emph{dual support} is a graph $Q^*$ on $\mathcal{H}$ such that 
for each $v\in V$, the induced subgraph $Q^*[\mathcal{H}_v]$ is connected, where $\mathcal{H}_v=\{H\in\mathcal{H}: v\in H\}$.
Given two families $\mathcal{H}$ and $\mathcal{K}$ of connected subgraphs of $G$, an \emph{intersection support}
$\tilde{Q}$ is a graph on $\mathcal{H}$ such that for each $K\in\mathcal{K}$, $K_{H}=\{H\in\mathcal{H}: V(H)\cap V(K)\neq\emptyset\}$
induces a connected subgraph of $\tilde{Q}$.
We present sufficient conditions on the host graph and subgraphs so that
the resulting primal/dual/intersection support comes from a restricted family.

We primarily study two classes of graphs: $(1)$ If the host graph has genus $g$
and the subgraphs satisfy a topological condition of being \emph{cross-free}, 
then there is a primal and a dual support of genus at most $g$.
$(2)$ If the host graph has treewidth $t$ and the subgraphs satisfy a combinatorial condition of being \emph{non-piercing}, 
then there exist 
primal and dual supports of treewidth $O(2^t)$.
We show that this exponential blow-up is sometimes necessary. As an intermediate case, we also study the case when the host graph is outerplanar.
Finally, we show applications of our results to packing and covering, and coloring problems on geometric hypergraphs.

\end{abstract}

\section{Introduction}
A hypergraph $(X,\mathcal{E})$ is defined by a set $X$ of elements and a collection $\mathcal{E}$ of subsets of $X$. 
In this paper, we study the notion of a \emph{support} for a hypergraph.
A support is a graph $Q$ on $X$ such that $\forall\; E\in\mathcal{E}$, the subgraph induced by $E$ in $Q$,  $Q[E]$ is connected.
The notion of a support was introduced by Voloshina and Feinberg \cite{voloshina1984planarity} in the context of VLSI circuits.
Since then, this notion has found wide applicability in several areas, such as visualizing hypergraphs 
\cite{bereg2015colored,bereg2011red,brandes2010blocks,brandes2012path,buchin2011planar,havet2022overlaying,hurtado2018colored},
in the design of networks \cite{anceaume2006semantic,baldoni2007tera,baldoni2007efficient,chand2005semantic,hosoda2012approximability,korach2003clustering,onus2011minimum}, and 
similar notions have been used in the analysis of local search algorithms for geometric problems \cite{Mustafa17,BasuRoy2018,Cohen-AddadM15,krohn2014guarding,mustafa2010improved,RR18}.

Any hypergraph clearly has a support - A complete graph on $X$ is a support.
The problem becomes interesting if we introduce a global constraint on the graph that is in \emph{tension}
with the \emph{local} connectivity requirement for each hyperedge.
In particular, we are interested in the restrictions on the hypergraph that guarantees
the existence of a support from a \emph{sparse} family of graphs.

Two classes of sparse graphs that have been studied intensively are
%We can broadly classify sparse graphs into two categories - 
those that are \emph{easily decomposable},
i.e., graphs with sublinear sized balanced separators\footnote{A graph has a sublinear sized balanced separator if there are constants $\epsilon >0$, and $c>0$ and a set $S$ of size $O(|V|^{1-\epsilon})$
such that $G\setminus S$ contains two disconnected components $A$ and $B$ such that $|A|,|B|\le c|V|$.
}, and graphs that satisfy various notions of \emph{expansion}
\cite{hoory2006expander}. Examples of the former are planar graphs \cite{LT79}, graphs of bounded genus \cite{gilbert1984separator}, graphs excluding a minor \cite{alon1990separator}, and more generally, graphs with polynomially bounded $t$-shallow minors \cite{nevsetvril2012sparsity}. 
The fact that a family of graphs have sub-linear separators has been exploited to develop faster algorithms, or algorithms with better approximation factors than in general graphs. Some results that use this paradigm are \cite{AW13,DBLP:conf/focs/Arora97,Frederickson87,DBLP:journals/cpc/FoxP14,lipton1980applications}.
Similarly, there are examples of algorithms that exploit expansion for faster algorithms, or to obtain algorithms with
better approximation factors \cite{DBLP:journals/corr/abs-2307-06113,DBLP:conf/stoc/AroraKKSTV08}.
In a similar vein, one would like to develop a notion of sparsity of hypergraphs that can be exploited algorithmically.
The existence of a sparse support for a hypergraph is one such notion. 

To motivate the study of hypergraphs defined by subgraphs of a host graph, consider the following geometric setting:
Let $\mathcal{D}$ be a set of disks in the plane. Let $G$ denote the \emph{dual arrangement graph} of $\mathcal{D}$
that has a vertex for each cell in the arrangement of $\mathcal{D}$, and two cells are adjacent if they are separated
by an arc of a disk in $\mathcal{D}$. $G$ is a planar graph, and
each $D\in\mathcal{D}$ corresponds to a connected subgraph of $G$. 
Note that the subgraphs defined by $\mathcal{D}$ have a special structure - 
they are pairwise \emph{non-piercing}, i.e., for any pair of disks $D$ and $D'$
in $\mathcal{D}$, $D\setminus D'$ induces a connected subgraph of $G$. 
Now, let $P$ be a set of points in the plane and let $\mathcal{D}$ be a set of disks in the plane, and consider the
hypergraph $(P,\mathcal{D})$ where each $D\in\mathcal{D}$ defines a hyperedge consisting of the points contained in $D$.
If a cell in the
arrangement of $\mathcal{D}$ does not contain a point of $P$, we color its corresponding vertex in $G$ \emph{red}.
Otherwise, we color it \emph{blue}. Let $V$ denote the vertices of $G$ and let $c:V\to\mathcal\{\R,\B\}$ be the coloring
assigned. If there is a planar support $Q$ on $\B(V)$ (i.e., a planar graph on $\B(V)=\{v\in V: c(v)=\B\}$ such that  for each
disk $D\in \mathcal{D}$, the set of blue vertices contained in $D$ induces a connected subgraph of $Q$), then, $Q$
yields a support for the hypergraph $(P,\mathcal{D})$. To see this, in each cell containing vertices of $P$, we keep
one representative vertex, and remove the remaining vertices. In the support $Q$, each vertex corresponds to the
representative vertex in the corresponding cell. To obtain a support on $P$, we attach the removed vertices in the cell
to its representative vertex in $Q$. The resulting graph remains planar, and is the desired support. Figure \ref{fig:disksandpoints}
shows an example of a hypergraph $(P,\mathcal{D})$, the dual arrangement graph $G$, and a support graph on $G$.

\begin{figure}[ht!]

\centering
\begin{subfigure}{0.42\textwidth}
\centering
\includegraphics[width=\textwidth]{disks_and_points.pdf}
 % \vspace{0.2cm}
\caption{Disks and Points}
\label{fig:disksAndPoints}
\end{subfigure}
\hfill
\begin{subfigure}{0.42\textwidth}
\centering
\includegraphics[width=\textwidth]{dual_arrangement_graph.pdf}
% \vspace{0.2cm}
\caption{Dual arrangement graph}
\label{fig:DualArrangemenGraph}
 \end{subfigure}\\
 \vspace{1cm}
 % \hspace{1cm}
 \begin{subfigure}{0.27\textwidth}
 \centering
\includegraphics[width=\textwidth]{support1.pdf}
% \vspace{0.85cm}
\caption{Support on blue points for hypergraph in (b)}
\label{fig:SupportOnBlue}
 \end{subfigure}
 \hspace{4.5cm}
 % \hfill
 \begin{subfigure}{0.26
\textwidth}
 \centering
\includegraphics[width=\textwidth]{support2.pdf}
\vspace{-0.2cm}
\caption{Support for the hypergraph defined in (a)}
\label{fig:SupportForHypergraph}
 \end{subfigure}
 \caption{Support for hypergraph defined by disks and points in the plane.}\label{fig:disksandpoints}
 \end{figure}



Raman and Ray \cite{RR18} showed that in fact, we can obtain a planar support even when instead of disks, we consider
general \emph{non-piercing regions} $\mathcal{R}$ - regions bounded by simple Jordan curves in the plane
such that for any pair of regions $R, R'\in\mathcal{R}$, $R\setminus R'$ is a connected set, and obtain a planar support for the hypergraph
$(P,\mathcal{R})$.

At this point, it is still not clear why we need to define hypergraphs defined on a host graph, since the hypergraphs we have
considered are geometric, and therefore it is natural to work directly with the geometric regions in question to construct the desired graph. 
There are three reasons for this: the first is that 
modeling this problem as subgraphs on a host graph makes the problem combinatorial and cleaner rather than having to deal with messy topological
modifications inherent in earlier works. 
Second, it turns out that for regions on higher genus surfaces, the non-piercing condition is insufficient for the existence of a sparse support
and the condition we require for the regions to satisfy is more naturally defined as a condition on subgraphs of a host graph. 
Finally, the model of subgraphs on a host graph could potentially have applications outside the geometric setting that was the motivation for our work.

The problem of the existence of support graphs studied in the geometric setting itself came from the analysis of classical 
Packing and Covering problems for geometric intersection graphs and hypergraphs.
The existence of a support in combination with earlier work (see Aschner, et. al. \cite{DBLP:conf/walcom/AschnerKMY13}) implies PTASes for several geometric hypergraphs.
Thus far, most of the research has been restricted to geometric hypergraphs in the plane (some of the results extend to halfspaces in $\mathbb{R}^3$). But, nothing
is known for regions defined on surfaces of higher genus. Our work can be seen as a generalization of the work of Raman and Ray \cite{RR18} to surfaces of higher genus.
In the plane, the non-piercing condition is sufficient for the existence of a planar support. For surfaces of higher genus however, this is not always true. It turns
out however, that if we require the subgraphs on the host graph to satisfy a \emph{cross-free condition} (which generalizes non-piercing condition in the plane), 
then we can show the existence of a support of bounded genus.
This result combined with the fact that graphs of bounded genus have sublinear sized separators \cite{alon1990separator} implies a PTAS for several packing and covering
problems on oriented surfaces of bounded genus.


\section{Related Work}
The notion of a planar analogue of a hypergraph was first suggested by Zykov \cite{zykov}, who defined a hypergraph to be planar if there is a plane graph on the elements of the hypergraph such that for each hyperedge, there
is a bounded face of the embedding containing only the elements of this hyperedge. Equivalently, a hypergraph is \emph{Zykov-planar} iff its incidence bipartite graph is planar. 
Voloshina and Feinberg \cite{voloshina1984planarity} introduced the notion of hypergraph planarity that is now called \emph{a planar support}
 in the context of planarizing VLSI circuits (see  the monograph by Feinberg et al., \cite{feinberg2012vlsi} and
references therein). 
Johnson and Pollak \cite{Johnson1987HypergraphPA} showed that the problem of deciding if a hypergraph has a planar support is NP-hard.

Since then, several authors have studied the question of deciding if a hypergraph admits a support from a restricted family of graphs.
Tarjan and Yannakakis \cite{tarjan1984simple} showed that we can decide in linear time if a hypergraph admits a tree support.
Buchin et. al., \cite{buchin2011planar} showed linear time algorithms to decide if a hypergraph admits a support that is a path or a cycle, 
and a polynomial time algorithm to decide if a hypergraph admits a support that is a tree with bounded degree.
Further, the authors sharpen the result of Johnson and Pollak \cite{Johnson1987HypergraphPA} by showing that deciding if a hypergraph admits a support that is a 2-outerplanar\footnote{A graph is 2-outerplanar if the graph can be embedded in the plane such that the vertices are on two concentric circles and removing all vertices of outer face results in an outerplanar graph.} graph, is NP-hard.
The notion of constructing a support with the fewest number of edges, or with minimum maximum degree have also been studied in \cite{baldoni2007tera,onus2011minimum}.

Another motivation for studying the existence of sparse supports, and the main motivation for our work comes from
the design and analysis of algorithms for packing and covering problems on geometric hypergraphs. 
Chan and Har-Peled \cite{ChanH12} gave a PTAS for the Independent
Set problem for an arrangement of pseudodisks in the plane. 
Mustafa and Ray \cite{mustafa2010improved} gave a PTAS for the Hitting Set problem for a set of points and pseudodisks in the plane
(both results apply more generally to k-admissible regions\footnote{A set of connected bounded regions $\mathcal{R}$ in the plane, each of whose boundary is a simple Jordan curve is $k$-admissible (for even $k$) if for any $R,R'\in\mathcal{R}$, $R\setminus R'$ and $R'\setminus R$ are connected, their boundaries are in general position, and 
intersect each other at most $k$ times. If $k=2$, the regions are called pseudodisks.}). 
For both the problems above, the algorithmic technique yielding a PTAS was a simple \emph{local search} algorithm.
The analysis of the local search algorithm in both cases was also similar, i.e., showing the existence of a planar graph
satisfying a \emph{locality property}. The locality captures the fact the solution returned is locally optimal, and
the existence of a sublinear separator for planar graphs implied that the algorithm
was a PTAS. The results of \cite{ChanH12,mustafa2010improved} were extended by Basu-Roy et. al., \cite{BasuRoy2018} who 
used the same paradigm to design a PTASes for the Set Cover, and Dominating Set problem.

While the definition of the locality property is problem specific, the existence of a planar support for an appropriate
hypergraph derived from the problem implies the desired locality property for all the problems above.
Raman and Ray \cite{RR18} showed the existence of a planar support for \emph{intersection hypergraphs of non-piercing regions}\footnote{This includes disks, pseudodisks, unit height-axis parallel rectangles, halfspaces, etc.}
(see Section \ref{sec:applications} for the definition), and this yields a unified analysis for the packing and covering problems described
above. In fact, the existence of a support for the \emph{intersection hypergraph} implies both the primal and dual supports (we leave the precise definitions to Section
\ref{sec:preliminaries}).

While the existence of a planar support is strictly more general than the locality property required for the problems
described above, it turns out to be essential if we consider the \emph{demand} or \emph{capcitated} version of the problems.
For example, if the input consists of a set of pseudodisks $\mathcal{D}$, each with a capacity $c_D$, $D\in\mathcal{D}$ 
bounded by a constant, and a set $P$ of points then the \emph{Point Packing problem} asks for a maximum number of points of $P$
that can be chosen so that for any pseudodisk $D\in\mathcal{D}$, no more than $c_D$ points are chosen.
We can extend the paradigm above to obtain a PTAS for this problem. The appropriate locality graph
turns out to be a non-planar graph, that is however built from a support graph for the \emph{dual hypergraph} (See \cite{BasuRoy2018, DBLP:journals/dcg/RamanR22}).

In \cite{DBLP:journals/dcg/RamanR22}, it was shown that 
LP-rounding can be combined with a local search algorithm to obtain 
$(2+\epsilon)$-approximations for the problem of Hitting Set and Set Cover with capacities for hypergraphs obtained from
points and pseudodisks. 
Besides packing and covering problems, the existence of a planar support implies that the hypergraph is 4-colorable, a question considered by Keller and Smorodinsky \cite{KellerS18} and Keszegh\cite{Keszegh20}. Further hypergraph coloring questions were studied by P{\'{a}}lv{\"{o}}lgyi and Keszegh \cite{DBLP:journals/jocg/KeszeghP19} and
Ackerman et al., \cite{ackerman2020coloring}. Ackerman et. al., studied $ABAB$-free hypergraphs
in \cite{ackerman2020coloring} and showed that these hypergraphs admit a 
representation with stabbed pseudodisks and points in the plane. We discuss the connection with our results
in Section \ref{sec:applications}.

From the discussion above, it follows that showing the existence of a support from a family of graphs with sublinear separators
is useful in the analysis of algorithmic and combinatorial questions involving hypergraphs. The problems described above
deal with hypergraphs defined by geometric objects in the plane. 
There have been some
works that use the local search paradigm above for problems that are not defined in the plane. 
Cabello and Gajser \cite{CabelloG14} showed that Independent Set and Vertex Cover admit a PTAS on graphs excluding a fixed minor, and
Aschner et al.,\cite{DBLP:conf/walcom/AschnerKMY13} study some packing and covering problems involving geometric non-planar graphs. In both these cases however, the authors do not need to show the existence of a support. 
A weaker notion is sufficient, and this follows in a straightforward manner.

Our results generalize the results of Raman and Ray \cite{RR18} from the plane to higher genus surfaces. We identify that a cross-free condition 
is sufficient for the existence of a support of bounded genus. We also show that for bounded-treewidth graphs, if the subgraphs are non-piercing
there is a support of bounded treewidth.

\section{Preliminaries}
\label{sec:preliminaries}

A graph $G=(V,E)$ and a collection of subgraphs $\mathcal{H}$ of $G$ naturally defines a hypergraph 
$(V(G),\{V(H): H\in\mathcal{H}\})$, where $V(H) = \{v\in V: v\in H\}$. We call the tuple $(G,\mathcal{H})$ a \emph{graph system},
since we will use the pair to define other hypergraphs. We implicitly make the assumption that $\mathcal{H}$ is a collection of
connected subgraphs of $G$, since this is the case we consider throughout the paper.

We primarily study two classes of graphs: graphs that are 2-cell embedded (See Section \ref{sec:basicgraphs} for a definition)
in an orientable surface of genus $g$, henceforth called \emph{embedded graphs}. A graph system $(G,\mathcal{H})$ along with an embedding
of $G$ in an oriented surface is called an \emph{embedded graph system}.
 We allow multi-edges and self-loops.
The second class of graphs we study are graphs of bounded treewidth. 
As an intermediate case, we also study the setting where $G$ is an outerplanar graph. The definitions of these terms can be found 
in Subsection \ref{sec:basicgraphs}.

For a vertex $v\in V(G)$, we use $N_G(v)$ to denote the neighbours of $v$ in $G$ 
(or just $N(v)$ if $G$ is clear from context), and we use $N_G[v]$ (or $N[v]$) to denote $N(v)\cup\{v\}$. 
We use $e\sim v$ to denote an edge $e$ incident to vertex $v$. For $S \subseteq V$, we use $G[S]$ to denote the subgraph of $G$ induced on $S$.
We use $\mathcal{H}_v = \{H\in\mathcal{H}: v\in H\}$.
Similarly, let $\mathcal{H}_e=\{H\in\mathcal{H}: e\in H\}$.
We let $\depth(v)=|\mathcal{H}_v|$. Similarly, let $\depth(e)=|\{H\in \mathcal{H}: e\in H|$.

Let $c:V\to\{\R,\B\}$ be an arbitrary coloring of the vertices of $G$ with two colors. 
Let $\B(V)$ and $\R(V)$ denote respectively $c^{-1}(\B)$ and $c^{-1}(\R)$. We also refer to the vertices in $\B(V)$ as \emph{blue vertices} and the vertices in $\R(V)$ as \emph{red vertices}.

A \emph{primal support} for $(G,\mathcal{H})$ is a graph $Q$ on $\B(V)$ such that $\forall\; H\in\mathcal{H}$, $Q[\B(H)]$ is connected\footnote{Note that we cannot simply project each $H$ on $\B(V)$ as the resulting subgraphs may not be connected in $G.$}. 
The system $(G,\mathcal{H})$ also defines a \emph{dual hypergraph} $(\mathcal{H}, \{\mathcal{H}_v\}_{v\in V(G)})$. A \emph{dual support} is a graph $Q^*$ on $\mathcal{H}$ such that $\forall\; v\in V(G)$, $Q^*[\mathcal{H}_v]$
is connected\footnote{ 
To make the definition symmetric, we could have considered a coloring $c:\mathcal{H}\to\{\R,\B\}$, and 
required that only $Q^*[\mathcal{H}^{\B_v}]$ be connected for each $v\in V$, where $\mathcal{H}^{\B_v} = \{H\in\mathcal{H}: v\in H \mbox{ and } c(H) =\B\}$. However, this problem reduces to constructing a dual support restricted to the hypergraphs $\mathcal{H}^{\B} = \{H\in\mathcal{H}: c(H)=\B\}$. Therefore, in the dual setting, it is sufficient to study the uncolored version of the problem.}.

Let $\mathcal{H}$ and $\mathcal{K}$ be two sets of connected subgraphs of a graph $G$, and let $(G,\mathcal{H},\mathcal{K})$ denote the \emph{intersection hypergraph}
$(\mathcal{H},\{\mathcal{H}_K\}_{K\in\mathcal{K}})$, where 
$\mathcal{H}_K = \{H\in\mathcal{H}: V(K)\cap V(H) \neq\emptyset\}$. Throughout the paper, we use the notation $H\cap K\neq\emptyset$ to mean $V(K)\cap V(H)\neq\emptyset$.
A support for $(G,\mathcal{H},\mathcal{K})$ is then a graph $\tilde{Q}=(\mathcal{H}, F)$ such that 
$\forall K\in\mathcal{K}$, $\tilde{Q}[\mathcal{H}_{K}]$ is connected. $\tilde{Q}$ is called the \emph{intersection support}.
The notion of an intersection hypergraph generalizes both the primal and dual hypergraphs defined above
- taking the blue vertices of $G$ as the singleton sets of $\mathcal{H}$ for the primal, 
and taking the vertices of $G$ as the singleton sets of $\mathcal{K}$ for the dual, respectively. Figure \ref{primaldualint} shows examples of
primal, dual and intersection supports.

%\clearpage
\begin{figure}[htp!]
\centering
\begin{subfigure}{0.43\textwidth}
\centering
\includegraphics[width=\textwidth]{primal.pdf}
 % \vspace{0.2cm}
 \caption{Primal hypergraph}
% \caption{Primal hypergraph with $H_1=\{a,b,c,d\}$, $H_2=\{c,d,e\}$, $H_3=\{a,b,f,e\}$, $H_4=\{a,,c,e\}$ as hyperedges.}
% \label{}
\end{subfigure}
% \hspace{1cm}
\hfill
\begin{subfigure}{0.32\textwidth}
\centering
\includegraphics[width=\textwidth]{primal_support.pdf}
% \vspace{0.2cm}
\caption{Primal support}
% \label{}
 \end{subfigure}
 % \vspace{1cm}
 \begin{subfigure}{0.43\textwidth}
 \centering
\includegraphics[width=\textwidth]{dual.pdf}
% \vspace{0.85cm}
\caption{Dual hypergraph}
% \caption{Dual hypergraph with $H_1=\{a,b,c,d\}$, $H_2=\{c,d,e\}$, $H_3=\{a,b,f,e\}$, $H_4=\{a,,c,e\}$ as hyperedges.}
% \label{}
 \end{subfigure}
 % \hspace{1cm}
 \hfill
 \begin{subfigure}{0.29\textwidth}
 \centering
\includegraphics[width=\textwidth]{dual_support.pdf}
\caption{Dual support}
% \label{}
 \end{subfigure}

 % \vspace{1cm}
 \begin{subfigure}{0.43\textwidth}
 \centering
\includegraphics[width=\textwidth]{intersection.pdf}
% \vspace{0.85cm}
\caption{Intersection hypergraph}
% \caption{Intersection hypergraph with $H_1=\{a,b,c\}$, $H_2=\{a,b,e\}$, $H_3=\{c,e,f\}$, $H_4=\{b,e,d\}$ and $K_1=\{c,d\}$, $K_2=\{b,f\}$, $H_3=\{e,d\}$ as hyperedges. as hyperedges.}
% \label{}
 \end{subfigure}
 % \hspace{1cm}
 \hfill
 \begin{subfigure}{0.29\textwidth}
 \centering
\includegraphics[width=\textwidth]{intersection_supprt.pdf}
\caption{Intersection support}
% \label{}
 \end{subfigure} 
 \caption{(a) and (c): Primal and Dual hypergraphs both having hyperedges $H_1=\{a,b,c,d\}$, $H_2=\{c,d,e\}$, $H_3=\{a,b,f,e\}$, $H_4=\{a,b,c,e\}$; (e): Intersection hypergraph with hyperedges $H_1=\{a,b,c\}$, $H_2=\{a,b,e\}$, $H_3=\{c,e,f\}$, $H_4=\{b,e,d\}$, and $K_1=\{c,d\}$, $K_2=\{b,f\}$, $K_3=\{e,d\}$.}
 
 % (b),(d) and(f): Supports for primal, dual and intersection hypergraphs respectively}
 \label{primaldualint}
 \end{figure}


Our goal is to consider restrictions of hypergraphs so that the support is guaranteed to be from a restricted family of graphs. 
To that end, we introduce a notion of cross-free hypergraphs and non-piercing hypergraphs.

\begin{definition}[Reduced graph]
Let $(G,\mathcal{H})$ be a graph system. 
For two subgraphs $H,H'\in\mathcal{H}$, the \emph{reduced graph} $R_G(H,H')$ (or just $R(H, H')$ if $G$ is clear from context) is the graph obtained by contracting all edges, both of whose end-points are in $H\cap H'$. 
We ignore self-loops formed during contraction, but keep the multi-edges formed in the process.
\end{definition}

Note that if $G$ is embedded in a surface $\Sigma$, then this induces an embedding of $R_G(H,H')$ in $\Sigma$.

\begin{definition}[Cross-free at $v$]
Let $(G,\mathcal{H})$ be an embedded graph system. 
Two subgraphs $H,H'\in\mathcal{H}$ are said to be cross-free at $v\in V(G)$
if the following holds: Consider the induced embedding of the reduced graph $R(H,H')$.
Let $\tilde{v}$ be the image of $v$ in $R(H,H')$. There are no 4 edges 
$e_i=\{\tilde{v},v_i\}$ in $R(H,H')$, $i=1,\ldots, 4$ incident to $\tilde{v}$ in cyclic order around $\tilde{v}$,
such that $v_1, v_3\in H\setminus H'$, and $v_2, v_4\in H'\setminus H$.
% See Figure \ref{}.
\end{definition}

For an embedded graph system $(G,\mathcal{H})$, if every pair $H, H'\in\mathcal{H}_v$ is cross-free at $v$, then
$(G,\mathcal{H})$ is said to be cross-free at $v$, and 
$(G,\mathcal{H})$ is cross-free if it is cross-free at all $v\in V(G)$.
If there exist $H, H'\in \mathcal{H}$ such that $H$ and $H'$ are not cross-free at $v$, we say that $H$ and $H'$ are \emph{crossing} at $v$.
Finally, a graph system $(G,\mathcal{H})$ is cross-free if there exists an embedding of $G$ 
such that the embedded graph system $(G,\mathcal{H})$ is cross-free with respect to $\mathcal{H}$. 
There exist graph systems $(G,\mathcal{H})$ where $G$ has genus $g$, but the graph system may not have a cross-free embedding on any surface of genus $g'\ge g$.
Throughout the paper, 
when we say that $G$ is embedded in the plane, we assume without loss of generality that the embedding is cross-free with respect to $\mathcal{H}$.

%Note that for cross-free system, we implicitly assume $\mathcal{H}$ is a collection of connected subgraphs.

An intersection hypergraph $(G,\mathcal{H},\mathcal{K})$ is cross-free
if there is an embedding of $G$ such that the embedded graph systems $(G,\mathcal{H})$ and $(G,\mathcal{K})$ are 
simultaneously cross-free.
Note that we can have $H\in\mathcal{H}$, $K\in\mathcal{K}$ that are crossing.
Finally, we use the term $(G,\mathcal{H})$ is a cross-free system of genus $g$ or the term
$(G,\mathcal{H},\mathcal{K})$ is a cross-free intersection system of genus $g$ to mean that the host
graph $G$ has genus $g$.


\begin{definition}[Non-piercing]
\label{defn:nonpiercing}
A graph system $(G,\mathcal{H})$ with $\mathcal{H}$ a collection of subgraphs of $G$
is non-piercing if each $H\in\mathcal{H}$ is connected and for any two 
subgraphs $H, H'\in\mathcal{H}$, $H\setminus H'$ induces a connected
subgraph of $H$.
\end{definition}

We say $(G,\mathcal{H})$ is a  non-piercing system of treewidth $t$ to mean that it is a non-piercing system
defined on a host graph $G$ whose treewidth is $t$.


Note that non-piercing is a purely combinatorial notion, 
and unlike the cross-free property above, it does not require an embedding of the graph. 
If $\exists\; H, H'\in\mathcal{H}$ such that either the induced subgraph $H\setminus H'$ of $H$ or the induced subgraph $H'\setminus H$ of $H'$ is not connected, then we say that $H$ and $H'$ are \emph{piercing}.
It should also be observed that if $\mathcal{H}$ is a collection of non-piercing subgraphs of $G$, then after contracting any edge in $G$, the resulting subgraphs in $\mathcal{H}$ remain non-piercing.

\subsection{Graph classes}
\label{sec:basicgraphs}

We now briefly describe the classes of graphs we study in this paper, and their properties.

\smallskip
\noindent
{\bf Bounded genus graphs:}
\begin{definition}[Embedding of a graph]
A graph $G$ is said to be \emph{embedded} in a surface $\Sigma$ if the vertices of $G$ are distinct points on $\Sigma$ and each edge of $G$ is a simple arc lying in $\Sigma$ whose endpoints are the vertices of the edge such that its interior is disjoint from other edges and vertices.
A 2-cell embedding is an embedding of a graph on a surface, where each face is homeomorphic to a disk in the plane.
\end{definition}

We say that a graph $G$ has an \emph{embedding} in a surface $\Sigma$ if there is a graph $G'$ embedded in $\Sigma$ such that $G'$ is isomorphic to $G$.
An orientable surface has genus $g$ if it is obtained from a sphere by adding $g$ \emph{handles} (See \cite{Mohar2001GraphsOS}, Chapter 3).

\begin{definition}[Genus] The \emph{genus} $g$ of a graph $G$ is the minimum genus of an oriented surface $\Sigma$ so that $G$ has an embedding in $\Sigma$.

\end{definition}
We say that a graph has bounded genus if it can be embedded in a surface whose genus is bounded.
It should be noted that contracting any edge of a graph does not increase the genus of the resulting graph and we will use this fact subsequently throughout the paper.


\smallskip
\noindent
% {\bf Bounded genus graphs:} 
%add defintions of genus of a surface, genus of a graph, embedding of graphs on surfaces - use defintions and cite Mohar-Thomassen.

\smallskip
\noindent
{\bf Bounded treewidth graphs:} 
Given a graph $G=(V,E)$, a tree-decomposition of $G$ is a pair $(T,\mathcal{B})$, 
where $T$ is a tree and $\mathcal{B}$ is a collection of \emph{bags} - subgraphs of $G$ indexed by the nodes of $T$\footnote{Throughout the paper, we use the term \emph{node} to refer
to the elements of $V(T)$ and \emph{vertices} to refer to the elements of $V(G)$.}, that satisfies the following properties:
$(i)$ For each edge, $\{u,v\}\in E(G)$ there is a bag $B\in\mathcal{B}$ that contains both $u$ and $v$. $(ii)$ For each vertex $v\in v(G)$, the set of bags containing $v$ induce
a connected sub-tree of $T$. 

The \emph{width} of a tree-decomposition is defined as $\max_{x\in V(T)} |B_x|-1$. The treewidth of $G$ is the minimum width of a tree-decomposition of $G$, denoted by $tw(G)$. A graph $G$ has bounded treewidth if $tw(G)$ is bounded.
See \cite{bodlaender1998partial} or Chapter 12 in \cite{diestel2005graph} for additional properties of a tree-decomposition. 

\smallskip
\noindent
{\bf Outerplanar graph:} 
An outerplanar graph is a graph that can be embedded in the plane such that all vertices lie on the outer face. It is a well-known
fact that an outerplanar graph has treewidth at most 2 and that an embedding of an outerplanar graph can be obtained in polynomial time.
If $(G,\mathcal{H})$ is a (cross-free/non-piercing) system where $G$ is an outerplanar graph, 
we call it an \emph{outerplanar (cross-free/non-piercing)} system.

When constructing the dual support, we can assume without loss of generality that there are no \emph{containments} (as shown in Proposition \ref{prop:nocontainment}), 
i.e., $\forall\; H,H'\in\mathcal{H},\; H\setminus H'\neq\emptyset$, and we say
that a graph system $(G,\mathcal{H})$ has no containments if this property holds. 

\begin{proposition}
\label{prop:nocontainment}
Let $(G,\mathcal{H})$ be a graph system. Let $\mathcal{H}'\subseteq\mathcal{H}$ be maximal such that 
$\forall\; H, H'\in\mathcal{H}'$, $H\setminus H'\neq\emptyset$. Let $Q'$ be a dual support for 
$(G,\mathcal{H}')$. Then, there is a dual support $Q^*$ for $(G,\mathcal{H})$ such that 
\begin{enumerate}
\item If $Q'$ has genus $g$, then $Q^*$ has genus $g$.
\item If $Q'$ has treewidth $t$, then $Q^*$ has treewidth $t$, and
\item If $Q'$ is outerplanar, then $Q^*$ is outerplanar.
\end{enumerate}
\end{proposition}
\begin{proof} 
Consider the containment order $\mathcal{P}=(\mathcal{H},\preceq)$ where for $H,H'\in\mathcal{H}$, $H\preceq H'$ iff $V(H)\subseteq V(H')$. 
We prove by induction on $(\ell(\mathcal{P}),c(\mathcal{P}))$, where $\ell(\mathcal{P})$ is the 
maximum length of a chain in $\mathcal{P}$, and $c(\mathcal{P})$ is the number of chains of length $\ell(\mathcal{P})$.

If $\ell(\mathcal{P})=1$, then the elements in $\mathcal{H}$ are pairwise incomparable.
Therefore, $Q'=Q^*$ is a dual support for $(G,\mathcal{H})$. Hence, all three conditions trivially follow.

Suppose for any $(G'',\mathcal{H}'')$, and containment order $\mathcal{P}''=(\mathcal{H}'',\preceq)$, such that 
$(\ell(\mathcal{P}''),c(\mathcal{P}''))$ is lexicographically smaller than the corresponding pair for the  containment
order $\mathcal{P} = (\mathcal{H},\preceq)$ on $(G,\mathcal{H})$, the statements of the proposition holds.

Consider a longest chain $C$ in $\mathcal{P}$, and let $H$ be the minimum element in $C$.
Let $\mathcal{H}'=\mathcal{H}\setminus\{H\}$. Then, for containment order $\mathcal{P}'$ on $\mathcal{H}'$,
$(\ell(\mathcal{P}'),c(\mathcal{P}'))$ is lexicographically smaller than $(\ell(\mathcal{P}),c(\mathcal{P}))$.
Therefore, by the inductive hypothesis, the statement of the proposition holds for $(G,\mathcal{H}')$, and let
$Q'$ denote its dual support. To obtain a dual support $Q^*$ for $(G,\mathcal{H})$, we add a vertex corresponding 
to $H$. Let $H'$ be an immediate successor of $H$ in $\mathcal{P}$ arbitrarily chosen. Connect $H$ to $H'$.

Since we added a new vertex of degree 1, it follows that if $Q'$ has genus $g$, then $Q^*$ has genus $g$.
Similarly, if $Q'$ has treewidth $t$, then so does $Q^*$, and finally if $Q'$ is outerplanar, so is $Q^*$.

To show that $Q^*$ is the desired support, let $v$ be a vertex in $G$. If $v\not\in V(H)$, then the fact that
subgraphs $\mathcal{H}_v$ correspond to a connected subgraph in $Q^*$ follows from the fact that $Q'$ is a dual
support for $(G,\mathcal{H}')$. So, consider a vertex $v\in V(H)$. Since $Q'$ is a dual support
for $(G,\mathcal{H}')$, the subgraphs $\mathcal{H}'_v$ induce a connected subgraph in $Q'$.
Since $H'$ is an immediate successor of $H$ in $\mathcal{P}$, $V(H)\subseteq V(H')$, i.e., $\mathcal{H}_v=\mathcal{H}'_v\cup\{H\}$. 
Since $H$ is adjacent to $H'$ in $Q^*$ it follows that the subgraphs $\mathcal{H}_v$ induce a connected subgraph in $Q^*$ Hence, $Q^*$ is the desired dual support for $(G,\mathcal{H)}$.


\end{proof}

\section{Contribution}
The existence of a \emph{local search graph} from a family of graphs with sublinear sized 
balanced separators
has been at the heart of the analysis of PTASes for several packing and covering problems with geometric hypergraphs
starting with the work of Chan and Har-Peled \cite{ChanH12}, and Mustafa and Ray \cite{mustafa2010improved}. The
algorithmic framework for all these problems is a simple \emph{local search} algorithm. The analyses for the algorithms
follow the same general framework - Let $R$ and $B$ be respectively, an optimal solution and a solution returned by the
local search algorithm. Show that there exists
a graph $G$ on $R\cup B$ that \emph{satisfies the local search condition}, and has sublinear sized balanced separators.
The main challenge in the analyses of this class of algorithms is therefore, showing the existence of
such a graph $G$. This had been done separately for each problem \cite{ChanH12, mustafa2010improved,BasuRoy2018}. However,
Raman and Ray in \cite{RR18} showed the existence of a planar support for the intersection hypergraph of non-piercing regions in the plane. 
The existence of such a support implies immediately, the existence of a graph satisfying the local search conditions required for
the packing and covering problems considered, and since planar graphs have sublinear sized balanced separators \cite{LT79}, this 
yields a unified PTAS for these problems.

While the result of Raman and Ray \cite{RR18} yields a general construction of a support graph, the result is restricted
to certain geometric hypergraphs in the plane. Our goal is to go beyond the geometric and planar setting and consider
hypergraphs defined on higher genus surfaces, and restricted hypergraphs that do not arise in geometric settings. 
To that end, we consider hypergraphs defined on a host graph. As stated in the introduction, this generalizes the planar setting.

We study two settings: When the host graph has bounded genus, 
and when the host graph has bounded treewidth.
While the results for bounded genus graphs roughly follow the proof outline of \cite{RR18}, 
several new ideas are required for the proofs to go through. In particular, it turns out that
for graphs of bounded genus, the non-piercing condition is insufficient for the existence of sparse supports. 
We introduce the notion of \emph{cross-free} subgraphs and show that if $(G,\mathcal{H})$ is a cross-free
system and $G$ has genus $g$, then there exist primal and dual supports of genus at most $g$.
Further, if $(G,\mathcal{H},\mathcal{K})$ is a cross-free system of genus $g$, then there exists an intersection support of genus at most $g$. 
Dealing with the cross-free condition on graphs is more challenging than in the geometric case. 
In particular, we show:
\begin{enumerate}
    \item Let $(G,\mathcal{H})$ be a cross-free system of genus $g$. Then,
    \begin{enumerate}
        \item For any 2-coloring $c:V\to\{\R,\B\}$ of the vertices $V$ of $G$, there exists a primal support $Q$ on $\B(V)$ 
        of genus at most $g$, i.e., for each $H\in\mathcal{H}$, the subgraph $Q[\B(H)]$ is connected, where
    $Q[\B(H)]$ is the subgraph of $Q$ induced by the vertices in $H$ colored $\B$.
    \item There exists a dual support $Q^*$ on $\mathcal{H}$ of genus at most $g$, i.e., for each $v\in V$, $Q^*[\mathcal{H}_v]$
    is connected, where $\mathcal{H}_v=\{H\in\mathcal{H}:v\ni H\}$.
    \end{enumerate}  

    \item If $(G,\mathcal{H},\mathcal{K})$ is a cross-free intersection system of genus $g$, there exists an intersection support $\tilde{Q}$ on $\mathcal{H}$ 
    of genus at most $g$, i.e., for each $K\in\mathcal{K}$, the induced subgraph $\tilde{Q}[\mathcal{H}_{K}]$, where 
    $\mathcal{H}_{K}=\{H\in\mathcal{H}: H\cap K\ne\emptyset\}$, is connected.
\end{enumerate}


Next, we study outerplanar graphs. Here, we show that there is a subtle difference between the primal and
dual settings:
\begin{enumerate}
    \item Let $(G,\mathcal{H})$ be a cross-free outerplanar system. For any coloring $c:V\to\{\R,\B\}$, 
    there exists an outerplanar primal support $Q$.
    \item Let $(G,\mathcal{H})$ be a non-piercing outerplanar system. 
    Then, there exists an outerplanar dual support $Q^*$.
    In this case, we show that the cross-free condition is insufficient.
\end{enumerate}


Finally, we consider the case where the host graphs have bounded treewidth. Here, we show the following:
\begin{enumerate}
    \item Let $(G,\mathcal{H})$ be a non-piercing system of treewidth $t$. Then,
    \begin{enumerate}
        \item For any coloring $c:V\to\{\R,\B\}$, of the vertices $V$ of $G$, there is a primal support $Q$ on $\B(V)$ of
        treewidth at most $3.2^t$. 
        \item There is a dual support $Q^*$ of treewidth at most $4.2^t$.
        \item There exist non-piercing systems of treewidth $t$ such that any primal or dual support has treewidth $\Omega(2^t)$.
    \end{enumerate}
\end{enumerate} 

As a consequence of the existence of supports, combined with earlier results we obtain the following results for packing and covering:
\begin{enumerate}
    \item Given a collection of regions on in oriented surface $\Sigma$ and a set of points such that the dual arrangement graph of the regions is cross-free. Then, a class of simple
    local search algorithms yield a PTAS for Hitting set, Set Cover, Dominating Set, Set Packing, and Point packing with demands or capacities bounded above by a constant.
    \item Given a collection of connected sub-graphs $\mathcal{H}$ of a bounded treewidth graph such that the sub-graphs are non-piercing, then a class of simple local search
    algorithms yield a PTAS for the Hitting Set, Set Cover, Set Packing, and Point Packing problems with demands or capacities bounded above by a constant.
\end{enumerate}

We obtain the following results for hypergraph coloring:
\begin{enumerate}
    \item Given a collection of regions on an oriented surface $\Sigma$ of genus $g$ and a set of points, such that the dual arrangement graph of the regions is cross-free, there is a coloring
    of the points with $c_g$ colors such that each region that contains more than one point, contains points of at least 2 colors.
    Similarly, there is a coloring of the regions with $c_g$ colors such that every point that is covered by more than one region, is covered by regions of at least 2 colors. Here, $c_g=(\sqrt{48g+1}+7)/2$ is an upper bound on the chromatic number of a graph of genus $g$ \cite{Mohar2001GraphsOS}.
    \item Given a graph $G=(V,E)$ of treewidth $t$ and non-piercing subgraphs $\mathcal{H}$ of $G$, there is a coloring of $V$ with $t+1$ colors such that each subgraph containing more than one vertex, contains vertices of
    at least 2 colors. Similarly, the subgraphs can be colored with at most $t+1$ colors such that each vertex $v\in V$ contained in at least 2 subgraphs, is contained in subgraphs of two
    different colors.
\end{enumerate}
    

The rest of the paper is organized as follows. In Section \ref{sec:nonpiercingimpliescrossfree}, we contrast non-piercing graph
systems with cross-free graph systems. In Sections \ref{sec:boundedgenus} and \ref{sec:constructiongenus}, we present our results for cross-free systems of bounded genus.
In Section \ref{sec:outerplanar} we present our results for outerplanar host graphs.
In Section \ref{sec:treewidth}, we present results for non-piercing systems of bounded treewidth. We describe some applications
in Section \ref{sec:applications}, and conclude in Section \ref{sec:conclusion} with open questions.


\section{Non-piercing and Cross-free Systems}
\label{sec:nonpiercingimpliescrossfree}
The non-piercing condition implies the cross-free condition in the plane, but they are incomparable in higher genus surfaces.
We start with the following result that shows that if a system is non-piercing in the plane, it is cross-free.

\begin{theorem}
\label{thm:planenp}
Let $(G,\mathcal{H})$ be a planar non-piercing system, 
then, $(G,\mathcal{H})$ is cross-free.
\end{theorem}

\begin{proof}
We show that if $(G,\mathcal{H})$ is crossing, then it cannot be non-piercing.
Consider an embedding of $G$ in the plane. Abusing notation, let $G$ also denote the embedding
of $G$ in the plane.
If $(G,\mathcal{H})$ is crossing, there are two subgraphs 
$H, H'\in\mathcal{H}$ and a vertex $x$ in $R_{G}(H, H')$ that lies in $H\cap H'$ and four neighbors
$x_1, \ldots, x_4$ in cyclic order around $x$ such that $x_1, x_3\in H\setminus H'$ and
$x_2, x_4\in H'\setminus H$. It cannot be that both $x_1=x_3$, and $x_2=x_4$ 
without violating planarity. 
So assume without loss of generality that $x_2\neq x_4$.

Since $\mathcal{H}$ is non-piercing, $H$ is connected and $H\setminus H'$ induces a connected subgraph of $G$. Further,
note that $H$ and $H'$ are non-piercing in $G$, then they remain non-piercing in $R_G(H, H')$.
Therefore, there is an $x_1$-$x_3$ path $P$ in $R_G(H, H')$ that lies in $H\setminus H'$. 
Again, since $\mathcal{H}$ is non-piercing, $H'\setminus H$ induces a connected subgraph of $G$. 
Therefore, there is a path $P'$ between $x_2$ and $x_4$ that lies in $H'\setminus H$.
Observe that $P\cup\{x_1,x\}\cup\{x,x_3\}$
induces a Jordan curve with $x_2$ and $x_4$ on either side of it. 
Thus $P$ and $P'$ intersect at a vertex that lies in $H\cap H'$, which is not possible since $P$ and $P'$ are disjoint. Therefore, there is no path $P'$ between $x_2$ and $x_4$ in $H'\setminus H$
which implies $H'\setminus H$ is not connected
and thus $\mathcal{H}$ is piercing.
\end{proof}

Note that the reverse implication does not hold. It is easy to construct examples of graph systems in the plane that are cross-free, but are piercing. 
Consider the graph system consisting of a graph $K_{1,4}$ embedded in the plane,
with central vertex $v$, and leaves $a,b,c,d$ in cyclic order.
Let $H$ and $H'$ be two subgraphs where $H$ is the graph induced on $\{v,a,b\}$ and $H'$ is the graph induced on $\{v,c,d\}$. Then, $H$ and $H'$ are cross-free, but neither $H\setminus H'$ nor $H'\setminus H$ is connected.

The proof of Theorem \ref{thm:planenp} relies on the Jordan curve theorem, and hence
the corresponding statement does not hold for surfaces of higher genus. For example, let $G$ be the
torus grid graph $T_{n,n} = C_n \Box C_n$\cite{weisstein2016torus}.  
The subgraphs $\mathcal{H}$ are the $n$ non-contractible cycles perpendicular to the hole, 
and the $n$ non-contractible cycles parallel to the hole. Note that the system $(T_{n,n},\mathcal{H})$ is non-piercing
but not cross-free.
Any pair of parallel and perpendicular cycles intersect at a unique vertex, and therefore in the dual support,
the vertices corresponding to these two cycles must be adjacent. Therefore, 
the dual support is
$K_{n,n}$ which is not embeddable on the torus for large enough $n$.
However, we show in Theorem \ref{thm:intsupport} 
that cross-free is a sufficient condition
for a system $(G,\mathcal{H})$ on a graph of genus $g$ to have a (primal/dual/intersection) support of genus at most $g$.

\section{Bounded Genus Graphs}
\label{sec:boundedgenus}
In this section we consider the setting where the host graph has bounded genus. 
We start in Section \ref{sec:vertexbypassing} where we define the Vertex Bypassing operation that we require to obtain the primal, dual, and
intersection supports.

\subsection{Vertex Bypassing}
\label{sec:vertexbypassing}
Vertex Bypassing (\VB$(v)$) takes a cross-free system $(G,\mathcal{H})$ as input and \emph{simplifies} the system around a vertex $v$
of $G$.
Since $(G,\mathcal{H})$ is cross-free, we assume that we are given a cross-free embedding of $G$ with respect to $\mathcal{H}$ on a surface of genus $g$. Vertex bypassing at a vertex $v$ is defined as follows: 

\begin{definition}[\VB$(v)$]
\label{defn:vertexbypassing}
Let $(G,\mathcal{H})$ be a cross-free system, and that we have a cross-free embedding of $G$ with respect to $\mathcal{H}$ in an oriented surface $\Sigma$.
Let $N(v)=(v_0,\ldots, v_{k-1}, v_0)$ be the cyclic order of neighbors of $v$ in that embedding.
\begin{enumerate}
\item Subdivide each edge $\{v,v_i\}$ by a vertex $u_i$. Connect consecutive vertices
$u_i, u_{i+1}$ (with indices taken~$\mathrm{mod}~ k$) with a simple arc not intersecting the edges of $G$ to
construct a cycle $C=(u_0,\ldots, u_{k-1}, u_0)$
such that  the resulting graph $G''$ remains embedded in $\Sigma$. Remove $v$.
\label{step:one}
\item For $H\in \mathcal{H}_v$, let $H'$ denote the
subgraph of $G''$ induced by $(V(H)\setminus \{v\})\cup \{u_i: v_i\in V(H)\}$. 
Let $\mathcal{H}'_v =\{H': H\in\mathcal{H}_v\}$.
Let $\mathcal{H}' = (\mathcal{H}\setminus\mathcal{H}_v)\cup\mathcal{H}'_v$ (Note that the subgraphs in $\mathcal{H}'_v$ may not be connected).\label{step:two}
\item Add a set $D$ of non-intersecting\footnote{We use the term non-intersecting to mean internally non-intersecting}  chords in $C$ so that $\forall H\in \mathcal{H}'$,
$H$ induces a connected subgraph in $C\cup D$, and $\mathcal{H}'$ remains cross-free. 
\label{step:three}
\end{enumerate}
Let $(G',\mathcal{H}')$ be the resulting system.
\end{definition}

It is easy to see that the graph $G'$ obtained from $G$ 
is also embedded in $\Sigma$ as each operation preserves
the embedding. 
%We show in Lemmas \ref{lem:keylem} and Lemma \ref{lem:remcrossfree} 
%that we can implement Step \ref{step:three} of \VB$(.)$.
At the end of Step \ref{step:one}, since we remove vertex $v$, the subgraphs $\mathcal{H}'_v$ in $G''$ 
may be disconnected. The main challenge is to add additional edges to the graph $G''$ so that each subgraph $\mathcal{H}'_v$,
$v\in V(G'')$ 
is connected, and the subgraphs remain cross-free.

In order to do so, we introduce the notion of $abab$-free hypergraphs.
An equivalent notion, namely $ABAB$-free hypergraphs was studied by Ackerman et al. 
\cite{ackerman2020coloring}, where the elements
of the hypergraph are placed in a linear order instead of a cyclic order.

\begin{definition}[$abab$-free]
\label{defn:abab}
A hypergraph $(X,\mathcal{H})$ is said to be $abab$-free if there is a cycle $C$ on $X$ such that  for any $H,H'\in\mathcal{H}$
there are no four vertices $x_1, x_2, x_3, x_4$ in cyclic order around $C$ such that $x_1,x_3\in H\setminus H'$,
and $x_2,x_4\in H'\setminus H$.
\end{definition}

Observe that the cycle $C=(u_0,\ldots, u_{k-1}, u_0)$ and subgraphs $\mathcal{H}'_v$ defined
in Steps (\ref{step:one}) and (\ref{step:two}) of Vertex Bypassing (Defn. \ref{defn:vertexbypassing}) induce an $abab$-free embedding
of an $abab$-free hypergraph.
Therefore, the problem of adding a set of non-intersecting chords $D$ in Step \ref{step:three} of vertex-bypassing 
reduces to the following: Given an $abab$-free embedding of an $abab$-free hypergraph, can we add a set of non-intersecting
chords in $C$ such that each subgraph is connected? We show in the following lemma, whose proof is in Section \ref{sec:nonblocking}
that we can always add such chords.

\begin{restatable}{lem}{nblockchord}
%\begin{theorem}
\label{lem:nblock}
Let $C$ be a cycle embedded in the plane, and let $\mathcal{K}$ be a set of $abab$-free subgraphs of $C$.
Then, we can add a set $D$ of non-intersecting chords in $C$ such that each $K\in\mathcal{K}$ 
induces a connected subgraph of $C\cup D$. Further, the set $D$ of non-intersecting chords to add can be computed in polynomial time.
%\end{theorem}
\end{restatable}

With Lemma \ref{lem:nblock} in hand, we can obtain the desired system $(G',\mathcal{H}')$.

\begin{lemma}
\label{lem:keylem}
Let $(G,\mathcal{H})$ be a cross-free system with a cross-free embedding of $G$ with respect to $\mathcal{H}$. 
Suppose we apply \VB$(v)$ to vertex $v\in V(G)$. Then, each subgraph $H$ in $(G',\mathcal{H}')$ is connected. Further,
\VB$(v)$ can be done in polynomial time.
\end{lemma}
\begin{proof}
Let $C$ be the cycle added on the subdividing vertices around vertex $v$. Since $(G,\mathcal{H})$ is cross-free, the subgraphs $\{H\cap C: H\in\mathcal{H}'_v\}$ 
satisfy the $abab$-free property on $C$. Therefore, by Lemma \ref{lem:nblock}, there is a collection $D$ of non-intersecting chords such that 
each subgraph in $\mathcal{H}'_v$ induces a connected subgraph of $C\cup D$. Hence, each subgraph $H\in \mathcal{H}'$ is a connected subgraph of $G'$ since each $H\in\mathcal{H}_v$ is modified only in the vertices of subdivision. Since Lemma \ref{lem:nblock} guarantees that the set $D$ of 
non-intersecting chords to add can be computed in polynomial time, it follows that \VB$(v)$ can be done in polynomial time.
\end{proof}

In the following, we argue that if $(G,\mathcal{H})$ is cross-free, it remains cross-free after bypassing vertex $v$. The proof is straightforward, but a bit tedious.

\begin{lemma}
\label{lem:remcrossfree}
Let $(G,\mathcal{H})$ be a cross-free system. Let $(G',\mathcal{H}')$ be the system obtained after applying vertex bypassing at a vertex
$v\in V(G)$. Then, $(G',\mathcal{H}')$ is cross-free.
\end{lemma}

\begin{proof}
By Lemma \ref{lem:keylem}, each subgraph $H'\in (G',\mathcal{H}')$ is connected. We will show that
the system is cross-free.
Let $N(v)=\{v_0,\ldots, v_{k-1}\}$ in $G$
and $S(v)=\{u_0,\ldots, u_{k-1}\}$ where $u_i$ subdivides the edge $\{v,v_i\}$.

For any two subgraphs $H_1$ and $H_2$ in $\mathcal{H}$, let $H'_1$ and $H'_2$ respectively denote the corresponding subgraphs in $\mathcal{H}'$. 
Consider the reduced graph $R_G(H_1, H_2)$. For a vertex $x\in V(G)$,
let $c_x$ denote its corresponding vertex in $R_G(H_1, H_2)$. Similarly, let $c'_x$ denote the vertex in $R_{G'}(H'_1, H'_2)$ corresponding
to a vertex $x$ in $G'$.
For a vertex $c_x\in V(R_G(H_1,H_2))$ with neighbors $x_1,\ldots, x_{\ell}, x_1$ in cyclic order around $c_x$, let  
the \emph{cyclic pattern} at $c_x$ be the cyclic sequence $(A_1, A_2, \ldots, A_{\ell}, A_1)$, 
where $A_i$ is the subset of $\{H_1, H_2\}$ containing $x_i$.

For any vertex $x\not\in N(v)$, the cyclic pattern of $c'_x$ in $R_{G'}(H'_1, H'_2)$ is identical to its
cyclic sequence in $R_G(H_1, H_2)$
up to relabeling $H_i$ by $H'_i$ (in the following, we say that the cyclic pattern is identical to mean that
it is unchanged up to relabeling $H_i$ by $H'_i$). Since 
$(G,\mathcal{H})$ is cross-free, the subgraphs in $R_{G'}(H'_1, H'_2)$ are cross-free at $c_x$.

We are left with showing that $H'_1$ and $H'_2$ are cross-free at each vertex in $N(v)\cup S(v)$.
We consider two cases depending on whether $v\in H_1\cap H_2$. 

{\bf Case 1:} $v\not\in H_1\cap H_2$.
Consider a vertex $v_i\in N(v)$ such that $v_i\in H_1\cap H_2$.
Since $(G,\mathcal{H})$ is cross-free, $H_1$ and $H_2$ are not crossing at $c_{v_i}$
in $R_G(H_1, H_2)$. The vertex $c'_{v_i}$, corresponding to $v_i$ in $R_{G'}(H'_1, H'_2)$ has the same
cyclic pattern as $c_{v_i}$, except that $v$ is replaced by $u_i$. 
Since $\mathcal{H}'_{u_i}\subseteq\mathcal{H}_{v_i}$, and $v_i\in H_1\cap H_2$, $\mathcal{H}'_{u_i}\cap\{H'_1,H'_2\}=\mathcal{H}_v\cap\{H_1,H_2\}$ (up to relabeling of $H_i$ by $H'_i$).
Hence, the cyclic pattern at $c'_{v_i}$ is identical to the cyclic pattern at $c_{v_i}$. Since $(G,\mathcal{H})$
is cross-free at $c_{v_i}$, it follows that $H'_1$ and $H'_2$ are cross-free at $c'_{v_i}$.

For each $u_i\in S(v)$, since $\mathcal{H}'_{u_i}\subseteq\mathcal{H}_{v}$, and we only need to show cross-freeness
at vertices in $H'_1\cap H'_2$, it follows that if $v\not\in H_1\cap H_2$, then $H'_1$ and $H'_2$ are cross-free
at each $c_{u_i}$.
Hence, $H'_1$ and $H'_2$ are cross-free at each vertex in $R_{G'}(H'_1, H'_2)$ when $v\notin H_1\cap H_2$.

{\bf Case 2:} $v\in H_1\cap H_2$.
If no vertex $v_i\in N(v)$ is in $H_1\cap H_2$, then $H'_1$ and $H'_2$ are trivially cross-free at each vertex $c'_{v_i}$ corresponding
to $v_i$ in $R_{G'}(H'_1, H'_2)$. Further, for vertices of $S(v)$, we have $\mathcal{H}'_{u_i}\subseteq\mathcal{H}_{v_i}$ It follows that no $u_i\in S(v)$
is contained in $H_1\cap H_2$. Thus, $H'_1$ and $H'_2$ are cross-free at each vertex $u_i\in S(v)$. Thus, $H'_1$ and $H'_2$
are cross-free at all vertices in $R_{G'}(H'_1, H'_2)$. 

Finally, suppose there is a vertex $v_i\in N(v)$ such that $v_i\in H_1\cap H_2$. Since $v$ and $v_i$ are adjacent in $G$,
they are contracted to the same vertex in $R_{G}(H_1, H_2)$. That is, $c_{v}=c_{v_i}$ in $R_G(H_1, H_2)$.
Since $u_i\in H'_1\cap H'_2$, it follows that $c'_{v_i}=c'_{u_i}$ in $R_{G'}(H'_1, H'_2)$. Let $u_j$ denote a vertex in $S(v)$
adjacent to $u_i$ in $G'$. Consider the cyclic pattern at $c'_{u_i}$.
The vertex $u_j$ corresponds to a vertex $v_j$ adjacent to $v$ in $G$. Observe that since $v\in H_1\cap H_2$
and $\mathcal{H}'_{u_{j}}\subseteq\mathcal{H}_{v_{j}}$, $\mathcal{H}'_{u_{j}}\cap\{H'_1,H'_2\}=\mathcal{H}_{v_{j}}\cap\{H_1,H_2\}$ (up to relabeling of $H_i$ by $H'_i$).
Hence, the cyclic pattern around $c'_{u_i}$ is identical to the cyclic pattern around $c_{v_i}$. Since $H_1$ and $H_2$
are cross-free at $v_i$, they remain cross-free at $c'_{u_i}$.

Since $H_1$ and $H_2$ were arbitrary subgraphs, it follows that $(G',\mathcal{H}')$ is a cross-free system.
\end{proof}

\subsection{Non-blocking chords in $abab$-free hypergraphs}
\label{sec:nonblocking}
In this section, we prove Lemma \ref{lem:nblock}. Let $C=(x_0,\ldots, x_{k-1},x_0)$ be a cycle embedded in the plane 
with vertices labelled in clockwise order. Let $\mathcal{K}$ be a collection of subgraphs of $C$ such that $\mathcal{K}$
is $abab$-free. 
For $i,j\in\{0,\ldots, k-1\}$, let $[x_i, x_j]$ denote the vertices on the arc from $i$ to $j$ in clockwise order.
Similarly, we use  $(x_i,x_j)$ to denote the open arc, i.e., consisting of the vertices on the arc from $i$ to $j$
except $x_i$ and $x_j$. The half-open arc $(x_i,x_j]$ that excludes $x_i$ but includes $x_j$ is defined similarly.

The addition of a chord $d=\{x_i, x_j\}$ divides $C$ into two open arcs - $(x_i,x_j)$ and $(x_j, x_i)$.
The chord $d$ \emph{blocks} a subgraph $K\in\mathcal{K}$ 
if both open arcs contain a run of $K$, and neither end-point of $d$ is contained in $K$. Here a \emph{run} refers to a connected component
of the subgraph $K$ in $C$.
Such a chord $d$ is called a \emph{blocking chord}. 
If $d$ does not block any subgraph in $\mathcal{K}$, it is called a \emph{non-blocking} chord. 
We show in Lemma \ref{lem:keylem2} that there always exists a non-blocking chord $d$ 
that connects two disjoint runs of some subgraph $K\in\mathcal{K}$. 

\begin{lemma}
\label{lem:keylem2}
Let $C$ be a cycle embedded in the plane, and let $\mathcal{K}$ be a collection of $abab$-free subgraphs in the embedding of $C$.
Then, for some disconnected $K\in\mathcal{K}$, there exists a non-blocking chord joining two disjoint runs of $K$. Further, such a chord can
be computed in polynomial time,
\end{lemma}
\begin{proof}
Assume wlog that each subgraph $K\in\mathcal{K}$ induces at least two runs in $C$,
and no two subgraphs contain the same subset of vertices of $C$.
Define a partial order $\prec_C$ on $\mathcal{K}$, where for $K, K'\in\mathcal{K}$, 
$K\prec_C K'$ iff $K\cap C\subset K'\cap C$. 
Let $K_0\in \mathcal{K}$ be a minimal subgraph with respect to the order $\prec_C$.

Let $K_0^0, \ldots, K_0^q$ denote the runs of $K_0$. We let $A$ denote the run $K_0^0$ and
let $B = \cup_{i=1}^q K_0^i$. For ease of exposition, we assume $C$ is drawn such that 
$A$ lies in the lower semi-circle of $C$ and that $B$ lies in the upper semi-circle of $C$, where the runs $K_0^1,\ldots, K_0^q$ appear in counter-clockwise order. Let $a_0,\ldots, a_r$ denote the
vertices of $A$ in clockwise order and let $b_0,\ldots, b_s$ denote the vertices of $B$ in 
counter-clockwise order. See Figure \ref{fig:circle}.

We show that there is a chord $d$ from a vertex in $A$ to a vertex in $B$ that
is non-blocking. In order to do so, we start with the chord $d_0 = a_0b_0$, and construct
a sequence of chords until we either find a non-blocking chord, or we end up with the chord $d_k = a_rb_s$,
which will turn out to be non-blocking. 
Having constructed chords $d_0,\ldots, d_{i-1}$, where $d_{i-1} = a_{\ell}b_{j}$, $d_i$ 
will be either the chord $a_{\ell}b_{j'}$ or
$a_{\ell'}b_{j}$, where $j'>j$ and $\ell' > \ell$. 

Next, we describe the construction of the chords. Each chord $d$ we construct satisfies the following invariant:
If $K$ is a subgraph blocked by a chord $d=a_{\ell}b_{j}$, then 
\begin{enumerate}
\item[$(i)$] $K$ is contained in $K_0$ in the arc $(a_{\ell},b_{j})$, and
\item[$(ii)$]
there is a vertex $k\in K\setminus K_0$ in the arc $(b_{j}, a_{\ell})$.
\end{enumerate}

Let $d_0$ denote the chord $a_0b_0$. If $d_0$ is non-blocking, we are done. Otherwise,
if any $K_1\in \mathcal{K}$ is blocked by $d_0$, there is a vertex $k\in K_1$ that lies in $(b_0,a_0)$. Since
$(b_0,a_0)$ does not contain a vertex of $K_0$, this implies $k\in K_1\setminus K_0$, and hence $d_0$
satisfies condition $(ii)$ of the invariant.
Since we assumed the subgraphs $\mathcal{K}$ are $abab$-free, 
this implies that any vertex of $K_1$ in arc $(a_0,b_0)$ is contained in $K_0$. 
This ensures that condition $(i)$ of the invariant is satisfied by $d_0$.

Having constructed $d_0=a_0b_0,\ldots, d_{i-1}=a_{\ell}b_j$, each of which satisfy conditions $(i)$ and $(ii)$
of the invariant, 
we construct $d_i$ as follows: We simultaneously scan the vertices of $B$ in counter-clockwise order from $b_j$, 
and the vertices of $A$ in clockwise order from $a_{\ell}$ 
until we find the
first vertex $x$ that belongs to a subgraph blocked by $d_{i-1}$. Let $K_i$ denote this subgraph.
If $x=b_{j'}\in B$, we set
$d_i = a_{\ell}b_{j'}$. Otherwise, $x=a_{\ell'}\in A$, and we set $d_i = a_{\ell'}b_j$. Assume without loss
of generality that $d_i = a_{\ell}b_{j'}$ (the other case is similar).


\begin{figure}[ht!]

\centering
\begin{subfigure}{0.4\textwidth}
\includegraphics[scale=.6]{circle.pdf}
 \vspace{0.2cm}
\caption{Ordering the vertices of $K_0$ in sets $A$ and $B$.
Here, $A=\{a_0,\ldots,a_r\}$ and $B=\{b_0,\ldots,b_s\}$.}
\label{fig:circle}
\end{subfigure}
\hfill
\begin{subfigure}{0.5\textwidth}
\includegraphics[scale=.6]{nextchord}\vspace{0.85cm}
\caption{Adding chords between $A$ and $B$.
If $K_{i+1}\setminus K_0\neq \emptyset$ in $(a_\ell,b_{j'})$, there is \emph{abab} among subgraphs $K_{i+1}$ and $K_i$ witnessed by the vertices $u,b_{j'},v$ and $w$.}
\label{fig:nextchord}
 \end{subfigure}\vspace{0.5cm}
 \caption{Finding a non-blocking chord to join two disjoint runs of $K_0$.}
 \end{figure}

If $d_i$ is a non-blocking chord, we are done. Otherwise, let $K_{i+1}$ denote a subgraph blocked by $d_i$. Then,
both the arcs $(a_{\ell}, b_{j'})$ and $(b_{j'},a_{\ell})$ contain a run of $K_{i+1}$, and $a_{\ell},b_{j'}\in K_0\setminus K_{i+1}$.
We now show that $d_i$ satisfies the invariant. Most of the work will go in showing that $d_i$ satisfies condition
$(i)$ of the invariant. We show this by contradiction - If $d_i$ does not satisfy invariant $(i)$, we will
exhibit a pair of subgraphs violating the $abab$-free property.

Suppose $d_i$ does not satisfy condition $(i)$ of the invariant, that is, there is a vertex $u\in K_{i+1}\setminus K_0$ 
that lies in $(a_{\ell}, b_{j'})$. 
Since $d_{i-1}$ satisfies both the conditions of the invariant, the subgraph
$K_i$ blocked by $d_{i-1}$ is contained in $K_0$ in $(a_{\ell}, b_j)$. Since $(a_{\ell}, b_{j'})\subset (a_{\ell}, b_j)$, it implies $u\not\in K_i$, and thus $u\in K_{i+1}\setminus K_i$.
By construction of $d_i$, the vertex $b_{j'}\in K_i$, and since $d_i$ blocks $K_{i+1}$, $b_{j'}\not\in K_{i+1}$. 
Thus, $b_{j'}\in K_i\setminus K_{i+1}$.

Now, we claim that $K_{i+1}$ is not blocked by $d_{i-1}$. To see this, since $d_{i-1}$ satisfies condition $(i)$ of
the invariant, for any subgraph $K'$ blocked by $d_{i-1}$, we have that $K'\subseteq K_0$ in $(a_{\ell},b_j)$.
Since $(a_{\ell}, b_{j'})\subset (a_{\ell}, b_{j})$, combined with the facts that
$u\in K_{i+1}\setminus K_0$ and that $u$ lies in $(a_{\ell}, b_{j'})$ implies that $K_{i+1}$ is not
blocked by $d_{i-1}$.
But $K_{i+1}$ is blocked by $d_i$, it follows that there is a vertex $v$ of $K_{i+1}$ in the
arc $(b_{j'},b_j]$.
Note that $v$ need not lie in $K_0$.
However, no vertex in the arc $(b_{j'},b_j]$ lies in $K_i$, since $b_{j'}$ was the first vertex encountered
that was contained in a subgraph blocked by $d_{i-1}$ when traversing the vertices of $B$ in counter-clockwise order
from $b_j$.
Therefore, $v\in K_{i+1}\setminus K_i$. 

Finally, since $d_{i-1}$ satisfies condition $(ii)$ of the invariant, it implies
that there is a vertex $w\in K_i\setminus K_0$ that lies in $(b_j, a_{\ell})$. 
However, since $u\in K_{i+1}\setminus K_0$, and $u$ lies in $(a_{\ell}, b_{j'})$, $K_{i+1}\subseteq K_0$ in
$(b_{j'}, a_{\ell})$ since the arrangement is $abab$-free. 
However, since $(b_{j}, a_{\ell})\subset (b_{j'}, a_{\ell})$, this implies 
$w\not\in K_{i+1}$. Therefore, $w\in K_i\setminus K_{i+1}$. See Figure \ref{fig:nextchord}.


From the above arguments, it follows that the subgraphs $K_{i+1}$ and $K_i$ are not $abab$-free, 
as witnessed by the sequence
of vertices $u, b_{j'}, v$ and $w$, a contradiction.
Thus, $d_i$ satisfies condition $(i)$ of the invariant. 
The fact that $d_i$ satisfies condition $(ii)$ of the invariant follows from the fact that $K_0$ is minimal.
Otherwise, $K_{i+1}\subseteq K_0$ in $(a_{\ell}, b_{j'})$ and in $(b_{j'}, a_{\ell})$, and therefore 
$K_{i+1}\subset K_0$.

Since the set of chords is finite, the sequence of chords constructed either ends in a 
non-blocking chord, or we end up with the chord $d=a_rb_s$. We claim
that $d$ must be a non-blocking chord. Suppose $d$ blocks a subgraph $K$. Then, $(a_r, b_s)$ contains
a vertex $u\in K\cap K_0$, as $d$ satisfies invariant $(i)$ and $(ii)$. However, $(a_r, b_s)$ 
does not contain a vertex in $K_0$. 
Therefore, $d$ must be a non-blocking chord.

We scan over at most $|C|^2$ chords, and for each chord we can check if it a non-blocking chord in polynomial time. Therefore,
we can compute a non-blocking chord in polynomial time.
\end{proof}

We are now ready to prove Lemma \ref{lem:nblock}. We do this by using Lemma \ref{lem:keylem2} 
to add a non-blocking chord connecting two disconnected components of a subgraph, and then recursively apply Lemma \ref{lem:keylem2}
to the two resulting cycles and their induced subgraphs.

\nblockchord*
%\begin{lemma}
%\label{lem:nblock}
%Let $C$ be a cycle and let $\mathcal{K}$ be an $abab$-free collection of subgraphs of $C$.
%Then, we can add a set $D$ of non-intersecting chords to $C$ such that each $K\in\mathcal{K}$ 
%induces a connected subgraph of $C\cup D$.
%\end{lemma}

\begin{proof}[Proof of Lemma \ref{lem:nblock}]
For any subgraph $K\in\mathcal{K}$, let $n_K$ denote the number of disjoint runs of $K$ on $C$.
Let 
\begin{align*}
\mathrm{cost}(C,\mathcal{K}) = \sum_{K\in\mathcal{K}} (n_K-1) 
\end{align*}
If $\mathrm{cost}(C,\mathcal{K})=0$, then every subgraph $K\in\mathcal{K}$ consists of one run, 
and therefore $C\cap K$ is connected for each $K\in\mathcal{K}$, and
$D=\emptyset$ suffices. 

Suppose the lemma holds for all $(C',\mathcal{K}')$ with $\mathrm{cost}(C',\mathcal{K}')<N$. Given an instance with $\mathrm{cost}(C,\mathcal{K}) = N$, by Lemma \ref{lem:keylem2}, 
there is a non-blocking chord $d=\{x,y\}$ joining two disjoint runs of some subgraph $K_0\in\mathcal{K}$.

The chord $d=\{x,y\}$ divides the cycle $C$ into two arcs, $[x,y]$, and $[y,x]$. 
We construct two disjoint sub-problems on the cycles $C_{\ell}$ and $C_r$ obtained from $C$, where $C_{\ell}$ is obtained by adding the edge
$\{x,y\}$ to the arc $[y,x]$, and $C_r$ is obtained by adding the edge $\{x,y\}$ to the arc $[x,y]$.
The subgraphs in
$C_{\ell}$ and $C_r$ are respectively those induced by $\mathcal{K}$, namely 
$\mathcal{K}_{\ell} =\{K\cap C_{\ell}: K\in\mathcal{K}\}$, 
and $\mathcal{K}_r = \{K\cap C_r:K\in\mathcal{K}\}$. 
Note that $\mathcal{K}_{\ell}$ and $\mathcal{K}_r$ 
are $abab$-free on $C_{\ell}$ and $C_r$, respectively. 
Let $n^{\ell}_K$ and $n^r_{K}$ denote respectively, the
number of runs of $K\in\mathcal{K}$ in $C_{\ell}$ and in $C_r$.
Clearly, $n^{\ell}_{K_0} < n_{K_0}$ and $n^r_{K_0} < n_{K_0}$.
Also, for all other subgraphs $K'\in\mathcal{K}$, $n^{\ell}_{K'}\le n_{K'}$ and $n^r_{K'}\le n_{K'}$,
it follows that $\mathrm{cost}(C_r,\mathcal{K}_r)<\mathrm{cost}(C,\mathcal{K})$ and $\mathrm{cost}(C_{\ell},\mathcal{K}_{\ell})<\mathrm{cost}(C,\mathcal{K})$.

Hence, by the inductive hypothesis on $C_{\ell},\mathcal{K}_{\ell}$ and $C_{r},\mathcal{K}_r$ respectively, there exists a set of non-intersecting
chords $D_{\ell}$ such that each $K\in\mathcal{K}_{\ell}$ induces a connected subgraph of $C_{\ell}\cup D_{\ell}$. Similarly, there exists a set of non-intersecting chords $D_r$ 
such that each $K\in\mathcal{K}_r$, induces a connected subgraph of $C_{r}\cup D_{r}$.
It follows that $D=D_{\ell}\cup D_r\cup d$ is a set of non-intersecting chords such that each $K\in\mathcal{K}$ induces  
a connected subgraph of $(C\cup D)$.

Since a non-blocking chord can be computed in polynomial time, and this yields two smaller instances where we want to find a non-blocking chord,
it follows that in polynomial time we can add chords so that each subgraph is connected.
\end{proof}



\section{Construction of Supports}
\label{sec:constructiongenus}
In this section, we show that for cross-free systems on a graph of genus $g$, there exist primal, dual
and intersection supports of genus at most $g$. While the existence of an intersection support implies the
existence of the primal and dual supports, we use the dual support and techniques from the primal support
in order to construct the intersection support, and hence present these first.

We obtain a polynomial time algorithm
to construct a primal support when a cross-free embedding of the graph system $(G,\mathcal{H})$ is given. However, we are
unable to prove a similar result for a dual support or intersection support, and this is an intriguing open question.


\subsection{Primal Support}
\label{sec:primalsupportgenus}
In this section, we show that for a cross-free system $(G,\mathcal{H})$ of genus $g$ with $c:V\to\{\R,\B\}$, 
there is a primal support $Q$ of genus at most $g$, and that $Q$ can be constructed in polynomial 
time in $|V(G)|$ and $|\mathcal{H}|$ if a cross-free embedding of $G$ is given. 
Recall that $\depth(v)=|\mathcal{H}_v|$, and $\depth(e)=|\{H\in\mathcal{H}: e\in H\}$.
We say that a vertex $v\in V(G)$ is \emph{maximal} if $\depth(v)>\depth(e)$ for all $e\sim v$.

We start with the following lemma for the construction of a primal support. In the lemma, we do not require
that $(G,\mathcal{H})$ is cross-free. In fact, the lemma 
also holds for any graph class closed under edge-contraction.
However, we state only for bounded genus, as this is the statement required for subsequent theorems.

\begin{lemma}
\label{lem:easypsupport}
Let $(G,\mathcal{H})$ be a graph system of genus $g$ with $c: V(G)\to\{\R,\B\}$ 
where each $H\in\mathcal{H}$ induces a connected subgraph
such that no vertex in $\R(V)$ is maximal. 
Then, there is a primal support $Q$ of genus at most $g$
for $(G,\mathcal{H})$ that can be computed in polynomial time.
\end{lemma}
\begin{proof}
We prove by induction on $|\R(V(G))|$. If $|\R(V(G))|=0$, then $G=Q$ is the desired support.
Suppose the statement holds for any graph system of genus $g$ with less than $k$ red vertices.

Consider a graph system $(G,\mathcal{H})$ with $|\R(V(G))|=k$.
Let $u\in\R(V(G))$. Since $u$ is not maximal, there is an edge $e=\{u,v\}\in E(G)$ such that 
$\mathcal{H}_e =\mathcal{H}_u$. Since $\mathcal{H}_e\subseteq\mathcal{H}_v$, it follows
that $\mathcal{H}_u\subseteq\mathcal{H}_v$. Let $H\in\mathcal{H}_u$. Since $H$ is connected
in $G$, between any pair of vertices $x,y\in H$, there is a path $P$ that lies in $H$.
Since $\mathcal{H}_u\subseteq\mathcal{H}_v$, it implies $v\in H$ and consequently, $H$
is connected in the graph $G'=G/\{u,v\}$, where the contracted vertex receives the
color of $v$. Further, $|\R(V(G'))|=k-1$ and $G'$ has genus
at most $g$. By the inductive hypothesis, there is a support $Q$ for $(G',\mathcal{H})$.
Note that $Q$ is also a support for $(G,\mathcal{H})$ 
since the set of blue vertices in $H$ remains unchanged in $(G',\mathcal{H})$. 
Since each edge contraction can be performed in polynomial time, and
the number of edge contractions is bounded by $|\R(V(G))|$, it follows that
a primal support for $(G,\mathcal{H})$ can be computed in polynomial time.
\end{proof}

\begin{theorem}
\label{thm:primalsupport}
Let $(G,\mathcal{H})$ be a cross-free system of genus $g$, 
with $c:V(G)\to\{\R,\B\}$. There exists a primal support $Q$ of genus at most $g$ on $\B(V)$.
Further, if a cross-free embedding of $G$ is given, $Q$ can be constructed in time polynomial in $|V(G)|$ and $|\mathcal{H}|$.
\end{theorem}

\begin{proof}
We prove by induction on the number of maximal red vertices. If no red vertex in $G$ is maximal,
then by Lemma \ref{lem:easypsupport}, there is a support $Q$ of genus $g$.

Suppose the statement holds for any cross-free graph system of genus $g$ with at most $k$ 
maximal red vertices.

Let $v$ be a maximal red vertex. 
We apply $\VB(v)$.
For $i=0,\ldots, \deg(v)-1$, let $u_i$ be the vertices added by $\VB(v)$ subdividing
edges $\{v,v_i\}$. Set $c(u_i)=\R$, $\forall\;i=0,\ldots, \deg(v)-1$.
No $u_i$ is maximal since $\mathcal{H}_{\{u_i,v_i\}}=\mathcal{H}_{u_i}$ for all $i$.
Further, by Lemma \ref{lem:keylem} and \ref{lem:remcrossfree}, the resulting
graph system $(G',\mathcal{H}')$ is cross-free and of genus at most $g$, and further, 
it can be computed in polynomial time.

By the inductive hypothesis, a primal support $Q$ for $(G',\mathcal{H}')$ can be computed in polynomial time. $Q$ is also a primal support for $(G,\mathcal{H})$ since the set of blue vertices in each subgraph $H\in\mathcal{H}'$ remains unchanged. 

A primal support $Q$ for $(G,\mathcal{H})$ can be computed in polynomial time: by Lemma \ref{lem:keylem}, each vertex bypassing operation can be done
in polynomial time. Since the number of maximal red vertices is at most $|V(G)|$, and a vertex bypassing operation
does not increase the number of maximal red vertices, we perform at most $|V(G)|$ vertex bypassing operations.
Finally, if no red vertex is maximal, by Lemma \ref{lem:easypsupport} we compute a primal support in polynomial time.

\hide{
Thus, if $depth(v)\ge 2$, we can reduce the number of vertices of maximum depth. 
We can assume that there is a maximal vertex in $\R(V)$. 
Note that
such a maximal vertex exists: Consider a vertex $u$ in $\R(V)$ of maximum depth. Since $\mathcal{H}_e\subseteq\mathcal{H}_u$ for
$e\sim u$.
If for an edge $e=\{u,w\}$, $\depth(e)=\depth(u)$,
then $\mathcal{H}_e=\mathcal{H}_u$, and $\mathcal{H}_e\subseteq\mathcal{H}_w$. Since $u$ has maximum depth, this implies
$\mathcal{H}_w=\mathcal{H}_e$. Hence, $\mathcal{H}_u=\mathcal{H}_w$ and we can contract the edge $e$ and set the color
of the contracted vertex to that of $w$.
A support for the resulting graph system is a support for the given system.

We apply \VB$(v)$.
For $i=1,\ldots, \deg(v)$, let $u_i$ be the vertices added by \VB$(v)$ subdividing
edges incident on $v$. Set $c(u_i)=\R$, $\forall\;i=1,\ldots, \deg(v)$.
Since $v$ was maximal, it follows that 
$\depth(u_i)<\depth(v)$, for all $i=1,\ldots, k$.
Further, no $u_i$ is maximal since $\depth(u_i)\le \depth(v_i)$, where $u_i$ subdivides the edge $\{v,v_i\}$, and they
remain non-maximal when we bypass other maximal vertices in $\R(V)$. To see this, note that 
if $u_i$ is maximal, the
vertex $u'_i$ subdividing $\{v_i, u_i\}$ satisfies $\mathcal{H}_{u'_i}=\mathcal{H}_{u_i}$.
Thus, by Lemmas \ref{lem:nblock} and \ref{lem:keylem}, the resulting system
is cross-free and we have one fewer maximal red vertex in the resulting system. We repeatedly apply vertex bypassing to
each maximal vertex in $\R(V)$ until we obtain the system $(G',\mathcal{H}')$ where no vertex in $\R(V(G'))$ is maximal.

Since no vertex in $\R(V(G'))$ is maximal in $(G',\mathcal{H}')$, we do the following for each $v\in \R(V(G'))$:
since $v$ is not maximal, 
there is an edge $e=\{v,u\}$, $e\sim v$ such that 
$\mathcal{H}_e = \mathcal{H}_v$, which implies
$\mathcal{H}_v\subseteq\mathcal{H}_u$. 
If $\mathcal{H}_v=\mathcal{H}_u$, then we contract the edge $e$ and set the color of the contracted vertex to the color of $u$. Otherwise, 
we orient the edge $\{v,u\}$ from $v$ to $u$.

Let $D$ denote the directed subgraph on $V(G')$ induced by these directed arcs. Observe that
in $D$, each vertex has out-degree at most 1, and for an arc $(v,u)$ in $D$, we have $\mathcal{H}_v\subset\mathcal{H}_u$,
which implies $D$ is acyclic. Thus, D is a forest. By construction, $D$ satisfies the following properties: 
In each tree $T$ of D, $(i)$ the arcs of $T$ are directed towards the
root of $T$, $(ii)$ all vertices in $T$ except the root are red, and $(iii)$ the root of $T$ is blue.
Further, for each arc $(v,u)$ of $D$, we have $\mathcal{H}_v\subset\mathcal{H}_u$.
Thus, contracting each tree of the forest to its root yields the desired support $Q$. Since $Q$ is obtained by applying 
vertex bypassing and edge-contraction, $Q$ is also an embedded graph of genus at most $g$.

Finally, to see that $Q$ can be constructed in polynomial time, note that there are at most $|V(G)|$ maximal vertices
in $\R(V)$. If a cross-free embedding of $G$ is given, then for a maximal vertex $v\in\R(V(G))$, \VB$(v)$ 
can be implemented in polynomial time by Lemma \ref{lem:keylem}, and 
moreover, we obtain a cross-free embedding of the resulting graph system. 
Further, no new vertex added during \VB$(v)$ becomes maximal on applying further vertex bypassing operations. 
Thus, while the size of the graph grows with each vertex bypassing operation, we apply
the vertex bypassing operation at most $|V(G)|$ times. Finally, constructing $D$ and contracting the arcs in $D$ to obtain $Q$ can clearly be done in polynomial time.
}
\end{proof}

\subsection{Dual Support}
\label{sec:dualsupportgenus}
In this section, we show that for a cross-free system on a graph of genus $g$, there is a dual support of genus at most $g$.
Unlike the primal case however, we are unable to prove that the algorithm implied in the
proof of Theorem \ref{thm:dualembedded} runs in polynomial time, even if a cross-free
embedding of the system is given. The main difficulty is that in the process of vertex bypassing, we add additional red vertices.
In the primal case, the newly added red vertices are not maximal vertices, and therefore we do not apply vertex bypassing
to them.
In the dual case however, we may repeatedly apply vertex bypassing to newly created vertices. As a consequence,
the size of the graph increases. In \cite{RR18}, Raman and Ray dealt with a similar difficulty in constructing a support
for non-piercing regions in the plane. In their setting, a \emph{cell-bypassing} operation
creates additional cells in the arrangement. However, they could overcome it
by showing additional structural restrictions using which, they could define a suitable potential function that was
polynomially bounded at the start, and decreased by at least 1 in each \emph{cell-bypassing} step.
In our case, however, it is not clear how to define a suitable potential function to show that the algorithm runs in polynomial time and
we leave it as an intriguing open problem. 

An edge $\{u,v\}\in E(G)$ is said to be a \emph{special edge} if 
$\mathcal{H}_u\neq\emptyset$, $\mathcal{H}_v\neq\emptyset$ and $\mathcal{H}_u\cap\mathcal{H}_v=\emptyset$.
Let $\Spl_{\mathcal{H}}(E)$ be the set of special edges in $E(G)$. 
A dual support $Q^*$ for  $(G,\mathcal{H})$ satisfies the \emph{special edge property} if 
for each $e=\{u,v\}\in \Spl_{\mathcal{H}}(E)$, there is an edge between some $H\in\mathcal{H}_u$ and $H'\in\mathcal{H}_v$ in $Q^*$.

\begin{lemma}
\label{lem:dualembeddedlem}
Let $(G,\mathcal{H})$ be a system of genus $g$ such that  $depth(v)\le 1$ for each $v\in V(G)$. Then, there
is a dual support $Q^*$ of genus $g$ on $\mathcal{H}$ with the special edge property.
\end{lemma}
\begin{proof}
Each vertex of $G$ has depth at most 1 and therefore, no two subgraphs in $\mathcal{H}$ share a vertex. Contracting
each subgraph $H\in\mathcal{H}$ to a vertex $v_H\in H$ yields a dual support $Q^*$. It is easy to check that $Q^*$ satisfies
the special edge property.
\end{proof}

The construction of a dual support in the general case repeatedly uses vertex bypassing to decrease the maximum depth
of a vertex until each vertex has depth at most 1. 

\begin{theorem}
\label{thm:dualembedded}
Let $(G,\mathcal{H})$ be a cross-free system of genus $g$. Then, there exists a dual support $Q^*$ on $\mathcal{H}$
of genus at most $g$ with the special edge property.
\end{theorem}

\begin{proof}
By Proposition \ref{prop:nocontainment}, we assume that there are no containments in $\mathcal{H}$. 
Since $(G,\mathcal{H})$ is cross-free,
there exists a cross-free embedding of $G$ in a surface of genus $g$. Consider such an embedding of $G$. We abuse notation
slightly and also refer to the embedded graph by $G$.
We prove by induction on cross-free graph systems lexicographically ordered by $(d, n_d)$, 
where $d$ is the maximum depth of a vertex in $G$ and $n_d$ is the number of vertices of depth $d$. 
%We need the following definition: An edge $\{u,v\}\in E(G)$ is said to be a \emph{special edge} if $\mathcal{H}_u\cap\mathcal{H}_v=\emptyset$.
%Let $\Spl_{\mathcal{H}}(E)$ be the set of special edges in $E(G)$. 
%We need a stronger inductive
%hypothesis, namely given $(G,\mathcal{H})$ there is a dual support with the \emph{special edge property}, i.e.,
%there is a dual support $Q^*$ on $\mathcal{H}$ such that for each $e=\{u,v\}\in \Spl_{\mathcal{H}}(E)$, there is an edge between some %$H\in\mathcal{H}_u$ and $H'\in\mathcal{H}_v$ in $Q^*$.

If $d = 1$, then Lemma \ref{lem:dualembeddedlem} guarantees a support satisfying the special edge property.
So suppose $d > 1$. Let $v$ be a vertex of maximum depth in $G$. 
Then, $\mathcal{H}_e\subset\mathcal{H}_v$ for all $e=\{u,v\}\in E(G)$. 
Otherwise, if $\mathcal{H}_e=\mathcal{H}_v$ for some edge $e=\{u,v\}$, then $\mathcal{H}_v\subseteq\mathcal{H}_u$ since $\mathcal{H}_e\subseteq\mathcal{H}_u$ for all $e\sim u$. 
This implies $\mathcal{H}_u=\mathcal{H}_v$ since $v$ has the maximum depth.
Hence, contracting $e$ we obtain a lexicographically smaller system $(G/e, \mathcal{H})$. 
By the inductive hypothesis, there is a dual support $Q^*$ with the special edge property 
for the cross-free system $(G/e, \mathcal{H})$. $Q^*$ is also a dual-support for $(G,\mathcal{H})$ with the special edge property, 
since $e$ is not a special edge.

Therefore, we can assume that for each $e\sim v$, $\mathcal{H}_e\subset \mathcal{H}_v$. We apply \VB$(v)$
to obtain the system $(G',\mathcal{H}')$.
Let $\prec$ denote the lexicographic ordering.
Then, $(d',n'_d) \prec (d,n_d)$, where $d'$ and $n'_d$ are respectively, the
depth of a maximum depth vertex, and the number of vertices of maximum depth in $(G',\mathcal{H}')$.
Further, there is an injective correspondence between the special edges in $G$ and the special edges in $G'$.
For a special edge $\{v,v_i\}$ in $G$, the edge $\{u_i, v_i\}$ is  special in $G'$, where $u_i$ is the vertex added
in the vertex bypassing operation subdividing the edge $\{v,v_i\}$.

By Lemma \ref{lem:keylem}, each subgraph $H\in\mathcal{H}'$ is connected in $G'$,
and by Lemma \ref{lem:remcrossfree}, $(G',\mathcal{H}')$ is cross-free.
By the inductive hypothesis, there is a dual support $Q_1^*$ for $(G',\mathcal{H}')$ satisfying the special edge property. 
We show that $Q_1^*$ is also a support for $(G,\mathcal{H})$. 
For each $u\neq v\in V(G)$, it follows from the inductive hypothesis that the support property is satisfied.
Let $C$ denote the cycle on $u_0,\ldots, u_{\deg(v)-1}$ added in \VB$(v)$.

Since we assumed (by Proposition \ref{prop:nocontainment}) that $\mathcal{H}$ has no containments, it follows that
$\cup_{i=0}^{\deg(v)-1} \mathcal{H}'_{u_i} = \mathcal{H}_v$ as there is no subgraph containing only the vertex $v$.
If none of the edges of $C$ are in $\Spl_{\mathcal{H}'}(E)$, then $\mathcal{H}_v$ is connected since adjacent vertices of $C$ 
share at least one subgraph. On the other hand, if an edge $e=\{u_i, u_{i+1}\}$ (where indices are taken $\mathrm{mod} \deg(v)$) of $C$ is in
$\Spl_{\mathcal{H}'}(E)$, by the inductive hypothesis, at least one subgraph from $\mathcal{H}'_{u_i}$ and one subgraph from $\mathcal{H}'_{u_{i+1}}$ are adjacent in $Q_1^*$.
Since $\cup_{i=0}^{\deg(v)-1}\mathcal{H}_{u_i}=\mathcal{H}_v$, and $C$ is a cycle,
it follows that $\mathcal{H}_v$ is connected and thus taking $Q^*=Q^*_1$, we get the desired dual support for $(G,\mathcal{H})$.
\end{proof}

\subsection{Intersection Support}
\label{sec:intsupportgenus}

We show that a cross-free intersection system $(G,\mathcal{H},\mathcal{K})$ has an intersection support of genus at most $g$
if $G$ has genus $g$. The proof builds on the construction of primal and dual supports, but is more involved. 
A vertex of $G$ that is contained only in subgraphs in $\mathcal{K}$ but not 
in any subgraph in $\mathcal{H}$, is called a $\mathcal{K}$-vertex. An edge $e\in E(G)$ is a $\mathcal{K}$-edge if $\mathcal{K}_e\neq\emptyset$ and $\mathcal{H}_e=\emptyset$. 
% Note that each edge incident on a $\mathcal{K}$-vertex is a $\mathcal{K}$-edge.
%For a $\mathcal{K}$-edge $\{u,v\}$ if both $u$ and
%$v$ are $\mathcal{K}$-vertices, then we can contract $\{u,v\}$ without modifying the intersection hypergraph.
%Therefore, we can assume that in the intersection system, 
%at least one end-point of each $\mathcal{K}$-edge is not a $\mathcal{K}$-vertex.\todo{at this point, K-subgraphs are not cross-free after this contraction.
%So, if we assume wlog that at least one end point of a K-edge is not a K-vertex, we should remove the assumption of corss-free for subgrpahs in $\mathcal{K}$.} 

An intersection support $\tilde{Q}$
for $(G,\mathcal{H},\mathcal{K})$ has the special $\mathcal{K}$-edge property if 
for each $\mathcal{K}$-edge $\{u,v\}\in E(G)$ such that neither $u$ nor $v$ is a $\mathcal{K}$-vertex, 
there is a subgraph in $\mathcal{H}_u$ that is adjacent to a subgraph in $\mathcal{H}_v$ in $\tilde{Q}$.

\begin{lemma}
\label{lem:conn}
Let $(G,\mathcal{H})$ be a cross-free intersection system with a set $\mathcal{K}$ of connected subgraphs of $G$.
If the graph $G$ does not contain a $\mathcal{K}$-vertex, then a dual support for $(G,\mathcal{H})$ with special edge property is an intersection support for $(G,\mathcal{H},\mathcal{K})$ with the special $\mathcal{K}$-edge property.
\end{lemma}
\begin{proof}
Let $Q^*$ denote a dual support for $(G,\mathcal{H})$ with the special edge property, as guaranteed by
Theorem \ref{thm:dualembedded}, i.e., for any vertex $v\in V(G)$, the subgraphs $\mathcal{H}_u$ induce a connected subgraph of $Q^*$, and 
for any edge $\{u,v\}\in E(G)$ such that $\mathcal{H}_u\neq\emptyset$, $\mathcal{H}_v\neq\emptyset$, and
$\mathcal{H}_{\{u,v\}}=\emptyset$, there is a subgraph $H\in\mathcal{H}_u$ and a subgraph $H'\in\mathcal{H}_v$
such that $H$ and $H'$ are adjacent in $Q^*$.
\hide{
By assumption, there is no $\mathcal{K}$-vertex in $G$. Therefore, for any $\mathcal{K}$-edge $e=\{u,v\}\in E(G)$, 
neither end-point of $e$ is a $\mathcal{K}$-vertex. 
If $e$ is not a special edge in $(G,\mathcal{H})$
then there is a subgraph $H\in\mathcal{H}_u\cap\mathcal{H}_v$.
If $e$ is a special edge in $(G,\mathcal{H})$, then by the assumption of special edge property as in Theorem \ref{thm:dualembedded}, 
there is an edge in $Q^*$ between a subgraph in $\mathcal{H}_u$ and a subgraph in $\mathcal{H}_v$.
In both cases, $Q^*$ satisfies the special $\mathcal{K}$-edge property for $(G,\mathcal{H,K})$.
}
The construction of the dual support for $(G,\mathcal{H})$ follows a sequence of vertex bypassing and edge-contraction 
operations. 
We modify the two operations slightly to account for the subgraphs in $\mathcal{K}$: 
When a vertex $v$ is bypassed, and $u_0,\ldots, u_{deg(v)-1}$
are the vertices created, we add the edges $\{u_i,v\}$ to each $K\in\mathcal{K}_v$,
we set $\mathcal{K}_{u_i}=\mathcal{K}_v$ for $i=0,\ldots, deg(v)-1$. In other words, the
subgraphs in $\mathcal{K}_v$ are modified to contain the sub-dividing vertices $u_i$ added on bypassing vertex $v$.
When an edge $\{u,v\}$ is contracted, we modify the subgraphs in $\mathcal{K}_u\cup\mathcal{K}_v$ to contain the
contracted vertex.

We show that $Q^*$ is an intersection support. For any $K\in\mathcal{K}$, let $H, H'\in\mathcal{H}_K$. 
Let $u\in H\cap K$ and $v\in H'\cap K$. Since $K$ is connected, there is a path $P=(u=u_0, u_1,\ldots, u_k=v)$
that lies in $K$. For any $i\in\{0,\ldots, k\}$, the subgraphs in $\mathcal{H}_{u_i}$ induce a connected subgraph in
$Q^*$, since $Q^*$ is a dual support. For any edge $\{u_i, u_{i+1}\}$, $i=0,\ldots, k-1$, either there is a subgraph $H\in\mathcal{H}$
that contains the edge $\{u_i,u_{i+1}\}$, or by the special edge property, there is a subgraph
in $\mathcal{H}_{u_i}$ that is adjacent to a subgraph in $\mathcal{H}_{u_{i+1}}$ in $Q^*$. Therefore, there is a path
in $Q^*$ between $H$ and $H'$ consisting only of subgraphs in $\mathcal{H}_K$. 
Since $K$ was arbitrary, $Q^*$ is an intersection support for $(G,\mathcal{H},\mathcal{K})$.

\end{proof}

If there is no $\mathcal{K}$-vertex in $G$, then Lemma \ref{lem:conn} guarantees the existence of an intersection
support of genus $g$. Otherwise, we first modify the arrangement so that no $\mathcal{K}$-vertex is maximal, and then
we add a \emph{dummy} subgraph corresponding to each $\mathcal{K}$-vertex so that the resulting system now
satisfies the conditions of Lemma \ref{lem:conn}. 
Let $\mathcal{F}$ denote
the set of dummy subgraphs added. We obtain a dual support on $\mathcal{H}\cup\mathcal{F}$ using Lemma \ref{lem:conn},
and finally obtain a support on just $\mathcal{H}$ by  removing the dummy subgraphs and carefully modifying the
underlying support graph. 


For a $\mathcal{K}$-vertex $v$, if an edge $e\sim v$ is such that $\mathcal{K}_v=\mathcal{K}_e$,
we say that $e$ is \emph{full} for $v$. If a $\mathcal{K}$-vertex does not have a full
edge incident on it, then we say that it is maximal. In this case, 
$\mathcal{K}_e \subset \mathcal{K}_v $ for all ${e}\sim v$.
In the following, we repeatedly apply vertex bypassing to a maximal $\mathcal{K}$-vertex of
maximum depth until no $\mathcal{K}$-vertex is maximal. 
Note that a maximum depth $\mathcal{K}$-vertex need not be maximal.

\begin{lemma}
\label{lem:removeMaximal}
Let $(G,\mathcal{H},\mathcal{K})$ be a cross-free intersection system of genus $g$. 
Then, we can modify the arrangement 
to a cross-free arrangement $(G',\mathcal{H},\mathcal{K}')$
so that $G'$ has genus $g$, no $\mathcal{K}$-vertex of $G'$ is maximal, and 
a support $Q'$ for $(G',\mathcal{H}, \mathcal{K}')$ is a support for 
$(G,\mathcal{H}, \mathcal{K})$.
\end{lemma}
\begin{proof}
We assume that a cross-free embedding of $(G,\mathcal{H},\mathcal{K})$ is given, and with a slight abuse of notation, we use $G$ to also refer
to the embedded graph.
Let $d$ denote the maximum depth of a maximal $\mathcal{K}$-vertex, and
let $n_d$ denote the number of maximal $\mathcal{K}$-vertices of depth $d$.
We repeatedly choose a maximal $\mathcal{K}$-vertex $v$ of maximum depth and apply \VB$(v)$. The operation
\VB$(v)$ modifies the graph $G$ and the subgraphs in $\mathcal{K}$, but does not modify any subgraph in
$\mathcal{H}$, as $v$ is a $\mathcal{K}$-vertex. Let $v$ be a $\mathcal{K}$-vertex to which we apply $\VB(v)$.
Let $u_0,\ldots, u_{deg(v)-1}$ be the new vertices added corresponding to the edges $\{v,v_i\}$ in $G$ for $0\le i\le deg(v)-1$.
The vertices
$u_i$, $i=0,\ldots, deg(v)-1$ are also $\mathcal{K}$-vertices.
Since the edge $\{u_i, v_i\}$ is full
for $u_i$, $u_i$ is not maximal.
Hence, none of the newly added vertices are maximal $\mathcal{K}$-vertices.

Let $(G',\mathcal{H},\mathcal{K}')$ denote
the new arrangement. Let $d'$ and $n'_{d'}$ denote respectively, the maximum depth of a maximal $\mathcal{K}$-vertex,
and the number of maximal $\mathcal{K}$-vertices in $(G',\mathcal{H},\mathcal{K}')$ of depth $d'$. 
It follows that $(d', n'_{d'})$ is lexicographically 
smaller than $(d, n_d)$ and hence the process eventually stops.
Since the newly added vertices and edges are not contained in any subgraph $H\in\mathcal{H}$,
it follows that $\mathcal{H}$ remains cross-free in $G'$.
Further, the fact that subgraphs in $\mathcal{K}'$ remain connected and cross-free follows from Lemmas \ref{lem:keylem} and \ref{lem:remcrossfree}.
The fact that $G'$ has genus $g$ follows from the fact that the operation of vertex bypassing preserves the embedding of the resulting graph on the same surface as that of the original graph.
Thus, 
$(G',\mathcal{H}, \mathcal{K}')$ is cross-free.
Since the underlying hypergraph $(\mathcal{H},\{\mathcal{H}_K\}_{K\in\mathcal{K}})$ is 
not modified, a support $Q'$ for $(G',\mathcal{H},\mathcal{K}')$ is also a support for $(G,\mathcal{H},\mathcal{K})$.
\end{proof}

\begin{lemma}
\label{lem:dummy}
Let $(G,\mathcal{H},\mathcal{K})$ be a cross-free intersection system of genus $g$ such that no $\mathcal{K}$-vertex
is maximal.
Then, we can add a collection of \emph{dummy subgraphs} $\mathcal{F}$ such that there are no $\mathcal{K}$-vertices in the resulting
system, $\mathcal{H}\cup\mathcal{F}$
remains cross-free, and a support $\tilde{Q}$ for $(G,\mathcal{H},\mathcal{K})$ can be obtained from 
the dual support $Q^*$ for the system $(G,\mathcal{H}\cup\mathcal{F})$, such that $\tilde{Q}$ has the same
genus as $Q^*$.
\end{lemma}
\begin{proof}
We assume a cross-free embedding of $G$ is given,
i.e., an embedding where both $\mathcal{H}$ and $\mathcal{K}$ are simultaneously cross-free.
By Lemma \ref{lem:removeMaximal}, we can assume that in $G$, no $\mathcal{K}$-vertex is maximal.

At each $\mathcal{K}$-vertex $u\in G$ we add a dummy subgraph $F_u$ containing $u$. 
Let $\mathcal{F}$ denote the set of dummy subgraphs added, and let $\mathcal{H}'=\mathcal{H}\cup\mathcal{F}$.
The graph system $(G,\mathcal{H}',\mathcal{K})$ does not contain a $\mathcal{K}$-vertex. Hence, there is
a dual support $Q^*$ with the special-edge property for $(G,\mathcal{H}')$ obtained by Lemma \ref{lem:conn}, which is an intersection
support for $(G,\mathcal{H}',\mathcal{K})$ that satisfies the special $\mathcal{K}$-edge property.
That is, for each $K\in\mathcal{K}$, the subgraphs in $\mathcal{H}'_K$ induce a connected subgraph
in $Q^*$. In other words, if $x_H$ is the vertex corresponding to the subgraph $H\in\mathcal{H}'$,
then
for each $K\in\mathcal{K}$, the set of vertices $\{x_{H}: H\in\mathcal{H}'\mbox{ and } H\cap K\neq\emptyset\}$
induce a connected subgraph in $Q^*$. We annotate each vertex $x_H$ in $Q^*$ with the set of subgraphs
in $\mathcal{K}$ that intersect $H$. This defines a new graph system $(Q^*,\mathcal{K}^*)$, where
for each $K\in\mathcal{K}$, we associate an induced subgraph $K^*\in\mathcal{K}^*$ consisting of the set of
vertices corresponding to $H\in\mathcal{H}'$ that intersect $K$. Note that since $Q^*$ is a dual support, each
$K^*\in\mathcal{K}^*$ induces a connected subgraph.
We let $\mathcal{K}_{x_H}$ to denote the set of subgraphs of $\mathcal{K}$ that
intersect $H$. 

In the graph system $(G,\mathcal{H}')$, each $\mathcal{K}$-vertex $u$ has depth 1 because it is covered only by the
dummy subgraph $F_u$. For
a $\mathcal{K}$-vertex $u$, if $x_u$ is the vertex corresponding to $F_u$ in $Q^*$, 
$\mathcal{K}_{x_u}=\mathcal{K}_u$.
By Lemma \ref{lem:removeMaximal}, $u$ is not maximal,
and therefore has a full edge $\{u,v\}$ incident to it.
Since $Q^*$ satisfies the special $\mathcal{K}$-edge
property for $(G,\mathcal{H}',\mathcal{K})$, 
it follows that $x_{u}$ is adjacent to a vertex $y$ in $Q^*$ corresponding to a subgraph in $\mathcal{H}'$ containing $v$, and
since $\{u,v\}$ is a full edge, it follows that the set of $\mathcal{K}$-subgraphs at $x_{u}$ is a subset of the set of $\mathcal{K}$-subgraphs
at $y$. That is, the vertex $x_u$ is not maximal with respect to the subgraphs $\mathcal{K}$.

We now color each vertex in $Q^*$ corresponding to a dummy subgraph to a red vertex, and a vertex corresponding
to a subgraph in $\mathcal{H}$ to a blue vertex. 
Since no red vertex in $Q^*$ is maximal, and
each $K\in\mathcal{K}$ induces a connected subgraph in $Q^*$, by Lemma \ref{lem:easypsupport},
there is a support $\tilde{Q}$ on the vertices corresponding to $\mathcal{H}$ that is connected for each $K\in\mathcal{K}$.
It follows that $\tilde{Q}$ is the desired support.
\end{proof}

We are now ready to prove the main theorem of this section.

\begin{theorem}
\label{thm:intsupport}
Let $(G,\mathcal{H},\mathcal{K})$ be a cross-free intersection system of genus $g$. 
Then, there exists an intersection support $\tilde{Q}$ on $\mathcal{H}$ of genus at most $g$.
\end{theorem}
\begin{proof}
If for each $K\in\mathcal{K}$, the intersection graph $G_K=(\mathcal{H}_K, E_K)$ is connected, from Lemma \ref{lem:conn},
we obtain a support of genus at most $g$. 
Otherwise, by Lemma \ref{lem:removeMaximal}, we obtain a cross-free system $(G',\mathcal{H},\mathcal{K}')$ 
such that no $\mathcal{K}$-vertex is maximal, 
and a support for $(G',\mathcal{H},\mathcal{K}')$ is a support for the original system. 
Finally, by Lemma \ref{lem:dummy}, we obtain
a support for $(G',\mathcal{H},\mathcal{K}')$, and thus a support for $(G,\mathcal{H},\mathcal{K})$, and the resulting support has
genus at most $g$.
\end{proof}


\section{Outerplanar Graphs}
\label{sec:outerplanar}
In this section, we consider the case when $G$ is outerplanar.
We assume an outerplanar embedding of
$G$ in the plane with $C$ denoting the outer face in the embedding.

\begin{theorem}
\label{thm:primalOuter}
Let $(G,\mathcal{H})$ be an outerplanar cross-free system, with $c:V\to\{\R,\B\}$.
Then, there is a support $Q$ on $\B(V)$ i.e. $Q[\B(H)]$ is connected for each 
$H\in\mathcal{H}$.
If a cross-free embedding is given, then an outerplanar support can be computed
in time polynomial in $|V(G)|$ and $|\mathcal{H}|$.
\end{theorem}
\begin{proof}
If $\R(V)=\emptyset$, $G$ itself is the desired support. 
Otherwise, let $C'$ be a cycle on $\B(V)$ in the same order as in the outer face of $G$. 
Wlog, let each $H\in\mathcal{H}$ induce a disjoint collection of runs on $C'$. It is easy to see that the collection of induced subgraphs $\{H\cap C'\}_{H\in\mathcal{H}}$ is \emph{abab-free} on $C'$.
By Lemma \ref{lem:nblock}, there is a collection of non-intersecting chords $D$ connecting all
the runs of $H$ for each subgraph $H\in\mathcal{H}$. Then, $Q=C'\cup D$ is 
the desired support.

Lemma \ref{lem:nblock} also yields a polynomial time algorithm to add a non-blocking diagonal.
For a fixed subgraph $H\in\mathcal{H}$, we try adding one of at most $\binom{n}{2}$ diagonals,
where $n=|V(G)|$. For each
choice, to check that it is non-blocking, we take $O(|\mathcal{H}|)$ time to check if the given diagonal
blocks a subgraph. Hence, we find a non-blocking diagonal in $O(n^2|\mathcal{H}|)$. A maximal outerplanar
graph has $n-3$ diagonals, and therefore the running time is $O(n^3|\mathcal{H}|)$.
\end{proof}

We show example of an outerplanar cross-free system $(G,\mathcal{H})$ that does not admit an outerplanar dual support. 
Let $G$ be a graph with vertex set $\{1,2,\ldots 6\}$ and edges $\{1,2\},\{2,3\},\{3,4\},\{4,5\},\{5,6\},\{6,1\},\{2,4\},\{2,6\}$ and $\{4,6\}$.
$G$ is called an \emph{asteroidal triple graph} as shown in Figure \ref{fig:asteroidal}.
Let $\mathcal{H}$ be a family of cross-free subgraphs induced by the vertex sets $\{1,2,3\}$, $\{3,4,5\}$, $\{5,6,1\}$ and $\{2,4,6\}$,
as in the Figure \ref{fig:asteroidal}. 
The support for the dual hypergraph is $K_4$ which is not outerplanar. 
A natural question is the following: If $(G,\mathcal{H})$ is a non-piercing system, and $G$ is a tree, is there
a support that is a tree? We show that the answer to this question is negative in both the primal and dual settings.
For the primal setting, consider the graph $K_{1,3}$ with $v$ being the central vertex colored red, 
and leaves $v_0,v_1,v_2$ colored blue. We put three subgraphs $H_0, H_1, H_2$, where $H_i = \{v_i, v, v_{i+1}\mod 3\}$.
It is easy to see that the primal support is a triangle. The same example without colors on the vertices shows that
the dual support is also a triangle which is not a tree.

We now show that if $\mathcal{H}$ 
is non-piercing and $G$ is an outerplanar graph, then $(G,\mathcal{H})$ admits an outerplanar dual support.
We start with the following definition:

\begin{definition}[$axax$-free]
\label{defn:axax}
Let $(C,\mathcal{H})$ be a graph system where $C$ is a cycle and $\mathcal{H}$ is a collection
of (not necessarily connected) subgraphs of $C$. Then, $(C,\mathcal{H})$ is $axax$-free if for
any two subgraphs $H, H'\in\mathcal{H}$, 
there are no four vertices $a_1,x_1, a_2, x_2$ in cyclic order 
around $C$ such that 
$a_1, a_2\in H\setminus H'$ and $x_1, x_2\in H'$.
\end{definition}

\begin{lemma}
\label{lem:nblockop}
Let $(G,\mathcal{H})$ be an embedded non-piercing outerplanar graph system.
Then, $(C,\mathcal{H})$ is $axax$-free, where $C$ is the cycle defining the outer face of $G$.
\end{lemma}
\begin{proof}
Suppose there exist $a_1, x_1, a_2, x_2$ in cyclic order around $C$ in the outerplanar embedding of $G$ so that $a_1, a_2\in H\setminus H'$ and
$x_1, x_2\in H'$. As
$a_1$ and $a_2$ are not consecutive along $C$, therefore, $x_1$ and $x_2$ lie in distinct arcs of $C$ defined
by $a_1$ and $a_2$. Since $(G,\mathcal{H})$ is a non-piercing system, $H\setminus H'$ is connected.
This implies
there is a path $P$ in $H\setminus H'$ between $a_1$ and $a_2$. 
But, then any path $P'$ in $H'\setminus H$ between $x_1$ and $x_2$ should cross $P$.
This contradicts the fact that $G$ is embedded as outerplanar graph in the plane. See Figure \ref{fig:axax}.


\hide{
We use the notation in Definition \ref{defn:axax}. Since $(G,\mathcal{H})$ is a non-piercing system,
$H'$ is connected. Therefore, $G$ contains a path between $x_1$ and $x_2$ that lies in $H'$.
Consider any path $P'$ in $H'$ between $x_1$ and $x_2$. Since $a_1, a_2\not\in H'$, 
$P'$ does not contain $a_1$ or $a_2$. Since $H$ is connected, there is a path
between $a_1$ and $a_2$ that lies in $H$.
By the Jordan curve theorem, any path $P$ in $H$ between $a_1$ and $a_2$ intersects
any path $P'$ in $H'$ between $x_1$ and $x_2$ as shown in Figure \ref{fig:axax}. 
Since $G$ is outerplanar, $P$ and $P'$ intersect at a vertex of $G$.
But, this implies that $P$ does not lie in $H\setminus H'$, or equivalently, that $H\setminus H'$ is
not connected, violating the assumption that the subgraphs are non-piercing.
}
\end{proof}

\begin{corollary}
\label{cor:axax}
Let $(G,\mathcal{H})$ be a non-piercing outerplanar system. Then, 
for any $H\in\mathcal{H}$, any chord $d$ whose end-points are in $H$ is non-blocking. 
\end{corollary}
\begin{proof}
This follows directly from Lemma \ref{lem:nblockop}.
\end{proof}

\begin{figure}[ht!]

    \centering
    \begin{subfigure}{0.4\textwidth}
    \vspace{0.31cm}
        \includegraphics[scale=.5]{asteroidal}\vspace{0.5cm}
\caption{Cross-free Asteroidal with the
 only dual support $K_4$.\vspace{1.4cm}}        \label{fig:asteroidal}
 \end{subfigure}
 \hfill
    \begin{subfigure}{0.4\textwidth}\vspace{1.5cm}
        \includegraphics[scale=.5]{outerplanardual}\vspace{0.5cm}
        \caption{Crossing between $P$ and $P'$ caused by $a_1,x_1,a_2,x_2$ sequence in outerplanar graph. In the figure, $a$ and $x$ are some vertices of $H\setminus H'$ and $H'$ respectively.}
        \label{fig:axax}
    \end{subfigure}
     \caption{}

\end{figure}

 Now we can obtain the result for the dual support.
 
\begin{theorem}
\label{thm:outerplanardual}
Let $(G,\mathcal{H})$ be a non-piercing outerplanar system.
Then, there is an outerplanar dual support $Q^*$ on $\mathcal{H}$. 
Further, an outerplanar dual support can be computed in time polynomial in $|\mathcal{H}|$ and $|V(G)|$.
\end{theorem}
\begin{proof}  
By Proposition \ref{prop:nocontainment}, we can assume there is no containment in $\mathcal{H}$.
By Lemma \ref{lem:nblockop}, $(C,\mathcal{H})$ is $axax$-free.
For $H\in\mathcal{H}$, let $n_H$ denote the number of runs of $H$
on $C$, and let
$N(C,\mathcal{H})=\sum_{H\in\mathcal{H}} (n_H-1)$. 
We prove by induction on $N(C,\mathcal{H})$ that if $(C,\mathcal{H})$ is $axax$-free, there is an outerplanar
support on $\mathcal{H}$. 
 
If $N(C,\mathcal{H})=0$, each subgraph consists of a single run. We claim that  
a cycle on $\mathcal{H}$ yields a support:
Let $v_1,\ldots, v_n$ be the cyclic order of vertices
on $C$. Traversing $C$ in clockwise order, we obtain a cyclic order 
on the subgraphs ordered on the last vertex of their run. 
Let $Q^*$ be the cycle on $\mathcal{H}$ in this order.
Since we assumed there is no containment in $\mathcal{H}$,
for any $v\in V(C)$, the subgraphs in $\mathcal{H}_v$ appear consecutively in $Q^*$, and
thus induce a connected subgraph of $Q^*$. Hence, $Q^*$ is a support. 

Suppose for any cycle $C'$ and subgraphs $\mathcal{H}'$ such that $(C',\mathcal{H}')$ is $axax$-free
and $N(C',\mathcal{H}')<N$,
there is an outerplanar support on $\mathcal{H}'$.
Consider $(C,\mathcal{H})$ with $N(C,\mathcal{H}) = N$. 
For $H\in\mathcal{H}$ with $n_H > 1$, a chord $d$ is a \emph{good chord}
if it connects the last vertex of a run of $H$ with the first vertex of the next run of $H$
along $C$.
Its length $\ell(d)$ is
the number of vertices along $C$ between its end-points. 
A good chord of minimum length, denoted $d_H$ is the
\emph{critical chord} of $H$. 

Let $H = \arg\min_{H\in\mathcal{H}} \ell(d_H)$, breaking ties arbitrarily.
Let $d_H=\{u_1, u_2\}$. $d_H$ partitions $C$ into two open arcs $\alpha_1=(u_1, u_2)$,
and $\alpha_2=(u_2, u_1)$. Since $d_H$ is a good chord of $H$, 
either $\alpha_1\cap H=\emptyset$, or $\alpha_2\cap H=\emptyset$. Assume the former. 
We obtain two induced sub-problems
on $C_R=\alpha_1\cup d_H$ and $C_L=\alpha_2\cup d_H$. Since $(G,\mathcal{H})$ is non-piercing,
it follows by Lemma \ref{lem:nblockop} that $(G,\mathcal{H})$ is $axax$-free. Therefore, 
by Corollary \ref{cor:axax}, $d_H$ is non-blocking. A subgraph $H'\in\mathcal{H}$ such that 
there is a vertex $v\in \alpha_1\cap H'$ is said to 
\emph{appear} in $C_R$. Since $(G,\mathcal{H})$ is non-piercing, %and $G$ is outerplanar, it implies
%by Lemma \ref{lem:axax} that $(G,\mathcal{H})$ is $axax$-free. Therefore, by Corollary \ref{cor:axax}, 
%it follows that $d_H$ is non-blocking. 
%
%from Lemma \ref{lem:nblockop} and Corollary \ref{cor:axax} that
it implies that 
if $H'$ appears in $C_R$, then $(H'\cap\alpha_2)\subseteq (H\cap\alpha_2)$ and if $(H'\cap\alpha_2)\ne\emptyset$, then $H'$ contains $u_1$ or $u_2$.
Thus, in the sub-problem induced on $C_L$, we can remove any $H'$ that appears in $C_R$.
Since $d_H$ joins two disjoint runs of $H$, $N(C_L, \mathcal{H}_{C_L}) < N$, where
$\mathcal{H}_{C_L}$ are the subgraphs $H\cap C_L$ for $H\in\mathcal{H}$ with containments removed. By the inductive
hypothesis, there is a support $Q_L$ on $\mathcal{H}_{C_L}$.

Now, consider the induced sub-problem on $C_R$. By the minimality of $d_H$, each subgraph
contributes at most one run to the outer face $C_R$. By the base case of the induction
hypothesis, there is a support $Q'$ on $\mathcal{H}_{C_R}$ that is a cycle, where
$\mathcal{H}_{C_R}$ are the subgraphs $H\cap C_R$ for $H\in\mathcal{H}$.
Since the original system did not have any containments, it follows
that each subgraph in $\mathcal{H}\setminus H$ is in $\mathcal{H}_{C_R}$ or 
$\mathcal{H}_{C_L}$.

We obtain a graph $Q_R$ from the support $Q'$ of $\mathcal{H}_{C_R}$ by adding a chord from $H$ to each $H'\in \mathcal{H}_{C_R}$.
By construction, $\mathcal{H}_{C_R}\cap\mathcal{H}_{C_L}=\{H\}$. We obtain the desired
support $Q^*$ by identifying $H$ in $Q_L$ and $Q_R$. It follows that $Q^*$ is outerplanar.
Let $v$ be a vertex in $C_L$ that is contained in a subgraph
$H'$ that appears in $C_R$. Since $(H'\cap\alpha_2)\subseteq(H\cap\alpha_2)$, it follows
from the induction hypothesis and the fact that $H$ and $H'$ are adjacent in $Q_R$ that
$Q^*[\mathcal{H}_v]$ is connected. Finally, by Lemma \ref{lem:nblockop},
for any subgraph $H'\in\mathcal{H}\setminus H$ having a vertex in $\alpha_1$ and $\alpha_2$, contains $u_1$ or $u_2$, 
and thus is adjacent to $H$ in $Q_R$. The theorem follows.

Finding a subgraph with a critical chord of minimal length can be done in $O(|V(G)||\mathcal{H}|)$ time.
Since the two sub-problems are smaller, the overall running time is upper bounded by $O(|V(G)|^2|\mathcal{H}|)$.
\end{proof}

\section{Graphs of Bounded Treewidth}
\label{sec:treewidth}
In this section, we show that if $(G,\mathcal{H})$ is a non-piercing system, then there exist both a primal and dual support of treewidth $O(2^{tw(G)})$. Further, the supports can be computed in polynomial time if $G$ has bounded treewidth, i.e., the algorithm is FPT in the treewidth of $G$.


\subsection{Basic tools for bounded treewidth graphs}
Let $G$ be a graph of treewidth $t$ and $\mathcal{H}$ be a collection of connected non-piercing subgraphs of $G$.
Throughout this section, we use $(T,\mathcal{B})$ to denote a tree decomposition, where
we assume without loss of generality that $T$ is a binary tree rooted
at a node $\rho$.
 
Let $CC(G)$ denote the \emph{chordal completion}
of $G$, i.e., for each bag $B\in\mathcal{B}$, we add edges between non-adjacent vertices such that each bag induces a complete subgraph. 
It is well-known that a chordal completion does not increase the treewidth of the underlying graph. 
It is easy to check that the subgraphs in $\mathcal{H}$ remain non-piercing in $CC(G)$ if they were non-piercing in $G$. 
Further, in both the primal and dual settings, 
a support for the subgraphs defined on $CC(G)$ is also a support for the subgraphs defined on $G$.
Therefore, we assume without loss of generality in this section that $G$ is a chordal graph of treewidth $t$. In other words,
the tree-decomposition of $G$ is \emph{complete}.

We use the following notation in this section: for a node $u$ of $T$,
let $T_u$ denote the sub-tree rooted at $u$, and let $G_u$ denote the subgraph induced by the union of bags associated with nodes in $T_u$ and
let $G'_u$ denote the graph induced by the union of bags in $T\setminus T_u$.
Let $A_{uv}=B_u\cap B_v$ denote the adhesion set between the bag $B_u$ at $u$ and the bag $B_v$ at its parent $v$
in $T$. $A_{uv}$ is a separator of $G$, and $G\setminus A_{uv}$ yields two disjoint induced subgraphs: $G_u\setminus A_{uv}$ and $G'_u\setminus A_{uv}$. 
Let $\mathcal{H}_{A_{uv}}=\{H\in\mathcal{H}: H\cap A_{uv}\neq\emptyset\}$. 

\begin{lemma}
\label{lem:sepnp}
Let $(G,\mathcal{H})$ be a non-piercing system with tree-decomposition $(T,\mathcal{B})$ of $G$. Then,
for any adhesion set $A_{uv}=B_u\cap B_v$, $(G_u,\mathcal{H}_u)$ and 
$(G'_u,\mathcal{H}'_u)$ are non-piercing systems, where 
$\mathcal{H}_u = \{H\cap G_u\neq\emptyset: H\in\mathcal{H}\}$ 
and $\mathcal{H}'_{u} = \{H\cap G'_u\neq\emptyset: H\in\mathcal{H}\}$.
\end{lemma}
\begin{proof}
We show that $\mathcal{H}_u$ is a collection of non-piercing subgraphs. An identical argument shows that $\mathcal{H}'_u$ is also a collection of non-piercing subgraphs. 
Let $H'_1$ and $H'_2$ be two arbitrary subgraphs in $\mathcal{H}_u$ corresponding respectively, to subgraphs $H_1$ and $H_2$ in $\mathcal{H}$. Since $\mathcal{H}$
is a non-piercing family, 
it follows that $H_1\setminus H_2$ and $H_2\setminus H_1$ are connected subgraphs of $G$. 
If $H_1\setminus H_2$ does
not intersect $A_{uv}$, then since $H_1\setminus H_2$ is connected and $A_{uv}$ is a separator in $G$, $H_1\setminus H_2$ lies entirely in $G_u$ or $G'_u$.
In this case, $H'_1\setminus H'_2=H_1\setminus H_2$ and hence connected.

Otherwise, let $H_1\setminus H_2$ intersect $A_{uv}$ at a vertex set $S$.
By assumption, $G$ is a chordal graph, and
hence $A_{uv}$ is a complete subgraph. Since $H'_1\setminus H'_2$ contains $S$, it follows that $H'_1\setminus H'_2$ is connected.
\end{proof}

We next show the following lemma that will be crucial for the construction of both the primal support and the dual support. For two sets $A$ and $B$ on the same ground set, we say that $A$ and $B$ properly intersect if $A\setminus B\neq\emptyset$ and $B\setminus A\neq\emptyset$.

\begin{lemma}
\label{lem:npsep}
Let $(G,\mathcal{H})$ be a non-piercing system with $(T,\mathcal{B})$,
a tree-decomposition of $G$.
Let $A_{uv}=B_u\cap B_v$ be an adhesion set corresponding to edge $e=(u,v)\in E(T)$ and 
let $H, H'\in\mathcal{H}_{A_{uv}}$. Then
\begin{enumerate}
    \item If $(H\cap G_u)\subset (H'\cap G_u)$ and $H\cap A_{uv} = H'\cap A_{uv}$, then $(H'\cap G'_u)\subseteq (H\cap G'_u)$.
    \item If $H\cap G_u$ and $H'\cap G_u$ properly intersect, and $H\cap A_{uv} = H'\cap A_{uv}$, then $H\cap G'_u = H'\cap G'_u$. \item If $H\cap G_u$ and $H'\cap G_u$ properly intersect and $(H\cap A_{uv}) \subset (H'\cap A_{uv})$, then $(H\cap G'_u) \subseteq (H'\cap G'_u)$.
 \end{enumerate}
\end{lemma}
\begin{proof}
\begin{enumerate}
    \item Suppose $(H\cap G_u)\subset (H'\cap G_u)$. Let $x\in (H'\setminus H)\cap G_u$.
If $\exists\;y\in (H'\setminus H)\cap G'_u$ then $H'\setminus H$ has $x$ and $y$ in two different components since $H\cap A_{uv} = H'\cap A_{uv}$ forms a separator in $H'$. This contradicts the fact that $H$ and $H'$ are non-piercing.

\item Since $H\cap G_u$ and $H'\cap G_u$ intersect properly, there exists $h\in (H\setminus H')\cap G_u$.
%
%Let $h\in (H\setminus H')\cap G_u$ and $h'\in(H'\setminus H)\cap G_u$ which exists since $H\cap G_u$ and $H'\cap G_u$ intersect properly.
If $\exists y\in (H\setminus H')\cap G'_u$, then $h$ and $y$ are not connected in $H\setminus H'$ since $H\cap A_{uv} = H'\cap A_{uv}$
separates $H\setminus H'$ into two components. Hence, $(H\cap G'_u)\subseteq (H'\cap G'_u)$. 
A symmetric argument shows $(H'\cap G'_u)\subseteq (H\cap G'_u)$.
Hence $(H'\cap G'_u)=(H\cap G'_u)$.

\item We have $(H\setminus H')\cap G_u\ne\emptyset\ne(H'\setminus H)\cap G_u$ since $H\cap G_u$ and $H'\cap G_u$ intersect properly.
Given that $(H\cap A_{uv}) \subset (H'\cap A_{uv})$, if $(H\setminus H')\cap G'_u\ne\emptyset$, then $H'\cap A_{uv}$ forms a separator for $H$; the result follows.

\end{enumerate}
\end{proof}



\subsection{Primal Support}
\label{sec:primal}
Let $(G,\mathcal{H})$ be a non-piercing system, and let 
$c:V(G)\to\{\R,\B\}$. We show that there is a primal support $Q$ on $\B(V)$ of 
treewidth $O(2^{tw(G)})$. The proof is algorithmic and yields a polynomial time algorithm if $tw(G)$ is bounded.
In other words, the algorithm is FPT in the tree-width of the graph.

Suppose the tree-decomposition $(T,\mathcal{B})$ of $G$ enjoys the additional property that for each adhesion set $A$, 
$\B(A)\neq\emptyset$, and for
each subgraph $H\in\mathcal{H}_A$, we have that $H\cap \B(A)\neq\emptyset$, i.e., for each adhesion set $A$ and each subgraph $H$ intersecting $A$,
$H$ intersects $A$ in a blue vertex. Then, $(T,\mathcal{B})$ is said to be an \emph{easy} tree-decomposition. 
If $(T,\mathcal{B})$ is an easy tree-decomposition, it is straightforward to obtain the desired support $Q$, and in fact $tw(Q)\le tw(G)$. 

\begin{lemma}
\label{lem:easytd}
Let $(G,\mathcal{H})$ be a non-piercing system with $c:V(G)\to\{\R,\B\}$. Let $(T,\mathcal{B})$ be an easy tree-decomposition of
width $t$. Then, there is a support $Q$ on $\B(V)$ of treewidth at most $t$.
\end{lemma}
\begin{proof}
Given $(T,\mathcal{B})$, we obtain a tree-decomposition for the support $Q$ on $\B(V)$ by removing vertices of $\R(V)$ from
each bag $B\in\mathcal{B}$. To see that $Q$ is a support, consider a subgraph $H\in\mathcal{H}$. Note that
$H\cap \B(B)$ is connected as each $B\in\mathcal{B}$ induces a complete graph and $(T,\mathcal{B})$ is an easy tree decomposition.
\end{proof}

If $(T,\mathcal{B})$ is not an easy tree-decomposition, we modify it to obtain an easy
tree-decomposition of width $O(2^t)$, where $t$ is the width of the tree-decomposition $(T,\mathcal{B})$. 
We then obtain a support by applying Lemma \ref{lem:easytd}.


Let $T$ be rooted at $\rho$. The algorithm to modify $(T,\mathcal{B})$ into an easy tree-decomposition 
consists of two phases: In the first phase, we go bottom-up adding carefully chosen vertices of $\B(V)$
to the bags such that the resulting structure is a valid tree-decomposition. At the end of the first phase, for each adhesion set $A$, only a subset of the subgraphs
intersecting $A$ do so at a blue vertex. In the second phase, we go top-down from $\rho$, again adding carefully chosen vertices of $\B(V)$. At the end of the second phase,
we end up with an easy tree-decomposition of width at most $3\cdot 2^t$. 

Let $e=(u,v)$ be an edge in $T$ where $v$ is a parent of $u$. 
Consider a non-empty set $S\subseteq A_e$ such that
$S\cap \B(V)=\emptyset$. We define $\mathcal{H}'_S = \{H\in\mathcal{H}_A: H\cap A = S,\; H\cap \B(G_u)\neq\emptyset \; \mbox{ and }\; H\cap \B(G'_u)\neq\emptyset\}$. We want to 
add vertices in $\B(V)$ to $A$ so that the subgraphs in $\mathcal{H}'_S$ intersect $A$ at a blue vertex.
In the rest of this section, we make the following
assumptions: when we consider an adhesion set $A_{uv}$ corresponding to edge $(u,v)$ of $T$, we assume that $v$ is the parent of $u$. 
When we consider subsets $S$ of an 
adhesion set $A$, we implicitly assume that $\mathcal{H}'_S\neq\emptyset$. Further, we use $\mathcal{M}_S\subseteq\mathcal{H}'_S$ to denote
the set of minimal subgraphs in the containment order $\preceq_{G_u}$ defined on $\{H\cap G_u: H\in \mathcal{H}'_S\}$, i.e., for $H, H'\in\mathcal{H}'_S$, 
$H\preceq_{G_u} H' \Leftrightarrow (H\cap G_u)\subseteq (H'\cap G_u)$.  We use $(\preceq_{G_u}, \mathcal{H}'_S)$ to denote this partial order.

For a tree-decomposition $(T,\mathcal{B})$ of $(G,\mathcal{H})$, 
we say that a tree-decomposition $(T,\mathcal{B}'')$ for $(G,\mathcal{H})$ satisfies the \emph{bottom-up property with respect to} $(T,\mathcal{B})$ at an adhesion set $A_{uv}$ if $\forall S\subseteq A_{uv}$, 
$\exists H\in\mathcal{M}_S$ such that $H\cap \B(A''_{uv})\neq\emptyset$, 
where $A''_{uv}$ is the adhesion set in $(T,\mathcal{B}'')$ corresponding to $A_{uv}$. 
$(T,\mathcal{B}'')$ satisfies the bottom-up property with respect to $(T,\mathcal{B})$ if it
satisfies the bottom-up property at each
adhesion set in $(T,\mathcal{B})$.

\begin{lemma}
\label{lem:bottomup}
Let $(T,\mathcal{B})$ be a tree-decomposition of width $t$ of a non-piercing system $(G,\mathcal{H})$. 
Then, there is a tree-decomposition $(T,\mathcal{B}'')$ of width at most $2\cdot 2^t$ 
that satisfies the bottom-up property with respect to $(T,\mathcal{B})$.
\end{lemma}
\begin{proof}
To construct $(T,\mathcal{B}'')$ we proceed bottom-up from the leaves of $T$.
If an adhesion set $A_{uv}=B_u\cap B_v$ satisfies the following condition:
\begin{itemize}
\item[$(*)$] For each $S\subseteq A_{uv}$, $\exists H\in\mathcal{M}_S$ such that $H\cap \B(B_u)\neq\emptyset$
\end{itemize}

Then, adding $\beta\in H\cap \B(B_u)$ to $B_v$ for each $S\subseteq A_{uv}$ ensures that the bottom-up property is satisfied in the resulting adhesion set $A''_{uv}$. Since $T$ is a binary tree, it follows that we at most $2\cdot 2^t$ blue vertices to 
each bag in this process.

We say that
$B_v$ satisfies the bottom-up property if each adhesion set $A_{wv}$ satisfies the bottom-up property for each child $w$ of $v$ in $T$.

By definition of $\mathcal{H}'_S$, condition $(*)$ is satisfied for each adhesion set $A_{uv}$ where
$u$ is a leaf of $T$. For a node $u$ with children $x$ and $y$, and parent $v$, we claim that if 
$A_{xu}$ and $A_{yu}$ satisfy the bottom-up property, then the adhesion set $B''_u\cap B_v$ satisfies the property $(*)$. 
This is sufficient to prove the lemma as we can process the adhesion sets bottom-up. 

So, suppose the bottom-up property is satisfied at $A_{xu}$ and $A_{yu}$. 
Let $S\subseteq B''_u\cap B_v$.
If there is a subgraph $H$ in $\mathcal{M}_S$ such that $H\cap A_{xu}=\emptyset$ and
$H\cap A_{yu}=\emptyset$, then, $\B(H)\cap B''_u\neq\emptyset$ by definition of $\mathcal{H}'_S$.
So, we can assume that $H\cap A_{xu}\neq\emptyset$ or $H\cap A_{yu}\neq\emptyset$ for each 
$H\in\mathcal{M}_S$.
Assume wlog the former holds for some $H\in\mathcal{M}_S$. 
%$\exists H\in\mathcal{M}_S$ such that 
Let $H\cap A_{xu}=S'$. Since $A''_{xu}$ satisfies the bottom-up property, 
there is a subgraph $H'\in\mathcal{M}_{S'}$ such that $H'\cap A_{xu}=S'$ and
$H'\cap \B(A''_{xu})\neq\emptyset$. 
Since $H'\in\mathcal{M}_{S'}$, it follows that in the partial order $(\preceq_{G_x},\mathcal{H}'_{S'})$,
either $H'\preceq_{G_x} H$, or $H'$ and $H$ are incomparable, i.e., $H\cap G_x$ and $H'\cap G_x$ intersect
properly. In the former case, 
$\B(H)\cap B''_u\neq\emptyset$ since $\B(H')\cap B''_u\neq\emptyset$. In the latter case, 
by Lemma \ref{lem:npsep},
$H'\cap G'_x=H\cap G'_x$ since $H\cap A_{xu}=S'=H'\cap A_{xu}$. Therefore, $H'\in\mathcal{M}_S$, and
$H'\cap \B(B''_u)\neq\emptyset$, since $x$ is a child of $u$.
\end{proof}

\begin{lemma}
\label{lem:topdown}
Let $(G,\mathcal{H})$ be a non-piercing system with a tree-decomposition $(T,\mathcal{B})$ of width $t$ that is not an easy tree-decomposition.
Then, there exists an easy tree-decomposition
$(T,\mathcal{B}')$ of width at most $3\cdot 2^t$.
\end{lemma}
\begin{proof}
By Lemma \ref{lem:bottomup}, we obtain a tree-decomposition $(T,\mathcal{B}'')$ of width at most 
$2\cdot 2^t$ that satisfies the bottom-up property with respect to $(T,\mathcal{B})$. 

For an adhesion set $A_{uv}$ in $(T,\mathcal{B})$ and $S\subseteq A_{uv}$, $S$ is said to be satisfied if for all $H\in\mathcal{H}'_S$, $H\cap \B(A''_{uv})\neq\emptyset$, 
where $A''_{uv}$ is the adhesion set corresponding to $A_{uv}$ in $(T,\mathcal{B}'')$.
An adhesion set $A_{uv}$ in $(T,\mathcal{B})$ is said to be \emph{satisfied} if $S$ is satisfied
for all $S\subseteq A_{uv}$. 
We say that $A_{uv}$ is \emph{nearly satisfied}
if for all edges $e$ closer to the root $\rho$ of $T$ than
$uv$, $A_e$ is satisfied. 

We claim that 
if $A_{uv}$ is nearly satisfied, then for each $S\subseteq A_{uv}$ and each $H\in\mathcal{H}'_S$,
$H\cap \B(B''_v)\neq\emptyset$. Further, it is sufficient to pick one vertex in $\B(B''_v)$ and add it to 
$B''_u$ to ensure that $A''_{uv}$ is satisfied, i.e., all subgraphs in $\mathcal{H}'_S$ intersect $A''_{uv}$ in a blue vertex.

Suppose there is a subgraph $H\in\mathcal{H}'_S$ such that
$H\cap \B(A''_{uv})=\emptyset$. Since $(T,\mathcal{B}'')$ satisfies the bottom-up property, 
there is a subgraph $H'\in\mathcal{M}_S$ such that $H'\cap \B(A''_{uv})\neq\emptyset$. Since 
$H'$ is a minimal subgraph in $\mathcal{H}'_S$, it follows
that $H$ and $H'$ are incomparable in $(\mathcal{H}'_S, \preceq_{G_u})$. 

Suppose $A_{uv}$ is nearly satisfied and $w\neq u$ is the other child of $v$.
Further, by definition of $\mathcal{H}'_S$, $H$ contains a blue vertex in
$G'_u$. 
Therefore, $H$ contains a vertex in the adhesion set $A_{vx}$ 
where $x$ is the parent of $v$ in $T$,
or $H$ intersects $A_{wv}$ or $H$ intersects $B''_v$ only. In the third case, $H\cap \B(B''_v)\neq\emptyset$.
In the first case, $H\in\mathcal{H}'_{S'}$
for some $S'\subseteq A_{vx}$. Since $A_{uv}$ is nearly satisfied, $H\cap \B(A''_{vx})\neq\emptyset$ and
therefore $H\cap \B(B''_v)\neq\emptyset$.

So suppose $H\cap A''_{vx}=\emptyset$, and $H\cap A_{wv}=S'$. Since $(T,\mathcal{B}'')$ 
satisfies the bottom-up property, there is a subgraph 
$H''\in\mathcal{M}_{S'}$ such that 
$H''\cap \B(A''_{wv})\neq\emptyset$ and
therefore $H''\cap \B(B''_{v})\neq\emptyset$. 
If $H\cap \B(B''_{v})=\emptyset$, it must be that $H$ and $H''$ are incomparable in $G_w$.
By Lemma \ref{lem:npsep}, therefore, $H\cap G'_w= H''\cap G'_w$. 
Since $H$ and $H'$ are incomparable
in $G_u$, it follows again by Lemma \ref{lem:npsep} that $H\cap G'_u = H'\cap G'_u$. 
This implies $H'\cap G_w$ and $H''\cap G_w$ are also incomparable.
But, $H''\cap G_u$ can't be equal to both $H\cap G_u$ and $H'\cap G_u$, as $H\cap G_u$ and $H'\cap G_u$ are incomparable.
Therefore, $H\cap \B(B''_{v})\neq\emptyset$ and since the argument above holds for any subgraph
in $\mathcal{H}'_S$ incomparable with $H'$, it follows that there is a vertex in $\B(B''_v)$
that is contained in all subgraphs in $\mathcal{H}'_S$ not comparable with $H'$.
Adding such a vertex $\beta\in H\cap \B(B''_{u})$ to $B''_u$ ensures that $S$ is satisfied. Repeating this 
process for each $S\subseteq A_{uv}$ ensures that $A_{uv}$ is satisfied.

$A_{u\rho}$ is clearly nearly satisfied, where $\rho$ is the root of $T$.
By the argument above, we can add a single blue node from $B''_{\rho}$ to $B''_u$ for each
$S\subseteq A_{u\rho}$, where $u$ is a child of $\rho$. This ensures that $A''_{yu}$ is nearly
satisfied for each child $y$ of $u$. Repeating this process top-down till the leaves of $T$ ensures
that the resulting tree-decomposition $(T,\mathcal{B}')$ is easy. For each subset of an adhesion set $A_{uv}$
we add at most one blue vertex from $B''_{v}$ to $B''_u$. Therefore the treewidth of $(T,\mathcal{B}')$ is at most $3\cdot 2^t$.
\end{proof}


\begin{theorem}
Let $(G,\mathcal{H})$ be a non-piercing system. Let $c:V(G)\to\{\R,\B\}$.
Then, there is a support $Q$ on $\B(V)$ such that  $tw(Q)\le 3\cdot 2^{tw(G)}$. Further, $Q$ can be computed in time polynomial in $|G|, |\mathcal{H}|$ if $G$ has bounded treewidth.
\end{theorem}
\begin{proof}
If $(T,\mathcal{B})$ is an easy tree-decomposition, then by Lemma \ref{lem:easytd}, we obtain a support $Q=(\B(V), F)$ of treewidth at most $tw(G)\le 3\cdot 2^{tw(G)}$. 
Otherwise, we apply Lemma \ref{lem:topdown} to obtain an easy tree-decomposition $(T,\mathcal{B}')$ of 
width at most $3\cdot 2^{tw(G)}$.
Applying Lemma \ref{lem:easytd} to $(T,\mathcal{B}')$ yields a support $Q=(\B(V), F)$ of
the treewidth at most $3\cdot 2^{tw(G)}$ for $(G,\mathcal{H})$.

If $G$ has treewidth bounded above by a constant $t$, then an optimal tree-decomposition of $G$ can be computed in
$O(2^{t}poly(n))$ time \cite{korhonen2022single} where $n$ is the number of vertices in $G$.
Now, Lemma \ref{lem:easytd}, Lemma \ref{lem:bottomup} and Lemma \ref{lem:topdown} suggest a natural two phase-algorithm:
Going bottom-up in $T$ and for each adhesion set $A$ and each $S\subseteq A$, adding a blue subgraph
corresponding to a minimal subgraph to a bag, and then doing a similar operation top-down.
Therefore, the time required to process an adhesion set is $O(2^{t}poly(|\mathcal{H}|)$. Thus, 
the overall running time is $O(poly(|G|,|\mathcal{H}|)2^{t})$, which is polynomial for bounded $t$.
\end{proof}



\subsection{Dual Support}
\label{sec:dual}

Let $(G,\mathcal{H})$ be a non-piercing system. We show in this
section that the system admits a dual support $Q^*$ such that  $tw(Q^*)\le 4\cdot 2^{tw(G)}$. Further, if $tw(G)$ is bounded above by a constant, then 
$Q^*$ can be computed in time polynomial in $|V(G)|,|\mathcal{H}|$. In other words, a dual support can be computed in FPT time parameterized
by the treewidth of $G$.
Recall that by proposition \ref{prop:nocontainment} we can assume that there are \emph{no containments}, i.e., there are no two subgraphs
$H, H'\in\mathcal{H}$ such that  $H\subseteq H'$. 
We start with a special case where it is easy to construct a support and then 
show how the general case can be reduced to this simple case. 
For a graph system $(G,\mathcal{H})$, where $\mathcal{H}$ is a collection of (possibly piercing) induced subgraphs of $G$, if a tree-decomposition $(T,\mathcal{B})$ of $G$ 
is such that for each bag
$B\in \mathcal{B}$, $|\mathcal{H}\cap B|\le k$, where $\mathcal{H}\cap B=\{H\in\mathcal{H}:H\cap B\neq\emptyset\}$  then, we call the system $(G,\mathcal{H})$ $k$-sparse.

\begin{lemma}
\label{lem:easydual}
Let $(G,\mathcal{H})$ be a (possibly piercing) system where $tw(G)=t$ and each $H\in\mathcal{H}$ induces a connected induced subgraph of $G$. Let $(T,\mathcal{B})$ be a tree-decomposition of $G$ that is $k$-sparse. Then 
there is a dual support $Q^*=(\mathcal{H},F)$ of treewidth at most $k$.
\end{lemma}
\begin{proof}
We obtain a tree-decomposition $(T',\mathcal{B}')$ of $Q^*=(\mathcal{H},F)$ as follows: the tree $T'$ is isomorphic to $T$.
Corresponding to each bag $B\in\mathcal{B}$, we construct a bag $B'\in\mathcal{B}'$ such that
$B'$ consists of a vertex $v_H$ for each subgraph $H\in\mathcal{H}$ such that $H\cap B\neq\emptyset$. 
Since each $H\in\mathcal{H}$ is a connected subgraph in $G$, each vertex $v_H$ corresponding to a subgraph $H\in\mathcal{H}$ lies in a connected subset of bags of $T'$. 
Further, since $|B\cap\mathcal{H}| \le k$, it follows that the resulting tree-decomposition has width at most $k$. Finally, we define $Q^*$ as follows: we add an edge between each pair of vertices $u_H$, and $v_{H'}$
such that $u_H$ and $v_{H'}$ lie in the same bag. 
To see that $Q^*$ is a dual support for $(G,\mathcal{H})$, consider a vertex $v\in V(G)$ and a bag $B\in\mathcal{B}$ containing $v$. The bag $B'\in\mathcal{B}'$ corresponding to $B$ contains the subgraphs $\mathcal{H}_v$ by construction. Adding edges between all pairs of subgraphs in $B'$ ensures that $\mathcal{H}_v$ is connected. The result follows.
\end{proof}

To obtain a dual support, we sparsify the input graph so that it satisfies the conditions of Lemma \ref{lem:easydual} and such that a support for the original system
can be obtained from the support for the new system. We proceed bottom-up in the tree-decomposition of $G$, and for each adhesion set $A_{uv}=B_u\cap B_v$,
and for each $S\subseteq A_{uv}$, we choose a minimal subgraph in $\mathcal{H}_S$ and use it to 
\emph{push out} a collection of subgraphs. 
This sparsification will ensure that at the end, there are at most $O(2^{t})$ distinct subgraphs
(subgraphs $H$ and $H'$ are distinct if $(H\cap V(G))\neq (H'\cap V(G))$) intersecting each bag as there are 
at most $2^t$ distinct subsets intersecting each adhesion set. In the following, for
an adhesion set $A$ and $S\subseteq A$, we let $\mathcal{H}'_S=\{H\in\mathcal{H}: H\cap A = S\}$. 

\begin{lemma}
\label{lem:mvi}
Let $(G,\mathcal{H})$ be a non-piercing system with tree-decomposition $(T,\mathcal{B})$ of width $t$.
Then, we can obtain a system $(G,\mathcal{H}')$ of (possibly piercing) connected induced subgraphs, 
where each $H'\in\mathcal{H'}$ is a subgraph of some $H\in\mathcal{H}$
such that each bag of $(T,\mathcal{B})$ intersects at most $4\cdot 2^t$ distinct subgraphs.
\end{lemma}
\begin{proof}
By Proposition \ref{prop:nocontainment}, we can assume that $(G,\mathcal{H})$ has no containments.
We process the adhesion sets bottom-up from the leaves of $T$. Let $A_{uv}$ be an adhesion set with
$v$ the parent of $u$ in $T$. Having processed the adhesion sets below $A_{uv}$, we do the following
at $A_{uv}$: Consider the containment order (for $S,S'\subseteq A_{uv},\; S\preceq S'\Leftrightarrow S\subseteq S'$)
on subsets of $A_{uv}$. We process the subsets of $A_{uv}$ according to the partial order $\preceq$. 
For each subset $S\subseteq A_{uv}$, let $H_S$ be a subgraph that is minimal in $(\mathcal{H}'_S,\preceq_{G_u})$.
For each $S'\preceq S$, if a subgraph $H\in\mathcal{H}'_{S'}$ is such that $(H\cap G_u)\setminus (H_S\cap G_u)\neq\emptyset$,
then $H_S$ \emph{pushes-out} $H$, i.e., we replace $H$ by $H''= (H\cap G_u)\setminus (H_S\cap G_u)$.
Let $\mathcal{H}''$ denote the set of subgraphs obtained by replacing each subgraph in $\mathcal{H}$ by its pushed-out copy.
Observe that $\mathcal{H}''$ may contain identical subgraphs 
even if $\mathcal{H}$ did not. Let $\unique(\mathcal{H}'')$ denote the subgraphs obtained by 
keeping a unique copy of each set of identical subgraphs.

\hide{
For each $S\subseteq A_{uv}$, choose a subgraph $H\in\mathcal{H}'_S$ 
that is \emph{minimal} in $(\mathcal{H}'_S, \preceq_{G_u})$. For each subgraph
$H'\in \mathcal{H}_{A_{uv}}$ such that $(H'\cap G_u)\setminus (H\cap G_u)\neq\emptyset$ and
$(H'\cap A_{uv})\subseteq S$, replace $H'$ by $H''=(H'\cap G_u)\setminus (H\cap G_u)$. We say that
$H'$ is \emph{pushed out} by $H$. The collection $\mathcal{H}''$ of subgraphs is obtained by 
replacing each subgraph
in $\mathcal{H}$ by its pushed out copy. Observe that $\mathcal{H}''$ may contain identical subgraphs 
even if $\mathcal{H}$ did not. Let $\unique(\mathcal{H}'')$ denote the subgraphs obtained by 
keeping a unique copy of each set of identical subgraphs.
}

We claim that at the end of this process, each subgraph is pushed out at most once,
each subgraph in $\mathcal{H}''$ is a connected induced subgraph of $G$, 
and that the resulting
system $(G,\unique(\mathcal{H}''))$ is $4\cdot 2^t$ sparse. 

For the first part, observe that since a subgraph
$H'$ is connected, it belongs to a connected subset of bags of $T$. Once $H'$ is pushed out at an
adhesion set $A_{uv}$, $H'$ does not intersect any adhesion set in $T'_u$. Thus, each subgraph
is pushed out at most once since we process the adhesion sets bottom-up.

The second part follows from the fact that the system $(G,\mathcal{H})$ is non-piercing, and therefore
$H''=(H'\cap G_u)\setminus (H\cap G_u)$ is a connected induced subgraph of $G_u$. The fact that $\mathcal{H}''$ consists
of connected induced subgraphs of $G$ follows from the fact that each subgraph is pushed out at most once.

For the third part, Once we have pushed out subgraphs at an adhesion set $A$, observe that for each
adhesion set $A$ and each $S\subseteq A$, 
there is at most one subgraph $H\in \unique(\mathcal{H}'')$ such that $H\cap A=S$.
Since $T$ is a binary tree, each bag intersects at most 3 adhesion sets and each adhesion set is
intersected by 
at most $2^t$ subgraphs in $\unique(\mathcal{H}'')$. Therefore,
there are at most $3\cdot 2^t$ subgraphs intersecting a bag $B\in\mathcal{B}$, and
an additional at most $2^t$ distinct subgraphs intersecting $B$ at vertices of $B$ not contained in any adhesion set intersecting $B$. The result follows.
\end{proof}

\begin{theorem}
Let  $(G,\mathcal{H})$ be a non-piercing system. There is a dual support $Q^*$ on $\mathcal{H}$
such that $tw(Q^*)\le 4\cdot 2^t$ where $t$ is the treewidth of $G$. Further, $Q^*$ can be
computed in time polynomial in $|G|, |\mathcal{H}|$ if $G$ has bounded treewidth.
\end{theorem}
\begin{proof}
If $(T,\mathcal{B})$ is at most $4\cdot 2^t$-sparse, we obtain a support $Q^*$ by Lemma \ref{lem:easydual}.
Otherwise, we apply Lemma \ref{lem:mvi} to obtain a system $(G, \unique(\mathcal{H}''))$ such that 
$(G,\unique(\mathcal{H}'')$ is $4\cdot 2^t$ sparse. Now, by Lemma \ref{lem:easydual}, 
we obtain a support $Q$ for $(G, \unique(\mathcal{H}''))$ of width at most $4\cdot 2^t$.

To obtain a support $Q^*$ for $(G,\mathcal{H})$, for each $H\in \unique(\mathcal{H}'')$, 
we add a new vertex $v_{H'}$ for each $H'\in\mathcal{H}''$ identical to $H$ and add the edges 
$\{v_{H}',v_{H}\}$ to $Q$. Since this operation does not increase the treewidth, $tw(Q^*)\le 4\cdot 2^t$.

We show that
$Q^*$ is a support for $(G,\mathcal{H})$. If $H'$ was pushed out by $H$, it follows that
$(H'\cap G'_u)\subseteq (H\cap G'_u)$ by Lemma \ref{lem:npsep}. Further, since $H'$ is connected, 
there is an edge $e=\{u,v\}$ in $G$ such that $u\in H'\setminus H$ and $v\in H\cap H'$. 
Hence, there is a bag $B\in\mathcal{B}$ containing $e$. By the way we construct a
support in Lemma \ref{lem:easydual}, it follows that $H$ and $H'$ are connected. 
Let $v\in V(G)$. The subgraphs in $\unique(\mathcal{H}'')$ containing $v$ induce a connected
subgraph of $Q^*$. If $H'\ni v$ was pushed out, there is a subgraph $H\in \unique(\mathcal{H}'')$
that contains $v$. By the argument above, it follows that there is a path from $H'$ to $H$ in $Q^*$
containing only subgraphs in $\mathcal{H}_v$.
Thus, $Q^*$ is a dual support for $(G,\mathcal{H})$.

If $G$ has treewidth $t$, bounded above by a constant, then by the result of Korhonen \cite{korhonen2022single},
a tree-decomposition of $G$ of width $t$ can be computed in time $O(2^t poly(n))$.
Lemma \ref{lem:easydual} and Lemma \ref{lem:mvi} suggest a natural bottom-up algorithm. The algorithm 
works by iterating over all subsets of each adhesion set and pushes out a subset of subgraphs. 
It is easy to see that the time taken to process an adhesion set is $O(2^t poly(|\mathcal{H}|))$, 
and the overall algorithm runs in $O(|G|,2^tpoly(|\mathcal{H}|)$, which is polynomial for bounded $t$.
\end{proof}
One may wonder if the non-piercing condition is necessary to obtain a support of bounded treewidth. The following examples
show that this is indeed the case. For the primal, consider a star $K_{1,n}$ with the leaves colored blue, and the central vertex
red.
Consider a collection of induced subgraphs defined by all pairs of leaves plus the central vertex.
It is easy to see that the subgraphs are not non-piercing and the support is a complete graph $K_n$ on the blue vertices.
For the dual, consider a star $K_{1, \binom{n}{2}}$. Each leaf of a star is labelled by a unique pair of $\{1,\ldots, n\}$.
There are $n$ subgraphs. The subgraph $i$ contains the central vertex and the leaves that contain the label $i$. The subgraphs
are piercing, and the dual support is $K_{n}$. Figures \ref{fig:primalpiercing} and \ref{fig:dualpiercing} show above examples of piercing subgraphs of a star such that
neither the primal nor dual supports have bounded treewidth. 

\begin{figure}[ht!]
    \centering
    \begin{subfigure}{0.3\textwidth}
    % \vspace{0.31cm}
    \begin{center}
        \includegraphics[scale=.55]{primal_piercing_tree.pdf}\vspace{0.5cm}
        \caption{$K_{1,n}$ with piercing subgraphs $H_{i,j}$ induced by the vertex sets $\{x,i,j\}$ for $i\ne j;\;i,j\in[n]$.}
        \label{fig:primalpiercing}
        \end{center}
        \end{subfigure}
        \hfill
    \begin{subfigure}{0.4\textwidth}
    % \vspace{1.3cm}
    \begin{center}
        \includegraphics[scale=.55]{dual_piercing_tree.pdf}\vspace{0.2cm}
        \caption{\vspace{0.4cm}$K_{1,\binom{n}{2}}$ with piercing subgraphs $H_i$ for $i\in[n]$, induced by vertex sets $\{x,(i,j):\;j\in[n]\setminus\{i\}\}$.}
        \label{fig:dualpiercing}
        \end{center}
    \end{subfigure}
    \caption{Primal and dual problems with piercing subgraphs that do not possess a primal or dual support of bounded treewidth.}
\end{figure}



\section{Lower Bounds}
In this section, we show that there exist graphs of treewidth $t$ whose (primal or dual) support requires treewidth $\Omega(2^t)$.

\begin{theorem}
\label{thm:primallb}
For any $\epsilon > 0$, there exists a graph $G=(V,E)$ with $c:V\to\{\R,\B\}$, and a collection of connected non-piercing induced subgraphs $\mathcal{H}$ such that 
any primal support $Q=(\B(V),F)$ of $(G,\mathcal{H})$ has treewidth $\Omega(2^{t/8(1+\epsilon)})$ where $t$ is the treewidth of $G$.
\end{theorem}
\begin{proof}
For any $t\in\mathbb{N}$, let $N=2^t$ and let $n=(1+\epsilon)\log N$. Since $\binom{2n}{n}\ge 2^n/(n+1)$,
it follows that we can choose $t$ large enough so that $\binom{2n}{n}\ge N$.

We construct the following graph $G(V,E)$: We start with an $N\times N$ grid of blue points $b_{i,j}$ for $i,j=1,\ldots, N$.
We construct four sets $U,D,L,R$ of $2n$ red points each.

\begin{figure}[ht!]
    \centering
    \begin{subfigure}{0.42\textwidth}
            \includegraphics[scale=.65]{primaltw}
        \caption{$H_{2,3}$ and $H'_{3,2}$ are shown on a $5\times 5$ gird.}
        \label{fig:primaltw}
    \end{subfigure}
    % \hspace{2cm}
    \hfill
    \begin{subfigure}{0.42\textwidth}
        \includegraphics[scale=.69]{primaltwnp}\vspace{0.5cm}
        \caption{The subgraphs $H_{ij}$ and $H_{k\ell}$ are shown non-piercing.}
        \label{fig:primaltwnp}
    \end{subfigure}
    \caption{Construction of primal hypergraph such that any support contains a grid as a subgraph.}
\end{figure}

Let $U_1,\ldots, U_{N-1}$ and $L_1,\ldots, L_{N-1}$ denote $N-1$ distinct subsets of $U$ and $L$, respectively, each of size $n$. Similarly, let $D_1,\ldots, D_N$ and $R_1,\ldots, R_N$ denote $N$ distinct 
subsets of $D$ and $L$, respectively, each of size $n$.
Since we assume that $\binom{2n}{n}\ge N$, this can indeed be done.

For each pair $b_{i,j}, b_{i,j+1}$ for $i=1,\ldots, N$ and $j=1,\ldots, N-1$,
we make $b_{i,j}$ and $b_{i,j+1}$ adjacent to all vertices in the set $U_j$ and all vertices
in the set $R_i$.
Next, for each pair $b_{i,j}, b_{i+1,j}$ for $i=1,\ldots, N-1$, $j=1,\ldots, N$ we 
make $b_{i,j}$ and $b_{i+1,j}$ adjacent to all vertices in the set $D_j$ and the set $L_i$.
This completes the construction of the graph (see Figure \ref{fig:primaltw}).

Since there are no edges between any pair of blue vertices or between any pair of red vertices, 
it follows that $G$ is bipartite.
Further, $|\R(V)| = 8(1+\epsilon)\log N$ and $|\B(V)| = N^2$. Thus, $tw(G)\le 8(1+\epsilon)\log N$.

Now we add a collection of non-piercing subgraphs to $G$ that will force the treewidth of the support to be $\Omega(N)$.
For each pair $b_{i,j},b_{i,j+1}$, $i=1,\ldots, N, j=1,\ldots, N-1$ we add the subgraph $H_{ij}$ induced by the vertices 
$b_{i,j}\cup b_{i,j+1}\cup U_j \cup R_i$.
Similarly, for each pair $b_{i,j},b_{i+1,j}$, $i=1,\ldots, N-1$ and $j=1,\ldots, N$ we add
a subgraph $H'_{ij}$ induced on the vertices $\{b_{i,j}\}\cup\{b_{i+1,j}\}\cup D_j\cup L_i$.
Let 
$\mathcal{H}=\{H_{ij}:i=1,\ldots, N, j=1,\ldots,N-1\}\cup\{H'_{ij}:i=1,\ldots, N-1,j=1,\ldots, N\}$.

We claim that $\mathcal{H}$ is a non-piercing collection of connected induced subgraphs of $G$. By definition, each subgraph in
$\mathcal{H}$ is a connected induced subgraph of $G$. It only remains to show that the subgraphs are non-piercing. Consider two
subgraphs $H_{ij}$ and $H_{k\ell}$ in $\mathcal{H}$ (as shown in Figure \ref{fig:primaltwnp}). $H_{ij}$ is the graph induced on the vertices $b_{i,j}\cup b_{i,j+1}\cup U_j\cup R_i$ and
$H_{k\ell}$ is the graph induced on the vertices $b_{k,\ell}\cup b_{k,\ell+1}\cup U_{\ell}\cup R_{k}$.
Thus, $H_{ij}\setminus H_{k\ell}$ consists
of the graph induced on the vertices $b_{i,j}\cup b_{i,j+1}\cup (U_j\setminus U_{\ell})\cup (R_i\setminus R_{k})\setminus\{b_{k,\ell},b_{k,\ell+1}\}$.
If $j\neq\ell$,  $U_j\setminus U_\ell$ is non-empty,
and if $ i\neq k$,  $R_i\setminus R_{k}$ is non-empty. Since $b_{i,j}$ and $b_{i,j+1}$ are adjacent to each vertex in $U_j$ and $R_i$, 
it follows that $H_{ij}\setminus H_{k\ell}$ is connected. A symmetric argument shows that $H_{k\ell}\setminus H_{ij}$ is connected. A similar argument
shows that subgraphs $H_{ij}$ and $H'_{k\ell}$ are non-piercing for any choice of $i,j,k$ and $\ell$. 

Each subgraph in $\mathcal{H}$ consists of exactly two blue vertices in consecutive rows or two blue vertices in consecutive columns. Therefore, any support 
$Q(\B(G),F)$
for the system $(G,\mathcal{H})$ must have an $N\times N$ grid as a subgraph. Therefore, $tw(Q)\ge N \geq 2^{tw(G)/8(1+\epsilon)}$.
\end{proof}

\begin{theorem}
\label{thm:dualtwlb}
For every $\epsilon > 0$ there exists a graph $G=(V,E)$ and a collection of connected non-piercing induced subgraphs $\mathcal{H}$ such that
any dual support $Q=(\mathcal{H},F)$ has treewidth $\Omega(2^{tw(G)})$.
\end{theorem}
\begin{proof}
Our construction of $G$ is similar to the construction for the primal support in Theorem \ref{thm:primallb}. Let $N=2^{tw(G)}$.
We start with a grid of $(2N+1)\times (2N+1)$ points $b_{ij}$, $i,j=1,\ldots, 2N+1$ where $N\in\mathbb{N}$.
At each point $b_{2i, 2j+1}, i=1,\ldots, N, j=0,\ldots, N$ add a vertex $g_{2i,2j+1}$ (See Figure \ref{fig:dualtw}). Similarly, at each point
$b_{2i+1,2j}, i=0,\ldots, N, j=1,\ldots, N$ add a vertex $g_{2i+1,2j}$. Let $K$ denote this set of vertices added.
Let $A$ and $B$ be two sets of vertices of size $2n$ each, where $n=d\log N$; $d>0$ such that $\binom{2n}{n}\ge N$. 
Let $A_1,\ldots, A_N$ be distinct subsets of $A$ of size $n$ each, and let $B_1,\ldots, B_N$ be distinct subsets of $B$ of size $n$ each.
Each point $g_{2i,2j+1}$ is adjacent to each vertex in $A_i,B_j$ and $B_{j+1}$ where $B_0$ and $B_{N+1}$ are empty sets.
Similarly, each vertex in $g_{2i+1, 2j}$ is adjacent to each vertex
in $A_i,A_{i+1}$ and $B_j$ where $A_0$ and $A_{N+1}$ are empty sets. This completes the construction of the graph. $G$ is a bipartite graph with bipartition $K$ and $A\cup B$, 
as the vertices in $K$ are pairwise non-adjacent, and so are the vertices in $A\cup B$. Further, $|K| = N^2$ and $ |A\cup B| = 4n$. 
Thus, $tw(G)\le 4n = 4d\log N$.

\begin{figure}[ht!]
    \centering
    \begin{subfigure}{0.4\textwidth}
    % \vspace{0.31cm}
    \begin{center}
        \includegraphics[scale=.62]{dualtw.pdf}\vspace{0.5cm}
        \caption{$H_{11}$ (blue) and $H_{22}$ (orange) are shown on a $5\times 5$ gird.}
        \label{fig:dualtw}
        \end{center}
        \end{subfigure}
        \hfill
        % \hspace{2.5cm}
    \begin{subfigure}{0.4\textwidth}
    % \vspace{1.3cm}
    \begin{center}
        \includegraphics[scale=.62]{dualtwnp.pdf}\vspace{.85cm}
        \caption{The subgraphs $H_{ij}$ and $H_{k\ell}$ are shown non-piercing.}
        \label{fig:dualtwnp}
        \end{center}
    \end{subfigure}
    \caption{Construction of a dual hypergraph such that any support contains a grid as a subgraph.}
\end{figure}
For each point $b_{2i,2j}, i,j=1,\ldots, N$ we construct a subgraph $H_{ij}$ induced on the vertices $g_{2i-1, 2j}, g_{2i+1,2j}, g_{2i,2j-1},g_{2i,2j+1}\cup A_i\cup B_j$.
It is easy to see that the subgraphs $\mathcal{H} = \cup_{i,j\in\{1,\ldots, N\}} H_{ij}$ are non-piercing: For $H_{ij}$ and $H_{k\ell}$,
it follows that $H_{ij}\setminus H_{k\ell}$ contains as a subset, the vertices in  $(A_i\setminus A_k)\cup(B_j\setminus B_{\ell})$ as shown in Figure \ref{fig:dualtwnp}. 
Since $H_{ij}$ and
$H_{k\ell}$ differ in at least one index, at least one of the sets $A_i\setminus A_k$, or $B_j\setminus B_{\ell}$ are non-empty, and the vertices in $K\cap (H_{ij}\setminus H_{k\ell})$ are adjacent to all vertices in $(A_i\setminus A_k)\cup(B_j\setminus B_{\ell})$. By the construction of $\mathcal{H}$, the set $K\cap (H_{ij}\setminus H_{k\ell})$ contains at least one vertex. Hence $H_{ij}\setminus H_{k\ell}$ is connected.

By construction, the vertex $g_{2i,2j+1}$ is contained only in the subgraphs $H_{ij}$ and $H_{i,j+1}$. This pair of subgraphs must be adjacent in any dual support $Q$.
Similarly, the vertex $g_{2i+1,2j}$ is contained only in subgraphs $H_{ij}$ and $H_{i+1,j}$ and this pair of subgraphs should also be adjacent in $Q$.
Therefore, $Q$ contains an $N\times N$ grid as an induced subgraph. It follows that $tw(Q)=\Omega(N) = \Omega(2^{tw(G)/4d})$. 
For any $\epsilon > 0$, setting $c = 1+\epsilon$, there is an $N$ large enough so that $\binom{2n}{n}\ge N$. Therefore, $tw(Q)=\Omega(2^{tw(G)/4(1+\epsilon)})$.
\end{proof}

\hide{
One can think of getting a support of treewidth $O(2^{tw(G)})$ without imposing the non-piercing condition on subgraphs of $G$. But we show in the following example that non-piercing is essential for both, the primal and the dual case even if the host graph $G$ has treewidth 1.\\
Let $G$ be the graph $K_{1,n+1}$ with vertex set $V_1=\{x,1,2,\ldots,n\}$ and edges $\{x,i\}$ for $i\in[n]$.
We color the central vertex $x$ red and all the leaves blue as shown in Figure \ref{fig:primalpiercing}. 
Let $\mathcal{H}_1$ be the collection of all subgraphs $H_{i,j}$ of $G$ induced by the vertices $\{x,i,j\}$ where $i\ne j,\; i,j\in[n]$.
Note that $H_{i,j}=H_{j,i}$
It is easy to see that the subgraphs in $\mathcal{H}_1$ are not non-piecing.
Further, any pair $\{i,j\}$ of blue vertices uniquely define a subgraph in $\mathcal{H}_1$, and hence, in any primal support, there is an edge $\{i,j\}$ for all $i\ne j;\;i,j\in[n]$.
So, the primal support is $K_n$ which has unbounded treewidth for large enough $n$.

For the dual, let $G'$ be the graph $K_{1,\binom{n}{2}+1}$ with the vertex set $V_2$ of $G'$ defined as $V_2=\{x,(i,j):\;i\ne j;\;i,j\in[n]\}$ where $(i,j)$ is an unordered pair. See Figure \ref{fig:dualpiercing}.
Let $\mathcal{H}_2$ be the collection of all the subgraphs $H_{i}$ of $G'$ induced by the vertices $\{x,(i,j):j\in[n]\setminus\{i\}\}$.
Again, the subgraphs in $\mathcal{H}_2$ are not non-piercing.
Further, each leaf $(i,j)$ of $G$ is contained in exactly two subgraphs $H_i$ and $H_j$.
Therefore, in any dual support, the vertex corresponding to $H_i$ must be adjacent to the vertex corresponding to $H_j$.
Since $|\mathcal{H}_2|=n$,
the dual support is $K_n$ which does not have bounded treewidth for large enough $n$.
}

\section{Applications}
\label{sec:applications}
In this section, we describe some applications of the existence of supports. We start with applications in
packing and covering problems.

\subsection{Packing and Covering via Local Search}
\label{sec:packcover}
Given a set $X$ and a collection $\mathcal{S}$ of subsets of $X$, 
the {\bf Set Packing} problem is the problem of
selecting a \emph{largest} sub-collection $\mathcal{S}'\subseteq\mathcal{S}$
such that no element in $X$ is covered by more than one set in $\mathcal{S}'$. 
The dual version of this problem is called {\bf Point Packing}\footnote{
In the Point Packing problem, we want to select the largest subset $X'\subseteq X$ so that each set $S\in\mathcal{S}$ 
contains at most one point of $X'$}.

In general set systems, since Point Packing is just the Set Packing problem on the dual set system,
algorithmic and hardness results that hold in the primal, also hold in the dual.
Set Packing problem contains
as a special case, the Independent Set problem on graphs. Since the
Independent Set problem on graphs is hard to approximate beyond $n^{1-\epsilon}$ for any $\epsilon > 0$ 
\cite{hastad1999clique}, the same hardness holds for the Set Packing and Point Packing problems.  In a geometric setting however,
one direction may be more amenable to geometric techniques, and hence easier. 

In the {\bf Set Cover} problem, the goal is to select a \emph{smallest} sub-collection $\mathcal{S}'\subseteq\mathcal{S}$ such that 
each element $x\in X$ is contained in at least one set in $\mathcal{S}'$.
The dual version of the problem is called {\bf Point Cover}\footnote{In the Point Cover problem for a set system $(X,\mathcal{S})$, the
goal is to select the smallest cardinality subset $X'\subseteq X$ s.t. $S\cap X'\neq\emptyset$ for all $S\in\mathcal{S}$.} problem. The Point Cover problem is more popularly called the {\bf Hitting Set} problem. 
Just as in the case for packing problems, in general set systems, the Hitting Set problem is just the
Set Cover problem on the dual set system. Lov\`asz \cite{lov1975notdef},
and later Chvat\'al \cite{chvatal1979greedy} gave $O(\log n)$-approximation algorithms. Feige \cite{feige1998threshold} showed that Set Cover cannot be approximated beyond $(1-\epsilon)\ln n$
for any $\epsilon > 0$ unless $NP\subseteq DTIME (n^{\log\log n})$.

In a geometric hypergraph, the elements are a set of points $P$ (or other geometric objects),
and the hyperedges are defined by a set $\mathcal{O}$ of geometric objects, 
where a hyperedge consists of all points contained in an object. 
In most cases of Packing and Covering problems that have been studied, the points and geometric objects are embedded in the Euclidean plane,
or in $\mathbb{R}^d$ for constant $d$.
In many cases, it is possible to exploit the structure of geometric hypergraphs to obtain
better algorithms than in the general setting. Our results imply that for a general class of geometric hypergraphs, 
there exists a unified algorithmic paradigm, namely \emph{local search}, and analysis that leads to a PTAS for a 
wide class of Packing and Covering problems. We next describe the paradigm and analysis. We also consider \emph{intersection graphs} 
defined by geometric objects. In this setting, there is a vertex for each region and an edge between two regions if their intersection is non-empty.

\emph{Local search} in the context of geometric packing and covering problems is the following (See \cite{DBLP:conf/walcom/AschnerKMY13, ChanH12, mustafa2010improved} for concrete algorithms under this paradigm for specific problems):

{\bf Local Search Paradigm:}
For a parameter $k\in\mathbb{N}$, start with an arbitrary feasible solution. While there is a feasible solution of
better value within a $k$-neighborhood of the current solution, replace the current solution with this better solution.
When no such improvement is possible, return the current solution.

Let $\mathcal{L}$ denote the solution returned by the local search algorithm, and let $\mathcal{O}$ denote an optimal
solution.
The key to analyze the local search paradigm is to show the existence of a 
\emph{local search graph}, i.e., a bipartite graph $G$ on $\mathcal{L}\cup\mathcal{O}$
that satisfies two properties, \emph{viz.,}
$(i)$ {\em Local Property:} For any $\mathcal{L}'\subseteq\mathcal{L}$, replacing $\mathcal{L}'$ by its neighborhood in
$G$ results in a feasible solution.
$(ii)$ {\em Global Property:} $G$ comes from a hereditary family that has sub-linear sized separators. 

The local property captures a subset of the local moves, and the global property is used to bound the approximation factor
guaranteed by the algorithm. See \cite{RR18, DBLP:conf/walcom/AschnerKMY13} for a description of this analysis.
Therefore,
the problem of showing that the local search algorithm satisfying the paradigm above yields a PTAS reduces to the combinatorial
question of the existence of a suitable local search graph.

Consider for example, the Set Packing problem for a geometric hypergraph defined by a set $P$ of points in the plane,
and a set $\mathcal{D}$ of \emph{pseudodisks} considered by Chan and Har-Peled \cite{ChanH12} (the authors consider a slightly
different problem where $P=\mathbb{R}^2$, but the same technique works here). Let $\mathcal{L}$ and $\mathcal{O}$ correspond
respectively, to a solution produced by the local search algorithm, and an optimal solution. We can assume that $\mathcal{L}\cap\mathcal{O}=\emptyset$ (otherwise, the ratio only improves). Since each point of $P$ is covered by at most
one pseudodisk of $\mathcal{L}$ and one pseudodisk of $\mathcal{D}$, we can put a vertex for each pseudodisk $L\in\mathcal{L}$ that lies in $L$
and one vertex for each $O\in\mathcal{O}$ that lies in $O$, and add edges via continuous, internally non-intersecting curves that 
go through a point in their intersection (if any) to obtain a
planar graph (See \cite{ChanH12} for a description of this graph construction). 
This graph satisfies the local property, as well as the global property since planar graphs have separators
of size $O(\sqrt{n})$ \cite{LT79}. By the arguments in \cite{ChanH12}, this is sufficient to obtain a PTAS.

When the points have arbitrary but bounded capacities, 
Basu Roy et. al., \cite{BasuRoy2018} used the existence of a planar support on the points with respect to the pseudodisks,
and used this to construct the desired local search graph on the pseudodisks. This graph is not planar, but the authors
showed that it nevertheless satisfied the sub-linear separator property. 

Raman and Ray \cite{RR18} obtained planar support graphs for the intersection hypergraph of non-piercing regions in the plane.
The existence of a support can be used to show the existence of a local search graph, and
hence their result implied a PTAS for several packing and covering problems for geometric hypergraphs defined by points
and \emph{non-piercing regions} in the plane. 

Our results on cross-free subgraphs on a host graph generalize the results of \cite{RR18}. While
our results hold for any dual arrangement graph that is cross-free, we give a concrete example here.
A collection $\mathcal{D}$ of \emph{non-piercing} regions on an oriented surface, is a set of regions where each $D\in\mathcal{D}$ is 
defined by a simple curve that bounds a disk, and such that for any pair of regions $D, D'\in\mathcal{D}$, $D\setminus D'$ is path
connected. 

\begin{lemma}
\label{lem:tdgraph}
Let $\mathcal{K}$ and $\mathcal{H}$ be two collections of non-piercing regions on an oriented surface of genus $g$. Let $G$ be the dual arrangement graph of $\mathcal{H}\cup\mathcal{K}$.
Then, both $\mathcal{K}$ and $\mathcal{H}$ induce cross-free systems on $G$.
\end{lemma}
\begin{proof}
For each region $R\in \mathcal{H}\cup\mathcal{K}$, we abuse terminology and use $R$ to also denote the subgraph of $G$ induced by $R$. Since each region is
disk-bounding, it follows that $R$ is a connected subgraph of $G$ that is bound by a cycle of $G$ separating the vertices in $R$ from the rest of $G$. 
As a consequence, it is easy to check that $R_G(H, H')$ is cross-free for any pair of subgraphs $H, H'\in\mathcal{H}$, as the subgraphs in $\mathcal{H}$
are non-piercing. Similarly, the subgraphs of $G$ corresponding to regions in $\mathcal{K}$ are cross-free. Therefore, $(G,\mathcal{H},\mathcal{K})$ is
a cross-free system of genus $g$.
\end{proof}

As a consequence, we obtain an intersection support of genus at most $g$. 

\begin{theorem}
\label{thm:intgraph}
Let $\mathcal{K}$ and $\mathcal{H}$ be two collections of non-piercing regions on an oriented surface of genus $g$. There
is an intersection support $\tilde{Q}$ for the intersection hypergraph $(\mathcal{H},\{\mathcal{H}_K\}_{K\in\mathcal{K}})$ of genus at most $g$.
\end{theorem}
\begin{proof}
By Lemma \ref{lem:tdgraph}, $(G,\mathcal{H},\mathcal{K})$ induces a cross-free system of genus $g$, where $G$ is the
dual arrangement graph of $\mathcal{H}\cup\mathcal{K}$. Hence, by Theorem \ref{thm:intsupport}, there is an intersection support
$\tilde{Q}$ of genus at most $g$.
\end{proof}

Since we have an intersection support of genus $g$, 
we obtain PTASes for several packing and covering problems for non-piercing regions on
an oriented surface by using the local search paradigm in \cite{RR18, DBLP:conf/walcom/AschnerKMY13}.

\begin{theorem}
\label{thm:disktorus}
Let $\mathcal{D}$ be a finite set of non-piercing regions in an oriented surface $\Sigma$ of genus $g$, and let $P$ be a set of points in $\Sigma$. 
Then, there is a PTAS for
\begin{enumerate}
\item The minimum Hitting Set problem for the hypergraph defined by $(P,\mathcal{D})$.\label{prb:one}
\item The minimum Set Cover problem for the hypergraph defined by $(P,\mathcal{D})$.\label{prb:two}
\item The Dominating Set problem for the intersection graph of the regions in $\mathcal{D}$.\label{prb:three}
\end{enumerate}
\end{theorem}
\begin{proof}
The PTAS for problems \ref{prb:one} and \ref{prb:two} follows directly by the local search paradigm in \cite{RR18,DBLP:conf/walcom/AschnerKMY13}. 
For problem \ref{prb:three}, i.e., the dominating set problem, we 
create a copy for each region in $\mathcal{D}$. We let $\mathcal{H}$ denote the original set of regions, and $\mathcal{K}$ to
denote the copies. $(G,\mathcal{H},\mathcal{K})$ is an intersection system of non-piercing regions, and by Theorem \ref{thm:intgraph}, there is a support $\tilde{Q}$ of
genus $g$. A PTAS now follows from the framework in \cite{RR18}. 
\end{proof}

\hide{
The results showing that local search is a PTAS relied on showing the existence of a local search graph, and this
was only done for geometric hypergraphs defined in the plane. If we consider problems on higher genus surfaces however,
the existing techniques are insufficient.

As a concrete example, consider the Hitting Set problem defined by a set $P$ of points and a set $\mathcal{D}$
of disks on a torus. 
Since the set of points in a cell of the arrangement belong to the same set of disks, we can
assume that each cell in the arrangement of $\mathcal{D}$ consists of at most one point of $P$. Let $G$ be the
\emph{dual arrangement graph}, whose vertex set are the cells of the arrangement, and two cells are adjacent
if they share an arc of the boundary of a disk. We color a vertex of $G$ \emph{red} if it corresponds to an
empty cell. Otherwise, we color it \emph{blue}. Each disk $D\in\mathcal{D}$ corresponds to a connected subgraph
of $G$. It is easy to check that the disks in $\mathcal{D}$ induce a cross-free arrangement on $G$.
Therefore, by Theorem \ref{thm:primalsupport}, there is a support $Q$ embedded on the torus such that for each
subgraph corresponding to a disk $D\in\mathcal{D}$, the graph induced by the non-empty cells in $D$ is connected.

Now, consider the local search paradigm applied to this problem. Let $L$ and $O$ denote respectively, a solution returned
by the local search algorithm, and an optimal solution. We can assume that $L\cap O=\emptyset$ (See \cite{mustafa2010improved}).
The local condition we require on a graph $G$ on $L\cup O$ is that for any subset $S\subseteq L$, 
$L'=(L\setminus S)\cup(N(S))$ is a feasible solution, where $N(S)$ is the set of neighbors of $S$ in $O$.
That is, for each disk $D$, $L'\cap D\neq\emptyset$. 
Since $L$ and $O$ are both feasible
solutions, each $D\in\mathcal{D}$ contains a point of $L$ and a point of $O$. Since $Q$ is a support, $D$ induces
a connected subgraph of $Q$. Hence, for each $D\in\mathcal{D}$ there is an edge between some point in $L\cap D$
and some point in $O\cap D$. Thus, $Q$ satisfies the locality property required.
Since graphs embedded on a surface of bounded genus have sub-linear sized separators \cite{gilbert1984separator}, 
combined with
previous results \cite{mustafa2010improved}, 
we obtain a PTAS for the Hitting Set problem of points and disks on a torus.

A similar reasoning holds for the other packing and covering problems considered in this section. 
Thus, we obtain the following theorem:

\begin{theorem}
\label{thm:ptascrossfree}
Let $(G,\mathcal{H},\mathcal{K})$ be a cross-free system of genus $g$, then there exists 
\begin{enumerate}
    \item a PTAS for the Dominating Set problem, which is a problem to find $\mathcal{H}'\subseteq\mathcal{H}$ of minimum 
    cardinality such that for each $H\in\mathcal{H}$, either $H\in\mathcal{H}'$ or $H\cap H'\neq\emptyset$ for
    some $H'\in\mathcal{H}'$.
\item a PTAS for the problem of packing points when each subgraph $H\in\mathcal{H}$ has capacity $D_H$ 
bounded by a constant, i.e.,
find $V'\subseteq V$ of maximum cardinality such that $|H\cap V'|\le D_H$ for each $H\in\mathcal{H}$.
\item a PTAS for the problem of packing subgraphs when each vertex $v\in V$ has capacity $D_v$ bounded by a constant, i.e.,
find $\mathcal{H}'\subseteq\mathcal{H}$ of maximum cardinality such that $|\{H\in\mathcal{H}': H\ni v\}|\le D_v$ for each $v\in V$.
\end{enumerate}
\end{theorem}
}

We believe that the cross-free condition is essential to obtain PTASes for packing
and covering problems when the host graph has bounded genus.
Chan and Grant \cite{DBLP:journals/comgeo/ChanG14}
proved that for a hypergraph defined by a set of horizontal and vertical slabs 
in the plane and a set of points $P$, the Hitting Set problem and
the Set Cover problems are  APX-hard. A simple modification of their result implies the following.

\begin{theorem}
\label{thm:apxhard}
There exist crossing non-piercing systems $(G,\mathcal{H})$ with $G$ embedded in the torus such that the
Hitting Set problem is APX-hard. Similarly, the Set Cover problem on such a set system is APX-hard.
\end{theorem}
\begin{proof}
The proof follows directly from the corresponding APX-hardness proof of 
Chan and Grant \cite{DBLP:journals/comgeo/ChanG14}.
We only sketch the modification required.
Consider the Set Cover problem: Given a set of horizontal slabs $H$ and a set $V$ of vertical slabs and a set $P$ of points in the
plane, the authors show that it is APX-hard to select a minimum cardinality subset of $H\cup V$ to cover $P$. 
To obtain the claimed APX-hardness proof on the torus for non-piercing regions, we embed this construction on a torus, and then
modify by boundary of each region in $H$ to be a pair of parallel non-separating closed curves parallel to the hole. Similarly,
we map each vertical slab in $V$ to a region bound by two parallel non-separating closed curves perpendicular to the hole.

Now, construct the dual arrangement graph $G$ with a representative point for each non-empty cell in the arrangement of the regions,
and let $\mathcal{H}$ denote the set of subgraphs of $G$ defined by the regions.
In $(G,\mathcal{H})$, the subgraphs are non-piercing,
but are crossing. The APX-hardness of the problem follows from the corresponding result of Chan and Grant \cite{DBLP:journals/comgeo/ChanG14}.

The proof of APX-hardness for the Hitting Set problem for non-piercing crossing subgraphs of a graph follows by a similar modification
of the construction of \cite{DBLP:journals/comgeo/ChanG14} for the Hitting Set problem with horizontal and vertical slabs in the plane.
\end{proof}

\subsection{Coloring Geometric Hypergraphs}
Keller and Smorodinsky \cite{KellerS18} showed that the intersection hypergraph of disks in the plane can be colored
with 4 colors, and this was generalized by Keszegh \cite{Keszegh20} for pseudodisks, which was 
further generalized in \cite{RR18} to show that the intersection hypergraph of non-piercing regions is 4-colorable. 
As a consequence of Theorem \ref{thm:intsupport}, 
we obtain the following.

\begin{theorem}
\label{thm:colorhypergraph}
Let $(G,\mathcal{H},\mathcal{K})$ be a cross-free intersection system where $G=(V,E)$ is embedded in an orientable 
surface of genus $g$. Then, $\mathcal{H}$ can be colored with at most $\frac{7 + \sqrt{1+24g}}{2}$ colors such that for any $K\in\mathcal{K}$,
no hyperedge ${\mathcal{H}_K}$ is monochromatic. 
\end{theorem}
\begin{proof}
By Theorem \ref{thm:intsupport}, $(G,\mathcal{H},\mathcal{K})$ has an intersection support $\tilde{Q}$ of genus at most $g$.
Now, $\chi(\tilde{Q})\le \frac{7 + \sqrt{1+24g}}{2}$ \cite{diestel2005graph}. Since $\tilde{Q}$ is a support,
for each $K\in\mathcal{K}$, there is an edge between some two subgraphs $H,H'\in {\mathcal{H}_K}$. Therefore,
no hyperedge ${\mathcal{H}_K}$ is monochromatic.
\end{proof}

Ackerman et al. \cite{ackerman2020coloring} considered a notion of $ABAB$-free hypergraphs, which is defined as follows: a 
hypergraph $(X,\mathcal{S})$ is $ABAB$-free if there exists a  linear ordering $x_1<\ldots< x_n$ of $X$ such that for any pair of hyperedges
$A,B\in\mathcal{S}$, there are no four elements $x_i < x_j < x_k < x_{\ell}$ such that $x_i,x_k\in A\setminus B$ and $x_j,x_{\ell}\in B\setminus A$.
The notion of $ABAB$-free hypergraphs is equivalent to the notion of $abab$-free hypergraphs (See Defn. \ref{defn:abab}).
Indeed, if there exists a linear ordering $x_1<\ldots<x_n$ that is $ABAB$-free, then the cyclic order $x_1<\ldots<x_n<x_1$ is $abab$-free, 
and similarly, if $x_1<\ldots, x_n<x_1$ is a cyclic order that is $abab$-free, then $x_1<\ldots<x_n$ is an $ABAB$-free linear order.

The authors show that $ABAB$-free hypergraphs are equivalent to 
hypergraphs with a 
\emph{stabbed pseudo-disk representation}, i.e.,
each $S\in\mathcal{S}$ is mapped to a closed and bounded region $D_S$ containing the origin whose boundary is a simple Jordan curve,
each $x\in X$ is mapped to a point $p_x$ in $\mathbb{R}^2$ such that $p_x\in D_S$ iff $x\in S$.
The regions $\mathcal{D}=\{D_S: S\in\mathcal{S}\}$
form a stabbed pseudodisk arrangement, i.e., the boundaries of any two of them are either disjoint or intersect exactly twice and all the regions in $\mathcal{D}$ contain the origin.

The authors show that to any stabbed pseudodisk arrangement $\mathcal{D}$ and a set $P$ of points,
we can add additional pseudodisks $\mathcal{D}'$ such that $(i)$ each $D'\in\mathcal{D}'$ contains exactly 2 points of $P$, 
$(ii)$ $\mathcal{D}\cup\mathcal{D}'$ is a pseudodisk arrangement, and
$(iii)$ Each $D\in\mathcal{D}$ such that $|D\cap P|\ge 3$ contains a pseudodisk $D'\in\mathcal{D}'$. The graph on $P$ whose edges are defined
by $\mathcal{D}'$ is called the \emph{Delaunay graph} of the arrangement. 


Our result, namely Theorem \ref{thm:primalOuter} in Section \ref{sec:outerplanar} is stronger. A Delaunay graph ensures that for each
pseudodisk $D\in\mathcal{D}$, the induced subgraph on the elements in $D$ is non-empty, while a support implies that the induced subgraph
of the support on the elements in $D$ is connected.
At the outset, it seems like the results of Ackerman et al. \cite{ackerman2020coloring},
especially the proof of Lemma $2.1$ can be used to prove Lemma \ref{lem:keylem}. However, there is a subtle difference between the two.
The authors show that there is a 2-element hyperedge, or equivalently a non-blocking diagonal that can be added between two elements of a hyperedge, but in the vertex bypassing operation we require this diagonal to be between two disjoint runs of a hyperedge (subgraph) which is a more stringent condition. 

The authors in \cite{ackerman2020coloring} show that for a stabbed pseudodisk arrangement, the Delaunay graph as constructed above is outerplanar, and hence
is 3-colorable. This implies that $ABAB$-free hypergraphs, and thus hypergraphs induced by stabbed pseudodisks can be colored with
3 colors so that no hyperedge with $\ge 2$ elements is monochromatic. 
This result also follows directly from Theorem \ref{thm:primalOuter}. 

Ackerman et al., (\cite{ackerman2020coloring}, See Conclusion) ask if we can 3 color the elements of the dual of an
$ABAB$-free hypergraph such that no hyperedge of the dual with at least two elements,
is monochromatic. In Figure \ref{fig:dualDColor} we show that this is not true - even if the regions are defined by unit disks in the plane. 
The dual hypergraph contains four elements corresponding to the four unit disks, and six hyperedges corresponding to the six points. Each hyperedge defined by a point is a pair of disks containing that point. The intersection of the disks is non-empty. Hence, this corresponds to a stabbed pseudodisk arrangement, and by the results in \cite{ackerman2020coloring}, the hypergraph is $ABAB$-free.
Each point is of depth 2, and therefore the dual support is $K_4$, which is not 3-colorable.
However, by Theorem \ref{thm:outerplanardual}, it follows that if a hypergraph $(X,\mathcal{S})$ admits a representation 
as non-piercing subgraphs on a host outerplanar graph, then the hyperedges of the dual hypergraph can be 3-colored so that no point
is monochromatic.

Consider the following natural extension of the result of Ackerman et al. \cite{ackerman2020coloring}: Call an arrangement of non-piercing regions \emph{stabbed} if their intersection is non-empty.
Given a collection of stabbed non-piercing regions in the plane, does there exist a coloring of the points with 3
colors such that no region is monochromatic?
We answer this question again in the negative by giving a counter-example (see Figure \ref{fig:primalNPColor}).
It is easy to check that in this case again, the primal support graph is $K_4$, and therefore the hypergraph is not $3$-colorable.
The reason why hypergraphs defined by stabbed pseudodisks are 3 colorable, but the ones defined by stabbed non-piercing regions are not, is
the following: Let $\mathcal{R}$ be an arrangement of non-piercing regions in the plane and $\partial R$ denote the boundary of a region $R\in\mathcal{R}$.
We call each connected component of $\mathbb{R}^2\setminus\cup_{R\in\mathcal{R}}\partial R$ a \emph{cell} in the arrangement and the depth of a cell is the number of regions containing it.
Let $\overrightarrow{H}$ denote the directed graph obtained from the dual arrangement graph
where each edge is directed from a cell to its adjacent cells of lower depth.
If $(P,\mathcal{D})$ is
a stabbed pseudodisk arrangement, then we can show that every cell is reachable from $o$, where $o$ is the
cell in the intersection of all pseudodisks (marked by $\times$ in Figure \ref{fig:primalNPColor}). This is not true for example, in the graph $\overrightarrow{H}$
corresponding to the arrangement of non-piercing regions as in Figure \ref{fig:primalNPColor}. In particular, the cell containing
$d$ is not reachable in $\overrightarrow{H}$ from the cell $o$ in the intersection of all the regions.

\begin{figure}[ht!]
\centering
\begin{subfigure}{0.4\textwidth}
 \includegraphics[scale=.7]{dualDisk.pdf}\vspace{.5cm}
\caption{Dual: Every point $a,b,\ldots,f$ is contained in two disks.}
\label{fig:dualDColor}
\end{subfigure}
\hfill
% \hspace{1.2cm}
\begin{subfigure}{0.5\textwidth}\vspace{0.5cm}
\includegraphics[scale=.63]{primalNPColor.pdf}\vspace{0.41cm}
\caption{Primal: Every region contains two points.}

\label{fig:primalNPColor}
 \end{subfigure}\vspace{0.5cm}
 \caption{Stabbed hypergraphs of disks (dual) and non-piercing regions (primal) requiring four colors.}
 \label{coloring}
 \end{figure}



\section{Conclusion}
\label{sec:conclusion}
In this paper, we studied the problem of construction of primal, dual supports for graph systems $(G,\mathcal{H})$ defined on a host graph
$G$. We also considered the more general problem of constructing a support for an intersection system $(G,\mathcal{H},\mathcal{K})$.
We primarily studied two settings, namely when $G$ has bounded genus, and when $G$ has bounded treewidth. We showed that
if $G$ has bounded genus, then the cross-free property is sufficient to obtain a support of genus at most that of $G$.
If $G$ has bounded treewidth, we showed that the non-piercing condition on $\mathcal{H}$ is a sufficient condition
to obtain a support of bounded treewidth in the primal and dual settings. However, an exponential blow-up of the
treewidth of the support is sometimes necessary.
Along the way, we also studied the settings of outerplanar graphs.

There are several intriguing open questions and research directions
and we mention a few: We do not know if the algorithm to construct a dual or intersection support 
in the bounded-genus case runs in polynomial time. 
A broader line of research is to obtain necessary and sufficient conditions for a hypergraph to have a \emph{sparse support} - 
where sparsity could be a graph with sublinear-sized separators or even just a graph with a linear number of edges.

\bibliography{ref}

\end{document}
    
