We consider an autonomous, semi-explicit system of \glspl{DAE}%
\begin{subequations}\label{eq:DAE_general}%
\begin{align}
    \frac{d}{dt} \bm{x} &= \bm{f}(\bm{x}(t),\bm{y}(t),\bm{u})\label{eq:DE_general}\\
    \bm{0} &= \bm{g}(\bm{x}(t),\bm{y}(t)) \label{eq:AE_general}
\end{align}
\end{subequations}
with differential variables $\bm{x}(t)\in \mathbb{R}^{m}$, algebraic variables $\bm{y}(t)\in \mathbb{R}^{n}$, control inputs $\bm{u}\in \mathbb{R}^{l}$, and the functions $\bm{f}:\mathbb{R}^{m+n+l+1}\mapsto\mathbb{R}^{m}$ and $\bm{g}: \mathbb{R}^{m+n+1}\mapsto\mathbb{R}^{n}$. We furthermore assume that $\bm{g}$ is continuously differentiable with respect to $\bm{y}$, hence, this system of \glspl{DAE} is of Hessenberg index-1 form~\cite{hill1990stability}. Given the initial state $\bm{x}_0=\bm{x}(t_0)$ of the system, our goal is to evaluate the trajectory of the dynamic variables $\bm{x}(t)$ and of the algebraic variables $\bm{y}(t)$ for the time interval $t \in [t_0, t_0 + \Delta t]$.

The \gls{DAE} in \eqref{eq:DAE_general} is nontrivial to solve and there has been a great deal of work dedicated to solving them in many applications \cite{brenan_numerical_1995,campbell_applications_2019}. In this paper, we consider solving~\eqref{eq:DAE_general} in the context of power systems, where $\bm{y}(t)$ describes bus terminal voltages and $\bm{x}(t)$ includes variables related to dynamics within the power system components. For example, $\bm{x}(t)$ includes the internal states of the generator (see, \cite{kundur_power_1994} and the references within). A particular challenge about the \glspl{DAE} that describe power systems dynamics is that they can be stiff, and a number of techniques have been developed to address this challenge \cite{stott_power_1979,milano_power_2010,sauer_power_1998,machowski_power_2008}. These techniques, however, tend to be computationally intensive and are difficult to use in real-time or when a large number of simulations need to be performed in a short time (e.g., for systems with high level of renewable resources).

To overcome these computational challenges, we leverage the underlying sparsity in power systems. Namely, we use the fact that the differential variables associated with each bus, evolve based only on the variables at that bus, except for a coupling constraint on the algebraic variables.\footnote{In this paper, we do not consider the very fast electromagnetic transients on the transmission lines.} This coupling constraint is the current balance, derived from Kirchhoff's and Ohm's laws, and constitutes \labelcref{eq:AE_general}:
\begin{align}
    \bm{0} &= \Bar{Y}_N \Bar{\bm{V}}(t) - \Bar{\bm{I}}(\bm{x}(t), \Bar{\bm{V}}(t)). \label{eq:AE_power}
\end{align}
 $\Bar{\bm{V}}\in \mathbb{C}^{n}$ contains the complex voltages at $n$ buses in the system and $\Bar{Y}\in \mathbb{C}^{n\times n}$ defines the complex admittance matrix of the network. The complex current injections from the $n$ buses, i.e., from components such as synchronous generators, converters, or loads, are collected in the vector $\Bar{\bm{I}} \in \mathbb{C}^{n}$. We assume $m$ components to be connected to the system with the differential states $\bm{x}_i \in \mathbb{R}^{p}$ and control inputs $\bm{u}_i \in \mathbb{R}^{q}$ for the $i$-th component.\footnote{All the results in the paper hold if $p$ is different for each bus $i$, but we avoid this notation to reduce clutter in the variables.} The $i$-th component is connected at the $i$-th bus and therefore dependent only on the complex voltage $\Bar{V}_i \in \mathbb{C}^{1}$. Each component is governed by a set of differential equations $\bm{f}_i: \mathbb{R}^{p+2+1+q} \mapsto \mathbb{R}^{p}$ and a function $h_i: \mathbb{R}^{p+2} \mapsto \mathbb{C}^{1}$ that determines the current injection $\Bar{I}_i$
\begin{subequations} \label{eq:DE_power}
\begin{alignat}{2}
    \frac{d}{dt} \bm{x}_i &= \bm{f}_i(\bm{x}_i(t), \Bar{V}_i(t), \bm{u}_i), \quad &&i=1, \dots, m\label{eq:DE_power_f} \\
    \Bar{I}_i &= h_i(\bm{x}_i(t), \Bar{V}_i(t)), &&i=1, \dots, m\label{eq:DE_power_h}
\end{alignat}
\end{subequations}
The concatenation of all differential variables is of dimension $m\cdot p$, i.e., $\bm{x}\in  \mathbb{R}^{m\cdot p}$. 

%  \begin{figure}[!ht]
%     \centering
%     \includegraphics[width=0.8\linewidth]{figures/ieee9_network.png}
%     \caption{Power system structure}
%     \label{fig:ieee9_network}
% \end{figure}