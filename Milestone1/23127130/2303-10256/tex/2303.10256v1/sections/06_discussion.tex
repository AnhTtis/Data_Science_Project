The above results illustrate the conceptual benefits of a simulator leveraging \gls{NN}-based predictors for the dynamic components of a power system. The upside of the approach should become more clear when integrating higher order models that would usually require very small time-steps for accurate resolution. A second direction of improvements lies in the implementation. All dynamic elements should be pre-compiled and distributed on the available cores to speed up the calculation of the residual and the Jacobian of the residual as this is the area where most computational time is spend. Furthermore, borrowing from classical numerical methods, dishonest Newton-Raphson-schemes and the exploitation of the structure in the Jacobians should be improved. The latter is due to the sparsity in the network equations and the dependence of the components current injection only on its terminal voltage. Another important aspect is to investigate the relation between the number of collocation points, the resulting current balance errors and overall errors. Analytical analysis of the scheme will be necessary to build confidence in the accuracy but also to identify limitations of the approach. Ultimately, larger test cases need to be studied.