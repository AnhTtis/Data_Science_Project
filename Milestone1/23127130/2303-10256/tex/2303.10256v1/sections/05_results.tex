We begin by briefly comparing the performance of the three predictors for $\hat{\bm{x}}_i$ (FE, RK4, NN) before testing \cref{algo:simulator}.

\subsection{Predictor comparison}\label{subsec:results_predictors}
The approximator $\hat{\bm{x}}_i$ should have two properties: being fast and being accurate for large time-steps $\Delta t$. \Cref{fig:predictor_characteristics} shows the two properties. The left panel displays the error in the differential variable $\Delta \omega$ across 200 points of random initial conditions $\bm{x}_{0}$ and voltage parameterizations $\yparams{}_i$. In terms of accuracy, the \gls{NN} performs well and only for smaller time-steps ($\Delta t < 0.05 \si{\second}$), the RK4 approximation becomes more accurate. The RK-schemes exhibit the expected dependency of the time-step - the local truncation error should follow $\mathcal{O}(\Delta t)$ and $\mathcal{O}(\Delta t^4)$ for FE and RK4. The error of the \gls{NN} in contrast is near independent of $\Delta t$. At the same time, the \gls{NN} is the fastest approximator. While the FE approximator is similarly fast, its poor accuracy makes it undesirable and while more accurate, the RK4 scheme has the drawback of comparably long run-time.     

\begin{figure}[!th]
    \centering
    \includegraphics[width=0.95\linewidth]{figures_pdf/predictor_error.pdf}
    \caption{Predictor characteristics: (left) prediction error of $\Delta \omega$ for 200 predictions with random $\bm{x}_0$ and \yparams{}, (right) run-time per point. These results show that \gls{NN} constitute an accurate and fast predictor.}
    \label{fig:predictor_characteristics}
\end{figure}

\subsection{Simulator results}\label{subsec:results_simulators}

We now focus on the performance of the simulator in \cref{algo:simulator}. \Cref{fig:ieee9_prediction} displays the prediction for $\Delta t = \SI{0.05}{\second}$ using \gls{NN}-based approximators $\hat{\bm{x}}_i^{NN}$. 
\begin{figure}[!ht]
    \centering
    \includegraphics[width=0.95\linewidth]{figures_pdf/simulation_results.pdf}
    \caption{Prediction of a state trajectory ($\Delta \omega_i$) and an algebraic variable $V_i$ for time-step size $\Delta t = \SI{0.05}{\second}$ with a \gls{NN}-based simulator. The predictions (dashed lines) coincide with the ground truth (gray lines).}
    \label{fig:ieee9_prediction}
\end{figure}
The results correspond to the first row in \cref{tbl:simulator_results} where we report the maximum absolute errors of $V$ and $\Delta \omega$ along the trajectory and the run-time. \Cref{tbl:simulator_results} shows further results for the \gls{RK4}-based simulator, for time-steps of $\Delta t = \SI{0.1}{\second}$ and $\Delta t = \SI{0.15}{\second}$ and for collocation points at $\bm{T}=[0.3, 0.7]\Delta t$ and $\bm{T}=[0.1, 0.3, 0.5, 0.7, 0.9]\Delta t$, i.e., $s=2$ and $s=5$. The \gls{NN}-based simulator is consistently faster and more accurate than the \gls{RK}-based simulator, except for the case of $\Delta t = \SI{0.05}{\second}$. These results confirm the predictor characteristics observed in \cref{subsec:results_predictors}. The simulation run-time scales approximately inversely proportional with $\Delta t$. Deviations can arise due to varying numbers of iteration in \cref{algo:parameter_update}, however, in the reported cases, we observe usually 5-7 iterations. Increasing the number of collocation points $s$ results in a small increase in run-time in all cases. In terms of accuracy, we observe that more collocation points can lead to better accuracy, when the overall solution quality is good. However, for too large time-steps, here $\Delta t = \SI{0.15}{\second}$, the effect might reverse. The choice of the location of the collocation points, i.e., $\bm{T}$, also matters.%\bz{Can we bold the best numbers in a box? This would help to highlight what people should look at. There are a lot of numbers.}
\begin{table}[!ht]
\renewcommand{\arraystretch}{1.2}
\caption{Comparison of simulators with different predictor schemes}
\label{tbl:simulator_results}
\centering
\begin{tabular}{ccc|ccc}
\toprule
$\Delta t$ & Predictor & s & run-time & $\max |V_i - \hat{V}_i|$ & $\max | \omega_i - \hat{\omega}_i|$ \\
$[\si{\second}]$ & & & [\si{\second}] & $\times 10^{-2} [\si{\pu}]$ & $\times 10^{-3} [\si{\pu}]$ \\ \midrule
\multirow{2}{*}{$0.05$} & NN & $5$ & $\bm{1.85}$ & $0.82$ & $0.35$\\
 & RK4 & $5$ & $3.88$ & $\bm{0.31}$ & $\bm{0.29}$\\ \midrule
\multirow{4}{*}{$0.10$} & \multirow{2}{*}{NN} & $2$ & $\bm{0.98}$ & $2.71$ & $1.29$ \\
& & $5$ & $1.05$ & $\bm{1.32}$ & $\bm{0.62}$ \\
 & \multirow{2}{*}{RK4} & $2$ & $2.21$ & $5.07$ & $2.40$ \\
& & $5$ & $2.27$ & $3.88$ & $2.01$ \\ \midrule
\multirow{4}{*}{$0.15$} & \multirow{2}{*}{NN} & $2$ & $\bm{0.60}$ & $\bm{6.28}$ & $2.93$ \\
& & $5$ & $0.77$ & $6.36$ & $\bm{2.90}$ \\
 & \multirow{2}{*}{RK4} & $2$ & $1.19$ & $11.8$ & $5.06$ \\
& & $5$ & $1.46$ & $19.0$ & $7.75$ \\
% $\approx$ 0.020 & BDF & - & 0.83 & 0.07 & 0.26\\
% $\approx$ 0.045 & BDF & - & 0.67 & 7.73 & 3.36\\
\bottomrule
\end{tabular}%
\end{table}



%The last two rows stem from the \gls{DAE}-solver in Assimulo based on a variable order backward differentiation formula (BDF) with different tolerance levels. We observe that the required time-step size is significantly smaller than for the proposed simulator. The resulting run-times are comparable, however, neither method was optimized for run-tme in this study. \bz{I would delete the the last two rows and skip this discussion. Not sure how this helps the paper.}