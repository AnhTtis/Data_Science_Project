Solving \glspl{DAE} constitutes one of the indispensable computational tasks for analyzing power system dynamics, e.g., for the analysis of transient stability phenomena. The size and complexity of the power system make solving these DAEs computationally demanding. In addition, the ongoing transition of the power system towards more distributed and heavily inverter-based generation increases this burden even further. Unsurprisingly, the acceleration of \gls{DAE} simulations has therefore been a research topic throughout many years. The fundamental problem setup as described in \cite{stott_power_1979} has undergone little change in the past decades. The main lines of work, as \cite{liu_solving_2020} reports, have been on 1) simplifying models, 2) decomposing the problem (spatially, temporally, and across the numerical schemes; see also \cite{gurrala_parareal_2016}) to allow for parallelization, and 3) using pre-computed analytical approximations for faster online evaluation. Additionally, the use of parallel processing offers potential for acceleration \cite{aristidou_time-domain_2015}.

In recent years, using \gls{ML} to solve differential equations has emerged as an active area of research. \Gls{ML}-based methods can be evaluated in real-time and have the capacity to approximate complicated functions, including ones that usually cannot be expressed in closed form. \glspl{PINN}, as reviewed in \cite{karniadakis_physics-informed_2021}, are used for approximating the solution of partial differential equation. The ideas of domain decomposition are used in~\cite{ameya_extended_2020, heinlein_combining_2021} and can extend beyond PINNs. 

In the area of power system dynamics, works in~\cite{misyris_physics-informed_2020,cui_2023,moya_dae-pinn_2023} have used \gls{ML}-based methods to predict transient behaviors of these systems. They effectively provide approximate \gls{DAE}-solvers. However, it is unclear whether these approaches are scalable with respect to the system size and different network configurations without becoming prohibitively expensive to train. A potential alternative to address this problem of scalability is the development of hybrid methods, that combine \gls{ML}-based approaches with conventional numerical \gls{DAE}-solution schemes to leverage their respective advantages. In~\cite{xiao_feasibility_2022}, the authors train a \gls{NN} to approximate the unknown dynamics of a system component which can then be integrated into a \gls{DAE}-solver. But this approach does not fundamentally overcome the scaling challenges in solving DAEs. 

In order to enable larger time-step sizes and hence to accelerate the simulation, we focus in this paper on the use of \glspl{NN} as approximators for the state evolution of single components; we thereby essentially apply a spatial decomposition to the system. In contrast to other explicit approximators for the state evolution of components, \glspl{NN} are fast, highly accurate, and numerically stable. However, it is nontrivial to model the interactions of these components in the network. The solution approach, we present, is structurally similar the \gls{SAS}-based method in~\cite{wang_timepower_2019}.

% The aim of novel methods involved in decomposition and parallelisation is often to enable larger time-steps, therefore reducing the number of iterations and avoiding large matrices. Among these approaches, a popular subset include the \gls{SAS}-methods. 
The \gls{SAS}-based methods make use of time-power series \cite{wang_timepower_2019} or Adomian Decomposition \cite{gurrala_large_2017} to approximate the state evolution. 
% To deal with the , but their radii of convergence are still limited. To deal with this difficulty, \cite{liu_solving_2020} introduces differential transformations to obtain an explicit approximation of the state evolution which is constructed from a recursive scheme. 
An essential part in these approaches is to model the interactions between various components (e.g., how generators interact with each other through power flow along the lines). Since the components are coupled with each other through algebraic equations, one way to \say{anticipate} their interaction is to predict or estimate the evolution of the algebraic variables as a function of time. For example, finding an approximate power series expansion in the time variable can lead to larger time-steps and therefore accelerate the simulations \cite{liu_solving_2020,wang_timepower_2019}. Hence the quality of these approximations is of critical importance. 

% is how they model the interactions of the components within the network which are determined by the evolution of the network algebraic variables. They approximate these algebraic variables as power series with respect to time and evaluate the components' dynamic evolution based on this approximation. This \say{anticipation} of how the interactions play out allows larger time-steps and can therefore accelerate simulations.


In this paper, we propose a \gls{NN}-based simulator for power system dynamics. The simulator consists of a \gls{NN} for each dynamic system component and a network model that governs their interaction. To model these interactions, the \glspl{NN} use the temporal evolution of the network algebraic variables as an input for their prediction. As the temporal evolution of the network algebraic variables is initially unknown -- when it is known, we \say{solved} the problem -- we first estimate it by means of a power series with respect to time. We then formulate an optimization problem to update this estimate by matching it with the prediction of the \glspl{NN}.

%\bz{better link to the next paragraph. What is the contribution of this paper compared to the papers mentioned in this paragraph?}

% Another data-driven approach is presented in \cite{pagnier_model_2021} where a mapping from the governing \glspl{ODE} to a three-dimensional \gls{PDE} is constructed which can then be used for transient analysis. 

%In the spirit of \gls{SciML}, we focus on an established spatial decomposition of the \glspl{DAE}, , but we approach the solution of the sub-domain \glspl{ODE}/\glspl{DAE} problems with \glspl{NN}. The limitation of \gls{SAS} methods is usually the limited range of applicability of the explicit expression, e.g., for time-power series (\cite{wang_timepower_2019}), Adomian Decomposition methods \cite{gurrala_large_2017}, or simple explicit \gls{RK}-schemes. In contrast, the approximation accuracy \glspl{NN} is not a priori tied to the prediction time-step size; it is a design choice in the setup and training process and hence even simple \glspl{NN} can be accurate for much larger time-steps. We, furthermore, adopt the concept of a dynamic bus and expressing the network variables (algebraic) as time-power series from \cite{wang_timepower_2019,liu_solving_2020}. Thereby, the entire solution approach remains in the continuous time-domain and does not require discretisation. 


%There are two common solution approaches: 1) Partitioned-explicit: The DEs and AEs are evaluated separately based on the previous iteration's values. 2) Simultaneous-implicit: The differential equations \labelcref{eq:DE2} are converted to AEs using a numerical scheme, usually the Trapezoidal rule, and then the new system of algebraic equations is solved. The former is fast to evaluate and easy to implement but suffers from \say{interface errors} and instability, while the latter requires the construction of a large Jacobian matrix.

%This paper proposes a multi-step simulator targeted at decomposable problems such as power system dynamics. The spatial decomposition is based on the algorithm in \cite{wang_timepower_2019} which parametrises the evolution of the algebraic network variables as a power series. \bz{It's not clear what the algorithm is actually doing. If it can accommodate any predictor, then what does it do? It seems the main thing that the algorithm does is to reflect the networked effect, where the components interact with each other. If that is the case, it need to be said much more explicitly.} The proposed algorithm can accommodate any explicit predictor function for the components dynamics, i.e., a \gls{SAS}, such as time-power series (\cite{wang_timepower_2019}), Adomian Decomposition methods \cite{gurrala_large_2017}, or explicit \gls{RK}-schemes. However, to circumvent the limited radius of convergence, we propose the use of \glspl{NN} as the explicit predictors for the dynamic components. The utilisation of \gls{AD} allows the necessary computation of derivatives of the predictor with respect to the parametrised network variables.  

\cref{sec:problem_formulation} introduces the problem formulation and \cref{sec:methodology} describes the proposed solution approach. \cref{sec:case_study} introduces the case study and \cref{sec:results} the results. \Cref{sec:discussion} discusses the results and \cref{sec:conclusion} concludes.

%Idea: The component complexity does not affect the size of the implicit set of equations that is to be solved. The NN as a powerful explicit approximator allows fast, stable, and arbitrary long current injection predictions, largely independent of model complexity.
