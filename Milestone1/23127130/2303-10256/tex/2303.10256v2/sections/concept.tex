\section{Concept}\label{sec:concept}

To facilitate the presentation of the methodology in \cref{sec:methodolgy}, we want to first describe the conceptual idea behind PINNSim. It originates from the problem that we face when solving \glspl{DAE}, a form of differential equations where a set of algebraic equations constraints the differential equations. It formulates as
\begin{subequations}\label{eq:DAE_general}%
\begin{align}
    \ddt \xstate &= \fupdateof{\xstate, \ystate}\\
    \bm{0} &= \gupdateof{\xstate, \ystate}
\end{align}
\end{subequations}
and we refer to $\xstate(t)$ as the differential variables and to $\ystate(t)$ as the algebraic variables. The update function \fupdateof{\xstate, \ystate} and the algebraic relationship  \gupdateof{\xstate, \ystate} govern the dynamics of the system\footnote{For notational clarity, we formulate an autonomous, unforced system. The conceptual idea can also accommodate non-autonomous and forced systems.}. Our interest focuses on the particular form of index-1 \glspl{DAE} or semi-analytical \glspl{DAE} \cite{brenan_numerical_1995}. This form implies that \gupdate{} can be differentiated once with respect to time $t$ which is possible when $\frac{\partial \gupdate{}}{\partial \ystate}$ is non-singular. Based on \cref{eq:DAE_general}, we could describe the temporal evolution of \xstate{} and \ystate{} as 
\begin{subequations}\label{eq:DAE_general_integral}
\begin{align}
    \xstate(t) &= \xinitial + \int_{t_0}^t \fupdateof{\xstate, \ystate} d\tau\\
    \ystate(t) &= \gupdate'(\xstate)
\end{align}
\end{subequations}
where $\xinitial = \xstate(t_0)$ represents the initial condition and $\gupdate'(\xstate)$ describes the solution to the algebraic equations given \xstate{}. However, usually no analytical expression describes the evolution $\xstate(t)$ and $\ystate(t)$ for a given \xinitial{}. Hence, we revert to numerical integration methods to obtain an approximate solution. 

To resolve the non-trivial integration operation and the implicit relationship \gupdate{}, numerical schemes often restrict the functional form of \xstate{} to a certain approximation \xstatehat{}. For instance, \gls{RK} methods assume a polynomial form of $\xstatehat{}(t) = \xinitial + a_1 (t-t_0) +  a_2 (t-t_0)^2 + \dots$ as they match the Taylor expansion up to a certain degree by construction. The different \gls{RK} schemes prescribe the order of the scheme and the computation of the coefficients. When algebraic variables are present, they have to be interfaced with the approximation of the differential variable to incorporate their interaction with each other. Simultaneous and partitioned integration methods describe such routines \cite{stott_power_1979} and are used in many variations. The accuracy of these constructions is dependent on the order of the used integration scheme and potential \say{interface} errors. These considerations and aspects of numerical stability limit the usable time step size. 

% We query this approximation at sufficiently many points to render the problem fully determined. For example, the explicit 4th-order \gls{RK} scheme uses the initial condition \xinitial{} and 4 additional conditions to fit a fourth-order polynomial, while the implicit trapezoidal rule defines a second-order polynomial based on \xinitial{} and two additional conditions. For \glspl{DAE} the algebraic variables \ystatehat{} will, depending on the approach, fulfil (simultaneous-implicit) or approximately fulfil (partitioned-explicit) the algebraic relationship at the points for which the above-mentioned conditions are evaluated \cite{stott_power_1979}. The accuracy of this construction is dependent on the order of the scheme and sets a practical limit on the time step size. 

By choosing a different functional form for \xstatehat{}, like Fourier-series based or around the Adomian decomposition \cite{adomian_review_1988}, the practical time step size might be increased. Due to their high flexibility of their functional form, \glspl{PINN}, and \glspl{NN} in general as suggested in \cite{lagaris_artificial_1998}, can allow significantly larger time steps. In fact, we can even choose a functional form of $\ystatehat{}(t)$ and approximate $\xstatehat{} = \PINN(t, \ystatehat{})$ in dependence of it. However, this great approximation flexibility of \glspl{PINN} comes with the challenge of generalising well across the entire domain of interest, i.e., being accurate over the entire domain and not only on the training dataset. When the dimensionality of \xstate{} and \ystate{} in \cref{eq:DAE_general} increases, the training of a \gls{PINN} to approximate a wide range of solutions becomes increasingly difficult and eventually intractable. With PINNSim, we avoid this problem by exploiting the structure of the power system specific \glspl{DAE}. The structure allows a decomposition of \cref{eq:DAE_general} into multiple smaller sub-problems which remain tractable from a learning perspective. At the same time, the use of \glspl{PINN} allows for large and accurate time steps of these sub-problems. To simulate the entire system, we need to align the sub-problems' solutions by enforcing the algebraic relationship \gupdateof{\xstate{}, \ystate{}}.