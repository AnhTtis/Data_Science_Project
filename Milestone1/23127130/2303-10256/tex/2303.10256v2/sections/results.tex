\section{Results}\label{sec:results}

The potential for accelerating \glspl{TDS} with PINNSim relies on increasing the time step size and thereby reducing the total number of time steps compared to established methods such as the trapezoidal method. We demonstrate this behaviour in \cref{subsec:full_simulation} on the test case. In \cref{subsec:single_step}, we then illustrate what allows PINNSim these larger time steps and describe in \cref{subsec:computational_cost} how the involved computational cost compare.

\subsection{Accuracy of PINNSim for a full TDS}\label{subsec:full_simulation}
First, we consider a simulation of the described test case over \SI{2.5}{\second}. \Cref{fig:trajectory_analysis} shows in the upper panels the resulting trajectory of the frequency deviation at machine 2, i.e., $\Delta \omega_2$. While PINNSim and the trapezoidal rule accurately capture the dynamics for a time step size of $\Delta t = \SI{0.05}{\second}$, the trapezoidal rule fails to track the evolution for a larger time step size of $\Delta t = \SI{0.25}{\second}$. In contrast, PINNSim captures the state evolution accurately.

We test this behaviour for more time step sizes and plot the maximum error over this time interval in the lower panel in \cref{fig:trajectory_analysis}. The trapezoidal rule performs well for very small time step sizes, but the errors quickly become very large - for reference, the overall variation of $\Delta \omega_2$ is around \SI{0.8}{\hertz}. PINNSim leads to an almost constant error characteristics, mostly with less than \SI{0.01}{\hertz} error. The slightly increasing error towards smaller time step sizes arises due to accumulating errors. The larger errors for $\Delta t > \SI{0.3}{\second}$ are expected, as we trained the \glspl{PINN} only up this time step size.

\begin{figure}[ht]
    \centering
    \includegraphics[width=\linewidth]{figures_pdf/trajectory_analysis.pdf}
    \caption{The upper panels show the simulated trajectories of machine 2 with a trapezoidal and PINNSim time stepping schemes for two time step sizes. The low panel shows the maximum error for $\Delta \omega_2$ over the trajectory for a range of time step sizes.}
    \label{fig:trajectory_analysis}
\end{figure}

% This very different error characteristic of PINNSim allows us to achieve the same accuracy as a trapezoidal rule while using a larger time step size, here up to five times. Each time step of PINNSim will be computationally more expensive but if we can keep this increase in checks, here less than five times, we will achieve an overall reduction in simulation time.
% To analyse this relation further, we consider in the following a single time step and how the elements of \cref{algo:simulator} affect accuracy and computation time.

\subsection{Accuracy of PINNSim on a single time step}\label{subsec:single_step}

In the following, we consider a single time step and show the influencing factors on the accuracy. To obtain a range of system states, we simulate the test case for \SI{10}{\second} with an accurate solver and then extract 200 instances, i.e., every \SI{0.05}{\second}.

\subsubsection{Voltage parametrisation} 

We use these 200 instances as initial values to test the performance of PINNSim on a single time step for different values of $\Delta t$, the voltage profile order $r$ and the number of query points $s$. 
\begin{figure}[th]
    \centering
    \includegraphics{figures_pdf/time_step_analysis.pdf}
    \caption{Maximum error for single time step predictions of different length $\Delta t$ starting from 200 different initial conditions. Comparison between the trapezoidal rule and PINNSim with different voltage scheme orders ($r$) and number of query points $s$. Larger $r$ and $s$, both improve the accuracy of PINNSim.}
    \label{fig:time_step_analysis}
\end{figure}
We evaluate these time steps with respect to the maximum error of $\Delta \omega_2$ at the end of the time step. \Cref{fig:time_step_analysis} reports the results. A main observation is that PINNSim with $r=1$ (linear voltage evolution) and $s=r+1$ has a comparable relationship between accuracy and time step size as the trapezoidal rule. However, when using PINNSim with $r=2$, the time step size can be significantly increased without incurring much larger errors. For very small time step sizes PINNSim offers no benefits over the trapezoidal rule as the error decreases slower when reducing $\Delta t$. We can improve this characteristic for PINNSim by evaluating \cref{eq:objective_approximation} on more points, i.e., choose a larger value for $s$, shown as the dashed lines. The improved performance, in particular for $r=1$, originates from the better approximation of \cref{eq:objective_voltage_update_exact}. However, for large time steps, here for around $\Delta t > \SI{0.2}{\second}$, we additionally require $r=2$ for accurate predictions; the voltage evolution becomes too non-linear, hence, the linear voltage approximation ($r=1$) is insufficient. For $r=2$, the limitation in time step size arises as training domain of the \glspl{PINN} was restricted to $\Delta t \leq \SI{0.3}{\second}$.

\subsubsection{PINN accuracy} The performance of PINNSim relies in large parts on how well each \gls{PINN} approximates the integration in \cref{eq:x_integral} for a given voltage profile $\Vcomplexihat{i}(t, \Vparamsi{i})$. \Cref{fig:error_characteristic_PINN} illustrates the results on a test dataset of 4000 points for generator 2. Both panels show the same results, i.e., the maximum and median error on the test dataset, but on a logarithmic and linear x-axis. The logarithmic axis clearly shows that for small time step sizes, the error of PINNs decreases as we included a time dependency in the final layer \cref{eq:NN_output}. The linear x-axis gives a more intuitive understanding of how much further \glspl{PINN} can predict the dynamics accurately. Only when reaching the limit of the training domain, the accuracy deteriorates. Revisiting the error plots in \cref{fig:trajectory_analysis,fig:time_step_analysis}, we can find that it is the particular error characteristic of \glspl{PINN} that allows the large time steps of PINNSim and why it outperforms the trapezoidal method. 
\begin{figure}[ht]
    \centering
    \includegraphics[width=\linewidth]{figures_pdf/pinn_accuracy.pdf}
    \caption{Error characteristics of the PINN for machine 2 on a test dataset. Both plots show the same results but on a logarithmic x-axis (left) and a linear x-axis (right) to highlight the accuracy of PINNs over large time steps.}
    \label{fig:error_characteristic_PINN}
\end{figure}

\subsection{Computation cost of a time step}\label{subsec:computational_cost}
The larger time step sizes of PINNSim come at increasd computational cost compared to the trapezoidal rule. Providing a fair comparison based on optimised implementations will be future work, instead, we will here focus on the main cost of PINNSim from a perspective of scalability. 

In \cref{algo:simulator}, the evaluation of the inner loop (lines 2-8) requires the calculation of network and component current injections \IcomplexNetworkHat{}, \IcomplexComponentHat{} and their sensitivity to \Vparams{}. For the network currents, this requires merely matrix-vector products and for the component currents the evaluations of the \glspl{PINN}. A single evaluation (pass) of a \gls{PINN} requires on the order of \SI{1}{\micro\second}, e.g., to compute the current injection, and the sensitivity calculations requires only two additional passes thanks to \gls{AD}. As all \glspl{PINN} could be evaluated in parallel, these computations can easily be scaled to large system. The \say{expensive} computation arises in the solution of the linear system of equations in line 9 of \cref{algo:simulator}, but its sparsity and known structure can be exploited. A closely related calculation is required for the trapezoidal rule.

The overall cost then scales linearly with the number of iterations $k$ per time step, i.e., line 1-10 in \cref{algo:simulator}, until convergence is reached. We show in \cref{fig:convergence_analysis} the value of the objective \cref{eq:objective_approximation} over the iterations for two time step sizes and different values of $r$. In the left panel, the least square problem is fully determined ($s=r+1$), hence the objective value continues to decrease. In the right panel, the five additional query points render the problem over-determined, hence the algorithm converges at a non-zero objective value. In either case, we observe that larger time step sizes require more iterations, while the order of the voltage profiles primarily affects the magnitude of the objective value and less its convergence.
\begin{figure}[th]
    \centering
    \includegraphics[width=\linewidth]{figures_pdf/convergence_analysis.pdf}
    \caption{Convergence of the objective value in \cref{eq:objective_approximation} for different number of query points $s$, voltage profile orders $r$ and time step sizes $\Delta t$.}
    \label{fig:convergence_analysis}
\end{figure}

As stated in \cite{stiasny_physics-informed_2023,stiasny_physics-informed_2023-1}, the cost of training the \glspl{PINN} has to be considered as well. Improving the efficiency of the training, however, can be decoupled from the analysis of PINNSim. 