\section{Introduction}\label{sec:introduction}

Differential equations form a centre piece in the modelling of the dynamic behaviour of power systems. They provide very widely applicable component and system models, however, their solution requires numerical integration methods. These tools for \glspl{TDS} constitute a workhorse of power system analysis and can reliably deliver accurate solutions. The used algorithms are scalable, versatile, and applicable without case-specific modifications, but they incur a high computational burden \cite{stott_power_1979}. Efforts around improved solvers and parallelisation can reduce this burden, see \cite{liu_solving_2020, gurrala_parareal_2016, aristidou_time-domain_2015} for an overview. However, a fundamental problem is the small time step size that is required by the schemes to ensure accurate results and numerical stability. Hence many time steps and function evaluations are necessary.

The proposition of \gls{NN}-based approaches \cite{lagaris_artificial_1998}, and in particular \glspl{PINN} \cite{raissi_physics-informed_2018}, has very different characteristics. In the training process, \glspl{PINN} learn the solution to the provided differential equations and subsequently provide fast and sufficiently accurate solutions over long time intervals -- addressing two main issues of conventional numerical integration methods. \glspl{PINN} have been adopted to power system dynamics in \cite{misyris_physics-informed_2020} to predict single machine dynamics. In \cite{moya_approximating_2023}, the related methodology of operator learning is also applied to single machine dynamics. The applicability of learning-based methods to multi-machine systems is explored in \cite{stiasny_physics-informed_2023} and with alternative architectures in \cite{cui_predicting_2021,moya_dae-pinn_2023}. The speed advantage persists and the accuracy remains sufficiently high for the trained setup. However, the common challenge lies in achieving generalisation, i.e., being able to apply the same \gls{NN} for an unseen task without retraining. Such flexibility would make \gls{NN}-based \gls{TDS} approaches comparable with conventional numerical integration methods. However, in the case of multi-machine systems, this requires the generalisation to a very large variety of setups. As a consequence, the dimensionality of the associated learning tasks increases dramatically with the system size and thereby becomes very or even too costly to learn \cite{stiasny_physics-informed_2023-1}.

To nonetheless harness the advantages of \glspl{PINN}, i.e., fast solutions for large time steps, we propose \textit{PINNSim}. This algorithm allows a modular integration of \glspl{PINN} into \glspl{TDS} for multi-machine systems. We learn the solution of all dynamic components in the system separately, i.e., with entirely independent \glspl{PINN}, and before executing any simulation. Thereby, the learning tasks are of relatively low dimensionality and hence remain tractable. To perform a \gls{TDS}, we solely need to determine the interaction between the components. To this end, we utilise a root-finding algorithm similar to conventional \gls{TDS} algorithms; at this stage, no learning is required. As a result, the scalability and flexibility of PINNSim become decoupled from the difficulties of learning high-dimensional problems. At the same time, we still can benefit from fast and accurate solutions for larger time step sizes due to the use of \glspl{PINN}.
% Include numerical consistency in PINN prediction - reducing time step size improves prediction quality

In \cref{sec:concept}, we present the conceptual motivation behind PINNSim. In \cref{sec:methodolgy}, we introduce the power system formulation and the methodology of PINNSim. \Cref{sec:case_study} describes the case study and in \cref{sec:results}, we discuss the results by highlighting key characteristics of PINNSim. \Cref{sec:discussion} discusses the steps for developing PINNSim from a proof of concept to a fully fledged simulator. \Cref{sec:conclusion} concludes.