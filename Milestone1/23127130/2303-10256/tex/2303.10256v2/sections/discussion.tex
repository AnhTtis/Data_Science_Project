\section{Discussion}\label{sec:discussion}

The presented results shall serve as a proof of concept of PINNSim as a novel time stepping simulator. We demonstrated that PINNSim outperforms the trapezoidal rule on the metric of the allowable time step size. The following describes four aspects that are necessary to turn PINNSim from a proof of concept into a fully fledged simulator for power system dynamics and to realise the potential acceleration of \glspl{TDS}.

\begin{enumerate}
    \item \textit{Speed}: Numerical methods require a high level of optimisation in the implementation to become competitive. For PINNSim this concerns primarily the optimisation of the calculation of the residual \residualError{}, the Jacobian \Jacobian{}, and the solution of the sparse linear system in \cref{eq:residual_least_square}. The focus should therefore lie on controlling memory allocations, utilising parallelisation, and exploiting sparsity patterns for these computations.
    \item \textit{Accuracy}: The accuracy of PINNSim hinges around accurately learned \glspl{PINN}. We envision that this process can be highly standardised, so that \glspl{PINN} can be trained reliably to high accuracy and with desirable error characteristics for a wide range of dynamic components.
    \item \textit{Scale}: The analysis of the computational complexity of PINNSim suggests its scalability but it remains to be shown in practice. Furthermore, we need to investigate the accuracy and convergence properties of the voltage update scheme for larger systems.
    \item \textit{Applications}: In this work, we have considered the dynamics of classical machine models under a set-point change. Future work should include more detailed dynamical models and apply PINNSim to short-circuits, topology changes, and other numerically more demanding setups.
\end{enumerate}