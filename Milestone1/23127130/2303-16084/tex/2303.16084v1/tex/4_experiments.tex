\section{Experiments}
\label{sec:experiments}

We report results on the three most commonly used benchmarks for few-shot action recognition, \ie \kinetics~\cite{zhu2018cmn}, Something-Something V2 (\ssvtwo) \cite{ssv2}, and \ucf \cite{ucf101}. 
\kinetics and \ucf contain videos collected from YouTube. Each video is trimmed to include only one coarse-grained human action such as ``playing trumpet'' or ``reading book''. \ssvtwo  contains egocentric videos where humans were instructed to perform predefined actions such as ``pushing something from left to right'' or ``dropping something into something''. 

We use the train/val/test splits from~\cite{zhu2018cmn} for all three datasets, containing 64/12/24 classes, respectively. 
We learn the parameters of the R(2+1)D backbone using the train split, similar to~\cite{tsl}. 
For matching-based approaches, we learn the projection $\phi$ together with any learnable parameters in the matching function $f$ using episodic training also on the train split. We use the val split for hyper-parameter tuning and early stopping.

\looseness=-1
We evaluate on the common 1-shot and 5-shot setups.
Unless otherwise stated, we use 5-way classification tasks.
Training episodes are randomly sampled from the train set.
We use the same fixed, predefined set of 10k test episodes sampled from the test set for all methods 
% (prior work randomly samples them each time~\cite{otam,trx,Zhu2021CloserLook}). 
(prior work samples them randomly). 
We always evaluate \textit{three} trained models and report mean and standard deviation.



\subsection{Implementation details}
\label{sec:implementationdetails}
We use the publicly available code for TSL\footnote{\url{https://github.com/xianyongqin/few-shot-video-classification}} for 
training the backbone as well as for reproducing the TSL method. We adapt the public TRX\footnote{\url{https://github.com/tobyperrett/trx}} codebase for learning parameters of the matching function and testing,
as well as for reproducing the TRX and OTAM methods. 


%\iffalse 
\begin{table*}[t]
\centering
\begin{tabular}{c|c|c|c|c|c||c|c|c}
\toprule
\multirow{2}{*}{Backbone} &    \multirow{2}{*}{Method} & \multicolumn{4}{c||}{ID: Pascal} & \multicolumn{3}{c}{ID: Cityscapes}  \\
 &           & Comic & Watercolor & Clipart & ID                           & Foggy & BDD  & ID    \\\hline
\multirow{4}{*}{ResNet50 Instagram~\cite{mahajan2018exploring}}&DP                                   & 15.7  & 21.2       & 15.3    & 44.6 &                                    13.9 & 7.7 & 28.3  \\
&FT                                   & 7.5   & 19.4       & 11.4    & 50.4 &                                     12.8  & 5.1  & 33.5 \\
%&FT + Augmix                          & 10.2  & 21.9       & 12.4    & 46.3 &                                    &       &             \\
& \CCG DP-FT                                &\CCG 9.1   & \CCG21.0       & \CCG12.9    &\CCG 52.6 &\CCG14.8  &\CCG 5.5  &\CCG\bf{34.7}  \\
 &\CCG DP-FT + WR           &\CCG \bf{16.8}  &\CCG \bf{26.5}       &\CCG \bf{17.6}    &\CCG \bf{52.9} &\CCG \bf{19.3}  &\CCG \bf{9.6}  &\CCG 34.5   \\\hline
 %&\CCG All                &\CCG \bf{18.9}  &\CCG \bf{27.5}       &\CCG \bf{21.4}    &\CCG 52.2 &\CCG  \\\hline                                     
\multirow{4}{*}{ConvNeXt IN21K~\cite{liu2022convnet}}&DP                                   & 11.7  & 17.3       & 14.0    & 39.7 &                                     14.7 & 7.8 & 31.1 \\
&FT                              &      11.5  & 22.9       & 16.8    & 60.6 &  18.1 & 9.7 & 35.8    \\
%&FT + Augmix                  &         15.5  & 28.6       & 20.7    & 61.4 &                                    &      &             \\
&\CCG  DP-FT                   & \CCG              13.6  &\CCG  24.7       &\CCG  19.1    &\CCG  \bf{62.3} & \CCG                                   20.5&\CCG 11.5&\CCG   37.1\\
%&\CCG  DP-FT + Seblock         &\CCG               15.4  &\CCG  27.5       &\CCG  20.9    &\CCG  61.6 &\CCG   \bf{22.0}  &\CCG  11.3 &\CCG  36.6            \\
&\CCG DP-FT + WR     &\CCG       \bf{14.6}  &\CCG  \bf{27.8}       &\CCG  \bf{19.7}    &\CCG  61.4 & \CCG                          \textbf{21.1}          &\CCG    \textbf{11.7}     &\CCG  \bf{37.2}   \\\hline
%&\CCG  All              &\CCG   \bf{17.8}  &\CCG  \bf{29.3}       &\CCG  \bf{23.8}    &\CCG  \bf{61.0}   &\CCG  21.7&\CCG  \bf{11.8}&\CCG  37.2            \\\hline
%&DP-FT + Reg + Augmix & 21.0  & 27.9       & 22.6    & 50.4 &            \\    \hline
\multirow{4}{*}{Eff-B2 JFT~\cite{xie2020self}}&  DP                                   & 12.6  & 20.4       & 15.1    & 40.2 &                                     11.1 & 6.9 & 25.2 \\
&FT                                   & 17.1  & 27.2       & 18.0    & 53.4 &          10.7                          &  5.1     &     31.5        \\
%&FT + Augmix                          & 20.1  & 31.2       & 19.7    & 52.7 &                                    &       &             \\
&\CCG  DP-FT                                &\CCG  17.4  &\CCG  29.4       &\CCG  20.7    &\CCG \bf{55.3} &\CCG   12.9&\CCG  7.3&\CCG   \bf{32.9}\\
&\CCG  DP-FT + WR        &\CCG  \bf{19.5}  &\CCG  \bf{30.0}         &\CCG  \bf{22.0}      &\CCG  54.2 &\CCG \bf{13.5} &\CCG  \bf{7.6} &\CCG  32.5\\

% \bf{13.1}&\CCG  \bf{7.5}&\CCG  32.7\\
\bottomrule
\end{tabular}
\vspace{-3mm}
\caption{Effect of weight regularization. DP, FT, and WR denote decoder-probing, fine-tuning, and weight regularization.} 
\label{tb:main}
\end{table*}

\iffalse 

\begin{table*}[]
\centering
\begin{tabular}{c|c|c|c|c|c|c}
\toprule
 \multirow{1}{*}{Backbone} &    \multirow{1}{*}{Method}           & Comic & Watercolor & Clipart & OOD Average& ID: Pascal          \\\hline
\multirow{6}{*}{ResNet50 Instagram}&DP                                   & 15.7  & 21.2       & 15.3    & 17.4&44.6   \\
&FT                                   & 7.5   & 19.4       & 11.4    & 12.8&50.4 \\
%&FT + Augmix                          & 10.2  & 21.9       & 12.4    & 46.3 &                                    &       &             \\
& \CCG DP-FT                                &\CCG 9.1   & \CCG21.0       & \CCG12.9  &\CCG 14.3  &\CCG 52.6 \\
&\CCG DP-FT + Seblock                      &\CCG 10.2  & \CCG 22.5       &\CCG 15.1    &\CCG 15.9 & \CCG \bf{53.4} \\
 &\CCG DP-FT + Reg           &\CCG 16.8  &\CCG 26.5       &\CCG 17.6    &\CCG 19.9&\CCG 52.9 \\
 &\CCG All                &\CCG \bf{18.9}  &\CCG \bf{27.5}       &\CCG \bf{21.4} & \CCG \bf{22.6} &\CCG 52.2 \\\hline                                     
\multirow{6}{*}{Convnext IN21K}&DP                                   & 11.7  & 17.3       & 14.0    & 14.3&39.7 \\
&FT                              &      11.5  & 22.9       & 16.8    &17.1& 60.6    \\
%&FT + Augmix                  &         15.5  & 28.6       & 20.7    & 61.4 &                                    &      &             \\
&\CCG  DP-FT                   & \CCG              13.6  &\CCG  24.7       &\CCG  19.1    &\CCG19.1&\CCG  62.3 \\
&\CCG  DP-FT + Seblock         &\CCG               15.4  &\CCG  27.5       &\CCG  20.9    &\CCG 21.3&\CCG  61.6 \\
&\CCG DP-FT + Reg     &\CCG       14.6  &\CCG  27.8       &\CCG  19.7    &\CCG20.7&\CCG  61.4 \\
&\CCG  All              &\CCG   \bf{17.8}  &\CCG  \bf{29.3}       &\CCG23.6&\CCG  \bf{23.8}    &\CCG  \bf{61.0}   \\\hline
%&DP-FT + Reg + Augmix & 21.0  & 27.9       & 22.6    & 50.4 &            \\    \hline
\multirow{4}{*}{Eff-B2 JFT}&  DP                                   & 12.6  & 20.4       & 15.1    &  16.0&40.2 \\
&FT                                   & 17.1  & 27.2       & 18.0    &20.8& 53.4 \\
%&FT + Augmix                          & 20.1  & 31.2       & 19.7    & 52.7 &                                    &       &             \\
&\CCG  DP-FT                                &\CCG  17.4  &\CCG  29.4       &\CCG  20.7    &\CCG 22.5&\CCG  \bf{55.3} \\
&\CCG  DP-FT + WR        &\CCG  \bf{19.5}  &\CCG  \bf{30.0}         &\CCG  \bf{22.0}      &\CCG 23.8&\CCG  54.2 \\
\bottomrule
\end{tabular}
\caption{Improvements by introducing our proposed modules.}
\label{tb:main}
\end{table*}
\fi

%\fi


\iffalse 
\begin{table*}[]
\centering
\begin{tabular}{c|c|c|c|c|c||c|c|c}
\toprule
\multirow{2}{*}{Model} &    \multirow{2}{*}{Method} & \multicolumn{4}{c||}{ID: Pascal} & \multicolumn{3}{c}{ID: Cityscape}  \\\cline{3-9}
 &           & Comic & Watercolor & Clipart & ID                           & Foggy & BDD  & ID    \\\hline
\multirow{6}{*}{ResNet50 Instagram}&DP                                   & 15.7  & 21.2       & 15.3    & 44.6 &                                    13.9 & 7.66 & 28.3  \\
&FT                                   & 7.5   & 19.4       & 11.4    & 50.4 &                                     12.8  & 5.1  & 33.52 \\
%&FT + Augmix                          & 10.2  & 21.9       & 12.4    & 46.3 &                                    &       &             \\
& \CCG DP-FT                                &\CCG 9.1   & \CCG21.0       & \CCG12.9    &\CCG 52.6 &\CCG14.8  &\CCG 5.5  &\CCG34.7  \\
&\CCG DP-FT + SE                      &\CCG 10.2  & \CCG 22.5       &\CCG 15.1    &\CCG \bf{53.4} &\CCG                                    &\CCG       &\CCG             \\
 &\CCG DP-FT + WR           &\CCG 16.8  &\CCG 26.5       &\CCG 17.6    &\CCG 52.9 &\CCG \bf{19.3}  &\CCG \bf{9.6}  &\CCG 34.5   \\
 &\CCG All                &\CCG \bf{18.9}  &\CCG \bf{27.5}       &\CCG \bf{21.4}    &\CCG 52.2 &\CCG  \\\hline                                     
\multirow{6}{*}{Convnext IN21K}&DP                                   & 11.7  & 17.3       & 14.0    & 39.7 &                                     14.7 & 7.8 & 31.1 \\
&FT                              &      11.5  & 22.9       & 16.8    & 60.6 &  18.1 & 9.7 & 35.8    \\
%&FT + Augmix                  &         15.5  & 28.6       & 20.7    & 61.4 &                                    &      &             \\
&\CCG  DP-FT                   & \CCG              13.6  &\CCG  24.7       &\CCG  19.1    &\CCG  62.3 & \CCG                                   20.5&\CCG 11.5&\CCG  37.1\\
&\CCG  DP-FT + Seblock         &\CCG               15.4  &\CCG  27.5       &\CCG  20.9    &\CCG  61.6 &\CCG   \bf{22.0}  &\CCG  11.3 &\CCG  36.6            \\
&\CCG DP-FT + Reg     &\CCG       14.6  &\CCG  27.8       &\CCG  19.7    &\CCG  61.4 & \CCG           20.9                         &\CCG   11.8     &\CCG  \bf{37.5}             \\
&\CCG  All              &\CCG   \bf{17.8}  &\CCG  \bf{29.3}       &\CCG  \bf{23.8}    &\CCG  \bf{61.0}   &\CCG  21.7&\CCG  \bf{11.8}&\CCG  37.2            \\\hline
%&DP-FT + Reg + Augmix & 21.0  & 27.9       & 22.6    & 50.4 &            \\    \hline
\multirow{4}{*}{Eff-B2 JFT}&  DP                                   & 12.6  & 20.4       & 15.1    & 40.2 &                                     11.1 & 6.9 & 25.2 \\
&FT                                   & 17.1  & 27.2       & 18.0    & \bf{53.4} &          10.7                          &  5.1     &     31.5        \\
%&FT + Augmix                          & 20.1  & 31.2       & 19.7    & 52.7 &                                    &       &             \\
&\CCG  DP-FT                                &\CCG  17.4  &\CCG  29.4       &\CCG  20.7    &\CCG  55.3 &\CCG   12.9&\CCG  7.3&\CCG   \bf{32.9}\\
&\CCG  DP-FT + Reg        &\CCG  \bf{19.5}  &\CCG  \bf{30.0}         &\CCG  \bf{22.0}      &\CCG  54.2 &\CCG  \bf{13.1}&\CCG  \bf{7.5}&\CCG  32.7\\
\bottomrule
\end{tabular}
\caption{Improvements by Applying weight regularization.}
\label{tb:main}
\end{table*}
\fi


\noindent\textbf{Learning the backbone parameters.}
We start from the publicly available 34-layer R(2+1)D backbone provided by the TSL codebase. This model is pre-trained on the large Sports-1M dataset~\cite{karpathy2014large}. We follow~\cite{tsl} and use a SGD optimizer with a constant learning rate of 0.001 for the backbone and 0.1 for the 64-class linear layer. We perform early-stopping using the 64-class validation dataset. 
Then, we use the backbone as a feature extractor to extract features from $n=8$ uniformly sampled clips\footnote{Note that although TSL uses randomly-sampled clips, we found that its performance is usually better when switching to uniformly-sampled clips.}. The input video clips are composed of 16 consecutive RGB frames with a spatial resolution of 112x112, and the dimensionality of the resulting feature vector is $d=512$.

\noindent\textbf{Training matching-based methods.}
We train the matching-based methods on the training episodes with an SGD optimizer and a constant learning rate of 0.001 for every method except for TRX where we use a learning rate of 0.01. 
Similar to prior work~\cite{otam,trx}, we select the best model using early stopping by measuring performance on the
validation set. We learn the projection $\phi$ jointly with any matching parameters. We set the projection dimension to $D=1152$ if not stated otherwise. This is equivalent to the dimensionality that TRX\cite{trx} uses for its attention layer. We train all the matching methods with the cross-entropy loss that uses softmax with a learnable temperature $\tau$.

\noindent\textbf{Learning classifiers for TSL.}
Instead of pairwise matching, TSL~\cite{tsl} learns classifiers at every test episode, using all available support examples. To reproduce TSL we
follow~\cite{tsl} and use the Adam optimizer with a constant learning rate of 0.01 for 10 epochs. 
The original TSL approach uses $n=10$ clips at test-time, but we set this number to $n=8$ to keep it the same with all matching methods for a fair comparison. Preliminary experiments show that this choice doesn't affect the performance of TSL at all.

\noindent\textbf{Data augmentation.} During 
training, videos are augmented with random cropping. 
We uniformly sample 8 clips from each video with temporal jittering, \ie  randomly perturb the starting point of each clip.
Additionally, for the \kinetics dataset, we also use random horizontal flipping as data augmentation. 
Since it is important for \ssvtwo to distinguish between
left-to-right and right-to-left, we do not use horizontal flipping for that dataset.
We only apply a center crop for videos during validation and testing.





\subsection{Results}
\label{sec:results}
\begin{figure}[t]
\vspace{-3ex}
\centering

\begin{subfigure}{\textwidth}
\begin{tikzpicture}
\begin{axis}[
  width=\linewidth,
  height=5cm,
  xtick = {0,1,2,3,4,5,6,7,8,9,10,11,12, 13, 14, 15, 16},
  xticklabels = {CMN++~\cite{otam},TRX~\cite{trx},Base+~\cite{Zhu2021CloserLook},OTAM~\cite{otam},  PAL \cite{Zhu2021pal}, TA$^2$N~\cite{li2021ta2n}, TAP~\cite{su2022temporal}, MTFAN~\cite{Wu_2022_CVPR}, TSL$^\dagger$~\cite{tsl},  Mean$^\dagger$, Max$^\dagger$, Chamfer++$^\dagger$, Diagonal$^\dagger$, Linear$^\dagger$, OTAM$^\dagger$~\cite{otam}, VISIL$^\dagger$~\cite{visil}, TRX$^\dagger$~\cite{trx}},
  x tick label style={rotate=40,anchor=east},
  ylabel={\small Top-1 Accuracy},
  legend pos=south east,
  label style={font=\scriptsize, row sep=0.5pt},
  tick label style={font=\tiny},
  ylabel near ticks, xlabel near ticks, 
  legend style={font=\scriptsize}, 
  minor y tick num=4,
  ]

    \addlegendimage{color=\resnetc,mark=square*,only marks} \addlegendentry{ResNet};
    \addlegendimage{color=\rtwoplusonec,mark=square*,only marks} \addlegendentry{R(2+1)D};
    \addlegendimage{color=black,mark=} \addlegendentry{};
    \addlegendimage{color=black,mark=\markerspatiotemporal,only marks}
    \addlegendentry{Spatio-temporal};
    \addlegendimage{color=black,mark=\markertemporal,only marks} \addlegendentry{Temporal};
    \addlegendimage{color=black,mark=\markernontemporal,only marks} \addlegendentry{Non-temporal};
    \addlegendimage{color=black,mark=o,only marks} \addlegendentry{No matching (classifier)};
    
% absolute
    \addplot[cmn] coordinates {(0, 40.62)};
    \addplot[trxresnet] coordinates {(1, 42.2)};
    \addplot[bmvc] coordinates {(2, 46.04)};
    \addplot[otamresnet] coordinates {(3, 48.8)};
    \addplot[pal] coordinates {(4, 46.4)};
    \addplot[titan] coordinates {(5, 47.6)};
    \addplot[tap] coordinates {(6, 45.2)};
    \addplot[mtfan] coordinates {(7, 45.7)};
    
    \addplot[tsl] coordinates {(8, 60.6)};
    \addplot[mean] coordinates {(9, 65.8)};
    \addplot[max] coordinates {(10, 65.0)};
    \addplot[chamfer] coordinates {(11, 67.8)};
    \addplot[diag] coordinates {(12, 66.7)};
    \addplot[linear] coordinates {(13, 66.6)};
    \addplot[otam] coordinates {(14, 67.1)};
    \addplot[visil] coordinates {(15, 67.7)};
    \addplot[trx] coordinates {(16, 65.5)};
            
\end{axis}
\end{tikzpicture}
 
\caption{\ssvtwo}
\label{fig:teaser_ssv2}
\end{subfigure}  

\begin{tabular}{c}

\begin{subfigure}{0.5\textwidth}
\begin{tikzpicture}
\begin{axis}[
  height=4.9cm,
  xtick = {0,1,2,3,4,5,6,7,8,9,10,11,12, 13, 14, 15, 16},
  xticklabels = {TA$^2$N~\cite{li2021ta2n}, PAL \cite{Zhu2021pal}, Base+~\cite{Zhu2021CloserLook},   MTFAN~\cite{Wu_2022_CVPR}, TSL$^\dagger$~\cite{tsl},  Mean$^\dagger$, Max$^\dagger$, Chamfer++$^\dagger$, Diagonal$^\dagger$, Linear$^\dagger$, OTAM$^\dagger$~\cite{otam}, VISIL$^\dagger$~\cite{visil}, TRX$^\dagger$~\cite{trx}},
  x tick label style={rotate=40,anchor=east},
  ylabel={\small Top-1 Accuracy},
  legend pos=south east,
  label style={font=\tiny, row sep=0.5pt},
  tick label style={font=\tiny},
  ylabel near ticks, xlabel near ticks, 
  legend style={font=\scriptsize}, 
  minor y tick num=4,
  ]


% absolute
    \addplot[titan] coordinates {(0, 72.8)};
    \addplot[pal] coordinates {(1, 74.2)};
    \addplot[bmvc] coordinates {(2, 74.6)};
    \addplot[mtfan] coordinates {(3, 74.3)};
    
    \addplot[tsl] coordinates {(4, 93.6)};
    \addplot[mean] coordinates {(5, 95.5)};
    \addplot[max] coordinates {(6, 95.3)};
    \addplot[chamfer] coordinates {(7, 96.1)};
    \addplot[diag] coordinates {(8, 95.3)};
    \addplot[linear] coordinates {(9, 95.5)};
    \addplot[otam] coordinates {(10, 95.9)};
    \addplot[visil] coordinates {(11, 95.9)};
    \addplot[trx] coordinates {(12, 93.4)};

\end{axis}
\end{tikzpicture}
 
 \vspace{-10pt}
\caption{\kinetics}
\label{fig:teaser_kinetics}
\end{subfigure}  
\begin{subfigure}{0.5\textwidth}
\begin{tikzpicture}
\begin{axis}[
  width=\linewidth,
  height=4.9cm,
  xtick = {0,1,2,3,4,5,6,7,8,9,10,11,12},
  xticklabels = {TA$^2$N~\cite{li2021ta2n}, PAL \cite{Zhu2021pal}, MTFAN~\cite{Wu_2022_CVPR}, TSL$^\dagger$~\cite{tsl}, Mean$^\dagger$, Max$^\dagger$, Chamfer++$^\dagger$, Diagonal$^\dagger$, Linear$^\dagger$, OTAM$^\dagger$~\cite{otam}, VISIL$^\dagger$~\cite{visil},TRX$^\dagger$~\cite{trx}},
  x tick label style={rotate=40,anchor=east},
  ylabel={\small Top-1 Accuracy},
  legend pos=south east,
  label style={font=\scriptsize, row sep=0.5pt},
  tick label style={font=\tiny},
  ylabel near ticks, xlabel near ticks, 
  legend style={font=\scriptsize}, 
  minor y tick num=4,
  ]

% absolute
    \addplot[titan] coordinates {(0, 81.9)};
    \addplot[pal] coordinates {(1, 85.3)};
    \addplot[mtfan] coordinates {(2, 84.8)};
    
    \addplot[tsl] coordinates {(3, 97.1)};
    \addplot[mean] coordinates {(4, 97.6)};
    \addplot[max] coordinates {(5, 97.9)};
    \addplot[chamfer] coordinates {(6, 97.7)};
    \addplot[diag] coordinates {(7, 97.6)};
    \addplot[linear] coordinates {(8, 97.6)};
    \addplot[otam] coordinates {(9, 97.8)};
    \addplot[visil] coordinates {(10, 97.8)};
    \addplot[trx] coordinates {(11, 96.6)};
    
\end{axis}
\end{tikzpicture}
 
 \vspace{-10pt}
\caption{\ucf}
\label{fig:teaser_ucf101}
\end{subfigure}  

\end{tabular}
\caption{\textbf{One-shot performance} for different backbones, types of matching, or use of parametric classifiers. The different colors account for the different backbones. The different shapes account for the type of matching.
$^\dagger$ denotes methods reproduced in this study.}
\label{fig:teasers}
\end{figure}



In this section, we report and analyse our results. We first discuss the gains from using temporal representations and the comparison between matching-based and classifier approaches. We then discuss the use of temporal information during matching and present results when varying the number of classes per classification task (test episode). Finally, we compare Chamfer++ to other recently published methods and show that our approach achieves state-of-the-art performance.

\myparagraph{Frame or clip-based features?} 
We start by evaluating a number of recent matching-based methods over temporal features. This is an important comparison that is missing from the current few-shot action recognition literature.
In Figure~\ref{fig:teasers}, we report one-shot performance for a number of matching methods under a common evaluation setup and using both frame-based (ResNet, blue points) and clip-based (R(2+1)D, orange points) features. 


We clearly see that using a spatio-temporal backbone significantly boosts accuracy by more than 10\% on all datasets. 
Interestingly, this is also true for the \kinetics and the \ucf datasets that are known to be more biased towards spatial context~\cite{huang2018makes}. Even for this case where context is important, we see that temporal dynamics
remain a valuable cue for few-shot action recognition. 
It is worth noting that the performance on \ucf and \kinetics appears to be saturated when using spatio-temporal representations.


\myparagraph{Pairwise matching or classifiers?}
Matching-based methods and 
classifiers are compared in Table~\ref{tab:main} and Figure~\ref{fig:teasers}.
All results in the table are computed under a common framework, \ie all methods share representations from an R(2+1)D backbone and are tested on the same set of episodes. We see that for
all setups and datasets, several matching approaches outperform TSL. In the 1-shot regime, \textit{most} matching-based methods outperform TSL.

\begin{figure*}[tp]
    \centering
    \includegraphics[width=0.9\linewidth]{figs/images/sota_result.pdf}
    \caption{Comparisons with other task-specific methods. We show the results  of: 
    SINet~\cite{fan2020camouflaged} and PFNet~\cite{mei2021camouflaged} on CAMO~\cite{le2019anabranch} dataset for camouflaged object detection (Top-left),
    ManTra~\cite{wu2019mantra} and SPAN~\cite{hu2020span} on CAISA~\cite{dong2013casia} dataset for forgery detection (Top-right),
    MTMT~\cite{mtmt} and FDRNet~\cite{zhu2021mitigating} on ISTD~\cite{wang2018stacked} dataset for shadow detection (Bottom-left), CENet~\cite{zhao2019cenet} and EFENet~\cite{zhao2021defocus} on CUHK~\cite{shi2014discriminative} dataset for defocus blur detection (Bottom-right).}
    \label{fig:sota_result}
\end{figure*}

\myparagraph{How useful is temporal matching?}
No significant difference in performance is observed between temporal and non-temporal matching approaches, as highlighted in Table~\ref{tab:main} and Figure~\ref{fig:teasers}.
On the \kinetics and \ucf datasets, where action classes are generally coarser and highly dependent on context~\cite{huang2018makes}, most methods we tested perform similarly well. Simply using $f(M)= \max_{ij} m_{ij}$ is enough to achieve top performance for \ucf, while the proposed Chamfer++ outperforms all other methods on \kinetics.
When it comes to the finer-grained \ssvtwo dataset, we see that, although most non-temporal matching methods lag behind the temporal ones for the 1-shot case, the proposed Chamfer++ method achieves the highest performance without temporal matching. For 5-shot action recognition, TRX, ViSiL, and Chamfer++ perform similarly well, with the last being parameter-free, more intuitive, and faster. 


\begin{figure*}[t!]
\centering
 \begin{subfigure}{\linewidth}
    \resizebox{\linewidth}{!}{
      \definecolor{C1}{RGB}{226, 43, 41}
\definecolor{C2}{RGB}{47, 96, 206}
\definecolor{C3}{RGB}{246, 175, 11} 

\begin{tikzpicture}
  \begin{axis}[
    width=0.5\linewidth,
    height=5cm, 
    xtick = {5,10,15,20,24},
	legend cell align={left},
	legend pos=north east,
    grid=both,
    ylabel={\small Top-1 Accuracy},
    xlabel={\small $C_f$},
    xlabel style={yshift=10pt, xshift=80pt,anchor=north east},
  ]
    
\addplot [thick, color=C3, mark=square*,  mark size=2] coordinates {(5, 67.83) (10, 54.8) (15, 47.5) (20, 42.8) (24, 39.7)};    
\addlegendentry{Chamfer++} 

\addplot [thick, color=C1, mark=x,  mark size=3] coordinates {(5, 65.5) (10, 51.5) (15, 44.2) (20, 39.6) (24, 36.4)};      
\addlegendentry{TRX-{2,3}} 
	
\addplot [thick, color=C2, mark=triangle*,  mark size=2] coordinates {(5, 60.6) (10, 47.8) (15, 40.87) (20, 36.8) (24, 34.03)};      
\addlegendentry{TSL} 
    
  \end{axis}
\end{tikzpicture}
      \definecolor{C1}{RGB}{226, 43, 41}
\definecolor{C2}{RGB}{47, 96, 206}
\definecolor{C3}{RGB}{246, 175, 11} 


\begin{tikzpicture}
  \begin{axis}[
    width=0.5\linewidth,
    height=5cm, 
    xtick = {5,10,15,20,24},
	legend cell align={left},
	legend pos=north east,
    grid=both,
    ylabel={\small Top-1 Accuracy},
    xlabel={\small $C_f$},
    xlabel style={yshift=10pt, xshift=80pt,anchor=north east},
  ]

\addplot [thick, color=C3, mark=square*,  mark size=2] coordinates {(5, 81.60) (10, 71) (15, 64.9) (20, 60.4) (24, 57.6)};   
\addlegendentry{Chamfer++} 

\addplot [thick, color=C1, mark=x,  mark size=3] coordinates {(5, 81.8) (10, 72.2) (15, 66) (20, 61.6) (24, 58.7)};     
\addlegendentry{TRX-{2,3}} 
	
\addplot [thick, color=C2, mark=triangle*,  mark size=2] coordinates {(5, 79.9) (10, 70.07) (15, 64.13) (20, 59.9) (24, 57.2)};  
\addlegendentry{TSL} 
    
  \end{axis}
\end{tikzpicture}
    }
    \vspace{-15pt}
    \caption{\textbf{\ssvtwo} dataset. 1-shot (left) and  5-shot (right)}
    \label{fig:ways_ssv2}
\end{subfigure}     

\vspace{20pt}
\begin{subfigure}{\linewidth}
    \resizebox{\linewidth}{!}{
      \definecolor{C1}{RGB}{226, 43, 41}
\definecolor{C2}{RGB}{47, 96, 206}
\definecolor{C3}{RGB}{246, 175, 11} 

\begin{tikzpicture}
  \begin{axis}[
    width=0.5\linewidth,
    height=5cm, 
    xtick = {5,10,15,20,24},
	legend cell align={left},
	legend pos=north east,
    grid=both,
    ylabel={\small Top-1 Accuracy},
    xlabel={\small $C_f$},
    xlabel style={yshift=10pt, xshift=80pt,anchor=north east},
  ]
    
    \addplot [thick, color=C3, mark=square*,  mark size=2] coordinates {(5, 96.1) (10, 92.7) (15, 90.4) (20, 88.47) (24, 87.13)};     
    \addlegendentry{Chamfer++}  

    \addplot [thick, color=C1, mark=x,  mark size=3] coordinates {(5, 93.4) (10, 88.8) (15, 86.1) (20, 83.7) (24, 82.3)};      
    \addlegendentry{TRX-{2,3}} 
	
    \addplot [thick, color=C2, mark=triangle*,  mark size=2] coordinates {(5, 93.6) (10, 89.5) (15, 87.07) (20, 85) (24, 83.7)};      
    \addlegendentry{TSL} 



  \end{axis}
\end{tikzpicture}
      \definecolor{C1}{RGB}{226, 43, 41}
\definecolor{C2}{RGB}{47, 96, 206}
\definecolor{C3}{RGB}{246, 175, 11} 


\begin{tikzpicture}
  \begin{axis}[
    width=0.5\linewidth,
    height=5cm, 
    xtick = {5,10,15,20,24},
	legend cell align={left},
	legend pos=south west,
    grid=both,
    ylabel={\small Top-1 Accuracy},
    xlabel={\small $C_f$},
    xlabel style={yshift=10pt, xshift=80pt,anchor=north east},
  ]

    \addplot [thick, color=C3, mark=square*,  mark size=2] coordinates {(5, 98.3) (10, 96.8) (15, 95.5) (20, 94.5) (24, 93.73)};   
    \addlegendentry{Chamfer++}  
	
    \addplot [thick, color=C1, mark=x,  mark size=3] coordinates {(5, 97.5) (10, 95.7) (15, 94.2) (20, 93.1) (24, 92.2)};      
    \addlegendentry{TRX-{2,3}} 
	
    \addplot [thick, color=C2, mark=triangle*,  mark size=2] coordinates {(5, 98) (10, 96.6) (15, 95.5) (20, 94.6) (24, 93.9)};      
    \addlegendentry{TSL} 


  \end{axis}
\end{tikzpicture}
    }
    \vspace{-15pt}
    \caption{\textbf{\kinetics} dataset. 1-shot (left) and  5-shot (right)}
    \label{fig:ways_kinetics}
\end{subfigure}     

\vspace{20pt}
\begin{subfigure}{\linewidth}
    \resizebox{\linewidth}{!}{
      \definecolor{C1}{RGB}{226, 43, 41}
\definecolor{C2}{RGB}{47, 96, 206}
\definecolor{C3}{RGB}{246, 175, 11} 


\begin{tikzpicture}
  \begin{axis}[
    width=0.5\linewidth,
    height=5cm, 
    xtick = {5,10,15,20,24},
	legend cell align={left},
	legend pos=north east,
    grid=both,
    ylabel={\small Top-1 Accuracy},
    xlabel={\small $C_f$},
    xlabel style={yshift=10pt, xshift=80pt,anchor=north east},
  ]

\addplot [thick, color=C3, mark=square*,  mark size=2] coordinates {(5, 97.7)(10, 96.2) (15, 95.07) (20, 94.1) (24, 93.53)}; 
\addlegendentry{Chamfer++} 

\addplot [thick, color=C1, mark=x,  mark size=3] coordinates {(5, 96.6) (10, 95.1) (15, 93.93) (20, 93.1) (24, 92.57)};       
\addlegendentry{TRX-{2,3}} 
	
\addplot [thick, color=C2, mark=triangle*,  mark size=2] coordinates {(5, 97.1) (10, 95.7) (15, 94.5) (20, 93.67) (24, 93.03)};   
\addlegendentry{TSL} 

  \end{axis}
\end{tikzpicture}
      \definecolor{C1}{RGB}{226, 43, 41}
\definecolor{C2}{RGB}{47, 96, 206}
\definecolor{C3}{RGB}{246, 175, 11} 


\begin{tikzpicture}
  \begin{axis}[
    width=0.5\linewidth,
    height=5cm, 
    xtick = {5,10,15,20,24},
	legend cell align={left},
	legend pos=south west,
    grid=both,
    ylabel={\small Top-1 Accuracy},
    xlabel={\small $C_f$},
    xlabel style={yshift=10pt, xshift=80pt,anchor=north east},
  ]
\addplot [thick, color=C3, mark=square*,  mark size=2] coordinates {(5, 99.33)(10, 98.6)(15, 98.1)(20, 97.6) (24, 97.3)};
\addlegendentry{Chamfer++} 

\addplot [thick, color=C1, mark=x,  mark size=3] coordinates {(5, 99.5) (10, 99.13) (15, 98.8) (20, 98.53) (24, 98.37)};   
\addlegendentry{TRX-{2,3}} 

\addplot [thick, color=C2, mark=triangle*,  mark size=2] coordinates {(5, 99.4) (10, 99.1) (15, 98.8) (20, 98.5) (24, 98.3)};     
\addlegendentry{TSL} 

  \end{axis}
\end{tikzpicture}
    }
    \vspace{-15pt}
    \caption{\textbf{\ucf} dataset. 1-shot (left) and  5-shot (right)}
    \label{fig:ways_ucf}
\end{subfigure}     

\caption{\textbf{Impact of the number of classes  per episode ($C_{f}$)} on three datasets.
\label{fig:number_ways}
}
\vspace{-7ex}
\end{figure*}
\myparagraph{Varying the number of classes in an episode.}
Figure~\ref{fig:number_ways} shows the performance when extending the case of $C_{f}=5$ classes to the maximum number of classes, $C_{f}=24$, in the 1-shot and 5-shot regime. 
We see that for 1-shot the proposed non-temporal Chamfer++ method highly outperforms TRX and the classifier-based TSL method in all datasets.

\begin{wraptable}[11]{R}{0.5\linewidth}
\vspace{-20pt}
\caption{Chamfer++ variants.} 
\label{tab:chamfer_ablation}
\centering
    \resizebox*{!}{95pt}
    {
        \setlength\extrarowheight{-3pt}
    \begin{tabular}{@{\xssp} l@{\ssp} l@{\ssp}l@{\ssp} l@{\ssp}l@{\ssp}l@{\nsp}} 
    \toprule
         Method & $l$ & {\scriptsize Joint} & {\scriptsize Tupl.} & \oneshot & \fiveshot \\
         \midrule
         Chamfer-Q  &  1 &   & &  65.7 \hspace{-3pt}\stddev{0.1} & 79.7 \hspace{-3pt}\stddev{0.1} \\
         Chamfer-S  &  1 &   & &  65.3 \hspace{-3pt}\stddev{0.1} & 79.1 \hspace{-3pt}\stddev{0.2}  \\
         Chamfer  &  1 &     & & 66.9 \hspace{-3pt}\stddev{0.1} & 80.0 \hspace{-3pt}\stddev{0.2} \\
         Chamfer+  &  1 & \checkmark   &  & 66.9 \hspace{-3pt}\stddev{0.1}& 80.7 \hspace{-3pt}\stddev{0.2} \\

         Chamfer++  &  2 &  \checkmark  & all & 67.1 \hspace{-3pt}\stddev{0.1} &  80.8 \hspace{-3pt}\stddev{0.2} \\
         Chamfer++  &  2 &  \checkmark  &  ord. & 67.7 \hspace{-3pt}\stddev{0.1} & 81.4 \hspace{-3pt}\stddev{0.2} \\
         Chamfer++  &  3 &  \checkmark  & all & 67.0 \hspace{-3pt}\stddev{0.3} & 80.8 \hspace{-3pt}\stddev{0.1} \\
         Chamfer++  &  3 &  \checkmark  & ord. & 67.8 \hspace{-3pt}\stddev{0.2} & 81.6 \hspace{-3pt}\stddev{0.1} \\
    \bottomrule
    \end{tabular}
        
    }
\vspace{-2ex}

\vspace{-3.5ex}
\end{wraptable}



\myparagraph{Comparison to the state-of-the-art.}
In Table~\ref{tab:sota-papers}, we compare the performance of the proposed Chamfer++ with the corresponding numbers reported in many recent few-shot action recognition papers. Although there is no direct comparison between all these methods, \ie, no common setup or backbones, we present all results jointly to show the overall progress in the task.



\subsection{Chamfer matching ablation and interpretability}
\label{sec:ablation}

In Table~\ref{tab:chamfer_ablation}, we present an ablative study with the 1-shot performance of all the different variants of Chamfer++ we discuss in Section~\ref{sec:newchamfer} on the \ssvtwo dataset. Combining support-based and query-based Chamfer matching helps learn a better projection $\phi$, while both joint matching and the use of clip tuples improve performance compared to the vanilla variant. Although using ordered clip tuples improves for 1-shot on \ssvtwo, overall, we see in Table~\ref{tab:main} that using all or ordered tuples results in more or less similar performance. 

\wrapfill

\looseness=-1
\myparagraph{Which are the most informative clips?}
As we presented in Section~\ref{sec:preliminaries}, we sample and encode a set of clips to represent each video. Not all clips are, however, equally valuable for the matching process.
To provide some insight and illustrate the proposed method's interpretability, we study a qualitative example from \ssvtwo in Figure \ref{fig:example}.
When matching the query video to the two support videos belonging to the same class (top), we see many clip-to-clip correspondences with high magnitudes (shown as thicker lines). This is not the case when matching to the support videos belonging to a negative class (bottom) and we can only see a single strong correspondence. Upon better inspection, we saw that the motion of picking up the object in the fourth clip of the query video does exhibit a left-to-right motion locally that matches the negative class. Nevertheless, the other correspondences are weaker.
Leveraging more than one clip helps Chamfer matching to disambiguate while, at the same time, its inherent selectivity makes it more robust against noisy correspondences.
% 
This example illustrates that not all the clip correspondences are equally valuable to compute a video-to-video similarity metric. The matching step is a dynamic way to select the most informative clip correspondences that highlight the similarity between two videos.

\section{Visualization On Demand} %Visualization Elements
\label{sec:visrisk}
Based on environment data and trajectory evaluation, we now present ways of communicating the situation and risks on a visual display to achieve an ADAS.
In this context, we employ a renderer that visualizes all the information in a joint Cartesian coordinate system (see section \ref{subsec:sim}). 
Once driving risks are detected, design elements are overlayed on the display with section \ref{subsec:active} and section \ref{subsec:warning}. 

\subsection{Simulator Environment}
\label{subsec:sim}
Nodes of the R-LDM have a range of potential attributes, such as the 3D position or geometrical shape of objects. 
% For instance, the road centerline is a polyline with bounderies to the left and right. Crosswalks have a defined width and buildings a polygonal outline description. 
In the renderer, we always visualize static and quasi-static data that lie in the field of view from the ego vehicle. 
For this, a local 3D model is generated by converting geographic points with (lat, lon, alt) into Cartesian coordinates of (x, y, z). 
% and project the positonal relations from a view perspective with a transformation matrix. 
Fig. \ref{fig:3Dsimulator} depicts an exemplary map section having several intersections in bird's-eye view.
% with several intersections, stop lines and crosswalks. 
On the top right, the first person view of a vehicle approaching a crosswalk is shown. 

The dynamic data is then added to this static view. A zoomed-in excerpt from the map is given at the bottom of Fig. \ref{fig:3Dsimulator} that includes a recorded GNSS trace (red).
We project the trace onto the connected lane center, which is pictured in green. 
% Because we project the ego position on the closest lane segment, on the bottom right the measured trace is changed in red and the aligned trace is marked in green.
Consequently, the virtual horizon and its possible paths are retrieved as described in section \ref{subsec:ldm}. 
We can lastly update and move the excerpt with the current position from the GNSS to obtain a live simulation.

\subsection{Proactive Support}
\label{subsec:active}
Communication of spatial as well as spatio-temporal relations is crucial for risk-averse driver support. 
% This has the reason that humans can estimate the time better than positions (especially for risks). 
% Velocity contains implicitly the time as well. 
Further sources of information are cause, likelihood and severity of a potential risks.  
% if a collision happens. 
The next step for RNS is the choice of suitable design elements. 
In this process, we suppose that we know where the ego vehicle is driving (i.e., the ego path) from its navigation route. 
Yet, for surrounding vehicles, all paths are considered.

\subsubsection{Hazard Route Element}
The so-called hazard route in Fig. \ref{fig:charts} is a concept that consists of a scale portraying distances to an upcoming risk element.
Furthermore, the geometrical area or length of risks is considered.
Risk is thus measured with respect to the ego path, ranging from the current position  $\Delta l \hspace{-0.03cm}=\hspace{-0.03cm} \unit[0]{m}$ to the end of the path $\Delta l_{h}$.
Here, the length $\Delta l_{h}$ can be chosen according to own preferences. 

At an upcoming intersection, risk is defined by the section of the path that lies within the junction.
Since risk corresponds to exposition time, we encode the path part from the intersection $I_z$ with a color, ranging from green for short intersections to red for long ones. 
%allgemein risiko entlang des pfades zu intersection zone
%share of junction segment to navigation route + 
%one case with large intersection far and one case with small intersection close
Fig. \ref{fig:charts}~a) gives two examples of the hazard route.
The left bar shows a large intersection (e.g. multi-lane four-way stop) in vicinity and the right bar has a small and consecutive medium junction. 
% In the case of collision risk, the intersection zone $I_z$ can be used.
% Depending on the value of $I_z$ (low, medium and large), the area is marked from green, to yellow until red for conveying the criticality. 
This emphasizes that we may include more than one intersection in our warnings.

\begin{figure}[t]
  \centering
  \includegraphics[width=0.95\linewidth]{./img/simulator.png}
  \caption{Rendered road network from two perspectives with the ego position being projected on the navigation route. \vspace{0.45cm}}
  \label{fig:3Dsimulator}
\end{figure}

\begin{figure}[t]
  \centering
  \resizebox{\linewidth}{!}{
  \import{img/}{velocity_scale_new.pdf_tex}}  
  \caption{Chart elements for proactive support. Hazard route (left) and velocity scale (right).} %\vspace{-0.3cm}}
  \label{fig:charts} 
\end{figure} 

\subsubsection{Velocity Scale Element}
The velocity scale, Fig. \ref{fig:charts}~b), is a second chart element which qualifies the difference between the current velocity of the vehicle $v_0$ and the target velocity $v_{\text{tar}}$ from the trajectory evaluation of section \ref{subsec:trajeval}. 
The scale shows possible velocity values, from standstill $v\hspace{-0.05cm}=\hspace{-0.05cm}\unit[0]{m/s}$ to a maximal velocity $v_{\text{max}}$. Depending on the difference $|v_0 \hspace{0.05cm} - \hspace{0.05cm} v_{\text{tar}}|$, the situation is rated as safe with $v_0 \hspace{-0.042cm} \approx \hspace{-0.042cm} v_{\text{tar}}$ (green, left), as dangerous with e.g. $v_0 \hspace{-0.05cm} < \hspace{-0.05cm} v_{\text{tar}}$ (yellow, middle) to critical with $v_0 \hspace{-0.07cm} \ll \hspace{-0.07cm} v_{\text{tar}}$ (red, right). The same cases hold true for the opposite circumstances, i.e., $v_0 \hspace{-0.032cm} > \hspace{-0.032cm} v_{\text{tar}}$. 
This velocity scale can be employed for curve or regulatory risks. 
Moreover, we may set an enforced speed limit as the target velocity $v_{\text{tar}}$ for proactive behavior, once there is no risk ahead. 
%\noindent -Warning vs behavior support \\
%-Ghost vehicle as in game \\

\subsection{Short-Term Warning Elements}
\label{subsec:warning}
In order to emphasize the criticality of the situation, we propose to add further intuitive warning elements as e.g. pop-up signs and lane colorings. 
The following elements augment the proactive elements.

\subsubsection{Pop-up Signs}
Explicit symbols indicate the risk cause accompanied with the event time for collisions ($s_E$), distances to the risk spot for turns (i.e., right curve with $d_r$ and left curve with $d_l$) or stopping distance for crosswalks ($d_c$). In Fig. \ref{fig:popups}~a), the pop-up signs are pictured. 
% Besides the velocity difference, the risk type is an indication for the severity of the situation.
%Examples for collision risk are car-to-car crash., curve risk can be  as a single-car accident and regulatory risks will be a car-to-object collision. 
We want to stress that this is just a selection and more risk causes can be added. 
The purpose is also to clarify the reason for the warning and give more human-understandable information.

\subsubsection{Colored Events}
Finally, we highlight lane parts or positions according to the corresponding risks.  
% the determined color rating from the hazard route and velocity scale and relate the risks to the simulator environment. 
In the instance of curve and regulatory risk, the lane is colored from the ego position up to the point of maximal risk. 
For collision risk, we mark the point of the closest encounter as a red cube.
An illustration for regulatory risk induced from a stop line is depicted in Fig. \ref{fig:popups}~b). Again, the color is defined by the deviation $|v_0-v_{\text{tar}}|$. It also shows the therein considered navigation route with length $\Delta l_h$ and another unlikely path. 

It should be noted that the visualization of warnings only occurs if the risks are actually present. 
%\textcolor{red}{improve language, repeat intersection zone and navigation route}
%eingrauen unlikely paths and navigation path and describe in text, maybe delete Iz -> put line from unlikely path to green arrow
Altogether, the RNS provides a variety of tools to analyze and circumvent critical situations in intersection scenarios, while not overloading the driver's awareness.

\begin{figure}[t]
  \centering
  \resizebox{\linewidth}{!}{
  \import{img/}{colored_lane_new.pdf_tex}}  
  \vspace{-0.53cm}
  \caption{Short-term warning elements. Selected pop-up warnings (left) and colored lane (right).}
  \label{fig:popups} 
\end{figure} 



