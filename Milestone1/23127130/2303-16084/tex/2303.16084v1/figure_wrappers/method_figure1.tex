\begin{figure}
  \vspace{-20pt}  
\centering
  \begin{subfigure}{0.75\textwidth}
    \includegraphics[width=\textwidth]{figures/method/spatio_temporal_fig_part1.pdf}
     \vspace{-15pt}
    \caption{}
    \label{fig:first}
\end{subfigure}     
\vspace{2pt}
\begin{subfigure}{0.75\textwidth}
\vspace{-0pt}
    \includegraphics[width=\textwidth]{figures/method/spatio_temporal_fig_part2.pdf}
    \vspace{-20pt}
    \caption{}
    \label{fig:second}
\end{subfigure}
\begin{subfigure}{0.75\textwidth}
\vspace{-8pt}
    \includegraphics[width=\textwidth]{figures/method/spatio_temporal_fig_part3.pdf}
    \vspace{-22pt}
    \caption{}
    \label{fig:third}
\end{subfigure}
  \vspace{-5pt}
\caption{\textbf{Different ways of using temporal information in few-shot action recognition.} Temporal information is utilized: a) during matching (existing \textit{matching-based} methods) b) in the backbone with a classifier (existing \textit{classifier-based} methods, \emph{e.g.}~\cite{tsl}) c) in the backbone with a matching step (this paper). 
  \label{fig:spatio_temporal}
\vspace{-15pt}
}
\end{figure}
