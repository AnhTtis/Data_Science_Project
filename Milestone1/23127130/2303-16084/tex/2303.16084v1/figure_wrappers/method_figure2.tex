\begin{figure*}[t]
\vspace{-5pt}

\begin{subfigure}[t]{0.27\textwidth}
    \centering
    \includegraphics[width=.9\linewidth]{figures/method/method_chamfer.pdf}
    \caption{Chamfer matching}
    \label{fig:chamfer_matching}
\end{subfigure}
\begin{subfigure}[t]{0.24\textwidth}
    \centering
    \includegraphics[width=.85\linewidth]{figures/method/method_joint_matching.pdf}
    \caption{Joint matching}
    \label{fig:joint_matching}
\end{subfigure}
\hspace{10pt}
\begin{subfigure}[t]{0.45\textwidth}
    \centering
    \includegraphics[width=\linewidth]{figures/method/method_tuples.pdf}
    \caption{Clip tuples}
    \label{fig:clip_tuples}
\end{subfigure}  

\vspace{3ex}
    \caption{\textbf{Details of the proposed matching approach}. a) The \textbf{Chamfer} matching function $f_{QS}$ from Eq.(\ref{eq:chamferqs}). b) Jointly matching multiple examples per class (Chamfer+). c) using  clip tuples as representation; one can use only ordered tuples or all tuples; in both cases, the matching part remains \textit{non-temporal}, \ie invariant to the temporal order of features.
    \vspace{-3ex}
    }
    \label{fig:method}
\end{figure*}
