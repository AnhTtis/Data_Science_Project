
\begin{figure}[!h]
\centering
  \includegraphics[width=0.48\textwidth]{Supplementary/PP_MassPlot.pdf}
  \caption{Dimuon invariant mass distribution in \pp collisions, integrated over the full kinematic range $\pt < 30$\GeVc and $\abs{y}<2.4$. The solid curves show the result of the fit, whereas the orange dashed and blue dash-dotted curves represent the three \PgU states and the background, respectively.}
  \label{fig:sigFitpp}
\end{figure}

\begin{figure}[!h]
\centering
\includegraphics[width=0.48\textwidth]{Supplementary/RAA_centrality_int_theory1_v2.pdf}
  \caption{Nuclear modification factors for the \PgUa, \PgUb, and \PgUc mesons as a function of $\langle \Npart \rangle$(from Figure 2 left), including the centrality integrated bin. The vertical lines and boxes correspond to statistical and systematic uncertainties, respectively. The left-most box at unity combines the uncertainties of pp luminosity and \PbPb \NMB, while the second (third) box corresponds to the uncertainty of pp yields for the \PgUb (\PgUc) state. Results for the \PgUa meson are taken from Ref.~\cite{HIN16023}. The bands represent calculations from Ref.~\cite{Xiaojun:2020}.
  }
  \label{fig:theory_cent1}
\end{figure}

\begin{figure}[ht!]
\centering
\includegraphics[width=0.48\textwidth]{Supplementary/RAA_centrality_int_theory2_v2.pdf}
\caption{Nuclear modification factors for the \PgUa, \PgUb, and \PgUc mesons as a function of $\langle \Npart \rangle$(from Figure 2 left), including the centrality integrated bin. The vertical lines and boxes correspond to statistical and systematic uncertainties, respectively. The left-most box at unity combines the uncertainties of pp luminosity and \PbPb \NMB, while the second (third) box corresponds to the uncertainty of pp yields for the \PgUb (\PgUc) state. Results for the \PgUa meson are taken from Ref.~\cite{HIN16023}. The OQS + pNRQCD theory calculations are taken from Ref.~\cite{Brambilla:2023Reg}.
    }
  \label{fig:theory_cent2}
\end{figure}

\begin{figure}[ht!]
\centering
\includegraphics[width=0.48\textwidth]{Supplementary/RAA_centrality_int_theory3_v2.pdf}
\caption{Nuclear modification factors for the \PgUa, \PgUb, and \PgUc mesons as a function of $\langle \Npart \rangle$(from Figure 2 left), including the centrality integrated bin. The vertical lines and boxes correspond to statistical and systematic uncertainties, respectively. The left-most box at unity combines the uncertainties of pp luminosity and \PbPb \NMB, while the second (third) box corresponds to the uncertainty of pp yields for the \PgUb (\PgUc) state. Results for the \PgUa meson are taken from Ref.~\cite{HIN16023}. The bands represent calculations from Ref.~\cite{FerreiroComp}.
    }
  \label{fig:theory_cent3}
\end{figure}

\begin{figure}[ht!]
\centering
\includegraphics[width=0.48\textwidth]{Supplementary/RAA_centrality_int_theory4_v2.pdf}
\caption{Nuclear modification factors for the \PgUa, \PgUb, and \PgUc mesons as a function of $\langle \Npart \rangle$(from Figure 2 left), including the centrality integrated bin. The vertical lines and boxes correspond to statistical and systematic uncertainties, respectively. The left-most box at unity combines the uncertainties of pp luminosity and \PbPb \NMB, while the second (third) box corresponds to the uncertainty of pp yields for the \PgUb (\PgUc) state. Results for the \PgUa meson are taken from Ref.~\cite{HIN16023}. The bands represent calculations from Ref.~\cite{Du:2017qkv}.
    }
  \label{fig:theory_cent4}
\end{figure}

\begin{figure}[ht!]
\centering
\includegraphics[width=0.48\textwidth]{Supplementary/RAA_centrality_int_theory5_v2.pdf}
\caption{Nuclear modification factors for the \PgUa, \PgUb, and \PgUc mesons as a function of $\langle \Npart \rangle$(from Figure 2 left), including the centrality integrated bin. The vertical lines and boxes correspond to statistical and systematic uncertainties, respectively. The left-most box at unity combines the uncertainties of pp luminosity and \PbPb \NMB, while the second (third) box corresponds to the uncertainty of pp yields for the \PgUb (\PgUc) state. Results for the \PgUa meson are taken from Ref.~\cite{HIN16023}. The lines represent calculations from Ref.~\cite{Wolschin:2020ijmpa}.}
\label{fig:theory_cent5}
\end{figure}


\begin{figure}[ht!]
\centering
\includegraphics[width=0.48\textwidth]{Supplementary/RAA_centrality_int_w1s_theory2_2reg.pdf}
\caption{Nuclear modification factors for the \PgUa, \PgUb, and \PgUc mesons as a function of $\langle \Npart \rangle$(from Figure 2 left), including the centrality integrated bin. The vertical lines and boxes correspond to statistical and systematic uncertainties, respectively. The left-most box at unity combines the uncertainties of pp luminosity and \PbPb \NMB, while the second (third) box corresponds to the uncertainty of pp yields for the \PgUb (\PgUc) state. Results for the \PgUa meson are taken from Ref.~\cite{HIN16023}. The two type of bands represent calculations from Ref.~\cite{Du:2017qkv}, with the solid filled bands calculated without the recombination component.}
\label{fig:theory_cent6}
\end{figure}

\begin{figure}[ht!]
\centering
\includegraphics[width=0.48\textwidth]{Supplementary/RAA_centrality_int_theory2_new_v2.pdf}
\caption{Nuclear modification factors for the \PgUa, \PgUb, and \PgUc mesons as a function of $\langle \Npart \rangle$(from Figure 2 left), including the centrality integrated bin. The vertical lines and boxes correspond to statistical and systematic uncertainties, respectively. The left-most box at unity combines the uncertainties of pp luminosity and \PbPb \NMB, while the second (third) box corresponds to the uncertainty of pp yields for the \PgUb (\PgUc) state. Results for the \PgUa meson are taken from Ref.~\cite{HIN16023}. The OQS + pNRQCD theory calculations are taken from Ref.~\cite{Brambilla:2023Reg}, with the dashed one calculated without the recombination component.}
\label{fig:theory_cent7}
\end{figure}

\begin{figure}[ht!]
\centering
\includegraphics[width=0.48\textwidth]{Supplementary/RAA_centrality_theorySHMb.pdf}
\caption{Nuclear modification factors for the \PgUa, \PgUb, and \PgUc mesons as a function of $\langle \Npart \rangle$(from Figure 2 left), including the centrality integrated bin. The vertical lines and boxes correspond to statistical and systematic uncertainties, respectively. The left-most box at unity combines the uncertainties of pp luminosity and \PbPb \NMB, while the second (third) box corresponds to the uncertainty of pp yields for the \PgUb (\PgUc) state. Results for the \PgUa meson are taken from Ref.~\cite{HIN16023}. The lines represent calculations from Ref.~\cite{Andronic2018}, with the solid line corresponding to 50\% thermalization of the $b\bar{b}$ pairs and the upper and lower dashed lines representing the 20\% uncertainty of the total $b\bar{b}$ cross section.}
\label{fig:theory_cent8}
\end{figure}

\begin{figure}[ht!]
\centering
\includegraphics[width=0.48\textwidth]{Supplementary/RAA_pt_theory1_v2.pdf}
\caption{Nuclear modification factors for the \PgUa, \PgUb, and \PgUc mesons as a function of \pt(from Figure 2 right). The vertical lines and boxes correspond to statistical and systematic uncertainties, respectively. The box at unity represents the global uncertainty, which combines uncertainties from \TAA, pp luminosity, and \PbPb \NMB. Results for the \PgUa meson are taken from Ref.~\cite{HIN16023}. The bands represent calculations from Ref.~\cite{Xiaojun:2020}.} 
\label{fig:theory_pt1}
\end{figure}

\begin{figure}[ht!]
\centering
\includegraphics[width=0.48\textwidth]{Supplementary/RAA_pt_theory2_v2.pdf}
\caption{Nuclear modification factors for the \PgUa, \PgUb, and \PgUc mesons as a function of \pt(from Figure 2 right). The vertical lines and boxes correspond to statistical and systematic uncertainties, respectively. The box at unity represents the global uncertainty, which combines uncertainties from \TAA, pp luminosity, and \PbPb \NMB. Results for the \PgUa meson are taken from Ref.~\cite{HIN16023}. The OQS + pNRQCD theory calculations are taken from Ref.~\cite{Brambilla:2023Reg}.} 
\label{fig:theory_pt2}
\end{figure}

\begin{figure}[ht!]
\centering
\includegraphics[width=0.48\textwidth]{Supplementary/RAA_pt_theory3_v2.pdf}
\caption{Nuclear modification factors for the \PgUa, \PgUb, and \PgUc mesons as a function of \pt(from Figure 2 right). The vertical lines and boxes correspond to statistical and systematic uncertainties, respectively. The box at unity represents the global uncertainty, which combines uncertainties from \TAA, pp luminosity, and \PbPb \NMB. Results for the \PgUa meson are taken from Ref.~\cite{HIN16023}. The bands represent calculations from Ref.~\cite{Du:2017qkv}.} 
\label{fig:theory_pt3}
\end{figure}

\begin{figure}[ht!]
\centering
\includegraphics[width=0.48\textwidth]{Supplementary/RAA_pt_theory4_v2.pdf}
\caption{Nuclear modification factors for the \PgUa, \PgUb, and \PgUc mesons as a function of \pt(from 2 right). The vertical lines and boxes correspond to statistical and systematic uncertainties, respectively. The box at unity represents the global uncertainty, which combines uncertainties from \TAA, pp luminosity, and \PbPb \NMB. Results for the \PgUa meson are taken from Ref.~\cite{HIN16023}. The lines represent calculations from Ref.~\cite{Wolschin:2020ijmpa}.} 
\label{fig:theory_pt4}
\end{figure}

\begin{figure}[ht!]
\centering
\includegraphics[width=0.48\textwidth]{Supplementary/RAA_pt_theory2_new_logy_v2.pdf}
\caption{Nuclear modification factors for the \PgUa, \PgUb, and \PgUc mesons as a function of \pt(from Figure 2 right). The vertical lines and boxes correspond to statistical and systematic uncertainties, respectively. The box at unity represents the global uncertainty, which combines uncertainties from \TAA, pp luminosity, and \PbPb \NMB. Results for the \PgUa meson are taken from Ref.~\cite{HIN16023}. The OQS + pNRQCD theory calculations are taken from Ref.~\cite{Brambilla:2023Reg}, with the dashed one calculated without the recombination component.}
\label{fig:theory_pt5}
\end{figure}


\begin{figure}[ht!]
\centering
\includegraphics[width=0.48\textwidth]{Supplementary/RAA_pt_w1s_theory2_2reg_logy.pdf}
\caption{Nuclear modification factors for the \PgUa, \PgUb, and \PgUc mesons as a function of \pt(from Figure 2 right). The vertical lines and boxes correspond to statistical and systematic uncertainties, respectively. The box at unity represents the global uncertainty, which combines uncertainties from \TAA, pp luminosity, and \PbPb \NMB. Results for the \PgUa meson are taken from Ref.~\cite{HIN16023}. The two type of bands represent calculations from Ref.~\cite{Du:2017qkv}, with the solid filled bands calculated without the recombination component.} 
\label{fig:theory_pt6}
\end{figure}


\begin{figure}[ht!]
\centering
\includegraphics[width=0.48\textwidth]{Supplementary/DR_centrality_int_theory_v2.pdf}
\caption{The double ratio of \PgUc/\PgUb as functions of $\langle \Npart \rangle$(from Figure 3 left). The vertical lines correspond to statistical uncertainties, while the boxes are the systematic uncertainties. The box at unity shows the combined systematic and statistical uncertainties from pp data. The six different types of lines and bands represent calculations from Ref.~\cite{Xiaojun:2020,Brambilla:2023Reg,FerreiroComp,Du:2017qkv,Wolschin:2020ijmpa,Andronic2018}.}
\label{fig:theory_dr_cent1}
\end{figure}



\begin{figure}[ht!]
\centering
\includegraphics[width=0.48\textwidth]{Supplementary/DR_centrality_int_theory2_wojump_v2.pdf}
          \caption{The double ratio of \PgUc/\PgUb as a function of $\langle \Npart \rangle$(from Figure 3 left). The vertical lines correspond to statistical uncertainties, while the boxes are the systematic uncertainties. The box at unity shows the combined systematic and statistical uncertainties from pp data. The orange and blue boxes represent calculations from Ref.~\cite{Brambilla:2023Reg}, with the latter showing the calculations without the recombination component.}
\label{fig:theory_dr_cent2}
\end{figure}

\begin{figure}[ht!]
\centering
\includegraphics[width=0.48\textwidth]{Supplementary/DR_centrality_int_theory3_noreg_v2.pdf}
          \caption{The double ratio of \PgUc/\PgUb as a function of $\langle \Npart \rangle$(from Figure 3 left). The vertical lines correspond to statistical uncertainties, while the boxes are the systematic uncertainties. The box at unity shows the combined systematic and statistical uncertainties from pp data. The red and blue lines represent calculations from Ref.~\cite{Du:2017qkv}, with the latter showing the calculations without the recombination component.}
\label{fig:theory_dr_cent3}
\end{figure}


\begin{figure}[ht!]
\centering
\includegraphics[width=0.48\textwidth]{Supplementary/DR_pt_theory_v2.pdf}
          \caption{The double ratio of \PgUc/\PgUb as a function of \pt(from Figure 3 right). The vertical lines correspond to statistical uncertainties, while the boxes are the systematic uncertainties. The bands and line represent calculations from Ref.~\cite{Xiaojun:2020,Brambilla:2023Reg,Du:2017qkv,Wolschin:2020ijmpa}.}
\label{fig:theory_dr_pt1}
\end{figure}


\begin{figure}[ht!]
\centering
\includegraphics[width=0.48\textwidth]{Supplementary/DR_pt_theory2_wojump_v2.pdf}
          \caption{The double ratio of \PgUc/\PgUb as a function of \pt(from Figure 3 right). The vertical lines correspond to statistical uncertainties, while the boxes are the systematic uncertainties. The orange and blue bands represent calculations from Ref.~\cite{Brambilla:2023Reg}, with the latter showing the calculations without the recombination component.}
\label{fig:theory_dr_pt2}
\end{figure}


\begin{figure}[ht!]
\centering
\includegraphics[width=0.48\textwidth]{Supplementary/DR_pt_theory3_noreg_v2.pdf}
          \caption{The double ratio of \PgUc/\PgUb as a function of \pt(from Figure 3 right). The vertical lines correspond to statistical uncertainties, while the boxes are the systematic uncertainties. The red band and blue line represent calculations from Ref.~\cite{Du:2017qkv}, with the latter showing the calculations without the recombination component.}
\label{fig:theory_dr_pt3}
\end{figure}

\begin{figure}[ht!]
\centering
\includegraphics[width=0.53\textwidth]{Supplementary/raa_vs_bindingE.pdf}
          \caption{The nuclear modification factors for various quarkonium mesons as a function of quarkonium binding energy at $\sqrt{\mathrm{s_{NN}}}=5.02~\mathrm{TeV}$. The \RAA values are taken from the data point with the most central collision bin, e.g., the values for the \PgU's correspond to the points of the hightest \Npart in Figure 2 left. The values for the binding energy of each quarkonium state are taken from Ref.~\cite{BindingESatz}. The error bars and boxes represent the statistical and systematic uncertainties, respectively. The results for the \PgUa meson and charmonium states (\PJGy and \Pgy) are taken from Refs.~\cite{HIN16023} and~\cite{Sirunyan:2018mp}, respectively.}
\label{fig:raa_bindingE}
\end{figure}

\begin{figure}[ht!]
\centering
\includegraphics[width=0.48\textwidth]{Supplementary/RpA_RAA_int.pdf}
          \caption{The nuclear modification factors for \PgUa, \PgUb, and \PgUc mesons in pPb and \PbPb collisions at $\sqrt{\mathrm{s_{NN}}}=5.02~\mathrm{TeV}$. The error bars and boxes represent the statistical and systematic uncertainties, respectively. The \PgUb and \PgUc values are from the integrated bin in Figure 2 left. The results for pPb collisions and the \PgUa meson are taken from Refs.~\cite{HIN18005} and~\cite{HIN16023}, respectively.} 
\label{fig:raa_rpa}
\end{figure}

\begin{figure*}[ht!]
\centering
\includegraphics[width=0.7\textwidth]{Supplementary/Quarkonia_vs_pT_RAA.pdf}
  \caption{Nuclear modification factors for the \PgUa, \PgUb, \PgUc, \PJGy, and \Pgy mesons as a function of \pt. The vertical lines and boxes correspond to statistical and systematic uncertainties, respectively. Results for the \PgUa meson are taken from Ref.~\cite{HIN16023}. The open and full cross points are the results for \PJGy mesons from Refs.~\cite{ALICE:JpsiRAA2019} and~\cite{Sirunyan:2018mp}, respectively. Results for \Pgy mesons are taken from Refs.~\cite{ALICE:2022psi2sRAA} and~\cite{Sirunyan:2018mp} for the open and full star points, respectively.}
\label{fig:raa_pt_exp}
\end{figure*}

\begin{figure}[h!]
\centering
      \includegraphics[width=0.48\textwidth]{Supplementary/SignificanceBDT.pdf}
      \caption{The significance $S/\sqrt{S+B}$ for \PgU mesons in \PbPb collisions as a function of BDT score normalized to range between -1 and 1. The quantities $S$ and $B$ represent the yields of signal and background dimuons used for the BDT training, respectively. The working point (WP) is determined to be the BDT score that maximizes the significance using a parametic fit.}
\label{fig:bdt_signif}
\end{figure}

\clearpage

\begin{table}[!hp]
  \centering
    \begin{tabular}{cccc}
      \pt range (\GeVc) & $\frac{1}{\langle \TAA \rangle}\frac{\rd^2 N^{\PgU}}{\rd\pt\rd y}$ & Stat. Unc.               & Syst. Unc.           \\
      \hline
      0--3                                             & 2.98            & 0.60 & 0.22  \\
      3--6                                             & 5.51            & 0.72 & 0.48  \\
      6--9                                             & 2.72            & 0.54 & 0.19  \\
      9--15                                            & 1.17            & 0.15 & 0.13  \\
      15--30                                           & 0.15            & 0.02 & 0.01 
    \end{tabular}
    \caption{Yields for \PgUb mesons in \PbPb collisions in centrality 0--90\% and $\abs{y} < 2.4$, corrected for acceptance and efficiency, and normalized by the nuclear thickness function $\langle \TAA \rangle$ and the number of minimum bias events $\NMB$. The values for the yields and their uncertainties are in units of pb$/\GeVc$.}
    \label{table:yield_pt_2S}
\end{table}
      
\begin{table}[!hp]
  \centering
    \begin{tabular}{ccccc}
      Centrality & $\langle \Npart \rangle$ & $\frac{1}{\langle \TAA \rangle}\frac{\rd^2 N^{\PgU}}{\rd\pt\rd y}$ & Stat. Unc. & Syst. Unc.           \\
      \hline
      0--5\% &  382.30    &  1.02 &  0.25 & 0.08  \\
      5--10\% &  331.50   &  0.95 &  0.25 & 0.11 \\
      10--20\% &  262.30  &  1.31 &  0.20 & 0.14  \\
      20--30\% &  188.20  &  1.25 &  0.24 & 0.17  \\
      30--40\% &  131.00  &  1.71 &  0.29 & 0.13  \\
      40--50\% &  87.19  &  2.32 &  0.37 &  0.15  \\
      50--60\% &  54.42  &  3.27 &  0.51 &  0.32  \\
      60--70\% &  31.21  &  5.74 &  0.82 &  0.69  \\
      70--90\% &  11.43  &  7.80 &  1.26 &  1.10 
    \end{tabular}
    \caption{Yields for \PgUb mesons in \PbPb collisions in $\pt < 30\GeVc$ and $\abs{y} < 2.4$, corrected for acceptance and efficiency, and normalized by the nuclear thickness function $\langle \TAA \rangle$ and the number of minimum bias events $\NMB$. The values for the yields and their uncertainties are in units of pb$/\GeVc$.}
    \label{table:yield_cent_2S}
  \end{table}


\begin{table}[!hp]
\centering
    \label{table:yield_pt_3S}
    \begin{tabular}{cccc}
      \pt range (\GeVc) & $\frac{1}{\langle \TAA \rangle}\frac{\rd^2 N^{\PgU}}{\rd\pt\rd y}$ & Stat. Unc.               & Syst. Unc.               \\
      \hline
      0--4 &   1.11 & 0.44 & 0.24 \\
      4--9 &   1.04 & 0.39 & 0.24 \\
      9--15 &   0.38 & 0.14 & 0.11 \\
      15--30 &   0.07 & 0.02 & 0.01 
    \end{tabular}
    \caption{Yields for \PgUc mesons in \PbPb collisions in centrality 0--90\% and $\abs{y} < 2.4$, corrected for acceptance and efficiency, and normalized by the nuclear thickness function $\langle \TAA \rangle$ and the number of minimum bias events $\NMB$. The values for the yields and their uncertainties are in units of pb$/\GeVc$.}
\end{table}

\begin{table}[!hp]
\centering
    \label{table:yield_cent_3S}
    \begin{tabular}{ccccc}
      Centrality & $\langle \Npart \rangle$ & $\frac{1}{\langle \TAA \rangle}\frac{\rd^2 N^{\PgU}}{\rd\pt\rd y}$ & Stat. Unc. & Syst. Unc.               \\
      \hline
      0--30\% &  269.10    &  0.33 & 0.10 & 0.07  \\
      30--50\% &  109.10   &  0.70 & 0.19 & 0.11  \\
      50--70\% &  42.81  &  1.85 & 0.36 & 0.21  \\
      70--90\% &  11.43  &  3.42 & 1.05 & 0.68 
    \end{tabular}
    \caption{Yields for \PgUc mesons in \PbPb collisions in $\pt < 30\GeVc$ and $\abs{y} < 2.4$, corrected for acceptance and efficiency, and normalized by the nuclear thickness function $\langle \TAA \rangle$ and the number of minimum bias events $\NMB$. The values for the yields and their uncertainties are in units of pb$/\GeVc$.}
\end{table}

\begin{table}[!hp]
  \centering
    \begin{tabular}{cccc}
      Centrality & $\frac{1}{\langle \TAA \rangle}\frac{\rd^2 N^{\PgU}}{\rd\pt\rd y}$ & Stat. Unc. & Syst. Unc.               \\
      \hline
      0--5\%      & 15.20 & 0.89 & $\begin{matrix} +0.88 \\ -0.95 \end{matrix}$  \\
      5--10\%     & 15.28 & 1.01 & $\begin{matrix} +0.89 \\ -0.96 \end{matrix}$  \\
      10--20\%    & 15.40 & 0.82 & $\begin{matrix} +0.94 \\ -1.00 \end{matrix}$  \\
      20--30\%    & 19.12 & 1.86 & $\begin{matrix} +1.22 \\ -1.27 \end{matrix}$  \\
      30--40\%    & 23.06 & 1.29 & $\begin{matrix} +1.68 \\ -1.71 \end{matrix}$  \\
      40--50\%    &  24.95 & 1.70 & $\begin{matrix} +2.18 \\ -2.14 \end{matrix}$  \\
      50--60\%    &  28.96 & 2.51 & $\begin{matrix} +3.10 \\ -3.00 \end{matrix}$  \\
      60--70\%    &  43.89 & 4.17 & $\begin{matrix} +6.04 \\ -5.56 \end{matrix}$  \\
      70--100\%   &  37.69 & 6.21 & $\begin{matrix} +6.68 \\ -4.61 \end{matrix}$
    \end{tabular}
    \caption{Yields for \PgUa mesons in \PbPb collisions in $\pt < 30\GeVc$ and $\abs{y} < 2.4$, corrected for acceptance and efficiency, and normalized by the nuclear thickness function $\langle \TAA \rangle$ and the number of minimum bias events $\NMB$ from Ref.~\cite{HIN16023}. The values for the yields and their uncertainties are in units of pb$/\GeVc$.}
    \label{table:yield_cent_1S}
  \end{table}
