


\section{\toolName}
\label{sec:venus}


Drawing from the iterative co-design process, we designed \toolName, a novel visualization approach that facilitates the observation of the single-qubit and two-qubit quantum states.
% Users can easily access the \toolName\ interface at 
To enhance accessibility, we implemented a web-based graphical interface to fit into routine tasks.
Users can access \toolName\ interface via
\textcolor{blue}{\url{https://venus-interface.github.io/}}.
% Design justification and a detailed introduction of the online interface are provided in Appendix A and B.
In this section, we first introduce the visual design of our visualization approach for single-qubit states and extend it to the form of two-qubit states.
% then present all design alternatives before we derive the final design at last,
% and detail the open-sourced interface of \toolName.



\subsection{Single-qubit State Representation}
\label{sec:single}

Informed by the design requirements introduced in Section \ref{sec:design_requirements}, we propose a novel visual design for single-qubit state representation. As shown in Figure \ref{fig:2}\component{A}, we utilize three right triangles and two circumscribed semicircles to represent the quantum states. 


\textbf{State vector.}
According to the quantum computing theory, the state vector of a quantum state is the fundamental discipline to represent a quantum state. Thus, it is of great importance to encode the state vector in the visual design (\textbf{R3}).
Specifically, we utilize each amplitude's real and imaginary parts (\textit{e.g.}, $a$ and $b$ of amplitude $\alpha$) to indicate the state vector according to the quantum theory illustrated in Section \ref{sec:3.3}.
Based on Equation \ref{equation:1}, we utilize two line segments to depict the amplitudes of the state vector as shown in Figure \ref{fig:2}\Subcomponent{1}.
Also, we use \modify{cyan} and red right triangles to indicate the two amplitudes $\alpha$ and $\beta$ of the state vector, respectively.
We encode the absolute values of real parts (\textit{i.e.}, $a$ and $c$) and imaginary parts (\textit{i.e.}, $b$ and $d$) of each amplitude by the line segments in black and grey, respectively.
We apply the double lines, as shown in Figure \ref{fig:2}\subcomponent{A1}, to reflect the negative real and imaginary values.
Note that if the number of the imaginary part is zero, the black line segment for the real part will coincide with the diameter of the semicircle.







\textbf{Probability of the quantum states.}
According to the refined requirement, the probability of each state (\textit{e.g.}, $\ket{0}$) is significant in revealing the superposition in quantum computing (\textbf{R2}).
Based on Equation \ref{equation:2_1}, we already know that the probability of a quantum state can be calculated by the absolute value of $a$ and $b$.
Meanwhile, for \toolName, according to the geometry of the visual design, the area of the semicircle circumscribed by the right triangle indicating the state vector can be calculated as follows:

\begin{equation}
\label{equation:2}
S_{semicircle} = \frac{\pi}{8} \cdot (|a|^2 + |b|^2),
\end{equation}

where $a$ and $b$ are the real and imaginary parts of $\alpha$.
Thus, building upon the above two equations, we can represent the probability of a specific quantum state by the area of the semicircle circumscribed by the right triangle since the area of the semicircle is proportional to the probability of the quantum state, as shown in Figure \ref{fig:2}\Subcomponent{2}.
% Thus, we made it possible to encode the probability of each quantum state by the intuitive area channel of the semicircles.
% For the color scheme,  we plot the semicircles for the two quantum states in two colors blue and red for $\ket{0}$ and $\ket{1}$ respectively, .
% In the scenario of the single-qubit quantum state, there are two possible quantum states (\textit{i.e.}, $\ket{0}$ and $\ket{1}$).
Thus, it is apparent to visually analyze the two probabilities of $\ket{0}$ and $\ket{1}$ by the semicircle area based on the numerical state vectors without any manual calculation for the probability.
Also, users are allowed to visually analyze how a certain state vector (\textit{i.e.}, line segments) affects the corresponding probability.






\textbf{Correlations between all elements.}
Through the co-design process, all participants pointed out that it is significant to visually link all various elements based on the normalization constraint, because this can
% correlate all above elements (i.e., state vector and corresponding probability), 
highlight how state vector and corresponding probability affect each other (\textbf{R4}).
As shown in Figure \ref{fig:2}\Subcomponent{3}, we arrange all elements in quantum states into a systematical form of shapes.
% Specifically, given that the sum of the probabilities of all single-qubit quantum states satisfies
% \begin{equation}
% \label{equation:3}
% Pr_{\ket{0}} + Pr_{\ket{1}} = 1.
% \end{equation}
We arrange the visual channels of all elements in the quantum state into a set of right triangles with the base side length of 1 according to Equation \ref{equation:4}.
Building upon this rule, the bottom right triangle in white is used to link the other two right triangles in \modify{cyan} and red for two states (\textit{i.e.}, $\ket{0}$ and $\ket{1}$).
% we visually correlate all elements and then form the primary visual design of \toolName.






\subsection{Two-qubit State Representation}

From the co-design process, all domain experts strongly agreed that it would be much more beneficial if the visualization could support the two-qubit state representation (\textbf{R1}), which is one of the major limitations for the widely-used approach, \textit{i.e.}, Bloch Sphere.
Recall that the two-qubit state representation is the basis to make the two-qubit entanglement representation available.
We extend from the approach for single-qubit state illustrated in Section \ref{sec:single} to unveil the mask of the two-qubit quantum states.


First, according to \textbf{R3}, we intend to present the two-qubit state based on the state vector. 
As shown in Figure \ref{fig:2}\component{B},
we utilize four pairs of line segments to visualize $\alpha, \beta, \gamma$ and $\delta$ building upon Equation \ref{equation:5}.
Each pair of line segments consists of a line segment in black to represent the real part and a line segment in grey to indicate the imaginary part of the complex number amplitudes (Figure \ref{fig:2}\Subcomponent{2}).

Second, 
the probability distribution of quantum states is also supported for \toolName's two-qubit mode (\textbf{R2}).
In this case, there are four possible quantum states, \textit{i.e.}, $\ket{00}, \ket{01},\ket{10},$ and $\ket{11}$.
By applying Equation \ref{equation:2_1}, the probabilities of the above four quantum states are proportional to the area of the corresponding circumscribed semicircle.
So we encode the four two-qubit quantum states by four semicircles colored blue, red, almond, and purple. 
In this way, we convert the probability distribution for the two-qubit entanglement case by the graphical area of the four semicircles.


Third,
inspired by \textbf{R4}, we correlate all elements for entangled two-qubit states (\textit{i.e.}, four quantum states' state vectors and corresponding probabilities) by the geometry of \toolName\ intrinsically.
Compared to the single-qubit representation, we utilize three auxiliary triangles (the white triangles in Figure \ref{fig:2}\component{B}) to correlate all colored triangles for quantum state representation based on Equation \ref{equation:7}.
It is clear that the base length of the other two triangles equals the bottom triangle's side length, which is always 1.
% As the three auxiliary triangles are right triangles, the sum of the square of each line segment for state vectors, which aligns with 
% Based on this, we linked all entangled two-qubit states together in the visual design by leveraging three auxiliary triangles.

% \subsection{Implementation of \toolName}
% Before the final implementation of the visual design, we came up with several design alternatives to meet the initial requirements, whose detailed justification is provided in Appendix A. To enhance the accessibility and impact of the design, we implemented a web-based graphical interface to fit into the routine tasks of all types of users in quantum computing. The online interface is introduced in Appendix B. 
% For the color selection of \toolName, we adopted different colors with high transparency to indicate the probability distribution due to the categorical quantum states.



% \begin{figure}[tbhp]
% \centering 
% \includegraphics[width=0.9\linewidth]{3_2.pdf}
% \caption{
% The design alternatives of \toolName. 
% (A) The visualization simply converts the three angles of the Bloch Sphere to 2D shapes.
% (B) A triangle-like design that visualizes the probabilities of two basic states (\textit{i.e.}, $\ket{0}$ and $\ket{1}$) using \textit{Viviani's theorem}.
% (C) The design utilizes the square and equilateral triangle area to explicitly display the correlation between the probability and amplitudes.
% % \yong{It is necessary to briefly introduce the three different designs in the caption.}
% % (A) The design alternative where the three radial axes convert the original Bloch Sphere to the 2D rotation of each axis. 
% % (B) The design alternative can visualize the state vector and the probability distribution by applying the geometry principle of the equilateral triangle.
% % (C) The design alternative can visually depict the state vector and probability distribution using line segment length and square area, respectively.
% }
% \label{fig:32}
% \end{figure}


\subsection{Design Justification}




\begin{figure*}[t]
\centering 
\includegraphics[width=0.9\linewidth]{case1.png}
\caption{
The case for the single-qubit quantum classifier. (A) The learning process of the quantum classifier for the Iris dataset. 
The four charts indicate the evolution of a quantum classifier from Epoch 1 to Epoch 100, where the probability of state 1 is 31\% at Epoch 1 and increased to 74\% after 100 learning epochs.
(B) The quantum states of an Iris data point in each step, which consists of two stages (\textit{i.e.}, quantum data embedding and the model layer).
`$P0$' indicates the probability the measurement result is 0.
}
\label{fig:case1}
\end{figure*}



\modify{We considered three design alternatives before we came up with the final design for quantum state representation, as shown in Figure \ref{fig:32}.}
\modify{Specifically, Figure \ref{fig:32}\component{A} shows the three axes (\textit{i.e.}, x-axis, y-axis, and z-axis) of Bloch Sphere, which simply converts the quantum states in Bloch Sphere representation to 2D shapes. However, this approach cannot display the state vector and probability distribution.}
\modify{Figure \ref{fig:32}\component{B} encodes the probabilities of the two single-qubit states (\textit{i.e.}, $\ket{0}$ and $\ket{1}$) by the length of blue and red line segments within an equilateral triangle,
% highlighted in blue and red, 
whose sum is a constant (\textit{e.g.}, the sum of two probabilities that equals 1) due to the geometrical principle of equilateral triangles, \textit{i.e.}, \textit{Viviani's theorem}~\cite{kawasaki2005proof}.
% \yong{1. Why must it be 1? It should depend on the side length of the triangle. 2. Any reference to the geometrical principle?}
}
% which applies the geometry that the sum of the distances from an arbitrary point on the base of an equilateral triangle to the two sides is a constant value. So we apply this rule to encode the two probabilities whose sum equals 1. 
\modify{However, 
% through the co-design process with domain users, they 
the domain experts pointed out in our co-design process
% argued 
that this design could not support the two-qubit state scenarios.}
\modify{Figure \ref{fig:32}\component{C} leverages two or four combinations of a square and a right triangle to visualize single-qubit and two-qubit states, respectively, where the length of two sides of triangles indicates the real and imaginary parts of each vector amplitudes.}
\modify{However, it is difficult to analyze the relationship of all quantum states and how the state vector will affect the probability. }
\modify{Thus, we further proposed the final designs (Figure \ref{fig:2}) that meet all the above requirements.}


