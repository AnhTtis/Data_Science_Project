\section{Informing The Design}




% In this section, we discuss the stages of our co-design process and the design requirements we summarized during the process.
In this section, we introduce our co-design process and the derived design requirements from it, which will inform our subsequent development of the visual design.



\subsection{Co-design Process}

The primary goal of our co-design process is to collect the task abstractions faced by quantum computing users.
% which are used to derive the design requirements of our visual design.
Thus, following the guideline of design study~\cite{sedlmair2012design}, we worked closely with five domain experts for over five months.
First, we conducted preliminary interviews with five quantum computing experts, where we sought to understand the practical challenges and difficulties.
Second, over the next four months, we revised our initial prototype iteratively according to the feedback collected from the expert test.
% The co-design process is shown in Figure \ref{fig:1}\component{B}.
% We present the co-design process and summarize the design requirements which are used to inform our visual design, \toolName.


\textbf{Participants.}
We invited five domain experts \textbf{P1-5} (5 males, $age_{mean}= 32.13$, $age_{sd}= 4.71$) to participate in our co-design process.
Specifically, \textbf{P1} is a research scientist from Pacific Northwest National Laboratory in the U.S., \textbf{P2-5} are either professors or post-doc researchers from three different universities in the U.S. 
Among them, the research direction of \textbf{P1-2} and \textbf{P5} is Quantum Machine Learning, while \textbf{P3-4} are working on Quantum Systems.
All the domain experts have an average of 6.3 years of research and development experience in quantum computing.



\textbf{Preliminary Interview.}
Following the methodology proposed by Sedlmair et al.~\cite{sedlmair2012design},
we began the preliminary interviews by performing one-on-one, semi-structured, hour-long interviews with Group 1 (\textbf{P1-3}), to collect current challenges the participants have when working on quantum computing.
First, each participant was asked to describe the major issues and challenges they faced using the most popular visualization, \textit{i.e.}, \textit{Bloch Sphere}.
Note that each participant in Group 1 used \textit{Bloch Sphere} as the common tool in their daily routine tasks.
Through this session, we collected a set of high-level requirements about the limitations of the \textit{Bloch Sphere} representation.
The aforementioned tasks last about 25 minutes.
Then, we further asked every participant to describe the future requirements in a think-aloud manner, including those that can facilitate the visual analysis of quantum states.
We summarized these initial requirements and expectations during this session.
This session lasted about 30 minutes.
We video-recorded and took notes for each interview and discussion.
All requirements collected in this round were used to inform our initial prototype.


\textbf{Expert Test.}
Over the next four months, we focused on iteratively testing the features of our initial prototype with Group 2 (\textbf{P4-5}).
We designed and implemented the visual design according to the qualitative requirements collected from Group 1
% in the previous session
and released it as an online interface, which allows Group 2 to access and use it.
Specifically, we began the expert test by briefly introducing the session's purpose and describing the initial prototype's functionality.
Each participant was encouraged to use our design when performing their daily domain tasks in quantum computing.
We collected their feedback by holding iterative meetings with the participants every two weeks, which were open and loosely structured to ensure they could express anything they thought of. 
We further tuned the several design alternatives (as shown in Fig. \ref{fig:32}) accordingly to guarantee our design meets all these practical needs.




\subsection{Design Requirements}
\label{sec:design_requirements}

We summarized all design requirements from the co-design process.
% , including both the initial feedback from Group 1 and subsequent requirements from Group 2.
% summarized
% from the stage of during the stage of the expert test. 
We reported six refined requirements and categorized them into \textbf{functionality} and \textbf{usability}.     


For the functionality, participants reported three major requirements to facilitate the observation of arbitrary quantum states:
% which need to be enhanced from the following aspects:

\begin{itemize}
  
    \item[\textbf{R1}] \hspace{7pt}\textbf{Visualize two-qubit states in addition to one-qubit states.}
    All participants (\textbf{P1-5}) reported a strong need to support the visualization of entanglement for two qubits. Thus, the basis for two-qubit entanglement representation is how to represent a two-qubit state.
    % entanglement of qubits, especially for the two-qubit state visualization. 
    \textbf{P3} also suggested that it would be helpful to represent single-qubit and two-qubit states in the same set of visualization other than using two different forms of approaches.
    
              
    \item[\textbf{R2}] \hspace{7pt} \textbf{Provide an intuitive representation for the probability distribution of different states.}
    All participants (\textbf{P1-5}) suggested that visually reflecting the probability distribution is essential for the quantum state exploration as the probabilities of different states explicitly reveal the \textit{superposition}, which is the fundamental ingredient of quantum computing. 
    % Thus, they believe it is crucial to
    Also, they preferred a visualization that can naturally show the probabilities with visual elements without any time-consuming manual calculation.
      
      
    \item[\textbf{R3}] \hspace{7pt} \textbf{Inform users of the state vector of each quantum state.}
    Four participants (\textbf{P1-3, P5}) confirmed that showing the state vector (\textit{e.g.}, amplitude $\alpha$ and $\beta$) as-is will greatly benefit quantum computing users. 
    % \textbf{P1} also mentioned that showing the essential individual element of each state in a state vector will benefit all users.
    \textbf{P5} also reported that forming the visualization with states vectors is better for any quantum computing users due to every visual element's intuitiveness.
    

    
\end{itemize}




For usability, participants focused on 
% the needs on 
how to make full use of the visualization for domain users, which was summarized as follows:



\begin{figure*}[t]
\centering 
\includegraphics[width=0.9\linewidth]{2_.png}
\caption{The visual design of \toolName\, which supports single-qubit (A) and two-qubit (B) state representation based on the same visualization form. 
\textbf{Line segments} visualize the state vector, where the black line denotes the real part, and the grey line denotes the imaginary part based on Equation \ref{equation:1}.
\textbf{Semicircles}'s area indicates the probability of measuring the corresponding state based on Equation \ref{equation:2_1}.
\textbf{Triangle base}'s length consistently equals to 1, because it encodes based on the constraint of normalization (\textit{e.g.}, Equation \ref{equation:4}).
}
\label{fig:2}
\end{figure*}






\begin{itemize}
  
    \item[\textbf{R4}] \hspace{7pt} \textbf{Visually correlate elements in quantum states.}
    % Three participants (\textbf{P1-2}, \textbf{P5}) emphasized that it is significant to make the visual design easy-to-understand for quantum computing users.
    Three participants (\textbf{P1-2}, \textbf{P5}) emphasized that it is significant to represent the state probabilities naturally by the basic element (\textit{e.g.}, state vectors) other than an additional individual visual channel,
    since visually correlating different individual elements by the quantum mechanics theory makes the evolution of quantum states more easy-to-understand.
    % visually correlating different individual elements for a given quantum state (\textit{i.e.}, state vectors and the corresponding probabilities) and 
    % encoding the quantum mechanics theory that the quantum computing users are familiar with into the visual elements will be helpful for domain users.
    
    
    \item[\textbf{R5}] \hspace{7pt} \textbf{Display all visual elements with 2D shapes.}
    Three participants (\textbf{P3-5}) confirmed that they prefer a 2D representation to 3D of a quantum state.
    For example, the \textit{Bloch Sphere} - the visualization tool that quantum computing users use most - is a 3D sphere. They commented that it is tough to locate the point's location and read the rotation angles accurately, even though the interaction of dragging is supported in some tools.
        
        
    \item[\textbf{R6}] \hspace{7pt}\textbf{Make the visualization tool accessible for all quantum computing users.}
    Even though four participants (\textbf{P1-2}, \textbf{P4-5}) gave positive feedback for the useful prototype interface to assist their routine tasks during the stage of expert tests, they also emphasized the necessity of making it publicly-available to benefit all quantum computing users. 
    Considering the power and popularity of the web-based cloud quantum computing platforms (\textit{e.g.}, IBM Quantum), an online web interface will be a good choice.
    

\end{itemize}



