

\section{Case Study}




% In this section, we conducted two case studies to demonstrate the effectiveness of \toolName. 
We utilized two applications, \textit{i.e.}, a two-class quantum classifier and \textit{Grover's algorithm}, to evaluate \toolName\ for single-qubit and two-qubit state representation, respectively.
% The users involved in the case studies are two quantum computing experts (E4 and E9) who also attended the expert interviews in Section \ref{sec:user_study}.
The participants were asked to 
% perform the applications with the online interface and 
use a monitor with 1920 $\times$ 1080 resolution in advance.





\subsection{Case Study \uppercase\expandafter{\romannumeral 1} - Single-qubit Quantum Classifier}







E4 employed \toolName\ to explore various quantum states at the different stages of a quantum classifier.
% Specifically, E4 expected to explore a quantum neural network for two-class classification.
% Meanwhile, E4 attempted to implement the quantum classifier with the quantum simulator to avoid noise and enhance the fidelity of the measurements.
Specifically, E4 utilized Iris datasets~\cite{iris} to train the quantum classifier with a quantum simulator.
Following the design methodology~\cite{Pennylane_example}, E4 encoded a single qubit by applying the two features of the Iris dataset.
According to the network architecture proposed by Stein et al.~\cite{SamMLSys22}, E4 first implemented the classifier circuit (Figure \ref{fig:case1}) using TorchQuantum~~\cite{TorchQuantum}, which can support a flexible output for quantum states at each stage of the classification.
% Figure \ref{fig:case1} shows the circuit to execute the quantum neural network.



\textbf{Understanding the learning process of the quantum classifier.}
E4 was curious about how the learning process will act on the data points. \textit{``Generally, I faced the density matrix of the gates and qubits only, so it will be interesting to check out the evolution of the quantum states visually.''}
Keeping this in mind, E4 first selected a data point from the validation sets, whose two features were 1.4595 and 0.6797, respectively.
After the selection, E4 glanced at the quantum state output at the first epoch out of 100.
As shown in Figure  \ref{fig:case1}\subcomponent{A1}, it is clear that the probability of state 0 is slightly larger than state 1, as indicated by the area of the two semicircles.
E4 was surprised that \toolName\ shows the state probability from the amplitude values directly without any calculation
% , which is very useful and time-saving.
% ``I just type in the two complex numbers of the state vector of the quantum state, and it shows the corresponding probability of the two states at once without any calculation from me. That is very useful and helpful.''E4 commented.
E4 then noticed that the label of the data point is 1
% , which is represented by state 1 in the quantum classifier.
\textit{``Due to Label 1, I expect to see when the semicircle on the right will be larger than the left one, and how the state vector affects this convergence process.''}
So E4 evenly output the results of the data point's quantum states in subsequent three epochs (\textit{i.e.}, 25, 50, 100) to see the evolution of the quantum classifier.
As shown in Figure \ref{fig:case1}\subcomponent{A2}, the area of the semicircles indicating the probability of state 0 and 1 are almost the same at Epoch 25, and then the semicircle area of state 1 is significantly larger than that of state 0 at Epoch 50, and almost remains the same at Epoch 100.
E4 commented that it is apparent that the quantum classifier has converged around Epoch 50 for this data point.
Meanwhile, E4 noticed that this probability change was caused by the evolution of the corresponding state vector indicated by the line segments within the semicircles.
% Based on the finding, E4 then reported that The change of the two sets of line segments in states 0 and 1 have the same pattern: the real parts have increased and the imaginary parts have decreased according to the line segments.
In addition, E4 found that the increase of the real part in state 1 mainly results in the larger area of the red semicircle.
% \textit{``''}
E4 then confirmed,
\textit{``These findings provide me with a good guideline for model tuning. For example, during the parameter initialization of the rotation gates, I will choose those rotation angles which could make the absolute value of real parts larger than the imaginary part for both state 0 and 1.''} 
% \yong{The first quotation mark should be ``, NOT ''. Please check it throughout the paper.}
% After the learning process exploration, E4 summarized that
% \toolName\ provides him with an intuitive representation of the state probability distribution and naturally reveals insights into how the state vector will affect the prediction via the valuable and concise 2D shapes.
% The data points' state vectors. More importantly, it makes me naturally aware of the probability distribution and 







\textbf{Unveiling the mask of the prediction process of QNN.}
Building upon the trained quantum classifier, E4 attempted to perform the prediction process and inspect the propagation of data points.
First, E4 mentioned that despite the importance of quantum data embedding, 
it is still hard for him to understand how the original decimal data points are transformed into abstract quantum states.
% is a crucial stage to initially embed original datasets into quantum states that is a valid form for the subsequent trainable quantum gates.
% However, E4 reported that, 
% image what the embedded quantum states look like as the complex matrix cannot reflect the original decimal data points intuitively.
% Thus, E4 expected to figure out the embedded quantum states by \toolName.
\modify{E4 hinted that the whole prediction process consists of four gates, where the first two gates are used for quantum data embedding and the last two gates correspond to the trained classification model,
% for the prediction, 
as shown in Figure \ref{fig:case1}.}
Hence, E4 first randomly selected a data point and then visualized it after the first gate (\textit{i.e.}, RY gate).
\modify{E4 was surprised that the probabilities of the two states were around 0.50 (Figure  \ref{fig:case1}\subcomponent{B1})}:
% \textit{``It is interesting to me because \toolName\ provides me an intuitive image of quantum states after embedding instead of the density matrix consisting of complex numbers.'' }E4 reported.
% the only thing I encountered during the QNN circuit design was the density matrix, which contains two totally different sets of complex numbers. I never thought the probability distribution was unchanged for the embedding.'' E4 reported.
\modify{After the second quantum gate RZ gate ( Figure  \ref{fig:case1}\subcomponent{B2})}, E4 further noticed that the line segments representing amplitudes coincide with the semicircles' diameters, while the amplitudes' imaginary parts convert to non-zero values.
% E4 then reported that it is apparent that the embedding for the first gate (\textit{i.e.}, RY gate) does not convert the decimal feature values into the complex number because the imaginary parts are still zero for both quantum states. Next, the RZ gate converts the original number into complex numbers with non-zero real and imaginary parts, as indicated by the line segments in Figure xx.
E4 explained this phenomenon, \textit{``RZ is better to encode information because RZ is for the phase rotation, so this converting makes the input data more resilient to errors due to non-zero imaginary parts.''}
% ``For me, '' E4 then added.
After exploring the two gates for quantum data embedding, 
% E4 was curious about the process of the model layer, which applies the trained model to the embedded quantum states.
% So 
\modify{E4 started to explore the two gates representing the classification model. He first inspected the quantum state after RY gate (Figure \ref{fig:case1}\subcomponent{B3})}.
% selected a data point with two features of 1.4595 and 0.7902, and then 
% output the quantum states after each gate in the model layer.
% As shown in Figure xx, the initial states after the quantum data embedding were converted two times by RY and RZ gates.
\modify{E4 noticed that the probability of state 0 decreased from 0.56 to 0.10, as indicated by the semicircles' area, and almost
% remains 
remained
the same after the last gate (Figure \ref{fig:case1}\subcomponent{B4})}. 
% \yong{1. Pls further check all the tense usage of your writing throughout the paper. 2. When revising some parts, please also read the sentences before and after them to guarantee the logic flow is smooth and makes sense.}
% E4 then found that this is because the absolute values of amplitudes have increased after comparing the corresponding line segments, making the area of the probability of state 0 increase.
\textit{``This is mainly because the absolute values of amplitudes have increased and then cause the area of the probability of State 0 to increase''}, E4 commented.
% \yong{Who said it?}
% E4 then explored the quantum states of several data points with \toolName\ and reported that
% ``This tool is appealing to me because I found no matter how the length of line segments changes, the vertex of the triangle will always locate on the semicircle representing state probability, which makes it much more intuitive to see how the density matrix affects the final measured probability. Integrating the math foundation of quantum computing theory and the visualization is clever and useful.''





\subsection{Case Study \uppercase\expandafter{\romannumeral 2} - Two-qubit Grover's Algorithm}


\begin{figure*}[t]
\centering 
\includegraphics[width=0.87\linewidth]{case2.png}
\caption{
The case for the two-qubit quantum algorithm, \textit{i.e.}, Grover's Algorithm. (A) The calculation process of Grover's Algorithm with one iteration, along with four consecutive quantum states and an interaction shown by \toolName.
(B) The Grover's Algorithm with one more iteration appended after the original circuit, along with two quantum states representation.
}
\label{fig:case2}
\end{figure*}





E9's research interest lies primarily in \textit{Grover's algorithm}~\cite{grover1996fast}, a famous algorithm for the unstructured searching problem.
% , a participant also participated in the expert interview, 
Thus, E9 planned to explore the evolution of quantum states in \textit{Grover's algorithm}.
Following ~\cite{grover}, E9 implemented the circuit with one iteration and the target ``winner'' of state $\ket{11}$.
% The circuit to execute the searching algorithm is shown in Figure \ref{fig:case2}\component{A}.


\textbf{Revealing the insights of the hidden quantum states in Grover's algorithm.}
E9 was curious about the insights of the quantum states provided by \toolName, 
\modify{and expected to see how different modules interact with each other from a view of the functionality block other than individual quantum gates.} 
% \yong{What is ``the systematic circuit''???}
% ``We designed the algorithm from a high-level perspective other than the specific quantum states, so I expect to see something different using this visualization tool.''
Thus, E9 input the density matrix of quantum states into \toolName.
After a glance at the results of the Hadamard gates (Figure \ref{fig:case2}\subcomponent{A1}), 
\modify{which is used to generate the superposition of qubits}, 
% \yong{please check it}
E9 noticed that the probabilities of the four states ($\ket{00}$, $\ket{01}$, $\ket{10}$, $\ket{11}$) are the same, as indicated by the white rectangle at the center and the same area of the four semicircles.
\textit{``It is clear that there is a uniform superposition at this stage because I found the probabilities are all the same. Besides,  the imaginary parts are 0 for all quantum states since the line segments coincide with the semicircle.''}
\modify{E9 then exported the chart after the oracle process (Figure \ref{fig:case2}\subcomponent{A2}) that is used to flip the phase of the searched state.}
% \yong{How can E9 output the chart? He can only download or export the chart.}
% to speculate how the quantum states will be pre-processed before the iteration.
% After comparing the first chart, E9 
He quickly identified a phase flip indicated by the double line for state $\ket{11}$.
Thus, E9 reported, \textit{``Clearly, the quantum state $\ket{11}$ is the one we marked before the initialization due to the negative amplitude shown by the double line.''}
Building upon these findings of the pre-process before the iteration, E9 expected to see \textit{``what the state ``looks like'' after the whole iteration stage.''}
\modify{Hence, E9 exported the visualization of the state (Figure \ref{fig:case2}\subcomponent{A5}) after the diffuser process, which is for the amplitude amplification of the target state. He quickly found that only a purple semicircle remained in the chart.}
\textit{``To my surprise, all semicircles and triangles disappeared compared to the previous figure. This is probably because the iteration found the winner, whose probability is 1.00''}, E9 said.
% \yong{1. Pls carefully check the logic of your claims. 2. Who said this? Please explicitly point it out before this sentence.}
E9 commented, 
\textit{``It is interesting to see the original two-qubit state (\textit{i.e.}, four semicircles) convert to only one semicircle, indicating the searching has converged.''}
% E9 also added that \toolName\ provided him with many insights and details for \textit{Grover's algorithm} analysis based on the intuitive visual evidence instead of the inflexible density matrix.



\textbf{Performing a what-if analysis for multiple iterations.}
\modify{
% Upon the experiment of \textit{Grover's algorithm} with \toolName, 
When using \toolName{} to explore the quantum state evolution of \textit{Grover's algorithm},
E9 was also interested in exploring what would happen if another iteration was implemented on the circuit, as more iteration processes may lead to a better result.
% or remain the same.
}
So he implemented one more iteration block and checked the results using \toolName\ (Figure \ref{fig:case2}\component{B}).
\textit{``As I expected, the searching did not work with another iteration, because the four semicircles with an equal area of 0.25 (Figure \ref{fig:case2}\subcomponent{B2}) unlike the single output in the previous execution} \textit{''}, E9 said.
So, E9 attempted to explore the possible reasons.
\modify{To this end, he compared all charts of each gate in the two iterations and quickly noticed that all states looked the same except those two after the last set of Hadamard gates in the respective iteration (Figure \ref{fig:case2}\subcomponent{A3} and Figure \ref{fig:case2}\subcomponent{B1}).}
Specifically, Figure \ref{fig:case2}\subcomponent{A3} shows that the two states (\textit{i.e.}, $\ket{10}$ and $\ket{11}$) are with the same probability of 0.50 after the first Hadamard gate, while Figure \ref{fig:case2}\subcomponent{B1} indicated that the two states (\textit{i.e.}, $\ket{00}$ and $\ket{11}$) are 0.50 at the same gate during the second iteration.
Then, E9 hovered on the base triangle in Figure \ref{fig:case2}\subcomponent{A3} and the popped tooltip indicated that the probability is 1.00 when the first qubit's state is 1 (Figure \ref{fig:case2}\subcomponent{A4}).
However, no tooltip popped out when he hovered over the second chart.
Thus, E9 concluded that the failure of one more iteration is due to the differences between the two density matrices after the second last Hadamard gate. The last Hadamard gate can convert the state to the one with a probability of 1.00 (Figure \ref{fig:case2}\subcomponent{A4}) with the common first qubit's state with a probability of 0.50. 
% In contrast, the one in the second iteration cannot output the probability of 1.00 because the quantum state does not contain the same first qubit's state (Figure \ref{fig:case2}\subcomponent{B1}). 
\textit{``This is because the Hadamard gate can output the unique value (\textit{i.e.}, 1.00) only if the operated state has the equal value on the first qubit's state.''} E9 reported.
% After the analysis, E9 also gave positive comments that 
% the user interaction impressed him because it provided an intuitive representation when analyzing the single-qubit state in a two-qubit state application. At the same time, it is tough to analyze the density matrix of the quantum state directly.
