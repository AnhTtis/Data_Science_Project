\section{Related Work}

Our work is relevant to prior studies on graphical tools in quantum computing and the visual design of quantum states. 


\subsection{Graphical Tools in Quantum Computing}

Researchers have attempted to visualize quantum circuits using graphical interfaces,
% Much prior work focused on, 
including quantum circuit implementation
% ~\cite{gheorghiu2018quantum,paykin2017qwire}
and building graphical simulators for quantum circuits.
% ~\cite{kelly2018simulating,zulehner2018advanced, moran2016quintuple}.
Specifically, Paykin et al.~\cite{paykin2017qwire} presented an interface to manipulate quantum circuits using a classical host approach. 
Zulehner et al.~\cite{zulehner2018advanced} proposed a new graph-based approach for quantum simulators.
Much prior work has also been studied to interpret quantum algorithms and workflow using a graphical approach.
% ~\cite{tao2017shorvis,karafyllidis2003visualization, ruan2022vacsen,miller2021graphstatevis,williams2021qcvis,lin2018quflow}.
Ruan et al.~\cite{ruan2022vacsen} introduced a real-time visualization system for noise awareness in quantum computers and compiled circuits.
Tao et al.~\cite{tao2017shorvis} propose a tool to assist users in understanding Shor's algorithms using a graphical interface.
% Karafyllidis et al.~\cite{karafyllidis2003visualization} also leveraged the visualization approach to help novices understand QFT (\textit{i.e.}, Quantum Fourier Transform) algorithm.
In addition, 
% cloud platforms support online quantum circuit implementation~\cite{quirk, qcircuits, ibCirqmq}.
% For example, 
\textit{Quirk}~\cite{quirk} also provides a graphical interface to make users aware of the quantum circuit's behavior. 
% Many large IT companies also provide cloud platforms that allow users to implement quantum circuits online, such as IBM Quantum~\cite{ibmq}.
The aforementioned interfaces focused on improving the interpretability of quantum circuits using 
% a graphical interface, 
% they mainly utilize 
multiple fundamental visualizations indivisually (\textit{e.g.}, bar charts, circles in \textit{Quirk}~\cite{quirk}). 
However, this type of visualization cannot intuitively visualize the relationship between properties in quantum computing, making the users feel struggle to understand the changes happening in quantum states.
Our work addressed this challenge by introducing a visual solution, assisting users in quickly understanding single-qubit and two-qubit states thoroughly. 


\subsection{Visual Design of Quantum States}

Many prior studies have focused on visualizing quantum states using 
% a variety of methods, which can be categorized into
3D and 2D visualizations.
For 3D representation, 
\modify{a} Block Sphere~\cite{bloch1946nuclear} visualizes single-qubit states based on a 3D geometrical representation and is still the widely-used visual representation till now~\cite{131387, havel2004bloch}.
Some prior work focused on extending \modify{Bloch Spheres}.
For example,
% ~\cite{makela2010n,altepeter2009multiple,wie2020two}.
Altepeter et al.~\cite{altepeter2009multiple} extended Bloch Sphere using the remote-state preparation protocol.
% Boyer et al.~\cite{boyer2017geometry} provided a geometrical analysis of entanglement and separability for all mixed states.
IBM also provides a Bloch Sphere-like design called Q-Sphere~\cite{ibmq} to represent multiple states in a single sphere.
% Tamate et al.~\cite{tamate2011bloch} proposed a way to represent three-vertex geometric phases for multiple state systems on the Bloch sphere.
In addition, many researchers have studied how to represent quantum states using 2D shapes other than extending Bloch Sphere.
For example,
% ~\cite{galambos2012visualizing, lopez2018geometry, avanesov2019unitary, chernega2017triangle}.
Galambos et al.~\cite{galambos2012visualizing} utilized fractal representation to visualize the multi-qubit qubit systems.
Chernega et al.~\cite{chernega2017triangle} mapped the density matrix of the qubit onto the vertices of a triangle. 
% Avanesov et al.~\cite{avanesov2019unitary} offered a method to visualize the quantum channel’s maps of qubit states.
While the prior studies provided different ways to portray the quantum states, 
none of the prior work focused on the representation of superposition, which reflects the probability of measuring each quantum state.
We aim to visually correlate multiple properties with the probability of each quantum state, making the users understand the probability more comprehensively.
%without any manual calculation.
