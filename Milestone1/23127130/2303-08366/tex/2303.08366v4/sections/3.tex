\section{Background}


In this section, we introduce a set of relevant concepts in quantum computing,
% that are  to our work, 
including the building block of quantum computing, the properties of the qubit, and the quantum state of the qubit. 

\def\Zero{
\begin{bmatrix}
    1 \\
    0 \\
\end{bmatrix}}

\def\One{
\begin{bmatrix}
    0 \\
    1 \\
\end{bmatrix}}

\def\sv{
\begin{bmatrix}
    \alpha \\
    \beta \\
\end{bmatrix}}

\def\svcomplex{
\begin{bmatrix}
    a+bi \\
    c+di \\
\end{bmatrix}}


\subsection{Building Block of Quantum Computing}

\textbf{Qubit},
the quantum version of the classic bit, is the basic unit in quantum computing. Similar to a classical bit, there are two computational basis states called state 0 and state 1 for a qubit~\cite{ciaran2021qc4c}. 
However, 
% in contrast to a classical bit, which is either in state 
% 0 or state 1, 
a qubit can also be in an arbitrary linear superposition of state 0 and state 1~\cite{ciaran2021qc4c, rieffel2000introQCnonP}, which is well-known as quantum superposition.
Mathematically, one can represent a qubit using the form of a state vector~\cite{rieffel2000introQCnonP}.
% Bra-Ket Notation or Bloch sphere. 
% Kets like $\ket{x}$ denote column vectors and are typically used to describe quantum states~\cite{rieffel2000introQCnonP}. 
% For instance, state 0 and state 1 can be expressed as $\ket{0}$ = $\Zero$, $\ket{1}$ = $\One$, respectively. 
% When dealing with qubits, and quantum computing in general, a fixed basis with respect to which all statements are made has been chosen in advance. In particular, unless otherwise specified, all measurements will be made concerning the standard basis for quantum computing, ${\ket{0}, \ket{1}}$~\cite{rieffel2000introQCnonP}. 
% As mentioned above, the quantum state $\ket{\psi}$ of a qubit can also be represented in a linear superposition of $\ket{0}$ and $\ket{1}$. 
% Section \ref{qubit_prop} includes more details about the concept of superposition. 

% The state of a single qubit can also be studied by Bloch Sphere. Bloch Sphere is a visual representation of a qubit with similar geometric properties to the unit circle from trigonometry. 
% % Each point on Bloch Sphere corresponds to a different possible superposition of a single qubit~\cite{ciaran2021qc4c}. The top and bottom of the Bloch Sphere correspond to the two measurable quantum states of the qubit, $\ket{0}$ and $\ket{1}$~\cite{ciaran2021qc4c}.
% A vector on the Bloch Sphere, which can point to any of the different locations on the surface of the sphere, indicates the current quantum state of the qubit~\cite{ciaran2021qc4c}. Figure \ref{fig:1}\component{A} is an example of Bloch Sphere visualizing the quantum state of a single qubit. Bloch Sphere is a helpful visualization for understanding how a qubit can have infinite possible quantum states. However, it can only represent one qubit and does not work for systems of two or more qubits~\cite{ciaran2021qc4c, bardin2021microwaves, wie2014bloch}.


% quantum state, superposition, |0>, |1>, psi-matrix, denoted by |0>
% quantum state is represented by |psi>
% similar to classical bit, |0>, |1>
% unlike classical bit, a|0> + b|1>, will discuss this in section 3.2. |psi> = a|0> + b|1> = a[1,0] + b[0,1] = [a,b]


\textbf{Quantum gate},
just like the manipulation of classical bits using classical logic instructions such as \textit{OR} and \textit{AND}, it is applied to qubits to change their quantum states. 
% and the quantum states of qubits change depending on which gate is applied. 
% In Bloch Sphere representation, gates provide instructions for rotating the qubit’s vector around the sphere~\cite{ciaran2021qc4c}. 
Mathematically, quantum gates are represented as operation matrices, acting on single qubit or multiple qubits.
Operations of quantum gates are equivalent to the dot products with the state vector of qubits.
% \yong{Applying xxx to qubits? Please double check the verb.}
% that act on qubits using matrix multiplication.
%Quantum Gate are represented by matrix. Single quantum gate like H, Multiple Quantum gate like CNOT.
%Pics of gate

\textbf{Quantum circuit},
similar to the classical circuit, is the implementation of the quantum program for execution.
A quantum circuit consists of a set of quantum gates, acting on one or multiple fixed qubits.
% in a quantum computer or quantum simulator.
As shown in Figure \ref{fig:case1} and \ref{fig:case2}, a quantum state will be initialized from the start of the quantum circuit and manipulated by quantum gates designed in the quantum circuit. 
% The final execution result is retrieved by measuring each quantum state.




\subsection{Properties of Qubit}
\label{qubit_prop}
\textbf{Superposition}
indicates that a qubit can not only be in one of the computational basis states $\ket{0}$ or $\ket{1}$, but also in a linear superposition of this two states~\cite {nara1999QC4B}.
% As mentioned in Section 3.1, compared with classical bits, the value can only be either 0 or 1, 
Thus, the quantum state $\ket{\psi}$ of a qubit is described by $\alpha\ket{0} + \beta\ket{1}$, where the complex numbers $\alpha$ and $\beta$ are called \textit{amplitudes} such that $|\alpha|^2 + |\beta|^2 = 1$~\cite{rieffel2000introQCnonP}. 
Meanwhile, 
% Once such superposition is measured with respect to the basis state {$\ket{0}$ and $\ket{1}$}, 
the probability of measuring $\ket{0}$ is $|\alpha|^2$ and the probability of $\ket{1}$ is $|\beta|^2$
~\cite{ciaran2021qc4c, rieffel2000introQCnonP,nara1999QC4B}.
% The real power of quantum computing derives from the exponentially increasing state space of multiple qubits,
as a quantum system with $n$ qubits can generate a linear superposition of $2^n$ basis states simultaneously~\cite{rieffel2000introQCnonP, Hey1999QuantumCA}.
% : since the quantum state of a single qubit can be in a linear superposition of $\ket{0}$ and $\ket{1}$, a quantum system with $n$ qubits can generate a linear superposition of $2^n$ basis states~\cite{rieffel2000introQCnonP, Hey1999QuantumCA}. 
% Thus, creating such a superposition is the critical properties of achieving the quantum advantage~\cite{Hey1999QuantumCA}.

% that gives quantum parallel processing its power~\cite{Hey1999QuantumCA}.

% Example.
% It is noteworthy that even though a qubit can have infinitely many superposition states before measurement, measuring a qubit collapses its superposition state into one of two possibilities~\cite{ciaran2021qc4c, rieffel2000introQCnonP}. That is, if the measurement of $\ket{\psi} = \alpha\ket{0} + \beta\ket{1}$ results in $\ket{0}$, then the state changes to $\ket{0}$, and a second measurement with respect to the same basis will return $\ket{0}$ with probability 1~\cite{rieffel2000introQCnonP}. Thus, unless the original state happens to be one of the basis vectors, the measurement will change that state, and it is impossible to determine what the original state was~\cite{rieffel2000introQCnonP}.


\textbf{Entanglement}
is an essential feature that differentiates qubits and classical bits. 
Specifically,
% In quantum computing, 
when two or more qubits are entangled, their quantum states are coupled with each other,
%they will exist in a single quantum state. 
so that changing the quantum state of any one qubit will instantaneously change the other qubit's quantum state in a predictable way~\cite{rieffel2000introQCnonP}.
% Instead, an easy test to determine if a system is entangled or not is to check if measuring the value of one qubit changes the probability distribution of the other qubit~\cite{ciaran2021qc4c}. 
% For instance, the quantum state $\ket{\psi} = \dfrac{1}{\sqrt{2}}\ket{00} + \dfrac{1}{\sqrt{2}}\ket{11}$ is entangled since the probability that the first qubit is measured to be $\ket{0}$ is $1/2$ if the second qubit has not been measured. 
Note that the entanglement operation requires more than one qubit, making it critically important to analyze the quantum states of multiple qubits instead of a single qubit.
% Note that the two-qubit entangle is a specific scenario of two-qubit state, where the two qubits are entangled.
% Thus, the approach of two-qubit state representation is of great value for solving the two-qubit entanglement representation.
% However, if the second qubit has been measured, the probability that the first qubit is measured as $\ket{0}$ is 1 or 0, if the second qubit was measured as $\ket{0}$ or $\ket{1}$, respectively. In this case, the result of measuring the first qubit depends on the measurement of the second qubit. 
% In contrast, in the case of $\ket{\psi} = \dfrac{1}{\sqrt{2}}\ket{00} + \dfrac{1}{\sqrt{2}}\ket{01}$, two qubits are not entangled since any measurement of the first qubit will yield $\ket{0}$ regardless of whether the second qubit was measured. Similarly, the second qubit has a fifty-fifty chance of being measured as $\ket{0}$ regardless of whether the first qubit was measured or not~\cite{rieffel2000introQCnonP}. Entanglement happens even if qubits are far away from each other. When two or more qubits are entangled, the quantum state can not be described in terms of the state of each of its components (qubits) separately. 


\subsection{Quantum State of Qubit}
\label{sec:3.3}
In quantum computing, a quantum state is a mathematical entity that provides a probability distribution of different basis states.
% on a single qubit or multiple qubits. 
For clarity, we start with a single-qubit state, and the case of a two-qubit state will be derived from these results.
In Section \ref{sec:venus}, we will illustrate how we apply the following quantum computing characteristics and encode them with a variety of 2D geometric shapes to form our visual design.

\textbf{Single-qubit state} represents the quantum state for a single qubit.
% From Section 3.1 and 3.2,
Recall that
the quantum state of a qubit can be a superposition of basis states {$\ket{0}$ and $\ket{1}$}, thus the quantum state $\ket{\psi}$ can be expressed as $\ket{\psi} = \alpha\ket{0} + \beta\ket{1} = \alpha\Zero + \beta\One = \sv$, where the amplitudes $\alpha$ and $\beta$ satisfy:
% are complex numbers
% , and $|\alpha|^2 + |\beta|^2 = 1$. 
% and can be calculated as follows:


\begin{equation}
\label{equation:1}
\alpha=a+b\cdot i, \beta=c+d\cdot i,
\end{equation}

where $i$ is the imaginary unit, and $a$, $b$, $c$, and $d$ are real numbers.
% Thus, the quantum state of a single qubit can be expressed as $\ket{\psi} = \sv = \svcomplex$. 
Based on the quantum 
% computing 
theory, the probability of a measured quantum state (\textit{e.g.}, $\ket{0}$) satisfies


\begin{equation}
\label{equation:2_1}
Pr(\ket{0}) = |\alpha|^2 = |a|^2 + |b|^2.
\end{equation}

% By definition, a complex number is a number of the form $a + bi$, where $a$ and $b$ are real numbers, and $i$ is an indeterminate satisfying $i^2=-1$. 
Meanwhile, since the amplitudes satisfy a normalization constraint,\textit{ i.e.}, the sum of the probabilities of the two quantum states for single qubits (\textit{i.e.}, $\ket{0}$ and $\ket{1}$) consistently equals 1, 
thus applying Equation \ref{equation:2_1}  yields


\begin{equation}
\label{equation:4}
|a|^2+|b|^2+|c|^2+|d|^2 = 1.
\end{equation}

% $a$, $b$, $c$ and $d$ satisfy 
% $a^2+b^2+c^2+d^2=1$.


\textbf{Two-qubit state} is the quantum states executing on a pair of qubits,
\modify{which can be calculated by the tensor product of two single-qubit states, \textit{e.g.}, $\ket{00} = \ket{0} \otimes \ket{0}$}.
% \yong{Pls check my comments on Overleaf Review.}
Meanwhile, any two qubits can be in the state $\ket{\psi} = \alpha\ket{00} + \beta\ket{01} + \gamma\ket{10} + \delta\ket{11}$, where the amplitudes $\alpha$, $\beta$, $\gamma$, and $\delta$ satisfy:

% For two-qubit entanglement, the state vector consists of four complex number amplitudes (\textit{i.e.}, $\alpha-$ $\beta-$ $\gamma-$ $\delta-$components), which can be represented as follows:


\begin{equation}
\label{equation:5}
\begin{aligned}
\alpha = a + b \cdot i, \beta = c + d \cdot i,\\
\gamma = e + f \cdot i, \delta = g + h \cdot i.
%  &\quad
\end{aligned}
\end{equation}



Similar to single-qubit states, since the probabilities of all possible qubits equal 1, the four amplitudes satisfy  $|\alpha|^2 + |\beta|^2 + |\gamma|^2 + |\delta|^2 = 1$. 
By applying Equation \ref{equation:5} to the above constraint,
we have
% By applying Equation \ref{equation:5} to Equation \ref{equation:6}, we can calculate the lengths of all line segments for state vector display as follows:


\begin{equation}
\label{equation:7}
|a|^2+|b|^2+|c|^2+|d|^2+|e|^2+|f|^2+|g|^2+|h|^2 = 1.
\end{equation}



% \begin{figure}[t]
% \centering 
% \includegraphics[width=0.7\linewidth]{figures/1_.pdf}
% \caption{A widely-used visualization called Bloch Sphere for quantum state representation.
% % (B) The workflow of the co-design process with domain experts. The steps connected by the green line indicate the stage of the preliminary interview for the initial prototype, while those steps connected by the blue line represent the iterative tuning process in the stage of the expert test.
% }
% \label{fig:1}
% \end{figure}