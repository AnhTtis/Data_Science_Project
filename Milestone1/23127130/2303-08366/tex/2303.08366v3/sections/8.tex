\section{Discussion}

In this section, we first summarize the lessons we learned during the development of \toolName, 
then we discuss the limitations of \toolName.

\vspace{-0.3cm}

\subsection{Lessons}
% We learned many valuable lessons from the design and evaluation of our visual design.

% \textbf{Fitting visualization into quantum computing.}
During the above evaluation processes, all participants gave highly positive feedback for \toolName.
Among all the responses, participants emphasized a strong need for visualization to fit into quantum computing regarding the complex quantum physics theory, non-transparency of quantum program process, and non-intuitive quantum computing properties (\textit{i.e.}, quantum entanglement and superposition).
The above challenges make it hard for novices and the general public to have a strong sense of quantum computing.
Thus, the quantum computing community urgently needs visualization to aid the transparency and interpretability of quantum computing with its scientific educational capability.


% \textbf{Lowering the learning curve of visual designs for quantum computing users.}
% During the co-design process, five domain experts strongly emphasized the difficulty for quantum computing users to learn and use cross-disciplinary visualizations. 
% All domain experts preferred concise and straightforward visual designs, which can really help them instead of those sophisticated visual solutions.
% Thus, \toolName\ received highly positive feedback concerning the simplicity and intuitiveness, giving it the potential to be widely spread by general users in quantum computing.
% Furthermore, we learned from the interviews that it is crucial to correlate the visual elements using quantum computing characteristics other than using many different and individual visual channels, such as the visual solution provided by \textit{Quirk}.


\vspace{-0.4cm}

\subsection{Limitations}

Our evaluation shows that \toolName\ can effectively facilitate quantum state observation. However, there are still some limitations.



\textbf{Limited support for quantum noise visualization.}
\toolName\ can effectively visualize various quantum states in situations where noise analysis is not required,
such as the design and debugging of quantum algorithms. 
% \yong{why? No noise in these applications??}
\modify{We do not consider the noise analysis of  \toolName\ because the design is built upon quantum simulators where the execution of quantum circuits is completely noise-free.}
% \modify{We do not consider the noise model of the simulators since too complex functionality might confuse the general users and learners under the education scenarios.}
% It would be better to support the comparison with noisy and noise-free quantum state results.
% \yong{Pls check my comments.}






\textbf{Scalability.} 
% As the current focus of \toolName\ is for 1-qubit and 2-qubit problems,
\toolName{} currently targets visualizing the quantum states of one or two qubits. Compared with Block Sphere, it can effectively visualize the quantum entanglement, a significant step towards effective qubit state visualization confirmed by the participants.
% Meanwhile, there are requests to prepare for the next generation of quantum computers with a scale difference of more than an order of magnitude. 
Also, E8 suggested enabling representation for more qubits by adding more triangles on top of \toolName. In the future, we will endeavor to extend the current design for more qubits.



\textbf{Time-consuming input for state vectors.}
The update of \toolName\ is driven by the inputted number of amplitudes, which requires users to input the real and imaginary parts of amplitudes manually.
Participants reported that it is inconvenient to type in the amplitude values.
However, due to our contribution to a design study, we plan to address this limitation in the future.
For example, as hinted by E4, converting from the popular visualization, Bloch Sphere, will also be helpful for users.


