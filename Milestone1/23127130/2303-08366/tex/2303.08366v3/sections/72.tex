
\section{Interview Results}
\label{sec:result}


In this section, we report the summarized results, including the Bloch Sphere's challenges, the responses for the three analytical tasks, and the suggestions provided by the participants. We reported the ratings from each participant in Figure \ref{fig:4}.


\subsection{Current Bloch Sphere Challenges}
We invited all participants to participate in the feedback collection for Bloch Sphere, collecting challenges faced by the participants when attempting to perform quantum state observation using Bloch Sphere. We use this session to confirm our previous findings from the co-design process.


\textbf{Challenge 1: Incapability of the two-qubit state representation.}
All participants agreed or strongly agreed that the inability to support two-qubit representation is a severe limitation for Bloch Sphere, they all agreed that it is unacceptable that Bloch Sphere cannot support multiple qubits' observation, which is the most critical property for quantum computing.
E7 also reported that Bloch Sphere needs to be more scalable to allow the visualization of quantum entanglement.


\textbf{Challenge 2: Non-intuitiveness of the state probability distribution.}
Most participants (11 out of 14) confirmed that Bloch Sphere could not intuitively visualize the probability of each possible quantum state.
E13 mentioned that he has to manually calculate the final measurement from the point's location in Bloch Sphere. However, it is not a trivial task \textit{``because the rotation angle in each axis of Bloch Sphere is hard to measure in a 3D model.''}




\textbf{Challenge 3: Lack of the state vector representation.}
Most participants (10 out of 14) agreed that the missing state vector in Bloch Sphere makes it challenging to understand.
E1 commented, \textit{``Despite the transparency of rotation gates for Bloch Sphere, but
I prefer a more intuitive way to reflect the density matrix
when designing  circuits with  quantum simulators.''}
E8 added that Bloch Sphere is unsuitable for educational purposes because of its complexity, especially for entry-level users.


 






\subsection{Results of Three Analytical Tasks}

We summarized all feedback regarding the three analytical tasks shown in Section \ref{sec:tasks}.



\textbf{Task 1: Two-qubit entanglement analysis.}
All participants agreed or strongly agreed that it was easy to support the two-qubit state observation using \toolName\ \textbf{($rating_{mean}= 6.01$, $rating_{sd}= 1.51$)}.
% E14 emphasized \textit{``I like the idea of having more than one qubit and being able to represent their states, on the Bloch sphere that's not so helpful, adding more qubits seems important to broaden its potential.''}
% Also, all participants confirmed that the feature of two-qubit state representation could yield significant benefits for the visual entanglement analysis.
Specifically, E3 believed that \toolName's most practical usage is to \textit{``make it possible for the entanglement display, which is the most important characteristic and needs for every quantum computing researcher.''}
% Most participants (9 out of 14) praised that \toolName\ enables the two-qubit state visualization using the same geometries for single-qubit visualization, which \textit{``makes the visualization more accessible for general quantum computing users.'' }
% For the specific research directions in quantum computing,
% all participants agreed that \toolName\ could practically assist them in their research domains in quantum computing.
% For example, 
E14, an expert in Quantum Chemistry, mentioned, \textit{``my research is mainly focused on cutting the large quantum circuit; usually, the circuit has more than two qubits. This visualization of more qubits will help me to
% better understand the deeper structure of a large circuit and will help me to 
find the optimal cutting point more accurately.''}
Furthermore, as an expert on Quantum Error Modeling, E9 noted, \textit{``I can see a great potential of \toolName. I think the quantum error correction (QEC) research can make good use of \toolName, for visualizing the encoding qubits and ancilla qubits separately.''}

\textbf{Task 2: State probability observation.}
Most participants agreed that they could identify the probability of all possible quantum states quickly ($rating_{mean}= 6.60$, $rating_{sd}= 1.42$).
E12 mentioned that the encoding area to visualize probability is
\textit{``easy and straightforward''}.
E4 praised that, \textit{``I like the idea of using the probability calculation equation to naturally visualize \modify{the} probability distributin. I can directly check the probability without any manual calculation.''}
% E4, who was working on Quantum Machine Learning and used Bloch Sphere frequently, commented, \textit{``I believe that it gives the user a more intuitive way to observe the probability compared to Bloch Sphere.''}
For the specific research directions in quantum computing,
most participants (9 out of 12) highly appreciated \toolName's usability and felt it could help them to handle their domain-specific tasks smoothly.
% E14 commented, \textit{``this tool helps me understand quantum circuit cutting method, especially this probability distribution provides me a better view for finding an optimal cutting point on a quantum circuit.''}
For example, E4 confirmed that \textit{``(for Quantum Machine Learning) \toolName\ can easily visualize the probability at breakpoints in  debugging, which saves time in in-line debugging.''}



\textbf{Task 3: State vector exploration.}
Most participants agreed that it was intuitive to get a sense of the state vectors in detail using \toolName\ ($rating_{mean}= 5.85$, $rating_{sd}= 1.24$).
Specifically, 8 out of 14 participants felt it helpful to show the amplitudes (\textit{i.e.}, $\alpha$ and $\beta$) separately using multiple right triangles.
% E12 said it was practical to represent quantum states using the state vectors because it makes domain users more easily understand the constructions of quantum states.
E9 agreed that it provides a better view of the relationships between the probability distribution, especially the entanglements.
Due to the accessibility of state vectors, E11 thought the visualizations of the real and imaginary parts of amplitudes were helpful \textit{``for fresh starters in quantum computing.''}
% Participants also commented that the state vector representation could be well integrated into their daily tasks for the specific research directions in quantum computing.
% E14 reported that knowing $a$ and/or $b$ can help him better visualize and understand the measurement basis for quantum circuit cut: \textit{``depending on the value of a and/or b, I can choose a proper measure basis for the post-processing step.''}
Moreover, E2 confirmed the \toolName's value in Quantum System, \textit{``I can quickly compare the consequence of using different gates as it can fit into the debugger.''}
% He also felt that \toolName\ could significantly reduce the requirements of developers to understand quantum physics, making it an excellent in-class teaching tool for non-physics background students.

\vspace{-0.3cm}

\subsection{Overall Feedback}
From the participants' responses and ratings (\textit{i.e.}, $rating_{mean}= 5.93$, $rating_{sd}= 1.03$), we summarize the overall feedback regarding the user interaction, the integration with the workflow, and the performance of 2D geometry.
\textit{1) User interactions.}
Participants agreed that the user interactions of \toolName\ are valuable and smooth. They enjoyed the overall interactions during the study. Among them, participants gave highly positive feedback for the feature of the switching of qubit display orders.
\textit{2) Integration with the workflow.}
Participants liked how \toolName\ can seamlessly fit into their specific domain tasks.
Building upon the publicly-available platform, \toolName\ can benefit all quantum computing developers and researchers.
% Besides, we open-sourced the code to enable any learning and extension for the current visualization.
\textit{3) Performance of 2D geometry.}
Participants agreed that using 2D shapes was more readable and apparent than 3D approaches.
For instance,
% E6 pointed out that \toolName\ flattens the states and makes the perception of each characteristic more accurate than 3D views.


