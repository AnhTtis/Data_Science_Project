\section{Expert Interview}
\label{sec:user_study}

% Following the evaluation method proposed by Lee et al.~\cite{lee2022collabally}, w
We conducted in-depth expert interviews with 14 domain experts
to evaluate the performance of \toolName\ for given tasks.
% Overall, our method used for the user study is task-based usability testing.
% We expect to understand the strengths and limitations of \toolName\ for quantum state representation tasks.

\vspace{-0.3cm}
\subsection{Participants and Apparatus}

We invited 14 experts (2 females, 12 males) in quantum computing to join our well-designed interviews.
Participants ($age_{mean}= 33.07$, $age_{sd}= 6.16$) were with an average of over five years of quantum computing research experience.
Specifically, E4-8 are working on Quantum Machine Learning;
E9-12 current research direction is Quantum Error Modeling; 
E13-14 are studying Quantum Chemistry, while the other three participants study Quantum Compiler (E1), Quantum Systems (E2), and Quantum Simulator (E3), respectively. 
Meanwhile, E3 and E12 are from the same research institute in the U.S., while others work in different educational institutions in the U.S.
The above participants differ from the domain experts involved in the co-design process.
% To guarantee the findings collected in the expert interviews are general for common users, 
% and none of them has a background in Visualization or HCI.
All participants were asked to use a monitor with a resolution of 1920 $\times$ 1080 in advance.

% Our study was reviewed and approved by our institution's IRB.


% In our study, E1 and E2 used our baseline visualization approach, \textit{i.e.}, Bloch Sphere, while E3-14 used \toolName\ as the apparatus.
% Apart from the different apparatus, all participants were given the same tasks described in Section \ref{sec:tasks}.
% We first report the results of E1 and E2 using Bloch Sphere in Section \ref{sec:result} to report the current challenges and limitations in quantum state observation using Bloch Sphere.
% We then present the results from the remaining 12 participants (E3-14), who used our visualization approach \toolName.




\subsection{Procedure}


Before the formal study began, participants were asked questions about their demographics and research directions.
% We then .
We first introduced the goal and detailed procedures of the study, and then showcased an example to illustrate the interface and usage of \toolName.
% including both single-qubit and two-qubit state representation.
The above process lasted about 20 minutes.


% Participants then accessed the \toolName\ package on their laptops or desktops.
After the introduction, they were expected to accomplish the pre-defined tasks described below in Section \ref{sec:tasks}, which were about to observe the qubit state using the \toolName\ interface.
Note that the application to perform the qubit state visualization was in line with the research direction of each participant so that we can evaluate the generalizability of \toolName\ for different domain tasks in quantum computing (\textit{e.g.}, quantum algorithm, quantum machine learning).
% The research directions of each participant are shown in Table \ref{tab:1}.
We recorded and took notes on each task and participants' interaction processes.
The above process lasted about 45 minutes.


Upon task completion, all participants were then invited to participate in the post-study interview.
Each participant was encouraged to describe the advantages and limitations in a think-aloud manner.
% Specifically, they provided feedback regarding how \toolName\ can visually support the analysis of quantum state vector, the probability distribution of quantum states, and quantum entanglement visualization.
Meanwhile, they were also encouraged to describe the issues of Bloch Sphere regarding these three aspects and how these issues affect their routine tasks.
We invited every participant to rate the \toolName\ using a 7-Likert scale based on the questionnaire (Table \ref{tab:1}).
The above interview and rating procedures took 30 minutes.
\subsection{Tasks}
\label{sec:tasks}

We designed the tasks to mimic the everyday tasks for analyzing the quantum states.
To enhance the generalizability and make it not limited to one single application in quantum computing, we asked each participant to complete the tasks based on a typical example in their research directions (\textit{e.g.}, variational quantum circuit for Quantum Machine Learning, QEC~\cite{lidar2013quantum} algorithm for Quantum Error Modeling, or VQE~\cite{kandala2017hardware} algorithm for Quantum Chemistry).  
We proposed three types of tasks as follows:


 
\begin{table}[t]
  \caption{
The questionnaire provided during the expert interview based on the tasks in the expert interview, \textit{i.e.},
Task 1 (\textbf{Q1-2}), Task 2 (\textbf{Q3-4}), Task 3 (\textbf{Q5-6}), and an overall feedback of \toolName\ (\textbf{Q7-9}).
  }
  \label{tab:1}
\begin{tabular}{p{0.4cm}|p{7.5cm}}
\hline
Q1  & It is useful to show the quantum entanglement when observing quantum states.                         \\
Q2  & It is easy to identify the entangled states via the visually correlated semicircles.                 \\ \hline
Q3  & It is helpful to show the probability distribution.                              \\
Q4  & It is intuitive to show probability distribution via the semicircle area.                            \\ \hline
Q5  & It is informative to represent states via the state vectors.                                 \\
Q6  & It is easy to identify the state vectors via the line pairs within each semicircle.                  \\ \hline
% Q7  & Overall, \toolName\ can better visualize qubit states than Bloch Sphere. \\
Q7  & The user interactions in the interface are useful and smooth.                                        \\
Q8  & The design can be integrated into the workflow well.                                                 \\
Q9 & The 2D visual design is easy to view.                                                                \\ \hline
\end{tabular}
\end{table}


\begin{figure}[t]
\centering 
\includegraphics[width=\linewidth]{4.png}
\caption{The summary of the user feedback, which consists of the results of the three analytical tasks and the overall feedback.}
\label{fig:4}
\end{figure}



\textbf{Task 1: Explore the two-qubit entanglement.}
Participants were asked to identify all two-qubit states in \toolName, describe the components in each two-qubit state, analyze the single-qubit state in the two-qubit quantum system states
% (using the interaction of ``display order'' in the control panel)
, and describe how this feature benefits their quantum state analysis.


\textbf{Task 2: Observe the state probability distribution.}
Participants were asked to identify the probability of each quantum state and compare all quantum state probabilities via the corresponding semicircle area. User interactions displaying the exact probability amplitudes are allowed after the answer is given.


\textbf{Task 3: Identify the state vector.}
Participants were asked to identify the real and imaginary parts of each state vector via the line segments on the right triangles and describe how the length of the line segments affects the semicircle area.
% User interactions are only allowed if the line segment coincides with the diameter of the semicircle.
