\section{Introduction}

% \yong{Need to re-write it.}

Quantum computing has been undergoing impressive development in recent years
% over classical computing
~\cite{arute2019quantum,moller2017impact}. 
An increasing number of applications have been proven possible to achieve quantum speedups
%the power of quantum computing, 
such as optimization problems, machine learning, cryptography, and finance~\cite{hassija2020present}.
% An increasing number of quantum algorithms have been developed and shown a significant speedup
% over their best-known classical counterparts,
% for example, \textit{Shor's} algorithm for integer factorization~\cite{shor1999polynomial} and \textit{Grover's} algorithm for unstructured data searching~\cite{grover1996fast,kwiat2000grover}.
Meanwhile, along with the evolution of real quantum computers provided by many quantum vendors such as IBM, Rigetti, Honeywell, and IonQ~\cite{saki2021survey}, the last decade represents a significant milestone in the field of quantum computing~\cite{corcoles2019challenges, tacchino2020quantum, gomes2018quantum}.
% Thus, there is an expectation that near-term quantum devices will soon support practical applications.


For the implementation of quantum computing, a quantum bit (\textit{a.k.a.} \textit{qubit}) is the basic unit of any quantum program.
Generally, the state of a single qubit or multiple qubits is called \textit{quantum state}, such as state $\ket{0}$ for a single-qubit state and state $\ket{01}$ for a two-qubit system.
Compared to classical computing, today's quantum computing is driven by two basic quantum-specific properties called \textit{quantum superposition} and \textit{quantum entanglement}~\cite{rieffel2011quantum, tannu2019not}, which makes it possible to achieve the power of the quantum advantages~\cite{steane1998quantum}.
Specifically, quantum superposition (\textit{a.k.a., superposition}) indicates the unknown quantum states before measurement. Upon measurement of the qubit, it collapses to either the $\ket{0}$ or $\ket{1}$ state according to the deterministic probabilities of all possible states~\cite{gokhale2020optimized}.
% (\textit{e.g.}, $Pr(\ket{0})=0.6$ and $Pr(\ket{1})=0.4$).
For example, the probability of the measured result 0 is 0.4, while that of 1 is 0.6.
Also, quantum entanglement (\textit{a.k.a. entanglement}) supports the correlations between multiple single qubits such that manipulating one qubit can impact the state of the other qubit(s). Among them, two-qubit entanglement plays a critical role in many popular applications (\textit{e.g.}, Grover's Algorithm~\cite{hayward2008quantum}, Deutsch Algorithm~\cite{gulde2003implementation}, and Quantum Machine Learning ~\cite{SamMLSys22, KerstinNature2022}), especially in the near-term Noisy Intermediate-Scale Quantum~\cite{bharti2021noisy} (NISQ) era where the qubit number is severely limited.
% The aforementioned two properties (\textit{i.e.}, superposition and entanglement) are the fundamental ingredients to .




To effectively study and execute quantum programs, observing the quantum state is a basic requirement. A \modify{visualization technique} named \textit{Bloch Sphere}~\cite{bloch1946nuclear} was proposed to meet this need. Bloch Sphere leverages a point on the unit sphere to represent the \textit{amplitudes} of a pure single-qubit state. 
Due to the intuitiveness, Bloch Sphere is widely-accepted in the quantum computing community to visually show the quantum state of a single qubit ~\cite{wie2014bloch,131387, havel2004bloch} and has been incorporated into many popular toolkits of quantum computing, such as Qiskit by IBM Quantum~\cite{ibmq}. 
Despite its prevalence, several issues still exist when using Bloch Sphere.
First, 
% as mentioned earlier, entanglement is critical to the power of quantum computing.
it is impossible to use Bloch Sphere 
%to perform entanglement using multiple qubits, not even for two qubits
to visualize quantum states of more than one qubit~\cite{bardin2021microwaves, wie2014bloch}. 
Once two qubits are entangled, the visualization approach Bloch Sphere breaks down because the quantum state cannot be condensed into three dimensions in the same form~\cite{zable2020investigating}.
Second, although Bloch Sphere can visualize quantum states, it cannot support an intuitive view of the probability of all basis states. 
However, the probability is a critical property of quantum states as it is used to directly reflect superposition~\cite{wie2020two}.
Third, three-dimension visualizations, like Bloch Sphere, have been proven to perform worse than two-dimension counterparts when conducting precise measurements~\cite{tory2005visualization, FA20153D2D}.
% Third, \textit{Bloch Sphere} is a three-dimensional visualization that performs worse than its two-dimensional counterparts
%when performing precise measurements
The prior study has found that the three-dimensional visualization of the \textit{Bloch Sphere} introduces the possibility of visual occlusion~\cite{williams2021qcvis}, which makes the observation of quantum states rather inaccurate.

% the  three above issues significantly hinder the performance and accuracy of quantum state visualization.



Thus, a novel 2D visualization approach is urgently needed to address the above issues for all quantum computing users (\textit{e.g.}, novices and experts).
% However, there are challenges that exist to propose such a visualization, which are mainly from two perspectives: \textit{individual properties of quantum states}, and \textit{complex construction of quantum entanglement}.
It is a challenging task to visualize quantum states and the challenges mainly come from two aspects:
\textit{individual properties of quantum states}, and \textit{complex construction of quantum entanglement}.
First, according to our co-design process with domain experts, 
% it is essential to lower quantum computing users' learning curve and make them learn and use the visualization smoothly.
it is essential to visually identify the correlation among all quantum state components (\textit{e.g.}, the amplitudes of the state vectors and probabilities of quantum states) other than representing the above components individually.
Because this can make users aware of the hidden logic of the evolution of quantum states.
% Using the original quantum computing characteristics to 
% visually correlate all quantum state components (\textit{e.g.}, the amplitudes of the state vectors and probabilities of quantum states) 
% other than representing the above components individually 
% can be significant for domain users.
However, how to seamlessly link the individual quantum state components by appropriate visual channels is not a trivial task.
% On the one hand, it is difficult to encode state vectors and probability distributions simultaneously using a set of visually correlated visual elements. Building upon the quantum theorems of quantum states, it is crucial to highlight the relationship between the state vectors and probability distributions, which makes it more easy-to-understand and interpretable for domain users in quantum computing.
% On the other hand,
% \yong{You are always using ``one the one hand, on the other hand'' in the wrong way.}
Second,
how to visualize the entangled quantum states remains a challenging task.
Given that the two-qubit entanglement state is a specific type of two-qubit state, the two-qubit state representation is still non-transparent for quantum computing researchers and developers since the two-qubit state is not a simple accumulation of multiple single-qubit states.
Meanwhile, how to utilize the same scalable form of visualization to represent the entangled states of both single qubits and multiple qubits would be even more difficult.




To address the above issues, we propose \toolName, a novel \underline{\textbf{V}}isual d\underline{\textbf{E}}sig\underline{\textbf{N}} for quant\underline{\textbf{U}}m state repre\underline{\textbf{S}}entation. 
\toolName\ supports quantum state representation for not only single-qubit states but also two-qubit states, which is the basis for two-qubit entanglement representation. Meanwhile, \toolName\ can inform users of the probability for the quantum superposition without any manual calculation of the probability, leading to more efficient and smooth analysis of arbitrary quantum states.
Following a user-centered design process~\cite{munzner2009nested},
% and a method~\cite{lee2022collabally} of informing the design, 
we work closely with five domain experts in quantum computing. A co-design process is conducted to iteratively derive design requirements for observing quantum states, which also guide our subsequent visual design.
\toolName\ mainly consists of two coordinated visual components: a set of right triangles to visualize state vectors, and the circumscribed semicircles of these right triangles explicitly reflect the probability distribution of the corresponding quantum states. 
To evaluate the usefulness and effectiveness of \toolName, we present two case studies on single-qubit (\textit{i.e.}, Quantum Neural Network) and two-qubit (\textit{i.e.}, \textit{Grover's algorithm}). Moreover, we conduct an in-depth interview with 14 domain experts in quantum computing. The results show that \toolName\ can effectively facilitate the visual analysis of various quantum programs.
The major contributions of this paper can be summarized as follows:

\begin{itemize}
    \item We co-design with five domain experts to identify and summarize the design requirements for visually analyzing the quantum states, pinpointing the common challenges when performing visual analysis of quantum states.
    % qua by quantum computing users.
    
    \item We present a novel visualization, \toolName, to tackle both quantum superposition and quantum entanglement representation challenges for both single-qubit and two-qubit scenarios with multiple visually correlated 2D geometrical shapes.
    
    \item We conduct two case studies and user interviews with 14 domain experts to validate the effectiveness and usefulness of \toolName.
\end{itemize}


