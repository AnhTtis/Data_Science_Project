{
\section{Proofs}
\subsection{Proof of Theorem~\ref{thm:iterdiff}}
\label{appendix:proof_iterdiff}

We present and prove three theorems first and then use them to prove Theorem~\ref{thm:iterdiff}.
\begin{thm}
    \label{thm:iterdiff_bot}
    Provided $P$ is a distribution, that is, $\isfinaldist(P)$. 
    \begin{align*}
        &\left(\lfun_P^b\right)^0(\ufzero) = \ufzero \\ 
        &\left(\lfun_P^b\right)^1(\ufzero) = \lambda (s, s'). ~\rvprfunsym{\ibracket{\lnot b(s)}*\ibracket{s'=s}}
    \end{align*}
    If $n>1$, then
    \begin{align*}
        & \left(\lfun_P^b\right)^n(\ufzero) = \\ 
        & \lambda (s, s'). \rvprfunsym{
            \left(
            \begin{array}[]{@{}l} 
                \sum\limits_{i=1}^{n-1} \left(
                \begin{array}{@{}l}
                    \infsum s_{i-1}.  
                    \ibracket{b(s)}*{\prrvfunsym{P}}(s,s_{i-1})* \\
                    \left(
                    \begin{array}{@{}l}
                        \infsum s_{i-2}. \ibracket{b(s_{i-1})}*{\prrvfunsym{P}}(s_{i-1}, s_{i-2})* \\\left(
                        \begin{array}{@{}l}
                            \vdots * \\
                            \left(
                            \infsum s_0. \ibracket{b(s_1)}* {\prrvfunsym{P}}(s_1, s_0) * \ibracket{\lnot b(s_0)}*\ibracket{s'=s_0} 
                            \right)
                        \end{array}\right)
                    \end{array}
                    \right)
                \end{array}\right) \\
                + \left({\ibracket{\lnot b(s)}}\right) * {\ibracket{s'=s}}
            \end{array} 
            \right) 
        } 
    \end{align*}
\end{thm}

\begin{proof}
    We show below that $\left(\lfun_P^b\right)^n(\ufzero)$ for $n=0$ to $3$ satisfies the theorem.
    \begin{align*}
        &\left(\lfun_P^b\right)^0(\ufzero) = \lambda (s, s'). 0 = \ufzero \\
        &\left(\lfun_P^b\right)^1(\ufzero) \\
        = & \cmt{Defintion~\ref{def:lfun}} \\
        & \pcchoice{b}{\left(\pseq{P}{\ufzero}\right)}{\pskip}\\ 
        = & \cmt{Law~\ref{thm:lfun_altdef}} \\
        & \rvprfunsym{{\ibracket{b}}*\prrvfunsym{\left(\pseq{P}{\ufzero}\right)} + \ibracket{\lnot b} * {\ibracket{\II}}} \\
        = & \cmt{Theorem~\ref{thm:prog_seq_comp} Law~\ref{thm:pseq_right_zero}} \\
        & \rvprfunsym{\ibracket{\lnot b} * {\ibracket{\II}}} \\
        = &\cmt{Expand as a function form, use $s$ and $s'$ for initial and final observation states} \\
        &\cmt{Substitution and Definition~\ref{def:uskip}} \\
         &\lambda (s, s'). ~\rvprfunsym{\ibracket{\lnot b(s)}*\ibracket{s'=s}}\\
%
        &\left(\lfun_P^b\right)^2 (\ufzero) \\
        = &\cmt{$\lfun^2(\ufzero) = \lfun(\lfun^1(\ufzero))$ and Law~\ref{thm:lfun_altdef}} \\
        & \pcchoice{b}{\left(\pseq{P}{\rvprfunsym{\ibracket{\lnot b} * {\ibracket{\II}}}}\right)}{\pskip}\\ 
        = &\cmt{Theorem~\ref{thm:prog_cond_choice} Law~\ref{thm:cchoice_pchoice} and Theorem~\ref{thm:prob_prob_choice} Law~\ref{thm:pchoice_altdef} } \\
        & \rvprfunsym{\prrvfunsym{\left({\rvprfunsym{\ibracket{b}}}\right)}*\prrvfunsym{\left(\pseq{P}{\rvprfunsym{\ibracket{\lnot b} * {\ibracket{\II}}}}\right)} + \left(\rfone - \prrvfunsym{\left({\rvprfunsym{\ibracket{b}}}\right)}\right) * \prrvfunsym{\pskip}}  \\
        = &\cmt{Theorem~\ref{thm:prrvfun_inverse_ibracket}, Definition~\ref{def:prog_skip}} \\
        & \rvprfunsym{{\ibracket{b}}*\prrvfunsym{\left(\pseq{P}{\rvprfunsym{\ibracket{\lnot b} * {\ibracket{\II}}}}\right)} + \left(\rfone - {\ibracket{b}}\right) * {\ibracket{\II}}} \\
        = &\cmt{Definition~\ref{def:prog_seq}, Theorem~\ref{thm:ib} Law~\ref{thm:ib_neg}} \\
        & \rvprfunsym{{\ibracket{b}}*\prrvfunsym{\left(\rvprfunsym{\left({\infsum v_0 @ {{\prrvfunsym{P}}}[v_0/\vv'] * {{\prrvfunsym{\left(\rvprfunsym{\ibracket{\lnot b} * {\ibracket{\II}}}\right)}}}[v_0/\vv]}\right)}\right)} + \left({\ibracket{\lnot b}}\right) * {\ibracket{\II}}} \\
        = &\cmt{Theorem~\ref{thm:prrvfun_inverse_ibracket}} \\
        & \rvprfunsym{{\ibracket{b}}*\prrvfunsym{\left(\rvprfunsym{\left({\infsum v_0 @ {{\prrvfunsym{P}}}[v_0/\vv'] * {{{\left({\ibracket{\lnot b} * {\ibracket{\II}}}\right)}}}[v_0/\vv]}\right)}\right)} + \left({\ibracket{\lnot b}}\right) * {\ibracket{\II}}} \\
        = &\cmt{Expand as a function form, use $s$ and $s'$ for initial and final observation states} \\
        &\cmt{Substitution and Definition~\ref{def:uskip}} \\
        & \lambda (s, s'). \rvprfunsym{{\ibracket{b(s)}}*\prrvfunsym{\left(\rvprfunsym{\left({\infsum s_0 @ {{\prrvfunsym{P}}}(s,s_0) * {{{\left({\ibracket{\lnot b(s_0)} * {\ibracket{s'=s_0}}}\right)}}}}\right)}\right)} + \left({\ibracket{\lnot b(s)}}\right) * {\ibracket{s'=s}}} \\
        = &\cmt{Theorem~\ref{thm:prrvfun_inverse} where the proof of $\isprob$ is omitted} \\
        & \lambda (s, s'). \rvprfunsym{{\ibracket{b(s)}}*{{\left({\infsum s_0 @ {{\prrvfunsym{P}}}(s,s_0) * {{{\left({\ibracket{\lnot b(s_0)} * {\ibracket{s'=s_0}}}\right)}}}}\right)}} + \left({\ibracket{\lnot b(s)}}\right) * {\ibracket{s'=s}}} \\
        = &\cmt{Law~\ref{thm:summation_cmult_left} and proof of summable is omitted } \\
        & \lambda (s, s'). \rvprfunsym{{{\left({\infsum s_0 @ {\ibracket{b(s)}}*{{\prrvfunsym{P}}}(s,s_0) * {{{\left({\ibracket{\lnot b(s_0)} * {\ibracket{s'=s_0}}}\right)}}}}\right)}} + \left({\ibracket{\lnot b(s)}}\right) * {\ibracket{s'=s}}} \\
        % = &\cmt{Only one state $s_0=s'$ satisfies predicates within the summation} \\
        % & \lambda (s, s'). \rvprfunsym{{\ibracket{b(s)}}*\prrvfunsym{\left(\rvprfunsym{\left({{{\prrvfunsym{P}}}(s,s')*\left({\ibracket{\lnot b(s')}}\right)}\right)}\right)} + \left({\ibracket{\lnot b(s)}}\right) * {\ibracket{s'=s}}} \\
        % = &\cmt{Theorem~\ref{thm:prrvfun_inverse} where $\isprob\left({{{\prrvfunsym{P}}}(s,s')*\left({\ibracket{\lnot b(s')}}\right)}\right)$ because of $\isfinaldist(P)$} \\
        % & \lambda (s, s'). \rvprfunsym{{\ibracket{b(s)}}*\prrvfunsym{P}(s,s')*{\ibracket{\lnot b(s')}} + {\ibracket{\lnot b(s)}} * {\ibracket{s'=s}}} \\
    %= &\lambda (s, s'). \left(b(s) \cond \left(P(s, s') * \ibracket{\lnot b(s')}\right) \typecolon \ibracket{s'=s}\right)\\
    %= &\lambda (s, s'). \left(\lnot b(s)\cond \ibracket{s'=s} \typecolon \left(P(s, s') * \ibracket{\lnot b(s')}\right) \right)\\
%	= &\lambda (s, s'). ~\ibracket{b(s)}*\left(\sum s_0. P(s, s_0) * \ibracket{\lnot b(s_0)}*\ibracket{s'=s_0} \right) + \ibracket{\lnot b(s)}*\ibracket{s'=s} \\
%	= &\lambda (s, s'). ~\ibracket{b(s)}*\left(P(s, s') * \ibracket{\lnot b(s')}\right) + \ibracket{\lnot b(s)}*\ibracket{s'=s} \\
    %= &\lambda (s, s'). \left(\lnot b(s)\cond \ibracket{s'=s} \typecolon 0\right) + \left(b(s) \land \lnot b(P(s,s'))\cond P(s,s')\typecolon0 \right)\\
    %
        &\left(\lfun_P^b\right)^3 (\ufzero) \\
        = &\cmt{$\lfun^3(\ufzero) = \lfun(\lfun^2(\ufzero))$ and same as previous proof} \\
        & \lambda (s, s'). \rvprfunsym{
            \left(
            \begin{array}[]{@{}l}
                {\ibracket{b(s)}}*\prrvfunsym{\left(\rvprfunsym{\left(
                    \begin{array}[]{@{}l}
                        \infsum s_1 @ {{\prrvfunsym{P}}}(s,s_1) * \\
                        \prrvfunsym{\left(
                            \rvprfunsym{
                                \begin{array}[]{@{}l}
                                    {{\left({\infsum s_0 @ {\ibracket{b(s_1)}}*{{\prrvfunsym{P}}}(s_1,s_0) * {{{\left({\ibracket{\lnot b(s_0)} * {\ibracket{s'=s_0}}}\right)}}}}\right)}} \\
                        %{\ibracket{b(s_0)}}*\prrvfunsym{P}(s_0,s')*{\ibracket{\lnot b(s')}} \\
                                    + {\ibracket{\lnot b(s_1)}} * {\ibracket{s'=s_1}}
                                \end{array} }\right)}
                            \end{array} \right)}\right)} \\
                            + \left({\ibracket{\lnot b(s)}}\right) * {\ibracket{s'=s}}
                        \end{array}
                        \right)
                    } \\
                    = &\cmt{Theorem~\ref{thm:prrvfun_inverse} where the proof of $\isprob$ is omitted} \\
                    & \lambda (s, s'). \rvprfunsym{
                        \left(
                        \begin{array}[]{@{}l}
                            {\ibracket{b(s)}}*\prrvfunsym{\left(\rvprfunsym{\left(
                                \begin{array}[]{@{}l}
                                    \infsum s_1 @ {{\prrvfunsym{P}}}(s,s_1) * \\
                                    {\left( {
                                        \begin{array}[]{@{}l}
                                            {{\left({\infsum s_0 @ {\ibracket{b(s_1)}}*{{\prrvfunsym{P}}}(s_1,s_0) * {{{\left({\ibracket{\lnot b(s_0)} * {\ibracket{s'=s_0}}}\right)}}}}\right)}} \\
                        %{\ibracket{b(s_0)}}*\prrvfunsym{P}(s_0,s')*{\ibracket{\lnot b(s')}} \\
                                            + {\ibracket{\lnot b(s_1)}} * {\ibracket{s'=s_1}}
                                        \end{array} }\right)}
                                    \end{array} \right)}\right)} \\
                                    + \left({\ibracket{\lnot b(s)}}\right) * {\ibracket{s'=s}}
                                \end{array}
                                \right)
                            } \\
                            = &\cmt{Law~\ref{thm:summation_add} and proofs of summable omitted } \\
                            & \lambda (s, s'). \rvprfunsym{
                                \left(
                                \begin{array}[]{@{}l}
                                    {\ibracket{b(s)}}*\prrvfunsym{\left(\rvprfunsym{\left(
                                        \begin{array}[]{@{}l}
                                            \infsum s_1 @ 
                                            {\prrvfunsym{P}}(s,s_1) * \\
                                            \;{{\left({\infsum s_0 @ {\ibracket{b(s_1)}}*{{\prrvfunsym{P}}}(s_1,s_0) * {{{\left({\ibracket{\lnot b(s_0)} * {\ibracket{s'=s_0}}}\right)}}}}\right)}} 
                                            \\
                                            + \infsum s_1 @ {\prrvfunsym{P}}(s,s_1) * {\ibracket{\lnot b(s_1)}} * {\ibracket{s'=s_1}}
                                        \end{array} \right)}\right)} \\
                                        + \left({\ibracket{\lnot b(s)}}\right) * {\ibracket{s'=s}}
                                    \end{array}
                                    \right)
                                } \\
                                = &\cmt{Theorem~\ref{thm:prrvfun_inverse} where the proof of $\isprob$ is omitted} \\
                                & \lambda (s, s'). \rvprfunsym{
                                    \left(
                                    \begin{array}[]{@{}l} 
                                        {\ibracket{b(s)}} *\infsum s_1 @ {\prrvfunsym{P}}(s,s_1) * \\
                                        \;{{\left({\infsum s_0 @ {\ibracket{b(s_1)}}*{{\prrvfunsym{P}}}(s_1,s_0) * {{{\left({\ibracket{\lnot b(s_0)} * {\ibracket{s'=s_0}}}\right)}}}}\right)}} \\
                                        {\ibracket{b(s)}} *\infsum s_1 @ {\prrvfunsym{P}}(s,s_1) * {\ibracket{\lnot b(s_1)}} * {\ibracket{s'=s_1}} \\ 
                                        + \left({\ibracket{\lnot b(s)}}\right) * {\ibracket{s'=s}}
                                    \end{array} 
                                    \right) 
                                } \\
                                = &\cmt{Law~\ref{thm:summation_cmult_left} and proof of summable is omitted } \\
                                & \lambda (s, s'). \rvprfunsym{
                                    \left(
                                    \begin{array}[]{@{}l} 
                                        \infsum s_1 @ {\ibracket{b(s)}} *{\prrvfunsym{P}}(s,s_1) * \\
                                        \;{{\left({\infsum s_0 @ {\ibracket{b(s_1)}}*{{\prrvfunsym{P}}}(s_1,s_0) * {{{\left({\ibracket{\lnot b(s_0)} * {\ibracket{s'=s_0}}}\right)}}}}\right)}} \\
                                        \infsum s_1 @ {\ibracket{b(s)}} *{\prrvfunsym{P}}(s,s_1) * {\ibracket{\lnot b(s_1)}} * {\ibracket{s'=s_1}} \\ 
                                        + \left({\ibracket{\lnot b(s)}}\right) * {\ibracket{s'=s}}
                                    \end{array} 
                                    \right) 
                                } \\
                                = &\cmt{Rewrite } \\
                                & \lambda (s, s'). \rvprfunsym{
                                    \left(
                                    \begin{array}[]{@{}l} 
                                        \sum\limits_{i=1}^{2} \left(
                                        \begin{array}{@{}l}
                                            \infsum s_{i-1}.  
                                            \ibracket{b(s)}*{\prrvfunsym{P}}(s,s_{i-1})* \\
                                            \left(
                                            \begin{array}{@{}l}
                                                \infsum s_{i-2}. \ibracket{b(s_{i-1})}*{\prrvfunsym{P}}(s_{i-1}, s_{i-2})* \\\left(
                                                \begin{array}{@{}l}
                                                    \vdots * \\
                                                    \left(
                                                    \infsum s_0. \ibracket{b(s_1)}* {\prrvfunsym{P}}(s_1, s_0) * \ibracket{\lnot b(s_0)}*\ibracket{s'=s_0} 
                                                    \right)
                                                \end{array}\right)
                                            \end{array}
                                            \right)
                                        \end{array}\right) \\
                                        + \left({\ibracket{\lnot b(s)}}\right) * {\ibracket{s'=s}}
                                    \end{array} 
                                    \right) 
                                } 
                            \end{align*}
%
Assume $(\lfun_P^b)^n(\ufzero)$ satisfies the theorem, we show below that $(\lfun_P^b)^{n+1}(\ufzero)$ also satisfies the theorem.
%
                            \begin{align*}
                                &\left(\lfun_P^b\right)^{n+1} (\ufzero) \\
                                = &\cmt{Assumption, $\lfun^{n+1}(\ufzero) = \lfun(\lfun^n(\ufzero))$, and same as previous proof} \\
                                & \lambda (s, s'). \rvprfunsym{
                                    \left(
                                    \begin{array}[]{@{}l}
                                        {\ibracket{b(s)}}*\\
                                        \prrvfunsym{\left(\rvprfunsym{\left(
                                            \begin{array}[]{@{}l}
                                                \infsum s_{n-1} @ {{\prrvfunsym{P}}}(s,s_{n-1}) * \\
                                                { \left(
                                                    \begin{array}[]{@{}l} 
                                                        \sum\limits_{i=1}^{n-1} \left(
                                                        \begin{array}{@{}l}
                                                            \infsum s_{i-1}. 
                                                            \ibracket{b(s_{n-1})}*{\prrvfunsym{P}}(s_{n-1},s_{i-1})* \\
                                                            \left(
                                                            \begin{array}{@{}l}
                                                                \infsum s_{i-2}. \ibracket{b(s_{i-1})}*{\prrvfunsym{P}}(s_{i-1}, s_{i-2})* \\\left(
                                                                \begin{array}{@{}l}
                                                                    \vdots * \\
                                                                    \left(
                                                                    \infsum s_0. \ibracket{b(s_1)}* {\prrvfunsym{P}}(s_1, s_0) * \ibracket{\lnot b(s_0)}*\ibracket{s'=s_0} 
                                                                    \right)
                                                                \end{array}\right)
                                                            \end{array}
                                                            \right)
                                                        \end{array}\right) \\
                                                        + \left({\ibracket{\lnot b(s_{n-1})}}\right) * {\ibracket{s'=s_{n-1}}}
                                                    \end{array} 
                                                    \right) 
                                                }
                                            \end{array} \right)}\right)} \\
                                            + \left({\ibracket{\lnot b(s)}}\right) * {\ibracket{s'=s}}
                                        \end{array}
                                        \right)
                                    } \\
                                    = &\cmt{ Expand finite summation } \\
                                    & \lambda (s, s'). \rvprfunsym{
                                        \left(
                                        \begin{array}[]{@{}l}
                                            {\ibracket{b(s)}}*\\
                                            \prrvfunsym{\left(\rvprfunsym{\left(
                                                \begin{array}[]{@{}l}
                                                    \infsum s_{n-1} @ {{\prrvfunsym{P}}}(s,s_{n-1}) * \\
                                                    { \left(
                                                        \begin{array}[]{@{}l} 
                                                            \left(
                                                            \begin{array}{@{}l}
                                                                \infsum s_{{n-1}-1}.  
                                                                \ibracket{b(s_{n-1})}*{\prrvfunsym{p}}(s_{n-1},s_{{n-1}-1})* \\
                                                                \left(
                                                                \begin{array}{@{}l}
                                                                    \infsum s_{{n-1}-2}. \ibracket{b(s_{{n-1}-1})}*{\prrvfunsym{p}}(s_{{n-1}-1}, s_{{n-1}-2})* \\\left(
                                                                    \begin{array}{@{}l}
                                                                        \vdots * \\
                                                                        \left(
                                                                        \infsum s_0. \ibracket{b(s_1)}* {\prrvfunsym{p}}(s_1, s_0) * \ibracket{\lnot b(s_0)}*\ibracket{s'=s_0} 
                                                                        \right)
                                                                    \end{array}\right)
                                                                \end{array}
                                                                \right)
                                                            \end{array}\right) \\
                                                            + \cdots + \\
                                                            \left(
                                                            \infsum s_0. \ibracket{b(s_{n-1})}* {\prrvfunsym{p}}(s_{n-1}, s_0) * \ibracket{\lnot b(s_0)}*\ibracket{s'=s_0} 
                                                            \right) \\
                                                            + \left({\ibracket{\lnot b(s_{n-1})}}\right) * {\ibracket{s'=s_{n-1}}}
                                                        \end{array} 
                                                        \right) 
                                                    }
                                                \end{array} \right)}\right)} \\
                                                + \left({\ibracket{\lnot b(s)}}\right) * {\ibracket{s'=s}}
                                            \end{array}
                                            \right)
                                        } \\
                                        = &\cmt{Law~\ref{thm:summation_add} and proofs of summable omitted } \\
                                        & \lambda (s, s'). \rvprfunsym{
                                            \left(
                                            \begin{array}[]{@{}l}
                                                {\ibracket{b(s)}}*\\
                                                \prrvfunsym{\left(\rvprfunsym{\left(
                                                    \begin{array}[]{@{}l}
                                                        { \left(
                                                            \begin{array}[]{@{}l} 
                                                                \left(
                                                                \begin{array}{@{}l}
                                                                    \infsum s_{n-1} @ {{\prrvfunsym{P}}}(s,s_{n-1}) * \\ 
                                                                    \infsum s_{{n-1}-1}.  
                                                                    \ibracket{b(s_{n-1})}*{\prrvfunsym{p}}(s_{n-1},s_{{n-1}-1})* \\
                                                                    \left(
                                                                    \begin{array}{@{}l}
                                                                        \infsum s_{{n-1}-2}. \ibracket{b(s_{{n-1}-1})}*{\prrvfunsym{p}}(s_{{n-1}-1}, s_{{n-1}-2})* \\\left(
                                                                        \begin{array}{@{}l}
                                                                            \vdots * \\
                                                                            \left(
                                                                            \infsum s_0. \ibracket{b(s_1)}* {\prrvfunsym{p}}(s_1, s_0) * \ibracket{\lnot b(s_0)}*\ibracket{s'=s_0} 
                                                                            \right)
                                                                        \end{array}\right)
                                                                    \end{array}
                                                                    \right)
                                                                \end{array}\right) \\
                                                                + \cdots + \\
                                                                \infsum s_{n-1} @ {{\prrvfunsym{P}}}(s,s_{n-1}) * \\
                                                                \infsum s_0. \ibracket{b(s_{n-1})}* {\prrvfunsym{p}}(s_{n-1}, s_0) * \ibracket{\lnot b(s_0)}*\ibracket{s'=s_0} \\ 
                                                                + 
                                                                \infsum s_{n-1} @ {{\prrvfunsym{P}}}(s,s_{n-1}) * 
                                                                \left({\ibracket{\lnot b(s_{n-1})}}\right) * {\ibracket{s'=s_{n-1}}}
                                                            \end{array} 
                                                            \right) 
                                                        }
                                                    \end{array} \right)}\right)} \\
                                                    + \left({\ibracket{\lnot b(s)}}\right) * {\ibracket{s'=s}}
                                                \end{array}
                                                \right)
                                            } \\
%
                    %
                                            = &\cmt{Law~\ref{thm:summation_cmult_left} and combine summation } \\
                                            & \lambda (s, s'). \rvprfunsym{
                                                \left(
                                                \begin{array}[]{@{}l} 
                                                    \sum\limits_{i=1}^{n} \left(
                                                    \begin{array}{@{}l}
                                                        \infsum s_{i-1}.  
                                                        \ibracket{b(s)}*{\prrvfunsym{P}}(s,s_{i-1})* \\
                                                        \left(
                                                        \begin{array}{@{}l}
                                                            \infsum s_{i-2}. \ibracket{b(s_{i-1})}*{\prrvfunsym{P}}(s_{i-1}, s_{i-2})* \\\left(
                                                            \begin{array}{@{}l}
                                                                \vdots * \\
                                                                \left(
                                                                \infsum s_0. \ibracket{b(s_1)}* {\prrvfunsym{P}}(s_1, s_0) * \ibracket{\lnot b(s_0)}*\ibracket{s'=s_0} 
                                                                \right)
                                                            \end{array}\right)
                                                        \end{array}
                                                        \right)
                                                    \end{array}\right) \\
                                                    + \left({\ibracket{\lnot b(s)}}\right) * {\ibracket{s'=s}}
                                                \end{array} 
                                                \right) 
                                            } \\
                                        \end{align*}
                                        This concludes the proof.
                                    \end{proof}

\begin{thm}
    \label{thm:iterdiff_top}
    Provided $P$ is a distribution, that is, $\isfinaldist(P)$. 
    \begin{align*}
        &\left(\lfun_P^b\right)^0(\ufone) = \ufone \\ 
        &\left(\lfun_P^b\right)^1(\ufone) = \lambda (s, s'). ~\rvprfunsym{\infsum s_0. \ibracket{b(s)}* {\prrvfunsym{P}}(s, s_0) + \ibracket{\lnot b(s)}*\ibracket{s'=s}}
    \end{align*}
    If $n>1$, then
    \begin{align*}
        & \left(\lfun_P^b\right)^n(\ufone) = \\ 
        & \lambda (s, s'). \rvprfunsym{
            \left(
            \begin{array}[]{@{}l} 
                \left(
                \begin{array}{@{}l}
                    \infsum s_{n-1}.  
                    \ibracket{b(s)}*{\prrvfunsym{P}}(s,s_{n-1})* \\
                    \left(
                    \begin{array}{@{}l}
                        \infsum s_{n-2}. \ibracket{b(s_{n-1})}*{\prrvfunsym{P}}(s_{n-1}, s_{n-2})* \\\left(
                        \begin{array}{@{}l}
                            \vdots * \\
                            \left(
                            \infsum s_0. \ibracket{b(s_1)}* {\prrvfunsym{P}}(s_1, s_0) 
                            \right)
                        \end{array}\right)
                    \end{array}
                    \right)
                \end{array}\right) + \hfill(\textnormal{\color{red} diff})\\
                \sum\limits_{i=1}^{n-1} \left(
                \begin{array}{@{}l}
                    \infsum s_{i-1}.  
                    \ibracket{b(s)}*{\prrvfunsym{P}}(s,s_{i-1})* \\
                    \left(
                    \begin{array}{@{}l}
                        \infsum s_{i-2}. \ibracket{b(s_{i-1})}*{\prrvfunsym{P}}(s_{i-1}, s_{i-2})* \\\left(
                        \begin{array}{@{}l}
                            \vdots * \\
                            \left(
                            \infsum s_0. \ibracket{b(s_1)}* {\prrvfunsym{P}}(s_1, s_0) * \ibracket{\lnot b(s_0)}*\ibracket{s'=s_0} 
                            \right)
                        \end{array}\right)
                    \end{array}
                    \right)
                \end{array}\right) \\
                + \left({\ibracket{\lnot b(s)}}\right) * {\ibracket{s'=s}}
            \end{array} 
            \right) 
        } 
    \end{align*}
    We note that the part marked with (\textnormal{\color{red} diff}) is the only difference of this $\left(\lfun_P^b\right)^n(\ufone)$ from $\left(\lfun_P^b\right)^n(\ufzero)$. 
\end{thm}

\begin{proof}
    We show below that $\left(\lfun_P^b\right)^n(\ufone)$ for $n=0$ to $3$ satisfies the theorem.
    \begin{align*}
        &\left(\lfun_P^b\right)^0(\ufone) = \lambda (s, s'). 0 = \ufone \\
        &\left(\lfun_P^b\right)^1(\ufone) \\
        = & \cmt{Defintion~\ref{def:lfun}} \\
        & \pcchoice{b}{\left(\pseq{P}{\ufone}\right)}{\pskip}\\ 
        = & \cmt{Law~\ref{thm:lfun_altdef}} \\
        & \rvprfunsym{{\ibracket{b}}*\prrvfunsym{\left(\pseq{P}{\ufone}\right)} + \ibracket{\lnot b} * {\ibracket{\II}}} \\
        = & \cmt{$\pseq{P}{\ufone}= \rvprfunsym{{\infsum v_0 @ \prrvfunsym{P}[v_0/\vv']}}$, see the proof of Theorem~\ref{thm:prog_seq_comp} Law~\ref{thm:pseq_one} } \\
        & \rvprfunsym{{\ibracket{b}}*\prrvfunsym{\left(\rvprfunsym{{\infsum v_0 @ \prrvfunsym{P}[v_0/\vv']}}\right)} + \ibracket{\lnot b} * {\ibracket{\II}}} \\
        = &\cmt{Expand as a function form, use $s$ and $s'$ for initial and final observation states} \\
        &\cmt{Substitution and Definition~\ref{def:uskip}, Theorem~\ref{thm:prrvfun_inverse}, and Law~\ref{thm:summation_cmult_left} } \\
         &\lambda (s, s'). ~\rvprfunsym{\infsum s_0. \ibracket{b(s)}* {\prrvfunsym{P}}(s, s_0) + \ibracket{\lnot b(s)}*\ibracket{s'=s}}\\
%
        &\left(\lfun_P^b\right)^2 (\ufone) \\
        = &\cmt{$\lfun^2(\ufone) = \lfun(\lfun^1(\ufone))$ and same as previous proof} \\
        & \lambda (s, s'). \rvprfunsym{
            \left(
            \begin{array}[]{@{}l}
                {\ibracket{b(s)}}*\prrvfunsym{\left(\rvprfunsym{\left(
                    \begin{array}[]{@{}l}
                        \infsum s_1 @ {{\prrvfunsym{P}}}(s,s_1) * \\
                        \prrvfunsym{\left(
                            \rvprfunsym{
                                \begin{array}[]{@{}l}
                                    \infsum s_0. \ibracket{b(s_1)}* {\prrvfunsym{P}}(s_1, s_0) 
                                    + {\ibracket{\lnot b(s_1)}} * {\ibracket{s'=s_1}}
                                \end{array} }\right)}
                            \end{array} \right)}\right)} \\
                            + \left({\ibracket{\lnot b(s)}}\right) * {\ibracket{s'=s}}
                        \end{array}
                        \right)
                    } \\
        = &\cmt{Theorem~\ref{thm:prrvfun_inverse} where the proof of $\isprob$ is omitted} \\
          &\cmt{Law~\ref{thm:summation_cmult_left} and proof of summable is omitted } \\
        & \lambda (s, s'). \rvprfunsym{
            \left(
            \begin{array}[]{@{}l}
                \infsum s_1 @ \ibracket{b(s)}*{{\prrvfunsym{P}}}(s,s_1) * 
                \left( \infsum s_0. \ibracket{b(s_1)}* {\prrvfunsym{P}}(s_1, s_0) \right) + \\
                \infsum s_1 @ \ibracket{b(s)}*{{\prrvfunsym{P}}}(s,s_1) * 
                \left( {\ibracket{\lnot b(s_1)}} * {\ibracket{s'=s_1}} \right) + \\
                {\ibracket{\lnot b(s)}} * {\ibracket{s'=s}}
            \end{array}
            \right)
        } 
\end{align*}
%
Assume $(\lfun_P^b)^n(\ufone)$ satisfies the theorem, we show below that $(\lfun_P^b)^{n+1}(\ufone)$ also satisfies the theorem.
%
\begin{align*}
    &\left(\lfun_P^b\right)^{n+1} (\ufone) \\
    = &\cmt{Assumption, $\lfun^{n+1}(\ufone) = \lfun(\lfun^n(\ufone))$, and same as previous proof} \\
    & \lambda (s, s'). \rvprfunsym{
        \left(
        \begin{array}[]{@{}l}
            {\ibracket{b(s)}}*\\
            \prrvfunsym{\left(\rvprfunsym{\left(
                \begin{array}[]{@{}l}
                    \infsum s_{n} @ {{\prrvfunsym{P}}}(s,s_{n}) * \\
            { \left(
            \begin{array}[]{@{}l} 
                \left(
                \begin{array}{@{}l}
                    \infsum s_{n-1}.  
                    \ibracket{b(s_n)}*{\prrvfunsym{P}}(s_n,s_{n-1})* \\
                    \left(
                    \begin{array}{@{}l}
                        \infsum s_{n-2}. \ibracket{b(s_{n-1})}*{\prrvfunsym{P}}(s_{n-1}, s_{n-2})* \\\left(
                        \begin{array}{@{}l}
                            \vdots * \\
                            \left(
                            \infsum s_0. \ibracket{b(s_1)}* {\prrvfunsym{P}}(s_1, s_0) 
                            \right)
                        \end{array}\right)
                    \end{array}
                    \right)
                \end{array}\right) + \\
                \sum\limits_{i=1}^{n-1} \left(
                \begin{array}{@{}l}
                    \infsum s_{i-1}.  
                    \ibracket{b(s_n)}*{\prrvfunsym{P}}(s_n,s_{i-1})* \\
                    \left(
                    \begin{array}{@{}l}
                        \infsum s_{i-2}. \ibracket{b(s_{i-1})}*{\prrvfunsym{P}}(s_{i-1}, s_{i-2})* \\\left(
                        \begin{array}{@{}l}
                            \vdots * \\
                            \left(
                            \infsum s_0. \ibracket{b(s_1)}* {\prrvfunsym{P}}(s_1, s_0) * \ibracket{\lnot b(s_0)}*\ibracket{s'=s_0} 
                            \right)
                        \end{array}\right)
                    \end{array}
                    \right)
                \end{array}\right) \\
                + \left({\ibracket{\lnot b(s)}}\right) * {\ibracket{s'=s}}
            \end{array} 
            \right) 
        }
                \end{array} \right)}\right)} \\
                + \left({\ibracket{\lnot b(s)}}\right) * {\ibracket{s'=s}}
            \end{array}
            \right)
        } \\
        = &\cmt{ Expand finite summation, Law~\ref{thm:summation_cmult_left}, same as previous proof } \\
        & \lambda (s, s'). \rvprfunsym{
            \left(
            \begin{array}[]{@{}l} 
                \left(
                \begin{array}{@{}l}
                    \infsum s_{n}.  
                    \ibracket{b(s)}*{\prrvfunsym{P}}(s,s_{n})* \\
                    \left(
                    \begin{array}{@{}l}
                        \infsum s_{n-1}. \ibracket{b(s_{n})}*{\prrvfunsym{P}}(s_{n}, s_{n-1})* \\\left(
                        \begin{array}{@{}l}
                            \vdots * \\
                            \left(
                            \infsum s_0. \ibracket{b(s_1)}* {\prrvfunsym{P}}(s_1, s_0) 
                            \right)
                        \end{array}\right)
                    \end{array}
                    \right)
                \end{array}\right) + \\
                \sum\limits_{i=1}^{n} \left(
                \begin{array}{@{}l}
                    \infsum s_{i-1}.  
                    \ibracket{b(s)}*{\prrvfunsym{P}}(s,s_{i-1})* \\
                    \left(
                    \begin{array}{@{}l}
                        \infsum s_{i-2}. \ibracket{b(s_{i-1})}*{\prrvfunsym{P}}(s_{i-1}, s_{i-2})* \\\left(
                        \begin{array}{@{}l}
                            \vdots * \\
                            \left(
                            \infsum s_0. \ibracket{b(s_1)}* {\prrvfunsym{P}}(s_1, s_0) * \ibracket{\lnot b(s_0)}*\ibracket{s'=s_0} 
                            \right)
                        \end{array}\right)
                    \end{array}
                    \right)
                \end{array}\right) \\
                + \left({\ibracket{\lnot b(s)}}\right) * {\ibracket{s'=s}}
            \end{array} 
            \right) 
        } 
\end{align*}
This concludes the proof.
\end{proof}

\begin{thm}
    \label{thm:iterdiff_eq}
    \begin{align*}
       \forall n:\nat | n \geq 1 \bullet  
\iterdiff(n, b, P) = \lambda (s, s'). \rvprfunsym{
\left(
                \begin{array}{@{}l}
                    \infsum s_{n-1}.  
                    \ibracket{b(s)}*{\prrvfunsym{P}}(s,s_{n-1})* \\
                    \left(
                    \begin{array}{@{}l}
                        \infsum s_{n-2}. \ibracket{b(s_{n-1})}*{\prrvfunsym{P}}(s_{n-1}, s_{n-2})* \\\left(
                        \begin{array}{@{}l}
                            \vdots * \\
                            \left(
                            \infsum s_0. \ibracket{b(s_1)}* {\prrvfunsym{P}}(s_1, s_0) 
                            \right)
                        \end{array}\right)
                    \end{array}
                    \right)
                \end{array}
            \right)}
    \end{align*}
\end{thm}

\begin{proof}
   If $n=1$, then  
   \begin{align*}
      & \iterdiff(1, b, P) = \\ 
     = &\cmt{Defintions~\ref{def:iterdiff} and $\lfundiff$} \\
     & \pcchoice{b}{\left(\pseq{P}{\ufone}\right)}{\ufzero} \\
     = &\cmt{Theorem~\ref{thm:prog_cond_choice} Law~\ref{thm:cchoice_pchoice} and Theorem~\ref{thm:prob_prob_choice} Law~\ref{thm:pchoice_altdef}, and Theorem~\ref{thm:prrvfun_inverse} } \\
     & \rvprfunsym{{\ibracket{b}}*\prrvfunsym{\left(\pseq{P}{\ufone}\right)}}\\
     = & \cmt{$\pseq{P}{\ufone}= \rvprfunsym{{\infsum v_0 @ \prrvfunsym{P}[v_0/\vv']}}$ } \\
       &\cmt{Expand as a function form, use $s$ and $s'$ for initial and final observation states} \\
       &\lambda (s, s').~\rvprfunsym{\infsum s_0. \ibracket{b(s)}* {\prrvfunsym{P}}(s, s_0)}
   \end{align*}
Assume $\iterdiff(n, b, P)$ satisfies the theorem, we show below that $\iterdiff(n+1, b, P)$ also satisfies the theorem.
   \begin{align*}
      & \iterdiff(n+1, b, P) = \\ 
     = &\cmt{Defintions~\ref{def:iterdiff} and $\lfundiff$} \\
     & \pcchoice{b}{\left(\pseq{P}{\iterdiff(n, b, P)}\right)}{\ufzero} \\
     = &\cmt{Theorem~\ref{thm:prog_cond_choice} Law~\ref{thm:cchoice_pchoice} and Theorem~\ref{thm:prob_prob_choice} Law~\ref{thm:pchoice_altdef}, and Theorem~\ref{thm:prrvfun_inverse} } \\
     & \rvprfunsym{{\ibracket{b}}*\prrvfunsym{\left(\pseq{P}{\iterdiff(n, b, P)}\right)}}\\
     = &\cmt{Definition~\ref{def:prog_seq}} \\
       & \rvprfunsym{{\ibracket{b}}*\prrvfunsym{\left(\rvprfunsym{\left({\infsum v_0 @ {{\prrvfunsym{P}}}[v_0/\vv'] * {{\prrvfunsym{\left(\rvprfunsym{\iterdiff(n, b, P)}\right)}}}[v_0/\vv]}\right)}\right)}} \\
     = &\cmt{Theorem~\ref{thm:prrvfun_inverse} where the proof of $\isprob$ is omitted} \\
       & \rvprfunsym{{\ibracket{b}}*{{\left({\infsum v_0 @ {{\prrvfunsym{P}}}[v_0/\vv'] * {{{\left({\iterdiff(n, b, P)}\right)}}}[v_0/\vv]}\right)}}} \\
     = &\cmt{Law~\ref{thm:summation_cmult_left} and proof of summable is omitted } \\
       & \rvprfunsym{{\left( {{\infsum v_0 @ {\ibracket{b}}*{{\prrvfunsym{P}}}[v_0/\vv'] * {{{\left({\iterdiff(n, b, P)}\right)}}}[v_0/\vv]}}\right)}} \\
     = &\cmt{Assumption, expand as a function form, use $s$ and $s'$ for initial and final observation states} \\
      &\lambda (s, s'). \rvprfunsym{ \left(
                \begin{array}{@{}l}
                    \infsum s_{n}.  
                    \ibracket{b(s)}*{\prrvfunsym{P}}(s,s_{n})* \\
                    \left(
                    \begin{array}{@{}l}
                        \infsum s_{n-1}. \ibracket{b(s_{n})}*{\prrvfunsym{P}}(s_{n}, s_{n-1})* \\\left(
                        \begin{array}{@{}l}
                            \vdots * \\
                            \left(
                            \infsum s_0. \ibracket{b(s_1)}* {\prrvfunsym{P}}(s_1, s_0) 
                            \right)
                        \end{array}\right)
                    \end{array}
                    \right)
                \end{array}
            \right)}
   \end{align*}
   This concludes the proof of Theorem~\ref{thm:iterdiff_eq}.
\end{proof}

We now show the proof of Theorem~\ref{thm:iterdiff}:
    $\forall n:\nat \bullet {\lfunbp}^n(\ufone) - {\lfunbp}^n(\ufzero) = \iterdiff(n, b, P)$
\begin{proof}
    For $n=0$, 
    \begin{align*}
        & {\lfunbp}^0(\ufone) - {\lfunbp}^0(\ufzero) \\ 
        = &\cmt{Theorems~\ref{thm:iterdiff_bot} and \ref{thm:iterdiff_top}} \\
        & \ufone - \ufzero \\
        = &\cmt{Theorem~\ref{thm:top_bot}} \\
        & \ufone \\
        = &\cmt{Definition~\ref{def:iterdiff}} \\
        & \iterdiff(0, b, P)
    \end{align*}
    For $n=1$, 
    \begin{align*}
        & {\lfunbp}^1(\ufone) - {\lfunbp}^1(\ufzero) \\ 
        = &\cmt{Theorems~\ref{thm:iterdiff_bot} and \ref{thm:iterdiff_top}} \\
        & \lambda (s, s'). ~\rvprfunsym{\infsum s_0. \ibracket{b(s)}* {\prrvfunsym{P}}(s, s_0) + \ibracket{\lnot b(s)}*\ibracket{s'=s}} - \lambda (s, s').~\rvprfunsym{\ibracket{\lnot b(s)}*\ibracket{s'=s}} \\
        = &\cmt{Definition~\ref{def:urf_pointwise}} \\
        & \lambda (s, s'). \left(\rvprfunsym{\infsum s_0. \ibracket{b(s)}* {\prrvfunsym{P}}(s, s_0) + \ibracket{\lnot b(s)}*\ibracket{s'=s}} - \rvprfunsym{\ibracket{\lnot b(s)}*\ibracket{s'=s}}\right) \\
        = &\cmt{Definition~\ref{def:bounded_plus_minus}} \\
        & \lambda (s, s'). \ru{\left(\umax\left(0, \ur{\left(\rvprfunsym{\infsum s_0. \ibracket{b(s)}* {\prrvfunsym{P}}(s, s_0) + \ibracket{\lnot b(s)}*\ibracket{s'=s}}\right)} - \ur{\left(\rvprfunsym{\ibracket{\lnot b(s)}*\ibracket{s'=s}}\right)}\right)\right)}\\
        = &\cmt{Theorem~\ref{thm:prrvfun_inverse} where the proof of $\isprob$ is omitted} \\
        & \lambda (s, s'). \ru{\left(\umax\left(0, {\left({\infsum s_0. \ibracket{b(s)}* {\prrvfunsym{P}}(s, s_0) + \ibracket{\lnot b(s)}*\ibracket{s'=s}}\right)} - {\left({\ibracket{\lnot b(s)}*\ibracket{s'=s}}\right)}\right)\right)}\\
        = &\cmt{$\forall s \bullet \infsum s_0. \ibracket{b(s)}* {\prrvfunsym{P}}(s, s_0) \geq 0$ because $P$ is a distribution, Theorem~\ref{thm:final_distribtion}, and Definition~\ref{def:isprob} } \\
        & \lambda (s, s').~\ru{\infsum s_0. \ibracket{b(s)}* {\prrvfunsym{P}}(s, s_0)}\\
        = &\cmt{Theorem~\ref{thm:iterdiff_eq}} \\
        & \iterdiff(1, b, P)
    \end{align*}
    For $n>1$, we assume ${\lfunbp}^n(\ufone) - {\lfunbp}^n(\ufzero)= \iterdiff(n, b, P)$, then 
    \begin{align*}
        & {\lfunbp}^{n+1}(\ufone) - {\lfunbp}^{n+1}(\ufzero) \\ 
        = &\cmt{Theorems~\ref{thm:iterdiff_bot} and \ref{thm:iterdiff_top}, and same as previous proof} \\
        & \lambda (s, s'). \rvprfunsym{
                \left(
                \begin{array}{@{}l}
                    \infsum s_{n}.  
                    \ibracket{b(s)}*{\prrvfunsym{P}}(s,s_{n})* \\
                    \left(
                    \begin{array}{@{}l}
                        \infsum s_{n-1}. \ibracket{b(s_{n})}*{\prrvfunsym{P}}(s_{n}, s_{n-1})* \\\left(
                        \begin{array}{@{}l}
                            \vdots * \\
                            \left(
                            \infsum s_0. \ibracket{b(s_1)}* {\prrvfunsym{P}}(s_1, s_0) 
                            \right)
                        \end{array}\right)
                    \end{array}
                    \right)
                \end{array}\right) 
        } \\ 
        = &\cmt{Theorem~\ref{thm:iterdiff_eq}} \\
        & \iterdiff(n+1, b, P)
    \end{align*}
   This concludes the proof of Theorem~\ref{thm:iterdiff}.
\end{proof}
}

{
\subsection{Proof of Theorem~\ref{thm:cflip_iterdiff_0}}
\begin{proof}
    \begin{align*}
        & \lambda n @ \prrvfunsym{\iterdiff\left(n, c=tl, cflip\right)}(s,s') \\
        = &\cmt{Theorem~\ref{thm:iterdiff_eq} where $b=(c=tl)$ and $P=cflip$} \\
        & \lambda n @ \prrvfunsym{ \left(
\rvprfunsym{\lambda (s, s').  \left(
                \begin{array}{@{}l}
                    \infsum s_{n-1}.  
                    \ibracket{b(s)}*{\prrvfunsym{P}}(s,s_{n-1})* \\
                    \left(
                    \begin{array}{@{}l}
                        \infsum s_{n-2}. \ibracket{b(s_{n-1})}*{\prrvfunsym{P}}(s_{n-1}, s_{n-2})* \\\left(
                        \begin{array}{@{}l}
                            \vdots * \\
                            \left(
                            \infsum s_0. \ibracket{b(s_1)}* {\prrvfunsym{P}}(s_1, s_0) 
                            \right)
                        \end{array}\right)
                    \end{array}
                    \right)
                \end{array}
            \right)}
    \right)}(s,s') \\
    = &\cmt{Definition~\ref{def:coin_flip} where $cstate$ only has one variable $c$ and so $s$ is replaced by $c$} \\
    &\cmt{${\prrvfunsym{cflip}}={{1/2}*{\ibracket{c' = hd}} + {1/2} * {\ibracket{c' = tl}}}$ according to Law~\ref{thm:cflip_altdef} and Theorem~\ref{thm:prrvfun_inverse_ibracket}} \\
        & \lambda n @ \prrvfunsym{ \left(
\rvprfunsym{\lambda (c, c').  \left(
                \begin{array}{@{}l}
                    \infsum s_{n-1}.  
                    \ibracket{c=tl}*({{1/2}*{\ibracket{s_{n-1} = hd}} + {1/2} * {\ibracket{s_{n-1} = tl}}})* \\
                    \left(
                    \begin{array}{@{}l}
                        \infsum s_{n-2}. 
                        \ibracket{s_{n-1}=tl}*({{1/2}*{\ibracket{s_{n-2} = hd}} + {1/2} * {\ibracket{s_{n-2} = tl}}})* \\
                        \left(
                        \begin{array}{@{}l}
                            \vdots * \\
                            \left(
                            \infsum s_0. 
                                \ibracket{s_1=tl}*({{1/2}*{\ibracket{s_0 = hd}} + {1/2} * {\ibracket{s_0 = tl}}})
                            \right)
                        \end{array}\right)
                    \end{array}
                    \right)
                \end{array}
            \right)}
    \right)}(c,c') \\
    = &\cmt{ Every summation variable such as $s_{n-1}$ only when it is equal to $tl$, then $\ibracket{s_{n-1}=tl}=1$. } \\
     &\cmt { Otherwise, it is 0 and the whole summation is also 0 because $a*\cdots*0*\cdots*b=0$.} \\
     &\cmt { All $\infsum$ are removed because of only one state $tl$ satisfying $s_i=tl$ } \\
     &\cmt { All $\ibracket{s_i=hd} = 0$ } \\
        & \lambda n @ \prrvfunsym{ \left(
\rvprfunsym{\lambda (c, c').  \left(
                \begin{array}{@{}l}
                    \ibracket{c=tl}*({{1/2}*1})* \\
                    \left(
                    \begin{array}{@{}l}
                        1*({{1/2} * 1})* \\
                        \left(
                        \begin{array}{@{}l}
                            \vdots * \\
                            \left(
                                1*({{1/2} * 1})
                            \right)
                        \end{array}\right)
                    \end{array}
                    \right)
                \end{array}
            \right)}
    \right)}(c,c') \\
    = &\cmt{ Rewrite } \\
    & \lambda n @ \prrvfunsym{ \left( \rvprfunsym{\lambda (c, c'). \ibracket{c=tl}*(1/2)^{n}} \right)}(c,c') \\
    = &\cmt{Theorem~\ref{thm:prrvfun_inverse} where $\isprob\left({\lambda (c, c'). \ibracket{c=tl}*(1/2)^{n}}\right)$} \\
    & \lambda n @ \ibracket{c=tl}*(1/2)^{n} 
\end{align*}
So if $c=hd$, then 
\begin{align*}
\left(\lambda n @ \prrvfunsym{\iterdiff\left(n, c=tl, cflip\right)}(s,s') = \lambda n @ 0\right)
\end{align*}
Otherwise, 
\begin{align*}
\left(\lambda n @ \prrvfunsym{\iterdiff\left(n, c=tl, cflip\right)}(s,s') = \lambda n @ (1/2)^n\right)
\end{align*}
We conclude that $\forall (c,c'):cstate\cross cstate$ such that 
\begin{align*}
\left(\lambda n @ \prrvfunsym{\iterdiff\left(n, c=tl, cflip\right)}(c,c')\right) \tendsto 0
\end{align*}
according to Definition~\ref{def:tendsto}.
\end{proof}
}

{
\subsection{The necessity of $\finstatesasc\left(\lambda n @ \iter\left(n, b, P, \ufzero\right) \right)$}
\label{appendix:necessity_finstatesasc}
We show below to establish the continuity Theorem~\ref{thm:continuity_lfun_bot}, $\finstatesasc\left(\lambda n @ \iter\left(n, b, P, \ufzero\right) \right)$ is required. Here we omit the details about type conversion to make the discussion clearer.

We aim to prove a continuity theorem below and then our proof later shows that without the premise, we cannot prove such theorem.
\begin{thm} \label{thm:continuity}
	If the final state of $P$ is a distribution, then $\lfunbp(X)$ is continuous. That is, %Let $S=\langle S_n | n \in \nat\rangle$ is an (countable infinite) increasing sequence and bound above (so has a supremum). 
    for an non-empty countable increasing chain $S_0 \leq S_1 \leq S_2 \leq ... $ of type $[s]prfun$ and bound above (so has a supremum), then
	\begin{align*}
		\lfunbp\left(\thnsup n @S_n\right) = %\thsup{} \lfunbp(S) \left(\defs 
        \thnsup n @ \lfunbp(S_n) \tag*{($\lfunbp$ continuous)} \label{thm:Fbp_continuous}
	\end{align*}
\end{thm}

\begin{proof}
    We define  
	\begin{align*}
        & F_1(X) \defs \pcchoice{b}{X}{\pskip} \\ 
        & F_2(X) \defs \pseq{P}{X}
	\end{align*}
    So $\lfunbp(X) \defs~F_1(F_2(X))$. 
    According to \cite[Lemma 5.3]{Nielson2007}, if both $F_1$ and $F_2$ are continuous, then $\lfunbp(X)$ is also continuous. It is trivial to show that $F_1$ is continuous, so our goal is to prove $F_2$ is continuous. That is, 
	\begin{align*}
		F_2\left(\thnsup n @S_n\right) = \thnsup n @ F_2(S_n) \tag*{($F_2$ continuous)} \label{thm:F2_continuous}
	\end{align*}

    Because $S_n$ is an increasing chain and bound above, according to the monotone sequence theorem, the limit of $S_n$ is just its supremum. That is,
	\begin{align*}
        \forall (s,s') @ \left(\lambda n @ S_n (s,s')\right)  \tendsto \left(\thnsup n @S_n\right)(s,s')
		%\tag*{($S_n$ supremum as limit)} \label{thm:Sn_limit}
	\end{align*}
where $(s,s')$ denotes the initial and final observations. 
    According to Definition~\ref{def:tendsto}, 
%According to \ref{thm:Sn_limit}, we can get  
    \begin{align*}
        &\forall (s,s') @ \forall \epsilon:\real > 0 \bullet \exists M:\nat \bullet \forall l \geq M \bullet \left({\left(\thnsup n @S_n\right)(s,s')} - {S_l(s,s')}\right)< \epsilon \tag*{($S_n$ supremum as limit)} \label{thm:Sn_limit_sup}
    \end{align*}

    According to Theorem~\ref{thm:prog_seq_comp} Law~\ref{thm:pseq_mono}, $F_2$ is monotonic. It is trivial to show that $F_2(S_n)$ is also an increasing chain and bound above. According to the monotone sequence theorem,
	\begin{align*}
        \forall (s,s') @ \left(\lambda n @ F_2(S_n) (s,s')\right) \tendsto \left(\thnsup n @F_2(S_n)\right)(s,s')
		\tag*{($F_2(S_n)$ supremum as limit)} \label{thm:F2_Sn_limit}
	\end{align*}
    Therefore, according to the unique sequence limit theorem (if exists), to prove \ref{thm:F2_continuous}, we need to prove that 
	\begin{align*}
        \forall (s,s') @ \left(\lambda n @ F_2(S_n) (s,s')\right) \tendsto F_2\left(\thnsup n @S_n\right) (s,s')
		\tag*{($F_2\left(\thnsup n @S_n\right)$ as limit)} \label{thm:F2_Sup_Sn_limit}
	\end{align*}
    According to Definition~\ref{def:tendsto}, this is equal to prove
    \begin{align*}
        &\forall (s,s') @ \forall \varepsilon:\real > 0 \bullet \exists N:\nat \bullet \forall l \geq N \bullet |F_2(S_l)(s,s') - F_2\left(\thnsup n @S_n\right)(s,s')| < \varepsilon \\
= & \cmt{$F_2$ is monotonic and $S_l \leq \left(\thnsup n @S_n\right)$, so $F_2(S_l)(s,s') \leq F_2\left(\thnsup n @S_n\right)(s,s')$} \\
        &\forall (s,s') @ \forall \varepsilon:\real > 0 \bullet \exists N:\nat \bullet \forall l \geq N \bullet F_2\left(\thnsup n @S_n\right)(s,s') - F_2(S_l)(s,s') < \varepsilon
    \end{align*}

    We can rewrite the non-quantifier part above to 
    \begin{align*}
        & |F_2(S_l)(s,s') - F_2\left(\thnsup n @S_n\right)(s,s')| < \varepsilon \\
        = & \cmt{$F_2$ is monotonic and $S_l \leq \left(\thnsup n @S_n\right)$, so $F_2(S_l)(s,s') \leq F_2\left(\thnsup n @S_n\right)(s,s')$} \\
        &F_2\left(\thnsup n @S_n\right)(s,s') - F_2(S_l)(s,s') < \varepsilon \\
        = & \cmt{Definitions of $F_2$ and $\pseq{}{}$~\ref{def:prog_seq}} \\
        &{\infsum s_0 @ {P}(s, s_0) * {\left(\thnsup n @S_n\right)(s_0,s')}} - {\infsum s_0 @ {P}(s, s_0) * {S_l(s_0,s')}} < \varepsilon \\
        = & \cmt{Theorem~\ref{thm:summation} Law~\ref{thm:summation_minus} and proof of summable is omitted } \\
          &{\infsum s_0 @ {P}(s, s_0) * \left({\left(\thnsup n @S_n\right)(s_0,s')} - {S_l(s_0,s')}\right)} < \varepsilon
    \end{align*}

    So our goal is to prove 
    \begin{align*}
        &\forall (s,s') @ \forall \varepsilon:\real > 0 \bullet \exists N:\nat \bullet \forall l \geq N \bullet \left({\infsum s_0 @ {P}(s, s_0) * \left({\left(\thnsup n @S_n\right)(s_0,s')} - {S_l(s_0,s')}\right)}\right) < \varepsilon
    \end{align*}
    The key step in proving the goal above is to supply a witness for $N$. Based on \ref{thm:Sn_limit_sup}, we can choose a $N$ to make $\left({\left(\thnsup n @S_n\right)(s_0,s')} - {S_l(s_0,s')}\right)$ any small (say $\epsilon(s_0)$), but this cannot guarantees 
    \begin{align*}
    \left({\infsum s_0 @ {P}(s, s_0) * \epsilon(s_0)}\right) < \varepsilon
    \end{align*}
    because this is an infinite sum. In other words, the question is to find a $N$ such that this summation converges to a value less than any number $\varepsilon$. The approach we use in this paper is to assume $\finstatesasc\left(\lambda n @ \iter\left(n, b, P, \ufzero\right) \right)$, and then we can construct such $N$.

    We omit further details of this proof.

%    New proof strategy:
%    1. prove $\left({\lambda s_0 @ {P}(s, s_0) * \left({\left(\thnsup n @S_n\right)(s_0,s')} - {S_l(s_0,s')}\right)}\right)$ is summable
%    2. Choose any N and let $\left({\infsum s_0 @ {P}(s, s_0) * \left({\left(\thnsup n @S_n\right)(s_0,s')} - {S_N(s_0,s')}\right)}\right)=y$
%    3. So $y>0$ (prove) 
%    \begin{align*}
%        & \varepsilon/(2*y) * \left({\infsum s_0 @ {P}(s, s_0) * \left({\left(\thnsup n @S_n\right)(s_0,s')} - {S_N(s_0,s')}\right)}\right) < \varepsilon \\
%        = & \cmt{Constant } \\
%        & \left({\infsum s_0 @ \varepsilon/(2*y) * {P}(s, s_0)*\left({\left(\thnsup n @S_n\right)(s_0,s')} - {S_N(s_0,s')}\right)}\right) < \varepsilon \\
%        = & \cmt{$\forall s_0 @ \exists m @ \left(
%            \begin{array}{@{}l}
%            \left({P}(s, s_0)*{\left({\left(\thnsup n @S_n\right)(s_0,s')} - {S_M(s_0,s')}\right)}\right) \\
%        < \left({\varepsilon/(2*y) * \left({P}(s, s_0)*{\left(\thnsup n @S_n\right)(s_0,s')} - {S_N(s_0,s')}\right)}\right)
%            \end{array}
%    \right)$} \\
%      = & \cmt{choose $M=\thsup{} \left\{s_0:S @ \left(\varepsilon m @ \left(
%            \begin{array}{@{}l}
%            \left({P}(s, s_0)*{\left({\left(\thnsup n @S_n\right)(s_0,s')} - {S_m(s_0,s')}\right)}\right) \\
%        < \left({\varepsilon/(2*y) * \left({P}(s, s_0)*{\left(\thnsup n @S_n\right)(s_0,s')} - {S_N(s_0,s')}\right)}\right)
%            \end{array}
%        \right)\right) \right\}$} \\
%        & \left({\infsum s_0 @ {P}(s, s_0) * \varepsilon/(2*y) * \left({\left(\thnsup n @S_n\right)(s_0,s')} - {S_M(s_0,s')}\right)}\right) < \varepsilon \\
%    \end{align*}
\end{proof}
}
