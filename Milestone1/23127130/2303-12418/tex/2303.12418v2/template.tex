\documentclass[lettersize, journal]{IEEEtran}
% \documentclass[letterpaper, 10 pt, conference]{ieeeconf}
% \linespread{0.98}
% \overrideIEEEmargins
\IEEEoverridecommandlockouts
\usepackage{makecell}
\usepackage{multirow}
\usepackage{booktabs}
\usepackage{threeparttable}

\usepackage{colortbl}
\usepackage{hhline}
\usepackage{color}
% \usepackage[flushleft]{threeparttable}
% \usepackage{rotating}
\usepackage{amsmath,amsfonts}
\usepackage[dvipsnames, svgnames, x11names]{xcolor}

\usepackage{algorithm}
\usepackage[english]{babel}
\usepackage{vcell}
\usepackage{tablefootnote}
\usepackage[utf8]{inputenc}

\usepackage{caption}

% \usepackage{algorithm}
\usepackage{algorithmic}
\renewcommand{\algorithmicrequire}{\textbf{Input:}}
\renewcommand{\algorithmicensure}{\textbf{Calculations:}}

% \usepackage{algorithmicx}
% \usepackage[noend]{algpseudocode}

\makeatletter
\newcommand{\removelatexerror}{\let\@latex@error\@gobble}
\makeatother

\usepackage{array}
\renewcommand\arraystretch{1.3} % double the height of table

\usepackage[caption=false,font=normalsize,labelfont=sf,textfont=sf]{subfig}
\usepackage{textcomp}
\usepackage{stfloats}
% \usepackage{url}
\usepackage{verbatim}
\usepackage{graphicx}
\usepackage{epsfig}
\usepackage{hyperref}
% \DeclareUnicodeCharacter{}
\DeclareUnicodeCharacter{2061}{}
\newcommand{\degree}{^\circ}%self-defined command to draw degree
% \newcommand{\removelatexerror}{\let\@latex@error\@gobble}
% \usepackage{cite}
\usepackage[backend=bibtex,style=ieee,sorting=none,isbn=false,url=false,doi=false]{biblatex}
\addbibresource{References.bib}
% \hyphenation{op-tical net-works semi-conduc-tor IEEE-Xplore}
% updated with editorial comments 8/9/2021

% \title{\LARGE \bf
% Theoretical Model Construction of Deformation-Force for Soft Grippers \Large{Part II: Displacement Control Based Intrinsic Force Sensing}}

\title{Theoretical Model Construction of Deformation-Force for Soft Grippers Part II: Displacement Control Based Intrinsic Force Sensing}

\author{Huixu Dong, Ziyi Zheng, Haotian Guo, Sihao Yang, Chen Qiu, Jiansheng Dai, I-Ming Chen, ~\IEEEmembership{Fellow,~IEEE}
\thanks{\noindent Huixu Dong, Ziyi Zheng, Haotian Guo, Sihao Yang are with Robot Perception
and Grasp Laboratory(Grasp Lab), Zhejiang University, Hangzhou 310058,
China (e-mail: huixudong@zju.edu.cn). I-Ming Chen is with Robotics Research Center, Nanyang Technological University, Singapore 639798. Chen Qiu is with Maider Medical Industry Equipment Co., Itd, China 317607. Jiansheng Dai is with Shenzhen Key Laboratory of Biomimetic Robotics and Intelligent Systems, SUSTech Institute of Robotics, Southern University of Science and Technology, Shenzhen, 518055, China, and Centre for Robotics Research, Department of Engineering, King's College London Strand, London WC2R 2LS, UK.}}
% \thanks{$^{2}$Bernard D. Researcheris with the Department of Electrical Engineering, Wright State University,
%         Dayton, OH 45435, USA
%         {\tt\small b.d.researcher@ieee.org}}%
% }% <-this % stops a space
% \thanks{Project page: \url{https://github.com/JasonLvernex}}


\begin{document}
\maketitle
\begin{abstract}
Force-aware grasping is an essential capability for most robots in practical applications. Especially for compliant grippers, such as Fin-Ray gripper, it still remains challenging to build a bidirectional mathematical model that mutually maps the shape deformation and contact force. Part I of this article has constructed the force-displacement relationship for design optimization through the co-rotational theory. In Part II, we further devise a displacement-force mathematical model, enabling the compliant gripper to precisely estimate contact force from deformations sensor-free. The presented displacement-force model elaborately investigates contact forces and provides force feedback for a force control system of a gripper, where deformation appears as displacements in contact points.
Afterwards, simulation experiments are conducted to evaluate the performance of the proposed model through comparisons with the finite-element analysis (FEA) in Ansys. Simulation results reveal that the proposed model accurately estimates contact force, with an average error of around 3\% and 4\% for single or multiple node cases, respectively, regardless of various design parameters (Part I of this article is released in Arxiv\footnote{Part I: \href{https://arxiv.org/pdf/2303.12987.pdf}{https://arxiv.org/pdf/2303.12987.pdf}}). 
\end{abstract}

\textbf{\textit{Index Terms---}Compliant gripper, Bidirectional modeling, Force sensing, Robotic grasp, Grasp performance }

\section{Introduction}

\IEEEPARstart{C}{ompliant} grasping plays an important role in robots' practical applications \cite{ref1,ref2,ref59}.  Despite significant progress in the field of compliant grasping, accurately sensing the forces involved in such grasps remains challenging as evidenced by existing literature \cite{ref2012grasping,refdong2022gsg,refdong2022construction}. Particularly, inadequate grasping forces may negatively influence grasping stability, even resulting in task failure. Meanwhile, excessive forces may cause damage to objects, especially fragile or soft ones. Therefore, it is well worth achieving the interaction forces between grippers and objects. As introduced in Part I, in this article we choose the Fin-Ray gripper as an example (see Fig. \ref{fig:1}) for its generality to other compliant grippers.

\begin{figure}[htbp]
\centerline{\includegraphics[width=0.85\columnwidth]{Fig/1.png}}
\vspace{-2mm}
\caption{\small{Festo’s soft gripper based on Fin-Ray® effect and its equivalent fin-ray structure, \textcolor{red}{retrived from \href{https://www.festo.com/}{Festo Gmbh}.}  The green circles represent the crossbeams; the yellow circles indicate the joints of the back side and the red circles are the joints of the contact side.}}
\vspace{-10pt}
\vspace{-2mm}
\label{fig:1}
\end{figure}
% In particular, the emphasis on applications of constructing deformation-force theoretical model of compliant grippers is summarized as two following points. First, the mapping models between contact force resultant, distribution and the finger deformation are employed to assess the grasp stability for optimizing the design parameters of a compliant gripper [3, 7]. Second, it is well-known that the contact force control based on the the model of mapping the finger deformation to contact force supervises grasping force, preventing excessive and insufficient contact forces from damaging objects and causing an unstable grasp [6, 8]. 


% Referring to the substantial literature available regarding compliant grasps, this work considers a type of compliant gripper, such as a popular fin-ray gripper [4, 6, 9]( a commercialized product Festo DAHS) that commonly appear in industrial or logistic scenarios, as an example to construct an accurate universal mathematical model of the finger deformation and contact force for compliant grasp (see Fig.1). Here we provide the main reason why the fin-ray gripper is selected as an example for modeling and analyzing the deformation-force relationship. A compliant gripper as aforementioned can passively fit its profile to an object’s shape against the object with the triangular or quadrilateral (fin-ray) or circle or even arbitrary geometry structure consisting of crossbeams and two sides. The individual contacting joint formed by a crossbeam and finger structure is independently rotational. As it appears, the deformations mostly occur at the contact side, the back, and the contacting joints of a compliant finger, as shown in Fig.1. In terms of modeling the compliant grasp, a soft/continuum gripper can be commonly considered a fin-ray gripper with numerous crossbeams, which performs a wide variety of DOF (Degree of Freedom) articulations with the low stiffness material for readily deforming. Moreover, a compliant rigid gripper with multiple joints can be regarded as a fin-ray gripper that has several crossbeams constructed by high stiffness material. A fin-ray gripper is the migration from a rigid gripper to a soft one. In particular, fin-ray grippers have an obvious advantage on strong force grasps compared with other purely soft robotic ones due to skeletons with a certain stiffness, which results in common applications for these grippers in industries. On the contrary, fin-ray grippers present more adaptive performance than compliant rigid grippers in grasping soft, fragile and delicate objects owing to their superior compliant characteristics. \textcolor{red}{In addition, there are various designs of fin-ray grippers. This lacks a universal theoretical model to optimize their designs.}
% Referring to the substantial literature available regarding compliant grasps, this work considers a type of compliant gripper, such as a popular fin-ray gripper [4, 6, 9](a commercialized product Festo DAHS) that commonly appear in industrial or logistic scenarios, as an example to construct an accurate universal mathematical model of the finger deformation and contact force for compliant grasp (see Fig.1). Fin-ray grippers present more adaptive performance than compliant rigid grippers in grasping soft, fragile and delicate objects owing to their superior compliant characteristics. Practically, fin-ray grippers are soft and triangular with crossbeams connecting the front and the back side of fin-rays. 

For a compliant grasp, it poses a considerable challenge to obtaining the contact forces. Some force sensors that rely heavily on force-sensitive resistors, strain gauges, and electrodes were developed and integrated into compliant grippers \cite{ref34,ref35}. However, they are not suitable for when large strain at finger is a need. A few new sensors appeared for obtaining contact forces of soft grippers based on the characteristics of capacitance or optical structures \cite{ref36,ref37}, but the complexity of compliant grippers increases accordingly.
Besides, there exist  good solutions for achieving contact forces through theoretical models of deformations and contact forces \cite{ref4,ref6,ref38,ref39}. Such a model revealing the deformation-force mathematical relationships is established by virtual work \cite{ref38}, or piecewise constant curvature and Cosserat-rod theory \cite{ref39,ref46}, or Newton–Euler iterations \cite{ref47}, or two recent methods based on machine learning \cite{ref6} and improved pseudo-rigid body model with the virtual work \cite{ref4}. The virtual work method \cite{ref38, ref43}, piecewise constant curvature and Cosserat-rod theory \cite{ref39, ref46, ref50}, Newton–Euler iterations \cite{ref47} typically focus on continuum robots with long thin structures while applications of such modelling methods on compliant grippers have not been reported, that is, outside of the field of continuum robotics, these methods are uncommon. Meanwhile, these robotic bodies are considered as a curve, whose cross-sections have the same rotations; however, the contact side and back of a compliant finger present significantly different deformations with non-uniformity joints.


\textcolor{Green}{Xu et al. \cite{ref6} introduced a neural network method to achieve the mapping relationships from the deformations to contact forces, skillfully avoiding building the corresponding model of compliant grasp. However, the supervised learning method yields two drawbacks, resulting in limited applications in real scenarios. In particular, the performance of the learning-based model highly relevant to the collected data, both quality and quantity. With the change in the geometrical parameters of the gripper, the model will be ineffective and new data needs to be collected, with repeated effort required.} In the meantime, they insert many rigid rods to modify the gripper structure, which improves gripper structure stiffness for the static model to be linear. By contrast, this also indicates that method is only suitable for linear-structured fin-ray grippers. Shan et al. \cite{ref4} proposed a theoretical model for deformation-force via an improved pseudo-rigid body model with the virtual work. However, this method's stiffness matrix assumes identical transformation frames or minor changes, the model cannot handle a scene where compliant fingers cause significant deformation. Shan's method can just tackle symmetrical-structure-compliant grippers, which is limited to broad applications. Moreover, these two methods are constrained to compliant grippers with fin-ray effects and rely on numerous assumptions for the deformation-force relationship.


To fill the above research gap, we seek to further construct the deformation-force model, providing crucial insights into design optimization and force control of fin-ray grippers. First, the proposed methods, based on the co-rotational theory, can map contact forces from deformations. The displacement-force modeling estimates contact forces, which potentially facilitates the force control of a gripper. Second, a series of comparison simulations based on the proposed model and the standard FEA in Ansys is implemented to systematically evaluate the performance of the proposed model.
% In particular, a family of fin-ray grippers with various design parameters are designed to investigate the optimal design of a fin-ray gripper. Then, when single-node or multiple-node contact occurs, simulation experiments are carried out to evaluate the performance of the proposed model in estimating contact forces. \textcolor{cyan}{Finally, after determining the design parameters optimized by the proposed method, we build a fin-ray gripper. Furthermore, we experimentally demonstrate that the proposed model can accurately predict both the contact forces and the overall grasp force.}

% We highlight the novelties of our work. Our core contribution is constructing the deformation-force mathematical models, giving the first generalizable deformation-force solution for compliant rigid and soft grippers with various structures and materials under limited assumptions, even when grippers deform significantly. The second novelty involves modeling and analyzing the mathematical relationships between finger deformation and contact forces using co-rotational theory. We are the first to establish the deformation-to-force mapping model, providing a critical insight into grasping force control by sensing contact forces in simulations. 

We highlight the novelties of this work. \textbf{Firstly}, this article, for the first time, explores the theoretical construction of the deformation-force relationship based on physical models through co-rational concepts. \textbf{Secondly}, the individual influence of the co-rotational modeling parameters, namely mesh density and node radius factor, on the model's accuracy is systematically investigated. 
% the proposed mathematical model can be easily generalized to other compliant grippers, such as compliant rigid and soft grippers with few assumptions made. 
% \textbf{Thirdly}, the comparison with FEA in simulation is innovatively adopted as \textcolor{Purple}{FEA has poor real-time force control and cannot apply to the displacement control of our gripper.} 
This article will not only endow compliant grippers with a powerful mathematical tool that provides force perception, laying the groundwork for further research into theoretical model construction, but inspire more systematical and optimized gripper design. 

% \section{RELATED WORK}\label{RELATED WORK}
% Generally, the current compliant grippers can be divided into two categories such as some traditional multiple-rigid-link grippers that can perform compliant grasps and soft grippers.  

% \subsection{Compliant Rigid Grippers}
% The grasp behaviors of rigid grippers as aforementioned mainly rely on joint kinematic pairs while the link deformation is ignored. To achieve the capability of compliant grasp, this type of grippers is usually designed as under-actuated ones by combining multiple rigid links and joints with only one actuator. Several universal simple rigid grippers have been proposed. A famous commercial gripper Robotiq, which is an improved version of Sarah hand [10], utilizes the parallelogram linkage to realize the compliant grasp. Dong et al. [3] proposed a tendon-driven compliant gripper that provides two classic grasping modes including pinch and enveloping grasps. An artful mechanism design in the Velo gripper allows pinch and enveloping modes of compliant grasp to be transited smoothly [11]. A compliant gripper consisting of two 4-bar links achieved both the pinching and power grasp as well as in-hand manipulation capability based on the various properties of passive deformable joints [12]. Such rigid grippers indeed benefit from simple mechanism and control, excellent reliability; however, they often suffer from weak compliance to unstructured objects. 

% To enhance compliance, the primary exploration on attempting to replicate the compliant grasp of the human hands steers the design and control of rigid robotic hands from two-finger ones to multiple-finger grippers [13]. A few representatives are considered as examples, such as the Pisa/IIT hand [14], the DLR-Hand [15], the SDM Hand [16] and Wang’s Hand [17]. The Pisa/IIT Hand is controlled by one single motor to drive more than 10 joints [14]. Dollar et al. [18] adopted polymeric materials to construct the rigid structure of the SDM Hand, allowing it to be superior robustness and compliance. The highly underactuated hand proposed by Wang et al. [17] includes three fingers with three joints and also perceives the contact forces by equipped tactile sensors. Although the superior compliant capability of the humanoid robotic hand is witnessed, their practical applications are limited owing to the considerably complex mechanism and control, low reliability as well as high cost compared with other rigid grippers.

% \subsection{Soft Gripper}
% In fact, soft grippers belonging to the second category of the compliant gripper are a natural extension of the multiple-link rigid gripper stated in prior works since a soft finger can be considered as a rigid finger with infinite or amounts of DOFs, which generates continuous deformations while holding objects. Thus, compared to rigid grippers, soft grippers have superior compliant performance in grasping objects with arbitrary shapes. The investigations of soft grippers mainly emphasize on the gripper design, executive structure design and actuation strategies [19]. The existing soft grippers have many morphologies, such as one-finger [20], two-finger [21], three-finger [22], and multiple-finger grippers [23], to adapt to objects for increasing the grasp stability. Moreover, soft anthropomorphic hands could implement dexterous manipulations by imitating the behaviors of a human’s hand, which provides a  promising solution for a robotic hand to reproduce human hand operations in the real world [18, 24]. Soft robots make use of various structures, such as inflatable rubber pockets [25], pneumatic channels embedded in elastomers [18, 26], hydrostatic skeletons [6, 20], material jamming [27] and vacuum suction [28], to realize structural deformations based on the intrinsic mechanical properties of soft materials. These compliant structures are driven by a variety of actuation strategies, such as deforming the chambers by pneumatic [24, 29] or hydraulic [21, 30] actuations, imposing high voltages on dielectric elastomers [31], using motors to control tendons [32] and implementing different temperatures to shape-memory actuators [33]. Therefore, various mechanisms and control methods can be easily applied to soft grippers; however, it is very difficult to realize precision control implementations for such soft compliant structures. 

% \subsection{Models of the Deformation-Force}
% To optimize the design of a soft gripper and implement force control, the contact force needs to be achieved. Thus, the important issue is contact force sensing. It poses a considerable challenge for obtaining the contact forces. Some force sensors based on force-sensitive resistors, strain gauges and electrodes were developed and integrated into compliant grippers [34, 35]. However, in terms of stretchable deformations, it is not available for them to be attached to large-strain fingers, especially for soft grippers.  A few new sensors appeared for obtaining contact forces of soft grippers based on the characteristics of capacitance or optical structures [36, 37], which also results in the complexity of compliant grippers. 

% In addition, the exists a good solution to achieving contact forces through establishing the theoretical models of deformations and contact forces [4, 6, 38, 39]. Such a model that reveals the deformation-force mathematical relationships is established by the pseudo-rigid body model [40], or Euler–Bernoulli bending method [41, 42] virtual work [38, 43], or piecewise constant curvature and Cosserat-rod theory [39, 44-46], or Newton–Euler iterations [47], or finite element method [48, 49], or two recent methods based on machine learning [6] and improved pseudo-rigid body model with the virtual work [4]. For the pseudo-rigid body model [42], Euler–Bernoulli bending method [43, 44], and finite element method [48, 49], these techniques mainly attempted to predict the deformations of compliant robots undertaking forces, but the inverse mathematical models are not available so that the contact forces cannot be provided. The virtual work method [38, 43], piecewise constant curvature and Cosserat-rod theory [39, 44-46, 50], Newton–Euler iterations [47] typically focus on continuum robots with long thin structures while applications of such modelling methods on compliant grippers have not been reported, that is, outside of the field of continuum robotics, these methods are uncommon.

%  Indeed, these robotic bodies are considered as a curve, whose cross-sections have the same rotations; however, the contact side and back of a compliant finger present significantly different deformations with nonuniformity joints. Xu et al. [6] introduced a neural network method to achieve the mapping relationships from the deformations to contact forces, skillfully avoiding building the corresponding model of compliant grasp; however, the learning-based method yields two fatal drawbacks to cause the limited applications. In particular, the performance of the learning-based model is determined by the quality and quantity of training samples; this model based on a supervised learning method - neural network model and cannot be generalized to other scenarios; when a geometrical parameter of the gripper is changed or the size range of objects to be grasped is different, the model will be ineffective. Also, they insert many rigid rods to modify the gripper structure, which improves the structure stiffness for allowing the static model to be linear. Thus, this method can just be suitable for fin-ray grippers with linear structures. In addition, when the contact occurs at the middle part of the fin-ray finger, this method can provide accurate force estimation; however, it presents a worse sensing performance when the gripper uses its tip or base to contact an object. Shan et al. [4] devised a theoretical model between the deformation-force through an improved pseudo-rigid body model with the virtual work. However, the stiffness matrix from the virtual work in this method considers that the transformation coordinate reference frames are acquiescently identical or have tiny changes and thus, the model cannot handle a scene where a compliant finger generates a large deformation while holding an object. Shan’ method can just tackle symmetrical-structure compliant gripper, which is limited to wide applications. The method also brings some concerns on computation time and sensing accuracy.  In addition, these two methods can just deal with the compliant grippers with a fin-ray effect based on too many assumptions to achieve the deformation-force relationship. 

\vspace{-1mm}
\section{modeling and analysis}

\subsection{Co-rotational Model}

As demonstrated in Part I of this article, we introduce the co-rotational modeling following Crisfield \cite{ref51} and Yaw \cite{ref52}, accurately and efficiently modeling any arbitrary large motion of objects. Then, two algorithms are proposed, where \textbf{Algorithm 1} acquires the current pose of beam element (current length \textbf{$L$}, the incline angle $\beta$, together with local internal force vector $\textbf{\textit{q}}_l$ and internal global force vector $\textbf{\textit{F}}_{int}$), and \textbf{Algorithm 2} updates the modified global tangent stiffness matrix $\textbf{\textit{K}}_s$ respectively. Besides, the force-deformation model, \textbf{Algorithm 3}, is devised for systematical design optimization. Considering nodes or other hinge-type connections between the separate beam elements \cite{ref53} in a real-world application, we introduce an index of effective length modification $R_m$ and thus,
\vspace{-2.5mm}
\begin{equation}
\begin{matrix}
L_0=L_0-2R_m\\
L=L-2R_m  
\end{matrix}
\label{eq:6}
\vspace{-2mm}
\end{equation}
\noindent The value of $R_m=\mu R_{node}$ is determined in terms of the connection shapes, with $R_{node}$ the connection node radius and $\mu$ the modification factor, determined by comparison with the standard finite element simulation.

% \subsubsection{Axial deflection modeling}
 
% \ 

% Without losing the generality, this paper provides 2-dimensional(D) modeling, closely following the concept of co-rotational modeling given by Crisfield[51] and Yaw[52], where any targeted structure can be modelled as a combination of elastic elements. Co-rotational modeling separates local deformation of each beam element and allows large motions between adjacent elements. This approach can model large deformation of objects with accuracy and efficiency. As for each beam element, it has two nodes. Fig.2 shows one beam element with node 1 and node 2. In the global coordinate frame $\left \{ O,X,Y \right \}$ , when the beam is at the initial configuration, node 1 is having a coordinate ($X_1,Y_1$) and node 2 is ($X_2,Y_2$). Then, both the initial angle $\beta_{0}$ of the beam and its length $L_0$ can be derived accordingly. When the beam element changes to its current configuration under an external load, each node will generate a displacement to reach its current location. For example, node 1 has a displacement of ($u_1,w_1$)  and node 2 has a displacement of ($u_2,w_2$), and the beam element has an incline angle $\beta$ and a current length $L$, which can also be derived from the current positions of both nodes as
% \vspace{-1mm}
% \begin{equation}
% \begin{matrix}
% L_0=\sqrt{(X_2-X_1 )^2+(Y_2-Y_1 )^2} \\
% dX=(X_2+u_2 )-(X_1+u_1 ) \\
% dY=(Y_2+w_2 )-(Y_1+w_1 )
% \end{matrix}
% \label{eq:1}
% \end{equation}
% \vspace{-2mm}
% \begin{equation}
%     L=\sqrt{(dX)^2+(dY)^2}
% \label{eq:2}
% \end{equation}
% \vspace{-5mm}
% \begin{equation}
%     \cos ⁡\beta=\frac{dX}{L}, \sin⁡\beta=\frac{dY}{L}
% \label{eq:3}
% \end{equation}
% \vspace{-5mm}

% With the resultant axial deformation $u_{l} =L-L_{0}$, the axial force N along the beam can be derived via
% % \begin{equation}
% %     u_l=\frac{(L^2-L_0^2 )}{L+L_0}
% % \label{eq:4}
% % \end{equation}
% \vspace{-2mm}
% \begin{equation}
%     N=\frac{EAu_1}{L_0}
% \label{eq:5}
% \end{equation}
% \vspace{-4.5mm}

% \begin{figure}
% 	\centering
% 	\begin{minipage}[!ht]{\linewidth}
% 		\centering
% 		\includegraphics[width=0.9\columnwidth]{Fig/Relationship global local.png}
%   \vspace{-2mm}
% 		\caption{\small{Relationship between global coordinates and local coordinates of the nodes of each beam element.}}
% 		\label{fig:2}
% 	\end{minipage}
% 	\\
% 	\begin{minipage}[!ht]{\linewidth}
% 		\centering
% 		\includegraphics[width=0.9\columnwidth]{Fig/global displacement for a beam element.png}
% 		\caption{\small{The global displacement for a beam element. $(X_l,Y_l )$ represents the node coordinate in the local coordinate reference frame, with $(e_1,e_2)$ the corresponding unit vector: the direction of $e_1$ is along the beam and that of $e_2$ is perpendicular to the beam; the purple line represents the beam after a certain displacement; $\delta d_{21}$ denotes the displacement vector; $\delta_{u_l}$ is the local translation displacement; $\delta\alpha$ represents the rotation angle of the beam.}}
% 		\label{fig:3}
%   \vspace{-6mm}
% 	\end{minipage}
% \end{figure}

% % \begin{figure}[htbp]
% % \centerline{\includegraphics[width=\columnwidth]{Fig/Relationship global local.png}}
% % \caption{Relationship between global coordinates and local coordinates of the nodes of each beam element.}
% % \label{fig:2}
% % \end{figure}

% % \begin{figure}[htbp]
% % \centerline{\includegraphics[width=\columnwidth]{Fig/global displacement for a beam element.png}}
% % \caption{The global displacement for a beam element. $(X_l,Y_l )$ represent the node coordinate in the local coordinate reference frame; $(e_1,e_2)$ is the unit vector of the displacement $(X_l,Y_l)$ on the beam: the direction of $e_1$ is along the beam and that of $e_2$ is vertical to the beam; the purple line represents the beam after a certain displacement; $\delta d_{21}$ denotes the displacement vector; $\delta_{u_l}$ is the local translation displacement; $\delta\alpha$ represents the rotation angle of the beam.
% % }
% % \label{fig:3}
% % \end{figure}

% % \begin{figure}[htbp]
% % \centerline{\includegraphics[width=\columnwidth]{Fig/Typical types of support nodes.png}}
% % \caption{Typical types of support nodes. The support node displacement is $(u=0,w=0,\theta=0)$ in (A), $(u=0,w=0,\theta\ne0)$ in (B), $(u=0,w\ne0,\theta=0)$ in (C).}
% % \label{fig:4}
% % \end{figure}

% \noindent where $E$ represents the module of elasticity and A denotes the cross-sectional area of the beam. Considering nodes or other hinge type connections between the separate beam elements[53], we introduce an index of effective length modification $R_m$ and thus,
% \vspace{-2mm}
% \begin{equation}
% \begin{matrix}
% L_0=L_0-2R_m\\
% L=L-2R_m  
% \end{matrix}
% \label{eq:6}
% \vspace{-3mm}
% \end{equation}

% The value of $R_m=\mu R$ is determined in terms of the connection shapes, where $R$ is the connection node radius, $\mu$ is the modification factor, which can be decided by comparing it with the standard finite element simulation. 

% % \vspace{-4mm}
% \subsubsection{Rotational deflection modeling}

% \ 

% % \vspace{-4mm}
% Apart from the axial deflection, each beam element also undertakes rotational motions at its two nodes. As shown in Fig. 2, $\theta_1$ and $\theta_2$ are the rotations of nodes 1 and 2, measured from the initial incline axis of the beam element. Together with node displacements, the vector $(u,w,\theta)$ represents the global displacement of each node (see Fig.3). Therefore, the local nodal rotations can be calculated via
% \vspace{-1.5mm}
% \begin{equation}
%     \begin{Bmatrix}
% \theta _{1l} 
%  \\
% \theta _{2l} 
% \end{Bmatrix}=\begin{Bmatrix}
% \theta _{1}+  \beta _{0}-\beta  
% \\
% \theta _{2}+  \beta _{0}-\beta
% \end{Bmatrix}
% \label{eq:7}
% \end{equation}
% \vspace{-4mm}

% Using the standard structural analysis results [54], the node moments of the beam (the moment of inertia is represented by $I$) can be related to the local nodal rotations with
% \vspace{-2mm}
% \begin{equation}
% \begin{Bmatrix}
% M_{1} 
%  \\
% M_{2} 
% \end{Bmatrix}=
% \frac{2EI}{L_{0} } \begin{bmatrix}
%  2 & 1\\
%  1 & 2
% \end{bmatrix}
% \begin{Bmatrix}
%  \theta _{1l} \\
% \theta _{2l} 
% \end{Bmatrix}  
% \label{eq:8}
% \end{equation}
% \vspace{-3.5mm}

% As a result, for a beam element, its global displacement $(u,w,\theta)$ can be used for calculating the local displacement $(\theta_1l,\theta_2l,u_l )$, and the latter can be further utilized to calculate the applied load $(N,M_1,M_2)$  through Eqs. (1) and (8). This lays the foundation of the co-rotational modeling[51, 52].

% \subsubsection{Variationally consistent tangent stiffness matrix}

% \ 

% As illustrated in Fig.3, we have the following equation as 
% \vspace{-3mm}
% \begin{equation}
%     \delta u_l=\textbf{e}_1^T \delta \textbf{d}_{21}=\textbf{r} ^T\delta \textbf{p} 
% \label{eq:9}
% \end{equation}
% \vspace{-6mm}

% \noindent where $\delta \textbf{p}$ is the variation of the global displacement vector $\textbf{p}=\begin{bmatrix}
% u_1&w_1&\theta_1&u_2&w_2&\theta_2 \end{bmatrix}^T$ and  $\textbf{r}= \begin{bmatrix}
% -\cos⁡\beta & -\sin⁡\beta & 0 & \cos\beta & \sin⁡\beta & 0 \end{bmatrix}^T$. When $\delta\alpha$ is small, we have 
% \vspace{-3mm}
% \begin{equation}
%     \delta \alpha =\frac{1}{L}\textbf{e}_2^T \delta \textbf{d}_{21}=\frac{1}{L} \textbf{z}^T\delta \textbf{p}                    
% \label{eq:10}
% \end{equation}
% \vspace{-6mm}

% \noindent where $\textbf{z}=\begin{bmatrix}\sin\beta&-\cos\beta& 0& -\sin⁡\beta&\cos⁡\beta&0\end{bmatrix}^T$.The changes of rotations can be provided as follows,
% \begin{small}
% \begin{equation}
%     \delta\theta_l = \left \{ \frac{\delta\theta_{1l}}{\delta\theta_{2l}} \right \}=\left \{ \frac{\delta\theta_{1}-\delta\alpha  }{\delta\theta_{2}-\delta\alpha } \right \}=\left [ T-\frac{1}{L}\left [ \begin{matrix}
%  \textbf{z}^T\\\textbf{z}^T\end{matrix} \right ]   \right ] \delta \textbf{p}
%  \label{eq:11}
% \end{equation}
% \end{small}
% \vspace{-2mm}

% \noindent where $\begin{small}T=\begin{bmatrix}
%   0& 0 & 1 &  0& 0 &0 \\
%  0 &  0& 0 & 0 & 0 &1
% \end{bmatrix}\end{small}$. The change $\delta \textbf{p}_l$ of the local displacement $\textbf{p}_l$ can be described by the change $\delta \textbf{p}$ of the global displacement vector $\textbf{p}$ as
% \vspace{-2mm}
% \begin{equation}
%         \delta \textbf{p} _l=\left \{ \begin{matrix} \delta u_l \\\delta\theta_{1l} \\\delta\theta _{2l} \end{matrix} \right \}=\textbf{B}\delta \textbf{p}   
% \label{eq:12}
% \end{equation}
% \vspace{-3mm}

% \noindent where \textbf{B} is the transformation matrix in the following form
% \vspace{-1mm}
% \begin{equation}
% \mathbf{B} =\begin{bmatrix}
%  -\cos\beta  & -\sin\beta & 0 & \cos\beta & \sin\beta & 0\\
%  -\frac{\sin\beta}{L}  &  \frac{\cos\beta}{L} & 1 & \frac{\sin\beta}{L} & -\frac{\cos\beta}{L} & 0\\
%  -\frac{\sin\beta}{L} & \frac{\cos\beta}{L} & 0 & \frac{\sin\beta}{L} & -\frac{\cos\beta}{L} & 1
% \end{bmatrix}
% \label{eq:13}
% \end{equation}

% In the local coordinate frame $\left \{ X_{l},Y_{l} \right \}$, the local internal force vector of i can be described by $\textbf{q}_{li}=\begin{bmatrix}
%  N & M_1 & M_2 
% \end{bmatrix}^T$ and local virtual displacement is $\delta \textbf{p}_{lv}=\begin{bmatrix}
% \delta u_{lv} & \delta \theta_{1lv}& \delta \theta_{2lv}
% \end{bmatrix}^T$. In terms of the global coordinate frame $\{X,Y\}$, $\textbf{q}_i$ is the vector of global internal forces for the element $i$ and $\delta \textbf{p}_v$ are the arbitrary virtual displacements. For each beam element, according to the equivalence of virtual work in the local and global systems, we have 
% \vspace{-1mm}
% \begin{equation}
%     \delta \textbf{p}_v^Tq_i=\delta p_{lv}^T \textbf{q}_{li}=(\textbf{B}\delta \textbf{p}_v )^T \textbf{q}_{li}=\delta \textbf{p}_v^T \textbf{B}^T \textbf{q}_{li}
% \label{eq:14}
% \end{equation}
% \vspace{-5mm}

% Thus,
% \vspace{-1mm}
% \begin{equation}
% \textbf{q}_i = \textbf{B}^T\textbf{q}_{li}
% \label{eq:15}
% \end{equation}
% \vspace{-6mm}

% Then, the vector $F_{int}$ of internal global forces is provided
% \vspace{-3mm}
% \begin{equation}
% \resizebox{.85\hsize}{!}{$\begin{matrix}
% \begin{split}
% &F_{int} = {A_s}_{i=1}^{n_m}q_i =  \\
% &{\textstyle \sum_{i=1}^{n_m}} 
% \begin{bmatrix}
% \cdots & \textbf{q}_{i,1} & \textbf{q}_{i,2} & \textbf{q}_{i,3} & \cdots & \textbf{q}_{i,4} & \textbf{q}_{i,5} & \textbf{q}_{i,6} & \cdots
% \end{bmatrix}^T
% \end{split}
% \end{matrix}$}
% \label{eq:16}
% \end{equation}
% where $A_s$ is the assembly index (see Hughes[55]). Take the derivative of $\textbf{q}_i=\textbf{B}^T \textbf{q}_{li}$, we have 
% \vspace{-1mm}
% \begin{equation}
%     \delta \textbf{q}_i=\textbf{k}_i\delta \textbf{p}
% \label{eq:17}
% \vspace{-1mm}
% \end{equation}

% \noindent in which $\textbf{k}_i$ is the tangent stiffness matrix. The detailed calculations are omitted here and the readers can refer to [51,52]. Here the final formula of $\textbf{k}_i$ is given as
% \vspace{-1mm}
% \begin{equation}
%     \textbf{k}_i  =\textbf{B}^T \textbf{C}_l\textbf{B}+\frac{F_N}{L}/\textbf{zz}^T+\frac{M_1+M_2}{L^2}(\textbf{r}\textbf{z}^T+\textbf{zr}^T )  
% \label{eq:18}
% \end{equation}
% \vspace{-4mm}
% \noindent where

% \begin{equation}
%     \textbf{C}_l= \frac{EA}{L_0} \begin{bmatrix}
%  1 &0  &0 \\
%  0 & \frac{4I}{A} & \frac{2I}{A}\\
%  0 & \frac{2I}{A} & \frac{4I}{A}
% \end{bmatrix}
% \label{eq:19}
% \end{equation}
% \vspace{-4mm}
% \begin{algorithm}[H]
% \caption{\textit{\textbf{Member data update}}}
% \begin{algorithmic}
% \STATE \textbf{\textit{Input:}}
% \STATE \noindent $n_{nodes},n_{mem}, m_{conn},\textit{\textbf{A, E, I}}, \textit{\textbf{x}}_0, \textit{\textbf{y}}_0, \textit{\textbf{L}}_0, \boldsymbol{\beta}_0, \textit{\textbf{u}},R_m $
% \STATE \textbf{\textit{Calculations:}}
% \STATE \textbf{\textit{For}} $i$=1: $n_{mem}$
% \STATE \quad\textit{Obtain $\textbf{L}, \cos\beta$, $\cos\beta$ according to Eqs. (1-3)}
% \STATE \quad\textit{Obtain $\textbf{q}_{li}$ according to Eqs. (4-8) and thus $\textbf{q}_l$}
% \STATE \quad\textit{Obtain $\textbf{\textit{F}}_{int}$ according to Eqs. (13-16)}
% \STATE \textbf{\textit{End}}     
% \STATE \textit{\textbf{Output:} $\textbf{L}, \textbf{c}, \textbf{s}, \textbf{q}_l, \textbf{F}_{int}$}
% \end{algorithmic}
% \label{alg:1}
% \end{algorithm}
% \vspace{-6mm}
% \begin{algorithm}[H]
% \caption{\textbf{\textit{Tangent stiffness matrix update}}}
% \begin{algorithmic}
% \STATE \textbf{\textit{Input:}}
% \STATE \begin{normalsize} $n_{nodes}, n_{mem}, m_{conn}, \textit{\textbf{sup, A, E, I, L, c, s}}, \textit{\textbf{q}}_l, R_m $ \end{normalsize}
% \STATE \textbf{\textit{Calculations:}}
% \STATE \textbf{\textit{For}} $i$=1: $n_{mem}$
% \STATE \ \ \textit{Obtain $\textbf{K}$ according to Eqs. (9-20) }
% \STATE \ \ \textit{Obtain \!the \!modified \!$\textbf{\small{K}}_s$\! based\! on the\! support\! condition \!$\textbf{\small{sup}}$}
% \STATE \textbf{\textit{End}}     
% \STATE \textit{\textbf{Output:} $\textbf{K}_s$}
% \end{algorithmic}
% \label{alg:2}
% \end{algorithm}

% \vspace{-4mm}
% \noindent Then we can further calculate the global tangent stiffness matrix K of the whole structure as
% \begin{equation}
%     \textbf{\textit{K}} = {A_s}_{i=1}^{n_m}\textbf{\textit{k}}_i=A_{i=1}^{n_m}\begin{bmatrix}
%   & \vdots &  & \vdots & \\
%  \cdots & k^i_{1,1} &  \cdots & k^i_{1,2} &  \cdots \\
%   & \vdots &  & \vdots & \\
% \cdots  &  k^i_{2,1} &  \cdots &  k^i_{2,2} &  \cdots\\
%   & \vdots &  & \vdots &
% \end{bmatrix}
% \label{eq:20}
% \end{equation}

% \noindent In which $A_s$ is the assembly operator and $n_m$ is the number of elements. The assembly rows and columns of $\textbf{\textit{k}}_i$ depends on the order of the first and second nodes in the element i. Allowing for the node supports, $\textbf{\textit{K}}$ can be changed to the modified global tangent stiffness matrix $\mathbf{\textit{K}}_s$. Typical types of support nodes are presented in Fig. 4, where they are constrained depending on the support conditions. For a particular support node, the displacements in the constrained directions are always zero regardless of external forces. The rows and columns of $\textbf{\textit{K}}_s$ concerning this displacement will be zero since the displacement on the support node is still zero. For example, if node i is defined as one support node and fully constrained, then the elements in the related rows $(3i-2,3i-1,3i)$ and related columns $(3i-2,3i-1,3i)$ of  $\textit{\textbf{K}}_s$ are all set to be zero. 

% \vspace{-4mm}
% \begin{figure}[htbp]
% \centerline{\includegraphics[width=\columnwidth]{Fig/Typical types of support nodes.png}}
% \caption{Typical types of support nodes. The support node displacement is $(u=0,w=0,\theta=0)$ in (A), $(u=0,w=0,\theta\ne0)$ in (B), $(u=0,w\ne0,\theta=0)$ in (C).}
% \label{fig:4}
% \vspace{-8mm}
% \end{figure}
% \subsection{Force Control}
% \begin{figure*}[t]
% \vspace{-5mm}
% \centerline{\includegraphics[width=2.05\columnwidth]{Fig/One incremental step in force control.png}}
% \caption{One incremental step in the force control. The sketch of finger deformation (A) and the relationship of force and deformation (B).}
% \label{Fig:5}
% \end{figure*}
% \begin{figure*}[t]
% \vspace{-8mm}
% \centerline{\includegraphics[width=2.05\columnwidth]{Fig/One incremental step in the displacement control.png}}
% \caption{One incremental step in the displacement control. The sketch of finger deformation (A) and the relationship of force and deformation (B)}
% \label{Fig:6}
% \vspace{-5mm}
% \end{figure*}
% “For control” indicates that the displacement is calculated by the contact force. In this section, we describe how to calculate the displacement on a finger of a gripper. This is an implicit formulation that uses the algorithm of Newton-Raphson iterations [56] at the global level to achieve equilibrium during each incremental load step. Different from the linear analysis, in the force control described here, the total increment is subdivided into a number of steps, each of which is represented by a cycle in which an equilibrium is reached within a certain tolerance. Specifically, the total vector of externally applied global nodal forces is defined as $\textbf{\textit{F}}_{total}$, as shown in Appendix. $\textbf{\textit{F}}_{total}$ is a $3n_m\times1$ vector, the non-zero elements of which represent the externally applied forces applied at selected nodes, so there are no constraints on the type of contact forces, which can be either concentrated load or distributed loads. Further, assuming a total number $n_{inc}$ of load increment steps is required to reach the final equilibrium from the initial equilibrium, and a load of each increment is $d\textbf{\textit{F}}$. Thus, we have
% \begin{equation}
% d\textbf{\textit{F}}=\lambda\textbf{\textit{F}}_{total}
% \label{eq:21}
% \end{equation}
% \noindent where $\lambda$ is the load ratio and $\lambda=\frac{1}{n_{inc}}$. In the $n^{th}$ increment, the vector of global nodal displacements is defined as $u^n$ and the vector of global nodal forces is $\textbf{\textit{F}}^n$, where the symbol in the superscript represents the order of the iterations. According to Algorithm 1 in Table I and Algorithm 2 in Table II, we can calculate $\textit{\textbf{L,c,s}},\textbf{\textit{q}}_{l},\textbf{\textit{F}}_{int}$ and obtain the modified stiffness matrix $\textbf{\textit{K}}_s$. Then, the vector $\textit{d\textbf{u}}$ of each incremental global nodal displacement can be calculated using $\textbf{\textit{K}}_s$ [51] as

% \begin{equation}
% d\textbf{\textit{u}}=\textbf{\textit{K}}_s^{-1}d\textbf{\textit{F}}
% \label{eq:22}
% \end{equation}

% \noindent As illustrated in Fig.5,  $\textbf{\textit{u}}^n$ and $\textbf{\textit{F}}^n$ can be further updated as

% \begin{equation}
%     \begin{matrix}
% \textbf{\textit{u}}^{n+1} = \textbf{\textit{u}}^n=d\textbf{\textit{u}}
%  \\
% \textbf{\textit{F}}^{n+1} = \textbf{\textit{F}}^n +d\textbf{\textit{F}}
% \end{matrix}
% \label{eq:23}
% \end{equation}
% which will later be used in the iteration cycle to achieve equilibrium. Further, we can update $\textbf{\textit{L,c,s}},\textbf{\textit{q}}_l^{n+1},\textbf{\textit{F}}_{int}^{n+1}$ according to \textbf{Algorithm \ref{alg:1}} in Table I based on $\textbf{\textit{u}}^{n+1}$. For the iterations, we need to determine the deviation that can be accepted to compare with a set tolerance by defining the residual \textbf{\textit{R}} as 

% \begin{equation}
%     \begin{matrix}
% \begin{aligned}
% &\textbf{\textit{R}} =  \textbf{\textit{F}}_{int}^{n=1}-\textbf{\textit{F}}^{n+1}
%  \\
% &R=\sqrt[]{\textbf{\textit{R}}\cdot \textbf{\textit{R}} } 
% \end{aligned}
% \end{matrix}
% \label{eq:24}
% \end{equation}
% Following the initial preparation process, we now enter into the iteration cycle which aims to reach the force equilibrium. A few iteration variables are defined as follows: Iteration number $k=0,tolerance=10^{-3}$. The maximum iteration step is limited by $maxiter=100$. The correction to incremental global nodal displacements is defined as $\delta \textbf{\textit{u}}^k=0$. The temporary vector of local forces in the $k$-th iteration cycle is set as
% \begin{equation}
%     \textbf{\textit{q}}_{l-temp}^k = \textbf{\textit{q}}_l^{n+1}
% \label{eq:25}
% \end{equation}
% \noindent In each step k, firstly, the stiffness matrix $\textbf{\textit{K}}_s$ is updated according to \textbf{Algorithm \ref{alg:2}} based on updated current values of inputs and $\textbf{\textit{q}}_{l-temp}^k$. Then the global nodal displacements and member data are updated as
% \begin{equation}
% \begin{matrix}
% \begin{aligned}
% &\delta \textbf{\textit{u}}^{k+1} = \delta \textbf{\textit{u}}^k-\textbf{\textit{K}}_s^{-1}\textbf{\textit{R}}
%  \\
% &\textbf{\textit{u}}_{(k+1)}^{n+1} = \textbf{\textit{u}}^{n+1}+\delta \textbf{\textit{u}}^{k+1}
% \end{aligned}
% \end{matrix}
% \label{eq:26}
% \end{equation}
% \noindent where $\textbf{\textit{u}}_{(k+1)}^{n+1}$ represents $\textbf{\textit{u}}^{n+1}$ in the $(k+1)$-th iteration. When the loop is stopped, $\textbf{\textit{u}}_{(k+1)}^{n+1}$ becomes $\textbf{\textit{u}}^{n+1}$. Update the $\textbf{\textit{q}}_{l-temp}^k,\textbf{\textit{F}}_{int}^{n+1}$, according to \textbf{Algorithm \ref{alg:1}} based on $\textbf{\textit{u}}_{(k+1)}^{n+1}$ obtained in Eqn. \ref{eq:25} and we can calculate the new residual
% \begin{equation}
%     \begin{matrix}
% \begin{aligned}
% &\textbf{\textit{R}} =  \textbf{\textit{F}}_{int}^{n=1}-\textbf{\textit{F}}^{n+1}
%  \\
% &R=\sqrt[]{\textbf{\textit{R}}\cdot \textbf{\textit{R}} } 
% \end{aligned}
% \end{matrix}
% \label{eq:27}
% \end{equation}
% Update iteration number $k=k+1$. The iteration cycle will terminate if $R\le tolerance$ or $k\ge maxiter$ (in this case, the convergence criteria is not met), and the variables will update to their final value in this $n$-th increment
% \begin{equation}
% \begin{aligned}
% \begin{matrix}
% \textbf{\textit{q}}^{n+1}_l = \textbf{\textit{q}}^{k+1}_{l-temp}
%  \\
% \textbf{\textit{u}}^{n+1} = \textbf{\textit{u}}^{n+1}_{(k+1)} = \textbf{\textit{u}}^{n+1}=\delta \textbf{\textit{u}}^{k+1}
% \end{matrix}
% \end{aligned}
% \label{eq:28}
% \end{equation}
% The complete force control in the $n^th$ increment is also illustrated in Figure. 5, where the variables updated in both the preliminary step and iteration cycle are demonstrated. The detailed algorithm is illustrated in Table III.

\vspace{-3mm}
\subsection{Displacement-Force Modeling}

% The force control algorithm given in the last section is satisfactory for many problems, for instance, when the external force is given to calculate the deformation of the gripper. However, in the case when a target deformation is given to estimate the reaction force, the force control algorithm may not well solve the problem. In addition, when problems of snap-through, snap-back, or equilibrium path tracing are encountered, force control algorithm will not be able to fully describe the system’s force-displacement behaviour. To solve this problem, displacement control algorithms [51, 55] are needed. 
In this section, a displacement-force relationship closely following controls schemes from McGuire et al. \cite{ref57} and Clarke and Hancock \cite{ref58} is provided, where a constantly increased external displacement load is applied at the target structure instead of an external force. 
% In addition, because of the nature of constantly increased displacement, the “snap-through” problem can be solved. 

The details of the proposed modeling are as follows. Each node has three DoFs, including x-axis and y-axis translations and a rotation. If the maximum displacement for the x-axis or y-axis translation or the rotation is defined as a vector $\textbf{\textit{D}}_{total}$, and a total number $n_{inc}$ of increments are required to reach the final equilibrium, we have:
\vspace{-2mm}
\begin{equation}
\Delta \bar{\textbf{\textit{u}}}_f = \frac{\textbf{\textit{D}}_{total}}{n_{inc}} 
    \label{eq:29}
    \vspace{-2mm}
\end{equation}

\noindent where $\Delta \bar{\textbf{\textit{u}}}_f$ represents an incremental value.

 A full equilibrium cycle in the $n$-th increment is demonstrated. We obtain the modified stiffness matrix $\textbf{\textit{K}}_s$ according to \textbf{Algorithm 1} and \textbf{Algorithm 2} based on the initial values resulting from the $(n-1)$-th cycle. Then, a global nodal displacements vector can be calculated as:
% As demonstrated in our prior work [arxiv], where any arbitrary large motion of objects can be accurately and efficiently modeled following the concept of the co-rotational modeling given by Crisfield[51] and Yaw[52], two  algorithms are proposed, where algorithm 1 acquires the current pose of beam element, together with global/local force vectors, and algorithm 2 updates the modified global tangent stiffness matrix $\textbf{\textit{K}}_s$ respectively
% We obtain the modified stiffness matrix $\textbf{\textit{K}}_s$ via \textbf{Algorithm \ref{alg:1}} and \textbf{Algorithm \ref{alg:2}} based on the $(n-1)$-th cycle's initial values. 
% Then, we calculate a global nodal displacements vector as
\vspace{-2mm}
\begin{equation}
\hat{\textbf{\textit{u}}}=\textbf{\textit{K}}_s^{-1}\textbf{\textit{F}}_{ref}
\label{eq:30}
\vspace{-2mm}
\end{equation}


\noindent where $\textbf{\textit{F}}_{ref}$ is the referenced value \cite{ref56}.
% rather than the total value of global nodal forces $\textbf{\textit{F}}_{total}$ used in the force control. 
The value of $\textbf{\textit{F}}_{ref}$ is determined according to either experience or an automatic strategy. In our case, it is found that the direction of $\textbf{\textit{F}}_{ref}$ is more important than its magnitude. As a result, the direction of $\textbf{\textit{F}}_{ref}$ is set the same as: $\textbf{\textit{D}}_{total}$. For the magnitude, we use a try-and-error approach. 
% by submitting a pre-defined $\textbf{\textit{F}}_{ref}$ into the force-control algorithm so that the resulted displacement is with the same order of magnitude of $\textbf{\textit{D}}_{total}$. 
Then we can obtain the incremental load ratio vector $d\boldsymbol{\lambda}^{n+1}$ as:
\vspace{-1mm}
\begin{equation}
d\boldsymbol{\lambda}^{n+1} = \frac{\Delta \bar{\textbf{\textit{u}}}_f}{\hat{\textbf{\textit{u}}}} 
\label{eq:31}
\vspace{-1mm}
\end{equation}
\noindent And then update the load ratio \cite{ref58}: 
\vspace{-1mm}
\begin{equation}
\boldsymbol{\lambda}^{n+1}=d\boldsymbol{\lambda}^{n+1}+\boldsymbol{\lambda}^{n}  
\label{eq:32}
\vspace{-2mm}
\end{equation}
\noindent Thus, the incremental force vector can be given as:
\vspace{-1mm}
\begin{equation}
d\textbf{\textit{F}} = d\boldsymbol{\lambda}^{n+1}\textbf{\textit{F}}_{ref}
\label{eq:33}
\vspace{-2mm}
\end{equation}
\noindent We calculate the incremental global nodal displacements as:
\vspace{-1mm}
\begin{equation}
d\textbf{\textit{u}} = \textbf{\textit{K}}^{-1}_sd\textbf{\textit{F}}
\label{eq:34}
\vspace{-2mm}
\end{equation}                           
\noindent Then the global nodal displacements and global nodal forces can be updated as:
\vspace{-4mm}
\begin{equation}
\vspace{-1mm}
\begin{matrix}
    \textbf{\textit{u}}^{n+1} = \textbf{\textit{u}}^n +d\textbf{\textit{u}} \\
    \textbf{\textit{F}}^{n+1} = \textbf{\textit{F}}^n +d\textbf{\textit{F}}
\end{matrix}
\label{eq:35}
\vspace{-2mm}
\end{equation}   


% \begin{algorithm}[H]
% \caption{\textit{\textbf{Force Control}}}
% \begin{algorithmic}
% \STATE \textbf{\textit{Input:}}
% \STATE $\!\!n_{nodes},n_{mem}, m_{conn},\textit{\textbf{A,E,I}}, \textit{\textbf{x}}_0, \textit{\textbf{y}}_0, \textit{\textbf{L}}_0, \!\boldsymbol{\beta}_0, \textit{\textbf{u}},R_m,\textbf{\textit{F}}_{total}$
% \STATE \textbf{\textit{Calculations:}}
% \STATE \textbf{\textit{For}} $n$=1: $n_{mem}$

% \STATE \quad\textit{calculate $d\textbf{F}$ by Eqn.(\ref{eq:21})}

% \STATE \quad\textit{calculate $\textbf{L}, \textbf{c}, \textbf{s}, \textbf{q}_l, \textbf{F}_{int}$ by \small{\textbf{Algorithm \ref{alg:1}}}}

% \STATE \quad\textit{calculate $\textbf{K}_s$ by \small{\textbf{Algorithm \ref{alg:2}}}}

% \STATE \quad\textit{Obtain $\textbf{\textit{d}}_{u}$ according to Eqs.(\ref{eq:22}) }

% \STATE \quad\textit{update $\textbf{u}^{n+1}$ and $\textbf{F}^{n+1}$ by Eqn.(\ref{eq:23})}

% \STATE \quad\textit{update $\textbf{L}, \textbf{c}, \textbf{s}, \textbf{q}_l^{n+1}, \!\textbf{F}_{int}^{n+1}$ \!by \small{\textbf{Algorithm \ref{alg:1}}} \normalsize{based on $\textbf{u}^{n+1}$}}

% \STATE \quad\textit{calculate the residual $\textbf{R}$ by Eqn.(\ref{eq:24})}

% \STATE \quad\textit{set up iteration variables $k, tolerance, maxiter, \delta \textbf{u}$ and $ \textbf{q}_{l-temp}^k$  }

% \STATE \quad\textit{
% start iterations while $R\! \ge\! tolerance$ and $k\! \le \!maxiter$}

% \STATE \qquad\enspace\textit{\footnotesize{\expandafter{\romannumeral1}. 
% calculate $\textbf{K}_s$ by \textbf{Algorithm \ref{alg:2}} and $\textbf{q}_{l-temp}^k$ by Eqn.(\ref{eq:25})}}

% \STATE \qquad\enspace\textit{\footnotesize{\expandafter{\romannumeral2}. 
% update member data $\delta \textbf{u}^{k+1}$ and $\textbf{u}_{(k+1)}^{n+1}$ by Eqn.(\ref{eq:26})
% }}

% \STATE \qquad\enspace\textit{\footnotesize{\expandafter{\romannumeral3}. 
% update $\textbf{q}_{l-temp}^{k+1},\textbf{F}_{int}^{n+1}$ by \textbf{Algorithm \ref{alg:1}} based on $\textbf{u}_{current}$
% }}

% \STATE \qquad\enspace\textit{\footnotesize{\expandafter{\romannumeral4}. 
% calculate the residual $\textbf{R}$ and $R$ by Eqn.(\ref{eq:27})
% }}

% \STATE \qquad\enspace\textit{\footnotesize{\expandafter{\romannumeral5}. 
% update iteration number k=k+1
% }}

% \STATE \quad\textit{End of while loop iterations}

% \STATE \quad\textit{Update variables $\textbf{q}_l^{n+1}$ and $\textbf{u}^{n+1}$ by Eqn. (\ref{eq:28})}

% \STATE \textbf{\textit{End}}     
% \STATE \textit{\textbf{Output:} $\textbf{q}_l^{n+1},\textbf{u}^{n+1}$}
% \end{algorithmic}
% \label{alg:3}
% \end{algorithm}
% \vspace{-4mm}

\noindent Further, we can update $\textbf{L}, \textbf{c}, \textbf{s}, \textbf{q}_l^{n+1}, \!\textbf{F}_{int}^{n+1}$ according to \textbf{Algorithm 1} based on $\textbf{u}^{n+1}$. By accounting for the required support, we calculate the residual as 
\vspace{-1mm}
\begin{equation}
    \textbf{\textit{R}} = \boldsymbol{\lambda}^{n+1}\textbf{\textit{F}}_{ref} - \textbf{\textit{F}}^{n+1}_{int}
\label{eq:36}
\vspace{-2mm}
\end{equation}

\noindent The norm of the residual is provided as
\begin{equation}
    R = \sqrt{\textbf{\textit{R}}\cdot \textbf{\textit{R}}}
    \label{eq:37}
    \vspace{-2mm}
\end{equation}

\noindent Subsequently, we implement an iterative strategy to obtain the final values of the node displacement and externally applied force. A few iteration variables are defined as follows: Iteration number $k=0$, $tolerance=10^{-3}$. The maximum iteration step is limited by $maxiter=100$. The correction to incremental global nodal displacements is defined as $\delta \textbf{\textit{u}}^k (\delta \textbf{\textit{u}}^0=\textbf{0})$, and the correction to load ratio is defined as $\delta \boldsymbol{\lambda}^k (\delta \boldsymbol{\lambda}^0=\textbf{0})$. The storage vector of local forces in this iteration cycle is set as $\textbf{\textit{q}}_{l-temp}^k=\textbf{\textit{q}}_l^{n+1}$.

\begin{figure*}[t]
\vspace{-1mm}
\centerline{\includegraphics[width=1.8\columnwidth]{Fig/2.png}}
\vspace{-2mm}
\caption{\small One incremental step in the displacement control. The sketch of finger deformation(A), the relationship of force and deformation(B).}
\label{Fig:6}
\vspace{-4mm}
\end{figure*}

% \begin{figure}
% \vspace{-4mm}
% 	\centering
% 	\begin{minipage}[t]{\linewidth}
% 		\centering
% 		\includegraphics[width=0.95\columnwidth]{Fig/A comparision between the co-rotational modeling and FEA simulation.png}
%   \vspace{-4mm}
% 		\caption{\small{A comparison between the co-rotational modelling and FEA simulation. In this case, external displacement loads are applied at nodes 8 and 9. The deformation shape of the soft finger is described in red line and the FEA simulation result of key nodes are described in blue circle, which shows a good agreement.}}
% 		\label{fig:6}
% 	\end{minipage}
% 	\\
% 	\begin{minipage}[t]{\linewidth}
% 		\centering
% 		\includegraphics[width=0.8\columnwidth]{Fig/The models meshed sparsely.png}
%   \vspace{-2mm}
% 		\caption{\small{The models meshed sparsely (A) and densely (B). m and n represent the width and height of a fin-ray finger, respectively. The numbers indicate the labels of nodes and the numbers in brackets denote the labels of flexible elements. The red frame with two nodes a and b represents the sparse connection style (A), where the bold circle indicates a physically existing node. For the dense connection style, an intermediate node c is introduced which is a virtual node used in the analytical model.}}
% 		\label{fig:7}
%   \vspace{-6mm}
% 	\end{minipage}
%  \vspace{-3mm}
% \end{figure}
% \vspace{-5mm}
% \begin{figure}[htbp]
% \vspace{-5mm}
% \centerline{\includegraphics[width=0.9\columnwidth]{Fig/A comparision between the co-rotational modeling and FEA simulation.png}}
% \vspace{-3mm}
% \caption{\small A comparison between the co-rotational modelling and FEA simulation. In this case, external displacement loads are applied at nodes 8 and 9. The deformation shape of the soft finger is described in red line and the FEA simulation result of key nodes are described in blue circle, which shows a good agreement}
% \vspace{-6mm}
% \label{A comparision between the co-rotational modeling and FEA}
% \end{figure}

% \begin{figure}[htbp]
% \centerline{\includegraphics[width=0.9\columnwidth]{Fig/The models meshed sparsely.png}}
% \vspace{-1.5mm}
% \caption{\small{The models meshed sparsely (A) and densely (B). m and n represent the width and height of a fin-ray finger, respectively. The numbers indicate the labels of nodes and the numbers in brackets denote the labels of flexible elements. The red frame with two nodes a and b represents the sparse connection style (A), where the bold circle indicates a physically existing node. For the dense connection style, an intermediate node c is introduced which is a virtual node used in the analytical model. }}
% \label{The models meshed sparsely}
% \vspace{-5mm}
% \end{figure}

In each step $k$, the stiffness matrix $\textbf{\textit{K}}_s$ is updated according to \textbf{Algorithm 2} based on updated current values of inputs and $\textbf{\textit{q}}_{l-temp}^k$. The load ratio is calculated following \cite{ref56} as
\vspace{-2mm}
\begin{equation}
    \begin{matrix}
        \acute{\textbf{\textit{u}}} = \textbf{\textit{K}}_s^{-1}\textbf{\textit{R}} \\
        \grave{\textbf{\textit{u}}} = \textbf{\textit{K}}_s^{-1}\textbf{\textit{F}}_{ref}\\
        \delta \boldsymbol{\lambda }^{k+1} = \delta\boldsymbol{\lambda }^k - \frac{\acute{\textbf{\textit{u}}}}{\grave{\textbf{\textit{u}}}} 
    \end{matrix}
    \label{eq:38}
    \vspace{-2mm}
\end{equation}
Then the correction to $\textbf{\textit{u}}^{n+1}$ can be calculated as
\vspace{-1mm}
\begin{equation}
\delta \textbf{\textit{u}}^{k+1} = \delta \textbf{\textit{u}}^{k} + \textbf{\textit{K}}_s^{-1}
\begin{bmatrix}
    \textbf{\textit{R}}-\frac{\acute{\textbf{\textit{u}}}}{\grave{\textbf{\textit{u}}}}\textbf{\textit{F}}_{ref}
\end{bmatrix}
    \label{eq:39}
    \vspace{-2mm}
\end{equation}
Update $\textbf{\textit{q}}_{l-temp}^{k+1}$ and $\textbf{\textit{F}}_{int}^{n+1}$ according to \textbf{Algorithm 1}, and the residual in this iteration can be calculated as
\vspace{-1mm}
\begin{equation}
    \textbf{\textit{R}} = (\boldsymbol{\lambda}^{n+1}+\delta\boldsymbol{\lambda^{k+1}})\textbf{\textit{F}}_{ref}-\textbf{\textit{F}}^{n+1}_{int}
    \label{eq:40}
    \vspace{-2mm}
\end{equation}
The norm of the residual is updated using  Eqn.(\ref{eq:37}).
% \vspace{-1mm}
% \begin{equation}
%     R = \sqrt{\textbf{\textit{R}}\cdot \textbf{\textit{R}}}
%     \label{eq:41}
%     \vspace{-2mm}
% \end{equation}

Then updating iteration number $k=k+1$, the iteration cycle will terminate if $R\le tolerance$ or $k\ge maxiter $ (in this case, the convergence criteria are not met), and the variables will update to their final value in this $n$-th increment
\vspace{-2mm}
\begin{equation}
\begin{matrix}
    \boldsymbol{\lambda}^{n+1} = \boldsymbol{\lambda}^{n+1} + \delta\boldsymbol{\lambda}^{k+1} \\
    \textbf{\textit{q}}_{l}^{n+1} = \textbf{\textit{q}}_{l-temp}^{k+1} \\
    \textbf{\textit{u}}^{n+1} = \textbf{\textit{u}}^{n} + d\textbf{\textit{u}}+\delta \textbf{\textit{u}}^{k+1}
\end{matrix}
    \label{eq:42}
    \vspace{-2mm}
\end{equation}

\begin{algorithm}[h]
\renewcommand{\thealgorithm}{4}
\caption{\textit{\textbf{Displacement-force Modeling}}}
\begin{algorithmic}
\STATE \textbf{\textit{Input:}}
\STATE \small$\!\!n_{nodes},n_{mem}, m_{conn},\textit{\textbf{A,E,I}}, \textit{\textbf{x}}_0, \textit{\textbf{y}}_0, \textit{\textbf{L}}_0, \!\boldsymbol{\beta}_0, \textit{\textbf{u}},R_m,\textbf{\textit{D}}_{total}, \textbf{\textit{F}}_{ref}$
\STATE \textbf{\textit{Calculations:}}
\STATE \textbf{\textit{For}} $n$=1: $n_{mem}$

\STATE \quad\textit{calculate $\Delta \bar{\textbf{u}}_f$ by Eqn.(\ref{eq:29})}

\STATE \quad\textit{calculate $\hat{\textbf{u}}$ by Eqn.(\ref{eq:30})}

\STATE \quad\textit{calculate $d\boldsymbol{\lambda}^{n+1},\boldsymbol{\lambda}^{n+1}$ by Eqns.(\ref{eq:31}-\ref{eq:32})}

\STATE \quad\textit{calculate $d\textbf{F}$ by Eqn.(\ref{eq:33})}

\STATE \quad\textit{calculate $\textbf{L}, \textbf{c}, \textbf{s}, \textbf{q}_l, \textbf{F}_{int}$ by \small{\textbf{Algorithm 1}}}

\STATE \quad\textit{calculate $\textbf{K}_s$ by \textbf{Algorithm 2}}

\STATE \quad\textit{calculate $d\textbf{u}$ by Eqn.(\ref{eq:34})}

\STATE \quad\textit{update $\textbf{u}^{n+1}$ and $\textbf{F}^{n+1}$ by Eqn.(\ref{eq:35})}

\STATE \quad\textit{update $\textbf{L}, \textbf{c}, \textbf{s}, \textbf{q}_l^{n+1}, \!\textbf{F}_{int}^{n+1}$ \!by \small{\textbf{Algorithm 1}} \normalsize{based on $\textbf{u}^{n+1}$}}

\STATE \quad\textit{calculate the residual $\textbf{R}$ and $R$ by Eqns.(\ref{eq:36}-\ref{eq:37})}

\STATE \quad\textit{set up iteration variables $k, tolerance, maxiter, \delta \textbf{u}, \delta \boldsymbol{\lambda}$ and $ \textbf{q}_{l-temp}^k$  }

\STATE \quad\textit{
start iterations while $R\! \ge\! tolerance$ and $k\! \le \!maxiter$}

\STATE \qquad\enspace\textit{\footnotesize{\expandafter{\romannumeral1}. calculate $\textbf{K}_s$ by \textbf{Algorithm 2}}}

\STATE \qquad\enspace\textit{\footnotesize{\expandafter{\romannumeral2}. update load ratio correction $\delta \boldsymbol{\lambda}^{k+1}$  by Eqn.(\ref{eq:38})}}

\STATE \qquad\enspace\textit{\footnotesize{\expandafter{\romannumeral3}. update $\delta\textbf{u}^{k+1}$ by Eqn. (\ref{eq:39})}}

\STATE \qquad\enspace\textit{\footnotesize{\expandafter{\romannumeral4}. update $\textbf{q}_{l-temp}^{k+1},\textbf{F}_{int}^{n+1}$ by \textbf{Algorithm 1}}}

\STATE \qquad\enspace\textit{\footnotesize{\expandafter{\romannumeral5}. 
calculate the residual $\textbf{R}$ by Eqn.(\ref{eq:40}), and its norm $R$ by Eqn.(\ref{eq:37})}}

\STATE \qquad\enspace\textit{\footnotesize{\expandafter{\romannumeral6}. update iteration number k=k+1
}}

\STATE \quad\textit{End of while loop iterations}

\STATE \quad\textit{Update variables $\boldsymbol{\lambda}^{n+1}, \textbf{q}_l^{n+1}$ and $\textbf{u}^{n+1}$ by Eqn. (\ref{eq:42})}

\STATE \textbf{\textit{End}}     
\STATE \textit{\textbf{Output:} $\boldsymbol{\lambda}^{n+1},\textbf{q}_l^{n+1},\textbf{u}^{n+1}$}
\end{algorithmic}
\label{alg:4}
\end{algorithm}
\vspace{-1mm}

The complete displacement control in the $n$-th increment is also illustrated in Fig. \ref{Fig:6}, where the variables updated in both the preliminary step and iteration cycle are demonstrated. The detailed algorithm is shown in \textbf{Algorithm 4}.


\vspace{-2mm}
\section{simulation experiments}

Simulation experiments were conducted to evaluate the performance of the proposed co-rotational approach by comparing it with the finite-element analysis (FEA), which is an important benchmark solution for the numerical analysis of mechanical models \cite{Force_FR}. \textcolor{Purple}{Especially for Fin-Ray grippers, research have proven its high accuracy compared with physical experiments, with an average error of around 3\%  \cite{ref4, ref6}}. Here a given compliant finger is meshed by rigid nodes, corresponding to the physical gripper in real scenarios, and detailed parameters are mentioned in Table I.
\begin{table}[thbp]
\vspace{-2mm}
\caption*{\small{Table \uppercase\expandafter{\romannumeral1}. The parameters of the sparsely and densely meshed models.}}
\vspace{-4mm}
\label{Table:1}
\begin{center}
\begin{threeparttable}[b]
\setlength{\tabcolsep}{1.8mm}{
\begin{tabular}{ccc}
\hline
Items & Sparse & Dense \\ \hline
Node number & $30$ & $66$ \\
Member & $38$ & $74$ \\
Width $m$(m) & $40e^{-3}$ & $40e^{-3}$ \\
Height $n$(m) & $80e^{-3}$ & $80e^{-3}$ \\
Node radius $R_{node}$(m) & $0.75e^{-3}$ & $0.75e^{-3}$ \\
Node radius modification factor & $1$ & $0.5$ \\
Cross section of each member $(b,h)$(m) & $20e^{-3},1e^{-3}$ & $20e^{-3},1e^{-3}$ \\
Young's modulus E (Pa) & $2e^7$ & $2e^7$ \\ \hline
\end{tabular}

}
% \begin{tablenotes}
%      \item \footnotesize{Dis$.^1$ denotes the displacement; Ave$.^2$ indicate the average value; N.A$.^3$ represents that the FE method is invalid.}
   % \end{tablenotes}
\end{threeparttable}
\end{center}
\vspace{-7mm}
\end{table}
 We  apply a pre-defined displacement load at the specified nodes (8 and 9) and estimate contact forces utilizing the proposed co-rotational theory. Similaly, the displacement load is exerted on the same finger using FEA. Fig. \ref{fig:6} reveals a good agreement between them. 
% Following the mathematical modeling, we conduct simulation experiments to evaluate the performance of proposed co-rotational approach via the comparison between the proposed model and the finite-element analysis (FEA), as=0 shown in Fig.7. FEA is considered as an important benchmark solution for the numerical analysis of mechanical models. Here a given compliant finger is meshed by nodes. Indeed, these nodes modeled as rigid elements are in accordance with the physical gripper in real scenarios. 
% % We indirectly conduct the comparison between the proposed method and FEA simulation in estimating contact force since FEA simulation cannot read contact force. 
% Specifically, contact forces can be first achieved by the proposed co-rotational approach; then, the calculated forces are exerted on the same soft finger, which allows the finger to generate displacement; finally, displacements from two methods are compared.  In terms of displacement estimation, we directly compare our method with FEA simulation since FEA can record displacements when a finger undertakes external forces. 
The rest of the section is organized as follows. 
\begin{enumerate}
\item The effects of two representative co-rotational modeling parameters, such as the mesh density parameter and node radius factor, on the model’s accuracy are investigated. 
\item We further evaluate the performance of the co-rotational approach in estimating contact forces via the proposed displacement control. 
% \item The optimization design of a soft gripper is conducted in terms of a series of key design parameters, such as the top angle of a gripper, the number and inclination angle of crossbeams, as well as the connection type between crossbeams and front \& back fin-rays.
\end{enumerate}

\begin{table*}[thbp]
\vspace{-0mm}
\renewcommand\arraystretch{0.9}
\caption*{\small{Table \uppercase\expandafter{\romannumeral2}. Results of the evaluations of single-node force estimation. The values with blue background represent the average values with different nodes at the same displacement; the values with orange background denote the average values with the same nodes at different displacements; the green-background value is the total average value; the row with grey background has the invalid data (N.A.), which is not considered when an average value is calculated. Note that all the values are considered absolute values when average values are calculated.}}
\vspace{-1mm}
\label{Table:5}
\vspace{-2mm}
\begin{center}
\begin{threeparttable}[b]
\setlength{\tabcolsep}{1.4mm}{
\begin{tabular}{c|cccccccc|cccccccc}
\hline
 & Dis$.^1$ & 2mm & 4mm & 6mm & 8mm & 10mm &  &  & Dis$.^1$ & 2mm & 4mm & 6mm & 8mm & 10mm &  &  \\ \hline
 & Node & \multicolumn{5}{c}{Error ratio(\%)} & A$.^2$(\%) & \multicolumn{1}{l|}{SD$.^3$(\%)} & Node & \multicolumn{5}{c}{Error ratio(\%)} & A$.^2$(\%) & \multicolumn{1}{l}{SD$.^3$(\%)} \\
 & \cellcolor[HTML]{9B9B9B}$1^{st}$ & \cellcolor[HTML]{9B9B9B}-4 & \cellcolor[HTML]{9B9B9B}-13 & \cellcolor[HTML]{9B9B9B}-13 & \cellcolor[HTML]{9B9B9B}-10 & \cellcolor[HTML]{9B9B9B}N.A$.^3$ & \cellcolor[HTML]{9B9B9B}10 & \cellcolor[HTML]{9B9B9B}3.67 & $6^{th}$ & -4 & -4 & -4 & -5 & -5 & \cellcolor[HTML]{F9D7AE}4.4 & \cellcolor[HTML]{F9D7AE}0.49 \\
 & $2^{nd}$ & -2 & -3 & -4 & -5 & -6 & \cellcolor[HTML]{F9D7AE}4 & \cellcolor[HTML]{F9D7AE}1.41 & $7^{th}$ & -3 & -3 & -3 & -4 & -4 & \cellcolor[HTML]{F9D7AE}3.4 & \cellcolor[HTML]{F9D7AE}0.49 \\
 & $3^{rd}$ & -4 & -4 & -4 & -5 & -5 & \cellcolor[HTML]{F9D7AE}4.4 & \cellcolor[HTML]{F9D7AE}0.49 & $8^{th}$ & 0 & 0 & 0 & 0 & 0 & \cellcolor[HTML]{F9D7AE}0 & \cellcolor[HTML]{F9D7AE}0 \\
 & $4^{th}$ & -4 & -4 & -5 & -5 & -6 & \cellcolor[HTML]{F9D7AE}4.8 & \cellcolor[HTML]{F9D7AE}0.75 & \cellcolor[HTML]{9B9B9B}$9^{th}$ & \cellcolor[HTML]{9B9B9B}8 & \cellcolor[HTML]{9B9B9B}8 & \cellcolor[HTML]{9B9B9B}9 & \cellcolor[HTML]{9B9B9B}11 & \cellcolor[HTML]{9B9B9B}N.A$.^3$ & \cellcolor[HTML]{9B9B9B}9 & \cellcolor[HTML]{9B9B9B}1.22 \\
\multirow{-6}{*}{Single-node} & $5^{th}$ & -4 & -4 & -5 & -5 & -5 & \cellcolor[HTML]{F9D7AE}4.6 & \cellcolor[HTML]{F9D7AE}0.49 & A$.^2$(\%) & \cellcolor[HTML]{ECF4FF}3.00 & \cellcolor[HTML]{ECF4FF}3.14 & \cellcolor[HTML]{ECF4FF}3.57 & \cellcolor[HTML]{ECF4FF}4.14 & \cellcolor[HTML]{ECF4FF}4.43 & \cellcolor[HTML]{D9F5B1}3.66 & \cellcolor[HTML]{D9F5B1}\textbackslash{} \\ \hline
\end{tabular}}
\begin{tablenotes}
     \item \footnotesize{Dis$.^1$ denotes the displacement; A$.^2$: the average value; N.A.$^3$ indicates that the FE method in Ansys fails; SD$.^3$: the standard deviation.}
   \end{tablenotes}
\end{threeparttable}
\end{center}
\vspace{-6mm}
\end{table*}

\vspace{-1mm}
\subsection{Co-rotational Modelling Parameter Comparison}
\vspace{-1mm}
\subsubsection{Comparison in terms of the mesh density}

\ 

We first examine the effect of the number of meshed nodes on the performance of the proposed co-rotational approach. In the sparse model, we discretize the front/rear beam of a finger into nine flexible elements using ten nodes and the crossbeam into two flexible elements with one node (seen in Fig. \ref{fig:7}), generating a total of 30 nodes. By contrast, the dense model adds one additional node in the flexible element of the sparse model, resulting in altogether 68 nodes. 

The accuracy of both sparse and dense models in estimating forces is examined. For each model, nine nodes that physically exist on the front beam of the soft finger are selected, and a displacement load ranging from 2mm to 10mm is applied. Fig. \ref{Fig:9} compares and illustrates the error ratios of both models. Generally, two models own similar error ratios at intermediate nodes (3,4,5,6,7) at various displacement loads except for the 10mm case. While, the proposed method provides poorer performance at bordering nodes (1,2 and 8,9) regardless of the load magnitudes. However, since the most commonly used nodes are 3,4,5,6,7 in practical grasping scenarios, the influence of these negative results at node 1,2,8,9 can be minimized. It is worth noting that the dense model shows more consistent error ratios compared to the sparse model regarding the varying displacement-load magnitudes at various nodes. Therefore, the co-rotational dense-meshed model is employed in the following simulations for its better performance. 

\vspace{0mm}
\subsubsection{Comparison of the node radius factor}

\ 

The node radius affects the accuracy of the co-rotational model in estimating contact forces. Inheriting the parameters set in the above dense-meshed model, we set three factors, 0.7, 0.6, and 0.5, to adjust the node radius in the simulation. The node radius affects the effective length of each beam. In particular, larger radium indicates a smaller effective length, resulting in a larger structural stiffness. Since the displacements of FEA simulation and co-rotational approach are compared at each node, the mathematical model with a larger node radius will have a smaller displacement at each node, thus the error line is higher (error line is the combination of error ratios at nine nodes). For the displacements from 2 to 10mm with the 2mm interval, the proposed co-rotational model with a radius factor of 0.7 has better performance than others, as shown in Fig. \ref{Fig:10}. 

\begin{figure}[H]
% \vspace{-2mm}
	\centering
 \vspace{-3mm}
	\begin{minipage}[t]{\linewidth}
		\centering
		\includegraphics[width=0.75\columnwidth]{Fig/3.png}
  \vspace{-5mm}
		\caption{\small{A comparison between the co-rotational modeling and FEA simulation. In this case, external displacement loads are applied at nodes 8 and 9. The deformation shape of the soft finger is described in the red line, and the FEA simulation result of key nodes is described in the blue circle, which shows a good agreement.}}
		\label{fig:6}
  \vspace{0mm}
	\end{minipage}
	\\
	\begin{minipage}[t]{\linewidth}
		\centering
		\includegraphics[width=0.75\columnwidth]{Fig/4.png}
  \vspace{-3mm}
		\caption{\small{The models meshed sparsely (A) and densely (B). m and n represent the width and height of a fin-ray finger, respectively. The numbers indicate the labels of nodes, and the numbers in brackets denote the labels of flexible elements. The red frame with two nodes, a and b, represents the sparse connection style (A), where the bold circle indicates a physically existing node. For the dense connection style, an intermediate node c is introduced, which is a virtual node used in the analytical model.}}
		\label{fig:7}
  \vspace{-2.5mm}
	\end{minipage}

\end{figure}


\begin{figure*}[thbp]
\centerline{\includegraphics[width=1.82\columnwidth]{Fig/5.png}}
\vspace{-3mm}
\caption{\small{The models meshed sparsely (blue) and densely (red) in terms of mesh density.}}
\label{Fig:9}
\vspace{-2mm}
\end{figure*}
\begin{figure*}[thbp]
\vspace{-2mm}
\centerline{\includegraphics[width=1.82\columnwidth]{Fig/6.png}}
\vspace{-3mm}
\caption{\small{The error ratio of the densely meshed model with three node radius factors.}}
% \caption{\small{The models meshed densely for the node radius factors.}}
\label{Fig:10}
\vspace{-1mm}
\end{figure*}

% \begin{figure}[htbp]
% \vspace{-2mm}
% \centerline{\includegraphics[width=0.95\columnwidth]{Fig/The deformations of a gripper for single-node displacement.png}}
% \caption{\small{The deformations of a gripper for single-node displacement loads. Displacements at nodes 1(A), 2(B), 3(C), 4(D), 5(E), 6(F), 7(G), 8(H), 9(I).}}
% \label{Fig:11}
% \vspace{-8mm}
% \end{figure}

\begin{table*}[t]
\vspace{-2mm}
\renewcommand\arraystretch{0.8}
\caption*{\small{Table \uppercase\expandafter{\romannumeral3}. Results of the evaluations of two-node force estimation. The color notations are the same as Table II. The positive error ratios for node 9 are considered negative when the corresponding average values are calculated, as highlighted by the deep blue color.}}
\vspace{-2mm}
\label{Table:6}
\begin{center}
\begin{threeparttable}[b]
\setlength{\tabcolsep}{2.0mm}{
\begin{tabular}{c|ccccccccccccc}
\hline
 & Dis$.^1$(mm) & \multicolumn{2}{c}{(2,2)} & \multicolumn{2}{c}{(4,4)} & \multicolumn{2}{c}{(6,6)} & \multicolumn{2}{c}{(8,8)} & \multicolumn{2}{c}{(10,10)} &  &  \\ \hline
 & Nodes & \multicolumn{10}{c}{Error ratio(\%)} & A-Ave$.^2$(\%) & SD$.^4$(\%) \\
 & 2+3 & -1 & -6 & -4 & -3 & -4 & -5 & -6 & -6 & -6 & -7 & \cellcolor[HTML]{F9D7AE}4.80 & \cellcolor[HTML]{F9D7AE}1.72 \\
 & 4+5 & -8 & -6 & -6 & -4 & -4 & -6 & -5 & -6 & -6 & -6 & \cellcolor[HTML]{F9D7AE}5.70 & \cellcolor[HTML]{F9D7AE}1.10 \\
 & 6+7 & -3 & -4 & -3 & -5 & -5 & -3 & -5 & -4 & -6 & -5 & \cellcolor[HTML]{F9D7AE}4.30 & \cellcolor[HTML]{F9D7AE}1.00 \\
 & \cellcolor[HTML]{9B9B9B}8+9 & \cellcolor[HTML]{9B9B9B}-3 & \cellcolor[HTML]{9B9B9B}3 & \cellcolor[HTML]{9B9B9B}-1 & \cellcolor[HTML]{9B9B9B}6 & \cellcolor[HTML]{9B9B9B}-1 & \cellcolor[HTML]{9B9B9B}5 & \cellcolor[HTML]{9B9B9B}0 & \cellcolor[HTML]{9B9B9B}7 & \cellcolor[HTML]{9B9B9B}N.A$.^3$ & \cellcolor[HTML]{9B9B9B}N.A$.^3$ & \cellcolor[HTML]{9B9B9B}3.25 & \cellcolor[HTML]{9B9B9B}2.38 \\
\multirow{-6}{*}{Two-adjacent nodes} & A-Ave$.^2$(\%) & \cellcolor[HTML]{ECF4FF}3.75 & \cellcolor[HTML]{72B8EE}4.75 & \cellcolor[HTML]{ECF4FF}3.50 & \cellcolor[HTML]{72B8EE}4.50 & \cellcolor[HTML]{ECF4FF}3.50 & \cellcolor[HTML]{72B8EE}4.75 & \cellcolor[HTML]{ECF4FF}4.00 & \cellcolor[HTML]{72B8EE}5.75 & \cellcolor[HTML]{ECF4FF}6.00 & \cellcolor[HTML]{ECF4FF}6.00 & \cellcolor[HTML]{D9F5B1}4.51 & \cellcolor[HTML]{D9F5B1}\textbackslash{} \\ \hline
 & Dis$.^1$(mm) & \multicolumn{2}{c}{(2,10)} & \multicolumn{2}{c}{(4, 8)} & \multicolumn{2}{c}{(6, 6)} & \multicolumn{2}{c}{(8, 4)} & \multicolumn{2}{c}{(10, 2)} &  &  \\ \hline
 & Nodes & \multicolumn{10}{c}{Error ratio(\%)} & A-Ave$.^2$(\%) & SD$.^4$(\%) \\
 & 2+8 & -6 & -1 & -5 & -2 & -6 & -2 & -5 & -4 & -5 & -6 & \cellcolor[HTML]{F9D7AE}4.20 & \cellcolor[HTML]{F9D7AE}1.78 \\
 & 4+7 & -5 & -4 & -6 & -4 & -5 & -4 & -5 & -5 & -5 & -6 & \cellcolor[HTML]{F9D7AE}4.9 & \cellcolor[HTML]{F9D7AE}0.7 \\
\multirow{-4}{*}{Two-non-adjacent nodes} & A-Ave$.^2$(\%) & \cellcolor[HTML]{ECF4FF}5.50 & \cellcolor[HTML]{ECF4FF}2.50 & \cellcolor[HTML]{ECF4FF}5.50 & \cellcolor[HTML]{ECF4FF}3.00 & \cellcolor[HTML]{ECF4FF}5.50 & \cellcolor[HTML]{ECF4FF}3.00 & \cellcolor[HTML]{ECF4FF}5.00 & \cellcolor[HTML]{ECF4FF}4.50 & \cellcolor[HTML]{ECF4FF}5.00 & \cellcolor[HTML]{ECF4FF}6.00 & \cellcolor[HTML]{D9F5B1}4.55 & \cellcolor[HTML]{D9F5B1}\textbackslash{} \\ \hline
\end{tabular}}
\begin{tablenotes}
     \item \footnotesize{Dis$.^1$ denotes the displacement; A$.^2$ indicates the average value; N.A.$^3$ indicates that the FE method in Ansys fails; SD$.^4$ indicates the standard deviations.}
   \end{tablenotes}
\end{threeparttable}
\end{center}
\vspace{-6mm}
\end{table*}


\vspace{-5mm}
\subsection{Contact Force Estimation}
\vspace{-1mm}
The proposed approach can estimate the applied contact forces by the developed displacement-force modeling, \textbf{Algorithm 4}. In addition, the parameters of the soft gripper are shown in Table I, except for the node radius modification factor of 0.7 based on the above analysis. 
\subsubsection{Evaluation of the single-node cases}

\ 

% \begin{figure*}[thbp]
% \centerline{\includegraphics[width=1.8\columnwidth]{Fig/The deformations of a gripper for two-adjacent-node displacement loads.png}}
% \caption{\small{The deformations of a gripper for two-adjacent-node displacement loads.}}
% \label{Fig:12}
% \vspace{-3mm}
% \end{figure*}
% \begin{figure*}[thbp]
% \centerline{\includegraphics[width=1.9\columnwidth]{Fig/The deformations of a gripper for two-non-adjacent-node displacement loads.png}}
% \caption{\small{The deformations of a gripper for two-non-adjacent-node displacement loads at nodes 3, 9 (A) and 5,8. }}
% \label{Fig:13}
% \vspace{-5mm}
% \end{figure*}

Here we compare the proposed method's ability to estimate force at single node. Each node is subjected to a horizontal external displacement load, as indicated in Fig. \ref{Fig:11}. A 10mm-displacement can be considered a large deformation with respect to the dimension of the gripper. For nodes 1 and 9, since the FEA simulation fail in estimating forces given the 10mm displacements, the maximum applied displacement is set to be 8mm (see Table II). The reaction force obtained from the proposed method is applied to an FEA simulation with 14278 elements and large deflection, and the resulting displacement is compared with the original displacement load from the mathematical model. For a quantitative comparison, Table II indicates the rates of displacement discrepancies/errors from compared results predicted by the proposed model and the FEA method, respectively. The overall average estimation error rate is 3.66\%. For the dense mesh model, displacement discrepancies are small \textcolor{DarkOrange}{(around 3\%-4\%, SD around 1\%)} for displacements between 2mm to 10mm with 2mm intervals, indicating that the proposed method can accurately analyze the displacement-to-force (estimate the contact forces from displacements) of a fin-ray finger. Surprisingly, the displacement discrepancy does not appear to be roughly proportional to the displacement at a node, suggesting the mathematical model can predict the large deformation without sacrificing accuracy. 

\begin{figure}[htbp]
\vspace{-3mm}
\centerline{\includegraphics[width=0.95\columnwidth]{Fig/7.png}}
\caption{\small{The deformations of a gripper for single-node displacement loads. Displacements at nodes 1(A), 2(B), 3(C), 4(D), 5(E), 6(F), 7(G), 8(H), 9(I).}}
\label{Fig:11}
\vspace{-6mm}
\end{figure}

\begin{figure*}[thbp]
\centerline{\includegraphics[width=1.7\columnwidth]{Fig/8.png}}
\caption{\small{The deformations of a gripper for two-adjacent-node displacement loads.}}
\label{Fig:12}
\vspace{-3mm}
\end{figure*}
\begin{figure*}[thbp]
\centerline{\includegraphics[width=1.65\columnwidth]{Fig/9.png}}
\caption{\small{The deformations of a gripper for two-non-adjacent-node displacement loads at nodes 3, 9 (A), and 5, 8 (B). }}
\label{Fig:13}
\vspace{-5mm}
\end{figure*}

\textcolor{Blue}{The FEA simulation in Ansys fail in estimating forces at the 10mm displacements at node 1 and node 9 (see Table II).} By contrast, the proposed model well handles these cases. Actually, error ratios at these nodes are not considered when calculating average values, as they are not usually used for contacting objects in gripper with fin-ray structure.  They differ slightly among most displacement-force cases. In particular, these error ratios are very close, within 6\% (Absolute value). Note that the proposed model demonstrates excellent consistency in terms of different positions and displacements. 
In particular, from node 3 to node 7, there are small fluctuations in displacement discrepancies at the same displacement. Moreover, displacements are gradually increased from 2mm to 10mm for all the nodes at the front beam; however, almost no obvious effect is generated on changing the displacement discrepancies/errors at the same node. At node 8, the proposed model illustrates an excellent performance with almost zero errors. 

% \begin{figure}[b]
% \vspace{-8mm}
% \centerline{\includegraphics[width=0.95\columnwidth]{Fig/The deformation of a gripper for three-non-adjacent-node.png}}
% \caption{\small{The deformations of a gripper for three-adjacent-node displacement loads.}}
% \label{Fig:14}
% \vspace{-2mm}
% \end{figure}


\subsubsection{Evaluation of more than one-node cases}

\ 

(1) Displacements at two nodes

To evaluate the performance of the proposed method in estimating contact forces at two nodes, we conduct experiments focusing on the two-adjacent-node and two-non-adjacent-node cases. The setup of the simulations is similar to the above subsection, except that those two horizontal displacements are applied simultaneously at two selected nodes, as shown in Fig. \ref{Fig:12} and Fig. \ref{Fig:13}. Statistical indicators of the force estimations from both the proposed model and the FEA method are illustrated in Table III. Therefore, it is reasonable to conclude that the proposed model can utilize finger deformations to accurately predict the contact forces when applied at two nodes. It is found that the proposed model can handle non-convergence at node 9 with 10mm displacement, while the FEA method fails, indicating the advantage of the proposed model over the FEA method. 

\begin{figure}[t]
\vspace{-0mm}
\centerline{\includegraphics[width=\columnwidth]{Fig/10.png}}
\caption{\small{The deformations of a gripper for three-adjacent-node displacement loads.}}
\label{Fig:14}
\vspace{-8mm}
\end{figure}

Regardless of two-adjacent-node and two-non-adjacent-node cases, the proposed method demonstrates a similar performance to the single-node case on the estimation accuracy, mostly maintaining around 4\%-6\% (absolute value) estimation errors for different positions and displacements, \textcolor{DarkOrange}{and SD is around 1\%.} The reason is that the proposed model considers stiffness rather than a specific position. In terms of the two-adjacent-node cases, the corresponding average error ratios from the first group to the third group (i.e., from base to tip) are around 5\% (see Fig. \ref{Fig:12} and Table III). 
% Note that it is unavailable for the FEA method to predict the contact force applied at node 8 and node 9 for the 10mm displacement.
Moreover, in terms of two-non-adjacent-node cases, the discrepancy/error ratios at nodes (2 and 8) and (4 and 7) are less than 6\% (see Fig. \ref{Fig:13} and Table III). To sum up, the proposed approach is capable of sensing contact forces exerted on its middle part but has a poor performance when the contact points are near the tip or base, especially the tip. Moreover, the nonlinearity of the compliant finger results in a systemic error. For instance, the calibrated Young’s modulus has a difference from the real one, which is a potential reason why the proposed model has a poor performance in estimating contact forces at the finger base and tip. 

(2) Displacements at three nodes

We test the proposed approach in estimating forces at multiple nodes, focusing on three-node cases, including adjacent and non-adjacent nodes. The simulation comparison is similar, where three pre-defined horizontal displacement loads are applied at selected nodes simultaneously, and their reaction forces are recorded, as illustrated in Fig. \ref{Fig:14} and Fig. \ref{Fig:15}.
% The force values are then passed to the standard FEA simulation, and the resulting deformation values are recorded, and their discrepancies with analytical results are compared.
The comparison result suggests the average sensing error ratios are around 5\%, \textcolor{DarkOrange}{the SD within 3\%,} which illustrates that the proposed model can well estimate forces at multiple nodes (see Table IV). Moreover, for force estimation on different positions and displacements, the proposed model shows excellent consistency, which is similar to the single-node cases.

The proposed model presents an unstable phenomenon in the first three-adjacent-node case, with the displacements gradually increasing. The high sensing error ratios at node 1 of the 8mm and 10mm displacements exceed 10\% (absolute value). While for the middle parts (node 3 to node 9), the average error ratios are within 5\%. One possible reason is that the joint points of the crossbeam and the beam have almost unchanged deformations at the finger base. \textcolor{blue}{To complete a precise pinch, high accuracy at the fingertip is necessary. However, due to the compliant nature of these grippers, they are more employed to compliantly envelop  the target objects, with the middle nodes first making contact and the peripheral nodes adaptively approaching to complete the grasp.} In practical applications, inaccurate estimations at the base of the finger, where contact with the object is infrequent, will not significantly affect the actual grasp prediction.
% Thus, inaccurate estimations at the compliant finger base do not generate a vital effect on practical grasp applications.
The estimating capability becomes better for estimating forces at three-non-adjacent-node cases without exceeding 5\%, as shown in Table IV. 
\textcolor{Yellow}{The overall average error for both adjacent and non-adjacent nodes is within 5\%, compared to 8\% reported by Xu\cite{ref6}}. 
The study demonstrates that the proposed model maintains good accuracy in estimating the contact forces with respect to large objects.

\begin{figure*}[th]
\vspace{-1mm}
\centerline{\includegraphics[width=1.8\columnwidth]{Fig/11.png}}
\vspace{-1mm}
\caption{\small{The deformations of a gripper for three-non-adjacent-node displacement loads.} }
\label{Fig:15}
\vspace{-3mm}
\end{figure*}

\begin{table*}[th]
\renewcommand\arraystretch{0.8}
\vspace{-1mm}
\caption*{\small{Table \uppercase\expandafter{\romannumeral4}. Results of the evaluations of three-node force estimation. The color notations are the same as Table III.}}
\vspace{-4mm}
\label{Table:7}
\begin{center}
\begin{threeparttable}[b]
\setlength{\tabcolsep}{2.1mm}{\begin{tabular}{c|cccccccccccccccccc}
\hline
 & Dis$.^1$(mm) & \multicolumn{3}{c}{(2,2,2)} & \multicolumn{3}{c}{(4,4,4)} & \multicolumn{3}{c}{(6,6,6)} & \multicolumn{3}{c}{(8,8,8)} & \multicolumn{3}{c}{(10,10,10)} &  & \multicolumn{1}{l}{} \\ \hline
 & Nodes & \multicolumn{15}{c}{Error ratio(\%)} & A-Ave$.^2$(\%) & \multicolumn{1}{l}{SD$.^5$(\%)} \\
 & 1+2+3 & -8 & -3 & -5 & -6 & -6 & -6 & -9 & -7 & -7 & -12 & -8 & -9 & -15 & -10 & -10 & \cellcolor[HTML]{F9D7AE}8 & \cellcolor[HTML]{F9D7AE}3 \\
 & 4+5+6 & -7 & -6 & -4 & -6 & -4 & -4 & -4 & -5 & -6 & -5 & -5 & -6 & -5 & -6 & -6 & \cellcolor[HTML]{F9D7AE}5 & \cellcolor[HTML]{F9D7AE}1 \\
 & 7+8+9 & -3 & 0 & 4 & -3 & 0 & 4 & -3 & 0 & 4 & -3 & -1 & 4 & -3 & -1 & 4 & \cellcolor[HTML]{F9D7AE}2 & \cellcolor[HTML]{F9D7AE}3 \\
\multirow{-5}{*}{3AN$^3$} & A-Ave$.^2$(\%) & \cellcolor[HTML]{ECF4FF}6 & \cellcolor[HTML]{ECF4FF}3 & \cellcolor[HTML]{72B8EE}4 & \cellcolor[HTML]{ECF4FF}5 & \cellcolor[HTML]{ECF4FF}3 & \cellcolor[HTML]{72B8EE}5 & \cellcolor[HTML]{ECF4FF}5 & \cellcolor[HTML]{ECF4FF}4 & \cellcolor[HTML]{72B8EE}6 & \cellcolor[HTML]{ECF4FF}7 & \cellcolor[HTML]{ECF4FF}5 & \cellcolor[HTML]{72B8EE}6 & \cellcolor[HTML]{ECF4FF}8 & \cellcolor[HTML]{ECF4FF}6 & \cellcolor[HTML]{72B8EE}7 & \cellcolor[HTML]{D9F5B1}5 & \cellcolor[HTML]{D9F5B1}\textbackslash{} \\ \hline
 & Dis$.^1$(mm) & \multicolumn{3}{c}{(2,6,10)} & \multicolumn{3}{c}{(4,6,8)} & \multicolumn{3}{c}{(6,6,6)} & \multicolumn{3}{c}{(8,6,4)} & \multicolumn{3}{c}{(10, 2)} &  & \multicolumn{1}{l}{} \\ \hline
 & Nodes & \multicolumn{15}{c}{Error ratio(\%)} & A-Ave$.^2$(\%) & \multicolumn{1}{l}{SD$.^5$(\%)} \\
 & 2+5+8 & -6 & -6 & -2 & -6 & -6 & -3 & -5 & -6 & -4 & -5 & -6 & -5 & -5 & -7 & -7 & \cellcolor[HTML]{F9D7AE}5 & \cellcolor[HTML]{F9D7AE}1 \\
 & 3+5+7 & -5 & -5 & -4 & -6 & -5 & -5 & -5 & -5 & -5 & -5 & -5 & -5 & -5 & -5 & -6 & \cellcolor[HTML]{F9D7AE}5 & \cellcolor[HTML]{F9D7AE}0 \\
\multirow{-4}{*}{3NAN} & A-Ave$.^2$(\%) & \cellcolor[HTML]{ECF4FF}6 & \cellcolor[HTML]{ECF4FF}6 & \cellcolor[HTML]{ECF4FF}3 & \cellcolor[HTML]{ECF4FF}6 & \cellcolor[HTML]{ECF4FF}6 & \cellcolor[HTML]{ECF4FF}4 & \cellcolor[HTML]{ECF4FF}5 & \cellcolor[HTML]{ECF4FF}6 & \cellcolor[HTML]{ECF4FF}5 & \cellcolor[HTML]{ECF4FF}5 & \cellcolor[HTML]{ECF4FF}6 & \cellcolor[HTML]{ECF4FF}5 & \cellcolor[HTML]{ECF4FF}5 & \cellcolor[HTML]{ECF4FF}6 & \cellcolor[HTML]{ECF4FF}7 & \cellcolor[HTML]{D9F5B1}5 & \cellcolor[HTML]{D9F5B1}\textbackslash{} \\ \hline
\end{tabular}}
\begin{tablenotes}
     \item \footnotesize{Dis$.^1$ denotes the displacement; A-Ave$.^2$ indicates around average values; 3AN$^3$ represents three-adjacent nodes; 3NAN$^4$ represents three-non-adjacent nodes; SD$.^5$ indicates the standard deviations.}
   \end{tablenotes}
\end{threeparttable}
\end{center}
\vspace{-7mm}
\end{table*}
\vspace{-2.5mm}
\section{Discussion}
\subsection{Advantages and limitations of the proposed Displacement-Force modeling}
Our proposed methods is not a substitute or supplement for existing methods. \textcolor{Purple}{The existing commercial FE software is designed to meet structural design optimization. Though it is good at detailed simulation, too much variables introduced would cause confusions for researchers and the key design index parameters are not intuitive. In addition, they have low iteration speed and is not suitable for application in the closed-loop control for the robot. By contrast, the proposed model is characterised with fair degree of accuracy and efficiency in displacement-force estimation. } 

\textcolor{Green}{Besides, the proposed method provides a general mathematical model depicting displacement-force relationship and can be further extended to more soft/rigid grippers. Especially, a soft/continuum gripper can be considered a Fin-Ray gripper with numerous crossbeams and low stiffness material for readily deforming. While a rigid gripper may be regarded as a Fin-Ray gripper with high-stiff material. Besides, unlike existing works, where a lot of assumptions are made to express axial/rotational deformation and the connection types, such as in Shan's model, they assume the crossbeams as inextensible beams and ignore their axial deformation\cite{ref4}. Thanks to the flexible nature of co-rotational methods, we could well handle the above constraints with fewer assumptions.}


\textcolor{Red}{Besides, the proposed model has the following limitations. The sources of differences between the different nodes can be ascribed to the following threefold aspects. First, the discretization degree based on the number of nodes meshing the model causes the estimation error of the stiffness. Second, the shapes of the crossbeams tend to be ‘S’-shaped when undergoing large deformations of the finger, which brings in inaccurate estimations for stiffness matrixes in the model. Third, the deformation at the base (node 1) of the beam structure of compliant lengths generates a significant deformation compared with other locations, but it is difficult to obtain a precision measurement.}

\subsection{Influence of the results on Physical Experiments}
\textcolor{Blue}{Although our experiments are simulated by FEA, the proposed model is able to be applied in a physical model. In FEA, we mentioned some physical parameters, such as the modification factor $\mu$, and Young's modulus, which not only have a large impact on the simulation results, but are also crucial in physics experiments. If physical experiments are performed, these physical parameters can be easily obtained from the physical model by measurement.}
\ 




\vspace{-2mm}
\section{conclusions}

In Part II of this article, detailed derivations of the displacement-force model have been devised to illustrate the intrinsic force-sensing principles behind the nonlinear deformation of grippers. We further explore the influence of modeling parameters on the algorithm's performance in accuracy, and results indicate that a denser mesh and a bigger node radius factor are preferred. Meanwhile, the comparison with FEA simulations validates that the proposed displacement-force model can accurately and efficiently estimate the loaded forces, with an average error $<5\%$, regardless of various design parameters. 

% Part I of this paper introduces the fundamental theory of the co-rotational approach, together with a force-displacement mapping providing gripper deformation under external forces.

To sum up, this paper innovatively devises, demonstrates, and experimentally verifies a universal theoretical model that mutually depicts the bidirectional relationship between the gripper's displacement and contact forces, facilitating both the optimization of gripper design and compliant grasping in a force-aware manner, sensor-free. Especially from the perspective of theoretical modeling, this work lays a solid foundation and provides a theoretical background for the development of more compliant grippers, providing a promising mathematical tool for the control of soft fin-ray grippers and represents a first step toward a more effective design of more compliant grippers.  

Future work will focus on the grasping experiments of the fin-ray gripper, where a gripper with two fin-ray-like fingers will be constructed, equipped with a camera that can capture shape deformations, to physically evaluate the performance and further validate the accuracy of the proposed model.
% This paper innovatively devises, demonstrates and experimentally verifies a universal theoretical model that mutually depicts the bidirectional relationship between gripper's displacement and contact forces, facilitating both the optimization of gripper design and compliant grasping in a force-aware manner, sensor-free. Part I of this paper introduces the fundamental theory of the co-rotational approach, together with a force-displacement mapping providing gripper deformation under external forces. Particularly in this part, detailed derivations of displacement-force model are implemented to illustrate the intrinsic force sensing principles behind the nonlinear deformation of grippers. We further explore the influence of modeling parameters on algorithm's performance in accuracy. Results of the simulation experiments have shown that the proposed displacement-force model can accurately and efficiently estimate the loaded forces, with an average error $<5\%$ compared with FEA simulations, regardless of various design parameters. From the perspectives of theoretical modeling, this work  lays a solid foundation and provides a theoretic background for the development of more compliant grippers, providing a promising mathematical tool for control of soft fin-ray grippers and represents a first step towards more effective design of more compliant grippers. 

\addtolength{\textheight}{+0.5cm}   % This command serves to balance the column lengths
\AtNextBibliography{\small}
\printbibliography
% \vfill

\end{document}


