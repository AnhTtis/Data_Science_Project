% mnras_template.tex 
%
% LaTeX template for creating an MNRAS paper
%
% v3.0 released 14 May 2015
% (version numbers match those of mnras.cls)
%
% Copyright (C) Royal Astronomical Society 2015
% Authors:
% Keith T. Smith (Royal Astronomical Society)

% Change log
%
% v3.0 May 2015
%    Renamed to match the new package name
%    Version number matches mnras.cls
%    A few minor tweaks to wording
% v1.0 September 2013
%    Beta testing only - never publicly released
%    First version: a simple (ish) template for creating an MNRAS paper

%%%%%%%%%%%%%%%%%%%%%%%%%%%%%%%%%%%%%%%%%%%%%%%%%%
% Basic setup. Most papers should leave these options alone.
\documentclass[fleqn,usenatbib]{mnras}

% MNRAS is set in Times font. If you don't have this installed (most LaTeX
% installations will be fine) or prefer the old Computer Modern fonts, comment
% out the following line
\usepackage{newtxtext,newtxmath}
% Depending on your LaTeX fonts installation, you might get better results with one of these:
%\usepackage{mathptmx}
%\usepackage{txfonts}

% Use vector fonts, so it zooms properly in on-screen viewing software
% Don't change these lines unless you know what you are doing
\usepackage[T1]{fontenc}

% Allow "Thomas van Noord" and "Simon de Laguarde" and alike to be sorted by "N" and "L" etc. in the bibliography.
% Write the name in the bibliography as "\VAN{Noord}{Van}{van} Noord, Thomas"
\DeclareRobustCommand{\VAN}[3]{#2}
\let\VANthebibliography\thebibliography
\def\thebibliography{\DeclareRobustCommand{\VAN}[3]{##3}\VANthebibliography}


%%%%% AUTHORS - PLACE YOUR OWN PACKAGES HERE %%%%%

% Only include extra packages if you really need them. Common packages are:
\usepackage[dvipdfmx]{graphicx}	% Including figure files
%\usepackage{graphicx}	% Including figure files
\usepackage{amsmath}	% Advanced maths commands
\usepackage{bm}  
%\usepackage{amssymb}	% Extra maths symbols
\usepackage{color}
\usepackage{ulem}

%%%%%%%%%%%%%%%%%%%%%%%%%%%%%%%%%%%%%%%%%%%%%%%%%%

%%%%% AUTHORS - PLACE YOUR OWN COMMANDS HERE %%%%%

% Please keep new commands to a minimum, and use \newcommand not \def to avoid
% overwriting existing commands. Example:
%\newcommand{\pcm}{\,cm$^{-2}$}	% per cm-squared
\definecolor{ao_en}{rgb}{0.0, 0.5, 0.0}
\newcommand{\adr}[1]{\textcolor{red}{#1}}
\newcommand{\adb}[1]{\textcolor{blue}{#1}}
\newcommand{\reff}[1]{\textcolor{blue}{#1}}
\newcommand{\adg}[1]{\textcolor{ao_en}{#1}}
\defcitealias{2021MNRAS.506.5512F}{Paper I}


%%%%%%%%%%%%%%%%%%%%%%%%%%%%%%%%%%%%%%%%%%%%%%%%%%

%%%%%%%%%%%%%%%%%%% TITLE PAGE %%%%%%%%%%%%%%%%%%%

% Title of the paper, and the short title which is used in the headers.
% Keep the title short and informative.
\title[GC formation with top-heavy IMF]{
The formation of globular clusters with top-heavy initial mass functions
}

% The list of authors, and the short list which is used in the headers.
% If you need two or more lines of authors, add an extra line using \newauthor
\author[H.Fukushima and H.Yajima]{
Hajime Fukushima$^{1}$\thanks{E-mail:fukushima@ccs.tsukuba.ac.jp},
Hidenobu Yajima$^{1}$
\\
% List of institutions
$^{1}$Center for Computational Sciences, University of Tsukuba, Ten-nodai, 1-1-1 Tsukuba, Ibaraki 305-8577, Japan\\
}
% These dates will be filled out by the publisher
\date{Accepted XXX. Received YYY; in original form ZZZ}

% Enter the current year, for the copyright statements etc.
\pubyear{2020}


% Don't change these lines
\begin{document}
\label{firstpage}
\pagerange{\pageref{firstpage}--\pageref{lastpage}}
\maketitle

% Abstract of the paper
\begin{abstract}
We study the formation of globular clusters in massive compact clouds with the low-metallicity of $Z=10^{-3}~Z_{\odot}$ by performing three-dimensional radiative-hydrodynamics simulations. 
Considering the uncertainty of the initial mass function (IMF) of stars formed in low-metallicity and high-density clouds, we investigate the impacts of the IMF on the cloud condition for the GC formation with the range of the power-law index of IMF as $\gamma = 1-2.35$. We find that the threshold surface density ($\Sigma_{\rm thr}$) for the GC formation increases from $800~M_{\odot} \; {\rm pc^{-2}}$ at $\gamma = 2.35$ to $1600~M_{\odot}\; {\rm pc^{-2}}$ at $\gamma = 1.5$ in the cases of clouds with $M_{\rm cl} = 10^6~M_{\odot}$ because the emissivity of ionizing photons per stellar mass increases as $\gamma$ decreases. For $\gamma 
 < 1.5$, $\Sigma_{\rm thr}$ saturates with $\sim 2000~M_{\odot}\; {\rm pc^{-2}}$ that is quite rare and observed only in local starburst galaxies due to e.g., merger processes.
Thus, we suggest that formation sites of low-metallicity GCs could be limited only in the very high-surface density regions. We also find that $\Sigma_{\rm thr}$ can be modelled by a power-law function with the cloud mass ($M_{\rm cl}$) and the emissivity of ionizing photons ($s_*$) as $\propto M_{\rm cl}^{-1/5} s_{*}^{2/5}$.
Based on the relation between the power-law slope of IMF and $\Sigma_{\rm thr}$, future observations with e.g., the James Webb Space Telescope can allow us to constrain the IMF of GCs.
\end{abstract}

% Select between one and six entries from the list of approved keywords.
% Don't make up new ones.
\begin{keywords}
stars:formation - stars:massive - stars:Population II - H{\sc ii} regions - galaxies: star clusters: general - galaxies: star formation.
\end{keywords}

%%%%%%%%%%%%%%%%%%%%%%%%%%%%%%%%%%%%%%%%%%%%%%%%%%

%%%%%%%%%%%%%%%%% BODY OF PAPER %%%%%%%%%%%%%%%%%%
\section{Introduction}\label{introduction} 
Globular clusters (GCs) are survivors of ancient star clusters born in the early Universe from the estimated age $\gtrsim 10~{\rm Gyr}$ \citep[e.g.,][]{2010ARA&A..48..431P}.
GCs can be keys for understanding cosmic star formation rate and galaxy evolution in the early Universe.
However, the formation mechanism of GCs has not been understood yet. 
Previous studies with cosmological simulations showed that massive and dense star clusters are formed in the early galaxies \citep{2016ApJ...831..204R, 2018MNRAS.475.4252A, 2021MNRAS.500.4578M}.
In observations, \citet{2017MNRAS.467.4304V} discovered the young stellar system heavier than $10^5~M_{\odot}$ at $z\sim 6$ \citep[see also][]{2019MNRAS.483.3618V}.
Very recently, James Webb Space Telescope (JWST) has allowed us to probe massive and compact star clusters more precisely. 
\citet{2022arXiv220800520V} discovered the star cluster whose stellar mass is comparable to GCs ($\sim 10^6~M_{\odot}$) at $z\sim 4$.





In the GC formation, efficient conversion from the gas to stars is a necessary condition. After evaporation of a star-forming cloud, the stellar density decreases significantly if the stellar system is gravitationally unbound.
$N$-body simulations showed that 
more than 10-30 percent of the gas should be converted into stars to make the star cluster gravitationally bound state 
\citep[e.g.,][]{1984ApJ...285..141L, 2001MNRAS.321..699K, 2007MNRAS.380.1589B, 2017A&A...605A.119S}.
Recent hydrodynamics simulations show that the bound fractions of star clusters are tightly related to the star formation efficiencies (SFEs)  \citep{2019MNRAS.487..364L, 2021MNRAS.506.3239G, 2022MNRAS.511.3346F}.
In the Milky Way, the SFEs are typically less than 10 percent \citep[e.g.,][]{1986ApJ...301..398M, 2020MNRAS.493.2872C}, and so most star clusters are dispersed after their birth \citep{2003ARA&A..41...57L}.

In star cluster formation, massive stars disrupt their host clouds with stellar feedback, such as radiative feedback, stellar wind, and supernovae (SNe) \citep[e.g.,][]{2019ARA&A..57..227K, 2022arXiv220309570C}.
Extreme ultraviolet (EUV; $13.6~{\rm eV} \lesssim h \nu \lesssim 1~{\rm keV}$ ) photons ionize the hydrogen atoms and heat gas to $10^4~{\rm K}$ of which the high thermal pressure can disrupt clouds
\citep[e.g.,][]{1997ApJ...476..166W, 2002ApJ...566..302M, 2009ApJ...703.1352K, 2010ApJ...710L.142F, 2016ApJ...819..137K, 2020MNRAS.497.5061I}.
This photoionization feedback can regulate the SFEs to be less than $\sim 0.1$ in the environments of the Milky Way
\citep[e.g.,][]{2010ApJ...715.1302V, 2012MNRAS.427.2852D, 2013MNRAS.430..234D, 2017MNRAS.470.3346H,2017MNRAS.471.4844G, 2017MNRAS.472.4155G, 2018ApJ...859...68K, 2019MNRAS.489.1880H, 2020MNRAS.497.4718D, 2018MNRAS.475.3511G, 2020MNRAS.497.3830F, 2021MNRAS.506.3239G, 2020MNRAS.499..668G,2020MNRAS.495.1672B, 2021MNRAS.501.4136A, 2021PASJ..tmp...65F, 2022MNRAS.509..954D, 2022ApJS..259...21K, 2022MNRAS.512..216G}.
Stellar wind also injects kinetic energy into the ambient matters \citep{1977ApJ...218..377W}.
However, \citet{2021ApJ...914...89L} showed that the mixing layers around the bubble efficiently lost energy, resulting in the weak impacts on the cloud disruption \citep[see also][]{2021ApJ...914...90L, 2021ApJ...922L...3L}.
Supernova feedback is powerful enough to evacuate the gas from a cloud \citep[e.g.,][]{2016MNRAS.463.3129G}.
Yet, recent observations indicate that clouds are disrupted before the end of the lifetime of OB stars \citep{2019Natur.569..519K}.
Thus, photoionization is likely to play a primary role in the star cluster formation \citep{2022arXiv220510413G}.
Note that the radiation pressure on dust grains also regulates the star formation \citep{2010ApJ...709..191M, 2015ApJ...809..187S, 2016ApJ...829..130R}.
However, it should be secondary because of the low-metallicities of clouds hosting GCs.


In understanding the relationship between the photoionization feedback and the star cluster formation, the initial mass function (IMF) is an essential factor. The production rate of ionizing photons sensitively depends on the shape of the IMF.
It is well known that the IMF in the nearby star-forming regions is universal \citep[e.g.][]{1955ApJ...121..161S, 2002Sci...295...82K, 2003PASP..115..763C}.  
Observations of massive star clusters indicated that the IMF is hop-heavy in these highly dense clusters \citep{2013ApJ...764..155L, 2016MNRAS.458.3027M, 2018Sci...359...69S, 2019ApJ...870...44H}.
The observed slopes of the IMFs are $\gamma = 1.7-1.9$, and there is an excess of massive stars.
\citet{2022arXiv220303276P} showed that the core mass function is shallower than the Salpeter IMF in the massive star-forming regions ($\gamma = 1.95$, W43-MM2\&MM3). 
Besides, indirect evidence of the top-heavy IMF in GCs is also found \citep{2012MNRAS.422.2246M}.
They found that the IMF of GCs with low-metallicities or high-stellar densities tends to be top-heavy. 
Using their results, \citet{2016ApJ...826...89Z} showed that the top-heavy IMF could nicely explain the relation between the mass-to-light ratios and metallicity of GCs in M31 \citep[see also,][]{2017ApJ...839...60H}.
On the other hand, \citet{2023arXiv230301636B} indicated that there is no evidence of top-heavy IMFs in GCs of the Galaxy and the Large/Small Magellanic Clouds in the metallicity range of $Z>10^{-2}~Z_{\odot}$.
Therefore, the IMF of low-metallicity GC is still under debate.
In particular, the case with a metallicity lower than $\sim 10^{-2}~Z_{\odot}$  is poorly understood.


Recently, \citet{2021MNRAS.508.4175C} performed hydrodynamics simulations, including the cooling effects of metal lines and dust grains.
They showed that the highly filamentary structures are induced, and the Chabrier-like IMF \citep{2003PASP..115..763C} is realized at $Z \gtrsim 10^{-1}~Z_{\odot}$.
In the low-metallicity environments with $Z\lesssim 10^{-2}~Z_{\odot}$, the formation of the filamentary structures is suppressed, resulting in the top-heavy IMF. 
Also, \citet{2022MNRAS.512.2573T} indicated that the IMF should be top-heavy with $\gamma < 1 $ around the active galactic nuclei to reproduce the observed Fe {\sc ii} / Mg {\sc ii} line-flux ratio.
Thus, the IMF is likely top-heavy in the low metallicity and the high gas density environments.




In this paper, we study the conditions of clouds for GC formation with top-heavy IMFs. We consider clouds with
 masses of $M_{\rm cl} = 10^6$ and $10^7~M_{\odot}$ and the cloud surface densities of $\Sigma_{\rm cl} \sim 400-2000~M_{\odot}{\rm pc^{-2}}$.
We utilize the 3D radiative-hydrodynamics (RHD) simulation code, \textsc{SFUMATO-M1} which is the modified code of a self-gravitational magnetohydrodynamics code with an Eulerian adaptive mesh refinement (AMR), \textsc{sfumato} \citep{2007PASJ...59..905M, 2015ApJ...801...77M}.
We adopt the radiation transfer scheme based on the momentum method with the M1-closure and the stochastic stellar population developed in \citet[][hereafter \citetalias{2021MNRAS.506.5512F}]{2021MNRAS.506.5512F} and \citet{2022MNRAS.511.3346F}.

We organize the rest of this paper as the following.
In Section \ref{Sec_simulation_method}, we describe the numerical method and the initial conditions of the simulations.
Then, we show the results of the simulations in Section \ref{Sec_results}.
In Section \ref{Sec:discussion}, we discuss the implications of our simulations on the low-metallicity GC formation.
Section \ref{Sec:summary} is for the summary.



\section{Simulation Method}\label{Sec_simulation_method}

We perform the RHD simulations with {\sc SFUMATO-M1} \citepalias{2021MNRAS.506.5512F}, the modified version of self-gravitational magnetohydrodynamics code {\sc sfumato} \citep{2007PASJ...59..905M} which utilizes the adaptive mesh refinement technique.
Our simulations include on-the-fly radiative transfer calculations with the M1-closure technique.
As in \citet{2013MNRAS.436.2188R}, we adopt the approximation with the reduced speed of light for the radiation transfer with $\tilde{c} = 3 \times 10^{-4} c$ where $c$ is the speed of light.
We set the four frequency bins for (1) extreme ultraviolet ($13.6~{\rm eV}< h\nu$), (2) Lyman-Werner ($11.2~{\rm eV} < h \nu < 13.6~{\rm eV}$), (3) far-ultraviolet ($6~{\rm eV} < h \nu < 13.6~{\rm eV}$), and (4) infrared photons.
The chemistry solver is developed in \citet{2020ApJ...892L..14S} which studied the primordial star formation.
We added the chemical network of CO and the oxygen ion in the H{\sc ii} regions \citep[O{\sc ii} and O{\sc iii},][]{2020MNRAS.497..829F}.
We also adopt the sink particle technique for modelling a star cluster as in \citet{2015ApJ...801...77M}.
Further detail of the simulation methods is described in \citetalias{2021MNRAS.506.5512F} and \citet{2022MNRAS.511.3346F}. 
In this study, we fixed the metallicity at $Z=10^{-3}~Z_{\odot}$.

\subsection{Model of stellar population}\label{sec:stellar_population}


% ---------------------------------------------------------------------------------
\begin{figure}
    \begin{center}
    	\includegraphics[width=\columnwidth]{./figures/sstar_ion_gam.pdf}
    \end{center}
    \caption{
    The emissivity of ionizing photons as a function of the slope of the stellar IMF.
}   
    \label{fig:sstar_ion_gam}
\end{figure}
% ---------------------------------------------------------------------------------




We adopt the stochastic stellar population model developed in \citet{2022MNRAS.511.3346F}.
This model is similar to {\sc SLUG} code developed by \citet{2012ApJ...745..145D} and \citet{2015MNRAS.452.1447K}.
We divide the masses into 150 bins between $0.1$ and $300~M_{\odot}$.
In the Milky way, it is known that there is an upper limit in stellar mass around $150~M_{\odot}$ \citep{2005Natur.434..192F}.
However, a massive star whose initial mass is more than $\sim 200~M_{\odot}$ was discovered in the R136 star cluster in the Large Magellanic Cloud \citep{2010MNRAS.408..731C, 2022arXiv220713078K}.
The upper stellar mass in the GCs is still unknown. Thus, we adopt $300~M_{\odot}$ as the upper stellar mass limit.
The newborn stars are stochastically distributed into bins based on the probability from the IMF.
We adopt the Chabrier IMF \citep{2003PASP..115..763C} but  change the slope of the massive end.
In the standard IMF, the slope is $\gamma = 2.35$ as the Salpeter IMF \citep{1955ApJ...121..161S}, where the number of stars within a specific mass range is $dn \propto m^{-\gamma} dm$.
Figure \ref{fig:sstar_ion_gam} shows  the emissivity of ionizing photons per unit stellar mass averaged over the IMF as a function of $\gamma$.
The shape of the IMF in a low-metallicity massive star cluster is still unknown.
\citet{2021MNRAS.508.4175C} indicated that the slope of IMF is shallower than $\gamma = 2$.
Thus, we mainly consider the case with $\gamma = 1.5$ as the fiducial model of the top-heavy IMF.
Besides, we calculate the models with $\gamma = 1$ and $2$.
In the low-mass star clusters ($M_*<10^4~M_{\odot}$), the emissivity is not uniform due to the stochastic stellar population \citep{2016ApJ...819..137K} even if the total stellar mass is same. 
However, the total stellar masses in our simulations are larger than $10^{4}~M_{\odot}$.
Therefore, the radiative properties of \adr{GCs} are almost the same as the IMF averaged values.


The stars are distributed in each sink particle at different times, and their radiative properties depend on their ages.
We consider the stellar evolution to estimate their radiative properties by adopting the {\sc PARSEC} tracks \citep{2012MNRAS.427..127B, 2014MNRAS.445.4287T, 2014MNRAS.444.2525C, 2015MNRAS.452.1068C, 2017ApJ...835...77M, 2019MNRAS.485.5666P, 2020MNRAS.498.3283P}.
The lower metallicity limit of {\sc PARSEC} tracks is $Z=10^{-2}~Z_{\odot}$, and thus we adopt the stellar evolution track at this value.
In addition, we use the SED models in \citet{1997A&AS..125..229L} and \citet{2019A&A...621A..85H} for OB-stars in SMC to calculate the emissivity of ionizing photons.


% ---------------------------------- tables ---------------------------------------
\begin{table}
    \caption{Models considered.}
    \label{Tab:models}
    \centering
    \begin{tabular}{|l|c|c|c|c|c|c|}
        \hline \hline
        model & $M_{\rm cl} $ & $\Sigma_{\rm cl}$ & $\gamma$ & $t_{\rm ff}$ & $v_{\rm esc} $  \\
        & $[M_{\odot} ]$ & $[M_{\odot} {\rm pc^{-2}}]$ & & $ [{\rm Myr} ]$ & $[{\rm km/s} ]$ \\
        \hline
M6R125G15 & $10^6$ & $2000$ & $1.5$ & $0.73$ & $26$ \\
M6R13G1 & $10^6$ & $2000$ & $1$ & $0.74$ & $26$ \\
M6R135G15 & $10^6$ & $1700$ & $1.5$ & $0.82$ & $25$ \\
M6R14G15 & $10^6$ & $1600$ & $1.5$ & $0.87$ & $25$ \\
M6R15G15 & $10^6$ & $1400$ & $1.5$ & $0.96$ & $24$ \\
M6R14G1 & $10^6$ & $1700$ & $1$ & $0.84$ & $25$ \\
M6R16G2 & $10^6$ & $1300$ & $2$ & $1.0$ & $23$ \\
M6R18G2 & $10^6$ & $1000$ & $2$ & $1.2$ & $22$ \\
M6R19G2 & $10^6$ & $900$ & $2$ & $1.3$ & $21$ \\
M6R20G15 & $10^6$ & $800$ & $1.5$ & $1.5$ & $21$ \\
M6R20G235 & $10^6$ & $800$ & $2.35$ & $1.5$ & $21$ \\
M6R23G235 & $10^6$ & $600$ & $2.35$ & $1.8$ & $19$ \\
M6R28G235 & $10^6$ & $410$ & $2.35$ & $2.5$ & $18$ \\
M7R46G15 & $10^7$ & $1500$ & $1.5$ & $1.6$ & $43$ \\
M7R50G15 & $10^7$ & $1300$ & $1.5$ & $1.9$ & $41$ \\
M7R56G15 & $10^7$ & $1000$ & $1.5$ & $2.2$ & $39$ \\
M7R63G15 & $10^7$ & $800$ & $1.5$ & $2.6$ & $37$ \\
M7R63G235 & $10^7$ & $800$ & $2.35$ & $2.6$ & $37$ \\
M7R73G235 & $10^7$ & $600$ & $2.35$ & $3.3$ & $34$ \\
M7R89G235 & $10^7$ & $400$ & $2.35$ & $4.4$ & $31$ \\
        \hline
    \end{tabular}
    \begin{minipage}{1 \hsize}
    Notes. Column 1: model names, Column 2: cloud masses, Column 3: surface densities, Column 4: high-mass IMF slope, Column 5: free-fall times, Column 6: escape velocities.
    \end{minipage}
\end{table}
%----------------------------------------------------------------------------------

\subsection{Initial condition}\label{Sec:Initial_condition}
We perform the simulations of clouds with the masses $M_{\rm cl} = 10^6~M_{\odot}$ and $10^{7}~M_{\odot}$.
We change the slopes of the IMF as $\gamma = 1$, $1.5$, $2$, and $2.35$.
By considering the variety of initial states of the clouds, we investigate the cases with $\Sigma_{\rm cl} =400-2000~M_{\odot}{\rm pc^{-2}}$.
The models are summarized in Table \ref{Tab:models}.
In each simulation, the maximum refinement level is fixed at $l_{\rm max} = 4$.
The minimum cell size is $\Delta x = 0.059 ~{\rm pc} (R_{\rm cl}/20~{\rm pc})$ where $R_{\rm cl}$ is the cloud radius.
As in \citetalias{2021MNRAS.506.5512F}, we set the turbulent motions in the initial condition with the velocity power spectrum as $P(k) \propto k^{-4}$ where $k$ is the wavenumber.
The strength of the turbulent motions is characterized by the virial parameter defined as 
\begin{align}
    \alpha_{0} = \frac{2E_{\rm kin}}{ | E_{\rm grav}|} = \frac{5 \sigma_0^2 R_{\rm cl}}{3 G M_{\rm cl}}, \label{eq_alpha_vir}
\end{align}
where $\sigma_0$, $E_{\rm kin}$, and $E_{\rm grav}$ are the 3D velocity dispersion, kinetic and gravitational energy.
Here, we adopt $\alpha_0 = 1$.




\section{Results} \label{Sec_results}

We first represent the impact of the top-heavy IMF on the dense star cluster formation in Section \ref{Sec:star_cluster_formation}.
In Section \ref{Sec:dependence_on_cloud_compact}, we investigate the condition for the formation of dense star clusters.
In Table \ref{Tab:results}, we summarize the results of our simulations.

% -------------------------------------- tables --------------------------------------------
\begin{table}
    \caption{Simulation results.}
    \label{Tab:results}
    \centering
    \begin{tabular}{|l|c|c|c|c|c|}
        \hline \hline
        model & $\varepsilon_*$ & $M_{\rm bd}$ & $f_{\rm bd}$ & $r_{\rm h}$ & $\rho_*$ \\
              &                 & $[\, M_{\odot} \,]$ &       & $[\, \rm pc \,]$ & $[\, M_{\odot} {\rm pc^{-3}} \,]$ \\
        \hline
M6R125G15 & $0.40$ & $3.9 \times 10^5$ & $0.98$ & $0.22$ & $9.1 \times 10^6$ \\
M6R13G1 & $0.18$ & $1.3 \times 10^5$ & $0.73$ & $1.8$ & $5.2 \times 10^3$ \\
M6R135G15 & $0.31$ & $3.0 \times 10^5$ & $0.96$ & $0.48$ & $6.5 \times 10^5$ \\
M6R14G15 & $0.18$ & $1.3 \times 10^5$ & $0.74$ & $1.9$ & $4.7 \times 10^3$ \\
M6R15G15 & $0.15$ & $7.8 \times 10^4$ & $0.53$ & $3.4$ & $5.0 \times 10^2$ \\
M6R14G1 & $0.13$ & $5.0 \times 10^4$ & $0.38$ & $2.8$ & $5.7 \times 10^2$ \\
M6R16G2 & $0.39$ & $3.8 \times 10^5$ & $0.98$ & $0.31$ & $3.0 \times 10^6$ \\
M6R18G2 & $0.19$ & $1.6 \times 10^5$ & $0.84$ & $2.8$ & $1.7 \times 10^3$ \\
M6R19G2 & $0.16$ & $1.2 \times 10^5$ & $0.75$ & $3.8$ & $5.1 \times 10^2$ \\
M6R20G15 & $0.078$ & $8.2 \times 10^3$ & $0.11$ & $1.5$ & $6.3 \times 10^2$ \\
M6R20G235 & $0.49$ & $4.8 \times 10^5$ & $0.99$ & $0.39$ & $2.0 \times 10^6$ \\
M6R23G235 & $0.17$ & $1.2 \times 10^5$ & $0.73$ & $5.1$ & $2.2 \times 10^2$ \\
M6R28G235 & $0.11$ & $2.6 \times 10^4$ & $0.23$ & $6.8$ & $1.9 \times 10^1$ \\
M7R46G15 & $0.48$ & $4.5 \times 10^6$ & $0.99$ & $0.91$ & $1.4 \times 10^6$ \\
M7R50G15 & $0.32$ & $2.9 \times 10^6$ & $0.96$ & $1.3$ & $3.5 \times 10^5$ \\
M7R56G15 & $0.11$ & $5.9 \times 10^5$ & $0.54$ & $12.0$ & $7.8 \times 10^1$ \\
M7R63G15 & $0.076$ & $8.3 \times 10^4$ & $0.11$ & $2.2$ & $2.4 \times 10^3$ \\
M7R63G235 & $0.60$ & $5.8 \times 10^6$ & $1.0$ & $1.4$ & $4.6 \times 10^5$ \\
M7R73G235 & $0.45$ & $4.4 \times 10^6$ & $0.99$ & $1.9$ & $1.6 \times 10^5$ \\
M7R89G235 & $0.13$ & $5.6 \times 10^5$ & $0.43$ & $19.0$ & $1.9 \times 10^1$ \\
      \hline
    \end{tabular}
    \begin{minipage}{1 \hsize}
    Notes. Column 1: model names, Column 2: star formation efficiencies, Column 3: gravitationally bound masses, Column 4: gravitational bound fraction of star clusters, Column 4: half-mass radii, Column 5: stellar densities of star clusters.
    \end{minipage}
\end{table}
% ----------------------------------------------------------------------------------------




\subsection{Star cluster formation with top-heavy IMF}\label{Sec:star_cluster_formation}



% ---------------------------------------------------------------------------------
\begin{figure*}
    \begin{center}
    	\includegraphics[width=185mm]{./figures/snap_gam.pdf}
    \end{center}
    \caption{
    Star cluster formation and cloud dispersion in the clouds with $M_{\rm cl} = 10^6~M_{\odot}$ and $\Sigma_{\rm cl} =800~M_{\odot}{\rm pc^{-2}}$. 
    The panels show the evolution of the surface densities.
    The top and bottom rows show the models with $\gamma = 1.5$ and $2.35$.
    The white dots represent the positions of stellar particles.
}   
    \label{fig:SNAP_SHOT2}
\end{figure*}
% ---------------------------------------------------------------------------------


We present the star cluster formation in the case of the models with $(M_{\rm cl}, \Sigma_{\rm cl}) = (10^{6}~M_{\odot}, 800~M_{\odot})$.
Figure \ref{fig:SNAP_SHOT2} shows the time evolution of the gas surface densities with the top-heavy 
 IMF ($\gamma = 1.5$) and the standard IMF ($\gamma = 2.35$).
The entire evolution is almost the same until the elapsed time $t \sim 1.0~t_{\rm ff}$ where $t_{\rm ff}$ is the free-fall time of the cloud.
Once a few percent of gas is converted into stars ($t \sim 1.5~t_{\rm ff}$), gas clumps around stars start to be evacuated only in the case of the top-heavy IMF.
As a result, the gravitational potential is too shallow to bind the star cluster compactly, allowing the expansion of stellar distribution.
In this case, the star cluster has a stellar mass density lower than that of GCs.
On the other hand, in the case of the standard IMF, gas inflow and star formation continue against the stellar feedback even at $t \gtrsim 1.5~t_{\rm ff}$. Finally, the compact star cluster form at the center of the cloud, and its density is comparable to the observed GCs.


% ---------------------------------------------------------------------------------
\begin{figure}
    \begin{center}
    	\includegraphics[width=\columnwidth]{./figures/mass_rho_star.pdf}
    \end{center}
    \caption{
    Upper panel: the time evolution of the stellar mass in the cases with $(M_{\rm cl}, \Sigma_{\rm cl}) = (10^6~M_{\odot}, 800~M_{\odot}{\rm pc^{-2}})$.
    Lower panel: the time evolution of the stellar density.
    Each line represents the different IMF with $\gamma = 1.5$ (solid) and $2.35$ (dashed line).
}   
    \label{fig:mass_rho_star}
\end{figure}
% ---------------------------------------------------------------------------------

Figure \ref{fig:mass_rho_star} shows the time evolution of the stellar mass and density in each model.
Here, we calculate the total mass and half-mass radius of the bound stars to obtain the stellar densities.
The details of the analytical method are described in \citetalias{2021MNRAS.506.5512F}.
The photoionization feedback suppresses the star formation significantly in the case of the top-heavy IMF ($\gamma = 1.5$).
The star formation efficiency (SFE) results in less than 10 percent.
In such a case, the stars are not bound by their own gravitational potential.
Thus, the resultant stellar density is lower than $10^3~M_{\odot} {\rm pc^{-3}}$.



In the case of the standard IMF ($\gamma = 2.35$), the star formation rate steeply increases at $t \gtrsim 1.3~t_{\rm ff}$.
In this phase, the high-density stellar core forms.
Finally, more than 40 percent of the gas is converted into stars, resulting in a massive compact star cluster. The stellar density shows $\sim 10^6~M_{\odot}{\rm pc^{-3}}$ which is similar to GCs.





\subsection{Dependence on IMF}\label{Sec:dependence_on_cloud_compact}


The large differences in the SFE and the stellar density are related to the stellar core formation as shown in Figure \ref{fig:SNAP_SHOT2}.
If the gas accretion continues against the photoionization feedback, the high-density stellar core forms at the center of the cloud.  
In \citetalias{2021MNRAS.506.5512F}, we analytically derive the threshold surface density for the stellar core formation as
\begin{align}
    \Sigma_{\rm cl} &> \Sigma_{\rm thr} = 670~M_{\odot}{\rm pc^{-2}} \left( \frac{M_{\rm cl}}{10^6~M_{\odot}} \right)^{-1/5} \left( \frac{s_*}{10^{47} ~M_{\odot}^{-1} s^{-1}} \right)^{2/5},
    \label{eq:thrsigma}
\end{align}
where $s_*$ is the number of ionizing photons per unit stellar mass.
Here, we consider the expanding shell model as in \citet{2002ApJ...566..302M} and \citet{2009ApJ...703.1352K}.
The deviation of $\Sigma_{\rm thr}$ is shown in Appendix \ref{section:shell_expansion_model}.
The cloud of $(M_{\rm cl}, \Sigma_{\rm cl}) = (10^6~M_{\odot}, 800~M_{\odot}{\rm pc^{-2}})$ satisfies this condition with the standard IMF. 
However, the threshold surface density increases up to $\Sigma_{\rm cl} > 1600~M_{\odot} {\rm pc^{-2}}$ in the case of the top-heavy IMF $(\gamma = 1.5)$.
Therefore, the massive and high-dense star clusters are difficult to form as the IMF is top-heavy.



% ---------------------------------------------------------------------------------
\begin{figure}
    \begin{center}
    	\includegraphics[width=\columnwidth]{./figures/eps_rhostar.pdf}
    \end{center}
    \caption{
    Upper panel:  SFEs of clouds as a function of surface densities. The back dashed line represents $\epsilon_* = 0.2$.
    Lower panel: Stellar densities of star clusters.
    Each symbol represents the different cloud masses: $M_{\rm cl} = 10^6~M_{\odot}$ (circle) and $10^7~M_{\odot}$ (square). Each color shows the slope of the IMF: $\gamma = 1.5$ (blue) and $\gamma = 2.35$ (orange).
}   
    \label{fig:eps_rhostar}
\end{figure}
% ---------------------------------------------------------------------------------

Figure \ref{fig:eps_rhostar} shows the dependencies of SFEs and the stellar densities on the surface density in the cases with the various cloud masses ($M_{\rm cl}=10^6~M_{\odot}$ and $10^7~M_{\odot}$) and the slope of the IMF ($\gamma = 1.5$ and $2.35$) as Tabel \ref{Tab:models}.
We estimate these values when the elapsed time of the simulations is $t=2~t_{\rm ff}$.
The SFE and the stellar densities increase with the higher surface densities.
The stellar core formation occurs when the SFE typically exceeds 0.15-0.2.
In such a case, the bound mass and the stellar densities are higher than $10^5~M_{\odot}$ and $>10^5~M_{\odot}{\rm pc^{-3}}$ which are similar to the properties of observed GCs.
With the top-heavy IMF ($\gamma = 1.5$), the core formation occurs at $\Sigma_{\rm cl} > 1600~M_{\odot}{\rm pc^{-2}}$ ($1000~M_{\odot}{\rm pc^{-2}}$) in the clouds with $M_{\rm cl} = 10^6~M_{\odot}$ ($10^7~M_{\odot}$).
In the case with the standard IMF ($\gamma = 2.35$), the threshold surface density decreases to $\Sigma_{\rm thr} = 750~M_{\odot}{\rm pc^{-2}}$ ($470~M_{\odot}{\rm pc^{-2}}$) with $M_{\rm cl} = 10^6~M_{\odot}$ ($10^7~M_{\odot}$) due to the decrease of the emissivity of the ionizing photons.
These results are consistent with the threshold surface density given as Equation \eqref{eq:thrsigma}.





% ---------------------------------------------------------------------------------
\begin{figure*}
    \begin{center}
     \includegraphics[width=13cm]
     {./figures/gam_sig.pdf}
    \end{center}
    \caption{The threshold surface density $(\Sigma_{\rm thr})$ for GC formation with $M_{\rm cl} = 10^{6}~M_{\odot}$ as a function of the slope of the IMF $(\gamma)$. 
    The dashed line shows the analytical result given as Eq. \eqref{eq:thrsigma}.
    The symbols indicate the stellar densities of star clusters.
    The circles, triangles, and crosses correspond to the cases with $\rho_* > 10^4~M_{\odot}{\rm pc^{-3}}$, $10^4~M_{\odot}{\rm pc^{-3}} > \rho_* > 10^3~M_{\odot}{\rm pc^{-3}}$, and $10^3~M_{\odot}{\rm pc^{-3}} > \rho_*$, respectively. 
    In the cases of triangles and circles, the bound stellar mass exceeds $10^5~M_{\odot}$.
    } 
    \label{fig:gam_sigma}
\end{figure*}
% ---------------------------------------------------------------------------------


To investigate the dependency of the IMF slope on $\Sigma_{\rm thr}$, we additionally perform the models with $\gamma = 1$ and $2$ as Table \ref{Tab:models}.
Figure \ref{fig:gam_sigma} shows the numerical results of the stellar densities in the plane of $\gamma$ and $\Sigma_{\rm cl}$ with the cloud mass of $M_{\rm cl}=10^6~M_{\odot}$.
The triangles and circles represent the cases where the star cluster satisfies the properties of GC progenitors \citep[$M_{\rm bh} > 10^5~M_{\odot}$ and $\rho_* > 10^3~M_{\odot}{\rm pc^{-3}}$, ][]{2010ARA&A..48..431P}. 
The threshold surface density $\Sigma_{\rm thr}$ derived in the analytical model (eq. \ref{eq:thrsigma}) reproduces the simulation results well.
The threshold value is $750~M_{\odot}{\rm pc^{-2}}$ for $\gamma = 2.35$, and it increases as $\gamma$ decreases. At $\gamma \lesssim 1.5$, the threshold surface density is not sensitive to $\gamma$ because $s_{*}$ does not change with $\gamma$ significantly. Thus, we suggest that once a high-density cloud with $\Sigma \gtrsim 2\times 10^{3} M_{\odot}{\rm pc^{-2}}$ forms, GCs are likely to form even if the IMF is top-heavy ($\gamma \ll 1.5$). 




\section{Discussion}\label{Sec:discussion}



In this paper, we study the impacts of the IMF on the condition for GC formation in low-metallicity clouds.
The observations of the GCs in the Milky Way showed that there is the metallicity floor at $Z\sim 10^{-2.5}~Z_{\odot}$ \citep[e.g.,][]{2005A&A...439..997P, 2019MNRAS.487.1986B}.
This metallicity floor can be related to the formation sites and mechanisms of GCs.
\citet{2019MNRAS.486L..20K} indicated that the metallicity floor results from the mass-metallicity relation of their host galaxies. 
They showed that the galaxies with metallicities less than the floor are too small to form GCs.
\citet{2018MNRAS.475L.130A} pointed out that radiation pressure caused by the Ly $\alpha$ photos suppresses the bound star cluster in the low-metallicity environments ($Z\lesssim 10^{-2.5}~Z_{\odot}$).
We suggested that more compact clouds ($\Sigma_{\rm cl} > 10^3~M_{\odot}{\rm pc^{-3}}$) are nessesary for the GC formation at $Z\lesssim 10^{-3}~Z_{\odot}$ if the IMF is a top-heavy \citep{2021MNRAS.508.4175C}.
In such high surface density environments, the formed bound star cluster is more likely to be destroyed due to the interactions with another star-forming cloud \citep[e.g.,][]{1987degc.book.....S, 1995ApJ...438..702K, 2006MNRAS.371..793G, 1999ApJ...522..935G, 2012MNRAS.426.3008K}.
Besides, the star clusters with the top-heavy IMF cannot survive for a long time \citep[e.g.,][]{2020MNRAS.491..440W, 2021MNRAS.504.5778W}.
In such a star cluster, more remnant black holes (BHs) form and migrate to the center by kicking neighbor low-mass stars. 
These star clusters can be finally evaporated or become dark star clusters that only contain BHs \citep{2011ApJ...741L..12B}.
On the other hand, observations have discovered the GCs with metallicities lower than the metallicity floor in most GCs \citep[e.g.,][]{2020Sci...370..970L, 2020Natur.583..768W, 2022Natur.601...45M}.
Therefore, further studies are needed to connect the formation of low-metallicity GCs, their stellar IMF, and their survival rates in the early galaxies.

The formation sites of young massive star clusters have been observed frequently in the starburst and merger galaxies \citep[e.g.,][]{2018ApJ...869..126L, 2021PASJ...73..417T}.
The numerical simulations have indicated massive clouds could be induced with galaxy merger processes \citep[e.g.,][]{2018MNRAS.475.4252A, 2019ApJ...879L..18L}.
In these environments, the colliding flow with thermal or gravitational instability can be key to forming massive compact clouds \citep{2012ApJ...759...35I, 2020ApJ...905...95K, 2021ApJ...908....2M, 2020MNRAS.496L...1D, 2022MNRAS.509..954D}.
The above large-scale conditions can determine the initial condition of clouds as used in our work. We will connect the global structure in a galaxy and the star cluster formation in a cloud in future work.




\section{Summary}\label{Sec:summary}

We have performed three-dimensional radiative-hydrodynamics  simulations of the formation of the low-metallicity globular clusters (GCs) with $Z=10^{-3}~Z_{\odot}$.
We have investigated the impacts of different initial mass functions (IMFs) on the properties of emergent star clusters.
Our simulations cover the various cloud masses and surface densities, $M_{\rm cl} = 10^6-10^7~M_{\odot}$ and $\Sigma_{\rm cl} =400-2000~M_{\odot}{\rm pc^{-2}}$.
Our findings are summarized as follows:

\begin{description}
\item[(i)] GCs form only in compact clouds with $\Sigma_{\rm cl} \sim 800~M_{\odot}{\rm pc^{-2}}$ for the cloud mass $M_{\rm cl}=10^6~M_{\odot}$ in the case of standard Salpeter-like IMF ($\gamma=2.35$).
With the top-heavy IMF of $\gamma = 1.5$, the threshold surface density ($\Sigma_{\rm thr}$) increases up to $\Sigma_{\rm thr} \sim 1600~M_{\odot}{\rm pc^{-2}}$ due to the higher emissivity of ionizing photons per unit stellar mass.
For $\gamma < 1.5$, $\Sigma_{\rm thr}$ saturates with $\Sigma_{\rm thr} \sim 2000~M_{\odot}{\rm pc^{-2}}$ because of saturation of the photon emissivity.

\item[(ii)] The simulation results about the threshold surface densities are reproduced well by a semi-analytical model updated from that derived in \citet{2021MNRAS.506.5512F}.
The threshold surface density mainly depends on the cloud mass and the emissivity of ionizing photons ($s_*$) as $\Sigma_{\rm cl} \propto M_{\rm cl}^{-1/5} s_{\rm *}^{2/5}$.



\end{description}

Theoretical studies predicted that GCs formed in dwarf galaxies in the early Universe
\citep[e.g.,][]{2016ApJ...831..204R, 2021MNRAS.500.4578M} due to the high Jeans mass \citep{1968ApJ...154..891P}, or thermal instability in low-metallicity gas \citep{1985ApJ...298...18F, 2018MNRAS.475.4252A}.
The collapse of the clouds under UV background radiation also induces the GC formation \citep{2009MNRAS.397.1338H, 2016MNRAS.463.2849A}.
Thanks to \textsc{James Webb Space Telescope} (JWST), star-formation sites in high-redshift galaxies have been observed directly \citep[e.g.,][]{2022arXiv220800520V}. Also, recent observations suggested the physical properties of the gas, e.g., density, temperature, and metallicity in star-forming regions of high-redshift galaxies \citep[e.g.,][]{2023arXiv230106811I}. 
Combining the GC formation condition obtained in this study with the observations, we will study the origin of GCs in future work.
Recently, \citet{2022ApJ...938L..10I} indicated that star-formation efficiencies (SFEs) can be higher than $\sim 0.1-0.3$ in galaxies at $z > 10$ from observed UV luminosity functions
\citep[e.g.,][]{2022ApJ...929....1H, 2022ApJS..259...20H, 2022arXiv220712474F}.
These high SFEs satisfy a condition for the GC formation.
With upcoming observations, we will study the GC formation at $z \gtrsim 10$ with cosmological simulations of galaxy formation \citep[e.g.,][]{2021MNRAS.508.3226A, 2022MNRAS.509.4037Y}.




\section*{Acknowledgements}
The authors wish to express their cordial thanks to Profs. Masayuki Umemura and Ken Ohsuga for their continual interest, advice, and encouragement.
We appreciate Tomoaki Matsumoto for his great contribution to code development.
We would like to thank Takashi Hosokawa for useful discussions and comments.
The numerical simulations were performed on the Cray XC50 (Aterui II) at the Center for Computational Astrophysics of National Astronomical Observatory of Japan and Yukawa-21 at Yukawa Institute for Theoretical Physics at Kyoto University.
This work is supported in part by MEXT/JSPS KAKENHI Grant Number 17H04827, 20H04724, 21H04489 (HY), NAOJ ALMA Scientific Research Grant Numbers 2019-11A, JST FOREST Program, Grant Number JP-MJFR202Z, and Astro Biology Center Project research AB041008 (HY). 

%%%%%%%%%%%%%%%%%%%%%%%%%%%%%%%%%%%%%%%%%%%%%%%%%%
\section*{Data Availability}

The data underlying this article will be shared on reasonable request to the corresponding author.




%%%%%%%%%%%%%%%%%%%% REFERENCES %%%%%%%%%%%%%%%%%%

% The best way to enter references is to use BibTeX:

\bibliographystyle{mnras}
%\bibliography{article} % if your bibtex file is called example.bib
% This must be in the first 5 lines to tell arXiv to use pdfLaTeX, which is strongly recommended.
\pdfoutput=1
% In particular, the hyperref package requires pdfLaTeX in order to break URLs across lines.

\documentclass[11pt]{article}

% Remove the "review" option to generate the final version.
%\usepackage[review]{ACL2023}
\usepackage{ACL2023}

% Standard package includes
\usepackage{times}
\usepackage{latexsym}

% For proper rendering and hyphenation of words containing Latin characters (including in bib files)
\usepackage[T1]{fontenc}
% For Vietnamese characters
% \usepackage[T5]{fontenc}
% See https://www.latex-project.org/help/documentation/encguide.pdf for other character sets

% This assumes your files are encoded as UTF8
\usepackage[utf8]{inputenc}

% This is not strictly necessary, and may be commented out.
% However, it will improve the layout of the manuscript,
% and will typically save some space.
\usepackage{microtype}

% This is also not strictly necessary, and may be commented out.
% However, it will improve the aesthetics of text in
% the typewriter font.
\usepackage{inconsolata}


% If the title and author information does not fit in the area allocated, uncomment the following
%
%\setlength\titlebox{10cm}
%
% and set <dim> to something 5cm or larger.

%%%%%%%%%%%%%%%%%%%%%%%%%%%%%%%%%%
\usepackage{graphicx}
\usepackage{amsfonts}
\usepackage{amsmath}
\usepackage{bigdelim}
\usepackage{diagbox}
\usepackage{amsthm}
\usepackage{makecell}
\usepackage{mathtools}
\usepackage{booktabs}
\usepackage[shortlabels]{enumitem}
\graphicspath{ {figs/} }

\theoremstyle{remark}
\newtheorem*{question}{Question}

\newcommand{\tk}[1]{\textcolor{blue}{{#1}}}
\newcommand{\sy}[1]{\textcolor{red}{{#1}}}
\newcommand{\mg}[1]{\textcolor{purple}{{#1}}}
\newcommand{\lh}[1]{\textcolor{green}{{#1}}}
\newcommand{\lc}[1]{\textcolor{green}{{#1}}}

% Rounded color box
\definecolor{light_blue}{HTML}{cfdfff}
\usepackage[most]{tcolorbox}
\tcbset{on line, 
        boxsep=1pt, left=0pt,right=0pt,top=0pt,bottom=0pt,
        colframe=white,colback=light_blue,  
        highlight math style={enhanced}
        }

\newcommand{\quash}[1]{}  %Anything in \quash is ignored
\newcommand{\gpt}{\textsc{GPT-2}}
\newcommand{\bert}{\textsc{BERT}}
\newcommand{\bertlarge}{\textsc{BERT-large}}
\newcommand{\mask}{\texttt{[MASK]}}
\newcommand{\cls}{\texttt{[CLS]}}
\newcommand{\sep}{\texttt{[SEP]}}
\newcommand{\mat}{\texttt{mat}}
\newcommand{\id}{\texttt{id}}
\newcommand{\matl}{\texttt{mat}_{\ell \rightarrow \ell'}}
\newcommand{\matattnl}{\texttt{mat\_attn}_{\ell \rightarrow \ell'}}
\newcommand{\matffl}{\texttt{mat\_ffn}_{\ell \rightarrow \ell'}}
\newcommand{\matlnl}{\texttt{mat\_ln1\_ln2}_{\ell \rightarrow \ell'}}
\newcommand{\idl}{\texttt{id}_{\ell \rightarrow \ell'}}
\newcommand{\matlL}{\texttt{mat}_{\ell \rightarrow L}}
\newcommand{\matattnlL}{\texttt{mat\_attn}_{\ell \rightarrow L}}
\newcommand{\matfflL}{\texttt{mat\_ffn}_{\ell \rightarrow L}}
\newcommand{\matlnlL}{\texttt{mat\_ln1\_ln2}_{\ell \rightarrow L}}
\newcommand{\idlL}{\texttt{id}_{\ell \rightarrow L}}

\definecolor{blue(munsell)}{rgb}{0.0, 0.5, 0.69}
%%%%%%%%%%%%%%%%%%%%%%%%%%%%%%%%%%

\title{Jump to Conclusions: Short-Cutting Transformers\\With Linear Transformations}

% Author information can be set in various styles:
% For several authors from the same institution:
% \author{Author 1 \and ... \and Author n \\
%         Address line \\ ... \\ Address line}
% if the names do not fit well on one line use
%         Author 1 \\ {\bf Author 2} \\ ... \\ {\bf Author n} \\
% For authors from different institutions:
% \author{Author 1 \\ Address line \\  ... \\ Address line
%         \And  ... \And
%         Author n \\ Address line \\ ... \\ Address line}
% To start a seperate ``row'' of authors use \AND, as in
% \author{Author 1 \\ Address line \\  ... \\ Address line
%         \AND
%         Author 2 \\ Address line \\ ... \\ Address line \And
%         Author 3 \\ Address line \\ ... \\ Address line}

\author{Alexander Yom Din$^{1}$ ~~~~~ Taelin Karidi$^{1}$ ~~~~~ Leshem Choshen$^{1}$ ~~~~~
Mor Geva$^{2}$ 
\vspace{0.2cm} \\
$^1$Hebrew University of Jerusalem ~~~ $^2$Google Research \\
\small{\texttt{\{alexander.yomdin, taelin.karidi, leshem.choshen\}@mail.huji.ac.il}}, \small{\texttt{pipek@google.com}}}

\quash{
\author{Alexander Yom Din \\
  Hebrew University of Jerusalem \\ \texttt{alexander.yomdin@mail.huji.ac.il} \\\And
  Taelin Karidi \\
  Hebrew University of Jerusalem \\
  \texttt{taelin.karidi@mail.huji.ac.il} \\\And
  Leshem Choshen \\
  Hebrew University of Jerusalem \\ \texttt{leshem.choshen@mail.huji.ac.il} \\\And
  Mor Geva \\
  Google Research \\
  \texttt{pipek@google.com} \\}
}

\begin{document}
\maketitle



\begin{abstract}
% \vspace{-1em}
The diffusion-based generative models have achieved remarkable success in text-based image generation. However, since it contains enormous randomness in generation progress, it is still challenging to apply such models for real-world visual content editing, especially in videos. 
In this paper, we propose \texttt{FateZero}, a zero-shot text-based editing method on real-world videos without per-prompt training or use-specific mask. 
\RM{Specifically, different from a pipeline of two independent inversion and then generation stages, we find the intermediate attention maps during inversions store better structure and motion information. We thus reform them to temporally casual attention and replace them in the generation progress. To further reduce the unnecessary semantic leakage of source video and enhance the editing quality, we then remix the temporally casual attentions via the cross-attention features of the source prompt as the mask.}
To edit videos consistently, we propose several techniques based on the pre-trained models. Firstly, in contrast to the straightforward DDIM inversion technique, our approach captures intermediate attention maps during inversion, which effectively retain both structural and motion information. These maps are directly fused in the editing process rather than generated during denoising. To further minimize semantic leakage of the source video, we then fuse self-attentions with a blending mask obtained by cross-attention features from the source prompt. Furthermore, we have implemented a reform of the self-attention mechanism in denoising UNet by introducing spatial-temporal attention to ensure frame consistency.
Yet succinct, our method is the first one to show the ability of zero-shot text-driven video style and local attribute editing from the trained text-to-image model. We also have a better zero-shot shape-aware editing ability based on the text-to-video model~\cite{tuneavideo}. \RM{Besides video, our unified method also achieves state-of-the-art performance in zero-shot image editing.\chenyang{Need exp or remove the zero-shot image}} Extensive experiments demonstrate our superior temporal consistency and editing capability than previous works.
% The code will be released.
% \chenyang{emphasize: our observation at inversion time} \xiaodong{replacing the bold part to the actual pipeline: \textbf{Specifically, we work on replacing and mixing the attention maps between the inversion and generation since the self-attention map keeps the structure of the original natural image and the cross-attention is semantic-related, after remixing, we replace them in the corresponding generation steps for denoising.}}
% \footnote{Since there is no general video diffusion model is publicly available, we use one-shot video generation method~(Tune-A-Video~\cite{tuneavideo}) as the pretrained video diffusion model for zero-shot video editing\xiaodong{can be removed if we actually zero-shot on video}.}.
\end{abstract}
\section{Introduction}

The ability to reason about plans is critical for performing long-horizon tasks \citep{erol1996hierarchical, sohn2018hierarchical, sharma-etal-2022-skill}, compositional generalization \citep{corona-etal-2021-modular} and generalization to unseen tasks and environments \citep{shridhar2020alfred}.
Consider a simple long-horizon planning scenario where a robot is tasked with preparing a meal and serving it on the table. 
This presents a non-trivial planning problem since the agent needs to understand the sequence of operations required to perform the task and search for the relevant objects in the unfamiliar environment by interacting with various objects. %



Large language models have been recently shown to possess commonsense knowledge about the world such as object affordances and physical dynamics \citep{ouyang2022training,chowdhery2022palm}.
Early approaches considered text based environments and fine-tuned PLMs to predict actions given the history of past observations and actions \citep{jansen-2020-visually,micheli-fleuret-2021-language,yao-etal-2020-keep}.
Recent work has used this ability to reason about plans from text instructions in simulated household environments with simplifying assumptions such as text-only environment observations or feedback \citep{huang2022language,ahn2022can,li2022pre,logeswaran-etal-2022-shot}.


We focus on \emph{visually grounded planning} with PLMs --- the ability to adapt plans based on interaction and visual feedback from the environment.
While PLMs have strong planning commonsense priors, predictions from a PLM may not be directly realizable in the environment since the observation and action spaces are unknown.
This requires \emph{grounding} the PLM in the environment and adapting it to observe visual feedback, which is highly non-trivial.
Some prior works assume the availability of a pre-trained affordance function \citep{ahn2022can} or a success detector \citep{mirchandani2021ella}.
Notably, SayCan \citep{ahn2022can} completely decouples the PLM from observation information by selecting actions that have both high affordability (through a pre-trained affordance model) and high PLM likelihood.
Although this partially addresses the grounding problem, the use of visual feedback for action affordance alone is limited.
Often an agent must choose one of many affordable actions using information from observations.
For example, a driving agent should re-navigate and possibly turn around when encountering a ``road closed'' sign, but both turning around and driving forward are indistinguishable to SayCan because they are both affordable and the PLM is blind to observations.

Another workaround explored in prior work is translating the information in the visual observations to text using a pre-trained captioning system \citep{shridhar2021alfworld,huang2022language}.
However, it can be difficult to faithfully describe an image in words and information is lost in this inherently noisy process, which limits the information available to the planner.



Recent work shows that PLMs can be adapted for various natural language tasks by inserting tunable embeddings or soft prompts at the input of the PLM (also called prompt tuning or prefix tuning)~\citep{li-liang-2021-prefix,lester-etal-2021-power}.
This approach also extends to multi-modal understanding tasks such as image captioning \citep{mokady2021clipcap} and VQA \citep{tsimpoukelli2021multimodal} where images are encoded as soft prompts and finetuned for the target task.
Transformer based architectures have also been successfully applied to offline Reinforcement Learning in recent work \citep{chen2021decision,janner2021offline,li2022pre,reid2022can}.

Taking inspiration from these works, we propose the simple approach of embedding visual observations (`visual prompts') and \textit{directly inserting them as PLM input embeddings}.
The visual encoder and PLM are jointly trained for the target task, an approach we call \textbf{\oursfull}~(\ours).
By teaching the PLM to use observations for planning in an end to end manner, we remove the dependency on external data such as captions and affordability information that was used in prior work.
We show that this simple approach performs better than prior PLM-based planning approaches on two embodied planning benchmarks based on ALFWorld~\citep{shridhar2021alfworld} and Virtualhome~\cite{puig2018virtualhome}.



\section{Related Work}

%Here we summarize prior work on transfer learning and property inference.

%\shortsection{Transfer Learning}
%%Transfer learning reuses features learned by pre-trained models for new tasks, with the pretext that inherent similarities in the generic features will be useful for the downstream tasks and hence reducing their cost of downstream training. Specifically, the downstream model trainer will use a pre-trained upstream model as the starting point for the downstream training, with inclusion of (or replacement with) the task-specific classification layer/module. The downstream model is then trained by either updating all layers of the model (including ones reused from upstream model) or freezing some earlier layers of the reused parts as the ``feature extractor'' and only updating the rest. The latter approach is more popular as the reused feature extractors can already learn useful feature representations and the training cost is also much lower and affordable for individuals with limited computational resources. We study the vulnerability of the latter transfer learning approach in this paper. 


%\shortsection{Transfer Learning} 
Several works have demonstrated risks associated with transfer learning across a variety of attack goals. Wang et al.~\cite{wang2018great} and Yao et al.~\cite{yao2019latent} consider manipulating the upstream model such that the fine-tuned downstream models contain backdoors, misclassifying test inputs that contain predefined backdoor triggers. These transfer manipulations are tailored to their particular attack goals and cannot be applied for the property inference goal considered in this paper. Zou et al.~\cite{zou2020privacy} study the threat of membership inference attacks on transfer learning, but with normally trained upstream models.  
%\dnote{its clear that the goals are different for these attacks, but how similar are the methods?} \ynote{similarity of the methods? more details about the methods? do not know what is expected here}
%In contrast, we investigate the possibility of boosting the effectiveness of property inference by manipulating the upstream model training. % Schuster et al.~\cite{schuster2020humpty} show that the attacker can modify the corpus on which the word embedding is trained such that the downstream NLP models which use that embedding will behave abnormally.

%\shortsection{Property Inference}
The risk of property inference was introduced by Ateniese et al.~\cite{ateniese2015hacking}, % introduces the threat of inferring properties of the training data from pre-trained models, 
and several subsequent works have developed property inference (also known as distribution inference) attacks~\cite{Wang2022GroupPI, suri2022formalizing, Jurez2022BlackBoxAF, Hartmann2022DistributionIR}.
% Ganju et al.~\cite{ganju2018property} and Suri and Evans~\cite{suri2022formalizing} 
These works study property inference against normally trained models, and they launch attacks using a variety of black-box and white-box attacks. All the white-box attacks use meta-classifiers, which take the permutation-invariant representation~\cite{ganju2018property} of the model parameters as the features. We use the state-of-the-art white-box attack~\cite{suri2022formalizing} in our experiments.
%We will use the state-of-the-art white-box method proposed by Ganju et al.~\cite{ganju2018property} and later extended by suri et al.~\cite{suri2022formalizing} in this paper.
%\dnote{do we use these attacks?} 
Melis et al.~\cite{melis2019exploiting} and Zhang et al.~\cite{zhang2021leakage} focus on property inference in distributed training scenarios. In their settings, the attacker is a participant in the global model training and conducts property inference using meta-classifiers that are trained on model outputs or gradients. Similarly, Suri et al.~\cite{suri2022subject} focus on federated learning settings where the attacker is a participant (or the central server) that utilizes black-box attacks for inferring membership of data from particular subjects. %\dnote{if we use black-box attacks, explain which ones, or how ours are related to previous ones} 
For our experiments, We improve the black-box meta-classifier proposed by Zhang et al.~\cite{zhang2021leakage} using the ``query tuning'' technique in Xu et al.~\cite{xu2019detecting}. 

The closest works to ours are Chase et al.~\cite{saeed} and Chaudhari et al.~\cite{Chaudhari2022SNAPEE}, which both consider a scenario where the attacker can manipulate some of the training data of the model to induce a model that significantly increases property inference risk.
% \dnote{it enables precise property inference attacks?}.
These works assume an adversary with the ability to poison the victim's training data, while the adversary in our scenario has no access to the victim's training data, and therefore, their methods are not applicable.
% \dnote{example how different from ours, and why the methods are not applicable}
%Thus, their methods are not applicable to our transfer learning scenario.
%Their methods rely on inducing certain behavior correlated with the properties to be inferred, and thus are not applicable to our transfer learning scenario. \anote{Still a bit unclear why that is the case.}
%
There are also works similar to ours that leverage ``adversarial initializations'' for attack purposes.
% \cite{grosse2019adversarial, boenisch2021curious, wen2022fishing, fowl2021robbing}.
Grosse et al.~\cite{grosse2019adversarial} focus on scenarios where the attacker can control the parameter initialization of a model, and demonstrate that the attacker can use special initializations to damage the performance of the trained model. %This attack is orthogonal to ours.
Other works \cite{boenisch2021curious, wen2022fishing, fowl2021robbing} show that the malicious central server in a federated learning protocol can reconstruct some training samples via falsifying the global model in some training rounds and then analyzing the submitted gradients. These kinds of attacks do not apply to our transfer-learning scenario since the attacker cannot access the downstream gradients, and can only manipulate the upstream training.

\iffalse %%%%%%%%%%%%%%%%%%%%%%%%%%%%%%%%

In this section, we provide the background and also the summary of prior attacks on transfer learning (Section~\ref{sec:transfer_learning}) and property inference (Section~\ref{sec:property_inference}). Then, we introduce the closely related manipulation attacks against machine learning models to boost different privacy risks in Section~\ref{sec:active_inference_attacks}.

%\anote{Do we really need a dedicated section for this? It's barely 2 paragraphs right now.}

%\dnote{the most closely related work to ours are works that attempt to amplify inference attacks by poisoning models, the two most relevant I know of are \url{https://www.computer.org/csdl/proceedings-article/sp/2022/131600b569/1CIO8nmuota} and \url{https://arxiv.org/abs/2204.00032}, but need to look thoroughly for others. We should definitely be describing this and relating it to our work, probably in the introduction. Most of what is here is Background, but should be clear what this section is for (not muddling background and related work)}

\subsection{Transfer Learning} \label{sec:transfer_learning}
Transfer learning reuses features learned by pre-trained models for new tasks, with the pretext that inherent similarities in generic features can be useful for downstream tasks, thus reducing the cost of downstream training. Specifically, the downstream model trainer uses a pre-trained upstream model as the starting point for downstream training, with the inclusion (or replacement) of task-specific classification layers/modules. The downstream model is then trained by either updating all layers of the model (including ones reused from the upstream model) or freezing some earlier layers of the reused parts as the ``feature extractor'' and only updating the rest. The latter approach is more popular as the reused feature extractors can already learn useful feature representations and the training cost is also much lower and affordable for individuals with limited computational resources. We study the vulnerability of the latter transfer learning approach in this paper. 
%mainly in two ways:  1) all the layers (including ones reused from ) and tune the full model; the other one is to freeze some earlier layers of the model as the feature extractor and only tune the rest later layers. The second update strategy could achieve better efficiency since the frozen layers can already produce meaningful feature representations~\cite{wang2018great,yao2019latent}, and we will study the transfer learning using this strategy. 

Recently, various attacks have been proposed for the transfer learning setting, but with different attack goals from ours. Wang et al.~\cite{wang2018great} generate adversarial examples against black-box student models that transfer knowledge from publicly available teacher models without repeated queries. Yao et al.~\cite{yao2019latent} propose to manipulate the upstream model such that the downstream models derived from the upstream model contain backdoors, which would misclassify test inputs that contain some predefined backdoor triggers. Zou et al.~\cite{zou2020privacy} study the threat of membership inference attacks on transfer learning and the upstream models are trained normally. In contrast, we investigate the possibility of boosting the effectiveness of property inference by manipulating the upstream model training. Schuster et al.~\cite{schuster2020humpty} show that the attacker can modify the corpus on which the word embedding is trained such that the downstream NLP models which use that embedding will behave abnormally.

%This additionally allows model trainers to achieve satisfactory performance with limited training samples, leading to reduced computational costs. The most common approach reuses parameters in the earlier layers of the pre-trained model, either by fixing them as the feature extractor or just using them for initialization, to conduct downstream training.

\subsection{Property Inference} \label{sec:property_inference}

\shortsection{Property Inference Attacks} In property inference attacks, the adversary aims to infer some sensitive properties of some data, given a model trained on it. For example, the adversary may be interested in sensitive properties like the presence of people of a specific race in the dataset~\cite{ateniese2015hacking, melis2019exploiting}), or even be curious about the 
the statistics of the training set (e.g, the ratio of people with a specific gender~\cite{saeed, ganju2018property, suri2022formalizing, zhang2021leakage}).


Ateniese et al.~\cite{ateniese2015hacking} were the first to identify the threat of inferring properties of the training data from pre-trained models. Ganju et al.~\cite{ganju2018property} and Suri and Evans~\cite{suri2022formalizing} 
study property inference against normally trained models, and they launch attacks using white-box meta-classifiers, which utilize the permutation-invariance representation~\cite{ganju2018property} of the model parameters, while other works focus on distributed training~\cite{zhang2021leakage} where the attacker is a participant in the global model training and conducts property inference using meta-classifiers trained on model outputs. Similarly, Suri et al.~\cite{suri2022subject} focus on federated learning, where the attacker is a participant (or the central server) that utilizes black-box attacks for inferring membership of data from particular subjects. Chase et al.~\cite{saeed} propose an active property inference attack for data poisoning scenarios, which we will cover and compare to in Section~\ref{sec:active_inference_attacks}.

%The closest work to ours are by Chase et al.~\cite{saeed} and Tramer et al.~\cite{tramer2022truth}. In their work, the attacker can manipulate some of the training data of the model such that a model trained (from scratch) on the poisoned data has an increased inference risk. However, their methods are not applicable to the transfer learning scenario. 
%In this work, we will focus on the property inference in transfer learning scenarios in which the attacker releases the upstream model and infer sensitive properties of the downstream models tuned from that upstream model.
% 

\shortsection{Defenses}
Defending against property inference attacks is an open problem. There are no studies in the current literature on active adversaries, and only a couple on passive ones. Ma et. al.~\cite{ma2021nosnoop} propose a defense against property inference attacks on data batches in the  collaborative learning setting. However, adversaries in the transfer-learning setting do not have access to batch-wise gradients of the downstream trainer. Chen and Ohrimenko~\cite{chen2022protecting} utilize mechanisms that add carefully-crafted noise to features to provide theoretical guarantees against inference adversaries, but focus on query-based access to the underlying dataset, not a machine learning model trained on it. These existing defenses thus do not apply to our threat model.

%propose a framework that reduces property inference to Boolean functions of individual members, posing the ratio of members satisfying the given function in a dataset as the property. These property inference attacks have since then been proposed as distribution inference attacks~\cite{suri2022formalizing}, presenting such attacks as inferring properties of the distributions used to sample datasets, differentiating them from exact inference attacks like dataset inference~\cite{maini2021dataset}. Nearly all property inference attacks use meta-classifiers to perform inference: training models on versions of datasets with and without the target property, followed by training a meta-classifier on top of these classifiers's model representations. These representations can take several forms: using model weights themselves with permutation-invariance~\cite{ganju2018property}, or model activations or logits for a generated set of query points~\cite{xu2019detecting}. However, the capability of such approaches is limited: the most that these attacks have been shown to work is medium-sized convolutional networks on the CelebA dataset~\cite{suri2022formalizing}.


\subsection{Active Privacy Attacks} \label{sec:active_inference_attacks}
% Perhaps the closely related works to ours as ones that proactively enhance the effectiveness of privacy attacks by manipulating the model training process in certain ways~\cite{saeed, melis2019exploiting, nasr2019comprehensive, tramer2022truth}. 
%shown that the adversary can, by using proactive ways, achieve stronger attacks that infer private information from deep learning systems~\cite{nasr2019comprehensive, melis2019exploiting, tramer2022truth, saeed}. In this section, we introduce the ones that are close to ours.

In the decentralized federated learning training, by submitting specially crafted gradients to the central server, malicious agents can increase membership inference risk~\cite{nasr2019comprehensive} and property inference risks~\cite{melis2019exploiting} of other benign agents' training data. However, these attacks do not apply to transfer learning scenario, as the attacker cannot control model gradients of downstream training. In the centralized setting, researchers propose attacks to poison the victim's training data such that the impacts of attribute inference and membership inference~\cite{tramer2022truth} and property inference~\cite{saeed} attacks are amplified on the poisoned model.
The ability to poison the victim's data is a threat model orthogonal to ours, since we have no access to the victim's downstream data. While there is scope to combine such approaches for stronger attacks (albeit with stronger access assumptions), we choose to focus on the scenario with no read/write access to the victim's data.

\fi %%%%%%%%%%%%%%%%%%%%%%%%%%%%%%%%

\section{Linear Shortcut Across Blocks}
\label{sec:layer_jump}

To use a hidden representation from layer $\ell<L$ as a final representation, we propose to cast it using linear regression, while skipping the computation in-between these layers. More generally, this approach can be applied to cast any $\ell$-th hidden representation to any subsequent layer $\ell'>\ell$.


\subsection{Method}
\label{subsec:methodology_linear_shortcut}

Given a source layer $\ell$ and a target layer $\ell'$ such that $0 \leq \ell < \ell' \leq L$, our goal is to learn a mapping
%$A_{\ell', \ell} \in \mathbb{R}^{d_h \times d_h}$
from hidden representations at layer $\ell$ to those at layer $\ell'$. To this end, we first collect a set of corresponding hidden representation pairs $(h^\ell, h^{\ell'})$. Concretely, we run a set $\mathcal{T}$ of input sequences through the model, and for each input $s$, we extract the hidden representations $h_{i_s}^{\ell}, h_{i_s}^{\ell'}$, where $i_s$ is a random position in $s$.
Next, we learn a matrix $A_{\ell', \ell} \in \mathbb{R}^{d_h \times d_h}$ by fitting linear regression over $\mathcal{T}$, i.e., $A_{\ell', \ell}$ is a numerical minimizer for:
$$ A \mapsto \sum_{s \in \mathcal{T}} || A \cdot h_{i_s}^\ell - h_{i_s}^{\ell'} ||^2,$$ 
and define the mapping of a representation $h$ from layer $\ell$ to layer $\ell'$ as:
\begin{equation}
\label{eq:linear_jump}
    \matl{} (h) \coloneqq A_{\ell', \ell} \cdot h.
\end{equation}


\subsection{Baseline}
\label{subsec:baseline}

We evaluate 
% our method against 
the prevalent approach of ``reading'' hidden representations directly, without any transformation. 
Namely, the propagation of a hidden representation from layer $\ell$ to layer $\ell'$ is given by the identity function, dubbed \id{}:

$$ \idl{} (h) \coloneqq h.$$

% Notably, 
This baseline 
assumes that representations at different layers operate in the same linear space.

\subsection{Quality of Fit}
\label{subsec:experiments_r2}

We first evaluate our method by measuring how well the learned linear mappings approximate the representations at the target layer. To this end, we calculate the (coordinate-averaged) $r^2$-score of our mapping's outputs with respect to the representations obtained from a full inference pass, and compare to the same for the \id{} baseline.


\paragraph{Models.}

We use \gpt{} \cite{radford2019language}, a decoder-only auto-regressive LM, with $L = 48$, $d_h = 1600$, and \bert{} \cite{devlin-etal-2019-bert}, an encoder-only model trained with masked language modeling, with $L=24$, $d_h=1024$.
% \footnote{\label{footnote:hf}We use models and data from Huggingface \cite{wolf-etal-2020-transformers,lhoest-etal-2021-datasets}.}
%For masked token prediction, we use a masked LM head pre-trained for our \bert{} model.

% \footnote{Specifically, we use the Huggingface Transformers \cite{wolf-etal-2020-transformers} implementations of all these models.}

%\sy{We use \gpt{} \cite{radford2019language}, a decoder-only auto-regressive LM, coming in four scales; $\texttt{gpt2}$ ($L = 12$, $d_h = 768$), $\texttt{gpt2-medium}$ ($L = 24$, $d_h = 1024$), $\texttt{gpt2-large}$ ($L = 36$, $d_h = 1280$) and $\texttt{gpt2-xl}$ ($L = 48$, $d_h = 1600$). Also, we use \bert{} \cite{devlin-etal-2019-bert}, an encoder-only model trained with masked language modeling, coming in two scales;  \texttt{bert-base-uncased} ($L=12$, $d_h=768$) and \texttt{bert-large-uncased} ($L=24$, $d_h=1024$). For masked token prediction, we use masked LM heads pre-trained for our models. Specifically, we use the Huggingface Transformers \cite{wolf-etal-2020-transformers} implementations of all these models. The plots presented in this section are for $48$-layered \gpt{} and $24$-layered \bert{}.}

%\sy{We use \gpt{} \cite{radford2019language}, a decoder-only auto-regressive LM, in the Huggingface \cite{wolf-etal-2020-transformers} implementation\footnote{\url{https://huggingface.co/gpt2}}, coming in four scales; $\texttt{gpt2}$ ($L = 12$, $d_h = 768$), $\texttt{gpt2-medium}$ ($L = 24$, $d_h = 1024$), $\texttt{gpt2-large}$ ($L = 36$, $d_h = 1280$) and $\texttt{gpt2-xl}$ ($L = 48$, $d_h = 1600$). Also, we use \bert{} \cite{devlin-etal-2019-bert}, an encoder-only model trained with masked language modeling, in the Hugginface implementation, coming in two scales;  \texttt{bert-base-uncased}\footnote{\url{https://huggingface.co/bert-base-uncased}} ($L=12$, $d_h=768$) and \texttt{bert-large-uncased}\footnote{\url{https://huggingface.co/bert-large-uncased}} ($L=24$, $d_h=1024$). For masked token prediction, we use the \texttt{BertForMaskedLM} heads from Huggingface, pretrained for these models. The plots presented in this section are for $48$-layered \gpt{} and $24$-layered \bert{}.}

\paragraph{Data.}
We sample random sentences from Wikipedia,
% \footref{footnote:hf} 
collecting 9,000 (resp. 3,000) sentences for the training set $\mathcal{T}$ (resp. validation set $\mathcal{V}$).\footnote{We use sentences rather than full documents to simplify the analysis.}
%\sy{We use two data sources to evaluate our method. One is Wikiepdia \cite{lhoest-etal-2021-datasets}\footnote{\url{https://huggingface.co/datasets/wikipedia}}; we use \texttt{spaCy}\footnote{\url{https://spacy.io/}} to divide documents into sentences\footnote{We use sentences rather than full documents to simplify the analysis.}\footnote{We pick randomly a Wikipedia document and then pick randomly a sentence ending in a newline character in it. \sy{[maybe this footnote is not needed?]}}, collecting 9,000 (resp. 3,000) random sentences for the training set $\mathcal{T}$ (resp. validation set $\mathcal{V}$). The second is a news article sentences dataset, the 10K English 2020 news sentences corpus
% \footnote{\url{https://downloads.wortschatz-leipzig.de/corpora/eng_news_2020_10K.tar.gz}} from the Leipzig Corpora Collection \cite{goldhahn-etal-2012-building}, which we randomly divide into a training set $\mathcal{T}$ consisting of 9,000 examples and a validation set $\mathcal{V}$ consisting of 1,000 examples.
% We truncate sentences to the maximal token length allowed by the model \mg{do we ever need to truncate? a sentence has about 10 words and the max. input len is thousands} \sy{[I surely did not need to in Leipzig, but discovered (via a transformers runtime warning) that I do need to for some (probably a minority) of the Wikipedia sentences. This probably has to do with that it is not really ``sentences" necessarily, for example, I noticed that it has some listings or something like that (bulleted items)... So some minority might get very long I guess...]}.
For each example $s$, we select a random position $i_s$ and extract the hidden representations $h_{i_s}^{\ell}$ at that position from all the layers.
For \bert{}, we first replace the input token at position $i_s$ with a \mask{} token, as our motivation is interpreting predictions, which are obtained via masked tokens in \bert{} (see \S\ref{subsec:BERT}).
Thus, in this case, the hidden representations we consider
%in the case of \bert{}
are of \mask{} tokens only.
%As we observed highly similar results for the two data sources across all our experiments, throughout the paper we will mainly report results for Wikipedia (except for \S\ref{sec:robustness}, where we cross-validate).


\begin{figure}[t]
\includegraphics[scale=0.2]{figs/r2_scores_48.pdf}
% \includegraphics[width=\columnwidth]{figs/r2_scores_48.pdf}
\caption{The coordinate-averaged $r^2$-score of $\matl{}$ (left) and $\idl{}$ (right) (\gpt{}).}
\label{fig:r2_scores}
\end{figure}


\begin{figure}[t]
\setlength{\belowcaptionskip}{-10pt}
\includegraphics[scale=0.2]{figs/bertmask_r2_scores_24.pdf}
% \includegraphics[width=\columnwidth]{figs/bertmask_r2_scores_24.pdf}
\caption{The coordinate-averaged $r^2$-score of $\matl{}$ (left) and $\idl{}$ (right) (\bert{}).}
\label{fig:bertmask_r2_scores}
\end{figure}



\paragraph{Evaluation.}
For every pair of layers $\ell, \ell'$, such that $0 \leq \ell < \ell' \leq L$, we use the training set $\mathcal{T}$ to fit linear regression as described in \S\ref{subsec:methodology_linear_shortcut}, and obtain a mapping $\matl{}$. 
Next, we evaluate the quality of $\matl{}$ as well as of $\idl{}$ using the $r^2$-coefficient, uniformly averaged over all coordinates. Concretely, we compute the $r^2$-coefficient of each of the predicted representations $\matl{} (h_{i_s}^{\ell})$ and $\idl{} (h_{i_s}^{\ell})$ versus the true representations $h_{i_s}^{\ell'}$
over all $s \in \mathcal{V}$.
%as we vary $s \in \mathcal{V}$.
%for every $s \in \mathcal{V}$.



\paragraph{Results.}
Results for \gpt{} and \bert{} are presented in Figs.~\ref{fig:r2_scores} and~\ref{fig:bertmask_r2_scores}, respectively.
In both models, \mat{} consistently yields better approximations than \id{}, as it obtains higher $r^2$-scores (in blue) across the network. 
This gap between \mat{} and \id{} is especially evident in \bert{}, where \id{} completely fails to map the representations between most layers, suggesting that hidden representations are modified  substantially by every transformer block.
Overall, this highlights the shortcoming of existing practices to inspect representations in the same linear space, and the gains from using our method to approximate future layers.
% in the network.
\section{Linear Shortcut for Language Modeling}
\label{sec:prediction}

We saw that our method approximates future hidden representations substantially better than a naive propagation. 
In this section, we will show that this improvement also translates to better predictive abilities from earlier layers. Specifically, we will use our method to estimate how often intermediate representations encode the final prediction, in the context of two fundamental LM tasks; next token prediction and masked token prediction.

\paragraph{Evaluation Metrics.}
Let $h, h' \in \mathbb{R}^{d_h}$ be a final representation and a substitute final representation obtained by some mapping, and denote by $\delta (h), \delta (h') \in \mathbb{R}^{d_v}$ their corresponding output probability distributions (obtained through projection to the output vocabulary -- see details below). 
We measure the prediction quality of $h'$ with respect to $h$ using two metrics:
\begin{itemize}
[leftmargin=*,topsep=1pt,parsep=1pt]
    \item \textbf{Precision@$k$} ($\uparrow$ is better): This checks whether the token with the highest probability according to $\delta(h')$ appears in the top-$k$ tokens according to $\delta(h)$. Namely, we sort $\delta(h)$ and assign a score of $1$ if $\arg\max(\delta(h'))$ appears in the top-$k$ tokens by $\delta(h)$, and $0$ otherwise.
    
    \item \textbf{Surprisal} ($\downarrow$ is better): We measure the minus log-probability according to $\delta(h)$, of the highest-probability token according to $\delta(h')$. Intuitively, low values mean that the model sees the substitute result as probable and hence not surprising.
\end{itemize}

\noindent We report the average Precision@$k$ and Surprisal over the validation set $\mathcal{V}$.



\subsection{Next Token Prediction}
\label{subsec:next_token_prediction_task}

Auto-regressive LMs output for every position a probability distribution over the vocabulary for the next token. Specifically, the output distribution for every position $i$ is given by $\delta (h_i^L)$, where:
\begin{equation}\label{eq:output_distribution}
    \delta (h) = \texttt{softmax} ( E^\top \cdot h) \in \mathbb{R}^{d_v}
\end{equation}
For some LMs, including \gpt{}, a layer normalization $\texttt{ln\_f}$ is applied to the final layer representation before this conversion (i.e., computing $\delta (\texttt{ln\_f}(h))$ rather than $\delta (h)$).

Recall that our goal is to measure how well this distribution can be estimated from intermediate representations, i.e. estimating $\delta (h_i^L)$ from $\delta (h_i^\ell)$ where $\ell<L$. To this end, we first run examples from the validation set through the model, while extracting for each example $s$ the hidden representation of a random position $i_s$ at every layer. Next, we apply our mappings $\matlL{}$ and the $\idlL{}$ baseline to cast the hidden representations of every layer $\ell$ to final layer substitutes (see \S\ref{sec:layer_jump}). Last, for each layer, we convert its corresponding final-layer substitute to an output distribution (Eq.~\ref{eq:output_distribution}) and compute the average Precision@$k$ (for $k=1,5,10$) and Surprisal scores with respect to the final output distribution, over the validation set.

\paragraph{Results.}
Figs.~\ref{fig:pre} and~\ref{fig:surp} show the average Precision@$k$ and Surprisal scores per layer in $48$-layered \gpt{}, respectively (the plots for the other \gpt{} models are presented in \S\ref{sec:app_scale}). Across all layers, \mat{} outperforms \id{} in terms of both scores, often by a large margin (e.g. till layer $44$ the Precision@$1$ achieved by \mat{} is bigger than that of $\id{}$ by more than $0.2$). 
This shows that linear mappings enable not just better estimation of final layer representations, but also of the predictions they induce. Moreover, the relatively high Precision@$k$ scores of \mat{} in early layers ($0.62$-$0.82$ for $k=10$, $0.52$-$0.74$ for $k=5$, and $0.28$-$0.45$ for $k=1$) suggest that early representations already encode a good estimation of the final prediction. Also, the substantially lower Surprisal scores of \mat{} compared to \id{} imply that our method allows for a more representative reading into the layer-wise prediction-formation of the model than allowed through direct projection to the vocabulary.

\begin{figure}[t]
\centering
\includegraphics[scale=0.4]{figs/pre_48.pdf}
\caption{Precision@$k$ ($k = 1,5, 10$) of $\matlL{}$ and $\idlL{}$ for next token prediction in $48$-layered \gpt{}.}
\label{fig:pre}
\end{figure}

\begin{figure}[t]
\centering
\includegraphics[scale=0.35]{figs/surp_48.pdf}
\caption{Surprisal for $\matlL$ and the baseline $\idlL{}$ ($48$-layered \gpt{} next token prediction task). A 95\% confidence interval surrounds the lines.}
\label{fig:surp}
\end{figure}

\subsection{Masked Token Prediction}
\label{subsec:BERT}

We now conduct the same experiment for the task of masked language modeling, where the model predicts a probability distribution of a masked token in the input rather than the token that follows the input. Unlike next token prediction, where the output distribution is computed from representations of varying input tokens, in masked token prediction the output is always obtained from representations of the same input token (i.e. \texttt{[MASK]}).

For this experiment, we use \bert{}, on top of which we use a pretrained masked language model head $\delta$; given a token sequence $s$, a \mask{} token inside it and its final representation $h$, $\delta (h) \in \mathbb{R}^{d_v}$
 is a probability distribution over tokens giving the model's assessment
 of the likelihood of tokens to be fitting in place of the \mask{} token in $s$.


\begin{figure}[t]
\centering
\includegraphics[scale=0.4]{figs/bertmask_pre_24.pdf}
\caption{Precision@$k$ ($k = 1,5, 10$) for  $\matlL{}$ and the baseline $\idlL{}$ ($24$-layered \bert{} masked token prediction task).}
\label{fig:bertmask_pre}
\end{figure}

\begin{figure}[t]
\centering
\includegraphics[scale=0.35]{figs/bertmask_surp_24.pdf}
\caption{Surprisal for $\matlL{}$ and the baseline $\idlL{}$ ($24$-layered \bert{} masked token prediction task). A 95\% confidence interval surrounds the lines.}
\label{fig:bertmask_surp}
\end{figure}

\paragraph{Results.}
Figs.~\ref{fig:bertmask_pre} and~\ref{fig:bertmask_surp} present the average Precision@$k$ and Surprisal scores per layer in $24$-layered \bert{} (the plots for the $12$-layered \bert{} model are presented in \S\ref{sec:app_scale}), overall showing trends similar to those observed for next token prediction in \gpt{} (\S\ref{subsec:next_token_prediction_task}). This is despite the differences between the two tasks and the considerable architectural differences between \bert{} and \gpt{}.
Notably, the superiority of \mat{} over \id{} in this setting is even more prominent; 
while \mat{}'s precision is between $0.2-0.6$ in the first ten layers (Fig.~\ref{fig:bertmask_pre}), \id{}'s precision for all values of $k$ is close to zero, again strongly indicating that our method allows for better reading into early layer hidden representations. 
More generally, \mat{} improves the Precision@$1$ of \id{} by more than $17\%$ at most layers, and unveils that a substantial amount of predictions ($>25\%$ starting from layer $3$) appear already in the very first layers.
Interestingly, the (rough) divide between the first half of layers and last half of layers for $\id{}$ in Figs.~\ref{fig:bertmask_pre},~\ref{fig:bertmask_surp} seems to align with the two-hump shape of the blue region for $\mat{}$ in Fig.~\ref{fig:bertmask_r2_scores}.

\paragraph{Analysis.}
We manually compare the predictions of our mapping $\matlL{}$ with $\idlL{}$, for a $24$-layered \bert{} model.  Concretely, we select 50 random sentences from the Leipzig dataset. Next, for each layer $\ell$, we manually analyze how many of the top-$5$ tokens according to $\matlL{}$ and $\idlL{}$ fit into context. We consider a token to fit into context if it is grammatically plausible within the sentence (see Tab.~\ref{tab:manual} for concrete examples).
In the resulting $1250$ instances (i.e. $50$ sentences $\times$ $25$ representations), we observe a substantially higher plausibility rate of $85.36\%$ for \mat{} compared to $52.8\%$ for \id{}. In fact, only in less than $4.3\%$ of the instances there are more plausible tokens among the top-$5$ tokens according to \id{} than among the top-$5$ tokens according to \mat{}, further supporting the Surprisal results above.

\begin{table*}
\footnotesize
\setlength{\belowcaptionskip}{-15pt}
\begin{tabular}{p{0.3\linewidth}ccccc}
& $\texttt{id}_{4 \rightarrow 24}$ & $\texttt{mat}_{4 \rightarrow 24}$ & $\texttt{id}_{12 \rightarrow 24}$ & $\texttt{mat}_{12 \rightarrow 24}$ & $\texttt{id}_{24 \rightarrow 24}$ \\ \midrule
\multirow{5}{=}{aldridge had shoulder surgery in \mask{}.} & fellowship & \tcbox{time} & cyclist & \tcbox{2009} & \tcbox{september} \\
& employment & \tcbox{it} & emergencies & \tcbox{2008} & \tcbox{november} \\
& agreement & her & seniors & \tcbox{2010} & \tcbox{december} \\
& \#\#ostal & them & cycling & \tcbox{2006} & \tcbox{august} \\
& \#\#com & work & \tcbox{pennsylvania} & \tcbox{2007} & \tcbox{july} \\ \midrule
\multirow{5}{=}{on your next view you will be asked to \mask{} continue reading.} & \#\#com & be & be & be & \tcbox{please} \\
& accreditation & get & undergo & \tcbox{please} & \tcbox{simply} \\ 
& $	\copyright$ & go & spartans & help & \tcbox{also} \\ 
& fellowship & \tcbox{help} & seniors & \tcbox{simply} & \tcbox{again} \\ 
& summer & have & * & say & \tcbox{immediately} \\ \bottomrule
\end{tabular}
\caption{Examples of top-$5$ predictions at layers $4$, $12$ and $24$, under the mappings $\matlL{}$ and $\idlL{}$, for a $24$-layered \bert{} model. Grammatically plausible predictions (according to a human annotator) are marked in \tcbox{blue}. Note that at layer $24$ the predictions of $\matlL{}$ and $\idlL{}$ are the same (by definition).} 
\label{tab:manual}
\end{table*}

\section{Implication to Early Exiting}
\label{sec:applications}

%The fact that it is often possible to approximate
The possibility of approximating
the final prediction already in the early layers has important implications for efficiency; applying our linear mapping instead of executing transformer blocks of quadratic time complexity, could save a substantial portion of the computation. In this section, we demonstrate this in the context of early exiting.

When 
% performing transformer model inference under 
using an early exit strategy \cite{schwartz-etal-2020-right, xin-etal-2020-deebert, schuster2022confident}, one aims at deciding dynamically at which layer to stop the computation and ``read'' the prediction from the hidden representation of that layer.
More precisely, under a confidence measure paradigm, one decides to stop the computation for a position $i$ at layer $\ell$ based on a confidence criterion, that is derived from casting the hidden representation $h_i^\ell$ as a final-layer representation and converting it to an output probability distribution. Specifically, following \citet{schuster2022confident}, a decision to exit is made if the difference between the highest and the second highest probabilities is bigger than $$ 0.9 \cdot \lambda + 0.1 \cdot {\rm exp} (-4 i / N),$$
where $N$ is the average length of the input until position $i_s$ for $s \in \mathcal{V}$, and $\lambda$ is a hyper-parameter.

\begin{figure}[t]
\setlength{\belowcaptionskip}{-10pt}
\centering
\includegraphics[width=\columnwidth]{figs/ee_gpt2bert.pdf}
\caption{Precision@$1$ with early exit and ``fixed exit'', applied to the $24$-layer \gpt{} for next token prediction (left) and the $24$-layer \bert{} for masked token prediction (right). Varying the confidence parameter $\lambda$, the $x$-coordinate is the average number of layers processed before an early exit decision is reached.}
\label{fig:ee_gpt2bert}
\end{figure}

\quash{
\begin{figure}[t]
\setlength{\belowcaptionskip}{-10pt}
\centering
\includegraphics[scale=0.35]{figs/ee_pre1_24.pdf}
\caption{Precision@$1$ for the various early exit methods, and previous ``fixed exit'' methods for comparison ($24$-layer \gpt{} next token prediction task). Varying the confidence parameter $\lambda$, the $x$-coordinate is the average number of layers processed before an early exit decision is reached.}
\label{fig:ee_pre1}
\end{figure}
}

\paragraph{Experiment.}
We assess the utility of our mapping $\matlL{}$ for early exit as a plug-and-play replacement for $\idlL{}$, through which intermediate representations are cast into final-layer representations.
We use \gpt{} for the next token prediction and \bert{} for masked token prediction (both with 24 layers).
We run each of the models over the validation set examples, while varying the confidence parameter $\lambda$ and using either $\idlL{}$ or $\matlL{}$ for casting intermediate representations.
Furthermore, we compare these early exit variants to the ``fixed exit'' strategy from \S\ref{sec:prediction}, where the computation is stopped after a pre-defined number of layers rather than relying on a dynamic decision.
We evaluate each variant in terms of both prediction's accuracy, using the Precision@$1$ metric (see \S\ref{sec:prediction}), and efficiency, measured as the average number of transformer layers processed during inference.


\paragraph{Results.}
%Figs.~\ref{fig:ee_pre1} and~\ref{fig:bertmask_ee_pre1}
Fig.~\ref{fig:ee_gpt2bert}
plots the average Precision@$1$ score against the average number of layers processed, for $24$-layer \gpt{} and $24$-layer \bert{}. For both models, under an early exit strategy our mapping \mat{} again provides a substantial improvement over \id{}.
For example, aiming at $95\%$ average precision, \mat{} saves $\sim3.3$ ($13.8$\%) layers in \gpt{} compared to only $\sim1.4$ ($5.9$\%) layers by \id{}, and $\sim4.8$ ($20$\%) layers in \bert{} versus $\sim3.5$ ($14.6$\%) layers by \id{}.
These results highlight the potential gains prominent early exit methods can obtain by using our method.
Notably, in both models and for each of the mapping methods, early exit obtains better results than fixed layer exit, as expected. 

\quash{
\begin{figure}[t]
\setlength{\belowcaptionskip}{-10pt}
\centering
\includegraphics[scale=0.35]{figs/bertmask_ee_pre1_24.pdf}
\caption{Precision@$1$ for the various early exit methods, and previous ``fixed exit'' methods for comparison ($24$-layer \bert{} masked token prediction task). Varying the confidence parameter $\lambda$, the $x$-coordinate is the average number of layers processed before an early exit decision is reached.}
\label{fig:bertmask_ee_pre1}
\end{figure}
}
\section{Linear Shortcut Across Sub-Modules}
\label{sec:submodules}

% Our experiments show that
% , despite the commonly-applied simplification by interpretability works, transformer layers do not operate in the same linear space and 
% there is a major gap in approximating future representations using an identity mapping (\S\ref{sec:layer_jump}, \S\ref{sec:prediction}).
% Here, 
In this section, we investigate whether discrepancies across layers result from specific sub-modules or are a general behaviour of all sub-modules in the network.  
This is done by extending our approach to test how well particular components in transformer blocks can be linearly approximated. 


\paragraph{Method.}

Consider \gpt{} for definiteness, then:
% we have 
$$ \texttt{b}_{\ell} = \texttt{b}_{\ell}^{\texttt{ffn}} \circ \texttt{b}_{\ell}^{\texttt{attn}}$$ 
% with
\begin{equation}\label{eq:attn} \texttt{b}^{\texttt{attn}}_{\ell} (H) = \texttt{attn}_{\ell} (\texttt{ln1}_{\ell} (H)) + H,\end{equation} 
where $\texttt{attn}_{\ell}$ is
%a multi-head self-attention
a MHSA
layer and \texttt{ln1} is a layer normalization (LN), and 
$$ \texttt{b}^{\texttt{ffn}}_{\ell} (H) = \texttt{ffn}_{\ell} (\texttt{ln2}_{\ell} (H)) + H,$$  
where $\texttt{ffn}_{\ell}$ is
%a feed-forward network
an FFN
layer and $\texttt{ln2}$ is a LN.
\quash{
Given a block $\texttt{b}_\ell$ and one of its sub-modules $\texttt{ln1}_\ell, \ \texttt{attn}_\ell, \ \texttt{ln2}_\ell$, or $\texttt{ffn}_\ell$, we fit linear regression approximating the output of the sub-module given its input and then use it in order to define mappings, as we now describe.
}
Given a block $\texttt{b}_\ell$ and one of its sub-modules $\texttt{ln1}_\ell, \ \texttt{attn}_\ell, \ \texttt{ln2}_\ell$, or $\texttt{ffn}_\ell$, we fit linear regression approximating the output of the sub-module given its input, and then use it to define mappings $\matattnl{}$, $\matlnl{}$ and $\matffl{}$.
%We provide the definition of $\matattnl{}$ below, and that of the other two in App. \ref{sec:app_submodule_skip_description}.
We provide the formal definitions of these mappings in App. \ref{sec:app_submodule_skip_description}.
\iffalse
\paragraph{$\matattnl{}$.}
%Illustrating this on $\texttt{attn}_\ell$ for definiteness,
For an input $s$, let $v^\ell_{i_s}$ be the vector at position $i_s$ in the output of $\texttt{attn}_\ell (\texttt{ln1}_\ell (H^{\ell - 1}))$. We denote by $A_\ell^{\texttt{attn}} \in \mathbb{R}^{d_h \times d_h}$ the matrix numerically minimizing 
$$ A \mapsto \sum_{s \in \mathcal{T}} || A \cdot \texttt{ln1}_\ell (h^{\ell-1}_{i_s}) - v^\ell_{i_s}||^2,$$
and define an attention sub-module replacement (Eq.~\ref{eq:attn}) by $$
\texttt{b}^{\overline{\texttt{attn}}}_\ell (h) \coloneqq A_{\ell}^{\texttt{attn}} \cdot \texttt{ln1}_\ell (h) + h. $$
We then define a mapping between two layers ${\ell \rightarrow \ell'}$ by:
$$ \matattnl{} (h) \coloneqq $$
$$ \texttt{b}^{\texttt{ffn}}_{\ell'} ( \texttt{b}^{\overline{\texttt{attn}}}_{\ell'} ( \ldots (\texttt{b}^{\texttt{ffn}}_{\ell+1} ( \texttt{b}^{\overline{\texttt{attn}}}_{\ell+1} (h)))\ldots)).$$ 
Namely, when applying each $\ell''$-th block, $\ell < \ell'' \leq \ell'$, we replace its attention sub-module $\texttt{attn}_{\ell''}$ by its linear approximation.
%In an analogous way, we consider the mappings $\matffl{}$ and $\matlnl{}$, where in the latter we perform the linear shortcut both for \texttt{ln1} and for \texttt{ln2} (see~\S\ref{sec:app_submodule_skip_description} for precise descriptions).
Importantly, unlike the original attention module, the approximation $\texttt{b}^{\overline{\texttt{attn}}}_\ell$ operates on each position independently, and therefore applying $\matattnl{}$ disables any contextualization between the layers $\ell$ and $\ell'$. Note that this is not the case for $\matffl{}$ and $\matlnl{}$, which retain the self-attention sub-modules and operate contextually.
\fi

\paragraph{Evaluation.}


We analyze the $24$-layered \gpt{}, and proceed completely analogously to \S\ref{subsec:next_token_prediction_task}, evaluating the Precision@$1$ and Surprisal metrics for the mappings $\matattnlL{}$, $\matfflL{}$ and $\matlnlL{}$.

\begin{figure}[t]
\setlength{\belowcaptionskip}{-0pt}
\centering
%\includegraphics[scale=0.2]
\includegraphics[width=\columnwidth]{figs/parts_presurp_24.pdf}
\caption{Precision@$1$ and Surprisal for the various sub-module linear mappings, and $\matlL{}$ for comparison ($24$-layer \gpt{} next token prediction task). A 95\% confidence interval surrounds the Surprisal lines.}
\label{fig:parts_presurp}
\end{figure}

\quash{
\begin{figure}[t]
\centering
\includegraphics[scale=0.4]{figs/parts_pre1_24.pdf}
\caption{Precision@$1$ for the various sub-module linear shortcut mappings, and the mapping $\matlL{}$ for comparison (\gpt{} next token prediction task).}
\label{fig:parts_pre1}
\end{figure}

\begin{figure}[t]
\centering
\includegraphics[scale=0.35]{figs/parts_surp_24.pdf}
\caption{Surprisal for the various sub-module linear shortcut mappings, and the mapping $\matlL{}$ for comparison (\gpt{} next token prediction task). A 95\% confidence interval surrounds the lines.}
\label{fig:parts_surp}
\end{figure}
}

\paragraph{Results.}
Fig.~\ref{fig:parts_presurp} shows the average Precision@$1$ and Surprisal scores per layer.
From a certain layer (\textasciitilde$7$), all sub-module mappings achieve better results than the full-block mapping $\matlL{}$. Thus, it is not just the cumulative effect of all the sub-modules in the transformer block that is amenable to linear approximation, but also individual sub-modules can be linearly approximated. 
Furthermore, the linear approximation of attention sub-modules is less harmful than that of the FFN or LN sub-modules. 
% Hypothetically, 
A possible reason is that the linear replacement of FFN or LN ``erodes'' the self-attention computation after a few layers. 
Moreover, the good performance of $\matattnlL{}$ suggests that contextualization often exhausts itself in early layers; speculatively, it is only in more delicate cases that the self-attention of late layers adds important information. Last, remark the sharp ascent of the scores for layer normalization in layers $5$-$8$, for which we do not currently see a particular reason. To conclude, we see that the possibility of linear approximation permeates
%the various
transformer components.


\section{Related Work}

Recently, there was a lot of interest in utilizing intermediate representations in transformer-based LMs, both for interpretability and for efficiency.

In the direction of interpretability, one seeks to understand the prediction construction process of the model \cite{tenney-etal-2019-bert, voita-etal-2019-bottom}.

More recent works use mechanistic interpretability and view the inference pass as a residual stream of information \cite{dar2022analyzing,geva-etal-2022-transformer}. Additionally, there are works on probing, attempting to understand what features are stored in the hidden representations \cite{adi2017finegrained, conneau-etal-2018-cram,liu-etal-2019-linguistic}. Our work is different in that it attempts to convert intermediate representations into a final-layer form, which is interpretable by design.

In the direction of efficiency, there is the thread of work on early exit, where computation is cut at a dynamically-decided earlier stage \cite{schwartz-etal-2020-right,xin-etal-2020-deebert,schuster2022confident}. Other works utilize a fixed early stage network to parallelize inference \citep{leviathan2022fast, chen2023accelerating}. However, intermediate representations are directly propagated in these works, which we show is substantially worse than our approach. Moreover, our method requires training considerably less parameters than methods such as \citet{schuster-etal-2021-consistent}, that learn a different output softmax for each intermediate layer.  

More broadly, skipping transformer layers and analyzing the linearity properties of transformer components have been discussed in prior works \cite{Zhao2021of,mickus-etal-2022-dissect,wang-etal-2022-skipbert,lamparth2023analyzing}.


\section{Conclusion and Future Work}

We present a simple and effective method for enhancing utilization of hidden representations in transformer-based LMs, that uses 
pre-fitted context-free and token-uniform linear mappings.
Through a series of experiments on different data sources, model architectures and scales, we show that our method consistently outperforms the prevalent practice of interpreting representations in the final-layer space of the model, yielding better approximations of succeeding representations and the predictions they induce, thus allowing a more faithful interpretation of the model's prediction-formation.
We demonstrate the practicality of our method for improving computation efficiency, saving a substantial amount of compute on top of prominent early exiting approaches. 
Also, by extending our method to sub-modules, 
% more specifically the attention sub-modules, 
we observe that replacing a part of the transformer inference by a non-contextual linear computation often results in a small deterioration of the prediction.
This opens new research directions for improving model efficiency,
% and parallelizability.
% including breaking the computation into several parallelizable tasks.
including breaking the computation into parallel tasks.

\section*{Limitations}

Although we see in this work that there is more linear structure to transformer inference than could be explained solely by the residual connection, we do not elucidate a reason for that. We also do not try to formulate formal criteria according to which to judge, in principle, the quality of ways of short-cutting transformer inference in-between layers. In addition, our experiments cover only English data.


%\section*{Ethics Statement}
%Scientific work published at ACL 2023 must comply with the ACL Ethics Policy.\footnote{\url{https://www.aclweb.org/portal/content/acl-code-ethics}} We encourage all authors to include an explicit ethics statement on the broader impact of the work, or other ethical considerations after the conclusion but before the references. The ethics statement will not count toward the page limit (8 pages for long, 4 pages for short papers).

\section*{Acknowledgements}

We thank Tal Schuster for constructive comments.

% Entries for the entire Anthology, followed by custom entries
\bibliography{anthology,custom}
\bibliographystyle{acl_natbib}

\appendix

\section{Descriptions of $\matattn{}$, $\matff{}$ and $\matln{}$}
\label{sec:app_submodule_skip_description}

Here we detail the definitions of the mappings $\matattnl{}$, $\matffl{}$ and $\matlnl{}$ utilized in \S\ref{sec:submodules}.

\paragraph{Description of $\matattnl{}$.}
%Illustrating this on $\texttt{attn}_\ell$ for definiteness,
For an input $s$, let $v^\ell_{i_s}$ be the vector at position $i_s$ in the output of $\texttt{attn}_\ell (\texttt{ln1}_\ell (H^{\ell - 1}))$. We denote by $A_\ell^{\texttt{attn}} \in \mathbb{R}^{d_h \times d_h}$ the matrix numerically minimizing 
$$ A \mapsto \sum_{s \in \mathcal{T}} || A \cdot \texttt{ln1}_\ell (h^{\ell-1}_{i_s}) - v^\ell_{i_s}||^2,$$
and define an attention sub-module replacement (Eq.~\ref{eq:attn}) by $$
\texttt{b}^{\overline{\texttt{attn}}}_\ell (h) \coloneqq A_{\ell}^{\texttt{attn}} \cdot \texttt{ln1}_\ell (h) + h. $$
We then define a mapping between two layers ${\ell \rightarrow \ell'}$ by:
$$ \matattnl{} (h) \coloneqq $$
$$ \texttt{b}^{\texttt{ffn}}_{\ell'} ( \texttt{b}^{\overline{\texttt{attn}}}_{\ell'} ( \ldots (\texttt{b}^{\texttt{ffn}}_{\ell+1} ( \texttt{b}^{\overline{\texttt{attn}}}_{\ell+1} (h)))\ldots)).$$ 
Namely, when applying each $\ell''$-th block, $\ell < \ell'' \leq \ell'$, we replace its attention sub-module $\texttt{attn}_{\ell''}$ by its linear approximation.
%In an analogous way, we consider the mappings $\matffl{}$ and $\matlnl{}$, where in the latter we perform the linear shortcut both for \texttt{ln1} and for \texttt{ln2} (see~\S\ref{sec:app_submodule_skip_description} for precise descriptions).
Importantly, unlike the original attention module, the approximation $\texttt{b}^{\overline{\texttt{attn}}}_\ell$ operates on each position independently, and therefore applying $\matattnl{}$ disables any contextualization between the layers $\ell$ and $\ell'$. Note that this is not the case for $\matffl{}$ and $\matlnl{}$, which retain the self-attention sub-modules and operate contextually.

\paragraph{Description of $\matffl{}$.}
Let $v^\ell_{i_s}$ be the vector at position $i_s$ in the output of $\texttt{ln2}_{\ell} (\texttt{b}_\ell^{\texttt{attn}} (H^{\ell - 1}))$, for a given input $s$. We denote by $A_\ell^{\texttt{ffn}} \in \mathbb{R}^{d_h \times d_h}$ the matrix numerically minimizing 
$$ A \mapsto \sum_{s \in \mathcal{T}} || A \cdot v^{\ell}_{i_s} - \texttt{ffn}_{\ell} (v^\ell_{i_s})||^2,$$
and define a replacement of the feed-forward sub-module $\texttt{b}_{\ell}^{\texttt{ffn}}$ by $$ \texttt{b}^{\overline{\texttt{ffn}}}_\ell (H) \coloneqq A_{\ell}^{\texttt{ffn}} \cdot \texttt{ln2}_\ell (H) + H.$$
We then define a mapping between two layers ${\ell \rightarrow \ell'}$ by:
$$ \matffl{} (H) \coloneqq $$
$$ \texttt{b}^{\overline{\texttt{ffn}}}_{\ell'} ( \texttt{b}^{\texttt{attn}}_{\ell'} ( \ldots (\texttt{b}^{\overline{\texttt{ffn}}}_{\ell+1} ( \texttt{b}^{\texttt{attn}}_{\ell+1} (H))\ldots)).$$

\paragraph{Description of $\matlnl{}$.}
Let $v^\ell_{i_s}$ be the vector at position $i_s$ in the output of $\texttt{b}^{\texttt{attn}}_{\ell} (H^{\ell - 1})$, for a given input $s$. We denote by $A_\ell^{\texttt{ln1}} \in \mathbb{R}^{d_h \times d_h}$ the matrix numerically minimizing 
$$ A \mapsto \sum_{s \in \mathcal{T}} || A \cdot h^{\ell}_{i_s} - \texttt{ln1}_{\ell} (h^\ell_{i_s})||^2$$ and we denote by $A_\ell^{\texttt{ln2}} \in \mathbb{R}^{d_h \times d_h}$ the matrix numerically minimizing $$ A \mapsto \sum_{s \in \mathcal{T}} || A \cdot v^{\ell}_{i_s} - \texttt{ln2}_{\ell} (v^\ell_{i_s})||^2.$$ We define a replacement of the block $\texttt{b}^{\texttt{attn}}_{\ell}$ by \begin{equation} \texttt{b}^{\overline{\texttt{ln1}}}_\ell (H) \coloneqq \texttt{attn}_{\ell} (A_{\ell}^{\texttt{ln1}} \cdot H) + H\end{equation} and we define a replacement of the block $\texttt{b}^{\texttt{ffn}}_{\ell}$ by \begin{equation} \texttt{b}^{\overline{\texttt{ln2}}}_\ell (H) \coloneqq \texttt{ffn}_{\ell} (A_{\ell}^{\texttt{ln2}} \cdot H) + H.\end{equation}
We then define a mapping between two layers ${\ell \rightarrow \ell'}$ by:
$$ \matlnl{} (H) \coloneqq $$
$$ \texttt{b}^{\overline{\texttt{ln2}}}_{\ell'} ( \texttt{b}^{\overline{\texttt{ln1}}}_{\ell'} ( \ldots (\texttt{b}^{\overline{\texttt{ln2}}}_{\ell+1} ( \texttt{b}^{\overline{\texttt{ln1}}}_{\ell+1} (H))\ldots)).$$


\end{document}




% Alternatively you could enter them by hand, like this:
% This method is tedious and prone to error if you have lots of references
%\begin{thebibliography}{99}
%\bibitem[\protect\citeauthoryear{Author}{2012}]{Author2012}
%Author A.~N., 2013, Journal of Improbable Astronomy, 1, 1
%\bibitem[\protect\citeauthoryear{Others}{2013}]{Others2013}
%Others S., 2012, Journal of Interesting Stuff, 17, 198
%\end{thebibliography}

%%%%%%%%%%%%%%%%%%%%%%%%%%%%%%%%%%%%%%%%%%%%%%%%%%

%%%%%%%%%%%%%%%%% APPENDICES %%%%%%%%%%%%%%%%%%%%

\appendix






\section{Threshold surface density of globular cluster formation}\label{section:shell_expansion_model}

We summarize the underlying physics in massive star cluster formation under radiative feedback as described in \citet{2021MNRAS.506.5512F}.
We consider an expanding shell around an H{\sc ii} region.
The equation of motion is described as \citep{2002ApJ...566..302M, 2009ApJ...703.1352K}
\begin{align}
    \frac{d}{dt} \left( M_{\rm sh} \dot r_{\rm sh} \right) = 4 \pi r_{\rm sh}^2 \rho_{\rm i} c_{\rm i}^2, \label{eq:shell_eom}
\end{align} 
where $\rho_{\rm i}$ and $c_{\rm i}$ are the density and the sound speed in H{\sc ii} regions.
Here, we consider the contribution from the thermal pressure of ionized gas alone and ignore other effects, such as radiation pressure and gravitational force from stars.
When the photon production rate balances with the total recombination rate in the H{\sc ii} region, the number density of H{\sc ii} regions is evaluated as 
\begin{align}
    n_{\rm i} = \left( \frac{\rho_i c_{i}^2}{k_{\rm B} T_{\rm i}} \right) = \left( \frac{3 S_{\rm ion}}{4 \pi r_{\rm sh}^3 \alpha_{\rm B}} \right)^{1/2}, \label{eq:number_density_inHII}
\end{align}
where $T_{\rm i}$, $S_{\rm ion}$, and $\alpha_{\rm B}$ are the temperature of ionized gas, the emissivity of ionizing photons, and the recombination coefficient $\alpha_{\rm B} = 2.6 \times 10^{-13} (T_{\rm i}/10^4~{\rm K})^{-0.8} \, {\rm cm^{3}s^{-1}}$ \citep{1989agna.book.....O}.
Here, we assume that the gas is converted into stars with the constant conversion rates $\epsilon_*$ after the shell passes.
The mass of expanding shell and the formed stars inside the shell are given as 
\begin{align}
    M_{\rm sh} = M_{\rm cl} (1-\epsilon_*) (r_{\rm sh}/R_{\rm cl})^3, \label{eq:shellmass}
\end{align}
and 
\begin{align}
    M_* = M_{\rm cl} \epsilon_* (r_{\rm sh}/R_{\rm cl})^3. \label{eq:starmass}
\end{align}
The emissivity of ionizing photons is estimated as
\begin{align}
    S_{\rm ion} = s_* M_{*}, \label{eq:sion}
\end{align}
where $s_*$ is the emissivity per unit mass.
Here, we adopt the following dimensionless parameters 
\begin{align}
    x = r_{\rm sh}/R_{\rm cl}, \label{eq:nondimensional1}
\end{align}
and
\begin{align}
    \tau =  \sqrt{3} t/t_{\rm HII}, \label{eq:nondimensional2}
\end{align}
where 
\begin{align}
    t_{\rm HII} = \left( \frac{1-\epsilon_*}{\epsilon_*} \right)^{1/4} \left( \frac{3 \alpha_{\rm B}}{4 s_* k_{\rm B}^2 T_{\rm i}^2} \right)^{1/4} \left( \frac{M_{\rm cl}^3}{\pi^3 \Sigma_{\rm cl}} \right)^{1/8}, \label{eq:tHII}
\end{align}
Substituting equation \eqref{eq:number_density_inHII}, \eqref{eq:nondimensional1}, \eqref{eq:nondimensional2} into equation \eqref{eq:shell_eom}, the equation of shell motion is rewritten as 
\begin{align}
    \frac{d}{d \tau} \left(x^3 \dot x \right) = x^2. \label{eq:eqmotion_dimensionless}
\end{align}
The self-similar solution of equation \eqref{eq:eqmotion_dimensionless} is found as $x = \tau / \sqrt{3}$ and $\dot x = 1/\sqrt{3}$ \citep{2021MNRAS.506.5512F}.
Thus, the shell arrives at the outer edge of the cloud within the timescale of $t_{\rm HII}$.
We assume that the shell expansion time of $t_{\rm HII}$ is equal to the duration time of the star formation.
The total stellar mass formed inside the cloud is given as 
\begin{align}
    M_* = \epsilon_* M_{\rm cl} = \dot M_* t_{\rm HII}. \label{eq:total_stellar_mass}
\end{align}
The star formation rates are expressed as 
\begin{align}
    \dot M_* = \epsilon_{\rm ff} \frac{M_{\rm cl}}{t_{\rm ff}}, \label{eq:star_formation_rates}
\end{align}
where $t_{\rm ff} = \sqrt{3 \pi / (32 G \rho)}$ is the free-fall time of the cloud.
Substituting equations \eqref{eq:tHII} and \eqref{eq:star_formation_rates} into equation \eqref{eq:total_stellar_mass}, we obtain the SFE as
\begin{align}
    \epsilon_* \simeq 0.07 & \left( \frac{\epsilon_{\rm ff}}{0.03} \right)^{4/5}  \left( \frac{\Sigma_{\rm cl}}{10^3~M_{\odot}{\rm pc^{-2}}} \right)^{1/2} \left( \frac{M_{\rm cl}}{10^6~M_{\odot}} \right)^{1/10} \nonumber \\
    & \left( \frac{T_{\rm i}}{3 \times 10^4 ~{\rm K}} \right)^{-14/25} \left( \frac{s_*}{5.8 \times 10^{46} ~M_{\odot}^{-1} s^{-1}} \right)^{-1/5} \label{eq:sfe}
\end{align}
The shell expansion velocity does not depend on the radial position.
With the SFE given as equation \eqref{eq:sfe}, the shell velocity is calculated as
\begin{align}
    v_{\rm exp} &= \frac{R_{\rm cl}}{t_{\rm HII}} \nonumber \\ 
    &\simeq 5.8~{\rm km/s} \left( \frac{\epsilon_{\rm ff}}{0.03} \right)^{1/5} \left( \frac{\Sigma_{\rm cl}}{10^3~M_{\odot}{\rm pc^{-2}}} \right)^{-1/4}  \left( \frac{M_{\rm cl}}{10^6~M_{\odot}} \right)^{3/20} \nonumber \\
    & \hspace{2cm} \left( \frac{T_{\rm i}}{3 \times 10^4~{\rm K}} \right)^{14/25} \left( \frac{s_*}{5.8 \times 10^{46} ~M_{\odot}^{-1} s^{-1}} \right)^{1/5}. \label{eq:velocity_of_HIIregions}
\end{align}

The formed stars start to be bound when the total stellar mass exceeds 0.1 times the cloud mass \citep{2021MNRAS.506.5512F}.
At this epoch, ionizing fronts propagate beyond the star cluster.
When the thermal pressure on the shell overcomes the gravitational force, the gas around the star clusters is dispersed.
On the other hand, the dense gas remains, and H{\sc ii} regions are confined if the gravitational force is strong enough to hold the ionized gas.
In such a case, the gas keeps accreting onto the star cluster and contributes to further star formation. 
As a result, the high-density stellar core forms.
Assuming that the expanding shell around the star cluster has the same velocity as the semi-analytical solution of equation \eqref{eq:velocity_of_HIIregions}, the condition for binding the shell is given as
\begin{align}
   v_{\rm exp} < v_{\rm esc} = \sqrt{2GM_{\rm core}}{R_{\rm core}}, \label{eq:formation_stellar_core}
\end{align}
where $M_{\rm core}$ and $R_{\rm core}$ are the core mass and radius.
These values are typically $R_{\rm core} \sim 0.1 R_{\rm cl}$ and $M_{\rm core} \sim 10^{-2}~M_{\rm cl}$.
We obtain the conditions for the formation of a massive star cluster as 
\begin{align}
    \Sigma_{\rm cl} > \Sigma_{\rm thr} &= 750~M_{\odot}{\rm pc^{-2}} \left( \frac{\epsilon_{\rm ff}}{0.03} \right)^{2/5}  \left( \frac{M_{\rm cl}}{10^6~M_{\odot}} \right)^{-1/5} \nonumber \\ 
    & \hspace{5mm} \left( \frac{T_{\rm i}}{2.5 \times 10^4~{\rm K}} \right)^{28/25} \left( \frac{s_*}{1.1 \times 10^{47} ~M_{\odot}^{-1} s^{-1}} \right)^{2/5}. \label{eq:sig_thr}
\end{align}




% ---------------------------------------------------------------------------------
\begin{figure}
    \begin{center}
    	\includegraphics[width=\columnwidth]{./figures/gam_sstar.pdf}
    \end{center}
    \caption{
    The threshold surface density ($\Sigma_{\rm thr}$) as the function of the slope of the IMF ($\gamma$). Each line shows the cases with the cloud of $M_{\rm cl}=10^6~M_{\odot}$ (blue) and $10^7~M_{\odot}$ (orange).
    The solid (dashed) line shows the cases that the maximum stellar mass is $300~M_{\odot}$ ($150~M_{\odot}$).}   
    \label{fig:gam_sstar}
\end{figure}
% ---------------------------------------------------------------------------------

Combining the emissivity of ionizing photons as shown in Figure \ref{fig:sstar_ion_gam}, we calculate the threshold surface densities as the function of the power-law index of the IMF ($\gamma$).
Figure \ref{fig:gam_sstar} shows the values of $\Sigma_{\rm thr}$ as a function of $\gamma$ in the clouds with $M_{\rm cl} = 10^6~M_{\odot}$ and $10^7~M_{\odot}$.
In both cases, the threshold surface densities are less than $10^3~M_{\odot}{\rm pc^{-2}}$ for the standard IMF ($\gamma \sim 2.35$).
At $\gamma \gtrsim 1.5$, the values of $\Sigma_{\rm thr}$ are larger than $\sim 10^{3}~M_{\odot}{\rm pc^{-2}}$ of which clouds are rare in  observations of local galaxies \citep[e.g.,][]{2010ApJ...723..492R}.
We also consider the dependence of $\Sigma_{\rm thr}$ on the maximum stellar mass ($M_{\rm max}$) of the IMF.
The dot and dashed lines in Figure \ref{fig:gam_sstar} show the cases that the maximum stellar masses are $M_{\rm max} = 300~M_{\odot}$ and $150~M_{\odot}$ with the Salpeter IMF.
At $\gamma < 1.5$, the values of $\Sigma_{\rm thr}$ depend on the maximum stellar mass.
However, the threshold surface density does not vary with the maximum stellar mass at $\gamma > 1.5$.







%%%%%%%%%%%%%%%%%%%%%%%%%%%%%%%%%%%%%%%%%%%%%%%%%%


% Don't change these lines
\bsp	% typesetting comment
\label{lastpage}
\end{document}

% End of mnras_template.tex
