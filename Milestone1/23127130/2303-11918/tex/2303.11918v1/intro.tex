Four decades after Freedman's celebrated work on 4-manifolds, the topological 4-genus $\gtop(K)$ of knots~$K$ remains difficult to determine~\cite{F}. The first challenge is posed by the figure-eight knot $4_1$,
which satisfies $\gtop(4_1)=1$ for no obvious reason. The problem with that knot is that its second power $4_1 \# 4_1$ bounds an embedded disc in the 4-ball, causing all its additive lower bounds on the 4-genus to be
trivial. In particular, the signature invariant $\sigma(4_1)$ is zero. Due to this example, there is little reason to believe that the inequality
$$|\sigma(K)| \leq 2\gtop(K)$$
has much to tell us about the (topological) 4-genus of knots in general. In this note, we show that closures of 3-braids are exceptional in this respect. More precisely, we will see that the figure-eight knot is
exceptional among closed 3-braids in that it is the only 3-braid knot~$K$ that satisfies $|\sigma(K)|<2g(K)$, yet $\gtop(K)=g(K)$, where $g(K)$ denotes the ordinary Seifert genus.

\begin{theorem} \label{thm:1} \label{theorem1}
Let $K$ be a $3$-braid knot other than the figure-eight knot. Then
\[ |\sigma(K)| = 2g(K) \quad\Longleftrightarrow\quad \gtop(K)=g(K). \]
These equalities hold precisely if $K$ or its mirror is one of the following knots:
\begin{itemize}
\item[--] $T(2,2n+1)\# T(2,2m+1)$, with $n,m\geq 0$,
\item[--] $P(2p, 2q+1, 2r+1, 1)$, with $p\geq 1$, $q,r\geq 0$,
\item[--] $T(3,4)$, or $T(3,5)$.
\end{itemize}
\end{theorem}
The equivalence of $|\sigma(K)| = 2g(K)$ and $\gtop(K)=g(K)$
also holds for braid positive knots $K$~\cite{L}.
The proof of \cref{thm:1} is based on a technique called twisting, used by McCoy for estimating the topological 4-genus from above \cite{McC}. We will also make use of a special presentation for 3-braids called Xu normal
form, which we will describe in the next section. The third section contains the proof of \cref{thm:1}, as well as a complementary result (\cref{thm:2}): a sharp lower bound on the difference $g(K)-\gtop(K)$ of so-called strongly
quasipositive 3-braid knots, in terms of two characteristic quantities associated with the Xu normal form of these.
In the last section, we determine the 
topological 4-genus of various families of braid positive 3-braid knots (almost) exactly, and display examples where our technique comes to a limit.\\

\emph{Acknowledgements:} The second author is supported by the Emmy Noether Programme of the DFG, Project number 412851057. 
The fourth author thanks Peter Feller for his constant encouragement and gratefully acknowledges support from the Swiss National Science Foundation Grant 181199.
