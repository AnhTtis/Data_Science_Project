Before beginning with the proof of \cref{thm:1}, we describe a technique that we use to detect topological four-genus defect in a given knot $K$, that is, to show $\gtop(K)<g(K)$. The main ingredient is the so-called {\em generalized crossing change}, also known as {\em null-homologous twist}, or simply {\em twist}. A null-homologous twist consists in performing a $\pm 1$ Dehn surgery on the boundary circle of an embedded disc $D\subset S^3$, such that $D$ intersects $K$ transversely in a finite number of interior points, with total algebraic intersection count~$0$. While $\pm 1$ Dehn surgery on an unknot in~$S^3$ gives back~$S^3$, the effect on~$K$ is that a (left- or right-handed) full twist is introduced into the strands of~$K$ that cross~$D$. The \emph{untwisting number} $\tu(K)$ of $K$, introduced by Ince~\cite{I}, is defined as the minimal number of null-homologous twists needed to turn $K$ into the unknot. Clearly, $\tu(K)\leq u(K)$, since crossing changes are special cases of null-homologous twists. McCoy~\cite{McC} showed 
that $\gtop(K)\leq \tu(K)$, that is, the untwisting number of $K$ is an upper bound on the topological four-genus of $K$. His result is based on Freedman's theorem, which states 
that knots of Alexander polynomial $1$ are topologically slice~\cite{F,FQ}. This bound can now be used to find topological four-genus defect: If we find a way to turn a knot $K$ into the unknot with strictly less than $g(K)$ null-homologous twists, this will show $\gtop(K)<g(K)$. This method was already applied by Baader--Banfield--Lewark~\cite{BBL} to $3$-stranded torus knots. For the proof of \cref{thm:1} below, we use a slightly refined version of the method, as follows.

\begin{figure}[b]
\begin{center}
\includegraphics[scale=1.5]{fig_saddlemoves.pdf}
\caption{Left: Saddle move. Middle: How to use isotopy and a saddle move to add or remove a braid crossing. Right: Example of a cobordism between $\cl(\alpha)$ and $\cl(\alpha\cdot abx)$ that consists of three saddle moves, for some $3$-braid $\alpha$.} \label{fig:saddle}
\end{center}
\end{figure}

Assume we find a cobordism $C\subset S^3\x[0,1]$ from a given knot~$K$ to some knot~$K'$
such that $g(C)=g(K)-g(K')$. If such a $C$ exists, we will write~$K\leadsto K'$.
Assume that furthermore $\tu(K')<g(K')$. Then, by McCoy's result, $\gtop(K')\leq \tu(K')<g(K')$. Composing the cobordism $C$ with a topological slice surface for $K'$, we obtain
\[
\gtop(K)\leq \gtop(K')+g(C)<g(K')+g(C)=g(K).
\]
In particular, for the topological four-genus defect we find
\begin{equation}\label{eq:defect2}
g(K) - \gtop(K)\geq g(K') - \gtop(K') \geq g(K') - \tu(K').
\end{equation}

One way to construct cobordisms is to apply saddle moves to knot diagrams, as in \cref{fig:saddle}. Note that such cobordisms will always be smooth.
Suppose the knot $K$ is the closure of a strongly 
quasipositive 3-braid of the form $\beta=\delta^n \tau_1^{u_1}\cdots\tau_t^{u_t}$ with $n\geq 0$ and $u_1, \ldots, u_t \geq 1$.
Out of saddles, one may build a cobordism $C$
that
lowers the exponents $u_i$ and $n$ 
(one saddle move 
per letter $\tau_i$, two saddle moves 
per letter $\delta$), or transforms $\delta$ into $\tau_i$ (one saddle move).
Suppose the exponents remain non-negative, and $C$ is a cobordism from $K$ to another knot $K'$.
Then $K'$ is also strongly quasipositive,
and it follows from the Bennequin inequality, see \eqref{eq:genus},
that $g(C) = g(K) - g(K')$, i.e.~$K \leadsto K'$.

\begin{proof}[Proof of \cref{thm:1}]
We organize the proof into two parts, which consist in verifying the following statements.

\begin{enumerate}
\item $|\sigma(K)|=2g(K)$ for all $K$ in the list of \cref{thm:1} and their mirrors,
\item $\gtop(K)<g(K)$ for all other $3$-braid knots except the figure-eight.
\end{enumerate}

In light of Kauffman and Taylor's signature bound $|\sigma(J)|\leq 2\gtop(J)$, valid for all knots $J$, see~\cite{KT,P}, these two statements together imply the theorem.\\[-0.7em]

{\em Part (1).} The genera and signatures of torus knots are well understood~\cite{Lit}; in particular, we know that the torus knots $K$ with $|\sigma(K)|=2g(K)$ are precisely the knots $T(3,4), T(3,5)$, $T(2,2n+1)$, $n\geq 0$, and their mirrors.
In fact, the signature of $T(2,2n+1)$, $n\geq 0$, is known to be $-2n$. Both genus and signature are additive under connected sum of knots; hence $|\sigma(K)|=2g(K)$ for all knots $K$ of the first type listed. If $K$ is one of the listed pretzel knots $K=P(2p,2q+1,2r+1,1)$ with $p\geq 1, q,r\geq 0$, then $K$ has a positive and alternating, hence special alternating diagram; see \cref{fig:pretzel}. Murasugi shows in this case that $|\sigma(K)|=2g(K)$; see~\cite[Corollary~10.3]{M1}.\\[-0.7em]

\begin{figure}[b]
\begin{center}
\includegraphics[scale=1.1]{fig_pretzel.pdf}
\caption{Isotopy (denoted $\approx$) from the closure of the braid $a^{u_1}b^{u_2}x^{u_3}$ to the pretzel knot $P(u_1,1,u_2,u_3)\approx P(u_1,u_2,u_3,1)$; here $(u_1,u_2,u_3)=(4,3,5)$.} \label{fig:pretzel}
\end{center}
\end{figure}

{\em Part (2).} 
Let $K$ be a $3$-braid knot other than the figure-eight knot such that neither $K$ nor its mirror appears in the above list. The goal is to show that $\gtop(K)<g(K)$. We distinguish several cases.

-- If $K$ is the closure of a positive $3$-braid, this is a special case of the analogous statement about all positive braids, on any number of strands, by Liechti \cite[Theorem~1, Corollary~2]{L}.

-- Next, we consider the case in which $K$ is strongly quasipositive without being braid positive; this is the main part of the proof. By \Cref{lem:xuform} and \cref{prop:positivity},
$K$ is the closure of a $3$-braid $\beta$ in Xu normal form $\beta=\delta^n \tau_1^{u_1}\tau_2^{u_2}\cdots \tau_t^{u_t}$ with $u_1,\ldots,u_t\geq 1$, $n\geq 0$, $t\geq 2$, $n+t\equiv 0\mod 3$, and $2n<t$ (note that the case $n = 0, t = 1$ is excluded since we assume that $\cl(\beta)$ is a knot). These conditions on $n$ and $t$ leave the following possibilities: $(n,t)=(0,3)$, or $(n,t)=(1,5)$, or $t\geq 6$. First, if $(n,t)=(0,3)$, then $K=P(u_1,u_2,u_3,1)$; see \cref{fig:pretzel}. If more than one of $u_1,u_2,u_3$ is even, $K$ is a link of more than one component; the same is true if all three parameters are odd. We may therefore assume that $u_1=2p$ is even and $u_2=2q+1$, $u_3=2r+1$ are odd, with $p\geq 1, q,r\geq 0$, in which case $K$ is a pretzel knot from the list, which we excluded.
\begin{figure}[b]
\begin{center}
\includegraphics[scale=1.2]{fig_abxabx.pdf}
\caption{$abxabx\to \varnothing$ using one twist on four strands, followed by another twist on two strands, at the locations marked $\star$. The first step is an isotopy, moving the blue strand. The last step is a crossing change near $\star$, followed by an isotopy fixing the endpoints of the braid strands.} \label{fig:abxabx}
\end{center}
\end{figure}

Secondly, if $(n,t)=(1,5)$, we have $\beta=\delta a^{u_1}b^{u_2}x^{u_3}a^{u_4}b^{u_5}$. If the exponents $u_i$ are all odd, $\beta$ closes to a two component link, a contradiction to $K$ being a knot. Therefore at least one of the $u_i$ is even. We may assume that $u_1$ is even, because the exponents $u_1,u_2,\ldots,u_t$ may be cyclically permuted without changing the braid closure, as explained in \cref{sec:Xu} after \cref{lem:xuform}. In particular, we may assume that $u_1\geq 2$. Then, since $\delta=xb$,
\[ \beta \ \sim \ a^{u_1} b^{u_2} x^{u_3} a^{u_4} b^{u_5} x b \ \leadsto \ a^2 b x a b x b \ =\ a (abx)^2 b \ \to \ ab, \]
where `$\sim$' denotes conjugation, `$\leadsto$' denotes the existence of a cobordism whose genus equals the difference of the 3-genera of the knots it connects and `$\to$' is shown in \cref{fig:abxabx}. Here, the cobordism `$\leadsto$' is built from saddles decreasing the exponents~$u_i$. Since the closure of $a(abx)^2 b$ has genus $3$ while only $2$ twists are used in `$\to$', we obtain $\gtop(K)\leq g(K)-1<g(K)$.

Finally, if $t\geq 6$, we proceed similarly as in the previous case.
First, assume via conjugation that $u_1$ is even. If $t > 6$ or $n > 0$, then
\[ \beta \ \leadsto \ a^{u_1} b^{u_2} x^{u_3} a^{u_4} b^{u_5} x^{u_6} b \ \leadsto \ a (abx)^2 b \ \to \ ab. \]
If $t = 6$ and $n = 0$, then $\beta=a^{u_1}b^{u_2}x^{u_3}a^{u_4}b^{u_5}x^{u_6}$. Again, the parity of the exponents $u_i$ determines whether $\beta$ closes to a knot or a multi-component link. In order to obtain a knot, at least two of them, say $u_i$ and $u_j$, need to be even. We may assume that $(i,j)=(1,2)$ or $(i,j)=(1,4)$. To see this, use cyclic permutation (as explained above in the case $(n,t)=(1,5)$) and the fact that $u_1,u_3$ cannot be the only even exponents, again because $\beta$ would not close to a knot if they were. Therefore, smooth cobordisms bring us to 
$a^2 b^2 x a b x$ or to $a^2 b x a^2 b x$. In the first case,
\[ a^2 b^2 x a b x\  =\ a^2 b \delta^{-1} \delta b x a b x \ =\ a^2 b \delta^{-1} b (abx)^2 \ \to \ a^2 b \delta^{-1} b \ =\ a^2 b a^{-1}\ \sim \ ab. \]
For the second case, \cref{fig:abxaabx} shows how to turn $a(abxa^2bx)$ into $a\gamma$ using two twists. Here, $\gamma$ is the tangle shown in the top right corner of the figure. Note that $a\gamma$ describes the unknot when closed like a braid. Since the closure of $a^2bxa^2bx$ has genus $3$ (see \eqref{eq:genus}), we obtain $\gtop(K)\leq g(K)-1<g(K)$. This concludes the case that $K$ is strongly quasipositive.

\begin{figure}[b]
\begin{center}
\includegraphics[scale=1.2]{fig_abxaabx.pdf}
\caption{How to turn the braid $abxa^2bx$ (top left) into the tangle $\gamma$ (top right) using one twist on four strands, followed by another twist on two strands, at the locations marked $\star$.} \label{fig:abxaabx}
\end{center}
\end{figure}

-- If the mirror of $K$ is strongly quasipositive, we apply the above argument to the mirror of $K$; since both $\gtop$ and $g$ are invariant under taking mirror images, we  obtain $\gtop(K)<g(K)$ again.

-- If $K$ or its mirror is the knot $T(2,2n+1)\#T(2,-2m-1)$, with $n\geq m\geq 1$, it has a Seifert surface $S$ of genus $n+m=g(K)$ that contains a copy of the ribbon knot $R\coloneqq T(2,2m+1)\#T(2,-2m-1)$, bounding a subsurface of~$S$ of genus $g(R)=2m$. A surgery that cuts this subsurface off $S$ and replaces it with a slice disk for $R$ gives rise to a smooth surface of genus $n+m-2m=n-m$ embedded in the four-ball, with boundary~$K$. This shows that $g_4(K) \leq n-m \leq n+m-2 < g(K)$, because $m\geq 1$ by assumption. Here, we use the standard notation $g_4(K)$ for the smooth $4$-genus. In particular, we obtain $\gtop(K)<g(K)$.

-- What remains are the $3$-braid knots $K$ such that neither $K$ nor its mirror is among the following: strongly quasipositive, a connected sum of the form $T(2,2n+1)\#T(2,-2m-1)$, $n\geq m\geq 1$, or the figure-eight knot. For such $K$, Lee--Lee~\cite{LL} prove the following bound on the unknotting number: $u(K)<g(K)$. Since $\gtop(J)\leq g_4(J)\leq u(J)$ holds for all knots $J$, this implies $\gtop(K)<g(K)$ and completes the proof.
\end{proof}

For comparison, we note the following analog of \cref{thm:1}, in which the smooth $4$-genus $g_4(K)$ replaces $\gtop(K)$, and Rasmussen's invariant $s(K)$
from Khovanov homology~\cite{Ras}
(or any other slice-torus invariant~\cite{Liv}, e.g.~the Heegaard--Floer $\tau$-invariant)
plays the role of the signature $\sigma(K)$.

\begin{proposition}
Let $K$ be a $3$-braid knot other than the figure-eight knot. Then
\[ |s(K)| = 2g(K) \quad \Longleftrightarrow\quad g_4(K) = g(K) \]
These equalities hold precisely when $K$ or its mirror is strongly quasipositive.
\end{proposition}

\begin{proof}
If $K$ or its mirror is strongly quasipositive, then $|s(K)|=2g(K)$ 
follows, see~\cite[Proposition~1.7]{Shu}. 
Moreover, the implication $|s(K)| = 2g(K) \Rightarrow g_4(K) = g(K)$ holds for all knots $K$,
because of the inequalities $|s(K)| \leq 2g_4(K) \leq 2g(K)$.
It remains to prove that if $K$ is a 3-braid knot other than the figure-eight knot with $g_4(K) = g(K)$, then $K$ or its mirror is strongly quasipositive. This follows from Lee--Lee's results~\cite{LL}. More precisely, for a 3-braid knot~$K$ with $g_4(K) = g(K)$ Theorem~1.1 in \cite{LL} implies $u(K)=g(K)$. By Theorem~1.3 of the same paper, $K$ or its mirror is either strongly quasipositive or a connected sum of two-strand torus knots $T(2,2n+1)\#T(2,-2m-1)$ with $n\geq m\geq 1$. The latter can be excluded as in the proof of our \cref{thm:1} by
showing $g_4(K)\leq n-m \leq n+m-2<g(K)$, a contradiction to $g_4(K)=g(K)$.
\end{proof}

\begin{theorem} \label{thm:2}
Let $K$ be a strongly quasipositive $3$-braid knot, written as the closure of a $3$-braid $\beta$ in Xu normal form $\beta=\delta^n \tau_1^{u_1}\tau_2^{u_2}\cdots \tau_t^{u_t}$, with $u_1,\ldots, u_t\geq 1$ and $n\geq 0$. Then the topological four-genus defect of $K$ is bounded as follows: 
\[ \frac{n}{3}+\frac{t}{3}-1\geq g(K)-\gtop(K) \geq \frac{n}{3}+\frac{t}{6}-3. \]
The constants $\frac13$ and $\frac16$ in the second inequality are optimal in the following sense:
Whenever $C>\frac13$ or $D>\frac16$, and $E\in\R$, there exists a $3$-braid in Xu normal form as above with $n\geq 0$ such that its braid closure $K$ satisfies
\[ g(K)-\gtop(K) < Cn+Dt-E. \]
\end{theorem}

\begin{proof}
We begin with the upper bound $\frac{n}{3}+\frac{t}{3}-1\geq g(K)-\gtop(K)$. The case $t=0$, in which $K=T(3,n)$, is covered by \cite[Theorem~1]{BBL}:
\[
g(K)-\gtop(K) = n-1 -\left\lceil \frac{2n}{3}\right\rceil \leq \frac{n}{3}-1.
\]
For $t > 0$, we first use \cref{prop:signature} to compute the absolute value of the signature $|\sigma(K)| = -\sigma(K) = U + \frac43n - \frac23t$, where $U=u_1+\ldots+u_t$, and recall that $g(K)=\frac{U}{2} + n-1$ from \eqref{eq:genus}, \cref{sec:Xu}. The bound then follows directly from Kauffman and Taylor's classical bound $|\sigma(K)|\leq 2\gtop(K)$.

To establish the lower bound, we first apply a smooth cobordism from $K$ to a knot $K'$, by suitably lowering the exponents $n,u_1,u_2,\ldots,u_t$ in $\beta$, as explained in \cref{fig:saddle} above.
First, assume $n > 0$.
We set $K'\coloneqq \cl(\delta^m(abx)^{\frac{s}{3}})$, where
\[ s=6\left\lfloor\frac{t}{6}\right\rfloor\quad\text{and}\quad m=\left\{\begin{array}{ll}n-2 & \text{if } n\equiv 0\mod 3\\ n-1 & \text{if }n\equiv 2\mod 3\\ n & \text{if }n\equiv 1\mod 3.\end{array}\right. \]
We have $K\leadsto K'$, and so this is a smooth cobordism which does not increase the topological four-genus defect. In other words,
\[ g(K)-\gtop(K)\geq g(K')-\gtop(K'), \]
as in \eqref{eq:defect2}.
Next, we apply the untwisting move $(abx)^2\to\varnothing$ from \cref{fig:abxabx} exactly $\frac{s}{6}$ times to the knot $K'$, resulting in $\cl(\delta^m)=T(3,m)$. Since $m\equiv 1\mod 3$, this is a knot again. For $m\geq 4$, \cite[Lemma~5~(1)]{BBL} yields $\tu(T(3,m))\leq \frac23 m+\frac13$.
This inequality still holds for $m = 1$
and adds up to
\[ \gtop(K')\leq \tu(K')\leq 2\cdot\frac{s}{6}+\frac23m+\frac13. \]
Since $g(K')=\frac{s}{2}+m-1$ by \eqref{eq:genus}, and since $\frac{s}{6}\geq \frac{t}{6}-\frac56$ and $m\geq n-2$, we obtain
\begin{eqnarray}
g(K)-\gtop(K) 
\geq \frac{s}{6}+\frac{m}{3}-\frac43 \nonumber 
     \geq\frac{t}{6}+\frac{n}{3}-3. \nonumber
\end{eqnarray}
In the case $n = 0$, the above procedure fails because $m$, as defined above, is negative and the smooth cobordism to $K'$ might therefore increase the four-genus defect. However, a simple cosmetic modification allows for a cobordism that only increases the four-genus defect by at most one. Specifically, we set $m=1$ (instead of $m=-2$) and $K'=\cl(\delta(abx)^{\frac{s}{3}})$. The cobordism from $K$ to~$K'$ is then given by lowering the exponents $u_1,u_2,\ldots,u_t$ in $\beta$ as above while increasing the exponent of $\delta$ from $0$ to $1$. This gives
\[ g(K)-\gtop(K)\geq g(K')-\gtop(K')-1. \]
Now we apply the $\frac{s}{6}$ untwisting moves $(abx)^2\to\varnothing$ as above. The result is the unknot $\cl(\delta)$. Since $n=0$, we again obtain the claimed bound
\[ g(K)-\gtop(K) \geq \frac{s}{6}-1\geq \frac{t}{6}-\frac56 -1\geq\frac{t}{6}+\frac{n}{3}-3. \]

In order to demonstrate optimality of the constants, we consider the two special families of $3$-braids $\delta^n$ and $(abx)^n$, which we slightly modify to $\delta^{3k+1}$ and $(abx)^{2k}abx^2abx^2$, to make sure that their braid closure is connected.

-- For $K=T(3,3k+1)$ with $k\geq 1$, the closure of the braid $\delta^{3k+1}$, which is in Xu normal form with $n=3k+1$ and $t=0$, we have $g(K)=3k$ and $\gtop(K)=2k+1$ by~\cite{BBL}, hence $g(K)-\gtop(K)=k-1$. Whenever $C>\frac13$ and $D,E$ are arbitrary constants, we will therefore have
\[ g(K)-\gtop(K)=k-1<C\cdot (3k+1)+D \cdot 0-E \]
for sufficiently large $k$.

-- If $K$ is the closure of $(abx)^{2k}abx^2abx^2$, which is in Xu normal form with $n=0$ and $t=6k+6$, then $g(K)=3k+3$. By Gambaudo--Ghys~\cite[Corollary~4.4]{GG},
for any $3$-braid~$\beta$ with closure $J$, the Levine--Tristram signature function of $J$, $[0,1]\to\Z$, $t\mapsto \sigma_{e^{2\pi i t}}(J)$, grows linearly on $(0,\frac13)$ with slope $-2$ times the writhe of $\beta$, up to a pointwise error of at most $2$ (see e.g.~\cref{fig:sigprofile} at the end of the paper). For strongly quasipositive $\beta$ with $J$ a knot, that slope is $-4(g(J)+1)$, see
\eqref{eq:genus}, and hence
\begin{equation}\label{eqn:sigmahat}
\widehat{\sigma}(J)\coloneqq \max_{\omega\in S^1\setminus\Delta_J^{-1}(0)} |\sigma_\omega(J)|\geq \frac43(g(J)+1)-2.
\end{equation}
Since $\frac12|\sigma_\omega(J)|\leq \gtop(J)$ whenever $\omega\in S^1$ is not a root of the Alexander polynomial of~$J$ (see~\cite{KT,P}), we obtain $\frac23(g(K)+1)-1=2k+\frac53\leq \gtop(K)$. Hence, if $C,E$ are arbitrary constants and $D>\frac16$, we have
\[ g(K)-\gtop(K)\leq 3k+3-2k-\frac53=k+\frac43 < C\cdot 0 + D\cdot (6k+6) - E, \]
for sufficiently large $k$. In this case, it does not suffice to consider the (classical) signature bound on $\gtop(K)$. Indeed, $|\sigma(K)|=2k+4$ (see~\cref{prop:signature}) is roughly half the maximal Levine--Tristram signature $\widehat{\sigma}(K)$. Substituting $|\sigma(K)|$ for $\widehat{\sigma}(K)$ in the above argument would therefore not work.
\end{proof}
