Our tool to handle 3-braids is what we call their \emph{Xu normal form}.
It was developed by Xu~\cite{Xu} (who called it \emph{representative symbol}),
as a variation of the Garside normal form~\cite{G}.
Using the Xu normal form, one may decide whether two given 3-braids are conjugate~\cite{Xu}, and whether their closures are equivalent links~\cite{BM1,BM2}. Later, the Xu normal form was generalized to braids on arbitrarily many strands by Birman--Ko--Lee~\cite{BKL}.
The Garside, Xu and BKL normal forms are all examples of Garside structures on (braid) %
groups~\cite{DDGKM}.
In this section, we will introduce the Xu normal form
and show how it determines the signature invariant $\sigma$ of its closure 
(\cref{prop:signature}),
as well as strong quasipositivity and braid positivity~(\cref{prop:positivity}).

A \emph{3-braid} $\beta$ is an element of the braid group $B_3 = \langle a,b \mid aba = bab \rangle$.
By closing off the three strands of $\beta$ and 
interpreting $a$ and $b$ as positive crossings between the first two and last two strands, respectively,
$\beta$ gives rise to a link called \emph{closure of $\beta$}, denoted $\cl(\beta)$.
Note that throughout the text, links are oriented.
Let us write $x \coloneqq a^{-1}ba \in B_3$ and $\delta \coloneqq ba = ax = xb \in B_3$. 
By a \emph{Xu word} or simply \emph{word}, we mean a word with letters $a,b,x,\delta$ and their inverses.
We reserve the equality sign $=$ for equality of braids, and use a dotted equality sign $\doteq$ for equality of words.
Moreover, we write $\beta \sim \gamma$ if the two braids $\beta, \gamma$ are conjugate. 
For efficiency, let us also introduce the following notation:
for any $i\in \Z$, set
$\tau_i \doteq a$ if $i\equiv 1 \pmod{3}$, $\tau_i \doteq b$ if $i\equiv 2 \pmod{3}$ and $\tau_i \doteq x$ if $i \equiv 0 \pmod{3}$.
\begin{definition}\label{def:bkl}
Let $w$ be a word of the form
\begin{equation}\label{eq:bkl}
w \doteq \delta^n \tau_1^{u_1} \tau_2^{u_2} \ldots \tau_t^{u_t}
\quad\text{for}\quad n\in \Z, t \geq 0, u_i \geq 1.
\end{equation}
We say that $w$ is in \emph{Xu normal form}
if the tuple $(-n,t,u_1, \ldots, u_t)$ is lexicographically minimal among all
words of the form \eqref{eq:bkl} representing the same conjugacy class of 3-braids.
\end{definition}
The condition of lexicographic minimality means, in particular, that
the Xu normal form maximizes $n$, and afterwards minimizes $t$.
The term `normal form' is justified by the following.
\begin{theorem}\label{thm:xuform}
Every 3-braid is conjugate to a unique word in Xu normal form.
\qed
\end{theorem}
The following lemma gives a criterion to easily decide whether a word is in Xu normal form.
\begin{lemma}\label{lem:xuform}\label{lem:bkl}
A word $w \doteq \delta^n \tau_1^{u_1} \tau_2^{u_2} \ldots \tau_t^{u_t}$
for some $n\in \Z$, $t \geq 0$, $u_i \geq 1$
is in Xu normal form if and only if one of the following conditions holds:
\begin{enumerate}[label=(\alph*)]
\item $t = 0$. In this case $w \doteq \delta^n$.
\item $t = 1$, and if $n\equiv 1\pmod{3}$ then $u_1 = 1$.
In this case, $w \doteq \delta^{3k} a^{u_1}$, or $w \doteq \delta^{3k+1} a$, or $w \doteq \delta^{3k+2} a^{u_1}$.
\item $t \geq 2$, $n + t \equiv 0\pmod{3}$ and the tuple $(u_1, \ldots, u_t)$ is lexicographically minimal among its cyclic permutations.
\end{enumerate}
\end{lemma}

The proofs of \cref{thm:xuform} and \cref{lem:xuform} are essentially contained in Xu's paper \cite[Section~4]{Xu}, albeit with slightly different conventions.
In our setup, \cref{thm:xuform} is not actually hard to prove, and makes a good exercise to get acquainted with the calculus of Xu words. The same is true for the `only if' direction of \cref{lem:bkl}.
Let us provide two hints. Firstly, the easily verifiable rules
\begin{equation}\label{eq:xurules}
\delta = \tau_i \tau_{i-1},\qquad
\tau_i\delta^n = \delta^n\tau_{i+n},\qquad
\tau_i^{-1}\delta^n = \delta^{n-1}\tau_{i+n+1}
\end{equation}
allow
to find Xu words for every 3-braid without the letters $a^{-1}, b^{-1}, x^{-1}$,
and to `pull all $\delta^{\pm 1}$ to the left' in a Xu word.
In this way, one can find a Xu word of the form 
$\delta^n \tau_{m+1}^{u_{1}} \ldots \tau_{m+t}^{u_{t}}$
with $m\in\Z$ and $u_i \geq 1$ for any 3-braid. Secondly, note that
\begin{multline*}
\delta^n \tau_1^{u_1} \ldots \tau_t^{u_t} \ =\ 
\tau_{1-n}^{u_1}\delta^n \tau_2^{u_2} \ldots \tau_t^{u_t}
\ \sim\  \delta^n \tau_2^{u_2} \ldots \tau_t^{u_t} \tau_{1-n}^{u_1}\\
\sim\ \delta^{n+1} \tau_2^{u_2} \ldots \tau_t^{u_t} \tau_{1-n}^{u_1} \delta^{-1}
\ =\ \delta^n \tau_1^{u_2} \ldots \tau_{t-1}^{u_t} \tau_{-n}^{u_1}.
\end{multline*}
So cyclically permuting the tuple $(u_1, \ldots, u_t)$
results in a
conjugate braid if $-n \equiv t\pmod{3}$.
\begin{proof}[Proof of the `if' direction of \cref{lem:bkl}]
Xu proves that condition (c) is sufficient for $w$ to be in Xu normal form (see the definition of the representative symbol and Theorem~5 in ~\cite{Xu}), but omits a discussion of conditions (a) and (b).
So suppose $w$ satisfies (a) or (b), $w' = \delta^{n'} \tau_1^{u'_1} \ldots \tau_{t'}^{u'_{t'}}$
is in Xu normal form, and $w'\sim w$. We need to show that $w \doteq w'$.
Let us distinguish the following cases.
\begin{itemize}
\item If $w$ satisfies (a), then $w \doteq \delta^n$.
The words $w$ and $w'$ must have the same writhe, so $2n = 2n' + u'_1 + \cdots + u'_{t'}$.
But since $w'$ is in Xu normal form, $n' \geq n$. Because $u'_i \geq 0$, it follows that $n' = n$ and $t' = 0$, thus $w \doteq w'$ as desired.
\item If $w$ satisfies (b) and $w\doteq \delta^n a$, a similar argument applies:
the equality of the
writhes of $w$ and $w'$ now reads as 
$2n + 1 = 2n' + u'_1 + \cdots + u'_{t'}$. Because of parity, we may again deduce $n' = n$. It follows that $t' = 1, u'_1 = 1$ and $w \doteq w'$ as desired.
\item The remaining case is that $w \doteq \delta^n a^{u_1}$ satisfies (b), $u_1 \geq 2$, and $n\not\equiv 1\pmod{3}$.
Now the so-called $r$-index of $w$ defined below Lemma~5 in~\cite{Xu} is~0,
and so \cite[Theorem~4]{Xu} implies that $n$ is maximal. It follows that $n' = n$,
and hence the minimality of $t'$ implies $t' \leq t = 1$. 
Finally, from the equality of the writhes of $w$ and $w'$, it follows that $t' = 1$ and $u_1 = u'_1$, and thus $w \doteq w'$.\qedhere
\end{itemize}
\end{proof}
To get a link invariant from the Xu 
normal form,
we need to understand the relationship between conjugacy classes of 3-braids and link equivalence classes of their closures. Birman--Menasco have shown that with a few well-understood exceptions, this relationship is one-to-one:
\begin{theorem}[\cite{BM1,BM2}]\label{thm:braidtolink}
Two 3-braids are conjugate if their closures are equivalent links, except in the following cases:
\begin{enumerate}
\item The non-conjugate braids $ab, ab^{-1}, a^{-1} b^{-1}$ have the unknot as closure.
\item For $n\in\Z\setminus\{\pm 1\}$, the non-conjugate braids $a^n b, a^n b^{-1}$ have the $T(2,n)$ torus link as closure.
\item For pairwise distinct integers $p,q,r\in \Z\setminus \{0,-1,-2\}$, the two non-conjugate braids $\beta = a^p b^q x^r$ and $\gamma = a^p b^r x^q$ have the $P(p,q,r,1)$ pretzel link as closure; and the two non-conjugate braids $\beta^{-1}$ and $\gamma^{-1}$ have the $P(-p,-q,-r,-1)$ pretzel link as closure.
\end{enumerate}
\end{theorem}
The following corollary allows us to sidestep the exceptional cases (1), (2), (3) in the above theorem
by focusing on links of braid index 3 instead of 3-braid links
(the latter class of links includes links with braid index 1 and 2, i.e.~2-stranded torus links).
Let the \emph{reverse} of a braid $\beta\in B_3$, denoted by $\rev(\beta)$,
be the braid given by reading $\beta$ backwards and switching $a$ with $b$,
and $a^{-1}$ with $b^{-1}$. Note that $\cl(\rev(\beta))$ is obtained from $\cl(\beta)$ by reversing the link's orientation.
\begin{corollary}\label{cor:xu}
Let $L$ be a link of braid index 3.
\begin{enumerate}
\item Either there is a unique conjugacy class of 3-braids with closure $L$,
or there are two of them, such that one consists of the reverses of the braids contained in the other.
\item
The numbers $n$, $t$, and $U = u_1 + \cdots + u_t$
of the Xu normal form of a braid with closure $L$
do not depend on the choice of braid.
Thus $n$, $t$ and $U$ are link invariants of links with braid index 3.
\end{enumerate}
\end{corollary}
\begin{proof}
Claim (1) follows quickly from \cref{thm:braidtolink},
since the unknot and the two-stranded torus links have braid index less than 3,
and $\rev(a^p b^q x^r) = x^r a^q b^p \sim a^p b^r x^q$.
For (2), note that $\rev(x) = x$ and so $\rev(\tau_i) = \tau_{-i}$. Also $\rev(\delta) = \delta$.
Thus the reverse of $\delta^n \tau_1^{u_1} \tau_2^{u_2} \ldots \tau_t^{u_t}$ is
$\tau_{-t}^{u_t} \ldots \tau_{-1}^{u_1} \delta^n$, which has Xu normal form
$\delta^n \tau_1^{u_t} \tau_2^{u_{t-1}} \ldots \tau_t^{u_1}$ (up to cyclically permuting the exponents $u_t, \ldots, u_1$).
So the numbers $n$, $t$ and $U$ do not change under braid reversal. Together with (1), this implies (2).
\end{proof}

Xu calls $n$ and $t$ the \emph{power} and \emph{syllable length},
while Birman--Ko--Lee use the terms \emph{infimum} and \emph{canonical length}, respectively.

Since the Xu normal form determines the link type, all link invariants may be read off it.
Let us first prove a formula for the signature invariant.
We will need the Garside normal form \cite{G}, which we introduce now. The reader will note many parallels between the Garside and Xu normal forms.
Let $\Delta\coloneqq aba$. A \emph{Garside word} is a word with letters $a, b, \Delta$ and their inverses. Again, we use $\doteq$ for equality of words.
We also use the following notation:
for any $i\in \Z$, set $\sigma_i \doteq a$ if $i\equiv 1 \pmod{2}$ and $\sigma_i \doteq b$ if $i\equiv 0 \pmod{2}$.
\begin{proposition}[{\cite[Proposition 3.2]{T}}]\label{prop:garside}
Every 3-braid contains in its conjugacy class a unique Garside word $v$ in
\emph{Garside normal form}, i.e.\ a word
\[
v \doteq \Delta^{\ell} \sigma_1^{p_1} \sigma_2^{p_2} \ldots \sigma_r^{p_r},
\]
with $\ell \in \Z$, $r\geq 0$, $p_i\geq 1$, satisfying one of the following conditions:
\begin{enumerate}[leftmargin=4em]
\item[(A)] $\ell$ is even and $r\in \{0,1\}$, i.e.~$v \doteq \Delta^{2k} a^{\geq 0}$,
\item[(B)] $\ell$ is even, $r = 2$, $p_1\in \{1,2,3\}$ and $p_2 = 1$, i.e.~$v \doteq \Delta^{2k} a^{\{1,2,3\}} b$,
\item[(C)/(D)] $r \geq 1$, $p_i \geq 2$, $\ell \equiv r \pmod{2}$, and the tuple
$(p_1, \ldots, p_r)$ is lexicographically minimal among its cyclic permutations.
\end{enumerate}
We refer by (C) and (D) to the case that $\ell$ is even and odd, respectively.
\end{proposition}
The following lemma tells us how to convert between the Xu and the Garside normal forms.
\begin{lemma}\label{lem:convert}
Let a word $w \doteq \delta^n \tau_1^{u_1} \tau_2^{u_2} \ldots \tau_t^{u_t}$ in Xu normal form be given.
Then the unique word $v$ in Garside normal form representing the same conjugacy class of 3-braids as $w$
is given by the following table.\medskip

\noindent
\begin{tabular}{ll|ll}\hline\rule{0pt}{5ex}%
\parbox[b]{5em}{\raggedright Case in\\\cref{lem:bkl}} &
\parbox[b]{5em}{\raggedright Xu normal form $w$} &
\parbox[b]{7em}{\raggedright Garside normal\\ form $v$} &
\parbox[b]{7em}{Case in\\\cref{prop:garside}} \\\hline
\rule{0pt}{3ex}%
\emph{(a)} & $\delta^{3k}$           & $\Delta^{2k}$             & \emph{(A)} \\[1ex]
\emph{(a)} & $\delta^{3k+1}$         & $\Delta^{2k}ab$           & \emph{(B)} \\[1ex]
\emph{(a)} & $\delta^{3k+2}$         & $\Delta^{2k} a^3b$        & \emph{(B)} \\[1ex]
\emph{(b)} & $\delta^{3k} a^{u_1}$   & $\Delta^{2k} a^{u_1}$     & \emph{(A)} \\[1ex]
\emph{(b)} & $\delta^{3k+1} a$       & $\Delta^{2k} a^2 b$       & \emph{(B)} \\[1ex]
\emph{(b)} & $\delta^{3k+2} a^{u_1}$ & $\Delta^{2k+1} a^{1+u_1}$ & \emph{(D)} \\[1ex]
\emph{(c)} & $\delta^n \tau_1^{u_1} \ldots \tau_t^{u_t}$
                              & $\Delta^{(2n-t)/3} \sigma_1^{1+u_1} \ldots \sigma_t^{1+u_t}$
                                                          & \emph{(C)/(D)}
\raisebox{-1.5ex}{}\\\hline
\end{tabular}
\end{lemma}
\begin{proof}
All rows in the table except for the last one may be checked quickly, using $\delta^3 = \Delta^2$.
Let us now prove $w\sim v$ for the words $w$ and $v$ in the last row.
Let $n = 3k + m$ and $t = 3s + 3 - m$ for $k,s\in\Z, s\geq 0, m\in\{1,2,3\}$.
In the Xu word $w$, replace $\delta^n$ by $(ba)^m\Delta^{2k}$.
Moreover, replace every $x^u$ by $\Delta^{-1} ab^{1+u} a$.
These replacements yield a Garside word $v_1$ with $v_1 = w$ and
\[
v_1 \doteq (ba)^m \Delta^{2k} a^{u_1} b^{u_2} (\Delta^{-1} ab^{1+u_3}a) a^{u_4} \ldots
(\Delta^{-1} ab^{1+u_{3s}}a) a^{u_{3s+1}}b^{u_{3s+2}},
\]
where we set $u_i = 0$ if $i > t$.
Now proceed by `pulling all the $\Delta^{-1}$ to the right', i.e.~replacing
$\Delta^{-1}a$ by $b\Delta^{-1}$ and $\Delta^{-1}b$ by $a\Delta^{-1}$ as long as possible.
These replacements produce a word $v_2$ with $v_2 = v_1$, where $v_2$ starts with $(ba)^m\Delta^{2k} a^{u_1} b^{u_2} (ba^{1+u_3}b) b^{u_4} \ldots$.
Using the $\sigma_i$-notation and noting that there are precisely $s$ occurrences of $\Delta^{-1}$ in $v_1$, we have
\begin{align*}
v_2 & \doteq (ba)^m\Delta^{2k} \sigma_1^{u_1} \sigma_2^{1+u_2} \sigma_3^{1+u_3} \sigma_4^{1+u_4} \ldots \sigma_{3s}^{1+u_{3s}} \sigma_{3s+1}^{1+u_{3s+1}}\sigma_{3s+2}^{u_{3s+2}} \Delta^{-s} \\
\sim v_3  & \doteq \Delta^{-s}(ba)^m\Delta^{2k} \sigma_1^{u_1} \sigma_2^{1+u_2}  \ldots \sigma_{3s}^{1+u_{3s}} \sigma_{3s+1}^{1+u_{3s+1}}\sigma_{3s+2}^{u_{3s+2}}.
\end{align*}
Let us now consider the three possibilities for $m$ case by case.
\begin{itemize}
\item If $m = 3$, then $u_{3s+2} = u_{3s+1} = 0$ and
\begin{align*}
v_3 & \doteq \Delta^{-s}(ba)^3\Delta^{2k} \sigma_1^{u_1} \sigma_2^{1+u_2} \ldots \sigma_{3s}^{1+u_{3s}} \sigma_{3s+1} \\
    & = \Delta^{2k + 2 - s} \sigma_1^{u_1} \sigma_2^{u_2 + 1} \ldots \sigma_{3s}^{1+u_{3s}}\sigma_{3s+1} \\
\sim v_4  & = \Delta^{2k + 2 - s} \sigma_1^{1+u_1} \sigma_2^{u_2 + 1} \ldots \sigma_{3s}^{1+u_{3s}}.
\end{align*}
We have $v_4 \doteq v$ as desired, since $2k + (m - 1) - s = (2n-t)/3$.
\item If $m = 2$, then $u_{3s+2} = 0$ 
and
\begin{align*}
v_3 & \doteq \Delta^{-s}(ba)^2\Delta^{2k} \sigma_1^{u_1} \sigma_2^{1+u_2} \ldots \sigma_{3s}^{1+u_{3s}} \sigma_{3s+1}^{1+u_{3s+1}} \\
    & = \Delta^{2k + 1 - s} \sigma_1^{1+u_1} \sigma_2^{1+u_2} \ldots \sigma_{3s}^{1+u_{3s}} \sigma_{3s+1}^{1+u_{3s+1}} \ \doteq v.
\end{align*}
\item If $m = 1$, then
\begin{align*}
 v_3  & \doteq \Delta^{-s}ba\Delta^{2k} \sigma_1^{u_1} \sigma_2^{1+u_2}  \ldots \sigma_{3s}^{1+u_{3s}} \sigma_{3s+1}^{1+u_{3s+1}} \sigma_{3s+2}^{u_{3s+2}} \\
      & = \sigma_{s} \Delta^{2k - s} \sigma_1^{1+u_1}\sigma_2^{1+u_2}  \ldots  \sigma_{3s+1}^{1+u_{3s+1}} \sigma_{3s+2}^{u_{3s+2}} \ \sim v.
\qedhere
\end{align*}
\end{itemize}
\end{proof}
We rely on the signature formula for 3-braids in Garside normal form deduced by the fourth author
\cite[Remark~1.6, Proposition~4.2, Remark~4.3]{T} from a result by Erle~\cite[Theorem~2.6]{E}.%
\begin{proposition}[\cite{T}]\label{prop:garsidesig}
Let $K$ be a knot that is the closure of a Garside normal form $\Delta^{\ell} \sigma_1^{p_1} \ldots \sigma_{r}^{p_r}$ in case (C)/(D) of \cref{prop:garside}. Then
\[
\sigma(K) = -2\ell + r - \sum_{i=1}^r p_i.
\]
\end{proposition}

We are now ready to state and prove our signature formula for 3-braids in Xu normal form.
\begin{proposition}\label{prop:signature}
Let $\delta^n \tau_1^{u_1} \tau_2^{u_2} \ldots \tau_t^{u_t}$ be the Xu normal form
of a 3-braid whose closure is a knot $K$ of braid index 3.
Set $U = u_1 + \cdots + u_t$. If $t > 0$ (equivalently, if $K$ is not a torus knot), then
\[
\sigma(K) = -U - \frac43 n + \frac23t.
\]
In the case $t = 0$, i.e.\ $U = 0$ and $K = T(3,n)$, the value $\sigma(K) =  -\frac43 n$
given by the above formula is only approximately true, with an error of at most $\frac43$.
In fact, 
in that case we have
\[
\sigma(K) = - \frac43 n + \left(2 + 4\left\lfloor\frac16 n\right\rfloor - \frac23 n\right)
=
2 - 2n + 4\left\lfloor\frac16 n\right\rfloor.
\]
\end{proposition}
\begin{proof}
Our signature formula for torus knots may be seen to agree with
the formula given e.g.~in~\cite[Proposition~9.1]{M2}.
So we are left with the case $t\geq 1$, i.e.~cases (b) and (c) in \cref{lem:bkl}.
Denote the Xu normal form in question by $w$.
Let us first consider case~(c).
Then $w$ has Garside normal form $\Delta^{(2n-t)/3} \sigma_1^{1+u_1}\ldots\sigma_t^{1+u_t}$,
see \cref{lem:convert}. By \cref{prop:garsidesig},
we have $\sigma(K) = -4n/3 + 2t/3 + t - (U + t)$, which is equal to the claimed formula.
In case~(b), since the closure of $w$ is a knot, the only possibility is $w \doteq \delta^{3k+2} a^{u_1}$. Then the Garside normal form of $w$ is $\Delta^{2k+1} a^{1+u_1}$,
which has the desired signature, again by \cref{prop:garsidesig}.
\end{proof}
Next, let us give complete criteria to decide braid positivity and strong quasipositivity for links of braid index 3.
A \emph{braid positive} link is the closure of some \emph{positive word}, i.e.~a word in positive powers of the standard generators $a_1, \ldots, a_{k-1}$ of the braid group $B_k$ on some number $k$ of strands. Similarly, a link is called \emph{strongly quasipositive} \cite{R1,R2} if it is the closure of a \emph{strongly quasipositive word} in some $B_k$, i.e.~a word in positive powers of
\[
a_{ij} = a_i^{-1} a_{i+1}^{-1} \ldots a_{j-1}^{-1} a_j a_{j-1} \ldots a_i
\]
with $1\leq i \leq j \leq k - 1$.
Note that for $k = 3$, positive words are words in $a = a_1$, $b = a_2$ and strongly quasipositive words are words in $a = a_{11}$, $b = a_{22}$ and $x = a_{12}$.
It is well-known and straightforward to show that a $k$-braid is the closure of some (strongly quasi-)positive word if and only if the power of $\Delta$ ($\delta$) in its Garside (Birman--Ko--Lee)
normal form is non-negative.
This makes (strong quasi-)positivity decidable for braids.
For links however, the problem is harder because
a priori, a braid positive (strongly quasipositive) link with braid index $k$
need not be the closure of a (strongly quasi-)positive word on $k$ strands.
For $k = 3$, however, this is the case:
\begin{theorem}[{\cite[Theorem~1.1 and 1.3]{S}}]\label{thm:posstoimenow}
The following hold. 
\begin{enumerate}
\item If a strongly quasipositive link is the closure of some 3-braid, then it is the closure of a strongly quasipositive 3-braid.
\item If a braid positive link is the closure of some 3-braid, then it is the closure of a positive 3-braid.
\end{enumerate}
\end{theorem}
We are now ready to prove our positivity characterizations. For braid positivity, we will once again need the Garside normal form introduced above.%
\begin{proposition}\label{prop:positivity}
Let $\delta^n \tau_1^{u_1} \tau_2^{u_2} \ldots \tau_t^{u_t}$ be in Xu normal form,
with closure a link $L$ of braid index 3. Then the following hold.
\begin{enumerate}
\item $L$ is a strongly quasipositive link if and only if $n \geq 0$.
\item $L$ is a braid positive link if and only if $n \geq t/2$ or $n = 0, t = 1$.
\end{enumerate}
\end{proposition}
\begin{proof}
{\em Part (1).} 
If $n\geq 0$, then the Xu normal form yields a strongly quasipositive word for $L$, so $L$ is strongly quasipositive. For the other direction, assume $L$ is strongly quasipositive and a Xu normal form $w$ with closure $L$ is given. By \cref{thm:posstoimenow}, there is a strongly quasipositive word in $B_3$ representing~$L$. It may be transformed to its Xu normal form $w'$ just by replacing $\tau_{i+1}\tau_i \to \delta$ and $\tau_i \delta \leftrightarrow \delta \tau_{i+1}$, and by passing from $y\tau_i$ to $\tau_i y$ for $y$ a word in positive powers of $a,b,x,\delta$. None of these transformations create negative powers of~$\delta$, and so we find that the Xu normal form $w'$ has $n\geq 0$.
By \cref{cor:xu}(2), all Xu normal forms with closure $L$ have the same $n$, so $w$ has $n\geq 0$ as well, which was to be proven.

{\em Part (2).} If $n \geq t/2$, then one may check using \cref{lem:convert} that the Garside normal form 
of $\delta^n \tau_1^{u_1} \tau_2^{u_2} \ldots \tau_t^{u_t}$ starts with a non-negative power of~$\Delta$,
and thus yields a positive word with closure~$L$.
If $n = 0$ and $t = 1$, then the Xu normal form is $a^{u_1}$, which is already a positive word with closure~$L$. In both cases, it follows that $L$ is braid positive.
For the other direction, assume $L$ is braid positive and let a Xu normal form $w$ with closure $L$ be given. By \cref{thm:posstoimenow},
there is a positive word in $B_3$ representing~$L$.
Similarly as in the proof of part (1), one sees that the Garside normal form of this positive word starts with $\Delta^{\ell}$ with $\ell \geq 0$. By going through the rows of the table in \cref{lem:convert}, one sees that this implies that $n \geq t/2$, with the sole exception in the fourth row if $k = 0$ and $u_1\neq 0$: then, the Xu normal form is $a^{u_1}$, so $n = 0$ and $t = 1$.
Again using \cref{cor:xu}(2), it follows that $w$ also satisfies $n \geq t/2$, or $n = 0$ and $t = 1$.
\end{proof}
If a knot $K$ is the closure of a strongly quasipositive 3-braid in Xu normal form $\delta^n \tau_1^{u_1}\ldots \tau_t^{u_t}$, there is also a simple formula for the Seifert genus of $K$:
\begin{equation}\label{eq:genus}
g(K) = \frac{U}{2} + n - 1,
\end{equation}
where 
$U = u_1 + \cdots + u_t$. This follows from the Bennequin inequality~\cite{bennequin}. 
