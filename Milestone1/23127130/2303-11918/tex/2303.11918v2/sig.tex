\begin{remark}\label{rem:sign}
The maximal Levine--Tristram signature, see \eqref{eqn:sigmahat} in \cref{sec:proofs}, provides a good computable lower bound for $\gtop$:
\begin{align*}
\widehat{\sigma}(K) = \underset{\omega \in S^1 \backslash \Delta_K^{-1}(0)}{\operatorname{max}} \left \vert \sigma_\omega(K) \right \vert \leq 2\gtop(K).
\end{align*}
The function $S^1 \to \Z$, $\omega\mapsto \sigma_{\omega}(K)$,
is piecewise constant and jumps only at zeroes of the Alexander polynomial $\Delta_K$. A priori, its maximum absolute value may be assumed anywhere on $S^1\setminus \{1\}$.

\begin{figure}[t]
\input{sig-profile.pdf_tex}
\caption{In blue, the graph of the Levine--Tristram signature $\sigma_{e^{2\pi i t}}(K)$
for $t \in \left[0,\frac12\right]$ and $K$ the closure of the 3-braid
$\left(a^2b^2\right)^8\left(a^5b^5\right)^4 \sim \delta^{12}\left(a^4b^4x^4\right)^2a^4b^4(xab)^5x$.
In black, the linear approximation by \cite[Corollary~4.4]{GG}
for $t \in [0,\frac13]$.
The maximum absolute value $\widehat{\sigma}(K)$
of $\sigma_{e^{2\pi i t}}(K)$,
which equals $|\sigma(K)| + 4 = |\sigma_{\zeta3}(K)| + 4$,
is assumed between $0.3599$ and $0.3826$ (rounded).
This graph was drawn with sage~\cite{Sage}, by computing
the signatures
of the Hermitian matrices $(1-\omega)A + (1-\overline{\omega})A^{\top}$,
for $A$ a Seifert matrix of $K$, and $\omega$
between the roots of $\Delta_K$ on $S^1$.}
\label{fig:sigprofile}
\end{figure}
In the above examples and \Cref{prop:braidposbound} 
we have seen that for certain families of $3$-braid knots, $\widehat{\sigma} = |\sigma| = 2\gtop$,
where $\sigma = \sigma_{-1} = \sigma_{e^{\pi i}}$ is the classical knot signature.
This also holds for the $T(3,3k+m)$ torus knots with $m\in\{1,2\}$ and odd $k\geq 1$ \cite{BBL}.
For even~$k$ on the other hand, e.g.~for $T(3,7)$,
one finds $\widehat{\sigma } = |\sigma_{\omega}| = |\sigma| + 2$
for $\omega$ chosen only one jump-point of the Levine--Tristram signature away from $e^{\pi i}$,
i.e.~$\omega = e^{2 \pi i t}$ for
\[ t \in \left(\frac12 - \frac5{18k + 6m},\ \frac12 - \frac1{18k + 6m}\right). \]
This observation relies on the fact that the jumps of the Levine--Tristram signatures of torus knots
are well understood~\cite{Lit, Ban}.
We have also seen examples where $\hatsigma = |\sigma_{\zeta3}|$, namely the closures of $(abx)^{2k} abx^2abx^2$ for $k \geq 0$; see \cref{rmk:abxexact}.
Overall, whenever we could precisely determine the topological 4-genus of a 3-braid knot $K$,
then the maximum absolute value of $\sigma_{e^{2 \pi i t}}$ was either assumed at $t = \frac{1}{3}$,
or at $t = \frac{1}{2}$, or close to $t = \frac{1}{2}$.

However, ${\widehat{\sigma}(K) - \max\left(|\sigma_{\zeta3}(K)|, |\sigma(K)|\right)}$ is unbounded for $K$ ranging over positive $3$-braid knots.
Indeed, let $K_n$ be the closure of $\left(a^2b^2\right)^{2n} \left(a^5b^5\right)^n$, which is a knot if $n$ is not a multiple of 3. The Levine--Tristram signatures of $K_4$ are shown in \cref{fig:sigprofile}.
Using \cref{prop:garsidesig} and 
\cite[Corollary~4.4]{GG} (as in \eqref{eqn:sigmahat}), we find
\begin{equation}\label{eq:sigandzeta3}
\sigma(K_n) = -12n, \quad \sigma_{\zeta3}(K_n) \approx -12n,
\end{equation}
with an error of at most $2$. Let us now estimate $\widehat{\sigma}(K_n)$.
We use the fact
that $K_n$ can be transformed by $n+10$ saddle moves into the
link $J_n = T(2,10)^{\#(n-1)} \# T(3,6n) \# T(2,-4n)$.
The latter satisfies, up to a constant error (i.e.~an error that is independent of $n$),
\[
\sigma_{e^{6 \pi i/7}}(J_n) \approx
-9n - 8n + \frac67\cdot 4n = -(13+\frac{4}{7})n,
\]
which implies that (again with constant error)
\[
\widehat{\sigma}(K_n) \geq |\sigma_{e^{6 \pi i/7}}(K_n)|
\geq |\sigma_{e^{6 \pi i/7}}(J_n)| - n - 10 \approx (12+\frac{4}{7})n.
\]
This estimate, combined with \eqref{eq:sigandzeta3}, shows that
\[
\lim_{n\to\infty} \widehat{\sigma}(K_n) - \max\left(|\sigma_{\zeta3}(K_n)|, |\sigma(K_n)|\right) = +\infty.
\]

This last example shows that determining $\widehat{\sigma}$ and $\gtop$ for all 3-braid knots, or even just closures of positive 3-braids, could be a hard problem.
\end{remark}
