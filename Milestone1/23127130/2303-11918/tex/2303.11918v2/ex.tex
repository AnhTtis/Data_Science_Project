Our results so far provide infinitely many strongly quasipositive 3-braid knots $K$
satisfying $\frac12 \hatsigma(K) = \gtop(K)$, see \cref{theorem1} and \cref{rmk:abxexact}.
This leads us to the following.
\begin{question}\label{q:equality}
Does the equality $\frac12 \hatsigma(K) = \gtop(K)$ hold for all strongly quasipositive 3-braid knots?
If not, does it at least hold for all braid positive 3-braid knots?
\end{question}
In this section, we exhibit large families of positive 3-braid knots
for which $\frac12 \hatsigma(K) = \gtop(K)$ holds up to an error of $1$,
see \Cref{prop:braidposbound}.
In certain cases, we can even determine $\gtop$ exactly (see \Cref{ex:exactex1,ex:exactex2} and \Cref{rem:exactfamilies}). 
%
For the knots covered by \Cref{prop:braidposbound}, $\widehat{\sigma}$ equals $|\sigma|$ or $|\sigma|+2$,
whereas in the example discussed in \cref{rmk:abxexact}, $\widehat{\sigma}$ equals $|\sigma_{\zeta3}|$.
However, as we will see in \cref{rem:sign}, the differences
$\widehat{\sigma} - |\sigma|$ and
$\widehat{\sigma} - |\sigma_{\zeta3}|$ can be arbitrarily large for positive 3-braid knots.
This fact contributes to the difficulty of \cref{q:equality}.

Throughout, we will use the Xu normal form of $3$-braids from \Cref{sec:Xu}. 
Recall that we write $\delta = ba = ax = xb$ such that $a\delta = \delta b$, $b \delta = \delta x$ and $x \delta = \delta a$ (see \eqref{eq:xurules} in \Cref{sec:Xu}).

\begin{example}\label{ex:exactex1}
Let $K$ be a knot that is the closure of a $3$-braid in Xu normal form $\beta = \delta^{3\ell +2} a^{u_1}$ for $\ell \geq 0, u_1 \geq 1$. Note that $u_1$ must be even for $K$ to be a knot. We claim that
\begin{align}\label{eq:g4topex1}
\gtop(K) = \tu(K) =\frac{|\sigma(K)|}{2}=  \frac{u_1}{2} +2\ell +1.
\end{align}
The last equality follows directly from \Cref{prop:signature}.
To prove \Cref{eq:g4topex1}, using $\frac12 |\sigma(K)| \leq \gtop(K) \leq \tu(K)$ (as explained in the beginning of \cref{sec:proofs}) 
it is enough to 
show that $\tu(K) \leq \frac12 |\sigma(K)|$. 
By $\frac{u_1-2}{2}$ crossing changes from positive crossings of $\beta$ to negative crossings we obtain the braid $\delta^{3\ell +2} a^{2}$.
We will prove by induction that this braid can be untwisted with $2\ell +2$ twists, which implies $\tu(K) \leq \frac{u_1}{2}+2\ell+1$. For $\ell = 0$, we have $\delta^2 a^2 = baba^3 = aba^4$ which becomes $ab$ (with closure the unknot) by two crossing changes. For $\ell = 1$, we have 
\begin{align*}
\delta^5 a^2 &= \delta^4 ax a^2 
= \delta^3 x^2 bxa^2 
= \delta^2 b^3 abxa^2 
= \delta a^4 xabxa^2 
= x^5 bxabxa^2\\
&\sim b^5 abxabx x \twist b^5 x,
\end{align*}
which turns into $bx\sim \delta$ using two crossing changes. Recall that the two twists needed to untwist $abxabx \twist \varnothing$ are shown in \cref{fig:abxabx} of \Cref{sec:proofs}.
Now, for $\ell \geq 2$, we have
\begin{align*}
\delta^{3\ell+2}a^2 &= \delta^{3\ell-3}x^5 bxabxa^2\sim \delta^{3\ell-3}b^5 abxabx x \twist \delta^{3\ell-3}b^5 x \\&\sim \delta^{3\ell-3} \delta b^4
= \delta^{3\ell-3} a \delta b^3 \sim \delta^{3\ell-1} b^2 \sim 
\delta^{3(\ell-1)+2} a^2,
\end{align*}
where we again used two twists for `$\twist$'. Inductively this shows that $\delta^{3\ell+2}a^2$ can be untwisted with $2\ell+2$ twists as claimed. 
\end{example}

\Cref{ex:exactex1} combined with the results from \cite{BBL} for $3$-strand torus knots shows that
the equalities $\gtop = \tu =\frac{\hatsigma}{2}$ hold for all strongly quasipositive $3$-braid knots in Xu normal form (a) or (b) from \Cref{lem:xuform}, where $\hatsigma = \left|\sigma \right|$ except for certain torus knots of braid index $3$; see also \Cref{rem:sign}. We next consider a sub-case of case (c) from \Cref{lem:xuform}. 

\begin{example}\label{ex:exactex2}
Let $K$ be a knot that is the closure of a $3$-braid in Xu normal form $\beta = \delta^{3\ell +1} a^{u_1}b^{u_2}$ for $\ell \geq 0, u_1, u_2 \geq 1$. Note that $u_1$ and $u_2$ must both be even for $K$ to be a knot. We claim that
\begin{align*}
\gtop(K) = \tu(K) =\frac{|\sigma(K)|}{2}=  \frac{u_1+u_2}{2} +2\ell.
\end{align*}
The proof works as in \Cref{ex:exactex1}. After $\frac{u_1+u_2-4}{2}$ positive to negative crossing changes in $\beta$ we obtain the braid $\delta^{3\ell +1}a^2 b^2$, which we can untwist with $2\ell +2$ twists as follows. For $\ell = 0$, the braid $\delta a^2 b^2$ turns into $\delta$ by two crossing changes. For $\ell = 1$, we have 
\begin{align*}
\delta^4 a^2b^2 &= 
\delta^3 xax ab^2 
= \delta^2 bx^2 b x ab^2 
= \delta ab^3 a b x ab^2 
= xa^4 x a b x ab^2 \\&\sim ab^4abxabxx \twist  ab^4x,
\end{align*}
which can be untwisted using two crossing changes. 
For $\ell \geq 2$, we have
\begin{align*}
\delta^{3\ell +1}a^2 b^2 &=\delta^{3\ell-3} xa^4 x a b x ab^2 \sim \delta^{3\ell-3} ab^4abxabxx \twist\delta^{3\ell-3} ab^4x \\
&= \delta^{3\ell-4} x a^3 ba bx  
= \delta^{3\ell-5} b x^3 a^2 x a bx 
= \delta^{3\ell-6} a b^3 x^3 b x a bx 
 \\&\sim \delta^{3\ell-6} a^3 b^3 abxabx \twist \delta^{3\ell-6} a^3 b^3 \sim \delta^{3\ell-5} a^2 b^2 = \delta^{3(\ell-2)+1} a^2 b^2,
\end{align*}
which we can untwist inductively using the two base cases above. 
\end{example}


The following proposition improves the statement from \Cref{thm:2} for braid positive $3$-braid knots under the additional assumption $u_i \geq 2$ for the exponents in the Xu normal form of their braid representatives.
In fact, we can determine $\gtop(K)$ in this case up to an error of $1$, using $\frac12 |\sigma(K)|$ as a lower bound. 

\begin{proposition}\label{prop:braidposbound}
Let $K$ be a knot that is the closure of a $3$-braid in Xu normal form
\begin{align*}
\delta^n \tau_1^{u_1} \tau_2^{u_2} \dots \tau_t^{u_t}
\quad\text{for}\quad t\geq 1,\ n \geq \frac{t}{2},\ u_1, \dots, u_t \geq 2.
\end{align*}
Then $K$ is a braid positive knot and 
\begin{align}\label{eq:braidposbound}
\frac{n+t}{3}-1 = g(K) - \frac{|\sigma(K)|}{2} \geq g(K) - \gtop(K) \geq 
\frac{n+t}{3}-2.
\end{align}
\end{proposition}


\begin{proof}
Set $U = u_1 + \cdots + u_t$.
\Cref{prop:positivity} implies that $K$ is braid positive. Moreover, we have $\frac{|\sigma(K)|}{2} = \frac{U}{2}+\frac{2n}{3}-\frac{t}{3}$ by \Cref{prop:signature} and $g(K) = \frac{U}{2}+n-1$ by \eqref{eq:genus}. Using $\frac12 |\sigma(K)| \leq \gtop(K)$, 
it remains to show that $\gtop(K) \leq \frac{|\sigma(K)|}{2} + 1$.

We distinguish two cases depending on the parity of $t$. First, let $t = 2r$ be even for $r \geq 1$. 
The conditions $2n \geq t$ and $n + t \equiv 0\pmod{3}$ imply that we can write $n = 3\ell + r$ for $\ell \geq 0$, and $K$ is the closure of
\begin{align*}
\beta = \delta^{3\ell + r} \tau_1^{u_1} \tau_2^{u_2} \dots \tau_{2r}^{u_{2r}}. 
\end{align*}
The case $r=1$ ($t=2$) is covered by \Cref{ex:exactex2}, so we can further assume that $r \geq 2$. 
There is a smooth cobordism of Euler characteristic $4r-U-4$ from $K$ to the knot that is the closure of 
\begin{align*}
\beta^\prime = \tau_{1-r}\delta^{3\ell + r-1} \prod_{i=1}^{r-2} \tau_i^2 \tau_{r-1} \tau_{r} \tau_{r+1} \prod_{i={r+2}}^{2r} \tau_i^2.
\end{align*}
Indeed, we can use $U-4r+3$ saddle moves to replace all but three of the exponents $u_i$ by $2$ and the other three by $1$. We use a last saddle move to replace $\delta$ by $\tau_{1-r}$. 
We will prove by induction on $r$ that $\beta^\prime$ turns into $\delta^{3\ell+1}$ by $2r-2$ twists.
Since the closure of $\delta^{3\ell+1}$ is the torus knot $T(3,3\ell+1)$, this will imply
\begin{align}\label{eq:claim1}
\tu\left(\cl\left(\beta^\prime\right)\right) \leq \tu(T(3,3\ell+1)) +2r-2 =
\begin{cases}
  2\ell+2r-1  & \text{if } \ell \geq 1,  \\
  2r-2 & \text{if } \ell = 0,
\end{cases}
\end{align}
where the equality follows from \cite[Lemma~5 and Theorem~1]{BBL}. 
Recall that $\delta = \tau_{i+1} \tau_{i}$, $\tau_i \delta = \delta \tau_{i+1}$ (see \eqref{eq:xurules}) and $\tau_i = \tau_{i+3m}$ for all $m \in \Z, i \in \Z$. For $r=2$, we thus have $\tau_{1-r} = \tau_2$ 
and 
\begin{align*}
\beta^\prime &= \tau_{2}\delta^{3\ell + 1}  \tau_{1} \tau_{2} \tau_{3}  \tau_4^2 
= \delta^{3\ell} \tau_2 \tau_1 \tau_0 \tau_{1} \tau_{2} \tau_{3}  \tau_4^2 
= \delta^{3\ell} \tau_0 \tau_{-1} \tau_0 \tau_{1} \tau_{2} \tau_{3}  \tau_4^2 
\\&\sim \delta^{3\ell} \tau_2 \tau_{1} \tau_2 \tau_{3} \tau_{4} \tau_{5}  \tau_6^2  
\twist \delta^{3\ell} \tau_2  \tau_6 \sim 
\delta^{3\ell+1},
\end{align*}
so $\beta^\prime$ indeed turns into $\delta^{3\ell+1}$ using $2r-2=2$ twists in this case.
Now, for $r \geq 3$, consider
\begin{align*}
\beta^\prime 
&= \tau_{1-r}\delta^{3\ell + r-2} \prod_{i=1}^{r-3} \tau_{i-1}^2 \delta \tau_{r-2}^2\tau_{r-1} \tau_{r} \tau_{r+1} \tau_{r+2}^2\prod_{i={r+3}}^{2r} \tau_i^2
\\&= \tau_{1-r}\delta^{3\ell + r-2} \prod_{i=1}^{r-3} \tau_{i-1}^2 \tau_{r-3}\tau_{r-2} \tau_{r-3} \tau_{r-2}\tau_{r-1} \tau_{r} \tau_{r+1} \tau_{r+2}^2\prod_{i={r+3}}^{2r} \tau_i^2
\\&\sim \tau_{(1-r)-(r-4)}\delta^{3\ell + r-2} \prod_{i=1}^{r-3} \tau_{i-1-(r-4)}^2 \tau_{1}\tau_{2} \tau_{1} \tau_{2}\tau_{3} \tau_{4} \tau_{5} \tau_{6}^2\prod_{i={r+3}}^{2r} \tau_{i-(r-4)}^2
\\&\twist \tau_{(1-r)-r+1}
\delta^{3\ell + r-2} \prod_{i=1}^{r-3} \tau_{i-r}^2 \tau_{1}\tau_{2}  \tau_{3}\prod_{i={r+3}}^{2r} \tau_{i-r+1}^2
\\&\sim 
\tau_{(1-r)+1}\delta^{3\ell + r-2} \prod_{i=1}^{r-3} \tau_{i}^2 \tau_{r-2}\tau_{r-1}  \tau_{r}\prod_{i={r+1}}^{2(r-1)} \tau_{i}^2.
\end{align*}
The braid $\beta^\prime$ hence turns into $\tau_{(1-r)+1}\delta^{3\ell + r-2} \prod_{i=1}^{r-3} \tau_{i}^2 \tau_{r-2}\tau_{r-1}  \tau_{r}\prod_{i={r+1}}^{2(r-1)} \tau_{i}^2$ by two twists and inductively we get that $\beta^\prime$ turns into
\begin{align*}
\tau_{(1-r)+r-2}\delta^{3\ell + 1} \tau_{1}\tau_{2}  \tau_{3}\tau_{4}^2
\end{align*}
by $2(r-2)$ twists. Since $\tau_{(1-r)+r-2} = \tau_2$, this braid is the same as the one from the base case $r=2$ and therefore can be untwisted with two twists. We obtain that $\beta^\prime$ becomes $\delta^{3\ell+1}$ by $2r-2$ twists.
\Cref{eq:claim1} follows and we get
\begin{align*}
\gtop(K) &\leq \gtop \left(\cl\left(\beta^\prime\right)\right) 
+ \frac{U-4r+4}{2} 
\leq \tu \left(\cl\left(\beta^\prime\right)\right)+
\frac{U}{2} -2r+2   
\\&\leq 
\begin{cases}
  \frac{U}{2} +2\ell+1 =  \frac{|\sigma(K)|}{2} + 1  & \text{if } \ell \geq 1,  \\
  \frac{U}{2} = \frac{|\sigma(K)|}{2} & \text{if } \ell = 0.
\end{cases}
\end{align*}

Next, let $t = 2r+1$ be odd for $r \geq 0$. The conditions $2n \geq t$ and $n + t \equiv 0\pmod{3}$ imply that we can write $n = 3\ell + r+2$ for $\ell \geq 0$, and $K$ is the closure of
\begin{align*}
\beta = \delta^{3\ell + r+2} \tau_1^{u_1} \tau_2^{u_2} \dots \tau_{2r+1}^{u_{2r+1}}. 
\end{align*}
The case $t = 1$ is covered by \Cref{ex:exactex1}, so we can further assume that $r \geq 1$. 
There is a smooth cobordism of Euler characteristic $4r-U-2$ from $K$ to the knot that is the closure of 
\begin{align*}
\beta^\prime = \tau_{1-r}\delta^{3\ell + r+1} \prod_{i=1}^{r-1} \tau_i^2 \tau_{r} \tau_{r+1} \tau_{r+2} \prod_{i={r+3}}^{2r+1} \tau_i^2,
\end{align*}
similar to the cobordism considered in the above case. 
We prove by induction on $r$ that $\beta^\prime$ turns into $\delta^{3(\ell+1)+1}$ by $2r-2$ twists, hence 
\begin{align*}
 \tu\left(\cl\left(\beta^\prime\right)\right) \leq \tu(T(3,3(\ell+1)+1)) +2r-2 
 = 2\ell+2r+1.
\end{align*}
For $r=1$, we have 
\begin{align}\label{eq:caser=1}
\beta^\prime = \tau_{0}\delta^{3\ell + 2}  \tau_{1} \tau_{2} \tau_{3} 
= \delta^{3\ell+3} \tau_{2} \tau_{3} \sim \delta^{3(\ell+1)+1}.
\end{align}
For $r=2$, we have
\begin{align*}
\beta^\prime &= \tau_{2}\delta^{3\ell + 3}  \tau_1^2 \tau_{2} \tau_{3} \tau_{4}  \tau_5^2
= \tau_{2}\delta^{3\ell + 2}  \tau_0\tau_1 \tau_0\tau_1 \tau_{2} \tau_{3} \tau_{4}  \tau_5^2 \\&\sim \tau_{0}\delta^{3\ell + 2}  \tau_1\tau_2 \tau_1\tau_2 \tau_{3} \tau_{4} \tau_{5}  \tau_6^2
\twist \tau_{0}\delta^{3\ell + 2}  \tau_1\tau_2 \tau_6 \sim \delta^{3(\ell+1)+1}
\end{align*}
using \Cref{eq:caser=1} in the last step.
Now, for $r \geq 3$, consider
\begin{align*}
\beta^\prime 
&= \tau_{1-r}\delta^{3\ell + r} \prod_{i=1}^{r-2} \tau_{i-1}^2 \delta \tau_{r-1}^2\tau_{r} \tau_{r+1} \tau_{r+2} \tau_{r+3}^2\prod_{i={r+4}}^{2r+1} \tau_i^2
\\&= \tau_{1-r}\delta^{3\ell + r} \prod_{i=1}^{r-2} \tau_{i-1}^2 \tau_{r-2}\tau_{r-1}\tau_{r-2} \tau_{r-1}\tau_{r} \tau_{r+1} \tau_{r+2} \tau_{r+3}^2\prod_{i={r+4}}^{2r+1} \tau_i^2
\\&\sim \tau_{(1-r)-(r-3)}\delta^{3\ell + r} \prod_{i=1}^{r-2} \tau_{i-1-(r-3)}^2 \tau_{1}\tau_{2}\tau_{1} \tau_{2}\tau_{3} \tau_{4} \tau_{5} \tau_{6}^2\prod_{i={r+4}}^{2r+1} \tau_{i-(r-3)}^2
\\&\twist \tau_{(1-r)-r}\delta^{3\ell + r} \prod_{i=1}^{r-2} \tau_{i-r-1}^2 \tau_{1}\tau_{2}\tau_{3}\prod_{i={r+4}}^{2r+1} \tau_{i-r}^2
\\&\sim \tau_{(1-r)+1}\delta^{3\ell + r} \prod_{i=1}^{r-2} \tau_{i}^2 \tau_{r-1}\tau_{r}\tau_{r+1}\prod_{i={r+2}}^{2(r-1)+1} \tau_{i}^2.
\end{align*}
Inductively we get that $\beta^\prime$ turns into
\begin{align*}
\tau_{(1-r)+r-2}\delta^{3\ell + 3} \tau_{1}^2\tau_{2}  \tau_{3}\tau_{4}\tau_5^2
=\tau_{2}\delta^{3\ell + 3} \tau_{1}^2\tau_{2}  \tau_{3}\tau_{4}\tau_5^2
\end{align*}
by $2(r-2)$ twists, so into $\delta^{3(\ell+1)+1}$ by $2r-2$ twists. We obtain
\begin{align*}
\gtop(K) &\leq \gtop \left(\cl\left(\beta^\prime\right)\right)+ \frac{U-4r+2}{2} 
\leq \tu \left(\cl\left(\beta^\prime\right)\right)+
\frac{U}{2} -2r+1  
\\&\leq
  \frac{U}{2} +2\ell+2 = \frac{|\sigma(K)|}{2} + 1.\qedhere
\end{align*}
\end{proof}

\begin{remark}\label{rem:exactfamilies}
The proof of \Cref{prop:braidposbound} (more precisely, the first case with~$\ell = 0$) shows that the first inequality in \eqref{eq:braidposbound} is an equality when~$2n=t$.
\end{remark}

\begin{example}\label{ex:galg}
For knots $K$ of low Seifert genus arising as closure of positive 3-braids,
we can often apply the untwisting moves shown in \cref{sec:proofs}
and show $\frac12 \widehat{\sigma}(K) = \tu(K)$, thus determining $\gtop(K)$.
Here is a selection of positive 3-braids
that close off to knots of Seifert genus 6 and 7,
for which that strategy did not succeed:
\begin{align*}
\delta^3 a^2b^2xabx &\sim a^3b^3a^2b^2a^2b^2\\
\delta^4 a^2 bxab &\sim \Delta a^3b^2a^2b^2a^2\\
\delta^4a^4bxab &\sim \Delta a^5b^2a^2b^2a^2\\
\delta^4a^2b^2xa^2b &\sim \Delta a^3b^3a^2b^3a^2\\
\delta^6 a^2bx &\sim \Delta^3a^3b^2a^2.
\end{align*}
First, we note that for all of these knots $K$, we have the lower bound $\frac12 \hatsigma(K) = \frac12|\sigma(K)| = g(K)-2 \leq \gtop(K)$.
Second, in the search for upper bounds, we are able to find a knot $J$ such that $K\leadsto J$ and $\tu(J) = g(J) - 1$, thus proving $\gtop(K) \leq g(K) - 1$, for each of these knots $K$.  However, we are unable to find $J$ with $K\leadsto J$ and $\tu(J) = g(J) - 2$.
Nevertheless, we can prove $\gtop(K) = g(K)-2$ for all of these knots $K$ in a different way,
which is practical for individual knots with low Seifert genus. 
Namely, a computer search~\cite{Lew} reveals that the \emph{algebraic genus} $g_{\text{alg}}(K)$, which is defined in terms of Seifert matrices of $K$ and provides 
an upper bound $\gtop(K) \leq g_{\text{alg}}(K)$~%
\cite{FL}, satisfies $g_{\text{alg}}(K) \leq g(K) - 2$ for all our knots $K$.
Yet it remains an open question whether $\tu(K) = g(K) - 1$ or $\tu(K) = g(K) -2$
for those knots $K$.
\end{example}
