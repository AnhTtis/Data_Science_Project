\section{Conclusion}\label{sec:conclusion}
The goal of this project is to gain familiarity with the ESP32 microcontroller, build toolchain, SDK and explore possibilities to emulate and containerize.  Espressif maintains both the SDK (ESP-IDF) and a customized fork of the QEMU project on GitHub.  Additionally, Espressif publishes a VSCode extension which wraps the SDK and provides a convenient integrated development environment for coding.  An example HTTP server was easily built and linked with the SDK's suite of support libraries and FreeRTOS kernel.  It loads and runs without issue on the FreeNove test board purchased for these experiments.

The QEMU fork for ESP32 maintains pre-built binaries for Linux and can also be natively built for macOS.  Both options were explored.  The former was run through building a Docker container with the QEMU emulator and mounting the ESP32 load from the host file system at runtime.  The latter required building QEMU for macOS and launching with the ESP32 load.  Both options successfully emulated the Xtensa instruction set and allowed the ESP32 load to run without error.

While running on the FreeNove board, the ESP32 joined the available Wi-Fi for networking.  During emulation, it was configured to use the native ethernet interface.  This interface is provided by the OpenCores Ethernet MAC driver that is part of the QEMU project.  The driver provides an API to the guest OS network stack to discover the underlying MAC as well as transmit and receive ethernet frames.  These frames are forwarded between QEMU and the host OS.  QEMU has a trace framework built in with calls placed in all major API functions of every subsystem.  For these experiments, the OpenCores driver API activity was observed through launching QEMU with the option to trace all activity in the ``open\_eth'' subsystem.  Both the macOS and containerized Linux versions of QEMU ran the HTTP server networking application with no errors.

In conclusion, the ESP32 is a feature rich and cost-effective system-on-a-chip providing Wi-Fi, Bluetooth and a large selection of I/O devices built in to a tiny package.  The \$10 investment provides a powerful microcontroller that is ideal for car chargers, stand mixers, robotic joints and many other embedded applications.  It is possible to natively build the ESP32 and emulate this load through QEMU.  Containerizing this combination adds a very convenient way to improve the rapid build and test cycle while scaling to many device deployments on a single host. 