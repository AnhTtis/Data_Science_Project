\section{Introduction}\label{sec:intro}
The ESP32 is a microcontroller developed by Espressif Systems Co. \cite{esp2023}, a semiconductor company headquartered in Shanghai, China.  The ESP32 provides a low-cost, low-power and reasonably performant all-in-one hardware package that is ideally suited to Internet of Things (IoT) applications.  The RISC processor is a 32-bit Xtensa core developed by Cadence Design Systems \cite{cadence2023}.  It is packaged as a system-on-a-chip (SoC) with Bluetooth, Wi-Fi and general purpose input output (GPIO) capabilities built-in.

In addition to the suite of microcontrollers, Espressif also supports a rich software ecosystem.  Developers around the world contribute to an open source project that provide a software development kit (SDK) for each microcontroller flavor.  The SDK provides a complete build system that utilize Python scripting on top of cmake C-language build projects.  Device drivers, a real-time operating system (OS) called FreeRTOS and a suite of software component projects are all included.  The component projects provide an excellent starting point for developers to expand their own software requirements.

QEMU \cite{qemu2023} is an emulator able to run guest OS and application binaries on a host operating system.  For this project, the overall goal is to see how QEMU can be used to emulate an ESP32 application image on a macOS host system.  

\subsection{Goals}\label{sub:goals}
At the completion of the project, the following target goals will be accomplished:
\begin{enumerate}
    \item Build an ESP32 target load containing the OS, device drivers and a simple HTTP application such as a web server that uses the TCP/IP stack.\label{goal:build}
    \item Execute this load on native hardware to first ensure it is functional.\label{goal:exec}
    \item Custom build QEMU for macOS (with ESP32 support) and modify as needed to run the target load.\label{goal:qemu}
    \item  Develop a Docker container around a QEMU tool chain.  A container will provide an isolated environment for all dependencies and is preferable to running natively.\label{goal:container}
    \item Generate some minimal HTTP traffic between the application and the external world.\label{goal:traffic}
    \item If working, trace through the call stack(s) as much as possible and determine how the emulation is being performed.\label{goal:trace}
    \item Detail and report on the experiments and system architecture.\label{goal:report}\\    
\end{enumerate}

The overall goal as detailed above is motivated by a desire to easily emulate the ESP32.  This microcontroller is being used as an edge device in the author's energy auction research.  As such, the need to easily deploy and destroy multiple containers containing ESP32 loads (with emulators) will facilitate load testing and easy automation for functional tests.  Both the Capstone project and follow-on research will benefit from being able to build the ESP32 load directly into a container and then immediately deploy multiple instances, each with a unique identifier, on either a local workstation or Cloud container service.  Additionally, the knowledge and understanding gained through digging into the QEMU architecture and APIs will  assist with the development and debugging of the edge device application. 

The system architecture is outlined in Section \ref{sec:system}.  The experiments performed and corresponding analysis are detailed in Sections \ref{sec:experiments} and \ref{sec:analysis} respectively.  The final discussion and conclusions are captured in Section \ref{sec:conclusion}.  