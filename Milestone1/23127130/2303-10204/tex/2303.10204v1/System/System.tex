\section{System Architecture}\label{sec:system}

The relevant system architecture for this paper primarily consists of the ESP32 microcontroller, its SDK, the QEMU emulator and Docker containers.  Visual Studio Code provides extensions that interface both to the ESP32 SDK and the Docker container management.  While the VSCode integrated development environment (IDE) does provide convenience features used during the experiments, they will only be mentioned in passing without going in depth in their implementation.

\subsection{ESP32 microcontroller}\label{sub:esp32}
The core component of this paper is the ESP32 microcontroller as this is the desired platform to both emulate and virtualize. Figure \ref{fig:esp_hw} shows the hardware development kit purchased for this experiment.  A micro USB connector provides both power and data.  The SoC chip itself can be purchased very cheaply on its own.  The dev kit as shown adds a few extra (unnecessary) peripherals such as a breadboard, GPIO breakout pins, LEDs, speakers and a camera.  The entire package was purchased for \$26.00 on Amazon which also shipped with a set of wires and resistors.  

\begin{figure}[H]
    \centerline{\includegraphics[width=1.0\linewidth, keepaspectratio]{esp_hw.png}}
    \captionsetup{width=.8\linewidth}
    \caption{FreeNove ESP32 development board containing the core SoC with some sample I/O devices such as LEDs and speaker.}\label{fig:esp_hw}
\end{figure}

The ESP32 chip is the silver component in the upper-left corner of Figure \ref{fig:esp_hw}. Its functional block diagram is shown in Figure \ref{fig:esp_block}. This diagram shows the richness of the SoC with dual Xtensa CPU cores, Wi-Fi, Bluetooth, RAM, flash memory and a large variety of I/O controllers. 

\begin{figure}[h]
    \centerline{\includegraphics[width=0.9\linewidth, keepaspectratio]{esp_block.png}}
    \captionsetup{width=.8\linewidth}
    \caption{The ESP32 block diagram showing the dual Xtensa cores with a wide variety of I/O controllers built in to the chip.}\label{fig:esp_block}   
\end{figure}

\subsection{Espressif ESP-IDF}\label{sub:esp_idf}
In addition to providing the ESP32 hardware itself, Espressif also maintains an open-source project for a complete SDK called the ESP integrated development framework (ESP-IDF).  Version 5.0 is the current release and publicly available on GitHub \cite{esp_git}.  ESP-IDF provides a suite of Python scripts that can build the application, flash to the remote ESP32 chip and monitor its stdout via the USB port powering the device.

ESP-IDF also consists of large set of cmake \cite{cmake} build projects.  Source code is C-language only and includes the FreeRTOS \cite{freertos} kernel, drivers, support libraries and many sample application projects.  The application developed for this paper is one of the sample projects that provides an HTTP server listening on port 80.  It provides a simple response message to a request message with the \emph{/hello} context.  

Building the application provides a linked binary containing the application, kernel and libraries all compiled for the Xtensa instruction set.  However, this monolith cannot run on the ESP32 target without providing some additional components for the flash device.  Both a bootloader and a partition table must be present in order for the microcontroller to bootstrap the application load.  ESP-IDF provides tools to assemble (merge) a final binary flash image.  The layout is shown in Figure \ref{fig:esp_flash}.  This merged binary is now able to be written to the target flash and booted.  It is ready for emulation as well. 

\begin{figure}[h]
    \centerline{\includegraphics[width=0.7\linewidth, keepaspectratio]{esp_flash.png}}
    \captionsetup{width=.8\linewidth}
    \caption{The ESP32 flash image layout including bootloader, partition table and application.}\label{fig:esp_flash}   
\end{figure}

\subsection{QEMU}\label{sub:qemu}
QEMU is a system emulator that provides a virtual model of a machine to run a guest OS.  CPU, memory and devices are all part of the emulation.  While the vanilla code does support the Xtensa processor, the ESP32 microcontroller currently requires a fork that is maintained by Espressif.  The source code project is available on GitHub \cite{esp_qemu}.  This source project has to be built in order to run QEMU and ESP32 on macOS.  There are no pre-built versions hosted.

QEMU is a user space application that requires an accelerator (hypervisor) in the host kernel.  However, this custom build utilizes Tiny Code Generator (TCG) which is pure emulation.  Theoretically, this trades performance for ease of implementation.  Features such as a block and character device are built-in which allow emulation of stdio, files, sockets and TCP networking.

\subsection{Containers}\label{sub:containers}
The container engine is provided by Docker \cite{docker}.  The purpose of this engine is to provide a virtualization of the file system and configuration while not incurring the overhead of booting a completely separate OS kernel.  The Docker image used starts with an existing Ubuntu 20.04 file system.  The ESP-IDF is cloned into the image as well as a pre-built QEMU from 2022-09-19.  Note that in the case of the container, we can use a pre-built QEMU (Linux binaries) that supports the ESP32.  Espressif has this tarball available in their GitHub repo.  Figure \ref{fig:container_image} shows a coarse outline of the layers in the image.  Each Docker image is composed of layers that can be reused among different images.  Each layer typically maps to a Docker build instruction. 

\begin{figure}[H]
    \centerline{\includegraphics[width=0.7\linewidth, keepaspectratio]{container_image.png}}
    \captionsetup{width=.8\linewidth}
    \caption{The Docker container image containing the Ubuntu base, ESP-IDF SDK toolchain and QEMU emulator.}\label{fig:container_image}   
\end{figure}