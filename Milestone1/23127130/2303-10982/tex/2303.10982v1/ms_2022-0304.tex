\documentclass{raa}
\usepackage{longtable}
\usepackage{threeparttablex}
\usepackage{natbib}
\usepackage{booktabs}
\usepackage{graphicx,times}             %for PS/EPS graphics inclusion, new
\usepackage{amssymb,amsmath}
% \usepackage[scheme=plain]{ctex}
\usepackage{epstopdf}
\setlength\LTleft{0pt}
\setlength\LTright{0pt}
\usepackage{pdflscape}
\bibpunct{(}{)}{;}{a}{}{,}
\usepackage[pagebackref=true]{hyperref}
\usepackage{ulem}
\hypersetup{colorlinks = true, linkcolor = green, anchorcolor = red, citecolor = blue, filecolor = red,  urlcolor = red}
\pdfoptionpdfminorversion = 7

\newcommand\kmps{\mbox{km s$^{-1}$}}
%\newcommand{\deg}{^\circ}
\newcommand{\abs[1]}{\left| #1 \right|}
\def\degree{${}^{\circ}$} 
%\newcommand\co[2][12]{\mbox{$^{#1}C^{#2}O$}}
\newcommand\co[1][12]{\mbox{$^{#1}$CO}}
\newcommand\CO{\mbox{C$^{18}$O}}
\newcommand\vlsr{V$_{LSR}$}
\newcommand{\nodata}{...}
\graphicspath{{./figures-new/}}

\begin{document}


\title{In search for infalling clumps in molecular clouds }
\subtitle{A catalogue of CO blue-profiles}

   \volnopage{Vol.0 (20xx) No.0, 000--000}      %%preserved for Editor. DOn't remove!
   \setcounter{page}{1}          %%starting page, preserved for Editor. DOn't remove!

   % \author{Zhibo Jiang (江治波) 
   %    \inst{1,2,3}
   %    \and Shaobo Zhang (张少博)
   %    \inst{1}      \and Zhiwei Chen (陈志维)
   %    \inst{1}
   %    \and Yang Yang (杨旸)
   %    \inst{1,2}
   %    \and Shuling Yu (于书岭)
   %    \inst{1,2}
   %    \and Haoran Feng (冯浩然)
   %    \inst{1,2}
   %    \and Ji Yang (杨戟)
   %    \inst{1}
   %    \and the MWISP group
   %    }

      \author{Zhibo Jiang
      \inst{1,2,3}
      \and Shaobo Zhang
      \inst{1}      \and Zhiwei Chen
      \inst{1}
      \and Yang Yang
      \inst{1,2}
      \and Shuling Yu
      \inst{1,2}
      \and Haoran Feng
      \inst{1,2}
      \and Ji Yang
      \inst{1}
      \and the MWISP group
      }


    
\institute{Purple Mountain Observatory, Chinese Academy of Sciences,
10 Yuanhua Road, 210023
Nanjing, China; {\it zbjiang@pmo.ac.cn} \\
\and  University of Science and Technology of China, Hefei 230026, China\\
\and Center for Astronomy and Space Sciences, Three Gorges University, Yichang 443002, China \\
}


\abstract{ 
We have started a systematic survey of molecular clumps with infall motions to study the very early phase of star formation. Our first step is to utilize the data products by MWISP to make an unbiased survey for blue asymmetric line profiles of CO isotopical molecules. Within a total area of $\sim$ 2400 square degrees nearby the Galactic plane, we have found 3533 candidates showing blue-profiles, in which 3329 are selected from the \co\&\co[13]{} pair and 204 are from the \co[13]\&\CO{} pair. Exploration of the parametric spaces suggests our samples are in the cold phase with relatively high column densities ready for star formation. Analysis of the spatial distribution of our samples suggests that they exist  virtually in all major components of the Galaxy. The vertical distribution suggest that the sources are located mainly in the thick disk of $\sim$ 85 parsec, but still a small part are located far beyond Galactic midplane.   Our follow-up observation indicates that these candidates are a good sample to start a search for infall motions, and to study the condition of very early phase of star formation. 
\keywords{line: profiles --- stars: formation  --- ISM: clouds
--- catalogues --- surveys}
}

   \authorrunning{Z. Jiang et al. }            %author_head in even pages
   \titlerunning{In Search for Infalls }  % title_head in odd pages
 
   \maketitle

\section{Introduction}           %% first-level sections will be auto-capitalized
   \label{sec:intro}
Gravitational collapse of dense molecular cloud cores is a key step in the formation of stars \citep[e.g.,][]{1987ARA&A..25...23S}. Presently due to the limitation of the observational facilities,  it is difficult to obtain a motion picture showing a core in the process of gravitational collapse. However, line profiles can provide signatures of inward collapse motion.  As theoretical modelings show \citep[e.g.,][]{1996ApJ...465L.133M}, after a collapse happens, the center of the core becomes warmer due to the accumulation of gravitational energy while the outside envelope remains relative cool.  In the configuration of an inflowing core surrounded with a static envelope, the radiation transfer along the line of sight can be simplified by a two layer model for both the gas flowing away and toward the observer \citep[see fig.5 in][]{1999ARA&A..37..311E}. For the gas flowing toward the observer, the inner layer with higher excitation is nearer to the observer than the outer layer with lower excitation; for the flow-away gas, emission from the inner layer with higher excitation is absorbed by the outer layer with lower excitation lying closer to the observer. A double-peaked line profile with an absorption dip is produced by the above radiation transfer. The blue peak (gas flowing toward the observer) is stronger than the red peak (gas flowing away from the observer). This effect is especially conspicuous for molecular transition lines with a suitable optical depth and critical density. This kind of profiles are commonly referred to as ``blue-profiles''. They have been discussed by many researchers, and are regarded as an indicator of gas inflow motion \citep[e.g.,][]{1993ApJ...404..232Z,1994ASPC...65..192M,1997ApJ...489..719M,1996ApJ...465L.133M}, or infall signature. To distinguish a blue-profile from  multi-component emissions in the line of sight that may also show double-peaks, one needs an optically thin line, whose profile is single-peaked without  a self-absorption feature.  The peak position of the optically thin line should be close to the self-absorption dip between the blue and red peaks.

   Observational studies of the blue-profiles started in early 1990's \citep{1993ApJ...404..232Z,1994ASPC...65..192M} and are seen accelerated recently. Up to date, the studies can be classified into two categories. One is case study, i.e. detailed observations toward some particular star-forming complexes to study the physical conditions of regions showing infall signature \citep[e.g.,][]{1993ApJ...404..232Z, 2011ApJ...740..114Z, 2012MNRAS.422.1098R, 2014MNRAS.437.3766M,2018ApJ...852...12Y}. The other is the deliberated search, i.e., systematical searches for infall signatures in deliberately selected samples of different properties, such as low-mass protostellar objects \citep{1997ApJ...489..719M}, high-mass star-forming regions \citep{2003ApJ...592L..79W,2007ApJ...669L..37W,2007ApJ...663.1092K,2009MNRAS.392..170S,0067-0049-225-2-21}, high-mass protostellar objects \citep{2018ApJS..235...31Y}, high infrared extinction clouds \citep{2013AA...549A...5R}, massive star-forming cores \citep{2009MNRAS.392..170S,2015MNRAS.450.1926H}, and early dense cores \citep{2018ApJ...862...63C}, extended green object \citep{2010ApJ...710..150C}, etc.


Above studies have presented us with diverse hints on the initial phase of star formation, e.g., when the gravitational collapse starts and ends, whether high- and low-mass stars start their life in similar ways, and the time scale of mass-infall phase. However, a comprehensive study on the infall phase is necessary to understand fully the process how dense molecular cores turn into stars, and to contribute to solve the long-standing problem of the origin of the stellar initial mass function.  Previous studies mainly utilized known targets that have shown star-forming activities as a start to look into. This kind of researches will inevitably introduce bias if a statistical study is carried out upon them. We therefore do the work in an opposite direction, i.e., by merely looking for infall signatures in the Galaxy without any other assumption,  we study the physical properties and star-forming activities where the signatures happen. The ongoing Milky Way Imaging Scroll Painting Project \citep[MWISP,][]{2019ApJS..240....9S}, a large scale survey of CO (J = 1 - 0) lines in the northern Galaxy, provides an excellent opportunity to start up such a kind of work. 

Our project is to set up a sample of CO blue-profiles using the MWISP data by blind search, and refine it by further observations using some more infall-sensitive lines (such as HCO$^+$ lines). The aim of this project is to obtain a comprehensive sample of infall candidates with a high confidence level. This paper is to present the very first result of our project, a preliminary catalogue of CO blue-profiles.    

The structure of this paper is arranged in this way: in Sec. \ref{sec:met} we give a brief introduction to data preparation, strategy of blind search for blue-profiles; in Sec. \ref{sec:cat} we present the main catalogue and the associated line profiles; in Sec. \ref{sec:disc} we discuss the distribution and parameter space of sources; and Sec. \ref{sec:sum} summarizes our results.

\section{Method} \label{sec:met}


\subsection{Data} \label{sec:dat}

All data used in this work are based on the MWISP data products. A detailed description of the data acquisition, quality control and archiving scheme can be found in \citet{2019ApJS..240....9S}. The archive data are in unit of ``cells'', {the main region of} which is an area of 30\arcmin $\times$ 30\arcmin. According to the survey strategy, 
{the outskirts of each cell are observed with the 9-beam Superconducting Spectroscopic Array Receiver \cite[SSAR,][]{2012ITTST...2..593S}} less than the main region. Consequently, the RMS noise levels are higher in the edge part than in the central. Therefore, a combination with the neighboring cells is necessary to achieve a uniform noise level. At this step, we intentionally extend the area of each cell to some extent (generally by 0.5 arcmin on each side) to avoid possible edge effects. We then cut off unnecessary velocity channels ($\abs[V_{LSR}] \ge$ 200 \kmps) to minimize the unnecessary computation, and possible {unreal} detections {due to untrue signals} which might be a problem in the manual check step. The data reduced as such are ready for  machine work.


\begin{figure}[h]
   \begin{minipage}[t]{0.495\linewidth}
   \centering
    \includegraphics[width=60mm]{ms2022-0304fig1a.pdf}
   \end{minipage}%
   \begin{minipage}[t]{0.495\textwidth}
   \centering
    \includegraphics[width=60mm]{ms2022-0304fig1b.pdf}
   \end{minipage}%
   \caption{(left) The distribution of main-beam temperatures of L134 on a time scale of seven years (2011-2017). The blue, green and red histograms represent \co, \co[13]{} and \CO{} peak intensities, respectively. (right) The distribution of velocity difference between \co[12]{} and \co[13]{} in the same period. The figure shows the radial velocities of \co[12] are overall less than that of \co[13] by $\sim$ 0.15 \kmps{} (about one channel). A secondary peak at $\sim$ 0.1 \kmps{} is found, mainly caused in the the 2011 observation season.  The results of all nine beams are included. \label{fig:l134} }
  
 \end{figure}

   
   Because our work relies on the comparison among different lines, it is an essential demand that the observation facility should be stable on a large time scale. Fortunately, MWISP has a scheme that immediately before and after the observation to each cell,  observations toward a ``standard source''  were  made to monitor the system performance.  As an example, in Fig. \ref{fig:l134} we show the distributions of T$_{MB}$ and V$_{LSR}$ difference of L134, one of the ``standard sources''. We choose this source as an illustration because its line-widths are relative small to allow accurate estimation of the central velocity. The gross data  were collected from 2011 November to 2017 December in all three lines.  The 1$\sigma$  variations of T$_{MB}$ are 0.66 K, 0.47 K, and  0.47 K, or 7.4\%, 8.4\%, and 20\% on relative scale for the three lines, respectively. This suggests the system was reasonably stable within the period of seven years. The radial velocities with respect to local standard of rest (LSR) are also checked within the period. As shown in Fig. \ref{fig:l134},  V$_{LSR}^{12CO}$ is roughly less than  V$_{LSR}^{13CO}$ by $\sim$ 0.15 km $s^{-1}$ (blue shifted). \citet{1978ApJS...36....1M} showed a slight velocity shift between \co{} and \co[13]{} which is consistent with this result. This  difference could be intrinsic or arise in the observations. Considering that the V$_{LSR}$ differences between $^{12}$CO blue peak and $^{13}$CO center in our selected sample are mostly significantly greater than that value, this discrepancy, even if could arise due to the instrument and observation, does not affect our result too much.
   
   \subsection{Search Strategy} \label{sec:ss}
   As stated above, two lines, one being optically thick and the other optically thin, are necessary to discriminate between blue-profiles and multi-components. We use the $^{12}$CO with $^{13}$CO (hereafter Pair-1), and $^{13}$CO with C$^{18}$O (hereafter Pair-2) as two line pairs to do the work, i.e., we arbitrarily assume  $^{13}$CO as an optically thin line and $^{12}$CO as the optically thick to search for blue-profiles, and then do it again with the other line pair. This scheme is reasonable because  the  $^{13}$CO line could be optically thin in some cases while thick elsewhere. We note that in the former case, the optically thin assumption of \co[13] lines, which might be not true, does not affect the judgement of \co{} blue-profiles, but might miss some candidates if \CO{} emissions are not detected. 
   
   Similar to the scheme of the MWISP survey,  {the search regions for machine work are split into cells. Then the automatical search is done pixel by pixel within each cell. } As stated in Sec. \ref{sec:dat}, the area for each cell has a 0.5\arcmin{} overlap with the surrounding cells, redundant detections are inevitable, which are cleaned out at later steps.

    \subsection{Automatic Search}
   Up to date, MWISP has been accomplished the region l = [12\degree, 230\degree], and b = [-5.25\degree, +5.25\degree], with some coverage at l $<$ 12\degree, covering a total area of $\sim$ 2400 square degrees,  and produced $\sim 10^8$ spectra.  It is impossible to deal with such a large amount of data manually. Initial automatic searches  have to be conducted to minimize the manual tasks. 
   At this step for each pixel, we assume \co[13] to be optically thin for Pair-1 and \CO{} for Pair-2. First of all, we decomposed the optically thin lines into components with a 1-dimensional watershed algorithm. Each component is Gaussian-like with a peak exceeding the nearby baseline or dip by over 5 RMS of the whole spectrum. Then the first and second moments were extracted for each component as center velocity and velocity dispersion. They were aligned with the optically thick counterpart to cut out segments of profiles to be selected. Two methods were then adopted to find profiles with double peaks.

   {Method one: we select optically thick profiles fulfilling the following criteria.}
   
   {1. A peak should appear on the blue side of central velocity. Such peak must exceed the nearby dip or baseline by 3 RMS, and away from the center velocity by more than velocity resolution.}

   {2. The peak on the blue side should be higher than the profile on the red side by at least 3 RMS. If a peak appears on the red side, the component is labelled with ``double-peak", or a "shoulder" label will be assigned.}
   
   {Method two: Gaussian fittings with two components were conducted to the optically thick profiles, and then results fulfilling the following criteria are selected.}

   {1. Two components should reside on both sides of the center velocity, and away from it by more than velocity resolution.}

   {2. The blue side Gaussian component must have a higher integrated intensity and peak.}
  
      

              
   
   \subsection{Manual Check}{\label{sec:man}}
   
   First look at the spectra of the candidates resulted from the automatic search suggests that many of them are not really blue-profiles, or hardly to say that. Therefore, further manual check is  necessary to refine the candidate list. Picking up the candidates of  likely blue-profiles is a hard work. We wrote some codes with graphic user interface to assist task. Then the pick-up work is only a couple of mouse clicks.  
   
   By checking the spectra of the candidates and comparing with the numeric simulations (c.f., Sec. \ref{sec:num}), the  candidates are rejected according to the following general rules: 
   

   \begin{itemize}
   \item[-]Spectra of either optically thick or thin lines are noisy: In the real data, the situation is much more complicated than the simulation due to the introduction of noises.  Further more, some candidates appear at the edge of a certain cell where the noise level is higher than those of central areas. Though this situation has been considered in the automatic search, there are still cases that the machine  would misjudge whether a peak is real or noise-induced, especially for optically thin lines.  
   \item[-]Two peaks of the optically thick line are far away (e.g. V$_{red}$-V$_{blue}$$>$ 10 \kmps): This is just empirical since we seldom find blue-profiles with very large peak separation in the literature. The line width (Full Width at Half Maximum, FWHM) of the optically thin line is too large, i.e., $>$10 \kmps.

   \item[-]Multi-peaks or flattened top in the optically thin line: It is usually very difficult to judge whether a blue-profile of the optically thick line comes from a single or multiple components if there exist multi-peaks or flattened top in the assumed optically thin line. Though there are cases that the peaks of the optically thin lines do not coincident with those of the optically thick lines, we suppose they still could arise from multi-components with very complicated physical conditions. However, the candidates with one situation are retained: for Pair-1 selected candidates, the \co[13]{} line also shows a blue-profile while the \CO{} emission is not detected. The two peaks of the $^{13}$CO line are between those of $^{12}$CO line. In this case it is likely that $^{13}$CO line is also optically thick, but the absorption is fainter than that of the $^{12}$CO line. In such a case self-absorption still exists to show a blue-profile, but the peak separation is smaller than that of the $^{12}$CO line.  

   
   
   \end{itemize}
   
   \subsection{Cleaning Candidates}
   The manual check is a tiresome work since the output of the automatic search returns hundreds of thousands of candidates. In the manual check step we cautiously relax the above criteria in order to avoid loss of some reliable candidates. After the manual checks, we make the Gaussian fittings to the line profiles. Since the fittings need human interactions in deciding parameters such as fitting ranges, line profiles of the candidates are checked again and some candidates are rejected at this step.  
   
   After the manual check, candidates with similar positions and velocities are grouped together. The similarity of two candidates is defined as {the following two conditions are satisfied}: 
   $$\Delta p = \sqrt{\Delta l^2cos{^2}b+\Delta b^2} \leq \Delta p_{0}; \Delta V_{c} \leq \Delta V_{0}$$ 
   where $\Delta l$, $\Delta b$ and $\Delta V_{c}$ are the differences of the Galactic coordinates and central velocities of two candidates; $\Delta p_0 = 1.0^\prime$ and $\Delta V_0 =0.2\, $\kmps{} in this work being based on the spatial and spectral resolution of the MWISP products.  For each group the sources are checked by eyes and those showing  blue-profile most significantly are  selected. 

   {Lastly, in a few cases blue-profiles are detected in both Pair-1 and Pair-2, we merge them to the Pair-2 group since the physical parameters (c.f., Fig. \ref{fig:specse}), such as column densities and system velocities, could be more correctly estimated due to the fact that the optically thin lines are single-peaked. }
     
   \section{The catalogue}\label{sec:cat}
   
   Finally, a total of 3533 sources are registered as possessing  blue-profiled lines, out of which 3329 sources are selected from Pair-1, and 204 sources are selected from Pair-2.  In Table \ref{tab:tab1} we present a subset of the candidates for illustration, while the whole catalogue is presented in the appendix\footnote{The full table can also be found at https://www.scidb.cn/en/detail?dataSetId=bc1c08c6e7ba426a9a07b8bad69b2ffb}. For each candidate we assign a name code, which is presented in Column 2. Name codes are based on the positions and velocities of the candidates, i.e., the (l, b) codes are reserved to the second decimal places and v to the first decimal places, and decimal points are removed from the final code.  As an example, if (l, b, v) = (123.450000, 1.230000, 123.4500), the code is assigned as ``12345+0123+1235''. Column 3 \& 4 give the Galactic coordinates of sources.  Column 5 gives the spatial association to the objects  from some well known catalogs, such as  IRAS PSC, NGC, the RAFGL catalog, the Sharpless catalog, and Lynds' dark nebular catalog, etc. We also checked some other catalogs such as H$_\textsc{II}$ candidates \citep{2014ApJS..212....1A,2018MNRAS.476.3981Y}, and the MSX dark cloud catalog \citep{2006ApJ...639..227S}.  In Column 6 we indicate whether our candidates show blue-profiles with other tracers in the literature. The associations in Column 5 \& 6 are shown in codes to save the table space, and noted as table footnotes. In Column 7 we give the distances of the sources. The distance of a source is obtained using the online distance calculator\footnote{http://bessel.vlbi-astrometry.org/node/378} which is based on the trigonometric parallax measurements of Galactic maser sources \citep{2016ApJ...823...77R,2019ApJ...885..131R} and assumes all molecular clouds are located on spiral arms. Using a Bayesian approach, this tool is believed to be a major improvement of the kinematic distance estimate of Galactic molecular clouds. For a few sources that are near the tangent direction (l $\sim$ 90\degree), the calculator does not give proper distances.     



   The second half of the table presents the derived parameters of the candidates.  Columns 8 \& 9 (V$_c$, V$_b$) are the characteristic velocities of sources.  V$_c$ represents the ``central'' velocities, which are derived from Gaussian fittings to the optically thin lines. V$_b$ represents positions where blue peaks are found, which are largely obtained from double-component Gaussian fittings. However, in some cases when Gaussian fittings do not converge, we use the peak positions by eyes. Column 10 presents the skewness parameter, $\delta$V = (V$_{thick}$-V$_{thin}$)/$\Delta$V$_{thin}$, as suggested by \citet{1997ApJ...489..719M}. 

   \small
%\begin{table}
\setlength{\tabcolsep}{3pt}
\begin{center}
    %\centering
    \begin{landscape}
        \begin{ThreePartTable}
            \begin{TableNotes}
                \item[] Column Heads: (1): internal serial number; (2): name code; (3)\&(4): Galactic Coordinates; (5): association to some known catalogues (c.f. next foot note); (6): infall signature reported in the literature (c.f. next foot note); (7): distance; (8): central velocity; (9): velocity at the blue peak; (10): skewness parameter (see text); (11): expected Tmb of optically thick line; (12): peak Tmb of optically thin line; (13): excitation temperature; (14): H$_2$ column density; (15): pair from which the blue-profile is obtained.
                \item[$\boldsymbol{\alpha}$] Ref. Codes: (1) IRAS Point Source Catalog; (2) New General Catalog (NGC); (3) \citeauthor{1999yCat.2094....0P} \citeyear{1999yCat.2094....0P}; (4) \citeauthor{1953ApJ...118..362S} \citeyear{1953ApJ...118..362S}; (5) \citeauthor{1996yCat.7007....0L} \citeyear{1996yCat.7007....0L}; (6) \citeauthor{2014ApJS..212....1A} \citeyear{2014ApJS..212....1A}; (7) \citeauthor{2006ApJ...639..227S} \citeyear{2006ApJ...639..227S}; (8) \citeauthor{2018MNRAS.476.3981Y} \citeyear{2018MNRAS.476.3981Y}; (9) \citeauthor{2003ApJS..149..375S} \citeyear{2003ApJS..149..375S}; (10) \citeauthor{2005A&A...442..949F} \citeyear{2005A&A...442..949F}; (11) \citeauthor{2007ApJ...663.1092K} \citeyear{2007ApJ...663.1092K}; (12) \citeauthor{2009MNRAS.392..170S} \citeyear{2009MNRAS.392..170S}; (13) \citeauthor{2011ApJ...740...40R} \citeyear{2011ApJ...740...40R}; (14) \citeauthor{2012A&A...538A.140K} \citeyear{2012A&A...538A.140K}; (15) \citeauthor{2013ApJ...776...29L} \citeyear{2013ApJ...776...29L}; (16) \citeauthor{2016MNRAS.456.2681Q} \citeyear{2016MNRAS.456.2681Q}; (17) \citeauthor{2016MNRAS.461.2288H} \citeyear{2016MNRAS.461.2288H}; (18) \citeauthor{2020RAA....20..115Y} \citeyear{2020RAA....20..115Y}; (19) \citeauthor{2021ApJ...910..112K} \citeyear{2021ApJ...910..112K}
                \item[$\boldsymbol{\beta}$] For a few sources the distances are not given because the distance calculator does not output proper values.
                \item[$\boldsymbol{\gamma}$] For a number of entries the values of N(H$_2$) are not given because the optical depths ($\tau_{thin}$) cannot be estimated due to the complex line profiles.
            \end{TableNotes}
            \begin{longtable}{rrrrllrrrrrrrrr}
                \caption{The catalog}\label{tab:tab1}                                                                                                                                                                                                                                                                                                                                                                                                                                                     \\
                \toprule
                \multicolumn{1}{c}{No}  & \multicolumn{1}{c}{Code} & \multicolumn{1}{c}{L}   & \multicolumn{1}{c}{B}   & \multicolumn{1}{c}{Association$^\alpha$} & \multicolumn{1}{c}{I.S.$^\alpha$} & \multicolumn{1}{c}{D$^\beta$} & \multicolumn{1}{c}{V$_c$}         & \multicolumn{1}{c}{V$_b$}         & \multicolumn{1}{c}{$\delta$V} & \multicolumn{1}{c}{T$_1$} & \multicolumn{1}{c}{T$_2$} & \multicolumn{1}{c}{T$_{ex}$} & \multicolumn{1}{c}{N$_{H2}${$^\gamma$}} & \multicolumn{1}{c}{Pair} \\
                \multicolumn{1}{c}{}    & \multicolumn{1}{c}{}     & \multicolumn{1}{c}{}    & \multicolumn{1}{c}{}    & \multicolumn{1}{c}{}                     & \multicolumn{1}{c}{}              & \multicolumn{1}{c}{(kpc)}     & \multicolumn{1}{c}{(km s$^{-1}$)} & \multicolumn{1}{c}{(km s$^{-1}$)} & \multicolumn{1}{c}{}          & \multicolumn{1}{c}{(K)}   & \multicolumn{1}{c}{(K)}   & \multicolumn{1}{c}{(K)}      & \multicolumn{1}{c}{(cm$^{-2}$)}         & \multicolumn{1}{c}{}     \\
                \multicolumn{1}{c}{(1)} & \multicolumn{1}{c}{(2)}  & \multicolumn{1}{c}{(3)} & \multicolumn{1}{c}{(4)} & \multicolumn{1}{c}{(5)}                  & \multicolumn{1}{c}{(6)}           & \multicolumn{1}{c}{(7)}       & \multicolumn{1}{c}{(8)}           & \multicolumn{1}{c}{(9)}           & \multicolumn{1}{c}{(10)}      & \multicolumn{1}{c}{(11)}  & \multicolumn{1}{c}{(12)}  & \multicolumn{1}{c}{(13)}     & \multicolumn{1}{c}{(14)}                & \multicolumn{1}{c}{(15)} \\

                \midrule
                \endfirsthead

                \multicolumn{15}{c}{Table \ref{tab:tab1}: Continued}                                                                                                                                                                                                                                                                                                                                                                                                                                      \\
                \toprule
                \multicolumn{1}{c}{(1)} & \multicolumn{1}{c}{(2)}  & \multicolumn{1}{c}{(3)} & \multicolumn{1}{c}{(4)} & \multicolumn{1}{c}{(5)}                  & \multicolumn{1}{c}{(6)}           & \multicolumn{1}{c}{(7)}       & \multicolumn{1}{c}{(8)}           & \multicolumn{1}{c}{(9)}           & \multicolumn{1}{c}{(10)}      & \multicolumn{1}{c}{(11)}  & \multicolumn{1}{c}{(12)}  & \multicolumn{1}{c}{(13)}     & \multicolumn{1}{c}{(14)}                & \multicolumn{1}{c}{(15)} \\
                \midrule
                \endhead

                \bottomrule
                \multicolumn{15}{c}{Continued on next page}                                                                                                                                                                                                                                                                                                                                                                                                                                               \\
                \endfoot

                \bottomrule
                \insertTableNotes
                \endlastfoot

                1001                    & 03105$+$0017$+$0118      & 31.0500                 & 0.1667                  & 1,  3,  6                                & \nodata                           & 1.90                          & 11.83                             & 11.20                             & $-$0.65                       & 4.1                       & 1.2                       & 11.2                         & 5.04E$+$20                              & 1                        \\
                1002                    & 03105$+$0076$+$0506      & 31.0500                 & 0.7583                  & 6,  7                                    & \nodata                           & 0.68                          & 50.62                             & 50.07                             & $-$0.44                       & 4.2                       & 1.3                       & 8.8                          & 5.82E$+$21                              & 2                        \\
                1003                    & 03105$+$0407$+$0063      & 31.0500                 & 4.0750                  & 5,  6                                    & \nodata                           & 0.24                          & 6.29                              & 5.68                              & $-$0.59                       & 5.8                       & 1.2                       & 8.4                          & 5.27E$+$20                              & 1                        \\
                1004                    & 03106$-$0067$+$0088      & 31.0583                 & $-$0.6667               & 6,  7                                    & \nodata                           & 0.24                          & 8.77                              & 8.41                              & $-$0.42                       & 4.1                       & 1.8                       & 8.3                          & 7.75E$+$20                              & 1                        \\
                1005                    & 03107$+$0078$+$0512      & 31.0667                 & 0.7833                  & 6,  7                                    & \nodata                           & 0.60                          & 51.22                             & 50.26                             & $-$0.48                       & 6.5                       & 2.9                       & 11.2                         & 3.02E$+$21                              & 1                        \\
                1006                    & 03107$+$0313$+$0103      & 31.0750                 & 3.1333                  & 5,  6                                    & \nodata                           & 0.24                          & 10.31                             & 9.32                              & $-$0.46                       & 6.8                       & 3.4                       & 10.0                         & 4.08E$+$21                              & 1                        \\
                1007                    & 03107$+$0524$+$0084      & 31.0750                 & 5.2417                  & 5,  6                                    & \nodata                           & 0.24                          & 8.41                              & 7.40                              & $-$0.51                       & 2.7                       & 3.4                       & 6.9                          & 6.93E$+$21                              & 1                        \\
                1008                    & 03107$-$0044$+$0085      & 31.0750                 & $-$0.4417               & 6                                        & \nodata                           & 0.24                          & 8.50                              & 8.14                              & $-$0.32                       & 4.3                       & 1.3                       & 10.9                         & 6.56E$+$20                              & 1                        \\
                1009                    & 03108$-$0042$+$0086      & 31.0833                 & $-$0.4167               & 6                                        & \nodata                           & 0.24                          & 8.61                              & 7.96                              & $-$0.68                       & 4.3                       & 1.3                       & 9.1                          & 5.32E$+$20                              & 1                        \\
                1010                    & 03109$-$0339$+$0100      & 31.0917                 & $-$3.3917               & 5,  6                                    & \nodata                           & 0.24                          & 10.00                             & 9.83                              & $-$0.23                       & 5.2                       & 2.1                       & 10.3                         & 7.41E$+$20                              & 1                        \\
            \end{longtable}
        \end{ThreePartTable}
        %\end{table}
    \end{landscape}
\end{center}

   
   Columns 11-13 (T$_1$, T$_2$, T$_{ex}$) are related temperatures of the sources. T$_2$ represents the peak main beam temperature of the optically thin line. T$_1$ represents the expected main beam temperature of the optically thick line. Since the line profile shows self-absorbed feature, the ``expected'' peak is derived by Gaussian fitting to the whole profile while excluding the portion that shows self-absorption, i.e., we use the waist and foot portions in the fittings. Though the fitting portions have been carefully adjusted,  {we caution that in some cases the derived values of T$_1$ are  sensitive to the fitting portions adopted, so that the errors could be very large -- they may reach to the values of T$_1$ themselves in some extreme cases depending on the environments of individual sources, e.g. when multi-components coexist with the blue-profiles. However, in most cases, we expect the uncertainties being less than 20\%.} T$_{ex}$ is the excitation temperature calculated from T$_1$\citep{2015PASP..127..266M}: 
    $$T_{ex} = T_0[ln(1+\frac{1}{\frac{T_1}{T_0}+\frac{1}{exp({T_0/T_{bg})}-1}})]^{-1} $$  where $T_0 = h\nu_0/k$ =  5.53 K  for Pair-1, or   5.29 K for Pair-2, T$_{bg}$ (= 2.73 K) is the background temperature of the Universe. {Here we assume the filling factor being close to unity. We use \co[13]{} to estimate the excitation temperatures of Pair-2 sources because in many cases \co{} profiles are very complicated so that we cannot obtain reliable ``expected'' main beam temperatures. In such cases T$_{ex}$ might be underestimated since the \co[13]{} lines are not optically thick enough}.
   
    The column densities of H$_2$ (Column 14) are estimated from the integrated intensities of the optically thin lines assuming local thermal equilibrium {(LTE)}. Under such assumption, the column densities of \co[13]{} or \CO{} are calculated by the following formula \citep{2016PASP..128b9201M}:
    $$ N(\co[13]|\CO) = f_1 \times f_0\times\frac{(T_{ex}+0.88)exp(\frac{T_0}{T_{ex}})}{exp(\frac{T_0}{T_{ex}})-1}\times\frac{\int T_{MB}\,dv \quad(K\,km\,s^{-1})}{f(J_\nu(T_{ex})-J_\nu(T_{bg}))} \qquad(cm^{-2})$$
    where f$_0$ is approximately 2.48 $\times 10^{14}$ for both \co[13]{} and \CO{}, T$_0$ equals 5.29 for \co[13]{} and 5.27 for \CO, f is the beam filling factor which we assume to be 1.0 in this work, and $J_\nu(T) = T_0/(exp(T_0/T)-1)$. The optical depth correction factor $f_1 = \tau/(1-exp(-\tau))$. Where $\tau$ is calculated by:
    $$\tau=-ln[1-\frac{T_2}{f(J_\nu(T_{ex})-J_\nu(T_{bg}))}]$$
    
    The total column densities of molecular hydrogen, $N(H_2)$, are obtained assuming the CO abundance ratios, $ N(H_2) \approx 3.9 \times 10^5 N(^{13}CO)$ or $N(H_2) \approx 3.0 \times 10^6 N(C^{18}O)$\citep{2019PASA...36...49A}, which are derived from the infrared dark cloud and close to those from star-forming regions\citep{2008ApJ...679..481P}. {Several factors affect estimations of N($_{H2}$): 1) Assumption of LTE which may not be true if the sources are actually in the process of infall; 2) T$_{ex}$s estimations; 3) Abundances of \co[13] and \CO{} relative to those of molecular hydrogen change from region to region \citep{2015ARA&A..53..583H};  4) Beam filling factors for the sources are not always close to unity. Therefore, it is difficult to estimate the overall uncertainties of N($_{H2}$).} In the last column (Column 15) we indicate the line pairs adopted in searching for the blue-profiles.      
   
   
    \begin{figure*}
        \subcaptionbox{05363+0004+0235\label{fig:specsa}}[0.495\textwidth]{\includegraphics[width=8.0cm,angle=0]{ms2022-0304fig2a.pdf}} %(Ideal 12CO & 13CO) 
     \subcaptionbox{02869+0377+0073\label{fig:specsb}}[0.495\textwidth]{\includegraphics[width=8.0cm,angle=0]{ms2022-0304fig2b.pdf}} %(Ideal 13CO & C18O, 12CO complicated)
     \subcaptionbox{01272+0069+0173\label{fig:specsc}}[0.495\textwidth]{\includegraphics[width=8.0cm,angle=0]{ms2022-0304fig2c.pdf}} %(thick line contaminated, red-side)
     \subcaptionbox{01282-0019+0349\label{fig:specsd}}[0.495\textwidth]{\includegraphics[width=8.0cm,angle=0]{ms2022-0304fig2d.pdf}} %(Velocity of 3 lines shift to red )
     \subcaptionbox{01432+0268+0156\label{fig:specse}}[0.495\textwidth]{\includegraphics[width=8.0cm,angle=0]{ms2022-0304fig2e.pdf}} %(wide thin line)
     \subcaptionbox{01288-0024+0355\label{fig:specsf}}[0.495\textwidth]{\includegraphics[width=8.0cm,angle=0]{ms2022-0304fig2f.pdf}} %(wide thin line)
     \subcaptionbox{01116-0014+1503\label{fig:specsg}}[0.495\textwidth]{\includegraphics[width=8.0cm,angle=0]{ms2022-0304fig2g.pdf}} %(Large vc)
     \subcaptionbox{02849+0399+0064\label{fig:specsh}}[0.495\textwidth]{\includegraphics[width=8.0cm,angle=0]{ms2022-0304fig2h.pdf}} %(13CO>12CO)
     
        \caption{The spectra (left) and gas distributions (right) of the candidates. On the left panels, the red, green and blue lines represent {\CO, \co[13] and \co} emissions, respectively. To enhance s/n ratios of the lines, the spectra are smoothed with median filter for each 3 velocity channels. For the same reason, the \CO{} spectra are also smoothed with 3$\times$3 spatial pixels,   The  dashed line indicates the system velocity. On the upper-right corners we indicate the Galactic coordinates (l, b), and the line-pairs used in the searching procedure. On the upper-left are the name codes of the candidates. The right panels show the integrated intensity map of the optically thin lines (contours) overlaid on those of the optically thick lines (grey scale). The red pluses indicate the positions where the blue-profiles are detected. \label{fig:specs}}
        \end{figure*}
        
   
    Fig. \ref{fig:specs} shows some examples of the spectra (left) and the relative location (right) where blue-profiles are detected. Again, for the limitation of the main body, the whole set of the figures is given in the appendix.  Fig. \ref{fig:specsa} (05363+0004+0235) shows an ideal blue-profile in \co{} emission with virtually symmetric profiles both in \co[13]{} and \CO{} lines, the blue-profile happens at the saddle point between two local maxima (right panel); Fig. \ref{fig:specsb} (02869+0377+0073) shows a candidate selected from Pair-2. The \co{} line of this source shows complicated peaks;      
    Fig. \ref{fig:specsc} (01272+0069+0173) shows a blue-profile selected from Pair-1, but the red side of the \co{} line is probably contaminated by another component. The \CO{} line, weak as it is,  also suggests such a possibility. However, we could recognize this candidate since the contamination is not too serious to affect our judgement;  Fig. \ref{fig:specsd} (01282-0019+0349) shows a candidate that does not show clear blue-profile, but the \CO{} and \co[13]{} peaks are shifted redward with respect to the \co{}  peak. Such a configuration happens when infall velocity is relative large while the optical depth is moderate (c.f., the lower-right panel of Fig. \ref{fig:nsim}). Fig. \ref{fig:specse} (01432+0268+0156) shows an example that both \co{} and \co[13]{} lines exhibit blue-profiles, with the peak separation of \co{} greater than that of the \co[13]{} line, while \CO{} is marginally detectable. The blue-profile happens at virtually the central part of a dense clump.  The examples presented here are typical cases in our candidates.
    
    In Fig. \ref{fig:specs} we also present some extreme cases, which might be of interests to readers. Fig. \ref{fig:specsf} (01288-0024+0355) shows a source with  wide profiles, even for the optically thin lines (\co[13]{} and C$^{18}$O), up to FWHM $\sim$ 6.6 and 4.0 \kmps, respectively. Fig. \ref{fig:specsg} (01116-0014+1503) shows a target that has a very large system velocity, $\sim$ 150 \kmps. The kinematic distance being estimated $\sim$ 8.0 kpc, it is one of the most distant object in the catalog. Interestingly, this source presents blue-profiles both in \co{} and \co[13]{} lines. Fig. \ref{fig:specsh} (02849+0399+0064) shows a source that is rarely seen in the Galactic molecular clouds, i.e., the \co[13]{} line intensity is stronger than that of the \co{} line. Aside of strong self-absorption of \co[12], there could be also a possibility of unusual \co[13]{} abundance and excitation environment.  
 
   
    In the right panel of each subset, we show the integrated intensity map of individual candidate. An interesting fact that should be noted here is that the infall-signatures do not always coincide with the clump centers (the points with the highest column densities). It would be interesting to ascertain whether this is a real fact or just an observational effect, which might be helpful in our understandings of the initial phase of star formation.
      
   
   \section{Discussion}\label{sec:disc}
   \subsection{Numerical Experiments}\label{sec:num}
   
   In order to see what a blue-profiled line appears in different environments, {and to assist us to identify infall candidates from a large number of outputs from the machine work, } we have done a series of numerical experiments. The experiments tried to simulate cores with inflowing gas, but with very simple models, e.g., fixed infall velocities, optical depths and line widths. Temperature and density profile variations are not considered in the experiments. To do this, we generate an emission line of Gaussian profile centered at 0.0 \kmps, and $\sigma$ = 2.0 \kmps, which has no physical meaning other than a scale in the velocity dimension, and upon which many other parameters are assumed.  The peak value is normalized to unity. A number of absorption lines, which are also of Gaussian profiles but with negative values, are put upon the emission line to mimic the absorption. The centers of the absorption lines are shifted a little to the red by fractions of $\sigma$, i.e., V$_{in}$ = f$_{dV}\sigma$. Here we use V$_{in}$ to designate the central velocity shift of the absorption feature because it is somehow related to the infall velocity in the reality. The negative maxima are assumed to be a few percent of unit (dip = f$_{dip}$) because the optical depths of infalling gas are expected to be optically thinner than the background emission. In the reality, the widths of the absorption lines are expected to be narrower than those of the emission lines, and so that are  set to some fractions of $\sigma$ ($\sigma_{abs}$ = f$_\sigma\sigma$). 
   \begin{figure}
   \includegraphics[width=16cm]{ms2022-0304fig3.pdf}
   \caption{One example of the numerical experiments with f$_\sigma$ = 0.5. The parameters are shown at the upper-right corner of each frame. The blue lines represent the self-absorbed optically thick lines showing  blue-profile lines, and the red lines indicate the un-absorbed optically thin lines for comparison. We also present $\delta V$ suggested by \citet{1997ApJ...489..719M} for different cases. \label{fig:nsim}}
   \end{figure}
   
   In Fig. \ref{fig:nsim} we show some results of the experiments with f$_\sigma$ = 0.5 and various f$_{dV}$'s and f$_{dip}$'s. The parameters are shown at the upper-right corner of each frame.  Though we also have done the experiments by varying f$_\sigma$, the results do not show very much difference and are not shown in this paper. 
   From Fig. \ref{fig:nsim} we see that the observability of a blue-profile line strongly depends on the strength of the absorption (f$_{dip}$), but less sensitive to the velocity shift (V$_{in}$). For f$_{dip}$ $\sim$ 0.15, approximately corresponding to the optical depth $\tau\sim$ 0.15, a blue-profile is hardly seen even in our noise-free experiments and is thus not likely to be detected in real observations. In the cases of f$_{dip}\sim$ 0.3 ($\tau\sim$ 0.36),  red shoulders are seen clearly. In such cases real detections are  mostly missed because our auto-search algorithm searches only for double peaks.   The blue-profiles become observable when f$_{dip}\sim$ 0.45 ($\tau\sim$ 0.6) with a small velocity shift but are still unclear at large velocity shifts. For nearly all velocity shifts, the signatures are clearly observable at f$_{dip}\sim$0.6 ($\tau\sim$ 0.9). 
   
   
   At the corner of each frame, we also present the dimensionless skewness parameter, $\delta$V,  suggested by \citet{1997ApJ...489..719M} as an indicator of the significance of blue asymmetry.  As can be seen from Fig. \ref{fig:nsim}, $\abs[\delta V]$ generally increase as f$_{dip}$ but decrease as f$_{dV}$. In fact, our experiments suggest the absolute value of $\delta$V decreases also with  f$_\sigma$. 
   
   \subsection{Parameter Distributions}
   Up to date, the MWISP project covers an area of $\sim$ 2400 square degrees toward the northern Galactic plane.  Though the sample is still quite far from complete, due to the fact that the blue-profiles are recognized without any other presumption, the blind search strategy enables us to avoid  bias toward known star-forming regions. This kind of work will provide a better approach to explore parametric space of the infalling molecular clumps. In the following we discuss the parametric distributions of our samples.   
   
   In Fig. \ref{fig:vc_distr} we present the distributions of central velocities of the optically thin lines. In the main frame we show the pair-separated distributions while the overall distribution is shown in the inset, and the rest figures are treated in the same way. The distribution is generally in Gaussian style peaked around 5 \kmps, and shows excess in both wings.  A large amount of sources gather within  $\abs[V_{LSR}] \le$   10 \,\kmps, suggesting that our sources are mainly located  close to the Sun. We notice here, though not shown in the figure, the smallest and largest central velocity are -75.5 and 150.3 \kmps, respectively. An interesting feature seen from Fig. \ref{fig:vc_distr} is that Pair-1 selected sources have lower radial velocities (with a median value of $\sim$ 6.8 \kmps) than  Pair-2 selected sources (median $\sim$ 7.9 \kmps).
   
   Fig. \ref{fig:tex_distr} shows the distributions of excitation temperatures T$_{ex}$. The excitation temperatures are generally low, all less than 50 K, with a minimum value $\lesssim$ 4.5 K. A vast majority of sources ($\sim$ 94\%) show T$_{ex}$ less than 20 K. The median value is $\sim$ 11.7 K, which is a little bit less than that of all infall sources in the literature \citep{2022RAA....22i5014Y}. The distribution of Pair-2 does not seem to follow a regular function, probably because of the paucity of samples. On the other hand, the profiles of Pair-1 as well as of the overall are quite smooth. The best log-normal fitting to the overall distribution gives a parameter-pair ($\sigma,\mu$) = (2.41, 0.25) with $\chi^2$ = 3.2. This results is very close to that derived from Planck cold clumps \citep{2012ApJ...756...76W}, indicating that our sources may be of similar properties to those cold clumps. {We note however, as stated in Section \ref{sec:cat}, the excitation temperatures derived from this work are subject to rather large uncertainties.}
   \begin{figure}
      \subcaptionbox{Vc\label{fig:vc_distr}}[0.495\textwidth]{\includegraphics[width=8cm]{ms2022-0304fig4a.pdf}}
      \subcaptionbox{T$_{ex}$\label{fig:tex_distr}}[0.495\textwidth]{\includegraphics[width=8cm]{ms2022-0304fig4b.pdf}} 
      \subcaptionbox{N(H$_2$)\label{fig:nh_distr}}[0.495\textwidth]{\includegraphics[width=8cm]{ms2022-0304fig4c.pdf}} 
      \subcaptionbox{FWHM\label{fig:fwhm_distr}}[0.495\textwidth]{\includegraphics[width=8cm]{ms2022-0304fig4d.pdf}} 

   \caption{Frequency distributions of (a) central velocities; (b) excitation temperatures;  (c) derived H$_2$ column densities (cm$^{-2}$ on log scale); (d) Line widths (FWHM) of the optically thin lines. In the inset of (b), the  dashed curve is a log-normal fit with ($\sigma, \mu$) = (2.41, 0.25) to overall T$_{ex}$ distribution. In the inset of (d), the dashed curve is the best log-normal fit with ($\sigma, \mu$) = (0.34, 0.076), to the overall distribution of FWHMs. For each parameter, the bins for both Pair-1 and Pair-2 as well as for the overall are set to same.  \label{fig:p_dist}}
   \end{figure}

   Fig. \ref{fig:nh_distr} presents the distributions of column densities of molecular hydrogen, N(H$_2$). It can be clearly seen the column densities traced by \co[13]{} (Pair-1) and \CO{} (Pair-2) are obviously separated. The values for the former group range from $2.6\times10^{20}$ to $1.1\times10^{23}$ cm$^{-2}$ with a median value of $1.3\times10^{21}$ cm$^{-2}$, while  those for the latter range between $1.1\times10^{21}$ and $1.8\times10^{23}$ cm$^{-2}$ with a median value of $6.4\times10^{21}$ cm$^{-2}$. The column densities of our samples are distributed narrower, and have smaller median value than those infall sources in the literature \citep{2022RAA....22i5014Y}. This may be because the latter utilized a large variety of molecular lines which include high density tracers, and used different observing facilities including ones with high angular resolution. 
   
   Our result means that a major part of the Pair-2  and a significant part of Pair-1 sources have higher column densities than the possible star formation threshold, i.e. 6.3$\times10^{21}$ cm$^{-2}$ \citep{2014Sci...344..183K,2010ApJ...724..687L,2021ApJS..254....3M}. All of our candidates have column densities less than the threshold of massive star formation proposed by \citet{2008Natur.451.1082K}, i.e., N(H$_2$) $\sim$2.1$\times10^{23}$ cm$^{-2}$. Three factors may reduce the estimated value of N(H$_2$): 1) large beam size of the telescope (hence low filling factor) may dilute the peak values; 2) our candidates are frequently seen outside of the central part of certain  clumps; and 3) T$_{ex}$'s of Pair-2 selected candidates are often underestimated hence the derived H$_2$ column densities. Therefore, whether our sources are massive star-forming candidates should await further deliberate studies.  
   
   The distribution of the column densities from \CO{} is roughly symmetric with respect to the median value, suggesting a log-normal profile (note the abscissa is in log-scale). On the other hand, that from \co[13]{} shows a log-normal style only in the first several bins, up to $\sim 4.0\times10^{21}$ cm$^{-2}$, and deviate from the log-normal distribution at higher column densities. Since Pair-1 objects account for a vast majority of our sample, the overall distribution generally resembles that from \co[13]{} (inset of Fig. \ref{fig:nh_distr}).
   

   Fig. \ref{fig:fwhm_distr} shows the distributions of the FWHMs of optically thin lines.  The distributions of the sources selected from two pairs do not show significant difference, except for a slightly shift of median values (1.21 vs 1.10 \kmps{} for Pair-1 and Pair-2, respectively). A Kolmogorov–Smirnov (K-S) test with a +0.10 \kmps{} shift of Pair-2 sources results in P-value of $\sim$0.47, rejecting the hypothesis that the two profiles are different. A log-normal fit to the overall distribution (inset) results in a $\sigma$ = 0.35 $\pm$ 0.013 and $\mu$ =0.076 $\pm$ 0.014. A K-S test results in P-value of 0.23, thus accepting the log-normal distribution hypothesis. 
   
   
   In Fig. \ref{fig:dv_distr} we show the distribution of the skewness parameters, $\delta$V, defined by \citet{1997ApJ...489..719M}.  As can be seen in the figure, the values range between -1.6 and 0.0 with a mean value $\sim$ -0.5. Since our samples are selected in favor of blue-profiles, it is not surprising that the values are mostly less than -0.25, a criterion as significant infall signature suggested by \citet{1997ApJ...489..719M}.

   \begin{figure}
      \centering
      \includegraphics[width=12cm,angle=0]{ms2022-0304fig5.pdf}
      \caption{The distributions of the skewness parameters, $\delta$V, of the sources selected from Pair-1 (blue), Pair-2 (red) and overall (inset). \label{fig:dv_distr}}
      \end{figure}



   Though  $\delta$V is defined as inversely correlated  to the line widths of the optically thin line, as shown in Fig. \ref{fig:dv_fwhm}, we have not found any clear trend of such correlation. 
    

   \begin{figure}
    \centering
   \includegraphics[width=10cm,angle=0]{ms2022-0304fig6.pdf}
   \caption{A plot of $\delta$V and line width (FWHM) of all candidates. The blue pluses and red crosses represent Pair-1 and Pair-2 selected sources, respectively.\label{fig:dv_fwhm}}
   \end{figure}
   
   \subsection{Spatial Distribution}

   Fig. \ref{fig:s_distr} shows the spatial distribution of our candidates, overlaid on the imaginary face-on view of the Milky Way (credit: Xing-Wu Zheng \& Mark Reid  BeSSeL/NJU/CFA\footnote{https://astronomy.nju.edu.cn/xtzl/EN/index.html}). It can be seen clearly that a vast majority of our candidates are  located  in the first quadrant, especially those selected from Pair-2, and outside of the 3-kpc molecular ring. This is not surprising because of the facility limitation and the confusion in the inner Galactic molecular clouds. It should be noted here that most of the sources are located in the spiral arms. This may be due to the distance estimator \citep{2016ApJ...823...77R,2019ApJ...885..131R}, which assumes that molecular clouds are presumably located in the spiral arms. Though this could be true  for a majority of the our sources, we advice the  researchers to interpret this result with caution.

   Fig. \ref{fig:z_distr} is the vertical distribution (source offset from the Galactic midplane) of our candidates. The main frame shows the pair-separated distribution while the inset is that of overall.  Gaussian fitting of the overall distribution shows the FWHM thickness is $\sim$ 85 pc, which is reminiscent of the thin molecular disk suggested  by \citet{2021ApJ...910..131S}. A majority (3034, $\sim$85.8 \%) of the sources are located within this range (i.e., $\abs[z]{} \le$ 42.5 pc). The profile of Pair-1 selected candidates is wider than that of the Pair-2 ones (85 pc vs 54 pc in FWHM). Interestingly, we notice the presence of excess of sources at the foot of the Gaussian fit (inset of Fig. \ref{fig:z_distr}), similar to that of the vertical distribution of molecular clouds by \citet{2021ApJ...910..131S}, and suggesting that star formation may also take place in the thick disk component. It is also interesting to mention that a number (16) of sources are located far beyond the Galactic midplane (i.e., $\abs[z]{} \ge$ 200 pc). These sources are worthy to investigate to show if there are star-forming activities even far from the Galactic plane.

   
   \begin{figure}
      \centering
      \includegraphics[width=12cm,angle=0]{ms2022-0304fig7.pdf}
      \caption{Spatial distribution of the infall candidates overlaid on the imaginary face-on view of the Milky Way (credit: Xing-Wu Zheng \& Mark Reid  BeSSeL/NJU/CFA). The green $\odot$ sign indicates the location of the Sun. The Galactic center is at the origin. Our candidates are designated by blue pluses (Pair-1) and red crosses(Pair-2). \label{fig:s_distr}}
   \end{figure}

   \begin{figure}
      \centering
      \includegraphics[width=12cm,angle=0]{ms2022-0304fig8.pdf}
      \caption{The vertical distribution. The blue and red bars represents the Pair-1 and Pair-2 selected sources, respectively. The pink dashed curve in the inset is a Gaussian fit to the overall distribution with $\sigma$ = 36 pc (FWHM = 85 pc).    \label{fig:z_distr}}
   \end{figure}

   In Fig. \ref{fig:l_v} we show source distribution in L - \vlsr{} space. Apart from the fact that sources selected from Pair-2 are more concentrated to the first quadrant, the overall distribution is reminiscent of the result by \citet[Fig. 3]{2001ApJ...547..792D}. Our sources exist in virtually all large-scale components in the northern Galaxy, such as  the Sagittarius-Scutum arm, the 3-kpc molecular ring, Lindblad Ring \& the local arm, the Perseus arm, and even the outer arm. If our sources represent quite a part of the infalling gas motion, the overall distribution indicates that there exist star-forming activities in nearly all large components in the Galaxy at present epoch. 

   \begin{figure}
      \centering
      \includegraphics[width=12cm,angle=0]{ms2022-0304fig9.pdf}
      \caption{The source distribution in L - \vlsr{} space. As in Fig. \ref{fig:s_distr}, our sources are designated as the blue pluses (Pair-1) and red crosses (Pair-2). \label{fig:l_v}}
      \end{figure}

   
\subsection{Confirmation Rates}
 As known to researchers, \co{} and its isotopic molecules are not ideal tracers to the infall motions, and HCO$^+$ would be much more efficient. Therefore,  shortly after we initiated this project, follow-up observations were made with HCO$^+$ (J = 1 - 0) and HCN (J = 1 - 0) lines. \citet{2020RAA....20..115Y} selected 133 candidates from our sample with a restriction that T$_{MB}$(\CO) $\geqq$ 1 K. They confirmed 56 sources to have infall signatures in HCO$^+$ and/or HCN lines. The overall confirmation rate is $\sim$ 42\%, the respective  rate being 40\% and 49\% for Pair-1 and Pair-2. Further, \citet{2021ApJ...922..144Y} used IRAM to carry out mapping observations towards 24 out of the 56 confirmed sources and finally confirmed 9 blue asymmetry profiles in HCO$^+$ (J = 1 - 0) line. These 9 sources are regarded as infall sources with high confidence. 
 
 On the other hand, Yu et al. (2023, in prep.) selected a sample of 210 sources with  more relaxed conditions to study the infall signatures. They found a total of 40 sources showing blue-profiles in HCO$^+$ emission line. The gross detection rate is $\sim$ 19\%, with 18\% for Pair-1 and 28\% for Pair-2 objects. In general, Pair-2 selected candidates have higher confirmation rate than Pair-1 selected ones. This is not surprising in light of that \CO{} traces higher column density than \co[13]{} does. If the sub-sample represents our overall sample adequately, we would expect the confirmation rate being no less $\sim$ 20\%. 

 In conclusion, starting from the \co[12], \co[13] and \CO{} (J = 1 - 0) lines, our approach to search for infall sources is effective. 

\section{Summary}\label{sec:sum}
Based on the MWISP data, we conducted a survey of infalling clumps in the Galaxy, which presumably show blue asymmetric profiles in the optically thick lines and single peaks in the  optically thin lines. Both \co\&\co[13]{} and \co[13]\&\CO{} pairs are utilized to carry out the work. The automatic search and manual check are conducted to select candidates out of $\sim$ 10$^8$ spectra. The main results are summarized as following:
\begin{enumerate}
\item A total of 3533 sources are finalized as candidates in the infalling phase of star formation, in which 3329 candidates are selected from the \co\&\co[13]{} pair and 204 are from the \co[13]\&\CO{} pair. 
\item Though being manually checked and filtered, the candidates show a wide range of spectra with complicated profiles. The locations that show blue-profiles do not always coincide with central parts of molecular clumps where the highest column densities are detected, but a significant part are rather located at the edges. 
\item The analysis of physical parameters of the sources suggests the Pair-2 candidates are colder, and have higher column densities, than Pair-1 ones. The overall distribution of T$_{ex}$ follows a log-normal style and is consistent to that by \citet{2012ApJ...756...76W}, suggesting the properties of our sources are quite similar to those of Planck cold  clumps. 
\item The line widths of the optically thin lines for both Pair-1 and Pair-2 sources are in log-normal style. The mean values of the two groups are different by < 0.1 \kmps.  A K-S test indicates that the distribution are the same when line widths of Pair-2 sources are added by 0.1 \kmps. 
\item Most of the sources are located within the first quadrant, especially those selected from the \co[13]\&\CO{} pair. The system velocities of the candidates ranges from $\sim$ -70 to $\sim$ 150 \kmps, and are present in virtually all large scale components in the northern Galaxy. The vertical distribution suggests that our sources are located primarily within the thin disk, but still present in the extended thick disk.
\item A sketchy estimation suggests that the confirmation rate of our sample could be no less than $\sim$ 20\%, indicating our strategy is a good start to study the very early phase of star formation systematically. 
\end{enumerate}
   \begin{acknowledgements}
      We are deeply obliged to the MWISP working group, Yang Su, Xin Zhou, Yan Sun, Jixian Sun, Dengrong Lu and Binggang Ju, and the observation assistants of the project, who have been working hard in instrument maintenance, taking and reducing the data, without which this work cannot be done. This work is supported by the National Key R\&D Program of China (Grant No. 2017YFA0402702), and the National Natural Science Foundation of China (NSFC, Grant Nos., 11873093, U2031202,11903083). MWISP is sponsored by National Key R\&D Program of China with grant 2017YFA0402701 and by CAS Key Research Program of Frontier Sciences with grant QYZDJ-SSW-SLH047.  We would like to express our thanks to the anonymous reviewer for constructive suggestions that enable improvement of the manuscript.  
   \end{acknowledgements}

   \appendix  

   \section{The full catalogue}
\small
%\begin{table}
\setlength{\tabcolsep}{3pt}
\begin{center}
    %\centering
    \begin{landscape}
        {\renewcommand{\arraystretch}{0.85}
            \begin{ThreePartTable}
                \begin{TableNotes}
                    \item[] Column Heads: (1): internal serial number; (2): name code; (3)\&(4): Galactic Coordinates; (5): association to some known catalogues (c.f. next foot note); (6): infall signature reported in the literature (c.f. next foot note); (7): distance; (8): central velocity; (9): velocity at the blue peak; (10): skewness parameter (see text); (11): expected Tmb of optically thick line; (12): peak Tmb of optically thin line; (13): excitation temperature; (14): H$_2$ column density; (15): pair from which the blue-profile is obtained.
                    \item[$\boldsymbol{\alpha}$] Ref. Codes: (1) IRAS Point Source Catalog; (2) New General Catalog (NGC); (3) \citeauthor{1999yCat.2094....0P} \citeyear{1999yCat.2094....0P}; (4) \citeauthor{1953ApJ...118..362S} \citeyear{1953ApJ...118..362S}; (5) \citeauthor{1996yCat.7007....0L} \citeyear{1996yCat.7007....0L}; (6) \citeauthor{2014ApJS..212....1A} \citeyear{2014ApJS..212....1A}; (7) \citeauthor{2006ApJ...639..227S} \citeyear{2006ApJ...639..227S}; (8) \citeauthor{2018MNRAS.476.3981Y} \citeyear{2018MNRAS.476.3981Y}; (9) \citeauthor{2003ApJS..149..375S} \citeyear{2003ApJS..149..375S}; (10) \citeauthor{2005A&A...442..949F} \citeyear{2005A&A...442..949F}; (11) \citeauthor{2007ApJ...663.1092K} \citeyear{2007ApJ...663.1092K}; (12) \citeauthor{2009MNRAS.392..170S} \citeyear{2009MNRAS.392..170S}; (13) \citeauthor{2011ApJ...740...40R} \citeyear{2011ApJ...740...40R}; (14) \citeauthor{2012A&A...538A.140K} \citeyear{2012A&A...538A.140K}; (15) \citeauthor{2013ApJ...776...29L} \citeyear{2013ApJ...776...29L}; (16) \citeauthor{2016MNRAS.456.2681Q} \citeyear{2016MNRAS.456.2681Q}; (17) \citeauthor{2016MNRAS.461.2288H} \citeyear{2016MNRAS.461.2288H}; (18) \citeauthor{2020RAA....20..115Y} \citeyear{2020RAA....20..115Y}; (19) \citeauthor{2021ApJ...910..112K} \citeyear{2021ApJ...910..112K}
                    \item[$\boldsymbol{\beta}$] For a few sources the distances are not given because the distance calculator does not output proper values.
                    \item[$\boldsymbol{\gamma}$] For a number of entries the values of N(H$_2$) are not given because the optical depths ($\tau_{thin}$) cannot be estimated due to the complex line profiles.
                \end{TableNotes}
                \begin{longtable}{rrrrllrrrrrrrrr}
                    \caption{The full catalog of the sources with CO blue-profiles.}\label{tab:tab_full}                                                                                                                                                                                                                                                                                                                                                                                                      \\
                    \toprule
                    \multicolumn{1}{c}{No}  & \multicolumn{1}{c}{Code} & \multicolumn{1}{c}{L}   & \multicolumn{1}{c}{B}   & \multicolumn{1}{c}{Association$^\alpha$} & \multicolumn{1}{c}{I.S.$^\alpha$} & \multicolumn{1}{c}{D$^\beta$} & \multicolumn{1}{c}{V$_c$}         & \multicolumn{1}{c}{V$_b$}         & \multicolumn{1}{c}{$\delta$V} & \multicolumn{1}{c}{T$_1$} & \multicolumn{1}{c}{T$_2$} & \multicolumn{1}{c}{T$_{ex}$} & \multicolumn{1}{c}{N$_{H2}${$^\gamma$}} & \multicolumn{1}{c}{Pair} \\
                    \multicolumn{1}{c}{}    & \multicolumn{1}{c}{}     & \multicolumn{1}{c}{}    & \multicolumn{1}{c}{}    & \multicolumn{1}{c}{}                     & \multicolumn{1}{c}{}              & \multicolumn{1}{c}{(kpc)}     & \multicolumn{1}{c}{(km s$^{-1}$)} & \multicolumn{1}{c}{(km s$^{-1}$)} & \multicolumn{1}{c}{}          & \multicolumn{1}{c}{(K)}   & \multicolumn{1}{c}{(K)}   & \multicolumn{1}{c}{(K)}      & \multicolumn{1}{c}{(cm$^{-2}$)}         & \multicolumn{1}{c}{}     \\
                    \multicolumn{1}{c}{(1)} & \multicolumn{1}{c}{(2)}  & \multicolumn{1}{c}{(3)} & \multicolumn{1}{c}{(4)} & \multicolumn{1}{c}{(5)}                  & \multicolumn{1}{c}{(6)}           & \multicolumn{1}{c}{(7)}       & \multicolumn{1}{c}{(8)}           & \multicolumn{1}{c}{(9)}           & \multicolumn{1}{c}{(10)}      & \multicolumn{1}{c}{(11)}  & \multicolumn{1}{c}{(12)}  & \multicolumn{1}{c}{(13)}     & \multicolumn{1}{c}{(14)}                & \multicolumn{1}{c}{(15)} \\
                    \midrule
                    \endfirsthead

                    \multicolumn{15}{c}{Table \ref{tab:tab_full}: Continued}                                                                                                                                                                                                                                                                                                                                                                                                                                  \\

                    \toprule
                    \multicolumn{1}{c}{(1)} & \multicolumn{1}{c}{(2)}  & \multicolumn{1}{c}{(3)} & \multicolumn{1}{c}{(4)} & \multicolumn{1}{c}{(5)}                  & \multicolumn{1}{c}{(6)}           & \multicolumn{1}{c}{(7)}       & \multicolumn{1}{c}{(8)}           & \multicolumn{1}{c}{(9)}           & \multicolumn{1}{c}{(10)}      & \multicolumn{1}{c}{(11)}  & \multicolumn{1}{c}{(12)}  & \multicolumn{1}{c}{(13)}     & \multicolumn{1}{c}{(14)}                & \multicolumn{1}{c}{(15)} \\
                    \midrule
                    \endhead

                    \bottomrule
                    \multicolumn{15}{c}{Continued on next page}                                                                                                                                                                                                                                                                                                                                                                                                                                               \\
                    \endfoot

                    \bottomrule
                    \insertTableNotes
                    \endlastfoot

                    1                       & 00137$-$0171$+$0069          & 1.3667                  & $-$1.7083                 & 5,  6                                    & \nodata                           & 1.34                          & 6.87                              & 6.43                              & $-$0.35                         & 5.2                       & 1.8                       & 10.8                         & 1.08E+21                                & 1                        \\
                    2                       & 00137$+$0168$-$0018          & 1.3750                  & 1.6833                  & 5,  6                                    & \nodata                           & 1.13                          & $-$1.82                             & $-$2.32                             & $-$0.21                         & 5.4                       & 1.6                       & 9.9                          & 1.75E+21                                & 1                        \\
                    3                       & 00140$-$0138$+$0081          & 1.4000                  & $-$1.3833                 & 3,  5,  6                                & \nodata                           & 1.37                          & 8.15                              & 7.67                              & $-$0.57                         & 6.6                       & 1.1                       & 12.2                         & 4.34E+20                                & 1                        \\
                    4                       & 00147$+$0052$+$0072          & 1.4750                  & 0.5250                  & 5,  6                                    & \nodata                           & 2.81                          & 7.22                              & 6.91                              & $-$0.27                         & 8.1                       & 3.0                       & 15.9                         & 1.91E+21                                & 1                        \\
                    5                       & 00148$-$0051$-$0111          & 1.4833                  & $-$0.5083                 & 3,  5,  6                                & \nodata                           & 2.80                          & $-$11.11                            & $-$12.14                            & $-$0.32                         & 5.7                       & 1.9                       & 11.3                         & 2.94E+21                                & 1                        \\
                    6                       & 00153$-$0177$+$0085          & 1.5333                  & $-$1.7667                 & 1,  5,  6                                & \nodata                           & 1.33                          & 8.48                              & 7.96                              & $-$0.41                         & 7.2                       & 2.2                       & 14.2                         & 1.48E+21                                & 1                        \\
                    7                       & 00158$-$0168$+$0084          & 1.5833                  & $-$1.6833                 & 5,  6                                    & \nodata                           & 1.34                          & 8.41                              & 7.40                              & $-$0.73                         & 7.0                       & 2.4                       & 15.7                         & 1.82E+21                                & 1                        \\
                    8                       & 00165$+$0060$+$0016          & 1.6500                  & 0.6000                  & 5,  6                                    & \nodata                           & 1.33                          & 1.57                              & 1.20                              & $-$0.33                         & 4.2                       & 1.5                       & 7.7                          & 7.94E+20                                & 1                        \\
                    9                       & 00166$-$0006$-$0347          & 1.6583                  & $-$0.0583                 & 5,  6,  7                                & \nodata                           & 3.74                          & $-$34.73                            & $-$35.72                            & $-$0.48                         & 3.5                       & 1.3                       & 7.6                          & 9.59E+21                                & 2                        \\
                    10                      & 00177$-$0067$+$0021          & 1.7750                  & $-$0.6750                 & 5,  6                                    & \nodata                           & 1.39                          & 2.08                              & 1.68                              & $-$0.51                         & 2.3                       & 0.8                       & 5.8                          & 2.65E+21                                & 2                        \\
                    11                      & 00187$+$0008$+$0076          & 1.8667                  & 0.0833                  & 5,  6,  7                                & \nodata                           & 2.81                          & 7.62                              & 7.11                              & $-$0.20                         & 7.8                       & 4.0                       & 11.5                         & 5.67E+21                                & 1                        \\
                    12                      & 00192$+$0205$+$0042          & 1.9167                  & 2.0500                  & 5,  6                                    & \nodata                           & 1.03                          & 4.22                              & 3.88                              & $-$0.39                         & 9.3                       & 2.6                       & 15.3                         & 1.19E+21                                & 1                        \\
                    13                      & 00198$+$0202$+$0043          & 1.9833                  & 2.0167                  & 5,  6                                    & \nodata                           & 1.04                          & 4.28                              & 3.83                              & $-$0.56                         & 7.9                       & 3.7                       & 14.6                         & 1.66E+21                                & 1                        \\
                    14                      & 00214$+$0190$+$0046          & 2.1417                  & 1.9000                  & 5,  6                                    & \nodata                           & 1.08                          & 4.60                              & 4.40                              & $-$0.24                         & 8.3                       & 2.4                       & 14.2                         & 1.02E+21                                & 1                        \\
                    15                      & 00230$-$0065$+$0005          & 2.3000                  & $-$0.6500                 & 5,  6                                    & \nodata                           & 1.39                          & 0.48                              & $-$1.23                             & $-$0.38                         & 8.2                       & 2.4                       & 13.7                         & 5.54E+21                                & 1                        \\
                    16                      & 00232$+$0000$+$0064          & 2.3167                  & 0.0000                  & 5,  6,  7                                & \nodata                           & 2.81                          & 6.42                              & 5.87                              & $-$0.34                         & 8.7                       & 5.2                       & 12.8                         & 5.33E+21                                & 1                        \\
                    17                      & 00233$+$0007$+$0158          & 2.3333                  & 0.0750                  & 5,  6,  7                                & \nodata                           & 2.84                          & 15.77                             & 14.98                             & $-$0.37                         & 2.9                       & 1.1                       & 6.5                          & 8.62E+21                                & 2                        \\
                    18                      & 00236$-$0160$+$0070          & 2.3583                  & $-$1.6000                 & 5,  6                                    & \nodata                           & 1.36                          & 6.97                              & 5.86                              & $-$0.71                         & 7.8                       & 3.7                       & 13.0                         & 3.14E+21                                & 1                        \\
                    19                      & 00256$-$0074$+$0056          & 2.5583                  & $-$0.7417                 & 3,  5,  6                                & \nodata                           & 1.41                          & 5.56                              & 5.09                              & $-$0.26                         & 12.7                      & 5.5                       & 17.2                         & 6.19E+21                                & 1                        \\
                    20                      & 00258$-$0040$+$0114          & 2.5833                  & $-$0.4000                 & 3,  5,  6                                & \nodata                           & 2.82                          & 11.36                             & 10.96                             & $-$0.20                         & 9.4                       & 3.5                       & 13.6                         & 3.89E+21                                & 1                        \\
                    21                      & 00261$+$0013$+$0113          & 2.6083                  & 0.1333                  & 1,  3,  5,  6,  7                        & \nodata                           & 2.82                          & 11.35                             & 10.60                             & $-$0.39                         & 8.8                       & 5.4                       & 12.1                         & 6.99E+21                                & 1                        \\
                    22                      & 00261$+$0189$+$0026          & 2.6083                  & 1.8917                  & 5,  6                                    & \nodata                           & 1.08                          & 2.56                              & 2.31                              & $-$0.24                         & 9.4                       & 2.3                       & 14.9                         & 1.29E+21                                & 1                        \\
                    23                      & 00262$+$0039$+$0025          & 2.6167                  & 0.3917                  & 1,  5,  6                                & \nodata                           & 2.78                          & 2.50                              & 1.80                              & $-$0.56                         & 5.8                       & 1.1                       & 10.0                         & 6.11E+20                                & 1                        \\
                    24                      & 00263$+$0012$+$0112          & 2.6333                  & 0.1250                  & 3,  5,  6,  7                            & \nodata                           & 2.82                          & 11.17                             & 10.48                             & $-$0.25                         & 10.9                      & 6.0                       & 13.3                         & 1.13E+22                                & 1                        \\
                    25                      & 00266$+$0018$+$0120          & 2.6583                  & 0.1833                  & 3,  5,  6,  7                            & \nodata                           & 2.83                          & 11.99                             & 10.65                             & $-$0.50                         & 9.9                       & 6.8                       & 16.6                         & 1.29E+22                                & 1                        \\
                    26                      & 00267$-$0051$-$0010          & 2.6667                  & $-$0.5083                 & 3,  5,  6                                & \nodata                           & 2.78                          & $-$0.95                             & $-$1.43                             & $-$0.23                         & 5.2                       & 1.8                       & 9.4                          & 1.73E+21                                & 1                        \\
                    27                      & 00274$-$0083$+$0171          & 2.7417                  & $-$0.8333                 & 5,  6                                    & \nodata                           & 1.42                          & 17.07                             & 16.15                             & $-$0.56                         & 5.5                       & 2.0                       & 10.1                         & 1.54E+21                                & 1                        \\
                    28                      & 00276$+$0031$-$0427          & 2.7583                  & 0.3083                  & 5,  6                                    & \nodata                           & 1.30                          & $-$42.75                            & $-$43.40                            & $-$0.39                         & 3.7                       & 1.1                       & 7.6                          & 8.40E+20                                & 1                        \\
                    29                      & 00282$-$0044$-$0384          & 2.8167                  & $-$0.4417                 & 5,  6,  7                                & \nodata                           & 3.74                          & $-$38.40                            & $-$39.52                            & $-$0.55                         & 3.6                       & 1.5                       & 6.7                          & 1.59E+21                                & 1                        \\
                    30                      & 00283$-$0022$+$0040          & 2.8333                  & $-$0.2167                 & 3,  5,  6,  7                            & \nodata                           & 2.81                          & 3.97                              & 3.75                              & $-$0.09                         & 10.0                      & 2.2                       & 13.2                         & 2.56E+21                                & 1                        \\
                    31                      & 00295$+$0110$-$0010          & 2.9500                  & 1.1000                  & 5,  6                                    & \nodata                           & 1.53                          & $-$0.99                             & $-$2.02                             & $-$0.60                         & 8.0                       & 3.7                       & 13.5                         & 3.56E+21                                & 1                        \\
                    32                      & 00297$+$0422$+$0197          & 2.9667                  & 4.2167                  & 6                                        & 17                                & 1.13                          & 19.73                             & 19.30                             & $-$0.57                         & 5.9                       & 2.3                       & 9.2                          & 8.68E+20                                & 1                        \\
                    33                      & 00297$+$0107$-$0011          & 2.9750                  & 1.0667                  & 5,  6                                    & \nodata                           & 1.53                          & $-$1.07                             & $-$1.52                             & $-$0.30                         & 9.3                       & 1.9                       & 13.7                         & 1.44E+21                                & 1                        \\
                    34                      & 00301$+$0137$+$0038          & 3.0083                  & 1.3667                  & 5,  6                                    & \nodata                           & 1.29                          & 3.81                              & 3.26                              & $-$0.34                         & 8.8                       & 2.0                       & 13.7                         & 1.67E+21                                & 1                        \\
                    35                      & 00302$+$0194$+$0035          & 3.0250                  & 1.9417                  & 5,  6                                    & \nodata                           & 0.99                          & 3.48                              & 3.09                              & $-$0.48                         & 5.8                       & 1.7                       & 12.3                         & 6.73E+20                                & 1                        \\
                    36                      & 00303$+$0097$-$0003          & 3.0333                  & 0.9667                  & 5,  6                                    & \nodata                           & 1.54                          & $-$0.34                             & $-$1.36                             & $-$0.44                         & 10.4                      & 5.0                       & 20.3                         & 7.61E+21                                & 1                        \\
                    37                      & 00307$+$0111$-$0011          & 3.0750                  & 1.1083                  & 5,  6                                    & \nodata                           & 1.31                          & $-$1.06                             & $-$1.83                             & $-$0.45                         & 7.0                       & 1.5                       & 10.5                         & 1.15E+21                                & 1                        \\
                    38                      & 00312$-$0097$-$0262          & 3.1167                  & $-$0.9750                 & 5,  6                                    & \nodata                           & 0.02                          & $-$26.19                            & $-$26.88                            & $-$0.31                         & 4.6                       & 1.3                       & 10.5                         & 1.37E+21                                & 1                        \\
                    39                      & 00315$+$0300$+$0030          & 3.1500                  & 3.0000                  & 6                                        & \nodata                           & 0.85                          & 3.01                              & 2.56                              & $-$0.44                         & 7.0                       & 1.4                       & 18.5                         & 7.84E+20                                & 1                        \\
                    40                      & 00315$-$0093$-$0267          & 3.1500                  & $-$0.9333                 & 5,  6                                    & \nodata                           & 4.49                          & $-$26.66                            & $-$27.29                            & $-$0.36                         & 4.8                       & 2.1                       & 8.9                          & 1.79E+21                                & 1                        \\
                    41                      & 00316$+$0149$+$0024          & 3.1583                  & 1.4917                  & 3,  5,  6                                & \nodata                           & 1.23                          & 2.39                              & 1.95                              & $-$0.33                         & 6.3                       & 1.3                       & 9.9                          & 7.35E+20                                & 1                        \\
                    42                      & 00317$-$0099$-$0259          & 3.1667                  & $-$0.9917                 & 5,  6                                    & \nodata                           & 0.16                          & $-$25.86                            & $-$26.80                            & $-$0.55                         & 5.0                       & 1.5                       & 12.4                         & 1.19E+21                                & 1                        \\
                    43                      & 00322$-$0017$+$0038          & 3.2250                  & $-$0.1667                 & 3,  5,  6                                & \nodata                           & 2.58                          & 3.81                              & 2.69                              & $-$0.32                         & 8.7                       & 2.1                       & 12.9                         & 3.63E+21                                & 1                        \\
                    44                      & 00325$+$0296$+$0029          & 3.2500                  & 2.9583                  & 6                                        & \nodata                           & 0.86                          & 2.87                              & 2.37                              & $-$0.48                         & 6.0                       & 1.1                       & 12.6                         & 5.09E+20                                & 1                        \\
                    45                      & 00330$+$0157$+$0033          & 3.3000                  & 1.5667                  & 3,  5,  6                                & \nodata                           & 1.24                          & 3.30                              & 2.40                              & $-$0.85                         & 6.3                       & 1.4                       & 13.2                         & 7.14E+20                                & 1                        \\
                    46                      & 00339$+$0171$+$0036          & 3.3917                  & 1.7083                  & 6                                        & \nodata                           & 1.20                          & 3.59                              & 3.09                              & $-$0.44                         & 5.8                       & 1.9                       & 10.4                         & 1.07E+21                                & 1                        \\
                    47                      & 00342$-$0002$+$0044          & 3.4250                  & $-$0.0250                 & 5,  6,  7                                & \nodata                           & 2.55                          & 4.43                              & 3.32                              & $-$0.54                         & 7.0                       & 4.3                       & 11.3                         & 5.30E+21                                & 1                        \\
                    48                      & 00347$+$0038$+$0188          & 3.4667                  & 0.3833                  & 5,  6                                    & \nodata                           & 2.95                          & 18.84                             & 18.28                             & $-$0.61                         & 4.3                       & 1.6                       & 8.8                          & 5.22E+21                                & 2                        \\
                    49                      & 00360$-$0083$+$0211          & 3.6000                  & $-$0.8333                 & 5,  6                                    & \nodata                           & 2.93                          & 21.11                             & 20.07                             & $-$0.37                         & 6.1                       & 1.6                       & 12.1                         & 2.18E+21                                & 1                        \\
                    50                      & 00360$-$0098$+$0096          & 3.6000                  & $-$0.9833                 & 5,  6                                    & \nodata                           & 1.54                          & 9.56                              & 8.88                              & $-$0.40                         & 10.2                      & 4.0                       & 17.6                         & 4.03E+21                                & 1                        \\
                    51                      & 00377$-$0013$+$0041          & 3.7667                  & $-$0.1333                 & 3,  5,  6                                & \nodata                           & 2.57                          & 4.13                              & 3.55                              & $-$0.26                         & 11.0                      & 5.5                       & 14.7                         & 7.94E+21                                & 1                        \\
                    52                      & 00379$-$0014$+$0040          & 3.7917                  & $-$0.1417                 & 3,  5,  6,  7                            & \nodata                           & 2.57                          & 4.02                              & 3.49                              & $-$0.26                         & 9.0                       & 5.5                       & 12.5                         & 7.39E+21                                & 1                        \\
                    53                      & 00379$-$0024$-$0303          & 3.7917                  & $-$0.2417                 & 5,  6                                    & \nodata                           & 3.76                          & $-$30.26                            & $-$30.60                            & $-$0.43                         & 4.1                       & 1.6                       & 7.1                          & 5.15E+21                                & 2                        \\
                    54                      & 00382$-$0103$+$0101          & 3.8250                  & $-$1.0333                 & 5,  6                                    & \nodata                           & 1.54                          & 10.14                             & 9.52                              & $-$0.39                         & 15.3                      & 6.4                       & 21.9                         & 7.31E+21                                & 1                        \\
                    55                      & 00383$-$0099$+$0106          & 3.8333                  & $-$0.9917                 & 5,  6,  7                                & \nodata                           & 1.54                          & 10.61                             & 10.25                             & $-$0.36                         & 7.0                       & 1.8                       & 11.0                         & 6.50E+21                                & 2                        \\
                    56                      & 00389$-$0021$-$0244          & 3.8917                  & $-$0.2083                 & 5,  6,  7                                & \nodata                           & 3.75                          & $-$24.45                            & $-$24.92                            & $-$0.28                         & 5.0                       & 1.2                       & 11.0                         & 9.43E+20                                & 1                        \\
                    57                      & 00397$+$0010$-$0387          & 3.9667                  & 0.1000                  & 3,  4,  5,  6                            & \nodata                           & 3.76                          & $-$38.73                            & $-$39.38                            & $-$0.27                         & 4.8                       & 1.2                       & 10.0                         & 1.25E+21                                & 1                        \\
                    58                      & 00397$+$0012$+$0659          & 3.9750                  & 0.1167                  & 3,  4,  5,  6                            & \nodata                           & 1.09                          & 65.85                             & 64.96                             & $-$0.69                         & 3.8                       & 1.1                       & 7.7                          & 6.54E+20                                & 1                        \\
                    59                      & 00400$+$0047$+$0133          & 4.0000                  & 0.4667                  & 4,  5,  6                                & \nodata                           & 2.94                          & 13.33                             & 12.71                             & $-$0.29                         & 10.6                      & 5.2                       & 16.6                         & 6.98E+21                                & 1                        \\
                    60                      & 00401$-$0084$+$0189          & 4.0083                  & $-$0.8417                 & 5,  6                                    & \nodata                           & 1.51                          & 18.89                             & 18.44                             & $-$0.14                         & 9.5                       & 3.7                       & 14.6                         & 6.71E+21                                & 1                        \\
                    61                      & 00403$-$0087$+$0189          & 4.0333                  & $-$0.8667                 & 5,  6                                    & \nodata                           & 1.51                          & 18.89                             & 17.81                             & $-$0.35                         & 9.0                       & 3.8                       & 13.1                         & 6.46E+21                                & 1                        \\
                    62                      & 00407$+$0041$+$0132          & 4.0750                  & 0.4083                  & 4,  5,  6,  7                            & \nodata                           & 2.94                          & 13.18                             & 12.77                             & $-$0.51                         & 4.8                       & 1.3                       & 9.0                          & 3.64E+21                                & 2                        \\
                    63                      & 00410$-$0012$+$0072          & 4.1000                  & $-$0.1250                 & 4,  5,  6                                & \nodata                           & 2.57                          & 7.16                              & 6.35                              & $-$0.54                         & 6.0                       & 2.1                       & 11.5                         & 1.57E+21                                & 1                        \\
                    64                      & 00412$-$0002$+$0063          & 4.1167                  & $-$0.0167                 & 3,  4,  5,  6,  7                        & \nodata                           & 2.93                          & 6.35                              & 5.70                              & $-$0.30                         & 9.0                       & 5.2                       & 13.0                         & 7.11E+21                                & 1                        \\
                    65                      & 00415$-$0192$+$0090          & 4.1500                  & $-$1.9250                 & 5,  6                                    & \nodata                           & 1.53                          & 8.99                              & 8.57                              & $-$0.39                         & 5.4                       & 1.5                       & 12.4                         & 7.60E+20                                & 1                        \\
                    66                      & 00417$-$0098$+$0111          & 4.1750                  & $-$0.9833                 & 5,  6                                    & \nodata                           & 1.54                          & 11.06                             & 10.70                             & $-$0.26                         & 15.6                      & 2.5                       & 20.7                         & 2.11E+21                                & 1                        \\
                    67                      & 00422$-$0032$+$0068          & 4.2250                  & $-$0.3167                 & 5,  6                                    & \nodata                           & 2.93                          & 6.80                              & 6.30                              & $-$0.32                         & 8.8                       & 4.1                       & 15.6                         & 3.72E+21                                & 1                        \\
                    68                      & 00434$-$0159$+$0078          & 4.3417                  & $-$1.5917                 & 3,  5,  6                                & \nodata                           & 1.54                          & 7.80                              & 7.07                              & $-$0.36                         & 9.6                       & 1.4                       & 19.0                         & 1.66E+21                                & 1                        \\
                    69                      & 00436$-$0029$-$0334          & 4.3583                  & $-$0.2917                 & 4,  5,  6                                & \nodata                           & 3.76                          & $-$33.43                            & $-$33.92                            & $-$0.35                         & 4.5                       & 1.8                       & 8.6                          & 1.28E+21                                & 1                        \\
                    70                      & 00443$+$0049$+$0141          & 4.4333                  & 0.4917                  & 4,  5,  6                                & \nodata                           & 2.94                          & 14.13                             & 13.52                             & $-$0.26                         & 8.8                       & 3.6                       & 11.9                         & 4.61E+21                                & 1                        \\
                    71                      & 00444$+$0040$-$0283          & 4.4417                  & 0.4000                  & 4,  5,  6                                & \nodata                           & 3.75                          & $-$28.28                            & $-$29.09                            & $-$0.35                         & 4.2                       & 1.7                       & 8.0                          & 1.90E+21                                & 1                        \\
                    72                      & 00444$-$0062$+$0130          & 4.4417                  & $-$0.6250                 & 5,  6                                    & \nodata                           & 1.54                          & 13.02                             & 11.98                             & $-$0.36                         & 10.2                      & 4.2                       & 15.9                         & 7.28E+21                                & 1                        \\
                    73                      & 00462$-$0012$+$0133          & 4.6167                  & $-$0.1167                 & 3,  4,  5,  6,  7                        & \nodata                           & 2.57                          & 13.29                             & 11.94                             & $-$0.43                         & 9.2                       & 3.4                       & 14.8                         & 5.86E+21                                & 1                        \\
                    74                      & 00507$-$0207$+$0040          & 5.0750                  & $-$2.0750                 & 5,  6                                    & \nodata                           & 1.21                          & 3.99                              & 3.65                              & $-$0.37                         & 3.8                       & 1.2                       & 8.5                          & 4.69E+20                                & 1                        \\
                    75                      & 00511$+$0227$+$0065          & 5.1083                  & 2.2750                  & 6                                        & \nodata                           & 1.16                          & 6.51                              & 6.10                              & $-$0.23                         & 5.1                       & 1.1                       & 8.7                          & 8.15E+20                                & 1                        \\
                    76                      & 00549$+$0078$+$0067          & 5.4917                  & 0.7833                  & 5,  6                                    & \nodata                           & 1.55                          & 6.69                              & 6.31                              & $-$0.38                         & 5.3                       & 1.4                       & 10.8                         & 6.26E+20                                & 1                        \\
                    77                      & 00553$+$0074$+$0068          & 5.5333                  & 0.7417                  & 5,  6                                    & \nodata                           & 1.55                          & 6.77                              & 6.46                              & $-$0.39                         & 4.8                       & 1.2                       & 10.4                         & 4.27E+20                                & 1                        \\
                    78                      & 00558$+$0055$+$0064          & 5.5833                  & 0.5500                  & 5,  6                                    & \nodata                           & 1.55                          & 6.40                              & 6.00                              & $-$0.50                         & 6.3                       & 2.3                       & 10.6                         & 9.18E+20                                & 1                        \\
                    79                      & 00558$+$0067$+$0067          & 5.5833                  & 0.6750                  & 5,  6                                    & \nodata                           & 1.55                          & 6.73                              & 6.09                              & $-$0.54                         & 5.5                       & 1.2                       & 11.1                         & 6.67E+20                                & 1                        \\
                    80                      & 00559$+$0049$+$0063          & 5.5917                  & 0.4917                  & 3,  5,  6                                & \nodata                           & 2.94                          & 6.32                              & 5.42                              & $-$0.84                         & 6.0                       & 1.1                       & 9.6                          & 4.93E+20                                & 1                        \\
                    81                      & 00559$+$0077$+$0067          & 5.5917                  & 0.7750                  & 5,  6                                    & \nodata                           & 1.55                          & 6.67                              & 6.27                              & $-$0.42                         & 5.8                       & 3.1                       & 11.7                         & 1.59E+21                                & 1                        \\
                    82                      & 00561$+$0075$+$0068          & 5.6083                  & 0.7500                  & 5,  6                                    & \nodata                           & 1.55                          & 6.79                              & 6.32                              & $-$0.50                         & 5.4                       & 2.7                       & 14.0                         & 1.29E+21                                & 1                        \\
                    83                      & 00562$+$0055$+$0063          & 5.6167                  & 0.5500                  & 5,  6                                    & \nodata                           & 1.55                          & 6.32                              & 5.76                              & $-$0.48                         & 6.9                       & 2.9                       & 12.9                         & 1.76E+21                                & 1                        \\
                    84                      & 00562$+$0070$+$0065          & 5.6167                  & 0.7000                  & 1,  5,  6                                & \nodata                           & 1.55                          & 6.51                              & 6.12                              & $-$0.35                         & 6.9                       & 2.0                       & 12.8                         & 1.05E+21                                & 1                        \\
                    85                      & 00566$+$0068$+$0066          & 5.6583                  & 0.6833                  & 5,  6                                    & \nodata                           & 1.55                          & 6.64                              & 6.01                              & $-$0.56                         & 5.4                       & 1.4                       & 11.6                         & 6.97E+20                                & 1                        \\
                    86                      & 00572$+$0052$+$0058          & 5.7167                  & 0.5250                  & 5,  6                                    & \nodata                           & 1.55                          & 5.77                              & 5.12                              & $-$0.37                         & 8.4                       & 1.7                       & 12.2                         & 1.47E+21                                & 1                        \\
                    87                      & 00574$+$0051$+$0060          & 5.7417                  & 0.5083                  & 5,  6                                    & \nodata                           & 1.55                          & 6.01                              & 5.41                              & $-$0.45                         & 9.0                       & 2.8                       & 14.0                         & 2.00E+21                                & 1                        \\
                    88                      & 00577$-$0027$+$0078          & 5.7667                  & $-$0.2667                 & 2,  3,  5,  6,  7                        & \nodata                           & 2.94                          & 7.82                              & 7.20                              & $-$0.23                         & 11.3                      & 4.8                       & 17.7                         & 8.30E+21                                & 1                        \\
                    89                      & 00577$-$0045$+$0076          & 5.7667                  & $-$0.4500                 & 2,  3,  5,  6                            & \nodata                           & 2.94                          & 7.64                              & 7.15                              & $-$0.30                         & 9.3                       & 4.9                       & 14.8                         & 4.93E+21                                & 1                        \\
                    90                      & 00582$-$0042$+$0082          & 5.8167                  & $-$0.4167                 & 2,  3,  5,  6                            & \nodata                           & 2.94                          & 8.23                              & 7.90                              & $-$0.20                         & 8.7                       & 1.7                       & 12.0                         & 1.04E+22                                & 2                        \\
                    91                      & 00582$-$0013$+$0068          & 5.8250                  & $-$0.1333                 & 2,  5,  6                                & \nodata                           & 2.56                          & 6.80                              & 6.41                              & $-$0.31                         & 7.2                       & 1.3                       & 10.0                         & 5.60E+21                                & 2                        \\
                    92                      & 00584$-$0046$+$0081          & 5.8417                  & $-$0.4583                 & 2,  3,  5,  6                            & \nodata                           & 2.94                          & 8.15                              & 7.59                              & $-$0.26                         & 18.7                      & 7.7                       & 20.9                         & 1.24E+22                                & 1                        \\
                    93                      & 00585$-$0035$+$0092          & 5.8500                  & $-$0.3500                 & 2,  3,  5,  6                            & \nodata                           & 2.94                          & 9.16                              & 8.91                              & $-$0.08                         & 20.1                      & 7.7                       & 22.8                         & 1.93E+22                                & 1                        \\
                    94                      & 00586$-$0006$+$0122          & 5.8583                  & $-$0.0583                 & 5,  6                                    & \nodata                           & 2.56                          & 12.22                             & 11.68                             & $-$0.19                         & 10.3                      & 3.1                       & 13.6                         & 4.77E+21                                & 1                        \\
                    95                      & 00587$-$0020$+$0092          & 5.8667                  & $-$0.2000                 & 2,  5,  6                                & \nodata                           & 2.94                          & 9.21                              & 8.71                              & $-$0.20                         & 9.5                       & 3.3                       & 13.2                         & 4.54E+21                                & 1                        \\
                    96                      & 00588$-$0047$+$0088          & 5.8833                  & $-$0.4667                 & 2,  3,  5,  6                            & 16                                & 2.94                          & 8.77                              & 8.26                              & $-$0.28                         & 8.7                       & 1.8                       & 12.0                         & 1.18E+22                                & 2                        \\
                    97                      & 00589$-$0017$-$0029          & 5.8917                  & $-$0.1750                 & 2,  5,  6                                & \nodata                           & 2.96                          & $-$2.91                             & $-$3.70                             & $-$0.46                         & 3.1                       & 0.7                       & 6.5                          & 5.50E+20                                & 1                        \\
                    98                      & 00589$-$0025$+$0093          & 5.8917                  & $-$0.2500                 & 2,  3,  5,  6,  7                        & \nodata                           & 2.94                          & 9.35                              & 8.60                              & $-$0.27                         & 12.8                      & 5.9                       & 17.3                         & 1.08E+22                                & 1                        \\
                    99                      & 00593$-$0040$+$0097          & 5.9333                  & $-$0.4000                 & 2,  3,  5,  6                            & \nodata                           & 2.94                          & 9.66                              & 9.22                              & $-$0.15                         & 23.7                      & 8.9                       & 28.6                         & 2.30E+22                                & 1                        \\
                    100                     & 00594$-$0133$+$0132          & 5.9417                  & $-$1.3333                 & 2,  3,  4,  5,  6                        & \nodata                           & 1.55                          & 13.23                             & 13.00                             & $-$0.14                         & 27.9                      & 8.0                       & 31.0                         & 1.21E+22                                & 1                        \\
                    101                     & 00595$-$0148$+$0124          & 5.9500                  & $-$1.4833                 & 2,  3,  4,  5,  6                        & \nodata                           & 1.55                          & 12.36                             & 11.82                             & $-$0.31                         & 17.2                      & 3.5                       & 19.5                         & 3.77E+21                                & 1                        \\
                    102                     & 00602$-$0032$+$0092          & 6.0250                  & $-$0.3167                 & 2,  3,  5,  6,  7                        & \nodata                           & 2.94                          & 9.20                              & 8.81                              & $-$0.22                         & 7.3                       & 1.6                       & 12.0                         & 1.03E+22                                & 2                        \\
                    103                     & 00619$-$0031$-$0333          & 6.1917                  & $-$0.3083                 & 5,  6,  7                                & \nodata                           & 3.80                          & $-$33.33                            & $-$33.95                            & $-$0.61                         & 3.6                       & 1.3                       & 8.3                          & 4.59E+21                                & 2                        \\
                    104                     & 00619$-$0053$+$0181          & 6.1917                  & $-$0.5333                 & 2,  3,  5,  6,  7                        & \nodata                           & 2.97                          & 18.13                             & 17.13                             & $-$0.22                         & 19.2                      & 4.8                       & 14.7                         & 1.35E+22                                & 1                        \\
                    105                     & 00656$+$0001$-$0214          & 6.5583                  & 0.0083                  & 3,  5,  6                                & \nodata                           & 3.78                          & $-$21.44                            & $-$22.28                            & $-$0.54                         & 4.6                       & 1.6                       & 7.1                          & 1.28E+21                                & 1                        \\
                    106                     & 00661$-$0040$+$0202          & 6.6083                  & $-$0.4000                 & 3,  4,  5,  6                            & \nodata                           & 3.50                          & 20.23                             & 19.79                             & $-$0.21                         & 13.9                      & 4.2                       & 18.7                         & 5.45E+21                                & 1                        \\
                    107                     & 00663$-$0051$-$0182          & 6.6333                  & $-$0.5083                 & 3,  4,  5,  6                            & \nodata                           & 3.80                          & $-$18.19                            & $-$18.68                            & $-$0.26                         & 5.0                       & 1.2                       & 9.5                          & 9.96E+20                                & 1                        \\
                    108                     & 00666$-$0059$-$0194          & 6.6583                  & $-$0.5917                 & 3,  4,  5,  6                            & \nodata                           & 3.70                          & $-$19.36                            & $-$19.88                            & $-$0.57                         & 5.1                       & 1.4                       & 8.8                          & 5.76E+20                                & 1                        \\
                    109                     & 00668$+$0076$+$0135          & 6.6833                  & 0.7583                  & 5,  6                                    & \nodata                           & 1.55                          & 13.52                             & 13.03                             & $-$0.32                         & 8.5                       & 2.8                       & 16.6                         & 2.38E+21                                & 1                        \\
                    110                     & 00672$-$0032$+$0086          & 6.7250                  & $-$0.3250                 & 2,  4,  5,  6,  7                        & \nodata                           & 2.95                          & 8.57                              & 7.53                              & $-$0.27                         & 15.3                      & 6.9                       & 24.8                         & 2.07E+22                                & 1                        \\
                    111                     & 00675$-$0041$+$0057          & 6.7500                  & $-$0.4083                 & 2,  3,  4,  5,  6                        & \nodata                           & 1.55                          & 5.73                              & 4.93                              & $-$0.28                         & 9.9                       & 2.5                       & 15.3                         & 3.78E+21                                & 1                        \\
                    112                     & 00685$+$0286$+$0048          & 6.8500                  & 2.8583                  & 6                                        & \nodata                           & 1.56                          & 4.80                              & 4.42                              & $-$0.50                         & 4.4                       & 1.5                       & 9.6                          & 5.31E+20                                & 1                        \\
                    113                     & 00692$-$0142$+$0231          & 6.9250                  & $-$1.4167                 & 3,  4,  6                                & \nodata                           & 1.36                          & 23.09                             & 22.66                             & $-$0.52                         & 4.1                       & 1.1                       & 8.6                          & 4.08E+20                                & 1                        \\
                    114                     & 00693$-$0061$-$0150          & 6.9333                  & $-$0.6083                 & 5,  6,  7                                & \nodata                           & 3.77                          & $-$15.02                            & $-$15.48                            & $-$0.37                         & 3.4                       & 1.7                       & 8.2                          & 1.01E+21                                & 1                        \\
                    115                     & 00698$-$0021$+$0191          & 6.9833                  & $-$0.2083                 & 2,  3,  5,  6                            & \nodata                           & 2.57                          & 19.10                             & 17.96                             & $-$0.39                         & 3.4                       & 1.2                       & 7.7                          & 1.28E+22                                & 2                        \\
                    116                     & 00722$+$0157$+$0067          & 7.2167                  & 1.5667                  & 5,  6                                    & \nodata                           & 1.24                          & 6.67                              & 6.15                              & $-$0.36                         & 5.9                       & 1.4                       & 11.0                         & 9.31E+20                                & 1                        \\
                    117                     & 00734$+$0007$+$0168          & 7.3417                  & 0.0750                  & 5,  6,  7                                & \nodata                           & 3.42                          & 16.84                             & 15.69                             & $-$0.33                         & 12.9                      & 5.9                       & 19.2                         & 1.36E+22                                & 1                        \\
                    118                     & 00734$-$0224$+$0084          & 7.3417                  & $-$2.2417                 & 3,  4,  5,  6                            & \nodata                           & 1.25                          & 8.42                              & 7.77                              & $-$0.83                         & 6.0                       & 1.1                       & 13.8                         & 4.03E+20                                & 1                        \\
                    119                     & 00756$+$0086$+$0145          & 7.5583                  & 0.8583                  & 5,  6                                    & \nodata                           & 1.25                          & 14.54                             & 13.88                             & $-$0.43                         & 7.2                       & 2.1                       & 11.3                         & 1.59E+21                                & 1                        \\
                    120                     & 00756$-$0025$+$0269          & 7.5583                  & $-$0.2500                 & 5,  6                                    & \nodata                           & 2.56                          & 26.86                             & 26.33                             & $-$0.51                         & 3.8                       & 1.3                       & 7.4                          & 6.37E+20                                & 1                        \\
                    121                     & 00783$-$0060$-$0062          & 7.8333                  & $-$0.6000                 & 6,  7                                    & \nodata                           & 1.55                          & $-$6.23                             & $-$7.00                             & $-$0.42                         & 4.1                       & 1.9                       & 8.4                          & 1.69E+21                                & 1                        \\
                    122                     & 00810$-$0177$+$0098          & 8.1000                  & $-$1.7667                 & 6                                        & \nodata                           & 1.25                          & 9.82                              & 9.64                              & $-$0.14                         & 9.6                       & 2.2                       & 17.0                         & 1.51E+21                                & 1                        \\
                    123                     & 00896$+$0037$+$0212          & 8.9583                  & 0.3667                  & 5,  6                                    & \nodata                           & 2.84                          & 21.24                             & 20.38                             & $-$0.43                         & 6.4                       & 2.9                       & 4.8                          & \nodata                                 & 1                        \\
                    124                     & 00901$+$0034$+$0218          & 9.0083                  & 0.3417                  & 5,  6,  7                                & \nodata                           & 2.83                          & 21.84                             & 20.68                             & $-$0.55                         & 6.0                       & 3.1                       & 12.1                         & 3.44E+21                                & 1                        \\
                    125                     & 00913$+$0027$+$0216          & 9.1333                  & 0.2750                  & 5,  6                                    & \nodata                           & 2.83                          & 21.59                             & 21.18                             & $-$0.24                         & 8.9                       & 2.2                       & 12.1                         & 1.85E+21                                & 1                        \\
                    126                     & 00921$+$0010$+$0325          & 9.2083                  & 0.1000                  & 3,  4,  5,  6                            & \nodata                           & 2.85                          & 32.49                             & 32.25                             & $-$0.11                         & 9.8                       & 4.0                       & 15.6                         & 5.29E+21                                & 1                        \\
                    127                     & 00921$+$0039$+$0217          & 9.2083                  & 0.3917                  & 5,  6                                    & \nodata                           & 2.82                          & 21.75                             & 20.90                             & $-$0.35                         & 6.5                       & 1.1                       & 10.4                         & 1.14E+21                                & 1                        \\
                    128                     & 00946$+$0161$+$0136          & 9.4583                  & 1.6083                  & 5,  6                                    & \nodata                           & 1.25                          & 13.64                             & 13.41                             & $-$0.30                         & 3.7                       & 1.0                       & 7.9                          & 2.79E+21                                & 2                        \\
                    129                     & 00958$+$0162$+$0117          & 9.5833                  & 1.6250                  & 5,  6                                    & \nodata                           & 1.25                          & 11.74                             & 11.43                             & $-$0.32                         & 5.5                       & 1.8                       & 9.8                          & 8.27E+20                                & 1                        \\
                    130                     & 00959$+$0220$+$0156          & 9.5917                  & 2.2000                  & 5,  6                                    & \nodata                           & 1.24                          & 15.61                             & 15.16                             & $-$0.38                         & 8.4                       & 3.8                       & 13.9                         & 2.56E+21                                & 1                        \\
                    131                     & 00967$+$0156$+$0125          & 9.6750                  & 1.5583                  & 5,  6                                    & \nodata                           & 1.25                          & 12.54                             & 11.72                             & $-$0.35                         & 6.4                       & 1.2                       & 12.2                         & 1.33E+21                                & 1                        \\
                    132                     & 00970$+$0126$+$0130          & 9.7000                  & 1.2583                  & 5,  6                                    & \nodata                           & 1.25                          & 12.99                             & 12.44                             & $-$0.34                         & 7.0                       & 1.3                       & 11.3                         & 9.48E+20                                & 1                        \\
                    133                     & 01027$+$0327$+$0055          & 10.2667                 & 3.2667                  & 5,  6                                    & \nodata                           & 1.25                          & 5.45                              & 4.73                              & $-$0.61                         & 6.7                       & 2.0                       & 11.0                         & 1.16E+21                                & 1                        \\
                    134                     & 01070$-$0288$+$0110          & 10.7000                 & $-$2.8833                 & 5,  6                                    & \nodata                           & 1.25                          & 10.97                             & 10.46                             & $-$0.36                         & 9.9                       & 3.7                       & 15.0                         & 2.90E+21                                & 1                        \\
                    135                     & 01083$-$0261$+$0126          & 10.8333                 & $-$2.6083                 & 3,  5,  6                                & \nodata                           & 1.25                          & 12.61                             & 11.59                             & $-$0.42                         & 18.5                      & 10.4                      & 27.6                         & 2.27E+22                                & 1                        \\
                    136                     & 01083$-$0264$+$0122          & 10.8333                 & $-$2.6417                 & 3,  5,  6                                & \nodata                           & 1.25                          & 12.17                             & 11.28                             & $-$0.42                         & 13.8                      & 6.8                       & 21.8                         & 1.05E+22                                & 1                        \\
                    137                     & 01087$-$0269$+$0119          & 10.8750                 & $-$2.6917                 & 3,  5,  6                                & \nodata                           & 1.25                          & 11.95                             & 11.49                             & $-$0.29                         & 13.8                      & 4.4                       & 24.2                         & 4.91E+21                                & 1                        \\
                    138                     & 01088$+$0177$+$0075          & 10.8833                 & 1.7667                  & 5,  6                                    & \nodata                           & 1.24                          & 7.47                              & 6.95                              & $-$0.61                         & 3.6                       & 1.4                       & 7.5                          & 5.97E+20                                & 1                        \\
                    139                     & 01103$-$0085$+$0266          & 11.0333                 & $-$0.8500                 & 5,  6,  7                                & \nodata                           & 2.81                          & 26.58                             & 26.20                             & $-$0.26                         & 4.9                       & 1.0                       & 11.0                         & 5.15E+21                                & 2                        \\
                    140                     & 01111$+$0342$+$0067          & 11.1083                 & 3.4250                  & 5,  6                                    & \nodata                           & 1.25                          & 6.67                              & 6.11                              & $-$0.72                         & 6.6                       & 3.3                       & 11.7                         & 1.35E+21                                & 1                        \\
                    141                     & 01113$-$0092$+$0254          & 11.1333                 & $-$0.9250                 & 5,  6                                    & \nodata                           & 1.21                          & 25.42                             & 25.07                             & $-$0.19                         & 11.9                      & 4.7                       & 15.3                         & 5.40E+21                                & 1                        \\
                    142                     & 01116$-$0014$+$1503          & 11.1583                 & $-$0.1417                 & 1,  5,  6,  7                            & \nodata                           & 8.00                          & 150.31                            & 149.51                            & $-$0.29                         & 7.9                       & 2.4                       & 15.3                         & 3.59E+21                                & 1                        \\
                    143                     & 01117$-$0023$+$0310          & 11.1750                 & $-$0.2333                 & 5,  6,  7                                & \nodata                           & 2.87                          & 30.99                             & 30.18                             & $-$0.35                         & 8.3                       & 3.4                       & 12.4                         & 4.21E+21                                & 1                        \\
                    144                     & 01119$-$0175$+$0113          & 11.1917                 & $-$1.7500                 & 5,  6                                    & \nodata                           & 1.25                          & 11.27                             & 10.71                             & $-$0.47                         & 6.8                       & 1.6                       & 12.1                         & 9.24E+20                                & 1                        \\
                    145                     & 01122$-$0010$+$0319          & 11.2167                 & $-$0.1000                 & 5,  6,  7                                & \nodata                           & 2.43                          & 31.88                             & 31.00                             & $-$0.43                         & 4.4                       & 2.2                       & 9.0                          & 1.77E+22                                & 2                        \\
                    146                     & 01128$+$0127$+$0061          & 11.2833                 & 1.2667                  & 5,  6                                    & \nodata                           & 1.25                          & 6.06                              & 5.63                              & $-$0.39                         & 7.7                       & 3.7                       & 16.8                         & 2.33E+21                                & 1                        \\
                    147                     & 01129$+$0122$+$0059          & 11.2917                 & 1.2167                  & 5,  6                                    & \nodata                           & 1.25                          & 5.91                              & 5.51                              & $-$0.37                         & 6.0                       & 3.3                       & 13.8                         & 1.99E+21                                & 1                        \\
                    148                     & 01135$+$0210$+$0232          & 11.3500                 & 2.1000                  & 5,  6                                    & \nodata                           & 1.12                          & 23.16                             & 22.35                             & $-$0.59                         & 7.4                       & 1.3                       & 16.2                         & 9.18E+20                                & 1                        \\
                    149                     & 01138$-$0174$+$0141          & 11.3833                 & $-$1.7417                 & 2,  3,  5,  6                            & \nodata                           & 1.25                          & 14.13                             & 13.48                             & $-$0.40                         & 22.5                      & 10.0                      & 32.8                         & 1.57E+22                                & 1                        \\
                    150                     & 01142$-$0168$+$0093          & 11.4167                 & $-$1.6833                 & 1,  2,  3,  5,  6                        & \nodata                           & 1.25                          & 9.27                              & 8.03                              & $-$0.44                         & 27.4                      & 16.5                      & 35.8                         & 5.57E+22                                & 1                        \\
                    151                     & 01161$-$0222$+$0125          & 11.6083                 & $-$2.2250                 & 5,  6                                    & \nodata                           & 1.25                          & 12.54                             & 11.57                             & $-$0.45                         & 9.2                       & 2.4                       & 17.3                         & 2.89E+21                                & 1                        \\
                    152                     & 01163$-$0100$+$0269          & 11.6333                 & $-$1.0000                 & 5,  6,  7                                & \nodata                           & 1.21                          & 26.87                             & 26.36                             & $-$0.24                         & 5.5                       & 0.9                       & 10.8                         & 6.36E+21                                & 2                        \\
                    153                     & 01178$+$0092$+$0252          & 11.7833                 & 0.9250                  & 5,  6                                    & \nodata                           & 1.17                          & 25.24                             & 24.67                             & $-$0.24                         & 10.2                      & 3.5                       & 15.0                         & 4.63E+21                                & 1                        \\
                    154                     & 01184$+$0077$+$0255          & 11.8417                 & 0.7750                  & 3,  5,  6                                & \nodata                           & 1.17                          & 25.52                             & 24.70                             & $-$0.35                         & 19.8                      & 8.7                       & 29.2                         & 1.79E+22                                & 1                        \\
                    155                     & 01234$-$0009$+$0338          & 12.3417                 & $-$0.0917                 & 4,  5,  6,  7                            & \nodata                           & 2.38                          & 33.83                             & 32.93                             & $-$0.35                         & 11.0                      & 4.8                       & 15.7                         & 7.51E+21                                & 1                        \\
                    156                     & 01235$+$0043$+$0021          & 12.3500                 & 0.4333                  & 3,  5,  6,  7                            & \nodata                           & 1.25                          & 2.12                              & 0.82                              & $-$1.48                         & 4.5                       & 1.4                       & 9.9                          & 5.76E+20                                & 1                        \\
                    157                     & 01236$+$0055$+$0189          & 12.3583                 & 0.5500                  & 3,  5,  6,  7                            & \nodata                           & 1.23                          & 18.95                             & 18.21                             & $-$0.46                         & 14.6                      & 5.4                       & 19.1                         & 5.63E+21                                & 1                        \\
                    158                     & 01243$+$0091$+$0187          & 12.4333                 & 0.9083                  & 5,  6                                    & \nodata                           & 1.22                          & 18.72                             & 17.64                             & $-$0.47                         & 7.8                       & 4.0                       & 12.2                         & 5.18E+21                                & 1                        \\
                    159                     & 01257$+$0054$+$0101          & 12.5750                 & 0.5417                  & 5,  6                                    & \nodata                           & 1.25                          & 10.11                             & 9.58                              & $-$0.49                         & 6.9                       & 1.4                       & 12.5                         & 6.99E+20                                & 1                        \\
                    160                     & 01266$-$0003$+$0216          & 12.6583                 & $-$0.0333                 & 3,  4,  5,  6,  7                        & \nodata                           & 2.36                          & 21.59                             & 20.55                             & $-$0.30                         & 7.6                       & 3.1                       & 14.5                         & 5.72E+21                                & 1                        \\
                    161                     & 01267$-$0007$+$0215          & 12.6750                 & $-$0.0750                 & 3,  4,  5,  6,  7                        & \nodata                           & 2.36                          & 21.49                             & 20.38                             & $-$0.33                         & 7.4                       & 2.7                       & 12.3                         & 4.59E+21                                & 1                        \\
                    162                     & 01272$+$0069$+$0173          & 12.7250                 & 0.6917                  & 1,  3,  5,  6                            & 17                                & 1.24                          & 17.32                             & 16.57                             & $-$0.51                         & 17.5                      & 9.9                       & 22.6                         & 1.23E+22                                & 1                        \\
                    163                     & 01274$-$0024$+$0359          & 12.7417                 & $-$0.2417                 & 3,  4,  5,  6                            & 16                                & 2.95                          & 35.86                             & 33.98                             & $-$0.37                         & 23.9                      & 11.4                      & 33.3                         & 5.89E+22                                & 1                        \\
                    164                     & 01274$-$0058$+$0062          & 12.7417                 & $-$0.5833                 & 4,  5,  6,  7                            & \nodata                           & 1.25                          & 6.22                              & 5.37                              & $-$0.58                         & 8.3                       & 4.7                       & 14.1                         & 4.15E+21                                & 1                        \\
                    165                     & 01275$-$0021$+$0351          & 12.7500                 & $-$0.2083                 & 3,  4,  5,  6,  7                        & 17                                & 2.94                          & 35.08                             & 34.52                             & $-$0.14                         & 27.8                      & 13.8                      & 33.7                         & 5.74E+22                                & 1                        \\
                    166                     & 01277$+$0039$+$0183          & 12.7667                 & 0.3917                  & 3,  4,  5,  6,  7                        & \nodata                           & 1.24                          & 18.25                             & 17.75                             & $-$0.17                         & 18.1                      & 6.3                       & 25.2                         & 1.39E+22                                & 1                        \\
                    167                     & 01277$-$0019$+$0356          & 12.7750                 & $-$0.1917                 & 3,  4,  5,  6,  7                        & 17                                & 2.95                          & 35.56                             & 34.36                             & $-$0.28                         & 29.6                      & 15.0                      & 37.2                         & 7.47E+22                                & 1                        \\
                    168                     & 01281$+$0050$+$0188          & 12.8083                 & 0.5000                  & 3,  5,  6                                & \nodata                           & 1.23                          & 18.76                             & 18.22                             & $-$0.17                         & 22.3                      & 7.6                       & 35.9                         & 2.27E+22                                & 1                        \\
                    169                     & 01282$+$0034$+$0186          & 12.8167                 & 0.3417                  & 3,  4,  5,  6,  7                        & \nodata                           & 1.24                          & 18.63                             & 18.06                             & $-$0.41                         & 7.6                       & 2.9                       & 12.3                         & 1.63E+22                                & 2                        \\
                    170                     & 01282$-$0019$+$0349          & 12.8167                 & $-$0.1917                 & 3,  4,  5,  6,  7                        & 17                                & 2.94                          & 34.87                             & 34.42                             & $-$0.10                         & 21.4                      & 6.1                       & 31.7                         & 1.84E+23                                & 2                        \\
                    171                     & 01283$+$0031$+$0190          & 12.8333                 & 0.3083                  & 5,  6,  7                                & \nodata                           & 1.23                          & 19.03                             & 18.53                             & $-$0.36                         & 5.7                       & 1.8                       & 9.9                          & 9.18E+21                                & 2                        \\
                    172                     & 01287$+$0054$+$0184          & 12.8750                 & 0.5417                  & 3,  5,  6,  7                            & \nodata                           & 1.23                          & 18.42                             & 17.77                             & $-$0.29                         & 18.0                      & 10.5                      & 28.0                         & 2.17E+22                                & 1                        \\
                    173                     & 01288$-$0024$+$0355          & 12.8833                 & $-$0.2417                 & 3,  4,  5,  6,  7                        & \nodata                           & 2.95                          & 35.50                             & 33.03                             & $-$0.37                         & 18.3                      & 10.3                      & 32.2                         & 6.53E+22                                & 1                        \\
                    174                     & 01289$+$0058$+$0187          & 12.8917                 & 0.5833                  & 3,  5,  6,  7                            & \nodata                           & 1.23                          & 18.73                             & 18.16                             & $-$0.28                         & 16.8                      & 5.0                       & 26.2                         & 7.55E+21                                & 1                        \\
                    175                     & 01294$-$0025$+$0349          & 12.9417                 & $-$0.2500                 & 3,  4,  5,  6,  7                        & 17                                & 2.94                          & 34.91                             & 33.19                             & $-$0.27                         & 14.9                      & 7.8                       & 23.5                         & 3.90E+22                                & 1                        \\
                    176                     & 01300$-$0022$+$0360          & 13.0000                 & $-$0.2250                 & 4,  5,  6,  7                            & \nodata                           & 2.95                          & 36.03                             & 34.44                             & $-$0.43                         & 5.4                       & 2.3                       & 9.4                          & 3.26E+22                                & 2                        \\
                    177                     & 01302$-$0015$+$0364          & 13.0167                 & $-$0.1500                 & 1,  4,  5,  6,  7                        & \nodata                           & 2.95                          & 36.36                             & 34.94                             & $-$0.47                         & 7.0                       & 2.1                       & 11.7                         & 2.39E+22                                & 2                        \\
                    178                     & 01306$+$0075$+$0251          & 13.0583                 & 0.7500                  & 5,  6                                    & \nodata                           & 1.18                          & 25.11                             & 24.33                             & $-$0.36                         & 7.9                       & 3.8                       & 13.9                         & 4.59E+21                                & 1                        \\
                    179                     & 01320$+$0019$+$0536          & 13.2000                 & 0.1917                  & 5,  6                                    & \nodata                           & 3.04                          & 53.59                             & 52.39                             & $-$0.38                         & 8.5                       & 5.0                       & 13.3                         & 9.90E+21                                & 1                        \\
                    180                     & 01379$-$0024$+$0381          & 13.7917                 & $-$0.2417                 & 1,  4,  6,  7                            & \nodata                           & 2.96                          & 38.10                             & 36.64                             & $-$0.40                         & 14.4                      & 7.4                       & 19.0                         & 1.93E+22                                & 1                        \\
                    181                     & 01382$-$0029$+$0392          & 13.8250                 & $-$0.2917                 & 3,  4,  6,  7                            & \nodata                           & 2.97                          & 39.21                             & 38.53                             & $-$0.25                         & 13.6                      & 5.0                       & 17.9                         & 8.61E+21                                & 1                        \\
                    182                     & 01398$+$0110$+$0212          & 13.9833                 & 1.1000                  & 6                                        & \nodata                           & 1.18                          & 21.17                             & 20.40                             & $-$0.42                         & 7.7                       & 3.1                       & 14.1                         & 3.07E+21                                & 1                        \\
                    183                     & 01399$-$0016$+$0396          & 13.9917                 & $-$0.1583                 & 3,  4,  6                                & 17                                & 2.99                          & 39.63                             & 39.00                             & $-$0.22                         & 12.4                      & 3.6                       & 17.2                         & 4.60E+22                                & 2                        \\
                    184                     & 01400$-$0013$+$0395          & 14.0000                 & $-$0.1333                 & 1,  3,  4,  6                            & \nodata                           & 2.27                          & 39.53                             & 39.29                             & $-$0.09                         & 10.6                      & 3.2                       & 14.1                         & 3.69E+22                                & 2                        \\
                    185                     & 01402$-$0019$+$0398          & 14.0250                 & $-$0.1917                 & 3,  4,  6                                & \nodata                           & 2.99                          & 39.84                             & 38.99                             & $-$0.27                         & 20.6                      & 11.0                      & 25.4                         & 3.20E+22                                & 1                        \\
                    186                     & 01410$-$0010$+$0368          & 14.1000                 & $-$0.1000                 & 3,  4,  6,  7                            & \nodata                           & 2.28                          & 36.82                             & 35.93                             & $-$0.25                         & 15.8                      & 5.4                       & 18.6                         & 1.26E+22                                & 1                        \\
                    187                     & 01411$-$0016$+$0390          & 14.1083                 & $-$0.1583                 & 3,  4,  6,  7                            & 17                                & 2.97                          & 39.04                             & 37.25                             & $-$0.70                         & 11.7                      & 7.6                       & 16.7                         & 1.43E+22                                & 1                        \\
                    188                     & 01414$-$0043$+$0202          & 14.1417                 & $-$0.4333                 & 4,  6,  7                                & \nodata                           & 1.84                          & 20.20                             & 18.30                             & $-$0.33                         & 18.1                      & 4.6                       & 24.8                         & 1.90E+22                                & 1                        \\
                    189                     & 01416$-$0015$+$0392          & 14.1583                 & $-$0.1500                 & 3,  4,  6,  7                            & \nodata                           & 2.98                          & 39.23                             & 35.74                             & $-$1.10                         & 11.4                      & 5.1                       & 16.0                         & 1.02E+22                                & 1                        \\
                    190                     & 01417$+$0274$+$0169          & 14.1667                 & 2.7417                  & 6                                        & \nodata                           & 1.12                          & 16.90                             & 16.46                             & $-$0.42                         & 5.0                       & 1.5                       & 9.2                          & 7.04E+20                                & 1                        \\
                    191                     & 01421$-$0054$+$0213          & 14.2083                 & $-$0.5417                 & 1,  3,  4,  5,  6,  7                    & 16                                & 1.85                          & 21.25                             & 20.32                             & $-$0.32                         & 8.9                       & 1.5                       & 12.9                         & 1.68E+22                                & 2                        \\
                    192                     & 01422$-$0052$+$0197          & 14.2250                 & $-$0.5250                 & 1,  3,  4,  5,  6,  7                    & 16                                & 1.85                          & 19.72                             & 18.90                             & $-$0.49                         & 13.8                      & 3.2                       & 17.7                         & 2.40E+22                                & 2                        \\
                    193                     & 01424$-$0050$+$0195          & 14.2417                 & $-$0.5000                 & 4,  5,  6,  7                            & \nodata                           & 1.84                          & 19.51                             & 18.39                             & $-$0.41                         & 19.3                      & 12.8                      & 26.0                         & 3.40E+22                                & 1                        \\
                    194                     & 01425$-$0017$+$0379          & 14.2500                 & $-$0.1750                 & 2,  3,  4,  6,  7                        & \nodata                           & 2.96                          & 37.94                             & 37.43                             & $-$0.11                         & 18.0                      & 8.5                       & 21.0                         & 3.14E+22                                & 1                        \\
                    195                     & 01427$+$0392$+$0301          & 14.2667                 & 3.9250                  & 6                                        & \nodata                           & 2.12                          & 30.07                             & 29.47                             & $-$0.39                         & 4.4                       & 1.8                       & 10.7                         & 1.28E+21                                & 1                        \\
                    196                     & 01428$-$0058$+$0205          & 14.2833                 & $-$0.5833                 & 3,  5,  6,  7                            & \nodata                           & 1.84                          & 20.48                             & 19.31                             & $-$0.36                         & 16.0                      & 9.2                       & 21.1                         & 2.46E+22                                & 1                        \\
                    197                     & 01429$-$0061$+$0210          & 14.2917                 & $-$0.6083                 & 3,  5,  6,  7                            & \nodata                           & 1.84                          & 21.04                             & 19.18                             & $-$0.58                         & 14.9                      & 8.4                       & 20.2                         & 2.11E+22                                & 1                        \\
                    198                     & 01432$+$0268$+$0156          & 14.3167                 & 2.6833                  & 6                                        & \nodata                           & 0.32                          & 15.56                             & 14.76                             & $-$0.41                         & 7.2                       & 2.1                       & 13.2                         & 2.04E+21                                & 1                        \\
                    199                     & 01432$+$0396$+$0300          & 14.3250                 & 3.9583                  & 6                                        & \nodata                           & 2.14                          & 30.04                             & 29.59                             & $-$0.32                         & 5.4                       & 2.4                       & 9.8                          & 1.71E+21                                & 1                        \\
                    200                     & 01444$+$0395$+$0298          & 14.4417                 & 3.9500                  & 1,  6                                    & \nodata                           & 2.15                          & 29.79                             & 29.04                             & $-$0.56                         & 6.5                       & 3.5                       & 16.3                         & 2.73E+21                                & 1                        \\
                    201                     & 01450$-$0003$+$0397          & 14.5000                 & $-$0.0333                 & 3,  4,  6                                & \nodata                           & 2.25                          & 39.68                             & 37.83                             & $-$0.37                         & 10.5                      & 4.9                       & 18.9                         & 1.57E+22                                & 1                        \\
                    202                     & 01458$+$0406$+$0289          & 14.5833                 & 4.0583                  & 6                                        & \nodata                           & 2.21                          & 28.92                             & 28.42                             & $-$0.63                         & 4.3                       & 1.0                       & 9.5                          & 3.49E+20                                & 1                        \\
                    203                     & 01468$-$0090$+$0191          & 14.6833                 & $-$0.9000                 & 4,  5,  6,  7                            & \nodata                           & 1.35                          & 19.09                             & 18.74                             & $-$0.25                         & 6.7                       & 1.3                       & 12.3                         & 6.36E+21                                & 2                        \\
                    204                     & 01469$+$0152$+$0116          & 14.6917                 & 1.5250                  & 6                                        & \nodata                           & 1.24                          & 11.59                             & 11.24                             & $-$0.30                         & 6.0                       & 1.4                       & 10.9                         & 7.44E+20                                & 1                        \\
                    205                     & 01473$+$0405$+$0289          & 14.7333                 & 4.0500                  & 6                                        & \nodata                           & 2.22                          & 28.89                             & 28.42                             & $-$0.46                         & 7.9                       & 3.6                       & 12.4                         & 2.06E+21                                & 1                        \\
                    206                     & 01473$+$0411$+$0290          & 14.7333                 & 4.1083                  & 6                                        & \nodata                           & 2.21                          & 29.05                             & 28.64                             & $-$0.31                         & 7.3                       & 2.4                       & 14.6                         & 1.64E+21                                & 1                        \\
                    207                     & 01476$-$0093$+$0187          & 14.7583                 & $-$0.9333                 & 4,  5,  6,  7                            & \nodata                           & 1.34                          & 18.69                             & 18.01                             & $-$0.29                         & 13.2                      & 4.0                       & 18.5                         & 5.67E+21                                & 1                        \\
                    208                     & 01494$+$0187$+$0279          & 14.9417                 & 1.8750                  & 6                                        & \nodata                           & 2.28                          & 27.94                             & 27.47                             & $-$0.29                         & 5.8                       & 1.0                       & 11.2                         & 7.40E+20                                & 1                        \\
                    209                     & 01497$-$0064$+$0206          & 14.9667                 & $-$0.6417                 & 2,  3,  4,  5,  6                        & \nodata                           & 1.84                          & 20.64                             & 18.86                             & $-$0.40                         & 38.3                      & 12.0                      & 49.6                         & 6.83E+22                                & 1                        \\
                    210                     & 01520$+$0335$+$0130          & 15.2000                 & 3.3500                  & 4,  6                                    & \nodata                           & 0.31                          & 13.03                             & 12.41                             & $-$0.86                         & 5.4                       & 1.3                       & 10.2                         & 4.10E+20                                & 1                        \\
                    211                     & 01528$+$0178$+$0237          & 15.2833                 & 1.7833                  & 3,  6                                    & \nodata                           & 1.17                          & 23.65                             & 23.17                             & $-$0.28                         & 5.8                       & 2.0                       & 9.3                          & 1.64E+21                                & 1                        \\
                    212                     & 01534$+$0004$+$0316          & 15.3417                 & 0.0417                  & 6,  7                                    & \nodata                           & 2.21                          & 31.59                             & 31.19                             & $-$0.30                         & 3.0                       & 0.7                       & 6.4                          & 3.44E+21                                & 2                        \\
                    213                     & 01556$+$0203$+$0288          & 15.5583                 & 2.0333                  & 6,  7                                    & \nodata                           & 2.26                          & 28.84                             & 27.72                             & $-$0.59                         & 6.6                       & 3.9                       & 13.9                         & 4.15E+21                                & 1                        \\
                    214                     & 01589$+$0221$+$0278          & 15.8917                 & 2.2083                  & 1,  5,  6                                & \nodata                           & 2.34                          & 27.82                             & 26.93                             & $-$0.86                         & 3.8                       & 1.6                       & 9.0                          & 7.85E+20                                & 1                        \\
                    215                     & 01614$-$0097$+$0579          & 16.1417                 & $-$0.9750                 & 3,  5,  6                                & \nodata                           & 3.65                          & 57.94                             & 57.00                             & $-$0.27                         & 7.1                       & 2.0                       & 13.5                         & 3.52E+21                                & 1                        \\
                    216                     & 01626$+$0100$+$0195          & 16.2583                 & 1.0000                  & 4,  6                                    & \nodata                           & 1.50                          & 19.52                             & 19.18                             & $-$0.32                         & 5.0                       & 1.5                       & 10.2                         & 5.36E+21                                & 2                        \\
                    217                     & 01627$+$0382$+$0054          & 16.2667                 & 3.8250                  & 5,  6                                    & \nodata                           & 0.29                          & 5.38                              & 4.36                              & $-$0.60                         & 9.6                       & 1.5                       & 13.6                         & 1.26E+21                                & 1                        \\
                    218                     & 01630$+$0247$+$0272          & 16.3000                 & 2.4667                  & 5,  6                                    & \nodata                           & 2.41                          & 27.24                             & 25.74                             & $-$1.02                         & 4.6                       & 1.5                       & 13.1                         & 1.06E+21                                & 1                        \\
                    219                     & 01640$+$0013$+$0284          & 16.4000                 & 0.1333                  & 4,  6,  7                                & \nodata                           & 3.10                          & 28.41                             & 27.49                             & $-$0.35                         & 7.0                       & 1.4                       & 11.6                         & 1.66E+21                                & 1                        \\
                    220                     & 01641$+$0426$+$0044          & 16.4083                 & 4.2583                  & 5,  6                                    & \nodata                           & 0.28                          & 4.36                              & 3.28                              & $-$0.55                         & 4.3                       & 1.1                       & 12.2                         & 9.43E+20                                & 1                        \\
                    221                     & 01644$+$0103$+$0237          & 16.4417                 & 1.0333                  & 2,  4,  5,  6,  7                        & \nodata                           & 1.50                          & 23.70                             & 23.37                             & $-$0.34                         & 7.1                       & 2.2                       & 12.2                         & 8.46E+21                                & 2                        \\
                    222                     & 01644$+$0416$+$0043          & 16.4417                 & 4.1583                  & 5,  6                                    & \nodata                           & 0.28                          & 4.34                              & 3.13                              & $-$1.10                         & 5.8                       & 1.3                       & 12.1                         & 6.71E+20                                & 1                        \\
                    223                     & 01646$+$0177$+$0199          & 16.4583                 & 1.7667                  & 2,  5,  6                                & \nodata                           & 1.20                          & 19.88                             & 19.37                             & $-$0.42                         & 11.9                      & 5.6                       & 19.4                         & 4.51E+21                                & 1                        \\
                    224                     & 01647$+$0182$+$0292          & 16.4667                 & 1.8250                  & 2,  5,  6                                & \nodata                           & 2.26                          & 29.19                             & 28.49                             & $-$0.38                         & 7.0                       & 2.5                       & 15.4                         & 2.46E+21                                & 1                        \\
                    225                     & 01647$+$0012$+$0196          & 16.4750                 & 0.1167                  & 4,  6,  7                                & \nodata                           & 1.84                          & 19.61                             & 19.11                             & $-$0.37                         & 8.9                       & 2.3                       & 13.8                         & 1.54E+21                                & 1                        \\
                    226                     & 01647$+$0247$+$0282          & 16.4750                 & 2.4750                  & 5,  6                                    & \nodata                           & 2.35                          & 28.16                             & 27.25                             & $-$0.64                         & 2.9                       & 1.5                       & 7.7                          & 1.02E+21                                & 1                        \\
                    227                     & 01649$+$0406$+$0042          & 16.4917                 & 4.0583                  & 5,  6                                    & \nodata                           & 0.27                          & 4.21                              & 3.29                              & $-$0.93                         & 6.8                       & 1.0                       & 15.5                         & 5.09E+20                                & 1                        \\
                    228                     & 01657$+$0227$+$0280          & 16.5667                 & 2.2667                  & 5,  6                                    & \nodata                           & 2.37                          & 28.00                             & 27.32                             & $-$0.80                         & 4.1                       & 1.0                       & 8.0                          & 3.90E+20                                & 1                        \\
                    229                     & 01667$+$0061$+$0198          & 16.6667                 & 0.6083                  & 2,  4,  6,  7                            & \nodata                           & 1.49                          & 19.76                             & 19.38                             & $-$0.35                         & 6.9                       & 1.5                       & 11.1                         & 5.72E+21                                & 2                        \\
                    230                     & 01668$+$0013$+$0155          & 16.6833                 & 0.1333                  & 4,  6                                    & \nodata                           & 1.43                          & 15.45                             & 15.10                             & $-$0.32                         & 9.8                       & 3.4                       & 15.6                         & 2.14E+21                                & 1                        \\
                    231                     & 01669$+$0515$+$0044          & 16.6917                 & 5.1500                  & 5,  6                                    & \nodata                           & 0.26                          & 4.41                              & 4.05                              & $-$0.49                         & 4.5                       & 0.9                       & 11.0                         & 2.92E+20                                & 1                        \\
                    232                     & 01680$+$0110$+$0197          & 16.8000                 & 1.1000                  & 2,  4,  5,  6,  7                        & \nodata                           & 1.50                          & 19.68                             & 19.06                             & $-$0.27                         & 16.7                      & 4.2                       & 29.2                         & 7.54E+21                                & 1                        \\
                    233                     & 01680$+$0250$+$0355          & 16.8000                 & 2.5000                  & 5,  6                                    & \nodata                           & 1.88                          & 35.46                             & 34.81                             & $-$0.35                         & 17.2                      & 7.4                       & 27.0                         & 1.13E+22                                & 1                        \\
                    234                     & 01683$+$0245$+$0346          & 16.8333                 & 2.4500                  & 5,  6                                    & \nodata                           & 1.95                          & 34.57                             & 34.16                             & $-$0.29                         & 12.5                      & 2.2                       & 21.7                         & 1.99E+21                                & 1                        \\
                    235                     & 01687$-$0256$+$0183          & 16.8667                 & $-$2.5583                 & 5,  6                                    & \nodata                           & 1.13                          & 18.25                             & 17.43                             & $-$0.39                         & 12.4                      & 5.6                       & 20.1                         & 8.08E+21                                & 1                        \\
                    236                     & 01688$-$0217$+$0181          & 16.8833                 & $-$2.1750                 & 3,  5,  6                                & \nodata                           & 1.15                          & 18.11                             & 16.88                             & $-$0.45                         & 14.3                      & 7.3                       & 23.2                         & 1.52E+22                                & 1                        \\
                    237                     & 01691$+$0232$+$0289          & 16.9083                 & 2.3250                  & 5,  6                                    & \nodata                           & 2.32                          & 28.89                             & 28.24                             & $-$0.68                         & 2.5                       & 1.1                       & 5.3                          & 6.44E+20                                & 1                        \\
                    238                     & 01692$+$0096$+$0206          & 16.9167                 & 0.9583                  & 1,  2,  3,  4,  5,  6                    & \nodata                           & 1.49                          & 20.64                             & 19.50                             & $-$0.46                         & 11.9                      & 3.7                       & 16.4                         & 4.14E+22                                & 2                        \\
                    239                     & 01692$-$0229$+$0178          & 16.9167                 & $-$2.2917                 & 3,  5,  6                                & \nodata                           & 1.15                          & 17.78                             & 16.97                             & $-$0.27                         & 11.4                      & 4.6                       & 19.5                         & 8.99E+21                                & 1                        \\
                    240                     & 01702$-$0222$+$0188          & 17.0167                 & $-$2.2167                 & 3,  5,  6                                & \nodata                           & 1.48                          & 18.78                             & 18.10                             & $-$0.24                         & 6.1                       & 1.0                       & 11.1                         & 9.73E+21                                & 2                        \\
                    241                     & 01702$-$0237$+$0191          & 17.0167                 & $-$2.3667                 & 5,  6                                    & \nodata                           & 1.16                          & 19.12                             & 18.19                             & $-$0.39                         & 13.1                      & 7.4                       & 23.9                         & 1.36E+22                                & 1                        \\
                    242                     & 01704$-$0241$+$0201          & 17.0417                 & $-$2.4083                 & 5,  6                                    & \nodata                           & 1.15                          & 20.07                             & 18.98                             & $-$0.48                         & 13.7                      & 7.6                       & 18.5                         & 1.26E+22                                & 1                        \\
                    243                     & 01717$+$0111$+$0170          & 17.1667                 & 1.1083                  & 2,  3,  4,  5,  6                        & \nodata                           & 1.49                          & 16.98                             & 16.26                             & $-$0.35                         & 13.1                      & 3.9                       & 20.9                         & 5.23E+21                                & 1                        \\
                    244                     & 01720$-$0256$+$0200          & 17.2000                 & $-$2.5583                 & 5,  6                                    & \nodata                           & 1.13                          & 20.00                             & 19.48                             & $-$0.39                         & 8.8                       & 2.1                       & 15.7                         & 1.50E+21                                & 1                        \\
                    245                     & 01723$+$0092$+$0186          & 17.2333                 & 0.9250                  & 2,  3,  4,  5,  6,  7                    & \nodata                           & 1.49                          & 18.59                             & 18.26                             & $-$0.17                         & 14.6                      & 4.3                       & 19.2                         & 5.27E+21                                & 1                        \\
                    246                     & 01724$+$0211$+$0282          & 17.2417                 & 2.1083                  & 5,  6                                    & \nodata                           & 2.00                          & 28.16                             & 27.76                             & $-$0.39                         & 4.3                       & 1.6                       & 8.4                          & 7.85E+20                                & 1                        \\
                    247                     & 01738$+$0212$+$0284          & 17.3833                 & 2.1167                  & 5,  6                                    & \nodata                           & 2.00                          & 28.41                             & 28.19                             & $-$0.29                         & 4.4                       & 1.8                       & 9.2                          & 6.55E+20                                & 1                        \\
                    248                     & 01756$-$0086$+$0057          & 17.5583                 & $-$0.8583                 & 6                                        & \nodata                           & 1.26                          & 5.70                              & 5.21                              & $-$0.48                         & 5.6                       & 1.1                       & 12.5                         & 5.03E+20                                & 1                        \\
                    249                     & 01757$+$0190$+$0266          & 17.5667                 & 1.9000                  & 4,  5,  6                                & \nodata                           & 1.49                          & 26.56                             & 26.14                             & $-$0.42                         & 6.3                       & 3.1                       & 15.8                         & 1.66E+21                                & 1                        \\
                    250                     & 01760$+$0192$+$0268          & 17.6000                 & 1.9250                  & 4,  5,  6                                & \nodata                           & 1.49                          & 26.76                             & 26.41                             & $-$0.38                         & 8.8                       & 6.0                       & 15.1                         & 3.73E+21                                & 1                        \\
                    251                     & 01761$+$0195$+$0270          & 17.6083                 & 1.9500                  & 4,  5,  6                                & \nodata                           & 1.49                          & 26.96                             & 26.62                             & $-$0.31                         & 6.2                       & 3.4                       & 11.3                         & 2.01E+21                                & 1                        \\
                    252                     & 01762$+$0199$+$0271          & 17.6250                 & 1.9917                  & 4,  5,  6                                & \nodata                           & 1.49                          & 27.09                             & 26.33                             & $-$0.57                         & 6.0                       & 1.4                       & 12.5                         & 8.77E+20                                & 1                        \\
                    253                     & 01765$+$0227$+$0289          & 17.6500                 & 2.2750                  & 4,  5,  6                                & \nodata                           & 2.35                          & 28.91                             & 28.26                             & $-$0.74                         & 4.2                       & 1.8                       & 7.6                          & 7.72E+20                                & 1                        \\
                    254                     & 01769$+$0287$+$0259          & 17.6917                 & 2.8750                  & 5,  6                                    & \nodata                           & 2.00                          & 25.88                             & 25.44                             & $-$0.42                         & 4.5                       & 1.7                       & 8.6                          & 8.79E+20                                & 1                        \\
                    255                     & 01772$+$0279$+$0256          & 17.7167                 & 2.7917                  & 5,  6                                    & \nodata                           & 1.49                          & 25.60                             & 24.94                             & $-$0.37                         & 5.5                       & 1.3                       & 13.5                         & 1.10E+21                                & 1                        \\
                    256                     & 01781$-$0054$+$0050          & 17.8083                 & $-$0.5417                 & 1,  2,  3,  6                            & \nodata                           & 1.26                          & 5.00                              & 4.73                              & $-$0.30                         & 6.5                       & 1.2                       & 12.2                         & 5.23E+20                                & 1                        \\
                    257                     & 01782$-$0051$+$0049          & 17.8167                 & $-$0.5083                 & 2,  3,  6                                & \nodata                           & 1.26                          & 4.87                              & 4.60                              & $-$0.29                         & 6.8                       & 1.5                       & 14.3                         & 6.83E+20                                & 1                        \\
                    258                     & 01783$-$0048$+$0051          & 17.8333                 & $-$0.4833                 & 2,  6                                    & \nodata                           & 1.26                          & 5.14                              & 4.74                              & $-$0.32                         & 5.9                       & 1.5                       & 13.3                         & 9.05E+20                                & 1                        \\
                    259                     & 01784$+$0250$+$0272          & 17.8417                 & 2.5000                  & 4,  5,  6                                & \nodata                           & 2.00                          & 27.20                             & 26.50                             & $-$0.41                         & 5.7                       & 1.5                       & 13.6                         & 1.24E+21                                & 1                        \\
                    260                     & 01785$-$0044$+$0051          & 17.8500                 & $-$0.4417                 & 2,  6                                    & \nodata                           & 1.26                          & 5.12                              & 4.71                              & $-$0.40                         & 7.1                       & 1.5                       & 13.1                         & 7.27E+20                                & 1                        \\
                    261                     & 01787$-$0022$+$0529          & 17.8667                 & $-$0.2167                 & 3,  6                                    & \nodata                           & 1.91                          & 52.90                             & 52.66                             & $-$0.20                         & 4.1                       & 1.0                       & 8.1                          & 4.11E+21                                & 2                        \\
                    262                     & 01787$-$0046$+$0054          & 17.8750                 & $-$0.4583                 & 2,  6,  7                                & \nodata                           & 1.26                          & 5.36                              & 4.77                              & $-$0.64                         & 6.3                       & 1.4                       & 13.9                         & 6.15E+20                                & 1                        \\
                    263                     & 01790$+$0312$+$0261          & 17.9000                 & 3.1167                  & 5,  6                                    & \nodata                           & 2.00                          & 26.10                             & 25.64                             & $-$0.41                         & 4.3                       & 1.6                       & 9.0                          & 8.12E+20                                & 1                        \\
                    264                     & 01792$+$0296$+$0266          & 17.9167                 & 2.9583                  & 5,  6                                    & \nodata                           & 2.00                          & 26.59                             & 25.74                             & $-$0.66                         & 4.1                       & 2.2                       & 9.8                          & 1.34E+21                                & 1                        \\
                    265                     & 01793$+$0007$+$1280          & 17.9333                 & 0.0667                  & 3,  5,  6,  7                            & \nodata                           & 7.91                          & 128.02                            & 126.78                            & $-$0.46                         & 3.7                       & 1.8                       & 10.9                         & 2.25E+21                                & 1                        \\
                    266                     & 01793$+$0307$+$0256          & 17.9333                 & 3.0750                  & 5,  6                                    & \nodata                           & 2.00                          & 25.58                             & 24.91                             & $-$0.65                         & 6.1                       & 3.0                       & 16.2                         & 1.71E+21                                & 1                        \\
                    267                     & 01799$+$0314$+$0259          & 17.9917                 & 3.1417                  & 5,  6                                    & \nodata                           & 2.00                          & 25.87                             & 25.54                             & $-$0.30                         & 7.0                       & 1.7                       & 11.9                         & 9.16E+20                                & 1                        \\
                    268                     & 01803$+$0270$+$0277          & 18.0333                 & 2.7000                  & 4,  5,  6                                & \nodata                           & 2.00                          & 27.71                             & 27.01                             & $-$0.46                         & 7.1                       & 2.5                       & 12.6                         & 1.90E+21                                & 1                        \\
                    269                     & 01806$+$0275$+$0262          & 18.0583                 & 2.7500                  & 4,  5,  6                                & \nodata                           & 2.00                          & 26.19                             & 25.69                             & $-$0.31                         & 5.5                       & 2.3                       & 9.3                          & 1.85E+21                                & 1                        \\
                    270                     & 01827$-$0044$+$0047          & 18.2667                 & $-$0.4417                 & 3,  6,  7                                & \nodata                           & 1.26                          & 4.70                              & 4.49                              & $-$0.22                         & 8.4                       & 1.6                       & 15.6                         & 7.67E+20                                & 1                        \\
                    271                     & 01827$-$0050$+$0671          & 18.2667                 & $-$0.5000                 & 3,  6,  7                                & \nodata                           & 3.94                          & 67.15                             & 65.55                             & $-$0.35                         & 7.4                       & 1.9                       & 11.8                         & 4.27E+21                                & 1                        \\
                    272                     & 01832$-$0044$+$0049          & 18.3167                 & $-$0.4417                 & 3,  6,  7                                & \nodata                           & 1.26                          & 4.87                              & 4.47                              & $-$0.35                         & 8.6                       & 1.8                       & 14.0                         & 1.00E+21                                & 1                        \\
                    273                     & 01833$+$0258$+$0275          & 18.3333                 & 2.5833                  & 4,  5,  6                                & \nodata                           & 2.00                          & 27.48                             & 26.54                             & $-$0.61                         & 5.3                       & 2.2                       & 11.2                         & 1.63E+21                                & 1                        \\
                    274                     & 01834$+$0264$+$0276          & 18.3417                 & 2.6417                  & 4,  5,  6                                & \nodata                           & 2.00                          & 27.56                             & 26.52                             & $-$0.48                         & 5.9                       & 1.6                       & 12.2                         & 1.64E+21                                & 1                        \\
                    275                     & 01834$-$0036$+$0049          & 18.3417                 & $-$0.3583                 & 3,  5,  6                                & \nodata                           & 1.26                          & 4.94                              & 4.48                              & $-$0.39                         & 7.3                       & 2.6                       & 16.9                         & 1.68E+21                                & 1                        \\
                    276                     & 01835$-$0044$+$0048          & 18.3500                 & $-$0.4417                 & 1,  3,  6                                & \nodata                           & 1.26                          & 4.76                              & 4.59                              & $-$0.18                         & 7.2                       & 1.6                       & 15.7                         & 7.58E+20                                & 1                        \\
                    277                     & 01837$+$0222$+$0179          & 18.3667                 & 2.2167                  & 4,  5,  6                                & \nodata                           & 1.20                          & 17.94                             & 17.45                             & $-$0.52                         & 5.1                       & 0.8                       & 10.7                         & 3.41E+20                                & 1                        \\
                    278                     & 01845$+$0000$+$0533          & 18.4500                 & 0.0000                  & 1,  5,  6                                & \nodata                           & 3.30                          & 53.30                             & 51.39                             & $-$0.44                         & 5.7                       & 2.0                       & 10.2                         & 3.24E+22                                & 2                        \\
                    279                     & 01847$-$0141$+$0360          & 18.4667                 & $-$1.4083                 & 5,  6                                    & \nodata                           & 1.94                          & 35.97                             & 35.39                             & $-$0.35                         & 6.9                       & 1.8                       & 12.9                         & 1.49E+21                                & 1                        \\
                    280                     & 01851$+$0256$+$0274          & 18.5083                 & 2.5583                  & 4,  5,  6                                & \nodata                           & 2.00                          & 27.42                             & 26.57                             & $-$0.75                         & 6.9                       & 1.8                       & 12.1                         & 9.84E+20                                & 1                        \\
                    281                     & 01855$+$0295$+$0256          & 18.5500                 & 2.9500                  & 4,  5,  6                                & \nodata                           & 2.00                          & 25.57                             & 25.04                             & $-$0.35                         & 4.8                       & 1.2                       & 10.3                         & 8.52E+20                                & 1                        \\
                    282                     & 01855$-$0137$+$0354          & 18.5500                 & $-$1.3667                 & 5,  6                                    & \nodata                           & 1.96                          & 35.44                             & 34.73                             & $-$0.44                         & 7.3                       & 2.5                       & 13.5                         & 2.05E+21                                & 1                        \\
                    283                     & 01859$+$0000$+$0487          & 18.5917                 & 0.0000                  & 1,  5,  6                                & \nodata                           & 3.25                          & 48.69                             & 48.28                             & $-$0.13                         & 11.6                      & 4.5                       & 16.7                         & 8.34E+21                                & 1                        \\
                    284                     & 01862$+$0351$+$0120          & 18.6250                 & 3.5083                  & 5,  6                                    & \nodata                           & 0.30                          & 11.98                             & 11.77                             & $-$0.23                         & 5.6                       & 2.2                       & 11.9                         & 9.76E+20                                & 1                        \\
                    285                     & 01862$-$0008$+$0454          & 18.6250                 & $-$0.0833                 & 3,  5,  6,  7                            & 9                                 & 1.92                          & 45.38                             & 44.43                             & $-$0.35                         & 13.5                      & 6.7                       & 21.3                         & 1.32E+22                                & 1                        \\
                    286                     & 01866$-$0008$+$0453          & 18.6583                 & $-$0.0833                 & 3,  5,  6,  7                            & 9                                 & 1.92                          & 45.34                             & 44.57                             & $-$0.33                         & 16.2                      & 8.1                       & 28.2                         & 1.60E+22                                & 1                        \\
                    287                     & 01873$+$0266$+$0267          & 18.7333                 & 2.6583                  & 4,  5,  6                                & \nodata                           & 2.00                          & 26.68                             & 25.90                             & $-$0.65                         & 6.7                       & 2.7                       & 13.8                         & 1.68E+21                                & 1                        \\
                    288                     & 01877$-$0022$+$0661          & 18.7667                 & $-$0.2250                 & 3,  5,  6,  7                            & \nodata                           & 3.90                          & 66.08                             & 65.56                             & $-$0.32                         & 5.5                       & 1.0                       & 8.9                          & 5.23E+21                                & 2                        \\
                    289                     & 01882$-$0008$+$1223          & 18.8167                 & $-$0.0833                 & 3,  5,  6                                & \nodata                           & 9.56                          & 122.28                            & 121.49                            & $-$0.37                         & 3.0                       & 1.4                       & 7.9                          & 1.04E+22                                & 2                        \\
                    290                     & 01892$+$0276$+$0050          & 18.9167                 & 2.7583                  & 4,  5,  6                                & \nodata                           & 0.27                          & 4.96                              & 4.68                              & $-$0.36                         & 3.4                       & 1.0                       & 10.0                         & 3.46E+20                                & 1                        \\
                    291                     & 01903$+$0160$+$0261          & 19.0333                 & 1.6000                  & 4,  5,  6,  7                            & \nodata                           & 1.49                          & 26.12                             & 25.83                             & $-$0.16                         & 9.2                       & 2.2                       & 12.4                         & 1.48E+22                                & 2                        \\
                    292                     & 01911$+$0026$+$0051          & 19.1083                 & 0.2583                  & 5,  6,  7                                & \nodata                           & 1.81                          & 5.10                              & 4.47                              & $-$0.58                         & 6.1                       & 2.2                       & 12.5                         & 1.21E+21                                & 1                        \\
                    293                     & 01912$+$0187$+$0268          & 19.1167                 & 1.8750                  & 4,  5,  6                                & \nodata                           & 1.49                          & 26.81                             & 25.77                             & $-$0.72                         & 6.6                       & 2.6                       & 12.0                         & 1.93E+21                                & 1                        \\
                    294                     & 01912$+$0022$+$0050          & 19.1250                 & 0.2167                  & 5,  6,  7                                & \nodata                           & 1.81                          & 5.03                              & 4.77                              & $-$0.20                         & 6.4                       & 1.3                       & 10.2                         & 7.61E+20                                & 1                        \\
                    295                     & 01914$+$0292$+$0050          & 19.1417                 & 2.9167                  & 4,  5,  6                                & \nodata                           & 0.27                          & 4.95                              & 4.43                              & $-$0.39                         & 4.5                       & 1.0                       & 10.4                         & 5.60E+20                                & 1                        \\
                    296                     & 01927$+$0432$+$0045          & 19.2750                 & 4.3167                  & 5,  6                                    & \nodata                           & 0.27                          & 4.46                              & 4.02                              & $-$0.55                         & 4.4                       & 1.0                       & 9.6                          & 3.28E+20                                & 1                        \\
                    297                     & 01953$-$0046$+$0638          & 19.5333                 & $-$0.4583                 & 5,  6,  7                                & \nodata                           & 3.27                          & 63.76                             & 63.11                             & $-$0.27                         & 6.7                       & 1.8                       & 12.0                         & 1.64E+22                                & 2                        \\
                    298                     & 01954$+$0443$+$0124          & 19.5417                 & 4.4333                  & 5,  6                                    & \nodata                           & 0.30                          & 12.43                             & 11.87                             & $-$0.44                         & 5.0                       & 1.5                       & 10.5                         & 8.70E+20                                & 1                        \\
                    299                     & 01955$+$0447$+$0124          & 19.5500                 & 4.4667                  & 5,  6                                    & \nodata                           & 0.30                          & 12.41                             & 11.60                             & $-$0.70                         & 4.5                       & 1.1                       & 14.8                         & 6.06E+20                                & 1                        \\
                    300                     & 01957$+$0291$+$0054          & 19.5667                 & 2.9083                  & 5,  6                                    & \nodata                           & 0.27                          & 5.38                              & 4.81                              & $-$0.60                         & 6.2                       & 1.1                       & 12.5                         & 4.83E+20                                & 1                        \\
                    301                     & 01957$+$0449$+$0126          & 19.5667                 & 4.4917                  & 5,  6                                    & \nodata                           & 0.30                          & 12.60                             & 11.63                             & $-$0.81                         & 5.2                       & 1.8                       & 12.8                         & 1.04E+21                                & 1                        \\
                    302                     & 01961$+$0022$+$0060          & 19.6083                 & 0.2250                  & 3,  5,  6                                & \nodata                           & 0.17                          & 5.97                              & 5.10                              & $-$0.61                         & 7.5                       & 3.1                       & 15.2                         & 2.43E+21                                & 1                        \\
                    303                     & 01965$+$0280$+$0053          & 19.6500                 & 2.8000                  & 5,  6                                    & \nodata                           & 0.27                          & 5.26                              & 4.77                              & $-$0.55                         & 6.7                       & 1.2                       & 13.8                         & 5.23E+20                                & 1                        \\
                    304                     & 01965$-$0379$+$0073          & 19.6500                 & $-$3.7917                 & 5,  6                                    & \nodata                           & 0.25                          & 7.32                              & 6.94                              & $-$0.35                         & 5.4                       & 1.6                       & 10.4                         & 8.23E+20                                & 1                        \\
                    305                     & 01966$+$0515$+$0144          & 19.6583                 & 5.1500                  & 6                                        & \nodata                           & 0.31                          & 14.45                             & 13.86                             & $-$0.39                         & 4.8                       & 1.2                       & 12.9                         & 8.71E+20                                & 1                        \\
                    306                     & 01970$+$0018$+$0064          & 19.7000                 & 0.1833                  & 5,  6                                    & \nodata                           & 1.47                          & 6.39                              & 5.53                              & $-$0.74                         & 8.0                       & 3.0                       & 18.6                         & 2.05E+21                                & 1                        \\
                    307                     & 01970$+$0022$+$0059          & 19.7000                 & 0.2250                  & 5,  6                                    & \nodata                           & 1.47                          & 5.89                              & 5.43                              & $-$0.34                         & 8.6                       & 3.2                       & 14.2                         & 2.33E+21                                & 1                        \\
                    308                     & 01973$-$0064$+$0233          & 19.7333                 & $-$0.6417                 & 5,  6,  7                                & \nodata                           & 1.50                          & 23.33                             & 22.90                             & $-$0.24                         & 5.6                       & 1.3                       & 11.7                         & 7.98E+21                                & 2                        \\
                    309                     & 01980$+$0162$+$0338          & 19.8000                 & 1.6250                  & 4,  5,  6                                & \nodata                           & 2.06                          & 33.81                             & 33.24                             & $-$0.34                         & 9.8                       & 1.9                       & 16.0                         & 1.69E+21                                & 1                        \\
                    310                     & 01981$-$0381$+$0075          & 19.8083                 & $-$3.8083                 & 5,  6                                    & \nodata                           & 0.25                          & 7.51                              & 7.14                              & $-$0.49                         & 4.7                       & 1.6                       & 14.6                         & 5.89E+20                                & 1                        \\
                    311                     & 01987$-$0277$+$0069          & 19.8667                 & $-$2.7750                 & 6                                        & \nodata                           & 0.26                          & 6.89                              & 5.74                              & $-$0.87                         & 5.0                       & 1.5                       & 10.2                         & 8.77E+20                                & 1                        \\
                    312                     & 01987$-$0282$+$0071          & 19.8667                 & $-$2.8167                 & 6                                        & \nodata                           & 0.26                          & 7.08                              & 5.54                              & $-$1.55                         & 6.3                       & 2.1                       & 15.6                         & 1.11E+21                                & 1                        \\
                    313                     & 01989$+$0047$+$0063          & 19.8917                 & 0.4667                  & 5,  6                                    & \nodata                           & 0.19                          & 6.25                              & 5.93                              & $-$0.43                         & 6.5                       & 2.6                       & 10.1                         & 1.01E+21                                & 1                        \\
                    314                     & 01993$+$0525$+$0060          & 19.9333                 & 5.2500                  & 6                                        & \nodata                           & 0.28                          & 5.98                              & 4.88                              & $-$1.21                         & 6.0                       & 3.2                       & 22.6                         & 1.91E+21                                & 1                        \\
                    315                     & 01994$-$0312$+$0066          & 19.9417                 & $-$3.1250                 & 5,  6                                    & \nodata                           & 0.26                          & 6.58                              & 6.09                              & $-$0.57                         & 5.5                       & 1.2                       & 17.0                         & 5.41E+20                                & 1                        \\
                    316                     & 02002$-$0250$+$0060          & 20.0250                 & $-$2.5000                 & 6                                        & \nodata                           & 0.25                          & 6.03                              & 5.40                              & $-$0.31                         & 5.8                       & 1.0                       & 11.0                         & 8.58E+20                                & 1                        \\
                    317                     & 02010$+$0071$+$0066          & 20.1000                 & 0.7083                  & 5,  6                                    & \nodata                           & 0.20                          & 6.60                              & 6.39                              & $-$0.28                         & 8.4                       & 2.5                       & 14.6                         & 9.66E+20                                & 1                        \\
                    318                     & 02011$-$0258$+$0074          & 20.1083                 & $-$2.5833                 & 6                                        & \nodata                           & 0.26                          & 7.35                              & 6.29                              & $-$0.73                         & 5.3                       & 1.5                       & 11.8                         & 1.02E+21                                & 1                        \\
                    319                     & 02012$+$0448$+$0035          & 20.1250                 & 4.4833                  & 5,  6                                    & \nodata                           & 0.27                          & 3.52                              & 3.12                              & $-$0.53                         & 5.0                       & 0.9                       & 9.7                          & 2.92E+20                                & 1                        \\
                    320                     & 02021$+$0087$+$0061          & 20.2083                 & 0.8750                  & 5,  6                                    & \nodata                           & 0.20                          & 6.12                              & 5.55                              & $-$0.54                         & 8.1                       & 3.8                       & 14.8                         & 2.29E+21                                & 1                        \\
                    321                     & 02024$-$0002$+$0065          & 20.2417                 & $-$0.0167                 & 5,  6,  7                                & \nodata                           & 0.17                          & 6.52                              & 6.21                              & $-$0.38                         & 7.3                       & 2.5                       & 14.8                         & 1.07E+21                                & 1                        \\
                    322                     & 02028$-$0017$+$0067          & 20.2833                 & $-$0.1667                 & 5,  6                                    & \nodata                           & 1.52                          & 6.67                              & 6.40                              & $-$0.34                         & 6.6                       & 1.8                       & 11.8                         & 6.44E+20                                & 1                        \\
                    323                     & 02032$-$0018$+$0067          & 20.3250                 & $-$0.1833                 & 5,  6,  7                                & \nodata                           & 1.52                          & 6.74                              & 6.31                              & $-$0.38                         & 6.4                       & 1.1                       & 11.9                         & 5.80E+20                                & 1                        \\
                    324                     & 02037$+$0032$+$0061          & 20.3667                 & 0.3250                  & 5,  6                                    & \nodata                           & 1.51                          & 6.08                              & 5.60                              & $-$0.44                         & 6.1                       & 2.3                       & 11.0                         & 1.19E+21                                & 1                        \\
                    325                     & 02037$-$0247$+$0078          & 20.3750                 & $-$2.4667                 & 6                                        & \nodata                           & 0.26                          & 7.83                              & 7.45                              & $-$0.40                         & 5.0                       & 2.2                       & 8.6                          & 1.07E+21                                & 1                        \\
                    326                     & 02041$+$0022$+$0064          & 20.4083                 & 0.2250                  & 5,  6                                    & \nodata                           & 0.17                          & 6.39                              & 5.81                              & $-$0.61                         & 6.6                       & 2.0                       & 13.2                         & 9.46E+20                                & 1                        \\
                    327                     & 02045$-$0022$+$0069          & 20.4500                 & $-$0.2167                 & 3,  5,  6                                & \nodata                           & 1.51                          & 6.85                              & 6.41                              & $-$0.43                         & 8.0                       & 1.3                       & 14.5                         & 6.59E+20                                & 1                        \\
                    328                     & 02046$-$0258$+$0073          & 20.4583                 & $-$2.5833                 & 3,  6                                    & \nodata                           & 0.26                          & 7.33                              & 6.73                              & $-$0.45                         & 5.3                       & 1.5                       & 11.3                         & 9.13E+20                                & 1                        \\
                    329                     & 02047$-$0014$+$0919          & 20.4667                 & $-$0.1417                 & 5,  6,  7                                & \nodata                           & 5.50                          & 91.94                             & 91.07                             & $-$0.51                         & 5.0                       & 1.7                       & 10.0                         & 1.37E+21                                & 1                        \\
                    330                     & 02052$-$0070$+$0601          & 20.5250                 & $-$0.7000                 & 5,  6                                    & \nodata                           & 3.47                          & 60.06                             & 59.37                             & $-$0.28                         & 6.3                       & 2.5                       & 10.2                         & 3.18E+21                                & 1                        \\
                    331                     & 02054$-$0243$+$0080          & 20.5417                 & $-$2.4333                 & 6                                        & \nodata                           & 0.26                          & 7.96                              & 7.35                              & $-$0.60                         & 6.2                       & 2.0                       & 12.7                         & 9.70E+20                                & 1                        \\
                    332                     & 02056$+$0402$+$0035          & 20.5583                 & 4.0250                  & 5,  6                                    & \nodata                           & 0.27                          & 3.52                              & 3.09                              & $-$0.44                         & 7.3                       & 2.2                       & 12.9                         & 1.08E+21                                & 1                        \\
                    333                     & 02056$-$0234$+$0085          & 20.5583                 & $-$2.3417                 & 6                                        & \nodata                           & 0.26                          & 8.51                              & 7.94                              & $-$0.47                         & 4.6                       & 3.2                       & 10.2                         & 2.08E+21                                & 1                        \\
                    334                     & 02056$-$0248$+$0082          & 20.5583                 & $-$2.4833                 & 6                                        & \nodata                           & 0.26                          & 8.23                              & 7.72                              & $-$0.51                         & 4.2                       & 1.5                       & 9.5                          & 6.78E+20                                & 1                        \\
                    335                     & 02057$+$0400$+$0036          & 20.5667                 & 4.0000                  & 5,  6                                    & \nodata                           & 0.27                          & 3.61                              & 3.04                              & $-$0.41                         & 6.8                       & 2.0                       & 12.6                         & 1.36E+21                                & 1                        \\
                    336                     & 02060$+$0074$+$0063          & 20.6000                 & 0.7417                  & 5,  6                                    & \nodata                           & 0.20                          & 6.25                              & 5.72                              & $-$0.50                         & 8.1                       & 3.6                       & 13.6                         & 2.08E+21                                & 1                        \\
                    337                     & 02068$+$0451$+$0032          & 20.6833                 & 4.5083                  & 5,  6                                    & \nodata                           & 0.27                          & 3.16                              & 2.68                              & $-$0.55                         & 4.5                       & 1.5                       & 11.5                         & 6.01E+20                                & 1                        \\
                    338                     & 02070$+$0025$+$0070          & 20.7000                 & 0.2500                  & 5,  6                                    & \nodata                           & 1.50                          & 7.00                              & 6.50                              & $-$0.59                         & 7.3                       & 1.5                       & 15.2                         & 6.69E+20                                & 1                        \\
                    339                     & 02070$+$0283$+$0042          & 20.7000                 & 2.8333                  & 1,  6                                    & \nodata                           & 0.26                          & 4.21                              & 3.09                              & $-$1.04                         & 4.0                       & 1.7                       & 9.2                          & 8.81E+20                                & 1                        \\
                    340                     & 02071$+$0029$+$0069          & 20.7083                 & 0.2917                  & 5,  6                                    & \nodata                           & 1.50                          & 6.87                              & 6.50                              & $-$0.40                         & 7.7                       & 1.5                       & 15.7                         & 7.23E+20                                & 1                        \\
                    341                     & 02075$+$0280$+$0042          & 20.7500                 & 2.8000                  & 6                                        & \nodata                           & 0.26                          & 4.21                              & 2.99                              & $-$1.54                         & 3.6                       & 1.6                       & 12.4                         & 6.20E+20                                & 1                        \\
                    342                     & 02091$+$0415$+$0051          & 20.9083                 & 4.1500                  & 3,  6                                    & \nodata                           & 0.27                          & 5.12                              & 4.72                              & $-$0.37                         & 6.4                       & 2.6                       & 11.6                         & 1.45E+21                                & 1                        \\
                    343                     & 02093$+$0019$+$0075          & 20.9333                 & 0.1917                  & 3,  5,  6,  7                            & \nodata                           & 1.20                          & 7.46                              & 7.08                              & $-$0.41                         & 11.3                      & 2.6                       & 19.3                         & 1.44E+21                                & 1                        \\
                    344                     & 02094$+$0413$+$0051          & 20.9417                 & 4.1333                  & 3,  6                                    & \nodata                           & 0.27                          & 5.12                              & 4.68                              & $-$0.53                         & 5.7                       & 2.4                       & 9.6                          & 1.02E+21                                & 1                        \\
                    345                     & 02096$+$0409$+$0051          & 20.9583                 & 4.0917                  & 3,  6                                    & \nodata                           & 0.27                          & 5.15                              & 4.60                              & $-$0.59                         & 5.3                       & 2.5                       & 10.0                         & 1.16E+21                                & 1                        \\
                    346                     & 02097$+$0047$+$0074          & 20.9750                 & 0.4750                  & 5,  6                                    & \nodata                           & 0.19                          & 7.41                              & 7.17                              & $-$0.31                         & 8.2                       & 2.3                       & 13.8                         & 8.99E+20                                & 1                        \\
                    347                     & 02098$+$0023$+$0075          & 20.9833                 & 0.2333                  & 3,  5,  6,  7                            & \nodata                           & 1.20                          & 7.53                              & 7.18                              & $-$0.37                         & 12.7                      & 3.3                       & 20.2                         & 1.96E+21                                & 1                        \\
                    348                     & 02098$+$0408$+$0052          & 20.9833                 & 4.0833                  & 3,  6                                    & \nodata                           & 0.27                          & 5.17                              & 4.62                              & $-$0.58                         & 5.6                       & 2.0                       & 12.5                         & 9.31E+20                                & 1                        \\
                    349                     & 02098$+$0411$+$0051          & 20.9833                 & 4.1083                  & 3,  5,  6                                & \nodata                           & 0.27                          & 5.12                              & 4.61                              & $-$0.62                         & 6.2                       & 3.2                       & 10.5                         & 1.41E+21                                & 1                        \\
                    350                     & 02098$+$0413$+$0051          & 20.9833                 & 4.1333                  & 3,  5,  6                                & \nodata                           & 0.27                          & 5.10                              & 4.59                              & $-$0.57                         & 5.3                       & 1.6                       & 9.8                          & 6.83E+20                                & 1                        \\
                    351                     & 02102$+$0046$+$0075          & 21.0167                 & 0.4583                  & 5,  6                                    & \nodata                           & 0.19                          & 7.47                              & 7.21                              & $-$0.22                         & 8.8                       & 1.6                       & 13.8                         & 9.15E+20                                & 1                        \\
                    352                     & 02102$+$0407$+$0051          & 21.0167                 & 4.0750                  & 3,  5,  6                                & \nodata                           & 0.27                          & 5.10                              & 4.54                              & $-$0.56                         & 5.6                       & 3.3                       & 10.7                         & 1.81E+21                                & 1                        \\
                    353                     & 02105$+$0031$+$0075          & 21.0500                 & 0.3083                  & 5,  6                                    & \nodata                           & 1.22                          & 7.48                              & 7.24                              & $-$0.32                         & 8.8                       & 3.2                       & 13.6                         & 1.30E+21                                & 1                        \\
                    354                     & 02106$+$0405$+$0050          & 21.0583                 & 4.0500                  & 3,  5,  6                                & \nodata                           & 0.27                          & 5.03                              & 4.76                              & $-$0.22                         & 5.6                       & 3.1                       & 14.6                         & 2.01E+21                                & 1                        \\
                    355                     & 02107$+$0409$+$0049          & 21.0667                 & 4.0917                  & 3,  5,  6                                & \nodata                           & 0.27                          & 4.93                              & 4.29                              & $-$0.57                         & 5.7                       & 1.6                       & 11.7                         & 8.34E+20                                & 1                        \\
                    356                     & 02109$+$0414$+$0052          & 21.0917                 & 4.1417                  & 3,  5,  6                                & \nodata                           & 0.27                          & 5.23                              & 4.33                              & $-$0.85                         & 4.5                       & 1.6                       & 9.3                          & 7.56E+20                                & 1                        \\
                    357                     & 02111$+$0409$+$0052          & 21.1083                 & 4.0917                  & 3,  5,  6                                & \nodata                           & 0.27                          & 5.21                              & 4.36                              & $-$0.79                         & 4.4                       & 1.8                       & 11.6                         & 9.25E+20                                & 1                        \\
                    358                     & 02112$+$0414$+$0053          & 21.1167                 & 4.1417                  & 3,  5,  6                                & \nodata                           & 0.27                          & 5.34                              & 4.24                              & $-$1.12                         & 4.5                       & 2.3                       & 9.2                          & 1.12E+21                                & 1                        \\
                    359                     & 02112$+$0305$+$0244          & 21.1250                 & 3.0500                  & 1,  6                                    & \nodata                           & 1.49                          & 24.42                             & 23.89                             & $-$0.47                         & 4.1                       & 1.2                       & 9.7                          & 5.99E+20                                & 1                        \\
                    360                     & 02113$+$0489$+$0030          & 21.1333                 & 4.8917                  & 6                                        & \nodata                           & 0.26                          & 3.00                              & 2.47                              & $-$0.66                         & 4.0                       & 2.4                       & 8.2                          & 1.03E+21                                & 1                        \\
                    361                     & 02119$+$0299$+$0242          & 21.1917                 & 2.9917                  & 6                                        & \nodata                           & 1.49                          & 24.20                             & 23.72                             & $-$0.38                         & 5.3                       & 1.5                       & 10.4                         & 8.58E+20                                & 1                        \\
                    362                     & 02127$+$0284$+$0038          & 21.2667                 & 2.8417                  & 6                                        & \nodata                           & 0.26                          & 3.76                              & 3.25                              & $-$0.52                         & 4.3                       & 1.3                       & 11.1                         & 5.64E+20                                & 1                        \\
                    363                     & 02154$+$0302$+$0138          & 21.5417                 & 3.0250                  & 6                                        & \nodata                           & 1.16                          & 13.76                             & 12.99                             & $-$0.66                         & 5.5                       & 1.8                       & 12.9                         & 1.03E+21                                & 1                        \\
                    364                     & 02165$+$0373$+$0069          & 21.6500                 & 3.7333                  & 5,  6                                    & \nodata                           & 0.28                          & 6.91                              & 6.36                              & $-$0.55                         & 3.9                       & 1.2                       & 8.4                          & 4.23E+21                                & 2                        \\
                    365                     & 02182$+$0477$+$0058          & 21.8167                 & 4.7750                  & 6                                        & \nodata                           & 0.28                          & 5.77                              & 4.59                              & $-$0.70                         & 5.3                       & 1.7                       & 10.1                         & 1.34E+21                                & 1                        \\
                    366                     & 02202$+$0491$+$0054          & 22.0167                 & 4.9083                  & 6                                        & \nodata                           & 0.24                          & 5.37                              & 4.60                              & $-$0.40                         & 5.7                       & 1.6                       & 10.8                         & 1.36E+21                                & 1                        \\
                    367                     & 02211$+$0068$+$0636          & 22.1083                 & 0.6833                  & 5,  6                                    & \nodata                           & 0.31                          & 63.62                             & 63.31                             & $-$0.35                         & 3.7                       & 1.4                       & 7.7                          & 4.61E+21                                & 2                        \\
                    368                     & 02211$+$0479$+$0050          & 22.1083                 & 4.7917                  & 6                                        & \nodata                           & 0.24                          & 5.01                              & 4.57                              & $-$0.43                         & 6.1                       & 2.3                       & 9.9                          & 1.17E+21                                & 1                        \\
                    369                     & 02215$+$0344$+$0082          & 22.1500                 & 3.4417                  & 6                                        & \nodata                           & 0.24                          & 8.22                              & 7.40                              & $-$0.89                         & 5.0                       & 1.3                       & 10.0                         & 5.44E+20                                & 1                        \\
                    370                     & 02222$+$0465$+$0051          & 22.2167                 & 4.6500                  & 6                                        & \nodata                           & 0.24                          & 5.11                              & 4.32                              & $-$0.50                         & 5.9                       & 1.3                       & 12.1                         & 9.31E+20                                & 1                        \\
                    371                     & 02225$+$0455$+$0053          & 22.2500                 & 4.5500                  & 6                                        & \nodata                           & 0.24                          & 5.32                              & 4.65                              & $-$0.55                         & 6.7                       & 2.0                       & 13.1                         & 1.19E+21                                & 1                        \\
                    372                     & 02227$+$0434$+$0060          & 22.2667                 & 4.3417                  & 6                                        & \nodata                           & 0.24                          & 6.04                              & 5.67                              & $-$0.43                         & 7.1                       & 1.5                       & 10.8                         & 6.10E+20                                & 1                        \\
                    373                     & 02230$+$0455$+$0055          & 22.3000                 & 4.5500                  & 6                                        & \nodata                           & 0.24                          & 5.55                              & 4.86                              & $-$0.80                         & 6.7                       & 2.6                       & 11.1                         & 1.14E+21                                & 1                        \\
                    374                     & 02232$+$0328$+$0044          & 22.3167                 & 3.2833                  & 5,  6                                    & \nodata                           & 0.24                          & 4.37                              & 3.64                              & $-$0.69                         & 3.9                       & 1.7                       & 10.9                         & 8.76E+20                                & 1                        \\
                    375                     & 02232$+$0397$+$0057          & 22.3167                 & 3.9750                  & 6                                        & \nodata                           & 0.24                          & 5.68                              & 5.03                              & $-$0.65                         & 5.8                       & 2.3                       & 12.3                         & 1.13E+21                                & 1                        \\
                    376                     & 02232$+$0427$+$0059          & 22.3167                 & 4.2667                  & 6                                        & \nodata                           & 0.24                          & 5.88                              & 5.38                              & $-$0.37                         & 6.4                       & 1.1                       & 9.7                          & 6.76E+20                                & 1                        \\
                    377                     & 02232$+$0511$+$0048          & 22.3250                 & 5.1083                  & 6                                        & \nodata                           & 0.24                          & 4.78                              & 4.35                              & $-$0.32                         & 6.1                       & 2.5                       & 15.5                         & 1.78E+21                                & 1                        \\
                    378                     & 02233$+$0452$+$0057          & 22.3333                 & 4.5167                  & 6                                        & \nodata                           & 0.24                          & 5.67                              & 4.76                              & $-$0.92                         & 5.9                       & 3.1                       & 11.6                         & 1.65E+21                                & 1                        \\
                    379                     & 02234$+$0504$+$0049          & 22.3417                 & 5.0417                  & 6                                        & \nodata                           & 0.24                          & 4.89                              & 4.33                              & $-$0.48                         & 6.1                       & 3.1                       & 10.5                         & 1.95E+21                                & 1                        \\
                    380                     & 02235$+$0457$+$0054          & 22.3500                 & 4.5750                  & 6                                        & \nodata                           & 0.24                          & 5.42                              & 4.76                              & $-$0.65                         & 6.1                       & 2.6                       & 10.8                         & 1.35E+21                                & 1                        \\
                    381                     & 02235$-$0108$+$0043          & 22.3500                 & $-$1.0833                 & 5,  6                                    & \nodata                           & 0.24                          & 4.26                              & 3.75                              & $-$0.49                         & 7.3                       & 3.8                       & 13.5                         & 2.25E+21                                & 1                        \\
                    382                     & 02236$-$0113$+$0043          & 22.3583                 & $-$1.1333                 & 5,  6                                    & \nodata                           & 0.24                          & 4.34                              & 4.06                              & $-$0.29                         & 6.2                       & 2.5                       & 10.1                         & 1.23E+21                                & 1                        \\
                    383                     & 02237$+$0494$+$0054          & 22.3667                 & 4.9417                  & 6                                        & \nodata                           & 0.24                          & 5.40                              & 4.83                              & $-$0.32                         & 6.0                       & 2.0                       & 11.5                         & 1.71E+21                                & 1                        \\
                    384                     & 02237$+$0508$+$0049          & 22.3750                 & 5.0833                  & 6                                        & \nodata                           & 0.24                          & 4.90                              & 4.36                              & $-$0.34                         & 6.6                       & 2.0                       & 13.9                         & 1.61E+21                                & 1                        \\
                    385                     & 02239$+$0499$+$0052          & 22.3917                 & 4.9917                  & 6                                        & \nodata                           & 0.24                          & 5.19                              & 4.66                              & $-$0.48                         & 5.5                       & 3.0                       & 11.1                         & 1.74E+21                                & 1                        \\
                    386                     & 02239$+$0519$+$0052          & 22.3917                 & 5.1917                  & 6                                        & \nodata                           & 0.24                          & 5.19                              & 4.69                              & $-$0.48                         & 6.5                       & 1.4                       & 10.2                         & 6.55E+20                                & 1                        \\
                    387                     & 02241$+$0511$+$0051          & 22.4083                 & 5.1083                  & 6                                        & \nodata                           & 0.24                          & 5.06                              & 4.51                              & $-$0.35                         & 6.4                       & 1.5                       & 16.9                         & 1.26E+21                                & 1                        \\
                    388                     & 02242$+$0507$+$0048          & 22.4167                 & 5.0667                  & 6                                        & \nodata                           & 0.24                          & 4.80                              & 4.17                              & $-$0.62                         & 5.4                       & 2.2                       & 11.0                         & 1.07E+21                                & 1                        \\
                    389                     & 02242$+$0518$+$0052          & 22.4250                 & 5.1833                  & 6                                        & \nodata                           & 0.24                          & 5.21                              & 4.66                              & $-$0.52                         & 7.4                       & 1.5                       & 12.0                         & 7.52E+20                                & 1                        \\
                    390                     & 02243$+$0492$+$0055          & 22.4333                 & 4.9250                  & 6                                        & \nodata                           & 0.24                          & 5.46                              & 4.95                              & $-$0.57                         & 5.5                       & 1.6                       & 12.3                         & 6.81E+20                                & 1                        \\
                    391                     & 02245$+$0517$+$0053          & 22.4500                 & 5.1750                  & 6                                        & \nodata                           & 0.24                          & 5.26                              & 4.63                              & $-$0.56                         & 6.1                       & 1.5                       & 10.0                         & 7.41E+20                                & 1                        \\
                    392                     & 02247$+$0508$+$0049          & 22.4750                 & 5.0833                  & 6                                        & \nodata                           & 0.24                          & 4.95                              & 4.50                              & $-$0.41                         & 5.6                       & 1.9                       & 13.9                         & 1.04E+21                                & 1                        \\
                    393                     & 02247$+$0514$+$0052          & 22.4750                 & 5.1417                  & 6                                        & \nodata                           & 0.24                          & 5.20                              & 4.75                              & $-$0.34                         & 6.2                       & 1.4                       & 12.0                         & 8.94E+20                                & 1                        \\
                    394                     & 02250$+$0517$+$0052          & 22.5000                 & 5.1667                  & 6                                        & \nodata                           & 0.24                          & 5.24                              & 4.77                              & $-$0.36                         & 6.1                       & 1.6                       & 10.1                         & 9.46E+20                                & 1                        \\
                    395                     & 02250$+$0522$+$0053          & 22.5000                 & 5.2250                  & 6                                        & \nodata                           & 0.24                          & 5.26                              & 4.82                              & $-$0.48                         & 6.0                       & 2.2                       & 10.9                         & 9.89E+20                                & 1                        \\
                    396                     & 02251$+$0525$+$0052          & 22.5083                 & 5.2500                  & 6                                        & \nodata                           & 0.24                          & 5.19                              & 4.88                              & $-$0.38                         & 6.4                       & 2.6                       & 10.6                         & 1.07E+21                                & 1                        \\
                    397                     & 02252$+$0517$+$0052          & 22.5250                 & 5.1750                  & 6                                        & \nodata                           & 0.24                          & 5.24                              & 4.66                              & $-$0.53                         & 6.1                       & 1.6                       & 15.0                         & 8.92E+20                                & 1                        \\
                    398                     & 02253$+$0352$+$0043          & 22.5333                 & 3.5250                  & 6                                        & \nodata                           & 0.24                          & 4.33                              & 3.86                              & $-$0.59                         & 5.6                       & 1.7                       & 9.5                          & 6.31E+20                                & 1                        \\
                    399                     & 02255$+$0358$+$0042          & 22.5500                 & 3.5833                  & 6                                        & \nodata                           & 0.24                          & 4.16                              & 3.60                              & $-$0.69                         & 5.3                       & 2.3                       & 12.7                         & 9.18E+20                                & 1                        \\
                    400                     & 02255$+$0523$+$0053          & 22.5500                 & 5.2333                  & 6                                        & \nodata                           & 0.24                          & 5.34                              & 4.63                              & $-$0.88                         & 5.8                       & 2.2                       & 13.5                         & 8.87E+20                                & 1                        \\
                    401                     & 02256$+$0520$+$0052          & 22.5583                 & 5.2000                  & 6                                        & \nodata                           & 0.24                          & 5.18                              & 4.78                              & $-$0.29                         & 5.9                       & 1.6                       & 13.3                         & 1.05E+21                                & 1                        \\
                    402                     & 02259$+$0511$+$0053          & 22.5917                 & 5.1083                  & 5,  6                                    & \nodata                           & 0.24                          & 5.35                              & 4.82                              & $-$0.54                         & 5.0                       & 1.8                       & 9.9                          & 8.08E+20                                & 1                        \\
                    403                     & 02262$+$0521$+$0050          & 22.6167                 & 5.2083                  & 6                                        & \nodata                           & 0.24                          & 5.00                              & 4.68                              & $-$0.37                         & 6.0                       & 2.7                       & 10.8                         & 1.21E+21                                & 1                        \\
                    404                     & 02264$+$0515$+$0053          & 22.6417                 & 5.1500                  & 5,  6                                    & \nodata                           & 0.23                          & 5.33                              & 4.91                              & $-$0.40                         & 6.1                       & 1.7                       & 12.8                         & 8.78E+20                                & 1                        \\
                    405                     & 02267$+$0511$+$0054          & 22.6750                 & 5.1083                  & 5,  6                                    & \nodata                           & 0.23                          & 5.36                              & 4.83                              & $-$0.48                         & 5.2                       & 1.5                       & 11.2                         & 7.64E+20                                & 1                        \\
                    406                     & 02267$-$0121$+$0039          & 22.6750                 & $-$1.2083                 & 5,  6                                    & \nodata                           & 0.23                          & 3.93                              & 3.43                              & $-$0.46                         & 7.4                       & 2.7                       & 12.7                         & 1.47E+21                                & 1                        \\
                    407                     & 02268$-$0136$+$0039          & 22.6833                 & $-$1.3583                 & 5,  6                                    & \nodata                           & 0.23                          & 3.89                              & 3.21                              & $-$0.48                         & 6.0                       & 1.1                       & 13.5                         & 7.16E+20                                & 1                        \\
                    408                     & 02269$+$0450$+$0058          & 22.6917                 & 4.5000                  & 5,  6                                    & \nodata                           & 0.23                          & 5.83                              & 5.00                              & $-$0.47                         & 5.0                       & 1.7                       & 15.6                         & 1.52E+21                                & 1                        \\
                    409                     & 02270$+$0352$+$0082          & 22.7000                 & 3.5167                  & 6                                        & \nodata                           & 0.23                          & 8.17                              & 7.59                              & $-$0.60                         & 4.5                       & 1.1                       & 10.9                         & 4.86E+20                                & 1                        \\
                    410                     & 02270$-$0131$+$0039          & 22.7000                 & $-$1.3083                 & 5,  6                                    & \nodata                           & 0.23                          & 3.87                              & 2.92                              & $-$0.82                         & 7.6                       & 3.4                       & 17.6                         & 2.28E+21                                & 1                        \\
                    411                     & 02272$+$0512$+$0054          & 22.7167                 & 5.1167                  & 5,  6                                    & \nodata                           & 0.23                          & 5.41                              & 4.88                              & $-$0.54                         & 5.6                       & 1.8                       & 15.3                         & 9.03E+20                                & 1                        \\
                    412                     & 02272$+$0516$+$0053          & 22.7250                 & 5.1583                  & 5,  6                                    & \nodata                           & 0.23                          & 5.28                              & 4.82                              & $-$0.48                         & 5.9                       & 2.5                       & 11.7                         & 1.19E+21                                & 1                        \\
                    413                     & 02273$+$0508$+$0055          & 22.7333                 & 5.0833                  & 5,  6                                    & \nodata                           & 0.23                          & 5.45                              & 4.92                              & $-$0.51                         & 5.5                       & 1.7                       & 10.2                         & 8.00E+20                                & 1                        \\
                    414                     & 02274$+$0504$+$0055          & 22.7417                 & 5.0417                  & 5,  6                                    & \nodata                           & 0.23                          & 5.48                              & 4.80                              & $-$0.65                         & 5.2                       & 1.5                       & 14.6                         & 7.98E+20                                & 1                        \\
                    415                     & 02274$-$0137$+$0041          & 22.7417                 & $-$1.3750                 & 5,  6                                    & \nodata                           & 0.23                          & 4.13                              & 3.36                              & $-$0.50                         & 8.0                       & 2.4                       & 12.2                         & 1.83E+21                                & 1                        \\
                    416                     & 02275$+$0514$+$0053          & 22.7500                 & 5.1417                  & 5,  6                                    & \nodata                           & 0.23                          & 5.26                              & 4.88                              & $-$0.40                         & 5.9                       & 2.5                       & 14.2                         & 1.29E+21                                & 1                        \\
                    417                     & 02277$+$0506$+$0055          & 22.7667                 & 5.0583                  & 5,  6                                    & \nodata                           & 0.23                          & 5.46                              & 4.84                              & $-$0.66                         & 5.2                       & 1.4                       & 11.2                         & 6.29E+20                                & 1                        \\
                    418                     & 02277$+$0520$+$0055          & 22.7667                 & 5.2000                  & 5,  6                                    & \nodata                           & 0.23                          & 5.54                              & 4.78                              & $-$0.85                         & 5.9                       & 1.8                       & 10.1                         & 7.79E+20                                & 1                        \\
                    419                     & 02277$-$0026$+$0727          & 22.7667                 & $-$0.2583                 & 1,  3,  5,  6,  7                        & \nodata                           & 4.75                          & 72.73                             & 71.08                             & $-$0.36                         & 5.1                       & 0.9                       & 9.9                          & 1.41E+22                                & 2                        \\
                    420                     & 02277$-$0140$+$0043          & 22.7667                 & $-$1.4000                 & 5,  6                                    & \nodata                           & 0.23                          & 4.29                              & 3.96                              & $-$0.24                         & 7.5                       & 3.4                       & 13.3                         & 2.49E+21                                & 1                        \\
                    421                     & 02279$+$0505$+$0055          & 22.7917                 & 5.0500                  & 5,  6                                    & \nodata                           & 0.23                          & 5.54                              & 5.02                              & $-$0.55                         & 5.0                       & 1.5                       & 9.5                          & 6.49E+20                                & 1                        \\
                    422                     & 02282$+$0501$+$0055          & 22.8167                 & 5.0083                  & 5,  6                                    & \nodata                           & 0.23                          & 5.54                              & 5.23                              & $-$0.40                         & 5.6                       & 1.9                       & 9.6                          & 6.99E+20                                & 1                        \\
                    423                     & 02282$+$0503$+$0055          & 22.8167                 & 5.0333                  & 5,  6                                    & \nodata                           & 0.23                          & 5.45                              & 4.98                              & $-$0.52                         & 5.2                       & 2.0                       & 10.9                         & 8.46E+20                                & 1                        \\
                    424                     & 02282$+$0260$+$0082          & 22.8250                 & 2.6000                  & 5,  6                                    & \nodata                           & 0.23                          & 8.21                              & 7.71                              & $-$0.47                         & 6.9                       & 3.2                       & 10.0                         & 1.86E+21                                & 1                        \\
                    425                     & 02282$+$0517$+$0055          & 22.8250                 & 5.1750                  & 5,  6                                    & \nodata                           & 0.23                          & 5.49                              & 4.85                              & $-$0.55                         & 5.3                       & 1.6                       & 12.0                         & 8.77E+20                                & 1                        \\
                    426                     & 02283$+$0512$+$0054          & 22.8333                 & 5.1250                  & 5,  6                                    & \nodata                           & 0.23                          & 5.38                              & 4.96                              & $-$0.39                         & 5.4                       & 2.2                       & 14.7                         & 1.25E+21                                & 1                        \\
                    427                     & 02284$+$0498$+$0056          & 22.8417                 & 4.9833                  & 5,  6                                    & \nodata                           & 0.23                          & 5.65                              & 5.19                              & $-$0.36                         & 5.5                       & 1.5                       & 9.8                          & 8.99E+20                                & 1                        \\
                    428                     & 02286$+$0507$+$0055          & 22.8583                 & 5.0750                  & 5,  6                                    & \nodata                           & 0.23                          & 5.51                              & 4.97                              & $-$0.45                         & 5.2                       & 1.8                       & 10.7                         & 1.00E+21                                & 1                        \\
                    429                     & 02287$+$0512$+$0053          & 22.8750                 & 5.1167                  & 5,  6                                    & \nodata                           & 0.23                          & 5.32                              & 4.88                              & $-$0.44                         & 5.5                       & 2.5                       & 12.3                         & 1.25E+21                                & 1                        \\
                    430                     & 02289$+$0509$+$0053          & 22.8917                 & 5.0917                  & 5,  6                                    & \nodata                           & 0.23                          & 5.29                              & 4.82                              & $-$0.50                         & 4.8                       & 2.8                       & 10.3                         & 1.39E+21                                & 1                        \\
                    431                     & 02290$+$0247$+$0086          & 22.9000                 & 2.4750                  & 5,  6                                    & \nodata                           & 0.23                          & 8.61                              & 7.69                              & $-$0.53                         & 7.4                       & 2.3                       & 14.5                         & 2.10E+21                                & 1                        \\
                    432                     & 02290$+$0504$+$0056          & 22.9000                 & 5.0417                  & 5,  6                                    & \nodata                           & 0.23                          & 5.64                              & 5.11                              & $-$0.50                         & 4.5                       & 2.3                       & 9.1                          & 1.28E+21                                & 1                        \\
                    433                     & 02297$+$0508$+$0054          & 22.9667                 & 5.0833                  & 5,  6                                    & \nodata                           & 0.23                          & 5.38                              & 4.89                              & $-$0.50                         & 5.9                       & 2.7                       & 15.7                         & 1.41E+21                                & 1                        \\
                    434                     & 02298$-$0017$+$0638          & 22.9833                 & $-$0.1750                 & 5,  6,  7                                & \nodata                           & 4.70                          & 63.76                             & 63.38                             & $-$0.31                         & 5.7                       & 1.8                       & 9.0                          & 7.92E+21                                & 2                        \\
                    435                     & 02300$-$0131$+$0035          & 23.0000                 & $-$1.3083                 & 5,  6                                    & \nodata                           & 0.24                          & 3.48                              & 3.02                              & $-$0.42                         & 6.4                       & 1.4                       & 11.2                         & 7.01E+20                                & 1                        \\
                    436                     & 02302$+$0048$+$0026          & 23.0167                 & 0.4833                  & 3,  5,  6                                & \nodata                           & 0.24                          & 2.61                              & 2.35                              & $-$0.27                         & 6.6                       & 2.2                       & 11.6                         & 1.02E+21                                & 1                        \\
                    437                     & 02303$+$0041$+$0031          & 23.0333                 & 0.4083                  & 3,  5,  6,  7                            & \nodata                           & 0.24                          & 3.05                              & 2.68                              & $-$0.38                         & 6.2                       & 2.9                       & 11.0                         & 1.50E+21                                & 1                        \\
                    438                     & 02304$+$0496$+$0061          & 23.0417                 & 4.9583                  & 5,  6                                    & \nodata                           & 0.24                          & 6.08                              & 5.39                              & $-$0.36                         & 5.8                       & 1.7                       & 16.1                         & 1.69E+21                                & 1                        \\
                    439                     & 02307$+$0498$+$0060          & 23.0750                 & 4.9833                  & 5,  6                                    & \nodata                           & 0.24                          & 5.97                              & 5.20                              & $-$0.53                         & 5.3                       & 2.3                       & 11.4                         & 1.59E+21                                & 1                        \\
                    440                     & 02310$+$0492$+$0060          & 23.1000                 & 4.9250                  & 5,  6                                    & \nodata                           & 0.24                          & 5.97                              & 5.37                              & $-$0.37                         & 5.8                       & 2.3                       & 13.3                         & 1.88E+21                                & 1                        \\
                    441                     & 02312$+$0140$+$0150          & 23.1250                 & 1.4000                  & 5,  6                                    & \nodata                           & 1.49                          & 14.95                             & 14.72                             & $-$0.21                         & 7.2                       & 1.7                       & 13.3                         & 9.28E+20                                & 1                        \\
                    442                     & 02312$+$0495$+$0060          & 23.1250                 & 4.9500                  & 5,  6                                    & \nodata                           & 0.24                          & 6.01                              & 5.15                              & $-$0.39                         & 5.6                       & 2.1                       & 15.6                         & 2.47E+21                                & 1                        \\
                    443                     & 02314$-$0036$+$0571          & 23.1417                 & $-$0.3583                 & 3,  5,  6,  7                            & \nodata                           & 3.49                          & 57.12                             & 56.60                             & $-$0.21                         & 6.8                       & 2.7                       & 10.9                         & 2.71E+22                                & 2                        \\
                    444                     & 02317$+$0475$+$0068          & 23.1667                 & 4.7500                  & 5,  6                                    & \nodata                           & 0.24                          & 6.76                              & 6.04                              & $-$0.76                         & 4.5                       & 2.6                       & 8.7                          & 1.30E+21                                & 1                        \\
                    445                     & 02321$+$0497$+$0062          & 23.2083                 & 4.9750                  & 5,  6                                    & \nodata                           & 0.24                          & 6.19                              & 5.60                              & $-$0.31                         & 5.6                       & 2.1                       & 11.8                         & 1.93E+21                                & 1                        \\
                    446                     & 02323$+$0497$+$0066          & 23.2333                 & 4.9750                  & 5,  6                                    & \nodata                           & 0.24                          & 6.57                              & 5.45                              & $-$0.81                         & 5.5                       & 2.4                       & 10.2                         & 1.69E+21                                & 1                        \\
                    447                     & 02327$+$0132$+$0147          & 23.2667                 & 1.3250                  & 5,  6                                    & \nodata                           & 1.49                          & 14.65                             & 14.35                             & $-$0.30                         & 7.1                       & 1.8                       & 12.5                         & 8.53E+20                                & 1                        \\
                    448                     & 02327$+$0183$+$0353          & 23.2750                 & 1.8333                  & 5,  6                                    & \nodata                           & 1.50                          & 35.29                             & 34.51                             & $-$0.83                         & 5.2                       & 1.2                       & 10.2                         & 4.90E+20                                & 1                        \\
                    449                     & 02327$+$0494$+$0067          & 23.2750                 & 4.9417                  & 5,  6                                    & \nodata                           & 0.24                          & 6.65                              & 5.55                              & $-$1.00                         & 5.3                       & 2.1                       & 9.8                          & 1.12E+21                                & 1                        \\
                    450                     & 02329$+$0130$+$0147          & 23.2917                 & 1.3000                  & 5,  6                                    & \nodata                           & 1.49                          & 14.69                             & 14.17                             & $-$0.61                         & 7.4                       & 2.0                       & 12.8                         & 8.30E+20                                & 1                        \\
                    451                     & 02331$+$0478$+$0064          & 23.3083                 & 4.7833                  & 5,  6                                    & \nodata                           & 0.24                          & 6.35                              & 5.82                              & $-$0.61                         & 4.8                       & 1.9                       & 9.0                          & 7.96E+20                                & 1                        \\
                    452                     & 02334$+$0411$+$0056          & 23.3417                 & 4.1083                  & 6                                        & \nodata                           & 0.24                          & 5.61                              & 5.09                              & $-$0.73                         & 4.0                       & 1.2                       & 9.3                          & 3.67E+20                                & 1                        \\
                    453                     & 02335$+$0124$+$0078          & 23.3500                 & 1.2417                  & 5,  6                                    & \nodata                           & 0.32                          & 7.84                              & 7.21                              & $-$0.32                         & 5.9                       & 1.5                       & 14.7                         & 1.47E+21                                & 1                        \\
                    454                     & 02335$+$0421$+$0058          & 23.3500                 & 4.2083                  & 6                                        & \nodata                           & 0.24                          & 5.76                              & 5.06                              & $-$0.92                         & 4.1                       & 2.0                       & 9.5                          & 7.48E+20                                & 1                        \\
                    455                     & 02337$+$0502$+$0059          & 23.3750                 & 5.0250                  & 6                                        & \nodata                           & 0.24                          & 5.87                              & 5.24                              & $-$0.38                         & 4.6                       & 1.2                       & 14.7                         & 9.88E+20                                & 1                        \\
                    456                     & 02337$+$0506$+$0059          & 23.3750                 & 5.0583                  & 6                                        & \nodata                           & 0.24                          & 5.93                              & 5.20                              & $-$0.82                         & 4.6                       & 1.1                       & 13.3                         & 4.75E+20                                & 1                        \\
                    457                     & 02340$+$0423$+$0058          & 23.4000                 & 4.2333                  & 6                                        & \nodata                           & 0.24                          & 5.84                              & 5.30                              & $-$0.52                         & 4.7                       & 1.6                       & 11.6                         & 7.67E+20                                & 1                        \\
                    458                     & 02342$+$0283$+$0093          & 23.4167                 & 2.8333                  & 5,  6                                    & \nodata                           & 0.24                          & 9.30                              & 8.53                              & $-$0.85                         & 6.0                       & 1.5                       & 11.5                         & 6.49E+20                                & 1                        \\
                    459                     & 02344$+$0430$+$0062          & 23.4417                 & 4.3000                  & 6                                        & \nodata                           & 0.24                          & 6.22                              & 5.39                              & $-$0.86                         & 4.4                       & 1.9                       & 11.2                         & 8.86E+20                                & 1                        \\
                    460                     & 02349$+$0157$+$0364          & 23.4917                 & 1.5667                  & 3,  6                                    & \nodata                           & 1.99                          & 36.41                             & 35.49                             & $-$0.62                         & 14.3                      & 6.3                       & 18.1                         & 6.33E+21                                & 1                        \\
                    461                     & 02352$-$0058$+$0048          & 23.5250                 & $-$0.5833                 & 5,  6                                    & \nodata                           & 1.67                          & 4.76                              & 4.26                              & $-$0.33                         & 4.9                       & 1.4                       & 8.8                          & 9.58E+20                                & 1                        \\
                    462                     & 02356$+$0112$+$0374          & 23.5583                 & 1.1167                  & 5,  6                                    & \nodata                           & 1.88                          & 37.38                             & 36.57                             & $-$0.60                         & 5.5                       & 1.0                       & 11.0                         & 5.88E+20                                & 1                        \\
                    463                     & 02360$+$0117$+$0142          & 23.6000                 & 1.1667                  & 5,  6                                    & \nodata                           & 0.24                          & 14.16                             & 13.59                             & $-$0.55                         & 5.7                       & 1.5                       & 12.4                         & 7.29E+20                                & 1                        \\
                    464                     & 02372$+$0333$+$0087          & 23.7167                 & 3.3333                  & 5,  6                                    & \nodata                           & 0.24                          & 8.68                              & 7.56                              & $-$1.08                         & 4.9                       & 1.6                       & 11.8                         & 7.69E+20                                & 1                        \\
                    465                     & 02373$+$0437$+$0063          & 23.7333                 & 4.3750                  & 6                                        & \nodata                           & 0.24                          & 6.29                              & 5.41                              & $-$0.97                         & 4.9                       & 2.0                       & 9.4                          & 8.80E+20                                & 1                        \\
                    466                     & 02374$+$0409$+$0059          & 23.7417                 & 4.0917                  & 5,  6                                    & \nodata                           & 0.24                          & 5.86                              & 5.07                              & $-$0.65                         & 4.8                       & 1.7                       & 10.6                         & 9.70E+20                                & 1                        \\
                    467                     & 02375$+$0442$+$0063          & 23.7500                 & 4.4167                  & 6                                        & \nodata                           & 0.24                          & 6.26                              & 5.20                              & $-$0.87                         & 4.9                       & 2.1                       & 10.1                         & 1.27E+21                                & 1                        \\
                    468                     & 02377$+$0438$+$0062          & 23.7667                 & 4.3833                  & 6                                        & \nodata                           & 0.24                          & 6.18                              & 5.44                              & $-$0.59                         & 5.4                       & 2.6                       & 9.5                          & 1.69E+21                                & 1                        \\
                    469                     & 02377$+$0222$+$0339          & 23.7750                 & 2.2167                  & 5,  6                                    & \nodata                           & 2.34                          & 33.85                             & 32.99                             & $-$0.84                         & 3.7                       & 1.3                       & 11.1                         & 5.98E+20                                & 1                        \\
                    470                     & 02379$+$0412$+$0059          & 23.7917                 & 4.1167                  & 5,  6                                    & \nodata                           & 0.24                          & 5.88                              & 5.32                              & $-$0.59                         & 5.6                       & 1.4                       & 9.3                          & 6.26E+20                                & 1                        \\
                    471                     & 02382$+$0442$+$0061          & 23.8167                 & 4.4167                  & 6                                        & \nodata                           & 0.24                          & 6.12                              & 5.20                              & $-$0.79                         & 5.4                       & 1.9                       & 10.5                         & 1.08E+21                                & 1                        \\
                    472                     & 02382$+$0067$+$0356          & 23.8250                 & 0.6750                  & 5,  6                                    & \nodata                           & 1.85                          & 35.56                             & 35.30                             & $-$0.33                         & 6.4                       & 1.2                       & 10.2                         & 4.11E+20                                & 1                        \\
                    473                     & 02383$-$0204$+$0091          & 23.8333                 & $-$2.0417                 & 5,  6                                    & \nodata                           & 0.24                          & 9.07                              & 8.58                              & $-$0.52                         & 4.7                       & 1.3                       & 13.5                         & 5.94E+20                                & 1                        \\
                    474                     & 02387$+$0107$+$0405          & 23.8750                 & 1.0750                  & 5,  6                                    & \nodata                           & 1.65                          & 40.50                             & 40.02                             & $-$0.35                         & 6.0                       & 1.2                       & 9.7                          & 7.46E+20                                & 1                        \\
                    475                     & 02396$+$0204$+$0093          & 23.9583                 & 2.0417                  & 5,  6                                    & \nodata                           & 0.23                          & 9.32                              & 8.81                              & $-$0.50                         & 6.7                       & 2.1                       & 11.1                         & 1.01E+21                                & 1                        \\
                    476                     & 02396$+$0443$+$0060          & 23.9583                 & 4.4333                  & 6                                        & \nodata                           & 0.23                          & 6.03                              & 5.68                              & $-$0.35                         & 5.2                       & 1.6                       & 9.0                          & 7.45E+20                                & 1                        \\
                    477                     & 02397$+$0202$+$0095          & 23.9667                 & 2.0167                  & 5,  6                                    & \nodata                           & 0.23                          & 9.45                              & 8.81                              & $-$0.36                         & 6.6                       & 1.4                       & 10.8                         & 1.12E+21                                & 1                        \\
                    478                     & 02400$+$0439$+$0063          & 24.0000                 & 4.3917                  & 5,  6                                    & \nodata                           & 0.24                          & 6.28                              & 5.62                              & $-$0.65                         & 4.5                       & 2.1                       & 10.2                         & 1.04E+21                                & 1                        \\
                    479                     & 02402$+$0462$+$0058          & 24.0167                 & 4.6167                  & 6                                        & \nodata                           & 0.24                          & 5.77                              & 5.35                              & $-$0.30                         & 4.7                       & 1.5                       & 9.2                          & 9.18E+20                                & 1                        \\
                    480                     & 02402$+$0332$+$0090          & 24.0250                 & 3.3250                  & 5,  6                                    & \nodata                           & 0.24                          & 8.97                              & 7.97                              & $-$1.03                         & 3.8                       & 2.0                       & 8.4                          & 9.54E+20                                & 1                        \\
                    481                     & 02402$+$0445$+$0059          & 24.0250                 & 4.4500                  & 5,  6                                    & \nodata                           & 0.24                          & 5.92                              & 5.44                              & $-$0.42                         & 4.9                       & 2.0                       & 8.8                          & 1.09E+21                                & 1                        \\
                    482                     & 02403$+$0022$+$1059          & 24.0333                 & 0.2167                  & 3,  6,  7                                & \nodata                           & 5.44                          & 105.93                            & 104.60                            & $-$0.43                         & 4.6                       & 2.0                       & 9.1                          & 2.30E+22                                & 2                        \\
                    483                     & 02407$+$0458$+$0058          & 24.0667                 & 4.5833                  & 5,  6                                    & \nodata                           & 0.24                          & 5.84                              & 5.35                              & $-$0.35                         & 5.4                       & 2.0                       & 12.0                         & 1.29E+21                                & 1                        \\
                    484                     & 02408$+$0437$+$0063          & 24.0833                 & 4.3667                  & 5,  6                                    & \nodata                           & 0.24                          & 6.32                              & 5.82                              & $-$0.41                         & 4.7                       & 1.8                       & 8.6                          & 1.06E+21                                & 1                        \\
                    485                     & 02410$+$0472$+$0059          & 24.1000                 & 4.7167                  & 6                                        & \nodata                           & 0.24                          & 5.85                              & 5.47                              & $-$0.24                         & 4.8                       & 1.6                       & 9.3                          & 1.21E+21                                & 1                        \\
                    486                     & 02415$+$0097$+$0407          & 24.1500                 & 0.9667                  & 6                                        & \nodata                           & 1.60                          & 40.72                             & 39.99                             & $-$0.70                         & 5.9                       & 2.0                       & 9.9                          & 1.02E+21                                & 1                        \\
                    487                     & 02415$+$0455$+$0060          & 24.1500                 & 4.5500                  & 6                                        & \nodata                           & 0.24                          & 6.05                              & 5.28                              & $-$0.54                         & 5.8                       & 2.3                       & 12.8                         & 1.60E+21                                & 1                        \\
                    488                     & 02417$+$0438$+$0063          & 24.1667                 & 4.3833                  & 5,  6                                    & \nodata                           & 0.24                          & 6.34                              & 5.73                              & $-$0.54                         & 4.3                       & 1.7                       & 9.2                          & 8.64E+20                                & 1                        \\
                    489                     & 02417$+$0450$+$0060          & 24.1750                 & 4.5000                  & 6                                        & \nodata                           & 0.24                          & 5.99                              & 5.35                              & $-$0.51                         & 5.3                       & 2.2                       & 10.7                         & 1.38E+21                                & 1                        \\
                    490                     & 02421$+$0024$+$0508          & 24.2083                 & 0.2417                  & 1,  3,  6,  7                            & \nodata                           & 1.48                          & 50.80                             & 49.30                             & $-$0.40                         & 7.8                       & 2.4                       & 15.4                         & 4.70E+21                                & 1                        \\
                    491                     & 02421$+$0191$+$0375          & 24.2083                 & 1.9083                  & 5,  6                                    & \nodata                           & 2.01                          & 37.52                             & 36.47                             & $-$0.54                         & 4.6                       & 2.3                       & 9.3                          & 2.21E+21                                & 1                        \\
                    492                     & 02422$+$0452$+$0061          & 24.2167                 & 4.5167                  & 6                                        & \nodata                           & 0.24                          & 6.06                              & 5.26                              & $-$0.46                         & 6.0                       & 2.3                       & 12.2                         & 2.01E+21                                & 1                        \\
                    493                     & 02424$+$0097$+$0408          & 24.2417                 & 0.9667                  & 6                                        & \nodata                           & 1.63                          & 40.79                             & 40.17                             & $-$0.68                         & 3.9                       & 1.1                       & 9.2                          & 4.26E+20                                & 1                        \\
                    494                     & 02425$+$0449$+$0063          & 24.2500                 & 4.4917                  & 6                                        & \nodata                           & 0.24                          & 6.28                              & 5.24                              & $-$0.74                         & 5.7                       & 2.8                       & 11.4                         & 1.99E+21                                & 1                        \\
                    495                     & 02425$+$0488$+$0054          & 24.2500                 & 4.8833                  & 6                                        & \nodata                           & 0.24                          & 5.45                              & 4.95                              & $-$0.61                         & 5.2                       & 1.2                       & 10.5                         & 4.30E+20                                & 1                        \\
                    496                     & 02426$+$0442$+$0064          & 24.2583                 & 4.4167                  & 5,  6                                    & \nodata                           & 0.24                          & 6.40                              & 5.37                              & $-$1.02                         & 6.1                       & 3.2                       & 10.2                         & 1.73E+21                                & 1                        \\
                    497                     & 02427$+$0462$+$0056          & 24.2667                 & 4.6250                  & 5,  6                                    & \nodata                           & 0.24                          & 5.59                              & 5.04                              & $-$0.50                         & 5.7                       & 2.8                       & 9.5                          & 1.69E+21                                & 1                        \\
                    498                     & 02427$+$0437$+$0063          & 24.2750                 & 4.3667                  & 5,  6                                    & \nodata                           & 0.24                          & 6.27                              & 5.52                              & $-$0.54                         & 5.0                       & 2.6                       & 9.6                          & 1.87E+21                                & 1                        \\
                    499                     & 02428$-$0167$+$0099          & 24.2833                 & $-$1.6750                 & 6                                        & \nodata                           & 0.24                          & 9.92                              & 9.49                              & $-$0.46                         & 5.3                       & 1.1                       & 10.3                         & 4.32E+20                                & 1                        \\
                    500                     & 02430$+$0447$+$0061          & 24.3000                 & 4.4667                  & 6                                        & \nodata                           & 0.24                          & 6.05                              & 4.87                              & $-$0.70                         & 6.2                       & 2.4                       & 13.1                         & 2.11E+21                                & 1                        \\
                    501                     & 02432$+$0438$+$0063          & 24.3167                 & 4.3833                  & 5,  6                                    & \nodata                           & 0.24                          & 6.29                              & 5.61                              & $-$0.46                         & 6.1                       & 2.7                       & 11.9                         & 2.07E+21                                & 1                        \\
                    502                     & 02432$+$0444$+$0061          & 24.3167                 & 4.4417                  & 5,  6                                    & \nodata                           & 0.24                          & 6.13                              & 5.04                              & $-$0.79                         & 5.7                       & 2.7                       & 10.6                         & 1.93E+21                                & 1                        \\
                    503                     & 02432$+$0436$+$0062          & 24.3250                 & 4.3583                  & 5,  6                                    & \nodata                           & 0.24                          & 6.21                              & 5.37                              & $-$0.63                         & 5.0                       & 2.5                       & 9.5                          & 1.73E+21                                & 1                        \\
                    504                     & 02435$+$0002$+$1173          & 24.3500                 & 0.0250                  & 3,  6,  7                                & \nodata                           & 5.47                          & 117.32                            & 114.85                            & $-$0.40                         & 13.3                      & 6.0                       & 24.2                         & 2.76E+22                                & 1                        \\
                    505                     & 02435$+$0433$+$0061          & 24.3500                 & 4.3333                  & 5,  6                                    & \nodata                           & 0.24                          & 6.08                              & 5.52                              & $-$0.53                         & 5.4                       & 2.4                       & 9.6                          & 1.28E+21                                & 1                        \\
                    506                     & 02436$+$0471$+$0055          & 24.3583                 & 4.7083                  & 5,  6                                    & \nodata                           & 0.24                          & 5.55                              & 4.93                              & $-$0.47                         & 6.2                       & 3.8                       & 12.2                         & 2.82E+21                                & 1                        \\
                    507                     & 02437$+$0434$+$0062          & 24.3750                 & 4.3417                  & 5,  6                                    & \nodata                           & 0.24                          & 6.16                              & 5.55                              & $-$0.70                         & 5.0                       & 2.3                       & 8.7                          & 1.05E+21                                & 1                        \\
                    508                     & 02440$+$0469$+$0055          & 24.4000                 & 4.6917                  & 5,  6                                    & \nodata                           & 0.24                          & 5.55                              & 5.11                              & $-$0.31                         & 6.3                       & 4.0                       & 10.7                         & 3.33E+21                                & 1                        \\
                    509                     & 02441$-$0036$+$0428          & 24.4083                 & $-$0.3583                 & 3,  4,  5,  6                            & \nodata                           & 1.22                          & 42.81                             & 42.44                             & $-$0.15                         & 9.0                       & 3.4                       & 19.0                         & 5.19E+21                                & 1                        \\
                    510                     & 02442$+$0176$+$0103          & 24.4250                 & 1.7583                  & 5,  6                                    & \nodata                           & 0.24                          & 10.32                             & 9.61                              & $-$0.38                         & 6.9                       & 1.8                       & 11.5                         & 1.65E+21                                & 1                        \\
                    511                     & 02445$+$0465$+$0055          & 24.4500                 & 4.6500                  & 5,  6                                    & \nodata                           & 0.24                          & 5.55                              & 4.98                              & $-$0.54                         & 5.4                       & 3.8                       & 11.0                         & 2.28E+21                                & 1                        \\
                    512                     & 02446$+$0486$+$0060          & 24.4583                 & 4.8583                  & 6                                        & \nodata                           & 0.24                          & 6.03                              & 5.16                              & $-$1.08                         & 6.3                       & 2.2                       & 11.9                         & 8.72E+20                                & 1                        \\
                    513                     & 02448$-$0005$+$0055          & 24.4833                 & $-$0.0500                 & 6,  7                                    & \nodata                           & 0.24                          & 5.46                              & 5.00                              & $-$0.34                         & 4.8                       & 1.5                       & 8.9                          & 9.24E+20                                & 1                        \\
                    514                     & 02448$-$0069$+$0484          & 24.4833                 & $-$0.6917                 & 5,  6,  7                                & \nodata                           & 3.72                          & 48.41                             & 46.74                             & $-$0.57                         & 11.5                      & 5.4                       & 17.0                         & 1.03E+22                                & 1                        \\
                    515                     & 02451$+$0507$+$0061          & 24.5083                 & 5.0667                  & 6                                        & \nodata                           & 0.24                          & 6.10                              & 5.65                              & $-$0.32                         & 5.8                       & 1.2                       & 10.9                         & 7.64E+20                                & 1                        \\
                    516                     & 02454$+$0017$+$0516          & 24.5417                 & 0.1750                  & 3,  6,  7                                & \nodata                           & 1.38                          & 51.63                             & 51.01                             & $-$0.62                         & 3.1                       & 1.0                       & 6.5                          & 3.61E+21                                & 2                        \\
                    517                     & 02454$+$0056$+$0437          & 24.5417                 & 0.5583                  & 3,  6,  7                                & \nodata                           & 1.40                          & 43.71                             & 43.16                             & $-$0.32                         & 5.7                       & 3.0                       & 11.1                         & 2.65E+21                                & 1                        \\
                    518                     & 02455$+$0418$+$0067          & 24.5500                 & 4.1833                  & 5,  6                                    & \nodata                           & 0.24                          & 6.72                              & 6.23                              & $-$0.51                         & 4.5                       & 1.1                       & 8.0                          & 4.87E+20                                & 1                        \\
                    519                     & 02455$-$0010$+$0445          & 24.5500                 & $-$0.1000                 & 3,  6                                    & \nodata                           & 1.29                          & 44.54                             & 43.50                             & $-$0.36                         & 10.5                      & 4.0                       & 13.6                         & 6.73E+21                                & 1                        \\
                    520                     & 02456$+$0502$+$0062          & 24.5583                 & 5.0167                  & 6                                        & \nodata                           & 0.24                          & 6.19                              & 5.07                              & $-$1.09                         & 5.6                       & 2.2                       & 11.0                         & 1.09E+21                                & 1                        \\
                    521                     & 02457$+$0043$+$1121          & 24.5667                 & 0.4333                  & 3,  6,  7                                & \nodata                           & 6.32                          & 112.08                            & 110.70                            & $-$0.33                         & 9.2                       & 2.7                       & 13.6                         & 6.00E+21                                & 1                        \\
                    522                     & 02457$+$0057$+$0433          & 24.5750                 & 0.5750                  & 1,  3,  6                                & \nodata                           & 1.42                          & 43.34                             & 42.41                             & $-$0.48                         & 5.6                       & 1.7                       & 9.5                          & 1.55E+21                                & 1                        \\
                    523                     & 02457$+$0489$+$0058          & 24.5750                 & 4.8917                  & 6                                        & \nodata                           & 0.24                          & 5.79                              & 5.07                              & $-$0.90                         & 5.7                       & 2.6                       & 10.7                         & 1.05E+21                                & 1                        \\
                    524                     & 02457$+$0506$+$0061          & 24.5750                 & 5.0583                  & 6                                        & \nodata                           & 0.24                          & 6.14                              & 5.90                              & $-$0.20                         & 5.4                       & 2.1                       & 9.6                          & 1.28E+21                                & 1                        \\
                    525                     & 02461$+$0207$+$0093          & 24.6083                 & 2.0667                  & 5,  6                                    & \nodata                           & 0.24                          & 9.34                              & 8.28                              & $-$0.64                         & 6.2                       & 1.9                       & 10.1                         & 1.47E+21                                & 1                        \\
                    526                     & 02461$+$0490$+$0057          & 24.6083                 & 4.9000                  & 1,  6                                    & \nodata                           & 0.24                          & 5.75                              & 5.13                              & $-$0.65                         & 6.4                       & 2.0                       & 11.4                         & 9.35E+20                                & 1                        \\
                    527                     & 02463$+$0490$+$0058          & 24.6333                 & 4.9000                  & 1,  6                                    & \nodata                           & 0.24                          & 5.83                              & 5.01                              & $-$0.94                         & 5.9                       & 2.8                       & 11.2                         & 1.25E+21                                & 1                        \\
                    528                     & 02464$-$0032$+$0429          & 24.6417                 & $-$0.3167                 & 1,  3,  4,  5,  6,  7                    & \nodata                           & 1.21                          & 42.91                             & 42.66                             & $-$0.17                         & 5.0                       & 2.1                       & 8.2                          & 1.19E+22                                & 2                        \\
                    529                     & 02467$+$0490$+$0061          & 24.6667                 & 4.9000                  & 6                                        & \nodata                           & 0.24                          & 6.07                              & 5.09                              & $-$0.92                         & 6.6                       & 2.5                       & 12.9                         & 1.35E+21                                & 1                        \\
                    530                     & 02471$+$0497$+$0058          & 24.7083                 & 4.9750                  & 6                                        & \nodata                           & 0.24                          & 5.80                              & 5.23                              & $-$0.31                         & 6.0                       & 1.5                       & 10.1                         & 1.24E+21                                & 1                        \\
                    531                     & 02475$+$0393$+$0070          & 24.7500                 & 3.9333                  & 5,  6                                    & \nodata                           & 0.24                          & 7.01                              & 6.59                              & $-$0.46                         & 4.6                       & 1.4                       & 8.9                          & 6.00E+20                                & 1                        \\
                    532                     & 02478$+$0063$+$0417          & 24.7833                 & 0.6333                  & 6,  7                                    & \nodata                           & 1.58                          & 41.69                             & 41.23                             & $-$0.39                         & 5.0                       & 2.1                       & 11.2                         & 9.14E+21                                & 2                        \\
                    533                     & 02478$+$0391$+$0070          & 24.7833                 & 3.9083                  & 5,  6                                    & \nodata                           & 0.24                          & 7.03                              & 6.58                              & $-$0.59                         & 4.0                       & 1.2                       & 10.6                         & 4.18E+20                                & 1                        \\
                    534                     & 02511$+$0047$+$0456          & 25.1083                 & 0.4750                  & 5,  6,  7                                & \nodata                           & 1.24                          & 45.59                             & 44.66                             & $-$0.46                         & 7.8                       & 3.6                       & 16.0                         & 4.18E+21                                & 1                        \\
                    535                     & 02512$+$0052$+$0456          & 25.1167                 & 0.5167                  & 5,  6,  7                                & \nodata                           & 1.27                          & 45.59                             & 45.23                             & $-$0.34                         & 4.9                       & 1.9                       & 9.7                          & 9.34E+20                                & 1                        \\
                    536                     & 02514$+$0525$+$0057          & 25.1417                 & 5.2500                  & 6                                        & \nodata                           & 0.24                          & 5.71                              & 4.93                              & $-$0.99                         & 5.1                       & 3.9                       & 8.8                          & 2.18E+21                                & 1                        \\
                    537                     & 02514$-$0030$+$0828          & 25.1417                 & $-$0.3000                 & 5,  6,  7                                & \nodata                           & 3.66                          & 82.75                             & 81.94                             & $-$0.57                         & 6.6                       & 4.0                       & 11.7                         & 3.26E+21                                & 1                        \\
                    538                     & 02516$+$0047$+$0452          & 25.1583                 & 0.4750                  & 5,  6                                    & \nodata                           & 1.30                          & 45.17                             & 44.26                             & $-$0.43                         & 6.2                       & 2.6                       & 11.1                         & 2.74E+21                                & 1                        \\
                    539                     & 02522$+$0429$+$0067          & 25.2167                 & 4.2917                  & 5,  6                                    & \nodata                           & 0.24                          & 6.73                              & 6.20                              & $-$0.56                         & 4.6                       & 1.3                       & 10.6                         & 5.56E+20                                & 1                        \\
                    540                     & 02532$+$0029$+$0417          & 25.3250                 & 0.2917                  & 3,  4,  5,  6,  7                        & \nodata                           & 3.73                          & 41.67                             & 41.27                             & $-$0.21                         & 13.3                      & 4.1                       & 16.8                         & 4.57E+21                                & 1                        \\
                    541                     & 02533$+$0501$+$0039          & 25.3333                 & 5.0083                  & 6                                        & \nodata                           & 0.24                          & 3.91                              & 3.15                              & $-$0.64                         & 5.5                       & 1.3                       & 9.8                          & 6.89E+20                                & 1                        \\
                    542                     & 02537$+$0087$+$0433          & 25.3667                 & 0.8667                  & 5,  6                                    & \nodata                           & 1.45                          & 43.26                             & 42.56                             & $-$0.62                         & 3.8                       & 2.3                       & 9.7                          & 1.28E+21                                & 1                        \\
                    543                     & 02537$+$0378$+$0070          & 25.3750                 & 3.7833                  & 5,  6                                    & \nodata                           & 0.24                          & 6.97                              & 6.38                              & $-$0.55                         & 4.1                       & 1.0                       & 10.0                         & 4.66E+20                                & 1                        \\
                    544                     & 02541$+$0103$+$0435          & 25.4083                 & 1.0333                  & 5,  6                                    & \nodata                           & 1.42                          & 43.53                             & 42.63                             & $-$0.65                         & 6.5                       & 1.8                       & 10.3                         & 1.20E+21                                & 1                        \\
                    545                     & 02541$+$0382$+$0072          & 25.4083                 & 3.8167                  & 5,  6                                    & \nodata                           & 0.24                          & 7.21                              & 6.74                              & $-$0.40                         & 4.5                       & 0.9                       & 8.4                          & 4.98E+20                                & 1                        \\
                    546                     & 02542$+$0164$+$0409          & 25.4250                 & 1.6417                  & 6                                        & \nodata                           & 1.62                          & 40.93                             & 40.08                             & $-$0.78                         & 4.7                       & 1.1                       & 12.8                         & 5.39E+20                                & 1                        \\
                    547                     & 02542$+$0415$+$0068          & 25.4250                 & 4.1500                  & 5,  6                                    & \nodata                           & 0.24                          & 6.80                              & 6.18                              & $-$0.66                         & 3.9                       & 1.3                       & 7.5                          & 5.99E+20                                & 1                        \\
                    548                     & 02544$+$0380$+$0074          & 25.4417                 & 3.8000                  & 5,  6                                    & \nodata                           & 0.24                          & 7.44                              & 6.75                              & $-$0.63                         & 4.8                       & 1.8                       & 10.7                         & 9.23E+20                                & 1                        \\
                    549                     & 02544$+$0412$+$0067          & 25.4417                 & 4.1250                  & 5,  6                                    & \nodata                           & 0.24                          & 6.70                              & 6.07                              & $-$0.65                         & 4.4                       & 1.6                       & 8.6                          & 7.43E+20                                & 1                        \\
                    550                     & 02545$+$0410$+$0065          & 25.4500                 & 4.1000                  & 5,  6                                    & \nodata                           & 0.24                          & 6.50                              & 6.29                              & $-$0.31                         & 4.7                       & 1.5                       & 9.3                          & 4.86E+20                                & 1                        \\
                    551                     & 02545$+$0423$+$0070          & 25.4500                 & 4.2333                  & 5,  6                                    & \nodata                           & 0.24                          & 7.04                              & 6.59                              & $-$0.33                         & 5.7                       & 1.6                       & 9.3                          & 1.00E+21                                & 1                        \\
                    552                     & 02546$+$0090$+$0439          & 25.4583                 & 0.9000                  & 5,  6                                    & \nodata                           & 1.37                          & 43.87                             & 43.51                             & $-$0.43                         & 4.6                       & 1.7                       & 8.0                          & 7.00E+20                                & 1                        \\
                    553                     & 02547$+$0413$+$0068          & 25.4750                 & 4.1333                  & 5,  6                                    & \nodata                           & 0.24                          & 6.76                              & 6.30                              & $-$0.62                         & 3.9                       & 1.7                       & 7.9                          & 6.04E+20                                & 1                        \\
                    554                     & 02549$+$0159$+$0403          & 25.4917                 & 1.5917                  & 6                                        & \nodata                           & 1.69                          & 40.32                             & 39.53                             & $-$0.61                         & 6.3                       & 1.4                       & 13.1                         & 8.47E+20                                & 1                        \\
                    555                     & 02552$-$0035$+$0943          & 25.5250                 & $-$0.3500                 & 5,  6,  7                                & \nodata                           & 4.83                          & 94.28                             & 93.74                             & $-$0.38                         & 7.2                       & 2.0                       & 10.5                         & 1.38E+21                                & 1                        \\
                    556                     & 02553$+$0406$+$0068          & 25.5333                 & 4.0583                  & 5,  6                                    & \nodata                           & 0.24                          & 6.82                              & 6.37                              & $-$0.53                         & 3.9                       & 2.1                       & 7.2                          & 9.87E+20                                & 1                        \\
                    557                     & 02555$+$0095$+$0437          & 25.5500                 & 0.9500                  & 3,  5,  6                                & \nodata                           & 1.38                          & 43.73                             & 43.38                             & $-$0.24                         & 6.6                       & 3.2                       & 10.5                         & 2.51E+21                                & 1                        \\
                    558                     & 02555$+$0297$+$0082          & 25.5500                 & 2.9750                  & 5,  6                                    & \nodata                           & 0.24                          & 8.17                              & 7.26                              & $-$0.60                         & 5.8                       & 1.2                       & 10.3                         & 8.31E+20                                & 1                        \\
                    559                     & 02556$-$0044$+$0615          & 25.5583                 & $-$0.4417                 & 5,  6,  7                                & \nodata                           & 3.89                          & 61.54                             & 60.54                             & $-$0.51                         & 6.9                       & 2.6                       & 11.0                         & 2.52E+21                                & 1                        \\
                    560                     & 02559$+$0412$+$0026          & 25.5917                 & 4.1250                  & 3,  5,  6                                & \nodata                           & 0.24                          & 2.63                              & 1.95                              & $-$0.51                         & 6.1                       & 2.1                       & 13.1                         & 1.37E+21                                & 1                        \\
                    561                     & 02561$+$0457$+$0029          & 25.6083                 & 4.5667                  & 6                                        & \nodata                           & 0.24                          & 2.87                              & 2.56                              & $-$0.30                         & 4.9                       & 1.3                       & 9.8                          & 5.74E+20                                & 1                        \\
                    562                     & 02562$+$0421$+$0025          & 25.6250                 & 4.2083                  & 3,  6                                    & \nodata                           & 0.24                          & 2.47                              & 1.85                              & $-$0.43                         & 5.3                       & 2.2                       & 10.8                         & 1.56E+21                                & 1                        \\
                    563                     & 02563$+$0482$+$0032          & 25.6333                 & 4.8167                  & 6                                        & \nodata                           & 0.24                          & 3.18                              & 2.84                              & $-$0.30                         & 4.9                       & 1.8                       & 10.0                         & 9.78E+20                                & 1                        \\
                    564                     & 02564$+$0413$+$0031          & 25.6417                 & 4.1333                  & 3,  5,  6                                & \nodata                           & 0.24                          & 3.10                              & 2.28                              & $-$0.53                         & 5.8                       & 1.9                       & 14.5                         & 1.45E+21                                & 1                        \\
                    565                     & 02565$-$0017$+$0919          & 25.6500                 & $-$0.1667                 & 5,  6,  7                                & \nodata                           & 5.48                          & 91.90                             & 90.43                             & $-$0.38                         & 15.1                      & 7.7                       & 19.7                         & 2.19E+22                                & 1                        \\
                    566                     & 02567$+$0345$+$0067          & 25.6667                 & 3.4500                  & 5,  6                                    & \nodata                           & 0.24                          & 6.66                              & 6.15                              & $-$0.44                         & 4.9                       & 1.1                       & 9.6                          & 5.83E+20                                & 1                        \\
                    567                     & 02570$+$0280$+$0078          & 25.7000                 & 2.8000                  & 5,  6                                    & \nodata                           & 0.24                          & 7.83                              & 7.40                              & $-$0.41                         & 5.2                       & 1.7                       & 11.5                         & 8.29E+20                                & 1                        \\
                    568                     & 02571$+$0336$+$0038          & 25.7083                 & 3.3583                  & 5,  6                                    & \nodata                           & 0.24                          & 3.85                              & 3.25                              & $-$0.55                         & 4.8                       & 1.1                       & 9.9                          & 5.42E+20                                & 1                        \\
                    569                     & 02571$+$0417$+$0030          & 25.7083                 & 4.1750                  & 3,  6                                    & \nodata                           & 0.24                          & 3.03                              & 2.22                              & $-$0.39                         & 4.7                       & 1.7                       & 12.1                         & 1.69E+21                                & 1                        \\
                    570                     & 02572$+$0275$+$0079          & 25.7167                 & 2.7500                  & 5,  6                                    & \nodata                           & 0.24                          & 7.86                              & 7.41                              & $-$0.55                         & 4.8                       & 1.2                       & 9.9                          & 4.39E+20                                & 1                        \\
                    571                     & 02572$+$0309$+$0088          & 25.7167                 & 3.0917                  & 5,  6                                    & \nodata                           & 0.24                          & 8.78                              & 7.74                              & $-$0.79                         & 5.8                       & 1.5                       & 10.4                         & 8.78E+20                                & 1                        \\
                    572                     & 02572$+$0407$+$0073          & 25.7250                 & 4.0667                  & 3,  6                                    & \nodata                           & 0.24                          & 7.34                              & 6.00                              & $-$1.42                         & 5.5                       & 2.4                       & 9.1                          & 1.19E+21                                & 1                        \\
                    573                     & 02573$+$0402$+$0032          & 25.7333                 & 4.0167                  & 3,  5,  6                                & \nodata                           & 0.24                          & 3.24                              & 2.86                              & $-$0.39                         & 5.5                       & 1.8                       & 11.8                         & 8.31E+20                                & 1                        \\
                    574                     & 02574$+$0270$+$0077          & 25.7417                 & 2.7000                  & 5,  6                                    & \nodata                           & 0.24                          & 7.68                              & 7.22                              & $-$0.51                         & 5.0                       & 1.4                       & 9.1                          & 5.68E+20                                & 1                        \\
                    575                     & 02574$+$0406$+$0032          & 25.7417                 & 4.0583                  & 3,  6                                    & \nodata                           & 0.24                          & 3.19                              & 2.74                              & $-$0.35                         & 5.2                       & 1.4                       & 12.1                         & 8.36E+20                                & 1                        \\
                    576                     & 02574$+$0468$+$0031          & 25.7417                 & 4.6833                  & 1,  6                                    & \nodata                           & 0.24                          & 3.13                              & 2.50                              & $-$0.40                         & 5.2                       & 2.1                       & 10.5                         & 1.66E+21                                & 1                        \\
                    577                     & 02575$+$0410$+$0033          & 25.7500                 & 4.1000                  & 3,  6                                    & \nodata                           & 0.24                          & 3.30                              & 2.68                              & $-$0.44                         & 4.8                       & 1.7                       & 9.3                          & 1.15E+21                                & 1                        \\
                    578                     & 02575$+$0477$+$0025          & 25.7500                 & 4.7750                  & 6                                        & \nodata                           & 0.24                          & 2.54                              & 1.92                              & $-$0.48                         & 4.3                       & 2.1                       & 11.8                         & 1.33E+21                                & 1                        \\
                    579                     & 02577$+$0407$+$0033          & 25.7667                 & 4.0667                  & 3,  6                                    & \nodata                           & 0.24                          & 3.33                              & 2.75                              & $-$0.56                         & 5.3                       & 1.5                       & 17.8                         & 8.29E+20                                & 1                        \\
                    580                     & 02577$+$0478$+$0029          & 25.7750                 & 4.7833                  & 6                                        & \nodata                           & 0.24                          & 2.86                              & 2.05                              & $-$0.58                         & 4.8                       & 1.9                       & 12.0                         & 1.27E+21                                & 1                        \\
                    581                     & 02577$+$0483$+$0035          & 25.7750                 & 4.8333                  & 6                                        & \nodata                           & 0.24                          & 3.47                              & 2.42                              & $-$1.00                         & 4.8                       & 2.0                       & 13.1                         & 1.02E+21                                & 1                        \\
                    582                     & 02578$+$0402$+$0035          & 25.7833                 & 4.0250                  & 3,  6                                    & \nodata                           & 0.24                          & 3.53                              & 3.00                              & $-$0.45                         & 4.0                       & 2.4                       & 10.1                         & 1.39E+21                                & 1                        \\
                    583                     & 02579$+$0338$+$0037          & 25.7917                 & 3.3833                  & 5,  6                                    & \nodata                           & 0.24                          & 3.69                              & 2.98                              & $-$0.54                         & 6.1                       & 2.9                       & 12.6                         & 2.04E+21                                & 1                        \\
                    584                     & 02581$+$0401$+$0040          & 25.8083                 & 4.0083                  & 6                                        & \nodata                           & 0.24                          & 3.97                              & 2.89                              & $-$0.76                         & 4.5                       & 1.6                       & 11.3                         & 1.07E+21                                & 1                        \\
                    585                     & 02581$+$0409$+$0036          & 25.8083                 & 4.0917                  & 3,  6                                    & \nodata                           & 0.24                          & 3.62                              & 2.69                              & $-$0.71                         & 4.5                       & 2.4                       & 12.7                         & 1.53E+21                                & 1                        \\
                    586                     & 02582$+$0406$+$0035          & 25.8250                 & 4.0583                  & 3,  6                                    & \nodata                           & 0.24                          & 3.51                              & 2.82                              & $-$0.56                         & 4.1                       & 1.7                       & 10.6                         & 9.61E+20                                & 1                        \\
                    587                     & 02582$-$0018$+$0934          & 25.8250                 & $-$0.1833                 & 3,  5,  6                                & 17                                & 5.51                          & 93.39                             & 93.14                             & $-$0.05                         & 6.6                       & 6.6                       & 10.4                         & 3.63E+22                                & 1                        \\
                    588                     & 02583$+$0086$+$0496          & 25.8333                 & 0.8583                  & 5,  6,  7                                & \nodata                           & 0.98                          & 49.60                             & 48.68                             & $-$0.48                         & 6.9                       & 2.9                       & 11.4                         & 2.93E+21                                & 1                        \\
                    589                     & 02583$+$0333$+$0036          & 25.8333                 & 3.3333                  & 5,  6                                    & \nodata                           & 0.24                          & 3.60                              & 3.10                              & $-$0.47                         & 5.3                       & 2.8                       & 9.9                          & 1.55E+21                                & 1                        \\
                    590                     & 02584$+$0336$+$0035          & 25.8417                 & 3.3583                  & 5,  6                                    & \nodata                           & 0.24                          & 3.51                              & 2.97                              & $-$0.55                         & 5.4                       & 2.7                       & 11.1                         & 1.34E+21                                & 1                        \\
                    591                     & 02585$+$0469$+$0028          & 25.8500                 & 4.6917                  & 6                                        & \nodata                           & 0.24                          & 2.82                              & 2.13                              & $-$0.54                         & 5.3                       & 2.5                       & 11.3                         & 1.57E+21                                & 1                        \\
                    592                     & 02587$+$0419$+$0032          & 25.8667                 & 4.1917                  & 3,  6                                    & \nodata                           & 0.24                          & 3.15                              & 2.67                              & $-$0.39                         & 5.4                       & 2.5                       & 11.1                         & 1.54E+21                                & 1                        \\
                    593                     & 02588$+$0392$+$0034          & 25.8833                 & 3.9167                  & 6                                        & \nodata                           & 0.24                          & 3.38                              & 2.88                              & $-$0.45                         & 4.6                       & 1.8                       & 8.8                          & 9.40E+20                                & 1                        \\
                    594                     & 02588$+$0472$+$0033          & 25.8833                 & 4.7250                  & 6                                        & \nodata                           & 0.24                          & 3.31                              & 2.59                              & $-$0.44                         & 5.6                       & 1.5                       & 12.1                         & 1.13E+21                                & 1                        \\
                    595                     & 02589$+$0334$+$0036          & 25.8917                 & 3.3417                  & 5,  6                                    & \nodata                           & 0.24                          & 3.59                              & 3.02                              & $-$0.41                         & 5.2                       & 1.6                       & 8.9                          & 1.07E+21                                & 1                        \\
                    596                     & 02601$+$0459$+$0034          & 26.0083                 & 4.5917                  & 6                                        & \nodata                           & 0.24                          & 3.42                              & 2.59                              & $-$0.43                         & 5.6                       & 4.0                       & 12.8                         & 4.29E+21                                & 1                        \\
                    597                     & 02604$+$0347$+$0033          & 26.0417                 & 3.4750                  & 5,  6                                    & \nodata                           & 0.24                          & 3.32                              & 2.59                              & $-$0.44                         & 6.7                       & 2.4                       & 17.3                         & 2.29E+21                                & 1                        \\
                    598                     & 02609$+$0416$+$0033          & 26.0917                 & 4.1583                  & 6                                        & \nodata                           & 0.24                          & 3.28                              & 2.77                              & $-$0.52                         & 5.3                       & 2.8                       & 11.0                         & 1.38E+21                                & 1                        \\
                    599                     & 02610$+$0429$+$0037          & 26.1000                 & 4.2917                  & 6                                        & \nodata                           & 0.24                          & 3.69                              & 2.99                              & $-$0.40                         & 4.9                       & 2.5                       & 11.4                         & 2.15E+21                                & 1                        \\
                    600                     & 02612$+$0127$+$0439          & 26.1167                 & 1.2750                  & 5,  6                                    & \nodata                           & 1.36                          & 43.89                             & 43.19                             & $-$0.39                         & 10.0                      & 4.5                       & 15.2                         & 4.83E+21                                & 1                        \\
                    601                     & 02612$+$0134$+$0436          & 26.1167                 & 1.3417                  & 5,  6                                    & \nodata                           & 1.41                          & 43.60                             & 43.36                             & $-$0.24                         & 5.0                       & 2.5                       & 8.3                          & 1.36E+21                                & 1                        \\
                    602                     & 02612$+$0306$+$0032          & 26.1167                 & 3.0583                  & 5,  6                                    & \nodata                           & 0.24                          & 3.17                              & 2.54                              & $-$0.44                         & 7.1                       & 3.0                       & 13.4                         & 2.35E+21                                & 1                        \\
                    603                     & 02612$+$0308$+$0034          & 26.1250                 & 3.0833                  & 5,  6                                    & \nodata                           & 0.24                          & 3.37                              & 2.23                              & $-$0.75                         & 6.2                       & 2.6                       & 14.8                         & 2.12E+21                                & 1                        \\
                    604                     & 02612$+$0470$+$0050          & 26.1250                 & 4.7000                  & 6                                        & \nodata                           & 0.24                          & 5.01                              & 4.26                              & $-$0.54                         & 4.6                       & 1.7                       & 9.5                          & 1.10E+21                                & 1                        \\
                    605                     & 02614$+$0312$+$0034          & 26.1417                 & 3.1250                  & 5,  6                                    & \nodata                           & 0.24                          & 3.38                              & 2.38                              & $-$0.76                         & 5.9                       & 2.2                       & 13.2                         & 1.45E+21                                & 1                        \\
                    606                     & 02616$+$0317$+$0030          & 26.1583                 & 3.1750                  & 5,  6                                    & \nodata                           & 0.24                          & 2.99                              & 2.26                              & $-$0.50                         & 5.2                       & 2.5                       & 10.7                         & 1.76E+21                                & 1                        \\
                    607                     & 02616$+$0449$+$0044          & 26.1583                 & 4.4917                  & 6                                        & \nodata                           & 0.24                          & 4.45                              & 3.45                              & $-$0.65                         & 4.8                       & 2.2                       & 10.2                         & 1.68E+21                                & 1                        \\
                    608                     & 02617$+$0445$+$0040          & 26.1750                 & 4.4500                  & 6                                        & \nodata                           & 0.24                          & 4.04                              & 2.94                              & $-$0.58                         & 5.2                       & 2.8                       & 11.8                         & 2.71E+21                                & 1                        \\
                    609                     & 02618$+$0447$+$0042          & 26.1833                 & 4.4750                  & 6                                        & \nodata                           & 0.24                          & 4.24                              & 3.25                              & $-$0.58                         & 4.9                       & 2.6                       & 11.3                         & 2.21E+21                                & 1                        \\
                    610                     & 02620$+$0196$+$0440          & 26.2000                 & 1.9583                  & 5,  6                                    & \nodata                           & 1.40                          & 44.01                             & 43.57                             & $-$0.38                         & 7.3                       & 2.6                       & 12.6                         & 1.52E+21                                & 1                        \\
                    611                     & 02621$+$0357$+$0034          & 26.2083                 & 3.5667                  & 6                                        & \nodata                           & 0.24                          & 3.44                              & 2.84                              & $-$0.41                         & 4.8                       & 2.2                       & 9.7                          & 1.58E+21                                & 1                        \\
                    612                     & 02622$+$0438$+$0039          & 26.2250                 & 4.3833                  & 6                                        & \nodata                           & 0.24                          & 3.95                              & 2.84                              & $-$0.87                         & 5.7                       & 3.0                       & 15.6                         & 2.12E+21                                & 1                        \\
                    613                     & 02625$+$0440$+$0040          & 26.2500                 & 4.4000                  & 6                                        & \nodata                           & 0.24                          & 4.00                              & 2.96                              & $-$0.61                         & 5.8                       & 3.3                       & 11.7                         & 3.06E+21                                & 1                        \\
                    614                     & 02626$+$0458$+$0051          & 26.2583                 & 4.5833                  & 6                                        & \nodata                           & 0.24                          & 5.08                              & 4.04                              & $-$0.97                         & 3.9                       & 2.0                       & 8.0                          & 1.10E+21                                & 1                        \\
                    615                     & 02627$+$0146$+$0429          & 26.2750                 & 1.4583                  & 1,  3,  5,  6                            & \nodata                           & 1.47                          & 42.93                             & 42.23                             & $-$0.32                         & 8.5                       & 1.4                       & 13.4                         & 1.17E+22                                & 2                        \\
                    616                     & 02630$+$0447$+$0044          & 26.3000                 & 4.4750                  & 6                                        & \nodata                           & 0.24                          & 4.44                              & 3.73                              & $-$0.48                         & 4.9                       & 2.7                       & 7.9                          & 2.28E+21                                & 1                        \\
                    617                     & 02630$+$0474$+$0058          & 26.3000                 & 4.7417                  & 6                                        & \nodata                           & 0.24                          & 5.82                              & 5.26                              & $-$0.55                         & 3.4                       & 1.6                       & 7.4                          & 8.23E+20                                & 1                        \\
                    618                     & 02632$+$0442$+$0041          & 26.3250                 & 4.4167                  & 6                                        & \nodata                           & 0.24                          & 4.14                              & 3.55                              & $-$0.42                         & 5.0                       & 2.5                       & 11.3                         & 1.79E+21                                & 1                        \\
                    619                     & 02632$-$0007$+$1005          & 26.3250                 & $-$0.0750                 & 1,  3,  5,  6                            & 17                                & 5.87                          & 100.45                            & 99.49                             & $-$0.33                         & 11.3                      & 5.7                       & 16.6                         & 1.09E+22                                & 1                        \\
                    620                     & 02636$+$0437$+$0041          & 26.3583                 & 4.3750                  & 1,  5,  6                                & \nodata                           & 0.24                          & 4.12                              & 3.00                              & $-$0.77                         & 4.6                       & 2.6                       & 10.2                         & 1.97E+21                                & 1                        \\
                    621                     & 02641$+$0439$+$0044          & 26.4083                 & 4.3917                  & 5,  6                                    & \nodata                           & 0.24                          & 4.37                              & 3.08                              & $-$0.73                         & 5.1                       & 2.1                       & 11.9                         & 1.75E+21                                & 1                        \\
                    622                     & 02643$+$0440$+$0046          & 26.4333                 & 4.4000                  & 5,  6                                    & \nodata                           & 0.24                          & 4.55                              & 3.14                              & $-$0.84                         & 4.2                       & 1.8                       & 8.5                          & 1.51E+21                                & 1                        \\
                    623                     & 02644$+$0136$+$0438          & 26.4417                 & 1.3583                  & 3,  5,  6                                & \nodata                           & 1.38                          & 43.79                             & 43.45                             & $-$0.23                         & 13.5                      & 6.2                       & 19.3                         & 6.29E+21                                & 1                        \\
                    624                     & 02647$+$0132$+$0446          & 26.4667                 & 1.3167                  & 3,  5,  6                                & \nodata                           & 1.33                          & 44.58                             & 43.96                             & $-$0.33                         & 7.1                       & 1.4                       & 13.3                         & 1.25E+21                                & 1                        \\
                    625                     & 02647$+$0364$+$0042          & 26.4667                 & 3.6417                  & 5,  6                                    & \nodata                           & 0.24                          & 4.16                              & 3.02                              & $-$0.53                         & 6.7                       & 2.7                       & 15.2                         & 3.15E+21                                & 1                        \\
                    626                     & 02649$+$0106$+$0436          & 26.4917                 & 1.0583                  & 5,  6                                    & \nodata                           & 1.41                          & 43.64                             & 42.91                             & $-$0.71                         & 4.5                       & 2.0                       & 8.9                          & 9.95E+20                                & 1                        \\
                    627                     & 02649$+$0113$+$0447          & 26.4917                 & 1.1333                  & 5,  6                                    & \nodata                           & 1.31                          & 44.72                             & 44.49                             & $-$0.38                         & 1.2                       & 0.5                       & 4.3                          & 1.74E+21                                & 2                        \\
                    628                     & 02652$+$0205$+$0429          & 26.5250                 & 2.0500                  & 5,  6                                    & \nodata                           & 1.49                          & 42.86                             & 42.52                             & $-$0.28                         & 5.6                       & 2.3                       & 10.1                         & 1.37E+21                                & 1                        \\
                    629                     & 02655$+$0273$+$0059          & 26.5500                 & 2.7333                  & 6                                        & \nodata                           & 0.24                          & 5.85                              & 5.28                              & $-$0.49                         & 5.8                       & 3.8                       & 9.5                          & 2.75E+21                                & 1                        \\
                    630                     & 02657$+$0262$+$0054          & 26.5750                 & 2.6167                  & 6                                        & \nodata                           & 0.24                          & 5.43                              & 4.97                              & $-$0.46                         & 4.3                       & 4.0                       & 7.6                          & 4.25E+21                                & 1                        \\
                    631                     & 02662$+$0255$+$0052          & 26.6167                 & 2.5500                  & 6                                        & \nodata                           & 0.24                          & 5.20                              & 4.97                              & $-$0.23                         & 3.5                       & 2.5                       & 9.0                          & 1.30E+21                                & 1                        \\
                    632                     & 02663$+$0227$+$0061          & 26.6333                 & 2.2667                  & 5,  6                                    & \nodata                           & 0.24                          & 6.07                              & 5.55                              & $-$0.42                         & 5.7                       & 3.4                       & 11.6                         & 2.33E+21                                & 1                        \\
                    633                     & 02665$+$0220$+$0063          & 26.6500                 & 2.2000                  & 5,  6                                    & \nodata                           & 0.24                          & 6.34                              & 5.90                              & $-$0.46                         & 4.7                       & 2.5                       & 9.0                          & 1.26E+21                                & 1                        \\
                    634                     & 02666$+$0321$+$0044          & 26.6583                 & 3.2083                  & 6                                        & \nodata                           & 0.24                          & 4.41                              & 3.66                              & $-$0.53                         & 5.0                       & 2.7                       & 8.9                          & 2.09E+21                                & 1                        \\
                    635                     & 02667$+$0387$+$0033          & 26.6667                 & 3.8750                  & 5,  6                                    & \nodata                           & 0.24                          & 3.28                              & 2.36                              & $-$0.65                         & 6.6                       & 3.5                       & 13.8                         & 2.74E+21                                & 1                        \\
                    636                     & 02668$+$0258$+$0054          & 26.6833                 & 2.5833                  & 6                                        & \nodata                           & 0.24                          & 5.38                              & 4.56                              & $-$0.59                         & 3.7                       & 2.0                       & 8.3                          & 1.36E+21                                & 1                        \\
                    637                     & 02670$+$0221$+$0062          & 26.7000                 & 2.2083                  & 5,  6                                    & \nodata                           & 0.24                          & 6.21                              & 5.67                              & $-$0.38                         & 5.3                       & 2.4                       & 8.1                          & 1.89E+21                                & 1                        \\
                    638                     & 02672$+$0266$+$0057          & 26.7167                 & 2.6583                  & 6                                        & \nodata                           & 0.24                          & 5.67                              & 4.62                              & $-$0.73                         & 6.0                       & 4.3                       & 16.9                         & 3.80E+21                                & 1                        \\
                    639                     & 02672$+$0263$+$0057          & 26.7250                 & 2.6333                  & 6                                        & \nodata                           & 0.24                          & 5.65                              & 4.93                              & $-$0.67                         & 4.8                       & 4.7                       & 8.4                          & 5.51E+21                                & 1                        \\
                    640                     & 02674$+$0258$+$0057          & 26.7417                 & 2.5833                  & 6                                        & \nodata                           & 0.24                          & 5.74                              & 5.32                              & $-$0.34                         & 4.3                       & 4.3                       & 7.7                          & 7.31E+21                                & 1                        \\
                    641                     & 02676$+$0299$+$0058          & 26.7583                 & 2.9917                  & 6                                        & \nodata                           & 0.24                          & 5.77                              & 5.28                              & $-$0.50                         & 4.3                       & 1.6                       & 8.5                          & 7.34E+20                                & 1                        \\
                    642                     & 02679$+$0227$+$0060          & 26.7917                 & 2.2667                  & 5,  6                                    & \nodata                           & 0.24                          & 5.99                              & 4.94                              & $-$0.90                         & 4.9                       & 2.4                       & 11.2                         & 1.41E+21                                & 1                        \\
                    643                     & 02681$+$0222$+$0062          & 26.8083                 & 2.2250                  & 5,  6                                    & \nodata                           & 0.24                          & 6.20                              & 5.01                              & $-$1.31                         & 4.7                       & 1.9                       & 13.8                         & 8.41E+20                                & 1                        \\
                    644                     & 02682$+$0231$+$0063          & 26.8167                 & 2.3083                  & 5,  6                                    & \nodata                           & 0.24                          & 6.31                              & 5.43                              & $-$0.60                         & 5.0                       & 2.9                       & 10.5                         & 2.26E+21                                & 1                        \\
                    645                     & 02682$-$0009$+$0514          & 26.8167                 & $-$0.0917                 & 5,  6                                    & \nodata                           & 1.01                          & 51.38                             & 50.86                             & $-$0.78                         & 3.8                       & 1.0                       & 7.4                          & 3.19E+20                                & 1                        \\
                    646                     & 02684$+$0227$+$0059          & 26.8417                 & 2.2750                  & 5,  6                                    & \nodata                           & 0.24                          & 5.87                              & 5.43                              & $-$0.38                         & 4.3                       & 2.1                       & 7.7                          & 1.28E+21                                & 1                        \\
                    647                     & 02690$+$0406$+$0037          & 26.9000                 & 4.0583                  & 5,  6                                    & \nodata                           & 0.24                          & 3.71                              & 3.30                              & $-$0.32                         & 7.6                       & 2.1                       & 10.9                         & 1.31E+21                                & 1                        \\
                    648                     & 02693$+$0344$+$0068          & 26.9333                 & 3.4417                  & 3,  6                                    & \nodata                           & 0.24                          & 6.80                              & 6.09                              & $-$0.73                         & 2.8                       & 1.3                       & 6.0                          & 5.40E+21                                & 2                        \\
                    649                     & 02693$+$0432$+$0097          & 26.9333                 & 4.3250                  & 5,  6                                    & \nodata                           & 0.24                          & 9.74                              & 9.27                              & $-$0.37                         & 6.8                       & 2.4                       & 10.6                         & 1.49E+21                                & 1                        \\
                    650                     & 02697$+$0048$+$0182          & 26.9750                 & 0.4833                  & 5,  6,  7                                & \nodata                           & 0.24                          & 18.23                             & 17.51                             & $-$0.52                         & 4.5                       & 2.5                       & 8.6                          & 1.88E+21                                & 1                        \\
                    651                     & 02707$+$0283$+$0052          & 27.0667                 & 2.8333                  & 6                                        & \nodata                           & 0.24                          & 5.25                              & 3.86                              & $-$0.66                         & 6.8                       & 1.8                       & 15.0                         & 1.95E+21                                & 1                        \\
                    652                     & 02726$+$0131$+$0040          & 27.2583                 & 1.3083                  & 5,  6                                    & \nodata                           & 0.24                          & 4.01                              & 3.32                              & $-$0.84                         & 4.4                       & 1.3                       & 9.2                          & 4.76E+20                                & 1                        \\
                    653                     & 02729$+$0186$+$0072          & 27.2917                 & 1.8583                  & 5,  6                                    & \nodata                           & 0.24                          & 7.24                              & 6.60                              & $-$0.39                         & 6.8                       & 1.7                       & 11.3                         & 1.33E+21                                & 1                        \\
                    654                     & 02730$+$0014$+$0314          & 27.3000                 & 0.1417                  & 3,  5,  6,  7                            & \nodata                           & 1.92                          & 31.41                             & 30.31                             & $-$1.09                         & 4.3                       & 1.3                       & 8.3                          & 4.44E+21                                & 2                        \\
                    655                     & 02772$+$0360$+$0087          & 27.7250                 & 3.6000                  & 5,  6                                    & \nodata                           & 0.24                          & 8.74                              & 8.23                              & $-$0.35                         & 5.8                       & 2.6                       & 9.2                          & 2.02E+21                                & 1                        \\
                    656                     & 02773$+$0066$+$0064          & 27.7333                 & 0.6583                  & 5,  6                                    & \nodata                           & 0.24                          & 6.41                              & 5.28                              & $-$0.99                         & 4.6                       & 1.5                       & 10.1                         & 7.79E+20                                & 1                        \\
                    657                     & 02776$-$0290$+$0205          & 27.7583                 & $-$2.9000                 & 5,  6                                    & \nodata                           & 0.41                          & 20.49                             & 19.91                             & $-$0.39                         & 6.1                       & 1.1                       & 10.7                         & 7.46E+20                                & 1                        \\
                    658                     & 02782$-$0033$+$0460          & 27.8167                 & $-$0.3333                 & 5,  6,  7                                & \nodata                           & 4.05                          & 46.03                             & 44.31                             & $-$1.06                         & 5.1                       & 3.4                       & 11.5                         & 2.95E+21                                & 1                        \\
                    659                     & 02785$+$0437$+$0058          & 27.8500                 & 4.3667                  & 5,  6                                    & \nodata                           & 0.24                          & 5.82                              & 4.66                              & $-$1.05                         & 4.2                       & 1.2                       & 9.5                          & 5.89E+20                                & 1                        \\
                    660                     & 02786$+$0151$+$0057          & 27.8583                 & 1.5083                  & 5,  6                                    & \nodata                           & 0.24                          & 5.70                              & 4.95                              & $-$0.85                         & 3.6                       & 1.1                       & 8.2                          & 4.55E+20                                & 1                        \\
                    661                     & 02795$+$0490$+$0080          & 27.9500                 & 4.9000                  & 5,  6                                    & \nodata                           & 0.24                          & 7.98                              & 7.23                              & $-$0.89                         & 4.4                       & 1.7                       & 9.2                          & 6.65E+20                                & 1                        \\
                    662                     & 02799$+$0165$+$0069          & 27.9917                 & 1.6500                  & 5,  6                                    & \nodata                           & 0.24                          & 6.85                              & 6.57                              & $-$0.32                         & 4.3                       & 1.7                       & 9.5                          & 7.05E+20                                & 1                        \\
                    663                     & 02799$-$0113$+$0029          & 27.9917                 & $-$1.1333                 & 5,  6                                    & \nodata                           & 0.24                          & 2.90                              & 2.45                              & $-$0.32                         & 5.1                       & 1.3                       & 10.2                         & 8.26E+20                                & 1                        \\
                    664                     & 02805$+$0130$+$0066          & 28.0500                 & 1.3000                  & 5,  6                                    & \nodata                           & 0.24                          & 6.63                              & 5.97                              & $-$0.72                         & 2.8                       & 1.6                       & 7.7                          & 6.98E+20                                & 1                        \\
                    665                     & 02809$-$0152$+$0036          & 28.0917                 & $-$1.5167                 & 6                                        & \nodata                           & 0.24                          & 3.56                              & 3.05                              & $-$0.39                         & 6.3                       & 1.8                       & 12.3                         & 1.10E+21                                & 1                        \\
                    666                     & 02812$+$0130$+$0066          & 28.1167                 & 1.3000                  & 5,  6                                    & \nodata                           & 0.24                          & 6.58                              & 6.19                              & $-$0.52                         & 3.5                       & 1.4                       & 7.8                          & 4.79E+20                                & 1                        \\
                    667                     & 02812$-$0152$+$0036          & 28.1167                 & $-$1.5167                 & 6                                        & \nodata                           & 0.24                          & 3.62                              & 3.08                              & $-$0.38                         & 6.4                       & 1.8                       & 11.6                         & 1.18E+21                                & 1                        \\
                    668                     & 02814$-$0002$+$0972          & 28.1417                 & $-$0.0167                 & 3,  5,  6                                & \nodata                           & 5.75                          & 97.22                             & 95.67                             & $-$0.42                         & 9.4                       & 4.0                       & 14.0                         & 8.47E+21                                & 1                        \\
                    669                     & 02816$+$0037$+$0072          & 28.1583                 & 0.3667                  & 5,  6                                    & \nodata                           & 1.78                          & 7.15                              & 6.26                              & $-$0.65                         & 4.4                       & 1.5                       & 8.1                          & 9.54E+20                                & 1                        \\
                    670                     & 02818$+$0408$+$0063          & 28.1833                 & 4.0833                  & 5,  6                                    & \nodata                           & 0.24                          & 6.31                              & 5.00                              & $-$0.97                         & 4.3                       & 1.9                       & 11.3                         & 1.19E+21                                & 1                        \\
                    671                     & 02820$-$0007$+$0972          & 28.2000                 & $-$0.0667                 & 3,  5,  6,  7                            & \nodata                           & 5.72                          & 97.19                             & 95.64                             & $-$0.36                         & 12.5                      & 6.6                       & 16.9                         & 1.95E+22                                & 1                        \\
                    672                     & 02824$+$0008$+$1074          & 28.2417                 & 0.0833                  & 3,  5,  6,  7                            & \nodata                           & 5.96                          & 107.42                            & 105.57                            & $-$0.42                         & 10.8                      & 4.3                       & 17.5                         & 1.16E+22                                & 1                        \\
                    673                     & 02824$+$0278$+$0054          & 28.2417                 & 2.7833                  & 5,  6                                    & \nodata                           & 0.24                          & 5.41                              & 4.83                              & $-$0.51                         & 6.1                       & 4.5                       & 11.2                         & 3.17E+21                                & 1                        \\
                    674                     & 02826$-$0177$+$0036          & 28.2583                 & $-$1.7667                 & 5,  6                                    & \nodata                           & 0.24                          & 3.65                              & 3.24                              & $-$0.40                         & 7.4                       & 2.9                       & 19.0                         & 1.72E+21                                & 1                        \\
                    675                     & 02827$+$0274$+$0059          & 28.2667                 & 2.7417                  & 5,  6                                    & \nodata                           & 0.24                          & 5.94                              & 5.11                              & $-$0.71                         & 6.7                       & 4.0                       & 11.8                         & 2.73E+21                                & 1                        \\
                    676                     & 02828$-$0036$+$0478          & 28.2833                 & $-$0.3583                 & 3,  6,  7                                & \nodata                           & 4.16                          & 47.83                             & 46.98                             & $-$0.26                         & 11.0                      & 7.4                       & 14.1                         & 1.92E+22                                & 1                        \\
                    677                     & 02829$-$0063$+$0056          & 28.2917                 & $-$0.6333                 & 5,  6                                    & \nodata                           & 0.24                          & 5.60                              & 5.08                              & $-$0.36                         & 6.0                       & 1.9                       & 13.0                         & 1.29E+21                                & 1                        \\
                    678                     & 02830$+$0283$+$0061          & 28.3000                 & 2.8333                  & 5,  6                                    & \nodata                           & 0.24                          & 6.11                              & 5.65                              & $-$0.39                         & 6.7                       & 4.9                       & 13.1                         & 3.55E+21                                & 1                        \\
                    679                     & 02830$+$0287$+$0060          & 28.3000                 & 2.8750                  & 5,  6                                    & 18                                & 0.24                          & 6.01                              & 5.43                              & $-$0.43                         & 6.2                       & 4.5                       & 11.5                         & 3.62E+21                                & 1                        \\
                    680                     & 02831$+$0255$+$0066          & 28.3083                 & 2.5500                  & 5,  6                                    & \nodata                           & 0.24                          & 6.64                              & 5.80                              & $-$0.75                         & 6.3                       & 3.3                       & 13.5                         & 2.04E+21                                & 1                        \\
                    681                     & 02832$-$0191$+$0034          & 28.3167                 & $-$1.9083                 & 5,  6                                    & \nodata                           & 0.24                          & 3.43                              & 2.91                              & $-$0.40                         & 7.5                       & 3.4                       & 14.2                         & 2.43E+21                                & 1                        \\
                    682                     & 02832$+$0286$+$0063          & 28.3250                 & 2.8583                  & 5,  6                                    & 18                                & 0.24                          & 6.31                              & 6.00                              & $-$0.43                         & 4.9                       & 1.3                       & 8.5                          & 3.17E+21                                & 2                        \\
                    683                     & 02833$+$0248$+$0063          & 28.3333                 & 2.4833                  & 5,  6                                    & \nodata                           & 0.24                          & 6.32                              & 5.78                              & $-$0.39                         & 6.8                       & 1.8                       & 13.1                         & 1.22E+21                                & 1                        \\
                    684                     & 02834$+$0253$+$0066          & 28.3417                 & 2.5333                  & 5,  6                                    & \nodata                           & 0.24                          & 6.57                              & 5.96                              & $-$0.64                         & 5.1                       & 4.6                       & 13.3                         & 2.63E+21                                & 1                        \\
                    685                     & 02840$+$0282$+$0066          & 28.4000                 & 2.8250                  & 5,  6                                    & \nodata                           & 0.24                          & 6.61                              & 5.48                              & $-$0.63                         & 5.9                       & 4.0                       & 10.5                         & 4.24E+21                                & 1                        \\
                    686                     & 02841$-$0043$+$0738          & 28.4083                 & $-$0.4333                 & 3,  6,  7                                & \nodata                           & 4.30                          & 73.77                             & 72.81                             & $-$0.35                         & 9.1                       & 4.1                       & 13.5                         & 6.41E+21                                & 1                        \\
                    687                     & 02842$-$0160$+$0037          & 28.4167                 & $-$1.6000                 & 5,  6                                    & \nodata                           & 0.24                          & 3.72                              & 3.15                              & $-$0.58                         & 6.2                       & 2.3                       & 13.1                         & 1.15E+21                                & 1                        \\
                    688                     & 02842$+$0262$+$0069          & 28.4250                 & 2.6250                  & 5,  6                                    & \nodata                           & 0.24                          & 6.94                              & 6.24                              & $-$0.52                         & 5.5                       & 3.5                       & 9.4                          & 2.82E+21                                & 1                        \\
                    689                     & 02842$-$0165$+$0036          & 28.4250                 & $-$1.6500                 & 5,  6                                    & \nodata                           & 0.24                          & 3.62                              & 3.07                              & $-$0.40                         & 6.4                       & 2.7                       & 13.6                         & 1.91E+21                                & 1                        \\
                    690                     & 02843$+$0403$+$0174          & 28.4333                 & 4.0333                  & 5,  6                                    & \nodata                           & 0.24                          & 17.35                             & 16.86                             & $-$0.53                         & 4.1                       & 1.5                       & 8.5                          & 6.25E+20                                & 1                        \\
                    691                     & 02844$+$0427$+$0066          & 28.4417                 & 4.2667                  & 5,  6                                    & \nodata                           & 0.24                          & 6.60                              & 4.97                              & $-$0.44                         & 4.9                       & 1.5                       & 7.0                          & 2.82E+21                                & 1                        \\
                    692                     & 02848$+$0406$+$0066          & 28.4833                 & 4.0583                  & 5,  6                                    & \nodata                           & 0.24                          & 6.61                              & 5.23                              & $-$0.87                         & 4.3                       & 3.2                       & 8.6                          & 3.00E+21                                & 1                        \\
                    693                     & 02849$+$0379$+$0066          & 28.4917                 & 3.7917                  & 5,  6                                    & \nodata                           & 0.24                          & 6.64                              & 5.98                              & $-$0.66                         & 3.7                       & 2.3                       & 9.7                          & 8.74E+21                                & 2                        \\
                    694                     & 02849$+$0399$+$0064          & 28.4917                 & 3.9917                  & 5,  6                                    & \nodata                           & 0.24                          & 6.37                              & 5.70                              & $-$0.62                         & 7.1                       & 4.2                       & 13.6                         & 2.02E+22                                & 2                        \\
                    695                     & 02851$+$0342$+$0071          & 28.5083                 & 3.4250                  & 5,  6                                    & \nodata                           & 0.24                          & 7.06                              & 5.89                              & $-$1.01                         & 2.7                       & 1.4                       & 7.0                          & 6.41E+21                                & 2                        \\
                    696                     & 02851$-$0193$+$0033          & 28.5083                 & $-$1.9333                 & 5,  6                                    & \nodata                           & 0.24                          & 3.34                              & 2.79                              & $-$0.31                         & 6.1                       & 2.4                       & 15.7                         & 2.36E+21                                & 1                        \\
                    697                     & 02852$+$0399$+$0059          & 28.5250                 & 3.9917                  & 5,  6                                    & \nodata                           & 0.24                          & 5.88                              & 5.26                              & $-$0.50                         & 6.2                       & 10.1                      & 10.6                         & \nodata                                 & 1                        \\
                    698                     & 02852$-$0166$+$0037          & 28.5250                 & $-$1.6583                 & 5,  6                                    & \nodata                           & 0.24                          & 3.72                              & 3.20                              & $-$0.28                         & 5.9                       & 2.0                       & 14.5                         & 1.93E+21                                & 1                        \\
                    699                     & 02853$+$0341$+$0071          & 28.5333                 & 3.4083                  & 5,  6                                    & \nodata                           & 0.24                          & 7.06                              & 6.30                              & $-$0.82                         & 2.5                       & 1.8                       & 6.7                          & 7.34E+21                                & 2                        \\
                    700                     & 02853$+$0357$+$0075          & 28.5333                 & 3.5667                  & 5,  6,  7                                & \nodata                           & 0.24                          & 7.49                              & 6.12                              & $-$1.39                         & 3.3                       & 1.6                       & 7.6                          & 5.83E+21                                & 2                        \\
                    701                     & 02854$+$0374$+$0069          & 28.5417                 & 3.7417                  & 5,  6                                    & \nodata                           & 0.24                          & 6.91                              & 5.86                              & $-$0.72                         & 4.2                       & 1.9                       & 8.7                          & 1.05E+22                                & 2                        \\
                    702                     & 02856$+$0383$+$0069          & 28.5583                 & 3.8333                  & 5,  6                                    & \nodata                           & 0.24                          & 6.90                              & 6.35                              & $-$0.38                         & 3.0                       & 2.2                       & 8.1                          & 1.25E+22                                & 2                        \\
                    703                     & 02857$+$0357$+$0073          & 28.5667                 & 3.5667                  & 5,  6                                    & \nodata                           & 0.24                          & 7.27                              & 6.61                              & $-$0.51                         & 3.1                       & 1.5                       & 6.0                          & 8.77E+21                                & 2                        \\
                    704                     & 02857$+$0375$+$0067          & 28.5667                 & 3.7500                  & 5,  6                                    & \nodata                           & 0.24                          & 6.66                              & 5.95                              & $-$0.62                         & 4.6                       & 1.6                       & 9.2                          & 6.28E+21                                & 2                        \\
                    705                     & 02857$+$0454$+$0070          & 28.5667                 & 4.5417                  & 5,  6                                    & \nodata                           & 0.24                          & 7.00                              & 6.53                              & $-$0.40                         & 3.7                       & 1.8                       & 11.4                         & 9.78E+20                                & 1                        \\
                    706                     & 02857$+$0342$+$0073          & 28.5750                 & 3.4167                  & 5,  6                                    & \nodata                           & 0.24                          & 7.31                              & 6.04                              & $-$1.30                         & 2.8                       & 1.4                       & 7.1                          & 5.38E+21                                & 2                        \\
                    707                     & 02859$+$0427$+$0078          & 28.5917                 & 4.2667                  & 5,  6                                    & \nodata                           & 0.24                          & 7.76                              & 7.17                              & $-$0.94                         & 3.4                       & 1.5                       & 7.1                          & 3.64E+21                                & 2                        \\
                    708                     & 02860$+$0320$+$0070          & 28.6000                 & 3.2000                  & 5,  6                                    & \nodata                           & 0.24                          & 7.03                              & 6.64                              & $-$0.44                         & 3.1                       & 1.4                       & 7.4                          & 4.47E+21                                & 2                        \\
                    709                     & 02860$+$0364$+$0071          & 28.6000                 & 3.6417                  & 5,  6                                    & \nodata                           & 0.24                          & 7.07                              & 5.84                              & $-$1.29                         & 3.1                       & 2.6                       & 6.5                          & 1.57E+22                                & 2                        \\
                    710                     & 02860$+$0462$+$0071          & 28.6000                 & 4.6250                  & 5,  6                                    & \nodata                           & 0.24                          & 7.07                              & 6.49                              & $-$0.40                         & 4.7                       & 1.5                       & 10.5                         & 1.01E+21                                & 1                        \\
                    711                     & 02861$+$0369$+$0067          & 28.6083                 & 3.6917                  & 5,  6,  7                                & 18                                & 0.24                          & 6.66                              & 5.66                              & $-$0.83                         & 3.8                       & 2.7                       & 7.8                          & 1.51E+22                                & 2                        \\
                    712                     & 02862$+$0354$+$0072          & 28.6167                 & 3.5417                  & 1,  5,  6                                & \nodata                           & 0.24                          & 7.16                              & 6.57                              & $-$0.78                         & 2.9                       & 1.5                       & 6.4                          & 4.70E+21                                & 2                        \\
                    713                     & 02862$+$0499$+$0060          & 28.6167                 & 4.9917                  & 5,  6                                    & \nodata                           & 0.24                          & 5.96                              & 5.23                              & $-$0.74                         & 6.7                       & 2.1                       & 12.6                         & 1.03E+21                                & 1                        \\
                    714                     & 02863$+$0322$+$0072          & 28.6333                 & 3.2250                  & 5,  6                                    & \nodata                           & 0.24                          & 7.25                              & 6.85                              & $-$0.52                         & 3.0                       & 1.4                       & 8.3                          & 3.84E+21                                & 2                        \\
                    715                     & 02863$+$0325$+$0076          & 28.6333                 & 3.2500                  & 5,  6                                    & \nodata                           & 0.24                          & 7.56                              & 6.88                              & $-$0.66                         & 2.4                       & 1.3                       & 6.0                          & 5.88E+21                                & 2                        \\
                    716                     & 02863$+$0403$+$0063          & 28.6333                 & 4.0333                  & 5,  6                                    & \nodata                           & 0.24                          & 6.33                              & 6.09                              & $-$0.32                         & 8.0                       & 1.4                       & 15.5                         & 4.16E+21                                & 2                        \\
                    717                     & 02865$+$0356$+$0069          & 28.6500                 & 3.5583                  & 3,  5,  6                                & \nodata                           & 0.24                          & 6.90                              & 5.99                              & $-$0.77                         & 3.0                       & 1.8                       & 6.7                          & 9.01E+21                                & 2                        \\
                    718                     & 02866$+$0362$+$0071          & 28.6583                 & 3.6167                  & 5,  6                                    & \nodata                           & 0.24                          & 7.13                              & 5.85                              & $-$1.13                         & 3.3                       & 1.1                       & 6.8                          & 4.83E+21                                & 2                        \\
                    719                     & 02866$+$0375$+$0071          & 28.6583                 & 3.7500                  & 5,  6                                    & \nodata                           & 0.24                          & 7.05                              & 6.22                              & $-$0.61                         & 3.6                       & 1.2                       & 9.2                          & 5.55E+21                                & 2                        \\
                    720                     & 02866$+$0400$+$0072          & 28.6583                 & 4.0000                  & 5,  6                                    & \nodata                           & 0.24                          & 7.24                              & 6.55                              & $-$0.64                         & 4.8                       & 2.1                       & 8.6                          & 8.66E+21                                & 2                        \\
                    721                     & 02867$+$0337$+$0068          & 28.6667                 & 3.3667                  & 5,  6                                    & \nodata                           & 0.24                          & 6.82                              & 6.13                              & $-$0.29                         & 3.7                       & 1.1                       & 7.6                          & 8.77E+21                                & 2                        \\
                    722                     & 02867$+$0368$+$0072          & 28.6667                 & 3.6833                  & 5,  6                                    & \nodata                           & 0.24                          & 7.22                              & 6.41                              & $-$0.80                         & 2.6                       & 1.9                       & 6.5                          & 9.11E+21                                & 2                        \\
                    723                     & 02867$+$0397$+$0073          & 28.6667                 & 3.9667                  & 5,  6                                    & \nodata                           & 0.24                          & 7.34                              & 6.81                              & $-$0.57                         & 4.0                       & 1.5                       & 9.0                          & 5.14E+21                                & 2                        \\
                    724                     & 02867$+$0407$+$0061          & 28.6667                 & 4.0750                  & 5,  6                                    & \nodata                           & 0.24                          & 6.14                              & 5.33                              & $-$0.63                         & 6.5                       & 6.7                       & 12.2                         & 6.81E+21                                & 1                        \\
                    725                     & 02869$+$0377$+$0073          & 28.6917                 & 3.7750                  & 5,  6                                    & \nodata                           & 0.24                          & 7.34                              & 6.71                              & $-$0.46                         & 3.3                       & 1.3                       & 6.9                          & 6.71E+21                                & 2                        \\
                    726                     & 02869$+$0403$+$0068          & 28.6917                 & 4.0333                  & 5,  6                                    & \nodata                           & 0.24                          & 6.81                              & 6.40                              & $-$0.53                         & 5.8                       & 2.2                       & 10.6                         & 6.64E+21                                & 2                        \\
                    727                     & 02870$+$0352$+$0070          & 28.7000                 & 3.5167                  & 3,  5,  6                                & \nodata                           & 0.24                          & 6.99                              & 6.14                              & $-$0.81                         & 2.8                       & 2.1                       & 6.4                          & 1.13E+22                                & 2                        \\
                    728                     & 02870$+$0357$+$0072          & 28.7000                 & 3.5750                  & 3,  5,  6                                & \nodata                           & 0.24                          & 7.16                              & 6.37                              & $-$0.78                         & 3.2                       & 1.7                       & 6.8                          & 7.26E+21                                & 2                        \\
                    729                     & 02870$+$0397$+$0074          & 28.7000                 & 3.9667                  & 5,  6                                    & \nodata                           & 0.24                          & 7.44                              & 6.87                              & $-$0.49                         & 4.2                       & 1.4                       & 8.1                          & 5.68E+21                                & 2                        \\
                    730                     & 02872$+$0351$+$0050          & 28.7167                 & 3.5083                  & 3,  5,  6                                & \nodata                           & 0.24                          & 4.99                              & 4.24                              & $-$0.33                         & 20.8                      & 6.0                       & 28.9                         & 1.12E+22                                & 1                        \\
                    731                     & 02872$+$0388$+$0077          & 28.7167                 & 3.8833                  & 5,  6                                    & \nodata                           & 0.24                          & 7.67                              & 6.53                              & $-$0.84                         & 3.9                       & 2.8                       & 8.3                          & 1.69E+22                                & 2                        \\
                    732                     & 02875$+$0384$+$0075          & 28.7500                 & 3.8417                  & 5,  6                                    & \nodata                           & 0.24                          & 7.53                              & 6.97                              & $-$0.43                         & 3.5                       & 1.5                       & 7.2                          & 7.39E+21                                & 2                        \\
                    733                     & 02875$+$0402$+$0067          & 28.7500                 & 4.0167                  & 5,  6                                    & \nodata                           & 0.24                          & 6.72                              & 5.26                              & $-$0.79                         & 6.7                       & 4.4                       & 12.9                         & 4.81E+21                                & 1                        \\
                    734                     & 02877$+$0357$+$0054          & 28.7667                 & 3.5750                  & 3,  5,  6                                & \nodata                           & 0.24                          & 5.40                              & 4.51                              & $-$0.58                         & 15.1                      & 10.7                      & 20.1                         & 1.46E+22                                & 1                        \\
                    735                     & 02877$+$0392$+$0074          & 28.7667                 & 3.9250                  & 5,  6                                    & \nodata                           & 0.24                          & 7.43                              & 6.98                              & $-$0.46                         & 3.7                       & 1.8                       & 6.7                          & 7.69E+21                                & 2                        \\
                    736                     & 02877$+$0388$+$0075          & 28.7750                 & 3.8833                  & 5,  6                                    & \nodata                           & 0.24                          & 7.45                              & 7.00                              & $-$0.32                         & 3.7                       & 1.7                       & 7.2                          & 9.39E+21                                & 2                        \\
                    737                     & 02879$+$0371$+$0071          & 28.7917                 & 3.7083                  & 5,  6                                    & \nodata                           & 0.24                          & 7.11                              & 4.53                              & $-$0.77                         & 6.1                       & 3.5                       & 38.8                         & 1.12E+22                                & 1                        \\
                    738                     & 02880$+$0395$+$0073          & 28.8000                 & 3.9500                  & 5,  6                                    & \nodata                           & 0.24                          & 7.34                              & 7.11                              & $-$0.21                         & 3.7                       & 1.7                       & 8.6                          & 7.07E+21                                & 2                        \\
                    739                     & 02882$+$0345$+$0072          & 28.8167                 & 3.4500                  & 3,  4,  5,  6                            & \nodata                           & 0.24                          & 7.16                              & 5.71                              & $-$0.97                         & 3.2                       & 1.2                       & 6.8                          & 6.84E+21                                & 2                        \\
                    740                     & 02883$+$0402$+$0073          & 28.8333                 & 4.0250                  & 5,  6                                    & \nodata                           & 0.24                          & 7.27                              & 7.01                              & $-$0.17                         & 5.2                       & 1.1                       & 8.5                          & 5.82E+21                                & 2                        \\
                    741                     & 02883$+$0411$+$0069          & 28.8333                 & 4.1083                  & 5,  6                                    & \nodata                           & 0.24                          & 6.91                              & 5.90                              & $-$0.52                         & 6.2                       & 5.5                       & 18.2                         & 6.94E+21                                & 1                        \\
                    742                     & 02883$-$0168$+$0044          & 28.8333                 & $-$1.6833                 & 5,  6                                    & \nodata                           & 0.24                          & 4.43                              & 3.33                              & $-$0.77                         & 5.7                       & 1.8                       & 13.6                         & 1.27E+21                                & 1                        \\
                    743                     & 02884$+$0341$+$0069          & 28.8417                 & 3.4083                  & 3,  4,  5,  6                            & \nodata                           & 0.24                          & 6.92                              & 5.88                              & $-$0.70                         & 5.5                       & 1.4                       & 10.0                         & 7.04E+21                                & 2                        \\
                    744                     & 02884$+$0400$+$0073          & 28.8417                 & 4.0000                  & 5,  6                                    & \nodata                           & 0.24                          & 7.35                              & 5.21                              & $-$0.95                         & 5.4                       & 4.5                       & 15.4                         & 6.04E+21                                & 1                        \\
                    745                     & 02885$+$0343$+$0071          & 28.8500                 & 3.4333                  & 3,  4,  5,  6                            & \nodata                           & 0.24                          & 7.10                              & 5.90                              & $-$0.62                         & 6.2                       & 1.4                       & 10.6                         & 9.82E+21                                & 2                        \\
                    746                     & 02886$+$0419$+$0069          & 28.8583                 & 4.1917                  & 1,  5,  6                                & \nodata                           & 0.24                          & 6.90                              & 5.08                              & $-$0.87                         & 5.9                       & 6.5                       & 11.3                         & 1.19E+22                                & 1                        \\
                    747                     & 02887$-$0167$+$0045          & 28.8667                 & $-$1.6750                 & 5,  6                                    & \nodata                           & 0.24                          & 4.45                              & 3.82                              & $-$0.64                         & 6.6                       & 2.0                       & 10.6                         & 9.75E+20                                & 1                        \\
                    748                     & 02888$+$0340$+$0071          & 28.8833                 & 3.4000                  & 3,  4,  5,  6                            & \nodata                           & 0.24                          & 7.07                              & 6.08                              & $-$0.81                         & 5.3                       & 2.4                       & 8.7                          & 1.21E+22                                & 2                        \\
                    749                     & 02888$+$0342$+$0069          & 28.8833                 & 3.4250                  & 3,  4,  5,  6                            & \nodata                           & 0.24                          & 6.93                              & 5.81                              & $-$1.08                         & 6.3                       & 3.2                       & 13.3                         & 1.37E+22                                & 2                        \\
                    750                     & 02890$+$0354$+$0053          & 28.9000                 & 3.5417                  & 3,  4,  5,  6                            & \nodata                           & 0.24                          & 5.31                              & 4.62                              & $-$0.34                         & 14.4                      & 6.3                       & 20.5                         & 8.91E+21                                & 1                        \\
                    751                     & 02891$+$0338$+$0071          & 28.9083                 & 3.3833                  & 4,  5,  6                                & \nodata                           & 0.24                          & 7.13                              & 6.05                              & $-$1.01                         & 5.0                       & 3.2                       & 9.9                          & 1.45E+22                                & 2                        \\
                    752                     & 02891$+$0342$+$0072          & 28.9083                 & 3.4250                  & 3,  4,  5,  6                            & \nodata                           & 0.24                          & 7.19                              & 6.01                              & $-$1.29                         & 6.1                       & 2.9                       & 13.0                         & 1.07E+22                                & 2                        \\
                    753                     & 02891$+$0345$+$0073          & 28.9083                 & 3.4500                  & 3,  4,  5,  6                            & \nodata                           & 0.24                          & 7.28                              & 5.61                              & $-$0.92                         & 7.0                       & 1.7                       & 16.4                         & 1.27E+22                                & 2                        \\
                    754                     & 02891$+$0413$+$0077          & 28.9083                 & 4.1333                  & 5,  6                                    & \nodata                           & 0.24                          & 7.70                              & 7.25                              & $-$0.41                         & 5.2                       & 1.3                       & 8.7                          & 4.89E+21                                & 2                        \\
                    755                     & 02893$+$0337$+$0072          & 28.9333                 & 3.3750                  & 4,  5,  6                                & \nodata                           & 0.24                          & 7.20                              & 5.87                              & $-$0.95                         & 4.8                       & 2.5                       & 10.7                         & 1.36E+22                                & 2                        \\
                    756                     & 02894$+$0453$+$0070          & 28.9417                 & 4.5333                  & 5,  6                                    & \nodata                           & 0.24                          & 7.04                              & 6.47                              & $-$0.41                         & 5.2                       & 3.7                       & 10.4                         & 2.93E+21                                & 1                        \\
                    757                     & 02896$+$0338$+$0073          & 28.9583                 & 3.3833                  & 4,  5,  6                                & \nodata                           & 0.24                          & 7.26                              & 5.82                              & $-$1.51                         & 7.2                       & 3.0                       & 16.9                         & 1.25E+22                                & 2                        \\
                    758                     & 02897$+$0354$+$0072          & 28.9667                 & 3.5417                  & 4,  5,  6                                & \nodata                           & 0.24                          & 7.17                              & 5.58                              & $-$1.14                         & 4.7                       & 1.3                       & 11.4                         & 6.58E+21                                & 2                        \\
                    759                     & 02897$+$0404$+$0077          & 28.9667                 & 4.0417                  & 5,  6                                    & \nodata                           & 0.24                          & 7.70                              & 7.46                              & $-$0.32                         & 4.0                       & 1.1                       & 8.7                          & 2.89E+21                                & 2                        \\
                    760                     & 02897$+$0345$+$0071          & 28.9750                 & 3.4500                  & 4,  5,  6                                & \nodata                           & 0.24                          & 7.14                              & 6.22                              & $-$0.89                         & 4.3                       & 2.8                       & 7.7                          & 1.37E+22                                & 2                        \\
                    761                     & 02899$+$0337$+$0074          & 28.9917                 & 3.3667                  & 4,  5,  6                                & \nodata                           & 0.24                          & 7.38                              & 6.24                              & $-$1.14                         & 2.8                       & 3.1                       & 8.1                          & 1.49E+22                                & 2                        \\
                    762                     & 02899$+$0450$+$0073          & 28.9917                 & 4.5000                  & 5,  6                                    & \nodata                           & 0.24                          & 7.30                              & 6.41                              & $-$0.58                         & 3.8                       & 3.0                       & 8.4                          & 2.69E+21                                & 1                        \\
                    763                     & 02899$+$0453$+$0069          & 28.9917                 & 4.5333                  & 5,  6                                    & \nodata                           & 0.24                          & 6.95                              & 6.50                              & $-$0.33                         & 4.6                       & 3.9                       & 8.6                          & 3.78E+21                                & 1                        \\
                    764                     & 02901$+$0257$+$0053          & 29.0083                 & 2.5750                  & 5,  6                                    & \nodata                           & 0.24                          & 5.32                              & 4.52                              & $-$0.86                         & 5.8                       & 2.1                       & 10.8                         & 9.66E+20                                & 1                        \\
                    765                     & 02901$-$0179$+$0046          & 29.0083                 & $-$1.7917                 & 5,  6                                    & \nodata                           & 0.24                          & 4.59                              & 3.56                              & $-$0.85                         & 4.8                       & 2.0                       & 9.0                          & 1.18E+21                                & 1                        \\
                    766                     & 02902$+$0452$+$0071          & 29.0167                 & 4.5250                  & 5,  6                                    & \nodata                           & 0.24                          & 7.13                              & 6.43                              & $-$0.48                         & 5.6                       & 3.7                       & 10.4                         & 3.14E+21                                & 1                        \\
                    767                     & 02903$+$0449$+$0074          & 29.0333                 & 4.4917                  & 5,  6                                    & \nodata                           & 0.24                          & 7.42                              & 6.75                              & $-$0.45                         & 4.6                       & 4.1                       & 8.3                          & 5.06E+21                                & 1                        \\
                    768                     & 02903$+$0455$+$0074          & 29.0333                 & 4.5500                  & 5,  6                                    & \nodata                           & 0.24                          & 7.41                              & 6.59                              & $-$0.60                         & 6.0                       & 4.2                       & 11.2                         & 3.36E+21                                & 1                        \\
                    769                     & 02904$+$0452$+$0074          & 29.0417                 & 4.5250                  & 5,  6                                    & \nodata                           & 0.24                          & 7.40                              & 6.54                              & $-$0.50                         & 4.9                       & 3.8                       & 8.9                          & 4.25E+21                                & 1                        \\
                    770                     & 02905$+$0517$+$0077          & 29.0500                 & 5.1750                  & 5,  6                                    & \nodata                           & 0.24                          & 7.65                              & 6.79                              & $-$0.78                         & 5.1                       & 2.9                       & 10.6                         & 1.66E+21                                & 1                        \\
                    771                     & 02906$+$0457$+$0074          & 29.0583                 & 4.5750                  & 5,  6                                    & \nodata                           & 0.24                          & 7.45                              & 6.79                              & $-$0.50                         & 4.8                       & 4.4                       & 10.3                         & 3.67E+21                                & 1                        \\
                    772                     & 02906$-$0142$+$0097          & 29.0583                 & $-$1.4167                 & 5,  6                                    & \nodata                           & 0.24                          & 9.74                              & 9.18                              & $-$0.71                         & 7.2                       & 1.7                       & 11.4                         & 6.24E+20                                & 1                        \\
                    773                     & 02906$-$0193$+$0051          & 29.0583                 & $-$1.9333                 & 5,  6                                    & \nodata                           & 0.24                          & 5.15                              & 4.17                              & $-$0.76                         & 5.2                       & 2.1                       & 8.8                          & 1.37E+21                                & 1                        \\
                    774                     & 02907$-$0191$+$0052          & 29.0667                 & $-$1.9083                 & 5,  6                                    & \nodata                           & 0.24                          & 5.19                              & 4.51                              & $-$0.56                         & 5.8                       & 1.9                       & 10.3                         & 1.09E+21                                & 1                        \\
                    775                     & 02907$+$0507$+$0060          & 29.0750                 & 5.0750                  & 5,  6                                    & \nodata                           & 0.24                          & 6.02                              & 5.68                              & $-$0.31                         & 5.3                       & 2.9                       & 11.7                         & 1.63E+21                                & 1                        \\
                    776                     & 02907$-$0182$+$0045          & 29.0750                 & $-$1.8167                 & 5,  6                                    & \nodata                           & 0.24                          & 4.45                              & 3.46                              & $-$0.61                         & 4.7                       & 1.7                       & 13.5                         & 1.35E+21                                & 1                        \\
                    777                     & 02909$+$0458$+$0079          & 29.0917                 & 4.5833                  & 5,  6                                    & \nodata                           & 0.24                          & 7.86                              & 6.81                              & $-$0.73                         & 4.5                       & 2.1                       & 11.2                         & 1.43E+21                                & 1                        \\
                    778                     & 02910$-$0177$+$0047          & 29.1000                 & $-$1.7667                 & 5,  6                                    & \nodata                           & 0.24                          & 4.74                              & 3.84                              & $-$0.97                         & 4.7                       & 2.4                       & 11.0                         & 1.10E+21                                & 1                        \\
                    779                     & 02910$-$0199$+$0045          & 29.1000                 & $-$1.9917                 & 5,  6                                    & \nodata                           & 0.24                          & 4.51                              & 3.83                              & $-$0.44                         & 6.9                       & 2.8                       & 14.1                         & 2.30E+21                                & 1                        \\
                    780                     & 02911$-$0194$+$0048          & 29.1083                 & $-$1.9417                 & 5,  6                                    & \nodata                           & 0.24                          & 4.84                              & 4.11                              & $-$0.48                         & 6.1                       & 3.0                       & 11.4                         & 2.36E+21                                & 1                        \\
                    781                     & 02911$-$0197$+$0049          & 29.1083                 & $-$1.9667                 & 5,  6                                    & \nodata                           & 0.24                          & 4.88                              & 4.16                              & $-$0.49                         & 5.8                       & 2.4                       & 14.4                         & 1.88E+21                                & 1                        \\
                    782                     & 02912$-$0172$+$0046          & 29.1167                 & $-$1.7250                 & 5,  6                                    & \nodata                           & 0.24                          & 4.61                              & 3.56                              & $-$0.73                         & 5.7                       & 2.1                       & 16.3                         & 1.58E+21                                & 1                        \\
                    783                     & 02912$+$0349$+$0076          & 29.1250                 & 3.4917                  & 4,  5,  6                                & \nodata                           & 0.24                          & 7.57                              & 6.56                              & $-$0.89                         & 4.0                       & 1.5                       & 7.8                          & 6.15E+21                                & 2                        \\
                    784                     & 02912$-$0199$+$0047          & 29.1250                 & $-$1.9917                 & 5,  6                                    & \nodata                           & 0.24                          & 4.74                              & 4.01                              & $-$0.52                         & 5.9                       & 2.7                       & 19.8                         & 2.27E+21                                & 1                        \\
                    785                     & 02914$+$0425$+$0075          & 29.1417                 & 4.2500                  & 5,  6                                    & \nodata                           & 0.24                          & 7.49                              & 6.59                              & $-$1.23                         & 3.8                       & 1.6                       & 7.7                          & 4.29E+21                                & 2                        \\
                    786                     & 02916$-$0147$+$0100          & 29.1583                 & $-$1.4667                 & 5,  6                                    & \nodata                           & 0.24                          & 10.04                             & 9.60                              & $-$0.52                         & 5.2                       & 2.0                       & 12.9                         & 8.42E+20                                & 1                        \\
                    787                     & 02916$-$0187$+$0049          & 29.1583                 & $-$1.8667                 & 5,  6                                    & \nodata                           & 0.24                          & 4.90                              & 4.15                              & $-$0.63                         & 4.8                       & 2.5                       & 11.8                         & 1.47E+21                                & 1                        \\
                    788                     & 02917$-$0177$+$0042          & 29.1667                 & $-$1.7667                 & 5,  6                                    & \nodata                           & 0.24                          & 4.17                              & 3.52                              & $-$0.57                         & 5.1                       & 2.3                       & 10.6                         & 1.27E+21                                & 1                        \\
                    789                     & 02917$-$0143$+$0099          & 29.1750                 & $-$1.4333                 & 5,  6                                    & \nodata                           & 0.24                          & 9.87                              & 9.24                              & $-$0.51                         & 6.0                       & 1.6                       & 10.9                         & 9.08E+20                                & 1                        \\
                    790                     & 02918$+$0399$+$0076          & 29.1833                 & 3.9917                  & 3,  5,  6                                & \nodata                           & 0.24                          & 7.64                              & 6.70                              & $-$0.45                         & 8.1                       & 3.7                       & 11.5                         & 4.25E+21                                & 1                        \\
                    791                     & 02918$+$0423$+$0076          & 29.1833                 & 4.2333                  & 5,  6                                    & \nodata                           & 0.24                          & 7.56                              & 6.57                              & $-$0.94                         & 3.3                       & 1.2                       & 8.1                          & 4.40E+21                                & 2                        \\
                    792                     & 02918$-$0159$+$0037          & 29.1833                 & $-$1.5917                 & 5,  6                                    & \nodata                           & 0.24                          & 3.67                              & 3.08                              & $-$0.78                         & 5.4                       & 1.4                       & 14.3                         & 5.29E+20                                & 1                        \\
                    793                     & 02918$-$0183$+$0046          & 29.1833                 & $-$1.8333                 & 5,  6                                    & \nodata                           & 0.24                          & 4.62                              & 4.12                              & $-$0.36                         & 5.0                       & 2.9                       & 10.4                         & 2.13E+21                                & 1                        \\
                    794                     & 02919$-$0197$+$0046          & 29.1917                 & $-$1.9750                 & 5,  6                                    & \nodata                           & 0.24                          & 4.57                              & 4.19                              & $-$0.33                         & 6.6                       & 2.8                       & 10.5                         & 1.68E+21                                & 1                        \\
                    795                     & 02921$+$0396$+$0077          & 29.2083                 & 3.9583                  & 3,  5,  6                                & \nodata                           & 0.24                          & 7.72                              & 6.72                              & $-$0.45                         & 6.0                       & 3.4                       & 10.6                         & 4.09E+21                                & 1                        \\
                    796                     & 02921$-$0137$+$0097          & 29.2083                 & $-$1.3667                 & 5,  6                                    & \nodata                           & 0.24                          & 9.68                              & 9.13                              & $-$0.45                         & 5.7                       & 2.5                       & 10.8                         & 1.56E+21                                & 1                        \\
                    797                     & 02922$-$0178$+$0041          & 29.2167                 & $-$1.7833                 & 5,  6                                    & \nodata                           & 0.24                          & 4.07                              & 3.61                              & $-$0.37                         & 5.2                       & 2.0                       & 13.3                         & 1.27E+21                                & 1                        \\
                    798                     & 02924$+$0397$+$0079          & 29.2417                 & 3.9667                  & 3,  5,  6                                & \nodata                           & 0.24                          & 7.91                              & 6.68                              & $-$0.51                         & 6.5                       & 3.1                       & 12.3                         & 3.93E+21                                & 1                        \\
                    799                     & 02927$-$0205$+$0048          & 29.2667                 & $-$2.0500                 & 1,  5,  6                                & \nodata                           & 0.24                          & 4.76                              & 3.71                              & $-$1.05                         & 5.4                       & 1.6                       & 12.6                         & 7.45E+20                                & 1                        \\
                    800                     & 02928$+$0397$+$0074          & 29.2833                 & 3.9667                  & 3,  5,  6                                & \nodata                           & 0.24                          & 7.37                              & 6.77                              & $-$0.50                         & 6.6                       & 3.1                       & 12.8                         & 2.01E+21                                & 1                        \\
                    801                     & 02930$-$0201$+$0047          & 29.3000                 & $-$2.0083                 & 5,  6                                    & \nodata                           & 0.24                          & 4.74                              & 3.44                              & $-$1.57                         & 5.0                       & 1.8                       & 11.5                         & 7.21E+20                                & 1                        \\
                    802                     & 02935$+$0319$+$0094          & 29.3500                 & 3.1917                  & 5,  6                                    & \nodata                           & 0.24                          & 9.40                              & 8.30                              & $-$0.62                         & 5.3                       & 3.7                       & 14.4                         & 3.70E+21                                & 1                        \\
                    803                     & 02937$+$0412$+$0075          & 29.3667                 & 4.1167                  & 3,  5,  6                                & \nodata                           & 0.24                          & 7.50                              & 6.59                              & $-$0.56                         & 6.4                       & 3.1                       & 27.0                         & 3.61E+21                                & 1                        \\
                    804                     & 02939$+$0293$+$0071          & 29.3917                 & 2.9333                  & 5,  6                                    & \nodata                           & 0.24                          & 7.09                              & 6.71                              & $-$0.57                         & 3.3                       & 0.8                       & 7.2                          & 1.87E+21                                & 2                        \\
                    805                     & 02940$+$0296$+$0070          & 29.4000                 & 2.9583                  & 5,  6                                    & \nodata                           & 0.24                          & 6.95                              & 6.20                              & $-$0.54                         & 5.1                       & 3.9                       & 11.3                         & 3.05E+21                                & 1                        \\
                    806                     & 02942$+$0505$+$0078          & 29.4167                 & 5.0500                  & 5,  6                                    & \nodata                           & 0.24                          & 7.83                              & 6.61                              & $-$1.02                         & 4.5                       & 2.1                       & 9.3                          & 1.24E+21                                & 1                        \\
                    807                     & 02942$-$0065$+$0615          & 29.4167                 & $-$0.6500                 & 6,  7                                    & \nodata                           & 4.26                          & 61.46                             & 60.52                             & $-$0.33                         & 7.1                       & 3.5                       & 13.3                         & 5.54E+21                                & 1                        \\
                    808                     & 02942$+$0494$+$0077          & 29.4250                 & 4.9417                  & 5,  6                                    & \nodata                           & 0.24                          & 7.66                              & 6.95                              & $-$0.72                         & 5.1                       & 2.3                       & 9.7                          & 1.13E+21                                & 1                        \\
                    809                     & 02943$+$0311$+$0087          & 29.4333                 & 3.1083                  & 5,  6                                    & \nodata                           & 0.24                          & 8.69                              & 7.41                              & $-$0.67                         & 4.8                       & 3.2                       & 11.7                         & 3.28E+21                                & 1                        \\
                    810                     & 02945$+$0298$+$0073          & 29.4500                 & 2.9833                  & 5,  6                                    & \nodata                           & 0.24                          & 7.28                              & 6.50                              & $-$0.49                         & 4.9                       & 4.4                       & 9.4                          & 4.79E+21                                & 1                        \\
                    811                     & 02946$+$0505$+$0078          & 29.4583                 & 5.0500                  & 5,  6                                    & \nodata                           & 0.24                          & 7.78                              & 6.09                              & $-$1.16                         & 3.7                       & 1.5                       & 8.5                          & 1.04E+21                                & 1                        \\
                    812                     & 02948$+$0294$+$0070          & 29.4833                 & 2.9417                  & 5,  6                                    & \nodata                           & 0.24                          & 7.02                              & 6.24                              & $-$0.47                         & 4.8                       & 3.0                       & 7.9                          & 3.24E+21                                & 1                        \\
                    813                     & 02948$-$0109$+$0096          & 29.4833                 & $-$1.0917                 & 5,  6                                    & \nodata                           & 0.24                          & 9.62                              & 9.00                              & $-$0.76                         & 4.0                       & 1.9                       & 7.8                          & 7.77E+20                                & 1                        \\
                    814                     & 02952$+$0299$+$0074          & 29.5250                 & 2.9917                  & 5,  6                                    & \nodata                           & 0.24                          & 7.43                              & 6.25                              & $-$0.70                         & 4.6                       & 2.5                       & 7.6                          & 2.50E+21                                & 1                        \\
                    815                     & 02953$+$0266$+$0070          & 29.5333                 & 2.6583                  & 5,  6                                    & \nodata                           & 0.24                          & 7.04                              & 5.86                              & $-$0.80                         & 5.4                       & 1.3                       & 10.0                         & 8.96E+20                                & 1                        \\
                    816                     & 02953$+$0317$+$0080          & 29.5333                 & 3.1750                  & 5,  6                                    & \nodata                           & 0.24                          & 8.04                              & 6.19                              & $-$1.33                         & 4.9                       & 2.3                       & 6.5                          & 2.24E+21                                & 1                        \\
                    817                     & 02954$+$0303$+$0079          & 29.5417                 & 3.0333                  & 5,  6                                    & \nodata                           & 0.24                          & 7.91                              & 6.48                              & $-$1.20                         & 4.5                       & 1.9                       & 7.8                          & 1.18E+21                                & 1                        \\
                    818                     & 02957$-$0106$+$0096          & 29.5750                 & $-$1.0583                 & 5,  6                                    & \nodata                           & 0.24                          & 9.61                              & 9.10                              & $-$0.52                         & 3.3                       & 1.5                       & 7.6                          & 7.19E+20                                & 1                        \\
                    819                     & 02958$-$0063$+$0759          & 29.5833                 & $-$0.6333                 & 6,  7                                    & \nodata                           & 4.26                          & 75.88                             & 75.67                             & $-$0.08                         & 5.2                       & 4.3                       & 9.8                          & 7.25E+21                                & 1                        \\
                    820                     & 02959$+$0493$+$0075          & 29.5917                 & 4.9333                  & 5,  6                                    & \nodata                           & 0.24                          & 7.46                              & 5.84                              & $-$1.52                         & 4.7                       & 1.5                       & 9.3                          & 7.57E+20                                & 1                        \\
                    821                     & 02960$-$0059$+$0753          & 29.6000                 & $-$0.5917                 & 6,  7                                    & \nodata                           & 4.26                          & 75.31                             & 73.90                             & $-$0.29                         & 11.6                      & 4.7                       & 14.9                         & 1.36E+22                                & 1                        \\
                    822                     & 02961$+$0314$+$0074          & 29.6083                 & 3.1417                  & 5,  6                                    & \nodata                           & 0.24                          & 7.43                              & 6.77                              & $-$0.36                         & 5.1                       & 2.3                       & 8.4                          & 2.16E+21                                & 1                        \\
                    823                     & 02961$+$0401$+$0089          & 29.6083                 & 4.0083                  & 5,  6                                    & \nodata                           & 0.24                          & 8.89                              & 7.98                              & $-$0.32                         & 8.3                       & 2.1                       & 13.0                         & 2.99E+21                                & 1                        \\
                    824                     & 02961$+$0424$+$0090          & 29.6083                 & 4.2417                  & 5,  6                                    & \nodata                           & 0.24                          & 9.03                              & 7.83                              & $-$0.75                         & 5.5                       & 1.5                       & 12.9                         & 1.14E+21                                & 1                        \\
                    825                     & 02962$+$0389$+$0092          & 29.6167                 & 3.8917                  & 5,  6                                    & \nodata                           & 0.24                          & 9.22                              & 8.02                              & $-$0.53                         & 6.1                       & 2.3                       & 12.3                         & 2.57E+21                                & 1                        \\
                    826                     & 02962$+$0396$+$0094          & 29.6167                 & 3.9583                  & 5,  6                                    & \nodata                           & 0.24                          & 9.37                              & 8.38                              & $-$0.36                         & 7.3                       & 1.7                       & 12.5                         & 2.17E+21                                & 1                        \\
                    827                     & 02963$+$0424$+$0086          & 29.6333                 & 4.2417                  & 5,  6                                    & \nodata                           & 0.24                          & 8.63                              & 7.92                              & $-$0.70                         & 5.6                       & 1.7                       & 11.0                         & 7.78E+20                                & 1                        \\
                    828                     & 02964$+$0212$+$0081          & 29.6417                 & 2.1250                  & 5,  6                                    & \nodata                           & 0.24                          & 8.10                              & 7.28                              & $-$0.52                         & 5.1                       & 2.5                       & 11.4                         & 1.97E+21                                & 1                        \\
                    829                     & 02964$+$0319$+$0079          & 29.6417                 & 3.1917                  & 5,  6                                    & \nodata                           & 0.24                          & 7.85                              & 6.27                              & $-$1.29                         & 5.6                       & 1.8                       & 9.9                          & 1.01E+21                                & 1                        \\
                    830                     & 02965$+$0322$+$0075          & 29.6500                 & 3.2167                  & 5,  6                                    & \nodata                           & 0.24                          & 7.50                              & 6.21                              & $-$0.68                         & 4.7                       & 1.6                       & 9.3                          & 1.46E+21                                & 1                        \\
                    831                     & 02966$+$0316$+$0076          & 29.6583                 & 3.1583                  & 5,  6                                    & \nodata                           & 0.24                          & 7.60                              & 6.85                              & $-$0.56                         & 4.4                       & 1.6                       & 8.2                          & 1.03E+21                                & 1                        \\
                    832                     & 02967$+$0207$+$0081          & 29.6667                 & 2.0750                  & 5,  6                                    & \nodata                           & 0.24                          & 8.05                              & 7.29                              & $-$1.09                         & 2.9                       & 1.3                       & 7.5                          & 3.18E+21                                & 2                        \\
                    833                     & 02967$+$0201$+$0075          & 29.6750                 & 2.0083                  & 5,  6                                    & \nodata                           & 0.24                          & 7.52                              & 7.17                              & $-$0.30                         & 4.7                       & 2.2                       & 18.1                         & 1.39E+21                                & 1                        \\
                    834                     & 02969$-$0347$+$0101          & 29.6917                 & $-$3.4750                 & 5,  6                                    & \nodata                           & 0.24                          & 10.08                             & 9.73                              & $-$0.37                         & 6.2                       & 3.2                       & 11.5                         & 1.56E+21                                & 1                        \\
                    835                     & 02970$+$0286$+$0072          & 29.7000                 & 2.8583                  & 5,  6                                    & \nodata                           & 0.24                          & 7.24                              & 6.65                              & $-$0.33                         & 5.8                       & 2.6                       & 10.6                         & 2.38E+21                                & 1                        \\
                    836                     & 02970$-$0365$+$0098          & 29.7000                 & $-$3.6500                 & 5,  6                                    & \nodata                           & 0.24                          & 9.83                              & 9.48                              & $-$0.52                         & 6.1                       & 2.1                       & 12.8                         & 7.05E+20                                & 1                        \\
                    837                     & 02973$+$0283$+$0075          & 29.7333                 & 2.8333                  & 5,  6                                    & \nodata                           & 0.24                          & 7.47                              & 6.41                              & $-$0.47                         & 5.4                       & 1.6                       & 9.6                          & 1.68E+21                                & 1                        \\
                    838                     & 02973$+$0398$+$0094          & 29.7333                 & 3.9833                  & 5,  6                                    & \nodata                           & 0.24                          & 9.39                              & 8.39                              & $-$0.91                         & 6.6                       & 2.0                       & 11.7                         & 1.04E+21                                & 1                        \\
                    839                     & 02973$-$0374$+$0097          & 29.7333                 & $-$3.7417                 & 5,  6                                    & \nodata                           & 0.24                          & 9.66                              & 9.26                              & $-$0.47                         & 5.1                       & 2.6                       & 11.6                         & 1.06E+21                                & 1                        \\
                    840                     & 02974$+$0207$+$0080          & 29.7417                 & 2.0750                  & 5,  6                                    & \nodata                           & 0.24                          & 7.97                              & 7.59                              & $-$0.36                         & 4.3                       & 3.7                       & 7.4                          & 3.74E+21                                & 1                        \\
                    841                     & 02974$+$0312$+$0076          & 29.7417                 & 3.1250                  & 5,  6                                    & \nodata                           & 0.24                          & 7.61                              & 6.26                              & $-$1.15                         & 4.5                       & 2.0                       & 10.7                         & 1.09E+21                                & 1                        \\
                    842                     & 02976$+$0323$+$0076          & 29.7583                 & 3.2333                  & 5,  6                                    & \nodata                           & 0.24                          & 7.62                              & 6.80                              & $-$0.70                         & 4.7                       & 2.4                       & 9.2                          & 1.42E+21                                & 1                        \\
                    843                     & 02977$+$0472$+$0080          & 29.7667                 & 4.7250                  & 5,  6                                    & \nodata                           & 0.24                          & 7.98                              & 7.10                              & $-$0.55                         & 4.3                       & 1.7                       & 13.7                         & 1.31E+21                                & 1                        \\
                    844                     & 02978$+$0312$+$0074          & 29.7833                 & 3.1250                  & 5,  6                                    & \nodata                           & 0.24                          & 7.42                              & 6.27                              & $-$0.85                         & 3.9                       & 1.8                       & 8.4                          & 1.19E+21                                & 1                        \\
                    845                     & 02978$+$0324$+$0078          & 29.7833                 & 3.2417                  & 5,  6                                    & \nodata                           & 0.24                          & 7.77                              & 6.91                              & $-$0.81                         & 4.9                       & 2.2                       & 8.7                          & 1.20E+21                                & 1                        \\
                    846                     & 02978$-$0347$+$0103          & 29.7833                 & $-$3.4750                 & 5,  6                                    & \nodata                           & 0.24                          & 10.32                             & 9.63                              & $-$0.56                         & 5.9                       & 1.6                       & 11.8                         & 9.57E+20                                & 1                        \\
                    847                     & 02979$-$0377$+$0097          & 29.7917                 & $-$3.7667                 & 5,  6                                    & \nodata                           & 0.24                          & 9.71                              & 9.20                              & $-$0.56                         & 4.8                       & 2.4                       & 10.0                         & 1.10E+21                                & 1                        \\
                    848                     & 02981$+$0017$+$0831          & 29.8083                 & 0.1750                  & 6,  7                                    & \nodata                           & 4.60                          & 83.14                             & 81.84                             & $-$0.56                         & 8.3                       & 4.7                       & 12.7                         & 6.65E+21                                & 1                        \\
                    849                     & 02981$+$0290$+$0069          & 29.8083                 & 2.9000                  & 5,  6                                    & \nodata                           & 0.24                          & 6.87                              & 6.25                              & $-$0.47                         & 3.9                       & 2.8                       & 8.2                          & 2.21E+21                                & 1                        \\
                    850                     & 02983$-$0383$+$0093          & 29.8333                 & $-$3.8333                 & 5,  6                                    & \nodata                           & 0.24                          & 9.34                              & 9.06                              & $-$0.29                         & 4.3                       & 1.7                       & 11.2                         & 7.58E+20                                & 1                        \\
                    851                     & 02989$-$0077$+$0833          & 29.8917                 & $-$0.7750                 & 1,  6                                    & \nodata                           & 4.31                          & 83.33                             & 82.83                             & $-$0.29                         & 5.9                       & 2.3                       & 10.2                         & 1.54E+22                                & 2                        \\
                    852                     & 02991$-$0001$+$0977          & 29.9083                 & $-$0.0083                 & 3,  6                                    & \nodata                           & 4.83                          & 97.67                             & 96.75                             & $-$0.20                         & 19.9                      & 8.3                       & 24.1                         & 3.15E+22                                & 1                        \\
                    853                     & 02992$+$0177$+$0069          & 29.9167                 & 1.7750                  & 5,  6                                    & \nodata                           & 0.24                          & 6.94                              & 6.06                              & $-$0.82                         & 3.4                       & 1.3                       & 7.2                          & 6.59E+20                                & 1                        \\
                    854                     & 02992$-$0372$+$0094          & 29.9250                 & $-$3.7250                 & 5,  6                                    & \nodata                           & 0.24                          & 9.38                              & 8.94                              & $-$0.53                         & 4.3                       & 1.7                       & 9.2                          & 6.87E+20                                & 1                        \\
                    855                     & 02993$-$0001$+$0987          & 29.9333                 & $-$0.0083                 & 3,  6                                    & \nodata                           & 4.84                          & 98.66                             & 96.49                             & $-$0.34                         & 18.5                      & 8.5                       & 28.9                         & 4.67E+22                                & 1                        \\
                    856                     & 02995$+$0170$+$0070          & 29.9500                 & 1.7000                  & 5,  6                                    & \nodata                           & 0.24                          & 7.04                              & 5.89                              & $-$0.91                         & 3.1                       & 1.6                       & 9.6                          & 9.54E+20                                & 1                        \\
                    857                     & 02997$+$0202$+$0090          & 29.9750                 & 2.0250                  & 1,  5,  6                                & \nodata                           & 0.24                          & 9.01                              & 8.44                              & $-$0.51                         & 3.8                       & 1.4                       & 9.1                          & 6.86E+20                                & 1                        \\
                    858                     & 02998$+$0522$+$0076          & 29.9833                 & 5.2250                  & 5,  6                                    & \nodata                           & 0.24                          & 7.57                              & 6.47                              & $-$1.05                         & 3.8                       & 1.2                       & 9.1                          & 5.65E+20                                & 1                        \\
                    859                     & 02998$-$0134$+$0090          & 29.9833                 & $-$1.3417                 & 1,  5,  6                                & \nodata                           & 0.24                          & 9.02                              & 8.63                              & $-$0.37                         & 4.5                       & 1.8                       & 13.3                         & 9.44E+20                                & 1                        \\
                    860                     & 03000$+$0187$+$0073          & 30.0000                 & 1.8750                  & 5,  6                                    & \nodata                           & 0.24                          & 7.28                              & 6.03                              & $-$0.56                         & 4.6                       & 2.0                       & 15.6                         & 2.33E+21                                & 1                        \\
                    861                     & 03002$+$0366$+$0095          & 30.0167                 & 3.6583                  & 5,  6                                    & \nodata                           & 0.24                          & 9.52                              & 9.00                              & $-$0.41                         & 7.0                       & 3.1                       & 13.2                         & 2.10E+21                                & 1                        \\
                    862                     & 03002$-$0003$+$0940          & 30.0167                 & $-$0.0333                 & 1,  3,  6                                & \nodata                           & 4.82                          & 94.00                             & 93.07                             & $-$0.18                         & 21.8                      & 9.8                       & 26.7                         & 4.34E+22                                & 1                        \\
                    863                     & 03002$+$0482$+$0073          & 30.0250                 & 4.8167                  & 5,  6                                    & \nodata                           & 0.24                          & 7.34                              & 6.65                              & $-$0.49                         & 4.9                       & 2.4                       & 11.5                         & 1.69E+21                                & 1                        \\
                    864                     & 03003$+$0008$+$0080          & 30.0333                 & 0.0833                  & 3,  6                                    & \nodata                           & 0.24                          & 7.96                              & 7.49                              & $-$0.31                         & 5.5                       & 1.3                       & 10.8                         & 9.10E+20                                & 1                        \\
                    865                     & 03003$+$0428$+$0091          & 30.0333                 & 4.2833                  & 5,  6                                    & \nodata                           & 0.24                          & 9.13                              & 7.58                              & $-$1.33                         & 5.8                       & 1.1                       & 12.5                         & 6.02E+20                                & 1                        \\
                    866                     & 03006$-$0137$+$0099          & 30.0583                 & $-$1.3667                 & 5,  6                                    & \nodata                           & 0.24                          & 9.93                              & 9.37                              & $-$0.27                         & 6.1                       & 1.8                       & 12.7                         & 1.78E+21                                & 1                        \\
                    867                     & 03007$-$0034$+$0708          & 30.0667                 & $-$0.3417                 & 6                                        & \nodata                           & 3.68                          & 70.82                             & 69.96                             & $-$0.39                         & 6.6                       & 2.8                       & 11.4                         & 3.16E+21                                & 1                        \\
                    868                     & 03012$+$0181$+$0071          & 30.1167                 & 1.8083                  & 5,  6                                    & \nodata                           & 0.24                          & 7.07                              & 6.10                              & $-$0.62                         & 4.3                       & 2.9                       & 11.2                         & 2.36E+21                                & 1                        \\
                    869                     & 03012$+$0250$+$0078          & 30.1167                 & 2.5000                  & 5,  6                                    & \nodata                           & 0.24                          & 7.78                              & 7.31                              & $-$0.24                         & 7.8                       & 3.0                       & 15.5                         & 3.20E+21                                & 1                        \\
                    870                     & 03014$+$0344$+$0093          & 30.1417                 & 3.4417                  & 5,  6                                    & \nodata                           & 0.24                          & 9.25                              & 8.17                              & $-$1.02                         & 5.9                       & 1.8                       & 13.5                         & 9.71E+20                                & 1                        \\
                    871                     & 03014$+$0497$+$0079          & 30.1417                 & 4.9667                  & 5,  6                                    & \nodata                           & 0.24                          & 7.85                              & 7.16                              & $-$0.82                         & 2.3                       & 1.3                       & 6.8                          & 4.10E+21                                & 2                        \\
                    872                     & 03016$+$0292$+$0071          & 30.1583                 & 2.9167                  & 5,  6                                    & \nodata                           & 0.24                          & 7.12                              & 6.66                              & $-$0.41                         & 4.9                       & 1.8                       & 11.2                         & 9.51E+20                                & 1                        \\
                    873                     & 03016$+$0322$+$0094          & 30.1583                 & 3.2250                  & 5,  6                                    & \nodata                           & 0.24                          & 9.43                              & 8.76                              & $-$0.48                         & 6.2                       & 3.3                       & 14.0                         & 2.50E+21                                & 1                        \\
                    874                     & 03017$+$0369$+$0088          & 30.1667                 & 3.6917                  & 5,  6                                    & 17                                & 0.24                          & 8.84                              & 8.06                              & $-$0.47                         & 7.2                       & 5.1                       & 17.8                         & 5.43E+21                                & 1                        \\
                    875                     & 03017$+$0194$+$0070          & 30.1750                 & 1.9417                  & 5,  6                                    & \nodata                           & 0.24                          & 6.98                              & 6.32                              & $-$0.42                         & 5.1                       & 2.5                       & 11.5                         & 1.98E+21                                & 1                        \\
                    876                     & 03017$+$0475$+$0078          & 30.1750                 & 4.7500                  & 5,  6                                    & \nodata                           & 0.24                          & 7.75                              & 7.17                              & $-$0.36                         & 5.5                       & 2.7                       & 9.0                          & 2.37E+21                                & 1                        \\
                    877                     & 03017$+$0481$+$0081          & 30.1750                 & 4.8083                  & 5,  6                                    & \nodata                           & 0.24                          & 8.13                              & 7.19                              & $-$0.42                         & 5.0                       & 2.1                       & 8.8                          & 2.33E+21                                & 1                        \\
                    878                     & 03017$+$0517$+$0074          & 30.1750                 & 5.1750                  & 5,  6                                    & \nodata                           & 0.24                          & 7.35                              & 6.71                              & $-$0.30                         & 5.3                       & 2.2                       & 14.3                         & 2.38E+21                                & 1                        \\
                    879                     & 03018$+$0137$+$0152          & 30.1833                 & 1.3667                  & 5,  6                                    & \nodata                           & 0.54                          & 15.24                             & 14.89                             & $-$0.24                         & 6.7                       & 1.5                       & 12.5                         & 1.04E+21                                & 1                        \\
                    880                     & 03018$+$0436$+$0083          & 30.1833                 & 4.3583                  & 5,  6                                    & \nodata                           & 0.24                          & 8.32                              & 6.66                              & $-$0.97                         & 8.3                       & 2.2                       & 20.6                         & 2.27E+21                                & 1                        \\
                    881                     & 03018$+$0486$+$0076          & 30.1833                 & 4.8583                  & 5,  6                                    & \nodata                           & 0.24                          & 7.57                              & 7.38                              & $-$0.32                         & 2.8                       & 0.7                       & 6.8                          & 1.56E+21                                & 2                        \\
                    882                     & 03019$+$0468$+$0078          & 30.1917                 & 4.6833                  & 5,  6                                    & \nodata                           & 0.24                          & 7.76                              & 7.05                              & $-$0.41                         & 5.3                       & 3.0                       & 13.0                         & 2.81E+21                                & 1                        \\
                    883                     & 03019$+$0483$+$0078          & 30.1917                 & 4.8333                  & 5,  6                                    & \nodata                           & 0.24                          & 7.81                              & 6.89                              & $-$0.64                         & 4.6                       & 2.7                       & 10.9                         & 1.95E+21                                & 1                        \\
                    884                     & 03019$+$0508$+$0073          & 30.1917                 & 5.0833                  & 5,  6                                    & \nodata                           & 0.24                          & 7.34                              & 6.02                              & $-$0.67                         & 5.2                       & 2.2                       & 11.2                         & 2.11E+21                                & 1                        \\
                    885                     & 03020$+$0179$+$0067          & 30.2000                 & 1.7917                  & 5,  6                                    & \nodata                           & 0.24                          & 6.67                              & 6.10                              & $-$0.34                         & 4.5                       & 2.3                       & 10.2                         & 1.89E+21                                & 1                        \\
                    886                     & 03020$+$0303$+$0072          & 30.2000                 & 3.0333                  & 5,  6                                    & \nodata                           & 0.24                          & 7.18                              & 6.45                              & $-$0.69                         & 5.1                       & 1.3                       & 10.5                         & 6.42E+20                                & 1                        \\
                    887                     & 03021$+$0481$+$0078          & 30.2083                 & 4.8083                  & 5,  6                                    & \nodata                           & 0.24                          & 7.79                              & 7.35                              & $-$0.35                         & 4.2                       & 2.9                       & 8.0                          & 2.28E+21                                & 1                        \\
                    888                     & 03022$+$0187$+$0069          & 30.2167                 & 1.8667                  & 5,  6                                    & \nodata                           & 0.24                          & 6.86                              & 5.91                              & $-$0.45                         & 5.3                       & 2.6                       & 17.1                         & 3.03E+21                                & 1                        \\
                    889                     & 03022$+$0284$+$0066          & 30.2167                 & 2.8417                  & 5,  6                                    & \nodata                           & 0.24                          & 6.58                              & 5.86                              & $-$0.53                         & 4.7                       & 1.7                       & 10.7                         & 1.09E+21                                & 1                        \\
                    890                     & 03022$+$0452$+$0078          & 30.2250                 & 4.5167                  & 5,  6                                    & \nodata                           & 0.24                          & 7.79                              & 7.05                              & $-$0.44                         & 7.3                       & 1.8                       & 12.9                         & 1.47E+21                                & 1                        \\
                    891                     & 03023$+$0481$+$0077          & 30.2333                 & 4.8083                  & 5,  6                                    & \nodata                           & 0.24                          & 7.74                              & 7.05                              & $-$0.57                         & 4.3                       & 3.1                       & 8.3                          & 2.28E+21                                & 1                        \\
                    892                     & 03024$+$0511$+$0075          & 30.2417                 & 5.1083                  & 5,  6                                    & \nodata                           & 0.24                          & 7.49                              & 6.37                              & $-$0.60                         & 6.0                       & 2.6                       & 13.2                         & 2.50E+21                                & 1                        \\
                    893                     & 03027$+$0187$+$0069          & 30.2750                 & 1.8667                  & 5,  6                                    & \nodata                           & 0.24                          & 6.93                              & 6.19                              & $-$0.42                         & 4.2                       & 2.7                       & 11.4                         & 2.44E+21                                & 1                        \\
                    894                     & 03027$+$0482$+$0078          & 30.2750                 & 4.8250                  & 5,  6                                    & \nodata                           & 0.24                          & 7.80                              & 7.10                              & $-$0.41                         & 4.0                       & 2.3                       & 7.4                          & 2.21E+21                                & 1                        \\
                    895                     & 03028$+$0475$+$0077          & 30.2833                 & 4.7500                  & 5,  6                                    & \nodata                           & 0.24                          & 7.74                              & 6.88                              & $-$0.73                         & 4.2                       & 2.1                       & 7.4                          & 1.39E+21                                & 1                        \\
                    896                     & 03028$+$0477$+$0076          & 30.2833                 & 4.7750                  & 5,  6                                    & \nodata                           & 0.24                          & 7.57                              & 6.74                              & $-$0.79                         & 4.3                       & 2.4                       & 8.6                          & 1.29E+21                                & 1                        \\
                    897                     & 03029$+$0189$+$0069          & 30.2917                 & 1.8917                  & 5,  6                                    & \nodata                           & 0.24                          & 6.93                              & 6.37                              & $-$0.37                         & 4.5                       & 2.4                       & 14.4                         & 1.90E+21                                & 1                        \\
                    898                     & 03030$+$0242$+$0084          & 30.3000                 & 2.4250                  & 5,  6                                    & \nodata                           & 0.24                          & 8.42                              & 8.12                              & $-$0.34                         & 3.0                       & 0.9                       & 7.1                          & 2.64E+21                                & 2                        \\
                    899                     & 03030$-$0359$+$0091          & 30.3000                 & $-$3.5917                 & 5,  6                                    & \nodata                           & 0.24                          & 9.05                              & 8.71                              & $-$0.39                         & 4.3                       & 1.5                       & 10.9                         & 6.14E+20                                & 1                        \\
                    900                     & 03032$+$0188$+$0070          & 30.3167                 & 1.8833                  & 5,  6                                    & \nodata                           & 0.24                          & 6.99                              & 6.45                              & $-$0.35                         & 4.9                       & 2.5                       & 11.9                         & 1.95E+21                                & 1                        \\
                    901                     & 03032$+$0262$+$0076          & 30.3167                 & 2.6167                  & 5,  6                                    & \nodata                           & 0.24                          & 7.57                              & 6.60                              & $-$0.78                         & 6.8                       & 3.2                       & 15.4                         & 2.15E+21                                & 1                        \\
                    902                     & 03032$+$0421$+$0083          & 30.3167                 & 4.2083                  & 5,  6                                    & \nodata                           & 0.24                          & 8.32                              & 7.40                              & $-$0.40                         & 9.1                       & 2.2                       & 15.4                         & 2.69E+21                                & 1                        \\
                    903                     & 03032$+$0234$+$0086          & 30.3250                 & 2.3417                  & 5,  6                                    & \nodata                           & 0.24                          & 8.64                              & 8.03                              & $-$0.32                         & 6.9                       & 3.7                       & 11.4                         & 3.96E+21                                & 1                        \\
                    904                     & 03032$+$0428$+$0077          & 30.3250                 & 4.2833                  & 5,  6                                    & \nodata                           & 0.24                          & 7.73                              & 6.88                              & $-$0.57                         & 7.2                       & 2.8                       & 13.2                         & 2.14E+21                                & 1                        \\
                    905                     & 03033$+$0423$+$0079          & 30.3333                 & 4.2333                  & 5,  6                                    & \nodata                           & 0.24                          & 7.88                              & 7.29                              & $-$0.43                         & 7.8                       & 2.4                       & 14.8                         & 1.73E+21                                & 1                        \\
                    906                     & 03034$+$0513$+$0076          & 30.3417                 & 5.1333                  & 5,  6                                    & \nodata                           & 0.24                          & 7.57                              & 6.18                              & $-$0.68                         & 5.9                       & 3.3                       & 14.6                         & 3.72E+21                                & 1                        \\
                    907                     & 03034$+$0525$+$0077          & 30.3417                 & 5.2500                  & 5,  6                                    & \nodata                           & 0.24                          & 7.70                              & 6.92                              & $-$0.57                         & 4.3                       & 1.7                       & 8.5                          & 1.12E+21                                & 1                        \\
                    908                     & 03035$+$0186$+$0068          & 30.3500                 & 1.8583                  & 5,  6                                    & \nodata                           & 0.24                          & 6.83                              & 6.39                              & $-$0.28                         & 4.4                       & 1.8                       & 8.9                          & 1.35E+21                                & 1                        \\
                    909                     & 03035$+$0234$+$0086          & 30.3500                 & 2.3417                  & 5,  6                                    & \nodata                           & 0.24                          & 8.59                              & 7.69                              & $-$0.39                         & 7.1                       & 4.4                       & 11.2                         & 6.17E+21                                & 1                        \\
                    910                     & 03035$+$0242$+$0083          & 30.3500                 & 2.4167                  & 5,  6                                    & \nodata                           & 0.24                          & 8.32                              & 7.78                              & $-$0.56                         & 4.1                       & 0.7                       & 7.8                          & 2.22E+21                                & 2                        \\
                    911                     & 03037$+$0033$+$1120          & 30.3667                 & 0.3333                  & 6                                        & \nodata                           & 4.74                          & 112.04                            & 111.21                            & $-$0.47                         & 3.3                       & 1.2                       & 7.9                          & 7.33E+21                                & 2                        \\
                    912                     & 03037$+$0428$+$0079          & 30.3667                 & 4.2833                  & 5,  6                                    & \nodata                           & 0.24                          & 7.87                              & 7.18                              & $-$0.41                         & 8.0                       & 1.9                       & 13.2                         & 1.58E+21                                & 1                        \\
                    913                     & 03037$+$0425$+$0081          & 30.3750                 & 4.2500                  & 5,  6                                    & \nodata                           & 0.24                          & 8.09                              & 7.24                              & $-$0.57                         & 6.7                       & 2.4                       & 12.8                         & 1.78E+21                                & 1                        \\
                    914                     & 03037$+$0516$+$0075          & 30.3750                 & 5.1583                  & 5,  6                                    & \nodata                           & 0.24                          & 7.49                              & 6.45                              & $-$0.37                         & 5.7                       & 2.8                       & 17.1                         & 4.37E+21                                & 1                        \\
                    915                     & 03041$+$0237$+$0090          & 30.4083                 & 2.3667                  & 1,  5,  6                                & \nodata                           & 0.24                          & 9.00                              & 7.75                              & $-$0.72                         & 5.9                       & 2.9                       & 9.8                          & 2.07E+22                                & 2                        \\
                    916                     & 03042$+$0234$+$0091          & 30.4167                 & 2.3417                  & 5,  6                                    & \nodata                           & 0.24                          & 9.10                              & 8.83                              & $-$0.22                         & 5.9                       & 2.8                       & 9.2                          & 1.46E+22                                & 2                        \\
                    917                     & 03042$+$0343$+$0382          & 30.4167                 & 3.4333                  & 5,  6                                    & \nodata                           & 1.76                          & 38.25                             & 37.82                             & $-$0.46                         & 3.6                       & 2.4                       & 9.2                          & 1.14E+21                                & 1                        \\
                    918                     & 03042$-$0038$+$0123          & 30.4167                 & $-$0.3833                 & 6,  7                                    & \nodata                           & 0.41                          & 12.32                             & 11.88                             & $-$0.32                         & 10.8                      & 2.9                       & 17.4                         & 2.27E+21                                & 1                        \\
                    919                     & 03042$-$0011$+$0883          & 30.4250                 & $-$0.1083                 & 3,  6                                    & \nodata                           & 4.80                          & 88.29                             & 87.88                             & $-$0.17                         & 19.3                      & 7.4                       & 26.2                         & 1.44E+22                                & 1                        \\
                    920                     & 03043$+$0515$+$0076          & 30.4333                 & 5.1500                  & 5,  6                                    & \nodata                           & 0.24                          & 7.59                              & 6.92                              & $-$0.36                         & 5.9                       & 4.0                       & 10.4                         & 4.44E+21                                & 1                        \\
                    921                     & 03045$-$0039$+$0121          & 30.4500                 & $-$0.3917                 & 6,  7                                    & \nodata                           & 0.41                          & 12.06                             & 11.57                             & $-$0.43                         & 11.5                      & 6.0                       & 22.1                         & 4.90E+21                                & 1                        \\
                    922                     & 03046$-$0113$+$0055          & 30.4583                 & $-$1.1333                 & 5,  6                                    & \nodata                           & 0.24                          & 5.53                              & 4.99                              & $-$0.62                         & 3.3                       & 1.1                       & 7.3                          & 4.24E+20                                & 1                        \\
                    923                     & 03047$-$0053$+$0118          & 30.4667                 & $-$0.5333                 & 3,  6                                    & \nodata                           & 0.41                          & 11.76                             & 11.29                             & $-$0.39                         & 18.2                      & 9.1                       & 31.6                         & 1.01E+22                                & 1                        \\
                    924                     & 03048$-$0036$+$0120          & 30.4833                 & $-$0.3583                 & 1,  6,  7                                & \nodata                           & 0.41                          & 12.01                             & 11.44                             & $-$0.56                         & 8.3                       & 9.0                       & 13.8                         & 9.56E+21                                & 1                        \\
                    925                     & 03048$-$0039$+$0119          & 30.4833                 & $-$0.3917                 & 6,  7                                    & \nodata                           & 0.41                          & 11.92                             & 11.41                             & $-$0.42                         & 15.1                      & 12.0                      & 29.8                         & 1.46E+22                                & 1                        \\
                    926                     & 03049$-$0097$+$0775          & 30.4917                 & $-$0.9750                 & 6                                        & \nodata                           & 3.76                          & 77.52                             & 76.48                             & $-$0.59                         & 4.6                       & 1.4                       & 8.2                          & 1.18E+21                                & 1                        \\
                    927                     & 03049$-$0206$+$0070          & 30.4917                 & $-$2.0583                 & 5,  6                                    & \nodata                           & 0.24                          & 6.99                              & 6.10                              & $-$0.87                         & 5.0                       & 1.3                       & 12.0                         & 6.17E+20                                & 1                        \\
                    928                     & 03050$-$0043$+$0118          & 30.5000                 & $-$0.4333                 & 6                                        & \nodata                           & 0.41                          & 11.77                             & 11.40                             & $-$0.38                         & 12.6                      & 11.7                      & 21.1                         & 1.08E+22                                & 1                        \\
                    929                     & 03051$+$0097$+$0503          & 30.5083                 & 0.9667                  & 3,  6,  7                                & \nodata                           & 0.73                          & 50.31                             & 49.70                             & $-$0.60                         & 1.8                       & 1.6                       & 5.5                          & 9.73E+21                                & 2                        \\
                    930                     & 03052$+$0182$+$0066          & 30.5167                 & 1.8250                  & 5,  6                                    & \nodata                           & 0.24                          & 6.63                              & 5.91                              & $-$0.39                         & 3.9                       & 1.6                       & 9.6                          & 1.33E+21                                & 1                        \\
                    931                     & 03052$+$0498$+$0079          & 30.5167                 & 4.9833                  & 5,  6                                    & \nodata                           & 0.24                          & 7.88                              & 6.75                              & $-$0.81                         & 5.1                       & 3.8                       & 9.1                          & 3.43E+21                                & 1                        \\
                    932                     & 03052$-$0033$+$0125          & 30.5167                 & $-$0.3333                 & 6,  7                                    & \nodata                           & 0.41                          & 12.51                             & 11.83                             & $-$0.55                         & 6.3                       & 9.7                       & 12.8                         & \nodata                                 & 1                        \\
                    933                     & 03052$-$0327$+$0096          & 30.5250                 & $-$3.2750                 & 5,  6                                    & \nodata                           & 0.24                          & 9.61                              & 9.19                              & $-$0.36                         & 7.8                       & 2.2                       & 14.3                         & 1.32E+21                                & 1                        \\
                    934                     & 03054$+$0187$+$0063          & 30.5417                 & 1.8667                  & 5,  6                                    & \nodata                           & 0.24                          & 6.32                              & 5.74                              & $-$0.45                         & 3.8                       & 1.6                       & 7.6                          & 1.02E+21                                & 1                        \\
                    935                     & 03054$-$0032$+$0127          & 30.5417                 & $-$0.3167                 & 6,  7                                    & \nodata                           & 0.41                          & 12.67                             & 11.68                             & $-$0.64                         & 5.8                       & 7.2                       & 13.4                         & 8.76E+21                                & 1                        \\
                    936                     & 03056$+$0493$+$0079          & 30.5583                 & 4.9333                  & 5,  6                                    & \nodata                           & 0.24                          & 7.94                              & 7.06                              & $-$0.66                         & 5.4                       & 3.3                       & 13.8                         & 2.41E+21                                & 1                        \\
                    937                     & 03057$+$0186$+$0064          & 30.5750                 & 1.8583                  & 5,  6                                    & \nodata                           & 0.24                          & 6.37                              & 5.79                              & $-$0.55                         & 3.2                       & 1.7                       & 10.0                         & 8.44E+20                                & 1                        \\
                    938                     & 03057$+$0340$+$0099          & 30.5750                 & 3.4000                  & 5,  6                                    & \nodata                           & 0.24                          & 9.87                              & 9.16                              & $-$0.68                         & 4.7                       & 3.2                       & 7.9                          & 2.17E+21                                & 1                        \\
                    939                     & 03058$+$0267$+$0075          & 30.5833                 & 2.6667                  & 5,  6                                    & \nodata                           & 0.24                          & 7.55                              & 7.26                              & $-$0.27                         & 6.1                       & 1.2                       & 9.9                          & 4.25E+21                                & 2                        \\
                    940                     & 03058$+$0336$+$0101          & 30.5833                 & 3.3583                  & 5,  6                                    & \nodata                           & 0.24                          & 10.07                             & 9.38                              & $-$0.45                         & 7.5                       & 3.8                       & 13.9                         & 3.23E+21                                & 1                        \\
                    941                     & 03060$+$0216$+$0073          & 30.6000                 & 2.1583                  & 5,  6                                    & \nodata                           & 0.24                          & 7.34                              & 6.64                              & $-$0.45                         & 6.2                       & 3.5                       & 14.5                         & 3.03E+21                                & 1                        \\
                    942                     & 03062$+$0432$+$0081          & 30.6167                 & 4.3250                  & 5,  6                                    & \nodata                           & 0.24                          & 8.10                              & 7.73                              & $-$0.30                         & 6.8                       & 3.8                       & 11.2                         & 2.60E+21                                & 1                        \\
                    943                     & 03062$+$0518$+$0081          & 30.6250                 & 5.1833                  & 5,  6                                    & \nodata                           & 0.24                          & 8.09                              & 7.67                              & $-$0.53                         & 4.9                       & 1.5                       & 10.6                         & 4.23E+21                                & 2                        \\
                    944                     & 03063$+$0214$+$0073          & 30.6333                 & 2.1417                  & 5,  6                                    & \nodata                           & 0.24                          & 7.28                              & 6.78                              & $-$0.38                         & 6.3                       & 4.3                       & 14.6                         & 3.31E+21                                & 1                        \\
                    945                     & 03063$+$0502$+$0082          & 30.6333                 & 5.0167                  & 5,  6                                    & \nodata                           & 0.24                          & 8.20                              & 7.47                              & $-$0.47                         & 5.7                       & 2.8                       & 12.6                         & 2.30E+21                                & 1                        \\
                    946                     & 03064$+$0323$+$0109          & 30.6417                 & 3.2333                  & 1,  5,  6                                & \nodata                           & 0.24                          & 10.86                             & 9.58                              & $-$0.82                         & 7.6                       & 3.6                       & 14.7                         & 3.12E+21                                & 1                        \\
                    947                     & 03065$+$0273$+$0078          & 30.6500                 & 2.7333                  & 5,  6                                    & \nodata                           & 0.24                          & 7.84                              & 7.28                              & $-$0.49                         & 7.1                       & 4.7                       & 12.8                         & 3.25E+21                                & 1                        \\
                    948                     & 03067$+$0212$+$0075          & 30.6667                 & 2.1250                  & 1,  5,  6                                & \nodata                           & 0.24                          & 7.46                              & 6.60                              & $-$0.67                         & 5.0                       & 4.0                       & 9.5                          & 3.25E+21                                & 1                        \\
                    949                     & 03067$+$0322$+$0108          & 30.6667                 & 3.2167                  & 5,  6                                    & \nodata                           & 0.24                          & 10.75                             & 9.60                              & $-$0.76                         & 6.8                       & 2.4                       & 14.2                         & 1.86E+21                                & 1                        \\
                    950                     & 03067$+$0502$+$0083          & 30.6667                 & 5.0250                  & 5,  6                                    & \nodata                           & 0.24                          & 8.31                              & 7.44                              & $-$0.44                         & 5.9                       & 3.0                       & 9.5                          & 3.23E+21                                & 1                        \\
                    951                     & 03067$+$0523$+$0080          & 30.6667                 & 5.2333                  & 5,  6                                    & \nodata                           & 0.24                          & 7.95                              & 7.33                              & $-$0.46                         & 5.2                       & 1.0                       & 11.0                         & 4.87E+21                                & 2                        \\
                    952                     & 03067$-$0132$+$0130          & 30.6750                 & $-$1.3167                 & 6                                        & \nodata                           & 0.24                          & 13.04                             & 12.69                             & $-$0.33                         & 5.0                       & 1.8                       & 12.1                         & 9.06E+20                                & 1                        \\
                    953                     & 03067$-$0134$+$0133          & 30.6750                 & $-$1.3417                 & 6                                        & \nodata                           & 0.24                          & 13.29                             & 12.67                             & $-$0.71                         & 4.5                       & 1.8                       & 9.5                          & 7.32E+20                                & 1                        \\
                    954                     & 03068$+$0208$+$0075          & 30.6833                 & 2.0833                  & 5,  6                                    & \nodata                           & 0.24                          & 7.45                              & 6.27                              & $-$1.35                         & 3.2                       & 1.9                       & 9.9                          & 7.81E+20                                & 1                        \\
                    955                     & 03068$+$0521$+$0078          & 30.6833                 & 5.2083                  & 5,  6                                    & \nodata                           & 0.24                          & 7.84                              & 7.38                              & $-$0.44                         & 5.6                       & 2.1                       & 8.8                          & 8.26E+21                                & 2                        \\
                    956                     & 03069$+$0322$+$0110          & 30.6917                 & 3.2167                  & 5,  6                                    & \nodata                           & 0.24                          & 10.99                             & 9.68                              & $-$0.82                         & 7.6                       & 3.1                       & 14.1                         & 2.68E+21                                & 1                        \\
                    957                     & 03069$+$0485$+$0078          & 30.6917                 & 4.8500                  & 5,  6                                    & \nodata                           & 0.24                          & 7.83                              & 6.88                              & $-$0.95                         & 5.0                       & 2.9                       & 9.5                          & 1.53E+21                                & 1                        \\
                    958                     & 03069$+$0502$+$0087          & 30.6917                 & 5.0250                  & 5,  6                                    & \nodata                           & 0.24                          & 8.69                              & 8.25                              & $-$0.48                         & 3.3                       & 0.6                       & 6.9                          & 1.71E+21                                & 2                        \\
                    959                     & 03071$+$0496$+$0082          & 30.7083                 & 4.9583                  & 5,  6                                    & \nodata                           & 0.24                          & 8.19                              & 6.95                              & $-$0.60                         & 4.0                       & 2.9                       & 10.5                         & 3.16E+21                                & 1                        \\
                    960                     & 03072$+$0518$+$0083          & 30.7167                 & 5.1833                  & 5,  6                                    & \nodata                           & 0.24                          & 8.29                              & 7.79                              & $-$0.44                         & 5.0                       & 1.5                       & 8.9                          & 6.03E+21                                & 2                        \\
                    961                     & 03072$-$0237$+$0069          & 30.7250                 & $-$2.3667                 & 5,  6                                    & \nodata                           & 0.24                          & 6.87                              & 6.49                              & $-$0.40                         & 4.0                       & 2.4                       & 8.4                          & 1.24E+21                                & 1                        \\
                    962                     & 03072$-$0327$+$0094          & 30.7250                 & $-$3.2750                 & 5,  6                                    & \nodata                           & 0.24                          & 9.41                              & 9.06                              & $-$0.29                         & 6.4                       & 1.7                       & 13.5                         & 9.96E+20                                & 1                        \\
                    963                     & 03073$+$0476$+$0079          & 30.7333                 & 4.7583                  & 5,  6                                    & \nodata                           & 0.24                          & 7.90                              & 6.99                              & $-$0.86                         & 5.9                       & 1.9                       & 10.5                         & 9.57E+20                                & 1                        \\
                    964                     & 03073$-$0132$+$0101          & 30.7333                 & $-$1.3250                 & 3,  6                                    & \nodata                           & 0.24                          & 10.05                             & 9.67                              & $-$0.53                         & 2.6                       & 1.6                       & 5.9                          & 7.12E+20                                & 1                        \\
                    965                     & 03074$+$0317$+$0104          & 30.7417                 & 3.1750                  & 5,  6                                    & \nodata                           & 0.24                          & 10.45                             & 9.35                              & $-$1.01                         & 7.4                       & 3.1                       & 15.3                         & 1.86E+21                                & 1                        \\
                    966                     & 03076$+$0274$+$0080          & 30.7583                 & 2.7417                  & 5,  6                                    & \nodata                           & 0.24                          & 8.04                              & 7.42                              & $-$0.45                         & 5.4                       & 4.8                       & 10.9                         & 4.29E+21                                & 1                        \\
                    967                     & 03076$+$0518$+$0080          & 30.7583                 & 5.1833                  & 5,  6                                    & \nodata                           & 0.24                          & 7.96                              & 7.12                              & $-$0.46                         & 5.4                       & 5.4                       & 8.6                          & \nodata                                 & 1                        \\
                    968                     & 03077$-$0027$+$0076          & 30.7667                 & $-$0.2750                 & 3,  6                                    & \nodata                           & 0.24                          & 7.56                              & 6.93                              & $-$0.88                         & 3.6                       & 1.1                       & 9.5                          & 3.34E+20                                & 1                        \\
                    969                     & 03077$+$0513$+$0078          & 30.7750                 & 5.1333                  & 5,  6                                    & \nodata                           & 0.24                          & 7.79                              & 6.77                              & $-$0.65                         & 5.7                       & 4.3                       & 10.0                         & 4.42E+21                                & 1                        \\
                    970                     & 03078$+$0320$+$0100          & 30.7833                 & 3.2000                  & 5,  6                                    & \nodata                           & 0.24                          & 10.04                             & 9.47                              & $-$0.39                         & 6.0                       & 3.4                       & 13.8                         & 2.68E+21                                & 1                        \\
                    971                     & 03079$+$0302$+$0102          & 30.7917                 & 3.0167                  & 1,  5,  6                                & \nodata                           & 0.24                          & 10.21                             & 9.53                              & $-$0.42                         & 7.0                       & 2.5                       & 12.9                         & 2.03E+21                                & 1                        \\
                    972                     & 03079$+$0502$+$0079          & 30.7917                 & 5.0167                  & 5,  6                                    & \nodata                           & 0.24                          & 7.94                              & 7.60                              & $-$0.53                         & 3.4                       & 1.1                       & 7.0                          & 2.65E+21                                & 2                        \\
                    973                     & 03079$+$0516$+$0077          & 30.7917                 & 5.1583                  & 5,  6                                    & \nodata                           & 0.24                          & 7.72                              & 6.67                              & $-$1.11                         & 6.8                       & 1.1                       & 11.5                         & 3.79E+21                                & 2                        \\
                    974                     & 03080$+$0482$+$0080          & 30.8000                 & 4.8250                  & 5,  6                                    & \nodata                           & 0.24                          & 8.00                              & 7.55                              & $-$0.32                         & 6.1                       & 2.3                       & 12.9                         & 1.65E+21                                & 1                        \\
                    975                     & 03080$+$0512$+$0078          & 30.8000                 & 5.1167                  & 5,  6                                    & \nodata                           & 0.24                          & 7.82                              & 6.88                              & $-$0.56                         & 5.7                       & 3.9                       & 9.5                          & 4.02E+21                                & 1                        \\
                    976                     & 03081$+$0524$+$0080          & 30.8083                 & 5.2417                  & 1,  5,  6                                & \nodata                           & 0.24                          & 8.02                              & 7.25                              & $-$0.64                         & 5.9                       & 2.5                       & 11.3                         & 1.15E+22                                & 2                        \\
                    977                     & 03081$-$0032$+$0075          & 30.8083                 & $-$0.3250                 & 3,  6                                    & \nodata                           & 0.24                          & 7.54                              & 7.16                              & $-$0.32                         & 4.4                       & 1.3                       & 9.3                          & 6.96E+20                                & 1                        \\
                    978                     & 03082$+$0517$+$0079          & 30.8250                 & 5.1667                  & 5,  6                                    & \nodata                           & 0.24                          & 7.88                              & 7.20                              & $-$0.57                         & 5.1                       & 1.1                       & 9.8                          & 4.68E+21                                & 2                        \\
                    979                     & 03083$+$0524$+$0081          & 30.8333                 & 5.2417                  & 5,  6                                    & \nodata                           & 0.24                          & 8.10                              & 7.51                              & $-$0.47                         & 6.3                       & 2.2                       & 10.8                         & 1.08E+22                                & 2                        \\
                    980                     & 03083$-$0034$+$0077          & 30.8333                 & $-$0.3417                 & 6                                        & \nodata                           & 0.24                          & 7.72                              & 7.08                              & $-$0.45                         & 3.6                       & 1.3                       & 10.4                         & 8.57E+20                                & 1                        \\
                    981                     & 03085$+$0516$+$0078          & 30.8500                 & 5.1583                  & 5,  6                                    & \nodata                           & 0.24                          & 7.81                              & 7.25                              & $-$0.49                         & 5.5                       & 1.3                       & 9.7                          & 4.90E+21                                & 2                        \\
                    982                     & 03085$+$0519$+$0078          & 30.8500                 & 5.1917                  & 5,  6                                    & \nodata                           & 0.24                          & 7.82                              & 7.13                              & $-$0.43                         & 8.2                       & 2.1                       & 15.7                         & 1.40E+22                                & 2                        \\
                    983                     & 03086$+$0510$+$0080          & 30.8583                 & 5.1000                  & 5,  6                                    & \nodata                           & 0.24                          & 8.03                              & 7.49                              & $-$0.63                         & 3.8                       & 2.1                       & 8.2                          & 7.12E+21                                & 2                        \\
                    984                     & 03086$+$0525$+$0078          & 30.8583                 & 5.2500                  & 5,  6                                    & \nodata                           & 0.24                          & 7.77                              & 7.24                              & $-$0.67                         & 4.9                       & 2.0                       & 8.8                          & 5.78E+21                                & 2                        \\
                    985                     & 03087$-$0404$+$0101          & 30.8667                 & $-$4.0417                 & 5,  6                                    & \nodata                           & 0.24                          & 10.08                             & 9.78                              & $-$0.34                         & 6.2                       & 1.4                       & 14.5                         & 6.33E+20                                & 1                        \\
                    986                     & 03087$+$0513$+$0079          & 30.8750                 & 5.1333                  & 5,  6                                    & \nodata                           & 0.24                          & 7.86                              & 7.15                              & $-$1.00                         & 4.8                       & 1.4                       & 9.2                          & 3.52E+21                                & 2                        \\
                    987                     & 03088$-$0346$+$0095          & 30.8833                 & $-$3.4583                 & 5,  6                                    & \nodata                           & 0.24                          & 9.48                              & 9.08                              & $-$0.55                         & 4.5                       & 1.3                       & 9.7                          & 4.34E+20                                & 1                        \\
                    988                     & 03089$+$0317$+$0102          & 30.8917                 & 3.1667                  & 5,  6                                    & \nodata                           & 0.24                          & 10.17                             & 9.25                              & $-$0.38                         & 11.8                      & 3.4                       & 22.8                         & 5.44E+21                                & 1                        \\
                    989                     & 03089$+$0517$+$0078          & 30.8917                 & 5.1750                  & 5,  6                                    & \nodata                           & 0.24                          & 7.80                              & 7.33                              & $-$0.57                         & 5.9                       & 2.6                       & 10.6                         & 8.43E+21                                & 2                        \\
                    990                     & 03089$+$0525$+$0079          & 30.8917                 & 5.2500                  & 5,  6                                    & \nodata                           & 0.24                          & 7.91                              & 7.58                              & $-$0.35                         & 4.8                       & 1.3                       & 8.2                          & 4.43E+21                                & 2                        \\
                    991                     & 03090$+$0071$+$0510          & 30.9000                 & 0.7083                  & 6,  7                                    & \nodata                           & 0.66                          & 50.95                             & 49.99                             & $-$0.34                         & 5.5                       & 1.8                       & 9.8                          & 2.44E+21                                & 1                        \\
                    992                     & 03090$+$0522$+$0083          & 30.9000                 & 5.2167                  & 5,  6                                    & \nodata                           & 0.24                          & 8.30                              & 7.22                              & $-$0.91                         & 5.1                       & 1.5                       & 10.7                         & 6.29E+21                                & 2                        \\
                    993                     & 03092$+$0502$+$0085          & 30.9250                 & 5.0167                  & 5,  6                                    & \nodata                           & 0.24                          & 8.45                              & 6.79                              & $-$0.73                         & 7.5                       & 3.7                       & 15.1                         & 4.70E+21                                & 1                        \\
                    994                     & 03095$+$0007$+$0392          & 30.9500                 & 0.0667                  & 3,  6                                    & \nodata                           & 0.28                          & 39.21                             & 38.10                             & $-$0.26                         & 8.9                       & 3.5                       & 13.5                         & 8.12E+21                                & 1                        \\
                    995                     & 03095$+$0519$+$0080          & 30.9500                 & 5.1917                  & 5,  6                                    & \nodata                           & 0.24                          & 8.05                              & 7.40                              & $-$0.65                         & 5.5                       & 1.7                       & 10.2                         & 6.31E+21                                & 2                        \\
                    996                     & 03097$+$0320$+$0102          & 30.9667                 & 3.2000                  & 5,  6                                    & \nodata                           & 0.24                          & 10.18                             & 9.97                              & $-$0.17                         & 4.9                       & 2.3                       & 8.7                          & 1.46E+21                                & 1                        \\
                    997                     & 03101$+$0077$+$0496          & 31.0083                 & 0.7667                  & 6,  7                                    & \nodata                           & 0.73                          & 49.64                             & 48.93                             & $-$0.45                         & 7.4                       & 4.5                       & 19.6                         & 4.46E+21                                & 1                        \\
                    998                     & 03102$+$0315$+$0100          & 31.0167                 & 3.1500                  & 5,  6                                    & \nodata                           & 0.24                          & 10.05                             & 9.48                              & $-$0.51                         & 5.3                       & 4.1                       & 8.9                          & 3.22E+21                                & 1                        \\
                    999                     & 03102$+$0511$+$0080          & 31.0167                 & 5.1083                  & 5,  6                                    & \nodata                           & 0.24                          & 7.98                              & 7.44                              & $-$0.47                         & 4.3                       & 0.7                       & 11.6                         & 2.57E+21                                & 2                        \\
                    1000                    & 03104$+$0267$+$0063          & 31.0417                 & 2.6667                  & 5,  6                                    & \nodata                           & 0.24                          & 6.32                              & 5.69                              & $-$0.57                         & 7.2                       & 2.0                       & 11.0                         & 1.07E+21                                & 1                        \\
                    1001                    & 03105$+$0017$+$0118          & 31.0500                 & 0.1667                  & 1,  3,  6                                & \nodata                           & 1.90                          & 11.83                             & 11.20                             & $-$0.65                         & 4.1                       & 1.2                       & 11.2                         & 5.04E+20                                & 1                        \\
                    1002                    & 03105$+$0076$+$0506          & 31.0500                 & 0.7583                  & 6,  7                                    & \nodata                           & 0.68                          & 50.62                             & 50.07                             & $-$0.44                         & 4.2                       & 1.3                       & 8.8                          & 5.82E+21                                & 2                        \\
                    1003                    & 03105$+$0407$+$0063          & 31.0500                 & 4.0750                  & 5,  6                                    & \nodata                           & 0.24                          & 6.29                              & 5.68                              & $-$0.59                         & 5.8                       & 1.2                       & 8.4                          & 5.27E+20                                & 1                        \\
                    1004                    & 03106$-$0067$+$0088          & 31.0583                 & $-$0.6667                 & 6,  7                                    & \nodata                           & 0.24                          & 8.77                              & 8.41                              & $-$0.42                         & 4.1                       & 1.8                       & 8.3                          & 7.75E+20                                & 1                        \\
                    1005                    & 03107$+$0078$+$0512          & 31.0667                 & 0.7833                  & 6,  7                                    & \nodata                           & 0.60                          & 51.22                             & 50.26                             & $-$0.48                         & 6.5                       & 2.9                       & 11.2                         & 3.02E+21                                & 1                        \\
                    1006                    & 03107$+$0313$+$0103          & 31.0750                 & 3.1333                  & 5,  6                                    & \nodata                           & 0.24                          & 10.31                             & 9.32                              & $-$0.46                         & 6.8                       & 3.4                       & 10.0                         & 4.08E+21                                & 1                        \\
                    1007                    & 03107$+$0524$+$0084          & 31.0750                 & 5.2417                  & 5,  6                                    & \nodata                           & 0.24                          & 8.41                              & 7.40                              & $-$0.51                         & 2.7                       & 3.4                       & 6.9                          & 6.93E+21                                & 1                        \\
                    1008                    & 03107$-$0044$+$0085          & 31.0750                 & $-$0.4417                 & 6                                        & \nodata                           & 0.24                          & 8.50                              & 8.14                              & $-$0.32                         & 4.3                       & 1.3                       & 10.9                         & 6.56E+20                                & 1                        \\
                    1009                    & 03108$-$0042$+$0086          & 31.0833                 & $-$0.4167                 & 6                                        & \nodata                           & 0.24                          & 8.61                              & 7.96                              & $-$0.68                         & 4.3                       & 1.3                       & 9.1                          & 5.32E+20                                & 1                        \\
                    1010                    & 03109$-$0339$+$0100          & 31.0917                 & $-$3.3917                 & 5,  6                                    & \nodata                           & 0.24                          & 10.00                             & 9.83                              & $-$0.23                         & 5.2                       & 2.1                       & 10.3                         & 7.41E+20                                & 1                        \\
                    1011                    & 03110$+$0520$+$0084          & 31.1000                 & 5.2000                  & 5,  6                                    & \nodata                           & 0.24                          & 8.37                              & 7.29                              & $-$0.54                         & 10.4                      & 3.5                       & 15.4                         & 3.96E+21                                & 1                        \\
                    1012                    & 03114$+$0313$+$0103          & 31.1417                 & 3.1333                  & 5,  6                                    & \nodata                           & 0.24                          & 10.32                             & 9.78                              & $-$0.55                         & 7.4                       & 5.1                       & 13.3                         & 3.09E+21                                & 1                        \\
                    1013                    & 03116$+$0451$+$0099          & 31.1583                 & 4.5083                  & 5,  6                                    & \nodata                           & 0.24                          & 9.89                              & 9.55                              & $-$0.33                         & 8.2                       & 3.6                       & 12.4                         & 1.99E+21                                & 1                        \\
                    1014                    & 03116$+$0522$+$0082          & 31.1583                 & 5.2167                  & 5,  6                                    & \nodata                           & 0.24                          & 8.22                              & 7.18                              & $-$0.60                         & 9.9                       & 4.8                       & 17.2                         & 5.22E+21                                & 1                        \\
                    1015                    & 03119$-$0183$+$1132          & 31.1917                 & $-$1.8333                 & 5,  6                                    & \nodata                           & 6.73                          & 113.21                            & 112.76                            & $-$0.47                         & 4.7                       & 1.1                       & 10.0                         & 4.83E+20                                & 1                        \\
                    1016                    & 03121$+$0522$+$0082          & 31.2083                 & 5.2167                  & 5,  6                                    & \nodata                           & 0.24                          & 8.18                              & 7.50                              & $-$0.54                         & 8.0                       & 4.8                       & 13.3                         & 3.64E+21                                & 1                        \\
                    1017                    & 03123$+$0207$+$0042          & 31.2333                 & 2.0750                  & 5,  6                                    & \nodata                           & 0.24                          & 4.19                              & 3.58                              & $-$0.67                         & 3.8                       & 1.8                       & 8.0                          & 8.32E+20                                & 1                        \\
                    1018                    & 03127$+$0016$+$1037          & 31.2667                 & 0.1583                  & 6,  7                                    & \nodata                           & 5.09                          & 103.72                            & 103.00                            & $-$0.28                         & 9.9                       & 2.9                       & 13.9                         & 3.85E+21                                & 1                        \\
                    1019                    & 03128$+$0311$+$0104          & 31.2833                 & 3.1083                  & 5,  6                                    & \nodata                           & 0.24                          & 10.41                             & 9.62                              & $-$0.47                         & 10.6                      & 4.6                       & 14.4                         & 4.63E+21                                & 1                        \\
                    1020                    & 03132$+$0418$+$0108          & 31.3167                 & 4.1833                  & 5,  6                                    & \nodata                           & 0.24                          & 10.77                             & 9.49                              & $-$1.54                         & 4.1                       & 3.1                       & 9.9                          & 1.38E+21                                & 1                        \\
                    1021                    & 03134$+$0411$+$0106          & 31.3417                 & 4.1083                  & 5,  6                                    & \nodata                           & 0.24                          & 10.60                             & 10.00                             & $-$0.61                         & 6.0                       & 4.0                       & 9.8                          & 2.43E+21                                & 1                        \\
                    1022                    & 03138$+$0018$+$0125          & 31.3833                 & 0.1833                  & 6,  7                                    & \nodata                           & 1.89                          & 12.54                             & 11.70                             & $-$0.64                         & 3.3                       & 1.0                       & 8.1                          & 5.58E+20                                & 1                        \\
                    1023                    & 03138$+$0277$+$0075          & 31.3833                 & 2.7750                  & 5,  6                                    & \nodata                           & 0.24                          & 7.52                              & 7.18                              & $-$0.37                         & 6.6                       & 4.8                       & 12.1                         & 2.75E+21                                & 1                        \\
                    1024                    & 03138$+$0412$+$0105          & 31.3833                 & 4.1250                  & 5,  6                                    & \nodata                           & 0.24                          & 10.52                             & 10.04                             & $-$0.46                         & 6.0                       & 4.7                       & 10.7                         & 3.20E+21                                & 1                        \\
                    1025                    & 03139$+$0028$+$0960          & 31.3917                 & 0.2833                  & 6,  7                                    & 10,  12,  13                      & 5.10                          & 96.04                             & 95.01                             & $-$0.33                         & 4.6                       & 2.2                       & 8.9                          & 2.66E+22                                & 2                        \\
                    1026                    & 03141$+$0031$+$0971          & 31.4083                 & 0.3083                  & 1,  6,  7                                & 10,  13                           & 5.09                          & 97.06                             & 96.06                             & $-$0.25                         & 6.5                       & 3.2                       & 10.6                         & 5.20E+22                                & 2                        \\
                    1027                    & 03142$+$0414$+$0104          & 31.4167                 & 4.1417                  & 5,  6                                    & \nodata                           & 0.24                          & 10.39                             & 9.94                              & $-$0.30                         & 5.7                       & 5.1                       & 15.5                         & 4.69E+21                                & 1                        \\
                    1028                    & 03142$+$0524$+$0080          & 31.4250                 & 5.2417                  & 5,  6                                    & \nodata                           & 0.24                          & 8.04                              & 7.17                              & $-$0.50                         & 14.0                      & 4.8                       & 20.5                         & 5.50E+21                                & 1                        \\
                    1029                    & 03142$-$0121$+$0130          & 31.4250                 & $-$1.2083                 & 5,  6                                    & \nodata                           & 0.24                          & 12.99                             & 12.58                             & $-$0.40                         & 6.4                       & 2.1                       & 12.1                         & 1.07E+21                                & 1                        \\
                    1030                    & 03149$+$0453$+$0101          & 31.4917                 & 4.5333                  & 5,  6                                    & \nodata                           & 0.24                          & 10.08                             & 9.13                              & $-$0.77                         & 6.7                       & 2.3                       & 15.7                         & 1.51E+21                                & 1                        \\
                    1031                    & 03152$+$0095$+$0102          & 31.5167                 & 0.9500                  & 6                                        & \nodata                           & 0.24                          & 10.17                             & 9.77                              & $-$0.46                         & 4.2                       & 1.3                       & 8.8                          & 4.97E+20                                & 1                        \\
                    1032                    & 03156$+$0452$+$0098          & 31.5583                 & 4.5250                  & 5,  6                                    & \nodata                           & 0.24                          & 9.76                              & 9.24                              & $-$0.39                         & 4.3                       & 1.8                       & 10.2                         & 1.12E+21                                & 1                        \\
                    1033                    & 03166$+$0257$+$0074          & 31.6583                 & 2.5750                  & 5,  6                                    & \nodata                           & 0.24                          & 7.36                              & 6.45                              & $-$0.65                         & 4.7                       & 5.2                       & 13.6                         & 4.52E+21                                & 1                        \\
                    1034                    & 03168$+$0313$+$0102          & 31.6833                 & 3.1333                  & 5,  6                                    & \nodata                           & 0.24                          & 10.20                             & 9.47                              & $-$0.35                         & 12.4                      & 3.8                       & 19.5                         & 4.87E+21                                & 1                        \\
                    1035                    & 03170$+$0271$+$0074          & 31.7000                 & 2.7083                  & 5,  6                                    & \nodata                           & 0.24                          & 7.45                              & 6.91                              & $-$0.42                         & 6.1                       & 2.8                       & 12.8                         & 1.90E+21                                & 1                        \\
                    1036                    & 03174$+$0335$+$0092          & 31.7417                 & 3.3500                  & 5,  6                                    & \nodata                           & 0.24                          & 9.19                              & 8.61                              & $-$0.30                         & 10.5                      & 1.8                       & 21.4                         & 2.17E+21                                & 1                        \\
                    1037                    & 03177$+$0329$+$0096          & 31.7750                 & 3.2917                  & 5,  6                                    & \nodata                           & 0.24                          & 9.59                              & 8.90                              & $-$0.29                         & 11.5                      & 2.7                       & 19.5                         & 3.83E+21                                & 1                        \\
                    1038                    & 03182$-$0362$+$0035          & 31.8167                 & $-$3.6167                 & 5,  6                                    & \nodata                           & 0.24                          & 3.54                              & 3.27                              & $-$0.35                         & 5.1                       & 1.5                       & 12.8                         & 5.27E+20                                & 1                        \\
                    1039                    & 03184$+$0262$+$0083          & 31.8417                 & 2.6250                  & 5,  6                                    & \nodata                           & 0.24                          & 8.34                              & 7.90                              & $-$0.36                         & 5.1                       & 1.3                       & 9.3                          & 5.36E+21                                & 2                        \\
                    1040                    & 03191$-$0052$+$0711          & 31.9083                 & $-$0.5167                 & 5,  6,  7                                & \nodata                           & 3.95                          & 71.05                             & 70.47                             & $-$0.36                         & 8.3                       & 3.9                       & 14.5                         & 3.63E+21                                & 1                        \\
                    1041                    & 03195$+$0103$+$0535          & 31.9500                 & 1.0333                  & 5,  6                                    & \nodata                           & 0.40                          & 53.49                             & 53.02                             & $-$0.49                         & 8.4                       & 1.7                       & 12.0                         & 7.56E+20                                & 1                        \\
                    1042                    & 03197$+$0297$+$0082          & 31.9750                 & 2.9667                  & 5,  6                                    & \nodata                           & 0.24                          & 8.22                              & 7.74                              & $-$0.38                         & 7.0                       & 2.4                       & 11.0                         & 1.14E+22                                & 2                        \\
                    1043                    & 03204$+$0005$+$0960          & 32.0417                 & 0.0500                  & 1,  6,  7                                & \nodata                           & 5.11                          & 96.04                             & 93.63                             & $-$0.34                         & 16.4                      & 6.8                       & 25.9                         & 3.80E+22                                & 1                        \\
                    1044                    & 03207$+$0065$+$0297          & 32.0750                 & 0.6500                  & 5,  6                                    & \nodata                           & 2.08                          & 29.67                             & 29.40                             & $-$0.26                         & 5.1                       & 1.4                       & 11.9                         & 6.34E+20                                & 1                        \\
                    1045                    & 03213$+$0032$+$0973          & 32.1333                 & 0.3167                  & 6,  7                                    & \nodata                           & 5.14                          & 97.27                             & 96.09                             & $-$0.64                         & 5.9                       & 2.7                       & 11.3                         & 2.54E+21                                & 1                        \\
                    1046                    & 03214$+$0013$+$0943          & 32.1417                 & 0.1333                  & 1,  6,  7                                & \nodata                           & 5.13                          & 94.35                             & 92.97                             & $-$0.38                         & 12.5                      & 6.5                       & 19.8                         & 1.68E+22                                & 1                        \\
                    1047                    & 03217$+$0266$+$0088          & 32.1750                 & 2.6583                  & 5,  6                                    & \nodata                           & 0.24                          & 8.80                              & 7.86                              & $-$0.74                         & 3.9                       & 2.3                       & 8.8                          & 1.51E+21                                & 1                        \\
                    1048                    & 03225$+$0051$+$0221          & 32.2500                 & 0.5083                  & 5,  6                                    & \nodata                           & 2.07                          & 22.11                             & 21.83                             & $-$0.33                         & 5.8                       & 2.2                       & 9.9                          & 8.80E+20                                & 1                        \\
                    1049                    & 03227$+$0040$+$0094          & 32.2667                 & 0.4000                  & 5,  6                                    & \nodata                           & 0.24                          & 9.41                              & 8.97                              & $-$0.46                         & 5.4                       & 2.2                       & 11.4                         & 9.97E+20                                & 1                        \\
                    1050                    & 03228$+$0301$+$0083          & 32.2833                 & 3.0083                  & 5,  6                                    & \nodata                           & 0.24                          & 8.31                              & 7.76                              & $-$0.41                         & 5.5                       & 1.4                       & 11.4                         & 8.86E+20                                & 1                        \\
                    1051                    & 03231$+$0091$+$0212          & 32.3083                 & 0.9083                  & 5,  6                                    & \nodata                           & 2.04                          & 21.24                             & 20.82                             & $-$0.48                         & 4.1                       & 0.9                       & 10.6                         & 3.41E+20                                & 1                        \\
                    1052                    & 03236$+$0091$+$0216          & 32.3583                 & 0.9083                  & 5,  6                                    & \nodata                           & 2.04                          & 21.59                             & 20.94                             & $-$0.62                         & 4.8                       & 1.2                       & 9.6                          & 5.78E+20                                & 1                        \\
                    1053                    & 03237$+$0057$+$0084          & 32.3667                 & 0.5750                  & 5,  6                                    & \nodata                           & 0.24                          & 8.37                              & 7.74                              & $-$0.53                         & 4.9                       & 2.6                       & 10.9                         & 1.60E+21                                & 1                        \\
                    1054                    & 03237$+$0058$+$0211          & 32.3667                 & 0.5833                  & 5,  6                                    & \nodata                           & 2.07                          & 21.11                             & 20.70                             & $-$0.36                         & 7.3                       & 3.1                       & 16.2                         & 1.95E+21                                & 1                        \\
                    1055                    & 03237$+$0087$+$0219          & 32.3667                 & 0.8750                  & 5,  6                                    & \nodata                           & 2.04                          & 21.91                             & 21.26                             & $-$0.55                         & 5.8                       & 1.5                       & 13.4                         & 8.56E+20                                & 1                        \\
                    1056                    & 03237$+$0016$+$0092          & 32.3750                 & 0.1583                  & 5,  6                                    & \nodata                           & 0.24                          & 9.16                              & 8.86                              & $-$0.31                         & 3.1                       & 2.2                       & 7.8                          & 1.12E+21                                & 1                        \\
                    1057                    & 03241$+$0021$+$0093          & 32.4083                 & 0.2083                  & 5,  6                                    & \nodata                           & 0.24                          & 9.26                              & 8.55                              & $-$0.66                         & 3.9                       & 1.4                       & 9.4                          & 6.93E+20                                & 1                        \\
                    1058                    & 03241$+$0092$+$0219          & 32.4083                 & 0.9167                  & 5,  6                                    & \nodata                           & 2.03                          & 21.86                             & 21.06                             & $-$0.69                         & 4.6                       & 1.6                       & 9.8                          & 8.40E+20                                & 1                        \\
                    1059                    & 03242$+$0352$+$0080          & 32.4167                 & 3.5250                  & 5,  6                                    & \nodata                           & 0.24                          & 8.00                              & 7.49                              & $-$0.43                         & 5.9                       & 1.6                       & 12.0                         & 9.07E+20                                & 1                        \\
                    1060                    & 03245$+$0064$+$0083          & 32.4500                 & 0.6417                  & 5,  6                                    & \nodata                           & 0.24                          & 8.29                              & 7.84                              & $-$0.49                         & 5.5                       & 1.9                       & 9.6                          & 8.52E+20                                & 1                        \\
                    1061                    & 03247$+$0047$+$0090          & 32.4750                 & 0.4750                  & 5,  6                                    & \nodata                           & 0.24                          & 9.00                              & 8.37                              & $-$0.62                         & 5.2                       & 1.5                       & 9.1                          & 7.20E+20                                & 1                        \\
                    1062                    & 03252$+$0058$+$0086          & 32.5250                 & 0.5833                  & 5,  6                                    & \nodata                           & 0.24                          & 8.65                              & 8.22                              & $-$0.49                         & 4.7                       & 2.5                       & 8.9                          & 1.13E+21                                & 1                        \\
                    1063                    & 03254$+$0122$+$1032          & 32.5417                 & 1.2250                  & 5,  6                                    & \nodata                           & 5.73                          & 103.24                            & 102.74                            & $-$0.40                         & 4.2                       & 1.0                       & 8.8                          & 5.89E+20                                & 1                        \\
                    1064                    & 03265$+$0267$+$0105          & 32.6500                 & 2.6750                  & 5,  6                                    & \nodata                           & 0.24                          & 10.49                             & 9.76                              & $-$0.40                         & 5.1                       & 1.8                       & 12.0                         & 1.54E+21                                & 1                        \\
                    1065                    & 03265$+$0326$+$0085          & 32.6500                 & 3.2583                  & 5,  6                                    & \nodata                           & 0.24                          & 8.47                              & 8.17                              & $-$0.31                         & 5.7                       & 1.0                       & 12.7                         & 4.58E+20                                & 1                        \\
                    1066                    & 03287$+$0251$+$0115          & 32.8750                 & 2.5083                  & 5,  6                                    & \nodata                           & 0.24                          & 11.51                             & 10.44                             & $-$0.74                         & 4.5                       & 1.9                       & 9.6                          & 1.28E+21                                & 1                        \\
                    1067                    & 03290$+$0028$+$0090          & 32.9000                 & 0.2833                  & 3,  5,  6,  7                            & \nodata                           & 0.24                          & 8.98                              & 8.65                              & $-$0.29                         & 6.3                       & 2.7                       & 11.6                         & 1.59E+21                                & 1                        \\
                    1068                    & 03292$+$0024$+$0089          & 32.9250                 & 0.2417                  & 3,  5,  6,  7                            & \nodata                           & 0.24                          & 8.93                              & 8.46                              & $-$0.44                         & 5.7                       & 2.6                       & 12.7                         & 1.41E+21                                & 1                        \\
                    1069                    & 03302$-$0379$+$0085          & 33.0167                 & $-$3.7917                 & 5,  6                                    & \nodata                           & 0.24                          & 8.53                              & 7.99                              & $-$0.55                         & 3.9                       & 1.0                       & 7.8                          & 4.57E+20                                & 1                        \\
                    1070                    & 03306$+$0166$+$0135          & 33.0583                 & 1.6583                  & 5,  6                                    & \nodata                           & 0.24                          & 13.48                             & 13.01                             & $-$0.43                         & 4.4                       & 1.8                       & 9.5                          & 9.48E+20                                & 1                        \\
                    1071                    & 03307$+$0277$+$0114          & 33.0667                 & 2.7667                  & 5,  6                                    & \nodata                           & 0.24                          & 11.43                             & 10.96                             & $-$0.52                         & 3.2                       & 1.7                       & 7.5                          & 7.51E+20                                & 1                        \\
                    1072                    & 03307$+$0272$+$0115          & 33.0750                 & 2.7167                  & 5,  6                                    & \nodata                           & 0.24                          & 11.50                             & 11.00                             & $-$0.52                         & 3.5                       & 3.1                       & 7.1                          & 2.41E+21                                & 1                        \\
                    1073                    & 03307$-$0227$+$0070          & 33.0750                 & $-$2.2750                 & 6                                        & \nodata                           & 0.24                          & 7.05                              & 6.29                              & $-$0.97                         & 3.1                       & 1.1                       & 8.8                          & 4.00E+20                                & 1                        \\
                    1074                    & 03309$-$0232$+$0070          & 33.0917                 & $-$2.3167                 & 6                                        & \nodata                           & 0.24                          & 7.01                              & 6.47                              & $-$0.57                         & 4.9                       & 1.1                       & 12.6                         & 4.73E+20                                & 1                        \\
                    1075                    & 03312$-$0227$+$0067          & 33.1167                 & $-$2.2750                 & 6                                        & \nodata                           & 0.24                          & 6.71                              & 6.39                              & $-$0.31                         & 4.3                       & 1.1                       & 8.5                          & 5.09E+20                                & 1                        \\
                    1076                    & 03320$+$0000$+$0103          & 33.2000                 & 0.0000                  & 1,  3,  5,  6                            & \nodata                           & 2.05                          & 10.32                             & 9.77                              & $-$0.49                         & 5.9                       & 2.4                       & 12.1                         & 1.34E+21                                & 1                        \\
                    1077                    & 03322$+$0006$+$0103          & 33.2167                 & 0.0583                  & 3,  5,  6                                & \nodata                           & 0.24                          & 10.27                             & 9.30                              & $-$0.40                         & 6.9                       & 2.3                       & 12.9                         & 2.82E+21                                & 1                        \\
                    1078                    & 03329$+$0082$+$0105          & 33.2917                 & 0.8167                  & 3,  5,  6                                & \nodata                           & 0.24                          & 10.48                             & 9.48                              & $-$0.75                         & 5.3                       & 2.1                       & 11.9                         & 1.35E+21                                & 1                        \\
                    1079                    & 03331$+$0342$+$0086          & 33.3083                 & 3.4167                  & 5,  6                                    & \nodata                           & 0.24                          & 8.63                              & 8.22                              & $-$0.37                         & 3.4                       & 1.7                       & 8.4                          & 8.91E+20                                & 1                        \\
                    1080                    & 03332$+$0081$+$0106          & 33.3167                 & 0.8083                  & 3,  5,  6                                & \nodata                           & 1.86                          & 10.58                             & 9.48                              & $-$0.80                         & 5.4                       & 2.9                       & 11.9                         & 2.08E+21                                & 1                        \\
                    1081                    & 03335$+$0086$+$0105          & 33.3500                 & 0.8583                  & 3,  5,  6                                & \nodata                           & 0.24                          & 10.48                             & 9.34                              & $-$0.93                         & 5.3                       & 1.6                       & 10.4                         & 9.26E+20                                & 1                        \\
                    1082                    & 03337$+$0031$+$0122          & 33.3667                 & 0.3083                  & 1,  5,  6                                & \nodata                           & 1.56                          & 12.22                             & 11.26                             & $-$0.79                         & 5.8                       & 1.7                       & 12.2                         & 9.93E+20                                & 1                        \\
                    1083                    & 03337$+$0088$+$0105          & 33.3667                 & 0.8833                  & 3,  5,  6                                & \nodata                           & 0.24                          & 10.55                             & 9.19                              & $-$0.92                         & 4.9                       & 1.6                       & 8.7                          & 1.08E+21                                & 1                        \\
                    1084                    & 03339$-$0015$+$0105          & 33.3917                 & $-$0.1500                 & 3,  5,  6                                & \nodata                           & 0.24                          & 10.48                             & 10.10                             & $-$0.34                         & 5.5                       & 2.1                       & 11.9                         & 1.09E+21                                & 1                        \\
                    1085                    & 03342$+$0000$+$0106          & 33.4250                 & 0.0000                  & 1,  3,  5,  6                            & \nodata                           & 2.05                          & 10.61                             & 10.03                             & $-$0.33                         & 5.3                       & 2.8                       & 8.6                          & 2.71E+21                                & 1                        \\
                    1086                    & 03342$+$0075$+$0106          & 33.4250                 & 0.7500                  & 3,  5,  6                                & \nodata                           & 1.89                          & 10.61                             & 9.96                              & $-$0.54                         & 5.6                       & 1.8                       & 10.8                         & 9.99E+20                                & 1                        \\
                    1087                    & 03342$-$0011$+$0108          & 33.4250                 & $-$0.1083                 & 3,  5,  6                                & \nodata                           & 0.24                          & 10.77                             & 10.13                             & $-$0.48                         & 5.3                       & 2.5                       & 10.5                         & 1.68E+21                                & 1                        \\
                    1088                    & 03345$+$0032$+$0575          & 33.4500                 & 0.3250                  & 5,  6                                    & \nodata                           & 0.08                          & 57.52                             & 56.20                             & $-$1.05                         & 4.3                       & 2.2                       & 8.6                          & 1.40E+21                                & 1                        \\
                    1089                    & 03351$-$0011$+$0112          & 33.5083                 & $-$0.1083                 & 3,  5,  6                                & \nodata                           & 0.24                          & 11.20                             & 10.23                             & $-$0.44                         & 5.4                       & 2.3                       & 14.2                         & 2.61E+21                                & 1                        \\
                    1090                    & 03354$+$0027$+$0573          & 33.5417                 & 0.2750                  & 5,  6                                    & \nodata                           & 0.02                          & 57.30                             & 56.19                             & $-$0.79                         & 3.7                       & 1.6                       & 7.6                          & 1.12E+21                                & 1                        \\
                    1091                    & 03356$+$0030$+$0574          & 33.5583                 & 0.3000                  & 5,  6                                    & \nodata                           & 0.04                          & 57.41                             & 56.42                             & $-$0.60                         & 4.2                       & 2.7                       & 10.9                         & 2.24E+21                                & 1                        \\
                    1092                    & 03362$-$0047$+$0627          & 33.6167                 & $-$0.4667                 & 3,  5,  6,  7                            & \nodata                           & 4.56                          & 62.75                             & 62.19                             & $-$0.37                         & 7.9                       & 2.8                       & 13.0                         & 2.20E+21                                & 1                        \\
                    1093                    & 03363$+$0110$+$0095          & 33.6333                 & 1.1000                  & 5,  6                                    & \nodata                           & 0.24                          & 9.54                              & 9.04                              & $-$0.53                         & 5.8                       & 1.6                       & 12.2                         & 7.02E+20                                & 1                        \\
                    1094                    & 03367$+$0189$+$0165          & 33.6667                 & 1.8917                  & 5,  6                                    & \nodata                           & 0.59                          & 16.45                             & 15.80                             & $-$0.81                         & 4.7                       & 1.4                       & 11.6                         & 5.35E+20                                & 1                        \\
                    1095                    & 03371$+$0310$+$0132          & 33.7083                 & 3.1000                  & 5,  6                                    & \nodata                           & 0.24                          & 13.17                             & 12.40                             & $-$0.90                         & 3.6                       & 1.5                       & 8.6                          & 6.03E+20                                & 1                        \\
                    1096                    & 03375$+$0002$+$0888          & 33.7500                 & 0.0167                  & 5,  6,  7                                & \nodata                           & 5.42                          & 88.79                             & 88.17                             & $-$0.32                         & 8.5                       & 3.4                       & 14.9                         & 3.59E+21                                & 1                        \\
                    1097                    & 03382$+$0314$+$0196          & 33.8250                 & 3.1417                  & 5,  6                                    & \nodata                           & 0.24                          & 19.62                             & 18.88                             & $-$0.46                         & 4.4                       & 1.7                       & 9.7                          & 1.30E+21                                & 1                        \\
                    1098                    & 03383$+$0181$+$0114          & 33.8333                 & 1.8083                  & 5,  6                                    & \nodata                           & 0.24                          & 11.43                             & 10.88                             & $-$0.37                         & 4.3                       & 1.2                       & 8.8                          & 8.05E+20                                & 1                        \\
                    1099                    & 03387$+$0240$+$0144          & 33.8667                 & 2.4000                  & 5,  6                                    & \nodata                           & 0.24                          & 14.38                             & 13.86                             & $-$0.56                         & 5.4                       & 2.0                       & 11.9                         & 9.29E+20                                & 1                        \\
                    1100                    & 03412$+$0005$+$0116          & 34.1167                 & 0.0500                  & 5,  6,  7                                & \nodata                           & 2.01                          & 11.65                             & 10.80                             & $-$0.97                         & 4.9                       & 1.4                       & 9.4                          & 5.72E+20                                & 1                        \\
                    1101                    & 03419$+$0112$+$0672          & 34.1917                 & 1.1250                  & 5,  6                                    & \nodata                           & 4.66                          & 67.17                             & 66.80                             & $-$0.22                         & 8.4                       & 2.1                       & 12.6                         & 1.71E+21                                & 1                        \\
                    1102                    & 03421$+$0067$+$0140          & 34.2083                 & 0.6750                  & 5,  6                                    & \nodata                           & 0.33                          & 13.99                             & 13.31                             & $-$0.62                         & 3.8                       & 1.2                       & 9.6                          & 5.82E+20                                & 1                        \\
                    1103                    & 03422$+$0016$+$0578          & 34.2250                 & 0.1583                  & 3,  5,  6,  7                            & 14,  17                           & 2.96                          & 57.76                             & 56.74                             & $-$0.25                         & 11.7                      & 2.8                       & 15.6                         & 4.80E+22                                & 2                        \\
                    1104                    & 03424$+$0009$+$0566          & 34.2417                 & 0.0917                  & 3,  5,  6                                & \nodata                           & 2.96                          & 56.59                             & 55.16                             & $-$0.46                         & 6.3                       & 1.6                       & 15.4                         & 2.03E+22                                & 2                        \\
                    1105                    & 03425$+$0016$+$0583          & 34.2500                 & 0.1583                  & 1,  3,  5,  6,  7                        & 14,  17                           & 2.96                          & 58.34                             & 56.47                             & $-$0.30                         & 11.9                      & 3.4                       & 21.5                         & 1.07E+23                                & 2                        \\
                    1106                    & 03427$-$0002$+$0554          & 34.2667                 & $-$0.0167                 & 5,  6                                    & \nodata                           & 2.95                          & 55.42                             & 54.77                             & $-$0.33                         & 6.1                       & 2.6                       & 13.4                         & 2.64E+21                                & 1                        \\
                    1107                    & 03430$+$0287$+$0149          & 34.3000                 & 2.8750                  & 5,  6                                    & \nodata                           & 0.24                          & 14.87                             & 14.45                             & $-$0.40                         & 4.6                       & 1.2                       & 8.6                          & 5.69E+20                                & 1                        \\
                    1108                    & 03431$-$0066$+$0127          & 34.3083                 & $-$0.6583                 & 5,  6                                    & \nodata                           & 0.24                          & 12.66                             & 11.68                             & $-$0.74                         & 6.6                       & 3.4                       & 10.1                         & 2.56E+21                                & 1                        \\
                    1109                    & 03432$+$0003$+$0563          & 34.3167                 & 0.0333                  & 3,  5,  6                                & \nodata                           & 2.96                          & 56.31                             & 55.53                             & $-$0.50                         & 4.5                       & 3.0                       & 9.1                          & 2.60E+21                                & 1                        \\
                    1110                    & 03432$+$0139$+$0159          & 34.3250                 & 1.3917                  & 5,  6                                    & \nodata                           & 0.78                          & 15.85                             & 14.59                             & $-$0.84                         & 3.3                       & 2.6                       & 9.3                          & 2.10E+21                                & 1                        \\
                    1111                    & 03440$+$0021$+$0575          & 34.4000                 & 0.2083                  & 3,  5,  6,  7                            & \nodata                           & 2.96                          & 57.50                             & 56.41                             & $-$0.38                         & 4.4                       & 1.1                       & 8.5                          & 1.07E+22                                & 2                        \\
                    1112                    & 03440$-$0098$+$0133          & 34.4000                 & $-$0.9833                 & 5,  6,  7                                & \nodata                           & 0.24                          & 13.29                             & 12.55                             & $-$1.03                         & 3.5                       & 1.6                       & 8.8                          & 4.12E+21                                & 2                        \\
                    1113                    & 03442$-$0064$+$0131          & 34.4167                 & $-$0.6417                 & 5,  6,  7                                & \nodata                           & 0.24                          & 13.13                             & 11.94                             & $-$1.45                         & 3.7                       & 1.3                       & 13.4                         & 4.00E+21                                & 2                        \\
                    1114                    & 03452$+$0182$+$0124          & 34.5167                 & 1.8167                  & 5,  6                                    & \nodata                           & 0.24                          & 12.36                             & 11.47                             & $-$0.70                         & 3.9                       & 1.1                       & 8.6                          & 6.30E+20                                & 1                        \\
                    1115                    & 03457$+$0044$+$0313          & 34.5667                 & 0.4417                  & 5,  6,  7                                & \nodata                           & 1.74                          & 31.31                             & 30.75                             & $-$0.64                         & 5.3                       & 1.0                       & 9.8                          & 3.79E+20                                & 1                        \\
                    1116                    & 03457$-$0138$+$0136          & 34.5750                 & $-$1.3833                 & 6                                        & \nodata                           & 0.24                          & 13.62                             & 13.38                             & $-$0.26                         & 4.2                       & 1.4                       & 8.5                          & 4.78E+21                                & 2                        \\
                    1117                    & 03470$+$0080$+$0125          & 34.7000                 & 0.8000                  & 5,  6                                    & \nodata                           & 0.89                          & 12.46                             & 11.50                             & $-$0.82                         & 4.7                       & 2.0                       & 10.9                         & 1.09E+21                                & 1                        \\
                    1118                    & 03470$+$0114$+$0142          & 34.7000                 & 1.1417                  & 5,  6                                    & \nodata                           & 0.38                          & 14.24                             & 13.08                             & $-$0.80                         & 3.5                       & 2.0                       & 7.0                          & 1.67E+21                                & 1                        \\
                    1119                    & 03471$+$0036$+$0427          & 34.7083                 & 0.3583                  & 3,  5,  6                                & \nodata                           & 1.09                          & 42.69                             & 42.18                             & $-$0.38                         & 6.1                       & 1.9                       & 10.1                         & 1.23E+21                                & 1                        \\
                    1120                    & 03473$+$0232$+$0207          & 34.7333                 & 2.3167                  & 5,  6                                    & \nodata                           & 0.71                          & 20.74                             & 20.27                             & $-$0.47                         & 3.7                       & 1.2                       & 9.1                          & 5.11E+20                                & 1                        \\
                    1121                    & 03475$+$0016$+$0568          & 34.7500                 & 0.1583                  & 5,  6                                    & \nodata                           & 2.96                          & 56.84                             & 56.31                             & $-$0.39                         & 5.1                       & 2.0                       & 8.9                          & 1.31E+21                                & 1                        \\
                    1122                    & 03478$-$0017$+$0131          & 34.7833                 & $-$0.1667                 & 3,  5,  6                                & \nodata                           & 0.24                          & 13.13                             & 12.54                             & $-$0.47                         & 4.8                       & 1.3                       & 9.0                          & 7.57E+20                                & 1                        \\
                    1123                    & 03478$-$0098$+$0403          & 34.7833                 & $-$0.9833                 & 5,  6                                    & \nodata                           & 2.44                          & 40.26                             & 39.58                             & $-$0.44                         & 6.6                       & 2.9                       & 12.9                         & 2.38E+21                                & 1                        \\
                    1124                    & 03479$+$0089$+$0127          & 34.7917                 & 0.8917                  & 5,  6                                    & \nodata                           & 0.33                          & 12.70                             & 11.52                             & $-$0.54                         & 5.1                       & 2.0                       & 12.8                         & 2.17E+21                                & 1                        \\
                    1125                    & 03482$+$0102$+$0133          & 34.8167                 & 1.0250                  & 1,  5,  6                                & \nodata                           & 0.35                          & 13.29                             & 12.33                             & $-$0.46                         & 6.2                       & 2.5                       & 15.5                         & 2.80E+21                                & 1                        \\
                    1126                    & 03484$-$0126$+$0419          & 34.8417                 & $-$1.2583                 & 5,  6                                    & \nodata                           & 2.45                          & 41.89                             & 41.12                             & $-$0.53                         & 4.8                       & 1.5                       & 8.8                          & 7.62E+21                                & 2                        \\
                    1127                    & 03487$+$0112$+$0138          & 34.8667                 & 1.1250                  & 5,  6                                    & \nodata                           & 0.24                          & 13.77                             & 12.43                             & $-$0.71                         & 4.3                       & 1.8                       & 9.6                          & 1.60E+21                                & 1                        \\
                    1128                    & 03497$+$0058$+$0130          & 34.9750                 & 0.5833                  & 5,  6                                    & \nodata                           & 1.78                          & 13.02                             & 12.23                             & $-$0.64                         & 5.6                       & 1.5                       & 9.9                          & 8.15E+20                                & 1                        \\
                    1129                    & 03501$-$0078$+$0141          & 35.0083                 & $-$0.7833                 & 3,  5,  6                                & \nodata                           & 0.44                          & 14.09                             & 13.55                             & $-$0.41                         & 6.3                       & 2.0                       & 12.0                         & 1.28E+21                                & 1                        \\
                    1130                    & 03502$+$0077$+$0131          & 35.0167                 & 0.7667                  & 5,  6                                    & \nodata                           & 0.33                          & 13.12                             & 12.68                             & $-$0.25                         & 5.5                       & 1.2                       & 10.3                         & 9.10E+20                                & 1                        \\
                    1131                    & 03502$-$0074$+$0139          & 35.0167                 & $-$0.7417                 & 3,  5,  6                                & \nodata                           & 0.44                          & 13.90                             & 13.34                             & $-$0.38                         & 6.3                       & 2.4                       & 12.6                         & 1.78E+21                                & 1                        \\
                    1132                    & 03507$-$0243$+$0394          & 35.0667                 & $-$2.4333                 & 6                                        & \nodata                           & 1.10                          & 39.40                             & 39.03                             & $-$0.29                         & 5.5                       & 1.8                       & 13.5                         & 1.11E+21                                & 1                        \\
                    1133                    & 03507$+$0023$+$0135          & 35.0750                 & 0.2333                  & 3,  5,  6                                & \nodata                           & 0.25                          & 13.46                             & 12.69                             & $-$0.39                         & 6.8                       & 2.6                       & 13.0                         & 2.63E+21                                & 1                        \\
                    1134                    & 03509$+$0028$+$0135          & 35.0917                 & 0.2833                  & 3,  5,  6                                & \nodata                           & 0.25                          & 13.48                             & 12.95                             & $-$0.47                         & 6.2                       & 3.2                       & 11.6                         & 1.95E+21                                & 1                        \\
                    1135                    & 03510$+$0019$+$0133          & 35.1000                 & 0.1917                  & 3,  5,  6                                & \nodata                           & 0.25                          & 13.33                             & 12.82                             & $-$0.29                         & 6.5                       & 2.0                       & 13.4                         & 1.75E+21                                & 1                        \\
                    1136                    & 03510$-$0132$+$0419          & 35.1000                 & $-$1.3250                 & 5,  6                                    & \nodata                           & 2.45                          & 41.88                             & 41.09                             & $-$0.50                         & 8.0                       & 3.2                       & 13.3                         & 2.69E+21                                & 1                        \\
                    1137                    & 03512$+$0019$+$0131          & 35.1250                 & 0.1917                  & 5,  6                                    & \nodata                           & 0.25                          & 13.06                             & 12.55                             & $-$0.42                         & 7.1                       & 2.6                       & 11.9                         & 1.56E+21                                & 1                        \\
                    1138                    & 03512$-$0168$+$0413          & 35.1250                 & $-$1.6833                 & 3,  6                                    & \nodata                           & 2.43                          & 41.29                             & 40.48                             & $-$0.45                         & 7.3                       & 2.1                       & 11.8                         & 1.47E+22                                & 2                        \\
                    1139                    & 03514$+$0058$+$0130          & 35.1417                 & 0.5833                  & 5,  6                                    & \nodata                           & 1.82                          & 12.99                             & 12.36                             & $-$0.51                         & 5.0                       & 2.4                       & 8.7                          & 1.53E+21                                & 1                        \\
                    1140                    & 03515$+$0055$+$0131          & 35.1500                 & 0.5500                  & 5,  6                                    & \nodata                           & 1.81                          & 13.06                             & 12.56                             & $-$0.42                         & 6.0                       & 2.1                       & 10.5                         & 1.22E+21                                & 1                        \\
                    1141                    & 03516$+$0040$+$0139          & 35.1583                 & 0.4000                  & 3,  5,  6                                & \nodata                           & 0.24                          & 13.92                             & 12.76                             & $-$0.74                         & 6.6                       & 2.7                       & 10.9                         & 2.15E+21                                & 1                        \\
                    1142                    & 03516$+$0052$+$0131          & 35.1583                 & 0.5250                  & 5,  6                                    & \nodata                           & 1.80                          & 13.14                             & 12.50                             & $-$0.40                         & 5.9                       & 2.0                       & 11.0                         & 1.51E+21                                & 1                        \\
                    1143                    & 03517$-$0228$+$0392          & 35.1667                 & $-$2.2833                 & 6                                        & \nodata                           & 1.08                          & 39.20                             & 37.77                             & $-$0.78                         & 5.6                       & 3.4                       & 10.4                         & 3.50E+21                                & 1                        \\
                    1144                    & 03517$-$0129$+$0133          & 35.1750                 & $-$1.2917                 & 5,  6                                    & \nodata                           & 0.24                          & 13.27                             & 12.59                             & $-$0.54                         & 5.5                       & 2.0                       & 11.3                         & 1.19E+21                                & 1                        \\
                    1145                    & 03518$+$0193$+$0089          & 35.1833                 & 1.9333                  & 5,  6                                    & \nodata                           & 0.24                          & 8.88                              & 8.18                              & $-$0.81                         & 4.4                       & 1.3                       & 8.2                          & 5.09E+20                                & 1                        \\
                    1146                    & 03520$-$0172$+$0423          & 35.2000                 & $-$1.7167                 & 3,  6                                    & \nodata                           & 0.91                          & 42.27                             & 40.65                             & $-$0.43                         & 14.4                      & 9.3                       & 18.7                         & 2.86E+22                                & 1                        \\
                    1147                    & 03521$+$0042$+$0137          & 35.2083                 & 0.4167                  & 5,  6                                    & \nodata                           & 0.24                          & 13.66                             & 12.29                             & $-$0.69                         & 5.6                       & 2.5                       & 12.3                         & 2.54E+21                                & 1                        \\
                    1148                    & 03521$-$0075$+$0344          & 35.2083                 & $-$0.7500                 & 1,  3,  5,  6,  7                        & \nodata                           & 2.20                          & 34.45                             & 31.53                             & $-$0.58                         & 20.9                      & 9.8                       & 27.6                         & 4.35E+22                                & 1                        \\
                    1149                    & 03526$+$0245$+$0150          & 35.2583                 & 2.4500                  & 5,  6                                    & \nodata                           & 0.24                          & 14.96                             & 14.38                             & $-$0.43                         & 4.6                       & 3.2                       & 7.7                          & 2.97E+21                                & 1                        \\
                    1150                    & 03526$-$0127$+$0136          & 35.2583                 & $-$1.2750                 & 1,  5,  6                                & \nodata                           & 0.42                          & 13.61                             & 13.19                             & $-$0.36                         & 5.7                       & 1.8                       & 11.3                         & 9.67E+20                                & 1                        \\
                    1151                    & 03528$+$0131$+$0131          & 35.2833                 & 1.3083                  & 5,  6                                    & \nodata                           & 0.24                          & 13.13                             & 12.24                             & $-$0.74                         & 4.3                       & 3.3                       & 8.9                          & 2.38E+21                                & 1                        \\
                    1152                    & 03528$+$0133$+$0129          & 35.2833                 & 1.3333                  & 5,  6                                    & \nodata                           & 0.24                          & 12.93                             & 12.06                             & $-$0.64                         & 5.1                       & 4.0                       & 10.6                         & 3.21E+21                                & 1                        \\
                    1153                    & 03529$-$0510$+$0093          & 35.2917                 & $-$5.1000                 & 6                                        & \nodata                           & 0.24                          & 9.27                              & 8.93                              & $-$0.40                         & 5.2                       & 1.1                       & 9.8                          & 4.24E+20                                & 1                        \\
                    1154                    & 03530$-$0003$+$0127          & 35.3000                 & $-$0.0333                 & 5,  6,  7                                & \nodata                           & 1.99                          & 12.66                             & 11.76                             & $-$0.33                         & 7.5                       & 2.7                       & 14.8                         & 3.94E+21                                & 1                        \\
                    1155                    & 03531$+$0246$+$0151          & 35.3083                 & 2.4583                  & 5,  6                                    & \nodata                           & 0.24                          & 15.05                             & 14.25                             & $-$0.43                         & 4.6                       & 2.8                       & 7.9                          & 3.05E+21                                & 1                        \\
                    1156                    & 03532$-$0064$+$0134          & 35.3250                 & $-$0.6417                 & 5,  6,  7                                & \nodata                           & 0.41                          & 13.35                             & 12.76                             & $-$0.45                         & 6.3                       & 2.8                       & 9.9                          & 1.97E+21                                & 1                        \\
                    1157                    & 03532$-$0131$+$0427          & 35.3250                 & $-$1.3083                 & 5,  6                                    & \nodata                           & 2.45                          & 42.67                             & 41.81                             & $-$0.76                         & 4.3                       & 1.8                       & 8.6                          & 9.85E+20                                & 1                        \\
                    1158                    & 03535$-$0133$+$0424          & 35.3500                 & $-$1.3333                 & 5,  6                                    & \nodata                           & 2.45                          & 42.38                             & 42.06                             & $-$0.24                         & 4.6                       & 1.7                       & 9.0                          & 1.06E+21                                & 1                        \\
                    1159                    & 03536$+$0185$+$0146          & 35.3583                 & 1.8500                  & 5,  6                                    & \nodata                           & 0.24                          & 14.62                             & 14.19                             & $-$0.45                         & 4.5                       & 1.9                       & 9.3                          & 8.94E+20                                & 1                        \\
                    1160                    & 03537$-$0091$+$0648          & 35.3667                 & $-$0.9083                 & 5,  6                                    & \nodata                           & 3.78                          & 64.76                             & 64.28                             & $-$0.40                         & 4.2                       & 1.1                       & 8.5                          & 4.37E+21                                & 2                        \\
                    1161                    & 03539$+$0142$+$0128          & 35.3917                 & 1.4250                  & 5,  6                                    & \nodata                           & 0.24                          & 12.77                             & 12.19                             & $-$0.60                         & 4.3                       & 1.3                       & 7.8                          & 6.02E+20                                & 1                        \\
                    1162                    & 03540$+$0127$+$0130          & 35.4000                 & 1.2750                  & 5,  6                                    & \nodata                           & 0.24                          & 13.04                             & 12.21                             & $-$0.60                         & 3.9                       & 2.2                       & 8.3                          & 1.52E+21                                & 1                        \\
                    1163                    & 03542$+$0177$+$0150          & 35.4250                 & 1.7750                  & 5,  6                                    & \nodata                           & 0.24                          & 15.04                             & 14.51                             & $-$0.44                         & 4.6                       & 2.4                       & 9.7                          & 1.42E+21                                & 1                        \\
                    1164                    & 03542$-$0119$+$0411          & 35.4250                 & $-$1.1917                 & 5,  6                                    & \nodata                           & 2.44                          & 41.08                             & 40.27                             & $-$0.41                         & 6.4                       & 2.2                       & 13.6                         & 2.16E+21                                & 1                        \\
                    1165                    & 03543$+$0052$+$0573          & 35.4333                 & 0.5250                  & 5,  6                                    & \nodata                           & 3.31                          & 57.27                             & 56.83                             & $-$0.28                         & 6.4                       & 1.7                       & 11.8                         & 1.24E+21                                & 1                        \\
                    1166                    & 03543$-$0094$+$0122          & 35.4333                 & $-$0.9417                 & 5,  6                                    & \nodata                           & 0.24                          & 12.20                             & 11.85                             & $-$0.40                         & 6.3                       & 1.8                       & 12.6                         & 7.77E+20                                & 1                        \\
                    1167                    & 03544$-$0199$+$0404          & 35.4417                 & $-$1.9917                 & 6                                        & \nodata                           & 0.97                          & 40.36                             & 39.59                             & $-$0.45                         & 6.4                       & 4.4                       & 11.1                         & 4.63E+21                                & 1                        \\
                    1168                    & 03544$-$0202$+$0407          & 35.4417                 & $-$2.0167                 & 6                                        & \nodata                           & 0.96                          & 40.71                             & 39.75                             & $-$0.65                         & 5.1                       & 3.3                       & 9.8                          & 2.80E+21                                & 1                        \\
                    1169                    & 03545$-$0081$+$0337          & 35.4500                 & $-$0.8083                 & 5,  6                                    & \nodata                           & 2.17                          & 33.71                             & 33.23                             & $-$0.43                         & 8.7                       & 4.0                       & 16.7                         & 2.60E+21                                & 1                        \\
                    1170                    & 03546$+$0124$+$0131          & 35.4583                 & 1.2417                  & 1,  5,  6                                & \nodata                           & 0.24                          & 13.09                             & 12.27                             & $-$0.57                         & 5.0                       & 2.1                       & 8.9                          & 1.55E+21                                & 1                        \\
                    1171                    & 03547$-$0144$+$0316          & 35.4667                 & $-$1.4417                 & 5,  6                                    & \nodata                           & 2.17                          & 31.56                             & 31.07                             & $-$0.39                         & 4.1                       & 1.4                       & 8.6                          & 5.96E+21                                & 2                        \\
                    1172                    & 03547$-$0210$+$0403          & 35.4667                 & $-$2.1000                 & 6                                        & \nodata                           & 0.97                          & 40.30                             & 39.54                             & $-$0.52                         & 5.1                       & 2.3                       & 9.5                          & 1.69E+21                                & 1                        \\
                    1173                    & 03548$+$0014$+$0778          & 35.4833                 & 0.1417                  & 3,  5,  6,  7                            & \nodata                           & 5.06                          & 77.84                             & 76.73                             & $-$0.32                         & 12.4                      & 5.5                       & 21.6                         & 1.32E+22                                & 1                        \\
                    1174                    & 03549$+$0123$+$0131          & 35.4917                 & 1.2333                  & 5,  6                                    & \nodata                           & 0.24                          & 13.06                             & 12.26                             & $-$0.66                         & 5.6                       & 2.9                       & 10.1                         & 1.86E+21                                & 1                        \\
                    1175                    & 03550$+$0126$+$0131          & 35.5000                 & 1.2583                  & 5,  6                                    & \nodata                           & 0.24                          & 13.10                             & 12.17                             & $-$0.83                         & 4.9                       & 2.9                       & 8.7                          & 1.84E+21                                & 1                        \\
                    1176                    & 03550$-$0188$+$0404          & 35.5000                 & $-$1.8833                 & 3,  6                                    & \nodata                           & 0.97                          & 40.39                             & 39.88                             & $-$0.46                         & 3.6                       & 2.1                       & 8.3                          & 1.17E+21                                & 1                        \\
                    1177                    & 03552$+$0126$+$0131          & 35.5250                 & 1.2583                  & 5,  6                                    & \nodata                           & 0.24                          & 13.11                             & 12.42                             & $-$0.74                         & 6.2                       & 2.9                       & 10.2                         & 1.39E+21                                & 1                        \\
                    1178                    & 03554$+$0126$+$0808          & 35.5417                 & 1.2583                  & 5,  6                                    & \nodata                           & 5.08                          & 80.78                             & 80.29                             & $-$0.31                         & 6.6                       & 2.8                       & 11.6                         & 2.25E+21                                & 1                        \\
                    1179                    & 03554$+$0129$+$0130          & 35.5417                 & 1.2917                  & 5,  6                                    & \nodata                           & 0.24                          & 12.95                             & 12.38                             & $-$0.54                         & 5.6                       & 2.5                       & 11.7                         & 1.32E+21                                & 1                        \\
                    1180                    & 03556$+$0087$+$0111          & 35.5583                 & 0.8667                  & 5,  6                                    & \nodata                           & 0.24                          & 11.08                             & 10.57                             & $-$0.38                         & 6.4                       & 2.8                       & 11.2                         & 1.87E+21                                & 1                        \\
                    1181                    & 03556$+$0125$+$0130          & 35.5583                 & 1.2500                  & 5,  6                                    & \nodata                           & 0.24                          & 13.04                             & 12.31                             & $-$0.86                         & 5.8                       & 2.7                       & 9.2                          & 1.19E+21                                & 1                        \\
                    1182                    & 03556$+$0132$+$0132          & 35.5583                 & 1.3250                  & 5,  6                                    & \nodata                           & 0.24                          & 13.19                             & 12.66                             & $-$0.34                         & 5.5                       & 2.3                       & 9.1                          & 1.79E+21                                & 1                        \\
                    1183                    & 03557$+$0092$+$0110          & 35.5667                 & 0.9250                  & 5,  6                                    & \nodata                           & 0.24                          & 11.01                             & 10.45                             & $-$0.44                         & 5.6                       & 2.6                       & 9.6                          & 1.73E+21                                & 1                        \\
                    1184                    & 03557$-$0092$+$0337          & 35.5667                 & $-$0.9167                 & 5,  6,  7                                & \nodata                           & 2.18                          & 33.69                             & 32.84                             & $-$0.66                         & 4.3                       & 1.6                       & 8.5                          & 7.44E+21                                & 2                        \\
                    1185                    & 03559$-$0005$+$0124          & 35.5917                 & $-$0.0500                 & 3,  5,  6                                & \nodata                           & 1.94                          & 12.45                             & 11.93                             & $-$0.58                         & 6.5                       & 1.3                       & 12.7                         & 5.48E+20                                & 1                        \\
                    1186                    & 03565$+$0033$+$0120          & 35.6500                 & 0.3333                  & 5,  6                                    & \nodata                           & 0.24                          & 12.01                             & 11.73                             & $-$0.15                         & 6.4                       & 1.4                       & 13.6                         & 1.24E+21                                & 1                        \\
                    1187                    & 03572$+$0121$+$0110          & 35.7167                 & 1.2083                  & 5,  6                                    & \nodata                           & 0.24                          & 10.99                             & 10.37                             & $-$0.63                         & 4.7                       & 1.5                       & 10.7                         & 6.49E+20                                & 1                        \\
                    1188                    & 03585$+$0023$+$0292          & 35.8500                 & 0.2333                  & 5,  6                                    & \nodata                           & 1.57                          & 29.20                             & 29.05                             & $-$0.15                         & 7.7                       & 1.1                       & 12.1                         & 3.91E+21                                & 2                        \\
                    1189                    & 03587$+$0032$+$0291          & 35.8667                 & 0.3167                  & 5,  6                                    & \nodata                           & 1.60                          & 29.09                             & 28.56                             & $-$0.43                         & 7.8                       & 3.4                       & 16.2                         & 2.41E+21                                & 1                        \\
                    1190                    & 03587$-$0147$+$0313          & 35.8667                 & $-$1.4667                 & 6                                        & \nodata                           & 2.17                          & 31.26                             & 30.75                             & $-$0.35                         & 7.6                       & 3.0                       & 14.2                         & 2.35E+21                                & 1                        \\
                    1191                    & 03588$-$0006$+$0286          & 35.8833                 & $-$0.0583                 & 5,  6,  7                                & \nodata                           & 2.11                          & 28.57                             & 28.36                             & $-$0.18                         & 11.8                      & 3.3                       & 16.2                         & 2.29E+21                                & 1                        \\
                    1192                    & 03589$-$0512$+$0084          & 35.8917                 & $-$5.1167                 & 5,  6                                    & \nodata                           & 0.24                          & 8.41                              & 8.14                              & $-$0.32                         & 5.5                       & 1.3                       & 12.2                         & 5.05E+20                                & 1                        \\
                    1193                    & 03592$+$0160$+$0123          & 35.9250                 & 1.6000                  & 5,  6                                    & \nodata                           & 0.24                          & 12.32                             & 11.30                             & $-$1.17                         & 4.1                       & 0.9                       & 7.7                          & 3.60E+20                                & 1                        \\
                    1194                    & 03592$-$0143$+$0324          & 35.9250                 & $-$1.4333                 & 6                                        & \nodata                           & 2.18                          & 32.38                             & 31.54                             & $-$0.57                         & 6.8                       & 3.0                       & 10.9                         & 2.33E+21                                & 1                        \\
                    1195                    & 03595$+$0233$+$0111          & 35.9500                 & 2.3333                  & 5,  6                                    & \nodata                           & 0.24                          & 11.07                             & 10.57                             & $-$0.49                         & 3.9                       & 1.3                       & 8.2                          & 5.84E+20                                & 1                        \\
                    1196                    & 03600$+$0078$+$0221          & 36.0000                 & 0.7833                  & 5,  6                                    & \nodata                           & 0.26                          & 22.06                             & 21.59                             & $-$0.68                         & 4.7                       & 1.4                       & 12.5                         & 4.70E+20                                & 1                        \\
                    1197                    & 03602$-$0136$+$0316          & 36.0250                 & $-$1.3583                 & 5,  6                                    & 17                                & 2.18                          & 31.59                             & 30.99                             & $-$0.35                         & 6.8                       & 3.9                       & 12.0                         & 3.80E+21                                & 1                        \\
                    1198                    & 03604$-$0158$+$0327          & 36.0417                 & $-$1.5833                 & 6                                        & \nodata                           & 2.20                          & 32.66                             & 31.76                             & $-$0.46                         & 6.8                       & 3.6                       & 11.7                         & 3.88E+21                                & 1                        \\
                    1199                    & 03607$-$0127$+$0314          & 36.0667                 & $-$1.2750                 & 5,  6                                    & \nodata                           & 2.18                          & 31.38                             & 30.91                             & $-$0.55                         & 3.9                       & 1.4                       & 8.5                          & 5.42E+20                                & 1                        \\
                    1200                    & 03607$-$0135$+$0317          & 36.0667                 & $-$1.3500                 & 5,  6                                    & \nodata                           & 2.18                          & 31.66                             & 31.04                             & $-$0.39                         & 6.1                       & 4.2                       & 11.9                         & 3.86E+21                                & 1                        \\
                    1201                    & 03609$+$0048$+$0174          & 36.0917                 & 0.4833                  & 5,  6                                    & \nodata                           & 1.90                          & 17.35                             & 16.87                             & $-$0.53                         & 4.3                       & 1.2                       & 14.4                         & 5.44E+20                                & 1                        \\
                    1202                    & 03613$+$0062$+$0222          & 36.1333                 & 0.6167                  & 5,  6,  7                                & \nodata                           & 1.91                          & 22.21                             & 21.86                             & $-$0.29                         & 6.2                       & 2.9                       & 12.3                         & 1.80E+21                                & 1                        \\
                    1203                    & 03618$-$0170$+$0677          & 36.1833                 & $-$1.7000                 & 6                                        & \nodata                           & 4.35                          & 67.74                             & 67.11                             & $-$0.50                         & 10.3                      & 2.5                       & 16.8                         & 1.77E+21                                & 1                        \\
                    1204                    & 03621$+$0021$+$0177          & 36.2083                 & 0.2083                  & 5,  6                                    & \nodata                           & 1.74                          & 17.69                             & 17.32                             & $-$0.42                         & 4.3                       & 1.9                       & 8.7                          & 7.95E+20                                & 1                        \\
                    1205                    & 03621$-$0149$+$0313          & 36.2083                 & $-$1.4917                 & 6                                        & \nodata                           & 2.16                          & 31.31                             & 30.59                             & $-$0.45                         & 5.9                       & 3.9                       & 13.7                         & 3.54E+21                                & 1                        \\
                    1206                    & 03626$+$0027$+$0177          & 36.2583                 & 0.2750                  & 5,  6                                    & \nodata                           & 1.76                          & 17.74                             & 17.24                             & $-$0.42                         & 4.3                       & 1.5                       & 9.7                          & 7.90E+20                                & 1                        \\
                    1207                    & 03628$-$0063$+$0107          & 36.2833                 & $-$0.6333                 & 5,  6                                    & \nodata                           & 0.25                          & 10.66                             & 10.25                             & $-$0.39                         & 4.5                       & 1.0                       & 8.6                          & 4.61E+20                                & 1                        \\
                    1208                    & 03632$-$0272$+$0349          & 36.3250                 & $-$2.7250                 & 6                                        & \nodata                           & 1.15                          & 34.87                             & 34.43                             & $-$0.41                         & 4.0                       & 1.2                       & 9.0                          & 5.78E+20                                & 1                        \\
                    1209                    & 03633$-$0127$+$0305          & 36.3333                 & $-$1.2750                 & 5,  6                                    & \nodata                           & 2.16                          & 30.49                             & 29.38                             & $-$0.74                         & 4.6                       & 3.6                       & 10.7                         & 3.00E+21                                & 1                        \\
                    1210                    & 03634$-$0093$+$0307          & 36.3417                 & $-$0.9333                 & 5,  6                                    & \nodata                           & 2.16                          & 30.73                             & 30.15                             & $-$0.51                         & 4.4                       & 2.6                       & 9.7                          & 1.47E+21                                & 1                        \\
                    1211                    & 03637$+$0012$+$0177          & 36.3750                 & 0.1250                  & 5,  6                                    & \nodata                           & 1.72                          & 17.71                             & 17.28                             & $-$0.48                         & 4.8                       & 3.3                       & 9.4                          & 1.71E+21                                & 1                        \\
                    1212                    & 03640$-$0007$+$0298          & 36.4000                 & $-$0.0750                 & 5,  6,  7                                & \nodata                           & 2.12                          & 29.85                             & 29.61                             & $-$0.26                         & 4.4                       & 1.3                       & 9.5                          & 4.11E+21                                & 2                        \\
                    1213                    & 03642$+$0116$+$0116          & 36.4250                 & 1.1583                  & 5,  6                                    & \nodata                           & 0.24                          & 11.56                             & 10.72                             & $-$0.52                         & 4.3                       & 1.2                       & 13.6                         & 9.53E+20                                & 1                        \\
                    1214                    & 03646$+$0024$+$0214          & 36.4583                 & 0.2417                  & 5,  6                                    & \nodata                           & 1.66                          & 21.45                             & 21.16                             & $-$0.35                         & 4.7                       & 2.0                       & 11.1                         & 7.98E+20                                & 1                        \\
                    1215                    & 03646$-$0093$+$0286          & 36.4583                 & $-$0.9333                 & 5,  6                                    & \nodata                           & 2.15                          & 28.58                             & 28.30                             & $-$0.19                         & 6.5                       & 1.3                       & 15.5                         & 9.47E+20                                & 1                        \\
                    1216                    & 03648$-$0008$+$0295          & 36.4833                 & $-$0.0833                 & 5,  6                                    & \nodata                           & 2.12                          & 29.53                             & 29.37                             & $-$0.14                         & 5.8                       & 1.4                       & 14.6                         & 8.25E+20                                & 1                        \\
                    1217                    & 03657$+$0123$+$0111          & 36.5667                 & 1.2333                  & 5,  6                                    & \nodata                           & 0.24                          & 11.06                             & 10.42                             & $-$0.63                         & 4.2                       & 1.3                       & 9.4                          & 5.91E+20                                & 1                        \\
                    1218                    & 03659$+$0123$+$0111          & 36.5917                 & 1.2333                  & 5,  6                                    & \nodata                           & 0.24                          & 11.07                             & 10.48                             & $-$0.59                         & 4.6                       & 1.3                       & 10.4                         & 5.83E+20                                & 1                        \\
                    1219                    & 03660$+$0102$+$0110          & 36.6000                 & 1.0250                  & 5,  6                                    & \nodata                           & 0.24                          & 10.95                             & 10.37                             & $-$0.78                         & 3.5                       & 1.0                       & 10.7                         & 3.24E+20                                & 1                        \\
                    1220                    & 03662$+$0107$+$0110          & 36.6250                 & 1.0750                  & 5,  6                                    & \nodata                           & 0.24                          & 11.04                             & 10.35                             & $-$0.79                         & 3.2                       & 1.2                       & 8.9                          & 4.57E+20                                & 1                        \\
                    1221                    & 03662$+$0124$+$0110          & 36.6250                 & 1.2417                  & 5,  6                                    & \nodata                           & 0.24                          & 11.01                             & 10.53                             & $-$0.51                         & 3.6                       & 1.5                       & 8.5                          & 6.43E+20                                & 1                        \\
                    1222                    & 03665$+$0123$+$0108          & 36.6500                 & 1.2333                  & 5,  6                                    & \nodata                           & 0.24                          & 10.85                             & 10.41                             & $-$0.51                         & 4.1                       & 1.7                       & 9.0                          & 6.98E+20                                & 1                        \\
                    1223                    & 03671$-$0032$+$0524          & 36.7083                 & $-$0.3250                 & 5,  6                                    & \nodata                           & 2.59                          & 52.38                             & 51.79                             & $-$0.32                         & 6.9                       & 2.7                       & 11.7                         & 2.52E+21                                & 1                        \\
                    1224                    & 03672$+$0057$+$0131          & 36.7250                 & 0.5750                  & 5,  6                                    & \nodata                           & 1.98                          & 13.13                             & 12.30                             & $-$0.70                         & 4.0                       & 1.4                       & 8.4                          & 7.58E+20                                & 1                        \\
                    1225                    & 03674$+$0062$+$0131          & 36.7417                 & 0.6250                  & 1,  5,  6                                & \nodata                           & 2.04                          & 13.14                             & 12.53                             & $-$0.50                         & 3.5                       & 1.5                       & 8.7                          & 8.77E+20                                & 1                        \\
                    1226                    & 03675$+$0132$+$0111          & 36.7500                 & 1.3250                  & 5,  6                                    & \nodata                           & 0.24                          & 11.08                             & 10.16                             & $-$0.68                         & 4.0                       & 1.3                       & 10.6                         & 8.06E+20                                & 1                        \\
                    1227                    & 03676$+$0058$+$0129          & 36.7583                 & 0.5833                  & 5,  6                                    & \nodata                           & 1.99                          & 12.94                             & 12.20                             & $-$0.78                         & 4.5                       & 1.5                       & 8.3                          & 6.65E+20                                & 1                        \\
                    1228                    & 03680$+$0036$+$0843          & 36.8000                 & 0.3583                  & 5,  6                                    & \nodata                           & 4.64                          & 84.32                             & 84.09                             & $-$0.18                         & 6.2                       & 3.1                       & 15.2                         & 2.21E+21                                & 1                        \\
                    1229                    & 03691$+$0069$+$0126          & 36.9083                 & 0.6917                  & 1,  5,  6                                & \nodata                           & 2.17                          & 12.62                             & 11.61                             & $-$0.59                         & 5.0                       & 2.8                       & 11.2                         & 2.42E+21                                & 1                        \\
                    1230                    & 03691$-$0043$+$0800          & 36.9083                 & $-$0.4333                 & 5,  6,  7                                & \nodata                           & 4.41                          & 80.01                             & 78.13                             & $-$0.44                         & 7.8                       & 3.9                       & 16.2                         & 9.66E+21                                & 1                        \\
                    1231                    & 03692$+$0092$+$0129          & 36.9167                 & 0.9250                  & 5,  6,  7                                & \nodata                           & 0.24                          & 12.92                             & 12.48                             & $-$0.49                         & 3.1                       & 1.4                       & 7.6                          & 4.70E+21                                & 2                        \\
                    1232                    & 03692$+$0287$+$0147          & 36.9250                 & 2.8750                  & 5,  6                                    & \nodata                           & 0.24                          & 14.73                             & 14.05                             & $-$0.66                         & 4.1                       & 1.3                       & 10.0                         & 5.80E+20                                & 1                        \\
                    1233                    & 03692$+$0304$+$0146          & 36.9250                 & 3.0417                  & 5,  6                                    & \nodata                           & 0.24                          & 14.61                             & 14.17                             & $-$0.36                         & 3.9                       & 1.0                       & 10.2                         & 5.31E+20                                & 1                        \\
                    1234                    & 03693$+$0068$+$0123          & 36.9333                 & 0.6833                  & 1,  5,  6                                & \nodata                           & 0.87                          & 12.31                             & 11.71                             & $-$0.41                         & 4.5                       & 2.5                       & 10.5                         & 1.78E+21                                & 1                        \\
                    1235                    & 03693$+$0090$+$0129          & 36.9333                 & 0.9000                  & 5,  6,  7                                & \nodata                           & 0.24                          & 12.86                             & 12.50                             & $-$0.43                         & 3.7                       & 1.5                       & 9.1                          & 4.52E+21                                & 2                        \\
                    1236                    & 03694$+$0109$+$0132          & 36.9417                 & 1.0917                  & 5,  6                                    & \nodata                           & 0.24                          & 13.17                             & 12.14                             & $-$0.83                         & 3.9                       & 2.4                       & 9.8                          & 1.48E+21                                & 1                        \\
                    1237                    & 03694$+$0226$+$0145          & 36.9417                 & 2.2583                  & 5,  6                                    & \nodata                           & 0.24                          & 14.50                             & 13.88                             & $-$0.72                         & 4.5                       & 1.6                       & 10.4                         & 6.48E+20                                & 1                        \\
                    1238                    & 03696$+$0228$+$0149          & 36.9583                 & 2.2833                  & 5,  6                                    & \nodata                           & 0.24                          & 14.91                             & 13.89                             & $-$0.67                         & 4.6                       & 1.5                       & 12.8                         & 1.11E+21                                & 1                        \\
                    1239                    & 03696$+$0290$+$0144          & 36.9583                 & 2.9000                  & 5,  6                                    & \nodata                           & 0.24                          & 14.43                             & 14.07                             & $-$0.43                         & 3.3                       & 1.4                       & 7.6                          & 5.33E+20                                & 1                        \\
                    1240                    & 03696$-$0132$+$0190          & 36.9583                 & $-$1.3167                 & 6                                        & \nodata                           & 0.51                          & 18.96                             & 18.54                             & $-$0.51                         & 4.4                       & 1.9                       & 10.5                         & 7.62E+20                                & 1                        \\
                    1241                    & 03698$+$0233$+$0146          & 36.9833                 & 2.3333                  & 5,  6                                    & \nodata                           & 0.24                          & 14.65                             & 14.12                             & $-$0.54                         & 4.5                       & 1.4                       & 9.6                          & 6.37E+20                                & 1                        \\
                    1242                    & 03702$+$0231$+$0148          & 37.0167                 & 2.3083                  & 5,  6                                    & \nodata                           & 0.31                          & 14.75                             & 14.30                             & $-$0.48                         & 3.8                       & 1.2                       & 8.5                          & 5.01E+20                                & 1                        \\
                    1243                    & 03704$+$0072$+$0126          & 37.0417                 & 0.7250                  & 5,  6                                    & \nodata                           & 0.29                          & 12.55                             & 11.84                             & $-$0.70                         & 4.0                       & 1.5                       & 7.7                          & 7.22E+20                                & 1                        \\
                    1244                    & 03704$-$0035$+$0070          & 37.0417                 & $-$0.3500                 & 5,  6                                    & \nodata                           & 0.27                          & 7.04                              & 6.58                              & $-$0.75                         & 3.4                       & 1.1                       & 7.4                          & 3.26E+20                                & 1                        \\
                    1245                    & 03708$+$0210$+$0144          & 37.0833                 & 2.1000                  & 5,  6                                    & \nodata                           & 0.30                          & 14.44                             & 13.79                             & $-$0.70                         & 4.4                       & 1.5                       & 10.1                         & 6.50E+20                                & 1                        \\
                    1246                    & 03713$+$0010$+$0216          & 37.1333                 & 0.1000                  & 5,  6,  7                                & \nodata                           & 1.55                          & 21.57                             & 21.39                             & $-$0.23                         & 5.9                       & 1.4                       & 11.7                         & 4.99E+20                                & 1                        \\
                    1247                    & 03713$+$0017$+$0181          & 37.1333                 & 0.1750                  & 5,  6                                    & \nodata                           & 1.63                          & 18.07                             & 17.59                             & $-$0.57                         & 3.4                       & 2.8                       & 7.7                          & 1.46E+21                                & 1                        \\
                    1248                    & 03717$+$0013$+$0216          & 37.1750                 & 0.1333                  & 1,  5,  6                                & \nodata                           & 1.55                          & 21.64                             & 21.17                             & $-$0.61                         & 6.2                       & 2.6                       & 11.4                         & 9.93E+20                                & 1                        \\
                    1249                    & 03722$+$0020$+$0497          & 37.2167                 & 0.2000                  & 5,  6                                    & \nodata                           & 2.53                          & 49.68                             & 49.09                             & $-$0.47                         & 6.9                       & 2.2                       & 11.8                         & 1.38E+21                                & 1                        \\
                    1250                    & 03723$+$0271$+$0145          & 37.2333                 & 2.7083                  & 5,  6                                    & \nodata                           & 0.31                          & 14.54                             & 14.06                             & $-$0.55                         & 4.0                       & 1.1                       & 12.9                         & 4.60E+20                                & 1                        \\
                    1251                    & 03727$+$0268$+$0148          & 37.2667                 & 2.6833                  & 5,  6                                    & \nodata                           & 0.31                          & 14.75                             & 14.22                             & $-$0.65                         & 3.6                       & 1.4                       & 9.0                          & 5.13E+20                                & 1                        \\
                    1252                    & 03727$+$0008$+$0916          & 37.2750                 & 0.0833                  & 1,  5,  6                                & \nodata                           & 4.99                          & 91.61                             & 90.10                             & $-$0.44                         & 9.7                       & 6.8                       & 17.6                         & 1.63E+22                                & 1                        \\
                    1253                    & 03730$+$0025$+$0210          & 37.3000                 & 0.2500                  & 5,  6                                    & \nodata                           & 1.57                          & 20.99                             & 19.36                             & $-$1.42                         & 4.7                       & 1.5                       & 11.7                         & 7.92E+20                                & 1                        \\
                    1254                    & 03732$+$0158$+$0322          & 37.3167                 & 1.5833                  & 1,  5,  6                                & \nodata                           & 1.90                          & 32.23                             & 31.48                             & $-$0.50                         & 6.0                       & 1.3                       & 11.0                         & 8.91E+20                                & 1                        \\
                    1255                    & 03738$+$0059$+$0353          & 37.3833                 & 0.5917                  & 5,  6                                    & \nodata                           & 1.92                          & 35.33                             & 34.84                             & $-$0.55                         & 5.9                       & 1.7                       & 11.1                         & 7.02E+20                                & 1                        \\
                    1256                    & 03744$-$0074$+$0127          & 37.4417                 & $-$0.7417                 & 3,  5,  6                                & \nodata                           & 0.34                          & 12.66                             & 12.05                             & $-$0.58                         & 4.0                       & 1.8                       & 8.8                          & 9.08E+20                                & 1                        \\
                    1257                    & 03746$+$0306$+$0156          & 37.4583                 & 3.0583                  & 5,  6                                    & \nodata                           & 0.31                          & 15.63                             & 14.79                             & $-$0.81                         & 5.5                       & 2.4                       & 12.2                         & 1.28E+21                                & 1                        \\
                    1258                    & 03746$+$0309$+$0156          & 37.4583                 & 3.0917                  & 5,  6                                    & \nodata                           & 0.31                          & 15.63                             & 14.81                             & $-$0.75                         & 5.8                       & 2.2                       & 11.2                         & 1.18E+21                                & 1                        \\
                    1259                    & 03747$+$0318$+$0156          & 37.4667                 & 3.1833                  & 5,  6                                    & \nodata                           & 0.31                          & 15.60                             & 15.04                             & $-$0.44                         & 6.8                       & 2.1                       & 16.0                         & 1.41E+21                                & 1                        \\
                    1260                    & 03747$+$0302$+$0153          & 37.4750                 & 3.0250                  & 5,  6                                    & \nodata                           & 0.31                          & 15.31                             & 14.87                             & $-$0.38                         & 5.4                       & 3.6                       & 9.3                          & 2.58E+21                                & 1                        \\
                    1261                    & 03750$+$0300$+$0154          & 37.5000                 & 3.0000                  & 5,  6                                    & \nodata                           & 0.31                          & 15.38                             & 14.86                             & $-$0.32                         & 6.1                       & 2.0                       & 11.3                         & 1.61E+21                                & 1                        \\
                    1262                    & 03750$+$0305$+$0155          & 37.5000                 & 3.0500                  & 5,  6                                    & \nodata                           & 0.31                          & 15.52                             & 14.91                             & $-$0.66                         & 6.5                       & 3.4                       & 12.5                         & 1.68E+21                                & 1                        \\
                    1263                    & 03750$+$0318$+$0157          & 37.5000                 & 3.1833                  & 5,  6                                    & \nodata                           & 0.31                          & 15.66                             & 14.92                             & $-$0.64                         & 5.9                       & 2.1                       & 14.0                         & 1.22E+21                                & 1                        \\
                    1264                    & 03756$+$0180$+$0328          & 37.5583                 & 1.8000                  & 5,  6                                    & \nodata                           & 1.89                          & 32.76                             & 32.35                             & $-$0.36                         & 3.2                       & 0.9                       & 6.7                          & 3.61E+21                                & 2                        \\
                    1265                    & 03757$+$0147$+$0298          & 37.5667                 & 1.4750                  & 5,  6                                    & \nodata                           & 1.90                          & 29.82                             & 29.50                             & $-$0.14                         & 3.3                       & 3.3                       & 6.8                          & 7.82E+21                                & 1                        \\
                    1266                    & 03757$+$0184$+$0325          & 37.5750                 & 1.8417                  & 5,  6                                    & \nodata                           & 1.89                          & 32.52                             & 32.13                             & $-$0.35                         & 3.2                       & 1.6                       & 7.3                          & 6.85E+21                                & 2                        \\
                    1267                    & 03758$-$0057$+$0135          & 37.5833                 & $-$0.5667                 & 5,  6,  7                                & \nodata                           & 0.32                          & 13.48                             & 12.82                             & $-$0.58                         & 4.9                       & 1.6                       & 9.9                          & 8.43E+20                                & 1                        \\
                    1268                    & 03762$+$0134$+$0098          & 37.6167                 & 1.3417                  & 5,  6                                    & \nodata                           & 0.30                          & 9.81                              & 9.20                              & $-$0.57                         & 3.7                       & 1.6                       & 8.0                          & 8.25E+20                                & 1                        \\
                    1269                    & 03762$-$0056$+$0135          & 37.6167                 & $-$0.5583                 & 5,  6,  7                                & \nodata                           & 0.31                          & 13.50                             & 12.92                             & $-$0.51                         & 5.5                       & 1.4                       & 10.5                         & 6.99E+20                                & 1                        \\
                    1270                    & 03763$-$0052$+$0133          & 37.6333                 & $-$0.5250                 & 5,  6,  7                                & \nodata                           & 0.30                          & 13.33                             & 12.80                             & $-$0.46                         & 3.8                       & 1.7                       & 9.0                          & 9.42E+20                                & 1                        \\
                    1271                    & 03764$+$0157$+$0302          & 37.6417                 & 1.5750                  & 5,  6                                    & \nodata                           & 1.89                          & 30.23                             & 29.19                             & $-$0.55                         & 5.5                       & 2.4                       & 9.9                          & 2.29E+21                                & 1                        \\
                    1272                    & 03767$-$0072$+$0136          & 37.6750                 & $-$0.7167                 & 6                                        & \nodata                           & 0.33                          & 13.64                             & 13.09                             & $-$0.59                         & 3.8                       & 0.9                       & 7.6                          & 3.69E+20                                & 1                        \\
                    1273                    & 03768$+$0184$+$0327          & 37.6833                 & 1.8417                  & 5,  6                                    & \nodata                           & 1.89                          & 32.74                             & 31.75                             & $-$0.41                         & 5.6                       & 1.9                       & 15.1                         & 2.39E+21                                & 1                        \\
                    1274                    & 03768$-$0053$+$0136          & 37.6833                 & $-$0.5333                 & 5,  6,  7                                & \nodata                           & 0.31                          & 13.57                             & 12.77                             & $-$0.53                         & 4.4                       & 1.1                       & 8.2                          & 7.58E+20                                & 1                        \\
                    1275                    & 03769$+$0117$+$0316          & 37.6917                 & 1.1750                  & 5,  6                                    & \nodata                           & 1.90                          & 31.64                             & 30.11                             & $-$0.63                         & 4.9                       & 2.0                       & 11.3                         & 2.27E+21                                & 1                        \\
                    1276                    & 03769$+$0120$+$0313          & 37.6917                 & 1.2000                  & 5,  6                                    & \nodata                           & 1.90                          & 31.33                             & 30.16                             & $-$0.86                         & 5.0                       & 2.3                       & 9.9                          & 1.57E+21                                & 1                        \\
                    1277                    & 03770$-$0050$+$0133          & 37.7000                 & $-$0.5000                 & 5,  6,  7                                & \nodata                           & 0.31                          & 13.34                             & 12.65                             & $-$0.56                         & 5.1                       & 2.3                       & 11.8                         & 1.43E+21                                & 1                        \\
                    1278                    & 03770$-$0072$+$0138          & 37.7000                 & $-$0.7167                 & 6                                        & \nodata                           & 0.33                          & 13.79                             & 13.24                             & $-$0.68                         & 3.3                       & 1.2                       & 6.8                          & 4.50E+20                                & 1                        \\
                    1279                    & 03770$-$0356$+$0481          & 37.7000                 & $-$3.5583                 & 6                                        & \nodata                           & 0.16                          & 48.13                             & 47.79                             & $-$0.35                         & 7.3                       & 3.8                       & 13.8                         & 2.00E+21                                & 1                        \\
                    1280                    & 03771$-$0047$+$0130          & 37.7083                 & $-$0.4750                 & 5,  6,  7                                & \nodata                           & 0.30                          & 13.04                             & 12.48                             & $-$0.54                         & 5.8                       & 3.0                       & 11.8                         & 1.59E+21                                & 1                        \\
                    1281                    & 03773$+$0346$+$0161          & 37.7333                 & 3.4583                  & 5,  6                                    & \nodata                           & 0.33                          & 16.13                             & 15.01                             & $-$1.17                         & 4.0                       & 2.4                       & 12.6                         & 1.13E+21                                & 1                        \\
                    1282                    & 03773$-$0069$+$0135          & 37.7333                 & $-$0.6917                 & 6                                        & \nodata                           & 0.32                          & 13.53                             & 12.99                             & $-$0.64                         & 2.9                       & 0.9                       & 7.1                          & 3.41E+20                                & 1                        \\
                    1283                    & 03773$-$0082$+$0137          & 37.7333                 & $-$0.8250                 & 6                                        & \nodata                           & 0.34                          & 13.69                             & 13.14                             & $-$0.63                         & 4.9                       & 1.5                       & 9.3                          & 5.76E+20                                & 1                        \\
                    1284                    & 03775$-$0015$+$0152          & 37.7500                 & $-$0.1500                 & 5,  6                                    & \nodata                           & 1.76                          & 15.20                             & 14.49                             & $-$0.52                         & 4.7                       & 1.5                       & 9.0                          & 9.18E+20                                & 1                        \\
                    1285                    & 03776$+$0303$+$0156          & 37.7583                 & 3.0333                  & 5,  6                                    & \nodata                           & 0.32                          & 15.62                             & 15.18                             & $-$0.50                         & 4.9                       & 2.1                       & 9.9                          & 9.07E+20                                & 1                        \\
                    1286                    & 03777$-$0032$+$0130          & 37.7667                 & $-$0.3167                 & 3,  5,  6                                & \nodata                           & 0.30                          & 12.96                             & 12.30                             & $-$0.39                         & 5.1                       & 1.8                       & 15.6                         & 1.53E+21                                & 1                        \\
                    1287                    & 03778$+$0130$+$0099          & 37.7833                 & 1.3000                  & 5,  6                                    & \nodata                           & 0.30                          & 9.85                              & 9.18                              & $-$0.54                         & 3.0                       & 1.1                       & 7.4                          & 6.53E+20                                & 1                        \\
                    1288                    & 03778$-$0027$+$0203          & 37.7833                 & $-$0.2667                 & 3,  5,  6                                & \nodata                           & 0.62                          & 20.35                             & 19.29                             & $-$0.91                         & 5.0                       & 1.3                       & 11.6                         & 6.63E+20                                & 1                        \\
                    1289                    & 03782$-$0022$+$0133          & 37.8250                 & $-$0.2167                 & 3,  5,  6,  7                            & \nodata                           & 0.30                          & 13.29                             & 12.79                             & $-$0.39                         & 4.8                       & 2.3                       & 8.5                          & 1.53E+21                                & 1                        \\
                    1290                    & 03787$+$0298$+$0154          & 37.8750                 & 2.9833                  & 5,  6                                    & \nodata                           & 0.32                          & 15.42                             & 14.94                             & $-$0.53                         & 4.4                       & 2.2                       & 10.6                         & 9.83E+20                                & 1                        \\
                    1291                    & 03790$-$0023$+$0134          & 37.9000                 & $-$0.2333                 & 3,  5,  6,  7                            & \nodata                           & 0.29                          & 13.35                             & 12.89                             & $-$0.53                         & 4.0                       & 2.5                       & 8.3                          & 1.23E+21                                & 1                        \\
                    1292                    & 03793$-$0024$+$0129          & 37.9333                 & $-$0.2417                 & 3,  5,  6                                & \nodata                           & 0.29                          & 12.90                             & 12.34                             & $-$0.65                         & 4.3                       & 2.4                       & 9.6                          & 1.04E+21                                & 1                        \\
                    1293                    & 03798$-$0033$+$0205          & 37.9833                 & $-$0.3333                 & 3,  5,  6                                & \nodata                           & 0.62                          & 20.48                             & 19.79                             & $-$0.74                         & 4.5                       & 1.8                       & 10.9                         & 7.68E+20                                & 1                        \\
                    1294                    & 03798$-$0036$+$0129          & 37.9833                 & $-$0.3583                 & 3,  5,  6                                & \nodata                           & 0.30                          & 12.95                             & 12.25                             & $-$0.96                         & 4.1                       & 1.7                       & 9.2                          & 5.77E+20                                & 1                        \\
                    1295                    & 03798$-$0091$+$0163          & 37.9833                 & $-$0.9083                 & 6                                        & \nodata                           & 0.59                          & 16.27                             & 15.86                             & $-$0.42                         & 5.7                       & 1.5                       & 14.6                         & 7.50E+20                                & 1                        \\
                    1296                    & 03800$-$0031$+$0204          & 38.0000                 & $-$0.3083                 & 5,  6                                    & \nodata                           & 0.63                          & 20.41                             & 19.74                             & $-$0.56                         & 4.4                       & 1.9                       & 12.4                         & 1.12E+21                                & 1                        \\
                    1297                    & 03801$+$0143$+$0100          & 38.0083                 & 1.4333                  & 5,  6                                    & \nodata                           & 0.30                          & 9.97                              & 9.30                              & $-$0.68                         & 2.7                       & 1.2                       & 7.3                          & 5.41E+20                                & 1                        \\
                    1298                    & 03803$+$0010$+$0188          & 38.0333                 & 0.1000                  & 5,  6,  7                                & \nodata                           & 1.52                          & 18.76                             & 18.47                             & $-$0.28                         & 3.1                       & 2.2                       & 6.4                          & 1.62E+21                                & 1                        \\
                    1299                    & 03803$+$0302$+$0149          & 38.0333                 & 3.0250                  & 5,  6                                    & \nodata                           & 0.31                          & 14.95                             & 14.59                             & $-$0.33                         & 5.1                       & 1.2                       & 10.8                         & 5.97E+20                                & 1                        \\
                    1300                    & 03803$-$0033$+$0203          & 38.0333                 & $-$0.3333                 & 6                                        & \nodata                           & 0.64                          & 20.34                             & 19.75                             & $-$0.70                         & 4.0                       & 1.5                       & 9.4                          & 5.85E+20                                & 1                        \\
                    1301                    & 03804$-$0042$+$0130          & 38.0417                 & $-$0.4167                 & 6                                        & \nodata                           & 0.29                          & 13.04                             & 12.44                             & $-$0.55                         & 5.2                       & 2.1                       & 11.8                         & 1.09E+21                                & 1                        \\
                    1302                    & 03806$+$0297$+$0153          & 38.0583                 & 2.9667                  & 5,  6                                    & \nodata                           & 0.31                          & 15.28                             & 14.98                             & $-$0.35                         & 3.9                       & 1.5                       & 8.7                          & 5.69E+20                                & 1                        \\
                    1303                    & 03806$-$0047$+$0126          & 38.0583                 & $-$0.4667                 & 6                                        & \nodata                           & 0.30                          & 12.65                             & 12.18                             & $-$0.43                         & 3.4                       & 1.4                       & 11.8                         & 6.87E+20                                & 1                        \\
                    1304                    & 03810$-$0232$+$0271          & 38.1000                 & $-$2.3250                 & 6                                        & \nodata                           & 0.72                          & 27.13                             & 26.30                             & $-$0.37                         & 5.5                       & 1.4                       & 11.1                         & 1.46E+21                                & 1                        \\
                    1305                    & 03811$-$0050$+$0125          & 38.1083                 & $-$0.5000                 & 1,  6                                    & \nodata                           & 0.30                          & 12.52                             & 12.08                             & $-$0.47                         & 4.6                       & 1.1                       & 13.0                         & 4.64E+20                                & 1                        \\
                    1306                    & 03812$+$0143$+$0099          & 38.1250                 & 1.4333                  & 5,  6                                    & \nodata                           & 0.29                          & 9.87                              & 8.67                              & $-$0.86                         & 3.3                       & 1.4                       & 11.2                         & 9.26E+20                                & 1                        \\
                    1307                    & 03813$-$0091$+$0166          & 38.1333                 & $-$0.9083                 & 6                                        & \nodata                           & 0.61                          & 16.59                             & 15.91                             & $-$0.60                         & 5.6                       & 2.7                       & 11.7                         & 1.58E+21                                & 1                        \\
                    1308                    & 03814$+$0146$+$0095          & 38.1417                 & 1.4583                  & 5,  6                                    & \nodata                           & 0.29                          & 9.47                              & 8.92                              & $-$0.52                         & 3.3                       & 1.4                       & 9.0                          & 6.97E+20                                & 1                        \\
                    1309                    & 03818$+$0287$+$0145          & 38.1833                 & 2.8750                  & 5,  6                                    & \nodata                           & 0.31                          & 14.50                             & 14.28                             & $-$0.26                         & 5.6                       & 2.7                       & 12.0                         & 1.15E+21                                & 1                        \\
                    1310                    & 03818$+$0291$+$0148          & 38.1833                 & 2.9083                  & 5,  6                                    & \nodata                           & 0.31                          & 14.77                             & 14.43                             & $-$0.32                         & 5.3                       & 1.5                       & 14.9                         & 8.27E+20                                & 1                        \\
                    1311                    & 03821$+$0244$+$0148          & 38.2083                 & 2.4417                  & 5,  6                                    & \nodata                           & 0.31                          & 14.80                             & 14.22                             & $-$0.80                         & 3.0                       & 1.2                       & 7.6                          & 3.90E+20                                & 1                        \\
                    1312                    & 03822$-$0008$+$0178          & 38.2250                 & $-$0.0833                 & 5,  6                                    & \nodata                           & 1.61                          & 17.77                             & 17.21                             & $-$0.42                         & 3.5                       & 2.2                       & 7.7                          & 1.61E+21                                & 1                        \\
                    1313                    & 03826$-$0087$+$0165          & 38.2583                 & $-$0.8750                 & 6                                        & \nodata                           & 0.61                          & 16.53                             & 15.62                             & $-$0.54                         & 6.7                       & 2.4                       & 13.6                         & 2.03E+21                                & 1                        \\
                    1314                    & 03828$+$0126$+$0091          & 38.2833                 & 1.2583                  & 5,  6                                    & \nodata                           & 0.29                          & 9.14                              & 8.67                              & $-$0.39                         & 3.5                       & 1.2                       & 9.7                          & 6.61E+20                                & 1                        \\
                    1315                    & 03831$+$0099$+$0300          & 38.3083                 & 0.9917                  & 5,  6                                    & \nodata                           & 1.91                          & 29.99                             & 29.11                             & $-$0.69                         & 4.1                       & 1.8                       & 8.7                          & 1.11E+21                                & 1                        \\
                    1316                    & 03832$-$0040$+$0194          & 38.3250                 & $-$0.4000                 & 1,  6                                    & \nodata                           & 0.62                          & 19.39                             & 18.86                             & $-$0.55                         & 4.5                       & 2.4                       & 11.0                         & 1.14E+21                                & 1                        \\
                    1317                    & 03837$+$0068$+$0105          & 38.3667                 & 0.6833                  & 5,  6                                    & \nodata                           & 0.29                          & 10.45                             & 9.87                              & $-$0.79                         & 3.7                       & 1.5                       & 7.3                          & 5.55E+20                                & 1                        \\
                    1318                    & 03837$-$0008$+$0172          & 38.3750                 & $-$0.0833                 & 6                                        & \nodata                           & 1.59                          & 17.15                             & 16.47                             & $-$0.71                         & 3.4                       & 1.9                       & 8.0                          & 8.96E+20                                & 1                        \\
                    1319                    & 03838$-$0016$+$0175          & 38.3833                 & $-$0.1583                 & 6                                        & \nodata                           & 1.64                          & 17.53                             & 16.22                             & $-$0.99                         & 3.8                       & 2.7                       & 10.0                         & 1.85E+21                                & 1                        \\
                    1320                    & 03845$-$0095$+$0127          & 38.4500                 & $-$0.9500                 & 6                                        & \nodata                           & 0.35                          & 12.66                             & 12.17                             & $-$0.51                         & 4.4                       & 1.6                       & 15.7                         & 7.91E+20                                & 1                        \\
                    1321                    & 03849$+$0080$+$0115          & 38.4917                 & 0.8000                  & 5,  6                                    & \nodata                           & 0.90                          & 11.50                             & 10.65                             & $-$0.52                         & 5.5                       & 1.6                       & 12.2                         & 1.25E+21                                & 1                        \\
                    1322                    & 03851$+$0071$+$0110          & 38.5083                 & 0.7083                  & 3,  5,  6                                & \nodata                           & 0.29                          & 10.99                             & 10.44                             & $-$0.43                         & 4.7                       & 1.4                       & 11.0                         & 8.28E+20                                & 1                        \\
                    1323                    & 03852$+$0074$+$0111          & 38.5167                 & 0.7417                  & 3,  5,  6                                & \nodata                           & 0.30                          & 11.12                             & 10.55                             & $-$0.51                         & 4.9                       & 1.3                       & 9.4                          & 6.29E+20                                & 1                        \\
                    1324                    & 03852$-$0092$+$0124          & 38.5250                 & $-$0.9250                 & 6                                        & \nodata                           & 0.34                          & 12.38                             & 12.01                             & $-$0.38                         & 5.5                       & 1.8                       & 13.8                         & 8.54E+20                                & 1                        \\
                    1325                    & 03855$+$0231$+$0151          & 38.5500                 & 2.3083                  & 5,  6                                    & \nodata                           & 0.32                          & 15.06                             & 14.64                             & $-$0.46                         & 5.8                       & 1.3                       & 11.0                         & 5.40E+20                                & 1                        \\
                    1326                    & 03856$+$0121$+$0109          & 38.5583                 & 1.2083                  & 5,  6                                    & \nodata                           & 0.30                          & 10.91                             & 9.88                              & $-$0.59                         & 4.7                       & 1.2                       & 9.0                          & 9.70E+20                                & 1                        \\
                    1327                    & 03857$+$0087$+$0119          & 38.5750                 & 0.8750                  & 5,  6                                    & \nodata                           & 0.30                          & 11.92                             & 10.56                             & $-$1.62                         & 4.5                       & 2.1                       & 9.8                          & 8.56E+20                                & 1                        \\
                    1328                    & 03857$-$0037$+$0350          & 38.5750                 & $-$0.3750                 & 6                                        & \nodata                           & 1.91                          & 35.03                             & 34.48                             & $-$0.39                         & 5.7                       & 2.0                       & 11.3                         & 1.38E+21                                & 1                        \\
                    1329                    & 03858$+$0076$+$0116          & 38.5833                 & 0.7583                  & 3,  5,  6                                & \nodata                           & 0.30                          & 11.55                             & 10.60                             & $-$0.58                         & 4.8                       & 1.3                       & 10.0                         & 9.66E+20                                & 1                        \\
                    1330                    & 03858$-$0017$+$0356          & 38.5833                 & $-$0.1750                 & 3,  6                                    & \nodata                           & 1.91                          & 35.59                             & 35.24                             & $-$0.25                         & 6.9                       & 1.4                       & 13.7                         & 9.29E+20                                & 1                        \\
                    1331                    & 03859$+$0096$+$0112          & 38.5917                 & 0.9583                  & 5,  6                                    & \nodata                           & 0.30                          & 11.24                             & 9.85                              & $-$0.73                         & 5.9                       & 1.5                       & 14.2                         & 1.36E+21                                & 1                        \\
                    1332                    & 03860$+$0100$+$0110          & 38.6000                 & 1.0000                  & 5,  6                                    & \nodata                           & 0.30                          & 10.99                             & 10.20                             & $-$0.73                         & 5.2                       & 1.1                       & 8.8                          & 5.48E+20                                & 1                        \\
                    1333                    & 03860$-$0092$+$0127          & 38.6000                 & $-$0.9167                 & 6                                        & \nodata                           & 0.35                          & 12.67                             & 11.96                             & $-$0.94                         & 5.6                       & 1.6                       & 12.3                         & 5.92E+20                                & 1                        \\
                    1334                    & 03860$-$0094$+$0125          & 38.6000                 & $-$0.9417                 & 6                                        & \nodata                           & 0.35                          & 12.53                             & 12.09                             & $-$0.51                         & 4.1                       & 1.8                       & 8.6                          & 7.36E+20                                & 1                        \\
                    1335                    & 03861$+$0225$+$0149          & 38.6083                 & 2.2500                  & 5,  6                                    & \nodata                           & 0.32                          & 14.86                             & 14.36                             & $-$0.39                         & 5.4                       & 2.6                       & 19.7                         & 2.00E+21                                & 1                        \\
                    1336                    & 03862$+$0126$+$0128          & 38.6250                 & 1.2583                  & 5,  6                                    & \nodata                           & 0.31                          & 12.82                             & 12.48                             & $-$0.39                         & 4.0                       & 1.4                       & 8.3                          & 5.56E+20                                & 1                        \\
                    1337                    & 03864$+$0224$+$0149          & 38.6417                 & 2.2417                  & 5,  6                                    & \nodata                           & 0.32                          & 14.86                             & 14.24                             & $-$0.36                         & 6.6                       & 2.7                       & 23.8                         & 3.07E+21                                & 1                        \\
                    1338                    & 03865$-$0086$+$0132          & 38.6500                 & $-$0.8583                 & 6                                        & \nodata                           & 0.35                          & 13.18                             & 12.27                             & $-$0.90                         & 4.8                       & 1.4                       & 12.6                         & 6.50E+20                                & 1                        \\
                    1339                    & 03868$-$0032$+$0354          & 38.6833                 & $-$0.3167                 & 6                                        & \nodata                           & 1.91                          & 35.35                             & 34.71                             & $-$0.42                         & 6.4                       & 2.1                       & 11.2                         & 1.53E+21                                & 1                        \\
                    1340                    & 03868$-$0094$+$0127          & 38.6833                 & $-$0.9417                 & 6                                        & \nodata                           & 0.35                          & 12.68                             & 12.18                             & $-$0.54                         & 5.3                       & 1.3                       & 11.7                         & 5.62E+20                                & 1                        \\
                    1341                    & 03868$-$0122$+$0158          & 38.6833                 & $-$1.2167                 & 5,  6                                    & \nodata                           & 0.60                          & 15.80                             & 15.19                             & $-$0.54                         & 4.9                       & 1.4                       & 14.8                         & 7.64E+20                                & 1                        \\
                    1342                    & 03870$-$0125$+$0161          & 38.7000                 & $-$1.2500                 & 5,  6                                    & \nodata                           & 0.60                          & 16.12                             & 15.50                             & $-$0.43                         & 6.0                       & 1.8                       & 17.3                         & 1.43E+21                                & 1                        \\
                    1343                    & 03871$-$0122$+$0158          & 38.7083                 & $-$1.2167                 & 5,  6                                    & \nodata                           & 0.60                          & 15.77                             & 15.18                             & $-$0.46                         & 5.9                       & 1.5                       & 12.3                         & 8.85E+20                                & 1                        \\
                    1344                    & 03873$-$0464$+$0074          & 38.7333                 & $-$4.6417                 & 6                                        & \nodata                           & 0.30                          & 7.35                              & 7.14                              & $-$0.24                         & 3.1                       & 1.0                       & 7.6                          & 4.19E+20                                & 1                        \\
                    1345                    & 03875$-$0097$+$0131          & 38.7500                 & $-$0.9667                 & 5,  6                                    & \nodata                           & 0.35                          & 13.12                             & 12.44                             & $-$0.43                         & 6.0                       & 1.7                       & 16.5                         & 1.44E+21                                & 1                        \\
                    1346                    & 03876$+$0282$+$0138          & 38.7583                 & 2.8250                  & 5,  6                                    & \nodata                           & 0.31                          & 13.83                             & 13.35                             & $-$0.55                         & 4.3                       & 1.1                       & 8.9                          & 4.09E+20                                & 1                        \\
                    1347                    & 03878$-$0098$+$0130          & 38.7833                 & $-$0.9833                 & 5,  6                                    & \nodata                           & 0.34                          & 12.98                             & 12.39                             & $-$0.53                         & 6.3                       & 2.1                       & 16.3                         & 1.22E+21                                & 1                        \\
                    1348                    & 03879$-$0101$+$0130          & 38.7917                 & $-$1.0083                 & 5,  6                                    & \nodata                           & 0.34                          & 13.05                             & 12.51                             & $-$0.40                         & 3.5                       & 1.5                       & 8.5                          & 9.04E+20                                & 1                        \\
                    1349                    & 03881$-$0098$+$0131          & 38.8083                 & $-$0.9833                 & 5,  6                                    & \nodata                           & 0.34                          & 13.10                             & 12.48                             & $-$0.55                         & 6.7                       & 1.7                       & 13.2                         & 9.40E+20                                & 1                        \\
                    1350                    & 03883$-$0102$+$0130          & 38.8333                 & $-$1.0250                 & 5,  6                                    & \nodata                           & 0.34                          & 13.00                             & 12.65                             & $-$0.36                         & 4.3                       & 1.1                       & 13.3                         & 5.13E+20                                & 1                        \\
                    1351                    & 03884$+$0372$+$0238          & 38.8417                 & 3.7167                  & 5,  6                                    & \nodata                           & 0.68                          & 23.77                             & 22.99                             & $-$0.43                         & 3.7                       & 1.2                       & 7.5                          & 1.04E+21                                & 1                        \\
                    1352                    & 03887$-$0018$+$0141          & 38.8667                 & $-$0.1833                 & 6                                        & \nodata                           & 0.30                          & 14.14                             & 13.53                             & $-$0.63                         & 5.2                       & 1.3                       & 12.4                         & 5.90E+20                                & 1                        \\
                    1353                    & 03890$-$0128$+$0168          & 38.9000                 & $-$1.2833                 & 5,  6                                    & \nodata                           & 0.60                          & 16.84                             & 16.58                             & $-$0.25                         & 4.7                       & 2.3                       & 12.3                         & 1.17E+21                                & 1                        \\
                    1354                    & 03900$-$0092$+$0128          & 39.0000                 & $-$0.9250                 & 6                                        & \nodata                           & 0.35                          & 12.82                             & 12.33                             & $-$0.61                         & 5.7                       & 1.0                       & 9.8                          & 3.57E+20                                & 1                        \\
                    1355                    & 03902$-$0249$+$0324          & 39.0250                 & $-$2.4917                 & 1,  6                                    & \nodata                           & 1.89                          & 32.40                             & 31.95                             & $-$0.39                         & 5.5                       & 2.2                       & 11.1                         & 1.25E+21                                & 1                        \\
                    1356                    & 03903$+$0036$+$0130          & 39.0333                 & 0.3583                  & 6                                        & \nodata                           & 0.32                          & 13.01                             & 11.88                             & $-$0.79                         & 4.4                       & 1.2                       & 8.8                          & 7.57E+20                                & 1                        \\
                    1357                    & 03908$-$0089$+$0086          & 39.0833                 & $-$0.8917                 & 6                                        & \nodata                           & 0.32                          & 8.62                              & 8.31                              & $-$0.39                         & 4.3                       & 1.2                       & 10.4                         & 4.48E+20                                & 1                        \\
                    1358                    & 03920$+$0046$+$0289          & 39.2000                 & 0.4583                  & 6                                        & \nodata                           & 1.27                          & 28.90                             & 27.93                             & $-$0.77                         & 6.9                       & 1.8                       & 13.5                         & 1.10E+21                                & 1                        \\
                    1359                    & 03928$-$0014$+$0152          & 39.2833                 & $-$0.1417                 & 3,  6                                    & \nodata                           & 0.30                          & 15.24                             & 14.86                             & $-$0.48                         & 4.0                       & 1.3                       & 9.8                          & 4.76E+20                                & 1                        \\
                    1360                    & 03928$-$0019$+$0699          & 39.2833                 & $-$0.1917                 & 3,  6,  7                                & 17                                & 4.05                          & 69.91                             & 69.13                             & $-$0.32                         & 4.5                       & 4.5                       & 9.5                          & 7.72E+21                                & 1                        \\
                    1361                    & 03930$+$0396$+$0292          & 39.3000                 & 3.9583                  & 5,  6                                    & \nodata                           & 1.01                          & 29.24                             & 28.58                             & $-$0.96                         & 3.1                       & 1.4                       & 6.8                          & 4.77E+20                                & 1                        \\
                    1362                    & 03930$-$0100$+$0127          & 39.3000                 & $-$1.0000                 & 6                                        & \nodata                           & 0.35                          & 12.67                             & 11.59                             & $-$0.89                         & 4.4                       & 2.0                       & 9.6                          & 1.20E+21                                & 1                        \\
                    1363                    & 03933$-$0102$+$0129          & 39.3333                 & $-$1.0250                 & 6                                        & \nodata                           & 0.35                          & 12.89                             & 11.85                             & $-$0.69                         & 5.7                       & 2.7                       & 11.9                         & 2.09E+21                                & 1                        \\
                    1364                    & 03934$-$0026$+$0696          & 39.3417                 & $-$0.2583                 & 3,  6                                    & \nodata                           & 4.01                          & 69.57                             & 69.37                             & $-$0.22                         & 5.3                       & 2.0                       & 8.9                          & 6.77E+21                                & 2                        \\
                    1365                    & 03934$-$0200$+$0314          & 39.3417                 & $-$2.0000                 & 5,  6                                    & \nodata                           & 1.20                          & 31.36                             & 30.70                             & $-$0.39                         & 6.4                       & 2.7                       & 11.1                         & 2.34E+21                                & 1                        \\
                    1366                    & 03935$+$0179$+$0312          & 39.3500                 & 1.7917                  & 5,  6                                    & \nodata                           & 1.90                          & 31.24                             & 30.30                             & $-$0.40                         & 5.8                       & 1.3                       & 12.2                         & 1.41E+21                                & 1                        \\
                    1367                    & 03937$-$0202$+$0313          & 39.3750                 & $-$2.0250                 & 5,  6                                    & \nodata                           & 1.18                          & 31.27                             & 30.60                             & $-$0.37                         & 5.6                       & 1.9                       & 11.2                         & 1.65E+21                                & 1                        \\
                    1368                    & 03941$-$0202$+$0313          & 39.4083                 & $-$2.0250                 & 5,  6                                    & \nodata                           & 1.19                          & 31.34                             & 30.70                             & $-$0.35                         & 6.2                       & 1.7                       & 11.5                         & 1.46E+21                                & 1                        \\
                    1369                    & 03942$-$0117$+$0126          & 39.4250                 & $-$1.1750                 & 6                                        & 17                                & 0.35                          & 12.64                             & 12.03                             & $-$0.45                         & 4.9                       & 3.0                       & 9.3                          & 2.26E+21                                & 1                        \\
                    1370                    & 03945$-$0117$+$0127          & 39.4500                 & $-$1.1667                 & 6                                        & \nodata                           & 0.36                          & 12.69                             & 12.09                             & $-$0.46                         & 4.9                       & 2.4                       & 12.1                         & 1.57E+21                                & 1                        \\
                    1371                    & 03945$-$0218$+$0306          & 39.4500                 & $-$2.1833                 & 5,  6                                    & \nodata                           & 1.10                          & 30.58                             & 30.12                             & $-$0.48                         & 5.9                       & 2.7                       & 11.3                         & 1.33E+21                                & 1                        \\
                    1372                    & 03952$-$0117$+$0129          & 39.5167                 & $-$1.1667                 & 6                                        & \nodata                           & 0.36                          & 12.87                             & 12.33                             & $-$0.40                         & 5.1                       & 1.8                       & 11.1                         & 1.15E+21                                & 1                        \\
                    1373                    & 03953$+$0194$+$0145          & 39.5333                 & 1.9417                  & 5,  6                                    & \nodata                           & 0.31                          & 14.52                             & 14.06                             & $-$0.56                         & 4.8                       & 2.2                       & 9.6                          & 8.69E+20                                & 1                        \\
                    1374                    & 03955$-$0043$+$0134          & 39.5500                 & $-$0.4333                 & 6                                        & \nodata                           & 0.30                          & 13.40                             & 12.94                             & $-$0.47                         & 4.0                       & 1.4                       & 9.8                          & 6.17E+20                                & 1                        \\
                    1375                    & 03957$-$0116$+$0130          & 39.5750                 & $-$1.1583                 & 6                                        & \nodata                           & 0.36                          & 12.97                             & 12.49                             & $-$0.46                         & 5.6                       & 2.3                       & 8.8                          & 1.24E+21                                & 1                        \\
                    1376                    & 03957$-$0260$+$0313          & 39.5750                 & $-$2.6000                 & 5,  6                                    & \nodata                           & 1.28                          & 31.26                             & 29.98                             & $-$1.18                         & 3.4                       & 1.5                       & 8.7                          & 7.56E+20                                & 1                        \\
                    1377                    & 03961$-$0114$+$0129          & 39.6083                 & $-$1.1417                 & 6                                        & \nodata                           & 0.36                          & 12.89                             & 12.05                             & $-$0.58                         & 5.2                       & 1.5                       & 10.9                         & 9.65E+20                                & 1                        \\
                    1378                    & 03962$-$0067$+$0131          & 39.6167                 & $-$0.6667                 & 6                                        & \nodata                           & 0.32                          & 13.12                             & 12.33                             & $-$1.00                         & 3.8                       & 1.8                       & 9.5                          & 6.84E+20                                & 1                        \\
                    1379                    & 03966$+$0195$+$0302          & 39.6583                 & 1.9500                  & 5,  6                                    & \nodata                           & 1.89                          & 30.17                             & 29.36                             & $-$0.52                         & 3.6                       & 2.3                       & 7.2                          & 2.10E+21                                & 1                        \\
                    1380                    & 03968$+$0192$+$0295          & 39.6833                 & 1.9250                  & 5,  6                                    & \nodata                           & 1.89                          & 29.52                             & 28.55                             & $-$0.69                         & 6.5                       & 4.0                       & 9.9                          & 3.48E+21                                & 1                        \\
                    1381                    & 03972$+$0197$+$0295          & 39.7250                 & 1.9667                  & 5,  6                                    & \nodata                           & 1.89                          & 29.54                             & 28.14                             & $-$0.49                         & 6.1                       & 3.9                       & 15.9                         & 6.60E+21                                & 1                        \\
                    1382                    & 03972$-$0255$+$0308          & 39.7250                 & $-$2.5500                 & 5,  6                                    & \nodata                           & 1.21                          & 30.83                             & 30.13                             & $-$0.62                         & 3.9                       & 1.4                       & 10.8                         & 7.39E+20                                & 1                        \\
                    1383                    & 03974$+$0190$+$0295          & 39.7417                 & 1.9000                  & 5,  6                                    & \nodata                           & 1.90                          & 29.50                             & 28.36                             & $-$0.37                         & 8.2                       & 4.9                       & 13.5                         & 9.20E+21                                & 1                        \\
                    1384                    & 03975$+$0177$+$0288          & 39.7500                 & 1.7667                  & 5,  6                                    & \nodata                           & 1.90                          & 28.81                             & 27.66                             & $-$0.46                         & 7.2                       & 3.6                       & 15.9                         & 5.09E+21                                & 1                        \\
                    1385                    & 03975$-$0047$+$0117          & 39.7500                 & $-$0.4750                 & 6                                        & \nodata                           & 0.29                          & 11.65                             & 11.15                             & $-$0.71                         & 3.8                       & 1.2                       & 9.0                          & 3.91E+20                                & 1                        \\
                    1386                    & 03977$+$0183$+$0292          & 39.7750                 & 1.8333                  & 5,  6                                    & \nodata                           & 1.89                          & 29.18                             & 27.81                             & $-$0.37                         & 8.0                       & 2.6                       & 16.7                         & 5.23E+21                                & 1                        \\
                    1387                    & 03979$-$0252$+$0306          & 39.7917                 & $-$2.5250                 & 5,  6                                    & \nodata                           & 1.18                          & 30.64                             & 30.07                             & $-$0.66                         & 3.2                       & 1.9                       & 7.2                          & 8.67E+20                                & 1                        \\
                    1388                    & 03982$-$0251$+$0308          & 39.8167                 & $-$2.5083                 & 5,  6                                    & \nodata                           & 1.18                          & 30.75                             & 30.22                             & $-$0.60                         & 4.2                       & 1.5                       & 10.6                         & 6.24E+20                                & 1                        \\
                    1389                    & 03982$+$0007$+$0428          & 39.8250                 & 0.0750                  & 6                                        & \nodata                           & 1.91                          & 42.77                             & 42.20                             & $-$0.58                         & 3.9                       & 2.4                       & 8.4                          & 1.27E+21                                & 1                        \\
                    1390                    & 03983$+$0165$+$0276          & 39.8333                 & 1.6500                  & 5,  6                                    & \nodata                           & 1.90                          & 27.57                             & 26.47                             & $-$0.50                         & 7.1                       & 3.7                       & 11.7                         & 4.47E+21                                & 1                        \\
                    1391                    & 03987$+$0167$+$0291          & 39.8750                 & 1.6667                  & 5,  6                                    & \nodata                           & 1.90                          & 29.13                             & 28.39                             & $-$0.46                         & 4.7                       & 2.6                       & 10.0                         & 2.15E+21                                & 1                        \\
                    1392                    & 03987$-$0068$+$0136          & 39.8750                 & $-$0.6833                 & 6                                        & \nodata                           & 0.31                          & 13.57                             & 13.12                             & $-$0.44                         & 4.1                       & 1.0                       & 9.3                          & 4.37E+20                                & 1                        \\
                    1393                    & 03988$-$0119$+$0133          & 39.8833                 & $-$1.1917                 & 6                                        & \nodata                           & 0.36                          & 13.28                             & 12.26                             & $-$0.85                         & 7.4                       & 2.7                       & 13.7                         & 1.68E+21                                & 1                        \\
                    1394                    & 03989$-$0034$+$0586          & 39.8917                 & $-$0.3417                 & 1,  6                                    & \nodata                           & 4.02                          & 58.64                             & 56.59                             & $-$0.44                         & 11.0                      & 4.8                       & 21.0                         & 1.49E+22                                & 1                        \\
                    1395                    & 03992$-$0120$+$0132          & 39.9167                 & $-$1.2000                 & 6                                        & \nodata                           & 0.36                          & 13.21                             & 12.36                             & $-$0.81                         & 7.8                       & 3.1                       & 14.2                         & 1.77E+21                                & 1                        \\
                    1396                    & 03992$+$0157$+$0078          & 39.9250                 & 1.5667                  & 5,  6                                    & \nodata                           & 0.29                          & 7.85                              & 7.26                              & $-$0.48                         & 6.5                       & 1.4                       & 11.5                         & 7.98E+20                                & 1                        \\
                    1397                    & 03993$+$0191$+$0302          & 39.9333                 & 1.9083                  & 5,  6                                    & \nodata                           & 1.89                          & 30.19                             & 29.52                             & $-$0.38                         & 5.1                       & 3.0                       & 9.5                          & 2.85E+21                                & 1                        \\
                    1398                    & 03995$+$0166$+$0281          & 39.9500                 & 1.6583                  & 5,  6                                    & \nodata                           & 1.90                          & 28.11                             & 26.69                             & $-$0.91                         & 5.1                       & 3.9                       & 10.1                         & 3.65E+21                                & 1                        \\
                    1399                    & 03996$+$0168$+$0278          & 39.9583                 & 1.6833                  & 5,  6                                    & \nodata                           & 1.90                          & 27.76                             & 26.67                             & $-$0.48                         & 6.0                       & 5.4                       & 15.5                         & 7.81E+21                                & 1                        \\
                    1400                    & 03996$-$0109$+$0136          & 39.9583                 & $-$1.0917                 & 6                                        & \nodata                           & 0.36                          & 13.61                             & 12.77                             & $-$0.62                         & 5.8                       & 2.7                       & 16.5                         & 1.99E+21                                & 1                        \\
                    1401                    & 03997$-$0118$+$0132          & 39.9667                 & $-$1.1833                 & 6                                        & \nodata                           & 0.36                          & 13.24                             & 12.53                             & $-$0.48                         & 7.1                       & 1.9                       & 24.3                         & 1.87E+21                                & 1                        \\
                    1402                    & 03997$-$0113$+$0136          & 39.9750                 & $-$1.1333                 & 6                                        & \nodata                           & 0.35                          & 13.61                             & 12.65                             & $-$0.58                         & 6.3                       & 2.2                       & 15.7                         & 1.97E+21                                & 1                        \\
                    1403                    & 03998$+$0167$+$0278          & 39.9833                 & 1.6667                  & 5,  6                                    & \nodata                           & 1.90                          & 27.82                             & 27.34                             & $-$0.31                         & 7.2                       & 6.1                       & 11.4                         & 7.29E+21                                & 1                        \\
                    1404                    & 03998$-$0108$+$0135          & 39.9833                 & $-$1.0833                 & 6                                        & \nodata                           & 0.35                          & 13.49                             & 12.88                             & $-$0.44                         & 6.2                       & 2.4                       & 14.7                         & 1.77E+21                                & 1                        \\
                    1405                    & 03998$-$0111$+$0136          & 39.9833                 & $-$1.1083                 & 6                                        & \nodata                           & 0.35                          & 13.56                             & 12.61                             & $-$0.59                         & 5.5                       & 2.2                       & 14.1                         & 1.78E+21                                & 1                        \\
                    1406                    & 03999$+$0307$+$0291          & 39.9917                 & 3.0750                  & 5,  6                                    & \nodata                           & 0.87                          & 29.10                             & 27.34                             & $-$1.47                         & 4.8                       & 2.1                       & 9.5                          & 1.22E+21                                & 1                        \\
                    1407                    & 04001$+$0301$+$0290          & 40.0083                 & 3.0083                  & 5,  6                                    & \nodata                           & 1.60                          & 29.01                             & 27.75                             & $-$0.84                         & 4.4                       & 2.1                       & 8.5                          & 1.56E+21                                & 1                        \\
                    1408                    & 04001$-$0094$+$0135          & 40.0083                 & $-$0.9417                 & 6                                        & \nodata                           & 0.35                          & 13.53                             & 12.98                             & $-$0.48                         & 5.5                       & 2.7                       & 10.7                         & 1.60E+21                                & 1                        \\
                    1409                    & 04001$-$0128$+$0131          & 40.0083                 & $-$1.2833                 & 6                                        & \nodata                           & 0.36                          & 13.10                             & 12.48                             & $-$0.49                         & 6.6                       & 2.1                       & 15.0                         & 1.37E+21                                & 1                        \\
                    1410                    & 04002$-$0073$+$0129          & 40.0167                 & $-$0.7333                 & 6                                        & \nodata                           & 0.32                          & 12.94                             & 12.39                             & $-$0.59                         & 3.9                       & 2.3                       & 11.3                         & 1.07E+21                                & 1                        \\
                    1411                    & 04003$-$0114$+$0132          & 40.0333                 & $-$1.1417                 & 6                                        & \nodata                           & 0.36                          & 13.24                             & 12.62                             & $-$0.53                         & 6.8                       & 2.3                       & 12.6                         & 1.38E+21                                & 1                        \\
                    1412                    & 04003$-$0128$+$0130          & 40.0333                 & $-$1.2833                 & 6                                        & \nodata                           & 0.36                          & 12.99                             & 12.46                             & $-$0.34                         & 6.5                       & 1.8                       & 16.1                         & 1.52E+21                                & 1                        \\
                    1413                    & 04005$-$0068$+$0129          & 40.0500                 & $-$0.6833                 & 6                                        & \nodata                           & 0.31                          & 12.91                             & 12.32                             & $-$0.51                         & 3.9                       & 1.5                       & 8.3                          & 7.90E+20                                & 1                        \\
                    1414                    & 04006$-$0113$+$0133          & 40.0583                 & $-$1.1333                 & 6                                        & \nodata                           & 0.36                          & 13.34                             & 12.77                             & $-$0.42                         & 6.2                       & 2.3                       & 13.4                         & 1.57E+21                                & 1                        \\
                    1415                    & 04006$-$0128$+$0132          & 40.0583                 & $-$1.2833                 & 6                                        & \nodata                           & 0.36                          & 13.20                             & 12.58                             & $-$0.33                         & 7.0                       & 1.6                       & 22.4                         & 1.90E+21                                & 1                        \\
                    1416                    & 04007$-$0077$+$0130          & 40.0667                 & $-$0.7667                 & 6                                        & \nodata                           & 0.32                          & 13.01                             & 12.27                             & $-$0.60                         & 4.9                       & 1.9                       & 9.4                          & 1.13E+21                                & 1                        \\
                    1417                    & 04007$-$0092$+$0135          & 40.0667                 & $-$0.9167                 & 1,  6                                    & \nodata                           & 0.35                          & 13.48                             & 12.86                             & $-$0.44                         & 5.5                       & 2.5                       & 10.1                         & 1.76E+21                                & 1                        \\
                    1418                    & 04007$-$0098$+$0136          & 40.0750                 & $-$0.9833                 & 6                                        & \nodata                           & 0.35                          & 13.55                             & 12.90                             & $-$0.51                         & 5.2                       & 1.9                       & 10.9                         & 1.18E+21                                & 1                        \\
                    1419                    & 04009$-$0067$+$0127          & 40.0917                 & $-$0.6750                 & 6                                        & \nodata                           & 0.31                          & 12.71                             & 12.21                             & $-$0.55                         & 3.9                       & 2.2                       & 10.2                         & 9.49E+20                                & 1                        \\
                    1420                    & 04011$-$0125$+$0131          & 40.1083                 & $-$1.2500                 & 6                                        & \nodata                           & 0.36                          & 13.13                             & 12.31                             & $-$0.58                         & 7.3                       & 2.5                       & 17.8                         & 1.96E+21                                & 1                        \\
                    1421                    & 04012$-$0067$+$0128          & 40.1167                 & $-$0.6750                 & 6                                        & \nodata                           & 0.31                          & 12.83                             & 12.12                             & $-$0.75                         & 3.7                       & 2.2                       & 8.9                          & 1.01E+21                                & 1                        \\
                    1422                    & 04012$-$0070$+$0128          & 40.1167                 & $-$0.7000                 & 6                                        & \nodata                           & 0.31                          & 12.76                             & 12.16                             & $-$0.59                         & 4.2                       & 1.8                       & 10.6                         & 8.85E+20                                & 1                        \\
                    1423                    & 04012$-$0075$+$0128          & 40.1167                 & $-$0.7500                 & 6                                        & \nodata                           & 0.32                          & 12.79                             & 12.15                             & $-$0.53                         & 4.2                       & 1.8                       & 8.4                          & 1.05E+21                                & 1                        \\
                    1424                    & 04014$-$0073$+$0128          & 40.1417                 & $-$0.7333                 & 6                                        & \nodata                           & 0.32                          & 12.80                             & 12.11                             & $-$0.59                         & 3.9                       & 1.6                       & 12.5                         & 8.73E+20                                & 1                        \\
                    1425                    & 04015$-$0070$+$0128          & 40.1500                 & $-$0.7000                 & 6                                        & \nodata                           & 0.31                          & 12.83                             & 12.19                             & $-$0.70                         & 3.7                       & 1.6                       & 8.7                          & 6.65E+20                                & 1                        \\
                    1426                    & 04035$-$0187$+$0113          & 40.3500                 & $-$1.8750                 & 6                                        & \nodata                           & 0.35                          & 11.26                             & 10.87                             & $-$0.42                         & 4.3                       & 1.2                       & 10.9                         & 5.08E+20                                & 1                        \\
                    1427                    & 04037$+$0255$+$0335          & 40.3750                 & 2.5500                  & 1,  3,  5,  6                            & \nodata                           & 0.54                          & 33.46                             & 32.97                             & $-$0.38                         & 6.0                       & 2.6                       & 10.3                         & 1.69E+21                                & 1                        \\
                    1428                    & 04037$-$0192$+$0343          & 40.3750                 & $-$1.9250                 & 6                                        & \nodata                           & 1.40                          & 34.26                             & 33.66                             & $-$0.57                         & 3.4                       & 1.2                       & 6.9                          & 6.04E+20                                & 1                        \\
                    1429                    & 04042$+$0252$+$0279          & 40.4167                 & 2.5250                  & 3,  5,  6                                & \nodata                           & 1.74                          & 27.95                             & 26.97                             & $-$0.61                         & 2.8                       & 1.1                       & 7.0                          & 6.31E+21                                & 2                        \\
                    1430                    & 04044$-$0128$+$0135          & 40.4417                 & $-$1.2833                 & 5,  6                                    & \nodata                           & 0.37                          & 13.51                             & 13.20                             & $-$0.41                         & 5.1                       & 1.3                       & 10.3                         & 4.35E+20                                & 1                        \\
                    1431                    & 04046$+$0099$+$0332          & 40.4583                 & 0.9917                  & 6                                        & \nodata                           & 1.92                          & 33.24                             & 32.67                             & $-$0.51                         & 6.4                       & 1.6                       & 10.4                         & 8.14E+20                                & 1                        \\
                    1432                    & 04048$-$0110$+$0061          & 40.4833                 & $-$1.1000                 & 5,  6                                    & \nodata                           & 0.32                          & 6.07                              & 5.83                              & $-$0.24                         & 5.3                       & 1.4                       & 12.2                         & 6.54E+20                                & 1                        \\
                    1433                    & 04055$-$0147$+$0311          & 40.5500                 & $-$1.4667                 & 5,  6                                    & \nodata                           & 1.29                          & 31.06                             & 29.77                             & $-$0.77                         & 5.4                       & 2.4                       & 11.7                         & 2.02E+21                                & 1                        \\
                    1434                    & 04062$+$0234$+$0286          & 40.6167                 & 2.3417                  & 5,  6                                    & \nodata                           & 1.69                          & 28.57                             & 27.91                             & $-$0.36                         & 5.6                       & 5.1                       & 9.9                          & 7.10E+21                                & 1                        \\
                    1435                    & 04063$+$0244$+$0293          & 40.6333                 & 2.4417                  & 3,  5,  6                                & \nodata                           & 1.68                          & 29.28                             & 28.60                             & $-$0.42                         & 2.8                       & 1.1                       & 6.9                          & 6.88E+21                                & 2                        \\
                    1436                    & 04071$-$0376$+$0066          & 40.7083                 & $-$3.7583                 & 1,  5,  6                                & \nodata                           & 0.31                          & 6.59                              & 6.22                              & $-$0.48                         & 5.5                       & 1.4                       & 11.6                         & 4.87E+20                                & 1                        \\
                    1437                    & 04075$-$0382$+$0068          & 40.7500                 & $-$3.8250                 & 5,  6                                    & \nodata                           & 0.31                          & 6.82                              & 6.37                              & $-$0.30                         & 3.0                       & 1.0                       & 7.0                          & 6.84E+20                                & 1                        \\
                    1438                    & 04076$-$0152$+$0107          & 40.7583                 & $-$1.5167                 & 5,  6                                    & \nodata                           & 0.34                          & 10.68                             & 10.07                             & $-$0.63                         & 4.8                       & 1.2                       & 11.0                         & 5.26E+20                                & 1                        \\
                    1439                    & 04077$-$0385$+$0069          & 40.7667                 & $-$3.8500                 & 5,  6                                    & \nodata                           & 0.31                          & 6.92                              & 6.30                              & $-$0.67                         & 4.9                       & 1.1                       & 13.3                         & 5.01E+20                                & 1                        \\
                    1440                    & 04077$-$0396$+$0073          & 40.7750                 & $-$3.9583                 & 1,  5,  6                                & \nodata                           & 0.31                          & 7.32                              & 6.68                              & $-$0.63                         & 3.1                       & 1.1                       & 9.5                          & 5.01E+20                                & 1                        \\
                    1441                    & 04082$-$0373$+$0069          & 40.8250                 & $-$3.7333                 & 5,  6                                    & \nodata                           & 0.31                          & 6.85                              & 6.34                              & $-$0.49                         & 4.1                       & 0.9                       & 11.1                         & 4.16E+20                                & 1                        \\
                    1442                    & 04092$-$0371$+$0067          & 40.9167                 & $-$3.7083                 & 5,  6                                    & \nodata                           & 0.31                          & 6.68                              & 6.27                              & $-$0.51                         & 3.6                       & 1.4                       & 9.4                          & 5.35E+20                                & 1                        \\
                    1443                    & 04092$-$0367$+$0067          & 40.9250                 & $-$3.6750                 & 5,  6                                    & \nodata                           & 0.31                          & 6.73                              & 6.30                              & $-$0.50                         & 3.6                       & 1.6                       & 13.0                         & 6.67E+20                                & 1                        \\
                    1444                    & 04095$-$0371$+$0068          & 40.9500                 & $-$3.7083                 & 5,  6                                    & \nodata                           & 0.31                          & 6.84                              & 6.41                              & $-$0.56                         & 3.7                       & 1.6                       & 10.3                         & 5.83E+20                                & 1                        \\
                    1445                    & 04097$-$0366$+$0068          & 40.9667                 & $-$3.6583                 & 5,  6                                    & \nodata                           & 0.31                          & 6.76                              & 6.47                              & $-$0.41                         & 4.3                       & 1.7                       & 9.2                          & 5.56E+20                                & 1                        \\
                    1446                    & 04097$-$0387$+$0069          & 40.9667                 & $-$3.8750                 & 5,  6                                    & \nodata                           & 0.31                          & 6.86                              & 6.46                              & $-$0.48                         & 2.9                       & 2.0                       & 7.1                          & 9.17E+20                                & 1                        \\
                    1447                    & 04097$-$0077$+$0071          & 40.9750                 & $-$0.7750                 & 5,  6                                    & \nodata                           & 0.31                          & 7.14                              & 6.49                              & $-$0.59                         & 5.5                       & 1.3                       & 15.7                         & 7.38E+20                                & 1                        \\
                    1448                    & 04099$+$0182$+$0333          & 40.9917                 & 1.8167                  & 5,  6                                    & \nodata                           & 1.90                          & 33.28                             & 32.11                             & $-$1.25                         & 4.6                       & 2.2                       & 10.0                         & 1.02E+21                                & 1                        \\
                    1449                    & 04100$+$0327$+$0342          & 41.0000                 & 3.2750                  & 5,  6                                    & \nodata                           & 0.39                          & 34.23                             & 33.28                             & $-$1.05                         & 4.0                       & 1.1                       & 8.4                          & 4.55E+20                                & 1                        \\
                    1450                    & 04102$-$0372$+$0067          & 41.0167                 & $-$3.7250                 & 5,  6                                    & \nodata                           & 0.31                          & 6.75                              & 6.22                              & $-$0.70                         & 2.9                       & 1.2                       & 8.6                          & 4.16E+20                                & 1                        \\
                    1451                    & 04105$-$0030$+$0393          & 41.0500                 & $-$0.3000                 & 3,  5,  6                                & \nodata                           & 1.94                          & 39.28                             & 38.63                             & $-$0.76                         & 2.9                       & 0.7                       & 6.4                          & 2.06E+21                                & 2                        \\
                    1452                    & 04106$-$0127$+$0155          & 41.0583                 & $-$1.2667                 & 5,  6                                    & \nodata                           & 0.26                          & 15.49                             & 15.14                             & $-$0.39                         & 4.8                       & 1.4                       & 9.5                          & 5.94E+20                                & 1                        \\
                    1453                    & 04106$-$0362$+$0067          & 41.0583                 & $-$3.6250                 & 6                                        & \nodata                           & 0.31                          & 6.71                              & 6.37                              & $-$0.28                         & 5.1                       & 1.3                       & 12.7                         & 7.44E+20                                & 1                        \\
                    1454                    & 04109$-$0108$+$0158          & 41.0917                 & $-$1.0833                 & 5,  6                                    & \nodata                           & 0.26                          & 15.78                             & 15.22                             & $-$0.61                         & 5.3                       & 1.7                       & 12.1                         & 7.21E+20                                & 1                        \\
                    1455                    & 04112$-$0124$+$0154          & 41.1167                 & $-$1.2417                 & 5,  6                                    & \nodata                           & 0.26                          & 15.44                             & 15.07                             & $-$0.41                         & 5.8                       & 1.4                       & 11.8                         & 5.78E+20                                & 1                        \\
                    1456                    & 04126$-$0144$+$0105          & 41.2583                 & $-$1.4417                 & 5,  6                                    & \nodata                           & 0.34                          & 10.46                             & 9.98                              & $-$0.66                         & 3.4                       & 1.2                       & 8.5                          & 4.04E+20                                & 1                        \\
                    1457                    & 04146$+$0202$+$0202          & 41.4583                 & 2.0167                  & 6                                        & \nodata                           & 0.26                          & 20.15                             & 19.16                             & $-$1.19                         & 3.7                       & 1.6                       & 10.9                         & 6.09E+20                                & 1                        \\
                    1458                    & 04147$+$0038$+$0178          & 41.4667                 & 0.3833                  & 5,  6                                    & \nodata                           & 0.62                          & 17.77                             & 17.08                             & $-$0.69                         & 3.0                       & 1.3                       & 7.3                          & 6.21E+20                                & 1                        \\
                    1459                    & 04152$-$0118$+$0347          & 41.5167                 & $-$1.1833                 & 5,  6                                    & \nodata                           & 0.24                          & 34.70                             & 34.09                             & $-$0.35                         & 3.6                       & 1.8                       & 7.7                          & 1.52E+21                                & 1                        \\
                    1460                    & 04179$-$0101$+$0109          & 41.7917                 & $-$1.0083                 & 5,  6                                    & \nodata                           & 0.33                          & 10.95                             & 10.25                             & $-$0.55                         & 6.5                       & 1.9                       & 14.0                         & 1.21E+21                                & 1                        \\
                    1461                    & 04184$-$0290$+$0067          & 41.8417                 & $-$2.9000                 & 5,  6                                    & \nodata                           & 0.32                          & 6.71                              & 6.42                              & $-$0.41                         & 3.8                       & 1.4                       & 9.0                          & 4.61E+20                                & 1                        \\
                    1462                    & 04185$-$0280$+$0068          & 41.8500                 & $-$2.8000                 & 5,  6                                    & \nodata                           & 0.32                          & 6.83                              & 6.58                              & $-$0.29                         & 4.0                       & 1.3                       & 14.8                         & 5.48E+20                                & 1                        \\
                    1463                    & 04188$-$0090$+$0110          & 41.8833                 & $-$0.9000                 & 5,  6                                    & \nodata                           & 0.33                          & 10.98                             & 10.64                             & $-$0.44                         & 6.5                       & 1.3                       & 12.9                         & 4.66E+20                                & 1                        \\
                    1464                    & 04192$-$0098$+$0110          & 41.9250                 & $-$0.9833                 & 5,  6                                    & \nodata                           & 0.33                          & 11.03                             & 10.48                             & $-$0.54                         & 5.2                       & 1.5                       & 11.1                         & 7.14E+20                                & 1                        \\
                    1465                    & 04193$+$0225$+$0187          & 41.9333                 & 2.2500                  & 6                                        & \nodata                           & 0.26                          & 18.74                             & 18.24                             & $-$0.41                         & 3.5                       & 1.4                       & 9.0                          & 7.96E+20                                & 1                        \\
                    1466                    & 04204$+$0219$+$0196          & 42.0417                 & 2.1917                  & 6                                        & \nodata                           & 0.26                          & 19.62                             & 19.06                             & $-$0.41                         & 2.8                       & 1.7                       & 7.2                          & 1.16E+21                                & 1                        \\
                    1467                    & 04207$+$0218$+$0196          & 42.0667                 & 2.1833                  & 6                                        & \nodata                           & 0.26                          & 19.55                             & 18.93                             & $-$0.57                         & 3.3                       & 1.9                       & 11.1                         & 9.61E+20                                & 1                        \\
                    1468                    & 04207$+$0222$+$0197          & 42.0750                 & 2.2250                  & 6                                        & \nodata                           & 0.26                          & 19.67                             & 19.16                             & $-$0.53                         & 3.9                       & 1.5                       & 10.1                         & 6.31E+20                                & 1                        \\
                    1469                    & 04211$-$0079$+$0113          & 42.1083                 & $-$0.7917                 & 5,  6                                    & \nodata                           & 0.32                          & 11.26                             & 11.01                             & $-$0.29                         & 5.6                       & 1.9                       & 12.0                         & 8.05E+20                                & 1                        \\
                    1470                    & 04212$+$0222$+$0197          & 42.1167                 & 2.2167                  & 6                                        & \nodata                           & 0.26                          & 19.75                             & 19.06                             & $-$0.53                         & 3.1                       & 1.3                       & 8.8                          & 7.82E+20                                & 1                        \\
                    1471                    & 04212$+$0219$+$0198          & 42.1250                 & 2.1917                  & 6                                        & \nodata                           & 0.26                          & 19.84                             & 19.28                             & $-$0.58                         & 2.8                       & 1.4                       & 8.5                          & 6.31E+20                                & 1                        \\
                    1472                    & 04212$-$0060$+$0679          & 42.1250                 & $-$0.6000                 & 1,  3,  5,  6                            & \nodata                           & 3.51                          & 67.94                             & 67.04                             & $-$0.34                         & 18.4                      & 8.8                       & 25.4                         & 1.92E+22                                & 1                        \\
                    1473                    & 04214$-$0057$+$0688          & 42.1417                 & $-$0.5667                 & 3,  5,  6                                & \nodata                           & 3.52                          & 68.80                             & 66.42                             & $-$0.90                         & 11.4                      & 5.4                       & 16.5                         & 9.08E+21                                & 1                        \\
                    1474                    & 04222$-$0005$+$0399          & 42.2250                 & $-$0.0500                 & 5,  6                                    & \nodata                           & 2.64                          & 39.93                             & 39.08                             & $-$0.60                         & 5.6                       & 1.7                       & 9.1                          & 1.11E+21                                & 1                        \\
                    1475                    & 04230$+$0036$+$0150          & 42.3000                 & 0.3583                  & 5,  6                                    & \nodata                           & 0.59                          & 14.99                             & 14.62                             & $-$0.38                         & 7.9                       & 2.8                       & 13.8                         & 1.44E+21                                & 1                        \\
                    1476                    & 04237$-$0045$+$0109          & 42.3667                 & $-$0.4500                 & 5,  6                                    & \nodata                           & 0.30                          & 10.93                             & 10.45                             & $-$0.59                         & 5.5                       & 2.8                       & 10.0                         & 1.21E+21                                & 1                        \\
                    1477                    & 04237$-$0288$+$0073          & 42.3750                 & $-$2.8833                 & 5,  6                                    & \nodata                           & 0.32                          & 7.28                              & 6.74                              & $-$0.52                         & 4.7                       & 1.1                       & 10.8                         & 5.25E+20                                & 1                        \\
                    1478                    & 04239$-$0047$+$0108          & 42.3917                 & $-$0.4667                 & 5,  6                                    & \nodata                           & 0.30                          & 10.82                             & 10.41                             & $-$0.50                         & 5.9                       & 3.6                       & 10.4                         & 1.65E+21                                & 1                        \\
                    1479                    & 04241$-$0055$+$0105          & 42.4083                 & $-$0.5500                 & 5,  6                                    & \nodata                           & 0.30                          & 10.47                             & 10.22                             & $-$0.28                         & 6.6                       & 3.1                       & 13.4                         & 1.51E+21                                & 1                        \\
                    1480                    & 04244$-$0033$+$0110          & 42.4417                 & $-$0.3333                 & 5,  6                                    & \nodata                           & 0.29                          & 10.99                             & 10.42                             & $-$0.48                         & 6.8                       & 1.5                       & 12.3                         & 8.61E+20                                & 1                        \\
                    1481                    & 04253$-$0043$+$0108          & 42.5333                 & $-$0.4333                 & 5,  6                                    & \nodata                           & 0.29                          & 10.83                             & 10.33                             & $-$0.41                         & 5.5                       & 2.6                       & 12.6                         & 1.59E+21                                & 1                        \\
                    1482                    & 04257$-$0047$+$0106          & 42.5750                 & $-$0.4667                 & 5,  6                                    & \nodata                           & 0.30                          & 10.60                             & 10.00                             & $-$0.43                         & 5.3                       & 1.4                       & 12.0                         & 8.78E+20                                & 1                        \\
                    1483                    & 04257$-$0305$+$0083          & 42.5750                 & $-$3.0500                 & 5,  6                                    & \nodata                           & 0.32                          & 8.33                              & 7.81                              & $-$0.48                         & 3.2                       & 1.5                       & 9.0                          & 7.28E+20                                & 1                        \\
                    1484                    & 04261$-$0046$+$0105          & 42.6083                 & $-$0.4583                 & 5,  6                                    & \nodata                           & 0.30                          & 10.54                             & 10.10                             & $-$0.43                         & 6.0                       & 1.5                       & 11.5                         & 7.17E+20                                & 1                        \\
                    1485                    & 04261$-$0297$+$0082          & 42.6083                 & $-$2.9667                 & 5,  6                                    & \nodata                           & 0.32                          & 8.21                              & 7.75                              & $-$0.38                         & 3.1                       & 1.0                       & 11.6                         & 5.42E+20                                & 1                        \\
                    1486                    & 04262$-$0252$+$0070          & 42.6167                 & $-$2.5167                 & 5,  6                                    & \nodata                           & 0.32                          & 7.04                              & 6.67                              & $-$0.48                         & 4.4                       & 1.4                       & 9.0                          & 4.93E+20                                & 1                        \\
                    1487                    & 04263$-$0001$+$0160          & 42.6333                 & $-$0.0083                 & 1,  5,  6                                & \nodata                           & 0.32                          & 16.04                             & 15.38                             & $-$0.98                         & 3.2                       & 1.5                       & 8.2                          & 4.71E+20                                & 1                        \\
                    1488                    & 04263$-$0300$+$0084          & 42.6333                 & $-$3.0000                 & 5,  6                                    & \nodata                           & 0.32                          & 8.45                              & 7.96                              & $-$0.44                         & 4.6                       & 1.7                       & 8.6                          & 8.94E+20                                & 1                        \\
                    1489                    & 04264$-$0249$+$0073          & 42.6417                 & $-$2.4917                 & 5,  6                                    & \nodata                           & 0.32                          & 7.30                              & 6.74                              & $-$0.59                         & 4.0                       & 1.8                       & 9.4                          & 7.99E+20                                & 1                        \\
                    1490                    & 04267$-$0262$+$0079          & 42.6667                 & $-$2.6167                 & 5,  6                                    & \nodata                           & 0.32                          & 7.86                              & 7.09                              & $-$0.82                         & 5.9                       & 2.8                       & 13.4                         & 1.37E+21                                & 1                        \\
                    1491                    & 04267$-$0297$+$0084          & 42.6667                 & $-$2.9667                 & 5,  6                                    & \nodata                           & 0.32                          & 8.39                              & 7.84                              & $-$0.59                         & 4.2                       & 1.8                       & 8.6                          & 8.28E+20                                & 1                        \\
                    1492                    & 04268$-$0075$+$0110          & 42.6833                 & $-$0.7500                 & 5,  6                                    & \nodata                           & 0.33                          & 11.02                             & 10.38                             & $-$0.77                         & 4.5                       & 1.4                       & 9.8                          & 5.35E+20                                & 1                        \\
                    1493                    & 04270$-$0252$+$0076          & 42.7000                 & $-$2.5250                 & 5,  6                                    & \nodata                           & 0.32                          & 7.64                              & 6.97                              & $-$0.50                         & 5.3                       & 2.2                       & 13.7                         & 1.48E+21                                & 1                        \\
                    1494                    & 04271$-$0073$+$0109          & 42.7083                 & $-$0.7333                 & 5,  6                                    & \nodata                           & 0.32                          & 10.94                             & 10.24                             & $-$0.76                         & 4.5                       & 1.8                       & 9.9                          & 7.74E+20                                & 1                        \\
                    1495                    & 04272$-$0069$+$0106          & 42.7167                 & $-$0.6917                 & 5,  6                                    & \nodata                           & 0.32                          & 10.62                             & 10.28                             & $-$0.30                         & 4.3                       & 1.2                       & 11.5                         & 6.20E+20                                & 1                        \\
                    1496                    & 04273$-$0063$+$0107          & 42.7333                 & $-$0.6333                 & 5,  6                                    & \nodata                           & 0.31                          & 10.72                             & 10.27                             & $-$0.35                         & 4.2                       & 1.4                       & 9.5                          & 8.37E+20                                & 1                        \\
                    1497                    & 04274$-$0248$+$0074          & 42.7417                 & $-$2.4833                 & 5,  6                                    & \nodata                           & 0.32                          & 7.38                              & 6.75                              & $-$0.53                         & 4.3                       & 2.0                       & 11.6                         & 1.14E+21                                & 1                        \\
                    1498                    & 04276$+$0273$+$0050          & 42.7583                 & 2.7333                  & 6                                        & \nodata                           & 0.31                          & 4.97                              & 4.61                              & $-$0.43                         & 2.9                       & 1.3                       & 6.5                          & 5.67E+20                                & 1                        \\
                    1499                    & 04277$-$0113$+$0396          & 42.7667                 & $-$1.1333                 & 5,  6                                    & \nodata                           & 2.27                          & 39.61                             & 39.13                             & $-$0.38                         & 8.0                       & 1.7                       & 12.5                         & 1.01E+21                                & 1                        \\
                    1500                    & 04278$-$0245$+$0073          & 42.7833                 & $-$2.4500                 & 5,  6                                    & \nodata                           & 0.32                          & 7.29                              & 6.83                              & $-$0.33                         & 3.8                       & 1.0                       & 9.2                          & 6.09E+20                                & 1                        \\
                    1501                    & 04281$-$0243$+$0073          & 42.8083                 & $-$2.4333                 & 5,  6                                    & \nodata                           & 0.32                          & 7.27                              & 6.87                              & $-$0.42                         & 3.9                       & 1.4                       & 10.3                         & 5.84E+20                                & 1                        \\
                    1502                    & 04281$-$0249$+$0074          & 42.8083                 & $-$2.4917                 & 5,  6                                    & \nodata                           & 0.32                          & 7.36                              & 6.71                              & $-$0.52                         & 4.3                       & 1.5                       & 10.0                         & 8.26E+20                                & 1                        \\
                    1503                    & 04282$-$0276$+$0079          & 42.8167                 & $-$2.7583                 & 5,  6                                    & \nodata                           & 0.32                          & 7.92                              & 7.50                              & $-$0.33                         & 5.6                       & 2.6                       & 10.5                         & 1.68E+21                                & 1                        \\
                    1504                    & 04294$-$0311$+$0081          & 42.9417                 & $-$3.1083                 & 5,  6                                    & \nodata                           & 0.32                          & 8.12                              & 7.75                              & $-$0.33                         & 7.1                       & 1.7                       & 13.3                         & 9.62E+20                                & 1                        \\
                    1505                    & 04296$-$0243$+$0073          & 42.9583                 & $-$2.4333                 & 5,  6                                    & \nodata                           & 0.32                          & 7.25                              & 6.82                              & $-$0.35                         & 5.0                       & 1.5                       & 11.3                         & 8.46E+20                                & 1                        \\
                    1506                    & 04301$-$0260$+$0082          & 43.0083                 & $-$2.6000                 & 5,  6                                    & \nodata                           & 0.32                          & 8.22                              & 7.50                              & $-$0.43                         & 5.4                       & 1.8                       & 17.2                         & 1.65E+21                                & 1                        \\
                    1507                    & 04303$-$0263$+$0081          & 43.0333                 & $-$2.6333                 & 5,  6                                    & \nodata                           & 0.32                          & 8.12                              & 7.59                              & $-$0.33                         & 5.1                       & 1.6                       & 11.9                         & 1.20E+21                                & 1                        \\
                    1508                    & 04305$-$0259$+$0080          & 43.0500                 & $-$2.5917                 & 5,  6                                    & \nodata                           & 0.32                          & 7.99                              & 7.38                              & $-$0.34                         & 6.2                       & 1.5                       & 13.7                         & 1.27E+21                                & 1                        \\
                    1509                    & 04307$-$0257$+$0078          & 43.0750                 & $-$2.5750                 & 5,  6                                    & \nodata                           & 0.32                          & 7.78                              & 7.21                              & $-$0.43                         & 6.4                       & 1.4                       & 13.0                         & 8.75E+20                                & 1                        \\
                    1510                    & 04308$-$0268$+$0082          & 43.0833                 & $-$2.6833                 & 5,  6                                    & \nodata                           & 0.32                          & 8.20                              & 7.56                              & $-$0.42                         & 4.4                       & 1.8                       & 10.2                         & 1.31E+21                                & 1                        \\
                    1511                    & 04308$-$0275$+$0080          & 43.0833                 & $-$2.7500                 & 5,  6                                    & \nodata                           & 0.32                          & 8.02                              & 7.49                              & $-$0.41                         & 5.7                       & 2.0                       & 10.3                         & 1.26E+21                                & 1                        \\
                    1512                    & 04308$-$0282$+$0081          & 43.0833                 & $-$2.8167                 & 5,  6                                    & \nodata                           & 0.32                          & 8.13                              & 7.70                              & $-$0.48                         & 4.7                       & 2.0                       & 12.3                         & 9.03E+20                                & 1                        \\
                    1513                    & 04309$-$0272$+$0080          & 43.0917                 & $-$2.7167                 & 5,  6                                    & \nodata                           & 0.32                          & 7.96                              & 7.30                              & $-$0.50                         & 5.1                       & 2.2                       & 11.4                         & 1.42E+21                                & 1                        \\
                    1514                    & 04312$-$0274$+$0081          & 43.1167                 & $-$2.7417                 & 5,  6                                    & \nodata                           & 0.32                          & 8.09                              & 7.26                              & $-$0.54                         & 5.7                       & 2.3                       & 18.0                         & 1.97E+21                                & 1                        \\
                    1515                    & 04312$-$0281$+$0082          & 43.1167                 & $-$2.8083                 & 5,  6                                    & \nodata                           & 0.32                          & 8.20                              & 7.44                              & $-$0.65                         & 6.3                       & 1.4                       & 11.8                         & 7.37E+20                                & 1                        \\
                    1516                    & 04317$-$0213$+$0069          & 43.1667                 & $-$2.1333                 & 5,  6                                    & \nodata                           & 0.32                          & 6.94                              & 6.30                              & $-$0.70                         & 4.7                       & 1.3                       & 10.5                         & 5.25E+20                                & 1                        \\
                    1517                    & 04317$-$0270$+$0084          & 43.1750                 & $-$2.7000                 & 5,  6                                    & \nodata                           & 0.33                          & 8.42                              & 7.65                              & $-$0.66                         & 5.5                       & 1.6                       & 13.0                         & 9.07E+20                                & 1                        \\
                    1518                    & 04319$-$0265$+$0086          & 43.1917                 & $-$2.6500                 & 5,  6                                    & \nodata                           & 0.33                          & 8.62                              & 7.85                              & $-$0.53                         & 5.6                       & 1.7                       & 11.1                         & 1.13E+21                                & 1                        \\
                    1519                    & 04320$-$0268$+$0083          & 43.2000                 & $-$2.6833                 & 5,  6                                    & \nodata                           & 0.33                          & 8.32                              & 7.84                              & $-$0.32                         & 4.8                       & 1.5                       & 10.3                         & 1.04E+21                                & 1                        \\
                    1520                    & 04322$-$0258$+$0086          & 43.2167                 & $-$2.5833                 & 5,  6                                    & \nodata                           & 0.33                          & 8.59                              & 7.66                              & $-$0.69                         & 7.5                       & 2.3                       & 17.1                         & 1.67E+21                                & 1                        \\
                    1521                    & 04322$-$0261$+$0085          & 43.2167                 & $-$2.6083                 & 5,  6                                    & \nodata                           & 0.33                          & 8.54                              & 7.87                              & $-$0.44                         & 5.9                       & 2.0                       & 13.9                         & 1.53E+21                                & 1                        \\
                    1522                    & 04326$-$0264$+$0085          & 43.2583                 & $-$2.6417                 & 5,  6                                    & \nodata                           & 0.33                          & 8.48                              & 7.96                              & $-$0.38                         & 6.4                       & 1.9                       & 13.4                         & 1.28E+21                                & 1                        \\
                    1523                    & 04328$-$0059$+$0090          & 43.2833                 & $-$0.5917                 & 5,  6                                    & \nodata                           & 0.30                          & 9.03                              & 8.77                              & $-$0.24                         & 7.7                       & 2.6                       & 14.2                         & 1.48E+21                                & 1                        \\
                    1524                    & 04332$+$0067$+$0110          & 43.3167                 & 0.6667                  & 5,  6                                    & \nodata                           & 0.31                          & 11.00                             & 10.24                             & $-$0.53                         & 4.0                       & 1.4                       & 8.1                          & 9.52E+20                                & 1                        \\
                    1525                    & 04332$-$0069$+$0094          & 43.3250                 & $-$0.6917                 & 5,  6                                    & \nodata                           & 0.31                          & 9.38                              & 9.20                              & $-$0.24                         & 4.9                       & 2.2                       & 12.4                         & 8.41E+20                                & 1                        \\
                    1526                    & 04337$-$0196$+$0072          & 43.3667                 & $-$1.9583                 & 5,  6                                    & \nodata                           & 0.32                          & 7.17                              & 6.43                              & $-$0.86                         & 4.0                       & 1.2                       & 8.2                          & 4.65E+20                                & 1                        \\
                    1527                    & 04342$-$0195$+$0072          & 43.4167                 & $-$1.9500                 & 5,  6                                    & \nodata                           & 0.32                          & 7.16                              & 6.34                              & $-$1.02                         & 4.1                       & 1.9                       & 12.0                         & 7.16E+20                                & 1                        \\
                    1528                    & 04344$-$0194$+$0071          & 43.4417                 & $-$1.9417                 & 5,  6                                    & \nodata                           & 0.32                          & 7.09                              & 6.34                              & $-$0.94                         & 3.5                       & 1.7                       & 11.0                         & 6.23E+20                                & 1                        \\
                    1529                    & 04357$-$0250$+$0077          & 43.5750                 & $-$2.5000                 & 5,  6                                    & \nodata                           & 0.32                          & 7.72                              & 7.31                              & $-$0.49                         & 5.1                       & 1.9                       & 12.4                         & 7.68E+20                                & 1                        \\
                    1530                    & 04362$-$0262$+$0088          & 43.6167                 & $-$2.6167                 & 5,  6                                    & \nodata                           & 0.33                          & 8.80                              & 8.19                              & $-$0.44                         & 4.6                       & 1.0                       & 13.8                         & 6.68E+20                                & 1                        \\
                    1531                    & 04362$-$0267$+$0087          & 43.6167                 & $-$2.6667                 & 5,  6                                    & \nodata                           & 0.33                          & 8.72                              & 8.09                              & $-$0.53                         & 4.2                       & 1.7                       & 11.8                         & 9.79E+20                                & 1                        \\
                    1532                    & 04372$-$0261$+$0090          & 43.7167                 & $-$2.6083                 & 5,  6                                    & \nodata                           & 0.33                          & 8.96                              & 8.34                              & $-$0.60                         & 5.0                       & 1.3                       & 10.5                         & 5.86E+20                                & 1                        \\
                    1533                    & 04372$-$0267$+$0089          & 43.7250                 & $-$2.6667                 & 5,  6                                    & \nodata                           & 0.33                          & 8.89                              & 8.18                              & $-$0.57                         & 4.6                       & 1.1                       & 9.2                          & 5.97E+20                                & 1                        \\
                    1534                    & 04374$-$0257$+$0089          & 43.7417                 & $-$2.5667                 & 5,  6                                    & \nodata                           & 0.33                          & 8.92                              & 8.32                              & $-$0.63                         & 4.6                       & 1.7                       & 10.1                         & 7.77E+20                                & 1                        \\
                    1535                    & 04374$-$0259$+$0088          & 43.7417                 & $-$2.5917                 & 5,  6                                    & \nodata                           & 0.33                          & 8.82                              & 8.24                              & $-$0.49                         & 5.1                       & 1.7                       & 10.6                         & 9.39E+20                                & 1                        \\
                    1536                    & 04376$-$0031$+$0652          & 43.7583                 & $-$0.3083                 & 5,  6                                    & \nodata                           & 7.40                          & 65.15                             & 64.22                             & $-$0.42                         & 6.6                       & 3.7                       & 11.7                         & 4.50E+21                                & 1                        \\
                    1537                    & 04376$-$0263$+$0089          & 43.7583                 & $-$2.6333                 & 5,  6                                    & \nodata                           & 0.33                          & 8.86                              & 8.31                              & $-$0.51                         & 4.9                       & 1.1                       & 11.5                         & 5.32E+20                                & 1                        \\
                    1538                    & 04376$-$0272$+$0091          & 43.7583                 & $-$2.7250                 & 5,  6                                    & \nodata                           & 0.33                          & 9.09                              & 8.31                              & $-$0.72                         & 4.2                       & 1.2                       & 9.4                          & 5.54E+20                                & 1                        \\
                    1539                    & 04377$-$0248$+$0083          & 43.7667                 & $-$2.4833                 & 5,  6                                    & \nodata                           & 0.33                          & 8.27                              & 7.88                              & $-$0.34                         & 5.1                       & 2.7                       & 11.7                         & 1.61E+21                                & 1                        \\
                    1540                    & 04377$-$0251$+$0086          & 43.7667                 & $-$2.5083                 & 5,  6                                    & \nodata                           & 0.33                          & 8.58                              & 8.06                              & $-$0.43                         & 5.2                       & 1.7                       & 10.4                         & 9.90E+20                                & 1                        \\
                    1541                    & 04377$-$0269$+$0089          & 43.7750                 & $-$2.6917                 & 5,  6                                    & \nodata                           & 0.33                          & 8.90                              & 8.43                              & $-$0.46                         & 4.7                       & 2.3                       & 10.3                         & 1.14E+21                                & 1                        \\
                    1542                    & 04378$-$0264$+$0088          & 43.7833                 & $-$2.6417                 & 5,  6                                    & \nodata                           & 0.33                          & 8.83                              & 8.11                              & $-$0.69                         & 5.5                       & 1.8                       & 14.0                         & 9.30E+20                                & 1                        \\
                    1543                    & 04379$-$0267$+$0089          & 43.7917                 & $-$2.6667                 & 5,  6                                    & \nodata                           & 0.33                          & 8.85                              & 8.17                              & $-$0.69                         & 5.1                       & 2.0                       & 11.4                         & 9.65E+20                                & 1                        \\
                    1544                    & 04379$-$0273$+$0088          & 43.7917                 & $-$2.7333                 & 5,  6                                    & \nodata                           & 0.33                          & 8.78                              & 8.21                              & $-$0.46                         & 5.2                       & 1.6                       & 10.8                         & 9.32E+20                                & 1                        \\
                    1545                    & 04381$-$0254$+$0087          & 43.8083                 & $-$2.5417                 & 5,  6                                    & \nodata                           & 0.33                          & 8.73                              & 8.37                              & $-$0.36                         & 4.0                       & 2.1                       & 9.0                          & 1.05E+21                                & 1                        \\
                    1546                    & 04382$-$0266$+$0090          & 43.8250                 & $-$2.6583                 & 5,  6                                    & \nodata                           & 0.33                          & 9.01                              & 8.10                              & $-$0.85                         & 4.9                       & 2.0                       & 12.4                         & 1.02E+21                                & 1                        \\
                    1547                    & 04384$-$0263$+$0090          & 43.8417                 & $-$2.6333                 & 5,  6                                    & \nodata                           & 0.33                          & 8.97                              & 8.30                              & $-$0.59                         & 5.5                       & 1.5                       & 10.3                         & 7.68E+20                                & 1                        \\
                    1548                    & 04385$-$0272$+$0086          & 43.8500                 & $-$2.7250                 & 5,  6                                    & \nodata                           & 0.33                          & 8.60                              & 8.16                              & $-$0.29                         & 4.1                       & 1.4                       & 8.1                          & 9.93E+20                                & 1                        \\
                    1549                    & 04388$-$0272$+$0087          & 43.8833                 & $-$2.7250                 & 5,  6                                    & \nodata                           & 0.33                          & 8.70                              & 7.97                              & $-$0.59                         & 3.9                       & 1.0                       & 9.8                          & 5.59E+20                                & 1                        \\
                    1550                    & 04397$-$0269$+$0088          & 43.9750                 & $-$2.6917                 & 5,  6                                    & \nodata                           & 0.33                          & 8.76                              & 8.05                              & $-$0.90                         & 4.1                       & 1.4                       & 8.8                          & 4.88E+20                                & 1                        \\
                    1551                    & 04398$-$0009$+$0434          & 43.9833                 & $-$0.0917                 & 1,  5,  6                                & \nodata                           & 2.85                          & 43.43                             & 42.14                             & $-$0.56                         & 7.0                       & 4.9                       & 11.1                         & 7.52E+21                                & 1                        \\
                    1552                    & 04399$-$0282$+$0089          & 43.9917                 & $-$2.8167                 & 5,  6                                    & \nodata                           & 0.33                          & 8.89                              & 8.12                              & $-$0.71                         & 3.5                       & 1.0                       & 10.3                         & 4.93E+20                                & 1                        \\
                    1553                    & 04401$+$0154$+$0075          & 44.0083                 & 1.5417                  & 5,  6                                    & \nodata                           & 0.30                          & 7.51                              & 7.12                              & $-$0.34                         & 3.5                       & 1.2                       & 11.6                         & 6.22E+20                                & 1                        \\
                    1554                    & 04402$-$0248$+$0082          & 44.0167                 & $-$2.4833                 & 5,  6                                    & \nodata                           & 0.33                          & 8.23                              & 7.81                              & $-$0.39                         & 5.2                       & 1.2                       & 9.4                          & 5.69E+20                                & 1                        \\
                    1555                    & 04407$-$0279$+$0084          & 44.0750                 & $-$2.7917                 & 5,  6                                    & \nodata                           & 0.33                          & 8.44                              & 8.11                              & $-$0.34                         & 4.3                       & 1.0                       & 10.3                         & 4.18E+20                                & 1                        \\
                    1556                    & 04424$-$0239$+$0278          & 44.2417                 & $-$2.3917                 & 5,  6                                    & \nodata                           & 0.13                          & 27.76                             & 27.38                             & $-$0.34                         & 5.2                       & 2.9                       & 12.4                         & 1.64E+21                                & 1                        \\
                    1557                    & 04427$-$0097$+$0256          & 44.2667                 & $-$0.9667                 & 3,  5,  6                                & \nodata                           & 1.46                          & 25.60                             & 25.22                             & $-$0.48                         & 4.0                       & 1.3                       & 8.4                          & 4.62E+20                                & 1                        \\
                    1558                    & 04427$-$0082$+$0621          & 44.2750                 & $-$0.8167                 & 5,  6                                    & \nodata                           & 3.60                          & 62.08                             & 61.34                             & $-$0.37                         & 6.3                       & 3.4                       & 15.8                         & 3.89E+21                                & 1                        \\
                    1559                    & 04429$-$0280$+$0082          & 44.2917                 & $-$2.8000                 & 5,  6                                    & \nodata                           & 0.33                          & 8.23                              & 7.89                              & $-$0.43                         & 4.8                       & 1.1                       & 13.1                         & 4.30E+20                                & 1                        \\
                    1560                    & 04430$-$0277$+$0084          & 44.3000                 & $-$2.7667                 & 5,  6                                    & \nodata                           & 0.33                          & 8.45                              & 7.83                              & $-$0.72                         & 4.3                       & 1.0                       & 9.7                          & 3.59E+20                                & 1                        \\
                    1561                    & 04431$+$0003$+$0564          & 44.3083                 & 0.0333                  & 3,  5,  6                                & \nodata                           & 3.71                          & 56.35                             & 54.21                             & $-$0.52                         & 9.1                       & 6.3                       & 14.9                         & 1.78E+22                                & 1                        \\
                    1562                    & 04431$-$0097$+$0260          & 44.3083                 & $-$0.9667                 & 3,  5,  6                                & \nodata                           & 1.48                          & 26.03                             & 25.60                             & $-$0.50                         & 4.7                       & 2.0                       & 9.1                          & 8.11E+20                                & 1                        \\
                    1563                    & 04432$-$0079$+$0620          & 44.3167                 & $-$0.7917                 & 5,  6                                    & \nodata                           & 4.45                          & 62.03                             & 61.11                             & $-$0.42                         & 6.5                       & 4.3                       & 12.9                         & 5.51E+21                                & 1                        \\
                    1564                    & 04432$-$0257$+$0275          & 44.3250                 & $-$2.5750                 & 5,  6                                    & \nodata                           & 0.13                          & 27.48                             & 26.92                             & $-$0.42                         & 5.0                       & 2.1                       & 11.8                         & 1.39E+21                                & 1                        \\
                    1565                    & 04442$-$0107$+$0628          & 44.4250                 & $-$1.0667                 & 5,  6                                    & \nodata                           & 5.56                          & 62.81                             & 62.36                             & $-$0.30                         & 4.4                       & 1.9                       & 9.0                          & 1.40E+21                                & 1                        \\
                    1566                    & 04442$-$0267$+$0272          & 44.4250                 & $-$2.6750                 & 5,  6                                    & \nodata                           & 0.13                          & 27.19                             & 26.70                             & $-$0.33                         & 6.0                       & 3.1                       & 10.9                         & 2.38E+21                                & 1                        \\
                    1567                    & 04442$-$0276$+$0086          & 44.4250                 & $-$2.7583                 & 5,  6                                    & \nodata                           & 0.33                          & 8.59                              & 7.99                              & $-$0.48                         & 4.4                       & 1.2                       & 10.1                         & 6.57E+20                                & 1                        \\
                    1568                    & 04445$-$0276$+$0080          & 44.4500                 & $-$2.7583                 & 5,  6                                    & \nodata                           & 0.33                          & 8.04                              & 7.66                              & $-$0.25                         & 5.2                       & 1.2                       & 10.5                         & 7.84E+20                                & 1                        \\
                    1569                    & 04446$-$0272$+$0081          & 44.4583                 & $-$2.7250                 & 5,  6                                    & \nodata                           & 0.33                          & 8.11                              & 7.54                              & $-$0.40                         & 5.4                       & 1.0                       & 14.8                         & 7.32E+20                                & 1                        \\
                    1570                    & 04451$-$0272$+$0082          & 44.5083                 & $-$2.7250                 & 5,  6                                    & \nodata                           & 0.33                          & 8.22                              & 7.58                              & $-$0.41                         & 5.2                       & 1.1                       & 12.7                         & 7.71E+20                                & 1                        \\
                    1571                    & 04452$-$0265$+$0080          & 44.5250                 & $-$2.6500                 & 5,  6                                    & \nodata                           & 0.33                          & 7.97                              & 7.36                              & $-$0.36                         & 6.0                       & 0.7                       & 20.2                         & 7.09E+20                                & 1                        \\
                    1572                    & 04456$-$0270$+$0080          & 44.5583                 & $-$2.7000                 & 5,  6                                    & \nodata                           & 0.33                          & 8.00                              & 7.30                              & $-$0.55                         & 5.4                       & 1.2                       & 14.4                         & 7.72E+20                                & 1                        \\
                    1573                    & 04458$-$0272$+$0081          & 44.5833                 & $-$2.7250                 & 5,  6                                    & \nodata                           & 0.33                          & 8.12                              & 7.33                              & $-$0.45                         & 5.5                       & 1.0                       & 11.8                         & 7.99E+20                                & 1                        \\
                    1574                    & 04463$-$0268$+$0077          & 44.6333                 & $-$2.6833                 & 5,  6                                    & \nodata                           & 0.33                          & 7.75                              & 7.11                              & $-$0.49                         & 5.8                       & 1.0                       & 12.8                         & 6.26E+20                                & 1                        \\
                    1575                    & 04466$-$0272$+$0078          & 44.6583                 & $-$2.7250                 & 5,  6                                    & \nodata                           & 0.33                          & 7.78                              & 6.90                              & $-$0.53                         & 6.2                       & 1.1                       & 12.9                         & 8.18E+20                                & 1                        \\
                    1576                    & 04467$-$0003$+$0406          & 44.6667                 & $-$0.0333                 & 5,  6                                    & \nodata                           & 2.83                          & 40.61                             & 40.08                             & $-$0.40                         & 4.3                       & 1.6                       & 11.6                         & 9.65E+20                                & 1                        \\
                    1577                    & 04467$-$0270$+$0078          & 44.6667                 & $-$2.7000                 & 5,  6                                    & \nodata                           & 0.33                          & 7.81                              & 6.91                              & $-$0.53                         & 6.0                       & 1.2                       & 15.0                         & 1.02E+21                                & 1                        \\
                    1578                    & 04468$-$0006$+$0508          & 44.6833                 & $-$0.0583                 & 5,  6                                    & \nodata                           & 3.42                          & 50.80                             & 50.02                             & $-$0.32                         & 6.0                       & 1.9                       & 9.8                          & 2.18E+21                                & 1                        \\
                    1579                    & 04474$-$0268$+$0074          & 44.7417                 & $-$2.6833                 & 5,  6                                    & \nodata                           & 0.33                          & 7.41                              & 7.09                              & $-$0.31                         & 5.1                       & 1.1                       & 14.6                         & 5.75E+20                                & 1                        \\
                    1580                    & 04476$-$0266$+$0076          & 44.7583                 & $-$2.6583                 & 5,  6                                    & \nodata                           & 0.33                          & 7.59                              & 7.16                              & $-$0.33                         & 5.2                       & 1.4                       & 12.5                         & 8.23E+20                                & 1                        \\
                    1581                    & 04482$-$0270$+$0075          & 44.8167                 & $-$2.7000                 & 5,  6                                    & \nodata                           & 0.33                          & 7.45                              & 7.04                              & $-$0.48                         & 5.4                       & 1.1                       & 11.7                         & 4.28E+20                                & 1                        \\
                    1582                    & 04482$-$0272$+$0076          & 44.8250                 & $-$2.7250                 & 6                                        & \nodata                           & 0.33                          & 7.65                              & 7.24                              & $-$0.26                         & 5.7                       & 1.2                       & 13.2                         & 9.04E+20                                & 1                        \\
                    1583                    & 04494$-$0247$+$0073          & 44.9417                 & $-$2.4667                 & 5,  6                                    & \nodata                           & 0.33                          & 7.34                              & 6.95                              & $-$0.43                         & 5.2                       & 1.3                       & 11.8                         & 5.18E+20                                & 1                        \\
                    1584                    & 04497$-$0283$+$0079          & 44.9750                 & $-$2.8333                 & 6                                        & \nodata                           & 0.33                          & 7.86                              & 7.32                              & $-$0.36                         & 6.1                       & 1.1                       & 18.0                         & 8.67E+20                                & 1                        \\
                    1585                    & 04502$-$0085$+$0428          & 45.0167                 & $-$0.8500                 & 5,  6                                    & \nodata                           & 2.65                          & 42.75                             & 42.17                             & $-$0.51                         & 3.8                       & 1.0                       & 10.8                         & 5.20E+20                                & 1                        \\
                    1586                    & 04506$-$0085$+$0433          & 45.0583                 & $-$0.8500                 & 5,  6                                    & \nodata                           & 2.68                          & 43.28                             & 42.83                             & $-$0.39                         & 3.3                       & 1.5                       & 7.0                          & 9.02E+20                                & 1                        \\
                    1587                    & 04513$-$0280$+$0075          & 45.1333                 & $-$2.8000                 & 6                                        & \nodata                           & 0.33                          & 7.51                              & 7.22                              & $-$0.40                         & 5.5                       & 1.2                       & 14.5                         & 4.35E+20                                & 1                        \\
                    1588                    & 04525$+$0021$+$0259          & 45.2500                 & 0.2083                  & 3,  5,  6                                & \nodata                           & 0.15                          & 25.93                             & 25.34                             & $-$0.48                         & 4.1                       & 2.1                       & 7.9                          & 1.32E+21                                & 1                        \\
                    1589                    & 04538$-$0075$+$0605          & 45.3833                 & $-$0.7500                 & 5,  6                                    & \nodata                           & 3.92                          & 60.51                             & 59.54                             & $-$0.26                         & 13.6                      & 5.4                       & 17.9                         & 1.31E+22                                & 1                        \\
                    1590                    & 04545$+$0005$+$0591          & 45.4500                 & 0.0500                  & 1,  3,  5,  6                            & \nodata                           & 4.94                          & 59.07                             & 57.35                             & $-$0.29                         & 17.0                      & 8.0                       & 23.1                         & 3.75E+22                                & 1                        \\
                    1591                    & 04549$+$0017$+$0574          & 45.4917                 & 0.1750                  & 3,  5,  6                                & \nodata                           & 3.96                          & 57.35                             & 54.95                             & $-$0.39                         & 9.1                       & 3.4                       & 16.2                         & 1.20E+22                                & 1                        \\
                    1592                    & 04553$-$0221$+$0074          & 45.5333                 & $-$2.2083                 & 6                                        & \nodata                           & 0.33                          & 7.39                              & 6.89                              & $-$0.49                         & 5.6                       & 1.2                       & 12.0                         & 5.54E+20                                & 1                        \\
                    1593                    & 04554$+$0007$+$0587          & 45.5417                 & 0.0750                  & 3,  5,  6                                & \nodata                           & 4.59                          & 58.74                             & 57.08                             & $-$0.34                         & 11.3                      & 5.1                       & 18.1                         & 1.58E+22                                & 1                        \\
                    1594                    & 04562$-$0199$+$0076          & 45.6167                 & $-$1.9917                 & 6                                        & \nodata                           & 0.33                          & 7.63                              & 7.15                              & $-$0.46                         & 5.6                       & 1.2                       & 12.2                         & 5.54E+20                                & 1                        \\
                    1595                    & 04566$-$0197$+$0078          & 45.6583                 & $-$1.9750                 & 6                                        & \nodata                           & 0.33                          & 7.83                              & 7.12                              & $-$0.60                         & 6.3                       & 1.6                       & 12.1                         & 8.67E+20                                & 1                        \\
                    1596                    & 04576$-$0188$+$0078          & 45.7583                 & $-$1.8833                 & 6                                        & \nodata                           & 0.33                          & 7.85                              & 7.36                              & $-$0.49                         & 5.7                       & 2.0                       & 11.0                         & 9.56E+20                                & 1                        \\
                    1597                    & 04587$-$0141$+$0235          & 45.8667                 & $-$1.4083                 & 6                                        & \nodata                           & 0.05                          & 23.49                             & 23.00                             & $-$0.47                         & 4.6                       & 1.4                       & 11.0                         & 6.64E+20                                & 1                        \\
                    1598                    & 04588$-$0163$+$0075          & 45.8833                 & $-$1.6333                 & 6                                        & \nodata                           & 0.33                          & 7.46                              & 6.92                              & $-$0.38                         & 7.2                       & 2.6                       & 12.8                         & 1.84E+21                                & 1                        \\
                    1599                    & 04590$-$0176$+$0074          & 45.9000                 & $-$1.7583                 & 6                                        & \nodata                           & 0.33                          & 7.36                              & 6.74                              & $-$0.71                         & 6.0                       & 2.1                       & 14.6                         & 9.50E+20                                & 1                        \\
                    1600                    & 04596$-$0177$+$0075          & 45.9583                 & $-$1.7750                 & 6                                        & \nodata                           & 0.33                          & 7.51                              & 6.79                              & $-$0.77                         & 6.4                       & 1.5                       & 12.2                         & 6.70E+20                                & 1                        \\
                    1601                    & 04598$-$0177$+$0073          & 45.9833                 & $-$1.7750                 & 6                                        & \nodata                           & 0.33                          & 7.26                              & 6.72                              & $-$0.42                         & 7.0                       & 1.4                       & 13.8                         & 8.60E+20                                & 1                        \\
                    1602                    & 04600$-$0167$+$0078          & 46.0000                 & $-$1.6667                 & 6                                        & \nodata                           & 0.33                          & 7.81                              & 7.13                              & $-$0.42                         & 7.3                       & 2.7                       & 14.9                         & 2.31E+21                                & 1                        \\
                    1603                    & 04600$-$0169$+$0078          & 46.0000                 & $-$1.6917                 & 6                                        & \nodata                           & 0.33                          & 7.78                              & 6.99                              & $-$0.47                         & 7.1                       & 2.5                       & 16.1                         & 2.26E+21                                & 1                        \\
                    1604                    & 04601$-$0172$+$0076          & 46.0083                 & $-$1.7250                 & 6                                        & \nodata                           & 0.33                          & 7.60                              & 6.87                              & $-$0.42                         & 7.3                       & 2.2                       & 16.5                         & 2.12E+21                                & 1                        \\
                    1605                    & 04602$-$0019$+$0255          & 46.0250                 & $-$0.1917                 & 6                                        & \nodata                           & 1.41                          & 25.52                             & 24.77                             & $-$0.49                         & 4.4                       & 2.2                       & 10.9                         & 1.67E+21                                & 1                        \\
                    1606                    & 04604$-$0155$+$0075          & 46.0417                 & $-$1.5500                 & 6                                        & \nodata                           & 0.33                          & 7.49                              & 6.63                              & $-$0.57                         & 7.7                       & 3.9                       & 14.0                         & 3.31E+21                                & 1                        \\
                    1607                    & 04608$-$0075$+$0259          & 46.0833                 & $-$0.7500                 & 6                                        & \nodata                           & 1.45                          & 25.91                             & 25.44                             & $-$0.44                         & 4.6                       & 1.7                       & 9.4                          & 8.49E+20                                & 1                        \\
                    1608                    & 04609$-$0016$+$0254          & 46.0917                 & $-$0.1583                 & 6                                        & \nodata                           & 1.40                          & 25.43                             & 24.97                             & $-$0.50                         & 5.6                       & 2.6                       & 8.6                          & 1.30E+21                                & 1                        \\
                    1609                    & 04609$-$0077$+$0466          & 46.0917                 & $-$0.7750                 & 6                                        & \nodata                           & 2.96                          & 46.60                             & 45.95                             & $-$0.54                         & 4.3                       & 1.5                       & 11.0                         & 8.23E+20                                & 1                        \\
                    1610                    & 04609$-$0155$+$0076          & 46.0917                 & $-$1.5500                 & 6                                        & \nodata                           & 0.33                          & 7.59                              & 6.64                              & $-$0.67                         & 7.6                       & 3.1                       & 17.2                         & 2.54E+21                                & 1                        \\
                    1611                    & 04611$-$0158$+$0077          & 46.1083                 & $-$1.5833                 & 6                                        & \nodata                           & 0.33                          & 7.72                              & 6.68                              & $-$0.70                         & 7.5                       & 2.7                       & 15.3                         & 2.19E+21                                & 1                        \\
                    1612                    & 04612$-$0152$+$0077          & 46.1250                 & $-$1.5167                 & 6                                        & \nodata                           & 0.33                          & 7.70                              & 6.86                              & $-$0.55                         & 7.5                       & 4.0                       & 14.9                         & 3.55E+21                                & 1                        \\
                    1613                    & 04613$-$0147$+$0079          & 46.1333                 & $-$1.4667                 & 1,  5,  6                                & \nodata                           & 0.33                          & 7.95                              & 6.77                              & $-$0.75                         & 7.5                       & 3.4                       & 12.6                         & 2.90E+21                                & 1                        \\
                    1614                    & 04615$-$0142$+$0076          & 46.1500                 & $-$1.4250                 & 5,  6                                    & \nodata                           & 0.33                          & 7.62                              & 6.95                              & $-$0.41                         & 7.3                       & 4.0                       & 16.0                         & 3.91E+21                                & 1                        \\
                    1615                    & 04615$-$0155$+$0076          & 46.1500                 & $-$1.5500                 & 6                                        & \nodata                           & 0.33                          & 7.61                              & 6.82                              & $-$0.46                         & 6.9                       & 3.1                       & 15.1                         & 2.89E+21                                & 1                        \\
                    1616                    & 04617$-$0150$+$0077          & 46.1750                 & $-$1.5000                 & 5,  6                                    & \nodata                           & 0.33                          & 7.74                              & 6.87                              & $-$0.49                         & 8.0                       & 2.9                       & 16.8                         & 2.90E+21                                & 1                        \\
                    1617                    & 04618$-$0136$+$0077          & 46.1833                 & $-$1.3583                 & 5,  6                                    & \nodata                           & 0.33                          & 7.72                              & 6.79                              & $-$0.63                         & 6.6                       & 3.9                       & 14.1                         & 3.27E+21                                & 1                        \\
                    1618                    & 04618$-$0142$+$0077          & 46.1833                 & $-$1.4250                 & 5,  6                                    & \nodata                           & 0.33                          & 7.73                              & 6.75                              & $-$0.62                         & 6.8                       & 3.4                       & 16.5                         & 3.03E+21                                & 1                        \\
                    1619                    & 04620$-$0147$+$0078          & 46.2000                 & $-$1.4750                 & 5,  6                                    & \nodata                           & 0.33                          & 7.82                              & 6.99                              & $-$0.54                         & 7.7                       & 3.1                       & 13.1                         & 2.61E+21                                & 1                        \\
                    1620                    & 04621$-$0134$+$0075          & 46.2083                 & $-$1.3417                 & 5,  6                                    & \nodata                           & 0.33                          & 7.53                              & 6.82                              & $-$0.42                         & 7.4                       & 4.0                       & 13.3                         & 3.87E+21                                & 1                        \\
                    1621                    & 04622$-$0137$+$0076          & 46.2250                 & $-$1.3750                 & 5,  6                                    & \nodata                           & 0.33                          & 7.61                              & 6.71                              & $-$0.50                         & 6.5                       & 2.9                       & 15.6                         & 2.82E+21                                & 1                        \\
                    1622                    & 04623$-$0147$+$0080          & 46.2333                 & $-$1.4667                 & 5,  6                                    & \nodata                           & 0.33                          & 7.97                              & 6.97                              & $-$0.62                         & 7.4                       & 2.2                       & 15.2                         & 1.87E+21                                & 1                        \\
                    1623                    & 04624$-$0103$+$0066          & 46.2417                 & $-$1.0333                 & 6                                        & \nodata                           & 0.33                          & 6.56                              & 5.59                              & $-$0.62                         & 4.7                       & 1.6                       & 12.2                         & 1.17E+21                                & 1                        \\
                    1624                    & 04624$-$0128$+$0070          & 46.2417                 & $-$1.2833                 & 5,  6                                    & \nodata                           & 0.33                          & 6.98                              & 6.39                              & $-$0.49                         & 7.8                       & 4.7                       & 12.8                         & 3.36E+21                                & 1                        \\
                    1625                    & 04624$-$0131$+$0072          & 46.2417                 & $-$1.3083                 & 5,  6                                    & \nodata                           & 0.33                          & 7.16                              & 6.75                              & $-$0.28                         & 8.0                       & 4.9                       & 15.0                         & 4.37E+21                                & 1                        \\
                    1626                    & 04626$-$0136$+$0071          & 46.2583                 & $-$1.3583                 & 5,  6                                    & \nodata                           & 0.33                          & 7.14                              & 6.64                              & $-$0.32                         & 7.2                       & 4.1                       & 13.9                         & 3.65E+21                                & 1                        \\
                    1627                    & 04627$+$0033$+$0244          & 46.2667                 & 0.3333                  & 6                                        & \nodata                           & 0.04                          & 24.44                             & 23.82                             & $-$0.64                         & 3.7                       & 1.3                       & 7.8                          & 6.00E+20                                & 1                        \\
                    1628                    & 04627$-$0067$+$0250          & 46.2667                 & $-$0.6750                 & 6                                        & \nodata                           & 1.31                          & 24.99                             & 24.14                             & $-$0.70                         & 5.8                       & 2.9                       & 9.8                          & 1.85E+21                                & 1                        \\
                    1629                    & 04627$-$0101$+$0068          & 46.2667                 & $-$1.0083                 & 6                                        & \nodata                           & 0.33                          & 6.84                              & 5.48                              & $-$0.74                         & 5.8                       & 1.3                       & 12.0                         & 1.09E+21                                & 1                        \\
                    1630                    & 04628$-$0132$+$0073          & 46.2833                 & $-$1.3250                 & 5,  6                                    & \nodata                           & 0.33                          & 7.29                              & 6.61                              & $-$0.45                         & 10.8                      & 4.3                       & 19.7                         & 4.06E+21                                & 1                        \\
                    1631                    & 04630$-$0024$+$0541          & 46.3000                 & $-$0.2417                 & 6                                        & \nodata                           & 3.92                          & 54.05                             & 53.30                             & $-$0.35                         & 4.2                       & 1.9                       & 8.5                          & 1.54E+22                                & 2                        \\
                    1632                    & 04630$-$0135$+$0075          & 46.3000                 & $-$1.3500                 & 5,  6                                    & \nodata                           & 0.33                          & 7.55                              & 6.77                              & $-$0.47                         & 6.3                       & 2.9                       & 15.1                         & 2.61E+21                                & 1                        \\
                    1633                    & 04637$-$0114$+$0069          & 46.3750                 & $-$1.1417                 & 5,  6                                    & \nodata                           & 0.33                          & 6.86                              & 5.89                              & $-$0.61                         & 7.9                       & 4.0                       & 12.6                         & 3.67E+21                                & 1                        \\
                    1634                    & 04637$-$0117$+$0071          & 46.3750                 & $-$1.1667                 & 5,  6                                    & \nodata                           & 0.33                          & 7.07                              & 5.73                              & $-$0.80                         & 7.7                       & 3.1                       & 12.6                         & 2.74E+21                                & 1                        \\
                    1635                    & 04642$-$0113$+$0068          & 46.4167                 & $-$1.1333                 & 5,  6                                    & \nodata                           & 0.33                          & 6.76                              & 6.07                              & $-$0.42                         & 7.1                       & 4.7                       & 13.3                         & 4.59E+21                                & 1                        \\
                    1636                    & 04644$-$0068$+$0253          & 46.4417                 & $-$0.6833                 & 6                                        & \nodata                           & 1.34                          & 25.29                             & 24.42                             & $-$0.45                         & 5.3                       & 2.3                       & 11.6                         & 2.21E+21                                & 1                        \\
                    1637                    & 04648$-$0143$+$0070          & 46.4833                 & $-$1.4333                 & 5,  6                                    & \nodata                           & 0.33                          & 7.04                              & 6.66                              & $-$0.27                         & 6.3                       & 4.6                       & 13.0                         & 3.81E+21                                & 1                        \\
                    1638                    & 04649$-$0118$+$0074          & 46.4917                 & $-$1.1833                 & 5,  6                                    & \nodata                           & 0.33                          & 7.42                              & 6.86                              & $-$0.28                         & 7.4                       & 2.4                       & 16.4                         & 2.65E+21                                & 1                        \\
                    1639                    & 04652$-$0067$+$0249          & 46.5250                 & $-$0.6750                 & 6                                        & \nodata                           & 1.44                          & 24.94                             & 24.22                             & $-$0.40                         & 6.1                       & 2.7                       & 15.2                         & 2.64E+21                                & 1                        \\
                    1640                    & 04652$-$0182$+$0071          & 46.5250                 & $-$1.8167                 & 6                                        & \nodata                           & 0.20                          & 7.15                              & 6.53                              & $-$0.49                         & 5.6                       & 1.5                       & 11.9                         & 8.77E+20                                & 1                        \\
                    1641                    & 04655$-$0109$+$0073          & 46.5500                 & $-$1.0917                 & 6                                        & \nodata                           & 0.22                          & 7.34                              & 6.57                              & $-$0.57                         & 7.2                       & 3.0                       & 11.3                         & 2.17E+21                                & 1                        \\
                    1642                    & 04655$-$0147$+$0071          & 46.5500                 & $-$1.4667                 & 6                                        & \nodata                           & 0.21                          & 7.13                              & 6.67                              & $-$0.33                         & 6.3                       & 2.0                       & 14.1                         & 1.40E+21                                & 1                        \\
                    1643                    & 04655$-$0155$+$0227          & 46.5500                 & $-$1.5500                 & 6                                        & \nodata                           & 1.26                          & 22.75                             & 22.43                             & $-$0.37                         & 4.6                       & 0.9                       & 10.4                         & 3.50E+20                                & 1                        \\
                    1644                    & 04657$-$0132$+$0070          & 46.5667                 & $-$1.3250                 & 5,  6                                    & \nodata                           & 0.20                          & 7.00                              & 6.33                              & $-$0.43                         & 6.6                       & 1.9                       & 11.8                         & 1.37E+21                                & 1                        \\
                    1645                    & 04657$-$0190$+$0073          & 46.5750                 & $-$1.9000                 & 6                                        & \nodata                           & 0.22                          & 7.32                              & 6.42                              & $-$0.81                         & 5.4                       & 1.3                       & 12.7                         & 6.71E+20                                & 1                        \\
                    1646                    & 04659$+$0036$+$0051          & 46.5917                 & 0.3583                  & 6                                        & \nodata                           & 0.49                          & 5.07                              & 4.21                              & $-$0.46                         & 5.6                       & 2.8                       & 9.9                          & 2.65E+21                                & 1                        \\
                    1647                    & 04660$-$0166$+$0068          & 46.6000                 & $-$1.6583                 & 6                                        & \nodata                           & 0.18                          & 6.79                              & 6.36                              & $-$0.49                         & 5.3                       & 2.6                       & 9.9                          & 1.17E+21                                & 1                        \\
                    1648                    & 04662$-$0148$+$0073          & 46.6167                 & $-$1.4833                 & 6                                        & \nodata                           & 0.22                          & 7.30                              & 6.79                              & $-$0.35                         & 6.2                       & 1.3                       & 11.2                         & 8.98E+20                                & 1                        \\
                    1649                    & 04662$+$0031$+$0068          & 46.6250                 & 0.3083                  & 6                                        & \nodata                           & 0.46                          & 6.79                              & 5.59                              & $-$0.39                         & 6.4                       & 3.5                       & 13.4                         & 5.79E+21                                & 1                        \\
                    1650                    & 04662$-$0143$+$0072          & 46.6250                 & $-$1.4333                 & 6                                        & \nodata                           & 0.22                          & 7.22                              & 6.49                              & $-$0.54                         & 6.4                       & 1.5                       & 14.9                         & 1.07E+21                                & 1                        \\
                    1651                    & 04667$-$0149$+$0071          & 46.6667                 & $-$1.4917                 & 6                                        & \nodata                           & 0.19                          & 7.12                              & 6.58                              & $-$0.35                         & 2.1                       & 2.0                       & 5.4                          & 3.33E+21                                & 1                        \\
                    1652                    & 04672$-$0168$+$0222          & 46.7250                 & $-$1.6833                 & 6                                        & \nodata                           & 1.24                          & 22.21                             & 21.59                             & $-$0.63                         & 3.0                       & 1.6                       & 9.5                          & 7.23E+20                                & 1                        \\
                    1653                    & 04673$-$0137$+$0072          & 46.7333                 & $-$1.3750                 & 6                                        & \nodata                           & 0.21                          & 7.17                              & 6.68                              & $-$0.26                         & 6.2                       & 1.3                       & 12.6                         & 1.17E+21                                & 1                        \\
                    1654                    & 04675$-$0134$+$0070          & 46.7500                 & $-$1.3417                 & 6                                        & \nodata                           & 0.21                          & 7.03                              & 6.59                              & $-$0.35                         & 6.0                       & 1.6                       & 9.8                          & 9.35E+20                                & 1                        \\
                    1655                    & 04675$-$0152$+$0072          & 46.7500                 & $-$1.5167                 & 5,  6                                    & \nodata                           & 0.21                          & 7.19                              & 6.26                              & $-$0.60                         & 5.4                       & 2.1                       & 11.6                         & 1.55E+21                                & 1                        \\
                    1656                    & 04677$-$0067$+$0546          & 46.7667                 & $-$0.6750                 & 6                                        & \nodata                           & 3.64                          & 54.60                             & 54.00                             & $-$0.72                         & 3.7                       & 1.0                       & 7.1                          & 3.96E+20                                & 1                        \\
                    1657                    & 04682$-$0127$+$0068          & 46.8167                 & $-$1.2667                 & 6                                        & \nodata                           & 0.16                          & 6.84                              & 6.26                              & $-$0.45                         & 6.2                       & 1.4                       & 10.6                         & 8.37E+20                                & 1                        \\
                    1658                    & 04683$-$0085$+$0078          & 46.8333                 & $-$0.8500                 & 6                                        & \nodata                           & 0.26                          & 7.83                              & 7.00                              & $-$0.79                         & 5.3                       & 1.1                       & 11.1                         & 5.33E+20                                & 1                        \\
                    1659                    & 04683$-$0113$+$0069          & 46.8333                 & $-$1.1333                 & 5,  6                                    & \nodata                           & 0.19                          & 6.90                              & 6.53                              & $-$0.48                         & 6.3                       & 2.4                       & 12.8                         & 9.60E+20                                & 1                        \\
                    1660                    & 04684$-$0161$+$0068          & 46.8417                 & $-$1.6083                 & 6                                        & \nodata                           & 0.16                          & 6.79                              & 6.39                              & $-$0.47                         & 5.4                       & 2.2                       & 12.5                         & 9.15E+20                                & 1                        \\
                    1661                    & 04687$-$0111$+$0075          & 46.8667                 & $-$1.1083                 & 5,  6                                    & \nodata                           & 0.22                          & 7.45                              & 6.70                              & $-$0.49                         & 6.3                       & 2.4                       & 15.6                         & 1.99E+21                                & 1                        \\
                    1662                    & 04687$-$0106$+$0073          & 46.8750                 & $-$1.0583                 & 5,  6                                    & \nodata                           & 0.22                          & 7.29                              & 6.60                              & $-$0.50                         & 7.1                       & 2.2                       & 14.2                         & 1.58E+21                                & 1                        \\
                    1663                    & 04691$-$0091$+$0071          & 46.9083                 & $-$0.9083                 & 6                                        & \nodata                           & 0.21                          & 7.08                              & 6.71                              & $-$0.35                         & 5.9                       & 1.8                       & 12.6                         & 9.08E+20                                & 1                        \\
                    1664                    & 04692$-$0127$+$0074          & 46.9250                 & $-$1.2667                 & 5,  6                                    & \nodata                           & 0.22                          & 7.37                              & 6.62                              & $-$0.49                         & 6.4                       & 2.3                       & 11.5                         & 1.75E+21                                & 1                        \\
                    1665                    & 04695$-$0133$+$0073          & 46.9500                 & $-$1.3333                 & 6                                        & \nodata                           & 0.23                          & 7.33                              & 7.07                              & $-$0.26                         & 5.9                       & 2.3                       & 13.1                         & 1.14E+21                                & 1                        \\
                    1666                    & 04695$-$0153$+$0072          & 46.9500                 & $-$1.5333                 & 6                                        & \nodata                           & 0.21                          & 7.21                              & 6.49                              & $-$0.41                         & 6.1                       & 1.4                       & 12.4                         & 1.13E+21                                & 1                        \\
                    1667                    & 04697$-$0147$+$0070          & 46.9667                 & $-$1.4750                 & 6                                        & \nodata                           & 0.18                          & 6.98                              & 6.46                              & $-$0.36                         & 6.1                       & 2.2                       & 15.4                         & 1.67E+21                                & 1                        \\
                    1668                    & 04697$-$0089$+$0070          & 46.9750                 & $-$0.8917                 & 6                                        & \nodata                           & 0.21                          & 7.04                              & 6.10                              & $-$0.91                         & 5.0                       & 2.1                       & 11.4                         & 1.02E+21                                & 1                        \\
                    1669                    & 04697$-$0092$+$0071          & 46.9750                 & $-$0.9250                 & 5,  6                                    & \nodata                           & 0.19                          & 7.08                              & 6.56                              & $-$0.41                         & 4.6                       & 2.0                       & 9.4                          & 1.23E+21                                & 1                        \\
                    1670                    & 04699$-$0141$+$0070          & 46.9917                 & $-$1.4083                 & 6                                        & \nodata                           & 0.19                          & 7.04                              & 6.66                              & $-$0.43                         & 6.2                       & 3.3                       & 12.4                         & 1.52E+21                                & 1                        \\
                    1671                    & 04701$-$0090$+$0071          & 47.0083                 & $-$0.9000                 & 6                                        & \nodata                           & 0.20                          & 7.06                              & 6.76                              & $-$0.29                         & 4.4                       & 2.3                       & 8.3                          & 1.24E+21                                & 1                        \\
                    1672                    & 04707$-$0207$+$0746          & 47.0667                 & $-$2.0667                 & 6                                        & \nodata                           & 5.55                          & 74.62                             & 74.18                             & $-$0.31                         & 6.6                       & 2.5                       & 13.3                         & 1.80E+21                                & 1                        \\
                    1673                    & 04707$-$0121$+$0074          & 47.0750                 & $-$1.2083                 & 5,  6                                    & \nodata                           & 0.23                          & 7.37                              & 6.63                              & $-$0.45                         & 7.5                       & 2.3                       & 13.7                         & 1.92E+21                                & 1                        \\
                    1674                    & 04709$-$0097$+$0076          & 47.0917                 & $-$0.9750                 & 6                                        & \nodata                           & 0.23                          & 7.56                              & 6.97                              & $-$0.63                         & 6.5                       & 1.9                       & 12.3                         & 8.57E+20                                & 1                        \\
                    1675                    & 04713$-$0108$+$0073          & 47.1333                 & $-$1.0833                 & 6                                        & \nodata                           & 0.22                          & 7.28                              & 6.60                              & $-$0.38                         & 7.1                       & 2.3                       & 15.0                         & 2.22E+21                                & 1                        \\
                    1676                    & 04719$-$0086$+$0071          & 47.1917                 & $-$0.8583                 & 6                                        & \nodata                           & 0.20                          & 7.10                              & 6.36                              & $-$0.77                         & 4.6                       & 2.4                       & 10.8                         & 1.16E+21                                & 1                        \\
                    1677                    & 04722$-$0102$+$0073          & 47.2167                 & $-$1.0167                 & 5,  6                                    & \nodata                           & 0.21                          & 7.25                              & 6.69                              & $-$0.40                         & 7.3                       & 3.1                       & 14.6                         & 2.33E+21                                & 1                        \\
                    1678                    & 04728$-$0082$+$0076          & 47.2833                 & $-$0.8167                 & 5,  6                                    & \nodata                           & 0.24                          & 7.62                              & 6.64                              & $-$0.45                         & 4.7                       & 1.7                       & 9.7                          & 1.67E+21                                & 1                        \\
                    1679                    & 04729$-$0087$+$0078          & 47.2917                 & $-$0.8667                 & 5,  6                                    & \nodata                           & 0.26                          & 7.84                              & 6.49                              & $-$0.63                         & 5.6                       & 1.8                       & 12.7                         & 1.89E+21                                & 1                        \\
                    1680                    & 04729$-$0099$+$0074          & 47.2917                 & $-$0.9917                 & 5,  6                                    & \nodata                           & 0.23                          & 7.40                              & 6.70                              & $-$0.41                         & 6.5                       & 2.5                       & 14.5                         & 2.19E+21                                & 1                        \\
                    1681                    & 04731$-$0081$+$0075          & 47.3083                 & $-$0.8083                 & 5,  6                                    & \nodata                           & 0.24                          & 7.55                              & 6.70                              & $-$0.59                         & 4.5                       & 2.1                       & 9.4                          & 1.45E+21                                & 1                        \\
                    1682                    & 04731$-$0095$+$0074          & 47.3083                 & $-$0.9500                 & 5,  6                                    & \nodata                           & 0.22                          & 7.41                              & 6.50                              & $-$0.66                         & 5.9                       & 3.8                       & 11.1                         & 3.02E+21                                & 1                        \\
                    1683                    & 04732$-$0097$+$0073          & 47.3167                 & $-$0.9750                 & 5,  6                                    & \nodata                           & 0.21                          & 7.30                              & 6.44                              & $-$0.61                         & 6.4                       & 3.0                       & 13.0                         & 2.23E+21                                & 1                        \\
                    1684                    & 04732$-$0160$+$0055          & 47.3250                 & $-$1.6000                 & 6                                        & \nodata                           & 0.06                          & 5.51                              & 5.10                              & $-$0.38                         & 5.8                       & 1.6                       & 11.5                         & 8.21E+20                                & 1                        \\
                    1685                    & 04734$-$0090$+$0073          & 47.3417                 & $-$0.9000                 & 1,  5,  6                                & \nodata                           & 0.22                          & 7.29                              & 6.44                              & $-$0.41                         & 5.0                       & 4.0                       & 12.7                         & 4.62E+21                                & 1                        \\
                    1686                    & 04735$-$0085$+$0075          & 47.3500                 & $-$0.8500                 & 5,  6                                    & \nodata                           & 0.22                          & 7.48                              & 6.93                              & $-$0.82                         & 3.0                       & 0.7                       & 7.2                          & 1.50E+21                                & 2                        \\
                    1687                    & 04737$-$0087$+$0070          & 47.3750                 & $-$0.8750                 & 5,  6                                    & \nodata                           & 0.17                          & 6.98                              & 6.07                              & $-$0.45                         & 5.7                       & 3.7                       & 11.6                         & 4.10E+21                                & 1                        \\
                    1688                    & 04739$-$0080$+$0070          & 47.3917                 & $-$0.8000                 & 5,  6                                    & \nodata                           & 0.19                          & 7.03                              & 6.48                              & $-$0.39                         & 5.6                       & 2.4                       & 11.4                         & 1.71E+21                                & 1                        \\
                    1689                    & 04740$-$0168$+$0063          & 47.4000                 & $-$1.6833                 & 6                                        & \nodata                           & 0.10                          & 6.26                              & 5.49                              & $-$0.68                         & 5.9                       & 2.7                       & 11.5                         & 1.56E+21                                & 1                        \\
                    1690                    & 04741$-$0072$+$0077          & 47.4083                 & $-$0.7250                 & 1,  5,  6                                & \nodata                           & 0.25                          & 7.67                              & 6.65                              & $-$0.73                         & 4.4                       & 1.6                       & 9.9                          & 1.00E+21                                & 1                        \\
                    1691                    & 04741$-$0087$+$0072          & 47.4083                 & $-$0.8750                 & 5,  6                                    & \nodata                           & 0.19                          & 7.21                              & 6.71                              & $-$0.55                         & 3.0                       & 1.4                       & 6.8                          & 5.00E+21                                & 2                        \\
                    1692                    & 04742$-$0085$+$0073          & 47.4167                 & $-$0.8500                 & 5,  6                                    & \nodata                           & 0.21                          & 7.28                              & 6.52                              & $-$1.16                         & 2.7                       & 1.3                       & 6.7                          & 3.18E+21                                & 2                        \\
                    1693                    & 04744$-$0077$+$0074          & 47.4417                 & $-$0.7667                 & 5,  6                                    & \nodata                           & 0.22                          & 7.40                              & 6.52                              & $-$0.67                         & 5.4                       & 2.9                       & 13.1                         & 1.97E+21                                & 1                        \\
                    1694                    & 04744$-$0161$+$0063          & 47.4417                 & $-$1.6083                 & 1,  6                                    & \nodata                           & 0.11                          & 6.32                              & 5.46                              & $-$0.76                         & 5.4                       & 1.8                       & 11.0                         & 9.65E+20                                & 1                        \\
                    1695                    & 04745$-$0171$+$0063          & 47.4500                 & $-$1.7083                 & 6                                        & \nodata                           & 0.11                          & 6.32                              & 5.62                              & $-$0.48                         & 5.2                       & 1.8                       & 13.6                         & 1.28E+21                                & 1                        \\
                    1696                    & 04747$-$0075$+$0073          & 47.4667                 & $-$0.7500                 & 5,  6                                    & \nodata                           & 0.22                          & 7.33                              & 6.59                              & $-$0.66                         & 5.7                       & 3.0                       & 14.1                         & 1.81E+21                                & 1                        \\
                    1697                    & 04749$-$0079$+$0072          & 47.4917                 & $-$0.7917                 & 5,  6                                    & \nodata                           & 0.21                          & 7.16                              & 6.59                              & $-$0.32                         & 5.9                       & 2.3                       & 15.9                         & 2.20E+21                                & 1                        \\
                    1698                    & 04753$-$0069$+$0076          & 47.5333                 & $-$0.6917                 & 6                                        & \nodata                           & 0.23                          & 7.60                              & 6.98                              & $-$0.53                         & 4.7                       & 1.8                       & 10.7                         & 1.02E+21                                & 1                        \\
                    1699                    & 04756$-$0050$+$0410          & 47.5583                 & $-$0.5000                 & 6                                        & \nodata                           & 3.69                          & 41.03                             & 40.63                             & $-$0.43                         & 4.6                       & 2.0                       & 8.7                          & 9.27E+20                                & 1                        \\
                    1700                    & 04759$-$0077$+$0073          & 47.5917                 & $-$0.7667                 & 6                                        & \nodata                           & 0.21                          & 7.27                              & 6.49                              & $-$0.47                         & 5.2                       & 2.7                       & 10.5                         & 2.25E+21                                & 1                        \\
                    1701                    & 04763$-$0082$+$0069          & 47.6333                 & $-$0.8250                 & 5,  6                                    & \nodata                           & 0.18                          & 6.87                              & 6.23                              & $-$0.42                         & 5.0                       & 2.2                       & 9.0                          & 1.69E+21                                & 1                        \\
                    1702                    & 04767$-$0119$+$0070          & 47.6667                 & $-$1.1917                 & 6                                        & \nodata                           & 0.17                          & 6.97                              & 6.30                              & $-$0.48                         & 5.6                       & 1.7                       & 11.6                         & 1.09E+21                                & 1                        \\
                    1703                    & 04767$-$0123$+$0073          & 47.6667                 & $-$1.2333                 & 6                                        & \nodata                           & 0.22                          & 7.25                              & 6.23                              & $-$0.63                         & 6.1                       & 1.6                       & 11.3                         & 1.21E+21                                & 1                        \\
                    1704                    & 04767$-$0139$+$0066          & 47.6667                 & $-$1.3917                 & 6                                        & \nodata                           & 0.14                          & 6.63                              & 5.86                              & $-$0.63                         & 4.9                       & 2.4                       & 10.8                         & 1.47E+21                                & 1                        \\
                    1705                    & 04768$-$0099$+$0064          & 47.6833                 & $-$0.9917                 & 6                                        & \nodata                           & 0.11                          & 6.45                              & 5.95                              & $-$0.51                         & 4.6                       & 1.7                       & 12.8                         & 8.17E+20                                & 1                        \\
                    1706                    & 04768$-$0188$+$0064          & 47.6833                 & $-$1.8833                 & 6                                        & \nodata                           & 0.11                          & 6.35                              & 6.08                              & $-$0.29                         & 4.3                       & 1.8                       & 11.0                         & 7.71E+20                                & 1                        \\
                    1707                    & 04770$-$0087$+$0069          & 47.7000                 & $-$0.8667                 & 6                                        & \nodata                           & 0.18                          & 6.87                              & 6.02                              & $-$0.71                         & 4.5                       & 1.7                       & 11.1                         & 9.58E+20                                & 1                        \\
                    1708                    & 04770$-$0089$+$0067          & 47.7000                 & $-$0.8917                 & 6                                        & \nodata                           & 0.17                          & 6.68                              & 6.14                              & $-$0.56                         & 5.4                       & 2.1                       & 11.8                         & 9.91E+20                                & 1                        \\
                    1709                    & 04770$-$0185$+$0064          & 47.7000                 & $-$1.8500                 & 6                                        & \nodata                           & 0.12                          & 6.42                              & 5.90                              & $-$0.63                         & 4.2                       & 2.6                       & 10.3                         & 1.08E+21                                & 1                        \\
                    1710                    & 04771$-$0096$+$0067          & 47.7083                 & $-$0.9583                 & 6                                        & \nodata                           & 0.16                          & 6.67                              & 6.12                              & $-$0.64                         & 5.1                       & 2.1                       & 11.2                         & 8.67E+20                                & 1                        \\
                    1711                    & 04771$-$0180$+$0063          & 47.7083                 & $-$1.8000                 & 6                                        & \nodata                           & 0.10                          & 6.28                              & 5.67                              & $-$0.56                         & 4.1                       & 1.1                       & 10.7                         & 5.43E+20                                & 1                        \\
                    1712                    & 04772$-$0082$+$0068          & 47.7250                 & $-$0.8250                 & 6                                        & \nodata                           & 0.17                          & 6.82                              & 6.05                              & $-$0.51                         & 5.2                       & 2.4                       & 11.7                         & 1.86E+21                                & 1                        \\
                    1713                    & 04772$-$0111$+$0073          & 47.7250                 & $-$1.1083                 & 6                                        & \nodata                           & 0.22                          & 7.31                              & 6.72                              & $-$0.50                         & 5.7                       & 1.3                       & 14.4                         & 7.56E+20                                & 1                        \\
                    1714                    & 04772$-$0185$+$0064          & 47.7250                 & $-$1.8500                 & 6                                        & \nodata                           & 0.11                          & 6.40                              & 6.04                              & $-$0.44                         & 3.6                       & 2.2                       & 8.6                          & 8.97E+20                                & 1                        \\
                    1715                    & 04773$-$0181$+$0063          & 47.7333                 & $-$1.8083                 & 1,  6                                    & \nodata                           & 0.11                          & 6.32                              & 5.81                              & $-$0.53                         & 4.8                       & 1.2                       & 10.9                         & 4.98E+20                                & 1                        \\
                    1716                    & 04777$-$0081$+$0067          & 47.7667                 & $-$0.8083                 & 6                                        & \nodata                           & 0.16                          & 6.70                              & 6.07                              & $-$0.44                         & 5.8                       & 2.2                       & 11.6                         & 1.52E+21                                & 1                        \\
                    1717                    & 04780$-$0083$+$0069          & 47.8000                 & $-$0.8333                 & 6                                        & \nodata                           & 0.19                          & 6.95                              & 6.25                              & $-$0.57                         & 5.0                       & 1.8                       & 11.9                         & 1.04E+21                                & 1                        \\
                    1718                    & 04793$-$0083$+$0067          & 47.9333                 & $-$0.8333                 & 5,  6                                    & \nodata                           & 0.14                          & 6.69                              & 6.19                              & $-$0.55                         & 4.1                       & 1.2                       & 8.2                          & 4.81E+20                                & 1                        \\
                    1719                    & 04797$-$0077$+$0066          & 47.9667                 & $-$0.7667                 & 5,  6                                    & \nodata                           & 0.15                          & 6.63                              & 5.89                              & $-$0.75                         & 4.1                       & 2.4                       & 10.2                         & 1.21E+21                                & 1                        \\
                    1720                    & 04810$-$0193$+$0218          & 48.1000                 & $-$1.9333                 & 6                                        & \nodata                           & 1.22                          & 21.83                             & 21.20                             & $-$0.52                         & 2.5                       & 1.2                       & 6.7                          & 7.23E+20                                & 1                        \\
                    1721                    & 04833$-$0236$+$0049          & 48.3333                 & $-$2.3583                 & 6                                        & \nodata                           & 0.04                          & 4.86                              & 4.53                              & $-$0.46                         & 4.2                       & 1.1                       & 9.4                          & 3.44E+20                                & 1                        \\
                    1722                    & 04835$+$0024$+$0057          & 48.3500                 & 0.2417                  & 5,  6                                    & \nodata                           & 0.29                          & 5.71                              & 5.07                              & $-$0.33                         & 4.7                       & 3.3                       & 8.0                          & 4.24E+21                                & 1                        \\
                    1723                    & 04852$-$0029$+$0362          & 48.5167                 & $-$0.2917                 & 6                                        & \nodata                           & 4.78                          & 36.18                             & 35.02                             & $-$0.31                         & 6.6                       & 2.1                       & 12.1                         & 3.87E+21                                & 1                        \\
                    1724                    & 04857$-$0048$+$0324          & 48.5667                 & $-$0.4833                 & 6,  7                                    & \nodata                           & 4.88                          & 32.38                             & 31.88                             & $-$0.25                         & 7.9                       & 3.3                       & 11.9                         & 3.59E+21                                & 1                        \\
                    1725                    & 04857$-$0183$+$0052          & 48.5667                 & $-$1.8333                 & 6                                        & \nodata                           & 0.05                          & 5.19                              & 4.77                              & $-$0.44                         & 6.4                       & 2.3                       & 11.4                         & 1.10E+21                                & 1                        \\
                    1726                    & 04857$-$0160$+$0197          & 48.5750                 & $-$1.6000                 & 6                                        & \nodata                           & 1.08                          & 19.69                             & 19.14                             & $-$0.48                         & 3.4                       & 1.3                       & 9.1                          & 6.71E+20                                & 1                        \\
                    1727                    & 04860$+$0036$+$0055          & 48.6000                 & 0.3583                  & 5,  6                                    & \nodata                           & 0.16                          & 5.55                              & 4.78                              & $-$0.47                         & 5.3                       & 1.1                       & 9.4                          & 7.98E+20                                & 1                        \\
                    1728                    & 04865$-$0027$+$0338          & 48.6500                 & $-$0.2750                 & 6,  7                                    & \nodata                           & 3.33                          & 33.77                             & 32.78                             & $-$0.40                         & 7.2                       & 4.2                       & 12.7                         & 5.92E+21                                & 1                        \\
                    1729                    & 04868$-$0023$+$0345          & 48.6833                 & $-$0.2333                 & 6,  7                                    & \nodata                           & 4.45                          & 34.47                             & 33.63                             & $-$0.40                         & 6.6                       & 4.5                       & 11.1                         & 5.81E+21                                & 1                        \\
                    1730                    & 04873$+$0254$+$0062          & 48.7333                 & 2.5417                  & 5,  6                                    & \nodata                           & 0.10                          & 6.24                              & 5.99                              & $-$0.36                         & 4.5                       & 1.5                       & 9.4                          & 5.08E+20                                & 1                        \\
                    1731                    & 04878$-$0001$+$0499          & 48.7833                 & $-$0.0083                 & 6                                        & \nodata                           & 4.05                          & 49.93                             & 49.34                             & $-$0.53                         & 4.1                       & 1.9                       & 8.9                          & 1.00E+21                                & 1                        \\
                    1732                    & 04879$+$0002$+$0501          & 48.7917                 & 0.0167                  & 6                                        & \nodata                           & 4.05                          & 50.08                             & 49.43                             & $-$0.40                         & 9.7                       & 5.9                       & 16.2                         & 6.13E+21                                & 1                        \\
                    1733                    & 04887$-$0148$+$0053          & 48.8750                 & $-$1.4833                 & 1,  6                                    & \nodata                           & 0.05                          & 5.27                              & 4.86                              & $-$0.31                         & 6.0                       & 1.7                       & 12.5                         & 1.06E+21                                & 1                        \\
                    1734                    & 04900$-$0052$+$0470          & 49.0000                 & $-$0.5250                 & 4,  6                                    & \nodata                           & 4.08                          & 46.99                             & 46.53                             & $-$0.26                         & 5.1                       & 2.6                       & 9.1                          & 2.44E+21                                & 1                        \\
                    1735                    & 04902$-$0059$+$0486          & 49.0250                 & $-$0.5917                 & 1,  4,  6                                & \nodata                           & 5.45                          & 48.56                             & 47.77                             & $-$0.38                         & 4.0                       & 1.2                       & 11.1                         & 1.18E+21                                & 1                        \\
                    1736                    & 04906$-$0153$+$0060          & 49.0583                 & $-$1.5333                 & 6                                        & \nodata                           & 0.07                          & 5.97                              & 5.51                              & $-$0.59                         & 4.5                       & 1.4                       & 10.4                         & 5.10E+20                                & 1                        \\
                    1737                    & 04908$-$0158$+$0059          & 49.0833                 & $-$1.5833                 & 6                                        & \nodata                           & 0.07                          & 5.91                              & 5.24                              & $-$0.67                         & 5.4                       & 1.3                       & 13.6                         & 6.16E+20                                & 1                        \\
                    1738                    & 04911$-$0153$+$0060          & 49.1083                 & $-$1.5333                 & 5,  6                                    & \nodata                           & 0.07                          & 5.98                              & 5.33                              & $-$0.72                         & 5.1                       & 1.5                       & 11.8                         & 6.23E+20                                & 1                        \\
                    1739                    & 04911$-$0160$+$0056          & 49.1083                 & $-$1.6000                 & 6                                        & \nodata                           & 0.06                          & 5.55                              & 5.23                              & $-$0.38                         & 5.5                       & 1.1                       & 13.7                         & 4.59E+20                                & 1                        \\
                    1740                    & 04912$+$0067$+$0499          & 49.1250                 & 0.6750                  & 5,  6                                    & \nodata                           & 3.44                          & 49.90                             & 49.08                             & $-$0.75                         & 4.0                       & 0.9                       & 7.5                          & 4.27E+20                                & 1                        \\
                    1741                    & 04914$-$0132$+$0058          & 49.1417                 & $-$1.3167                 & 5,  6                                    & \nodata                           & 0.06                          & 5.76                              & 5.25                              & $-$0.80                         & 2.4                       & 0.5                       & 6.5                          & 1.08E+21                                & 2                        \\
                    1742                    & 04917$-$0132$+$0057          & 49.1667                 & $-$1.3250                 & 5,  6                                    & \nodata                           & 0.06                          & 5.75                              & 5.19                              & $-$0.56                         & 2.5                       & 1.0                       & 7.6                          & 3.29E+21                                & 2                        \\
                    1743                    & 04918$-$0129$+$0058          & 49.1833                 & $-$1.2917                 & 5,  6                                    & \nodata                           & 0.07                          & 5.75                              & 5.33                              & $-$0.63                         & 2.8                       & 1.7                       & 6.5                          & 4.94E+21                                & 2                        \\
                    1744                    & 04918$-$0153$+$0053          & 49.1833                 & $-$1.5333                 & 5,  6                                    & \nodata                           & 0.05                          & 5.34                              & 4.68                              & $-$0.62                         & 4.7                       & 1.6                       & 12.0                         & 8.12E+20                                & 1                        \\
                    1745                    & 04919$-$0148$+$0053          & 49.1917                 & $-$1.4833                 & 5,  6                                    & \nodata                           & 0.05                          & 5.32                              & 4.84                              & $-$0.42                         & 5.9                       & 2.2                       & 12.0                         & 1.20E+21                                & 1                        \\
                    1746                    & 04920$-$0125$+$0058          & 49.2000                 & $-$1.2500                 & 5,  6                                    & \nodata                           & 0.06                          & 5.81                              & 5.21                              & $-$0.83                         & 6.3                       & 1.1                       & 13.7                         & 2.92E+21                                & 2                        \\
                    1747                    & 04920$-$0140$+$0052          & 49.2000                 & $-$1.4000                 & 5,  6                                    & \nodata                           & 0.05                          & 5.15                              & 4.69                              & $-$0.37                         & 5.6                       & 2.9                       & 12.4                         & 1.87E+21                                & 1                        \\
                    1748                    & 04920$-$0143$+$0052          & 49.2000                 & $-$1.4333                 & 5,  6                                    & \nodata                           & 0.05                          & 5.21                              & 4.52                              & $-$0.60                         & 4.9                       & 2.7                       & 12.2                         & 1.57E+21                                & 1                        \\
                    1749                    & 04921$-$0128$+$0058          & 49.2083                 & $-$1.2833                 & 1,  5,  6                                & \nodata                           & 0.06                          & 5.80                              & 5.34                              & $-$0.66                         & 3.1                       & 1.9                       & 7.2                          & 5.57E+21                                & 2                        \\
                    1750                    & 04921$-$0132$+$0058          & 49.2083                 & $-$1.3250                 & 5,  6                                    & \nodata                           & 0.06                          & 5.80                              & 5.35                              & $-$0.68                         & 2.5                       & 1.0                       & 6.0                          & 2.58E+21                                & 2                        \\
                    1751                    & 04921$-$0157$+$0053          & 49.2083                 & $-$1.5750                 & 6                                        & \nodata                           & 0.05                          & 5.33                              & 4.86                              & $-$0.49                         & 4.9                       & 2.4                       & 11.4                         & 1.12E+21                                & 1                        \\
                    1752                    & 04922$-$0152$+$0052          & 49.2250                 & $-$1.5250                 & 5,  6                                    & \nodata                           & 0.05                          & 5.16                              & 4.79                              & $-$0.31                         & 5.8                       & 2.2                       & 13.3                         & 1.30E+21                                & 1                        \\
                    1753                    & 04923$-$0140$+$0053          & 49.2333                 & $-$1.4000                 & 5,  6                                    & 17                                & 0.06                          & 5.32                              & 4.78                              & $-$0.51                         & 5.0                       & 2.7                       & 11.7                         & 1.47E+21                                & 1                        \\
                    1754                    & 04926$-$0009$+$0487          & 49.2583                 & $-$0.0917                 & 6,  7                                    & \nodata                           & 4.20                          & 48.73                             & 47.16                             & $-$0.42                         & 10.1                      & 4.5                       & 16.0                         & 1.03E+22                                & 1                        \\
                    1755                    & 04927$-$0141$+$0055          & 49.2667                 & $-$1.4083                 & 5,  6                                    & 17                                & 0.06                          & 5.51                              & 5.28                              & $-$0.41                         & 2.4                       & 1.0                       & 5.7                          & 2.39E+21                                & 2                        \\
                    1756                    & 04930$-$0007$+$0478          & 49.3000                 & $-$0.0750                 & 6,  7                                    & \nodata                           & 4.06                          & 47.78                             & 46.59                             & $-$0.37                         & 8.0                       & 3.1                       & 14.7                         & 5.37E+21                                & 1                        \\
                    1757                    & 04930$-$0142$+$0054          & 49.3000                 & $-$1.4167                 & 5,  6                                    & \nodata                           & 0.06                          & 5.38                              & 5.03                              & $-$0.34                         & 5.2                       & 2.8                       & 11.8                         & 1.47E+21                                & 1                        \\
                    1758                    & 04932$-$0141$+$0053          & 49.3250                 & $-$1.4083                 & 5,  6                                    & \nodata                           & 0.05                          & 5.26                              & 4.72                              & $-$0.56                         & 5.3                       & 2.7                       & 9.4                          & 1.41E+21                                & 1                        \\
                    1759                    & 04934$-$0136$+$0052          & 49.3417                 & $-$1.3583                 & 5,  6                                    & \nodata                           & 0.05                          & 5.24                              & 4.76                              & $-$0.39                         & 7.0                       & 2.3                       & 12.9                         & 1.41E+21                                & 1                        \\
                    1760                    & 04935$-$0132$+$0053          & 49.3500                 & $-$1.3167                 & 6                                        & \nodata                           & 0.05                          & 5.29                              & 4.74                              & $-$0.45                         & 6.6                       & 2.1                       & 13.0                         & 1.26E+21                                & 1                        \\
                    1761                    & 04938$-$0134$+$0052          & 49.3833                 & $-$1.3417                 & 6                                        & \nodata                           & 0.05                          & 5.21                              & 4.70                              & $-$0.52                         & 5.7                       & 1.4                       & 11.8                         & 6.42E+20                                & 1                        \\
                    1762                    & 04956$-$0041$+$0624          & 49.5583                 & $-$0.4083                 & 3,  6                                    & \nodata                           & 5.46                          & 62.38                             & 60.78                             & $-$0.24                         & 19.8                      & 3.8                       & 28.1                         & 1.93E+22                                & 1                        \\
                    1763                    & 04960$+$0391$+$0053          & 49.6000                 & 3.9083                  & 5,  6                                    & \nodata                           & 0.05                          & 5.30                              & 4.88                              & $-$0.57                         & 6.0                       & 1.3                       & 10.7                         & 4.37E+20                                & 1                        \\
                    1764                    & 04960$-$0035$+$0456          & 49.6000                 & $-$0.3500                 & 3,  6                                    & \nodata                           & 4.05                          & 45.59                             & 44.92                             & $-$0.67                         & 3.5                       & 1.5                       & 7.4                          & 7.29E+20                                & 1                        \\
                    1765                    & 04979$-$0180$+$0032          & 49.7917                 & $-$1.8000                 & 6                                        & \nodata                           & 0.54                          & 3.20                              & 2.89                              & $-$0.34                         & 6.7                       & 2.4                       & 11.9                         & 1.10E+21                                & 1                        \\
                    1766                    & 04983$+$0037$+$0053          & 49.8333                 & 0.3667                  & 1,  5,  6                                & \nodata                           & 9.90                          & 5.32                              & 4.53                              & $-$0.23                         & 13.1                      & 4.8                       & 15.5                         & 9.82E+21                                & 1                        \\
                    1767                    & 05010$+$0262$+$0061          & 50.1000                 & 2.6167                  & 5,  6                                    & \nodata                           & 0.07                          & 6.11                              & 5.62                              & $-$0.66                         & 2.6                       & 4.1                       & 8.6                          & 2.30E+21                                & 1                        \\
                    1768                    & 05020$-$0170$+$0042          & 50.2000                 & $-$1.7000                 & 5,  6                                    & \nodata                           & 0.02                          & 4.19                              & 3.82                              & $-$0.48                         & 5.7                       & 1.4                       & 10.9                         & 4.80E+20                                & 1                        \\
                    1769                    & 05025$-$0167$+$0042          & 50.2500                 & $-$1.6750                 & 5,  6                                    & \nodata                           & 0.02                          & 4.21                              & 3.81                              & $-$0.45                         & 6.0                       & 2.9                       & 11.3                         & 1.35E+21                                & 1                        \\
                    1770                    & 05051$+$0100$+$0557          & 50.5083                 & 1.0000                  & 5,  6                                    & \nodata                           & 4.56                          & 55.68                             & 55.07                             & $-$0.36                         & 10.6                      & 5.7                       & 18.3                         & 6.47E+21                                & 1                        \\
                    1771                    & 05073$+$0243$+$0152          & 50.7333                 & 2.4333                  & 5,  6                                    & \nodata                           & 0.81                          & 15.24                             & 14.55                             & $-$0.77                         & 4.8                       & 1.3                       & 9.2                          & 5.48E+20                                & 1                        \\
                    1772                    & 05075$+$0251$+$0153          & 50.7500                 & 2.5083                  & 5,  6                                    & \nodata                           & 0.81                          & 15.26                             & 14.24                             & $-$0.83                         & 4.2                       & 1.5                       & 9.7                          & 8.28E+20                                & 1                        \\
                    1773                    & 05080$+$0067$+$0060          & 50.8000                 & 0.6750                  & 5,  6                                    & \nodata                           & 9.81                          & 5.98                              & 5.13                              & $-$0.69                         & 2.9                       & 1.0                       & 7.8                          & 5.57E+20                                & 1                        \\
                    1774                    & 05081$+$0034$+$0340          & 50.8083                 & 0.3417                  & 5,  6                                    & \nodata                           & 5.73                          & 33.98                             & 33.65                             & $-$0.32                         & 4.2                       & 1.0                       & 7.4                          & 4.84E+20                                & 1                        \\
                    1775                    & 05083$+$0035$+$0404          & 50.8333                 & 0.3500                  & 5,  6                                    & \nodata                           & 4.49                          & 40.38                             & 39.15                             & $-$0.52                         & 4.6                       & 3.5                       & 10.7                         & 4.49E+21                                & 1                        \\
                    1776                    & 05084$+$0061$+$0052          & 50.8417                 & 0.6083                  & 5,  6                                    & \nodata                           & 9.81                          & 5.17                              & 4.56                              & $-$0.54                         & 3.2                       & 1.4                       & 7.3                          & 7.98E+20                                & 1                        \\
                    1777                    & 05100$+$0317$+$0157          & 51.0000                 & 3.1667                  & 5,  6                                    & \nodata                           & 0.84                          & 15.69                             & 15.18                             & $-$0.47                         & 4.1                       & 2.0                       & 8.5                          & 1.06E+21                                & 1                        \\
                    1778                    & 05103$+$0057$+$0053          & 51.0333                 & 0.5667                  & 5,  6                                    & \nodata                           & 9.77                          & 5.26                              & 4.49                              & $-$0.43                         & 4.5                       & 2.0                       & 10.1                         & 1.74E+21                                & 1                        \\
                    1779                    & 05107$+$0298$+$0166          & 51.0750                 & 2.9833                  & 5,  6                                    & \nodata                           & 0.89                          & 16.57                             & 15.69                             & $-$0.86                         & 5.0                       & 1.6                       & 8.9                          & 7.76E+20                                & 1                        \\
                    1780                    & 05107$+$0302$+$0165          & 51.0750                 & 3.0167                  & 5,  6                                    & \nodata                           & 0.88                          & 16.49                             & 15.79                             & $-$0.78                         & 5.3                       & 2.0                       & 8.9                          & 8.71E+20                                & 1                        \\
                    1781                    & 05118$+$0282$+$0147          & 51.1833                 & 2.8250                  & 5,  6                                    & \nodata                           & 0.78                          & 14.68                             & 14.27                             & $-$0.34                         & 5.4                       & 1.2                       & 13.0                         & 7.04E+20                                & 1                        \\
                    1782                    & 05122$-$0082$+$0456          & 51.2167                 & $-$0.8250                 & 5,  6                                    & \nodata                           & 3.13                          & 45.60                             & 44.60                             & $-$0.49                         & 6.1                       & 2.6                       & 11.5                         & 2.65E+21                                & 1                        \\
                    1783                    & 05122$-$0096$+$0439          & 51.2250                 & $-$0.9583                 & 1,  5,  6                                & \nodata                           & 2.88                          & 43.90                             & 43.70                             & $-$0.17                         & 2.9                       & 1.4                       & 6.4                          & 8.40E+20                                & 1                        \\
                    1784                    & 05123$+$0408$+$0158          & 51.2333                 & 4.0833                  & 5,  6                                    & \nodata                           & 0.85                          & 15.84                             & 15.67                             & $-$0.19                         & 6.0                       & 1.7                       & 13.7                         & 7.88E+20                                & 1                        \\
                    1785                    & 05125$-$0098$+$0118          & 51.2500                 & $-$0.9833                 & 5,  6                                    & \nodata                           & 0.56                          & 11.82                             & 11.50                             & $-$0.42                         & 3.9                       & 1.3                       & 9.0                          & 4.41E+20                                & 1                        \\
                    1786                    & 05126$+$0314$+$0166          & 51.2583                 & 3.1417                  & 5,  6                                    & \nodata                           & 0.91                          & 16.59                             & 16.33                             & $-$0.32                         & 1.9                       & 0.7                       & 5.9                          & 2.08E+21                                & 2                        \\
                    1787                    & 05127$+$0407$+$0161          & 51.2667                 & 4.0667                  & 5,  6                                    & \nodata                           & 0.87                          & 16.11                             & 15.46                             & $-$0.58                         & 6.6                       & 1.6                       & 12.8                         & 8.76E+20                                & 1                        \\
                    1788                    & 05136$+$0020$+$0511          & 51.3583                 & 0.2000                  & 5,  6                                    & \nodata                           & 4.65                          & 51.07                             & 50.39                             & $-$0.47                         & 5.0                       & 1.7                       & 9.6                          & 1.16E+21                                & 1                        \\
                    1789                    & 05138$+$0101$+$0230          & 51.3833                 & 1.0083                  & 5,  6                                    & \nodata                           & 1.36                          & 23.02                             & 22.39                             & $-$0.57                         & 3.3                       & 1.2                       & 7.3                          & 6.36E+20                                & 1                        \\
                    1790                    & 05140$+$0405$+$0165          & 51.4000                 & 4.0500                  & 6                                        & \nodata                           & 0.90                          & 16.47                             & 16.11                             & $-$0.35                         & 4.4                       & 1.9                       & 10.2                         & 9.15E+20                                & 1                        \\
                    1791                    & 05147$+$0282$+$0156          & 51.4667                 & 2.8250                  & 5,  6                                    & \nodata                           & 0.84                          & 15.63                             & 14.42                             & $-$1.05                         & 5.3                       & 2.3                       & 12.0                         & 1.28E+21                                & 1                        \\
                    1792                    & 05155$+$0282$+$0156          & 51.5500                 & 2.8250                  & 5,  6                                    & \nodata                           & 0.84                          & 15.57                             & 13.95                             & $-$1.21                         & 5.8                       & 2.7                       & 11.6                         & 1.81E+21                                & 1                        \\
                    1793                    & 05164$+$0302$+$0160          & 51.6417                 & 3.0250                  & 5,  6                                    & \nodata                           & 0.86                          & 16.00                             & 14.43                             & $-$1.03                         & 5.0                       & 2.2                       & 11.9                         & 1.68E+21                                & 1                        \\
                    1794                    & 05168$+$0305$+$0159          & 51.6833                 & 3.0500                  & 5,  6                                    & \nodata                           & 0.86                          & 15.92                             & 14.48                             & $-$0.93                         & 6.1                       & 3.3                       & 12.7                         & 2.78E+21                                & 1                        \\
                    1795                    & 05171$+$0382$+$0155          & 51.7083                 & 3.8250                  & 5,  6                                    & \nodata                           & 0.82                          & 15.46                             & 14.65                             & $-$0.65                         & 4.3                       & 2.1                       & 12.6                         & 1.28E+21                                & 1                        \\
                    1796                    & 05177$+$0150$+$0232          & 51.7667                 & 1.5000                  & 5,  6                                    & \nodata                           & 1.37                          & 23.20                             & 22.68                             & $-$0.53                         & 3.9                       & 1.4                       & 9.4                          & 6.19E+20                                & 1                        \\
                    1797                    & 05180$+$0406$+$0158          & 51.8000                 & 4.0583                  & 5,  6                                    & \nodata                           & 0.85                          & 15.77                             & 15.09                             & $-$0.45                         & 5.7                       & 1.2                       & 11.8                         & 8.02E+20                                & 1                        \\
                    1798                    & 05182$+$0152$+$0229          & 51.8167                 & 1.5250                  & 5,  6                                    & \nodata                           & 1.35                          & 22.87                             & 22.49                             & $-$0.44                         & 4.8                       & 1.2                       & 11.2                         & 4.54E+20                                & 1                        \\
                    1799                    & 05182$+$0401$+$0155          & 51.8167                 & 4.0083                  & 5,  6                                    & \nodata                           & 0.83                          & 15.55                             & 14.79                             & $-$0.64                         & 5.6                       & 1.7                       & 11.5                         & 9.67E+20                                & 1                        \\
                    1800                    & 05182$+$0188$+$0242          & 51.8250                 & 1.8833                  & 5,  6                                    & \nodata                           & 1.44                          & 24.23                             & 23.35                             & $-$0.88                         & 5.1                       & 1.2                       & 11.4                         & 5.60E+20                                & 1                        \\
                    1801                    & 05185$+$0410$+$0156          & 51.8500                 & 4.1000                  & 5,  6                                    & \nodata                           & 0.84                          & 15.58                             & 14.85                             & $-$0.93                         & 6.9                       & 1.3                       & 11.0                         & 4.81E+20                                & 1                        \\
                    1802                    & 05186$+$0302$+$0139          & 51.8583                 & 3.0167                  & 5,  6                                    & \nodata                           & 0.72                          & 13.87                             & 12.70                             & $-$1.04                         & 5.1                       & 2.8                       & 12.8                         & 1.66E+21                                & 1                        \\
                    1803                    & 05187$+$0152$+$0228          & 51.8750                 & 1.5167                  & 5,  6                                    & \nodata                           & 1.34                          & 22.77                             & 22.56                             & $-$0.28                         & 4.5                       & 1.4                       & 10.6                         & 4.74E+20                                & 1                        \\
                    1804                    & 05192$+$0189$+$0236          & 51.9167                 & 1.8917                  & 5,  6                                    & \nodata                           & 1.40                          & 23.61                             & 23.23                             & $-$0.39                         & 4.6                       & 1.0                       & 10.2                         & 4.36E+20                                & 1                        \\
                    1805                    & 05210$+$0262$+$0158          & 52.1000                 & 2.6167                  & 5,  6                                    & \nodata                           & 0.86                          & 15.79                             & 15.07                             & $-$0.39                         & 6.2                       & 1.5                       & 12.0                         & 1.32E+21                                & 1                        \\
                    1806                    & 05212$+$0254$+$0153          & 52.1167                 & 2.5417                  & 5,  6                                    & \nodata                           & 0.82                          & 15.28                             & 14.52                             & $-$0.49                         & 4.9                       & 1.2                       & 12.3                         & 8.91E+20                                & 1                        \\
                    1807                    & 05213$-$0026$+$0218          & 52.1333                 & $-$0.2583                 & 5,  6                                    & \nodata                           & 1.32                          & 21.84                             & 21.31                             & $-$0.50                         & 3.4                       & 1.3                       & 8.3                          & 6.35E+20                                & 1                        \\
                    1808                    & 05214$+$0020$+$0566          & 52.1417                 & 0.2000                  & 5,  6                                    & \nodata                           & 5.67                          & 56.62                             & 56.25                             & $-$0.25                         & 5.9                       & 1.4                       & 12.2                         & 9.45E+20                                & 1                        \\
                    1809                    & 05217$+$0250$+$0155          & 52.1750                 & 2.5000                  & 5,  6                                    & \nodata                           & 0.84                          & 15.54                             & 14.75                             & $-$0.58                         & 4.9                       & 1.4                       & 10.6                         & 8.72E+20                                & 1                        \\
                    1810                    & 05222$+$0156$+$0065          & 52.2250                 & 1.5583                  & 5,  6                                    & \nodata                           & 0.11                          & 6.55                              & 6.06                              & $-$0.63                         & 4.2                       & 1.5                       & 9.3                          & 5.21E+20                                & 1                        \\
                    1811                    & 05227$-$0059$+$0498          & 52.2667                 & $-$0.5917                 & 5,  6                                    & \nodata                           & 4.52                          & 49.83                             & 49.38                             & $-$0.50                         & 3.9                       & 1.0                       & 7.5                          & 3.91E+20                                & 1                        \\
                    1812                    & 05227$-$0027$+$0208          & 52.2750                 & $-$0.2750                 & 5,  6                                    & \nodata                           & 1.24                          & 20.82                             & 20.33                             & $-$0.34                         & 5.2                       & 2.6                       & 10.7                         & 1.93E+21                                & 1                        \\
                    1813                    & 05228$-$0022$+$0214          & 52.2833                 & $-$0.2167                 & 1,  5,  6                                & \nodata                           & 1.29                          & 21.42                             & 20.63                             & $-$0.55                         & 3.7                       & 1.1                       & 9.3                          & 7.06E+20                                & 1                        \\
                    1814                    & 05229$-$0024$+$0212          & 52.2917                 & $-$0.2417                 & 5,  6                                    & \nodata                           & 1.27                          & 21.22                             & 20.79                             & $-$0.38                         & 3.6                       & 1.8                       & 8.9                          & 1.01E+21                                & 1                        \\
                    1815                    & 05232$-$0012$+$0512          & 52.3167                 & $-$0.1250                 & 1,  5,  6                                & \nodata                           & 4.58                          & 51.21                             & 50.33                             & $-$0.38                         & 6.1                       & 2.5                       & 11.0                         & 2.83E+21                                & 1                        \\
                    1816                    & 05235$+$0032$+$0464          & 52.3500                 & 0.3167                  & 1,  5,  6                                & \nodata                           & 5.64                          & 46.41                             & 44.83                             & $-$0.62                         & 5.3                       & 3.1                       & 12.8                         & 4.17E+21                                & 1                        \\
                    1817                    & 05237$+$0073$+$0058          & 52.3750                 & 0.7333                  & 5,  6                                    & \nodata                           & 9.56                          & 5.76                              & 5.21                              & $-$0.22                         & 7.0                       & 1.9                       & 10.1                         & 2.25E+21                                & 1                        \\
                    1818                    & 05238$-$0056$+$0341          & 52.3833                 & $-$0.5583                 & 5,  6                                    & \nodata                           & 3.54                          & 34.09                             & 33.67                             & $-$0.25                         & 4.8                       & 1.3                       & 8.3                          & 9.98E+20                                & 1                        \\
                    1819                    & 05251$+$0191$+$0255          & 52.5083                 & 1.9083                  & 5,  6                                    & \nodata                           & 1.56                          & 25.51                             & 24.95                             & $-$0.47                         & 6.4                       & 1.2                       & 13.3                         & 6.76E+20                                & 1                        \\
                    1820                    & 05270$+$0214$+$0230          & 52.7000                 & 2.1417                  & 5,  6                                    & \nodata                           & 1.38                          & 23.02                             & 22.25                             & $-$0.71                         & 5.0                       & 1.9                       & 10.3                         & 9.93E+20                                & 1                        \\
                    1821                    & 05271$+$0144$+$0223          & 52.7083                 & 1.4417                  & 5,  6                                    & \nodata                           & 1.32                          & 22.33                             & 21.94                             & $-$0.40                         & 5.9                       & 2.7                       & 12.5                         & 1.33E+21                                & 1                        \\
                    1822                    & 05273$+$0148$+$0224          & 52.7333                 & 1.4833                  & 5,  6                                    & \nodata                           & 1.33                          & 22.42                             & 21.99                             & $-$0.40                         & 6.5                       & 2.7                       & 12.8                         & 1.52E+21                                & 1                        \\
                    1823                    & 05275$+$0034$+$0152          & 52.7500                 & 0.3417                  & 1,  5,  6                                & \nodata                           & 0.85                          & 15.24                             & 14.53                             & $-$0.28                         & 12.8                      & 3.0                       & 16.6                         & 4.41E+21                                & 1                        \\
                    1824                    & 05281$-$0070$+$0494          & 52.8083                 & $-$0.7000                 & 5,  6                                    & \nodata                           & 4.47                          & 49.43                             & 48.24                             & $-$0.57                         & 8.2                       & 3.9                       & 13.4                         & 4.57E+21                                & 1                        \\
                    1825                    & 05283$-$0069$+$0493          & 52.8333                 & $-$0.6917                 & 5,  6                                    & \nodata                           & 4.48                          & 49.33                             & 48.49                             & $-$0.40                         & 8.7                       & 3.4                       & 15.0                         & 3.87E+21                                & 1                        \\
                    1826                    & 05284$+$0299$+$0098          & 52.8417                 & 2.9917                  & 6                                        & \nodata                           & 0.40                          & 9.84                              & 9.37                              & $-$0.49                         & 3.6                       & 1.6                       & 9.3                          & 7.46E+20                                & 1                        \\
                    1827                    & 05288$+$0300$+$0101          & 52.8833                 & 3.0000                  & 6                                        & \nodata                           & 0.43                          & 10.09                             & 9.44                              & $-$0.68                         & 4.2                       & 1.2                       & 11.0                         & 5.02E+20                                & 1                        \\
                    1828                    & 05289$+$0306$+$0104          & 52.8917                 & 3.0583                  & 6                                        & \nodata                           & 0.47                          & 10.45                             & 10.00                             & $-$0.51                         & 4.1                       & 1.7                       & 9.5                          & 6.85E+20                                & 1                        \\
                    1829                    & 05291$+$0000$+$0222          & 52.9083                 & 0.0000                  & 5,  6                                    & \nodata                           & 1.36                          & 22.17                             & 21.51                             & $-$0.54                         & 8.9                       & 3.1                       & 15.8                         & 2.09E+21                                & 1                        \\
                    1830                    & 05291$+$0038$+$0226          & 52.9083                 & 0.3833                  & 5,  6                                    & \nodata                           & 1.39                          & 22.63                             & 22.04                             & $-$0.35                         & 5.8                       & 1.6                       & 13.6                         & 1.34E+21                                & 1                        \\
                    1831                    & 05291$+$0297$+$0100          & 52.9083                 & 2.9750                  & 6                                        & \nodata                           & 0.42                          & 9.99                              & 9.47                              & $-$0.51                         & 4.5                       & 1.6                       & 9.4                          & 7.46E+20                                & 1                        \\
                    1832                    & 05293$+$0015$+$0205          & 52.9333                 & 0.1500                  & 5,  6                                    & \nodata                           & 1.23                          & 20.49                             & 19.66                             & $-$0.62                         & 5.9                       & 1.8                       & 14.0                         & 1.20E+21                                & 1                        \\
                    1833                    & 05293$+$0302$+$0107          & 52.9333                 & 3.0167                  & 6                                        & \nodata                           & 0.48                          & 10.65                             & 9.98                              & $-$0.45                         & 4.8                       & 1.5                       & 11.1                         & 1.06E+21                                & 1                        \\
                    1834                    & 05294$+$0304$+$0107          & 52.9417                 & 3.0417                  & 6                                        & \nodata                           & 0.48                          & 10.68                             & 9.94                              & $-$0.89                         & 4.7                       & 2.1                       & 10.0                         & 8.28E+20                                & 1                        \\
                    1835                    & 05295$+$0005$+$0219          & 52.9500                 & 0.0500                  & 5,  6                                    & \nodata                           & 1.34                          & 21.88                             & 21.42                             & $-$0.41                         & 6.8                       & 1.6                       & 14.9                         & 9.08E+20                                & 1                        \\
                    1836                    & 05295$-$0001$+$0220          & 52.9500                 & $-$0.0083                 & 5,  6                                    & \nodata                           & 1.34                          & 21.99                             & 21.53                             & $-$0.34                         & 9.1                       & 4.8                       & 17.2                         & 4.01E+21                                & 1                        \\
                    1837                    & 05295$-$0006$+$0219          & 52.9500                 & $-$0.0583                 & 5,  6                                    & \nodata                           & 1.34                          & 21.92                             & 21.26                             & $-$0.53                         & 7.6                       & 4.1                       & 15.1                         & 2.94E+21                                & 1                        \\
                    1838                    & 05296$+$0300$+$0107          & 52.9583                 & 3.0000                  & 6                                        & \nodata                           & 0.49                          & 10.70                             & 9.76                              & $-$0.46                         & 6.1                       & 1.5                       & 10.5                         & 1.40E+21                                & 1                        \\
                    1839                    & 05297$+$0042$+$0232          & 52.9667                 & 0.4250                  & 5,  6                                    & \nodata                           & 1.43                          & 23.18                             & 22.24                             & $-$0.59                         & 6.2                       & 2.2                       & 13.3                         & 1.74E+21                                & 1                        \\
                    1840                    & 05299$+$0027$+$0217          & 52.9917                 & 0.2750                  & 5,  6                                    & \nodata                           & 1.31                          & 21.69                             & 21.38                             & $-$0.35                         & 5.3                       & 1.5                       & 10.2                         & 6.33E+20                                & 1                        \\
                    1841                    & 05299$+$0044$+$0226          & 52.9917                 & 0.4417                  & 5,  6                                    & \nodata                           & 1.39                          & 22.62                             & 22.24                             & $-$0.44                         & 5.1                       & 2.0                       & 12.7                         & 8.33E+20                                & 1                        \\
                    1842                    & 05299$-$0006$+$0219          & 52.9917                 & $-$0.0583                 & 5,  6                                    & \nodata                           & 1.34                          & 21.94                             & 21.32                             & $-$0.44                         & 7.7                       & 2.8                       & 13.9                         & 2.08E+21                                & 1                        \\
                    1843                    & 05300$+$0312$+$0100          & 53.0000                 & 3.1250                  & 6                                        & \nodata                           & 0.44                          & 10.05                             & 9.73                              & $-$0.28                         & 5.9                       & 2.2                       & 11.6                         & 1.20E+21                                & 1                        \\
                    1844                    & 05301$+$0300$+$0102          & 53.0083                 & 3.0000                  & 5,  6                                    & \nodata                           & 0.44                          & 10.19                             & 9.64                              & $-$0.35                         & 8.7                       & 3.2                       & 15.0                         & 2.82E+21                                & 1                        \\
                    1845                    & 05302$+$0002$+$0220          & 53.0167                 & 0.0250                  & 1,  5,  6                                & \nodata                           & 1.33                          & 22.05                             & 21.39                             & $-$0.63                         & 12.3                      & 5.2                       & 17.7                         & 3.50E+21                                & 1                        \\
                    1846                    & 05302$+$0043$+$0227          & 53.0167                 & 0.4333                  & 5,  6                                    & \nodata                           & 1.39                          & 22.70                             & 22.27                             & $-$0.42                         & 5.0                       & 1.9                       & 11.6                         & 9.53E+20                                & 1                        \\
                    1847                    & 05302$+$0310$+$0104          & 53.0250                 & 3.1000                  & 6                                        & \nodata                           & 0.46                          & 10.41                             & 9.69                              & $-$0.47                         & 6.7                       & 4.0                       & 17.1                         & 3.70E+21                                & 1                        \\
                    1848                    & 05303$+$0027$+$0218          & 53.0333                 & 0.2667                  & 5,  6                                    & \nodata                           & 1.33                          & 21.77                             & 21.46                             & $-$0.42                         & 5.4                       & 3.7                       & 9.1                          & 1.71E+21                                & 1                        \\
                    1849                    & 05304$+$0007$+$0056          & 53.0417                 & 0.0667                  & 3,  5,  6                                & \nodata                           & 0.07                          & 5.56                              & 3.27                              & $-$0.41                         & 11.8                      & 4.1                       & 22.6                         & 1.57E+22                                & 1                        \\
                    1850                    & 05307$+$0044$+$0228          & 53.0667                 & 0.4417                  & 5,  6                                    & \nodata                           & 1.40                          & 22.77                             & 22.27                             & $-$0.52                         & 5.7                       & 1.8                       & 10.8                         & 8.04E+20                                & 1                        \\
                    1851                    & 05307$+$0008$+$0217          & 53.0750                 & 0.0833                  & 3,  5,  6                                & 17                                & 1.31                          & 21.72                             & 21.12                             & $-$0.46                         & 12.4                      & 4.5                       & 16.8                         & 3.63E+21                                & 1                        \\
                    1852                    & 05308$-$0024$+$0236          & 53.0833                 & $-$0.2417                 & 5,  6                                    & \nodata                           & 4.41                          & 23.60                             & 22.53                             & $-$0.68                         & 6.3                       & 3.7                       & 12.4                         & 3.31E+21                                & 1                        \\
                    1853                    & 05311$+$0013$+$0219          & 53.1083                 & 0.1333                  & 3,  5,  6                                & \nodata                           & 1.33                          & 21.93                             & 21.52                             & $-$0.28                         & 11.2                      & 4.0                       & 15.0                         & 3.28E+21                                & 1                        \\
                    1854                    & 05312$+$0142$+$0215          & 53.1167                 & 1.4167                  & 5,  6                                    & \nodata                           & 1.27                          & 21.53                             & 21.21                             & $-$0.38                         & 7.0                       & 2.1                       & 13.1                         & 9.22E+20                                & 1                        \\
                    1855                    & 05312$+$0000$+$0223          & 53.1250                 & 0.0000                  & 5,  6                                    & \nodata                           & 1.36                          & 22.34                             & 21.34                             & $-$0.58                         & 10.6                      & 4.4                       & 17.1                         & 4.66E+21                                & 1                        \\
                    1856                    & 05314$+$0007$+$0219          & 53.1417                 & 0.0750                  & 1,  3,  5,  6                            & 17                                & 1.34                          & 21.89                             & 20.56                             & $-$0.47                         & 12.8                      & 9.7                       & 19.4                         & 2.26E+22                                & 1                        \\
                    1857                    & 05314$+$0380$+$0065          & 53.1417                 & 3.8000                  & 5,  6                                    & \nodata                           & 0.09                          & 6.50                              & 6.20                              & $-$0.28                         & 2.7                       & 1.1                       & 8.3                          & 5.20E+20                                & 1                        \\
                    1858                    & 05315$+$0385$+$0066          & 53.1500                 & 3.8500                  & 5,  6                                    & \nodata                           & 0.09                          & 6.57                              & 5.81                              & $-$1.02                         & 2.4                       & 0.9                       & 7.7                          & 2.95E+20                                & 1                        \\
                    1859                    & 05316$+$0024$+$0230          & 53.1583                 & 0.2417                  & 3,  5,  6                                & \nodata                           & 1.42                          & 22.98                             & 22.27                             & $-$0.43                         & 5.8                       & 1.5                       & 12.4                         & 1.18E+21                                & 1                        \\
                    1860                    & 05317$+$0272$+$0096          & 53.1667                 & 2.7250                  & 6                                        & \nodata                           & 0.40                          & 9.60                              & 9.16                              & $-$0.49                         & 2.7                       & 1.7                       & 10.4                         & 7.15E+20                                & 1                        \\
                    1861                    & 05317$+$0306$+$0104          & 53.1667                 & 3.0583                  & 6                                        & \nodata                           & 0.46                          & 10.38                             & 9.85                              & $-$0.42                         & 4.6                       & 1.3                       & 12.0                         & 7.44E+20                                & 1                        \\
                    1862                    & 05321$+$0003$+$0233          & 53.2083                 & 0.0333                  & 3,  5,  6                                & \nodata                           & 5.22                          & 23.30                             & 21.75                             & $-$0.69                         & 10.9                      & 5.3                       & 16.8                         & 7.60E+21                                & 1                        \\
                    1863                    & 05323$-$0012$+$0439          & 53.2333                 & $-$0.1250                 & 5,  6                                    & \nodata                           & 4.53                          & 43.94                             & 42.81                             & $-$0.52                         & 6.9                       & 3.8                       & 11.2                         & 4.62E+21                                & 1                        \\
                    1864                    & 05324$+$0270$+$0099          & 53.2417                 & 2.7000                  & 6                                        & \nodata                           & 0.42                          & 9.89                              & 8.92                              & $-$0.66                         & 3.4                       & 1.3                       & 12.5                         & 8.55E+20                                & 1                        \\
                    1865                    & 05330$+$0006$+$0245          & 53.3000                 & 0.0583                  & 3,  5,  6                                & \nodata                           & 1.51                          & 24.46                             & 23.25                             & $-$0.57                         & 7.0                       & 2.9                       & 11.6                         & 3.15E+21                                & 1                        \\
                    1866                    & 05354$-$0097$+$0248          & 53.5417                 & $-$0.9750                 & 5,  6                                    & \nodata                           & 1.60                          & 24.84                             & 24.52                             & $-$0.30                         & 5.8                       & 2.0                       & 12.1                         & 1.01E+21                                & 1                        \\
                    1867                    & 05356$-$0091$+$0249          & 53.5583                 & $-$0.9083                 & 5,  6                                    & \nodata                           & 1.62                          & 24.89                             & 24.55                             & $-$0.35                         & 5.7                       & 3.4                       & 12.0                         & 1.81E+21                                & 1                        \\
                    1868                    & 05361$-$0027$+$0114          & 53.6083                 & $-$0.2750                 & 5,  6                                    & \nodata                           & 0.56                          & 11.41                             & 11.06                             & $-$0.45                         & 3.4                       & 1.1                       & 8.6                          & 4.04E+20                                & 1                        \\
                    1869                    & 05362$+$0052$+$0240          & 53.6167                 & 0.5167                  & 5,  6                                    & \nodata                           & 1.53                          & 24.01                             & 23.50                             & $-$0.47                         & 6.0                       & 2.6                       & 10.7                         & 1.42E+21                                & 1                        \\
                    1870                    & 05363$+$0004$+$0235          & 53.6333                 & 0.0417                  & 3,  4,  5,  6                            & \nodata                           & 1.47                          & 23.54                             & 23.03                             & $-$0.25                         & 29.6                      & 11.4                      & 37.4                         & 2.49E+22                                & 1                        \\
                    1871                    & 05364$+$0000$+$0240          & 53.6417                 & 0.0000                  & 3,  4,  5,  6                            & \nodata                           & 1.50                          & 23.97                             & 23.47                             & $-$0.34                         & 26.6                      & 7.8                       & 36.5                         & 1.10E+22                                & 1                        \\
                    1872                    & 05365$+$0057$+$0245          & 53.6500                 & 0.5750                  & 5,  6                                    & \nodata                           & 1.56                          & 24.45                             & 24.00                             & $-$0.29                         & 6.6                       & 4.2                       & 11.5                         & 3.80E+21                                & 1                        \\
                    1873                    & 05368$+$0002$+$0237          & 53.6833                 & 0.0250                  & 3,  5,  6                                & \nodata                           & 1.49                          & 23.72                             & 23.00                             & $-$0.43                         & 19.9                      & 9.8                       & 29.3                         & 1.48E+22                                & 1                        \\
                    1874                    & 05368$+$0040$+$0231          & 53.6833                 & 0.4000                  & 5,  6                                    & \nodata                           & 1.45                          & 23.07                             & 22.66                             & $-$0.51                         & 4.1                       & 1.7                       & 9.8                          & 6.37E+20                                & 1                        \\
                    1875                    & 05369$-$0093$+$0243          & 53.6917                 & $-$0.9333                 & 5,  6                                    & \nodata                           & 1.57                          & 24.33                             & 24.03                             & $-$0.33                         & 5.7                       & 1.9                       & 11.2                         & 8.33E+20                                & 1                        \\
                    1876                    & 05371$+$0047$+$0234          & 53.7083                 & 0.4667                  & 5,  6                                    & \nodata                           & 1.49                          & 23.37                             & 22.33                             & $-$0.84                         & 6.1                       & 3.8                       & 12.7                         & 2.65E+21                                & 1                        \\
                    1877                    & 05373$+$0043$+$0231          & 53.7333                 & 0.4333                  & 5,  6                                    & \nodata                           & 1.45                          & 23.15                             & 22.37                             & $-$0.50                         & 5.6                       & 3.2                       & 15.6                         & 2.76E+21                                & 1                        \\
                    1878                    & 05374$+$0002$+$0239          & 53.7417                 & 0.0250                  & 5,  6                                    & \nodata                           & 1.52                          & 23.94                             & 22.75                             & $-$0.55                         & 11.9                      & 5.3                       & 17.8                         & 7.38E+21                                & 1                        \\
                    1879                    & 05375$+$0040$+$0228          & 53.7500                 & 0.4000                  & 5,  6                                    & \nodata                           & 1.43                          & 22.79                             & 22.11                             & $-$0.78                         & 6.4                       & 1.9                       & 11.9                         & 7.93E+20                                & 1                        \\
                    1880                    & 05376$-$0099$+$0241          & 53.7583                 & $-$0.9917                 & 5,  6                                    & \nodata                           & 1.55                          & 24.12                             & 23.85                             & $-$0.33                         & 5.5                       & 1.5                       & 11.4                         & 5.82E+20                                & 1                        \\
                    1881                    & 05378$+$0045$+$0232          & 53.7833                 & 0.4500                  & 5,  6                                    & \nodata                           & 1.48                          & 23.25                             & 22.71                             & $-$0.48                         & 6.0                       & 2.9                       & 14.4                         & 1.76E+21                                & 1                        \\
                    1882                    & 05379$+$0004$+$0238          & 53.7917                 & 0.0417                  & 5,  6                                    & \nodata                           & 1.51                          & 23.84                             & 22.86                             & $-$0.48                         & 10.8                      & 4.2                       & 19.7                         & 5.43E+21                                & 1                        \\
                    1883                    & 05381$+$0070$+$0246          & 53.8083                 & 0.7000                  & 1,  5,  6                                & \nodata                           & 1.55                          & 24.63                             & 24.10                             & $-$0.43                         & 6.5                       & 2.3                       & 11.8                         & 1.39E+21                                & 1                        \\
                    1884                    & 05384$+$0070$+$0250          & 53.8417                 & 0.7000                  & 1,  5,  6                                & \nodata                           & 1.57                          & 25.05                             & 24.26                             & $-$0.54                         & 7.1                       & 3.2                       & 14.4                         & 2.48E+21                                & 1                        \\
                    1885                    & 05386$+$0000$+$0240          & 53.8583                 & 0.0000                  & 5,  6                                    & \nodata                           & 1.54                          & 23.98                             & 23.51                             & $-$0.45                         & 3.8                       & 1.0                       & 9.1                          & 3.58E+21                                & 2                        \\
                    1886                    & 05386$-$0005$+$0233          & 53.8583                 & $-$0.0500                 & 5,  6                                    & \nodata                           & 1.47                          & 23.31                             & 22.73                             & $-$0.41                         & 8.6                       & 3.5                       & 15.0                         & 2.80E+21                                & 1                        \\
                    1887                    & 05388$-$0003$+$0238          & 53.8833                 & $-$0.0333                 & 5,  6                                    & \nodata                           & 1.52                          & 23.81                             & 23.03                             & $-$0.38                         & 7.7                       & 2.8                       & 18.1                         & 3.33E+21                                & 1                        \\
                    1888                    & 05391$-$0002$+$0234          & 53.9083                 & $-$0.0250                 & 5,  6                                    & \nodata                           & 1.49                          & 23.41                             & 22.92                             & $-$0.40                         & 8.0                       & 3.6                       & 15.1                         & 2.48E+21                                & 1                        \\
                    1889                    & 05393$-$0002$+$0235          & 53.9333                 & $-$0.0167                 & 3,  5,  6                                & \nodata                           & 1.49                          & 23.47                             & 22.89                             & $-$0.50                         & 6.0                       & 3.0                       & 12.9                         & 1.80E+21                                & 1                        \\
                    1890                    & 05396$+$0312$+$0088          & 53.9583                 & 3.1167                  & 5,  6                                    & \nodata                           & 0.32                          & 8.77                              & 8.32                              & $-$0.37                         & 3.1                       & 1.9                       & 7.3                          & 1.18E+21                                & 1                        \\
                    1891                    & 05396$-$0010$+$0233          & 53.9583                 & $-$0.1000                 & 3,  5,  6                                & \nodata                           & 1.48                          & 23.26                             & 23.04                             & $-$0.15                         & 6.1                       & 3.0                       & 14.9                         & 2.22E+21                                & 1                        \\
                    1892                    & 05397$-$0241$+$0174          & 53.9667                 & $-$2.4083                 & 5,  6                                    & \nodata                           & 1.00                          & 17.41                             & 17.16                             & $-$0.26                         & 6.5                       & 3.0                       & 13.2                         & 1.44E+21                                & 1                        \\
                    1893                    & 05398$+$0312$+$0087          & 53.9833                 & 3.1167                  & 5,  6                                    & \nodata                           & 0.30                          & 8.69                              & 8.09                              & $-$0.52                         & 3.1                       & 2.0                       & 7.7                          & 1.21E+21                                & 1                        \\
                    1894                    & 05402$-$0009$+$0233          & 54.0167                 & $-$0.0917                 & 3,  5,  6                                & \nodata                           & 1.43                          & 23.28                             & 22.86                             & $-$0.41                         & 5.2                       & 1.9                       & 10.0                         & 9.09E+20                                & 1                        \\
                    1895                    & 05403$-$0232$+$0176          & 54.0333                 & $-$2.3250                 & 5,  6                                    & \nodata                           & 1.00                          & 17.59                             & 17.28                             & $-$0.30                         & 4.7                       & 3.6                       & 11.2                         & 2.09E+21                                & 1                        \\
                    1896                    & 05404$-$0237$+$0174          & 54.0417                 & $-$2.3750                 & 5,  6                                    & \nodata                           & 1.00                          & 17.42                             & 17.15                             & $-$0.33                         & 4.6                       & 3.9                       & 12.7                         & 1.77E+21                                & 1                        \\
                    1897                    & 05406$-$0006$+$0234          & 54.0583                 & $-$0.0583                 & 3,  5,  6                                & \nodata                           & 1.44                          & 23.41                             & 23.03                             & $-$0.47                         & 6.8                       & 1.4                       & 13.1                         & 5.43E+20                                & 1                        \\
                    1898                    & 05406$-$0011$+$0233          & 54.0583                 & $-$0.1083                 & 3,  5,  6                                & \nodata                           & 1.45                          & 23.30                             & 22.89                             & $-$0.40                         & 4.8                       & 1.6                       & 10.6                         & 7.66E+20                                & 1                        \\
                    1899                    & 05417$-$0113$+$0242          & 54.1667                 & $-$1.1333                 & 5,  6                                    & \nodata                           & 1.55                          & 24.25                             & 23.96                             & $-$0.25                         & 5.1                       & 1.0                       & 11.9                         & 5.13E+20                                & 1                        \\
                    1900                    & 05432$-$0117$+$0245          & 54.3167                 & $-$1.1667                 & 5,  6                                    & \nodata                           & 1.56                          & 24.47                             & 23.84                             & $-$0.73                         & 5.8                       & 1.6                       & 10.5                         & 6.27E+20                                & 1                        \\
                    1901                    & 05437$-$0123$+$0238          & 54.3750                 & $-$1.2333                 & 1,  5,  6                                & \nodata                           & 1.51                          & 23.81                             & 23.50                             & $-$0.42                         & 5.6                       & 1.1                       & 9.8                          & 3.44E+20                                & 1                        \\
                    1902                    & 05438$-$0052$+$0343          & 54.3833                 & $-$0.5167                 & 5,  6                                    & \nodata                           & 4.27                          & 34.29                             & 34.04                             & $-$0.26                         & 3.4                       & 1.2                       & 7.1                          & 4.02E+21                                & 2                        \\
                    1903                    & 05439$+$0424$+$0093          & 54.3917                 & 4.2417                  & 5,  6                                    & \nodata                           & 0.37                          & 9.33                              & 9.04                              & $-$0.25                         & 4.8                       & 1.4                       & 11.6                         & 7.23E+20                                & 1                        \\
                    1904                    & 05450$+$0098$+$0074          & 54.5000                 & 0.9833                  & 5,  6                                    & \nodata                           & 0.23                          & 7.44                              & 6.38                              & $-$1.05                         & 3.2                       & 1.1                       & 7.7                          & 5.26E+20                                & 1                        \\
                    1905                    & 05454$+$0280$+$0077          & 54.5417                 & 2.8000                  & 5,  6                                    & \nodata                           & 0.22                          & 7.67                              & 6.96                              & $-$0.90                         & 2.9                       & 3.3                       & 6.3                          & \nodata                                 & 1                        \\
                    1906                    & 05458$+$0282$+$0077          & 54.5833                 & 2.8250                  & 5,  6                                    & \nodata                           & 0.22                          & 7.67                              & 6.82                              & $-$0.85                         & 3.0                       & 2.5                       & 12.0                         & 1.23E+21                                & 1                        \\
                    1907                    & 05462$+$0276$+$0076          & 54.6250                 & 2.7583                  & 5,  6                                    & \nodata                           & 0.20                          & 7.55                              & 7.30                              & $-$0.33                         & 2.5                       & 2.3                       & 7.2                          & 1.06E+21                                & 1                        \\
                    1908                    & 05462$+$0281$+$0075          & 54.6250                 & 2.8083                  & 5,  6                                    & \nodata                           & 0.20                          & 7.47                              & 6.93                              & $-$0.60                         & 2.5                       & 3.2                       & 7.2                          & 2.34E+21                                & 1                        \\
                    1909                    & 05466$+$0294$+$0078          & 54.6583                 & 2.9417                  & 5,  6                                    & \nodata                           & 0.23                          & 7.84                              & 7.14                              & $-$0.74                         & 4.2                       & 2.5                       & 8.9                          & 1.27E+21                                & 1                        \\
                    1910                    & 05470$+$0300$+$0079          & 54.7000                 & 3.0000                  & 5,  6                                    & \nodata                           & 0.23                          & 7.87                              & 7.31                              & $-$0.42                         & 5.0                       & 2.3                       & 8.2                          & 1.57E+21                                & 1                        \\
                    1911                    & 05472$+$0284$+$0076          & 54.7167                 & 2.8417                  & 5,  6                                    & \nodata                           & 0.21                          & 7.64                              & 6.92                              & $-$0.58                         & 3.5                       & 2.1                       & 8.0                          & 1.33E+21                                & 1                        \\
                    1912                    & 05473$+$0288$+$0077          & 54.7333                 & 2.8833                  & 5,  6                                    & \nodata                           & 0.21                          & 7.68                              & 7.02                              & $-$0.61                         & 3.5                       & 2.2                       & 10.1                         & 1.15E+21                                & 1                        \\
                    1913                    & 05473$+$0299$+$0079          & 54.7333                 & 2.9917                  & 5,  6                                    & \nodata                           & 0.23                          & 7.87                              & 7.10                              & $-$0.34                         & 6.1                       & 1.9                       & 12.2                         & 2.09E+21                                & 1                        \\
                    1914                    & 05485$+$0255$+$0082          & 54.8500                 & 2.5500                  & 5,  6                                    & \nodata                           & 0.25                          & 8.19                              & 7.75                              & $-$0.54                         & 2.5                       & 1.5                       & 9.9                          & 5.29E+20                                & 1                        \\
                    1915                    & 05485$-$0198$+$0165          & 54.8500                 & $-$1.9833                 & 6                                        & \nodata                           & 0.96                          & 16.54                             & 16.01                             & $-$0.53                         & 5.1                       & 1.5                       & 11.3                         & 7.09E+20                                & 1                        \\
                    1916                    & 05487$+$0357$+$0072          & 54.8667                 & 3.5750                  & 5,  6                                    & \nodata                           & 0.16                          & 7.24                              & 6.92                              & $-$0.42                         & 4.7                       & 1.6                       & 9.5                          & 5.76E+20                                & 1                        \\
                    1917                    & 05487$+$0361$+$0070          & 54.8667                 & 3.6083                  & 5,  6                                    & \nodata                           & 0.14                          & 7.05                              & 6.72                              & $-$0.33                         & 4.6                       & 2.0                       & 11.1                         & 9.11E+20                                & 1                        \\
                    1918                    & 05488$+$0319$+$0076          & 54.8833                 & 3.1917                  & 5,  6                                    & \nodata                           & 0.21                          & 7.61                              & 7.22                              & $-$0.46                         & 4.9                       & 2.0                       & 11.3                         & 7.97E+20                                & 1                        \\
                    1919                    & 05489$+$0253$+$0084          & 54.8917                 & 2.5333                  & 5,  6                                    & \nodata                           & 0.28                          & 8.42                              & 8.12                              & $-$0.37                         & 2.7                       & 1.3                       & 8.5                          & 4.91E+20                                & 1                        \\
                    1920                    & 05492$+$0357$+$0072          & 54.9250                 & 3.5750                  & 5,  6                                    & \nodata                           & 0.17                          & 7.24                              & 6.84                              & $-$0.39                         & 4.1                       & 2.4                       & 10.0                         & 1.26E+21                                & 1                        \\
                    1921                    & 05493$+$0360$+$0071          & 54.9333                 & 3.6000                  & 5,  6                                    & \nodata                           & 0.16                          & 7.15                              & 6.66                              & $-$0.46                         & 4.4                       & 2.1                       & 12.9                         & 1.08E+21                                & 1                        \\
                    1922                    & 05494$+$0321$+$0075          & 54.9417                 & 3.2083                  & 5,  6                                    & \nodata                           & 0.19                          & 7.52                              & 7.18                              & $-$0.33                         & 4.7                       & 1.7                       & 8.7                          & 8.57E+20                                & 1                        \\
                    1923                    & 05495$+$0323$+$0077          & 54.9500                 & 3.2333                  & 5,  6                                    & \nodata                           & 0.22                          & 7.72                              & 7.23                              & $-$0.27                         & 6.2                       & 1.2                       & 20.6                         & 1.23E+21                                & 1                        \\
                    1924                    & 05497$+$0362$+$0072          & 54.9750                 & 3.6167                  & 5,  6                                    & \nodata                           & 0.16                          & 7.21                              & 6.86                              & $-$0.35                         & 4.4                       & 1.7                       & 8.6                          & 7.87E+20                                & 1                        \\
                    1925                    & 05513$+$0322$+$0072          & 55.1333                 & 3.2167                  & 5,  6                                    & \nodata                           & 0.17                          & 7.22                              & 6.83                              & $-$0.33                         & 5.4                       & 1.5                       & 15.6                         & 9.11E+20                                & 1                        \\
                    1926                    & 05516$+$0316$+$0072          & 55.1583                 & 3.1583                  & 5,  6                                    & \nodata                           & 0.17                          & 7.25                              & 6.64                              & $-$0.65                         & 4.1                       & 0.9                       & 9.7                          & 3.79E+20                                & 1                        \\
                    1927                    & 05517$+$0326$+$0074          & 55.1750                 & 3.2583                  & 5,  6                                    & \nodata                           & 0.18                          & 7.42                              & 6.81                              & $-$0.53                         & 6.2                       & 1.4                       & 14.2                         & 7.93E+20                                & 1                        \\
                    1928                    & 05518$+$0321$+$0073          & 55.1833                 & 3.2083                  & 5,  6                                    & \nodata                           & 0.16                          & 7.29                              & 6.85                              & $-$0.48                         & 4.9                       & 1.1                       & 10.9                         & 4.31E+20                                & 1                        \\
                    1929                    & 05519$+$0329$+$0075          & 55.1917                 & 3.2917                  & 5,  6                                    & \nodata                           & 0.19                          & 7.55                              & 6.91                              & $-$0.49                         & 6.2                       & 1.2                       & 12.7                         & 7.64E+20                                & 1                        \\
                    1930                    & 05520$+$0324$+$0073          & 55.2000                 & 3.2417                  & 1,  5,  6                                & \nodata                           & 0.17                          & 7.28                              & 6.79                              & $-$0.53                         & 6.0                       & 1.2                       & 11.0                         & 5.04E+20                                & 1                        \\
                    1931                    & 05522$+$0323$+$0072          & 55.2250                 & 3.2333                  & 5,  6                                    & \nodata                           & 0.16                          & 7.24                              & 6.81                              & $-$0.40                         & 6.3                       & 1.2                       & 11.0                         & 5.74E+20                                & 1                        \\
                    1932                    & 05523$+$0331$+$0076          & 55.2333                 & 3.3083                  & 5,  6                                    & \nodata                           & 0.20                          & 7.58                              & 6.84                              & $-$0.57                         & 4.9                       & 1.3                       & 13.5                         & 8.06E+20                                & 1                        \\
                    1933                    & 05525$+$0322$+$0073          & 55.2500                 & 3.2250                  & 5,  6                                    & \nodata                           & 0.18                          & 7.29                              & 6.68                              & $-$0.67                         & 5.2                       & 1.1                       & 10.6                         & 4.49E+20                                & 1                        \\
                    1934                    & 05526$+$0310$+$0073          & 55.2583                 & 3.1000                  & 5,  6                                    & \nodata                           & 0.17                          & 7.28                              & 6.63                              & $-$0.40                         & 4.7                       & 1.2                       & 9.0                          & 8.61E+20                                & 1                        \\
                    1935                    & 05527$+$0102$+$0068          & 55.2667                 & 1.0167                  & 5,  6                                    & \nodata                           & 0.15                          & 6.82                              & 6.09                              & $-$0.51                         & 6.1                       & 3.3                       & 14.1                         & 2.59E+21                                & 1                        \\
                    1936                    & 05527$+$0327$+$0073          & 55.2667                 & 3.2750                  & 5,  6                                    & \nodata                           & 0.18                          & 7.30                              & 6.80                              & $-$0.41                         & 5.4                       & 1.2                       & 12.0                         & 6.84E+20                                & 1                        \\
                    1937                    & 05528$+$0334$+$0077          & 55.2833                 & 3.3417                  & 5,  6                                    & \nodata                           & 0.21                          & 7.70                              & 6.89                              & $-$0.74                         & 5.4                       & 1.1                       & 11.7                         & 5.48E+20                                & 1                        \\
                    1938                    & 05532$+$0303$+$0072          & 55.3167                 & 3.0333                  & 5,  6                                    & \nodata                           & 0.16                          & 7.17                              & 6.77                              & $-$0.31                         & 5.0                       & 2.4                       & 12.6                         & 1.53E+21                                & 1                        \\
                    1939                    & 05536$+$0309$+$0074          & 55.3583                 & 3.0917                  & 5,  6                                    & \nodata                           & 0.17                          & 7.37                              & 6.83                              & $-$0.44                         & 5.5                       & 2.1                       & 13.1                         & 1.30E+21                                & 1                        \\
                    1940                    & 05541$+$0301$+$0072          & 55.4083                 & 3.0083                  & 5,  6                                    & \nodata                           & 0.16                          & 7.19                              & 6.65                              & $-$0.34                         & 6.4                       & 1.6                       & 20.2                         & 1.51E+21                                & 1                        \\
                    1941                    & 05544$+$0302$+$0073          & 55.4417                 & 3.0250                  & 5,  6                                    & \nodata                           & 0.17                          & 7.25                              & 6.67                              & $-$0.59                         & 4.6                       & 1.8                       & 9.8                          & 8.67E+20                                & 1                        \\
                    1942                    & 05545$+$0305$+$0073          & 55.4500                 & 3.0500                  & 5,  6                                    & \nodata                           & 0.17                          & 7.33                              & 6.71                              & $-$0.62                         & 4.9                       & 1.6                       & 9.5                          & 7.23E+20                                & 1                        \\
                    1943                    & 05579$+$0036$+$0339          & 55.7917                 & 0.3583                  & 6                                        & \nodata                           & 4.04                          & 33.88                             & 33.25                             & $-$0.56                         & 5.7                       & 2.7                       & 10.5                         & 1.58E+21                                & 1                        \\
                    1944                    & 05580$+$0332$+$0078          & 55.8000                 & 3.3167                  & 5,  6                                    & \nodata                           & 0.22                          & 7.77                              & 7.30                              & $-$0.28                         & 6.3                       & 1.4                       & 14.4                         & 1.12E+21                                & 1                        \\
                    1945                    & 05582$+$0330$+$0077          & 55.8250                 & 3.3000                  & 5,  6                                    & \nodata                           & 0.21                          & 7.67                              & 7.33                              & $-$0.20                         & 6.7                       & 1.8                       & 15.6                         & 1.52E+21                                & 1                        \\
                    1946                    & 05584$+$0316$+$0082          & 55.8417                 & 3.1583                  & 5,  6                                    & \nodata                           & 0.27                          & 8.19                              & 7.50                              & $-$0.56                         & 5.7                       & 1.1                       & 11.1                         & 6.26E+20                                & 1                        \\
                    1947                    & 05586$+$0312$+$0080          & 55.8583                 & 3.1250                  & 5,  6                                    & \nodata                           & 0.26                          & 8.04                              & 7.54                              & $-$0.50                         & 5.3                       & 1.3                       & 11.0                         & 6.18E+20                                & 1                        \\
                    1948                    & 05592$+$0202$+$0099          & 55.9167                 & 2.0167                  & 5,  6                                    & \nodata                           & 0.43                          & 9.92                              & 9.48                              & $-$0.58                         & 6.9                       & 4.2                       & 12.2                         & 1.82E+21                                & 1                        \\
                    1949                    & 05592$+$0332$+$0080          & 55.9167                 & 3.3167                  & 5,  6                                    & \nodata                           & 0.24                          & 8.05                              & 7.61                              & $-$0.51                         & 4.2                       & 1.7                       & 9.7                          & 6.76E+20                                & 1                        \\
                    1950                    & 05597$+$0206$+$0098          & 55.9667                 & 2.0583                  & 5,  6                                    & \nodata                           & 0.42                          & 9.77                              & 9.54                              & $-$0.29                         & 5.4                       & 1.6                       & 10.5                         & 6.01E+20                                & 1                        \\
                    1951                    & 05600$+$0043$+$0325          & 56.0000                 & 0.4333                  & 6                                        & \nodata                           & 1.88                          & 32.51                             & 31.73                             & $-$0.67                         & 4.4                       & 1.3                       & 11.3                         & 7.17E+20                                & 1                        \\
                    1952                    & 05600$+$0048$+$0319          & 56.0000                 & 0.4833                  & 1,  6                                    & \nodata                           & 3.38                          & 31.88                             & 31.06                             & $-$0.49                         & 4.3                       & 1.9                       & 8.6                          & 1.52E+21                                & 1                        \\
                    1953                    & 05604$-$0292$+$0153          & 56.0417                 & $-$2.9250                 & 6                                        & \nodata                           & 1.03                          & 15.32                             & 14.95                             & $-$0.37                         & 5.9                       & 2.3                       & 14.5                         & 1.18E+21                                & 1                        \\
                    1954                    & 05636$-$0052$+$0319          & 56.3583                 & $-$0.5250                 & 6                                        & \nodata                           & 1.89                          & 31.86                             & 30.53                             & $-$1.05                         & 5.2                       & 2.5                       & 9.2                          & 1.61E+21                                & 1                        \\
                    1955                    & 05657$+$0040$+$0087          & 56.5667                 & 0.4000                  & 6                                        & \nodata                           & 2.16                          & 8.65                              & 8.11                              & $-$0.40                         & 3.6                       & 1.9                       & 9.1                          & 1.23E+21                                & 1                        \\
                    1956                    & 05657$+$0052$+$0080          & 56.5667                 & 0.5167                  & 6                                        & \nodata                           & 2.16                          & 8.02                              & 7.55                              & $-$0.48                         & 3.3                       & 1.2                       & 6.9                          & 5.60E+20                                & 1                        \\
                    1957                    & 05672$+$0482$+$0106          & 56.7250                 & 4.8250                  & 5,  6                                    & \nodata                           & 0.50                          & 10.64                             & 10.34                             & $-$0.38                         & 4.2                       & 1.1                       & 9.8                          & 3.80E+20                                & 1                        \\
                    1958                    & 05676$+$0338$+$0114          & 56.7583                 & 3.3833                  & 5,  6                                    & \nodata                           & 0.57                          & 11.40                             & 11.10                             & $-$0.28                         & 7.2                       & 3.9                       & 14.6                         & 2.45E+21                                & 1                        \\
                    1959                    & 05677$+$0477$+$0108          & 56.7750                 & 4.7667                  & 5,  6                                    & \nodata                           & 0.52                          & 10.84                             & 10.57                             & $-$0.23                         & 6.6                       & 2.0                       & 15.0                         & 1.22E+21                                & 1                        \\
                    1960                    & 05682$+$0329$+$0111          & 56.8250                 & 3.2917                  & 5,  6                                    & \nodata                           & 0.53                          & 11.08                             & 10.63                             & $-$0.30                         & 12.5                      & 3.8                       & 18.7                         & 3.42E+21                                & 1                        \\
                    1961                    & 05684$+$0332$+$0112          & 56.8417                 & 3.3250                  & 5,  6                                    & \nodata                           & 0.56                          & 11.21                             & 10.78                             & $-$0.32                         & 9.4                       & 3.8                       & 16.1                         & 2.99E+21                                & 1                        \\
                    1962                    & 05685$+$0345$+$0116          & 56.8500                 & 3.4500                  & 5,  6                                    & \nodata                           & 0.59                          & 11.64                             & 11.07                             & $-$0.51                         & 7.4                       & 3.5                       & 14.5                         & 2.18E+21                                & 1                        \\
                    1963                    & 05687$+$0313$+$0112          & 56.8750                 & 3.1333                  & 5,  6                                    & \nodata                           & 0.55                          & 11.22                             & 10.74                             & $-$0.36                         & 7.8                       & 2.9                       & 15.4                         & 2.08E+21                                & 1                        \\
                    1964                    & 05687$+$0317$+$0112          & 56.8750                 & 3.1667                  & 5,  6                                    & \nodata                           & 0.55                          & 11.21                             & 10.58                             & $-$0.44                         & 7.6                       & 2.7                       & 15.3                         & 2.09E+21                                & 1                        \\
                    1965                    & 05687$+$0322$+$0113          & 56.8750                 & 3.2167                  & 5,  6                                    & \nodata                           & 0.56                          & 11.28                             & 10.70                             & $-$0.41                         & 7.5                       & 3.1                       & 14.5                         & 2.34E+21                                & 1                        \\
                    1966                    & 05687$+$0344$+$0115          & 56.8750                 & 3.4417                  & 5,  6                                    & \nodata                           & 0.58                          & 11.50                             & 10.92                             & $-$0.41                         & 11.0                      & 3.7                       & 18.9                         & 3.21E+21                                & 1                        \\
                    1967                    & 05687$+$0347$+$0115          & 56.8750                 & 3.4667                  & 5,  6                                    & \nodata                           & 0.58                          & 11.52                             & 11.00                             & $-$0.39                         & 8.4                       & 3.8                       & 15.2                         & 2.86E+21                                & 1                        \\
                    1968                    & 05688$+$0483$+$0112          & 56.8833                 & 4.8333                  & 5,  6                                    & \nodata                           & 0.54                          & 11.22                             & 10.90                             & $-$0.34                         & 7.6                       & 3.9                       & 15.8                         & 2.13E+21                                & 1                        \\
                    1969                    & 05690$+$0478$+$0113          & 56.9000                 & 4.7833                  & 5,  6                                    & \nodata                           & 0.55                          & 11.32                             & 10.91                             & $-$0.54                         & 6.4                       & 2.2                       & 11.9                         & 8.25E+20                                & 1                        \\
                    1970                    & 05691$+$0347$+$0114          & 56.9083                 & 3.4750                  & 5,  6                                    & \nodata                           & 0.56                          & 11.36                             & 10.90                             & $-$0.36                         & 10.3                      & 4.1                       & 22.8                         & 3.51E+21                                & 1                        \\
                    1971                    & 05691$+$0350$+$0113          & 56.9083                 & 3.5000                  & 5,  6                                    & \nodata                           & 0.56                          & 11.32                             & 10.91                             & $-$0.34                         & 9.4                       & 3.8                       & 22.3                         & 3.06E+21                                & 1                        \\
                    1972                    & 05692$+$0475$+$0111          & 56.9167                 & 4.7500                  & 5,  6                                    & \nodata                           & 0.54                          & 11.14                             & 10.80                             & $-$0.37                         & 6.5                       & 2.4                       & 12.3                         & 1.09E+21                                & 1                        \\
                    1973                    & 05692$+$0344$+$0113          & 56.9250                 & 3.4417                  & 5,  6                                    & \nodata                           & 0.55                          & 11.27                             & 10.65                             & $-$0.47                         & 11.1                      & 4.7                       & 20.8                         & 4.06E+21                                & 1                        \\
                    1974                    & 05692$+$0354$+$0114          & 56.9250                 & 3.5417                  & 5,  6                                    & \nodata                           & 0.56                          & 11.38                             & 10.93                             & $-$0.47                         & 10.5                      & 2.6                       & 17.6                         & 1.48E+21                                & 1                        \\
                    1975                    & 05693$+$0330$+$0112          & 56.9333                 & 3.3000                  & 5,  6                                    & \nodata                           & 0.56                          & 11.20                             & 10.82                             & $-$0.35                         & 9.0                       & 4.0                       & 16.5                         & 2.56E+21                                & 1                        \\
                    1976                    & 05697$+$0362$+$0115          & 56.9667                 & 3.6167                  & 5,  6                                    & \nodata                           & 0.58                          & 11.53                             & 10.98                             & $-$0.41                         & 10.2                      & 2.7                       & 20.0                         & 2.21E+21                                & 1                        \\
                    1977                    & 05699$+$0361$+$0115          & 56.9917                 & 3.6083                  & 6                                        & \nodata                           & 0.58                          & 11.46                             & 10.75                             & $-$0.35                         & 10.8                      & 3.3                       & 17.2                         & 3.90E+21                                & 1                        \\
                    1978                    & 05700$+$0332$+$0114          & 57.0000                 & 3.3250                  & 5,  6                                    & \nodata                           & 0.57                          & 11.39                             & 10.95                             & $-$0.36                         & 9.4                       & 3.3                       & 16.6                         & 2.32E+21                                & 1                        \\
                    1979                    & 05702$+$0337$+$0114          & 57.0167                 & 3.3667                  & 5,  6                                    & \nodata                           & 0.58                          & 11.41                             & 11.06                             & $-$0.35                         & 8.2                       & 2.1                       & 16.3                         & 1.13E+21                                & 1                        \\
                    1980                    & 05702$+$0330$+$0114          & 57.0250                 & 3.3000                  & 5,  6                                    & \nodata                           & 0.58                          & 11.40                             & 10.83                             & $-$0.45                         & 9.8                       & 2.2                       & 16.6                         & 1.55E+21                                & 1                        \\
                    1981                    & 05705$+$0326$+$0113          & 57.0500                 & 3.2583                  & 5,  6                                    & \nodata                           & 0.57                          & 11.34                             & 10.89                             & $-$0.45                         & 8.3                       & 3.3                       & 16.8                         & 1.91E+21                                & 1                        \\
                    1982                    & 05705$+$0337$+$0116          & 57.0500                 & 3.3667                  & 5,  6                                    & \nodata                           & 0.58                          & 11.55                             & 10.89                             & $-$0.57                         & 7.8                       & 2.5                       & 16.2                         & 1.57E+21                                & 1                        \\
                    1983                    & 05706$+$0342$+$0115          & 57.0583                 & 3.4167                  & 5,  6                                    & \nodata                           & 0.59                          & 11.55                             & 11.10                             & $-$0.39                         & 8.5                       & 2.6                       & 14.6                         & 1.55E+21                                & 1                        \\
                    1984                    & 05707$+$0025$+$0098          & 57.0667                 & 0.2500                  & 6                                        & \nodata                           & 0.47                          & 9.81                              & 8.41                              & $-$0.78                         & 3.7                       & 2.3                       & 8.0                          & 2.23E+21                                & 1                        \\
                    1985                    & 05707$+$0275$+$0119          & 57.0750                 & 2.7500                  & 6                                        & \nodata                           & 0.62                          & 11.95                             & 11.40                             & $-$0.40                         & 9.0                       & 3.1                       & 14.5                         & 2.33E+21                                & 1                        \\
                    1986                    & 05709$+$0133$+$0099          & 57.0917                 & 1.3333                  & 5,  6                                    & \nodata                           & 0.46                          & 9.94                              & 9.29                              & $-$0.36                         & 4.8                       & 1.7                       & 14.4                         & 1.52E+21                                & 1                        \\
                    1987                    & 05709$+$0281$+$0119          & 57.0917                 & 2.8083                  & 6                                        & \nodata                           & 0.61                          & 11.85                             & 11.09                             & $-$0.74                         & 8.6                       & 5.7                       & 13.9                         & 3.87E+21                                & 1                        \\
                    1988                    & 05709$+$0285$+$0118          & 57.0917                 & 2.8500                  & 6                                        & \nodata                           & 0.60                          & 11.77                             & 11.29                             & $-$0.38                         & 11.5                      & 6.1                       & 17.3                         & 5.11E+21                                & 1                        \\
                    1989                    & 05709$+$0363$+$0117          & 57.0917                 & 3.6333                  & 5,  6                                    & \nodata                           & 0.61                          & 11.67                             & 11.24                             & $-$0.44                         & 5.9                       & 1.9                       & 11.7                         & 6.97E+21                                & 2                        \\
                    1990                    & 05711$+$0317$+$0113          & 57.1083                 & 3.1667                  & 5,  6                                    & \nodata                           & 0.56                          & 11.29                             & 10.75                             & $-$0.53                         & 8.0                       & 3.3                       & 15.2                         & 1.92E+21                                & 1                        \\
                    1991                    & 05714$+$0369$+$0113          & 57.1417                 & 3.6917                  & 6                                        & \nodata                           & 0.57                          & 11.28                             & 10.70                             & $-$0.39                         & 9.4                       & 4.8                       & 17.2                         & 4.51E+21                                & 1                        \\
                    1992                    & 05717$+$0212$+$0117          & 57.1750                 & 2.1250                  & 6                                        & \nodata                           & 0.59                          & 11.70                             & 11.05                             & $-$0.67                         & 4.8                       & 2.1                       & 11.1                         & 9.88E+20                                & 1                        \\
                    1993                    & 05719$+$0203$+$0114          & 57.1917                 & 2.0333                  & 6                                        & \nodata                           & 0.57                          & 11.37                             & 10.93                             & $-$0.50                         & 4.6                       & 1.7                       & 9.4                          & 6.90E+20                                & 1                        \\
                    1994                    & 05719$+$0216$+$0117          & 57.1917                 & 2.1583                  & 6                                        & \nodata                           & 0.60                          & 11.72                             & 11.21                             & $-$0.47                         & 5.5                       & 1.5                       & 11.9                         & 7.60E+20                                & 1                        \\
                    1995                    & 05719$+$0396$+$0110          & 57.1917                 & 3.9583                  & 6                                        & \nodata                           & 0.54                          & 11.00                             & 10.64                             & $-$0.41                         & 7.9                       & 1.9                       & 14.9                         & 8.39E+20                                & 1                        \\
                    1996                    & 05720$+$0222$+$0116          & 57.2000                 & 2.2167                  & 6                                        & \nodata                           & 0.59                          & 11.63                             & 11.19                             & $-$0.45                         & 6.5                       & 2.6                       & 14.3                         & 1.35E+21                                & 1                        \\
                    1997                    & 05721$+$0007$+$0245          & 57.2083                 & 0.0667                  & 1,  6                                    & \nodata                           & 2.16                          & 24.45                             & 23.67                             & $-$0.32                         & 7.2                       & 4.3                       & 13.0                         & 6.18E+21                                & 1                        \\
                    1998                    & 05721$+$0334$+$0110          & 57.2083                 & 3.3417                  & 5,  6                                    & \nodata                           & 0.54                          & 11.05                             & 10.18                             & $-$0.52                         & 7.4                       & 2.6                       & 16.4                         & 2.43E+21                                & 1                        \\
                    1999                    & 05722$+$0161$+$0109          & 57.2250                 & 1.6083                  & 5,  6                                    & \nodata                           & 0.53                          & 10.88                             & 10.28                             & $-$0.67                         & 6.6                       & 2.2                       & 16.0                         & 1.06E+21                                & 1                        \\
                    2000                    & 05723$+$0106$+$0101          & 57.2333                 & 1.0583                  & 5,  6                                    & \nodata                           & 0.50                          & 10.09                             & 9.73                              & $-$0.36                         & 4.0                       & 1.3                       & 8.3                          & 5.62E+20                                & 1                        \\
                    2001                    & 05724$+$0232$+$0116          & 57.2417                 & 2.3167                  & 6                                        & \nodata                           & 0.59                          & 11.56                             & 11.18                             & $-$0.33                         & 4.8                       & 2.1                       & 10.5                         & 1.13E+21                                & 1                        \\
                    2002                    & 05724$+$0342$+$0106          & 57.2417                 & 3.4250                  & 5,  6                                    & \nodata                           & 0.51                          & 10.65                             & 10.17                             & $-$0.35                         & 6.7                       & 2.8                       & 14.9                         & 2.02E+21                                & 1                        \\
                    2003                    & 05725$+$0338$+$0105          & 57.2500                 & 3.3833                  & 5,  6                                    & \nodata                           & 0.50                          & 10.48                             & 10.03                             & $-$0.41                         & 6.5                       & 2.5                       & 13.3                         & 1.44E+21                                & 1                        \\
                    2004                    & 05725$-$0002$+$0092          & 57.2500                 & $-$0.0250                 & 6                                        & \nodata                           & 0.43                          & 9.18                              & 8.48                              & $-$0.57                         & 4.0                       & 1.0                       & 7.8                          & 5.66E+20                                & 1                        \\
                    2005                    & 05727$+$0232$+$0115          & 57.2667                 & 2.3167                  & 6                                        & \nodata                           & 0.58                          & 11.48                             & 11.07                             & $-$0.56                         & 5.5                       & 2.5                       & 10.4                         & 8.93E+20                                & 1                        \\
                    2006                    & 05727$+$0340$+$0105          & 57.2750                 & 3.4000                  & 5,  6                                    & \nodata                           & 0.50                          & 10.53                             & 9.89                              & $-$0.60                         & 7.1                       & 1.6                       & 12.8                         & 8.33E+20                                & 1                        \\
                    2007                    & 05728$+$0387$+$0115          & 57.2833                 & 3.8750                  & 5,  6                                    & \nodata                           & 0.59                          & 11.51                             & 11.09                             & $-$0.48                         & 9.6                       & 3.0                       & 15.0                         & 1.45E+21                                & 1                        \\
                    2008                    & 05729$+$0012$+$0098          & 57.2917                 & 0.1167                  & 6                                        & \nodata                           & 0.47                          & 9.84                              & 8.53                              & $-$0.88                         & 3.9                       & 1.6                       & 13.8                         & 1.17E+21                                & 1                        \\
                    2009                    & 05730$+$0394$+$0111          & 57.3000                 & 3.9417                  & 6                                        & \nodata                           & 0.55                          & 11.11                             & 10.74                             & $-$0.41                         & 9.4                       & 4.0                       & 15.8                         & 2.12E+21                                & 1                        \\
                    2010                    & 05732$+$0231$+$0116          & 57.3167                 & 2.3083                  & 6                                        & \nodata                           & 0.59                          & 11.58                             & 11.09                             & $-$0.54                         & 7.2                       & 2.5                       & 14.0                         & 1.16E+21                                & 1                        \\
                    2011                    & 05733$+$0292$+$0116          & 57.3333                 & 2.9167                  & 6                                        & \nodata                           & 0.59                          & 11.60                             & 10.96                             & $-$0.41                         & 6.8                       & 1.6                       & 12.8                         & 1.16E+21                                & 1                        \\
                    2012                    & 05733$-$0032$+$0368          & 57.3333                 & $-$0.3250                 & 6                                        & \nodata                           & 1.98                          & 36.75                             & 35.75                             & $-$0.48                         & 4.9                       & 2.3                       & 9.7                          & 2.37E+21                                & 1                        \\
                    2013                    & 05734$+$0224$+$0113          & 57.3417                 & 2.2417                  & 6                                        & \nodata                           & 0.57                          & 11.33                             & 10.73                             & $-$0.51                         & 6.3                       & 2.3                       & 15.3                         & 1.46E+21                                & 1                        \\
                    2014                    & 05735$+$0238$+$0116          & 57.3500                 & 2.3833                  & 6                                        & \nodata                           & 0.60                          & 11.62                             & 11.27                             & $-$0.32                         & 6.3                       & 3.3                       & 13.7                         & 1.96E+21                                & 1                        \\
                    2015                    & 05736$+$0227$+$0113          & 57.3583                 & 2.2667                  & 6                                        & \nodata                           & 0.57                          & 11.32                             & 10.72                             & $-$0.55                         & 6.6                       & 2.4                       & 15.6                         & 1.41E+21                                & 1                        \\
                    2016                    & 05736$+$0308$+$0111          & 57.3583                 & 3.0833                  & 6                                        & \nodata                           & 0.55                          & 11.14                             & 10.51                             & $-$0.52                         & 6.7                       & 1.2                       & 15.8                         & 7.57E+20                                & 1                        \\
                    2017                    & 05736$+$0312$+$0111          & 57.3583                 & 3.1167                  & 6                                        & \nodata                           & 0.55                          & 11.13                             & 10.44                             & $-$0.64                         & 6.6                       & 1.4                       & 12.4                         & 7.08E+20                                & 1                        \\
                    2018                    & 05737$+$0231$+$0115          & 57.3667                 & 2.3083                  & 6                                        & \nodata                           & 0.58                          & 11.47                             & 11.17                             & $-$0.30                         & 6.5                       & 2.2                       & 11.6                         & 1.08E+21                                & 1                        \\
                    2019                    & 05737$+$0243$+$0117          & 57.3667                 & 2.4333                  & 6                                        & \nodata                           & 0.60                          & 11.68                             & 11.27                             & $-$0.38                         & 5.4                       & 3.7                       & 12.2                         & 2.21E+21                                & 1                        \\
                    2020                    & 05738$+$0392$+$0114          & 57.3833                 & 3.9167                  & 5,  6                                    & \nodata                           & 0.58                          & 11.37                             & 10.93                             & $-$0.48                         & 8.9                       & 2.3                       & 16.1                         & 1.14E+21                                & 1                        \\
                    2021                    & 05738$-$0007$+$0086          & 57.3833                 & $-$0.0750                 & 6                                        & \nodata                           & 2.16                          & 8.59                              & 8.02                              & $-$0.55                         & 3.5                       & 1.4                       & 7.7                          & 6.98E+20                                & 1                        \\
                    2022                    & 05739$+$0226$+$0113          & 57.3917                 & 2.2583                  & 6                                        & \nodata                           & 0.57                          & 11.33                             & 10.82                             & $-$0.53                         & 5.7                       & 2.0                       & 12.9                         & 9.45E+20                                & 1                        \\
                    2023                    & 05739$+$0235$+$0115          & 57.3917                 & 2.3500                  & 6                                        & \nodata                           & 0.58                          & 11.55                             & 11.06                             & $-$0.44                         & 6.2                       & 2.4                       & 14.7                         & 1.40E+21                                & 1                        \\
                    2024                    & 05739$+$0243$+$0117          & 57.3917                 & 2.4333                  & 6                                        & \nodata                           & 0.60                          & 11.71                             & 11.27                             & $-$0.52                         & 5.3                       & 2.9                       & 11.3                         & 1.30E+21                                & 1                        \\
                    2025                    & 05742$+$0242$+$0117          & 57.4167                 & 2.4167                  & 6                                        & \nodata                           & 0.59                          & 11.66                             & 11.25                             & $-$0.48                         & 4.5                       & 2.9                       & 9.4                          & 1.34E+21                                & 1                        \\
                    2026                    & 05752$+$0057$+$0087          & 57.5250                 & 0.5667                  & 1,  5,  6                                & \nodata                           & 0.38                          & 8.75                              & 7.98                              & $-$0.90                         & 4.8                       & 2.6                       & 9.6                          & 1.14E+21                                & 1                        \\
                    2027                    & 05755$+$0060$+$0089          & 57.5500                 & 0.6000                  & 5,  6                                    & \nodata                           & 0.40                          & 8.88                              & 8.53                              & $-$0.39                         & 3.9                       & 2.2                       & 9.7                          & 1.00E+21                                & 1                        \\
                    2028                    & 05759$+$0058$+$0090          & 57.5917                 & 0.5833                  & 5,  6                                    & \nodata                           & 0.40                          & 9.00                              & 8.38                              & $-$0.74                         & 4.6                       & 2.9                       & 9.7                          & 1.28E+21                                & 1                        \\
                    2029                    & 05759$-$0041$+$0360          & 57.5917                 & $-$0.4083                 & 6                                        & \nodata                           & 2.04                          & 36.01                             & 35.38                             & $-$0.51                         & 3.4                       & 1.5                       & 7.4                          & 9.01E+20                                & 1                        \\
                    2030                    & 05760$+$0055$+$0089          & 57.6000                 & 0.5500                  & 5,  6                                    & \nodata                           & 0.38                          & 8.87                              & 8.33                              & $-$0.60                         & 4.4                       & 2.4                       & 9.0                          & 1.11E+21                                & 1                        \\
                    2031                    & 05762$+$0058$+$0090          & 57.6167                 & 0.5833                  & 5,  6                                    & \nodata                           & 0.40                          & 9.01                              & 8.38                              & $-$0.68                         & 4.5                       & 1.6                       & 9.9                          & 6.64E+20                                & 1                        \\
                    2032                    & 05764$-$0013$+$0341          & 57.6417                 & $-$0.1333                 & 6                                        & \nodata                           & 2.12                          & 34.13                             & 33.62                             & $-$0.46                         & 2.6                       & 1.5                       & 8.5                          & 7.78E+20                                & 1                        \\
                    2033                    & 05772$-$0016$+$0345          & 57.7167                 & $-$0.1583                 & 6                                        & \nodata                           & 2.06                          & 34.48                             & 33.87                             & $-$0.59                         & 2.6                       & 1.9                       & 7.6                          & 1.02E+21                                & 1                        \\
                    2034                    & 05782$+$0057$+$0093          & 57.8167                 & 0.5667                  & 6                                        & \nodata                           & 0.43                          & 9.27                              & 8.51                              & $-$0.97                         & 3.6                       & 1.0                       & 8.5                          & 3.61E+20                                & 1                        \\
                    2035                    & 05787$+$0017$+$0087          & 57.8667                 & 0.1750                  & 6                                        & \nodata                           & 2.16                          & 8.67                              & 8.18                              & $-$0.38                         & 3.4                       & 1.9                       & 8.9                          & 1.22E+21                                & 1                        \\
                    2036                    & 05794$+$0278$+$0104          & 57.9417                 & 2.7833                  & 6                                        & \nodata                           & 0.49                          & 10.37                             & 9.80                              & $-$0.55                         & 5.5                       & 1.8                       & 11.0                         & 8.53E+20                                & 1                        \\
                    2037                    & 05794$-$0024$+$0106          & 57.9417                 & $-$0.2417                 & 6                                        & \nodata                           & 2.16                          & 10.55                             & 9.99                              & $-$0.45                         & 2.8                       & 2.1                       & 6.4                          & 1.69E+21                                & 1                        \\
                    2038                    & 05799$+$0287$+$0100          & 57.9917                 & 2.8750                  & 6                                        & \nodata                           & 0.46                          & 10.00                             & 9.55                              & $-$0.46                         & 5.5                       & 2.5                       & 10.6                         & 1.20E+21                                & 1                        \\
                    2039                    & 05800$+$0320$+$0106          & 58.0000                 & 3.2000                  & 5,  6                                    & \nodata                           & 0.50                          & 10.56                             & 10.06                             & $-$0.38                         & 4.9                       & 1.7                       & 10.5                         & 1.01E+21                                & 1                        \\
                    2040                    & 05801$+$0303$+$0100          & 58.0083                 & 3.0333                  & 1,  6                                    & \nodata                           & 0.46                          & 10.01                             & 9.43                              & $-$0.64                         & 5.1                       & 3.3                       & 11.0                         & 1.63E+21                                & 1                        \\
                    2041                    & 05802$-$0027$+$0105          & 58.0167                 & $-$0.2667                 & 6                                        & \nodata                           & 2.16                          & 10.49                             & 9.45                              & $-$0.99                         & 3.6                       & 2.1                       & 11.8                         & 1.09E+21                                & 1                        \\
                    2042                    & 05802$+$0290$+$0102          & 58.0250                 & 2.9000                  & 6                                        & \nodata                           & 0.47                          & 10.19                             & 9.70                              & $-$0.43                         & 6.2                       & 2.5                       & 14.2                         & 1.50E+21                                & 1                        \\
                    2043                    & 05803$+$0276$+$0104          & 58.0333                 & 2.7583                  & 6                                        & \nodata                           & 0.49                          & 10.38                             & 9.71                              & $-$0.58                         & 5.5                       & 1.9                       & 12.7                         & 1.07E+21                                & 1                        \\
                    2044                    & 05804$+$0299$+$0102          & 58.0417                 & 2.9917                  & 6                                        & \nodata                           & 0.46                          & 10.16                             & 9.83                              & $-$0.34                         & 6.4                       & 3.9                       & 15.4                         & 2.18E+21                                & 1                        \\
                    2045                    & 05805$+$0287$+$0103          & 58.0500                 & 2.8750                  & 6                                        & \nodata                           & 0.48                          & 10.28                             & 9.67                              & $-$0.61                         & 5.2                       & 2.2                       & 11.3                         & 1.06E+21                                & 1                        \\
                    2046                    & 05806$+$0276$+$0103          & 58.0583                 & 2.7583                  & 6                                        & \nodata                           & 0.47                          & 10.29                             & 9.81                              & $-$0.44                         & 5.8                       & 1.7                       & 10.0                         & 8.73E+20                                & 1                        \\
                    2047                    & 05807$+$0297$+$0101          & 58.0667                 & 2.9667                  & 6                                        & \nodata                           & 0.46                          & 10.05                             & 9.63                              & $-$0.44                         & 6.3                       & 3.0                       & 11.7                         & 1.51E+21                                & 1                        \\
                    2048                    & 05807$+$0315$+$0099          & 58.0667                 & 3.1500                  & 5,  6                                    & \nodata                           & 0.45                          & 9.92                              & 9.37                              & $-$0.37                         & 4.9                       & 3.1                       & 13.7                         & 2.46E+21                                & 1                        \\
                    2049                    & 05807$+$0323$+$0099          & 58.0667                 & 3.2333                  & 6                                        & \nodata                           & 0.45                          & 9.89                              & 9.37                              & $-$0.53                         & 4.6                       & 3.1                       & 10.2                         & 1.63E+21                                & 1                        \\
                    2050                    & 05807$-$0029$+$0100          & 58.0667                 & $-$0.2917                 & 6                                        & \nodata                           & 2.16                          & 9.98                              & 9.30                              & $-$0.54                         & 3.1                       & 1.8                       & 7.6                          & 1.17E+21                                & 1                        \\
                    2051                    & 05807$-$0023$+$0098          & 58.0750                 & $-$0.2333                 & 6                                        & \nodata                           & 2.16                          & 9.82                              & 9.32                              & $-$0.60                         & 2.9                       & 2.2                       & 7.8                          & 1.01E+21                                & 1                        \\
                    2052                    & 05809$-$0029$+$0099          & 58.0917                 & $-$0.2917                 & 6                                        & \nodata                           & 2.16                          & 9.90                              & 9.08                              & $-$0.75                         & 3.0                       & 2.3                       & 10.3                         & 1.26E+21                                & 1                        \\
                    2053                    & 05810$+$0030$+$0090          & 58.1000                 & 0.3000                  & 6                                        & \nodata                           & 2.16                          & 9.00                              & 8.53                              & $-$0.49                         & 4.4                       & 1.3                       & 9.7                          & 5.46E+20                                & 1                        \\
                    2054                    & 05810$+$0320$+$0100          & 58.1000                 & 3.2000                  & 1,  5,  6                                & \nodata                           & 0.46                          & 10.05                             & 9.54                              & $-$0.50                         & 4.2                       & 2.6                       & 10.5                         & 1.32E+21                                & 1                        \\
                    2055                    & 05811$+$0224$+$0113          & 58.1083                 & 2.2417                  & 6                                        & \nodata                           & 0.58                          & 11.31                             & 10.58                             & $-$0.70                         & 4.8                       & 1.4                       & 9.8                          & 6.39E+20                                & 1                        \\
                    2056                    & 05811$+$0227$+$0113          & 58.1083                 & 2.2667                  & 6                                        & \nodata                           & 0.58                          & 11.33                             & 10.45                             & $-$0.81                         & 3.9                       & 1.0                       & 10.8                         & 4.68E+20                                & 1                        \\
                    2057                    & 05813$+$0312$+$0101          & 58.1333                 & 3.1167                  & 5,  6                                    & \nodata                           & 0.46                          & 10.07                             & 9.32                              & $-$0.50                         & 5.3                       & 1.9                       & 14.3                         & 1.44E+21                                & 1                        \\
                    2058                    & 05813$-$0042$+$0096          & 58.1333                 & $-$0.4250                 & 5,  6                                    & \nodata                           & 0.47                          & 9.63                              & 8.95                              & $-$0.66                         & 3.0                       & 1.2                       & 8.6                          & 5.30E+20                                & 1                        \\
                    2059                    & 05813$-$0055$+$0095          & 58.1333                 & $-$0.5500                 & 1,  5,  6                                & \nodata                           & 0.45                          & 9.49                              & 8.85                              & $-$0.58                         & 3.3                       & 1.3                       & 11.5                         & 6.71E+20                                & 1                        \\
                    2060                    & 05814$-$0040$+$0097          & 58.1417                 & $-$0.4000                 & 5,  6                                    & \nodata                           & 2.16                          & 9.71                              & 8.74                              & $-$0.96                         & 3.0                       & 1.5                       & 7.5                          & 7.18E+20                                & 1                        \\
                    2061                    & 05815$+$0309$+$0101          & 58.1500                 & 3.0917                  & 5,  6                                    & \nodata                           & 0.46                          & 10.12                             & 9.38                              & $-$0.73                         & 5.8                       & 2.1                       & 17.0                         & 1.17E+21                                & 1                        \\
                    2062                    & 05817$-$0025$+$0095          & 58.1750                 & $-$0.2500                 & 6                                        & \nodata                           & 2.16                          & 9.50                              & 9.00                              & $-$0.51                         & 4.7                       & 1.8                       & 8.8                          & 8.11E+20                                & 1                        \\
                    2063                    & 05818$-$0018$+$0096          & 58.1833                 & $-$0.1833                 & 6                                        & \nodata                           & 2.16                          & 9.59                              & 9.17                              & $-$0.58                         & 3.3                       & 1.5                       & 7.2                          & 5.50E+20                                & 1                        \\
                    2064                    & 05819$+$0306$+$0101          & 58.1917                 & 3.0583                  & 5,  6                                    & \nodata                           & 0.47                          & 10.10                             & 9.55                              & $-$0.36                         & 6.2                       & 1.8                       & 16.2                         & 1.42E+21                                & 1                        \\
                    2065                    & 05819$+$0308$+$0100          & 58.1917                 & 3.0833                  & 5,  6                                    & \nodata                           & 0.46                          & 10.04                             & 9.59                              & $-$0.42                         & 6.3                       & 1.9                       & 19.1                         & 1.21E+21                                & 1                        \\
                    2066                    & 05821$+$0034$+$0268          & 58.2083                 & 0.3417                  & 6                                        & \nodata                           & 2.17                          & 26.78                             & 25.66                             & $-$0.40                         & 5.6                       & 2.2                       & 13.7                         & 3.15E+21                                & 1                        \\
                    2067                    & 05824$+$0034$+$0271          & 58.2417                 & 0.3417                  & 6                                        & \nodata                           & 2.17                          & 27.09                             & 25.54                             & $-$0.66                         & 5.7                       & 3.8                       & 10.3                         & 5.27E+21                                & 1                        \\
                    2068                    & 05824$+$0197$+$0105          & 58.2417                 & 1.9667                  & 6                                        & \nodata                           & 0.49                          & 10.48                             & 10.12                             & $-$0.36                         & 5.1                       & 2.0                       & 12.7                         & 1.01E+21                                & 1                        \\
                    2069                    & 05825$+$0047$+$0284          & 58.2500                 & 0.4667                  & 6                                        & \nodata                           & 2.17                          & 28.43                             & 28.08                             & $-$0.22                         & 6.3                       & 2.6                       & 11.2                         & 2.06E+21                                & 1                        \\
                    2070                    & 05826$+$0303$+$0101          & 58.2583                 & 3.0333                  & 5,  6                                    & \nodata                           & 0.47                          & 10.10                             & 9.40                              & $-$0.44                         & 5.1                       & 1.1                       & 14.3                         & 8.67E+20                                & 1                        \\
                    2071                    & 05827$+$0307$+$0101          & 58.2750                 & 3.0750                  & 5,  6                                    & \nodata                           & 0.47                          & 10.13                             & 9.61                              & $-$0.41                         & 4.5                       & 1.1                       & 12.8                         & 6.44E+20                                & 1                        \\
                    2072                    & 05829$+$0305$+$0097          & 58.2917                 & 3.0500                  & 5,  6                                    & \nodata                           & 0.43                          & 9.75                              & 9.42                              & $-$0.35                         & 5.4                       & 1.4                       & 15.5                         & 6.56E+20                                & 1                        \\
                    2073                    & 05832$+$0321$+$0101          & 58.3250                 & 3.2083                  & 5,  6                                    & \nodata                           & 0.46                          & 10.06                             & 9.29                              & $-$0.50                         & 6.8                       & 3.3                       & 11.6                         & 2.71E+21                                & 1                        \\
                    2074                    & 05837$+$0302$+$0099          & 58.3667                 & 3.0167                  & 5,  6                                    & \nodata                           & 0.44                          & 9.86                              & 9.32                              & $-$0.53                         & 4.4                       & 1.2                       & 11.7                         & 5.62E+20                                & 1                        \\
                    2075                    & 05839$+$0312$+$0104          & 58.3917                 & 3.1167                  & 5,  6                                    & \nodata                           & 0.49                          & 10.36                             & 9.66                              & $-$0.76                         & 5.1                       & 1.1                       & 11.2                         & 4.75E+20                                & 1                        \\
                    2076                    & 05842$+$0308$+$0101          & 58.4167                 & 3.0833                  & 5,  6                                    & \nodata                           & 0.47                          & 10.09                             & 9.78                              & $-$0.36                         & 5.7                       & 1.2                       & 12.5                         & 4.68E+20                                & 1                        \\
                    2077                    & 05845$+$0300$+$0099          & 58.4500                 & 3.0000                  & 5,  6                                    & \nodata                           & 0.44                          & 9.87                              & 9.47                              & $-$0.52                         & 5.9                       & 2.4                       & 18.1                         & 1.05E+21                                & 1                        \\
                    2078                    & 05845$+$0302$+$0100          & 58.4500                 & 3.0250                  & 5,  6                                    & \nodata                           & 0.46                          & 10.00                             & 9.57                              & $-$0.39                         & 5.7                       & 1.6                       & 13.4                         & 8.43E+20                                & 1                        \\
                    2079                    & 05846$+$0312$+$0103          & 58.4583                 & 3.1167                  & 5,  6                                    & \nodata                           & 0.48                          & 10.28                             & 9.47                              & $-$0.51                         & 7.3                       & 1.1                       & 14.3                         & 8.42E+20                                & 1                        \\
                    2080                    & 05850$+$0307$+$0103          & 58.5000                 & 3.0667                  & 5,  6                                    & \nodata                           & 0.49                          & 10.30                             & 9.69                              & $-$0.43                         & 6.8                       & 1.3                       & 16.0                         & 9.29E+20                                & 1                        \\
                    2081                    & 05854$+$0268$+$0106          & 58.5417                 & 2.6833                  & 6                                        & \nodata                           & 0.50                          & 10.60                             & 9.71                              & $-$0.78                         & 4.4                       & 1.8                       & 12.1                         & 9.75E+20                                & 1                        \\
                    2082                    & 05855$+$0307$+$0102          & 58.5500                 & 3.0750                  & 5,  6                                    & \nodata                           & 0.47                          & 10.17                             & 9.49                              & $-$0.64                         & 6.3                       & 1.3                       & 14.4                         & 7.01E+20                                & 1                        \\
                    2083                    & 05856$+$0273$+$0097          & 58.5583                 & 2.7333                  & 5,  6                                    & \nodata                           & 0.42                          & 9.74                              & 9.13                              & $-$0.66                         & 4.3                       & 1.1                       & 11.2                         & 4.69E+20                                & 1                        \\
                    2084                    & 05857$+$0302$+$0102          & 58.5750                 & 3.0250                  & 5,  6                                    & \nodata                           & 0.47                          & 10.16                             & 9.65                              & $-$0.36                         & 6.1                       & 1.3                       & 14.8                         & 8.78E+20                                & 1                        \\
                    2085                    & 05858$+$0267$+$0104          & 58.5833                 & 2.6667                  & 6                                        & \nodata                           & 0.49                          & 10.39                             & 9.55                              & $-$0.78                         & 3.5                       & 1.6                       & 9.6                          & 7.84E+20                                & 1                        \\
                    2086                    & 05858$+$0305$+$0102          & 58.5833                 & 3.0500                  & 5,  6                                    & \nodata                           & 0.47                          & 10.16                             & 9.62                              & $-$0.45                         & 7.0                       & 1.6                       & 13.5                         & 9.58E+20                                & 1                        \\
                    2087                    & 05859$+$0291$+$0101          & 58.5917                 & 2.9083                  & 5,  6                                    & \nodata                           & 0.46                          & 10.08                             & 9.62                              & $-$0.32                         & 6.5                       & 1.3                       & 14.4                         & 9.26E+20                                & 1                        \\
                    2088                    & 05861$+$0309$+$0102          & 58.6083                 & 3.0917                  & 5,  6                                    & \nodata                           & 0.48                          & 10.22                             & 9.57                              & $-$0.68                         & 4.4                       & 1.4                       & 10.8                         & 6.21E+20                                & 1                        \\
                    2089                    & 05862$+$0266$+$0103          & 58.6167                 & 2.6583                  & 6                                        & \nodata                           & 0.47                          & 10.32                             & 9.48                              & $-$0.68                         & 4.4                       & 1.5                       & 10.6                         & 8.49E+20                                & 1                        \\
                    2090                    & 05878$+$0275$+$0106          & 58.7833                 & 2.7500                  & 6                                        & \nodata                           & 0.51                          & 10.57                             & 9.94                              & $-$0.42                         & 5.1                       & 1.0                       & 11.6                         & 7.07E+20                                & 1                        \\
                    2091                    & 05882$+$0027$+$0072          & 58.8250                 & 0.2667                  & 6                                        & \nodata                           & 8.26                          & 7.20                              & 6.85                              & $-$0.44                         & 3.9                       & 1.8                       & 11.2                         & 6.99E+20                                & 1                        \\
                    2092                    & 05886$-$0051$+$0257          & 58.8583                 & $-$0.5083                 & 5,  6                                    & \nodata                           & 2.17                          & 25.66                             & 24.78                             & $-$0.97                         & 5.8                       & 2.6                       & 12.0                         & 1.18E+21                                & 1                        \\
                    2093                    & 05891$-$0029$+$0269          & 58.9083                 & $-$0.2917                 & 1,  5,  6                                & \nodata                           & 2.17                          & 26.85                             & 25.99                             & $-$0.59                         & 10.8                      & 4.5                       & 18.7                         & 4.10E+21                                & 1                        \\
                    2094                    & 05897$-$0139$+$0099          & 58.9667                 & $-$1.3917                 & 5,  6                                    & \nodata                           & 0.56                          & 9.90                              & 9.49                              & $-$0.27                         & 6.2                       & 1.0                       & 12.1                         & 7.14E+20                                & 1                        \\
                    2095                    & 05901$-$0127$+$0095          & 59.0083                 & $-$1.2750                 & 5,  6                                    & \nodata                           & 0.41                          & 9.53                              & 9.14                              & $-$0.31                         & 6.0                       & 1.3                       & 11.4                         & 7.09E+20                                & 1                        \\
                    2096                    & 05907$+$0266$+$0096          & 59.0750                 & 2.6583                  & 5,  6                                    & \nodata                           & 0.42                          & 9.60                              & 9.01                              & $-$0.54                         & 3.9                       & 1.7                       & 9.7                          & 8.89E+20                                & 1                        \\
                    2097                    & 05941$+$0064$+$0346          & 59.4083                 & 0.6417                  & 6                                        & \nodata                           & 3.19                          & 34.61                             & 34.24                             & $-$0.37                         & 4.3                       & 1.4                       & 7.7                          & 6.86E+20                                & 1                        \\
                    2098                    & 05941$-$0147$+$0265          & 59.4083                 & $-$1.4667                 & 5,  6                                    & \nodata                           & 2.06                          & 26.55                             & 26.01                             & $-$0.49                         & 7.7                       & 2.0                       & 13.9                         & 1.12E+21                                & 1                        \\
                    2099                    & 05946$-$0012$+$0334          & 59.4583                 & $-$0.1250                 & 4,  5,  6                                & \nodata                           & 3.31                          & 33.41                             & 32.80                             & $-$0.33                         & 7.8                       & 2.1                       & 14.1                         & 1.96E+21                                & 1                        \\
                    2100                    & 05951$-$0012$+$0333          & 59.5083                 & $-$0.1250                 & 4,  5,  6                                & \nodata                           & 3.31                          & 33.33                             & 32.89                             & $-$0.34                         & 8.7                       & 3.9                       & 14.8                         & 2.89E+21                                & 1                        \\
                    2101                    & 05952$+$0228$+$0113          & 59.5167                 & 2.2833                  & 1,  6                                    & \nodata                           & 0.57                          & 11.25                             & 10.70                             & $-$0.64                         & 3.4                       & 1.5                       & 8.5                          & 6.00E+20                                & 1                        \\
                    2102                    & 05953$+$0235$+$0112          & 59.5333                 & 2.3500                  & 6                                        & \nodata                           & 0.59                          & 11.25                             & 10.59                             & $-$0.49                         & 4.1                       & 1.3                       & 11.4                         & 8.01E+20                                & 1                        \\
                    2103                    & 05955$+$0227$+$0113          & 59.5500                 & 2.2750                  & 1,  6                                    & \nodata                           & 0.59                          & 11.32                             & 10.77                             & $-$0.57                         & 3.5                       & 1.3                       & 8.3                          & 5.57E+20                                & 1                        \\
                    2104                    & 05967$+$0062$+$0333          & 59.6667                 & 0.6250                  & 6                                        & \nodata                           & 3.16                          & 33.27                             & 32.54                             & $-$0.40                         & 6.1                       & 1.9                       & 13.6                         & 1.72E+21                                & 1                        \\
                    2105                    & 05977$+$0188$+$0113          & 59.7667                 & 1.8833                  & 6                                        & \nodata                           & 0.59                          & 11.31                             & 10.74                             & $-$0.48                         & 2.6                       & 0.9                       & 6.5                          & 5.38E+20                                & 1                        \\
                    2106                    & 05980$-$0078$+$0199          & 59.8000                 & $-$0.7833                 & 1,  5,  6                                & \nodata                           & 2.17                          & 19.94                             & 18.98                             & $-$0.85                         & 4.3                       & 3.1                       & 9.2                          & 1.94E+21                                & 1                        \\
                    2107                    & 06029$-$0069$+$0294          & 60.2917                 & $-$0.6917                 & 5,  6                                    & \nodata                           & 2.18                          & 29.44                             & 28.69                             & $-$0.64                         & 5.7                       & 1.6                       & 12.5                         & 9.02E+20                                & 1                        \\
                    2108                    & 06032$-$0071$+$0291          & 60.3167                 & $-$0.7083                 & 5,  6                                    & \nodata                           & 2.18                          & 29.06                             & 28.34                             & $-$0.60                         & 4.7                       & 1.7                       & 9.5                          & 9.69E+20                                & 1                        \\
                    2109                    & 06037$-$0072$+$0288          & 60.3750                 & $-$0.7250                 & 1,  5,  6                                & \nodata                           & 2.18                          & 28.85                             & 28.03                             & $-$0.68                         & 5.3                       & 3.1                       & 10.8                         & 1.94E+21                                & 1                        \\
                    2110                    & 06042$+$0266$+$0028          & 60.4167                 & 2.6583                  & 6                                        & \nodata                           & 8.11                          & 2.83                              & 2.20                              & $-$0.77                         & 5.3                       & 1.5                       & 9.4                          & 5.46E+20                                & 1                        \\
                    2111                    & 06056$+$0077$+$0059          & 60.5583                 & 0.7750                  & 6                                        & \nodata                           & 0.08                          & 5.91                              & 5.32                              & $-$0.56                         & 3.1                       & 1.3                       & 7.2                          & 6.35E+20                                & 1                        \\
                    2112                    & 06058$-$0069$+$0293          & 60.5833                 & $-$0.6917                 & 5,  6                                    & \nodata                           & 2.18                          & 29.30                             & 28.59                             & $-$0.45                         & 6.7                       & 2.3                       & 12.3                         & 1.78E+21                                & 1                        \\
                    2113                    & 06061$-$0068$+$0296          & 60.6083                 & $-$0.6833                 & 5,  6                                    & \nodata                           & 2.18                          & 29.57                             & 28.87                             & $-$0.60                         & 5.5                       & 2.7                       & 11.1                         & 1.59E+21                                & 1                        \\
                    2114                    & 06064$-$0071$+$0293          & 60.6417                 & $-$0.7083                 & 5,  6                                    & \nodata                           & 2.18                          & 29.28                             & 28.60                             & $-$0.57                         & 6.3                       & 2.9                       & 9.8                          & 1.84E+21                                & 1                        \\
                    2115                    & 06067$-$0089$+$0114          & 60.6667                 & $-$0.8917                 & 5,  6                                    & \nodata                           & 2.17                          & 11.39                             & 10.82                             & $-$0.41                         & 7.4                       & 3.8                       & 13.9                         & 2.96E+21                                & 1                        \\
                    2116                    & 06070$-$0030$+$0308          & 60.7000                 & $-$0.3000                 & 5,  6                                    & \nodata                           & 3.31                          & 30.75                             & 30.40                             & $-$0.29                         & 6.1                       & 2.8                       & 12.5                         & 1.74E+21                                & 1                        \\
                    2117                    & 06070$-$0097$+$0114          & 60.7000                 & $-$0.9667                 & 5,  6                                    & \nodata                           & 2.17                          & 11.42                             & 10.89                             & $-$0.40                         & 6.0                       & 3.1                       & 11.2                         & 2.13E+21                                & 1                        \\
                    2118                    & 06072$-$0101$+$0113          & 60.7167                 & $-$1.0083                 & 5,  6                                    & \nodata                           & 2.17                          & 11.28                             & 10.69                             & $-$0.67                         & 4.9                       & 3.3                       & 11.5                         & 1.57E+21                                & 1                        \\
                    2119                    & 06072$-$0091$+$0116          & 60.7250                 & $-$0.9083                 & 5,  6                                    & \nodata                           & 2.17                          & 11.62                             & 11.00                             & $-$0.54                         & 7.4                       & 3.5                       & 15.2                         & 2.30E+21                                & 1                        \\
                    2120                    & 06090$+$0172$+$0103          & 60.9000                 & 1.7250                  & 5,  6                                    & \nodata                           & 0.49                          & 10.32                             & 9.77                              & $-$0.63                         & 5.4                       & 1.4                       & 12.0                         & 5.78E+20                                & 1                        \\
                    2121                    & 06091$+$0222$+$0110          & 60.9083                 & 2.2167                  & 5,  6                                    & \nodata                           & 0.58                          & 11.04                             & 10.30                             & $-$0.37                         & 6.5                       & 1.5                       & 14.4                         & 1.48E+21                                & 1                        \\
                    2122                    & 06095$+$0221$+$0110          & 60.9500                 & 2.2083                  & 5,  6                                    & \nodata                           & 0.56                          & 10.96                             & 10.29                             & $-$0.52                         & 7.3                       & 2.0                       & 14.6                         & 1.34E+21                                & 1                        \\
                    2123                    & 06109$+$0073$+$0059          & 61.0917                 & 0.7333                  & 6                                        & \nodata                           & 0.08                          & 5.91                              & 5.20                              & $-$0.70                         & 4.1                       & 1.2                       & 12.3                         & 5.49E+20                                & 1                        \\
                    2124                    & 06115$-$0111$+$0114          & 61.1500                 & $-$1.1083                 & 5,  6                                    & \nodata                           & 2.16                          & 11.37                             & 10.87                             & $-$0.36                         & 3.5                       & 1.4                       & 8.1                          & 8.79E+20                                & 1                        \\
                    2125                    & 06121$+$0079$+$0105          & 61.2083                 & 0.7917                  & 6                                        & \nodata                           & 0.57                          & 10.46                             & 9.74                              & $-$0.87                         & 4.8                       & 1.5                       & 13.9                         & 6.02E+20                                & 1                        \\
                    2126                    & 06124$+$0076$+$0105          & 61.2417                 & 0.7583                  & 6                                        & \nodata                           & 0.56                          & 10.46                             & 9.96                              & $-$0.55                         & 4.7                       & 1.6                       & 9.5                          & 6.91E+20                                & 1                        \\
                    2127                    & 06132$-$0207$+$0287          & 61.3250                 & $-$2.0667                 & 5,  6                                    & \nodata                           & 1.87                          & 28.72                             & 28.12                             & $-$0.45                         & 7.9                       & 3.3                       & 12.9                         & 2.34E+21                                & 1                        \\
                    2128                    & 06142$-$0136$+$0107          & 61.4167                 & $-$1.3583                 & 5,  6                                    & \nodata                           & 0.39                          & 10.74                             & 9.51                              & $-$1.11                         & 4.7                       & 2.5                       & 14.4                         & 1.44E+21                                & 1                        \\
                    2129                    & 06142$+$0136$+$0053          & 61.4250                 & 1.3583                  & 6                                        & \nodata                           & 0.05                          & 5.29                              & 4.88                              & $-$0.47                         & 4.3                       & 1.1                       & 9.3                          & 4.38E+20                                & 1                        \\
                    2130                    & 06155$+$0048$+$0045          & 61.5500                 & 0.4833                  & 4,  5,  6                                & \nodata                           & 7.86                          & 4.53                              & 4.15                              & $-$0.37                         & 6.6                       & 1.4                       & 12.3                         & 6.97E+20                                & 1                        \\
                    2131                    & 06155$+$0142$+$0051          & 61.5500                 & 1.4250                  & 6                                        & \nodata                           & 0.04                          & 5.10                              & 4.75                              & $-$0.48                         & 3.6                       & 1.3                       & 8.9                          & 4.21E+20                                & 1                        \\
                    2132                    & 06157$-$0117$+$0095          & 61.5750                 & $-$1.1667                 & 5,  6                                    & \nodata                           & 0.26                          & 9.45                              & 9.02                              & $-$0.36                         & 5.0                       & 1.1                       & 8.6                          & 6.07E+20                                & 1                        \\
                    2133                    & 06157$-$0122$+$0097          & 61.5750                 & $-$1.2250                 & 5,  6                                    & \nodata                           & 0.27                          & 9.72                              & 9.11                              & $-$0.54                         & 6.1                       & 1.8                       & 12.3                         & 1.00E+21                                & 1                        \\
                    2134                    & 06158$-$0151$+$0269          & 61.5833                 & $-$1.5083                 & 5,  6                                    & \nodata                           & 1.88                          & 26.90                             & 26.33                             & $-$0.54                         & 4.0                       & 1.4                       & 10.9                         & 6.80E+20                                & 1                        \\
                    2135                    & 06162$-$0200$+$0303          & 61.6167                 & $-$2.0000                 & 1,  5,  6                                & \nodata                           & 1.87                          & 30.28                             & 29.49                             & $-$0.55                         & 5.9                       & 1.0                       & 12.9                         & 6.70E+20                                & 1                        \\
                    2136                    & 06165$+$0081$+$0050          & 61.6500                 & 0.8083                  & 5,  6                                    & \nodata                           & 0.05                          & 4.96                              & 4.58                              & $-$0.42                         & 5.3                       & 1.4                       & 10.4                         & 5.53E+20                                & 1                        \\
                    2137                    & 06182$+$0033$+$0211          & 61.8250                 & 0.3333                  & 1,  5,  6                                & \nodata                           & 2.16                          & 21.09                             & 20.60                             & $-$0.34                         & 12.9                      & 6.2                       & 17.6                         & 6.11E+21                                & 1                        \\
                    2138                    & 06205$-$0210$+$0234          & 62.0500                 & $-$2.1000                 & 5,  6                                    & \nodata                           & 1.87                          & 23.45                             & 22.90                             & $-$0.40                         & 6.5                       & 1.8                       & 12.0                         & 1.19E+21                                & 1                        \\
                    2139                    & 06207$-$0206$+$0239          & 62.0667                 & $-$2.0583                 & 5,  6                                    & \nodata                           & 1.87                          & 23.89                             & 23.08                             & $-$0.55                         & 8.2                       & 2.3                       & 14.9                         & 1.79E+21                                & 1                        \\
                    2140                    & 06211$-$0402$+$0112          & 62.1083                 & $-$4.0250                 & 6                                        & \nodata                           & 0.56                          & 11.15                             & 10.72                             & $-$0.45                         & 3.7                       & 1.8                       & 7.4                          & 9.24E+20                                & 1                        \\
                    2141                    & 06212$+$0103$+$0042          & 62.1250                 & 1.0333                  & 5,  6                                    & \nodata                           & 0.02                          & 4.17                              & 3.66                              & $-$0.45                         & 7.3                       & 2.1                       & 12.9                         & 1.17E+21                                & 1                        \\
                    2142                    & 06215$-$0454$+$0124          & 62.1500                 & $-$4.5417                 & 6                                        & \nodata                           & 0.71                          & 12.40                             & 11.77                             & $-$0.51                         & 5.6                       & 3.2                       & 12.7                         & 2.08E+21                                & 1                        \\
                    2143                    & 06217$+$0038$+$0397          & 62.1667                 & 0.3833                  & 1,  5,  6                                & \nodata                           & 1.89                          & 39.73                             & 39.09                             & $-$0.46                         & 8.2                       & 3.8                       & 15.6                         & 3.00E+21                                & 1                        \\
                    2144                    & 06230$+$0145$+$0045          & 62.3000                 & 1.4500                  & 6                                        & \nodata                           & 0.02                          & 4.54                              & 3.65                              & $-$1.00                         & 5.5                       & 1.1                       & 15.0                         & 4.71E+20                                & 1                        \\
                    2145                    & 06234$-$0058$+$0066          & 62.3417                 & $-$0.5833                 & 6                                        & \nodata                           & 1.01                          & 6.64                              & 6.05                              & $-$0.50                         & 4.3                       & 1.2                       & 9.9                          & 6.58E+20                                & 1                        \\
                    2146                    & 06236$+$0193$+$0023          & 62.3583                 & 1.9333                  & 6                                        & \nodata                           & 7.67                          & 2.32                              & 2.01                              & $-$0.40                         & 4.3                       & 1.5                       & 10.0                         & 5.40E+20                                & 1                        \\
                    2147                    & 06249$-$0053$+$0065          & 62.4917                 & $-$0.5333                 & 6                                        & \nodata                           & 1.24                          & 6.50                              & 5.99                              & $-$0.36                         & 4.8                       & 1.9                       & 12.4                         & 1.32E+21                                & 1                        \\
                    2148                    & 06250$-$0051$+$0072          & 62.5000                 & $-$0.5083                 & 6                                        & \nodata                           & 1.34                          & 7.19                              & 6.43                              & $-$0.79                         & 3.7                       & 1.4                       & 9.5                          & 6.00E+20                                & 1                        \\
                    2149                    & 06251$+$0168$+$0029          & 62.5083                 & 1.6833                  & 6                                        & \nodata                           & 7.58                          & 2.93                              & 2.39                              & $-$0.53                         & 6.0                       & 1.9                       & 11.7                         & 9.43E+20                                & 1                        \\
                    2150                    & 06254$+$0147$+$0023          & 62.5417                 & 1.4750                  & 5,  6                                    & \nodata                           & 7.63                          & 2.30                              & 1.90                              & $-$0.42                         & 5.7                       & 2.5                       & 11.6                         & 1.21E+21                                & 1                        \\
                    2151                    & 06257$+$0151$+$0024          & 62.5667                 & 1.5083                  & 5,  6                                    & \nodata                           & 7.62                          & 2.38                              & 1.96                              & $-$0.40                         & 5.2                       & 2.8                       & 12.0                         & 1.50E+21                                & 1                        \\
                    2152                    & 06260$+$0182$+$0027          & 62.6000                 & 1.8167                  & 6                                        & \nodata                           & 7.58                          & 2.69                              & 2.33                              & $-$0.37                         & 5.7                       & 3.1                       & 13.6                         & 1.61E+21                                & 1                        \\
                    2153                    & 06262$+$0184$+$0027          & 62.6250                 & 1.8417                  & 6                                        & \nodata                           & 7.57                          & 2.70                              & 2.32                              & $-$0.41                         & 5.4                       & 3.6                       & 13.8                         & 1.88E+21                                & 1                        \\
                    2154                    & 06266$-$0352$+$0096          & 62.6583                 & $-$3.5167                 & 5,  6                                    & \nodata                           & 0.53                          & 9.58                              & 9.17                              & $-$0.50                         & 5.0                       & 1.9                       & 9.1                          & 7.68E+20                                & 1                        \\
                    2155                    & 06273$+$0178$+$0020          & 62.7333                 & 1.7833                  & 6                                        & \nodata                           & 7.61                          & 1.96                              & 1.52                              & $-$0.41                         & 6.6                       & 1.5                       & 11.3                         & 7.17E+20                                & 1                        \\
                    2156                    & 06275$+$0157$+$0025          & 62.7500                 & 1.5750                  & 6                                        & \nodata                           & 7.56                          & 2.50                              & 1.62                              & $-$0.95                         & 4.0                       & 1.8                       & 9.4                          & 8.02E+20                                & 1                        \\
                    2157                    & 06278$+$0163$+$0021          & 62.7833                 & 1.6333                  & 6                                        & \nodata                           & 7.59                          & 2.09                              & 1.43                              & $-$0.41                         & 4.7                       & 1.1                       & 13.1                         & 8.50E+20                                & 1                        \\
                    2158                    & 06282$+$0169$+$0020          & 62.8167                 & 1.6917                  & 1,  6                                    & \nodata                           & 7.59                          & 1.95                              & 1.40                              & $-$0.50                         & 5.6                       & 3.3                       & 13.1                         & 1.96E+21                                & 1                        \\
                    2159                    & 06283$+$0162$+$0019          & 62.8333                 & 1.6250                  & 6                                        & \nodata                           & 7.58                          & 1.94                              & 1.51                              & $-$0.45                         & 5.9                       & 1.4                       & 13.1                         & 6.26E+20                                & 1                        \\
                    2160                    & 06290$+$0189$+$0016          & 62.9000                 & 1.8917                  & 6                                        & \nodata                           & 7.60                          & 1.59                              & 0.82                              & $-$0.50                         & 5.9                       & 2.4                       & 11.1                         & 1.88E+21                                & 1                        \\
                    2161                    & 06292$-$0040$+$0270          & 62.9167                 & $-$0.4000                 & 6                                        & \nodata                           & 2.18                          & 26.98                             & 25.83                             & $-$0.42                         & 6.3                       & 2.7                       & 12.3                         & 3.70E+21                                & 1                        \\
                    2162                    & 06294$+$0165$+$0022          & 62.9417                 & 1.6500                  & 6                                        & \nodata                           & 7.54                          & 2.15                              & 1.44                              & $-$0.54                         & 6.6                       & 3.4                       & 13.4                         & 2.48E+21                                & 1                        \\
                    2163                    & 06321$-$0271$+$0098          & 63.2083                 & $-$2.7083                 & 5,  6                                    & \nodata                           & 0.74                          & 9.80                              & 9.47                              & $-$0.55                         & 3.3                       & 1.5                       & 7.6                          & 4.28E+20                                & 1                        \\
                    2164                    & 06342$-$0465$+$0148          & 63.4167                 & $-$4.6500                 & 6                                        & \nodata                           & 1.10                          & 14.81                             & 14.54                             & $-$0.30                         & 3.0                       & 1.2                       & 7.9                          & 4.79E+20                                & 1                        \\
                    2165                    & 06345$-$0034$+$0299          & 63.4500                 & $-$0.3417                 & 6                                        & \nodata                           & 2.18                          & 29.88                             & 28.92                             & $-$0.57                         & 6.1                       & 1.3                       & 10.4                         & 9.82E+20                                & 1                        \\
                    2166                    & 06350$-$0146$+$0198          & 63.5000                 & $-$1.4583                 & 6                                        & \nodata                           & 1.88                          & 19.82                             & 19.45                             & $-$0.27                         & 6.6                       & 3.5                       & 13.5                         & 2.65E+21                                & 1                        \\
                    2167                    & 06379$-$0259$+$0080          & 63.7917                 & $-$2.5917                 & 5,  6                                    & \nodata                           & 0.53                          & 8.03                              & 7.40                              & $-$0.63                         & 3.0                       & 1.5                       & 6.8                          & 8.10E+20                                & 1                        \\
                    2168                    & 06388$-$0256$+$0209          & 63.8833                 & $-$2.5583                 & 6                                        & \nodata                           & 1.88                          & 20.92                             & 20.56                             & $-$0.43                         & 5.4                       & 1.4                       & 12.2                         & 5.20E+20                                & 1                        \\
                    2169                    & 06390$-$0172$+$0103          & 63.9000                 & $-$1.7250                 & 1,  6                                    & \nodata                           & 0.70                          & 10.32                             & 9.68                              & $-$0.60                         & 2.6                       & 0.9                       & 6.6                          & 4.31E+20                                & 1                        \\
                    2170                    & 06402$-$0475$+$0155          & 64.0250                 & $-$4.7500                 & 6                                        & \nodata                           & 1.23                          & 15.48                             & 15.11                             & $-$0.44                         & 4.7                       & 1.1                       & 9.5                          & 3.93E+20                                & 1                        \\
                    2171                    & 06407$-$0309$+$0070          & 64.0750                 & $-$3.0917                 & 5,  6                                    & \nodata                           & 0.26                          & 6.97                              & 6.79                              & $-$0.20                         & 3.2                       & 1.2                       & 7.6                          & 5.12E+20                                & 1                        \\
                    2172                    & 06416$-$0262$+$0065          & 64.1583                 & $-$2.6167                 & 6                                        & \nodata                           & 0.41                          & 6.47                              & 6.13                              & $-$0.35                         & 4.1                       & 1.1                       & 10.0                         & 5.02E+20                                & 1                        \\
                    2173                    & 06452$-$0067$+$0196          & 64.5167                 & $-$0.6667                 & 6                                        & \nodata                           & 2.36                          & 19.63                             & 19.02                             & $-$0.48                         & 5.0                       & 2.1                       & 13.0                         & 1.33E+21                                & 1                        \\
                    2174                    & 06455$-$0024$+$0227          & 64.5500                 & $-$0.2417                 & 6                                        & \nodata                           & 2.41                          & 22.65                             & 21.88                             & $-$0.64                         & 4.8                       & 1.4                       & 10.1                         & 7.56E+20                                & 1                        \\
                    2175                    & 06468$-$0249$+$0093          & 64.6833                 & $-$2.4917                 & 6                                        & \nodata                           & 0.82                          & 9.31                              & 8.83                              & $-$0.49                         & 3.8                       & 2.5                       & 7.8                          & 1.37E+21                                & 1                        \\
                    2176                    & 06482$-$0245$+$0093          & 64.8167                 & $-$2.4500                 & 5,  6                                    & \nodata                           & 0.83                          & 9.25                              & 8.45                              & $-$0.47                         & 4.9                       & 2.3                       & 13.2                         & 1.96E+21                                & 1                        \\
                    2177                    & 06483$-$0242$+$0091          & 64.8333                 & $-$2.4167                 & 5,  6                                    & \nodata                           & 0.83                          & 9.12                              & 8.24                              & $-$0.50                         & 5.8                       & 2.1                       & 12.8                         & 1.78E+21                                & 1                        \\
                    2178                    & 06518$-$0266$+$0050          & 65.1833                 & $-$2.6583                 & 3,  6                                    & \nodata                           & 0.13                          & 5.02                              & 4.40                              & $-$0.46                         & 3.6                       & 1.5                       & 10.5                         & 9.26E+20                                & 1                        \\
                    2179                    & 06523$-$0166$+$0178          & 65.2333                 & $-$1.6583                 & 6                                        & \nodata                           & 2.21                          & 17.85                             & 17.28                             & $-$0.44                         & 5.3                       & 1.7                       & 10.0                         & 1.02E+21                                & 1                        \\
                    2180                    & 06538$-$0265$+$0063          & 65.3833                 & $-$2.6500                 & 3,  6                                    & \nodata                           & 0.37                          & 6.29                              & 5.88                              & $-$0.42                         & 5.5                       & 2.3                       & 12.1                         & 1.14E+21                                & 1                        \\
                    2181                    & 06543$-$0309$+$0062          & 65.4333                 & $-$3.0917                 & 6                                        & \nodata                           & 0.12                          & 6.23                              & 5.26                              & $-$0.92                         & 5.2                       & 2.1                       & 12.0                         & 1.08E+21                                & 1                        \\
                    2182                    & 06550$-$0249$+$0060          & 65.5000                 & $-$2.4917                 & 6                                        & \nodata                           & 0.39                          & 6.04                              & 5.23                              & $-$0.72                         & 4.3                       & 1.8                       & 8.3                          & 1.00E+21                                & 1                        \\
                    2183                    & 06551$-$0266$+$0066          & 65.5083                 & $-$2.6583                 & 6                                        & \nodata                           & 0.43                          & 6.64                              & 6.36                              & $-$0.29                         & 4.1                       & 3.0                       & 12.3                         & 1.48E+21                                & 1                        \\
                    2184                    & 06563$-$0263$+$0064          & 65.6333                 & $-$2.6333                 & 6                                        & \nodata                           & 0.40                          & 6.39                              & 5.54                              & $-$0.64                         & 4.7                       & 4.2                       & 8.5                          & 4.38E+21                                & 1                        \\
                    2185                    & 06564$-$0270$+$0066          & 65.6417                 & $-$2.7000                 & 6                                        & \nodata                           & 0.41                          & 6.61                              & 6.19                              & $-$0.33                         & 3.4                       & 2.5                       & 7.8                          & 1.83E+21                                & 1                        \\
                    2186                    & 06565$-$0257$+$0064          & 65.6500                 & $-$2.5750                 & 6                                        & \nodata                           & 0.41                          & 6.37                              & 5.56                              & $-$0.69                         & 4.7                       & 3.2                       & 9.3                          & 2.09E+21                                & 1                        \\
                    2187                    & 06567$-$0267$+$0070          & 65.6667                 & $-$2.6667                 & 6                                        & \nodata                           & 0.49                          & 7.02                              & 6.23                              & $-$0.56                         & 3.5                       & 3.6                       & 9.0                          & 3.20E+21                                & 1                        \\
                    2188                    & 06569$-$0252$+$0058          & 65.6917                 & $-$2.5167                 & 6                                        & \nodata                           & 0.32                          & 5.78                              & 5.08                              & $-$0.43                         & 5.6                       & 2.2                       & 12.7                         & 1.81E+21                                & 1                        \\
                    2189                    & 06580$-$0315$+$0061          & 65.8000                 & $-$3.1500                 & 6                                        & \nodata                           & 0.12                          & 6.09                              & 5.74                              & $-$0.30                         & 3.9                       & 2.9                       & 8.4                          & 1.98E+21                                & 1                        \\
                    2190                    & 06582$-$0125$+$0053          & 65.8167                 & $-$1.2500                 & 6                                        & \nodata                           & 0.56                          & 5.28                              & 4.33                              & $-$0.64                         & 4.3                       & 1.8                       & 8.9                          & 1.30E+21                                & 1                        \\
                    2191                    & 06582$-$0272$+$0059          & 65.8167                 & $-$2.7250                 & 6                                        & \nodata                           & 0.27                          & 5.89                              & 4.91                              & $-$0.63                         & 3.7                       & 1.9                       & 11.5                         & 1.41E+21                                & 1                        \\
                    2192                    & 06584$-$0254$+$0049          & 65.8417                 & $-$2.5417                 & 1,  6                                    & \nodata                           & 0.16                          & 4.93                              & 4.32                              & $-$0.55                         & 4.4                       & 2.3                       & 10.2                         & 9.75E+21                                & 2                        \\
                    2193                    & 06586$-$0314$+$0059          & 65.8583                 & $-$3.1417                 & 6                                        & \nodata                           & 0.10                          & 5.91                              & 5.04                              & $-$0.56                         & 4.0                       & 3.2                       & 7.8                          & 3.37E+21                                & 1                        \\
                    2194                    & 06587$-$0200$+$0085          & 65.8750                 & $-$2.0000                 & 6                                        & \nodata                           & 0.79                          & 8.52                              & 7.75                              & $-$0.58                         & 4.2                       & 1.3                       & 10.7                         & 8.04E+20                                & 1                        \\
                    2195                    & 06603$-$0295$+$0067          & 66.0333                 & $-$2.9500                 & 6                                        & \nodata                           & 0.23                          & 6.75                              & 6.12                              & $-$0.51                         & 4.5                       & 3.3                       & 9.5                          & 2.32E+21                                & 1                        \\
                    2196                    & 06607$-$0316$+$0051          & 66.0750                 & $-$3.1583                 & 1,  6                                    & \nodata                           & 0.05                          & 5.10                              & 4.47                              & $-$0.48                         & 8.9                       & 4.7                       & 14.6                         & 3.72E+21                                & 1                        \\
                    2197                    & 06612$-$0310$+$0052          & 66.1250                 & $-$3.1000                 & 6                                        & \nodata                           & 0.07                          & 5.19                              & 4.32                              & $-$0.68                         & 6.8                       & 5.3                       & 14.8                         & 4.21E+21                                & 1                        \\
                    2198                    & 06613$-$0318$+$0054          & 66.1333                 & $-$3.1833                 & 6                                        & \nodata                           & 0.07                          & 5.39                              & 4.96                              & $-$0.25                         & 21.8                      & 11.5                      & 27.4                         & 1.88E+22                                & 1                        \\
                    2199                    & 06615$-$0310$+$0051          & 66.1500                 & $-$3.1000                 & 6                                        & \nodata                           & 0.07                          & 5.14                              & 4.31                              & $-$0.54                         & 7.3                       & 4.8                       & 17.7                         & 4.70E+21                                & 1                        \\
                    2200                    & 06616$-$0319$+$0055          & 66.1583                 & $-$3.1917                 & 1,  6                                    & \nodata                           & 0.08                          & 5.53                              & 5.19                              & $-$0.17                         & 14.2                      & 10.5                      & 18.0                         & 1.97E+22                                & 1                        \\
                    2201                    & 06617$-$0313$+$0049          & 66.1667                 & $-$3.1333                 & 6                                        & \nodata                           & 0.06                          & 4.94                              & 4.19                              & $-$0.46                         & 10.4                      & 8.1                       & 17.6                         & 9.99E+21                                & 1                        \\
                    2202                    & 06617$-$0317$+$0054          & 66.1667                 & $-$3.1667                 & 6                                        & \nodata                           & 0.08                          & 5.38                              & 4.07                              & $-$0.68                         & 10.6                      & 9.6                       & 16.8                         & 1.63E+22                                & 1                        \\
                    2203                    & 06617$-$0286$+$0081          & 66.1750                 & $-$2.8583                 & 6                                        & \nodata                           & 0.62                          & 8.07                              & 7.39                              & $-$0.92                         & 4.0                       & 1.1                       & 8.2                          & 3.55E+20                                & 1                        \\
                    2204                    & 06624$-$0312$+$0051          & 66.2417                 & $-$3.1167                 & 6                                        & \nodata                           & 0.07                          & 5.08                              & 4.18                              & $-$0.48                         & 9.0                       & 3.9                       & 23.0                         & 5.00E+21                                & 1                        \\
                    2205                    & 06629$-$0307$+$0052          & 66.2917                 & $-$3.0750                 & 6                                        & \nodata                           & 0.07                          & 5.23                              & 4.36                              & $-$0.56                         & 5.7                       & 3.5                       & 11.8                         & 2.99E+21                                & 1                        \\
                    2206                    & 06632$-$0308$+$0051          & 66.3250                 & $-$3.0833                 & 6                                        & \nodata                           & 0.07                          & 5.10                              & 4.28                              & $-$0.47                         & 6.2                       & 3.2                       & 10.7                         & 3.04E+21                                & 1                        \\
                    2207                    & 06634$-$0330$+$0047          & 66.3417                 & $-$3.3000                 & 6                                        & \nodata                           & 2.46                          & 4.67                              & 3.83                              & $-$0.95                         & 3.4                       & 2.2                       & 11.0                         & 9.59E+20                                & 1                        \\
                    2208                    & 06652$-$0143$+$0043          & 66.5167                 & $-$1.4333                 & 6                                        & \nodata                           & 0.54                          & 4.34                              & 3.95                              & $-$0.44                         & 3.9                       & 1.3                       & 9.2                          & 5.23E+20                                & 1                        \\
                    2209                    & 06655$-$0144$+$0042          & 66.5500                 & $-$1.4417                 & 6                                        & \nodata                           & 0.50                          & 4.21                              & 3.66                              & $-$0.54                         & 3.3                       & 1.4                       & 9.6                          & 6.46E+20                                & 1                        \\
                    2210                    & 06657$+$0065$+$0201          & 66.5750                 & 0.6500                  & 1,  6                                    & \nodata                           & 2.46                          & 20.07                             & 19.19                             & $-$0.50                         & 10.5                      & 5.3                       & 16.2                         & 5.94E+21                                & 1                        \\
                    2211                    & 06661$-$0320$+$0050          & 66.6083                 & $-$3.2000                 & 6                                        & \nodata                           & 0.06                          & 4.96                              & 4.00                              & $-$0.54                         & 4.7                       & 3.4                       & 11.4                         & 3.33E+21                                & 1                        \\
                    2212                    & 06661$-$0325$+$0051          & 66.6083                 & $-$3.2500                 & 6                                        & \nodata                           & 0.05                          & 5.08                              & 3.87                              & $-$0.74                         & 5.7                       & 3.2                       & 12.7                         & 2.83E+21                                & 1                        \\
                    2213                    & 06671$-$0314$+$0045          & 66.7083                 & $-$3.1417                 & 6                                        & \nodata                           & 2.46                          & 4.54                              & 3.94                              & $-$0.43                         & 8.0                       & 2.1                       & 13.6                         & 1.49E+21                                & 1                        \\
                    2214                    & 06684$-$0118$+$0041          & 66.8417                 & $-$1.1833                 & 6                                        & \nodata                           & 0.16                          & 4.11                              & 3.24                              & $-$0.56                         & 5.2                       & 1.1                       & 12.2                         & 7.60E+20                                & 1                        \\
                    2215                    & 06685$-$0242$+$0157          & 66.8500                 & $-$2.4250                 & 1,  6                                    & \nodata                           & 1.30                          & 15.74                             & 15.29                             & $-$0.30                         & 7.1                       & 3.7                       & 14.0                         & 3.04E+21                                & 1                        \\
                    2216                    & 06689$-$0146$+$0037          & 66.8917                 & $-$1.4583                 & 6                                        & \nodata                           & 0.42                          & 3.70                              & 3.15                              & $-$0.35                         & 3.9                       & 1.7                       & 11.5                         & 1.25E+21                                & 1                        \\
                    2217                    & 06699$-$0233$+$0156          & 66.9917                 & $-$2.3333                 & 6                                        & \nodata                           & 1.31                          & 15.62                             & 15.03                             & $-$0.65                         & 4.7                       & 1.1                       & 8.5                          & 4.33E+20                                & 1                        \\
                    2218                    & 06703$-$0307$+$0037          & 67.0333                 & $-$3.0750                 & 6                                        & \nodata                           & 2.46                          & 3.70                              & 3.23                              & $-$0.47                         & 11.7                      & 5.0                       & 18.5                         & 3.19E+21                                & 1                        \\
                    2219                    & 06704$-$0144$+$0045          & 67.0417                 & $-$1.4417                 & 1,  6                                    & \nodata                           & 0.58                          & 4.46                              & 3.49                              & $-$1.44                         & 3.3                       & 1.2                       & 8.1                          & 3.55E+20                                & 1                        \\
                    2220                    & 06704$-$0300$+$0032          & 67.0417                 & $-$3.0000                 & 1,  6                                    & \nodata                           & 2.46                          & 3.17                              & 2.61                              & $-$0.55                         & 5.1                       & 4.0                       & 16.4                         & 2.38E+21                                & 1                        \\
                    2221                    & 06707$-$0293$+$0032          & 67.0750                 & $-$2.9333                 & 6                                        & \nodata                           & 2.46                          & 3.15                              & 2.55                              & $-$0.49                         & 5.7                       & 1.4                       & 10.3                         & 8.03E+20                                & 1                        \\
                    2222                    & 06709$-$0312$+$0042          & 67.0917                 & $-$3.1250                 & 1,  6                                    & \nodata                           & 2.46                          & 4.25                              & 3.96                              & $-$0.36                         & 6.3                       & 5.3                       & 9.7                          & 3.55E+21                                & 1                        \\
                    2223                    & 06709$-$0320$+$0040          & 67.0917                 & $-$3.2000                 & 6                                        & \nodata                           & 2.46                          & 3.99                              & 3.53                              & $-$0.53                         & 8.7                       & 4.6                       & 13.0                         & 2.34E+21                                & 1                        \\
                    2224                    & 06711$-$0315$+$0041          & 67.1083                 & $-$3.1500                 & 6                                        & \nodata                           & 2.46                          & 4.15                              & 3.82                              & $-$0.40                         & 12.4                      & 7.6                       & 15.6                         & 4.74E+21                                & 1                        \\
                    2225                    & 06722$-$0131$+$0045          & 67.2250                 & $-$1.3083                 & 6                                        & \nodata                           & 0.50                          & 4.47                              & 3.89                              & $-$0.50                         & 5.1                       & 1.3                       & 12.8                         & 7.35E+20                                & 1                        \\
                    2226                    & 06727$-$0149$+$0047          & 67.2750                 & $-$1.4917                 & 6                                        & \nodata                           & 0.64                          & 4.70                              & 4.09                              & $-$0.77                         & 4.4                       & 1.3                       & 11.8                         & 4.79E+20                                & 1                        \\
                    2227                    & 06732$-$0142$+$0048          & 67.3167                 & $-$1.4250                 & 6                                        & \nodata                           & 0.64                          & 4.80                              & 4.27                              & $-$0.51                         & 4.5                       & 1.6                       & 11.1                         & 8.08E+20                                & 1                        \\
                    2228                    & 06767$-$0064$+$0068          & 67.6667                 & $-$0.6417                 & 6                                        & \nodata                           & 2.46                          & 6.76                              & 6.01                              & $-$0.45                         & 3.7                       & 1.4                       & 7.6                          & 1.11E+21                                & 1                        \\
                    2229                    & 06787$-$0267$+$0053          & 67.8750                 & $-$2.6667                 & 6                                        & \nodata                           & 0.25                          & 5.33                              & 4.77                              & $-$0.56                         & 3.1                       & 1.9                       & 7.8                          & 9.83E+20                                & 1                        \\
                    2230                    & 06831$-$0115$+$0046          & 68.3083                 & $-$1.1500                 & 5,  6                                    & \nodata                           & 2.46                          & 4.57                              & 4.10                              & $-$0.65                         & 2.6                       & 1.4                       & 9.1                          & 4.47E+20                                & 1                        \\
                    2231                    & 06835$-$0076$+$0050          & 68.3500                 & $-$0.7583                 & 5,  6                                    & \nodata                           & 2.46                          & 4.98                              & 4.55                              & $-$0.36                         & 6.1                       & 1.3                       & 10.4                         & 6.73E+20                                & 1                        \\
                    2232                    & 06842$-$0313$+$0096          & 68.4250                 & $-$3.1333                 & 5,  6                                    & \nodata                           & 0.67                          & 9.59                              & 8.75                              & $-$0.69                         & 4.9                       & 2.1                       & 9.9                          & 1.24E+21                                & 1                        \\
                    2233                    & 06859$-$0056$+$0051          & 68.5917                 & $-$0.5583                 & 5,  6                                    & \nodata                           & 2.46                          & 5.10                              & 4.30                              & $-$0.64                         & 5.4                       & 2.5                       & 10.1                         & 1.53E+21                                & 1                        \\
                    2234                    & 06863$-$0055$+$0051          & 68.6333                 & $-$0.5500                 & 5,  6                                    & \nodata                           & 2.46                          & 5.13                              & 4.16                              & $-$0.56                         & 5.2                       & 2.1                       & 11.4                         & 1.76E+21                                & 1                        \\
                    2235                    & 06922$+$0066$+$0134          & 69.2250                 & 0.6583                  & 5,  6                                    & \nodata                           & 2.45                          & 13.40                             & 12.69                             & $-$0.74                         & 4.7                       & 1.7                       & 9.4                          & 7.77E+20                                & 1                        \\
                    2236                    & 06927$-$0093$+$0111          & 69.2750                 & $-$0.9333                 & 3,  5,  6                                & \nodata                           & 2.46                          & 11.10                             & 10.22                             & $-$0.40                         & 5.8                       & 1.9                       & 10.2                         & 1.97E+21                                & 1                        \\
                    2237                    & 06952$-$0097$+$0122          & 69.5167                 & $-$0.9667                 & 5,  6                                    & \nodata                           & 2.46                          & 12.22                             & 10.90                             & $-$0.57                         & 9.9                       & 4.8                       & 17.0                         & 6.85E+21                                & 1                        \\
                    2238                    & 06953$-$0098$+$0111          & 69.5333                 & $-$0.9833                 & 1,  5,  6                                & \nodata                           & 2.46                          & 11.12                             & 9.25                              & $-$0.50                         & 11.8                      & 7.7                       & 20.2                         & 2.14E+22                                & 1                        \\
                    2239                    & 06957$-$0094$+$0108          & 69.5750                 & $-$0.9417                 & 5,  6                                    & \nodata                           & 2.46                          & 10.78                             & 9.76                              & $-$0.36                         & 8.7                       & 3.4                       & 13.1                         & 5.22E+21                                & 1                        \\
                    2240                    & 06958$-$0091$+$0111          & 69.5833                 & $-$0.9083                 & 5,  6                                    & \nodata                           & 2.46                          & 11.06                             & 10.69                             & $-$0.36                         & 3.5                       & 1.0                       & 6.9                          & 3.43E+21                                & 2                        \\
                    2241                    & 06982$-$0161$+$0135          & 69.8167                 & $-$1.6083                 & 5,  6                                    & \nodata                           & 2.46                          & 13.47                             & 12.58                             & $-$0.73                         & 7.6                       & 2.9                       & 11.8                         & 1.81E+21                                & 1                        \\
                    2242                    & 06984$-$0161$+$0132          & 69.8417                 & $-$1.6083                 & 5,  6                                    & \nodata                           & 2.46                          & 13.22                             & 12.12                             & $-$0.62                         & 7.6                       & 2.1                       & 15.5                         & 1.95E+21                                & 1                        \\
                    2243                    & 06992$-$0125$+$0129          & 69.9167                 & $-$1.2500                 & 5,  6                                    & \nodata                           & 2.46                          & 12.91                             & 11.72                             & $-$0.45                         & 8.0                       & 3.1                       & 14.9                         & 4.58E+21                                & 1                        \\
                    2244                    & 07068$-$0389$+$0052          & 70.6833                 & $-$3.8917                 & 6                                        & \nodata                           & 0.55                          & 5.17                              & 4.62                              & $-$0.59                         & 3.4                       & 1.3                       & 7.6                          & 5.40E+20                                & 1                        \\
                    2245                    & 07087$-$0037$+$0084          & 70.8667                 & $-$0.3667                 & 4,  5,  6                                & \nodata                           & 3.79                          & 8.36                              & 7.78                              & $-$0.56                         & 5.1                       & 1.7                       & 9.1                          & 8.43E+20                                & 1                        \\
                    2246                    & 07127$-$0226$+$0060          & 71.2750                 & $-$2.2583                 & 4,  5,  6                                & \nodata                           & 1.59                          & 6.01                              & 5.59                              & $-$0.44                         & 5.5                       & 1.1                       & 11.6                         & 4.51E+20                                & 1                        \\
                    2247                    & 07157$-$0192$+$0083          & 71.5750                 & $-$1.9167                 & 4,  5,  6                                & \nodata                           & 1.58                          & 8.34                              & 7.56                              & $-$0.65                         & 6.2                       & 2.9                       & 11.9                         & 1.77E+21                                & 1                        \\
                    2248                    & 07160$-$0187$+$0079          & 71.6000                 & $-$1.8750                 & 4,  5,  6                                & \nodata                           & 1.25                          & 7.93                              & 7.16                              & $-$0.55                         & 6.5                       & 2.1                       & 12.2                         & 1.46E+21                                & 1                        \\
                    2249                    & 07167$-$0201$+$0075          & 71.6667                 & $-$2.0083                 & 4,  5,  6                                & \nodata                           & 1.59                          & 7.49                              & 6.76                              & $-$0.48                         & 6.1                       & 3.6                       & 13.4                         & 3.03E+21                                & 1                        \\
                    2250                    & 07182$-$0193$+$0072          & 71.8250                 & $-$1.9333                 & 1,  4,  5,  6                            & \nodata                           & 1.59                          & 7.18                              & 6.62                              & $-$0.70                         & 5.7                       & 2.0                       & 13.2                         & 8.18E+20                                & 1                        \\
                    2251                    & 07349$-$0500$+$0114          & 73.4917                 & $-$5.0000                 & 3,  4,  5,  6                            & \nodata                           & 1.10                          & 11.39                             & 10.40                             & $-$0.75                         & 5.0                       & 1.6                       & 10.9                         & 1.01E+21                                & 1                        \\
                    2252                    & 07393$-$0178$+$0058          & 73.9333                 & $-$1.7833                 & 4,  5,  6                                & \nodata                           & 1.59                          & 5.75                              & 4.38                              & $-$0.64                         & 6.4                       & 3.3                       & 12.8                         & 3.85E+21                                & 1                        \\
                    2253                    & 07408$-$0487$+$0095          & 74.0833                 & $-$4.8667                 & 4,  5,  6                                & \nodata                           & 0.90                          & 9.52                              & 8.93                              & $-$0.57                         & 3.4                       & 0.9                       & 8.1                          & 4.21E+20                                & 1                        \\
                    2254                    & 07411$-$0482$+$0093          & 74.1083                 & $-$4.8250                 & 4,  5,  6                                & \nodata                           & 0.84                          & 9.27                              & 8.67                              & $-$0.49                         & 4.8                       & 1.0                       & 10.2                         & 5.33E+20                                & 1                        \\
                    2255                    & 07447$-$0249$+$0048          & 74.4750                 & $-$2.4917                 & 4,  5,  6                                & \nodata                           & 1.59                          & 4.78                              & 4.33                              & $-$0.42                         & 6.5                       & 1.7                       & 11.9                         & 8.50E+20                                & 1                        \\
                    2256                    & 07483$+$0192$-$0219          & 74.8333                 & 1.9250                  & 4,  6                                    & \nodata                           & 3.53                          & $-$21.91                            & $-$22.36                            & $-$0.37                         & 3.4                       & 1.0                       & 9.3                          & 5.28E+20                                & 1                        \\
                    2257                    & 07487$-$0472$+$0120          & 74.8750                 & $-$4.7250                 & 4,  5,  6                                & \nodata                           & 1.59                          & 12.02                             & 11.78                             & $-$0.25                         & 5.7                       & 2.2                       & 11.1                         & 1.04E+21                                & 1                        \\
                    2258                    & 07488$-$0217$+$0023          & 74.8833                 & $-$2.1667                 & 4,  5,  6                                & \nodata                           & 1.59                          & 2.27                              & 1.27                              & $-$0.84                         & 6.6                       & 2.3                       & 12.4                         & 1.35E+21                                & 1                        \\
                    2259                    & 07532$-$0212$-$0041          & 75.3250                 & $-$2.1250                 & 4,  5,  6                                & \nodata                           & 1.30                          & $-$4.06                             & $-$4.53                             & $-$0.45                         & 5.2                       & 1.3                       & 11.8                         & 6.59E+20                                & 1                        \\
                    2260                    & 07564$+$0031$-$0011          & 75.6417                 & 0.3083                  & 4,  6                                    & \nodata                           & 1.31                          & $-$1.08                             & $-$2.25                             & $-$0.32                         & 10.7                      & 4.0                       & 16.0                         & 8.61E+21                                & 1                        \\
                    2261                    & 07567$+$0032$-$0011          & 75.6750                 & 0.3250                  & 4,  6                                    & \nodata                           & 1.31                          & $-$1.13                             & $-$2.05                             & $-$0.24                         & 10.8                      & 3.5                       & 15.4                         & 7.42E+21                                & 1                        \\
                    2262                    & 07582$+$0041$+$0014          & 75.8250                 & 0.4083                  & 1,  3,  4,  6                            & \nodata                           & 1.33                          & 1.37                              & 0.38                              & $-$0.38                         & 18.3                      & 6.0                       & 25.2                         & 1.17E+22                                & 1                        \\
                    2263                    & 07585$+$0030$+$0036          & 75.8500                 & 0.3000                  & 3,  4,  6                                & \nodata                           & 1.58                          & 3.63                              & 2.38                              & $-$0.54                         & 10.0                      & 3.8                       & 15.0                         & 5.03E+21                                & 1                        \\
                    2264                    & 07607$+$0026$+$0012          & 76.0750                 & 0.2583                  & 4,  6                                    & \nodata                           & 1.33                          & 1.16                              & 0.03                              & $-$0.43                         & 6.8                       & 2.5                       & 12.9                         & 3.41E+21                                & 1                        \\
                    2265                    & 07625$-$0077$-$0004          & 76.2500                 & $-$0.7750                 & 4,  5,  6                                & \nodata                           & 2.88                          & $-$0.39                             & $-$1.84                             & $-$0.44                         & 15.1                      & 5.4                       & 26.5                         & 1.37E+22                                & 1                        \\
                    2266                    & 07634$-$0062$-$0005          & 76.3417                 & $-$0.6250                 & 3,  4,  6                                & 11                                & 2.87                          & $-$0.54                             & $-$1.43                             & $-$0.37                         & 21.3                      & 6.5                       & 25.3                         & 1.19E+22                                & 1                        \\
                    2267                    & 07637$-$0091$+$0020          & 76.3667                 & $-$0.9083                 & 4,  5,  6                                & \nodata                           & 1.30                          & 2.05                              & 0.99                              & $-$0.69                         & 11.6                      & 1.9                       & 22.5                         & 1.82E+21                                & 1                        \\
                    2268                    & 07642$-$0073$+$0022          & 76.4167                 & $-$0.7333                 & 3,  4,  5,  6                            & \nodata                           & 1.30                          & 2.24                              & 1.29                              & $-$0.49                         & 10.2                      & 2.6                       & 14.8                         & 2.70E+21                                & 1                        \\
                    2269                    & 07642$-$0083$+$0027          & 76.4167                 & $-$0.8333                 & 4,  5,  6                                & \nodata                           & 1.30                          & 2.68                              & 2.01                              & $-$0.31                         & 13.7                      & 3.4                       & 19.6                         & 4.50E+21                                & 1                        \\
                    2270                    & 07673$-$0038$-$0048          & 76.7333                 & $-$0.3833                 & 4,  6                                    & \nodata                           & 3.25                          & $-$4.81                             & $-$5.34                             & $-$0.27                         & 8.0                       & 1.9                       & 15.3                         & 1.93E+21                                & 1                        \\
                    2271                    & 07693$+$0032$+$0012          & 76.9333                 & 0.3167                  & 3,  4,  6                                & \nodata                           & 1.31                          & 1.20                              & 0.37                              & $-$0.64                         & 8.2                       & 2.8                       & 14.3                         & 1.91E+21                                & 1                        \\
                    2272                    & 07695$-$0057$+$0012          & 76.9500                 & $-$0.5750                 & 3,  4,  6                                & \nodata                           & 1.31                          & 1.16                              & 0.45                              & $-$0.73                         & 3.4                       & 1.1                       & 10.2                         & 4.66E+20                                & 1                        \\
                    2273                    & 07707$+$0356$-$0124          & 77.0667                 & 3.5583                  & 4,  6                                    & \nodata                           & 1.55                          & $-$12.41                            & $-$13.24                            & $-$0.46                         & 4.0                       & 2.2                       & 8.7                          & 1.99E+21                                & 1                        \\
                    2274                    & 07718$-$0032$-$0206          & 77.1833                 & $-$0.3167                 & 4,  6                                    & \nodata                           & 3.55                          & $-$20.59                            & $-$20.96                            & $-$0.33                         & 2.3                       & 1.3                       & 6.1                          & 8.19E+20                                & 1                        \\
                    2275                    & 07727$+$0018$+$0012          & 77.2667                 & 0.1833                  & 4,  6                                    & \nodata                           & 2.59                          & 1.17                              & 0.73                              & $-$0.37                         & 7.6                       & 2.4                       & 13.3                         & 1.42E+21                                & 1                        \\
                    2276                    & 07735$+$0002$-$0332          & 77.3500                 & 0.0167                  & 4,  6                                    & \nodata                           & 6.18                          & $-$33.21                            & $-$33.74                            & $-$0.48                         & 5.9                       & 1.5                       & 14.2                         & 7.85E+20                                & 1                        \\
                    2277                    & 07740$+$0187$+$0013          & 77.4000                 & 1.8750                  & 3,  4,  5,  6                            & \nodata                           & 2.60                          & 1.34                              & 0.73                              & $-$0.37                         & 16.7                      & 4.8                       & 21.6                         & 5.23E+21                                & 1                        \\
                    2278                    & 07746$+$0173$+$0013          & 77.4583                 & 1.7333                  & 3,  4,  5,  6                            & \nodata                           & 1.38                          & 1.28                              & 0.33                              & $-$0.38                         & 22.3                      & 11.8                      & 34.4                         & 3.05E+22                                & 1                        \\
                    2279                    & 07747$+$0177$+$0019          & 77.4750                 & 1.7667                  & 1,  3,  4,  5,  6                        & \nodata                           & 1.37                          & 1.93                              & 0.72                              & $-$0.45                         & 19.3                      & 10.6                      & 24.9                         & 2.54E+22                                & 1                        \\
                    2280                    & 07747$+$0184$+$0016          & 77.4750                 & 1.8417                  & 3,  4,  5,  6                            & \nodata                           & 1.78                          & 1.57                              & $-$0.14                             & $-$0.41                         & 18.2                      & 6.4                       & 32.4                         & 2.35E+22                                & 1                        \\
                    2281                    & 07749$+$0180$+$0015          & 77.4917                 & 1.8000                  & 3,  4,  5,  6                            & \nodata                           & 1.38                          & 1.52                              & 0.54                              & $-$0.44                         & 18.7                      & 6.4                       & 27.4                         & 1.15E+22                                & 1                        \\
                    2282                    & 07758$+$0161$+$0032          & 77.5833                 & 1.6083                  & 4,  5,  6                                & \nodata                           & 1.34                          & 3.18                              & 2.79                              & $-$0.35                         & 8.2                       & 3.2                       & 12.3                         & 1.88E+21                                & 1                        \\
                    2283                    & 07780$+$0041$+$0015          & 77.8000                 & 0.4083                  & 4,  6                                    & \nodata                           & 1.61                          & 1.49                              & 0.65                              & $-$0.68                         & 6.8                       & 4.3                       & 14.1                         & 3.10E+21                                & 1                        \\
                    2284                    & 07780$-$0176$+$0007          & 77.8000                 & $-$1.7583                 & 3,  4,  5,  6                            & \nodata                           & 1.70                          & 0.72                              & $-$0.30                             & $-$0.56                         & 7.7                       & 1.5                       & 12.3                         & 1.31E+21                                & 1                        \\
                    2285                    & 07783$+$0052$+$0013          & 77.8333                 & 0.5167                  & 4,  6                                    & \nodata                           & 1.67                          & 1.31                              & 0.39                              & $-$0.57                         & 6.7                       & 2.8                       & 11.0                         & 2.30E+21                                & 1                        \\
                    2286                    & 07787$-$0113$-$0019          & 77.8750                 & $-$1.1333                 & 4,  6                                    & \nodata                           & 2.62                          & $-$1.88                             & $-$2.94                             & $-$0.38                         & 12.3                      & 4.2                       & 17.6                         & 7.13E+21                                & 1                        \\
                    2287                    & 07791$-$0116$-$0007          & 77.9083                 & $-$1.1583                 & 1,  4,  6                                & \nodata                           & 2.60                          & $-$0.67                             & $-$2.71                             & $-$0.55                         & 11.0                      & 5.3                       & 19.2                         & 1.27E+22                                & 1                        \\
                    2288                    & 07792$+$0164$-$0024          & 77.9250                 & 1.6417                  & 3,  4,  5,  6                            & \nodata                           & 2.71                          & $-$2.41                             & $-$3.02                             & $-$0.34                         & 10.1                      & 10.3                      & 20.3                         & 1.56E+22                                & 1                        \\
                    2289                    & 07792$+$0177$-$0014          & 77.9250                 & 1.7750                  & 3,  4,  5,  6                            & \nodata                           & 2.70                          & $-$1.41                             & $-$2.56                             & $-$0.69                         & 13.9                      & 4.8                       & 20.7                         & 5.33E+21                                & 1                        \\
                    2290                    & 07793$+$0174$-$0015          & 77.9333                 & 1.7417                  & 3,  4,  5,  6                            & \nodata                           & 2.70                          & $-$1.53                             & $-$2.41                             & $-$0.29                         & 22.4                      & 6.1                       & 34.5                         & 1.69E+22                                & 1                        \\
                    2291                    & 07794$-$0069$+$0017          & 77.9417                 & $-$0.6917                 & 4,  6                                    & \nodata                           & 2.51                          & 1.67                              & 1.16                              & $-$0.51                         & 4.8                       & 1.7                       & 8.6                          & 8.02E+20                                & 1                        \\
                    2292                    & 07795$-$0112$-$0015          & 77.9500                 & $-$1.1167                 & 4,  6                                    & \nodata                           & 2.61                          & $-$1.52                             & $-$2.61                             & $-$0.52                         & 13.6                      & 5.5                       & 19.3                         & 7.72E+21                                & 1                        \\
                    2293                    & 07796$-$0118$-$0002          & 77.9583                 & $-$1.1833                 & 4,  6                                    & \nodata                           & 2.70                          & $-$0.23                             & $-$1.85                             & $-$0.58                         & 9.7                       & 4.0                       & 15.4                         & 6.62E+21                                & 1                        \\
                    2294                    & 07797$-$0002$-$0028          & 77.9750                 & $-$0.0250                 & 3,  4,  6                                & \nodata                           & 2.66                          & $-$2.79                             & $-$3.67                             & $-$0.37                         & 9.5                       & 4.0                       & 16.2                         & 5.63E+21                                & 1                        \\
                    2295                    & 07797$-$0180$+$0020          & 77.9750                 & $-$1.8000                 & 3,  4,  5,  6                            & \nodata                           & 1.29                          & 1.98                              & 1.32                              & $-$0.32                         & 7.7                       & 2.6                       & 13.0                         & 2.69E+21                                & 1                        \\
                    2296                    & 07798$-$0118$-$0003          & 77.9833                 & $-$1.1833                 & 4,  6                                    & \nodata                           & 2.67                          & $-$0.31                             & $-$1.49                             & $-$0.50                         & 8.6                       & 4.4                       & 14.9                         & 6.17E+21                                & 1                        \\
                    2297                    & 07804$-$0101$+$0007          & 78.0417                 & $-$1.0083                 & 4,  6,  7                                & \nodata                           & 1.70                          & 0.67                              & $-$0.53                             & $-$0.45                         & 10.8                      & 4.2                       & 17.7                         & 6.86E+21                                & 1                        \\
                    2298                    & 07808$+$0206$+$0137          & 78.0833                 & 2.0583                  & 3,  4,  5,  6                            & \nodata                           & 1.29                          & 13.72                             & 13.30                             & $-$0.54                         & 7.3                       & 4.6                       & 15.2                         & 2.21E+21                                & 1                        \\
                    2299                    & 07811$+$0146$+$0001          & 78.1083                 & 1.4583                  & 4,  5,  6,  7                            & \nodata                           & 1.66                          & 0.07                              & $-$0.62                             & $-$0.27                         & 7.7                       & 4.3                       & 15.2                         & 6.42E+21                                & 1                        \\
                    2300                    & 07813$+$0137$+$0006          & 78.1333                 & 1.3667                  & 4,  5,  6                                & \nodata                           & 1.66                          & 0.62                              & $-$0.27                             & $-$0.51                         & 6.7                       & 4.9                       & 12.9                         & 5.32E+21                                & 1                        \\
                    2301                    & 07817$-$0200$+$0013          & 78.1667                 & $-$2.0000                 & 3,  4,  6                                & \nodata                           & 1.31                          & 1.27                              & 0.65                              & $-$0.76                         & 3.2                       & 1.1                       & 7.7                          & 4.25E+20                                & 1                        \\
                    2302                    & 07817$-$0054$+$0005          & 78.1750                 & $-$0.5417                 & 3,  4,  6                                & \nodata                           & 1.71                          & 0.47                              & $-$0.09                             & $-$0.27                         & 11.1                      & 2.8                       & 16.5                         & 3.25E+21                                & 1                        \\
                    2303                    & 07819$-$0129$+$0009          & 78.1917                 & $-$1.2917                 & 4,  6                                    & \nodata                           & 1.64                          & 0.86                              & 0.04                              & $-$0.31                         & 11.9                      & 3.3                       & 17.5                         & 5.06E+21                                & 1                        \\
                    2304                    & 07821$-$0150$+$0008          & 78.2083                 & $-$1.5000                 & 4,  6                                    & \nodata                           & 1.32                          & 0.82                              & $-$0.88                             & $-$0.83                         & 10.9                      & 3.9                       & 19.5                         & 5.04E+21                                & 1                        \\
                    2305                    & 07822$-$0103$-$0004          & 78.2167                 & $-$1.0333                 & 4,  6                                    & \nodata                           & 2.42                          & $-$0.40                             & $-$2.35                             & $-$0.52                         & 13.9                      & 5.7                       & 20.5                         & 1.46E+22                                & 1                        \\
                    2306                    & 07822$+$0036$+$0002          & 78.2250                 & 0.3583                  & 4,  6                                    & \nodata                           & 1.77                          & 0.15                              & $-$0.49                             & $-$0.38                         & 8.7                       & 4.3                       & 15.4                         & 4.24E+21                                & 1                        \\
                    2307                    & 07826$+$0026$-$0275          & 78.2583                 & 0.2583                  & 4,  6                                    & \nodata                           & 5.99                          & $-$27.45                            & $-$27.76                            & $-$0.37                         & 4.8                       & 1.1                       & 11.4                         & 4.07E+20                                & 1                        \\
                    2308                    & 07829$+$0106$+$0014          & 78.2917                 & 1.0583                  & 4,  5,  6                                & \nodata                           & 1.69                          & 1.41                              & 0.68                              & $-$0.34                         & 4.9                       & 1.8                       & 11.7                         & 1.38E+22                                & 2                        \\
                    2309                    & 07831$+$0132$+$0003          & 78.3083                 & 1.3250                  & 4,  5,  6,  7                            & \nodata                           & 1.67                          & 0.27                              & $-$0.63                             & $-$0.42                         & 7.2                       & 3.8                       & 14.6                         & 4.59E+21                                & 1                        \\
                    2310                    & 07837$+$0030$+$0090          & 78.3667                 & 0.3000                  & 4,  6                                    & \nodata                           & 1.30                          & 9.03                              & 8.23                              & $-$0.38                         & 10.3                      & 6.2                       & 18.6                         & 8.90E+21                                & 1                        \\
                    2311                    & 07843$+$0160$+$0028          & 78.4333                 & 1.6000                  & 4,  5,  6                                & \nodata                           & 1.66                          & 2.80                              & 2.20                              & $-$0.53                         & 5.1                       & 3.1                       & 10.7                         & 1.87E+21                                & 1                        \\
                    2312                    & 07843$-$0103$+$0114          & 78.4333                 & $-$1.0333                 & 4,  6                                    & \nodata                           & 1.30                          & 11.41                             & 11.08                             & $-$0.28                         & 10.9                      & 3.6                       & 15.9                         & 2.41E+21                                & 1                        \\
                    2313                    & 07844$+$0115$+$0008          & 78.4417                 & 1.1500                  & 4,  5,  6                                & \nodata                           & 2.87                          & 0.79                              & 0.17                              & $-$0.54                         & 7.0                       & 4.1                       & 11.7                         & 2.75E+21                                & 1                        \\
                    2314                    & 07844$+$0117$+$0006          & 78.4417                 & 1.1750                  & 4,  5,  6,  7                            & \nodata                           & 2.85                          & 0.63                              & 0.21                              & $-$0.38                         & 7.8                       & 4.2                       & 14.0                         & 2.67E+21                                & 1                        \\
                    2315                    & 07846$+$0122$+$0005          & 78.4583                 & 1.2167                  & 4,  5,  6,  7                            & \nodata                           & 2.40                          & 0.48                              & $-$0.06                             & $-$0.50                         & 6.1                       & 4.4                       & 11.3                         & 2.89E+21                                & 1                        \\
                    2316                    & 07846$-$0134$-$0026          & 78.4583                 & $-$1.3417                 & 1,  4,  6                                & \nodata                           & 2.43                          & $-$2.58                             & $-$3.82                             & $-$0.41                         & 12.4                      & 3.3                       & 19.3                         & 6.09E+21                                & 1                        \\
                    2317                    & 07848$+$0120$+$0004          & 78.4833                 & 1.2000                  & 4,  5,  6,  7                            & \nodata                           & 1.70                          & 0.38                              & $-$0.16                             & $-$0.45                         & 5.8                       & 3.7                       & 10.5                         & 2.51E+21                                & 1                        \\
                    2318                    & 07856$+$0116$+$0004          & 78.5583                 & 1.1583                  & 3,  4,  5,  6                            & \nodata                           & 2.41                          & 0.43                              & $-$0.48                             & $-$0.78                         & 7.2                       & 2.7                       & 13.0                         & 1.66E+21                                & 1                        \\
                    2319                    & 07857$+$0041$-$0040          & 78.5750                 & 0.4083                  & 4,  6                                    & \nodata                           & 2.57                          & $-$3.95                             & $-$4.89                             & $-$0.32                         & 10.7                      & 4.7                       & 15.2                         & 8.17E+21                                & 1                        \\
                    2320                    & 07859$+$0026$+$0091          & 78.5917                 & 0.2583                  & 4,  6                                    & \nodata                           & 1.30                          & 9.07                              & 7.85                              & $-$0.54                         & 8.6                       & 2.3                       & 13.3                         & 2.56E+21                                & 1                        \\
                    2321                    & 07862$+$0272$+$0092          & 78.6250                 & 2.7167                  & 4,  5,  6                                & \nodata                           & 1.59                          & 9.22                              & 8.64                              & $-$0.50                         & 10.3                      & 2.4                       & 15.5                         & 1.49E+21                                & 1                        \\
                    2322                    & 07864$-$0127$+$0013          & 78.6417                 & $-$1.2750                 & 4,  6                                    & \nodata                           & 1.67                          & 1.34                              & 0.47                              & $-$0.35                         & 9.1                       & 1.6                       & 13.6                         & 1.96E+21                                & 1                        \\
                    2323                    & 07868$-$0166$+$0008          & 78.6833                 & $-$1.6583                 & 1,  4,  6                                & \nodata                           & 1.66                          & 0.80                              & $-$0.64                             & $-$0.91                         & 4.8                       & 3.7                       & 9.6                          & 3.51E+21                                & 1                        \\
                    2324                    & 07870$+$0031$+$0011          & 78.7000                 & 0.3083                  & 4,  6                                    & \nodata                           & 2.44                          & 1.15                              & 0.01                              & $-$0.58                         & 5.5                       & 1.8                       & 10.2                         & 1.62E+21                                & 1                        \\
                    2325                    & 07876$+$0261$+$0028          & 78.7583                 & 2.6083                  & 4,  5,  6                                & \nodata                           & 1.61                          & 2.77                              & 2.26                              & $-$0.44                         & 22.3                      & 6.5                       & 29.7                         & 6.42E+21                                & 1                        \\
                    2326                    & 07877$+$0016$-$0013          & 78.7667                 & 0.1583                  & 3,  4,  6                                & \nodata                           & 2.45                          & $-$1.35                             & $-$2.10                             & $-$0.46                         & 6.7                       & 2.7                       & 10.9                         & 2.23E+21                                & 1                        \\
                    2327                    & 07886$+$0017$+$0086          & 78.8583                 & 0.1667                  & 3,  4,  6                                & \nodata                           & 1.29                          & 8.65                              & 7.84                              & $-$0.45                         & 7.7                       & 1.8                       & 14.8                         & 1.61E+21                                & 1                        \\
                    2328                    & 07886$-$0119$+$0024          & 78.8583                 & $-$1.1917                 & 4,  6                                    & \nodata                           & 1.67                          & 2.40                              & 1.89                              & $-$0.38                         & 7.4                       & 2.4                       & 11.7                         & 1.59E+21                                & 1                        \\
                    2329                    & 07887$+$0012$+$0087          & 78.8667                 & 0.1167                  & 3,  4,  6                                & \nodata                           & 1.29                          & 8.73                              & 8.17                              & $-$0.34                         & 7.8                       & 2.6                       & 13.0                         & 2.17E+21                                & 1                        \\
                    2330                    & 07888$-$0172$+$0002          & 78.8833                 & $-$1.7167                 & 4,  6                                    & \nodata                           & 1.56                          & 0.23                              & $-$0.29                             & $-$0.26                         & 7.1                       & 3.4                       & 13.2                         & 3.65E+21                                & 1                        \\
                    2331                    & 07901$+$0035$+$0063          & 79.0083                 & 0.3500                  & 4,  6                                    & \nodata                           & 1.64                          & 6.28                              & 5.16                              & $-$0.50                         & 13.2                      & 5.7                       & 17.9                         & 8.42E+21                                & 1                        \\
                    2332                    & 07903$-$0161$+$0010          & 79.0333                 & $-$1.6083                 & 4,  6                                    & \nodata                           & 1.62                          & 1.03                              & 0.05                              & $-$0.68                         & 5.4                       & 2.2                       & 10.2                         & 1.58E+21                                & 1                        \\
                    2333                    & 07904$+$0064$+$0066          & 79.0417                 & 0.6417                  & 4,  6                                    & \nodata                           & 1.62                          & 6.64                              & 6.04                              & $-$0.31                         & 22.2                      & 11.3                      & 27.6                         & 2.08E+22                                & 1                        \\
                    2334                    & 07906$+$0217$+$0101          & 79.0583                 & 2.1750                  & 4,  5,  6                                & \nodata                           & 1.30                          & 10.13                             & 9.35                              & $-$0.52                         & 6.7                       & 3.1                       & 13.9                         & 2.50E+21                                & 1                        \\
                    2335                    & 07914$+$0056$-$0018          & 79.1417                 & 0.5583                  & 4,  5,  6                                & \nodata                           & 2.35                          & $-$1.76                             & $-$2.17                             & $-$0.27                         & 12.4                      & 3.8                       & 20.1                         & 3.64E+21                                & 1                        \\
                    2336                    & 07914$+$0059$-$0005          & 79.1417                 & 0.5917                  & 4,  5,  6                                & \nodata                           & 1.63                          & $-$0.46                             & $-$1.95                             & $-$0.49                         & 10.5                      & 4.4                       & 16.4                         & 8.00E+21                                & 1                        \\
                    2337                    & 07924$+$0047$+$0006          & 79.2417                 & 0.4750                  & 4,  5,  6,  7                            & \nodata                           & 1.63                          & 0.61                              & $-$0.37                             & $-$0.43                         & 7.2                       & 4.7                       & 13.5                         & 6.58E+21                                & 1                        \\
                    2338                    & 07924$+$0053$+$0003          & 79.2417                 & 0.5333                  & 4,  5,  6,  7                            & 17                                & 1.63                          & 0.27                              & $-$0.45                             & $-$0.51                         & 4.5                       & 2.5                       & 9.0                          & 1.41E+22                                & 2                        \\
                    2339                    & 07926$-$0128$+$0071          & 79.2583                 & $-$1.2833                 & 3,  4,  6,  7                            & \nodata                           & 1.62                          & 7.06                              & 5.57                              & $-$0.37                         & 10.5                      & 2.9                       & 16.6                         & 6.53E+21                                & 1                        \\
                    2340                    & 07928$+$0305$+$0141          & 79.2833                 & 3.0500                  & 4,  5,  6                                & \nodata                           & 1.29                          & 14.14                             & 13.32                             & $-$0.45                         & 5.2                       & 1.4                       & 11.1                         & 1.16E+21                                & 1                        \\
                    2341                    & 07939$+$0019$-$0003          & 79.3917                 & 0.1917                  & 3,  4,  5,  6                            & \nodata                           & 1.63                          & $-$0.35                             & $-$1.35                             & $-$0.69                         & 7.9                       & 3.2                       & 13.4                         & 2.44E+21                                & 1                        \\
                    2342                    & 07939$+$0027$+$0004          & 79.3917                 & 0.2667                  & 3,  4,  5,  6                            & \nodata                           & 1.63                          & 0.37                              & $-$0.60                             & $-$0.42                         & 8.8                       & 4.1                       & 20.0                         & 5.93E+21                                & 1                        \\
                    2343                    & 07942$+$0013$+$0004          & 79.4167                 & 0.1333                  & 4,  5,  6                                & \nodata                           & 1.63                          & 0.35                              & $-$0.73                             & $-$0.91                         & 7.0                       & 4.4                       & 13.5                         & 3.07E+21                                & 1                        \\
                    2344                    & 07942$+$0017$+$0002          & 79.4167                 & 0.1667                  & 3,  4,  5,  6                            & \nodata                           & 1.63                          & 0.19                              & $-$0.91                             & $-$0.76                         & 6.8                       & 3.9                       & 11.5                         & 3.19E+21                                & 1                        \\
                    2345                    & 07942$+$0021$-$0002          & 79.4167                 & 0.2083                  & 3,  4,  5,  6                            & \nodata                           & 1.63                          & $-$0.17                             & $-$1.07                             & $-$0.52                         & 7.2                       & 4.6                       & 12.9                         & 4.77E+21                                & 1                        \\
                    2346                    & 07943$+$0095$-$0029          & 79.4333                 & 0.9500                  & 4,  5,  6                                & \nodata                           & 1.62                          & $-$2.92                             & $-$3.96                             & $-$0.53                         & 7.3                       & 4.3                       & 12.1                         & 5.06E+21                                & 1                        \\
                    2347                    & 07944$+$0018$+$0001          & 79.4417                 & 0.1833                  & 3,  4,  5,  6                            & \nodata                           & 1.63                          & 0.14                              & $-$0.66                             & $-$0.53                         & 6.1                       & 5.2                       & 12.2                         & 5.07E+21                                & 1                        \\
                    2348                    & 07945$+$0021$-$0001          & 79.4500                 & 0.2083                  & 3,  4,  5,  6                            & \nodata                           & 1.63                          & $-$0.10                             & $-$1.05                             & $-$0.51                         & 6.6                       & 4.1                       & 15.1                         & 4.44E+21                                & 1                        \\
                    2349                    & 07945$-$0087$+$0075          & 79.4500                 & $-$0.8667                 & 1,  4,  6                                & \nodata                           & 1.73                          & 7.48                              & 6.61                              & $-$0.41                         & 10.6                      & 3.2                       & 16.9                         & 3.91E+21                                & 1                        \\
                    2350                    & 07948$+$0025$-$0001          & 79.4833                 & 0.2500                  & 4,  5,  6                                & \nodata                           & 1.63                          & $-$0.13                             & $-$1.16                             & $-$0.47                         & 7.5                       & 4.6                       & 12.4                         & 6.00E+21                                & 1                        \\
                    2351                    & 07954$-$0125$+$0073          & 79.5417                 & $-$1.2500                 & 4,  6                                    & \nodata                           & 1.62                          & 7.32                              & 6.29                              & $-$0.36                         & 8.0                       & 1.7                       & 12.6                         & 2.37E+21                                & 1                        \\
                    2352                    & 07955$+$0004$+$0008          & 79.5500                 & 0.0417                  & 4,  5,  6                                & \nodata                           & 1.63                          & 0.84                              & 0.15                              & $-$0.44                         & 7.0                       & 4.6                       & 14.4                         & 4.19E+21                                & 1                        \\
                    2353                    & 07957$+$0007$+$0008          & 79.5750                 & 0.0667                  & 4,  5,  6                                & \nodata                           & 1.63                          & 0.84                              & 0.31                              & $-$0.37                         & 6.1                       & 4.1                       & 11.1                         & 3.49E+21                                & 1                        \\
                    2354                    & 07984$-$0038$+$0067          & 79.8417                 & $-$0.3833                 & 4,  5,  6                                & \nodata                           & 1.63                          & 6.73                              & 5.80                              & $-$0.42                         & 8.1                       & 2.4                       & 13.6                         & 2.67E+21                                & 1                        \\
                    2355                    & 07991$-$0037$+$0069          & 79.9083                 & $-$0.3667                 & 4,  5,  6                                & \nodata                           & 1.63                          & 6.89                              & 5.93                              & $-$0.50                         & 6.9                       & 2.0                       & 11.6                         & 1.87E+21                                & 1                        \\
                    2356                    & 07992$+$0241$+$0055          & 79.9167                 & 2.4083                  & 4,  5,  6                                & \nodata                           & 1.60                          & 5.53                              & 4.77                              & $-$0.37                         & 8.9                       & 2.7                       & 15.4                         & 3.05E+21                                & 1                        \\
                    2357                    & 07996$-$0297$-$0008          & 79.9583                 & $-$2.9750                 & 4,  5,  6                                & \nodata                           & 1.57                          & $-$0.75                             & $-$1.56                             & $-$0.39                         & 8.7                       & 1.2                       & 13.1                         & 1.13E+21                                & 1                        \\
                    2358                    & 08012$-$0122$-$0038          & 80.1167                 & $-$1.2167                 & 4,  6                                    & \nodata                           & 1.60                          & $-$3.79                             & $-$4.63                             & $-$0.67                         & 6.2                       & 3.8                       & 12.7                         & 2.63E+21                                & 1                        \\
                    2359                    & 08013$-$0033$-$0017          & 80.1333                 & $-$0.3333                 & 4,  5,  6                                & \nodata                           & 1.61                          & $-$1.74                             & $-$2.22                             & $-$0.37                         & 5.0                       & 1.1                       & 9.9                          & 6.24E+20                                & 1                        \\
                    2360                    & 08015$-$0092$-$0028          & 80.1500                 & $-$0.9167                 & 4,  5,  6                                & \nodata                           & 1.60                          & $-$2.82                             & $-$3.74                             & $-$0.47                         & 6.8                       & 3.2                       & 12.4                         & 3.26E+21                                & 1                        \\
                    2361                    & 08015$-$0119$-$0039          & 80.1500                 & $-$1.1917                 & 4,  6                                    & \nodata                           & 1.60                          & $-$3.86                             & $-$4.66                             & $-$0.39                         & 6.4                       & 2.2                       & 12.6                         & 2.22E+21                                & 1                        \\
                    2362                    & 08017$-$0223$-$0039          & 80.1750                 & $-$2.2333                 & 4,  5,  6                                & \nodata                           & 1.59                          & $-$3.90                             & $-$4.56                             & $-$0.29                         & 14.4                      & 4.9                       & 24.0                         & 8.02E+21                                & 1                        \\
                    2363                    & 08021$+$0292$+$0070          & 80.2083                 & 2.9167                  & 4,  6                                    & \nodata                           & 1.47                          & 7.02                              & 6.60                              & $-$0.24                         & 9.6                       & 4.2                       & 12.8                         & 4.23E+21                                & 1                        \\
                    2364                    & 08027$+$0283$+$0052          & 80.2667                 & 2.8333                  & 4,  6                                    & \nodata                           & 1.56                          & 5.18                              & 4.32                              & $-$0.52                         & 10.2                      & 3.8                       & 25.3                         & 4.43E+21                                & 1                        \\
                    2365                    & 08032$-$0267$-$0011          & 80.3250                 & $-$2.6750                 & 1,  4,  5,  6                            & \nodata                           & 1.59                          & $-$1.15                             & $-$1.66                             & $-$0.31                         & 5.4                       & 3.5                       & 9.6                          & 3.39E+21                                & 1                        \\
                    2366                    & 08039$-$0266$-$0038          & 80.3917                 & $-$2.6583                 & 4,  5,  6                                & \nodata                           & 1.58                          & $-$3.77                             & $-$4.51                             & $-$0.35                         & 7.4                       & 4.6                       & 13.4                         & 5.69E+21                                & 1                        \\
                    2367                    & 08042$+$0022$+$0076          & 80.4167                 & 0.2250                  & 4,  5,  6                                & \nodata                           & 2.09                          & 7.57                              & 6.75                              & $-$0.78                         & 5.9                       & 2.7                       & 12.0                         & 1.47E+21                                & 1                        \\
                    2368                    & 08046$-$0269$-$0037          & 80.4583                 & $-$2.6917                 & 4,  5,  6                                & \nodata                           & 1.58                          & $-$3.67                             & $-$4.60                             & $-$0.65                         & 5.2                       & 2.6                       & 10.8                         & 1.91E+21                                & 1                        \\
                    2369                    & 08055$-$0002$+$0016          & 80.5500                 & $-$0.0167                 & 4,  5,  6                                & \nodata                           & 1.95                          & 1.62                              & 0.80                              & $-$0.35                         & 10.1                      & 3.1                       & 15.7                         & 4.00E+21                                & 1                        \\
                    2370                    & 08075$-$0082$+$0039          & 80.7500                 & $-$0.8167                 & 4,  5,  6                                & \nodata                           & 1.60                          & 3.92                              & 3.50                              & $-$0.27                         & 7.0                       & 2.9                       & 12.7                         & 2.39E+21                                & 1                        \\
                    2371                    & 08084$-$0284$-$0009          & 80.8417                 & $-$2.8417                 & 4,  5,  6                                & \nodata                           & 1.58                          & $-$0.86                             & $-$1.67                             & $-$0.77                         & 5.6                       & 1.9                       & 9.0                          & 9.90E+20                                & 1                        \\
                    2372                    & 08085$+$0044$-$0030          & 80.8500                 & 0.4417                  & 3,  4,  5,  6                            & \nodata                           & 1.59                          & $-$3.02                             & $-$4.06                             & $-$0.39                         & 13.6                      & 7.4                       & 16.9                         & 1.47E+22                                & 1                        \\
                    2373                    & 08102$-$0082$+$0060          & 81.0167                 & $-$0.8167                 & 4,  5,  6                                & \nodata                           & 1.54                          & 6.01                              & 5.36                              & $-$0.57                         & 4.1                       & 1.6                       & 9.2                          & 8.48E+20                                & 1                        \\
                    2374                    & 08103$-$0021$+$0049          & 81.0333                 & $-$0.2083                 & 3,  4,  5,  6                            & \nodata                           & 1.55                          & 4.94                              & 4.11                              & $-$0.45                         & 6.5                       & 3.0                       & 13.2                         & 3.02E+21                                & 1                        \\
                    2375                    & 08104$-$0046$+$0057          & 81.0417                 & $-$0.4583                 & 4,  5,  6                                & \nodata                           & 1.54                          & 5.66                              & 4.85                              & $-$0.79                         & 3.6                       & 1.5                       & 7.3                          & 5.83E+21                                & 2                        \\
                    2376                    & 08105$-$0031$+$0057          & 81.0500                 & $-$0.3083                 & 4,  5,  6                                & \nodata                           & 1.54                          & 5.72                              & 4.76                              & $-$0.42                         & 7.6                       & 2.9                       & 14.8                         & 3.60E+21                                & 1                        \\
                    2377                    & 08108$-$0014$-$0044          & 81.0833                 & $-$0.1417                 & 4,  5,  6                                & \nodata                           & 1.57                          & $-$4.44                             & $-$5.05                             & $-$0.28                         & 16.5                      & 4.4                       & 27.0                         & 7.09E+21                                & 1                        \\
                    2378                    & 08114$-$0012$-$0045          & 81.1417                 & $-$0.1167                 & 4,  5,  6                                & \nodata                           & 1.57                          & $-$4.50                             & $-$5.20                             & $-$0.32                         & 19.3                      & 9.5                       & 30.8                         & 1.95E+22                                & 1                        \\
                    2379                    & 08122$-$0055$+$0059          & 81.2167                 & $-$0.5500                 & 4,  5,  6                                & \nodata                           & 1.54                          & 5.94                              & 4.93                              & $-$0.69                         & 5.9                       & 1.5                       & 10.1                         & 1.02E+21                                & 1                        \\
                    2380                    & 08124$+$0087$+$0145          & 81.2417                 & 0.8667                  & 4,  5,  6                                & \nodata                           & 1.31                          & 14.46                             & 12.99                             & $-$0.44                         & 19.1                      & 7.2                       & 30.5                         & 2.07E+22                                & 1                        \\
                    2381                    & 08138$-$0140$+$0035          & 81.3833                 & $-$1.4000                 & 4,  5,  6                                & \nodata                           & 0.67                          & 3.48                              & 2.95                              & $-$0.53                         & 6.3                       & 1.3                       & 10.9                         & 6.07E+20                                & 1                        \\
                    2382                    & 08147$+$0027$+$0090          & 81.4667                 & 0.2750                  & 4,  5,  6                                & \nodata                           & 1.41                          & 8.98                              & 7.97                              & $-$0.71                         & 8.2                       & 1.3                       & 17.5                         & 9.54E+20                                & 1                        \\
                    2383                    & 08147$-$0008$+$0104          & 81.4667                 & $-$0.0833                 & 3,  4,  5,  6                            & \nodata                           & 1.68                          & 10.43                             & 9.88                              & $-$0.57                         & 8.7                       & 1.2                       & 14.1                         & 5.74E+20                                & 1                        \\
                    2384                    & 08155$+$0103$+$0027          & 81.5500                 & 1.0333                  & 4,  5,  6                                & \nodata                           & 1.54                          & 2.75                              & 2.23                              & $-$0.33                         & 11.9                      & 5.5                       & 21.5                         & 5.96E+21                                & 1                        \\
                    2385                    & 08157$+$0104$+$0029          & 81.5750                 & 1.0417                  & 4,  5,  6                                & \nodata                           & 1.54                          & 2.88                              & 2.28                              & $-$0.34                         & 8.6                       & 3.6                       & 13.7                         & 3.48E+21                                & 1                        \\
                    2386                    & 08159$+$0151$+$0002          & 81.5917                 & 1.5083                  & 4,  5,  6                                & \nodata                           & 1.55                          & 0.17                              & $-$0.58                             & $-$0.32                         & 11.7                      & 6.2                       & 18.2                         & 9.95E+21                                & 1                        \\
                    2387                    & 08162$+$0111$+$0038          & 81.6167                 & 1.1083                  & 4,  5,  6,  7                            & \nodata                           & 1.53                          & 3.76                              & 2.87                              & $-$0.56                         & 12.8                      & 7.6                       & 18.3                         & 8.92E+21                                & 1                        \\
                    2388                    & 08167$+$0054$-$0025          & 81.6750                 & 0.5417                  & 3,  4,  5,  6                            & \nodata                           & 1.56                          & $-$2.47                             & $-$5.42                             & $-$0.78                         & 30.8                      & 18.6                      & 37.9                         & 8.91E+22                                & 1                        \\
                    2389                    & 08168$+$0106$+$0027          & 81.6833                 & 1.0583                  & 4,  5,  6                                & \nodata                           & 1.54                          & 2.67                              & 2.31                              & $-$0.20                         & 10.6                      & 5.4                       & 14.6                         & 6.28E+21                                & 1                        \\
                    2390                    & 08171$-$0152$+$0026          & 81.7083                 & $-$1.5167                 & 4,  5,  6                                & \nodata                           & 0.67                          & 2.59                              & 2.15                              & $-$0.41                         & 7.6                       & 3.2                       & 14.7                         & 1.85E+21                                & 1                        \\
                    2391                    & 08172$+$0057$-$0030          & 81.7250                 & 0.5750                  & 3,  4,  5,  6,  7                        & 17                                & 1.56                          & $-$3.05                             & $-$5.18                             & $-$0.53                         & 34.3                      & 19.3                      & 49.1                         & 1.11E+23                                & 1                        \\
                    2392                    & 08172$+$0128$+$0037          & 81.7250                 & 1.2833                  & 4,  5,  6                                & 17                                & 1.53                          & 3.73                              & 2.98                              & $-$0.42                         & 10.3                      & 7.1                       & 17.2                         & 9.10E+21                                & 1                        \\
                    2393                    & 08172$-$0160$+$0025          & 81.7250                 & $-$1.6000                 & 3,  4,  5,  6                            & \nodata                           & 0.67                          & 2.54                              & 1.84                              & $-$0.56                         & 6.8                       & 4.0                       & 11.5                         & 2.89E+21                                & 1                        \\
                    2394                    & 08173$+$0134$+$0039          & 81.7333                 & 1.3417                  & 4,  5,  6                                & \nodata                           & 1.53                          & 3.92                              & 2.99                              & $-$0.37                         & 10.2                      & 6.0                       & 22.4                         & 1.10E+22                                & 1                        \\
                    2395                    & 08173$+$0165$+$0024          & 81.7333                 & 1.6500                  & 4,  5,  6                                & \nodata                           & 1.54                          & 2.40                              & 1.61                              & $-$0.40                         & 7.0                       & 4.8                       & 12.6                         & 5.71E+21                                & 1                        \\
                    2396                    & 08175$+$0053$+$0078          & 81.7500                 & 0.5333                  & 3,  4,  5,  6                            & 11,  12,  17                      & 1.46                          & 7.75                              & 6.69                              & $-$0.43                         & 10.8                      & 3.5                       & 17.7                         & 5.03E+21                                & 1                        \\
                    2397                    & 08175$+$0128$+$0040          & 81.7500                 & 1.2833                  & 4,  5,  6                                & 17                                & 1.53                          & 4.00                              & 3.37                              & $-$0.34                         & 11.6                      & 5.7                       & 19.7                         & 7.02E+21                                & 1                        \\
                    2398                    & 08176$+$0160$+$0045          & 81.7583                 & 1.6000                  & 4,  5,  6                                & \nodata                           & 1.52                          & 4.46                              & 3.42                              & $-$0.99                         & 4.2                       & 2.4                       & 9.0                          & 1.30E+21                                & 1                        \\
                    2399                    & 08177$-$0167$+$0030          & 81.7667                 & $-$1.6750                 & 3,  4,  5,  6                            & \nodata                           & 0.67                          & 3.04                              & 2.38                              & $-$0.77                         & 4.7                       & 1.3                       & 8.6                          & 3.89E+21                                & 2                        \\
                    2400                    & 08177$+$0117$+$0047          & 81.7750                 & 1.1667                  & 4,  5,  6,  7                            & \nodata                           & 1.52                          & 4.67                              & 3.93                              & $-$0.51                         & 7.2                       & 3.2                       & 11.4                         & 2.41E+21                                & 1                        \\
                    2401                    & 08177$-$0139$+$0024          & 81.7750                 & $-$1.3917                 & 4,  5,  6                                & \nodata                           & 0.67                          & 2.38                              & 1.99                              & $-$0.41                         & 5.9                       & 2.0                       & 12.0                         & 9.39E+20                                & 1                        \\
                    2402                    & 08182$+$0147$+$0119          & 81.8250                 & 1.4667                  & 4,  5,  6                                & \nodata                           & 1.33                          & 11.88                             & 11.30                             & $-$0.33                         & 7.9                       & 3.7                       & 15.7                         & 3.72E+21                                & 1                        \\
                    2403                    & 08182$+$0149$+$0034          & 81.8250                 & 1.4917                  & 4,  5,  6                                & \nodata                           & 1.53                          & 3.40                              & 2.53                              & $-$0.49                         & 7.9                       & 3.6                       & 15.4                         & 3.65E+21                                & 1                        \\
                    2404                    & 08184$+$0072$+$0091          & 81.8417                 & 0.7250                  & 3,  4,  5,  6                            & \nodata                           & 1.48                          & 9.11                              & 8.26                              & $-$0.33                         & 13.8                      & 3.5                       & 20.8                         & 5.80E+21                                & 1                        \\
                    2405                    & 08184$+$0080$+$0089          & 81.8417                 & 0.8000                  & 3,  4,  5,  6,  7                        & \nodata                           & 1.49                          & 8.91                              & 7.50                              & $-$0.51                         & 14.2                      & 4.6                       & 21.5                         & 8.55E+21                                & 1                        \\
                    2406                    & 08184$+$0082$-$0027          & 81.8417                 & 0.8167                  & 3,  4,  5,  6                            & \nodata                           & 1.56                          & $-$2.73                             & $-$3.37                             & $-$0.33                         & 17.5                      & 8.4                       & 26.1                         & 1.34E+22                                & 1                        \\
                    2407                    & 08186$+$0144$+$0118          & 81.8583                 & 1.4417                  & 4,  5,  6                                & \nodata                           & 1.33                          & 11.77                             & 11.05                             & $-$0.28                         & 9.1                       & 4.4                       & 18.0                         & 6.81E+21                                & 1                        \\
                    2408                    & 08186$-$0172$+$0036          & 81.8583                 & $-$1.7250                 & 3,  4,  5,  6                            & \nodata                           & 0.67                          & 3.57                              & 2.98                              & $-$0.52                         & 4.2                       & 2.0                       & 8.6                          & 8.58E+21                                & 2                        \\
                    2409                    & 08191$+$0143$+$0110          & 81.9083                 & 1.4333                  & 4,  5,  6                                & \nodata                           & 1.34                          & 11.01                             & 10.35                             & $-$0.34                         & 9.9                       & 3.8                       & 18.9                         & 4.51E+21                                & 1                        \\
                    2410                    & 08198$+$0247$-$0036          & 81.9833                 & 2.4750                  & 4,  5,  6                                & \nodata                           & 0.67                          & $-$3.56                             & $-$4.58                             & $-$0.48                         & 8.5                       & 3.9                       & 15.0                         & 4.81E+21                                & 1                        \\
                    2411                    & 08201$+$0031$+$0083          & 82.0083                 & 0.3083                  & 4,  5,  6,  7                            & \nodata                           & 1.38                          & 8.32                              & 7.49                              & $-$0.33                         & 9.7                       & 3.3                       & 15.2                         & 4.65E+21                                & 1                        \\
                    2412                    & 08206$-$0163$+$0028          & 82.0583                 & $-$1.6333                 & 4,  5,  6                                & \nodata                           & 0.67                          & 2.79                              & 2.34                              & $-$0.47                         & 6.5                       & 1.8                       & 12.2                         & 8.24E+20                                & 1                        \\
                    2413                    & 08216$-$0147$+$0025          & 82.1583                 & $-$1.4750                 & 4,  5,  6                                & \nodata                           & 0.67                          & 2.54                              & 2.03                              & $-$0.53                         & 8.0                       & 4.6                       & 12.3                         & 2.70E+21                                & 1                        \\
                    2414                    & 08217$+$0007$+$0101          & 82.1667                 & 0.0667                  & 4,  5,  6                                & 17                                & 1.67                          & 10.05                             & 8.99                              & $-$0.61                         & 13.8                      & 7.2                       & 18.3                         & 8.89E+21                                & 1                        \\
                    2415                    & 08221$-$0153$+$0027          & 82.2083                 & $-$1.5333                 & 4,  5,  6                                & 17                                & 0.67                          & 2.66                              & 2.46                              & $-$0.24                         & 5.8                       & 1.8                       & 12.0                         & 5.48E+21                                & 2                        \\
                    2416                    & 08225$+$0217$-$0028          & 82.2500                 & 2.1750                  & 4,  5,  6                                & \nodata                           & 1.54                          & $-$2.79                             & $-$3.65                             & $-$0.48                         & 6.1                       & 1.6                       & 12.2                         & 1.34E+21                                & 1                        \\
                    2417                    & 08226$-$0151$+$0029          & 82.2583                 & $-$1.5083                 & 4,  5,  6                                & \nodata                           & 0.67                          & 2.94                              & 2.33                              & $-$0.73                         & 9.6                       & 3.3                       & 16.5                         & 1.59E+21                                & 1                        \\
                    2418                    & 08235$+$0006$+$0084          & 82.3500                 & 0.0583                  & 4,  5,  6                                & \nodata                           & 1.36                          & 8.45                              & 7.88                              & $-$0.36                         & 7.0                       & 3.4                       & 11.9                         & 2.83E+21                                & 1                        \\
                    2419                    & 08241$-$0175$+$0040          & 82.4083                 & $-$1.7500                 & 4,  5,  6                                & \nodata                           & 0.67                          & 4.00                              & 3.49                              & $-$0.36                         & 6.7                       & 4.0                       & 14.1                         & 3.21E+21                                & 1                        \\
                    2420                    & 08242$+$0121$+$0045          & 82.4250                 & 1.2083                  & 4,  5,  6                                & \nodata                           & 1.49                          & 4.53                              & 3.81                              & $-$0.75                         & 4.2                       & 2.3                       & 8.1                          & 1.22E+21                                & 1                        \\
                    2421                    & 08243$-$0171$+$0039          & 82.4333                 & $-$1.7083                 & 4,  5,  6                                & \nodata                           & 0.67                          & 3.94                              & 3.52                              & $-$0.41                         & 8.2                       & 3.9                       & 16.8                         & 2.38E+21                                & 1                        \\
                    2422                    & 08243$-$0177$+$0040          & 82.4333                 & $-$1.7750                 & 4,  5,  6                                & \nodata                           & 0.67                          & 4.02                              & 3.29                              & $-$0.43                         & 7.6                       & 4.2                       & 16.8                         & 4.27E+21                                & 1                        \\
                    2423                    & 08246$-$0172$+$0041          & 82.4583                 & $-$1.7167                 & 4,  5,  6                                & \nodata                           & 0.67                          & 4.10                              & 3.72                              & $-$0.29                         & 9.0                       & 2.7                       & 15.9                         & 1.93E+21                                & 1                        \\
                    2424                    & 08248$-$0178$+$0044          & 82.4833                 & $-$1.7833                 & 4,  5,  6                                & \nodata                           & 0.67                          & 4.41                              & 3.76                              & $-$0.43                         & 7.5                       & 2.9                       & 14.4                         & 2.28E+21                                & 1                        \\
                    2425                    & 08249$-$0174$+$0042          & 82.4917                 & $-$1.7417                 & 4,  5,  6                                & \nodata                           & 0.67                          & 4.18                              & 3.73                              & $-$0.32                         & 8.9                       & 3.1                       & 19.2                         & 2.59E+21                                & 1                        \\
                    2426                    & 08251$-$0178$+$0041          & 82.5083                 & $-$1.7833                 & 4,  5,  6                                & \nodata                           & 0.67                          & 4.13                              & 3.70                              & $-$0.35                         & 7.9                       & 3.9                       & 14.5                         & 2.79E+21                                & 1                        \\
                    2427                    & 08251$-$0195$+$0044          & 82.5083                 & $-$1.9500                 & 4,  5,  6                                & \nodata                           & 0.67                          & 4.36                              & 3.65                              & $-$0.45                         & 7.0                       & 4.3                       & 13.7                         & 3.88E+21                                & 1                        \\
                    2428                    & 08252$+$0317$-$0018          & 82.5167                 & 3.1750                  & 4,  6                                    & \nodata                           & 0.67                          & $-$1.78                             & $-$2.22                             & $-$0.58                         & 4.2                       & 1.5                       & 9.1                          & 5.00E+20                                & 1                        \\
                    2429                    & 08252$-$0192$+$0045          & 82.5167                 & $-$1.9167                 & 4,  5,  6                                & \nodata                           & 0.67                          & 4.45                              & 3.65                              & $-$0.52                         & 7.9                       & 4.9                       & 13.8                         & 4.59E+21                                & 1                        \\
                    2430                    & 08253$+$0005$+$0101          & 82.5333                 & 0.0500                  & 4,  5,  6                                & \nodata                           & 1.34                          & 10.10                             & 8.93                              & $-$0.42                         & 10.6                      & 6.4                       & 23.3                         & 1.31E+22                                & 1                        \\
                    2431                    & 08258$-$0196$+$0042          & 82.5833                 & $-$1.9583                 & 4,  5,  6                                & \nodata                           & 0.67                          & 4.23                              & 3.15                              & $-$0.57                         & 7.9                       & 4.5                       & 16.1                         & 5.08E+21                                & 1                        \\
                    2432                    & 08261$+$0136$+$0000          & 82.6083                 & 1.3583                  & 4,  5,  6                                & \nodata                           & 1.53                          & $-$0.01                             & $-$0.73                             & $-$0.65                         & 4.6                       & 1.9                       & 10.1                         & 1.01E+21                                & 1                        \\
                    2433                    & 08262$+$0130$+$0019          & 82.6250                 & 1.3000                  & 4,  5,  6                                & \nodata                           & 1.52                          & 1.86                              & 0.62                              & $-$0.75                         & 5.8                       & 2.0                       & 12.9                         & 1.64E+21                                & 1                        \\
                    2434                    & 08264$-$0198$+$0041          & 82.6417                 & $-$1.9833                 & 4,  5,  6                                & \nodata                           & 0.67                          & 4.13                              & 3.20                              & $-$0.47                         & 8.7                       & 4.9                       & 18.2                         & 6.10E+21                                & 1                        \\
                    2435                    & 08265$+$0134$+$0005          & 82.6500                 & 1.3417                  & 4,  5,  6                                & \nodata                           & 1.53                          & 0.47                              & $-$0.21                             & $-$0.46                         & 5.9                       & 4.6                       & 9.7                          & 4.70E+21                                & 1                        \\
                    2436                    & 08267$+$0227$+$0044          & 82.6667                 & 2.2667                  & 4,  5,  6                                & \nodata                           & 1.49                          & 4.43                              & 3.89                              & $-$0.46                         & 4.5                       & 1.7                       & 8.8                          & 9.37E+20                                & 1                        \\
                    2437                    & 08271$+$0329$-$0023          & 82.7083                 & 3.2917                  & 3,  4,  6                                & \nodata                           & 0.67                          & $-$2.33                             & $-$2.83                             & $-$0.46                         & 5.4                       & 1.6                       & 12.0                         & 8.47E+20                                & 1                        \\
                    2438                    & 08272$-$0207$+$0042          & 82.7167                 & $-$2.0750                 & 4,  5,  6                                & \nodata                           & 0.67                          & 4.25                              & 3.31                              & $-$0.54                         & 8.4                       & 4.6                       & 20.2                         & 5.27E+21                                & 1                        \\
                    2439                    & 08274$+$0112$+$0033          & 82.7417                 & 1.1250                  & 4,  5,  6                                & \nodata                           & 1.50                          & 3.26                              & 2.45                              & $-$0.48                         & 8.3                       & 5.5                       & 13.7                         & 5.95E+21                                & 1                        \\
                    2440                    & 08275$+$0115$+$0030          & 82.7500                 & 1.1500                  & 4,  5,  6                                & \nodata                           & 1.51                          & 2.98                              & 2.06                              & $-$0.33                         & 8.3                       & 4.8                       & 16.9                         & 8.28E+21                                & 1                        \\
                    2441                    & 08275$-$0207$+$0043          & 82.7500                 & $-$2.0667                 & 4,  5,  6                                & \nodata                           & 0.67                          & 4.28                              & 3.07                              & $-$0.74                         & 8.5                       & 5.3                       & 15.4                         & 5.45E+21                                & 1                        \\
                    2442                    & 08277$+$0332$-$0027          & 82.7750                 & 3.3167                  & 3,  4,  6                                & \nodata                           & 0.67                          & $-$2.67                             & $-$3.48                             & $-$0.57                         & 5.1                       & 2.8                       & 12.1                         & 2.02E+21                                & 1                        \\
                    2443                    & 08279$-$0179$+$0047          & 82.7917                 & $-$1.7917                 & 4,  5,  6                                & \nodata                           & 0.67                          & 4.73                              & 4.16                              & $-$0.40                         & 8.1                       & 3.0                       & 15.3                         & 2.40E+21                                & 1                        \\
                    2444                    & 08280$+$0334$-$0031          & 82.8000                 & 3.3417                  & 3,  4,  6                                & \nodata                           & 0.67                          & $-$3.14                             & $-$3.76                             & $-$0.31                         & 5.6                       & 2.5                       & 10.8                         & 2.49E+21                                & 1                        \\
                    2445                    & 08280$-$0138$+$0036          & 82.8000                 & $-$1.3833                 & 4,  5,  6                                & \nodata                           & 0.67                          & 3.57                              & 2.84                              & $-$0.74                         & 6.8                       & 4.2                       & 15.4                         & 2.40E+21                                & 1                        \\
                    2446                    & 08283$-$0307$+$0024          & 82.8333                 & $-$3.0667                 & 4,  5,  6                                & \nodata                           & 0.67                          & 2.42                              & 1.99                              & $-$0.43                         & 7.7                       & 2.2                       & 16.6                         & 1.20E+21                                & 1                        \\
                    2447                    & 08287$+$0332$-$0028          & 82.8667                 & 3.3250                  & 3,  4,  6                                & \nodata                           & 0.67                          & $-$2.81                             & $-$3.65                             & $-$0.46                         & 5.1                       & 2.2                       & 11.5                         & 1.91E+21                                & 1                        \\
                    2448                    & 08287$-$0133$+$0033          & 82.8750                 & $-$1.3333                 & 4,  5,  6                                & \nodata                           & 0.67                          & 3.26                              & 2.40                              & $-$0.78                         & 9.4                       & 2.8                       & 17.0                         & 1.77E+21                                & 1                        \\
                    2449                    & 08289$+$0332$-$0028          & 82.8917                 & 3.3250                  & 3,  4,  6                                & \nodata                           & 0.67                          & $-$2.84                             & $-$3.63                             & $-$0.46                         & 4.0                       & 1.7                       & 8.1                          & 1.43E+21                                & 1                        \\
                    2450                    & 08293$-$0152$+$0033          & 82.9333                 & $-$1.5167                 & 4,  5,  6                                & \nodata                           & 0.67                          & 3.30                              & 2.75                              & $-$0.58                         & 8.9                       & 2.7                       & 12.6                         & 1.34E+21                                & 1                        \\
                    2451                    & 08293$-$0191$+$0045          & 82.9333                 & $-$1.9083                 & 4,  5,  6                                & \nodata                           & 0.67                          & 4.46                              & 3.24                              & $-$1.33                         & 5.3                       & 3.6                       & 11.5                         & 1.85E+21                                & 1                        \\
                    2452                    & 08294$-$0166$+$0029          & 82.9417                 & $-$1.6583                 & 4,  5,  6                                & \nodata                           & 0.67                          & 2.87                              & 2.54                              & $-$0.33                         & 8.0                       & 2.5                       & 13.8                         & 1.29E+21                                & 1                        \\
                    2453                    & 08297$+$0183$+$0046          & 82.9750                 & 1.8333                  & 4,  5,  6                                & \nodata                           & 1.50                          & 4.64                              & 3.13                              & $-$0.56                         & 6.6                       & 1.4                       & 12.7                         & 1.72E+21                                & 1                        \\
                    2454                    & 08302$+$0210$+$0069          & 83.0167                 & 2.1000                  & 4,  5,  6                                & \nodata                           & 0.67                          & 6.94                              & 5.63                              & $-$0.47                         & 5.9                       & 1.7                       & 15.5                         & 2.39E+21                                & 1                        \\
                    2455                    & 08308$+$0187$+$0058          & 83.0833                 & 1.8667                  & 4,  5,  6                                & \nodata                           & 1.42                          & 5.76                              & 4.97                              & $-$0.34                         & 8.5                       & 2.9                       & 12.9                         & 3.54E+21                                & 1                        \\
                    2456                    & 08310$-$0209$+$0021          & 83.1000                 & $-$2.0917                 & 1,  4,  5,  6                            & \nodata                           & 0.67                          & 2.14                              & 1.67                              & $-$0.56                         & 6.4                       & 2.0                       & 14.7                         & 8.44E+20                                & 1                        \\
                    2457                    & 08312$+$0192$+$0051          & 83.1167                 & 1.9167                  & 4,  5,  6                                & \nodata                           & 1.44                          & 5.15                              & 4.16                              & $-$0.35                         & 8.6                       & 2.8                       & 13.4                         & 4.13E+21                                & 1                        \\
                    2458                    & 08312$+$0233$+$0053          & 83.1167                 & 2.3333                  & 4,  6                                    & \nodata                           & 0.67                          & 5.26                              & 4.26                              & $-$0.34                         & 6.2                       & 1.2                       & 13.8                         & 1.71E+21                                & 1                        \\
                    2459                    & 08337$-$0039$+$0006          & 83.3750                 & $-$0.3917                 & 4,  6                                    & \nodata                           & 1.52                          & 0.65                              & 0.23                              & $-$0.36                         & 5.7                       & 2.4                       & 10.0                         & 1.38E+21                                & 1                        \\
                    2460                    & 08354$-$0210$+$0018          & 83.5417                 & $-$2.1000                 & 4,  5,  6                                & \nodata                           & 0.67                          & 1.85                              & 0.52                              & $-$0.88                         & 6.4                       & 4.2                       & 14.9                         & 3.63E+21                                & 1                        \\
                    2461                    & 08365$-$0058$+$0022          & 83.6500                 & $-$0.5833                 & 4,  6                                    & \nodata                           & 1.50                          & 2.17                              & 1.10                              & $-$0.78                         & 7.7                       & 1.4                       & 14.0                         & 9.11E+20                                & 1                        \\
                    2462                    & 08380$-$0049$+$0011          & 83.8000                 & $-$0.4917                 & 4,  6                                    & \nodata                           & 1.51                          & 1.09                              & 0.25                              & $-$0.86                         & 7.3                       & 1.9                       & 15.7                         & 9.54E+20                                & 1                        \\
                    2463                    & 08382$-$0208$+$0020          & 83.8250                 & $-$2.0833                 & 4,  5,  6                                & \nodata                           & 0.67                          & 1.96                              & 1.25                              & $-$0.48                         & 8.9                       & 3.7                       & 13.3                         & 2.97E+21                                & 1                        \\
                    2464                    & 08407$-$0124$+$0003          & 84.0667                 & $-$1.2417                 & 4,  6                                    & \nodata                           & 0.67                          & 0.34                              & $-$0.31                             & $-$0.61                         & 5.6                       & 2.0                       & 12.7                         & 1.06E+21                                & 1                        \\
                    2465                    & 08409$+$0094$-$0022          & 84.0917                 & 0.9417                  & 4,  6                                    & \nodata                           & 1.51                          & $-$2.15                             & $-$2.46                             & $-$0.38                         & 9.8                       & 4.1                       & 15.7                         & 1.92E+21                                & 1                        \\
                    2466                    & 08427$-$0049$+$0013          & 84.2667                 & $-$0.4917                 & 4,  5,  6                                & \nodata                           & 0.67                          & 1.30                              & $-$0.03                             & $-$0.86                         & 7.6                       & 2.2                       & 13.4                         & 1.71E+21                                & 1                        \\
                    2467                    & 08454$-$0068$+$0028          & 84.5417                 & $-$0.6833                 & 4,  5,  6                                & \nodata                           & 0.67                          & 2.77                              & 1.13                              & $-$0.54                         & 8.7                       & 3.3                       & 22.7                         & 6.63E+21                                & 1                        \\
                    2468                    & 08455$-$0113$+$0062          & 84.5500                 & $-$1.1333                 & 4,  5,  6                                & \nodata                           & 0.67                          & 6.21                              & 4.53                              & $-$1.60                         & 8.0                       & 3.6                       & 12.8                         & 2.10E+21                                & 1                        \\
                    2469                    & 08456$-$0042$+$0005          & 84.5583                 & $-$0.4250                 & 4,  5,  6                                & \nodata                           & 1.50                          & 0.45                              & $-$0.42                             & $-$0.41                         & 6.7                       & 3.0                       & 12.3                         & 3.43E+21                                & 1                        \\
                    2470                    & 08457$-$0072$+$0027          & 84.5667                 & $-$0.7250                 & 4,  5,  6                                & \nodata                           & 0.67                          & 2.68                              & 1.69                              & $-$0.61                         & 8.6                       & 3.4                       & 16.2                         & 3.10E+21                                & 1                        \\
                    2471                    & 08460$-$0087$-$0049          & 84.6000                 & $-$0.8667                 & 4,  5,  6                                & \nodata                           & 1.51                          & $-$4.86                             & $-$6.08                             & $-$0.53                         & 11.3                      & 5.8                       & 17.2                         & 8.81E+21                                & 1                        \\
                    2472                    & 08462$-$0075$+$0035          & 84.6167                 & $-$0.7500                 & 4,  5,  6                                & \nodata                           & 0.67                          & 3.50                              & 2.24                              & $-$0.42                         & 8.5                       & 2.9                       & 17.8                         & 4.96E+21                                & 1                        \\
                    2473                    & 08463$+$0256$-$0075          & 84.6333                 & 2.5583                  & 4,  6                                    & \nodata                           & 1.52                          & $-$7.45                             & $-$8.28                             & $-$0.81                         & 6.1                       & 1.9                       & 11.3                         & 9.08E+20                                & 1                        \\
                    2474                    & 08467$-$0163$+$0010          & 84.6750                 & $-$1.6333                 & 4,  5,  6                                & \nodata                           & 0.67                          & 1.00                              & $-$0.22                             & $-$0.57                         & 13.2                      & 1.9                       & 18.6                         & 2.27E+21                                & 1                        \\
                    2475                    & 08468$+$0264$-$0021          & 84.6833                 & 2.6417                  & 4,  6                                    & \nodata                           & 0.67                          & $-$2.12                             & $-$2.74                             & $-$0.35                         & 7.7                       & 2.1                       & 13.7                         & 1.85E+21                                & 1                        \\
                    2476                    & 08469$+$0067$-$0138          & 84.6917                 & 0.6667                  & 4,  6                                    & \nodata                           & 1.85                          & $-$13.83                            & $-$14.87                            & $-$0.58                         & 7.0                       & 3.1                       & 15.1                         & 3.09E+21                                & 1                        \\
                    2477                    & 08469$-$0122$+$0015          & 84.6917                 & $-$1.2167                 & 2,  4,  5,  6                            & \nodata                           & 0.67                          & 1.48                              & 0.91                              & $-$0.59                         & 5.3                       & 1.7                       & 8.9                          & 5.95E+21                                & 2                        \\
                    2478                    & 08470$-$0074$-$0054          & 84.7000                 & $-$0.7417                 & 4,  5,  6                                & \nodata                           & 1.51                          & $-$5.42                             & $-$6.40                             & $-$0.49                         & 11.7                      & 5.0                       & 18.5                         & 6.43E+21                                & 1                        \\
                    2479                    & 08470$-$0139$+$0014          & 84.7000                 & $-$1.3917                 & 2,  4,  5,  6                            & \nodata                           & 0.67                          & 1.45                              & 1.20                              & $-$0.17                         & 17.7                      & 6.7                       & 22.4                         & 7.12E+21                                & 1                        \\
                    2480                    & 08472$-$0090$+$0037          & 84.7167                 & $-$0.9000                 & 4,  5,  6                                & \nodata                           & 0.67                          & 3.74                              & 2.12                              & $-$0.86                         & 9.3                       & 3.8                       & 13.7                         & 4.03E+21                                & 1                        \\
                    2481                    & 08472$-$0082$-$0052          & 84.7250                 & $-$0.8167                 & 4,  5,  6                                & \nodata                           & 1.51                          & $-$5.19                             & $-$6.41                             & $-$0.58                         & 14.5                      & 6.3                       & 20.6                         & 9.30E+21                                & 1                        \\
                    2482                    & 08473$-$0118$+$0010          & 84.7333                 & $-$1.1833                 & 2,  4,  5,  6                            & \nodata                           & 0.67                          & 0.99                              & $-$0.63                             & $-$0.58                         & 9.8                       & 6.6                       & 17.6                         & 1.27E+22                                & 1                        \\
                    2483                    & 08478$-$0119$+$0008          & 84.7833                 & $-$1.1917                 & 2,  4,  5,  6,  7                        & \nodata                           & 0.67                          & 0.84                              & 0.15                              & $-$0.47                         & 5.8                       & 2.6                       & 11.4                         & 1.51E+22                                & 2                        \\
                    2484                    & 08485$+$0027$+$0012          & 84.8500                 & 0.2750                  & 4,  6                                    & \nodata                           & 1.49                          & 1.15                              & 0.59                              & $-$0.49                         & 6.4                       & 1.4                       & 12.7                         & 5.63E+21                                & 2                        \\
                    2485                    & 08489$+$0257$-$0014          & 84.8917                 & 2.5667                  & 4,  6                                    & \nodata                           & 0.67                          & $-$1.39                             & $-$2.44                             & $-$0.46                         & 8.0                       & 3.5                       & 16.3                         & 4.51E+21                                & 1                        \\
                    2486                    & 08490$+$0253$-$0020          & 84.9000                 & 2.5333                  & 1,  4,  6                                & \nodata                           & 0.67                          & $-$1.97                             & $-$2.93                             & $-$0.44                         & 10.5                      & 5.9                       & 19.7                         & 8.76E+21                                & 1                        \\
                    2487                    & 08502$-$0094$-$0149          & 85.0167                 & $-$0.9417                 & 2,  4,  5,  6                            & \nodata                           & 1.50                          & $-$14.87                            & $-$15.30                            & $-$0.38                         & 5.4                       & 2.4                       & 9.2                          & 1.38E+21                                & 1                        \\
                    2488                    & 08505$-$0012$-$0038          & 85.0500                 & $-$0.1167                 & 4,  6                                    & \nodata                           & 1.50                          & $-$3.84                             & $-$5.11                             & $-$0.47                         & 16.7                      & 7.6                       & 20.7                         & 1.54E+22                                & 1                        \\
                    2489                    & 08505$-$0125$-$0375          & 85.0500                 & $-$1.2500                 & 2,  4,  5,  6                            & 17                                & 5.12                          & $-$37.51                            & $-$37.99                            & $-$0.45                         & 4.1                       & 1.2                       & 7.8                          & 4.64E+21                                & 2                        \\
                    2490                    & 08506$+$0045$-$0020          & 85.0583                 & 0.4500                  & 4,  5,  6                                & \nodata                           & 1.50                          & $-$2.00                             & $-$3.05                             & $-$0.58                         & 6.9                       & 4.9                       & 11.5                         & 5.71E+21                                & 1                        \\
                    2491                    & 08507$+$0033$-$0013          & 85.0667                 & 0.3333                  & 4,  5,  6                                & \nodata                           & 1.50                          & $-$1.34                             & $-$2.85                             & $-$0.68                         & 6.1                       & 4.8                       & 14.4                         & 6.45E+21                                & 1                        \\
                    2492                    & 08509$+$0050$-$0020          & 85.0917                 & 0.5000                  & 4,  5,  6,  7                            & \nodata                           & 1.50                          & $-$1.97                             & $-$2.87                             & $-$0.46                         & 6.8                       & 3.9                       & 13.8                         & 4.38E+21                                & 1                        \\
                    2493                    & 08510$+$0007$-$0008          & 85.1000                 & 0.0750                  & 4,  5,  6                                & \nodata                           & 0.67                          & $-$0.79                             & $-$1.60                             & $-$0.66                         & 8.9                       & 2.5                       & 16.8                         & 1.72E+21                                & 1                        \\
                    2494                    & 08510$+$0042$-$0025          & 85.1000                 & 0.4167                  & 4,  5,  6                                & \nodata                           & 1.50                          & $-$2.48                             & $-$3.34                             & $-$0.42                         & 7.7                       & 3.9                       & 14.6                         & 4.46E+21                                & 1                        \\
                    2495                    & 08510$-$0337$+$0019          & 85.1000                 & $-$3.3667                 & 4,  5,  6                                & \nodata                           & 0.67                          & 1.86                              & 1.22                              & $-$0.38                         & 5.0                       & 1.1                       & 9.5                          & 7.76E+20                                & 1                        \\
                    2496                    & 08512$+$0050$-$0017          & 85.1167                 & 0.5000                  & 4,  5,  6                                & \nodata                           & 1.50                          & $-$1.67                             & $-$2.48                             & $-$1.08                         & 4.3                       & 1.3                       & 8.5                          & 3.57E+21                                & 2                        \\
                    2497                    & 08512$-$0100$+$0039          & 85.1250                 & $-$1.0000                 & 2,  4,  5,  6                            & \nodata                           & 0.67                          & 3.90                              & 2.54                              & $-$0.88                         & 8.7                       & 3.3                       & 13.7                         & 2.79E+21                                & 1                        \\
                    2498                    & 08515$+$0057$-$0025          & 85.1500                 & 0.5667                  & 4,  5,  6                                & \nodata                           & 1.50                          & $-$2.47                             & $-$3.05                             & $-$0.43                         & 8.4                       & 3.2                       & 12.5                         & 2.25E+21                                & 1                        \\
                    2499                    & 08515$-$0037$-$0435          & 85.1500                 & $-$0.3750                 & 4,  5,  6                                & \nodata                           & 5.05                          & $-$43.55                            & $-$44.29                            & $-$0.61                         & 4.1                       & 1.4                       & 8.4                          & 7.97E+20                                & 1                        \\
                    2500                    & 08518$-$0112$+$0042          & 85.1833                 & $-$1.1167                 & 2,  4,  5,  6                            & \nodata                           & 0.67                          & 4.20                              & 3.71                              & $-$0.31                         & 9.7                       & 2.9                       & 12.7                         & 2.35E+21                                & 1                        \\
                    2501                    & 08522$-$0414$+$0060          & 85.2167                 & $-$4.1417                 & 4,  5,  6                                & \nodata                           & 0.67                          & 5.97                              & 4.64                              & $-$0.58                         & 5.3                       & 1.3                       & 17.9                         & 1.62E+21                                & 1                        \\
                    2502                    & 08522$-$0079$-$0436          & 85.2250                 & $-$0.7917                 & 2,  4,  5,  6                            & \nodata                           & 5.03                          & $-$43.63                            & $-$44.25                            & $-$0.63                         & 5.5                       & 1.3                       & 9.9                          & 5.74E+20                                & 1                        \\
                    2503                    & 08525$-$0414$+$0059          & 85.2500                 & $-$4.1417                 & 4,  5,  6                                & \nodata                           & 0.67                          & 5.91                              & 4.98                              & $-$0.51                         & 4.8                       & 1.4                       & 15.1                         & 1.26E+21                                & 1                        \\
                    2504                    & 08529$-$0411$+$0058          & 85.2917                 & $-$4.1083                 & 4,  5,  6                                & \nodata                           & 0.67                          & 5.84                              & 4.84                              & $-$0.58                         & 4.7                       & 0.9                       & 9.9                          & 6.89E+20                                & 1                        \\
                    2505                    & 08534$+$0302$-$0088          & 85.3417                 & 3.0167                  & 1,  4,  5,  6                            & \nodata                           & 1.50                          & $-$8.80                             & $-$9.28                             & $-$0.51                         & 5.8                       & 1.9                       & 10.8                         & 8.71E+20                                & 1                        \\
                    2506                    & 08534$-$0215$+$0021          & 85.3417                 & $-$2.1500                 & 2,  4,  5,  6                            & \nodata                           & 0.67                          & 2.09                              & 1.77                              & $-$0.39                         & 5.7                       & 1.6                       & 11.1                         & 6.29E+20                                & 1                        \\
                    2507                    & 08537$-$0173$+$0026          & 85.3667                 & $-$1.7333                 & 2,  4,  6                                & \nodata                           & 0.67                          & 2.57                              & 2.09                              & $-$0.41                         & 4.3                       & 2.0                       & 8.5                          & 1.19E+21                                & 1                        \\
                    2508                    & 08542$-$0317$+$0113          & 85.4250                 & $-$3.1667                 & 4,  5,  6                                & \nodata                           & 0.67                          & 11.29                             & 10.71                             & $-$0.59                         & 5.3                       & 1.4                       & 10.6                         & 6.27E+20                                & 1                        \\
                    2509                    & 08547$-$0112$+$0043          & 85.4750                 & $-$1.1250                 & 2,  4,  5,  6                            & \nodata                           & 0.67                          & 4.29                              & 3.63                              & $-$0.57                         & 6.7                       & 3.5                       & 12.1                         & 2.24E+21                                & 1                        \\
                    2510                    & 08547$-$0317$+$0114          & 85.4750                 & $-$3.1667                 & 4,  5,  6                                & \nodata                           & 0.67                          & 11.40                             & 10.99                             & $-$0.30                         & 5.9                       & 1.7                       & 12.9                         & 1.14E+21                                & 1                        \\
                    2511                    & 08551$-$0425$+$0064          & 85.5083                 & $-$4.2500                 & 4,  5,  6                                & \nodata                           & 0.67                          & 6.35                              & 5.20                              & $-$0.74                         & 5.5                       & 1.1                       & 10.5                         & 7.22E+20                                & 1                        \\
                    2512                    & 08572$-$0474$+$0084          & 85.7167                 & $-$4.7417                 & 4,  5,  6                                & \nodata                           & 0.67                          & 8.42                              & 7.62                              & $-$0.52                         & 6.2                       & 1.9                       & 11.2                         & 1.35E+21                                & 1                        \\
                    2513                    & 08599$-$0230$+$0023          & 85.9917                 & $-$2.3000                 & 2,  4,  5,  6                            & \nodata                           & 0.67                          & 2.27                              & 1.97                              & $-$0.41                         & 5.4                       & 1.6                       & 10.5                         & 5.42E+20                                & 1                        \\
                    2514                    & 08603$+$0227$-$0028          & 86.0333                 & 2.2750                  & 4,  6                                    & \nodata                           & 0.67                          & $-$2.76                             & $-$3.30                             & $-$0.60                         & 8.1                       & 2.1                       & 12.9                         & 9.17E+20                                & 1                        \\
                    2515                    & 08649$+$0022$+$0005          & 86.4917                 & 0.2250                  & 4,  5,  6                                & \nodata                           & 1.45                          & 0.46                              & $-$0.37                             & $-$0.43                         & 7.7                       & 3.9                       & 16.2                         & 4.41E+21                                & 1                        \\
                    2516                    & 08715$+$0499$+$0059          & 87.1500                 & 4.9917                  & 5,  6                                    & \nodata                           & 0.67                          & 5.91                              & 5.61                              & $-$0.34                         & 6.2                       & 1.7                       & 12.7                         & 7.31E+20                                & 1                        \\
                    2517                    & 08724$+$0404$-$0061          & 87.2417                 & 4.0417                  & 1,  4,  5,  6                            & \nodata                           & 0.67                          & $-$6.09                             & $-$6.98                             & $-$0.39                         & 7.9                       & 4.1                       & 18.1                         & 5.77E+21                                & 1                        \\
                    2518                    & 08841$+$0142$+$0021          & 88.4083                 & 1.4167                  & 5,  6                                    & \nodata                           & 0.67                          & 2.08                              & 1.51                              & $-$0.44                         & 5.0                       & 2.4                       & 9.9                          & 1.59E+21                                & 1                        \\
                    2519                    & 08847$+$0140$+$0019          & 88.4667                 & 1.4000                  & 5,  6                                    & \nodata                           & 0.67                          & 1.92                              & 1.42                              & $-$0.55                         & 4.8                       & 1.7                       & 9.3                          & 7.24E+20                                & 1                        \\
                    2520                    & 08857$+$0084$+$0017          & 88.5750                 & 0.8417                  & 6                                        & \nodata                           & 0.67                          & 1.71                              & 0.25                              & $-$0.42                         & 5.3                       & 1.4                       & 12.6                         & 2.30E+21                                & 1                        \\
                    2521                    & 08858$+$0152$+$0013          & 88.5833                 & 1.5250                  & 5,  6                                    & \nodata                           & 0.67                          & 1.25                              & 0.47                              & $-$0.58                         & 5.2                       & 1.4                       & 10.3                         & 8.23E+20                                & 1                        \\
                    2522                    & 08867$+$0087$+$0011          & 88.6750                 & 0.8667                  & 6                                        & \nodata                           & 0.67                          & 1.09                              & $-$0.94                             & $-$0.74                         & 6.9                       & 1.3                       & 18.2                         & 2.03E+21                                & 1                        \\
                    2523                    & 08870$+$0083$+$0002          & 88.7000                 & 0.8333                  & 6                                        & \nodata                           & 0.67                          & 0.22                              & $-$0.76                             & $-$0.50                         & 7.7                       & 2.3                       & 13.6                         & 2.31E+21                                & 1                        \\
                    2524                    & 08875$+$0139$+$0008          & 88.7500                 & 1.3917                  & 5,  6                                    & \nodata                           & 0.67                          & 0.80                              & 0.34                              & $-$0.42                         & 4.3                       & 1.2                       & 7.6                          & 6.34E+20                                & 1                        \\
                    2525                    & 08877$+$0137$+$0011          & 88.7750                 & 1.3750                  & 5,  6                                    & \nodata                           & 0.67                          & 1.09                              & 0.33                              & $-$0.44                         & 4.4                       & 1.8                       & 9.4                          & 1.48E+21                                & 1                        \\
                    2526                    & 08878$+$0133$+$0010          & 88.7833                 & 1.3333                  & 5,  6                                    & \nodata                           & 0.67                          & 0.97                              & 0.37                              & $-$0.53                         & 4.2                       & 1.5                       & 10.1                         & 7.72E+20                                & 1                        \\
                    2527                    & 08882$+$0146$+$0008          & 88.8167                 & 1.4583                  & 5,  6                                    & \nodata                           & 0.67                          & 0.81                              & 0.27                              & $-$0.39                         & 5.8                       & 1.4                       & 11.1                         & 8.78E+20                                & 1                        \\
                    2528                    & 08882$-$0162$-$0189          & 88.8167                 & $-$1.6167                 & 6                                        & \nodata                           & 1.63                          & $-$18.86                            & $-$19.38                            & $-$0.48                         & 5.7                       & 1.7                       & 9.7                          & 8.52E+20                                & 1                        \\
                    2529                    & 08884$+$0130$+$0015          & 88.8417                 & 1.3000                  & 5,  6                                    & \nodata                           & 0.67                          & 1.47                              & 0.65                              & $-$0.60                         & 3.4                       & 1.7                       & 8.6                          & 1.13E+21                                & 1                        \\
                    2530                    & 08884$+$0147$+$0009          & 88.8417                 & 1.4667                  & 5,  6                                    & \nodata                           & 0.67                          & 0.92                              & 0.23                              & $-$0.63                         & 3.9                       & 1.4                       & 8.9                          & 7.04E+20                                & 1                        \\
                    2531                    & 08887$+$0142$+$0010          & 88.8667                 & 1.4250                  & 5,  6                                    & \nodata                           & 0.67                          & 0.99                              & 0.55                              & $-$0.39                         & 5.9                       & 2.1                       & 9.5                          & 1.18E+21                                & 1                        \\
                    2532                    & 08900$-$0035$+$0011          & 89.0000                 & $-$0.3500                 & 5,  6                                    & \nodata                           & 0.67                          & 1.10                              & 0.64                              & $-$0.35                         & 5.8                       & 1.3                       & 11.3                         & 7.93E+20                                & 1                        \\
                    2533                    & 08908$+$0357$-$0076          & 89.0833                 & 3.5750                  & 6                                        & \nodata                           & 1.48                          & $-$7.60                             & $-$8.17                             & $-$0.41                         & 5.6                       & 2.1                       & 9.2                          & 1.48E+21                                & 1                        \\
                    2534                    & 08925$-$0268$+$0009          & 89.2500                 & $-$2.6833                 & 5,  6                                    & \nodata                           & 0.67                          & 0.91                              & 0.64                              & $-$0.27                         & 8.1                       & 1.6                       & 16.9                         & 8.59E+20                                & 1                        \\
                    2535                    & 08927$+$0392$-$0047          & 89.2667                 & 3.9167                  & 1,  6                                    & \nodata                           & 0.67                          & $-$4.67                             & $-$5.61                             & $-$0.38                         & 7.1                       & 2.8                       & 15.4                         & 3.73E+21                                & 1                        \\
                    2536                    & 08932$-$0263$+$0010          & 89.3250                 & $-$2.6333                 & 5,  6                                    & \nodata                           & 0.67                          & 0.99                              & 0.72                              & $-$0.38                         & 6.2                       & 1.2                       & 12.0                         & 3.98E+20                                & 1                        \\
                    2537                    & 08935$-$0267$+$0008          & 89.3500                 & $-$2.6750                 & 5,  6                                    & \nodata                           & 0.67                          & 0.81                              & 0.59                              & $-$0.27                         & 5.6                       & 2.3                       & 11.7                         & 9.33E+20                                & 1                        \\
                    2538                    & 08994$-$0195$+$0017          & 89.9417                 & $-$1.9500                 & 6                                        & \nodata                           & 0.67                          & 1.67                              & 1.24                              & $-$0.51                         & 5.8                       & 1.5                       & 10.9                         & 4.31E+21                                & 2                        \\
                    2539                    & 09002$-$0493$+$0104          & 90.0250                 & $-$4.9333                 & 6                                        & \nodata                           & 0.67                          & 10.43                             & 10.09                             & $-$0.37                         & 6.8                       & 1.5                       & 11.0                         & 6.29E+20                                & 1                        \\
                    2540                    & 09028$-$0077$-$0355          & 90.2833                 & $-$0.7667                 & 3,  6                                    & \nodata                           & 4.56                          & $-$35.54                            & $-$36.29                            & $-$0.70                         & 3.8                       & 1.2                       & 13.4                         & 5.84E+20                                & 1                        \\
                    2541                    & 09050$-$0146$+$0034          & 90.5000                 & $-$1.4583                 & 5,  6                                    & \nodata                           & 0.67                          & 3.40                              & 2.63                              & $-$0.88                         & 4.6                       & 2.5                       & 11.2                         & 1.09E+21                                & 1                        \\
                    2542                    & 09073$-$0457$+$0104          & 90.7333                 & $-$4.5750                 & 6                                        & \nodata                           & 0.67                          & 10.43                             & 9.75                              & $-$0.39                         & 8.9                       & 2.5                       & 14.4                         & 2.27E+21                                & 1                        \\
                    2543                    & 09084$-$0244$+$0016          & 90.8417                 & $-$2.4417                 & 6                                        & \nodata                           & 0.67                          & 1.60                              & 1.33                              & $-$0.28                         & 6.2                       & 1.4                       & 10.6                         & 6.34E+20                                & 1                        \\
                    2544                    & 09097$+$0350$-$0019          & 90.9750                 & 3.5000                  & 5,  6                                    & \nodata                           & 0.67                          & $-$1.88                             & $-$3.22                             & $-$1.12                         & 4.4                       & 1.9                       & 10.2                         & 1.07E+21                                & 1                        \\
                    2545                    & 09104$+$0452$-$0060          & 91.0417                 & 4.5250                  & 5,  6                                    & \nodata                           & 1.32                          & $-$5.97                             & $-$6.58                             & $-$0.47                         & 4.6                       & 1.5                       & 9.4                          & 8.72E+20                                & 1                        \\
                    2546                    & 09104$-$0212$+$0049          & 91.0417                 & $-$2.1250                 & 5,  6                                    & \nodata                           & 0.67                          & 4.87                              & 4.40                              & $-$0.45                         & 4.6                       & 1.5                       & 8.7                          & 7.12E+20                                & 1                        \\
                    2547                    & 09134$+$0480$-$0041          & 91.3417                 & 4.8000                  & 5,  6                                    & \nodata                           & 1.31                          & $-$4.05                             & $-$4.95                             & $-$0.73                         & 6.1                       & 1.9                       & 10.7                         & 1.09E+21                                & 1                        \\
                    2548                    & 09136$+$0482$-$0040          & 91.3583                 & 4.8250                  & 5,  6                                    & \nodata                           & 1.31                          & $-$3.98                             & $-$5.07                             & $-$0.74                         & 6.3                       & 2.4                       & 10.3                         & 1.73E+21                                & 1                        \\
                    2549                    & 09170$+$0439$-$0041          & 91.7000                 & 4.3917                  & 5,  6                                    & \nodata                           & 1.31                          & $-$4.07                             & $-$5.11                             & $-$0.52                         & 6.4                       & 3.5                       & 15.0                         & 4.01E+21                                & 1                        \\
                    2550                    & 09187$-$0027$-$0529          & 91.8750                 & $-$0.2750                 & 1,  6                                    & \nodata                           & 4.28                          & $-$52.89                            & $-$53.42                            & $-$0.42                         & 9.1                       & 4.2                       & 17.9                         & 3.20E+21                                & 1                        \\
                    2551                    & 09189$-$0025$-$0533          & 91.8917                 & $-$0.2500                 & 6                                        & \nodata                           & 4.27                          & $-$53.31                            & $-$53.80                            & $-$0.45                         & 5.2                       & 1.5                       & 9.6                          & 7.29E+20                                & 1                        \\
                    2552                    & 09200$+$0494$-$0041          & 92.0000                 & 4.9417                  & 5,  6                                    & \nodata                           & 1.27                          & $-$4.09                             & $-$4.66                             & $-$0.43                         & 5.6                       & 1.8                       & 9.2                          & 1.14E+21                                & 1                        \\
                    2553                    & 09215$-$0270$+$0015          & 92.1500                 & $-$2.7000                 & 6                                        & \nodata                           & 0.68                          & 1.48                              & 0.69                              & $-$0.78                         & 7.3                       & 2.8                       & 12.6                         & 1.44E+21                                & 1                        \\
                    2554                    & 09215$-$0273$+$0016          & 92.1500                 & $-$2.7333                 & 6                                        & \nodata                           & 0.68                          & 1.58                              & 1.11                              & $-$0.40                         & 7.1                       & 2.7                       & 10.8                         & 1.60E+21                                & 1                        \\
                    2555                    & 09229$-$0024$+$0045          & 92.2917                 & $-$0.2417                 & 6                                        & \nodata                           & 0.68                          & 4.49                              & 3.75                              & $-$0.64                         & 4.2                       & 1.0                       & 13.2                         & 5.30E+20                                & 1                        \\
                    2556                    & 09233$-$0049$+$0052          & 92.3333                 & $-$0.4917                 & 6                                        & \nodata                           & 0.68                          & 5.22                              & 4.70                              & $-$0.55                         & 4.6                       & 1.3                       & 9.7                          & 5.71E+20                                & 1                        \\
                    2557                    & 09239$-$0062$+$0057          & 92.3917                 & $-$0.6250                 & 6                                        & \nodata                           & 0.68                          & 5.65                              & 5.15                              & $-$0.42                         & 5.4                       & 1.6                       & 11.8                         & 8.70E+20                                & 1                        \\
                    2558                    & 09240$-$0023$+$0044          & 92.4000                 & $-$0.2333                 & 6                                        & \nodata                           & 0.68                          & 4.35                              & 3.80                              & $-$0.55                         & 5.9                       & 1.9                       & 12.4                         & 9.42E+20                                & 1                        \\
                    2559                    & 09241$-$0265$+$0024          & 92.4083                 & $-$2.6500                 & 6                                        & \nodata                           & 0.68                          & 2.36                              & 1.92                              & $-$0.39                         & 6.6                       & 2.1                       & 13.1                         & 1.21E+21                                & 1                        \\
                    2560                    & 09266$-$0080$+$0039          & 92.6583                 & $-$0.8000                 & 6                                        & \nodata                           & 0.68                          & 3.94                              & 3.56                              & $-$0.42                         & 4.8                       & 2.5                       & 9.7                          & 1.15E+21                                & 1                        \\
                    2561                    & 09269$-$0011$+$0039          & 92.6917                 & $-$0.1083                 & 1,  6                                    & \nodata                           & 0.68                          & 3.86                              & 3.04                              & $-$0.60                         & 4.6                       & 3.4                       & 10.4                         & 2.62E+21                                & 1                        \\
                    2562                    & 09272$-$0087$+$0033          & 92.7167                 & $-$0.8667                 & 6                                        & \nodata                           & 0.68                          & 3.29                              & 2.98                              & $-$0.26                         & 4.3                       & 2.2                       & 8.9                          & 1.34E+21                                & 1                        \\
                    2563                    & 09277$+$0003$+$0028          & 92.7750                 & 0.0333                  & 6                                        & \nodata                           & 1.21                          & 2.78                              & 1.98                              & $-$0.95                         & 4.1                       & 1.4                       & 12.4                         & 5.55E+20                                & 1                        \\
                    2564                    & 09321$-$0463$+$0037          & 93.2083                 & $-$4.6333                 & 6                                        & \nodata                           & $-$1.00                         & 3.65                              & 3.16                              & $-$0.46                         & 5.4                       & 2.5                       & 10.4                         & 1.38E+21                                & 1                        \\
                    2565                    & 09334$-$0027$-$0454          & 93.3417                 & $-$0.2750                 & 1,  5,  6                                & \nodata                           & 4.25                          & $-$45.38                            & $-$45.90                            & $-$0.39                         & 6.7                       & 2.9                       & 12.3                         & 2.01E+21                                & 1                        \\
                    2566                    & 09341$-$0036$-$0470          & 93.4083                 & $-$0.3583                 & 1,  5,  6                                & \nodata                           & 4.25                          & $-$47.02                            & $-$47.95                            & $-$0.46                         & 7.0                       & 3.9                       & 13.9                         & 4.48E+21                                & 1                        \\
                    2567                    & 09347$+$0010$+$0024          & 93.4667                 & 0.1000                  & 5,  6                                    & \nodata                           & 0.68                          & 2.41                              & 2.04                              & $-$0.27                         & 5.0                       & 1.7                       & 10.9                         & 1.06E+21                                & 1                        \\
                    2568                    & 09349$-$0427$+$0036          & 93.4917                 & $-$4.2667                 & 5,  6                                    & \nodata                           & $-$1.00                         & 3.60                              & 2.82                              & $-$0.40                         & 8.7                       & 3.4                       & 16.1                         & 3.74E+21                                & 1                        \\
                    2569                    & 09353$-$0428$+$0041          & 93.5333                 & $-$4.2833                 & 5,  6                                    & 18                                & $-$1.00                         & 4.07                              & 3.36                              & $-$0.40                         & 8.8                       & 5.0                       & 17.4                         & 5.58E+21                                & 1                        \\
                    2570                    & 09370$-$0458$+$0037          & 93.7000                 & $-$4.5833                 & 6                                        & \nodata                           & $-$1.00                         & 3.69                              & 2.99                              & $-$0.46                         & 7.9                       & 3.8                       & 16.2                         & 3.36E+21                                & 1                        \\
                    2571                    & 09381$+$0057$+$0005          & 93.8083                 & 0.5667                  & 6                                        & \nodata                           & 0.68                          & 0.54                              & $-$0.11                             & $-$0.66                         & 3.9                       & 1.4                       & 9.2                          & 6.12E+20                                & 1                        \\
                    2572                    & 09382$-$0101$+$0044          & 93.8250                 & $-$1.0083                 & 5,  6                                    & \nodata                           & 0.68                          & 4.44                              & 4.25                              & $-$0.26                         & 4.1                       & 1.4                       & 9.0                          & 4.87E+20                                & 1                        \\
                    2573                    & 09383$+$0057$+$0006          & 93.8333                 & 0.5750                  & 6                                        & \nodata                           & 0.68                          & 0.64                              & $-$0.15                             & $-$0.89                         & 3.8                       & 1.5                       & 8.8                          & 6.04E+20                                & 1                        \\
                    2574                    & 09392$-$0471$+$0022          & 93.9167                 & $-$4.7083                 & 5,  6                                    & \nodata                           & $-$1.00                         & 2.22                              & 1.81                              & $-$0.31                         & 7.0                       & 3.3                       & 11.9                         & 2.35E+21                                & 1                        \\
                    2575                    & 09407$-$0162$-$0382          & 94.0750                 & $-$1.6167                 & 4,  5,  6                                & \nodata                           & 4.39                          & $-$38.24                            & $-$39.00                            & $-$0.70                         & 5.7                       & 3.2                       & 10.1                         & 1.91E+21                                & 1                        \\
                    2576                    & 09487$-$0085$-$0444          & 94.8667                 & $-$0.8500                 & 5,  6                                    & \nodata                           & 4.40                          & $-$44.35                            & $-$44.89                            & $-$0.43                         & 4.4                       & 1.3                       & 8.6                          & 7.34E+20                                & 1                        \\
                    2577                    & 09572$-$0082$-$0454          & 95.7250                 & $-$0.8250                 & 3,  6                                    & \nodata                           & 3.98                          & $-$45.44                            & $-$46.19                            & $-$0.51                         & 5.5                       & 3.8                       & 11.8                         & 3.14E+21                                & 1                        \\
                    2578                    & 09592$-$0023$-$0554          & 95.9250                 & $-$0.2333                 & 1,  6                                    & \nodata                           & 3.84                          & $-$55.40                            & $-$56.30                            & $-$0.43                         & 7.7                       & 2.6                       & 12.9                         & 2.77E+21                                & 1                        \\
                    2579                    & 09595$-$0182$+$0039          & 95.9500                 & $-$1.8250                 & 5,  6                                    & \nodata                           & 0.68                          & 3.95                              & 3.29                              & $-$0.41                         & 7.0                       & 2.4                       & 15.4                         & 1.98E+21                                & 1                        \\
                    2580                    & 09596$-$0173$+$0043          & 95.9583                 & $-$1.7333                 & 6                                        & \nodata                           & 0.68                          & 4.27                              & 3.21                              & $-$0.84                         & 4.9                       & 1.7                       & 12.3                         & 1.03E+21                                & 1                        \\
                    2581                    & 09604$-$0177$+$0037          & 96.0417                 & $-$1.7667                 & 6                                        & \nodata                           & 0.68                          & 3.73                              & 2.89                              & $-$0.61                         & 5.4                       & 1.6                       & 11.8                         & 1.07E+21                                & 1                        \\
                    2582                    & 09607$-$0178$+$0035          & 96.0667                 & $-$1.7833                 & 6                                        & \nodata                           & 0.68                          & 3.49                              & 2.65                              & $-$0.93                         & 6.0                       & 1.7                       & 14.9                         & 7.51E+20                                & 1                        \\
                    2583                    & 09626$+$0023$-$0356          & 96.2583                 & 0.2333                  & 6                                        & \nodata                           & 4.38                          & $-$35.65                            & $-$36.35                            & $-$0.55                         & 5.2                       & 2.3                       & 12.5                         & 1.49E+21                                & 1                        \\
                    2584                    & 09676$+$0000$-$0530          & 96.7583                 & 0.0000                  & 6                                        & \nodata                           & 3.66                          & $-$52.97                            & $-$53.60                            & $-$0.59                         & 5.0                       & 2.0                       & 9.9                          & 1.00E+21                                & 1                        \\
                    2585                    & 09735$+$0292$+$0101          & 97.3500                 & 2.9167                  & 5,  6                                    & \nodata                           & 0.68                          & 10.11                             & 9.51                              & $-$0.78                         & 6.3                       & 1.6                       & 13.0                         & 5.94E+20                                & 1                        \\
                    2586                    & 09745$+$0281$-$0019          & 97.4500                 & 2.8083                  & 5,  6                                    & \nodata                           & 0.68                          & $-$1.92                             & $-$2.16                             & $-$0.27                         & 6.5                       & 1.7                       & 13.4                         & 7.48E+20                                & 1                        \\
                    2587                    & 09749$+$0297$+$0105          & 97.4917                 & 2.9750                  & 5,  6                                    & \nodata                           & 0.67                          & 10.45                             & 9.80                              & $-$0.74                         & 5.2                       & 1.7                       & 9.9                          & 7.13E+20                                & 1                        \\
                    2588                    & 09780$+$0299$+$0072          & 97.8000                 & 2.9917                  & 5,  6                                    & \nodata                           & 0.68                          & 7.22                              & 6.98                              & $-$0.27                         & 6.4                       & 1.7                       & 13.3                         & 7.46E+20                                & 1                        \\
                    2589                    & 09798$+$0150$+$0009          & 97.9833                 & 1.5000                  & 3,  5,  6                                & \nodata                           & 0.68                          & 0.91                              & 0.23                              & $-$0.61                         & 7.7                       & 2.8                       & 13.2                         & 1.65E+21                                & 1                        \\
                    2590                    & 09822$+$0480$-$0011          & 98.2250                 & 4.8000                  & 5,  6                                    & \nodata                           & 0.68                          & $-$1.08                             & $-$1.37                             & $-$0.31                         & 9.6                       & 3.2                       & 13.3                         & 1.63E+21                                & 1                        \\
                    2591                    & 09824$+$0469$-$0012          & 98.2417                 & 4.6917                  & 5,  6                                    & \nodata                           & 0.68                          & $-$1.16                             & $-$1.84                             & $-$0.66                         & 9.4                       & 5.7                       & 16.1                         & 3.86E+21                                & 1                        \\
                    2592                    & 09826$+$0472$-$0014          & 98.2583                 & 4.7167                  & 5,  6                                    & \nodata                           & 0.68                          & $-$1.37                             & $-$2.08                             & $-$0.42                         & 10.7                      & 5.0                       & 16.7                         & 5.12E+21                                & 1                        \\
                    2593                    & 09830$+$0376$+$0036          & 98.3000                 & 3.7583                  & 3,  4,  6                                & \nodata                           & 0.68                          & 3.55                              & 3.22                              & $-$0.40                         & 4.7                       & 1.6                       & 9.7                          & 6.13E+20                                & 1                        \\
                    2594                    & 09832$+$0519$+$0001          & 98.3167                 & 5.1917                  & 6                                        & \nodata                           & 0.68                          & 0.09                              & $-$0.60                             & $-$0.47                         & 5.6                       & 1.2                       & 9.8                          & 7.55E+20                                & 1                        \\
                    2595                    & 09833$+$0339$+$0079          & 98.3333                 & 3.3917                  & 4,  5,  6                                & \nodata                           & 0.68                          & 7.86                              & 7.32                              & $-$0.32                         & 6.3                       & 1.6                       & 15.7                         & 1.41E+21                                & 1                        \\
                    2596                    & 09834$+$0517$+$0000          & 98.3417                 & 5.1750                  & 6                                        & \nodata                           & 0.68                          & $-$0.03                             & $-$0.61                             & $-$0.44                         & 7.7                       & 1.9                       & 14.3                         & 1.31E+21                                & 1                        \\
                    2597                    & 09835$+$0343$+$0080          & 98.3500                 & 3.4333                  & 4,  5,  6                                & \nodata                           & 0.68                          & 8.00                              & 7.19                              & $-$0.43                         & 6.0                       & 1.3                       & 14.4                         & 1.20E+21                                & 1                        \\
                    2598                    & 09835$+$0352$+$0079          & 98.3500                 & 3.5250                  & 4,  5,  6                                & \nodata                           & 0.68                          & 7.89                              & 7.16                              & $-$0.89                         & 7.1                       & 1.4                       & 13.5                         & 5.64E+20                                & 1                        \\
                    2599                    & 09837$+$0348$+$0081          & 98.3667                 & 3.4833                  & 4,  5,  6                                & \nodata                           & 0.68                          & 8.09                              & 7.61                              & $-$0.48                         & 7.4                       & 2.5                       & 13.0                         & 1.26E+21                                & 1                        \\
                    2600                    & 09839$+$0517$-$0004          & 98.3917                 & 5.1667                  & 6                                        & \nodata                           & 0.68                          & $-$0.37                             & $-$0.88                             & $-$0.45                         & 10.9                      & 4.5                       & 19.4                         & 3.26E+21                                & 1                        \\
                    2601                    & 09889$+$0039$-$0001          & 98.8917                 & 0.3917                  & 5,  6                                    & \nodata                           & 0.68                          & $-$0.12                             & $-$0.55                             & $-$0.46                         & 5.8                       & 1.1                       & 11.8                         & 4.71E+20                                & 1                        \\
                    2602                    & 09935$+$0163$-$0012          & 99.3500                 & 1.6333                  & 5,  6                                    & \nodata                           & 0.68                          & $-$1.17                             & $-$1.95                             & $-$0.56                         & 5.0                       & 1.4                       & 11.2                         & 9.26E+20                                & 1                        \\
                    2603                    & 09971$+$0155$-$0015          & 99.7083                 & 1.5500                  & 5,  6                                    & \nodata                           & 0.68                          & $-$1.50                             & $-$1.94                             & $-$0.44                         & 6.6                       & 1.4                       & 11.6                         & 6.46E+20                                & 1                        \\
                    2604                    & 09978$+$0174$-$0012          & 99.7833                 & 1.7417                  & 6                                        & \nodata                           & 0.68                          & $-$1.16                             & $-$1.77                             & $-$0.45                         & 6.5                       & 1.4                       & 19.0                         & 1.04E+21                                & 1                        \\
                    2605                    & 09982$+$0169$-$0014          & 99.8167                 & 1.6917                  & 6                                        & \nodata                           & 0.68                          & $-$1.39                             & $-$1.94                             & $-$0.51                         & 4.0                       & 1.1                       & 9.7                          & 5.26E+20                                & 1                        \\
                    2606                    & 09986$+$0173$-$0011          & 99.8583                 & 1.7333                  & 6                                        & \nodata                           & 0.68                          & $-$1.08                             & $-$1.66                             & $-$0.60                         & 4.7                       & 1.5                       & 9.6                          & 6.86E+20                                & 1                        \\
                    2607                    & 10027$+$0328$-$0019          & 100.2750                & 3.2833                  & 4,  5,  6                                & \nodata                           & 0.69                          & $-$1.93                             & $-$3.21                             & $-$0.67                         & 16.5                      & 3.9                       & 41.3                         & 7.40E+21                                & 1                        \\
                    2608                    & 10027$+$0333$-$0020          & 100.2750                & 3.3333                  & 4,  5,  6                                & \nodata                           & 0.69                          & $-$1.96                             & $-$2.28                             & $-$0.23                         & 17.6                      & 4.6                       & 23.1                         & 4.34E+21                                & 1                        \\
                    2609                    & 10030$+$0327$-$0022          & 100.3000                & 3.2750                  & 4,  5,  6                                & \nodata                           & 0.69                          & $-$2.22                             & $-$2.84                             & $-$0.43                         & 16.8                      & 5.7                       & 31.7                         & 7.09E+21                                & 1                        \\
                    2610                    & 10047$+$0347$-$0027          & 100.4667                & 3.4667                  & 3,  4,  6                                & \nodata                           & 0.72                          & $-$2.71                             & $-$3.35                             & $-$0.31                         & 14.2                      & 4.8                       & 26.7                         & 7.24E+21                                & 1                        \\
                    2611                    & 10061$+$0165$-$0011          & 100.6083                & 1.6500                  & 6                                        & \nodata                           & 0.69                          & $-$1.10                             & $-$1.62                             & $-$0.56                         & 5.2                       & 1.3                       & 12.5                         & 5.39E+20                                & 1                        \\
                    2612                    & 10070$+$0520$-$0013          & 100.7000                & 5.2000                  & 5,  6                                    & \nodata                           & 0.73                          & $-$1.34                             & $-$1.89                             & $-$0.56                         & 6.6                       & 3.4                       & 15.7                         & 1.93E+21                                & 1                        \\
                    2613                    & 10103$+$0459$+$0000          & 101.0333                & 4.5917                  & 6                                        & \nodata                           & 0.71                          & 0.05                              & $-$0.49                             & $-$0.51                         & 6.3                       & 1.3                       & 12.8                         & 6.42E+20                                & 1                        \\
                    2614                    & 10105$+$0344$+$0002          & 101.0500                & 3.4417                  & 6                                        & \nodata                           & 0.70                          & 0.20                              & $-$0.04                             & $-$0.29                         & 4.9                       & 1.5                       & 11.3                         & 5.96E+20                                & 1                        \\
                    2615                    & 10123$+$0307$-$0002          & 101.2333                & 3.0750                  & 6                                        & \nodata                           & 0.71                          & $-$0.25                             & $-$1.17                             & $-$0.45                         & 7.0                       & 1.7                       & 14.6                         & 1.79E+21                                & 1                        \\
                    2616                    & 10132$+$0338$-$0014          & 101.3250                & 3.3833                  & 6                                        & \nodata                           & 0.73                          & $-$1.43                             & $-$1.91                             & $-$0.52                         & 7.4                       & 1.5                       & 13.7                         & 6.63E+20                                & 1                        \\
                    2617                    & 10145$+$0308$+$0002          & 101.4500                & 3.0833                  & 6                                        & \nodata                           & 0.72                          & 0.22                              & $-$0.91                             & $-$0.41                         & 6.5                       & 1.7                       & 10.3                         & 2.10E+21                                & 1                        \\
                    2618                    & 10147$+$0292$-$0006          & 101.4750                & 2.9250                  & 6                                        & \nodata                           & 0.72                          & $-$0.59                             & $-$1.47                             & $-$0.54                         & 7.2                       & 2.1                       & 12.5                         & 1.69E+21                                & 1                        \\
                    2619                    & 10149$+$0309$-$0003          & 101.4917                & 3.0917                  & 6                                        & \nodata                           & 0.72                          & $-$0.27                             & $-$1.32                             & $-$0.48                         & 5.8                       & 1.7                       & 12.1                         & 1.77E+21                                & 1                        \\
                    2620                    & 10152$+$0283$-$0014          & 101.5250                & 2.8333                  & 6                                        & \nodata                           & 0.74                          & $-$1.37                             & $-$1.89                             & $-$0.34                         & 6.6                       & 1.5                       & 13.8                         & 1.13E+21                                & 1                        \\
                    2621                    & 10157$+$0291$-$0014          & 101.5667                & 2.9083                  & 6                                        & \nodata                           & 0.74                          & $-$1.41                             & $-$2.18                             & $-$0.41                         & 6.9                       & 1.4                       & 14.9                         & 1.28E+21                                & 1                        \\
                    2622                    & 10157$+$0322$-$0016          & 101.5750                & 3.2250                  & 6                                        & \nodata                           & 0.74                          & $-$1.64                             & $-$2.19                             & $-$0.32                         & 6.9                       & 2.4                       & 12.6                         & 2.02E+21                                & 1                        \\
                    2623                    & 10160$+$0302$-$0014          & 101.6000                & 3.0167                  & 6                                        & \nodata                           & 0.74                          & $-$1.40                             & $-$2.39                             & $-$0.43                         & 7.8                       & 1.9                       & 18.4                         & 2.44E+21                                & 1                        \\
                    2624                    & 10163$+$0303$-$0015          & 101.6333                & 3.0333                  & 6                                        & \nodata                           & 0.74                          & $-$1.46                             & $-$2.58                             & $-$0.46                         & 7.0                       & 1.5                       & 14.5                         & 1.79E+21                                & 1                        \\
                    2625                    & 10164$+$0301$-$0013          & 101.6417                & 3.0083                  & 6                                        & \nodata                           & 0.74                          & $-$1.26                             & $-$2.23                             & $-$0.38                         & 7.5                       & 1.6                       & 16.6                         & 2.12E+21                                & 1                        \\
                    2626                    & 10165$+$0276$-$0017          & 101.6500                & 2.7583                  & 6                                        & \nodata                           & 0.74                          & $-$1.71                             & $-$2.61                             & $-$0.45                         & 7.2                       & 1.6                       & 14.9                         & 1.61E+21                                & 1                        \\
                    2627                    & 10167$+$0500$-$0023          & 101.6667                & 5.0000                  & 5,  6                                    & \nodata                           & 0.81                          & $-$2.35                             & $-$2.83                             & $-$0.44                         & 6.5                       & 3.4                       & 13.2                         & 2.05E+21                                & 1                        \\
                    2628                    & 10167$+$0502$-$0025          & 101.6667                & 5.0250                  & 5,  6                                    & \nodata                           & 0.81                          & $-$2.46                             & $-$2.96                             & $-$0.46                         & 6.1                       & 1.9                       & 11.6                         & 1.02E+21                                & 1                        \\
                    2629                    & 10167$+$0357$-$0011          & 101.6750                & 3.5667                  & 6                                        & \nodata                           & 0.74                          & $-$1.08                             & $-$1.74                             & $-$0.48                         & 6.4                       & 1.7                       & 12.4                         & 1.12E+21                                & 1                        \\
                    2630                    & 10171$+$0273$-$0013          & 101.7083                & 2.7333                  & 6                                        & \nodata                           & 0.74                          & $-$1.31                             & $-$1.85                             & $-$0.49                         & 5.7                       & 1.4                       & 12.1                         & 6.92E+20                                & 1                        \\
                    2631                    & 10180$+$0516$-$0025          & 101.8000                & 5.1583                  & 5,  6                                    & \nodata                           & 0.81                          & $-$2.53                             & $-$3.04                             & $-$0.62                         & 6.4                       & 2.5                       & 11.9                         & 1.02E+21                                & 1                        \\
                    2632                    & 10185$+$0392$-$0015          & 101.8500                & 3.9167                  & 6                                        & \nodata                           & 0.79                          & $-$1.46                             & $-$2.01                             & $-$0.53                         & 4.8                       & 1.3                       & 9.6                          & 6.11E+20                                & 1                        \\
                    2633                    & 10189$+$0378$-$0026          & 101.8917                & 3.7833                  & 6                                        & \nodata                           & 0.79                          & $-$2.65                             & $-$3.16                             & $-$0.56                         & 5.2                       & 1.9                       & 9.1                          & 8.24E+20                                & 1                        \\
                    2634                    & 10197$+$0329$-$0016          & 101.9667                & 3.2917                  & 6                                        & \nodata                           & 0.78                          & $-$1.56                             & $-$2.03                             & $-$0.35                         & 7.9                       & 2.7                       & 15.2                         & 1.90E+21                                & 1                        \\
                    2635                    & 10200$+$0322$-$0016          & 102.0000                & 3.2250                  & 6                                        & \nodata                           & 0.77                          & $-$1.63                             & $-$2.09                             & $-$0.46                         & 7.1                       & 3.0                       & 15.3                         & 1.64E+21                                & 1                        \\
                    2636                    & 10202$+$0329$-$0016          & 102.0167                & 3.2917                  & 6                                        & \nodata                           & 0.77                          & $-$1.56                             & $-$2.20                             & $-$0.50                         & 7.0                       & 3.0                       & 12.7                         & 2.02E+21                                & 1                        \\
                    2637                    & 10203$+$0324$-$0016          & 102.0333                & 3.2417                  & 6                                        & \nodata                           & 0.77                          & $-$1.64                             & $-$2.27                             & $-$0.65                         & 7.1                       & 3.2                       & 12.0                         & 1.65E+21                                & 1                        \\
                    2638                    & 10203$+$0334$-$0019          & 102.0333                & 3.3417                  & 6                                        & \nodata                           & 0.77                          & $-$1.87                             & $-$2.21                             & $-$0.31                         & 6.8                       & 2.5                       & 13.6                         & 1.42E+21                                & 1                        \\
                    2639                    & 10206$+$0330$-$0015          & 102.0583                & 3.3000                  & 6                                        & \nodata                           & 0.77                          & $-$1.54                             & $-$2.15                             & $-$0.33                         & 7.4                       & 2.3                       & 16.2                         & 2.31E+21                                & 1                        \\
                    2640                    & 10210$+$0250$-$0024          & 102.1000                & 2.5000                  & 5,  6                                    & \nodata                           & 0.76                          & $-$2.36                             & $-$2.85                             & $-$0.39                         & 6.9                       & 2.3                       & 14.0                         & 1.45E+21                                & 1                        \\
                    2641                    & 10212$+$0517$-$0025          & 102.1167                & 5.1750                  & 5,  6                                    & \nodata                           & 0.81                          & $-$2.50                             & $-$3.02                             & $-$0.52                         & 6.6                       & 2.1                       & 13.1                         & 1.05E+21                                & 1                        \\
                    2642                    & 10212$+$0252$-$0024          & 102.1250                & 2.5167                  & 5,  6                                    & \nodata                           & 0.77                          & $-$2.39                             & $-$2.83                             & $-$0.48                         & 6.8                       & 2.2                       & 13.7                         & 1.00E+21                                & 1                        \\
                    2643                    & 10213$+$0496$-$0020          & 102.1333                & 4.9583                  & 6                                        & \nodata                           & 0.80                          & $-$2.03                             & $-$2.63                             & $-$0.63                         & 5.3                       & 1.4                       & 12.2                         & 6.25E+20                                & 1                        \\
                    2644                    & 10217$+$0252$-$0022          & 102.1750                & 2.5167                  & 5,  6                                    & \nodata                           & 0.77                          & $-$2.21                             & $-$3.08                             & $-$0.84                         & 5.9                       & 2.3                       & 11.6                         & 1.19E+21                                & 1                        \\
                    2645                    & 10228$+$0332$-$0017          & 102.2833                & 3.3167                  & 6                                        & \nodata                           & 0.77                          & $-$1.67                             & $-$2.17                             & $-$0.40                         & 6.8                       & 1.9                       & 15.3                         & 1.19E+21                                & 1                        \\
                    2646                    & 10236$+$0303$-$0016          & 102.3583                & 3.0333                  & 5,  6                                    & \nodata                           & 0.77                          & $-$1.62                             & $-$2.27                             & $-$0.53                         & 7.0                       & 3.4                       & 13.5                         & 2.28E+21                                & 1                        \\
                    2647                    & 10245$+$0304$-$0017          & 102.4500                & 3.0417                  & 5,  6                                    & \nodata                           & 0.77                          & $-$1.70                             & $-$2.51                             & $-$0.44                         & 7.3                       & 3.4                       & 15.1                         & 3.39E+21                                & 1                        \\
                    2648                    & 10251$+$0523$-$0026          & 102.5083                & 5.2333                  & 6                                        & \nodata                           & 0.81                          & $-$2.63                             & $-$3.06                             & $-$0.49                         & 7.4                       & 1.7                       & 14.5                         & 7.48E+20                                & 1                        \\
                    2649                    & 10274$+$0270$-$0015          & 102.7417                & 2.7000                  & 4,  5,  6                                & \nodata                           & 0.77                          & $-$1.48                             & $-$2.06                             & $-$0.44                         & 6.1                       & 2.9                       & 10.5                         & 1.98E+21                                & 1                        \\
                    2650                    & 10279$+$0272$-$0014          & 102.7917                & 2.7250                  & 4,  5,  6                                & \nodata                           & 0.77                          & $-$1.39                             & $-$2.02                             & $-$0.47                         & 5.7                       & 2.4                       & 10.6                         & 1.63E+21                                & 1                        \\
                    2651                    & 10292$+$0237$-$0029          & 102.9250                & 2.3667                  & 4,  5,  6                                & \nodata                           & 0.77                          & $-$2.86                             & $-$3.26                             & $-$0.51                         & 4.4                       & 2.1                       & 12.4                         & 7.95E+20                                & 1                        \\
                    2652                    & 10297$+$0278$-$0016          & 102.9667                & 2.7833                  & 4,  5,  6                                & \nodata                           & 0.77                          & $-$1.60                             & $-$2.23                             & $-$0.50                         & 6.4                       & 1.5                       & 12.9                         & 9.16E+20                                & 1                        \\
                    2653                    & 10343$+$0285$-$0026          & 103.4333                & 2.8500                  & 4,  5,  6                                & \nodata                           & 0.78                          & $-$2.60                             & $-$3.39                             & $-$0.73                         & 5.6                       & 1.5                       & 9.6                          & 7.42E+20                                & 1                        \\
                    2654                    & 10344$+$0282$-$0030          & 103.4417                & 2.8250                  & 4,  5,  6                                & \nodata                           & 0.78                          & $-$3.02                             & $-$3.52                             & $-$0.33                         & 5.8                       & 2.8                       & 12.9                         & 2.17E+21                                & 1                        \\
                    2655                    & 10485$+$0285$-$0240          & 104.8500                & 2.8500                  & 4,  6                                    & \nodata                           & 0.83                          & $-$23.99                            & $-$24.71                            & $-$0.74                         & 4.2                       & 3.3                       & 9.6                          & 1.83E+21                                & 1                        \\
                    2656                    & 10490$+$0240$-$0238          & 104.9000                & 2.4000                  & 1,  3,  4,  5,  6                        & \nodata                           & 0.83                          & $-$23.80                            & $-$24.26                            & $-$0.59                         & 4.8                       & 1.5                       & 11.3                         & 5.54E+20                                & 1                        \\
                    2657                    & 10552$+$0300$-$0031          & 105.5167                & 3.0000                  & 5,  6                                    & \nodata                           & 0.78                          & $-$3.08                             & $-$3.52                             & $-$0.42                         & 5.1                       & 3.3                       & 9.2                          & 1.95E+21                                & 1                        \\
                    2658                    & 10577$+$0287$-$0042          & 105.7667                & 2.8667                  & 6                                        & \nodata                           & 0.79                          & $-$4.16                             & $-$4.77                             & $-$0.48                         & 5.2                       & 1.4                       & 10.2                         & 8.23E+20                                & 1                        \\
                    2659                    & 10577$+$0284$-$0042          & 105.7750                & 2.8417                  & 5,  6                                    & \nodata                           & 0.78                          & $-$4.22                             & $-$4.64                             & $-$0.44                         & 5.4                       & 2.5                       & 9.9                          & 1.19E+21                                & 1                        \\
                    2660                    & 10584$+$0048$-$0037          & 105.8417                & 0.4833                  & 5,  6                                    & \nodata                           & 0.77                          & $-$3.66                             & $-$4.18                             & $-$0.66                         & 4.3                       & 2.4                       & 9.2                          & 9.29E+20                                & 1                        \\
                    2661                    & 10591$+$0050$-$0534          & 105.9083                & 0.5000                  & 1,  3,  5,  6                            & \nodata                           & 2.82                          & $-$53.39                            & $-$53.84                            & $-$0.33                         & 8.4                       & 2.9                       & 14.5                         & 2.13E+21                                & 1                        \\
                    2662                    & 10602$+$0396$-$0077          & 106.0250                & 3.9583                  & 6                                        & \nodata                           & 0.81                          & $-$7.71                             & $-$8.26                             & $-$0.30                         & 13.7                      & 4.3                       & 18.3                         & 5.02E+21                                & 1                        \\
                    2663                    & 10604$+$0384$-$0072          & 106.0417                & 3.8417                  & 6                                        & \nodata                           & 0.81                          & $-$7.20                             & $-$8.00                             & $-$0.46                         & 8.5                       & 2.7                       & 15.5                         & 2.50E+21                                & 1                        \\
                    2664                    & 10606$+$0395$-$0079          & 106.0583                & 3.9500                  & 6                                        & \nodata                           & 0.81                          & $-$7.87                             & $-$8.49                             & $-$0.34                         & 11.8                      & 4.2                       & 22.0                         & 5.15E+21                                & 1                        \\
                    2665                    & 10658$+$0099$-$0124          & 106.5833                & 0.9917                  & 5,  6                                    & \nodata                           & 0.80                          & $-$12.40                            & $-$12.90                            & $-$0.47                         & 6.2                       & 3.8                       & 11.2                         & 2.29E+21                                & 1                        \\
                    2666                    & 10668$+$0107$-$0613          & 106.6833                & 1.0667                  & 5,  6                                    & \nodata                           & 2.77                          & $-$61.31                            & $-$62.17                            & $-$0.80                         & 4.4                       & 2.1                       & 9.5                          & 1.12E+21                                & 1                        \\
                    2667                    & 10688$+$0407$-$0060          & 106.8833                & 4.0750                  & 6                                        & \nodata                           & 0.80                          & $-$6.01                             & $-$6.45                             & $-$0.46                         & 10.3                      & 5.4                       & 15.9                         & 3.25E+21                                & 1                        \\
                    2668                    & 10697$+$0521$-$0090          & 106.9667                & 5.2083                  & 4,  5,  6                                & \nodata                           & 0.83                          & $-$9.01                             & $-$9.74                             & $-$0.52                         & 14.6                      & 9.7                       & 26.5                         & 1.17E+22                                & 1                        \\
                    2669                    & 10728$-$0027$-$0033          & 107.2833                & $-$0.2667                 & 5,  6                                    & \nodata                           & 0.76                          & $-$3.28                             & $-$3.65                             & $-$0.42                         & 7.0                       & 3.5                       & 10.1                         & 1.78E+21                                & 1                        \\
                    2670                    & 10750$+$0447$-$0022          & 107.5000                & 4.4667                  & 5,  6                                    & \nodata                           & 0.78                          & $-$2.22                             & $-$3.43                             & $-$0.60                         & 15.2                      & 9.0                       & 22.4                         & 1.48E+22                                & 1                        \\
                    2671                    & 10831$+$0379$-$0093          & 108.3083                & 3.7917                  & 6                                        & \nodata                           & 0.80                          & $-$9.33                             & $-$9.91                             & $-$0.64                         & 8.1                       & 2.7                       & 14.0                         & 1.28E+21                                & 1                        \\
                    2672                    & 10868$+$0087$-$0111          & 108.6833                & 0.8667                  & 6                                        & \nodata                           & 0.79                          & $-$11.07                            & $-$11.61                            & $-$0.40                         & 9.1                       & 2.4                       & 16.5                         & 1.77E+21                                & 1                        \\
                    2673                    & 10882$+$0020$-$0528          & 108.8167                & 0.2000                  & 6                                        & \nodata                           & 2.87                          & $-$52.77                            & $-$53.52                            & $-$0.53                         & 9.2                       & 3.8                       & 17.0                         & 3.13E+21                                & 1                        \\
                    2674                    & 10887$+$0254$-$0107          & 108.8667                & 2.5417                  & 6                                        & \nodata                           & 0.80                          & $-$10.74                            & $-$11.43                            & $-$0.57                         & 15.0                      & 4.0                       & 22.8                         & 3.24E+21                                & 1                        \\
                    2675                    & 10887$+$0271$-$0099          & 108.8750                & 2.7083                  & 6                                        & \nodata                           & 0.80                          & $-$9.87                             & $-$10.40                            & $-$0.42                         & 12.5                      & 7.8                       & 21.1                         & 7.55E+21                                & 1                        \\
                    2676                    & 10889$+$0260$-$0103          & 108.8917                & 2.6000                  & 6                                        & \nodata                           & 0.80                          & $-$10.29                            & $-$11.19                            & $-$0.72                         & 14.4                      & 8.3                       & 21.1                         & 8.07E+21                                & 1                        \\
                    2677                    & 10891$+$0266$-$0101          & 108.9083                & 2.6583                  & 1,  6                                    & \nodata                           & 0.80                          & $-$10.06                            & $-$10.96                            & $-$0.56                         & 14.6                      & 7.2                       & 23.8                         & 8.98E+21                                & 1                        \\
                    2678                    & 10892$+$0272$-$0099          & 108.9250                & 2.7250                  & 5,  6                                    & \nodata                           & 0.80                          & $-$9.94                             & $-$10.69                            & $-$0.50                         & 13.6                      & 10.3                      & 23.0                         & 1.31E+22                                & 1                        \\
                    2679                    & 10892$-$0002$-$0536          & 108.9250                & $-$0.0167                 & 6                                        & \nodata                           & 2.86                          & $-$53.61                            & $-$54.13                            & $-$0.46                         & 5.0                       & 2.7                       & 9.5                          & 1.66E+21                                & 1                        \\
                    2680                    & 10895$+$0273$-$0101          & 108.9500                & 2.7333                  & 1,  5,  6                                & 17                                & 0.80                          & $-$10.05                            & $-$10.75                            & $-$0.46                         & 12.5                      & 8.6                       & 22.6                         & 1.04E+22                                & 1                        \\
                    2681                    & 10896$+$0269$-$0097          & 108.9583                & 2.6917                  & 5,  6                                    & \nodata                           & 0.80                          & $-$9.75                             & $-$10.42                            & $-$0.41                         & 17.5                      & 9.4                       & 31.2                         & 1.42E+22                                & 1                        \\
                    2682                    & 10897$+$0274$-$0104          & 108.9750                & 2.7417                  & 5,  6                                    & \nodata                           & 0.80                          & $-$10.40                            & $-$11.11                            & $-$0.46                         & 12.5                      & 9.8                       & 25.2                         & 1.30E+22                                & 1                        \\
                    2683                    & 10909$+$0267$-$0115          & 109.0917                & 2.6667                  & 5,  6                                    & \nodata                           & 0.80                          & $-$11.52                            & $-$12.11                            & $-$0.48                         & 12.3                      & 6.9                       & 19.4                         & 6.09E+21                                & 1                        \\
                    2684                    & 10964$+$0252$-$0090          & 109.6417                & 2.5167                  & 5,  6                                    & \nodata                           & 0.79                          & $-$9.00                             & $-$9.68                             & $-$0.62                         & 12.1                      & 5.7                       & 19.9                         & 4.25E+21                                & 1                        \\
                    2685                    & 10967$+$0253$-$0088          & 109.6667                & 2.5333                  & 5,  6                                    & \nodata                           & 0.79                          & $-$8.82                             & $-$9.29                             & $-$0.35                         & 11.8                      & 7.5                       & 21.2                         & 7.58E+21                                & 1                        \\
                    2686                    & 10970$+$0251$-$0085          & 109.7000                & 2.5083                  & 5,  6                                    & \nodata                           & 0.79                          & $-$8.54                             & $-$9.22                             & $-$0.39                         & 13.1                      & 6.9                       & 20.9                         & 8.65E+21                                & 1                        \\
                    2687                    & 10977$+$0173$-$0056          & 109.7750                & 1.7333                  & 3,  6                                    & \nodata                           & 0.78                          & $-$5.56                             & $-$6.29                             & $-$0.59                         & 5.3                       & 1.1                       & 10.6                         & 6.02E+20                                & 1                        \\
                    2688                    & 10980$+$0134$-$0108          & 109.8000                & 1.3417                  & 6                                        & \nodata                           & 0.79                          & $-$10.79                            & $-$11.16                            & $-$0.36                         & 7.4                       & 2.7                       & 14.2                         & 1.47E+21                                & 1                        \\
                    2689                    & 10982$+$0215$-$0100          & 109.8167                & 2.1500                  & 3,  6                                    & \nodata                           & 0.79                          & $-$9.99                             & $-$10.77                            & $-$0.54                         & 16.4                      & 6.5                       & 22.9                         & 6.98E+21                                & 1                        \\
                    2690                    & 10983$+$0157$+$0050          & 109.8333                & 1.5667                  & 3,  6                                    & \nodata                           & 0.03                          & 4.99                              & 4.55                              & $-$0.39                         & 5.6                       & 1.6                       & 9.8                          & 8.12E+20                                & 1                        \\
                    2691                    & 10985$+$0210$-$0098          & 109.8500                & 2.1000                  & 3,  6                                    & 10,  11,  13                      & 0.79                          & $-$9.80                             & $-$10.28                            & $-$0.23                         & 8.3                       & 1.2                       & 12.2                         & 9.60E+21                                & 2                        \\
                    2692                    & 10992$+$0211$-$0110          & 109.9167                & 2.1083                  & 3,  5,  6                                & 10,  11,  13                      & 0.79                          & $-$10.95                            & $-$12.01                            & $-$0.46                         & 18.8                      & 9.3                       & 25.2                         & 1.78E+22                                & 1                        \\
                    2693                    & 11007$+$0161$-$0064          & 110.0667                & 1.6083                  & 5,  6                                    & \nodata                           & 0.78                          & $-$6.42                             & $-$7.18                             & $-$0.66                         & 5.9                       & 1.5                       & 10.1                         & 7.66E+20                                & 1                        \\
                    2694                    & 11011$+$0159$-$0068          & 110.1083                & 1.5917                  & 5,  6                                    & \nodata                           & 0.78                          & $-$6.78                             & $-$7.33                             & $-$0.40                         & 5.6                       & 1.5                       & 11.4                         & 9.20E+20                                & 1                        \\
                    2695                    & 11013$+$0161$-$0063          & 110.1333                & 1.6083                  & 5,  6                                    & \nodata                           & 0.78                          & $-$6.34                             & $-$7.03                             & $-$0.58                         & 5.3                       & 1.3                       & 10.8                         & 6.64E+20                                & 1                        \\
                    2696                    & 11019$+$0096$-$0527          & 110.1917                & 0.9583                  & 6                                        & \nodata                           & 2.71                          & $-$52.66                            & $-$53.47                            & $-$0.51                         & 5.6                       & 1.8                       & 10.3                         & 1.35E+21                                & 1                        \\
                    2697                    & 11022$+$0095$-$0524          & 110.2250                & 0.9500                  & 6                                        & \nodata                           & 2.71                          & $-$52.44                            & $-$53.31                            & $-$0.44                         & 6.5                       & 3.2                       & 12.0                         & 3.31E+21                                & 1                        \\
                    2698                    & 11027$+$0160$-$0064          & 110.2667                & 1.6000                  & 5,  6                                    & \nodata                           & 0.78                          & $-$6.43                             & $-$6.90                             & $-$0.53                         & 4.4                       & 1.2                       & 10.6                         & 4.66E+20                                & 1                        \\
                    2699                    & 11032$+$0254$-$0120          & 110.3167                & 2.5417                  & 3,  4,  5,  6                            & \nodata                           & 0.79                          & $-$11.97                            & $-$12.60                            & $-$0.39                         & 20.1                      & 13.3                      & 36.8                         & 2.36E+22                                & 1                        \\
                    2700                    & 11040$+$0167$-$0112          & 110.4000                & 1.6667                  & 5,  6                                    & 17                                & 0.79                          & $-$11.18                            & $-$11.94                            & $-$0.53                         & 9.1                       & 4.4                       & 16.5                         & 3.79E+21                                & 1                        \\
                    2701                    & 11047$+$0158$-$0098          & 110.4667                & 1.5833                  & 5,  6                                    & \nodata                           & 0.78                          & $-$9.80                             & $-$10.74                            & $-$0.59                         & 8.8                       & 3.4                       & 16.2                         & 3.02E+21                                & 1                        \\
                    2702                    & 11054$+$0163$-$0098          & 110.5417                & 1.6333                  & 5,  6                                    & \nodata                           & 0.78                          & $-$9.84                             & $-$10.49                            & $-$0.39                         & 8.4                       & 3.9                       & 17.5                         & 3.90E+21                                & 1                        \\
                    2703                    & 11060$+$0160$-$0101          & 110.6000                & 1.6000                  & 5,  6                                    & \nodata                           & 0.78                          & $-$10.07                            & $-$10.71                            & $-$0.42                         & 8.7                       & 3.7                       & 15.5                         & 3.25E+21                                & 1                        \\
                    2704                    & 11091$-$0242$-$0344          & 110.9083                & $-$2.4167                 & 6                                        & \nodata                           & 3.01                          & $-$34.39                            & $-$35.38                            & $-$0.63                         & 6.5                       & 2.3                       & 13.2                         & 1.80E+21                                & 1                        \\
                    2705                    & 11107$+$0205$-$0103          & 111.0667                & 2.0500                  & 5,  6                                    & \nodata                           & 0.78                          & $-$10.26                            & $-$11.10                            & $-$0.42                         & 11.4                      & 6.6                       & 22.4                         & 9.68E+21                                & 1                        \\
                    2706                    & 11112$+$0212$-$0099          & 111.1167                & 2.1250                  & 5,  6                                    & \nodata                           & 0.78                          & $-$9.90                             & $-$10.92                            & $-$0.49                         & 12.5                      & 6.1                       & 29.1                         & 1.04E+22                                & 1                        \\
                    2707                    & 11114$+$0212$-$0101          & 111.1417                & 2.1167                  & 5,  6                                    & \nodata                           & 0.78                          & $-$10.06                            & $-$10.69                            & $-$0.31                         & 10.9                      & 5.4                       & 21.0                         & 7.44E+21                                & 1                        \\
                    2708                    & 11123$+$0207$-$0089          & 111.2333                & 2.0750                  & 5,  6                                    & \nodata                           & 0.78                          & $-$8.93                             & $-$9.71                             & $-$0.36                         & 10.9                      & 5.1                       & 20.4                         & 7.38E+21                                & 1                        \\
                    2709                    & 11126$+$0205$-$0092          & 111.2583                & 2.0500                  & 5,  6                                    & \nodata                           & 0.78                          & $-$9.23                             & $-$9.96                             & $-$0.45                         & 11.1                      & 5.5                       & 17.9                         & 5.91E+21                                & 1                        \\
                    2710                    & 11147$+$0158$-$0089          & 111.4750                & 1.5833                  & 6                                        & \nodata                           & 0.78                          & $-$8.94                             & $-$9.60                             & $-$0.87                         & 6.4                       & 2.2                       & 10.5                         & 8.28E+20                                & 1                        \\
                    2711                    & 11149$+$0161$-$0092          & 111.4917                & 1.6083                  & 6                                        & \nodata                           & 0.78                          & $-$9.19                             & $-$9.59                             & $-$0.28                         & 7.3                       & 1.3                       & 13.6                         & 8.90E+20                                & 1                        \\
                    2712                    & 11152$+$0074$-$0549          & 111.5250                & 0.7417                  & 2,  3,  6                                & 8,  11                            & 2.69                          & $-$54.92                            & $-$57.54                            & $-$0.35                         & 21.7                      & 5.6                       & 33.2                         & 3.66E+22                                & 1                        \\
                    2713                    & 11156$+$0078$-$0556          & 111.5583                & 0.7833                  & 1,  2,  3,  4,  6                        & 8                                 & 2.69                          & $-$55.58                            & $-$58.14                            & $-$0.46                         & 17.1                      & 5.5                       & 23.7                         & 2.24E+22                                & 1                        \\
                    2714                    & 11156$+$0197$-$0101          & 111.5583                & 1.9667                  & 6                                        & \nodata                           & 0.78                          & $-$10.11                            & $-$10.68                            & $-$0.54                         & 10.6                      & 3.9                       & 17.3                         & 2.47E+21                                & 1                        \\
                    2715                    & 11173$+$0002$-$0308          & 111.7333                & 0.0250                  & 1,  6                                    & \nodata                           & 3.47                          & $-$30.76                            & $-$32.59                            & $-$0.79                         & 11.4                      & 4.4                       & 17.2                         & 6.24E+21                                & 1                        \\
                    2716                    & 11176$-$0270$-$0397          & 111.7583                & $-$2.7000                 & 5,  6                                    & \nodata                           & 2.79                          & $-$39.74                            & $-$40.30                            & $-$0.46                         & 5.8                       & 2.5                       & 9.6                          & 1.51E+21                                & 1                        \\
                    2717                    & 11187$+$0082$-$0511          & 111.8750                & 0.8250                  & 1,  4,  6                                & \nodata                           & 2.69                          & $-$51.13                            & $-$52.44                            & $-$0.37                         & 17.8                      & 7.8                       & 29.9                         & 2.35E+22                                & 1                        \\
                    2718                    & 11208$+$0150$-$0052          & 112.0833                & 1.5000                  & 3,  4,  6                                & \nodata                           & 0.77                          & $-$5.17                             & $-$5.67                             & $-$0.45                         & 6.1                       & 2.3                       & 11.8                         & 1.28E+21                                & 1                        \\
                    2719                    & 11374$+$0089$-$0755          & 113.7417                & 0.8917                  & 5,  6                                    & \nodata                           & 3.39                          & $-$75.55                            & $-$75.89                            & $-$0.37                         & 4.2                       & 1.4                       & 9.6                          & 5.83E+20                                & 1                        \\
                    2720                    & 11436$-$0244$-$0026          & 114.3583                & $-$2.4417                 & 5,  6                                    & \nodata                           & 0.73                          & $-$2.59                             & $-$3.12                             & $-$0.60                         & 5.0                       & 1.0                       & 12.0                         & 3.77E+20                                & 1                        \\
                    2721                    & 11439$-$0243$-$0027          & 114.3917                & $-$2.4333                 & 5,  6                                    & \nodata                           & 0.73                          & $-$2.65                             & $-$3.24                             & $-$0.45                         & 5.5                       & 1.8                       & 12.4                         & 1.11E+21                                & 1                        \\
                    2722                    & 11442$-$0242$-$0028          & 114.4250                & $-$2.4250                 & 5,  6                                    & \nodata                           & 0.73                          & $-$2.82                             & $-$3.27                             & $-$0.31                         & 6.5                       & 1.5                       & 13.4                         & 1.01E+21                                & 1                        \\
                    2723                    & 11452$-$0053$-$0478          & 114.5167                & $-$0.5333                 & 5,  6                                    & \nodata                           & 2.78                          & $-$47.82                            & $-$49.34                            & $-$0.58                         & 17.9                      & 6.5                       & 26.8                         & 1.33E+22                                & 1                        \\
                    2724                    & 11474$+$0192$-$0113          & 114.7417                & 1.9167                  & 6                                        & \nodata                           & 0.76                          & $-$11.26                            & $-$12.16                            & $-$0.76                         & 7.8                       & 2.4                       & 15.1                         & 1.48E+21                                & 1                        \\
                    2725                    & 11484$+$0077$+$0020          & 114.8417                & 0.7750                  & 5,  6                                    & \nodata                           & 0.74                          & 2.00                              & 1.41                              & $-$0.63                         & 3.7                       & 1.1                       & 7.5                          & 4.84E+20                                & 1                        \\
                    2726                    & 11494$+$0195$-$0042          & 114.9417                & 1.9500                  & 6                                        & \nodata                           & 0.75                          & $-$4.24                             & $-$4.89                             & $-$0.68                         & 6.7                       & 3.8                       & 14.3                         & 2.04E+21                                & 1                        \\
                    2727                    & 11498$-$0161$-$0419          & 114.9833                & $-$1.6083                 & 5,  6                                    & \nodata                           & 2.79                          & $-$41.88                            & $-$42.21                            & $-$0.40                         & 7.2                       & 1.7                       & 13.0                         & 6.70E+20                                & 1                        \\
                    2728                    & 11500$+$0111$-$0136          & 115.0000                & 1.1083                  & 6                                        & \nodata                           & 0.76                          & $-$13.56                            & $-$13.89                            & $-$0.44                         & 7.6                       & 2.4                       & 10.8                         & 9.25E+20                                & 1                        \\
                    2729                    & 11517$+$0179$-$0119          & 115.1667                & 1.7917                  & 6                                        & \nodata                           & 0.76                          & $-$11.88                            & $-$12.40                            & $-$0.59                         & 4.9                       & 1.4                       & 10.0                         & 5.82E+20                                & 1                        \\
                    2730                    & 11517$+$0497$-$0061          & 115.1750                & 4.9667                  & 6                                        & \nodata                           & 0.75                          & $-$6.05                             & $-$6.50                             & $-$0.36                         & 4.3                       & 1.8                       & 9.4                          & 1.04E+21                                & 1                        \\
                    2731                    & 11524$+$0387$-$0245          & 115.2417                & 3.8667                  & 6                                        & \nodata                           & 0.93                          & $-$24.47                            & $-$24.86                            & $-$0.33                         & 6.0                       & 1.6                       & 10.5                         & 8.80E+20                                & 1                        \\
                    2732                    & 11527$+$0383$-$0244          & 115.2667                & 3.8333                  & 6                                        & \nodata                           & 0.93                          & $-$24.44                            & $-$25.03                            & $-$0.61                         & 5.5                       & 1.6                       & 11.9                         & 7.10E+20                                & 1                        \\
                    2733                    & 11536$+$0374$-$0247          & 115.3583                & 3.7417                  & 6                                        & \nodata                           & 0.93                          & $-$24.70                            & $-$25.32                            & $-$0.52                         & 6.7                       & 2.3                       & 11.7                         & 1.40E+21                                & 1                        \\
                    2734                    & 11542$+$0279$-$0039          & 115.4167                & 2.7917                  & 6                                        & \nodata                           & 0.74                          & $-$3.93                             & $-$4.50                             & $-$0.48                         & 6.5                       & 2.2                       & 11.3                         & 1.27E+21                                & 1                        \\
                    2735                    & 11546$+$0357$-$0250          & 115.4583                & 3.5667                  & 6                                        & \nodata                           & 0.93                          & $-$24.98                            & $-$25.51                            & $-$0.51                         & 5.6                       & 1.9                       & 11.0                         & 9.34E+20                                & 1                        \\
                    2736                    & 11550$+$0366$-$0251          & 115.5000                & 3.6583                  & 6                                        & \nodata                           & 0.93                          & $-$25.11                            & $-$25.58                            & $-$0.49                         & 5.3                       & 2.3                       & 10.6                         & 1.09E+21                                & 1                        \\
                    2737                    & 11554$+$0390$-$0255          & 115.5417                & 3.9000                  & 6                                        & \nodata                           & 0.93                          & $-$25.50                            & $-$25.92                            & $-$0.39                         & 5.0                       & 2.8                       & 8.9                          & 1.65E+21                                & 1                        \\
                    2738                    & 11562$+$0199$-$0104          & 115.6167                & 1.9917                  & 6                                        & \nodata                           & 0.75                          & $-$10.41                            & $-$10.97                            & $-$0.57                         & 6.2                       & 2.7                       & 13.9                         & 1.41E+21                                & 1                        \\
                    2739                    & 11565$+$0129$-$0090          & 115.6500                & 1.2917                  & 5,  6                                    & \nodata                           & 0.75                          & $-$8.97                             & $-$9.27                             & $-$0.31                         & 8.4                       & 4.2                       & 15.4                         & 2.34E+21                                & 1                        \\
                    2740                    & 11565$+$0465$-$0083          & 115.6500                & 4.6500                  & 6                                        & \nodata                           & 0.75                          & $-$8.35                             & $-$9.35                             & $-$0.89                         & 4.3                       & 1.8                       & 11.0                         & 9.64E+20                                & 1                        \\
                    2741                    & 11569$+$0126$-$0093          & 115.6917                & 1.2583                  & 5,  6                                    & \nodata                           & 0.75                          & $-$9.30                             & $-$9.66                             & $-$0.35                         & 7.5                       & 3.5                       & 15.0                         & 1.98E+21                                & 1                        \\
                    2742                    & 11573$+$0142$-$0087          & 115.7333                & 1.4250                  & 5,  6                                    & \nodata                           & 0.75                          & $-$8.73                             & $-$9.13                             & $-$0.39                         & 6.5                       & 2.8                       & 12.4                         & 1.47E+21                                & 1                        \\
                    2743                    & 11577$+$0213$-$0102          & 115.7750                & 2.1333                  & 6                                        & \nodata                           & 0.75                          & $-$10.17                            & $-$10.59                            & $-$0.51                         & 7.3                       & 2.2                       & 14.2                         & 9.34E+20                                & 1                        \\
                    2744                    & 11580$+$0143$-$0085          & 115.8000                & 1.4333                  & 5,  6                                    & \nodata                           & 0.75                          & $-$8.49                             & $-$8.88                             & $-$0.34                         & 4.9                       & 2.4                       & 10.2                         & 1.31E+21                                & 1                        \\
                    2745                    & 11588$+$0143$-$0081          & 115.8833                & 1.4333                  & 6                                        & \nodata                           & 0.75                          & $-$8.12                             & $-$8.35                             & $-$0.31                         & 6.1                       & 2.9                       & 10.7                         & 1.10E+21                                & 1                        \\
                    2746                    & 11592$+$0202$-$0099          & 115.9250                & 2.0167                  & 6                                        & \nodata                           & 0.75                          & $-$9.94                             & $-$10.49                            & $-$0.45                         & 5.7                       & 1.5                       & 13.0                         & 8.83E+20                                & 1                        \\
                    2747                    & 11592$-$0159$-$0414          & 115.9250                & $-$1.5917                 & 6                                        & \nodata                           & 2.77                          & $-$41.35                            & $-$41.97                            & $-$0.37                         & 11.2                      & 3.5                       & 19.8                         & 3.68E+21                                & 1                        \\
                    2748                    & 11593$+$0457$-$0065          & 115.9333                & 4.5667                  & 6                                        & \nodata                           & 0.74                          & $-$6.46                             & $-$7.12                             & $-$0.47                         & 3.6                       & 1.8                       & 7.0                          & 1.41E+21                                & 1                        \\
                    2749                    & 11597$+$0417$-$0244          & 115.9667                & 4.1667                  & 6                                        & \nodata                           & 0.93                          & $-$24.39                            & $-$25.13                            & $-$0.46                         & 5.5                       & 1.7                       & 10.5                         & 1.26E+21                                & 1                        \\
                    2750                    & 11602$+$0204$-$0102          & 116.0167                & 2.0417                  & 6                                        & \nodata                           & 0.74                          & $-$10.20                            & $-$10.88                            & $-$0.68                         & 4.9                       & 2.2                       & 11.4                         & 1.09E+21                                & 1                        \\
                    2751                    & 11605$-$0211$-$0030          & 116.0500                & $-$2.1083                 & 5,  6                                    & \nodata                           & 0.41                          & $-$3.01                             & $-$3.54                             & $-$0.53                         & 5.8                       & 1.6                       & 22.1                         & 9.54E+20                                & 1                        \\
                    2752                    & 11609$+$0196$-$0101          & 116.0917                & 1.9583                  & 6                                        & \nodata                           & 0.74                          & $-$10.06                            & $-$10.60                            & $-$0.51                         & 6.3                       & 1.8                       & 15.2                         & 9.47E+20                                & 1                        \\
                    2753                    & 11611$+$0140$-$0073          & 116.1083                & 1.4000                  & 6                                        & \nodata                           & 0.73                          & $-$7.28                             & $-$7.75                             & $-$0.59                         & 6.3                       & 1.7                       & 11.4                         & 6.42E+20                                & 1                        \\
                    2754                    & 11612$+$0194$-$0095          & 116.1167                & 1.9417                  & 6                                        & \nodata                           & 0.74                          & $-$9.51                             & $-$10.17                            & $-$0.41                         & 5.8                       & 2.6                       & 10.5                         & 2.09E+21                                & 1                        \\
                    2755                    & 11612$+$0394$-$0250          & 116.1167                & 3.9417                  & 6                                        & \nodata                           & 0.93                          & $-$24.95                            & $-$25.32                            & $-$0.31                         & 5.2                       & 1.6                       & 9.2                          & 8.53E+20                                & 1                        \\
                    2756                    & 11613$+$0145$-$0076          & 116.1333                & 1.4500                  & 6                                        & \nodata                           & 0.73                          & $-$7.60                             & $-$8.21                             & $-$0.56                         & 5.8                       & 2.7                       & 11.9                         & 1.47E+21                                & 1                        \\
                    2757                    & 11615$+$0138$-$0072          & 116.1500                & 1.3833                  & 6                                        & \nodata                           & 0.73                          & $-$7.24                             & $-$7.64                             & $-$0.39                         & 7.3                       & 1.9                       & 14.1                         & 9.63E+20                                & 1                        \\
                    2758                    & 11617$+$0192$-$0096          & 116.1667                & 1.9250                  & 6                                        & \nodata                           & 0.74                          & $-$9.61                             & $-$10.36                            & $-$0.52                         & 5.1                       & 2.5                       & 10.0                         & 1.84E+21                                & 1                        \\
                    2759                    & 11648$-$0275$+$0008          & 116.4833                & $-$2.7500                 & 6                                        & \nodata                           & 0.71                          & 0.83                              & 0.43                              & $-$0.37                         & 6.6                       & 2.4                       & 10.7                         & 1.27E+21                                & 1                        \\
                    2760                    & 11722$+$0462$-$0075          & 117.2167                & 4.6167                  & 4,  5,  6                                & \nodata                           & 0.73                          & $-$7.48                             & $-$8.18                             & $-$0.50                         & 4.1                       & 1.5                       & 10.0                         & 9.63E+20                                & 1                        \\
                    2761                    & 11728$+$0314$-$0183          & 117.2833                & 3.1417                  & 3,  6                                    & \nodata                           & 0.71                          & $-$18.31                            & $-$19.39                            & $-$0.60                         & 9.1                       & 7.0                       & 24.5                         & 9.59E+21                                & 1                        \\
                    2762                    & 11730$+$0310$-$0187          & 117.3000                & 3.1000                  & 3,  6                                    & \nodata                           & 0.71                          & $-$18.73                            & $-$19.78                            & $-$0.67                         & 11.1                      & 7.1                       & 15.8                         & 8.00E+21                                & 1                        \\
                    2763                    & 11732$+$0314$-$0187          & 117.3250                & 3.1417                  & 1,  3,  6                                & \nodata                           & 0.71                          & $-$18.66                            & $-$19.84                            & $-$0.59                         & 10.4                      & 8.1                       & 15.4                         & 1.26E+22                                & 1                        \\
                    2764                    & 11749$+$0441$-$0074          & 117.4917                & 4.4083                  & 4,  5,  6                                & \nodata                           & 0.73                          & $-$7.39                             & $-$8.29                             & $-$0.54                         & 6.1                       & 1.7                       & 13.8                         & 1.41E+21                                & 1                        \\
                    2765                    & 11752$+$0440$-$0077          & 117.5167                & 4.4000                  & 4,  5,  6                                & \nodata                           & 0.73                          & $-$7.70                             & $-$8.57                             & $-$0.44                         & 6.1                       & 1.3                       & 14.0                         & 1.29E+21                                & 1                        \\
                    2766                    & 11752$-$0055$+$0008          & 117.5250                & $-$0.5500                 & 6                                        & \nodata                           & 0.71                          & 0.84                              & 0.43                              & $-$0.40                         & 6.9                       & 2.4                       & 13.1                         & 1.28E+21                                & 1                        \\
                    2767                    & 11762$+$0481$-$0079          & 117.6250                & 4.8083                  & 4,  5,  6                                & \nodata                           & 0.73                          & $-$7.87                             & $-$8.35                             & $-$0.49                         & 4.5                       & 1.8                       & 9.7                          & 8.52E+20                                & 1                        \\
                    2768                    & 11763$+$0492$-$0076          & 117.6333                & 4.9167                  & 3,  4,  5,  6                            & \nodata                           & 0.73                          & $-$7.64                             & $-$8.10                             & $-$0.42                         & 6.0                       & 2.5                       & 12.5                         & 1.37E+21                                & 1                        \\
                    2769                    & 11765$+$0520$-$0079          & 117.6500                & 5.2000                  & 4,  5,  6                                & \nodata                           & 0.73                          & $-$7.94                             & $-$8.59                             & $-$0.63                         & 4.9                       & 1.7                       & 12.0                         & 8.51E+20                                & 1                        \\
                    2770                    & 11781$+$0485$-$0007          & 117.8083                & 4.8500                  & 4,  5,  6                                & \nodata                           & 0.71                          & $-$0.72                             & $-$1.15                             & $-$0.50                         & 6.1                       & 1.8                       & 10.6                         & 7.35E+20                                & 1                        \\
                    2771                    & 11782$+$0488$-$0007          & 117.8167                & 4.8833                  & 3,  4,  5,  6                            & \nodata                           & 0.71                          & $-$0.68                             & $-$0.99                             & $-$0.51                         & 4.8                       & 1.0                       & 11.9                         & 2.62E+20                                & 1                        \\
                    2772                    & 11793$+$0499$-$0132          & 117.9333                & 4.9917                  & 3,  4,  5,  6                            & \nodata                           & 0.74                          & $-$13.20                            & $-$14.12                            & $-$0.41                         & 15.1                      & 3.4                       & 22.0                         & 4.93E+21                                & 1                        \\
                    2773                    & 11812$+$0519$-$0148          & 118.1167                & 5.1917                  & 4,  5,  6                                & \nodata                           & 0.75                          & $-$14.78                            & $-$15.66                            & $-$0.44                         & 16.4                      & 9.0                       & 25.7                         & 1.53E+22                                & 1                        \\
                    2774                    & 11814$+$0506$-$0131          & 118.1417                & 5.0583                  & 2,  3,  4,  5,  6                        & \nodata                           & 0.74                          & $-$13.12                            & $-$14.20                            & $-$0.50                         & 15.1                      & 5.1                       & 19.4                         & 7.18E+21                                & 1                        \\
                    2775                    & 11817$+$0456$-$0075          & 118.1667                & 4.5583                  & 4,  5,  6                                & \nodata                           & 0.72                          & $-$7.46                             & $-$7.97                             & $-$0.62                         & 4.8                       & 1.5                       & 10.6                         & 5.50E+20                                & 1                        \\
                    2776                    & 11817$+$0459$-$0073          & 118.1667                & 4.5917                  & 4,  5,  6                                & \nodata                           & 0.72                          & $-$7.31                             & $-$7.63                             & $-$0.39                         & 5.0                       & 1.4                       & 9.6                          & 5.30E+20                                & 1                        \\
                    2777                    & 11825$+$0479$-$0068          & 118.2500                & 4.7917                  & 4,  5,  6                                & \nodata                           & 0.72                          & $-$6.80                             & $-$7.06                             & $-$0.36                         & 5.8                       & 2.7                       & 9.9                          & 9.98E+20                                & 1                        \\
                    2778                    & 11828$+$0480$-$0069          & 118.2833                & 4.8000                  & 4,  5,  6                                & \nodata                           & 0.72                          & $-$6.91                             & $-$7.33                             & $-$0.51                         & 4.2                       & 3.5                       & 9.3                          & 1.74E+21                                & 1                        \\
                    2779                    & 11830$+$0474$-$0069          & 118.3000                & 4.7417                  & 4,  5,  6                                & \nodata                           & 0.72                          & $-$6.88                             & $-$7.30                             & $-$0.42                         & 5.5                       & 2.8                       & 11.0                         & 1.41E+21                                & 1                        \\
                    2780                    & 11831$+$0472$-$0068          & 118.3083                & 4.7167                  & 4,  5,  6                                & \nodata                           & 0.72                          & $-$6.83                             & $-$7.52                             & $-$0.70                         & 5.6                       & 1.7                       & 11.6                         & 7.94E+20                                & 1                        \\
                    2781                    & 11832$+$0480$-$0071          & 118.3167                & 4.8000                  & 4,  5,  6                                & \nodata                           & 0.72                          & $-$7.11                             & $-$7.45                             & $-$0.41                         & 5.0                       & 2.9                       & 10.3                         & 1.28E+21                                & 1                        \\
                    2782                    & 11832$+$0321$-$0049          & 118.3250                & 3.2083                  & 4,  5,  6                                & \nodata                           & 0.71                          & $-$4.91                             & $-$5.16                             & $-$0.25                         & 6.9                       & 1.2                       & 10.6                         & 5.18E+20                                & 1                        \\
                    2783                    & 11832$+$0482$-$0070          & 118.3250                & 4.8250                  & 4,  5,  6                                & \nodata                           & 0.72                          & $-$7.03                             & $-$7.40                             & $-$0.36                         & 5.6                       & 2.9                       & 13.3                         & 1.61E+21                                & 1                        \\
                    2784                    & 11833$+$0485$-$0071          & 118.3333                & 4.8500                  & 4,  5,  6                                & \nodata                           & 0.72                          & $-$7.13                             & $-$7.65                             & $-$0.50                         & 4.7                       & 3.0                       & 11.6                         & 1.62E+21                                & 1                        \\
                    2785                    & 11837$+$0318$-$0048          & 118.3667                & 3.1833                  & 6                                        & \nodata                           & 0.71                          & $-$4.82                             & $-$5.36                             & $-$0.38                         & 17.1                      & 5.0                       & 22.9                         & 4.94E+21                                & 1                        \\
                    2786                    & 11837$+$0321$-$0048          & 118.3750                & 3.2083                  & 4,  5,  6                                & \nodata                           & 0.71                          & $-$4.81                             & $-$5.39                             & $-$0.47                         & 13.2                      & 4.3                       & 20.8                         & 3.42E+21                                & 1                        \\
                    2787                    & 11837$+$0480$-$0073          & 118.3750                & 4.8000                  & 4,  5,  6                                & \nodata                           & 0.72                          & $-$7.26                             & $-$7.69                             & $-$0.50                         & 5.2                       & 1.3                       & 11.5                         & 4.97E+20                                & 1                        \\
                    2788                    & 11839$+$0482$-$0073          & 118.3917                & 4.8250                  & 4,  5,  6                                & \nodata                           & 0.72                          & $-$7.32                             & $-$7.85                             & $-$0.58                         & 4.3                       & 1.8                       & 13.1                         & 7.90E+20                                & 1                        \\
                    2789                    & 11839$+$0496$-$0071          & 118.3917                & 4.9583                  & 4,  5,  6                                & \nodata                           & 0.72                          & $-$7.13                             & $-$8.05                             & $-$0.95                         & 4.6                       & 1.3                       & 11.4                         & 5.98E+20                                & 1                        \\
                    2790                    & 11842$+$0317$-$0047          & 118.4250                & 3.1750                  & 6                                        & \nodata                           & 0.71                          & $-$4.70                             & $-$5.14                             & $-$0.34                         & 11.7                      & 6.6                       & 24.0                         & 6.51E+21                                & 1                        \\
                    2791                    & 11845$+$0318$-$0047          & 118.4500                & 3.1833                  & 6                                        & \nodata                           & 0.71                          & $-$4.67                             & $-$5.25                             & $-$0.47                         & 9.8                       & 4.6                       & 18.5                         & 3.56E+21                                & 1                        \\
                    2792                    & 11862$+$0323$-$0052          & 118.6167                & 3.2333                  & 4,  5,  6                                & \nodata                           & 0.71                          & $-$5.22                             & $-$5.56                             & $-$0.36                         & 6.8                       & 3.3                       & 11.0                         & 1.72E+21                                & 1                        \\
                    2793                    & 11867$+$0306$-$0056          & 118.6667                & 3.0583                  & 6                                        & \nodata                           & 0.71                          & $-$5.62                             & $-$5.85                             & $-$0.31                         & 6.4                       & 2.1                       & 12.3                         & 8.04E+20                                & 1                        \\
                    2794                    & 11873$+$0510$-$0036          & 118.7333                & 5.1000                  & 4,  5,  6                                & \nodata                           & 0.71                          & $-$3.58                             & $-$4.08                             & $-$0.47                         & 7.1                       & 3.1                       & 11.9                         & 1.76E+21                                & 1                        \\
                    2795                    & 11885$+$0359$-$0072          & 118.8500                & 3.5917                  & 4,  5,  6                                & \nodata                           & 0.71                          & $-$7.18                             & $-$7.72                             & $-$0.62                         & 5.5                       & 2.9                       & 10.6                         & 1.35E+21                                & 1                        \\
                    2796                    & 11890$+$0421$+$0011          & 118.9000                & 4.2083                  & 4,  5,  6                                & \nodata                           & 0.71                          & 1.12                              & 0.27                              & $-$0.62                         & 4.9                       & 1.6                       & 12.2                         & 1.02E+21                                & 1                        \\
                    2797                    & 11892$+$0300$-$0043          & 118.9167                & 3.0000                  & 1,  6                                    & \nodata                           & 0.71                          & $-$4.34                             & $-$4.61                             & $-$0.29                         & 11.4                      & 6.1                       & 20.5                         & 3.94E+21                                & 1                        \\
                    2798                    & 11894$+$0515$-$0040          & 118.9417                & 5.1500                  & 4,  5,  6                                & \nodata                           & 0.71                          & $-$4.04                             & $-$4.69                             & $-$0.50                         & 6.3                       & 1.9                       & 11.9                         & 1.22E+21                                & 1                        \\
                    2799                    & 11895$+$0012$-$0466          & 118.9500                & 0.1250                  & 6                                        & \nodata                           & 2.68                          & $-$46.55                            & $-$46.85                            & $-$0.28                         & 5.9                       & 1.8                       & 11.8                         & 8.82E+20                                & 1                        \\
                    2800                    & 11907$+$0137$-$0036          & 119.0750                & 1.3667                  & 6                                        & \nodata                           & 0.71                          & $-$3.60                             & $-$4.14                             & $-$0.38                         & 7.0                       & 4.3                       & 12.5                         & 3.61E+21                                & 1                        \\
                    2801                    & 11910$+$0293$-$0046          & 119.1000                & 2.9333                  & 6                                        & \nodata                           & 0.70                          & $-$4.59                             & $-$4.77                             & $-$0.21                         & 5.4                       & 2.4                       & 9.8                          & 1.03E+21                                & 1                        \\
                    2802                    & 11925$+$0298$-$0050          & 119.2500                & 2.9833                  & 1,  6                                    & \nodata                           & 0.70                          & $-$4.97                             & $-$5.83                             & $-$0.42                         & 5.8                       & 1.1                       & 9.7                          & 1.01E+21                                & 1                        \\
                    2803                    & 11929$-$0138$-$0356          & 119.2917                & $-$1.3833                 & 6                                        & \nodata                           & 0.93                          & $-$35.59                            & $-$36.18                            & $-$0.38                         & 8.0                       & 1.5                       & 14.6                         & 1.19E+21                                & 1                        \\
                    2804                    & 11937$+$0314$-$0050          & 119.3750                & 3.1417                  & 6                                        & \nodata                           & 0.70                          & $-$4.97                             & $-$5.85                             & $-$0.49                         & 5.5                       & 3.1                       & 12.1                         & 2.94E+21                                & 1                        \\
                    2805                    & 11939$+$0317$-$0051          & 119.3917                & 3.1667                  & 6                                        & \nodata                           & 0.70                          & $-$5.08                             & $-$5.86                             & $-$0.43                         & 5.9                       & 2.8                       & 12.4                         & 2.66E+21                                & 1                        \\
                    2806                    & 11948$+$0453$-$0022          & 119.4833                & 4.5333                  & 4,  5,  6                                & \nodata                           & 0.70                          & $-$2.18                             & $-$2.57                             & $-$0.45                         & 4.1                       & 1.5                       & 7.8                          & 6.23E+20                                & 1                        \\
                    2807                    & 11963$+$0318$-$0464          & 119.6333                & 3.1833                  & 6                                        & \nodata                           & 2.56                          & $-$46.35                            & $-$46.98                            & $-$0.55                         & 10.2                      & 4.8                       & 20.6                         & 3.61E+21                                & 1                        \\
                    2808                    & 11977$+$0307$-$0196          & 119.7750                & 3.0750                  & 6                                        & \nodata                           & 0.93                          & $-$19.58                            & $-$20.46                            & $-$0.55                         & 5.6                       & 1.5                       & 12.4                         & 1.12E+21                                & 1                        \\
                    2809                    & 11980$+$0282$-$0213          & 119.8000                & 2.8250                  & 6                                        & \nodata                           & 0.93                          & $-$21.32                            & $-$21.73                            & $-$0.52                         & 6.7                       & 1.3                       & 14.3                         & 4.83E+20                                & 1                        \\
                    2810                    & 11985$+$0309$-$0199          & 119.8500                & 3.0917                  & 6                                        & \nodata                           & 0.93                          & $-$19.87                            & $-$20.64                            & $-$0.85                         & 5.5                       & 1.3                       & 10.6                         & 5.16E+20                                & 1                        \\
                    2811                    & 11989$+$0380$-$0014          & 119.8917                & 3.8000                  & 6                                        & \nodata                           & 0.71                          & $-$1.38                             & $-$1.64                             & $-$0.34                         & 3.8                       & 1.2                       & 7.7                          & 4.25E+20                                & 1                        \\
                    2812                    & 12014$+$0379$-$0075          & 120.1417                & 3.7917                  & 6                                        & \nodata                           & 0.70                          & $-$7.51                             & $-$7.82                             & $-$0.43                         & 2.6                       & 1.3                       & 7.1                          & 4.48E+20                                & 1                        \\
                    2813                    & 12019$+$0332$-$0688          & 120.1917                & 3.3250                  & 6                                        & \nodata                           & 3.59                          & $-$68.82                            & $-$69.38                            & $-$0.54                         & 4.9                       & 1.7                       & 9.8                          & 8.11E+20                                & 1                        \\
                    2814                    & 12040$+$0302$-$0184          & 120.4000                & 3.0250                  & 6                                        & \nodata                           & 0.93                          & $-$18.41                            & $-$19.71                            & $-$0.80                         & 8.1                       & 3.9                       & 14.3                         & 3.56E+21                                & 1                        \\
                    2815                    & 12048$+$0274$-$0174          & 120.4833                & 2.7417                  & 6                                        & \nodata                           & 0.94                          & $-$17.35                            & $-$18.98                            & $-$0.80                         & 8.9                       & 4.6                       & 16.7                         & 5.61E+21                                & 1                        \\
                    2816                    & 12052$+$0323$-$0060          & 120.5167                & 3.2333                  & 6                                        & \nodata                           & 0.70                          & $-$5.99                             & $-$6.67                             & $-$0.82                         & 4.5                       & 1.4                       & 9.7                          & 5.46E+20                                & 1                        \\
                    2817                    & 12052$-$0184$-$0182          & 120.5250                & $-$1.8417                 & 6                                        & \nodata                           & 0.93                          & $-$18.18                            & $-$18.72                            & $-$0.50                         & 7.6                       & 1.2                       & 12.3                         & 5.85E+20                                & 1                        \\
                    2818                    & 12056$+$0324$-$0058          & 120.5583                & 3.2417                  & 6                                        & \nodata                           & 0.70                          & $-$5.80                             & $-$6.24                             & $-$0.52                         & 4.9                       & 2.0                       & 9.2                          & 8.37E+20                                & 1                        \\
                    2819                    & 12059$+$0332$-$0058          & 120.5917                & 3.3250                  & 6                                        & \nodata                           & 0.70                          & $-$5.80                             & $-$6.26                             & $-$0.52                         & 4.7                       & 2.1                       & 8.4                          & 9.52E+20                                & 1                        \\
                    2820                    & 12060$+$0295$-$0180          & 120.6000                & 2.9500                  & 6                                        & \nodata                           & 0.94                          & $-$18.04                            & $-$18.44                            & $-$0.32                         & 7.4                       & 2.5                       & 13.0                         & 1.63E+21                                & 1                        \\
                    2821                    & 12061$+$0327$-$0060          & 120.6083                & 3.2750                  & 6                                        & \nodata                           & 0.70                          & $-$5.99                             & $-$6.33                             & $-$0.42                         & 5.7                       & 2.0                       & 10.5                         & 8.07E+20                                & 1                        \\
                    2822                    & 12071$+$0410$-$0100          & 120.7083                & 4.1000                  & 6                                        & \nodata                           & 0.70                          & $-$10.01                            & $-$11.07                            & $-$0.82                         & 4.0                       & 1.5                       & 10.2                         & 9.12E+20                                & 1                        \\
                    2823                    & 12081$+$0316$-$0054          & 120.8083                & 3.1583                  & 6                                        & \nodata                           & 0.70                          & $-$5.41                             & $-$5.78                             & $-$0.42                         & 4.3                       & 2.5                       & 8.1                          & 1.25E+21                                & 1                        \\
                    2824                    & 12081$+$0492$-$0096          & 120.8083                & 4.9250                  & 6                                        & \nodata                           & 0.70                          & $-$9.59                             & $-$10.18                            & $-$0.46                         & 4.0                       & 1.4                       & 8.7                          & 8.32E+20                                & 1                        \\
                    2825                    & 12090$+$0297$-$0054          & 120.9000                & 2.9667                  & 6                                        & \nodata                           & 0.70                          & $-$5.40                             & $-$5.94                             & $-$0.59                         & 4.3                       & 1.8                       & 9.0                          & 7.71E+20                                & 1                        \\
                    2826                    & 12091$+$0312$-$0058          & 120.9083                & 3.1250                  & 6                                        & \nodata                           & 0.70                          & $-$5.79                             & $-$6.12                             & $-$0.36                         & 4.8                       & 1.6                       & 10.1                         & 6.74E+20                                & 1                        \\
                    2827                    & 12092$+$0295$-$0054          & 120.9250                & 2.9500                  & 6                                        & \nodata                           & 0.70                          & $-$5.42                             & $-$6.08                             & $-$0.59                         & 4.3                       & 3.8                       & 11.0                         & 2.43E+21                                & 1                        \\
                    2828                    & 12094$+$0301$-$0053          & 120.9417                & 3.0083                  & 6                                        & \nodata                           & 0.70                          & $-$5.29                             & $-$5.73                             & $-$0.49                         & 3.7                       & 3.3                       & 9.0                          & 1.74E+21                                & 1                        \\
                    2829                    & 12095$+$0298$-$0053          & 120.9500                & 2.9833                  & 6                                        & \nodata                           & 0.70                          & $-$5.28                             & $-$5.89                             & $-$0.45                         & 3.8                       & 3.1                       & 14.0                         & 2.26E+21                                & 1                        \\
                    2830                    & 12096$+$0308$-$0058          & 120.9583                & 3.0833                  & 6                                        & \nodata                           & 0.70                          & $-$5.78                             & $-$6.19                             & $-$0.39                         & 4.3                       & 2.9                       & 9.7                          & 1.64E+21                                & 1                        \\
                    2831                    & 12097$+$0301$-$0055          & 120.9667                & 3.0083                  & 6                                        & \nodata                           & 0.70                          & $-$5.47                             & $-$6.05                             & $-$0.56                         & 4.5                       & 3.5                       & 14.7                         & 2.04E+21                                & 1                        \\
                    2832                    & 12097$+$0322$-$0064          & 120.9667                & 3.2167                  & 6                                        & \nodata                           & 0.70                          & $-$6.38                             & $-$6.73                             & $-$0.37                         & 3.5                       & 1.0                       & 7.0                          & 4.18E+20                                & 1                        \\
                    2833                    & 12097$+$0292$-$0052          & 120.9750                & 2.9250                  & 6                                        & \nodata                           & 0.70                          & $-$5.24                             & $-$6.14                             & $-$0.92                         & 4.4                       & 2.0                       & 9.1                          & 9.79E+20                                & 1                        \\
                    2834                    & 12097$+$0297$-$0053          & 120.9750                & 2.9750                  & 6                                        & \nodata                           & 0.70                          & $-$5.27                             & $-$5.69                             & $-$0.33                         & 4.3                       & 2.9                       & 10.6                         & 1.95E+21                                & 1                        \\
                    2835                    & 12100$+$0291$-$0054          & 121.0000                & 2.9083                  & 6                                        & \nodata                           & 0.70                          & $-$5.43                             & $-$5.93                             & $-$0.61                         & 3.3                       & 1.1                       & 8.5                          & 4.06E+20                                & 1                        \\
                    2836                    & 12102$+$0297$-$0052          & 121.0167                & 2.9750                  & 6                                        & \nodata                           & 0.70                          & $-$5.20                             & $-$5.73                             & $-$0.48                         & 4.0                       & 2.9                       & 7.8                          & 1.93E+21                                & 1                        \\
                    2837                    & 12106$+$0303$-$0058          & 121.0583                & 3.0333                  & 6                                        & \nodata                           & 0.70                          & $-$5.76                             & $-$6.23                             & $-$0.53                         & 3.8                       & 2.4                       & 8.6                          & 1.13E+21                                & 1                        \\
                    2838                    & 12107$+$0317$-$0062          & 121.0667                & 3.1750                  & 6                                        & \nodata                           & 0.70                          & $-$6.18                             & $-$7.12                             & $-$0.65                         & 3.9                       & 1.4                       & 8.7                          & 9.53E+20                                & 1                        \\
                    2839                    & 12122$+$0347$-$0052          & 121.2250                & 3.4750                  & 6                                        & \nodata                           & 0.70                          & $-$5.23                             & $-$6.47                             & $-$1.21                         & 4.6                       & 1.4                       & 8.1                          & 6.93E+20                                & 1                        \\
                    2840                    & 12131$+$0064$-$0172          & 121.3083                & 0.6417                  & 6                                        & 12,  17                           & 0.93                          & $-$17.17                            & $-$18.19                            & $-$0.50                         & 13.1                      & 8.2                       & 24.5                         & 1.36E+22                                & 1                        \\
                    2841                    & 12132$+$0344$-$0051          & 121.3250                & 3.4417                  & 5,  6                                    & \nodata                           & 0.70                          & $-$5.13                             & $-$6.86                             & $-$1.11                         & 3.4                       & 2.7                       & 7.5                          & 2.62E+21                                & 1                        \\
                    2842                    & 12133$+$0068$-$0171          & 121.3333                & 0.6833                  & 6                                        & \nodata                           & 0.93                          & $-$17.14                            & $-$17.77                            & $-$0.37                         & 10.5                      & 4.4                       & 17.9                         & 4.62E+21                                & 1                        \\
                    2843                    & 12134$+$0342$-$0055          & 121.3417                & 3.4167                  & 5,  6                                    & 17                                & 0.70                          & $-$5.51                             & $-$6.10                             & $-$0.39                         & 3.8                       & 3.8                       & 8.6                          & 3.92E+21                                & 1                        \\
                    2844                    & 12135$+$0350$-$0054          & 121.3500                & 3.5000                  & 5,  6                                    & \nodata                           & 0.70                          & $-$5.39                             & $-$6.38                             & $-$0.76                         & 3.6                       & 1.7                       & 8.3                          & 1.03E+21                                & 1                        \\
                    2845                    & 12136$+$0344$-$0053          & 121.3583                & 3.4417                  & 5,  6                                    & \nodata                           & 0.70                          & $-$5.31                             & $-$6.43                             & $-$1.02                         & 2.7                       & 3.5                       & 9.1                          & 2.34E+21                                & 1                        \\
                    2846                    & 12137$+$0340$-$0054          & 121.3667                & 3.4000                  & 5,  6                                    & 17                                & 0.70                          & $-$5.43                             & $-$5.71                             & $-$0.40                         & 2.8                       & 1.4                       & 6.9                          & 3.68E+21                                & 2                        \\
                    2847                    & 12139$+$0330$-$0053          & 121.3917                & 3.3000                  & 5,  6                                    & \nodata                           & 0.70                          & $-$5.31                             & $-$6.11                             & $-$0.66                         & 4.4                       & 1.8                       & 9.1                          & 9.98E+20                                & 1                        \\
                    2848                    & 12141$+$0351$-$0064          & 121.4083                & 3.5083                  & 5,  6                                    & \nodata                           & 0.70                          & $-$6.40                             & $-$6.76                             & $-$0.31                         & 3.9                       & 2.3                       & 9.9                          & 1.32E+21                                & 1                        \\
                    2849                    & 12142$+$0342$-$0055          & 121.4167                & 3.4167                  & 5,  6                                    & \nodata                           & 0.70                          & $-$5.51                             & $-$6.58                             & $-$0.77                         & 4.1                       & 2.7                       & 10.8                         & 1.88E+21                                & 1                        \\
                    2850                    & 12145$+$0332$-$0060          & 121.4500                & 3.3167                  & 5,  6                                    & \nodata                           & 0.70                          & $-$5.97                             & $-$6.81                             & $-$0.63                         & 3.7                       & 2.7                       & 7.9                          & 2.16E+21                                & 1                        \\
                    2851                    & 12145$+$0449$-$0093          & 121.4500                & 4.4917                  & 6                                        & \nodata                           & 0.70                          & $-$9.28                             & $-$9.55                             & $-$0.39                         & 4.3                       & 1.2                       & 9.8                          & 3.82E+20                                & 1                        \\
                    2852                    & 12147$+$0330$-$0058          & 121.4750                & 3.3000                  & 5,  6                                    & \nodata                           & 0.70                          & $-$5.83                             & $-$6.46                             & $-$0.42                         & 4.0                       & 1.8                       & 8.3                          & 1.33E+21                                & 1                        \\
                    2853                    & 12152$+$0339$-$0060          & 121.5167                & 3.3917                  & 5,  6                                    & \nodata                           & 0.70                          & $-$5.98                             & $-$6.86                             & $-$0.83                         & 4.1                       & 1.3                       & 8.6                          & 6.34E+20                                & 1                        \\
                    2854                    & 12152$+$0342$-$0064          & 121.5250                & 3.4250                  & 5,  6                                    & \nodata                           & 0.70                          & $-$6.43                             & $-$7.16                             & $-$0.78                         & 4.3                       & 1.0                       & 10.6                         & 4.24E+20                                & 1                        \\
                    2855                    & 12154$+$0337$-$0063          & 121.5417                & 3.3667                  & 5,  6                                    & \nodata                           & 0.70                          & $-$6.26                             & $-$6.77                             & $-$0.39                         & 4.3                       & 1.3                       & 8.1                          & 7.62E+20                                & 1                        \\
                    2856                    & 12159$+$0332$-$0063          & 121.5917                & 3.3250                  & 5,  6                                    & \nodata                           & 0.70                          & $-$6.28                             & $-$6.92                             & $-$0.60                         & 4.4                       & 1.4                       & 12.6                         & 7.09E+20                                & 1                        \\
                    2857                    & 12160$+$0123$-$0054          & 121.6000                & 1.2333                  & 6                                        & \nodata                           & 0.70                          & $-$5.36                             & $-$6.35                             & $-$1.11                         & 3.5                       & 1.4                       & 8.1                          & 6.02E+20                                & 1                        \\
                    2858                    & 12162$+$0120$-$0051          & 121.6250                & 1.2000                  & 6                                        & \nodata                           & 0.70                          & $-$5.10                             & $-$6.42                             & $-$1.28                         & 4.5                       & 1.6                       & 12.4                         & 7.70E+20                                & 1                        \\
                    2859                    & 12174$-$0277$-$0217          & 121.7417                & $-$2.7750                 & 6                                        & \nodata                           & 0.93                          & $-$21.72                            & $-$22.26                            & $-$0.60                         & 4.7                       & 1.5                       & 12.4                         & 6.45E+20                                & 1                        \\
                    2860                    & 12175$+$0343$-$0069          & 121.7500                & 3.4333                  & 5,  6                                    & \nodata                           & 0.70                          & $-$6.88                             & $-$7.51                             & $-$0.56                         & 4.3                       & 1.1                       & 10.8                         & 5.55E+20                                & 1                        \\
                    2861                    & 12179$+$0069$-$0184          & 121.7917                & 0.6917                  & 6                                        & \nodata                           & 0.93                          & $-$18.39                            & $-$18.81                            & $-$0.42                         & 5.1                       & 1.5                       & 11.4                         & 7.04E+20                                & 1                        \\
                    2862                    & 12186$+$0384$-$0098          & 121.8583                & 3.8417                  & 6                                        & \nodata                           & 0.70                          & $-$9.84                             & $-$10.48                            & $-$0.86                         & 4.0                       & 1.6                       & 8.5                          & 5.55E+20                                & 1                        \\
                    2863                    & 12188$+$0301$-$0040          & 121.8833                & 3.0083                  & 6                                        & \nodata                           & 0.70                          & $-$4.00                             & $-$4.22                             & $-$0.19                         & 6.2                       & 1.2                       & 10.0                         & 6.29E+20                                & 1                        \\
                    2864                    & 12188$+$0336$-$0068          & 121.8833                & 3.3583                  & 5,  6                                    & \nodata                           & 0.70                          & $-$6.83                             & $-$7.24                             & $-$0.39                         & 4.5                       & 1.1                       & 9.5                          & 5.30E+20                                & 1                        \\
                    2865                    & 12190$+$0332$-$0066          & 121.9000                & 3.3250                  & 6                                        & \nodata                           & 0.70                          & $-$6.58                             & $-$6.97                             & $-$0.42                         & 4.3                       & 1.2                       & 7.7                          & 4.97E+20                                & 1                        \\
                    2866                    & 12194$-$0075$+$0054          & 121.9417                & $-$0.7500                 & 6                                        & \nodata                           & 0.23                          & 5.35                              & 4.93                              & $-$0.52                         & 5.9                       & 2.3                       & 11.8                         & 8.97E+20                                & 1                        \\
                    2867                    & 12195$-$0055$-$0121          & 121.9500                & $-$0.5500                 & 6                                        & \nodata                           & 0.70                          & $-$12.11                            & $-$12.54                            & $-$0.48                         & 6.1                       & 4.0                       & 10.9                         & 2.11E+21                                & 1                        \\
                    2868                    & 12197$-$0159$-$0148          & 121.9750                & $-$1.5917                 & 6                                        & \nodata                           & 0.93                          & $-$14.77                            & $-$15.34                            & $-$0.40                         & 4.7                       & 2.7                       & 12.1                         & 1.91E+21                                & 1                        \\
                    2869                    & 12199$-$0148$-$0142          & 121.9917                & $-$1.4833                 & 6                                        & \nodata                           & 0.93                          & $-$14.21                            & $-$14.78                            & $-$0.46                         & 6.8                       & 4.0                       & 13.5                         & 2.84E+21                                & 1                        \\
                    2870                    & 12200$-$0156$-$0145          & 122.0000                & $-$1.5583                 & 6                                        & \nodata                           & 0.93                          & $-$14.55                            & $-$14.95                            & $-$0.35                         & 3.4                       & 2.5                       & 8.8                          & 1.55E+21                                & 1                        \\
                    2871                    & 12202$-$0107$-$0133          & 122.0250                & $-$1.0750                 & 2,  6                                    & \nodata                           & 0.69                          & $-$13.30                            & $-$13.58                            & $-$0.34                         & 6.1                       & 3.2                       & 12.9                         & 1.37E+21                                & 1                        \\
                    2872                    & 12207$-$0082$+$0046          & 122.0750                & $-$0.8250                 & 6                                        & \nodata                           & 0.02                          & 4.64                              & 4.12                              & $-$0.42                         & 7.7                       & 2.8                       & 19.0                         & 2.06E+21                                & 1                        \\
                    2873                    & 12217$-$0083$-$0423          & 122.1750                & $-$0.8333                 & 6                                        & \nodata                           & 2.52                          & $-$42.27                            & $-$42.97                            & $-$0.58                         & 6.6                       & 2.8                       & 11.5                         & 1.74E+21                                & 1                        \\
                    2874                    & 12219$-$0107$-$0133          & 122.1917                & $-$1.0750                 & 1,  6                                    & \nodata                           & 0.93                          & $-$13.27                            & $-$13.73                            & $-$0.45                         & 6.8                       & 1.6                       & 10.4                         & 7.68E+20                                & 1                        \\
                    2875                    & 12257$+$0416$-$0046          & 122.5750                & 4.1583                  & 6                                        & \nodata                           & 0.27                          & $-$4.55                             & $-$4.97                             & $-$0.41                         & 4.8                       & 2.0                       & 10.3                         & 9.82E+20                                & 1                        \\
                    2876                    & 12264$-$0117$-$0155          & 122.6417                & $-$1.1750                 & 5,  6                                    & \nodata                           & 0.93                          & $-$15.54                            & $-$16.32                            & $-$0.71                         & 5.6                       & 1.5                       & 13.4                         & 7.88E+20                                & 1                        \\
                    2877                    & 12277$+$0430$-$0057          & 122.7667                & 4.3000                  & 6                                        & \nodata                           & 0.55                          & $-$5.65                             & $-$6.47                             & $-$0.84                         & 4.0                       & 1.3                       & 8.6                          & 5.87E+20                                & 1                        \\
                    2878                    & 12277$+$0240$-$0366          & 122.7750                & 2.4000                  & 6                                        & \nodata                           & 0.93                          & $-$36.62                            & $-$36.98                            & $-$0.42                         & 9.6                       & 2.1                       & 16.3                         & 9.49E+20                                & 1                        \\
                    2879                    & 12286$-$0233$-$0164          & 122.8583                & $-$2.3333                 & 6                                        & \nodata                           & 0.93                          & $-$16.44                            & $-$17.07                            & $-$0.80                         & 5.2                       & 1.3                       & 11.9                         & 4.88E+20                                & 1                        \\
                    2880                    & 12297$-$0107$-$0191          & 122.9667                & $-$1.0667                 & 3,  5,  6                                & \nodata                           & 0.93                          & $-$19.15                            & $-$19.60                            & $-$0.36                         & 5.1                       & 1.7                       & 11.0                         & 9.79E+20                                & 1                        \\
                    2881                    & 12305$-$0091$-$0179          & 123.0500                & $-$0.9083                 & 5,  6                                    & \nodata                           & 0.93                          & $-$17.94                            & $-$18.91                            & $-$0.56                         & 4.9                       & 1.7                       & 11.9                         & 1.41E+21                                & 1                        \\
                    2882                    & 12317$-$0125$-$0189          & 123.1667                & $-$1.2500                 & 5,  6                                    & \nodata                           & 0.93                          & $-$18.89                            & $-$19.37                            & $-$0.55                         & 5.0                       & 1.1                       & 10.6                         & 4.40E+20                                & 1                        \\
                    2883                    & 12351$+$0295$-$0104          & 123.5083                & 2.9500                  & 6                                        & \nodata                           & 0.74                          & $-$10.40                            & $-$11.21                            & $-$0.39                         & 5.1                       & 2.4                       & 10.7                         & 2.54E+21                                & 1                        \\
                    2884                    & 12355$+$0511$-$0065          & 123.5500                & 5.1083                  & 5,  6                                    & \nodata                           & 0.52                          & $-$6.53                             & $-$7.26                             & $-$0.57                         & 4.5                       & 1.1                       & 9.9                          & 6.19E+20                                & 1                        \\
                    2885                    & 12377$+$0339$-$0096          & 123.7667                & 3.3917                  & 6                                        & \nodata                           & 0.72                          & $-$9.61                             & $-$10.47                            & $-$0.77                         & 3.9                       & 1.7                       & 8.1                          & 9.01E+20                                & 1                        \\
                    2886                    & 12391$-$0092$+$0045          & 123.9083                & $-$0.9250                 & 4,  5,  6                                & \nodata                           & 0.23                          & 4.54                              & 3.92                              & $-$0.52                         & 5.9                       & 2.7                       & 10.7                         & 1.68E+21                                & 1                        \\
                    2887                    & 12407$+$0320$-$0115          & 124.0750                & 3.2000                  & 5,  6                                    & \nodata                           & 0.70                          & $-$11.55                            & $-$12.12                            & $-$0.54                         & 6.1                       & 2.9                       & 16.1                         & 1.68E+21                                & 1                        \\
                    2888                    & 12414$+$0297$-$0107          & 124.1417                & 2.9667                  & 5,  6                                    & \nodata                           & 0.72                          & $-$10.74                            & $-$11.49                            & $-$0.77                         & 6.0                       & 3.0                       & 11.1                         & 1.55E+21                                & 1                        \\
                    2889                    & 12418$+$0315$-$0116          & 124.1833                & 3.1500                  & 5,  6                                    & \nodata                           & 0.70                          & $-$11.55                            & $-$12.03                            & $-$0.45                         & 6.1                       & 3.0                       & 16.9                         & 1.84E+21                                & 1                        \\
                    2890                    & 12420$+$0309$-$0115          & 124.2000                & 3.0917                  & 5,  6                                    & \nodata                           & 0.70                          & $-$11.50                            & $-$12.15                            & $-$0.54                         & 6.7                       & 1.9                       & 19.3                         & 1.29E+21                                & 1                        \\
                    2891                    & 12420$+$0312$-$0115          & 124.2000                & 3.1250                  & 5,  6                                    & \nodata                           & 0.70                          & $-$11.47                            & $-$11.80                            & $-$0.28                         & 6.5                       & 2.6                       & 12.4                         & 1.57E+21                                & 1                        \\
                    2892                    & 12427$+$0331$-$0066          & 124.2667                & 3.3083                  & 6                                        & \nodata                           & 0.57                          & $-$6.58                             & $-$7.27                             & $-$0.57                         & 5.0                       & 1.0                       & 11.5                         & 5.48E+20                                & 1                        \\
                    2893                    & 12428$+$0265$-$0101          & 124.2833                & 2.6500                  & 5,  6                                    & \nodata                           & 0.83                          & $-$10.10                            & $-$10.85                            & $-$0.84                         & 4.3                       & 2.0                       & 9.1                          & 8.55E+20                                & 1                        \\
                    2894                    & 12428$+$0267$-$0104          & 124.2833                & 2.6750                  & 5,  6                                    & \nodata                           & 0.82                          & $-$10.38                            & $-$10.73                            & $-$0.36                         & 4.9                       & 2.0                       & 11.9                         & 9.03E+20                                & 1                        \\
                    2895                    & 12438$-$0065$+$0040          & 124.3833                & $-$0.6500                 & 5,  6                                    & \nodata                           & 0.05                          & 3.95                              & 3.57                              & $-$0.36                         & 6.5                       & 1.8                       & 13.2                         & 9.04E+20                                & 1                        \\
                    2896                    & 12444$-$0063$+$0033          & 124.4417                & $-$0.6333                 & 5,  6                                    & \nodata                           & 0.23                          & 3.33                              & 2.83                              & $-$0.41                         & 5.5                       & 1.2                       & 12.2                         & 6.66E+20                                & 1                        \\
                    2897                    & 12450$+$0210$-$0079          & 124.5000                & 2.1000                  & 5,  6                                    & \nodata                           & 0.59                          & $-$7.87                             & $-$8.85                             & $-$0.72                         & 6.7                       & 2.2                       & 11.0                         & 1.47E+21                                & 1                        \\
                    2898                    & 12450$+$0249$-$0113          & 124.5000                & 2.4917                  & 1,  3,  5,  6                            & \nodata                           & 0.67                          & $-$11.30                            & $-$11.72                            & $-$0.47                         & 5.0                       & 3.3                       & 9.9                          & 1.63E+21                                & 1                        \\
                    2899                    & 12454$+$0252$-$0113          & 124.5417                & 2.5250                  & 1,  3,  5,  6                            & \nodata                           & 0.68                          & $-$11.27                            & $-$12.29                            & $-$0.90                         & 5.9                       & 3.3                       & 14.0                         & 2.03E+21                                & 1                        \\
                    2900                    & 12457$+$0242$-$0112          & 124.5667                & 2.4250                  & 3,  5,  6                                & \nodata                           & 0.68                          & $-$11.16                            & $-$11.98                            & $-$0.57                         & 5.2                       & 3.5                       & 10.7                         & 2.88E+21                                & 1                        \\
                    2901                    & 12460$-$0064$+$0035          & 124.6000                & $-$0.6417                 & 5,  6                                    & \nodata                           & 0.23                          & 3.49                              & 2.77                              & $-$0.86                         & 5.4                       & 2.0                       & 10.9                         & 8.19E+20                                & 1                        \\
                    2902                    & 12481$+$0242$+$0004          & 124.8083                & 2.4167                  & 5,  6                                    & \nodata                           & 0.24                          & 0.39                              & $-$0.09                             & $-$0.37                         & 5.4                       & 1.7                       & 13.2                         & 1.03E+21                                & 1                        \\
                    2903                    & 12481$+$0274$-$0047          & 124.8083                & 2.7417                  & 5,  6                                    & \nodata                           & 0.27                          & $-$4.69                             & $-$5.05                             & $-$0.44                         & 4.0                       & 1.3                       & 9.1                          & 4.96E+20                                & 1                        \\
                    2904                    & 12484$+$0244$+$0003          & 124.8417                & 2.4417                  & 5,  6                                    & \nodata                           & 0.24                          & 0.34                              & $-$0.18                             & $-$0.42                         & 6.2                       & 1.4                       & 13.1                         & 8.08E+20                                & 1                        \\
                    2905                    & 12484$-$0043$+$0041          & 124.8417                & $-$0.4333                 & 5,  6                                    & \nodata                           & 0.05                          & 4.09                              & 3.63                              & $-$0.31                         & 8.2                       & 3.3                       & 14.2                         & 2.64E+21                                & 1                        \\
                    2906                    & 12486$+$0252$+$0009          & 124.8583                & 2.5250                  & 5,  6                                    & \nodata                           & 0.24                          & 0.91                              & 0.14                              & $-$0.66                         & 4.9                       & 1.2                       & 12.5                         & 6.62E+20                                & 1                        \\
                    2907                    & 12487$-$0046$+$0041          & 124.8750                & $-$0.4583                 & 5,  6                                    & \nodata                           & 0.05                          & 4.08                              & 3.40                              & $-$0.43                         & 6.2                       & 2.6                       & 14.6                         & 2.19E+21                                & 1                        \\
                    2908                    & 12499$+$0351$-$0064          & 124.9917                & 3.5083                  & 6                                        & \nodata                           & 0.58                          & $-$6.44                             & $-$7.01                             & $-$0.65                         & 5.5                       & 1.5                       & 10.6                         & 5.89E+20                                & 1                        \\
                    2909                    & 12504$-$0253$-$0104          & 125.0417                & $-$2.5333                 & 6                                        & \nodata                           & 0.63                          & $-$10.42                            & $-$10.92                            & $-$0.60                         & 5.9                       & 1.2                       & 12.3                         & 4.75E+20                                & 1                        \\
                    2910                    & 12509$-$0051$+$0034          & 125.0917                & $-$0.5083                 & 5,  6                                    & \nodata                           & 0.24                          & 3.42                              & 2.73                              & $-$0.43                         & 5.7                       & 2.5                       & 10.3                         & 2.07E+21                                & 1                        \\
                    2911                    & 12528$+$0293$+$0000          & 125.2833                & 2.9333                  & 6                                        & \nodata                           & 0.25                          & $-$0.01                             & $-$0.60                             & $-$0.49                         & 4.7                       & 1.4                       & 9.7                          & 7.53E+20                                & 1                        \\
                    2912                    & 12558$+$0343$-$0087          & 125.5833                & 3.4333                  & 6                                        & \nodata                           & 0.65                          & $-$8.74                             & $-$9.59                             & $-$0.73                         & 3.6                       & 1.9                       & 8.2                          & 1.09E+21                                & 1                        \\
                    2913                    & 12581$+$0113$-$0087          & 125.8083                & 1.1333                  & 6                                        & \nodata                           & 0.78                          & $-$8.66                             & $-$9.12                             & $-$0.47                         & 5.2                       & 2.6                       & 14.2                         & 1.30E+21                                & 1                        \\
                    2914                    & 12582$+$0367$-$0068          & 125.8167                & 3.6750                  & 6                                        & \nodata                           & 0.58                          & $-$6.78                             & $-$7.24                             & $-$0.49                         & 3.4                       & 1.9                       & 7.4                          & 9.45E+20                                & 1                        \\
                    2915                    & 12582$+$0378$-$0071          & 125.8167                & 3.7833                  & 6                                        & \nodata                           & 0.58                          & $-$7.06                             & $-$7.78                             & $-$0.73                         & 4.2                       & 1.3                       & 7.8                          & 6.01E+20                                & 1                        \\
                    2916                    & 12585$+$0383$-$0065          & 125.8500                & 3.8333                  & 6                                        & \nodata                           & 0.57                          & $-$6.47                             & $-$7.25                             & $-$0.72                         & 3.6                       & 1.9                       & 13.0                         & 1.02E+21                                & 1                        \\
                    2917                    & 12590$-$0108$-$0109          & 125.9000                & $-$1.0833                 & 5,  6                                    & \nodata                           & 0.93                          & $-$10.92                            & $-$11.87                            & $-$0.81                         & 7.7                       & 3.2                       & 13.3                         & 2.01E+21                                & 1                        \\
                    2918                    & 12591$-$0097$-$0111          & 125.9083                & $-$0.9667                 & 5,  6                                    & \nodata                           & 0.93                          & $-$11.06                            & $-$11.40                            & $-$0.28                         & 6.8                       & 3.0                       & 13.0                         & 1.93E+21                                & 1                        \\
                    2919                    & 12592$-$0112$-$0113          & 125.9167                & $-$1.1167                 & 5,  6                                    & \nodata                           & 0.93                          & $-$11.26                            & $-$11.81                            & $-$0.37                         & 7.4                       & 2.3                       & 12.7                         & 1.73E+21                                & 1                        \\
                    2920                    & 12594$-$0108$-$0109          & 125.9417                & $-$1.0833                 & 5,  6                                    & \nodata                           & 0.93                          & $-$10.91                            & $-$11.90                            & $-$0.88                         & 7.0                       & 3.8                       & 11.7                         & 2.39E+21                                & 1                        \\
                    2921                    & 12598$-$0089$-$0117          & 125.9833                & $-$0.8917                 & 5,  6                                    & \nodata                           & 0.93                          & $-$11.75                            & $-$12.36                            & $-$0.52                         & 7.1                       & 2.1                       & 14.5                         & 1.29E+21                                & 1                        \\
                    2922                    & 12619$+$0507$-$0125          & 126.1917                & 5.0667                  & 6                                        & \nodata                           & 0.83                          & $-$12.52                            & $-$13.00                            & $-$0.50                         & 7.0                       & 1.9                       & 16.5                         & 9.53E+20                                & 1                        \\
                    2923                    & 12632$-$0049$-$0108          & 126.3167                & $-$0.4917                 & 6                                        & \nodata                           & 0.93                          & $-$10.80                            & $-$11.43                            & $-$0.56                         & 6.9                       & 1.4                       & 11.3                         & 7.47E+20                                & 1                        \\
                    2924                    & 12637$+$0522$-$0124          & 126.3667                & 5.2250                  & 6                                        & \nodata                           & 0.82                          & $-$12.40                            & $-$12.79                            & $-$0.40                         & 6.9                       & 2.8                       & 12.9                         & 1.38E+21                                & 1                        \\
                    2925                    & 12637$-$0120$-$0125          & 126.3750                & $-$1.2000                 & 6                                        & \nodata                           & 0.93                          & $-$12.54                            & $-$13.48                            & $-$0.51                         & 8.8                       & 3.5                       & 21.9                         & 4.19E+21                                & 1                        \\
                    2926                    & 12640$+$0012$-$0119          & 126.4000                & 0.1250                  & 6                                        & \nodata                           & 0.93                          & $-$11.90                            & $-$12.71                            & $-$0.77                         & 6.6                       & 2.3                       & 10.9                         & 1.16E+21                                & 1                        \\
                    2927                    & 12642$+$0007$-$0120          & 126.4167                & 0.0667                  & 6                                        & \nodata                           & 0.93                          & $-$11.95                            & $-$12.81                            & $-$0.64                         & 6.0                       & 1.1                       & 9.7                          & 6.66E+20                                & 1                        \\
                    2928                    & 12644$+$0040$-$0128          & 126.4417                & 0.4000                  & 5,  6                                    & \nodata                           & 0.93                          & $-$12.77                            & $-$13.29                            & $-$0.36                         & 7.9                       & 1.7                       & 14.9                         & 1.17E+21                                & 1                        \\
                    2929                    & 12647$-$0145$-$0110          & 126.4667                & $-$1.4500                 & 6                                        & \nodata                           & 0.93                          & $-$11.01                            & $-$11.79                            & $-$0.49                         & 6.6                       & 3.3                       & 12.5                         & 2.81E+21                                & 1                        \\
                    2930                    & 12651$-$0130$-$0118          & 126.5083                & $-$1.3000                 & 5,  6                                    & \nodata                           & 0.93                          & $-$11.82                            & $-$12.67                            & $-$0.49                         & 8.2                       & 4.0                       & 19.1                         & 4.36E+21                                & 1                        \\
                    2931                    & 12657$+$0028$-$0124          & 126.5750                & 0.2833                  & 6                                        & \nodata                           & 0.93                          & $-$12.39                            & $-$13.35                            & $-$0.80                         & 6.8                       & 1.7                       & 13.2                         & 9.73E+20                                & 1                        \\
                    2932                    & 12662$-$0062$-$0157          & 126.6167                & $-$0.6167                 & 6                                        & \nodata                           & 0.93                          & $-$15.72                            & $-$16.38                            & $-$0.31                         & 9.8                       & 3.4                       & 14.8                         & 4.02E+21                                & 1                        \\
                    2933                    & 12667$-$0082$-$0138          & 126.6667                & $-$0.8167                 & 3,  4,  5,  6                            & \nodata                           & 0.93                          & $-$13.78                            & $-$14.92                            & $-$0.46                         & 14.8                      & 7.9                       & 21.9                         & 1.49E+22                                & 1                        \\
                    2934                    & 12669$-$0077$-$0130          & 126.6917                & $-$0.7750                 & 3,  4,  6                                & \nodata                           & 0.93                          & $-$13.03                            & $-$14.33                            & $-$0.73                         & 14.6                      & 4.8                       & 23.9                         & 6.01E+21                                & 1                        \\
                    2935                    & 12670$-$0089$-$0131          & 126.7000                & $-$0.8917                 & 1,  3,  6                                & \nodata                           & 0.93                          & $-$13.10                            & $-$13.85                            & $-$0.46                         & 6.1                       & 1.3                       & 9.3                          & 7.18E+21                                & 2                        \\
                    2936                    & 12673$-$0083$-$0143          & 126.7333                & $-$0.8333                 & 3,  4,  6                                & \nodata                           & 0.93                          & $-$14.31                            & $-$15.93                            & $-$0.59                         & 18.5                      & 9.8                       & 29.7                         & 2.48E+22                                & 1                        \\
                    2937                    & 12677$-$0084$-$0147          & 126.7750                & $-$0.8417                 & 3,  6                                    & \nodata                           & 0.93                          & $-$14.73                            & $-$15.95                            & $-$0.47                         & 17.0                      & 5.1                       & 29.3                         & 1.07E+22                                & 1                        \\
                    2938                    & 12677$-$0087$-$0145          & 126.7750                & $-$0.8667                 & 3,  6                                    & \nodata                           & 0.93                          & $-$14.47                            & $-$15.65                            & $-$0.54                         & 14.9                      & 5.0                       & 21.9                         & 7.35E+21                                & 1                        \\
                    2939                    & 12678$-$0134$-$0111          & 126.7833                & $-$1.3417                 & 5,  6                                    & \nodata                           & 0.93                          & $-$11.14                            & $-$11.78                            & $-$0.70                         & 6.1                       & 1.0                       & 11.1                         & 3.89E+20                                & 1                        \\
                    2940                    & 12680$+$0070$-$0129          & 126.8000                & 0.7000                  & 6                                        & \nodata                           & 0.93                          & $-$12.92                            & $-$13.56                            & $-$0.84                         & 6.5                       & 2.1                       & 13.1                         & 7.99E+20                                & 1                        \\
                    2941                    & 12684$-$0104$-$0125          & 126.8417                & $-$1.0417                 & 6                                        & \nodata                           & 0.93                          & $-$12.49                            & $-$13.19                            & $-$0.54                         & 5.4                       & 2.9                       & 10.6                         & 1.99E+21                                & 1                        \\
                    2942                    & 12692$-$0087$-$0110          & 126.9250                & $-$0.8750                 & 6                                        & \nodata                           & 0.93                          & $-$10.98                            & $-$11.65                            & $-$0.45                         & 7.4                       & 1.5                       & 12.5                         & 1.01E+21                                & 1                        \\
                    2943                    & 12712$-$0204$-$0099          & 127.1167                & $-$2.0417                 & 6                                        & \nodata                           & 0.62                          & $-$9.88                             & $-$10.27                            & $-$0.43                         & 6.1                       & 1.3                       & 12.5                         & 5.39E+20                                & 1                        \\
                    2944                    & 12712$-$0198$-$0098          & 127.1250                & $-$1.9833                 & 6                                        & \nodata                           & 0.62                          & $-$9.80                             & $-$10.50                            & $-$0.41                         & 6.4                       & 1.5                       & 15.4                         & 1.30E+21                                & 1                        \\
                    2945                    & 12718$-$0239$-$0103          & 127.1833                & $-$2.3917                 & 6                                        & \nodata                           & 0.62                          & $-$10.30                            & $-$10.90                            & $-$0.51                         & 6.1                       & 1.4                       & 11.9                         & 7.48E+20                                & 1                        \\
                    2946                    & 12731$-$0087$-$0111          & 127.3083                & $-$0.8667                 & 3,  6                                    & \nodata                           & 0.63                          & $-$11.06                            & $-$11.51                            & $-$0.50                         & 5.8                       & 1.3                       & 11.6                         & 5.47E+20                                & 1                        \\
                    2947                    & 12742$+$0457$-$0022          & 127.4250                & 4.5667                  & 5,  6                                    & \nodata                           & 0.28                          & $-$2.18                             & $-$2.88                             & $-$0.64                         & 4.4                       & 1.3                       & 10.5                         & 6.51E+20                                & 1                        \\
                    2948                    & 12752$+$0459$-$0020          & 127.5167                & 4.5917                  & 5,  6                                    & \nodata                           & 0.28                          & $-$2.03                             & $-$2.50                             & $-$0.59                         & 3.3                       & 1.9                       & 8.3                          & 7.54E+20                                & 1                        \\
                    2949                    & 12755$+$0457$-$0019          & 127.5500                & 4.5667                  & 5,  6                                    & \nodata                           & 0.28                          & $-$1.92                             & $-$2.58                             & $-$0.59                         & 4.4                       & 2.0                       & 9.7                          & 1.10E+21                                & 1                        \\
                    2950                    & 12757$+$0452$-$0019          & 127.5667                & 4.5250                  & 5,  6                                    & \nodata                           & 0.28                          & $-$1.88                             & $-$2.49                             & $-$0.53                         & 4.5                       & 2.3                       & 9.8                          & 1.36E+21                                & 1                        \\
                    2951                    & 12757$+$0460$-$0016          & 127.5750                & 4.6000                  & 5,  6                                    & \nodata                           & 0.27                          & $-$1.62                             & $-$2.31                             & $-$0.61                         & 3.9                       & 1.4                       & 9.7                          & 7.06E+20                                & 1                        \\
                    2952                    & 12760$+$0452$-$0017          & 127.6000                & 4.5250                  & 5,  6                                    & \nodata                           & 0.28                          & $-$1.71                             & $-$2.38                             & $-$0.54                         & 4.4                       & 1.8                       & 8.7                          & 1.05E+21                                & 1                        \\
                    2953                    & 12788$+$0267$-$0115          & 127.8833                & 2.6750                  & 6                                        & \nodata                           & 0.88                          & $-$11.45                            & $-$12.00                            & $-$0.36                         & 9.2                       & 5.9                       & 17.6                         & 5.83E+21                                & 1                        \\
                    2954                    & 12791$+$0264$-$0113          & 127.9083                & 2.6417                  & 6                                        & \nodata                           & 0.88                          & $-$11.29                            & $-$11.86                            & $-$0.39                         & 10.4                      & 5.6                       & 22.4                         & 5.85E+21                                & 1                        \\
                    2955                    & 12820$+$0480$-$0082          & 128.2000                & 4.8000                  & 6                                        & \nodata                           & 0.58                          & $-$8.25                             & $-$9.04                             & $-$0.55                         & 5.2                       & 1.9                       & 10.5                         & 1.32E+21                                & 1                        \\
                    2956                    & 12830$+$0452$-$0059          & 128.3000                & 4.5167                  & 6                                        & \nodata                           & 0.28                          & $-$5.90                             & $-$6.53                             & $-$0.49                         & 3.7                       & 0.8                       & 8.8                          & 4.60E+20                                & 1                        \\
                    2957                    & 12832$+$0454$-$0057          & 128.3167                & 4.5417                  & 6                                        & \nodata                           & 0.28                          & $-$5.66                             & $-$6.48                             & $-$0.44                         & 3.9                       & 1.0                       & 8.3                          & 7.82E+20                                & 1                        \\
                    2958                    & 12841$+$0344$-$0054          & 128.4083                & 3.4417                  & 6                                        & \nodata                           & 0.27                          & $-$5.40                             & $-$6.09                             & $-$0.78                         & 4.0                       & 1.3                       & 8.2                          & 5.16E+20                                & 1                        \\
                    2959                    & 12867$+$0457$-$0051          & 128.6667                & 4.5667                  & 5,  6                                    & \nodata                           & 0.28                          & $-$5.08                             & $-$6.05                             & $-$0.67                         & 4.0                       & 1.1                       & 10.0                         & 7.33E+20                                & 1                        \\
                    2960                    & 12870$-$0040$-$0151          & 128.7000                & $-$0.4000                 & 6                                        & \nodata                           & 0.93                          & $-$15.07                            & $-$15.58                            & $-$0.35                         & 7.7                       & 2.1                       & 14.8                         & 1.55E+21                                & 1                        \\
                    2961                    & 12871$-$0033$-$0153          & 128.7083                & $-$0.3333                 & 6                                        & \nodata                           & 0.93                          & $-$15.28                            & $-$15.92                            & $-$0.48                         & 8.4                       & 2.7                       & 15.5                         & 1.91E+21                                & 1                        \\
                    2962                    & 12896$+$0442$-$0051          & 128.9583                & 4.4250                  & 5,  6                                    & \nodata                           & 0.28                          & $-$5.08                             & $-$6.20                             & $-$0.61                         & 5.1                       & 2.1                       & 12.0                         & 1.87E+21                                & 1                        \\
                    2963                    & 12902$+$0303$-$0084          & 129.0250                & 3.0333                  & 6                                        & \nodata                           & 0.56                          & $-$8.37                             & $-$9.06                             & $-$0.66                         & 4.1                       & 1.6                       & 8.7                          & 7.99E+20                                & 1                        \\
                    2964                    & 12947$-$0239$-$0131          & 129.4750                & $-$2.3917                 & 6                                        & \nodata                           & 0.86                          & $-$13.14                            & $-$13.66                            & $-$0.52                         & 5.9                       & 1.3                       & 12.1                         & 5.94E+20                                & 1                        \\
                    2965                    & 12957$-$0037$-$0149          & 129.5750                & $-$0.3750                 & 6                                        & \nodata                           & 0.93                          & $-$14.94                            & $-$15.71                            & $-$0.61                         & 8.0                       & 2.4                       & 13.1                         & 1.51E+21                                & 1                        \\
                    2966                    & 12959$-$0238$-$0129          & 129.5917                & $-$2.3833                 & 6                                        & \nodata                           & 0.93                          & $-$12.93                            & $-$13.50                            & $-$0.54                         & 5.8                       & 1.2                       & 13.2                         & 6.18E+20                                & 1                        \\
                    2967                    & 12988$+$0206$-$0029          & 129.8833                & 2.0583                  & 6                                        & \nodata                           & 0.27                          & $-$2.95                             & $-$3.55                             & $-$0.54                         & 4.1                       & 1.9                       & 10.2                         & 9.78E+20                                & 1                        \\
                    2968                    & 13073$+$0041$-$0106          & 130.7333                & 0.4083                  & 6                                        & \nodata                           & 0.56                          & $-$10.57                            & $-$11.21                            & $-$0.53                         & 4.4                       & 1.4                       & 9.2                          & 7.93E+20                                & 1                        \\
                    2969                    & 13075$+$0048$-$0108          & 130.7500                & 0.4833                  & 6                                        & \nodata                           & 0.56                          & $-$10.78                            & $-$11.35                            & $-$0.62                         & 4.8                       & 2.5                       & 10.7                         & 1.14E+21                                & 1                        \\
                    2970                    & 13107$+$0411$-$0022          & 131.0667                & 4.1083                  & 5,  6                                    & \nodata                           & 0.28                          & $-$2.19                             & $-$2.81                             & $-$0.53                         & 3.2                       & 1.0                       & 9.2                          & 5.23E+20                                & 1                        \\
                    2971                    & 13115$+$0407$-$0022          & 131.1500                & 4.0667                  & 6                                        & \nodata                           & 0.28                          & $-$2.20                             & $-$3.10                             & $-$1.10                         & 2.9                       & 1.3                       & 8.9                          & 4.86E+20                                & 1                        \\
                    2972                    & 13227$+$0497$-$0069          & 132.2667                & 4.9750                  & 6                                        & \nodata                           & 0.28                          & $-$6.92                             & $-$7.47                             & $-$0.49                         & 4.6                       & 1.2                       & 10.8                         & 6.12E+20                                & 1                        \\
                    2973                    & 13297$+$0074$-$0422          & 132.9667                & 0.7417                  & 6                                        & \nodata                           & 1.96                          & $-$42.20                            & $-$43.09                            & $-$0.40                         & 11.5                      & 6.5                       & 18.6                         & 1.00E+22                                & 1                        \\
                    2974                    & 13323$+$0051$-$0493          & 133.2333                & 0.5083                  & 6                                        & \nodata                           & 1.96                          & $-$49.25                            & $-$49.76                            & $-$0.38                         & 9.2                       & 5.6                       & 16.9                         & 4.75E+21                                & 1                        \\
                    2975                    & 13327$-$0002$-$0146          & 133.2667                & $-$0.0250                 & 6                                        & \nodata                           & 1.66                          & $-$14.59                            & $-$15.13                            & $-$0.42                         & 7.7                       & 2.2                       & 13.7                         & 1.46E+21                                & 1                        \\
                    2976                    & 13332$-$0004$-$0150          & 133.3167                & $-$0.0417                 & 6                                        & \nodata                           & 1.47                          & $-$15.05                            & $-$15.45                            & $-$0.32                         & 7.6                       & 2.9                       & 14.5                         & 1.94E+21                                & 1                        \\
                    2977                    & 13333$+$0024$-$0506          & 133.3333                & 0.2417                  & 6                                        & \nodata                           & 1.96                          & $-$50.63                            & $-$51.32                            & $-$0.81                         & 6.6                       & 1.3                       & 11.8                         & 5.10E+20                                & 1                        \\
                    2978                    & 13334$-$0005$-$0149          & 133.3417                & $-$0.0500                 & 6                                        & \nodata                           & 1.73                          & $-$14.88                            & $-$15.32                            & $-$0.31                         & 8.0                       & 2.7                       & 15.7                         & 2.05E+21                                & 1                        \\
                    2979                    & 13337$+$0002$-$0151          & 133.3667                & 0.0167                  & 6                                        & \nodata                           & 1.39                          & $-$15.05                            & $-$15.34                            & $-$0.21                         & 7.6                       & 3.1                       & 13.7                         & 2.28E+21                                & 1                        \\
                    2980                    & 13337$+$0098$-$0430          & 133.3750                & 0.9833                  & 3,  6                                    & \nodata                           & 1.96                          & $-$43.03                            & $-$44.02                            & $-$0.56                         & 8.1                       & 4.6                       & 16.2                         & 4.89E+21                                & 1                        \\
                    2981                    & 13342$+$0000$-$0152          & 133.4250                & 0.0000                  & 6                                        & 17                                & 1.47                          & $-$15.20                            & $-$15.54                            & $-$0.26                         & 7.1                       & 3.9                       & 14.5                         & 2.87E+21                                & 1                        \\
                    2982                    & 13342$+$0002$-$0151          & 133.4250                & 0.0250                  & 6                                        & \nodata                           & 1.47                          & $-$15.15                            & $-$15.52                            & $-$0.31                         & 7.0                       & 3.1                       & 13.8                         & 2.03E+21                                & 1                        \\
                    2983                    & 13347$-$0015$-$0154          & 133.4750                & $-$0.1500                 & 6                                        & \nodata                           & 1.51                          & $-$15.39                            & $-$16.13                            & $-$0.49                         & 7.7                       & 2.4                       & 15.9                         & 1.94E+21                                & 1                        \\
                    2984                    & 13349$+$0011$-$0491          & 133.4917                & 0.1083                  & 6                                        & \nodata                           & 1.96                          & $-$49.07                            & $-$49.67                            & $-$0.38                         & 9.3                       & 3.7                       & 15.8                         & 3.40E+21                                & 1                        \\
                    2985                    & 13358$+$0010$-$0496          & 133.5833                & 0.1000                  & 6                                        & \nodata                           & 1.96                          & $-$49.64                            & $-$50.11                            & $-$0.46                         & 5.0                       & 1.0                       & 8.9                          & 3.31E+21                                & 2                        \\
                    2986                    & 13372$+$0120$-$0382          & 133.7167                & 1.2000                  & 2,  3,  4,  5,  6                        & \nodata                           & 1.96                          & $-$38.21                            & $-$38.96                            & $-$0.32                         & 21.7                      & 7.6                       & 27.5                         & 1.48E+22                                & 1                        \\
                    2987                    & 13392$+$0047$-$0460          & 133.9167                & 0.4750                  & 4,  5,  6                                & \nodata                           & 1.96                          & $-$45.95                            & $-$46.97                            & $-$0.41                         & 9.3                       & 3.5                       & 18.7                         & 5.14E+21                                & 1                        \\
                    2988                    & 13393$+$0107$-$0462          & 133.9333                & 1.0667                  & 1,  3,  4,  5,  6                        & 15                                & 1.96                          & $-$46.21                            & $-$47.91                            & $-$0.38                         & 21.7                      & 8.7                       & 40.5                         & 4.15E+22                                & 1                        \\
                    2989                    & 13397$-$0037$-$0139          & 133.9750                & $-$0.3750                 & 5,  6                                    & \nodata                           & 1.57                          & $-$13.86                            & $-$14.66                            & $-$0.85                         & 4.9                       & 1.6                       & 9.0                          & 7.23E+20                                & 1                        \\
                    2990                    & 13405$+$0072$-$0479          & 134.0500                & 0.7167                  & 4,  5,  6                                & \nodata                           & 1.96                          & $-$47.93                            & $-$48.89                            & $-$0.40                         & 11.5                      & 6.2                       & 18.9                         & 1.00E+22                                & 1                        \\
                    2991                    & 13420$+$0078$-$0497          & 134.2000                & 0.7833                  & 1,  3,  4,  5,  6                        & \nodata                           & 1.96                          & $-$49.69                            & $-$50.61                            & $-$0.35                         & 21.3                      & 12.6                      & 32.4                         & 3.35E+22                                & 1                        \\
                    2992                    & 13491$-$0021$-$0125          & 134.9083                & $-$0.2083                 & 4,  5,  6                                & \nodata                           & 0.56                          & $-$12.54                            & $-$13.07                            & $-$0.41                         & 6.9                       & 2.0                       & 12.0                         & 1.24E+21                                & 1                        \\
                    2993                    & 13493$-$0117$-$0121          & 134.9333                & $-$1.1667                 & 6                                        & \nodata                           & 0.54                          & $-$12.14                            & $-$12.65                            & $-$0.52                         & 3.5                       & 1.9                       & 8.4                          & 9.05E+20                                & 1                        \\
                    2994                    & 13510$-$0032$-$0114          & 135.1000                & $-$0.3167                 & 5,  6                                    & \nodata                           & 0.54                          & $-$11.41                            & $-$12.11                            & $-$0.38                         & 5.5                       & 2.1                       & 10.6                         & 1.86E+21                                & 1                        \\
                    2995                    & 13564$-$0037$-$0089          & 135.6417                & $-$0.3750                 & 5,  6                                    & \nodata                           & 0.27                          & $-$8.88                             & $-$9.29                             & $-$0.34                         & 6.1                       & 2.6                       & 11.5                         & 1.59E+21                                & 1                        \\
                    2996                    & 13668$-$0153$-$0109          & 136.6833                & $-$1.5333                 & 6                                        & \nodata                           & 0.56                          & $-$10.94                            & $-$11.81                            & $-$0.57                         & 6.8                       & 1.6                       & 12.2                         & 1.16E+21                                & 1                        \\
                    2997                    & 13669$+$0116$-$0357          & 136.6917                & 1.1583                  & 4,  5,  6                                & \nodata                           & 1.96                          & $-$35.66                            & $-$36.02                            & $-$0.31                         & 8.9                       & 3.1                       & 15.5                         & 2.04E+21                                & 1                        \\
                    2998                    & 13759$+$0450$-$0104          & 137.5917                & 4.5000                  & 6                                        & \nodata                           & 0.62                          & $-$10.37                            & $-$10.73                            & $-$0.44                         & 4.7                       & 1.2                       & 10.6                         & 4.39E+20                                & 1                        \\
                    2999                    & 13782$+$0234$-$0119          & 137.8250                & 2.3417                  & 6                                        & \nodata                           & 0.62                          & $-$11.88                            & $-$12.55                            & $-$0.50                         & 5.7                       & 2.0                       & 12.2                         & 1.30E+21                                & 1                        \\
                    3000                    & 13783$+$0397$-$0090          & 137.8333                & 3.9667                  & 6                                        & \nodata                           & 0.37                          & $-$9.00                             & $-$9.77                             & $-$0.30                         & 5.9                       & 1.5                       & 11.1                         & 1.72E+21                                & 1                        \\
                    3001                    & 13893$-$0097$-$0044          & 138.9333                & $-$0.9750                 & 6                                        & \nodata                           & 0.27                          & $-$4.39                             & $-$4.90                             & $-$0.61                         & 3.5                       & 1.1                       & 8.2                          & 4.26E+20                                & 1                        \\
                    3002                    & 13921$+$0457$-$0098          & 139.2083                & 4.5667                  & 6                                        & \nodata                           & 0.39                          & $-$9.81                             & $-$10.51                            & $-$0.62                         & 3.5                       & 1.7                       & 8.1                          & 9.54E+20                                & 1                        \\
                    3003                    & 13930$+$0520$-$0147          & 139.3000                & 5.2000                  & 6                                        & \nodata                           & 0.56                          & $-$14.69                            & $-$15.25                            & $-$0.42                         & 5.8                       & 3.6                       & 14.0                         & 2.67E+21                                & 1                        \\
                    3004                    & 13951$+$0237$-$0107          & 139.5083                & 2.3667                  & 4,  6                                    & \nodata                           & 0.55                          & $-$10.66                            & $-$11.25                            & $-$0.58                         & 6.5                       & 2.5                       & 10.9                         & 1.24E+21                                & 1                        \\
                    3005                    & 13962$+$0389$-$0105          & 139.6167                & 3.8917                  & 6                                        & \nodata                           & 0.55                          & $-$10.47                            & $-$10.92                            & $-$0.33                         & 4.3                       & 1.2                       & 9.4                          & 6.95E+20                                & 1                        \\
                    3006                    & 13981$+$0267$-$0118          & 139.8083                & 2.6750                  & 4,  6                                    & \nodata                           & 0.55                          & $-$11.75                            & $-$12.27                            & $-$0.53                         & 8.2                       & 2.0                       & 13.2                         & 9.81E+20                                & 1                        \\
                    3007                    & 14017$+$0358$-$0100          & 140.1750                & 3.5833                  & 6                                        & \nodata                           & 0.53                          & $-$9.98                             & $-$10.40                            & $-$0.44                         & 5.0                       & 1.9                       & 11.4                         & 8.71E+20                                & 1                        \\
                    3008                    & 14042$+$0404$-$0107          & 140.4250                & 4.0417                  & 6                                        & \nodata                           & 0.55                          & $-$10.75                            & $-$11.57                            & $-$0.72                         & 5.4                       & 2.4                       & 10.2                         & 1.40E+21                                & 1                        \\
                    3009                    & 14056$+$0411$-$0107          & 140.5583                & 4.1083                  & 6                                        & \nodata                           & 0.56                          & $-$10.69                            & $-$11.68                            & $-$0.62                         & 6.2                       & 1.9                       & 16.3                         & 1.60E+21                                & 1                        \\
                    3010                    & 14057$+$0152$-$0143          & 140.5667                & 1.5250                  & 4,  6                                    & \nodata                           & 1.09                          & $-$14.25                            & $-$14.90                            & $-$0.73                         & 4.9                       & 1.6                       & 10.2                         & 6.66E+20                                & 1                        \\
                    3011                    & 14061$+$0153$-$0143          & 140.6083                & 1.5333                  & 4,  6                                    & \nodata                           & 1.38                          & $-$14.26                            & $-$14.59                            & $-$0.40                         & 4.4                       & 1.6                       & 10.0                         & 6.13E+20                                & 1                        \\
                    3012                    & 14062$+$0151$-$0145          & 140.6167                & 1.5083                  & 4,  6                                    & \nodata                           & 1.34                          & $-$14.45                            & $-$14.73                            & $-$0.40                         & 4.2                       & 1.9                       & 10.2                         & 6.26E+20                                & 1                        \\
                    3013                    & 14070$+$0306$-$0069          & 140.7000                & 3.0583                  & 3,  4,  6                                & \nodata                           & 0.28                          & $-$6.93                             & $-$7.67                             & $-$0.83                         & 4.8                       & 1.2                       & 8.7                          & 4.79E+20                                & 1                        \\
                    3014                    & 14072$+$0231$-$0126          & 140.7250                & 2.3083                  & 4,  6                                    & \nodata                           & 0.55                          & $-$12.56                            & $-$13.44                            & $-$0.74                         & 6.2                       & 2.4                       & 11.4                         & 1.40E+21                                & 1                        \\
                    3015                    & 14075$+$0231$-$0126          & 140.7500                & 2.3083                  & 4,  6                                    & \nodata                           & 0.55                          & $-$12.62                            & $-$13.57                            & $-$0.68                         & 6.9                       & 2.7                       & 12.4                         & 1.93E+21                                & 1                        \\
                    3016                    & 14094$+$0144$-$0125          & 140.9417                & 1.4417                  & 4,  6                                    & \nodata                           & 0.55                          & $-$12.49                            & $-$12.98                            & $-$0.54                         & 5.4                       & 1.8                       & 9.5                          & 7.73E+20                                & 1                        \\
                    3017                    & 14140$+$0359$-$0104          & 141.4000                & 3.5917                  & 6                                        & \nodata                           & 0.53                          & $-$10.39                            & $-$10.80                            & $-$0.43                         & 4.3                       & 1.6                       & 9.6                          & 7.09E+20                                & 1                        \\
                    3018                    & 14167$+$0243$-$0103          & 141.6750                & 2.4333                  & 4,  6                                    & \nodata                           & 0.53                          & $-$10.29                            & $-$11.00                            & $-$0.53                         & 7.8                       & 2.0                       & 12.9                         & 1.29E+21                                & 1                        \\
                    3019                    & 14180$+$0196$-$0099          & 141.8000                & 1.9583                  & 4,  6                                    & \nodata                           & 0.53                          & $-$9.87                             & $-$10.23                            & $-$0.29                         & 12.6                      & 4.1                       & 20.8                         & 3.24E+21                                & 1                        \\
                    3020                    & 14209$+$0046$-$0151          & 142.0917                & 0.4583                  & 6                                        & \nodata                           & 1.53                          & $-$15.11                            & $-$15.91                            & $-$0.59                         & 7.1                       & 3.0                       & 13.6                         & 2.15E+21                                & 1                        \\
                    3021                    & 14210$+$0101$-$0149          & 142.1000                & 1.0083                  & 6                                        & \nodata                           & 1.52                          & $-$14.88                            & $-$15.37                            & $-$0.42                         & 8.0                       & 1.6                       & 14.9                         & 9.35E+20                                & 1                        \\
                    3022                    & 14211$+$0061$-$0152          & 142.1083                & 0.6083                  & 1,  6                                    & \nodata                           & 1.54                          & $-$15.20                            & $-$15.99                            & $-$0.36                         & 7.7                       & 4.3                       & 17.7                         & 5.77E+21                                & 1                        \\
                    3023                    & 14212$+$0039$-$0154          & 142.1250                & 0.3917                  & 6                                        & \nodata                           & 1.54                          & $-$15.41                            & $-$16.14                            & $-$0.61                         & 5.3                       & 3.1                       & 10.8                         & 1.97E+21                                & 1                        \\
                    3024                    & 14213$+$0049$-$0151          & 142.1333                & 0.4917                  & 6                                        & \nodata                           & 1.54                          & $-$15.10                            & $-$15.86                            & $-$0.44                         & 6.7                       & 3.9                       & 14.3                         & 3.86E+21                                & 1                        \\
                    3025                    & 14213$+$0057$-$0152          & 142.1333                & 0.5667                  & 6                                        & \nodata                           & 1.54                          & $-$15.24                            & $-$16.23                            & $-$0.68                         & 6.8                       & 4.6                       & 14.7                         & 3.98E+21                                & 1                        \\
                    3026                    & 14218$+$0157$-$0130          & 142.1833                & 1.5750                  & 3,  6                                    & \nodata                           & 0.53                          & $-$13.00                            & $-$13.89                            & $-$0.28                         & 8.5                       & 2.3                       & 15.1                         & 3.88E+21                                & 1                        \\
                    3027                    & 14219$+$0072$-$0145          & 142.1917                & 0.7167                  & 6                                        & \nodata                           & 1.50                          & $-$14.46                            & $-$15.94                            & $-$0.94                         & 8.2                       & 3.0                       & 19.0                         & 2.83E+21                                & 1                        \\
                    3028                    & 14220$+$0168$-$0116          & 142.2000                & 1.6833                  & 6                                        & \nodata                           & 0.53                          & $-$11.62                            & $-$13.01                            & $-$0.64                         & 9.1                       & 4.0                       & 21.3                         & 5.60E+21                                & 1                        \\
                    3029                    & 14222$+$0082$-$0144          & 142.2250                & 0.8167                  & 6                                        & \nodata                           & 1.49                          & $-$14.37                            & $-$15.64                            & $-$0.67                         & 7.6                       & 2.4                       & 17.6                         & 2.58E+21                                & 1                        \\
                    3030                    & 14225$+$0065$-$0153          & 142.2500                & 0.6500                  & 6                                        & \nodata                           & 1.54                          & $-$15.27                            & $-$16.51                            & $-$0.51                         & 8.2                       & 2.8                       & 17.9                         & 3.84E+21                                & 1                        \\
                    3031                    & 14225$+$0515$-$0100          & 142.2500                & 5.1500                  & 6                                        & \nodata                           & 0.53                          & $-$9.98                             & $-$10.32                            & $-$0.31                         & 5.5                       & 1.9                       & 14.3                         & 1.09E+21                                & 1                        \\
                    3032                    & 14232$+$0135$-$0126          & 142.3167                & 1.3500                  & 3,  6                                    & \nodata                           & 0.53                          & $-$12.64                            & $-$13.65                            & $-$0.54                         & 9.8                       & 4.7                       & 16.0                         & 5.35E+21                                & 1                        \\
                    3033                    & 14243$+$0158$-$0137          & 142.4333                & 1.5833                  & 6                                        & \nodata                           & 0.53                          & $-$13.68                            & $-$14.25                            & $-$0.41                         & 6.9                       & 1.7                       & 12.5                         & 1.13E+21                                & 1                        \\
                    3034                    & 14251$+$0147$-$0122          & 142.5083                & 1.4667                  & 6                                        & \nodata                           & 0.53                          & $-$12.22                            & $-$12.80                            & $-$0.37                         & 9.4                       & 3.4                       & 15.1                         & 3.00E+21                                & 1                        \\
                    3035                    & 14263$+$0164$-$0126          & 142.6333                & 1.6417                  & 1,  6                                    & \nodata                           & 0.53                          & $-$12.59                            & $-$13.87                            & $-$0.59                         & 6.3                       & 2.0                       & 11.5                         & 2.07E+21                                & 1                        \\
                    3036                    & 14270$+$0097$-$0125          & 142.7000                & 0.9750                  & 6                                        & \nodata                           & 0.53                          & $-$12.53                            & $-$13.28                            & $-$0.56                         & 5.0                       & 1.8                       & 11.7                         & 1.12E+21                                & 1                        \\
                    3037                    & 14272$+$0240$-$0107          & 142.7250                & 2.4000                  & 6                                        & \nodata                           & 0.53                          & $-$10.70                            & $-$11.09                            & $-$0.39                         & 5.0                       & 1.6                       & 10.9                         & 7.19E+20                                & 1                        \\
                    3038                    & 14304$+$0176$-$0088          & 143.0417                & 1.7583                  & 6                                        & \nodata                           & 0.27                          & $-$8.75                             & $-$9.74                             & $-$0.64                         & 11.0                      & 5.7                       & 16.6                         & 5.78E+21                                & 1                        \\
                    3039                    & 14324$+$0170$-$0083          & 143.2417                & 1.7000                  & 6                                        & \nodata                           & 0.27                          & $-$8.29                             & $-$8.97                             & $-$0.56                         & 9.1                       & 3.6                       & 17.8                         & 2.56E+21                                & 1                        \\
                    3040                    & 14326$+$0017$-$0145          & 143.2583                & 0.1667                  & 6                                        & \nodata                           & 1.51                          & $-$14.52                            & $-$15.04                            & $-$0.56                         & 4.4                       & 1.4                       & 7.8                          & 6.05E+20                                & 1                        \\
                    3041                    & 14329$+$0019$-$0131          & 143.2917                & 0.1917                  & 6                                        & \nodata                           & 0.52                          & $-$13.12                            & $-$13.63                            & $-$0.57                         & 4.6                       & 1.2                       & 9.6                          & 4.73E+20                                & 1                        \\
                    3042                    & 14366$-$0359$-$0353          & 143.6583                & $-$3.5917                 & 6                                        & \nodata                           & 2.05                          & $-$35.32                            & $-$35.72                            & $-$0.30                         & 6.6                       & 2.6                       & 11.9                         & 1.79E+21                                & 1                        \\
                    3043                    & 14379$-$0350$-$0348          & 143.7917                & $-$3.5000                 & 6                                        & \nodata                           & 2.04                          & $-$34.82                            & $-$35.21                            & $-$0.33                         & 7.5                       & 2.3                       & 15.4                         & 1.49E+21                                & 1                        \\
                    3044                    & 14407$+$0000$-$0093          & 144.0667                & 0.0000                  & 6                                        & \nodata                           & 0.53                          & $-$9.26                             & $-$9.61                             & $-$0.41                         & 6.7                       & 2.0                       & 13.5                         & 8.34E+20                                & 1                        \\
                    3045                    & 14427$-$0044$-$0399          & 144.2667                & $-$0.4417                 & 1,  6                                    & \nodata                           & 2.04                          & $-$39.92                            & $-$40.62                            & $-$0.57                         & 6.1                       & 3.0                       & 10.3                         & 1.93E+21                                & 1                        \\
                    3046                    & 14446$-$0129$-$0304          & 144.4583                & $-$1.2917                 & 6                                        & \nodata                           & 2.07                          & $-$30.38                            & $-$31.25                            & $-$0.55                         & 5.9                       & 2.2                       & 12.3                         & 1.70E+21                                & 1                        \\
                    3047                    & 14507$-$0104$-$0304          & 145.0750                & $-$1.0417                 & 6                                        & \nodata                           & 2.05                          & $-$30.44                            & $-$30.79                            & $-$0.36                         & 7.0                       & 1.5                       & 12.1                         & 6.81E+20                                & 1                        \\
                    3048                    & 14530$-$0320$-$0196          & 145.3000                & $-$3.2000                 & 6                                        & \nodata                           & 1.93                          & $-$19.56                            & $-$19.94                            & $-$0.42                         & 6.0                       & 1.7                       & 12.5                         & 7.60E+20                                & 1                        \\
                    3049                    & 14612$+$0127$+$0007          & 146.1250                & 1.2667                  & 6                                        & \nodata                           & 0.28                          & 0.73                              & 0.28                              & $-$0.56                         & 5.6                       & 1.5                       & 13.1                         & 5.71E+20                                & 1                        \\
                    3050                    & 14630$+$0512$-$0025          & 146.3000                & 5.1167                  & 6                                        & \nodata                           & 0.28                          & $-$2.54                             & $-$2.81                             & $-$0.31                         & 6.7                       & 1.3                       & 13.0                         & 5.40E+20                                & 1                        \\
                    3051                    & 14634$+$0507$-$0024          & 146.3417                & 5.0667                  & 6                                        & \nodata                           & 0.28                          & $-$2.38                             & $-$2.71                             & $-$0.42                         & 6.6                       & 1.2                       & 11.8                         & 4.37E+20                                & 1                        \\
                    3052                    & 14635$+$0509$-$0023          & 146.3500                & 5.0917                  & 6                                        & \nodata                           & 0.28                          & $-$2.35                             & $-$2.64                             & $-$0.40                         & 6.4                       & 1.1                       & 13.7                         & 3.98E+20                                & 1                        \\
                    3053                    & 14641$+$0506$-$0023          & 146.4083                & 5.0583                  & 6                                        & \nodata                           & 0.28                          & $-$2.30                             & $-$2.68                             & $-$0.50                         & 6.1                       & 1.1                       & 10.7                         & 3.66E+20                                & 1                        \\
                    3054                    & 14802$-$0443$-$0264          & 148.0250                & $-$4.4333                 & 6                                        & \nodata                           & 2.03                          & $-$26.44                            & $-$27.02                            & $-$0.54                         & 5.3                       & 1.4                       & 10.2                         & 6.88E+20                                & 1                        \\
                    3055                    & 14808$+$0022$-$0337          & 148.0833                & 0.2167                  & 1,  3,  4,  6                            & \nodata                           & 2.04                          & $-$33.68                            & $-$34.61                            & $-$0.69                         & 6.6                       & 6.3                       & 16.4                         & 5.71E+21                                & 1                        \\
                    3056                    & 14832$-$0071$-$0056          & 148.3250                & $-$0.7083                 & 1,  4,  6                                & \nodata                           & 0.28                          & $-$5.57                             & $-$5.76                             & $-$0.11                         & 11.3                      & 3.7                       & 15.2                         & 3.71E+21                                & 1                        \\
                    3057                    & 14836$-$0020$-$0041          & 148.3583                & $-$0.2000                 & 4,  5,  6                                & \nodata                           & 0.28                          & $-$4.10                             & $-$4.84                             & $-$0.61                         & 11.8                      & 5.2                       & 19.4                         & 4.04E+21                                & 1                        \\
                    3058                    & 14841$+$0204$+$0023          & 148.4083                & 2.0417                  & 5,  6                                    & \nodata                           & 0.27                          & 2.32                              & 1.93                              & $-$0.47                         & 6.3                       & 3.5                       & 14.6                         & 1.64E+21                                & 1                        \\
                    3059                    & 14877$+$0217$+$0016          & 148.7750                & 2.1667                  & 5,  6                                    & \nodata                           & 0.27                          & 1.61                              & 0.96                              & $-$0.75                         & 6.0                       & 3.1                       & 17.1                         & 1.52E+21                                & 1                        \\
                    3060                    & 14878$+$0255$+$0027          & 148.7833                & 2.5500                  & 5,  6                                    & \nodata                           & 0.27                          & 2.70                              & 2.28                              & $-$0.43                         & 6.0                       & 1.9                       & 13.1                         & 9.09E+20                                & 1                        \\
                    3061                    & 14886$+$0308$+$0033          & 148.8583                & 3.0833                  & 6                                        & \nodata                           & 0.26                          & 3.35                              & 2.91                              & $-$0.55                         & 6.6                       & 1.5                       & 11.8                         & 5.73E+20                                & 1                        \\
                    3062                    & 14891$+$0310$+$0033          & 148.9083                & 3.1000                  & 6                                        & \nodata                           & 0.26                          & 3.29                              & 2.79                              & $-$0.59                         & 5.1                       & 2.5                       & 10.6                         & 1.05E+21                                & 1                        \\
                    3063                    & 14892$+$0307$+$0033          & 148.9250                & 3.0750                  & 6                                        & \nodata                           & 0.26                          & 3.28                              & 2.78                              & $-$0.53                         & 6.4                       & 2.5                       & 13.8                         & 1.19E+21                                & 1                        \\
                    3064                    & 14895$+$0302$+$0033          & 148.9500                & 3.0250                  & 6                                        & \nodata                           & 0.26                          & 3.26                              & 2.65                              & $-$0.76                         & 5.2                       & 1.5                       & 10.0                         & 5.38E+20                                & 1                        \\
                    3065                    & 14897$+$0316$+$0035          & 148.9667                & 3.1583                  & 6                                        & \nodata                           & 0.26                          & 3.46                              & 2.93                              & $-$0.60                         & 5.9                       & 1.3                       & 11.1                         & 5.37E+20                                & 1                        \\
                    3066                    & 14911$+$0262$-$0036          & 149.1083                & 2.6250                  & 5,  6                                    & \nodata                           & 0.28                          & $-$3.62                             & $-$3.84                             & $-$0.26                         & 8.6                       & 3.1                       & 13.9                         & 1.41E+21                                & 1                        \\
                    3067                    & 14912$+$0318$-$0034          & 149.1167                & 3.1833                  & 6                                        & \nodata                           & 0.28                          & $-$3.43                             & $-$4.19                             & $-$0.88                         & 4.3                       & 1.2                       & 11.6                         & 4.96E+20                                & 1                        \\
                    3068                    & 14917$+$0299$+$0032          & 149.1667                & 2.9917                  & 5,  6                                    & \nodata                           & 0.27                          & 3.18                              & 2.60                              & $-$0.55                         & 6.5                       & 3.7                       & 12.5                         & 2.16E+21                                & 1                        \\
                    3069                    & 14921$+$0298$+$0028          & 149.2083                & 2.9833                  & 5,  6                                    & \nodata                           & 0.27                          & 2.84                              & 2.31                              & $-$0.44                         & 7.5                       & 3.7                       & 13.4                         & 2.47E+21                                & 1                        \\
                    3070                    & 14936$+$0218$+$0015          & 149.3583                & 2.1833                  & 5,  6                                    & \nodata                           & 0.27                          & 1.53                              & 1.08                              & $-$0.45                         & 5.0                       & 1.3                       & 12.1                         & 6.10E+20                                & 1                        \\
                    3071                    & 14937$+$0215$+$0015          & 149.3667                & 2.1500                  & 5,  6                                    & \nodata                           & 0.27                          & 1.47                              & 1.08                              & $-$0.36                         & 4.8                       & 2.1                       & 11.8                         & 1.10E+21                                & 1                        \\
                    3072                    & 14939$+$0214$+$0018          & 149.3917                & 2.1417                  & 5,  6                                    & \nodata                           & 0.27                          & 1.81                              & 1.16                              & $-$0.87                         & 4.6                       & 1.4                       & 10.8                         & 4.92E+20                                & 1                        \\
                    3073                    & 14947$+$0342$+$0034          & 149.4750                & 3.4250                  & 5,  6                                    & \nodata                           & 0.26                          & 3.39                              & 2.75                              & $-$0.59                         & 5.9                       & 2.9                       & 13.0                         & 1.61E+21                                & 1                        \\
                    3074                    & 14952$+$0347$+$0036          & 149.5250                & 3.4667                  & 5,  6                                    & \nodata                           & 0.26                          & 3.56                              & 2.73                              & $-$0.70                         & 5.6                       & 3.0                       & 10.8                         & 1.89E+21                                & 1                        \\
                    3075                    & 14956$+$0348$+$0035          & 149.5583                & 3.4833                  & 5,  6                                    & \nodata                           & 0.26                          & 3.50                              & 2.77                              & $-$0.60                         & 5.3                       & 3.2                       & 10.7                         & 2.08E+21                                & 1                        \\
                    3076                    & 14956$+$0351$+$0037          & 149.5583                & 3.5083                  & 5,  6                                    & \nodata                           & 0.26                          & 3.69                              & 3.22                              & $-$0.41                         & 4.6                       & 2.2                       & 9.3                          & 1.23E+21                                & 1                        \\
                    3077                    & 14983$+$0361$+$0029          & 149.8333                & 3.6083                  & 5,  6                                    & \nodata                           & 0.27                          & 2.86                              & 2.47                              & $-$0.44                         & 6.2                       & 2.8                       & 12.4                         & 1.29E+21                                & 1                        \\
                    3078                    & 14988$+$0424$+$0048          & 149.8833                & 4.2417                  & 6                                        & \nodata                           & 0.25                          & 4.77                              & 4.25                              & $-$0.72                         & 5.0                       & 1.5                       & 10.7                         & 4.81E+20                                & 1                        \\
                    3079                    & 14992$+$0422$+$0046          & 149.9167                & 4.2250                  & 6                                        & \nodata                           & 0.25                          & 4.64                              & 4.21                              & $-$0.44                         & 5.3                       & 1.5                       & 10.2                         & 6.82E+20                                & 1                        \\
                    3080                    & 14998$+$0376$+$0029          & 149.9833                & 3.7583                  & 5,  6                                    & \nodata                           & 0.27                          & 2.86                              & 2.54                              & $-$0.37                         & 5.7                       & 2.8                       & 11.6                         & 1.24E+21                                & 1                        \\
                    3081                    & 14998$+$0428$+$0046          & 149.9833                & 4.2833                  & 6                                        & \nodata                           & 0.26                          & 4.56                              & 4.17                              & $-$0.49                         & 4.9                       & 1.2                       & 10.9                         & 4.33E+20                                & 1                        \\
                    3082                    & 14999$+$0371$+$0029          & 149.9917                & 3.7083                  & 5,  6                                    & \nodata                           & 0.27                          & 2.90                              & 2.61                              & $-$0.26                         & 6.2                       & 3.3                       & 12.8                         & 1.96E+21                                & 1                        \\
                    3083                    & 15000$+$0422$+$0046          & 150.0000                & 4.2250                  & 6                                        & \nodata                           & 0.25                          & 4.61                              & 4.20                              & $-$0.43                         & 5.0                       & 1.8                       & 10.4                         & 7.98E+20                                & 1                        \\
                    3084                    & 15012$+$0392$+$0033          & 150.1167                & 3.9167                  & 5,  6                                    & \nodata                           & 0.27                          & 3.27                              & 2.98                              & $-$0.33                         & 5.6                       & 3.4                       & 11.5                         & 1.59E+21                                & 1                        \\
                    3085                    & 15017$+$0427$+$0045          & 150.1667                & 4.2750                  & 6                                        & \nodata                           & 0.26                          & 4.55                              & 3.91                              & $-$0.73                         & 5.0                       & 3.0                       & 11.3                         & 1.36E+21                                & 1                        \\
                    3086                    & 15020$+$0385$+$0031          & 150.2000                & 3.8500                  & 5,  6                                    & \nodata                           & 0.27                          & 3.09                              & 2.59                              & $-$0.43                         & 7.5                       & 3.5                       & 13.8                         & 2.24E+21                                & 1                        \\
                    3087                    & 15035$+$0347$+$0014          & 150.3500                & 3.4667                  & 6                                        & \nodata                           & 0.28                          & 1.36                              & 0.98                              & $-$0.44                         & 6.3                       & 2.2                       & 11.3                         & 9.26E+20                                & 1                        \\
                    3088                    & 15035$+$0391$+$0030          & 150.3500                & 3.9083                  & 5,  6                                    & \nodata                           & 0.27                          & 3.02                              & 2.64                              & $-$0.39                         & 8.9                       & 5.0                       & 15.2                         & 2.99E+21                                & 1                        \\
                    3089                    & 15035$+$0393$+$0031          & 150.3500                & 3.9333                  & 5,  6                                    & \nodata                           & 0.27                          & 3.06                              & 2.67                              & $-$0.35                         & 8.6                       & 5.1                       & 14.3                         & 3.43E+21                                & 1                        \\
                    3090                    & 15035$+$0424$+$0045          & 150.3500                & 4.2417                  & 6                                        & \nodata                           & 0.26                          & 4.49                              & 3.98                              & $-$0.48                         & 6.0                       & 1.6                       & 11.1                         & 7.81E+20                                & 1                        \\
                    3091                    & 15037$+$0270$+$0028          & 150.3667                & 2.7000                  & 5,  6                                    & \nodata                           & 0.27                          & 2.79                              & 2.31                              & $-$0.45                         & 4.3                       & 1.7                       & 10.5                         & 8.30E+20                                & 1                        \\
                    3092                    & 15037$+$0427$+$0045          & 150.3667                & 4.2750                  & 6                                        & \nodata                           & 0.26                          & 4.54                              & 4.05                              & $-$0.59                         & 5.1                       & 1.6                       & 10.8                         & 5.90E+20                                & 1                        \\
                    3093                    & 15046$+$0357$-$0091          & 150.4583                & 3.5667                  & 5,  6                                    & \nodata                           & 0.28                          & $-$9.09                             & $-$9.42                             & $-$0.36                         & 6.3                       & 2.6                       & 14.5                         & 1.25E+21                                & 1                        \\
                    3094                    & 15046$+$0363$-$0087          & 150.4583                & 3.6333                  & 5,  6                                    & \nodata                           & 0.28                          & $-$8.75                             & $-$9.29                             & $-$0.56                         & 5.6                       & 3.6                       & 11.6                         & 1.92E+21                                & 1                        \\
                    3095                    & 15046$+$0369$-$0085          & 150.4583                & 3.6917                  & 5,  6                                    & \nodata                           & 0.28                          & $-$8.54                             & $-$9.33                             & $-$0.66                         & 6.2                       & 3.2                       & 10.2                         & 2.04E+21                                & 1                        \\
                    3096                    & 15048$+$0359$-$0089          & 150.4833                & 3.5917                  & 5,  6                                    & \nodata                           & 0.28                          & $-$8.93                             & $-$9.40                             & $-$0.39                         & 7.4                       & 2.3                       & 14.9                         & 1.44E+21                                & 1                        \\
                    3097                    & 15051$+$0448$+$0043          & 150.5083                & 4.4833                  & 6                                        & \nodata                           & 0.26                          & 4.33                              & 3.74                              & $-$0.63                         & 5.9                       & 1.5                       & 11.8                         & 6.74E+20                                & 1                        \\
                    3098                    & 15059$+$0348$+$0013          & 150.5917                & 3.4833                  & 5,  6                                    & \nodata                           & 0.28                          & 1.34                              & 0.93                              & $-$0.41                         & 7.0                       & 2.3                       & 12.1                         & 1.13E+21                                & 1                        \\
                    3099                    & 15062$+$0345$+$0012          & 150.6167                & 3.4500                  & 5,  6                                    & \nodata                           & 0.28                          & 1.22                              & 0.67                              & $-$0.58                         & 6.1                       & 2.2                       & 10.6                         & 1.02E+21                                & 1                        \\
                    3100                    & 15116$+$0522$+$0018          & 151.1583                & 5.2250                  & 5,  6                                    & \nodata                           & 0.27                          & 1.85                              & 1.54                              & $-$0.30                         & 6.5                       & 1.7                       & 13.0                         & 8.00E+20                                & 1                        \\
                    3101                    & 15121$+$0247$+$0017          & 151.2083                & 2.4667                  & 6                                        & \nodata                           & 0.27                          & 1.68                              & 1.23                              & $-$0.53                         & 7.3                       & 1.9                       & 14.2                         & 8.30E+20                                & 1                        \\
                    3102                    & 15124$+$0388$+$0017          & 151.2417                & 3.8833                  & 6                                        & \nodata                           & 0.27                          & 1.73                              & 1.12                              & $-$0.51                         & 6.0                       & 2.6                       & 12.5                         & 1.59E+21                                & 1                        \\
                    3103                    & 15125$+$0391$+$0017          & 151.2500                & 3.9083                  & 5,  6                                    & \nodata                           & 0.27                          & 1.71                              & 1.10                              & $-$0.48                         & 6.4                       & 2.9                       & 12.8                         & 1.97E+21                                & 1                        \\
                    3104                    & 15126$+$0246$+$0014          & 151.2583                & 2.4583                  & 6                                        & \nodata                           & 0.28                          & 1.40                              & 1.01                              & $-$0.57                         & 6.9                       & 1.6                       & 12.5                         & 5.29E+20                                & 1                        \\
                    3105                    & 15141$+$0387$+$0016          & 151.4083                & 3.8667                  & 5,  6                                    & \nodata                           & 0.27                          & 1.62                              & 0.96                              & $-$0.46                         & 6.3                       & 4.7                       & 15.7                         & 4.11E+21                                & 1                        \\
                    3106                    & 15148$+$0384$+$0015          & 151.4833                & 3.8417                  & 5,  6                                    & \nodata                           & 0.27                          & 1.52                              & 0.72                              & $-$0.59                         & 7.7                       & 4.1                       & 13.8                         & 3.23E+21                                & 1                        \\
                    3107                    & 15152$+$0512$-$0064          & 151.5167                & 5.1250                  & 6                                        & \nodata                           & 0.28                          & $-$6.44                             & $-$6.81                             & $-$0.40                         & 8.2                       & 2.1                       & 13.0                         & 9.62E+20                                & 1                        \\
                    3108                    & 15161$+$0440$-$0072          & 151.6083                & 4.4000                  & 5,  6                                    & \nodata                           & 0.28                          & $-$7.16                             & $-$7.69                             & $-$0.49                         & 8.0                       & 3.5                       & 15.0                         & 2.16E+21                                & 1                        \\
                    3109                    & 15242$+$0150$-$0379          & 152.4250                & 1.5000                  & 6                                        & \nodata                           & 2.06                          & $-$37.91                            & $-$39.31                            & $-$0.56                         & 8.2                       & 2.5                       & 15.7                         & 3.42E+21                                & 1                        \\
                    3110                    & 15257$+$0386$-$0071          & 152.5750                & 3.8583                  & 5,  6                                    & \nodata                           & 0.28                          & $-$7.06                             & $-$7.37                             & $-$0.38                         & 4.7                       & 1.3                       & 9.3                          & 4.63E+20                                & 1                        \\
                    3111                    & 15312$+$0304$+$0009          & 153.1167                & 3.0417                  & 5,  6                                    & \nodata                           & 0.28                          & 0.91                              & 0.13                              & $-$0.74                         & 7.1                       & 2.6                       & 14.0                         & 1.46E+21                                & 1                        \\
                    3112                    & 15320$+$0307$-$0065          & 153.2000                & 3.0667                  & 5,  6                                    & \nodata                           & 0.28                          & $-$6.47                             & $-$6.92                             & $-$0.58                         & 5.5                       & 3.2                       & 11.7                         & 1.30E+21                                & 1                        \\
                    3113                    & 15324$-$0107$-$0215          & 153.2417                & $-$1.0750                 & 6                                        & \nodata                           & 2.02                          & $-$21.49                            & $-$22.30                            & $-$0.47                         & 7.7                       & 4.0                       & 14.7                         & 3.90E+21                                & 1                        \\
                    3114                    & 15327$+$0373$-$0058          & 153.2667                & 3.7333                  & 5,  6                                    & \nodata                           & 0.28                          & $-$5.82                             & $-$6.32                             & $-$0.43                         & 7.7                       & 2.7                       & 12.9                         & 1.68E+21                                & 1                        \\
                    3115                    & 15327$-$0107$-$0214          & 153.2667                & $-$1.0750                 & 6                                        & \nodata                           & 2.02                          & $-$21.43                            & $-$22.31                            & $-$0.56                         & 7.7                       & 4.5                       & 14.2                         & 4.15E+21                                & 1                        \\
                    3116                    & 15327$+$0128$-$0051          & 153.2750                & 1.2833                  & 6                                        & \nodata                           & 0.28                          & $-$5.07                             & $-$5.54                             & $-$0.54                         & 5.5                       & 2.0                       & 11.7                         & 8.59E+20                                & 1                        \\
                    3117                    & 15332$+$0356$-$0059          & 153.3167                & 3.5583                  & 5,  6                                    & \nodata                           & 0.28                          & $-$5.91                             & $-$6.43                             & $-$0.51                         & 5.7                       & 2.1                       & 11.5                         & 1.03E+21                                & 1                        \\
                    3118                    & 15337$+$0362$-$0056          & 153.3750                & 3.6250                  & 5,  6                                    & \nodata                           & 0.28                          & $-$5.58                             & $-$6.35                             & $-$0.92                         & 6.9                       & 1.8                       & 12.4                         & 7.38E+20                                & 1                        \\
                    3119                    & 15352$+$0157$-$0050          & 153.5167                & 1.5750                  & 6                                        & \nodata                           & 0.28                          & $-$5.02                             & $-$5.74                             & $-$0.71                         & 5.1                       & 1.8                       & 9.6                          & 8.69E+20                                & 1                        \\
                    3120                    & 15352$+$0168$-$0054          & 153.5250                & 1.6833                  & 6                                        & \nodata                           & 0.28                          & $-$5.35                             & $-$6.03                             & $-$0.62                         & 5.1                       & 1.6                       & 10.3                         & 7.99E+20                                & 1                        \\
                    3121                    & 15370$+$0256$-$0002          & 153.7000                & 2.5583                  & 6                                        & \nodata                           & 0.28                          & $-$0.21                             & $-$1.25                             & $-$1.00                         & 5.7                       & 1.8                       & 19.6                         & 1.06E+21                                & 1                        \\
                    3122                    & 15376$+$0261$-$0008          & 153.7583                & 2.6083                  & 6                                        & \nodata                           & 0.28                          & $-$0.75                             & $-$1.20                             & $-$0.44                         & 5.4                       & 1.9                       & 10.4                         & 8.89E+20                                & 1                        \\
                    3123                    & 15378$+$0256$-$0008          & 153.7833                & 2.5583                  & 6                                        & \nodata                           & 0.28                          & $-$0.83                             & $-$1.15                             & $-$0.36                         & 5.3                       & 1.6                       & 9.3                          & 6.63E+20                                & 1                        \\
                    3124                    & 15405$+$0508$+$0044          & 154.0500                & 5.0833                  & 5,  6                                    & 17                                & 0.26                          & 4.40                              & 3.66                              & $-$0.58                         & 6.8                       & 5.3                       & 12.9                         & 4.28E+21                                & 1                        \\
                    3125                    & 15412$+$0166$-$0010          & 154.1250                & 1.6583                  & 6                                        & \nodata                           & 0.28                          & $-$0.99                             & $-$1.58                             & $-$0.32                         & 5.3                       & 2.1                       & 9.7                          & 1.89E+21                                & 1                        \\
                    3126                    & 15417$+$0172$-$0017          & 154.1667                & 1.7250                  & 6                                        & \nodata                           & 0.28                          & $-$1.70                             & $-$2.26                             & $-$0.45                         & 5.1                       & 2.0                       & 11.3                         & 1.21E+21                                & 1                        \\
                    3127                    & 15419$-$0265$-$0117          & 154.1917                & $-$2.6500                 & 6                                        & \nodata                           & 0.50                          & $-$11.75                            & $-$12.63                            & $-$0.58                         & 5.7                       & 1.6                       & 10.2                         & 1.17E+21                                & 1                        \\
                    3128                    & 15422$+$0423$+$0006          & 154.2167                & 4.2333                  & 6                                        & \nodata                           & 0.28                          & 0.56                              & $-$0.10                             & $-$0.66                         & 5.1                       & 1.4                       & 10.2                         & 6.18E+20                                & 1                        \\
                    3129                    & 15426$+$0172$-$0016          & 154.2583                & 1.7250                  & 6                                        & \nodata                           & 0.28                          & $-$1.56                             & $-$2.01                             & $-$0.38                         & 4.9                       & 1.7                       & 10.0                         & 9.30E+20                                & 1                        \\
                    3130                    & 15449$+$0352$-$0007          & 154.4917                & 3.5250                  & 6                                        & \nodata                           & 0.28                          & $-$0.75                             & $-$1.07                             & $-$0.41                         & 6.0                       & 1.8                       & 11.8                         & 6.94E+20                                & 1                        \\
                    3131                    & 15450$+$0313$-$0028          & 154.5000                & 3.1333                  & 6                                        & \nodata                           & 0.28                          & $-$2.79                             & $-$3.31                             & $-$0.30                         & 6.6                       & 2.4                       & 10.6                         & 2.00E+21                                & 1                        \\
                    3132                    & 15456$+$0326$-$0025          & 154.5583                & 3.2583                  & 6                                        & \nodata                           & 0.28                          & $-$2.55                             & $-$2.83                             & $-$0.27                         & 6.3                       & 1.8                       & 10.0                         & 8.65E+20                                & 1                        \\
                    3133                    & 15457$+$0303$-$0028          & 154.5667                & 3.0333                  & 6                                        & \nodata                           & 0.28                          & $-$2.76                             & $-$3.26                             & $-$0.36                         & 6.1                       & 2.4                       & 11.6                         & 1.66E+21                                & 1                        \\
                    3134                    & 15475$-$0053$+$0052          & 154.7500                & $-$0.5333                 & 6                                        & \nodata                           & 0.25                          & 5.22                              & 4.78                              & $-$0.42                         & 5.2                       & 1.4                       & 14.0                         & 7.18E+20                                & 1                        \\
                    3135                    & 15476$+$0332$-$0019          & 154.7583                & 3.3167                  & 6                                        & \nodata                           & 0.28                          & $-$1.93                             & $-$2.30                             & $-$0.39                         & 6.3                       & 2.0                       & 14.8                         & 9.79E+20                                & 1                        \\
                    3136                    & 15479$+$0334$-$0019          & 154.7917                & 3.3417                  & 6                                        & \nodata                           & 0.28                          & $-$1.89                             & $-$2.48                             & $-$0.53                         & 6.5                       & 2.2                       & 11.5                         & 1.22E+21                                & 1                        \\
                    3137                    & 15479$-$0052$+$0053          & 154.7917                & $-$0.5167                 & 6                                        & \nodata                           & 0.25                          & 5.28                              & 4.86                              & $-$0.45                         & 5.9                       & 2.1                       & 11.3                         & 9.32E+20                                & 1                        \\
                    3138                    & 15482$+$0333$-$0017          & 154.8167                & 3.3333                  & 6                                        & \nodata                           & 0.28                          & $-$1.74                             & $-$2.15                             & $-$0.33                         & 6.2                       & 1.8                       & 12.2                         & 1.09E+21                                & 1                        \\
                    3139                    & 15482$-$0050$+$0053          & 154.8167                & $-$0.5000                 & 6                                        & \nodata                           & 0.25                          & 5.29                              & 4.91                              & $-$0.49                         & 5.7                       & 2.2                       & 10.8                         & 8.25E+20                                & 1                        \\
                    3140                    & 15483$+$0477$+$0042          & 154.8333                & 4.7750                  & 6                                        & \nodata                           & 0.26                          & 4.15                              & 3.56                              & $-$0.47                         & 7.5                       & 2.4                       & 16.9                         & 1.70E+21                                & 1                        \\
                    3141                    & 15486$-$0037$+$0049          & 154.8583                & $-$0.3667                 & 6                                        & \nodata                           & 0.26                          & 4.93                              & 4.64                              & $-$0.23                         & 6.5                       & 2.0                       & 14.0                         & 1.24E+21                                & 1                        \\
                    3142                    & 15486$-$0041$+$0052          & 154.8583                & $-$0.4083                 & 6                                        & \nodata                           & 0.25                          & 5.18                              & 4.70                              & $-$0.54                         & 4.9                       & 1.4                       & 10.3                         & 5.42E+20                                & 1                        \\
                    3143                    & 15490$+$0248$+$0007          & 154.9000                & 2.4833                  & 6                                        & \nodata                           & 0.28                          & 0.70                              & $-$0.05                             & $-$0.75                         & 5.4                       & 2.1                       & 9.7                          & 1.01E+21                                & 1                        \\
                    3144                    & 15492$+$0246$+$0009          & 154.9167                & 2.4583                  & 6                                        & \nodata                           & 0.28                          & 0.90                              & $-$0.29                             & $-$1.02                         & 7.4                       & 2.3                       & 13.5                         & 1.37E+21                                & 1                        \\
                    3145                    & 15501$+$0023$+$0047          & 155.0083                & 0.2333                  & 6                                        & \nodata                           & 0.26                          & 4.68                              & 4.22                              & $-$0.44                         & 5.2                       & 1.3                       & 13.5                         & 6.52E+20                                & 1                        \\
                    3146                    & 15502$+$0030$+$0047          & 155.0167                & 0.3000                  & 6                                        & \nodata                           & 0.26                          & 4.71                              & 4.34                              & $-$0.40                         & 4.7                       & 1.6                       & 12.4                         & 7.09E+20                                & 1                        \\
                    3147                    & 15502$-$0084$+$0050          & 155.0250                & $-$0.8417                 & 6                                        & \nodata                           & 0.26                          & 4.99                              & 4.60                              & $-$0.54                         & 6.0                       & 1.4                       & 11.9                         & 4.66E+20                                & 1                        \\
                    3148                    & 15504$+$0029$+$0048          & 155.0417                & 0.2917                  & 6                                        & \nodata                           & 0.26                          & 4.80                              & 4.25                              & $-$0.70                         & 4.3                       & 1.8                       & 11.3                         & 6.45E+20                                & 1                        \\
                    3149                    & 15505$+$0026$+$0047          & 155.0500                & 0.2583                  & 6                                        & \nodata                           & 0.26                          & 4.75                              & 4.17                              & $-$0.68                         & 4.9                       & 1.5                       & 9.9                          & 5.77E+20                                & 1                        \\
                    3150                    & 15509$-$0063$+$0051          & 155.0917                & $-$0.6333                 & 6                                        & \nodata                           & 0.26                          & 5.13                              & 4.86                              & $-$0.30                         & 8.1                       & 1.4                       & 13.5                         & 6.20E+20                                & 1                        \\
                    3151                    & 15509$-$0067$+$0051          & 155.0917                & $-$0.6667                 & 6                                        & \nodata                           & 0.26                          & 5.10                              & 4.86                              & $-$0.33                         & 7.0                       & 2.1                       & 13.0                         & 7.86E+20                                & 1                        \\
                    3152                    & 15530$+$0406$-$0032          & 155.3000                & 4.0583                  & 6                                        & \nodata                           & 0.28                          & $-$3.19                             & $-$3.92                             & $-$0.80                         & 4.6                       & 1.5                       & 9.3                          & 6.24E+20                                & 1                        \\
                    3153                    & 15531$+$0487$+$0048          & 155.3083                & 4.8750                  & 6                                        & \nodata                           & 0.26                          & 4.76                              & 4.21                              & $-$0.58                         & 3.8                       & 0.9                       & 8.9                          & 3.00E+21                                & 2                        \\
                    3154                    & 15547$-$0058$+$0050          & 155.4750                & $-$0.5833                 & 6                                        & \nodata                           & 0.26                          & 5.00                              & 4.65                              & $-$0.45                         & 6.8                       & 2.1                       & 11.5                         & 7.72E+20                                & 1                        \\
                    3155                    & 15548$+$0493$+$0045          & 155.4833                & 4.9333                  & 5,  6                                    & \nodata                           & 0.26                          & 4.55                              & 4.06                              & $-$0.46                         & 6.7                       & 2.1                       & 12.1                         & 1.06E+21                                & 1                        \\
                    3156                    & 15563$+$0464$+$0033          & 155.6333                & 4.6417                  & 5,  6                                    & \nodata                           & 0.27                          & 3.29                              & 2.88                              & $-$0.50                         & 5.7                       & 1.4                       & 13.3                         & 5.59E+20                                & 1                        \\
                    3157                    & 15566$-$0061$+$0051          & 155.6583                & $-$0.6083                 & 6                                        & \nodata                           & 0.26                          & 5.09                              & 4.73                              & $-$0.40                         & 6.5                       & 1.2                       & 11.7                         & 5.07E+20                                & 1                        \\
                    3158                    & 15566$-$0075$+$0053          & 155.6583                & $-$0.7500                 & 6                                        & \nodata                           & 0.26                          & 5.31                              & 4.94                              & $-$0.51                         & 4.8                       & 1.0                       & 10.8                         & 3.29E+20                                & 1                        \\
                    3159                    & 15567$-$0042$+$0053          & 155.6667                & $-$0.4250                 & 6                                        & \nodata                           & 0.26                          & 5.30                              & 4.90                              & $-$0.50                         & 6.4                       & 2.3                       & 11.5                         & 9.06E+20                                & 1                        \\
                    3160                    & 15567$-$0049$+$0052          & 155.6667                & $-$0.4917                 & 6                                        & \nodata                           & 0.26                          & 5.18                              & 4.83                              & $-$0.50                         & 5.7                       & 2.3                       & 11.2                         & 8.09E+20                                & 1                        \\
                    3161                    & 15567$-$0040$+$0052          & 155.6750                & $-$0.4000                 & 6                                        & \nodata                           & 0.26                          & 5.24                              & 4.71                              & $-$0.62                         & 6.8                       & 1.6                       & 12.5                         & 6.45E+20                                & 1                        \\
                    3162                    & 15568$-$0055$+$0052          & 155.6833                & $-$0.5500                 & 6                                        & \nodata                           & 0.26                          & 5.17                              & 4.68                              & $-$0.65                         & 7.1                       & 2.0                       & 11.8                         & 7.40E+20                                & 1                        \\
                    3163                    & 15568$-$0057$+$0052          & 155.6833                & $-$0.5750                 & 6                                        & \nodata                           & 0.26                          & 5.17                              & 4.86                              & $-$0.42                         & 7.6                       & 2.2                       & 13.3                         & 7.84E+20                                & 1                        \\
                    3164                    & 15568$-$0062$-$0353          & 155.6833                & $-$0.6250                 & 6                                        & \nodata                           & 2.05                          & $-$35.29                            & $-$35.72                            & $-$0.42                         & 4.1                       & 1.8                       & 9.3                          & 8.65E+20                                & 1                        \\
                    3165                    & 15569$-$0044$+$0053          & 155.6917                & $-$0.4417                 & 6                                        & \nodata                           & 0.26                          & 5.27                              & 4.91                              & $-$0.43                         & 6.8                       & 2.2                       & 13.3                         & 9.26E+20                                & 1                        \\
                    3166                    & 15570$-$0067$+$0053          & 155.7000                & $-$0.6750                 & 6                                        & \nodata                           & 0.26                          & 5.26                              & 4.80                              & $-$0.60                         & 6.1                       & 1.2                       & 12.0                         & 4.08E+20                                & 1                        \\
                    3167                    & 15570$-$0097$+$0052          & 155.7000                & $-$0.9750                 & 6                                        & \nodata                           & 0.26                          & 5.18                              & 4.75                              & $-$0.45                         & 4.2                       & 1.1                       & 10.5                         & 4.80E+20                                & 1                        \\
                    3168                    & 15572$-$0040$+$0052          & 155.7167                & $-$0.4000                 & 6                                        & \nodata                           & 0.26                          & 5.19                              & 4.73                              & $-$0.60                         & 7.0                       & 2.3                       & 13.7                         & 9.07E+20                                & 1                        \\
                    3169                    & 15573$-$0048$+$0052          & 155.7333                & $-$0.4833                 & 6                                        & \nodata                           & 0.26                          & 5.18                              & 4.70                              & $-$0.60                         & 7.2                       & 2.4                       & 12.3                         & 9.67E+20                                & 1                        \\
                    3170                    & 15574$-$0041$+$0052          & 155.7417                & $-$0.4083                 & 6                                        & \nodata                           & 0.26                          & 5.19                              & 4.68                              & $-$0.53                         & 7.0                       & 2.4                       & 13.4                         & 1.20E+21                                & 1                        \\
                    3171                    & 15574$-$0053$+$0053          & 155.7417                & $-$0.5333                 & 6                                        & \nodata                           & 0.26                          & 5.30                              & 4.78                              & $-$0.54                         & 7.0                       & 2.0                       & 12.7                         & 9.49E+20                                & 1                        \\
                    3172                    & 15575$-$0064$+$0053          & 155.7500                & $-$0.6417                 & 6                                        & \nodata                           & 0.26                          & 5.33                              & 4.78                              & $-$0.73                         & 6.1                       & 1.3                       & 11.9                         & 4.31E+20                                & 1                        \\
                    3173                    & 15576$-$0085$+$0053          & 155.7583                & $-$0.8500                 & 6                                        & \nodata                           & 0.26                          & 5.35                              & 4.98                              & $-$0.53                         & 5.4                       & 1.3                       & 10.3                         & 4.05E+20                                & 1                        \\
                    3174                    & 15578$-$0059$+$0053          & 155.7833                & $-$0.5917                 & 6                                        & \nodata                           & 0.26                          & 5.32                              & 4.85                              & $-$0.54                         & 7.0                       & 1.9                       & 12.9                         & 8.05E+20                                & 1                        \\
                    3175                    & 15581$-$0057$+$0053          & 155.8083                & $-$0.5667                 & 6                                        & \nodata                           & 0.26                          & 5.26                              & 4.90                              & $-$0.44                         & 6.8                       & 2.3                       & 12.1                         & 9.46E+20                                & 1                        \\
                    3176                    & 15582$-$0090$+$0053          & 155.8167                & $-$0.9000                 & 6                                        & \nodata                           & 0.26                          & 5.30                              & 4.83                              & $-$0.56                         & 4.7                       & 1.3                       & 11.1                         & 4.97E+20                                & 1                        \\
                    3177                    & 15582$-$0048$+$0052          & 155.8250                & $-$0.4833                 & 6                                        & \nodata                           & 0.26                          & 5.22                              & 4.74                              & $-$0.53                         & 6.3                       & 1.8                       & 11.4                         & 7.64E+20                                & 1                        \\
                    3178                    & 15582$-$0051$+$0051          & 155.8250                & $-$0.5083                 & 6                                        & \nodata                           & 0.26                          & 5.11                              & 4.68                              & $-$0.58                         & 7.0                       & 2.8                       & 11.3                         & 1.06E+21                                & 1                        \\
                    3179                    & 15582$-$0054$+$0052          & 155.8250                & $-$0.5417                 & 6                                        & \nodata                           & 0.26                          & 5.17                              & 4.90                              & $-$0.35                         & 7.3                       & 2.5                       & 13.2                         & 1.02E+21                                & 1                        \\
                    3180                    & 15582$-$0066$+$0052          & 155.8250                & $-$0.6583                 & 6                                        & \nodata                           & 0.26                          & 5.24                              & 4.83                              & $-$0.54                         & 6.5                       & 1.5                       & 14.1                         & 5.46E+20                                & 1                        \\
                    3181                    & 15583$-$0042$+$0054          & 155.8333                & $-$0.4167                 & 6                                        & \nodata                           & 0.26                          & 5.39                              & 5.07                              & $-$0.37                         & 5.6                       & 1.4                       & 10.3                         & 5.50E+20                                & 1                        \\
                    3182                    & 15583$-$0057$+$0052          & 155.8333                & $-$0.5667                 & 6                                        & \nodata                           & 0.26                          & 5.19                              & 4.75                              & $-$0.56                         & 7.7                       & 2.2                       & 13.3                         & 8.64E+20                                & 1                        \\
                    3183                    & 15587$-$0054$+$0052          & 155.8667                & $-$0.5417                 & 6                                        & \nodata                           & 0.26                          & 5.19                              & 4.72                              & $-$0.54                         & 8.0                       & 2.1                       & 14.2                         & 9.44E+20                                & 1                        \\
                    3184                    & 15587$-$0058$+$0052          & 155.8667                & $-$0.5833                 & 6                                        & \nodata                           & 0.26                          & 5.17                              & 4.94                              & $-$0.26                         & 7.3                       & 2.6                       & 13.2                         & 1.14E+21                                & 1                        \\
                    3185                    & 15588$-$0051$+$0053          & 155.8833                & $-$0.5083                 & 6                                        & \nodata                           & 0.26                          & 5.30                              & 4.89                              & $-$0.47                         & 5.9                       & 1.4                       & 11.3                         & 5.54E+20                                & 1                        \\
                    3186                    & 15590$-$0072$+$0053          & 155.9000                & $-$0.7167                 & 6                                        & \nodata                           & 0.26                          & 5.31                              & 4.73                              & $-$0.69                         & 7.4                       & 2.6                       & 13.2                         & 1.15E+21                                & 1                        \\
                    3187                    & 15590$-$0076$+$0053          & 155.9000                & $-$0.7583                 & 6                                        & \nodata                           & 0.26                          & 5.30                              & 4.84                              & $-$0.60                         & 6.5                       & 2.3                       & 12.6                         & 8.73E+20                                & 1                        \\
                    3188                    & 15592$-$0057$+$0051          & 155.9250                & $-$0.5750                 & 6                                        & \nodata                           & 0.26                          & 5.12                              & 4.79                              & $-$0.39                         & 7.2                       & 3.0                       & 12.7                         & 1.33E+21                                & 1                        \\
                    3189                    & 15604$-$0045$+$0056          & 156.0417                & $-$0.4500                 & 6                                        & \nodata                           & 0.25                          & 5.57                              & 5.23                              & $-$0.37                         & 5.6                       & 1.3                       & 12.7                         & 5.45E+20                                & 1                        \\
                    3190                    & 15605$-$0050$+$0053          & 156.0500                & $-$0.5000                 & 6                                        & \nodata                           & 0.26                          & 5.30                              & 4.93                              & $-$0.43                         & 4.9                       & 1.1                       & 9.8                          & 4.48E+20                                & 1                        \\
                    3191                    & 15644$+$0430$+$0007          & 156.4417                & 4.3000                  & 6                                        & \nodata                           & 0.28                          & 0.71                              & 0.11                              & $-$0.47                         & 4.3                       & 1.2                       & 10.6                         & 6.62E+20                                & 1                        \\
                    3192                    & 15650$+$0352$+$0011          & 156.5000                & 3.5250                  & 6                                        & \nodata                           & 0.28                          & 1.07                              & 0.46                              & $-$0.39                         & 6.4                       & 2.8                       & 16.1                         & 2.41E+21                                & 1                        \\
                    3193                    & 15656$-$0163$-$0157          & 156.5583                & $-$1.6333                 & 1,  6                                    & \nodata                           & 0.52                          & $-$15.74                            & $-$16.35                            & $-$0.39                         & 7.6                       & 3.2                       & 14.8                         & 2.73E+21                                & 1                        \\
                    3194                    & 15657$-$0082$+$0054          & 156.5667                & $-$0.8250                 & 3,  6                                    & \nodata                           & 0.26                          & 5.38                              & 4.92                              & $-$0.59                         & 5.4                       & 1.3                       & 11.0                         & 4.78E+20                                & 1                        \\
                    3195                    & 15657$-$0080$+$0054          & 156.5750                & $-$0.8000                 & 3,  6                                    & \nodata                           & 0.26                          & 5.35                              & 4.90                              & $-$0.65                         & 4.4                       & 1.3                       & 10.2                         & 4.17E+20                                & 1                        \\
                    3196                    & 15661$-$0084$+$0054          & 156.6083                & $-$0.8417                 & 3,  6                                    & \nodata                           & 0.26                          & 5.39                              & 4.92                              & $-$0.62                         & 5.1                       & 1.4                       & 12.8                         & 4.83E+20                                & 1                        \\
                    3197                    & 15663$-$0084$+$0054          & 156.6333                & $-$0.8417                 & 3,  6                                    & \nodata                           & 0.26                          & 5.42                              & 5.13                              & $-$0.35                         & 4.2                       & 1.5                       & 8.9                          & 5.83E+20                                & 1                        \\
                    3198                    & 15667$-$0087$+$0054          & 156.6667                & $-$0.8750                 & 6                                        & \nodata                           & 0.26                          & 5.41                              & 5.01                              & $-$0.44                         & 4.0                       & 1.4                       & 9.5                          & 5.55E+20                                & 1                        \\
                    3199                    & 15670$-$0082$+$0054          & 156.7000                & $-$0.8250                 & 6                                        & \nodata                           & 0.26                          & 5.41                              & 4.83                              & $-$0.80                         & 4.6                       & 2.5                       & 9.9                          & 9.30E+20                                & 1                        \\
                    3200                    & 15673$-$0083$+$0054          & 156.7333                & $-$0.8333                 & 6                                        & \nodata                           & 0.26                          & 5.44                              & 4.88                              & $-$0.71                         & 4.6                       & 2.5                       & 11.3                         & 9.81E+20                                & 1                        \\
                    3201                    & 15676$+$0047$+$0012          & 156.7583                & 0.4750                  & 1,  6                                    & \nodata                           & 0.28                          & 1.21                              & 0.62                              & $-$0.56                         & 5.4                       & 1.1                       & 11.8                         & 5.40E+20                                & 1                        \\
                    3202                    & 15677$-$0082$+$0054          & 156.7667                & $-$0.8250                 & 6                                        & \nodata                           & 0.26                          & 5.39                              & 4.96                              & $-$0.56                         & 3.5                       & 2.6                       & 11.0                         & 9.97E+20                                & 1                        \\
                    3203                    & 15680$-$0090$+$0055          & 156.8000                & $-$0.9000                 & 6                                        & \nodata                           & 0.26                          & 5.45                              & 4.86                              & $-$0.78                         & 4.6                       & 2.0                       & 12.1                         & 7.32E+20                                & 1                        \\
                    3204                    & 15684$-$0100$+$0052          & 156.8417                & $-$1.0000                 & 6                                        & \nodata                           & 0.26                          & 5.24                              & 4.80                              & $-$0.53                         & 5.0                       & 2.6                       & 11.8                         & 1.11E+21                                & 1                        \\
                    3205                    & 15687$-$0100$+$0052          & 156.8750                & $-$1.0000                 & 6                                        & \nodata                           & 0.26                          & 5.22                              & 4.74                              & $-$0.66                         & 5.4                       & 2.4                       & 12.7                         & 8.99E+20                                & 1                        \\
                    3206                    & 15691$-$0105$+$0053          & 156.9083                & $-$1.0500                 & 6                                        & \nodata                           & 0.26                          & 5.31                              & 4.87                              & $-$0.59                         & 3.7                       & 1.8                       & 8.6                          & 6.39E+20                                & 1                        \\
                    3207                    & 15692$-$0096$+$0053          & 156.9167                & $-$0.9583                 & 6                                        & \nodata                           & 0.26                          & 5.30                              & 4.78                              & $-$0.64                         & 4.9                       & 3.2                       & 12.3                         & 1.35E+21                                & 1                        \\
                    3208                    & 15692$-$0102$+$0053          & 156.9167                & $-$1.0250                 & 6                                        & \nodata                           & 0.26                          & 5.26                              & 4.90                              & $-$0.46                         & 4.8                       & 1.9                       & 15.2                         & 7.59E+20                                & 1                        \\
                    3209                    & 15696$-$0099$+$0052          & 156.9583                & $-$0.9917                 & 6                                        & \nodata                           & 0.26                          & 5.22                              & 4.69                              & $-$0.75                         & 4.7                       & 2.8                       & 9.8                          & 1.03E+21                                & 1                        \\
                    3210                    & 15697$-$0137$-$0018          & 156.9667                & $-$1.3667                 & 6                                        & \nodata                           & 0.28                          & $-$1.82                             & $-$2.39                             & $-$0.39                         & 6.9                       & 3.1                       & 11.8                         & 2.33E+21                                & 1                        \\
                    3211                    & 15697$-$0093$+$0052          & 156.9750                & $-$0.9333                 & 5,  6                                    & \nodata                           & 0.26                          & 5.22                              & 4.83                              & $-$0.48                         & 5.1                       & 4.4                       & 13.0                         & 2.02E+21                                & 1                        \\
                    3212                    & 15698$-$0089$+$0053          & 156.9833                & $-$0.8917                 & 5,  6                                    & \nodata                           & 0.26                          & 5.25                              & 4.70                              & $-$0.66                         & 4.8                       & 3.8                       & 10.7                         & 1.81E+21                                & 1                        \\
                    3213                    & 15701$-$0159$-$0019          & 157.0083                & $-$1.5917                 & 6                                        & \nodata                           & 0.28                          & $-$1.94                             & $-$2.41                             & $-$0.49                         & 5.6                       & 1.5                       & 12.3                         & 6.78E+20                                & 1                        \\
                    3214                    & 15706$-$0092$+$0053          & 157.0583                & $-$0.9250                 & 5,  6                                    & \nodata                           & 0.26                          & 5.29                              & 4.80                              & $-$0.70                         & 5.1                       & 4.8                       & 11.5                         & 2.15E+21                                & 1                        \\
                    3215                    & 15707$-$0095$+$0053          & 157.0750                & $-$0.9500                 & 5,  6                                    & \nodata                           & 0.26                          & 5.34                              & 4.99                              & $-$0.47                         & 4.9                       & 4.6                       & 11.9                         & 2.07E+21                                & 1                        \\
                    3216                    & 15708$-$0142$-$0023          & 157.0833                & $-$1.4250                 & 6                                        & \nodata                           & 0.28                          & $-$2.30                             & $-$2.77                             & $-$0.59                         & 5.2                       & 1.4                       & 10.5                         & 5.15E+20                                & 1                        \\
                    3217                    & 15711$-$0095$+$0053          & 157.1083                & $-$0.9500                 & 5,  6                                    & \nodata                           & 0.26                          & 5.31                              & 4.86                              & $-$0.51                         & 5.9                       & 4.9                       & 13.1                         & 2.64E+21                                & 1                        \\
                    3218                    & 15712$-$0083$+$0051          & 157.1167                & $-$0.8333                 & 5,  6                                    & \nodata                           & 0.26                          & 5.14                              & 4.64                              & $-$0.66                         & 5.7                       & 3.1                       & 12.1                         & 1.23E+21                                & 1                        \\
                    3219                    & 15716$-$0100$+$0053          & 157.1583                & $-$1.0000                 & 5,  6                                    & \nodata                           & 0.26                          & 5.33                              & 5.01                              & $-$0.47                         & 5.6                       & 5.5                       & 11.3                         & 2.59E+21                                & 1                        \\
                    3220                    & 15725$-$0092$+$0050          & 157.2500                & $-$0.9167                 & 5,  6                                    & \nodata                           & 0.26                          & 5.01                              & 4.66                              & $-$0.47                         & 7.2                       & 2.8                       & 12.3                         & 1.07E+21                                & 1                        \\
                    3221                    & 15725$-$0275$-$0095          & 157.2500                & $-$2.7500                 & 6                                        & \nodata                           & 0.47                          & $-$9.48                             & $-$9.87                             & $-$0.30                         & 4.5                       & 2.2                       & 8.9                          & 1.42E+21                                & 1                        \\
                    3222                    & 15727$+$0394$+$0006          & 157.2667                & 3.9417                  & 3,  6                                    & \nodata                           & 0.28                          & 0.65                              & 0.11                              & $-$0.57                         & 4.7                       & 1.5                       & 11.6                         & 6.44E+20                                & 1                        \\
                    3223                    & 15727$-$0277$-$0012          & 157.2667                & $-$2.7750                 & 6                                        & \nodata                           & 0.28                          & $-$1.17                             & $-$1.56                             & $-$0.52                         & 4.9                       & 1.9                       & 9.7                          & 7.04E+20                                & 1                        \\
                    3224                    & 15727$-$0422$-$0084          & 157.2750                & $-$4.2250                 & 6                                        & \nodata                           & 0.46                          & $-$8.38                             & $-$8.75                             & $-$0.42                         & 5.7                       & 1.9                       & 12.1                         & 7.98E+20                                & 1                        \\
                    3225                    & 15737$-$0401$-$0324          & 157.3667                & $-$4.0083                 & 6                                        & \nodata                           & 1.95                          & $-$32.38                            & $-$33.03                            & $-$0.30                         & 6.3                       & 2.1                       & 12.7                         & 2.22E+21                                & 1                        \\
                    3226                    & 15737$+$0394$+$0008          & 157.3750                & 3.9417                  & 6                                        & \nodata                           & 0.28                          & 0.85                              & 0.42                              & $-$0.45                         & 5.5                       & 1.7                       & 11.1                         & 7.38E+20                                & 1                        \\
                    3227                    & 15745$-$0020$-$0016          & 157.4500                & $-$0.2000                 & 6                                        & \nodata                           & 0.28                          & $-$1.59                             & $-$1.86                             & $-$0.22                         & 7.3                       & 1.8                       & 14.2                         & 1.07E+21                                & 1                        \\
                    3228                    & 15746$+$0395$+$0010          & 157.4583                & 3.9500                  & 6                                        & \nodata                           & 0.28                          & 0.95                              & 0.54                              & $-$0.43                         & 5.5                       & 1.3                       & 9.8                          & 5.66E+20                                & 1                        \\
                    3229                    & 15746$-$0405$-$0319          & 157.4583                & $-$4.0500                 & 6                                        & \nodata                           & 1.96                          & $-$31.90                            & $-$32.85                            & $-$0.48                         & 5.5                       & 2.1                       & 10.7                         & 1.98E+21                                & 1                        \\
                    3230                    & 15747$-$0402$-$0315          & 157.4750                & $-$4.0250                 & 6                                        & \nodata                           & 1.96                          & $-$31.54                            & $-$32.46                            & $-$0.45                         & 3.9                       & 1.9                       & 11.4                         & 1.81E+21                                & 1                        \\
                    3231                    & 15756$-$0417$-$0325          & 157.5583                & $-$4.1667                 & 6                                        & \nodata                           & 1.95                          & $-$32.51                            & $-$33.25                            & $-$0.44                         & 5.4                       & 1.8                       & 15.2                         & 1.60E+21                                & 1                        \\
                    3232                    & 15758$-$0416$-$0326          & 157.5833                & $-$4.1583                 & 6                                        & \nodata                           & 1.95                          & $-$32.63                            & $-$33.12                            & $-$0.40                         & 4.4                       & 2.1                       & 10.6                         & 1.23E+21                                & 1                        \\
                    3233                    & 15759$-$0416$-$0325          & 157.5917                & $-$4.1583                 & 6                                        & \nodata                           & 1.95                          & $-$32.50                            & $-$33.19                            & $-$0.49                         & 4.6                       & 1.7                       & 11.3                         & 1.09E+21                                & 1                        \\
                    3234                    & 15775$-$0098$-$0025          & 157.7500                & $-$0.9833                 & 6                                        & \nodata                           & 0.28                          & $-$2.48                             & $-$3.18                             & $-$0.62                         & 5.8                       & 1.7                       & 13.0                         & 9.00E+20                                & 1                        \\
                    3235                    & 15801$-$0250$-$0026          & 158.0083                & $-$2.5000                 & 6                                        & \nodata                           & 0.28                          & $-$2.57                             & $-$2.89                             & $-$0.46                         & 5.7                       & 1.5                       & 11.8                         & 4.95E+20                                & 1                        \\
                    3236                    & 15850$-$0104$-$0030          & 158.5000                & $-$1.0417                 & 6                                        & \nodata                           & 0.28                          & $-$3.02                             & $-$3.62                             & $-$0.47                         & 6.0                       & 1.9                       & 11.3                         & 1.13E+21                                & 1                        \\
                    3237                    & 15892$-$0254$-$0193          & 158.9167                & $-$2.5417                 & 6                                        & \nodata                           & 2.00                          & $-$19.34                            & $-$19.74                            & $-$0.46                         & 4.1                       & 2.3                       & 10.1                         & 1.01E+21                                & 1                        \\
                    3238                    & 15929$-$0067$+$0004          & 159.2917                & $-$0.6750                 & 6                                        & \nodata                           & 0.28                          & 0.41                              & 0.08                              & $-$0.26                         & 7.1                       & 1.6                       & 13.9                         & 9.84E+20                                & 1                        \\
                    3239                    & 15941$+$0117$-$0058          & 159.4083                & 1.1750                  & 6                                        & \nodata                           & 0.46                          & $-$5.81                             & $-$6.34                             & $-$0.64                         & 3.6                       & 1.1                       & 11.1                         & 4.02E+20                                & 1                        \\
                    3240                    & 15941$+$0120$-$0060          & 159.4083                & 1.2000                  & 6                                        & \nodata                           & 0.46                          & $-$6.00                             & $-$6.47                             & $-$0.59                         & 3.1                       & 1.6                       & 7.2                          & 6.28E+20                                & 1                        \\
                    3241                    & 15947$-$0061$+$0011          & 159.4667                & $-$0.6083                 & 6                                        & \nodata                           & 0.28                          & 1.12                              & 0.79                              & $-$0.30                         & 6.1                       & 1.2                       & 13.6                         & 6.03E+20                                & 1                        \\
                    3242                    & 15983$+$0104$+$0043          & 159.8333                & 1.0417                  & 6                                        & \nodata                           & 0.27                          & 4.25                              & 3.90                              & $-$0.44                         & 7.5                       & 3.6                       & 14.0                         & 1.59E+21                                & 1                        \\
                    3243                    & 16031$+$0293$-$0214          & 160.3083                & 2.9333                  & 4,  6                                    & \nodata                           & 2.02                          & $-$21.37                            & $-$22.17                            & $-$0.41                         & 5.9                       & 2.3                       & 11.1                         & 2.23E+21                                & 1                        \\
                    3244                    & 16126$+$0351$-$0038          & 161.2583                & 3.5083                  & 4,  6                                    & \nodata                           & 0.28                          & $-$3.79                             & $-$4.16                             & $-$0.41                         & 4.0                       & 1.3                       & 10.4                         & 5.57E+20                                & 1                        \\
                    3245                    & 16127$+$0348$-$0038          & 161.2667                & 3.4833                  & 4,  6                                    & \nodata                           & 0.28                          & $-$3.75                             & $-$3.99                             & $-$0.24                         & 4.5                       & 1.4                       & 13.0                         & 6.95E+20                                & 1                        \\
                    3246                    & 16284$+$0122$+$0006          & 162.8417                & 1.2250                  & 3,  6                                    & \nodata                           & 0.28                          & 0.60                              & 0.20                              & $-$0.42                         & 7.7                       & 3.9                       & 14.0                         & 2.07E+21                                & 1                        \\
                    3247                    & 16974$-$0067$-$0078          & 169.7417                & $-$0.6750                 & 6                                        & \nodata                           & 1.80                          & $-$7.76                             & $-$8.25                             & $-$0.56                         & 5.8                       & 1.4                       & 12.0                         & 5.94E+20                                & 1                        \\
                    3248                    & 16987$-$0417$+$0056          & 169.8750                & $-$4.1667                 & 6                                        & \nodata                           & 0.23                          & 5.65                              & 5.10                              & $-$0.61                         & 5.4                       & 2.4                       & 10.6                         & 1.08E+21                                & 1                        \\
                    3249                    & 17061$+$0263$-$0193          & 170.6083                & 2.6333                  & 6                                        & \nodata                           & 1.70                          & $-$19.29                            & $-$19.56                            & $-$0.20                         & 5.2                       & 1.1                       & 10.5                         & 6.85E+20                                & 1                        \\
                    3250                    & 17065$-$0026$-$0154          & 170.6500                & $-$0.2583                 & 1,  6                                    & \nodata                           & 1.85                          & $-$15.43                            & $-$16.76                            & $-$0.59                         & 6.9                       & 4.6                       & 14.9                         & 6.13E+21                                & 1                        \\
                    3251                    & 17067$-$0022$-$0151          & 170.6750                & $-$0.2167                 & 6                                        & \nodata                           & 1.85                          & $-$15.11                            & $-$15.99                            & $-$0.42                         & 7.6                       & 3.6                       & 12.6                         & 4.20E+21                                & 1                        \\
                    3252                    & 17072$+$0177$-$0173          & 170.7167                & 1.7750                  & 6                                        & \nodata                           & 1.74                          & $-$17.30                            & $-$17.65                            & $-$0.29                         & 7.5                       & 2.2                       & 12.5                         & 1.35E+21                                & 1                        \\
                    3253                    & 17072$-$0015$-$0152          & 170.7250                & $-$0.1500                 & 6                                        & \nodata                           & 1.83                          & $-$15.15                            & $-$16.05                            & $-$0.57                         & 7.7                       & 4.2                       & 16.0                         & 3.90E+21                                & 1                        \\
                    3254                    & 17080$+$0267$-$0200          & 170.8000                & 2.6750                  & 6                                        & \nodata                           & 1.70                          & $-$20.01                            & $-$20.57                            & $-$0.32                         & 5.2                       & 1.5                       & 10.1                         & 1.17E+21                                & 1                        \\
                    3255                    & 17105$+$0257$-$0208          & 171.0500                & 2.5750                  & 6                                        & \nodata                           & 1.71                          & $-$20.76                            & $-$21.28                            & $-$0.56                         & 6.1                       & 1.5                       & 12.5                         & 6.68E+20                                & 1                        \\
                    3256                    & 17121$-$0087$-$0136          & 171.2083                & $-$0.8667                 & 3,  4,  5,  6                            & \nodata                           & 1.81                          & $-$13.59                            & $-$14.37                            & $-$1.04                         & 5.3                       & 1.4                       & 9.8                          & 4.61E+20                                & 1                        \\
                    3257                    & 17137$-$0086$-$0133          & 171.3750                & $-$0.8583                 & 4,  5,  6                                & \nodata                           & 1.80                          & $-$13.27                            & $-$13.92                            & $-$0.48                         & 5.2                       & 1.3                       & 11.2                         & 8.16E+20                                & 1                        \\
                    3258                    & 17144$+$0252$-$0187          & 171.4417                & 2.5167                  & 5,  6                                    & \nodata                           & 1.69                          & $-$18.74                            & $-$19.58                            & $-$0.63                         & 7.7                       & 4.5                       & 12.9                         & 3.58E+21                                & 1                        \\
                    3259                    & 17147$+$0112$-$0164          & 171.4667                & 1.1250                  & 6                                        & \nodata                           & 1.71                          & $-$16.42                            & $-$17.17                            & $-$0.41                         & 7.3                       & 2.1                       & 13.8                         & 1.94E+21                                & 1                        \\
                    3260                    & 17154$+$0245$-$0186          & 171.5417                & 2.4500                  & 1,  5,  6                                & \nodata                           & 1.69                          & $-$18.58                            & $-$19.55                            & $-$0.44                         & 9.3                       & 4.7                       & 17.2                         & 6.43E+21                                & 1                        \\
                    3261                    & 17162$+$0110$-$0183          & 171.6250                & 1.1000                  & 6                                        & \nodata                           & 1.77                          & $-$18.31                            & $-$19.05                            & $-$0.40                         & 6.4                       & 2.1                       & 12.8                         & 1.88E+21                                & 1                        \\
                    3262                    & 17199$-$0162$-$0039          & 171.9917                & $-$1.6250                 & 4,  5,  6                                & \nodata                           & 1.69                          & $-$3.93                             & $-$4.41                             & $-$0.54                         & 5.3                       & 1.4                       & 10.8                         & 5.59E+20                                & 1                        \\
                    3263                    & 17272$+$0257$-$0178          & 172.7167                & 2.5667                  & 5,  6                                    & \nodata                           & 1.69                          & $-$17.81                            & $-$19.11                            & $-$0.60                         & 9.2                       & 4.2                       & 18.2                         & 5.61E+21                                & 1                        \\
                    3264                    & 17277$+$0209$-$0153          & 172.7667                & 2.0917                  & 5,  6                                    & 17                                & 1.69                          & $-$15.31                            & $-$16.17                            & $-$0.49                         & 7.7                       & 2.9                       & 12.9                         & 2.59E+21                                & 1                        \\
                    3265                    & 17287$+$0229$-$0173          & 172.8750                & 2.2917                  & 5,  6                                    & \nodata                           & 1.69                          & $-$17.25                            & $-$18.77                            & $-$0.54                         & 7.7                       & 4.5                       & 16.2                         & 7.45E+21                                & 1                        \\
                    3266                    & 17290$+$0227$-$0173          & 172.9000                & 2.2667                  & 5,  6                                    & \nodata                           & 1.69                          & $-$17.30                            & $-$19.09                            & $-$0.55                         & 10.5                      & 5.3                       & 20.5                         & 1.18E+22                                & 1                        \\
                    3267                    & 17301$+$0259$-$0176          & 173.0083                & 2.5917                  & 5,  6                                    & \nodata                           & 0.97                          & $-$17.62                            & $-$18.18                            & $-$0.32                         & 10.3                      & 3.9                       & 17.0                         & 4.09E+21                                & 1                        \\
                    3268                    & 17303$-$0355$-$0023          & 173.0333                & $-$3.5500                 & 4,  6                                    & \nodata                           & 0.23                          & $-$2.32                             & $-$2.91                             & $-$0.73                         & 7.9                       & 1.2                       & 14.7                         & 4.73E+20                                & 1                        \\
                    3269                    & 17316$+$0256$-$0174          & 173.1583                & 2.5583                  & 1,  5,  6                                & \nodata                           & 0.97                          & $-$17.42                            & $-$18.27                            & $-$0.41                         & 12.7                      & 6.1                       & 17.8                         & 8.51E+21                                & 1                        \\
                    3270                    & 17316$+$0262$-$0169          & 173.1583                & 2.6250                  & 5,  6                                    & \nodata                           & 0.97                          & $-$16.87                            & $-$17.55                            & $-$0.39                         & 10.1                      & 1.9                       & 15.3                         & 1.73E+21                                & 1                        \\
                    3271                    & 17318$+$0198$-$0126          & 173.1833                & 1.9833                  & 6                                        & \nodata                           & 0.97                          & $-$12.63                            & $-$13.02                            & $-$0.46                         & 5.2                       & 1.7                       & 10.3                         & 6.66E+20                                & 1                        \\
                    3272                    & 17334$-$0030$-$0141          & 173.3417                & $-$0.3000                 & 3,  4,  5,  6                            & \nodata                           & 0.97                          & $-$14.12                            & $-$14.83                            & $-$0.45                         & 6.7                       & 1.7                       & 11.8                         & 1.29E+21                                & 1                        \\
                    3273                    & 17348$+$0244$-$0168          & 173.4833                & 2.4417                  & 1,  6                                    & \nodata                           & 0.97                          & $-$16.83                            & $-$18.24                            & $-$0.37                         & 20.1                      & 6.8                       & 29.7                         & 2.20E+22                                & 1                        \\
                    3274                    & 17355$+$0245$-$0171          & 173.5500                & 2.4500                  & 6                                        & \nodata                           & 0.97                          & $-$17.14                            & $-$18.55                            & $-$0.45                         & 12.2                      & 4.0                       & 20.9                         & 8.28E+21                                & 1                        \\
                    3275                    & 17357$+$0247$-$0168          & 173.5667                & 2.4750                  & 6                                        & \nodata                           & 0.97                          & $-$16.81                            & $-$17.20                            & $-$0.17                         & 10.3                      & 2.9                       & 15.1                         & 3.76E+21                                & 1                        \\
                    3276                    & 17374$+$0267$-$0163          & 173.7417                & 2.6750                  & 3,  6                                    & \nodata                           & 0.97                          & $-$16.33                            & $-$17.37                            & $-$0.45                         & 18.7                      & 8.0                       & 29.7                         & 1.62E+22                                & 1                        \\
                    3277                    & 17376$+$0267$-$0111          & 173.7583                & 2.6667                  & 1,  3,  6                                & \nodata                           & 1.69                          & $-$11.12                            & $-$11.88                            & $-$0.53                         & 5.4                       & 1.7                       & 14.0                         & 1.18E+21                                & 1                        \\
                    3278                    & 17471$+$0199$-$0211          & 174.7083                & 1.9917                  & 6                                        & \nodata                           & 1.68                          & $-$21.07                            & $-$21.42                            & $-$0.30                         & 8.9                       & 3.8                       & 16.1                         & 2.59E+21                                & 1                        \\
                    3279                    & 17508$+$0384$-$0162          & 175.0833                & 3.8417                  & 6                                        & \nodata                           & 0.98                          & $-$16.18                            & $-$16.60                            & $-$0.53                         & 4.6                       & 1.3                       & 9.1                          & 4.46E+20                                & 1                        \\
                    3280                    & 17627$+$0168$-$0092          & 176.2667                & 1.6833                  & 6                                        & \nodata                           & 1.68                          & $-$9.20                             & $-$9.85                             & $-$0.66                         & 4.7                       & 1.2                       & 11.0                         & 5.35E+20                                & 1                        \\
                    3281                    & 17635$+$0190$-$0099          & 176.3500                & 1.9000                  & 6                                        & \nodata                           & 1.68                          & $-$9.86                             & $-$10.46                            & $-$0.45                         & 3.9                       & 1.3                       & 7.5                          & 7.78E+20                                & 1                        \\
                    3282                    & 17642$+$0017$-$0182          & 176.4250                & 0.1667                  & 3,  6                                    & \nodata                           & 2.09                          & $-$18.25                            & $-$18.94                            & $-$0.29                         & 7.0                       & 1.6                       & 12.9                         & 1.90E+21                                & 1                        \\
                    3283                    & 17649$+$0015$-$0189          & 176.4917                & 0.1500                  & 3,  6                                    & \nodata                           & 2.09                          & $-$18.88                            & $-$20.04                            & $-$0.51                         & 10.1                      & 3.5                       & 19.2                         & 4.97E+21                                & 1                        \\
                    3284                    & 17650$+$0021$-$0183          & 176.5000                & 0.2083                  & 3,  6                                    & \nodata                           & 2.09                          & $-$18.30                            & $-$19.53                            & $-$0.58                         & 11.0                      & 6.2                       & 18.8                         & 8.95E+21                                & 1                        \\
                    3285                    & 17659$-$0035$-$0187          & 176.5917                & $-$0.3500                 & 6                                        & \nodata                           & 2.10                          & $-$18.70                            & $-$19.19                            & $-$0.43                         & 6.6                       & 1.1                       & 12.1                         & 5.79E+20                                & 1                        \\
                    3286                    & 17692$-$0043$+$0045          & 176.9167                & $-$0.4333                 & 6                                        & \nodata                           & 0.23                          & 4.51                              & 4.03                              & $-$0.39                         & 5.0                       & 1.2                       & 9.4                          & 6.38E+20                                & 1                        \\
                    3287                    & 17722$-$0127$-$0169          & 177.2167                & $-$1.2667                 & 3,  6                                    & \nodata                           & 2.14                          & $-$16.86                            & $-$17.24                            & $-$0.33                         & 11.3                      & 4.4                       & 17.7                         & 3.07E+21                                & 1                        \\
                    3288                    & 17730$-$0135$-$0184          & 177.3000                & $-$1.3500                 & 1,  3,  6                                & \nodata                           & 2.08                          & $-$18.43                            & $-$19.29                            & $-$0.54                         & 13.7                      & 5.7                       & 19.0                         & 6.07E+21                                & 1                        \\
                    3289                    & 17874$+$0118$-$0182          & 178.7417                & 1.1833                  & 6                                        & \nodata                           & 1.99                          & $-$18.17                            & $-$18.79                            & $-$0.46                         & 9.5                       & 3.6                       & 15.7                         & 2.81E+21                                & 1                        \\
                    3290                    & 17875$+$0437$+$0008          & 178.7500                & 4.3750                  & 5,  6                                    & \nodata                           & 0.23                          & 0.77                              & 0.25                              & $-$0.42                         & 6.8                       & 3.0                       & 15.0                         & 2.01E+21                                & 1                        \\
                    3291                    & 17883$-$0257$+$0045          & 178.8333                & $-$2.5667                 & 6                                        & \nodata                           & 0.23                          & 4.49                              & 3.86                              & $-$0.55                         & 5.6                       & 1.5                       & 13.0                         & 8.37E+20                                & 1                        \\
                    3292                    & 17887$+$0442$+$0008          & 178.8667                & 4.4250                  & 5,  6                                    & \nodata                           & 0.23                          & 0.85                              & 0.10                              & $-$0.50                         & 7.1                       & 2.2                       & 13.8                         & 1.67E+21                                & 1                        \\
                    3293                    & 17888$-$0248$+$0041          & 178.8833                & $-$2.4833                 & 6                                        & \nodata                           & 0.23                          & 4.11                              & 3.65                              & $-$0.53                         & 6.1                       & 2.4                       & 11.8                         & 1.04E+21                                & 1                        \\
                    3294                    & 17891$+$0434$+$0002          & 178.9083                & 4.3417                  & 3,  5,  6                                & \nodata                           & 0.23                          & 0.22                              & $-$0.55                             & $-$0.69                         & 7.0                       & 2.4                       & 19.3                         & 1.56E+21                                & 1                        \\
                    3295                    & 17897$-$0052$-$0010          & 178.9750                & $-$0.5250                 & 4,  6                                    & \nodata                           & 0.23                          & $-$0.97                             & $-$1.56                             & $-$0.41                         & 5.3                       & 2.0                       & 13.0                         & 1.41E+21                                & 1                        \\
                    3296                    & 17898$+$0427$+$0004          & 178.9833                & 4.2667                  & 3,  6                                    & \nodata                           & 0.23                          & 0.45                              & $-$0.22                             & $-$0.54                         & 7.3                       & 4.2                       & 14.1                         & 3.01E+21                                & 1                        \\
                    3297                    & 17898$+$0434$+$0006          & 178.9833                & 4.3417                  & 3,  6                                    & \nodata                           & 0.23                          & 0.63                              & 0.07                              & $-$0.33                         & 7.9                       & 4.1                       & 18.3                         & 4.25E+21                                & 1                        \\
                    3298                    & 17901$-$0253$+$0046          & 179.0083                & $-$2.5333                 & 6                                        & \nodata                           & 0.23                          & 4.57                              & 4.27                              & $-$0.30                         & 11.6                      & 3.9                       & 18.8                         & 2.43E+21                                & 1                        \\
                    3299                    & 17902$+$0430$+$0006          & 179.0167                & 4.3000                  & 3,  6                                    & \nodata                           & 0.23                          & 0.63                              & $-$0.42                             & $-$0.50                         & 10.9                      & 5.0                       & 23.6                         & 7.39E+21                                & 1                        \\
                    3300                    & 17903$+$0437$+$0008          & 179.0333                & 4.3667                  & 3,  6                                    & \nodata                           & 0.23                          & 0.75                              & $-$0.06                             & $-$0.50                         & 8.3                       & 2.8                       & 12.5                         & 2.29E+21                                & 1                        \\
                    3301                    & 17904$-$0251$+$0048          & 179.0417                & $-$2.5083                 & 6                                        & \nodata                           & 0.23                          & 4.77                              & 4.31                              & $-$0.62                         & 8.4                       & 4.3                       & 15.2                         & 1.87E+21                                & 1                        \\
                    3302                    & 17907$+$0427$+$0005          & 179.0667                & 4.2667                  & 3,  6                                    & \nodata                           & 0.23                          & 0.52                              & $-$0.03                             & $-$0.30                         & 7.8                       & 3.8                       & 15.9                         & 3.90E+21                                & 1                        \\
                    3303                    & 17917$+$0436$+$0013          & 179.1750                & 4.3583                  & 3,  5,  6                                & \nodata                           & 0.23                          & 1.27                              & 0.35                              & $-$0.71                         & 5.2                       & 1.7                       & 12.1                         & 1.07E+21                                & 1                        \\
                    3304                    & 17920$-$0236$+$0052          & 179.2000                & $-$2.3583                 & 4,  6                                    & \nodata                           & 0.23                          & 5.19                              & 4.90                              & $-$0.39                         & 8.3                       & 3.6                       & 13.5                         & 1.47E+21                                & 1                        \\
                    3305                    & 17924$-$0385$+$0073          & 179.2417                & $-$3.8500                 & 6                                        & \nodata                           & 0.23                          & 7.29                              & 6.78                              & $-$0.53                         & 6.0                       & 1.1                       & 14.5                         & 5.17E+20                                & 1                        \\
                    3306                    & 17953$-$0282$+$0018          & 179.5333                & $-$2.8167                 & 6                                        & \nodata                           & 0.23                          & 1.81                              & 1.37                              & $-$0.37                         & 9.0                       & 2.1                       & 14.8                         & 1.26E+21                                & 1                        \\
                    3307                    & 17957$+$0415$+$0004          & 179.5750                & 4.1500                  & 6                                        & \nodata                           & 0.23                          & 0.45                              & $-$0.63                             & $-$1.00                         & 6.3                       & 1.4                       & 12.2                         & 6.97E+20                                & 1                        \\
                    3308                    & 18027$+$0396$+$0022          & 180.2667                & 3.9583                  & 6                                        & \nodata                           & 0.23                          & 2.21                              & 1.62                              & $-$0.38                         & 5.8                       & 1.3                       & 11.8                         & 9.43E+20                                & 1                        \\
                    3309                    & 18032$+$0419$+$0019          & 180.3167                & 4.1917                  & 6                                        & \nodata                           & 0.23                          & 1.89                              & 0.87                              & $-$0.56                         & 6.0                       & 1.9                       & 10.5                         & 1.68E+21                                & 1                        \\
                    3310                    & 18039$+$0415$+$0020          & 180.3917                & 4.1500                  & 6                                        & \nodata                           & 0.23                          & 2.00                              & 0.78                              & $-$0.76                         & 6.3                       & 2.4                       & 15.9                         & 2.04E+21                                & 1                        \\
                    3311                    & 18039$+$0426$+$0014          & 180.3917                & 4.2583                  & 6                                        & \nodata                           & 0.23                          & 1.38                              & 0.67                              & $-$0.64                         & 6.3                       & 1.5                       & 10.5                         & 7.50E+20                                & 1                        \\
                    3312                    & 18041$+$0417$+$0021          & 180.4083                & 4.1750                  & 6                                        & \nodata                           & 0.23                          & 2.10                              & 0.73                              & $-$0.82                         & 6.9                       & 2.3                       & 13.6                         & 1.94E+21                                & 1                        \\
                    3313                    & 18041$-$0193$-$0061          & 180.4083                & $-$1.9333                 & 4,  6                                    & \nodata                           & 2.07                          & $-$6.07                             & $-$6.38                             & $-$0.34                         & 5.8                       & 1.3                       & 13.0                         & 5.63E+20                                & 1                        \\
                    3314                    & 18045$+$0419$+$0022          & 180.4500                & 4.1917                  & 6                                        & \nodata                           & 0.23                          & 2.21                              & 1.03                              & $-$0.68                         & 7.7                       & 1.5                       & 15.0                         & 1.33E+21                                & 1                        \\
                    3315                    & 18051$+$0424$+$0023          & 180.5083                & 4.2417                  & 6                                        & \nodata                           & 0.23                          & 2.30                              & 1.49                              & $-$0.70                         & 7.7                       & 2.4                       & 14.3                         & 1.48E+21                                & 1                        \\
                    3316                    & 18053$-$0199$-$0065          & 180.5250                & $-$1.9917                 & 4,  6                                    & \nodata                           & 2.07                          & $-$6.51                             & $-$6.98                             & $-$0.43                         & 8.8                       & 2.9                       & 17.2                         & 1.75E+21                                & 1                        \\
                    3317                    & 18123$+$0439$+$0019          & 181.2250                & 4.3917                  & 5,  6                                    & \nodata                           & 0.23                          & 1.85                              & 1.14                              & $-$0.35                         & 5.5                       & 2.1                       & 10.6                         & 2.08E+21                                & 1                        \\
                    3318                    & 18123$+$0434$+$0015          & 181.2333                & 4.3417                  & 5,  6                                    & \nodata                           & 0.23                          & 1.46                              & 0.64                              & $-$0.40                         & 7.2                       & 2.7                       & 13.6                         & 2.82E+21                                & 1                        \\
                    3319                    & 18125$+$0437$+$0016          & 181.2500                & 4.3667                  & 5,  6                                    & \nodata                           & 0.23                          & 1.65                              & 0.81                              & $-$0.46                         & 6.4                       & 3.2                       & 12.1                         & 3.03E+21                                & 1                        \\
                    3320                    & 18144$-$0430$+$0070          & 181.4417                & $-$4.3000                 & 6                                        & \nodata                           & 1.60                          & 6.96                              & 6.52                              & $-$0.63                         & 6.6                       & 2.3                       & 13.7                         & 8.09E+20                                & 1                        \\
                    3321                    & 18145$+$0397$+$0013          & 181.4500                & 3.9667                  & 5,  6                                    & \nodata                           & 0.23                          & 1.29                              & 0.52                              & $-$0.52                         & 6.4                       & 1.5                       & 13.8                         & 1.10E+21                                & 1                        \\
                    3322                    & 18157$-$0432$+$0070          & 181.5667                & $-$4.3167                 & 6                                        & \nodata                           & 1.60                          & 7.01                              & 6.45                              & $-$0.56                         & 8.6                       & 1.0                       & 14.8                         & 4.97E+20                                & 1                        \\
                    3323                    & 18163$+$0411$+$0034          & 181.6250                & 4.1083                  & 5,  6                                    & \nodata                           & 0.23                          & 3.39                              & 2.28                              & $-$0.58                         & 5.8                       & 1.8                       & 12.6                         & 1.68E+21                                & 1                        \\
                    3324                    & 18168$-$0515$+$0060          & 181.6833                & $-$5.1500                 & 6                                        & \nodata                           & 1.60                          & 5.96                              & 5.49                              & $-$0.52                         & 7.5                       & 2.2                       & 13.6                         & 1.00E+21                                & 1                        \\
                    3325                    & 18173$-$0440$+$0070          & 181.7333                & $-$4.4000                 & 6                                        & \nodata                           & 1.60                          & 6.99                              & 6.56                              & $-$0.49                         & 8.9                       & 1.3                       & 18.7                         & 6.20E+20                                & 1                        \\
                    3326                    & 18192$-$0082$-$0107          & 181.9167                & $-$0.8250                 & 6                                        & \nodata                           & 2.07                          & $-$10.71                            & $-$11.35                            & $-$0.56                         & 4.8                       & 1.5                       & 9.2                          & 7.84E+20                                & 1                        \\
                    3327                    & 18202$+$0028$+$0014          & 182.0167                & 0.2833                  & 6                                        & \nodata                           & 1.59                          & 1.43                              & 0.75                              & $-$0.65                         & 5.4                       & 3.1                       & 12.0                         & 1.74E+21                                & 1                        \\
                    3328                    & 18203$+$0028$+$0015          & 182.0333                & 0.2833                  & 6                                        & \nodata                           & 1.59                          & 1.49                              & 0.96                              & $-$0.54                         & 5.8                       & 3.0                       & 11.5                         & 1.57E+21                                & 1                        \\
                    3329                    & 18204$-$0016$+$0035          & 182.0417                & $-$0.1583                 & 1,  6                                    & \nodata                           & 1.59                          & 3.51                              & 2.84                              & $-$0.50                         & 7.7                       & 4.1                       & 14.1                         & 3.12E+21                                & 1                        \\
                    3330                    & 18206$+$0040$+$0023          & 182.0583                & 0.4000                  & 1,  6                                    & \nodata                           & 1.59                          & 2.33                              & 1.89                              & $-$0.38                         & 4.4                       & 3.9                       & 8.8                          & 3.15E+21                                & 1                        \\
                    3331                    & 18207$+$0462$+$0046          & 182.0667                & 4.6250                  & 6                                        & \nodata                           & 0.23                          & 4.58                              & 3.81                              & $-$0.67                         & 4.8                       & 1.0                       & 9.2                          & 5.22E+20                                & 1                        \\
                    3332                    & 18227$-$0088$-$0104          & 182.2667                & $-$0.8833                 & 6                                        & \nodata                           & 2.07                          & $-$10.40                            & $-$11.50                            & $-$0.66                         & 5.6                       & 1.8                       & 12.4                         & 1.45E+21                                & 1                        \\
                    3333                    & 18233$+$0023$+$0021          & 182.3250                & 0.2333                  & 3,  6                                    & \nodata                           & 1.59                          & 2.12                              & 1.02                              & $-$0.56                         & 9.1                       & 5.0                       & 16.0                         & 6.09E+21                                & 1                        \\
                    3334                    & 18286$-$0097$-$0101          & 182.8583                & $-$0.9750                 & 6                                        & \nodata                           & 2.07                          & $-$10.08                            & $-$10.51                            & $-$0.42                         & 5.5                       & 2.4                       & 10.0                         & 1.24E+21                                & 1                        \\
                    3335                    & 18288$-$0095$-$0108          & 182.8750                & $-$0.9500                 & 6                                        & \nodata                           & 2.07                          & $-$10.78                            & $-$11.70                            & $-$0.63                         & 6.7                       & 2.7                       & 12.8                         & 2.03E+21                                & 1                        \\
                    3336                    & 18323$-$0244$+$0018          & 183.2250                & $-$2.4417                 & 6                                        & \nodata                           & 1.59                          & 1.80                              & 1.23                              & $-$0.49                         & 6.1                       & 1.6                       & 13.4                         & 9.12E+20                                & 1                        \\
                    3337                    & 18323$-$0249$+$0017          & 183.2250                & $-$2.4917                 & 6                                        & \nodata                           & 1.59                          & 1.75                              & 0.90                              & $-$0.44                         & 6.0                       & 2.2                       & 14.2                         & 2.14E+21                                & 1                        \\
                    3338                    & 18326$-$0077$-$0084          & 183.2583                & $-$0.7750                 & 6                                        & \nodata                           & 1.59                          & $-$8.36                             & $-$8.85                             & $-$0.48                         & 6.5                       & 1.3                       & 13.4                         & 6.19E+20                                & 1                        \\
                    3339                    & 18333$+$0480$+$0037          & 183.3333                & 4.8000                  & 5,  6                                    & \nodata                           & 0.23                          & 3.72                              & 3.15                              & $-$0.45                         & 4.7                       & 1.0                       & 10.2                         & 5.42E+20                                & 1                        \\
                    3340                    & 18334$-$0059$-$0090          & 183.3417                & $-$0.5917                 & 1,  6                                    & \nodata                           & 2.05                          & $-$8.98                             & $-$9.89                             & $-$0.45                         & 14.8                      & 6.9                       & 25.0                         & 1.08E+22                                & 1                        \\
                    3341                    & 18337$+$0482$+$0041          & 183.3667                & 4.8167                  & 5,  6                                    & \nodata                           & 0.23                          & 4.13                              & 3.58                              & $-$0.60                         & 3.8                       & 0.9                       & 11.6                         & 3.79E+20                                & 1                        \\
                    3342                    & 18338$-$0057$-$0092          & 183.3750                & $-$0.5750                 & 6                                        & \nodata                           & 2.05                          & $-$9.25                             & $-$10.39                            & $-$0.75                         & 14.5                      & 8.6                       & 22.7                         & 1.04E+22                                & 1                        \\
                    3343                    & 18338$-$0235$+$0029          & 183.3833                & $-$2.3500                 & 6                                        & \nodata                           & 1.59                          & 2.86                              & 2.14                              & $-$0.69                         & 4.6                       & 2.1                       & 9.0                          & 1.07E+21                                & 1                        \\
                    3344                    & 18345$-$0062$-$0084          & 183.4500                & $-$0.6167                 & 6                                        & \nodata                           & 1.59                          & $-$8.44                             & $-$8.96                             & $-$0.47                         & 9.0                       & 3.2                       & 13.4                         & 1.87E+21                                & 1                        \\
                    3345                    & 18346$-$0294$+$0013          & 183.4583                & $-$2.9417                 & 6                                        & \nodata                           & 1.59                          & 1.30                              & 0.77                              & $-$0.47                         & 4.9                       & 2.2                       & 9.6                          & 1.23E+21                                & 1                        \\
                    3346                    & 18346$-$0302$+$0017          & 183.4583                & $-$3.0250                 & 6                                        & \nodata                           & 1.59                          & 1.65                              & 0.95                              & $-$0.52                         & 4.8                       & 1.4                       & 10.9                         & 8.69E+20                                & 1                        \\
                    3347                    & 18356$-$0360$+$0015          & 183.5583                & $-$3.6000                 & 6                                        & \nodata                           & 1.59                          & 1.47                              & 0.76                              & $-$0.52                         & 4.6                       & 2.4                       & 10.1                         & 1.62E+21                                & 1                        \\
                    3348                    & 18385$-$0272$+$0008          & 183.8500                & $-$2.7250                 & 6                                        & \nodata                           & 1.59                          & 0.83                              & 0.11                              & $-$0.65                         & 5.4                       & 1.5                       & 9.9                          & 7.63E+20                                & 1                        \\
                    3349                    & 18398$+$0503$-$0012          & 183.9750                & 5.0333                  & 5,  6                                    & \nodata                           & 0.23                          & $-$1.21                             & $-$1.96                             & $-$0.66                         & 4.1                       & 1.6                       & 7.8                          & 9.02E+20                                & 1                        \\
                    3350                    & 18413$-$0452$+$0017          & 184.1333                & $-$4.5167                 & 6                                        & \nodata                           & 1.59                          & 1.66                              & 0.94                              & $-$0.74                         & 6.6                       & 1.2                       & 12.1                         & 5.38E+20                                & 1                        \\
                    3351                    & 18433$+$0033$-$0016          & 184.3250                & 0.3333                  & 6                                        & \nodata                           & 1.59                          & $-$1.61                             & $-$2.39                             & $-$0.55                         & 5.1                       & 1.4                       & 11.7                         & 9.13E+20                                & 1                        \\
                    3352                    & 18435$+$0032$-$0017          & 184.3500                & 0.3250                  & 6                                        & \nodata                           & 1.59                          & $-$1.72                             & $-$2.24                             & $-$0.45                         & 5.1                       & 1.6                       & 10.9                         & 8.71E+20                                & 1                        \\
                    3353                    & 18443$+$0038$-$0014          & 184.4250                & 0.3833                  & 6                                        & \nodata                           & 1.59                          & $-$1.42                             & $-$2.32                             & $-$0.86                         & 5.1                       & 1.9                       & 10.8                         & 9.47E+20                                & 1                        \\
                    3354                    & 18467$+$0421$+$0031          & 184.6667                & 4.2083                  & 6                                        & \nodata                           & 0.23                          & 3.13                              & 2.85                              & $-$0.26                         & 3.3                       & 1.7                       & 8.9                          & 8.71E+20                                & 1                        \\
                    3355                    & 18474$-$0399$+$0021          & 184.7417                & $-$3.9917                 & 1,  6                                    & \nodata                           & 0.50                          & 2.13                              & 1.03                              & $-$0.94                         & 3.8                       & 2.5                       & 8.4                          & 1.56E+21                                & 1                        \\
                    3356                    & 18477$-$0412$+$0020          & 184.7667                & $-$4.1167                 & 6                                        & \nodata                           & 0.50                          & 2.02                              & 1.34                              & $-$0.56                         & 3.3                       & 2.1                       & 7.7                          & 1.37E+21                                & 1                        \\
                    3357                    & 18479$-$0384$+$0030          & 184.7917                & $-$3.8417                 & 6                                        & \nodata                           & 0.50                          & 2.97                              & 2.26                              & $-$0.73                         & 4.2                       & 3.4                       & 10.1                         & 1.83E+21                                & 1                        \\
                    3358                    & 18479$-$0390$+$0027          & 184.7917                & $-$3.9000                 & 6                                        & \nodata                           & 0.50                          & 2.75                              & 2.29                              & $-$0.42                         & 3.2                       & 2.5                       & 7.4                          & 1.66E+21                                & 1                        \\
                    3359                    & 18481$-$0405$+$0022          & 184.8083                & $-$4.0500                 & 6                                        & \nodata                           & 0.50                          & 2.20                              & 1.81                              & $-$0.45                         & 3.1                       & 1.5                       & 9.0                          & 6.13E+20                                & 1                        \\
                    3360                    & 18482$-$0387$+$0028          & 184.8167                & $-$3.8750                 & 6                                        & \nodata                           & 0.50                          & 2.85                              & 2.41                              & $-$0.39                         & 4.7                       & 4.1                       & 10.1                         & 2.77E+21                                & 1                        \\
                    3361                    & 18485$-$0417$+$0026          & 184.8500                & $-$4.1667                 & 6                                        & \nodata                           & 0.62                          & 2.58                              & 1.84                              & $-$0.70                         & 4.4                       & 2.0                       & 9.9                          & 1.03E+21                                & 1                        \\
                    3362                    & 18491$-$0176$+$0021          & 184.9083                & $-$1.7583                 & 6                                        & \nodata                           & 1.59                          & 2.11                              & 1.43                              & $-$0.53                         & 5.5                       & 1.8                       & 11.0                         & 1.08E+21                                & 1                        \\
                    3363                    & 18494$-$0084$+$0060          & 184.9417                & $-$0.8417                 & 3,  6                                    & \nodata                           & 1.59                          & 5.99                              & 5.32                              & $-$0.69                         & 5.4                       & 2.3                       & 10.3                         & 1.07E+21                                & 1                        \\
                    3364                    & 18500$-$0357$+$0019          & 185.0000                & $-$3.5667                 & 6                                        & \nodata                           & 0.62                          & 1.89                              & 1.51                              & $-$0.42                         & 2.5                       & 2.5                       & 7.9                          & 1.26E+21                                & 1                        \\
                    3365                    & 18504$-$0364$+$0022          & 185.0417                & $-$3.6417                 & 6                                        & \nodata                           & 0.62                          & 2.25                              & 1.85                              & $-$0.43                         & 3.8                       & 2.1                       & 8.3                          & 1.00E+21                                & 1                        \\
                    3366                    & 18606$-$0208$+$0011          & 186.0583                & $-$2.0833                 & 6                                        & \nodata                           & 1.59                          & 1.13                              & 0.50                              & $-$0.64                         & 4.4                       & 1.8                       & 11.0                         & 8.51E+20                                & 1                        \\
                    3367                    & 18606$-$0244$+$0012          & 186.0583                & $-$2.4417                 & 6                                        & \nodata                           & 0.53                          & 1.23                              & 0.30                              & $-$0.73                         & 7.1                       & 2.5                       & 11.6                         & 1.59E+21                                & 1                        \\
                    3368                    & 18633$-$0364$-$0014          & 186.3250                & $-$3.6417                 & 1,  6                                    & \nodata                           & 0.52                          & $-$1.35                             & $-$1.75                             & $-$0.43                         & 6.9                       & 2.8                       & 12.2                         & 1.30E+21                                & 1                        \\
                    3369                    & 18683$-$0522$+$0034          & 186.8250                & $-$5.2167                 & 6                                        & \nodata                           & 0.51                          & 3.44                              & 2.64                              & $-$0.49                         & 5.3                       & 1.3                       & 14.0                         & 1.07E+21                                & 1                        \\
                    3370                    & 18695$-$0382$+$0008          & 186.9500                & $-$3.8167                 & 6                                        & \nodata                           & 0.23                          & 0.84                              & 0.16                              & $-$0.37                         & 5.9                       & 3.7                       & 11.0                         & 3.91E+21                                & 1                        \\
                    3371                    & 18715$-$0010$+$0022          & 187.1500                & $-$0.1000                 & 6                                        & \nodata                           & 2.10                          & 2.23                              & 1.66                              & $-$0.81                         & 4.1                       & 1.2                       & 8.5                          & 3.67E+20                                & 1                        \\
                    3372                    & 18717$-$0088$+$0015          & 187.1667                & $-$0.8833                 & 6                                        & \nodata                           & 1.59                          & 1.52                              & 1.03                              & $-$0.60                         & 5.6                       & 1.9                       & 10.8                         & 7.50E+20                                & 1                        \\
                    3373                    & 18718$+$0002$+$0023          & 187.1750                & 0.0167                  & 6                                        & \nodata                           & 2.10                          & 2.34                              & 1.80                              & $-$0.67                         & 4.4                       & 1.6                       & 10.3                         & 6.02E+20                                & 1                        \\
                    3374                    & 18727$-$0132$+$0018          & 187.2667                & $-$1.3167                 & 6                                        & \nodata                           & 1.59                          & 1.79                              & 1.40                              & $-$0.31                         & 5.1                       & 1.7                       & 9.6                          & 9.56E+20                                & 1                        \\
                    3375                    & 18791$-$0472$-$0028          & 187.9083                & $-$4.7167                 & 6                                        & \nodata                           & 0.23                          & $-$2.78                             & $-$3.42                             & $-$0.60                         & 5.3                       & 1.1                       & 10.3                         & 5.04E+20                                & 1                        \\
                    3376                    & 18801$-$0368$+$0030          & 188.0083                & $-$3.6833                 & 6                                        & \nodata                           & 0.23                          & 2.95                              & 2.56                              & $-$0.35                         & 3.4                       & 2.8                       & 9.8                          & 1.66E+21                                & 1                        \\
                    3377                    & 18803$-$0369$+$0028          & 188.0333                & $-$3.6917                 & 6                                        & \nodata                           & 0.23                          & 2.78                              & 2.25                              & $-$0.41                         & 4.3                       & 2.1                       & 10.9                         & 1.29E+21                                & 1                        \\
                    3378                    & 18806$-$0368$+$0027          & 188.0583                & $-$3.6833                 & 6                                        & \nodata                           & 0.23                          & 2.71                              & 2.28                              & $-$0.48                         & 3.3                       & 1.9                       & 7.3                          & 9.09E+20                                & 1                        \\
                    3379                    & 18832$-$0380$+$0026          & 188.3167                & $-$3.8000                 & 6                                        & \nodata                           & 0.23                          & 2.58                              & 2.18                              & $-$0.50                         & 4.5                       & 1.5                       & 10.6                         & 5.29E+20                                & 1                        \\
                    3380                    & 18843$+$0359$-$0044          & 188.4333                & 3.5917                  & 3,  6                                    & \nodata                           & 1.91                          & $-$4.38                             & $-$5.10                             & $-$0.53                         & 7.0                       & 1.5                       & 12.0                         & 9.42E+20                                & 1                        \\
                    3381                    & 18851$+$0025$+$0011          & 188.5083                & 0.2500                  & 6                                        & \nodata                           & 1.86                          & 1.09                              & 0.62                              & $-$0.45                         & 4.4                       & 1.3                       & 8.9                          & 5.89E+20                                & 1                        \\
                    3382                    & 18873$+$0283$-$0036          & 188.7250                & 2.8333                  & 4,  6                                    & \nodata                           & 1.91                          & $-$3.61                             & $-$4.17                             & $-$0.56                         & 4.4                       & 1.8                       & 10.8                         & 8.54E+20                                & 1                        \\
                    3383                    & 18881$+$0100$-$0001          & 188.8083                & 1.0000                  & 3,  6                                    & \nodata                           & 1.91                          & $-$0.15                             & $-$1.01                             & $-$0.45                         & 12.6                      & 3.7                       & 22.5                         & 4.71E+21                                & 1                        \\
                    3384                    & 18882$+$0100$-$0002          & 188.8167                & 1.0000                  & 3,  6                                    & \nodata                           & 1.91                          & $-$0.17                             & $-$0.98                             & $-$0.39                         & 12.6                      & 3.3                       & 25.7                         & 4.90E+21                                & 1                        \\
                    3385                    & 18882$+$0110$+$0001          & 188.8167                & 1.1000                  & 3,  6                                    & \nodata                           & 1.92                          & 0.10                              & $-$0.91                             & $-$0.43                         & 9.8                       & 4.2                       & 18.7                         & 6.16E+21                                & 1                        \\
                    3386                    & 18893$-$0196$-$0012          & 188.9333                & $-$1.9583                 & 6                                        & \nodata                           & 1.58                          & $-$1.18                             & $-$1.66                             & $-$0.38                         & 6.7                       & 3.5                       & 14.6                         & 2.42E+21                                & 1                        \\
                    3387                    & 18894$+$0142$+$0019          & 188.9417                & 1.4250                  & 6                                        & \nodata                           & 1.92                          & 1.87                              & 0.99                              & $-$0.39                         & 7.8                       & 2.0                       & 17.3                         & 2.44E+21                                & 1                        \\
                    3388                    & 18905$+$0110$+$0020          & 189.0500                & 1.1000                  & 6                                        & \nodata                           & 1.88                          & 1.98                              & 0.86                              & $-$0.42                         & 10.4                      & 5.8                       & 18.6                         & 1.03E+22                                & 1                        \\
                    3389                    & 18908$+$0071$+$0026          & 189.0833                & 0.7083                  & 3,  6                                    & \nodata                           & 1.85                          & 2.59                              & 1.63                              & $-$0.52                         & 14.1                      & 6.4                       & 20.0                         & 8.19E+21                                & 1                        \\
                    3390                    & 18911$+$0064$+$0026          & 189.1083                & 0.6417                  & 1,  3,  6                                & \nodata                           & 1.84                          & 2.59                              & 1.87                              & $-$0.35                         & 13.1                      & 6.2                       & 24.5                         & 9.43E+21                                & 1                        \\
                    3391                    & 18930$+$0416$-$0036          & 189.3000                & 4.1583                  & 4,  6                                    & \nodata                           & 1.80                          & $-$3.63                             & $-$4.26                             & $-$0.67                         & 6.0                       & 1.6                       & 11.7                         & 7.17E+20                                & 1                        \\
                    3392                    & 18932$+$0005$+$0017          & 189.3167                & 0.0500                  & 6                                        & \nodata                           & 1.71                          & 1.69                              & 1.11                              & $-$0.49                         & 7.2                       & 1.5                       & 13.2                         & 8.86E+20                                & 1                        \\
                    3393                    & 18934$+$0007$+$0015          & 189.3417                & 0.0667                  & 6                                        & \nodata                           & 1.71                          & 1.52                              & 0.75                              & $-$0.44                         & 8.1                       & 1.8                       & 15.1                         & 1.63E+21                                & 1                        \\
                    3394                    & 18936$+$0161$+$0046          & 189.3583                & 1.6083                  & 6                                        & \nodata                           & 1.89                          & 4.61                              & 3.93                              & $-$0.50                         & 3.8                       & 2.5                       & 10.6                         & 1.72E+21                                & 1                        \\
                    3395                    & 18943$+$0157$+$0044          & 189.4250                & 1.5750                  & 6                                        & \nodata                           & 1.88                          & 4.42                              & 3.84                              & $-$0.64                         & 4.0                       & 1.0                       & 10.9                         & 4.01E+20                                & 1                        \\
                    3396                    & 18960$-$0006$+$0078          & 189.6000                & $-$0.0583                 & 6                                        & \nodata                           & 1.73                          & 7.77                              & 7.01                              & $-$0.36                         & 10.8                      & 3.7                       & 18.5                         & 4.76E+21                                & 1                        \\
                    3397                    & 18963$+$0007$+$0078          & 189.6333                & 0.0667                  & 2,  6                                    & \nodata                           & 1.73                          & 7.85                              & 6.91                              & $-$0.29                         & 12.1                      & 4.1                       & 18.8                         & 8.29E+21                                & 1                        \\
                    3398                    & 18966$+$0024$+$0068          & 189.6583                & 0.2417                  & 2,  3,  6                                & \nodata                           & 1.74                          & 6.83                              & 5.97                              & $-$0.35                         & 10.3                      & 3.5                       & 19.4                         & 5.29E+21                                & 1                        \\
                    3399                    & 18967$+$0017$+$0074          & 189.6667                & 0.1667                  & 1,  2,  3,  6                            & 17                                & 1.74                          & 7.37                              & 6.05                              & $-$0.50                         & 12.8                      & 7.4                       & 22.0                         & 1.44E+22                                & 1                        \\
                    3400                    & 18968$+$0027$+$0071          & 189.6833                & 0.2667                  & 1,  2,  3,  6                            & \nodata                           & 1.74                          & 7.07                              & 6.13                              & $-$0.47                         & 11.1                      & 4.2                       & 22.6                         & 5.69E+21                                & 1                        \\
                    3401                    & 18984$+$0285$-$0011          & 189.8417                & 2.8500                  & 6                                        & \nodata                           & 1.72                          & $-$1.11                             & $-$1.94                             & $-$0.67                         & 5.4                       & 1.4                       & 10.1                         & 7.93E+20                                & 1                        \\
                    3402                    & 19008$+$0237$-$0017          & 190.0750                & 2.3667                  & 6                                        & \nodata                           & 1.71                          & $-$1.67                             & $-$2.13                             & $-$0.40                         & 4.4                       & 1.1                       & 10.6                         & 5.65E+20                                & 1                        \\
                    3403                    & 19024$-$0232$+$0010          & 190.2417                & $-$2.3250                 & 6                                        & \nodata                           & 1.57                          & 1.04                              & 0.12                              & $-$0.49                         & 6.1                       & 2.4                       & 10.3                         & 2.19E+21                                & 1                        \\
                    3404                    & 19070$-$0043$-$0003          & 190.7000                & $-$0.4333                 & 6                                        & \nodata                           & 1.67                          & $-$0.33                             & $-$0.80                             & $-$0.44                         & 7.6                       & 3.7                       & 14.7                         & 2.17E+21                                & 1                        \\
                    3405                    & 19086$-$0377$+$0052          & 190.8583                & $-$3.7667                 & 6                                        & \nodata                           & 0.24                          & 5.19                              & 4.77                              & $-$0.44                         & 5.5                       & 2.7                       & 10.6                         & 1.28E+21                                & 1                        \\
                    3406                    & 19087$-$0401$-$0001          & 190.8667                & $-$4.0083                 & 6                                        & \nodata                           & 0.24                          & $-$0.07                             & $-$0.33                             & $-$0.21                         & 5.9                       & 2.0                       & 13.8                         & 1.24E+21                                & 1                        \\
                    3407                    & 19099$-$0026$+$0111          & 190.9917                & $-$0.2583                 & 6                                        & \nodata                           & 1.70                          & 11.13                             & 10.40                             & $-$0.51                         & 7.2                       & 3.6                       & 11.7                         & 2.84E+21                                & 1                        \\
                    3408                    & 19161$+$0083$+$0027          & 191.6083                & 0.8333                  & 6                                        & \nodata                           & 1.70                          & 2.75                              & 1.98                              & $-$0.59                         & 4.3                       & 1.0                       & 10.9                         & 5.65E+20                                & 1                        \\
                    3409                    & 19178$+$0081$+$0072          & 191.7833                & 0.8083                  & 6                                        & \nodata                           & 1.68                          & 7.16                              & 6.74                              & $-$0.41                         & 9.7                       & 3.2                       & 14.3                         & 1.76E+21                                & 1                        \\
                    3410                    & 19183$-$0076$-$0013          & 191.8333                & $-$0.7583                 & 6                                        & \nodata                           & 1.64                          & $-$1.28                             & $-$1.71                             & $-$0.35                         & 4.8                       & 1.3                       & 9.3                          & 7.40E+20                                & 1                        \\
                    3411                    & 19184$-$0082$-$0011          & 191.8417                & $-$0.8250                 & 6                                        & \nodata                           & 1.64                          & $-$1.10                             & $-$1.94                             & $-$0.51                         & 6.0                       & 2.3                       & 12.3                         & 1.91E+21                                & 1                        \\
                    3412                    & 19187$-$0083$-$0011          & 191.8667                & $-$0.8333                 & 5,  6                                    & \nodata                           & 1.64                          & $-$1.08                             & $-$1.63                             & $-$0.38                         & 6.1                       & 2.7                       & 13.0                         & 2.01E+21                                & 1                        \\
                    3413                    & 19195$-$0157$-$0006          & 191.9500                & $-$1.5667                 & 6                                        & \nodata                           & 1.58                          & $-$0.61                             & $-$1.23                             & $-$0.56                         & 5.3                       & 1.6                       & 10.6                         & 8.27E+20                                & 1                        \\
                    3414                    & 19207$-$0175$-$0008          & 192.0667                & $-$1.7500                 & 6                                        & \nodata                           & 1.59                          & $-$0.78                             & $-$1.53                             & $-$0.36                         & 6.1                       & 2.4                       & 11.4                         & 2.48E+21                                & 1                        \\
                    3415                    & 19229$-$0263$+$0027          & 192.2917                & $-$2.6333                 & 5,  6                                    & \nodata                           & 1.58                          & 2.70                              & 1.73                              & $-$0.75                         & 5.9                       & 3.0                       & 9.7                          & 2.08E+21                                & 1                        \\
                    3416                    & 19260$-$0372$+$0049          & 192.6000                & $-$3.7167                 & 6                                        & \nodata                           & 0.27                          & 4.93                              & 4.34                              & $-$0.37                         & 5.9                       & 2.6                       & 11.8                         & 2.06E+21                                & 1                        \\
                    3417                    & 19261$-$0014$+$0070          & 192.6083                & $-$0.1417                 & 1,  3,  6                                & \nodata                           & 1.68                          & 6.98                              & 6.05                              & $-$0.42                         & 25.4                      & 6.9                       & 32.6                         & 1.36E+22                                & 1                        \\
                    3418                    & 19271$-$0367$+$0053          & 192.7083                & $-$3.6667                 & 5,  6                                    & \nodata                           & 0.27                          & 5.31                              & 4.16                              & $-$0.76                         & 5.1                       & 2.7                       & 13.6                         & 2.14E+21                                & 1                        \\
                    3419                    & 19273$-$0367$+$0052          & 192.7333                & $-$3.6667                 & 5,  6                                    & \nodata                           & 0.27                          & 5.25                              & 4.20                              & $-$0.63                         & 4.7                       & 2.3                       & 10.8                         & 1.85E+21                                & 1                        \\
                    3420                    & 19274$-$0369$+$0052          & 192.7417                & $-$3.6917                 & 5,  6                                    & \nodata                           & 0.28                          & 5.20                              & 4.43                              & $-$0.60                         & 4.4                       & 2.0                       & 9.2                          & 1.26E+21                                & 1                        \\
                    3421                    & 19278$-$0372$+$0042          & 192.7833                & $-$3.7167                 & 5,  6                                    & \nodata                           & 0.28                          & 4.22                              & 3.31                              & $-$0.71                         & 4.4                       & 1.3                       & 9.2                          & 7.81E+20                                & 1                        \\
                    3422                    & 19280$+$0011$+$0089          & 192.8000                & 0.1083                  & 3,  6                                    & \nodata                           & 1.69                          & 8.93                              & 7.86                              & $-$0.47                         & 10.5                      & 4.9                       & 20.2                         & 7.35E+21                                & 1                        \\
                    3423                    & 19292$-$0369$+$0041          & 192.9167                & $-$3.6917                 & 5,  6                                    & \nodata                           & 0.25                          & 4.10                              & 3.66                              & $-$0.35                         & 5.1                       & 2.3                       & 10.8                         & 1.43E+21                                & 1                        \\
                    3424                    & 19298$+$0017$+$0085          & 192.9750                & 0.1667                  & 3,  6                                    & \nodata                           & 1.68                          & 8.48                              & 7.61                              & $-$0.38                         & 14.7                      & 7.2                       & 22.4                         & 1.27E+22                                & 1                        \\
                    3425                    & 19334$-$0141$+$0025          & 193.3417                & $-$1.4083                 & 5,  6                                    & \nodata                           & 1.67                          & 2.49                              & 2.16                              & $-$0.36                         & 7.7                       & 3.2                       & 12.3                         & 1.59E+21                                & 1                        \\
                    3426                    & 19335$-$0144$+$0029          & 193.3500                & $-$1.4417                 & 5,  6                                    & \nodata                           & 1.67                          & 2.89                              & 2.30                              & $-$0.58                         & 6.0                       & 1.9                       & 10.8                         & 9.38E+20                                & 1                        \\
                    3427                    & 19368$-$0105$+$0229          & 193.6833                & $-$1.0500                 & 1,  5,  6                                & \nodata                           & 4.55                          & 22.86                             & 21.93                             & $-$0.43                         & 8.1                       & 2.6                       & 14.2                         & 2.89E+21                                & 1                        \\
                    3428                    & 19389$-$0098$+$0031          & 193.8917                & $-$0.9833                 & 5,  6                                    & \nodata                           & 1.68                          & 3.07                              & 2.37                              & $-$0.44                         & 6.3                       & 2.2                       & 13.6                         & 1.76E+21                                & 1                        \\
                    3429                    & 19473$-$0338$+$0126          & 194.7250                & $-$3.3833                 & 1,  5,  6                                & \nodata                           & 1.62                          & 12.55                             & 11.58                             & $-$0.49                         & 6.4                       & 4.2                       & 12.2                         & 4.73E+21                                & 1                        \\
                    3430                    & 19473$-$0111$+$0147          & 194.7333                & $-$1.1083                 & 5,  6                                    & \nodata                           & 1.71                          & 14.65                             & 13.35                             & $-$0.57                         & 5.7                       & 2.5                       & 11.7                         & 2.84E+21                                & 1                        \\
                    3431                    & 19488$+$0097$+$0126          & 194.8833                & 0.9750                  & 6                                        & \nodata                           & 1.70                          & 12.56                             & 12.20                             & $-$0.33                         & 4.6                       & 1.0                       & 9.7                          & 4.91E+20                                & 1                        \\
                    3432                    & 19620$+$0027$+$0032          & 196.2000                & 0.2750                  & 6                                        & \nodata                           & 1.68                          & 3.24                              & 2.66                              & $-$0.39                         & 7.2                       & 2.0                       & 11.5                         & 1.39E+21                                & 1                        \\
                    3433                    & 19630$-$0232$+$0151          & 196.3000                & $-$2.3250                 & 4,  6                                    & \nodata                           & 2.15                          & 15.13                             & 13.70                             & $-$0.41                         & 7.0                       & 2.4                       & 17.6                         & 4.66E+21                                & 1                        \\
                    3434                    & 19686$+$0079$+$0213          & 196.8583                & 0.7917                  & 6                                        & \nodata                           & 1.70                          & 21.31                             & 20.65                             & $-$0.70                         & 4.4                       & 1.3                       & 9.5                          & 5.37E+20                                & 1                        \\
                    3435                    & 19695$+$0061$+$0211          & 196.9500                & 0.6083                  & 6                                        & \nodata                           & 1.70                          & 21.14                             & 20.08                             & $-$0.52                         & 6.5                       & 2.3                       & 15.8                         & 2.52E+21                                & 1                        \\
                    3436                    & 19879$+$0162$+$0059          & 198.7917                & 1.6167                  & 5,  6                                    & \nodata                           & 0.51                          & 5.89                              & 5.43                              & $-$0.53                         & 5.0                       & 1.2                       & 10.0                         & 4.58E+20                                & 1                        \\
                    3437                    & 19883$+$0164$+$0062          & 198.8333                & 1.6417                  & 5,  6                                    & \nodata                           & 0.51                          & 6.20                              & 5.68                              & $-$0.58                         & 4.0                       & 1.1                       & 7.9                          & 4.56E+20                                & 1                        \\
                    3438                    & 19927$+$0102$+$0083          & 199.2667                & 1.0250                  & 6                                        & \nodata                           & 0.42                          & 8.33                              & 8.03                              & $-$0.30                         & 5.0                       & 1.8                       & 8.5                          & 8.78E+20                                & 1                        \\
                    3439                    & 19988$+$0122$+$0211          & 199.8833                & 1.2167                  & 1,  6                                    & \nodata                           & 4.61                          & 21.12                             & 20.18                             & $-$0.52                         & 7.4                       & 3.7                       & 16.2                         & 3.90E+21                                & 1                        \\
                    3440                    & 19998$+$0061$+$0069          & 199.9833                & 0.6083                  & 6                                        & \nodata                           & 0.42                          & 6.94                              & 6.17                              & $-$0.49                         & 5.8                       & 3.0                       & 10.4                         & 2.48E+21                                & 1                        \\
                    3441                    & 20035$+$0060$+$0066          & 200.3500                & 0.6000                  & 6                                        & \nodata                           & 0.42                          & 6.64                              & 6.04                              & $-$0.39                         & 7.1                       & 2.3                       & 15.2                         & 1.80E+21                                & 1                        \\
                    3442                    & 20036$+$0091$+$0061          & 200.3583                & 0.9083                  & 6                                        & \nodata                           & 0.42                          & 6.13                              & 5.61                              & $-$0.50                         & 5.7                       & 2.9                       & 10.2                         & 1.59E+21                                & 1                        \\
                    3443                    & 20044$+$0048$+$0055          & 200.4417                & 0.4833                  & 6                                        & \nodata                           & 0.42                          & 5.47                              & 5.12                              & $-$0.33                         & 8.1                       & 2.6                       & 15.5                         & 1.49E+21                                & 1                        \\
                    3444                    & 20047$+$0062$+$0059          & 200.4667                & 0.6167                  & 6                                        & \nodata                           & 0.42                          & 5.85                              & 5.51                              & $-$0.24                         & 6.4                       & 1.6                       & 10.3                         & 1.08E+21                                & 1                        \\
                    3445                    & 20051$+$0057$+$0060          & 200.5083                & 0.5750                  & 6                                        & \nodata                           & 0.42                          & 5.99                              & 5.38                              & $-$0.41                         & 6.7                       & 2.4                       & 14.8                         & 1.89E+21                                & 1                        \\
                    3446                    & 20051$+$0147$+$0086          & 200.5083                & 1.4750                  & 6                                        & \nodata                           & 0.53                          & 8.64                              & 8.17                              & $-$0.47                         & 5.2                       & 2.3                       & 11.1                         & 1.11E+21                                & 1                        \\
                    3447                    & 20056$+$0056$+$0060          & 200.5583                & 0.5583                  & 6                                        & \nodata                           & 0.42                          & 5.97                              & 5.51                              & $-$0.34                         & 7.1                       & 2.2                       & 12.4                         & 1.49E+21                                & 1                        \\
                    3448                    & 20059$+$0053$+$0060          & 200.5917                & 0.5333                  & 6                                        & \nodata                           & 0.42                          & 5.97                              & 5.46                              & $-$0.46                         & 6.7                       & 3.1                       & 14.3                         & 1.86E+21                                & 1                        \\
                    3449                    & 20099$+$0049$+$0048          & 200.9917                & 0.4917                  & 5,  6                                    & \nodata                           & 0.42                          & 4.79                              & 4.29                              & $-$0.55                         & 5.5                       & 2.8                       & 11.6                         & 1.31E+21                                & 1                        \\
                    3450                    & 20108$+$0041$+$0050          & 201.0833                & 0.4083                  & 5,  6                                    & \nodata                           & 0.42                          & 5.04                              & 4.60                              & $-$0.39                         & 6.6                       & 1.7                       & 13.1                         & 9.22E+20                                & 1                        \\
                    3451                    & 20116$+$0037$+$0054          & 201.1583                & 0.3750                  & 5,  6                                    & \nodata                           & 0.42                          & 5.38                              & 4.41                              & $-$0.63                         & 7.9                       & 4.1                       & 15.1                         & 3.68E+21                                & 1                        \\
                    3452                    & 20147$-$0007$-$0010          & 201.4667                & $-$0.0750                 & 5,  6                                    & \nodata                           & 0.42                          & $-$1.03                             & $-$1.44                             & $-$0.38                         & 7.6                       & 2.5                       & 16.2                         & 1.46E+21                                & 1                        \\
                    3453                    & 20157$-$0003$-$0013          & 201.5667                & $-$0.0333                 & 6                                        & \nodata                           & 0.42                          & $-$1.29                             & $-$1.91                             & $-$0.39                         & 10.0                      & 3.7                       & 19.1                         & 3.59E+21                                & 1                        \\
                    3454                    & 20191$+$0008$+$0003          & 201.9083                & 0.0833                  & 6                                        & \nodata                           & 0.42                          & 0.32                              & $-$0.08                             & $-$0.33                         & 13.2                      & 3.2                       & 23.0                         & 2.60E+21                                & 1                        \\
                    3455                    & 20194$+$0017$+$0002          & 201.9417                & 0.1667                  & 6                                        & \nodata                           & 0.42                          & 0.17                              & $-$0.24                             & $-$0.55                         & 12.5                      & 2.5                       & 17.7                         & 1.06E+21                                & 1                        \\
                    3456                    & 20198$+$0150$+$0065          & 201.9750                & 1.5000                  & 4,  6                                    & \nodata                           & 0.47                          & 6.51                              & 5.81                              & $-$0.51                         & 7.8                       & 3.1                       & 13.4                         & 2.33E+21                                & 1                        \\
                    3457                    & 20219$+$0268$+$0058          & 202.1917                & 2.6833                  & 4,  6                                    & \nodata                           & 0.46                          & 5.83                              & 5.40                              & $-$0.30                         & 10.4                      & 5.0                       & 13.8                         & 4.44E+21                                & 1                        \\
                    3458                    & 20247$+$0160$+$0016          & 202.4667                & 1.6000                  & 4,  6                                    & \nodata                           & 0.49                          & 1.61                              & 1.18                              & $-$0.44                         & 4.8                       & 2.1                       & 11.8                         & 9.97E+20                                & 1                        \\
                    3459                    & 20253$+$0155$+$0020          & 202.5250                & 1.5500                  & 4,  6                                    & \nodata                           & 0.50                          & 1.99                              & 1.52                              & $-$0.49                         & 4.1                       & 1.5                       & 10.2                         & 6.70E+20                                & 1                        \\
                    3460                    & 20254$+$0152$+$0021          & 202.5417                & 1.5250                  & 4,  6                                    & \nodata                           & 0.48                          & 2.05                              & 1.36                              & $-$0.78                         & 4.5                       & 1.8                       & 10.4                         & 7.59E+20                                & 1                        \\
                    3461                    & 20258$+$0138$+$0017          & 202.5750                & 1.3833                  & 4,  6                                    & \nodata                           & 0.42                          & 1.71                              & 1.25                              & $-$0.62                         & 5.1                       & 2.0                       & 10.2                         & 7.18E+20                                & 1                        \\
                    3462                    & 20279$+$0152$+$0332          & 202.7917                & 1.5167                  & 4,  6                                    & \nodata                           & 4.91                          & 33.22                             & 32.63                             & $-$0.60                         & 3.3                       & 1.5                       & 8.7                          & 6.72E+20                                & 1                        \\
                    3463                    & 20325$+$0207$+$0061          & 203.2500                & 2.0667                  & 3,  4,  6                                & \nodata                           & 0.50                          & 6.15                              & 4.90                              & $-$0.52                         & 23.2                      & 12.2                      & 28.8                         & 2.86E+22                                & 1                        \\
                    3464                    & 20377$+$0280$+$0096          & 203.7667                & 2.8000                  & 4,  6                                    & \nodata                           & 0.51                          & 9.56                              & 9.09                              & $-$0.42                         & 4.8                       & 1.1                       & 10.2                         & 5.47E+20                                & 1                        \\
                    3465                    & 20410$+$0238$+$0091          & 204.1000                & 2.3833                  & 4,  6                                    & \nodata                           & 0.51                          & 9.11                              & 8.52                              & $-$0.60                         & 5.0                       & 1.3                       & 8.5                          & 5.81E+20                                & 1                        \\
                    3466                    & 20468$+$0047$+$0098          & 204.6833                & 0.4750                  & 6                                        & \nodata                           & 0.42                          & 9.84                              & 9.34                              & $-$0.41                         & 4.0                       & 1.4                       & 8.4                          & 7.51E+20                                & 1                        \\
                    3467                    & 20472$+$0047$+$0102          & 204.7167                & 0.4667                  & 6                                        & \nodata                           & 0.42                          & 10.16                             & 9.39                              & $-$0.55                         & 4.6                       & 1.5                       & 11.8                         & 9.47E+20                                & 1                        \\
                    3468                    & 20486$+$0050$+$0093          & 204.8583                & 0.5000                  & 6                                        & \nodata                           & 0.42                          & 9.33                              & 8.75                              & $-$0.34                         & 6.2                       & 2.2                       & 10.9                         & 1.79E+21                                & 1                        \\
                    3469                    & 20529$-$0275$+$0172          & 205.2917                & $-$2.7500                 & 1,  2,  5,  6                            & \nodata                           & 0.87                          & 17.24                             & 16.79                             & $-$0.37                         & 11.9                      & 5.7                       & 20.4                         & 4.80E+21                                & 1                        \\
                    3470                    & 20588$+$0014$+$0199          & 205.8833                & 0.1417                  & 6                                        & \nodata                           & 2.28                          & 19.90                             & 19.63                             & $-$0.36                         & 5.9                       & 1.9                       & 11.4                         & 6.87E+20                                & 1                        \\
                    3471                    & 20703$-$0188$+$0106          & 207.0333                & $-$1.8833                 & 2,  4,  5,  6                            & \nodata                           & 0.42                          & 10.58                             & 9.67                              & $-$0.42                         & 6.3                       & 3.5                       & 15.9                         & 4.25E+21                                & 1                        \\
                    3472                    & 20708$-$0195$+$0105          & 207.0750                & $-$1.9500                 & 2,  4,  5,  6                            & \nodata                           & 0.42                          & 10.48                             & 9.91                              & $-$0.37                         & 5.6                       & 1.9                       & 10.8                         & 1.41E+21                                & 1                        \\
                    3473                    & 20709$-$0175$+$0098          & 207.0917                & $-$1.7500                 & 2,  5,  6                                & \nodata                           & 0.42                          & 9.84                              & 8.89                              & $-$0.71                         & 7.4                       & 3.9                       & 13.7                         & 2.96E+21                                & 1                        \\
                    3474                    & 20727$-$0154$+$0102          & 207.2667                & $-$1.5417                 & 2,  6                                    & \nodata                           & 0.42                          & 10.18                             & 9.19                              & $-$0.63                         & 6.5                       & 2.2                       & 12.3                         & 1.68E+21                                & 1                        \\
                    3475                    & 20748$+$0193$+$0127          & 207.4750                & 1.9333                  & 6                                        & \nodata                           & 0.64                          & 12.67                             & 12.01                             & $-$0.78                         & 4.4                       & 1.0                       & 8.8                          & 3.83E+20                                & 1                        \\
                    3476                    & 20757$-$0172$+$0122          & 207.5667                & $-$1.7250                 & 1,  6                                    & \nodata                           & 0.42                          & 12.24                             & 11.37                             & $-$0.39                         & 12.1                      & 7.4                       & 18.9                         & 1.20E+22                                & 1                        \\
                    3477                    & 20765$-$0168$+$0117          & 207.6500                & $-$1.6833                 & 6                                        & \nodata                           & 0.42                          & 11.68                             & 11.07                             & $-$0.30                         & 7.0                       & 2.3                       & 13.3                         & 2.34E+21                                & 1                        \\
                    3478                    & 20778$-$0182$+$0109          & 207.7833                & $-$1.8167                 & 6                                        & \nodata                           & 0.42                          & 10.93                             & 10.36                             & $-$0.44                         & 6.6                       & 4.8                       & 16.5                         & 3.87E+21                                & 1                        \\
                    3479                    & 20779$-$0196$+$0113          & 207.7917                & $-$1.9583                 & 6                                        & \nodata                           & 0.42                          & 11.29                             & 10.56                             & $-$0.40                         & 7.9                       & 3.2                       & 16.5                         & 3.28E+21                                & 1                        \\
                    3480                    & 20780$-$0184$+$0112          & 207.8000                & $-$1.8417                 & 6                                        & \nodata                           & 0.42                          & 11.17                             & 10.42                             & $-$0.41                         & 7.6                       & 3.9                       & 14.8                         & 4.04E+21                                & 1                        \\
                    3481                    & 20780$-$0192$+$0113          & 207.8000                & $-$1.9250                 & 6                                        & \nodata                           & 0.42                          & 11.27                             & 10.57                             & $-$0.41                         & 9.2                       & 3.7                       & 15.6                         & 3.65E+21                                & 1                        \\
                    3482                    & 20787$-$0181$+$0110          & 207.8667                & $-$1.8083                 & 6                                        & \nodata                           & 0.42                          & 10.97                             & 10.63                             & $-$0.33                         & 5.3                       & 3.0                       & 13.4                         & 1.59E+21                                & 1                        \\
                    3483                    & 20788$-$0241$+$0105          & 207.8750                & $-$2.4083                 & 6                                        & \nodata                           & 0.42                          & 10.51                             & 10.17                             & $-$0.37                         & 4.9                       & 1.3                       & 8.5                          & 5.51E+20                                & 1                        \\
                    3484                    & 20868$+$0229$+$0092          & 208.6750                & 2.2917                  & 3,  5,  6                                & \nodata                           & 0.53                          & 9.15                              & 8.86                              & $-$0.25                         & 6.6                       & 1.4                       & 13.0                         & 7.58E+20                                & 1                        \\
                    3485                    & 20919$-$0013$+$0296          & 209.1917                & $-$0.1333                 & 6                                        & \nodata                           & 2.45                          & 29.60                             & 28.91                             & $-$0.60                         & 4.1                       & 2.0                       & 9.0                          & 1.14E+21                                & 1                        \\
                    3486                    & 20944$-$0166$+$0375          & 209.4417                & $-$1.6583                 & 1,  6                                    & \nodata                           & 2.43                          & 37.52                             & 36.60                             & $-$0.53                         & 6.5                       & 2.7                       & 12.3                         & 2.38E+21                                & 1                        \\
                    3487                    & 21502$-$0191$+$0267          & 215.0167                & $-$1.9083                 & 6                                        & \nodata                           & 2.59                          & 26.67                             & 26.06                             & $-$0.61                         & 5.6                       & 2.3                       & 11.5                         & 1.15E+21                                & 1                        \\
                    3488                    & 21509$-$0175$+$0270          & 215.0917                & $-$1.7500                 & 6                                        & \nodata                           & 2.60                          & 27.02                             & 26.49                             & $-$0.59                         & 6.0                       & 2.2                       & 10.6                         & 9.82E+20                                & 1                        \\
                    3489                    & 21730$-$0005$+$0265          & 217.3000                & $-$0.0500                 & 1,  3,  6                                & \nodata                           & 2.56                          & 26.51                             & 25.83                             & $-$0.30                         & 18.7                      & 9.3                       & 25.8                         & 1.77E+22                                & 1                        \\
                    3490                    & 21763$-$0018$+$0257          & 217.6333                & $-$0.1833                 & 1,  3,  6                                & \nodata                           & 2.53                          & 25.72                             & 25.18                             & $-$0.37                         & 15.3                      & 6.6                       & 28.3                         & 7.75E+21                                & 1                        \\
                    3491                    & 21765$+$0016$+$0218          & 217.6500                & 0.1583                  & 6                                        & \nodata                           & 0.66                          & 21.82                             & 21.34                             & $-$0.49                         & 6.2                       & 2.0                       & 11.7                         & 9.38E+20                                & 1                        \\
                    3492                    & 22033$-$0124$+$0182          & 220.3333                & $-$1.2417                 & 6                                        & \nodata                           & 0.70                          & 18.16                             & 17.78                             & $-$0.38                         & 5.7                       & 1.8                       & 11.4                         & 8.34E+20                                & 1                        \\
                    3493                    & 22048$-$0163$+$0142          & 220.4833                & $-$1.6333                 & 6                                        & \nodata                           & 0.42                          & 14.21                             & 13.70                             & $-$0.52                         & 7.4                       & 1.8                       & 15.5                         & 9.17E+20                                & 1                        \\
                    3494                    & 22080$-$0161$+$0136          & 220.8000                & $-$1.6083                 & 3,  6                                    & \nodata                           & 0.42                          & 13.55                             & 13.15                             & $-$0.34                         & 6.5                       & 1.5                       & 12.5                         & 8.29E+20                                & 1                        \\
                    3495                    & 22080$-$0166$+$0136          & 220.8000                & $-$1.6583                 & 3,  5,  6                                & \nodata                           & 0.42                          & 13.56                             & 12.72                             & $-$0.52                         & 9.2                       & 3.4                       & 17.9                         & 3.26E+21                                & 1                        \\
                    3496                    & 22098$-$0252$+$0130          & 220.9750                & $-$2.5167                 & 6                                        & \nodata                           & 0.42                          & 12.98                             & 12.02                             & $-$0.62                         & 7.8                       & 3.9                       & 14.8                         & 3.41E+21                                & 1                        \\
                    3497                    & 22172$-$0279$+$0121          & 221.7167                & $-$2.7917                 & 3,  6                                    & \nodata                           & 0.43                          & 12.12                             & 11.39                             & $-$0.54                         & 4.5                       & 2.1                       & 9.4                          & 1.36E+21                                & 1                        \\
                    3498                    & 22199$-$0477$+$0159          & 221.9917                & $-$4.7750                 & 6                                        & \nodata                           & 0.87                          & 15.91                             & 15.31                             & $-$0.59                         & 5.4                       & 3.1                       & 11.8                         & 1.64E+21                                & 1                        \\
                    3499                    & 22199$-$0481$+$0160          & 221.9917                & $-$4.8083                 & 6                                        & \nodata                           & 0.87                          & 15.99                             & 15.23                             & $-$0.67                         & 4.7                       & 2.5                       & 13.6                         & 1.45E+21                                & 1                        \\
                    3500                    & 22220$+$0117$+$0284          & 222.2000                & 1.1750                  & 6                                        & \nodata                           & 0.69                          & 28.40                             & 27.78                             & $-$0.58                         & 5.6                       & 1.4                       & 10.6                         & 6.66E+20                                & 1                        \\
                    3501                    & 22233$-$0301$+$0120          & 222.3250                & $-$3.0083                 & 6                                        & \nodata                           & 0.43                          & 12.05                             & 11.57                             & $-$0.30                         & 8.9                       & 2.3                       & 15.0                         & 1.92E+21                                & 1                        \\
                    3502                    & 22236$-$0310$+$0117          & 222.3583                & $-$3.1000                 & 6                                        & \nodata                           & 0.43                          & 11.69                             & 11.20                             & $-$0.53                         & 7.4                       & 2.3                       & 12.8                         & 1.04E+21                                & 1                        \\
                    3503                    & 22238$-$0313$+$0113          & 222.3750                & $-$3.1333                 & 6                                        & \nodata                           & 0.43                          & 11.25                             & 10.72                             & $-$0.61                         & 6.7                       & 2.3                       & 12.7                         & 1.03E+21                                & 1                        \\
                    3504                    & 22248$-$0035$+$0162          & 222.4750                & $-$0.3500                 & 6                                        & \nodata                           & 0.42                          & 16.25                             & 15.61                             & $-$0.97                         & 4.9                       & 1.9                       & 10.4                         & 5.75E+20                                & 1                        \\
                    3505                    & 22251$-$0035$+$0162          & 222.5083                & $-$0.3500                 & 6                                        & \nodata                           & 0.42                          & 16.23                             & 15.55                             & $-$0.80                         & 3.8                       & 1.4                       & 9.2                          & 5.55E+20                                & 1                        \\
                    3506                    & 22255$-$0357$+$0126          & 222.5500                & $-$3.5750                 & 6                                        & \nodata                           & 0.44                          & 12.65                             & 12.19                             & $-$0.39                         & 6.1                       & 1.5                       & 12.6                         & 8.65E+20                                & 1                        \\
                    3507                    & 22283$-$0039$+$0154          & 222.8333                & $-$0.3917                 & 6                                        & \nodata                           & 0.42                          & 15.38                             & 14.70                             & $-$0.64                         & 5.1                       & 2.6                       & 11.8                         & 1.39E+21                                & 1                        \\
                    3508                    & 22288$-$0042$+$0154          & 222.8833                & $-$0.4167                 & 6                                        & \nodata                           & 0.42                          & 15.37                             & 14.67                             & $-$0.43                         & 5.7                       & 1.5                       & 12.2                         & 1.18E+21                                & 1                        \\
                    3509                    & 22304$-$0131$+$0130          & 223.0417                & $-$1.3083                 & 6                                        & \nodata                           & 0.42                          & 12.97                             & 12.53                             & $-$0.45                         & 6.1                       & 2.4                       & 11.2                         & 1.17E+21                                & 1                        \\
                    3510                    & 22314$-$0514$+$0174          & 223.1417                & $-$5.1417                 & 6                                        & \nodata                           & 0.87                          & 17.42                             & 16.84                             & $-$0.70                         & 4.2                       & 1.2                       & 8.9                          & 4.34E+20                                & 1                        \\
                    3511                    & 22315$-$0187$+$0182          & 223.1500                & $-$1.8750                 & 4,  5,  6                                & \nodata                           & 0.77                          & 18.18                             & 17.69                             & $-$0.38                         & 11.0                      & 2.7                       & 16.3                         & 1.93E+21                                & 1                        \\
                    3512                    & 22331$-$0092$+$0156          & 223.3083                & $-$0.9167                 & 6                                        & \nodata                           & 0.42                          & 15.57                             & 14.86                             & $-$0.85                         & 5.5                       & 2.5                       & 12.9                         & 1.07E+21                                & 1                        \\
                    3513                    & 22332$-$0096$+$0158          & 223.3167                & $-$0.9583                 & 4,  6                                    & \nodata                           & 0.42                          & 15.78                             & 15.24                             & $-$0.61                         & 4.9                       & 1.5                       & 10.6                         & 5.95E+20                                & 1                        \\
                    3514                    & 22340$-$0097$+$0151          & 223.4000                & $-$0.9667                 & 4,  6                                    & \nodata                           & 0.42                          & 15.12                             & 14.72                             & $-$0.36                         & 5.5                       & 1.6                       & 11.6                         & 7.95E+20                                & 1                        \\
                    3515                    & 22364$-$0091$+$0163          & 223.6417                & $-$0.9083                 & 4,  6                                    & \nodata                           & 0.42                          & 16.27                             & 15.79                             & $-$0.59                         & 7.4                       & 3.2                       & 13.3                         & 1.41E+21                                & 1                        \\
                    3516                    & 22400$-$0169$+$0170          & 224.0000                & $-$1.6917                 & 4,  5,  6                                & \nodata                           & 0.42                          & 17.04                             & 16.67                             & $-$0.26                         & 19.4                      & 7.5                       & 24.3                         & 8.48E+21                                & 1                        \\
                    3517                    & 22426$-$0107$+$0147          & 224.2583                & $-$1.0667                 & 4,  6                                    & \nodata                           & 0.42                          & 14.66                             & 12.75                             & $-$0.91                         & 10.5                      & 6.0                       & 18.0                         & 8.41E+21                                & 1                        \\
                    3518                    & 22428$-$0082$+$0142          & 224.2833                & $-$0.8250                 & 4,  6                                    & \nodata                           & 0.42                          & 14.23                             & 13.54                             & $-$0.53                         & 5.1                       & 3.4                       & 10.2                         & 1.92E+22                                & 2                        \\
                    3519                    & 22448$-$0068$+$0160          & 224.4750                & $-$0.6833                 & 4,  6                                    & \nodata                           & 0.42                          & 16.00                             & 15.07                             & $-$0.74                         & 4.6                       & 1.9                       & 9.9                          & 8.71E+21                                & 2                        \\
                    3520                    & 22448$-$0380$+$0153          & 224.4750                & $-$3.8000                 & 6                                        & \nodata                           & 0.42                          & 15.32                             & 14.69                             & $-$0.50                         & 8.6                       & 2.8                       & 16.8                         & 1.96E+21                                & 1                        \\
                    3521                    & 22450$-$0067$+$0160          & 224.5000                & $-$0.6667                 & 4,  6                                    & \nodata                           & 0.42                          & 15.96                             & 15.23                             & $-$0.89                         & 3.5                       & 1.4                       & 7.9                          & 4.16E+21                                & 2                        \\
                    3522                    & 22461$-$0053$+$0153          & 224.6083                & $-$0.5333                 & 4,  6                                    & \nodata                           & 0.42                          & 15.27                             & 14.46                             & $-$0.37                         & 7.0                       & 3.3                       & 15.3                         & 3.93E+21                                & 1                        \\
                    3523                    & 22482$-$0179$+$0160          & 224.8167                & $-$1.7917                 & 4,  6                                    & \nodata                           & 0.42                          & 16.03                             & 15.27                             & $-$0.44                         & 11.9                      & 3.3                       & 17.8                         & 3.40E+21                                & 1                        \\
                    3524                    & 22488$-$0040$+$0163          & 224.8833                & $-$0.4000                 & 4,  6                                    & \nodata                           & 0.42                          & 16.31                             & 15.52                             & $-$0.37                         & 7.0                       & 1.6                       & 13.4                         & 1.60E+21                                & 1                        \\
                    3525                    & 22491$+$0067$+$0175          & 224.9083                & 0.6667                  & 6                                        & \nodata                           & 0.42                          & 17.46                             & 16.86                             & $-$0.54                         & 4.7                       & 2.5                       & 9.8                          & 1.44E+21                                & 1                        \\
                    3526                    & 22505$-$0012$+$0170          & 225.0500                & $-$0.1167                 & 6                                        & \nodata                           & 0.42                          & 17.00                             & 16.65                             & $-$0.37                         & 6.7                       & 1.9                       & 10.8                         & 8.48E+20                                & 1                        \\
                    3527                    & 22539$-$0263$+$0115          & 225.3917                & $-$2.6333                 & 3,  4,  6                                & \nodata                           & 0.42                          & 11.49                             & 10.96                             & $-$0.36                         & 16.0                      & 7.6                       & 27.7                         & 9.38E+21                                & 1                        \\
                    3528                    & 22540$-$0260$+$0114          & 225.4000                & $-$2.6000                 & 3,  4,  6                                & \nodata                           & 0.42                          & 11.43                             & 10.99                             & $-$0.27                         & 16.1                      & 7.8                       & 24.2                         & 9.95E+21                                & 1                        \\
                    3529                    & 22568$-$0040$+$0164          & 225.6750                & $-$0.4000                 & 6                                        & \nodata                           & 0.42                          & 16.44                             & 15.29                             & $-$0.59                         & 5.3                       & 1.5                       & 11.1                         & 1.29E+21                                & 1                        \\
                    3530                    & 22609$-$0036$+$0161          & 226.0917                & $-$0.3583                 & 6                                        & \nodata                           & 0.42                          & 16.08                             & 15.52                             & $-$0.45                         & 4.3                       & 2.4                       & 7.8                          & 1.73E+21                                & 1                        \\
                    3531                    & 22614$-$0032$+$0155          & 226.1417                & $-$0.3250                 & 6                                        & \nodata                           & 0.42                          & 15.48                             & 15.06                             & $-$0.34                         & 6.2                       & 3.5                       & 10.3                         & 2.43E+21                                & 1                        \\
                    3532                    & 22643$-$0060$+$0151          & 226.4250                & $-$0.6000                 & 1,  6                                    & \nodata                           & 0.42                          & 15.13                             & 14.56                             & $-$0.72                         & 9.6                       & 5.7                       & 16.5                         & 7.50E+21                                & 1                        \\
                    3533                    & 22666$-$0062$+$0168          & 226.6583                & $-$0.6167                 & 6                                        & \nodata                           & 0.42                          & 16.84                             & 16.35                             & $-$0.71                         & 3.5                       & 1.3                       & 9.7                          & 3.94E+20                                & 1                        \\
                \end{longtable}
            \end{ThreePartTable}
            %\end{table}
        }
    \end{landscape}
\end{center}


   \section{Spectra and gas distributions of the sources}

\begin{figure}
\includegraphics[width=9.0cm,angle=0]{ms2022-0304fig10-0001.pdf}
\includegraphics[width=9.0cm,angle=0]{ms2022-0304fig10-0002.pdf}
\vspace{-0.5cm}

\includegraphics[width=9.0cm,angle=0]{ms2022-0304fig10-0003.pdf}
\includegraphics[width=9.0cm,angle=0]{ms2022-0304fig10-0004.pdf}
\vspace{-0.5cm}

\includegraphics[width=9.0cm,angle=0]{ms2022-0304fig10-0005.pdf}
\includegraphics[width=9.0cm,angle=0]{ms2022-0304fig10-0006.pdf}
\vspace{-0.5cm}

\includegraphics[width=9.0cm,angle=0]{ms2022-0304fig10-0007.pdf}
\includegraphics[width=9.0cm,angle=0]{ms2022-0304fig10-0008.pdf}
\vspace{-0.5cm}

\includegraphics[width=9.0cm,angle=0]{ms2022-0304fig10-0009.pdf}
\includegraphics[width=9.0cm,angle=0]{ms2022-0304fig10-0010.pdf}
\end{figure}
\clearpage

\begin{figure}
\includegraphics[width=9.0cm,angle=0]{ms2022-0304fig10-0011.pdf}
\includegraphics[width=9.0cm,angle=0]{ms2022-0304fig10-0012.pdf}
\vspace{-0.5cm}

\includegraphics[width=9.0cm,angle=0]{ms2022-0304fig10-0013.pdf}
\includegraphics[width=9.0cm,angle=0]{ms2022-0304fig10-0014.pdf}
\vspace{-0.5cm}

\includegraphics[width=9.0cm,angle=0]{ms2022-0304fig10-0015.pdf}
\includegraphics[width=9.0cm,angle=0]{ms2022-0304fig10-0016.pdf}
\vspace{-0.5cm}

\includegraphics[width=9.0cm,angle=0]{ms2022-0304fig10-0017.pdf}
\includegraphics[width=9.0cm,angle=0]{ms2022-0304fig10-0018.pdf}
\vspace{-0.5cm}

\includegraphics[width=9.0cm,angle=0]{ms2022-0304fig10-0019.pdf}
\includegraphics[width=9.0cm,angle=0]{ms2022-0304fig10-0020.pdf}
\end{figure}
\clearpage

\begin{figure}
\includegraphics[width=9.0cm,angle=0]{ms2022-0304fig10-0021.pdf}
\includegraphics[width=9.0cm,angle=0]{ms2022-0304fig10-0022.pdf}
\vspace{-0.5cm}

\includegraphics[width=9.0cm,angle=0]{ms2022-0304fig10-0023.pdf}
\includegraphics[width=9.0cm,angle=0]{ms2022-0304fig10-0024.pdf}
\vspace{-0.5cm}

\includegraphics[width=9.0cm,angle=0]{ms2022-0304fig10-0025.pdf}
\includegraphics[width=9.0cm,angle=0]{ms2022-0304fig10-0026.pdf}
\vspace{-0.5cm}

\includegraphics[width=9.0cm,angle=0]{ms2022-0304fig10-0027.pdf}
\includegraphics[width=9.0cm,angle=0]{ms2022-0304fig10-0028.pdf}
\vspace{-0.5cm}

\includegraphics[width=9.0cm,angle=0]{ms2022-0304fig10-0029.pdf}
\includegraphics[width=9.0cm,angle=0]{ms2022-0304fig10-0030.pdf}
\end{figure}
\clearpage

\begin{figure}
\includegraphics[width=9.0cm,angle=0]{ms2022-0304fig10-0031.pdf}
\includegraphics[width=9.0cm,angle=0]{ms2022-0304fig10-0032.pdf}
\vspace{-0.5cm}

\includegraphics[width=9.0cm,angle=0]{ms2022-0304fig10-0033.pdf}
\includegraphics[width=9.0cm,angle=0]{ms2022-0304fig10-0034.pdf}
\vspace{-0.5cm}

\includegraphics[width=9.0cm,angle=0]{ms2022-0304fig10-0035.pdf}
\includegraphics[width=9.0cm,angle=0]{ms2022-0304fig10-0036.pdf}
\vspace{-0.5cm}

\includegraphics[width=9.0cm,angle=0]{ms2022-0304fig10-0037.pdf}
\includegraphics[width=9.0cm,angle=0]{ms2022-0304fig10-0038.pdf}
\vspace{-0.5cm}

\includegraphics[width=9.0cm,angle=0]{ms2022-0304fig10-0039.pdf}
\includegraphics[width=9.0cm,angle=0]{ms2022-0304fig10-0040.pdf}
\end{figure}
\clearpage

\begin{figure}
\includegraphics[width=9.0cm,angle=0]{ms2022-0304fig10-0041.pdf}
\includegraphics[width=9.0cm,angle=0]{ms2022-0304fig10-0042.pdf}
\vspace{-0.5cm}

\includegraphics[width=9.0cm,angle=0]{ms2022-0304fig10-0043.pdf}
\includegraphics[width=9.0cm,angle=0]{ms2022-0304fig10-0044.pdf}
\vspace{-0.5cm}

\includegraphics[width=9.0cm,angle=0]{ms2022-0304fig10-0045.pdf}
\includegraphics[width=9.0cm,angle=0]{ms2022-0304fig10-0046.pdf}
\vspace{-0.5cm}

\includegraphics[width=9.0cm,angle=0]{ms2022-0304fig10-0047.pdf}
\includegraphics[width=9.0cm,angle=0]{ms2022-0304fig10-0048.pdf}
\vspace{-0.5cm}

\includegraphics[width=9.0cm,angle=0]{ms2022-0304fig10-0049.pdf}
\includegraphics[width=9.0cm,angle=0]{ms2022-0304fig10-0050.pdf}
\end{figure}
\clearpage

\begin{figure}
\includegraphics[width=9.0cm,angle=0]{ms2022-0304fig10-0051.pdf}
\includegraphics[width=9.0cm,angle=0]{ms2022-0304fig10-0052.pdf}
\vspace{-0.5cm}

\includegraphics[width=9.0cm,angle=0]{ms2022-0304fig10-0053.pdf}
\includegraphics[width=9.0cm,angle=0]{ms2022-0304fig10-0054.pdf}
\vspace{-0.5cm}

\includegraphics[width=9.0cm,angle=0]{ms2022-0304fig10-0055.pdf}
\includegraphics[width=9.0cm,angle=0]{ms2022-0304fig10-0056.pdf}
\vspace{-0.5cm}

\includegraphics[width=9.0cm,angle=0]{ms2022-0304fig10-0057.pdf}
\includegraphics[width=9.0cm,angle=0]{ms2022-0304fig10-0058.pdf}
\vspace{-0.5cm}

\includegraphics[width=9.0cm,angle=0]{ms2022-0304fig10-0059.pdf}
\includegraphics[width=9.0cm,angle=0]{ms2022-0304fig10-0060.pdf}
\end{figure}
\clearpage

\begin{figure}
\includegraphics[width=9.0cm,angle=0]{ms2022-0304fig10-0061.pdf}
\includegraphics[width=9.0cm,angle=0]{ms2022-0304fig10-0062.pdf}
\vspace{-0.5cm}

\includegraphics[width=9.0cm,angle=0]{ms2022-0304fig10-0063.pdf}
\includegraphics[width=9.0cm,angle=0]{ms2022-0304fig10-0064.pdf}
\vspace{-0.5cm}

\includegraphics[width=9.0cm,angle=0]{ms2022-0304fig10-0065.pdf}
\includegraphics[width=9.0cm,angle=0]{ms2022-0304fig10-0066.pdf}
\vspace{-0.5cm}

\includegraphics[width=9.0cm,angle=0]{ms2022-0304fig10-0067.pdf}
\includegraphics[width=9.0cm,angle=0]{ms2022-0304fig10-0068.pdf}
\vspace{-0.5cm}

\includegraphics[width=9.0cm,angle=0]{ms2022-0304fig10-0069.pdf}
\includegraphics[width=9.0cm,angle=0]{ms2022-0304fig10-0070.pdf}
\end{figure}
\clearpage

\begin{figure}
\includegraphics[width=9.0cm,angle=0]{ms2022-0304fig10-0071.pdf}
\includegraphics[width=9.0cm,angle=0]{ms2022-0304fig10-0072.pdf}
\vspace{-0.5cm}

\includegraphics[width=9.0cm,angle=0]{ms2022-0304fig10-0073.pdf}
\includegraphics[width=9.0cm,angle=0]{ms2022-0304fig10-0074.pdf}
\vspace{-0.5cm}

\includegraphics[width=9.0cm,angle=0]{ms2022-0304fig10-0075.pdf}
\includegraphics[width=9.0cm,angle=0]{ms2022-0304fig10-0076.pdf}
\vspace{-0.5cm}

\includegraphics[width=9.0cm,angle=0]{ms2022-0304fig10-0077.pdf}
\includegraphics[width=9.0cm,angle=0]{ms2022-0304fig10-0078.pdf}
\vspace{-0.5cm}

\includegraphics[width=9.0cm,angle=0]{ms2022-0304fig10-0079.pdf}
\includegraphics[width=9.0cm,angle=0]{ms2022-0304fig10-0080.pdf}
\end{figure}
\clearpage

\begin{figure}
\includegraphics[width=9.0cm,angle=0]{ms2022-0304fig10-0081.pdf}
\includegraphics[width=9.0cm,angle=0]{ms2022-0304fig10-0082.pdf}
\vspace{-0.5cm}

\includegraphics[width=9.0cm,angle=0]{ms2022-0304fig10-0083.pdf}
\includegraphics[width=9.0cm,angle=0]{ms2022-0304fig10-0084.pdf}
\vspace{-0.5cm}

\includegraphics[width=9.0cm,angle=0]{ms2022-0304fig10-0085.pdf}
\includegraphics[width=9.0cm,angle=0]{ms2022-0304fig10-0086.pdf}
\vspace{-0.5cm}

\includegraphics[width=9.0cm,angle=0]{ms2022-0304fig10-0087.pdf}
\includegraphics[width=9.0cm,angle=0]{ms2022-0304fig10-0088.pdf}
\vspace{-0.5cm}

\includegraphics[width=9.0cm,angle=0]{ms2022-0304fig10-0089.pdf}
\includegraphics[width=9.0cm,angle=0]{ms2022-0304fig10-0090.pdf}
\end{figure}
\clearpage

\begin{figure}
\includegraphics[width=9.0cm,angle=0]{ms2022-0304fig10-0091.pdf}
\includegraphics[width=9.0cm,angle=0]{ms2022-0304fig10-0092.pdf}
\vspace{-0.5cm}

\includegraphics[width=9.0cm,angle=0]{ms2022-0304fig10-0093.pdf}
\includegraphics[width=9.0cm,angle=0]{ms2022-0304fig10-0094.pdf}
\vspace{-0.5cm}

\includegraphics[width=9.0cm,angle=0]{ms2022-0304fig10-0095.pdf}
\includegraphics[width=9.0cm,angle=0]{ms2022-0304fig10-0096.pdf}
\vspace{-0.5cm}

\includegraphics[width=9.0cm,angle=0]{ms2022-0304fig10-0097.pdf}
\includegraphics[width=9.0cm,angle=0]{ms2022-0304fig10-0098.pdf}
\vspace{-0.5cm}

\includegraphics[width=9.0cm,angle=0]{ms2022-0304fig10-0099.pdf}
\includegraphics[width=9.0cm,angle=0]{ms2022-0304fig10-0100.pdf}
\end{figure}
\clearpage

\begin{figure}
\includegraphics[width=9.0cm,angle=0]{ms2022-0304fig10-0101.pdf}
\includegraphics[width=9.0cm,angle=0]{ms2022-0304fig10-0102.pdf}
\vspace{-0.5cm}

\includegraphics[width=9.0cm,angle=0]{ms2022-0304fig10-0103.pdf}
\includegraphics[width=9.0cm,angle=0]{ms2022-0304fig10-0104.pdf}
\vspace{-0.5cm}

\includegraphics[width=9.0cm,angle=0]{ms2022-0304fig10-0105.pdf}
\includegraphics[width=9.0cm,angle=0]{ms2022-0304fig10-0106.pdf}
\vspace{-0.5cm}

\includegraphics[width=9.0cm,angle=0]{ms2022-0304fig10-0107.pdf}
\includegraphics[width=9.0cm,angle=0]{ms2022-0304fig10-0108.pdf}
\vspace{-0.5cm}

\includegraphics[width=9.0cm,angle=0]{ms2022-0304fig10-0109.pdf}
\includegraphics[width=9.0cm,angle=0]{ms2022-0304fig10-0110.pdf}
\end{figure}
\clearpage

\begin{figure}
\includegraphics[width=9.0cm,angle=0]{ms2022-0304fig10-0111.pdf}
\includegraphics[width=9.0cm,angle=0]{ms2022-0304fig10-0112.pdf}
\vspace{-0.5cm}

\includegraphics[width=9.0cm,angle=0]{ms2022-0304fig10-0113.pdf}
\includegraphics[width=9.0cm,angle=0]{ms2022-0304fig10-0114.pdf}
\vspace{-0.5cm}

\includegraphics[width=9.0cm,angle=0]{ms2022-0304fig10-0115.pdf}
\includegraphics[width=9.0cm,angle=0]{ms2022-0304fig10-0116.pdf}
\vspace{-0.5cm}

\includegraphics[width=9.0cm,angle=0]{ms2022-0304fig10-0117.pdf}
\includegraphics[width=9.0cm,angle=0]{ms2022-0304fig10-0118.pdf}
\vspace{-0.5cm}

\includegraphics[width=9.0cm,angle=0]{ms2022-0304fig10-0119.pdf}
\includegraphics[width=9.0cm,angle=0]{ms2022-0304fig10-0120.pdf}
\end{figure}
\clearpage

\begin{figure}
\includegraphics[width=9.0cm,angle=0]{ms2022-0304fig10-0121.pdf}
\includegraphics[width=9.0cm,angle=0]{ms2022-0304fig10-0122.pdf}
\vspace{-0.5cm}

\includegraphics[width=9.0cm,angle=0]{ms2022-0304fig10-0123.pdf}
\includegraphics[width=9.0cm,angle=0]{ms2022-0304fig10-0124.pdf}
\vspace{-0.5cm}

\includegraphics[width=9.0cm,angle=0]{ms2022-0304fig10-0125.pdf}
\includegraphics[width=9.0cm,angle=0]{ms2022-0304fig10-0126.pdf}
\vspace{-0.5cm}

\includegraphics[width=9.0cm,angle=0]{ms2022-0304fig10-0127.pdf}
\includegraphics[width=9.0cm,angle=0]{ms2022-0304fig10-0128.pdf}
\vspace{-0.5cm}

\includegraphics[width=9.0cm,angle=0]{ms2022-0304fig10-0129.pdf}
\includegraphics[width=9.0cm,angle=0]{ms2022-0304fig10-0130.pdf}
\end{figure}
\clearpage

\begin{figure}
\includegraphics[width=9.0cm,angle=0]{ms2022-0304fig10-0131.pdf}
\includegraphics[width=9.0cm,angle=0]{ms2022-0304fig10-0132.pdf}
\vspace{-0.5cm}

\includegraphics[width=9.0cm,angle=0]{ms2022-0304fig10-0133.pdf}
\includegraphics[width=9.0cm,angle=0]{ms2022-0304fig10-0134.pdf}
\vspace{-0.5cm}

\includegraphics[width=9.0cm,angle=0]{ms2022-0304fig10-0135.pdf}
\includegraphics[width=9.0cm,angle=0]{ms2022-0304fig10-0136.pdf}
\vspace{-0.5cm}

\includegraphics[width=9.0cm,angle=0]{ms2022-0304fig10-0137.pdf}
\includegraphics[width=9.0cm,angle=0]{ms2022-0304fig10-0138.pdf}
\vspace{-0.5cm}

\includegraphics[width=9.0cm,angle=0]{ms2022-0304fig10-0139.pdf}
\includegraphics[width=9.0cm,angle=0]{ms2022-0304fig10-0140.pdf}
\end{figure}
\clearpage

\begin{figure}
\includegraphics[width=9.0cm,angle=0]{ms2022-0304fig10-0141.pdf}
\includegraphics[width=9.0cm,angle=0]{ms2022-0304fig10-0142.pdf}
\vspace{-0.5cm}

\includegraphics[width=9.0cm,angle=0]{ms2022-0304fig10-0143.pdf}
\includegraphics[width=9.0cm,angle=0]{ms2022-0304fig10-0144.pdf}
\vspace{-0.5cm}

\includegraphics[width=9.0cm,angle=0]{ms2022-0304fig10-0145.pdf}
\includegraphics[width=9.0cm,angle=0]{ms2022-0304fig10-0146.pdf}
\vspace{-0.5cm}

\includegraphics[width=9.0cm,angle=0]{ms2022-0304fig10-0147.pdf}
\includegraphics[width=9.0cm,angle=0]{ms2022-0304fig10-0148.pdf}
\vspace{-0.5cm}

\includegraphics[width=9.0cm,angle=0]{ms2022-0304fig10-0149.pdf}
\includegraphics[width=9.0cm,angle=0]{ms2022-0304fig10-0150.pdf}
\end{figure}
\clearpage

\begin{figure}
\includegraphics[width=9.0cm,angle=0]{ms2022-0304fig10-0151.pdf}
\includegraphics[width=9.0cm,angle=0]{ms2022-0304fig10-0152.pdf}
\vspace{-0.5cm}

\includegraphics[width=9.0cm,angle=0]{ms2022-0304fig10-0153.pdf}
\includegraphics[width=9.0cm,angle=0]{ms2022-0304fig10-0154.pdf}
\vspace{-0.5cm}

\includegraphics[width=9.0cm,angle=0]{ms2022-0304fig10-0155.pdf}
\includegraphics[width=9.0cm,angle=0]{ms2022-0304fig10-0156.pdf}
\vspace{-0.5cm}

\includegraphics[width=9.0cm,angle=0]{ms2022-0304fig10-0157.pdf}
\includegraphics[width=9.0cm,angle=0]{ms2022-0304fig10-0158.pdf}
\vspace{-0.5cm}

\includegraphics[width=9.0cm,angle=0]{ms2022-0304fig10-0159.pdf}
\includegraphics[width=9.0cm,angle=0]{ms2022-0304fig10-0160.pdf}
\end{figure}
\clearpage

\begin{figure}
\includegraphics[width=9.0cm,angle=0]{ms2022-0304fig10-0161.pdf}
\includegraphics[width=9.0cm,angle=0]{ms2022-0304fig10-0162.pdf}
\vspace{-0.5cm}

\includegraphics[width=9.0cm,angle=0]{ms2022-0304fig10-0163.pdf}
\includegraphics[width=9.0cm,angle=0]{ms2022-0304fig10-0164.pdf}
\vspace{-0.5cm}

\includegraphics[width=9.0cm,angle=0]{ms2022-0304fig10-0165.pdf}
\includegraphics[width=9.0cm,angle=0]{ms2022-0304fig10-0166.pdf}
\vspace{-0.5cm}

\includegraphics[width=9.0cm,angle=0]{ms2022-0304fig10-0167.pdf}
\includegraphics[width=9.0cm,angle=0]{ms2022-0304fig10-0168.pdf}
\vspace{-0.5cm}

\includegraphics[width=9.0cm,angle=0]{ms2022-0304fig10-0169.pdf}
\includegraphics[width=9.0cm,angle=0]{ms2022-0304fig10-0170.pdf}
\end{figure}
\clearpage

\begin{figure}
\includegraphics[width=9.0cm,angle=0]{ms2022-0304fig10-0171.pdf}
\includegraphics[width=9.0cm,angle=0]{ms2022-0304fig10-0172.pdf}
\vspace{-0.5cm}

\includegraphics[width=9.0cm,angle=0]{ms2022-0304fig10-0173.pdf}
\includegraphics[width=9.0cm,angle=0]{ms2022-0304fig10-0174.pdf}
\vspace{-0.5cm}

\includegraphics[width=9.0cm,angle=0]{ms2022-0304fig10-0175.pdf}
\includegraphics[width=9.0cm,angle=0]{ms2022-0304fig10-0176.pdf}
\vspace{-0.5cm}

\includegraphics[width=9.0cm,angle=0]{ms2022-0304fig10-0177.pdf}
\includegraphics[width=9.0cm,angle=0]{ms2022-0304fig10-0178.pdf}
\vspace{-0.5cm}

\includegraphics[width=9.0cm,angle=0]{ms2022-0304fig10-0179.pdf}
\includegraphics[width=9.0cm,angle=0]{ms2022-0304fig10-0180.pdf}
\end{figure}
\clearpage

\begin{figure}
\includegraphics[width=9.0cm,angle=0]{ms2022-0304fig10-0181.pdf}
\includegraphics[width=9.0cm,angle=0]{ms2022-0304fig10-0182.pdf}
\vspace{-0.5cm}

\includegraphics[width=9.0cm,angle=0]{ms2022-0304fig10-0183.pdf}
\includegraphics[width=9.0cm,angle=0]{ms2022-0304fig10-0184.pdf}
\vspace{-0.5cm}

\includegraphics[width=9.0cm,angle=0]{ms2022-0304fig10-0185.pdf}
\includegraphics[width=9.0cm,angle=0]{ms2022-0304fig10-0186.pdf}
\vspace{-0.5cm}

\includegraphics[width=9.0cm,angle=0]{ms2022-0304fig10-0187.pdf}
\includegraphics[width=9.0cm,angle=0]{ms2022-0304fig10-0188.pdf}
\vspace{-0.5cm}

\includegraphics[width=9.0cm,angle=0]{ms2022-0304fig10-0189.pdf}
\includegraphics[width=9.0cm,angle=0]{ms2022-0304fig10-0190.pdf}
\end{figure}
\clearpage

\begin{figure}
\includegraphics[width=9.0cm,angle=0]{ms2022-0304fig10-0191.pdf}
\includegraphics[width=9.0cm,angle=0]{ms2022-0304fig10-0192.pdf}
\vspace{-0.5cm}

\includegraphics[width=9.0cm,angle=0]{ms2022-0304fig10-0193.pdf}
\includegraphics[width=9.0cm,angle=0]{ms2022-0304fig10-0194.pdf}
\vspace{-0.5cm}

\includegraphics[width=9.0cm,angle=0]{ms2022-0304fig10-0195.pdf}
\includegraphics[width=9.0cm,angle=0]{ms2022-0304fig10-0196.pdf}
\vspace{-0.5cm}

\includegraphics[width=9.0cm,angle=0]{ms2022-0304fig10-0197.pdf}
\includegraphics[width=9.0cm,angle=0]{ms2022-0304fig10-0198.pdf}
\vspace{-0.5cm}

\includegraphics[width=9.0cm,angle=0]{ms2022-0304fig10-0199.pdf}
\includegraphics[width=9.0cm,angle=0]{ms2022-0304fig10-0200.pdf}
\end{figure}
\clearpage

\begin{figure}
\includegraphics[width=9.0cm,angle=0]{ms2022-0304fig10-0201.pdf}
\includegraphics[width=9.0cm,angle=0]{ms2022-0304fig10-0202.pdf}
\vspace{-0.5cm}

\includegraphics[width=9.0cm,angle=0]{ms2022-0304fig10-0203.pdf}
\includegraphics[width=9.0cm,angle=0]{ms2022-0304fig10-0204.pdf}
\vspace{-0.5cm}

\includegraphics[width=9.0cm,angle=0]{ms2022-0304fig10-0205.pdf}
\includegraphics[width=9.0cm,angle=0]{ms2022-0304fig10-0206.pdf}
\vspace{-0.5cm}

\includegraphics[width=9.0cm,angle=0]{ms2022-0304fig10-0207.pdf}
\includegraphics[width=9.0cm,angle=0]{ms2022-0304fig10-0208.pdf}
\vspace{-0.5cm}

\includegraphics[width=9.0cm,angle=0]{ms2022-0304fig10-0209.pdf}
\includegraphics[width=9.0cm,angle=0]{ms2022-0304fig10-0210.pdf}
\end{figure}
\clearpage

\begin{figure}
\includegraphics[width=9.0cm,angle=0]{ms2022-0304fig10-0211.pdf}
\includegraphics[width=9.0cm,angle=0]{ms2022-0304fig10-0212.pdf}
\vspace{-0.5cm}

\includegraphics[width=9.0cm,angle=0]{ms2022-0304fig10-0213.pdf}
\includegraphics[width=9.0cm,angle=0]{ms2022-0304fig10-0214.pdf}
\vspace{-0.5cm}

\includegraphics[width=9.0cm,angle=0]{ms2022-0304fig10-0215.pdf}
\includegraphics[width=9.0cm,angle=0]{ms2022-0304fig10-0216.pdf}
\vspace{-0.5cm}

\includegraphics[width=9.0cm,angle=0]{ms2022-0304fig10-0217.pdf}
\includegraphics[width=9.0cm,angle=0]{ms2022-0304fig10-0218.pdf}
\vspace{-0.5cm}

\includegraphics[width=9.0cm,angle=0]{ms2022-0304fig10-0219.pdf}
\includegraphics[width=9.0cm,angle=0]{ms2022-0304fig10-0220.pdf}
\end{figure}
\clearpage

\begin{figure}
\includegraphics[width=9.0cm,angle=0]{ms2022-0304fig10-0221.pdf}
\includegraphics[width=9.0cm,angle=0]{ms2022-0304fig10-0222.pdf}
\vspace{-0.5cm}

\includegraphics[width=9.0cm,angle=0]{ms2022-0304fig10-0223.pdf}
\includegraphics[width=9.0cm,angle=0]{ms2022-0304fig10-0224.pdf}
\vspace{-0.5cm}

\includegraphics[width=9.0cm,angle=0]{ms2022-0304fig10-0225.pdf}
\includegraphics[width=9.0cm,angle=0]{ms2022-0304fig10-0226.pdf}
\vspace{-0.5cm}

\includegraphics[width=9.0cm,angle=0]{ms2022-0304fig10-0227.pdf}
\includegraphics[width=9.0cm,angle=0]{ms2022-0304fig10-0228.pdf}
\vspace{-0.5cm}

\includegraphics[width=9.0cm,angle=0]{ms2022-0304fig10-0229.pdf}
\includegraphics[width=9.0cm,angle=0]{ms2022-0304fig10-0230.pdf}
\end{figure}
\clearpage

\begin{figure}
\includegraphics[width=9.0cm,angle=0]{ms2022-0304fig10-0231.pdf}
\includegraphics[width=9.0cm,angle=0]{ms2022-0304fig10-0232.pdf}
\vspace{-0.5cm}

\includegraphics[width=9.0cm,angle=0]{ms2022-0304fig10-0233.pdf}
\includegraphics[width=9.0cm,angle=0]{ms2022-0304fig10-0234.pdf}
\vspace{-0.5cm}

\includegraphics[width=9.0cm,angle=0]{ms2022-0304fig10-0235.pdf}
\includegraphics[width=9.0cm,angle=0]{ms2022-0304fig10-0236.pdf}
\vspace{-0.5cm}

\includegraphics[width=9.0cm,angle=0]{ms2022-0304fig10-0237.pdf}
\includegraphics[width=9.0cm,angle=0]{ms2022-0304fig10-0238.pdf}
\vspace{-0.5cm}

\includegraphics[width=9.0cm,angle=0]{ms2022-0304fig10-0239.pdf}
\includegraphics[width=9.0cm,angle=0]{ms2022-0304fig10-0240.pdf}
\end{figure}
\clearpage

\begin{figure}
\includegraphics[width=9.0cm,angle=0]{ms2022-0304fig10-0241.pdf}
\includegraphics[width=9.0cm,angle=0]{ms2022-0304fig10-0242.pdf}
\vspace{-0.5cm}

\includegraphics[width=9.0cm,angle=0]{ms2022-0304fig10-0243.pdf}
\includegraphics[width=9.0cm,angle=0]{ms2022-0304fig10-0244.pdf}
\vspace{-0.5cm}

\includegraphics[width=9.0cm,angle=0]{ms2022-0304fig10-0245.pdf}
\includegraphics[width=9.0cm,angle=0]{ms2022-0304fig10-0246.pdf}
\vspace{-0.5cm}

\includegraphics[width=9.0cm,angle=0]{ms2022-0304fig10-0247.pdf}
\includegraphics[width=9.0cm,angle=0]{ms2022-0304fig10-0248.pdf}
\vspace{-0.5cm}

\includegraphics[width=9.0cm,angle=0]{ms2022-0304fig10-0249.pdf}
\includegraphics[width=9.0cm,angle=0]{ms2022-0304fig10-0250.pdf}
\end{figure}
\clearpage

\begin{figure}
\includegraphics[width=9.0cm,angle=0]{ms2022-0304fig10-0251.pdf}
\includegraphics[width=9.0cm,angle=0]{ms2022-0304fig10-0252.pdf}
\vspace{-0.5cm}

\includegraphics[width=9.0cm,angle=0]{ms2022-0304fig10-0253.pdf}
\includegraphics[width=9.0cm,angle=0]{ms2022-0304fig10-0254.pdf}
\vspace{-0.5cm}

\includegraphics[width=9.0cm,angle=0]{ms2022-0304fig10-0255.pdf}
\includegraphics[width=9.0cm,angle=0]{ms2022-0304fig10-0256.pdf}
\vspace{-0.5cm}

\includegraphics[width=9.0cm,angle=0]{ms2022-0304fig10-0257.pdf}
\includegraphics[width=9.0cm,angle=0]{ms2022-0304fig10-0258.pdf}
\vspace{-0.5cm}

\includegraphics[width=9.0cm,angle=0]{ms2022-0304fig10-0259.pdf}
\includegraphics[width=9.0cm,angle=0]{ms2022-0304fig10-0260.pdf}
\end{figure}
\clearpage

\begin{figure}
\includegraphics[width=9.0cm,angle=0]{ms2022-0304fig10-0261.pdf}
\includegraphics[width=9.0cm,angle=0]{ms2022-0304fig10-0262.pdf}
\vspace{-0.5cm}

\includegraphics[width=9.0cm,angle=0]{ms2022-0304fig10-0263.pdf}
\includegraphics[width=9.0cm,angle=0]{ms2022-0304fig10-0264.pdf}
\vspace{-0.5cm}

\includegraphics[width=9.0cm,angle=0]{ms2022-0304fig10-0265.pdf}
\includegraphics[width=9.0cm,angle=0]{ms2022-0304fig10-0266.pdf}
\vspace{-0.5cm}

\includegraphics[width=9.0cm,angle=0]{ms2022-0304fig10-0267.pdf}
\includegraphics[width=9.0cm,angle=0]{ms2022-0304fig10-0268.pdf}
\vspace{-0.5cm}

\includegraphics[width=9.0cm,angle=0]{ms2022-0304fig10-0269.pdf}
\includegraphics[width=9.0cm,angle=0]{ms2022-0304fig10-0270.pdf}
\end{figure}
\clearpage

\begin{figure}
\includegraphics[width=9.0cm,angle=0]{ms2022-0304fig10-0271.pdf}
\includegraphics[width=9.0cm,angle=0]{ms2022-0304fig10-0272.pdf}
\vspace{-0.5cm}

\includegraphics[width=9.0cm,angle=0]{ms2022-0304fig10-0273.pdf}
\includegraphics[width=9.0cm,angle=0]{ms2022-0304fig10-0274.pdf}
\vspace{-0.5cm}

\includegraphics[width=9.0cm,angle=0]{ms2022-0304fig10-0275.pdf}
\includegraphics[width=9.0cm,angle=0]{ms2022-0304fig10-0276.pdf}
\vspace{-0.5cm}

\includegraphics[width=9.0cm,angle=0]{ms2022-0304fig10-0277.pdf}
\includegraphics[width=9.0cm,angle=0]{ms2022-0304fig10-0278.pdf}
\vspace{-0.5cm}

\includegraphics[width=9.0cm,angle=0]{ms2022-0304fig10-0279.pdf}
\includegraphics[width=9.0cm,angle=0]{ms2022-0304fig10-0280.pdf}
\end{figure}
\clearpage

\begin{figure}
\includegraphics[width=9.0cm,angle=0]{ms2022-0304fig10-0281.pdf}
\includegraphics[width=9.0cm,angle=0]{ms2022-0304fig10-0282.pdf}
\vspace{-0.5cm}

\includegraphics[width=9.0cm,angle=0]{ms2022-0304fig10-0283.pdf}
\includegraphics[width=9.0cm,angle=0]{ms2022-0304fig10-0284.pdf}
\vspace{-0.5cm}

\includegraphics[width=9.0cm,angle=0]{ms2022-0304fig10-0285.pdf}
\includegraphics[width=9.0cm,angle=0]{ms2022-0304fig10-0286.pdf}
\vspace{-0.5cm}

\includegraphics[width=9.0cm,angle=0]{ms2022-0304fig10-0287.pdf}
\includegraphics[width=9.0cm,angle=0]{ms2022-0304fig10-0288.pdf}
\vspace{-0.5cm}

\includegraphics[width=9.0cm,angle=0]{ms2022-0304fig10-0289.pdf}
\includegraphics[width=9.0cm,angle=0]{ms2022-0304fig10-0290.pdf}
\end{figure}
\clearpage

\begin{figure}
\includegraphics[width=9.0cm,angle=0]{ms2022-0304fig10-0291.pdf}
\includegraphics[width=9.0cm,angle=0]{ms2022-0304fig10-0292.pdf}
\vspace{-0.5cm}

\includegraphics[width=9.0cm,angle=0]{ms2022-0304fig10-0293.pdf}
\includegraphics[width=9.0cm,angle=0]{ms2022-0304fig10-0294.pdf}
\vspace{-0.5cm}

\includegraphics[width=9.0cm,angle=0]{ms2022-0304fig10-0295.pdf}
\includegraphics[width=9.0cm,angle=0]{ms2022-0304fig10-0296.pdf}
\vspace{-0.5cm}

\includegraphics[width=9.0cm,angle=0]{ms2022-0304fig10-0297.pdf}
\includegraphics[width=9.0cm,angle=0]{ms2022-0304fig10-0298.pdf}
\vspace{-0.5cm}

\includegraphics[width=9.0cm,angle=0]{ms2022-0304fig10-0299.pdf}
\includegraphics[width=9.0cm,angle=0]{ms2022-0304fig10-0300.pdf}
\end{figure}
\clearpage

\begin{figure}
\includegraphics[width=9.0cm,angle=0]{ms2022-0304fig10-0301.pdf}
\includegraphics[width=9.0cm,angle=0]{ms2022-0304fig10-0302.pdf}
\vspace{-0.5cm}

\includegraphics[width=9.0cm,angle=0]{ms2022-0304fig10-0303.pdf}
\includegraphics[width=9.0cm,angle=0]{ms2022-0304fig10-0304.pdf}
\vspace{-0.5cm}

\includegraphics[width=9.0cm,angle=0]{ms2022-0304fig10-0305.pdf}
\includegraphics[width=9.0cm,angle=0]{ms2022-0304fig10-0306.pdf}
\vspace{-0.5cm}

\includegraphics[width=9.0cm,angle=0]{ms2022-0304fig10-0307.pdf}
\includegraphics[width=9.0cm,angle=0]{ms2022-0304fig10-0308.pdf}
\vspace{-0.5cm}

\includegraphics[width=9.0cm,angle=0]{ms2022-0304fig10-0309.pdf}
\includegraphics[width=9.0cm,angle=0]{ms2022-0304fig10-0310.pdf}
\end{figure}
\clearpage

\begin{figure}
\includegraphics[width=9.0cm,angle=0]{ms2022-0304fig10-0311.pdf}
\includegraphics[width=9.0cm,angle=0]{ms2022-0304fig10-0312.pdf}
\vspace{-0.5cm}

\includegraphics[width=9.0cm,angle=0]{ms2022-0304fig10-0313.pdf}
\includegraphics[width=9.0cm,angle=0]{ms2022-0304fig10-0314.pdf}
\vspace{-0.5cm}

\includegraphics[width=9.0cm,angle=0]{ms2022-0304fig10-0315.pdf}
\includegraphics[width=9.0cm,angle=0]{ms2022-0304fig10-0316.pdf}
\vspace{-0.5cm}

\includegraphics[width=9.0cm,angle=0]{ms2022-0304fig10-0317.pdf}
\includegraphics[width=9.0cm,angle=0]{ms2022-0304fig10-0318.pdf}
\vspace{-0.5cm}

\includegraphics[width=9.0cm,angle=0]{ms2022-0304fig10-0319.pdf}
\includegraphics[width=9.0cm,angle=0]{ms2022-0304fig10-0320.pdf}
\end{figure}
\clearpage

\begin{figure}
\includegraphics[width=9.0cm,angle=0]{ms2022-0304fig10-0321.pdf}
\includegraphics[width=9.0cm,angle=0]{ms2022-0304fig10-0322.pdf}
\vspace{-0.5cm}

\includegraphics[width=9.0cm,angle=0]{ms2022-0304fig10-0323.pdf}
\includegraphics[width=9.0cm,angle=0]{ms2022-0304fig10-0324.pdf}
\vspace{-0.5cm}

\includegraphics[width=9.0cm,angle=0]{ms2022-0304fig10-0325.pdf}
\includegraphics[width=9.0cm,angle=0]{ms2022-0304fig10-0326.pdf}
\vspace{-0.5cm}

\includegraphics[width=9.0cm,angle=0]{ms2022-0304fig10-0327.pdf}
\includegraphics[width=9.0cm,angle=0]{ms2022-0304fig10-0328.pdf}
\vspace{-0.5cm}

\includegraphics[width=9.0cm,angle=0]{ms2022-0304fig10-0329.pdf}
\includegraphics[width=9.0cm,angle=0]{ms2022-0304fig10-0330.pdf}
\end{figure}
\clearpage

\begin{figure}
\includegraphics[width=9.0cm,angle=0]{ms2022-0304fig10-0331.pdf}
\includegraphics[width=9.0cm,angle=0]{ms2022-0304fig10-0332.pdf}
\vspace{-0.5cm}

\includegraphics[width=9.0cm,angle=0]{ms2022-0304fig10-0333.pdf}
\includegraphics[width=9.0cm,angle=0]{ms2022-0304fig10-0334.pdf}
\vspace{-0.5cm}

\includegraphics[width=9.0cm,angle=0]{ms2022-0304fig10-0335.pdf}
\includegraphics[width=9.0cm,angle=0]{ms2022-0304fig10-0336.pdf}
\vspace{-0.5cm}

\includegraphics[width=9.0cm,angle=0]{ms2022-0304fig10-0337.pdf}
\includegraphics[width=9.0cm,angle=0]{ms2022-0304fig10-0338.pdf}
\vspace{-0.5cm}

\includegraphics[width=9.0cm,angle=0]{ms2022-0304fig10-0339.pdf}
\includegraphics[width=9.0cm,angle=0]{ms2022-0304fig10-0340.pdf}
\end{figure}
\clearpage

\begin{figure}
\includegraphics[width=9.0cm,angle=0]{ms2022-0304fig10-0341.pdf}
\includegraphics[width=9.0cm,angle=0]{ms2022-0304fig10-0342.pdf}
\vspace{-0.5cm}

\includegraphics[width=9.0cm,angle=0]{ms2022-0304fig10-0343.pdf}
\includegraphics[width=9.0cm,angle=0]{ms2022-0304fig10-0344.pdf}
\vspace{-0.5cm}

\includegraphics[width=9.0cm,angle=0]{ms2022-0304fig10-0345.pdf}
\includegraphics[width=9.0cm,angle=0]{ms2022-0304fig10-0346.pdf}
\vspace{-0.5cm}

\includegraphics[width=9.0cm,angle=0]{ms2022-0304fig10-0347.pdf}
\includegraphics[width=9.0cm,angle=0]{ms2022-0304fig10-0348.pdf}
\vspace{-0.5cm}

\includegraphics[width=9.0cm,angle=0]{ms2022-0304fig10-0349.pdf}
\includegraphics[width=9.0cm,angle=0]{ms2022-0304fig10-0350.pdf}
\end{figure}
\clearpage

\begin{figure}
\includegraphics[width=9.0cm,angle=0]{ms2022-0304fig10-0351.pdf}
\includegraphics[width=9.0cm,angle=0]{ms2022-0304fig10-0352.pdf}
\vspace{-0.5cm}

\includegraphics[width=9.0cm,angle=0]{ms2022-0304fig10-0353.pdf}
\includegraphics[width=9.0cm,angle=0]{ms2022-0304fig10-0354.pdf}
\vspace{-0.5cm}

\includegraphics[width=9.0cm,angle=0]{ms2022-0304fig10-0355.pdf}
\includegraphics[width=9.0cm,angle=0]{ms2022-0304fig10-0356.pdf}
\vspace{-0.5cm}

\includegraphics[width=9.0cm,angle=0]{ms2022-0304fig10-0357.pdf}
\includegraphics[width=9.0cm,angle=0]{ms2022-0304fig10-0358.pdf}
\vspace{-0.5cm}

\includegraphics[width=9.0cm,angle=0]{ms2022-0304fig10-0359.pdf}
\includegraphics[width=9.0cm,angle=0]{ms2022-0304fig10-0360.pdf}
\end{figure}
\clearpage

\begin{figure}
\includegraphics[width=9.0cm,angle=0]{ms2022-0304fig10-0361.pdf}
\includegraphics[width=9.0cm,angle=0]{ms2022-0304fig10-0362.pdf}
\vspace{-0.5cm}

\includegraphics[width=9.0cm,angle=0]{ms2022-0304fig10-0363.pdf}
\includegraphics[width=9.0cm,angle=0]{ms2022-0304fig10-0364.pdf}
\vspace{-0.5cm}

\includegraphics[width=9.0cm,angle=0]{ms2022-0304fig10-0365.pdf}
\includegraphics[width=9.0cm,angle=0]{ms2022-0304fig10-0366.pdf}
\vspace{-0.5cm}

\includegraphics[width=9.0cm,angle=0]{ms2022-0304fig10-0367.pdf}
\includegraphics[width=9.0cm,angle=0]{ms2022-0304fig10-0368.pdf}
\vspace{-0.5cm}

\includegraphics[width=9.0cm,angle=0]{ms2022-0304fig10-0369.pdf}
\includegraphics[width=9.0cm,angle=0]{ms2022-0304fig10-0370.pdf}
\end{figure}
\clearpage

\begin{figure}
\includegraphics[width=9.0cm,angle=0]{ms2022-0304fig10-0371.pdf}
\includegraphics[width=9.0cm,angle=0]{ms2022-0304fig10-0372.pdf}
\vspace{-0.5cm}

\includegraphics[width=9.0cm,angle=0]{ms2022-0304fig10-0373.pdf}
\includegraphics[width=9.0cm,angle=0]{ms2022-0304fig10-0374.pdf}
\vspace{-0.5cm}

\includegraphics[width=9.0cm,angle=0]{ms2022-0304fig10-0375.pdf}
\includegraphics[width=9.0cm,angle=0]{ms2022-0304fig10-0376.pdf}
\vspace{-0.5cm}

\includegraphics[width=9.0cm,angle=0]{ms2022-0304fig10-0377.pdf}
\includegraphics[width=9.0cm,angle=0]{ms2022-0304fig10-0378.pdf}
\vspace{-0.5cm}

\includegraphics[width=9.0cm,angle=0]{ms2022-0304fig10-0379.pdf}
\includegraphics[width=9.0cm,angle=0]{ms2022-0304fig10-0380.pdf}
\end{figure}
\clearpage

\begin{figure}
\includegraphics[width=9.0cm,angle=0]{ms2022-0304fig10-0381.pdf}
\includegraphics[width=9.0cm,angle=0]{ms2022-0304fig10-0382.pdf}
\vspace{-0.5cm}

\includegraphics[width=9.0cm,angle=0]{ms2022-0304fig10-0383.pdf}
\includegraphics[width=9.0cm,angle=0]{ms2022-0304fig10-0384.pdf}
\vspace{-0.5cm}

\includegraphics[width=9.0cm,angle=0]{ms2022-0304fig10-0385.pdf}
\includegraphics[width=9.0cm,angle=0]{ms2022-0304fig10-0386.pdf}
\vspace{-0.5cm}

\includegraphics[width=9.0cm,angle=0]{ms2022-0304fig10-0387.pdf}
\includegraphics[width=9.0cm,angle=0]{ms2022-0304fig10-0388.pdf}
\vspace{-0.5cm}

\includegraphics[width=9.0cm,angle=0]{ms2022-0304fig10-0389.pdf}
\includegraphics[width=9.0cm,angle=0]{ms2022-0304fig10-0390.pdf}
\end{figure}
\clearpage

\begin{figure}
\includegraphics[width=9.0cm,angle=0]{ms2022-0304fig10-0391.pdf}
\includegraphics[width=9.0cm,angle=0]{ms2022-0304fig10-0392.pdf}
\vspace{-0.5cm}

\includegraphics[width=9.0cm,angle=0]{ms2022-0304fig10-0393.pdf}
\includegraphics[width=9.0cm,angle=0]{ms2022-0304fig10-0394.pdf}
\vspace{-0.5cm}

\includegraphics[width=9.0cm,angle=0]{ms2022-0304fig10-0395.pdf}
\includegraphics[width=9.0cm,angle=0]{ms2022-0304fig10-0396.pdf}
\vspace{-0.5cm}

\includegraphics[width=9.0cm,angle=0]{ms2022-0304fig10-0397.pdf}
\includegraphics[width=9.0cm,angle=0]{ms2022-0304fig10-0398.pdf}
\vspace{-0.5cm}

\includegraphics[width=9.0cm,angle=0]{ms2022-0304fig10-0399.pdf}
\includegraphics[width=9.0cm,angle=0]{ms2022-0304fig10-0400.pdf}
\end{figure}
\clearpage

\begin{figure}
\includegraphics[width=9.0cm,angle=0]{ms2022-0304fig10-0401.pdf}
\includegraphics[width=9.0cm,angle=0]{ms2022-0304fig10-0402.pdf}
\vspace{-0.5cm}

\includegraphics[width=9.0cm,angle=0]{ms2022-0304fig10-0403.pdf}
\includegraphics[width=9.0cm,angle=0]{ms2022-0304fig10-0404.pdf}
\vspace{-0.5cm}

\includegraphics[width=9.0cm,angle=0]{ms2022-0304fig10-0405.pdf}
\includegraphics[width=9.0cm,angle=0]{ms2022-0304fig10-0406.pdf}
\vspace{-0.5cm}

\includegraphics[width=9.0cm,angle=0]{ms2022-0304fig10-0407.pdf}
\includegraphics[width=9.0cm,angle=0]{ms2022-0304fig10-0408.pdf}
\vspace{-0.5cm}

\includegraphics[width=9.0cm,angle=0]{ms2022-0304fig10-0409.pdf}
\includegraphics[width=9.0cm,angle=0]{ms2022-0304fig10-0410.pdf}
\end{figure}
\clearpage

\begin{figure}
\includegraphics[width=9.0cm,angle=0]{ms2022-0304fig10-0411.pdf}
\includegraphics[width=9.0cm,angle=0]{ms2022-0304fig10-0412.pdf}
\vspace{-0.5cm}

\includegraphics[width=9.0cm,angle=0]{ms2022-0304fig10-0413.pdf}
\includegraphics[width=9.0cm,angle=0]{ms2022-0304fig10-0414.pdf}
\vspace{-0.5cm}

\includegraphics[width=9.0cm,angle=0]{ms2022-0304fig10-0415.pdf}
\includegraphics[width=9.0cm,angle=0]{ms2022-0304fig10-0416.pdf}
\vspace{-0.5cm}

\includegraphics[width=9.0cm,angle=0]{ms2022-0304fig10-0417.pdf}
\includegraphics[width=9.0cm,angle=0]{ms2022-0304fig10-0418.pdf}
\vspace{-0.5cm}

\includegraphics[width=9.0cm,angle=0]{ms2022-0304fig10-0419.pdf}
\includegraphics[width=9.0cm,angle=0]{ms2022-0304fig10-0420.pdf}
\end{figure}
\clearpage

\begin{figure}
\includegraphics[width=9.0cm,angle=0]{ms2022-0304fig10-0421.pdf}
\includegraphics[width=9.0cm,angle=0]{ms2022-0304fig10-0422.pdf}
\vspace{-0.5cm}

\includegraphics[width=9.0cm,angle=0]{ms2022-0304fig10-0423.pdf}
\includegraphics[width=9.0cm,angle=0]{ms2022-0304fig10-0424.pdf}
\vspace{-0.5cm}

\includegraphics[width=9.0cm,angle=0]{ms2022-0304fig10-0425.pdf}
\includegraphics[width=9.0cm,angle=0]{ms2022-0304fig10-0426.pdf}
\vspace{-0.5cm}

\includegraphics[width=9.0cm,angle=0]{ms2022-0304fig10-0427.pdf}
\includegraphics[width=9.0cm,angle=0]{ms2022-0304fig10-0428.pdf}
\vspace{-0.5cm}

\includegraphics[width=9.0cm,angle=0]{ms2022-0304fig10-0429.pdf}
\includegraphics[width=9.0cm,angle=0]{ms2022-0304fig10-0430.pdf}
\end{figure}
\clearpage

\begin{figure}
\includegraphics[width=9.0cm,angle=0]{ms2022-0304fig10-0431.pdf}
\includegraphics[width=9.0cm,angle=0]{ms2022-0304fig10-0432.pdf}
\vspace{-0.5cm}

\includegraphics[width=9.0cm,angle=0]{ms2022-0304fig10-0433.pdf}
\includegraphics[width=9.0cm,angle=0]{ms2022-0304fig10-0434.pdf}
\vspace{-0.5cm}

\includegraphics[width=9.0cm,angle=0]{ms2022-0304fig10-0435.pdf}
\includegraphics[width=9.0cm,angle=0]{ms2022-0304fig10-0436.pdf}
\vspace{-0.5cm}

\includegraphics[width=9.0cm,angle=0]{ms2022-0304fig10-0437.pdf}
\includegraphics[width=9.0cm,angle=0]{ms2022-0304fig10-0438.pdf}
\vspace{-0.5cm}

\includegraphics[width=9.0cm,angle=0]{ms2022-0304fig10-0439.pdf}
\includegraphics[width=9.0cm,angle=0]{ms2022-0304fig10-0440.pdf}
\end{figure}
\clearpage

\begin{figure}
\includegraphics[width=9.0cm,angle=0]{ms2022-0304fig10-0441.pdf}
\includegraphics[width=9.0cm,angle=0]{ms2022-0304fig10-0442.pdf}
\vspace{-0.5cm}

\includegraphics[width=9.0cm,angle=0]{ms2022-0304fig10-0443.pdf}
\includegraphics[width=9.0cm,angle=0]{ms2022-0304fig10-0444.pdf}
\vspace{-0.5cm}

\includegraphics[width=9.0cm,angle=0]{ms2022-0304fig10-0445.pdf}
\includegraphics[width=9.0cm,angle=0]{ms2022-0304fig10-0446.pdf}
\vspace{-0.5cm}

\includegraphics[width=9.0cm,angle=0]{ms2022-0304fig10-0447.pdf}
\includegraphics[width=9.0cm,angle=0]{ms2022-0304fig10-0448.pdf}
\vspace{-0.5cm}

\includegraphics[width=9.0cm,angle=0]{ms2022-0304fig10-0449.pdf}
\includegraphics[width=9.0cm,angle=0]{ms2022-0304fig10-0450.pdf}
\end{figure}
\clearpage

\begin{figure}
\includegraphics[width=9.0cm,angle=0]{ms2022-0304fig10-0451.pdf}
\includegraphics[width=9.0cm,angle=0]{ms2022-0304fig10-0452.pdf}
\vspace{-0.5cm}

\includegraphics[width=9.0cm,angle=0]{ms2022-0304fig10-0453.pdf}
\includegraphics[width=9.0cm,angle=0]{ms2022-0304fig10-0454.pdf}
\vspace{-0.5cm}

\includegraphics[width=9.0cm,angle=0]{ms2022-0304fig10-0455.pdf}
\includegraphics[width=9.0cm,angle=0]{ms2022-0304fig10-0456.pdf}
\vspace{-0.5cm}

\includegraphics[width=9.0cm,angle=0]{ms2022-0304fig10-0457.pdf}
\includegraphics[width=9.0cm,angle=0]{ms2022-0304fig10-0458.pdf}
\vspace{-0.5cm}

\includegraphics[width=9.0cm,angle=0]{ms2022-0304fig10-0459.pdf}
\includegraphics[width=9.0cm,angle=0]{ms2022-0304fig10-0460.pdf}
\end{figure}
\clearpage

\begin{figure}
\includegraphics[width=9.0cm,angle=0]{ms2022-0304fig10-0461.pdf}
\includegraphics[width=9.0cm,angle=0]{ms2022-0304fig10-0462.pdf}
\vspace{-0.5cm}

\includegraphics[width=9.0cm,angle=0]{ms2022-0304fig10-0463.pdf}
\includegraphics[width=9.0cm,angle=0]{ms2022-0304fig10-0464.pdf}
\vspace{-0.5cm}

\includegraphics[width=9.0cm,angle=0]{ms2022-0304fig10-0465.pdf}
\includegraphics[width=9.0cm,angle=0]{ms2022-0304fig10-0466.pdf}
\vspace{-0.5cm}

\includegraphics[width=9.0cm,angle=0]{ms2022-0304fig10-0467.pdf}
\includegraphics[width=9.0cm,angle=0]{ms2022-0304fig10-0468.pdf}
\vspace{-0.5cm}

\includegraphics[width=9.0cm,angle=0]{ms2022-0304fig10-0469.pdf}
\includegraphics[width=9.0cm,angle=0]{ms2022-0304fig10-0470.pdf}
\end{figure}
\clearpage

\begin{figure}
\includegraphics[width=9.0cm,angle=0]{ms2022-0304fig10-0471.pdf}
\includegraphics[width=9.0cm,angle=0]{ms2022-0304fig10-0472.pdf}
\vspace{-0.5cm}

\includegraphics[width=9.0cm,angle=0]{ms2022-0304fig10-0473.pdf}
\includegraphics[width=9.0cm,angle=0]{ms2022-0304fig10-0474.pdf}
\vspace{-0.5cm}

\includegraphics[width=9.0cm,angle=0]{ms2022-0304fig10-0475.pdf}
\includegraphics[width=9.0cm,angle=0]{ms2022-0304fig10-0476.pdf}
\vspace{-0.5cm}

\includegraphics[width=9.0cm,angle=0]{ms2022-0304fig10-0477.pdf}
\includegraphics[width=9.0cm,angle=0]{ms2022-0304fig10-0478.pdf}
\vspace{-0.5cm}

\includegraphics[width=9.0cm,angle=0]{ms2022-0304fig10-0479.pdf}
\includegraphics[width=9.0cm,angle=0]{ms2022-0304fig10-0480.pdf}
\end{figure}
\clearpage

\begin{figure}
\includegraphics[width=9.0cm,angle=0]{ms2022-0304fig10-0481.pdf}
\includegraphics[width=9.0cm,angle=0]{ms2022-0304fig10-0482.pdf}
\vspace{-0.5cm}

\includegraphics[width=9.0cm,angle=0]{ms2022-0304fig10-0483.pdf}
\includegraphics[width=9.0cm,angle=0]{ms2022-0304fig10-0484.pdf}
\vspace{-0.5cm}

\includegraphics[width=9.0cm,angle=0]{ms2022-0304fig10-0485.pdf}
\includegraphics[width=9.0cm,angle=0]{ms2022-0304fig10-0486.pdf}
\vspace{-0.5cm}

\includegraphics[width=9.0cm,angle=0]{ms2022-0304fig10-0487.pdf}
\includegraphics[width=9.0cm,angle=0]{ms2022-0304fig10-0488.pdf}
\vspace{-0.5cm}

\includegraphics[width=9.0cm,angle=0]{ms2022-0304fig10-0489.pdf}
\includegraphics[width=9.0cm,angle=0]{ms2022-0304fig10-0490.pdf}
\end{figure}
\clearpage

\begin{figure}
\includegraphics[width=9.0cm,angle=0]{ms2022-0304fig10-0491.pdf}
\includegraphics[width=9.0cm,angle=0]{ms2022-0304fig10-0492.pdf}
\vspace{-0.5cm}

\includegraphics[width=9.0cm,angle=0]{ms2022-0304fig10-0493.pdf}
\includegraphics[width=9.0cm,angle=0]{ms2022-0304fig10-0494.pdf}
\vspace{-0.5cm}

\includegraphics[width=9.0cm,angle=0]{ms2022-0304fig10-0495.pdf}
\includegraphics[width=9.0cm,angle=0]{ms2022-0304fig10-0496.pdf}
\vspace{-0.5cm}

\includegraphics[width=9.0cm,angle=0]{ms2022-0304fig10-0497.pdf}
\includegraphics[width=9.0cm,angle=0]{ms2022-0304fig10-0498.pdf}
\vspace{-0.5cm}

\includegraphics[width=9.0cm,angle=0]{ms2022-0304fig10-0499.pdf}
\includegraphics[width=9.0cm,angle=0]{ms2022-0304fig10-0500.pdf}
\end{figure}
\clearpage

\begin{figure}
\includegraphics[width=9.0cm,angle=0]{ms2022-0304fig10-0501.pdf}
\includegraphics[width=9.0cm,angle=0]{ms2022-0304fig10-0502.pdf}
\vspace{-0.5cm}

\includegraphics[width=9.0cm,angle=0]{ms2022-0304fig10-0503.pdf}
\includegraphics[width=9.0cm,angle=0]{ms2022-0304fig10-0504.pdf}
\vspace{-0.5cm}

\includegraphics[width=9.0cm,angle=0]{ms2022-0304fig10-0505.pdf}
\includegraphics[width=9.0cm,angle=0]{ms2022-0304fig10-0506.pdf}
\vspace{-0.5cm}

\includegraphics[width=9.0cm,angle=0]{ms2022-0304fig10-0507.pdf}
\includegraphics[width=9.0cm,angle=0]{ms2022-0304fig10-0508.pdf}
\vspace{-0.5cm}

\includegraphics[width=9.0cm,angle=0]{ms2022-0304fig10-0509.pdf}
\includegraphics[width=9.0cm,angle=0]{ms2022-0304fig10-0510.pdf}
\end{figure}
\clearpage

\begin{figure}
\includegraphics[width=9.0cm,angle=0]{ms2022-0304fig10-0511.pdf}
\includegraphics[width=9.0cm,angle=0]{ms2022-0304fig10-0512.pdf}
\vspace{-0.5cm}

\includegraphics[width=9.0cm,angle=0]{ms2022-0304fig10-0513.pdf}
\includegraphics[width=9.0cm,angle=0]{ms2022-0304fig10-0514.pdf}
\vspace{-0.5cm}

\includegraphics[width=9.0cm,angle=0]{ms2022-0304fig10-0515.pdf}
\includegraphics[width=9.0cm,angle=0]{ms2022-0304fig10-0516.pdf}
\vspace{-0.5cm}

\includegraphics[width=9.0cm,angle=0]{ms2022-0304fig10-0517.pdf}
\includegraphics[width=9.0cm,angle=0]{ms2022-0304fig10-0518.pdf}
\vspace{-0.5cm}

\includegraphics[width=9.0cm,angle=0]{ms2022-0304fig10-0519.pdf}
\includegraphics[width=9.0cm,angle=0]{ms2022-0304fig10-0520.pdf}
\end{figure}
\clearpage

\begin{figure}
\includegraphics[width=9.0cm,angle=0]{ms2022-0304fig10-0521.pdf}
\includegraphics[width=9.0cm,angle=0]{ms2022-0304fig10-0522.pdf}
\vspace{-0.5cm}

\includegraphics[width=9.0cm,angle=0]{ms2022-0304fig10-0523.pdf}
\includegraphics[width=9.0cm,angle=0]{ms2022-0304fig10-0524.pdf}
\vspace{-0.5cm}

\includegraphics[width=9.0cm,angle=0]{ms2022-0304fig10-0525.pdf}
\includegraphics[width=9.0cm,angle=0]{ms2022-0304fig10-0526.pdf}
\vspace{-0.5cm}

\includegraphics[width=9.0cm,angle=0]{ms2022-0304fig10-0527.pdf}
\includegraphics[width=9.0cm,angle=0]{ms2022-0304fig10-0528.pdf}
\vspace{-0.5cm}

\includegraphics[width=9.0cm,angle=0]{ms2022-0304fig10-0529.pdf}
\includegraphics[width=9.0cm,angle=0]{ms2022-0304fig10-0530.pdf}
\end{figure}
\clearpage

\begin{figure}
\includegraphics[width=9.0cm,angle=0]{ms2022-0304fig10-0531.pdf}
\includegraphics[width=9.0cm,angle=0]{ms2022-0304fig10-0532.pdf}
\vspace{-0.5cm}

\includegraphics[width=9.0cm,angle=0]{ms2022-0304fig10-0533.pdf}
\includegraphics[width=9.0cm,angle=0]{ms2022-0304fig10-0534.pdf}
\vspace{-0.5cm}

\includegraphics[width=9.0cm,angle=0]{ms2022-0304fig10-0535.pdf}
\includegraphics[width=9.0cm,angle=0]{ms2022-0304fig10-0536.pdf}
\vspace{-0.5cm}

\includegraphics[width=9.0cm,angle=0]{ms2022-0304fig10-0537.pdf}
\includegraphics[width=9.0cm,angle=0]{ms2022-0304fig10-0538.pdf}
\vspace{-0.5cm}

\includegraphics[width=9.0cm,angle=0]{ms2022-0304fig10-0539.pdf}
\includegraphics[width=9.0cm,angle=0]{ms2022-0304fig10-0540.pdf}
\end{figure}
\clearpage

\begin{figure}
\includegraphics[width=9.0cm,angle=0]{ms2022-0304fig10-0541.pdf}
\includegraphics[width=9.0cm,angle=0]{ms2022-0304fig10-0542.pdf}
\vspace{-0.5cm}

\includegraphics[width=9.0cm,angle=0]{ms2022-0304fig10-0543.pdf}
\includegraphics[width=9.0cm,angle=0]{ms2022-0304fig10-0544.pdf}
\vspace{-0.5cm}

\includegraphics[width=9.0cm,angle=0]{ms2022-0304fig10-0545.pdf}
\includegraphics[width=9.0cm,angle=0]{ms2022-0304fig10-0546.pdf}
\vspace{-0.5cm}

\includegraphics[width=9.0cm,angle=0]{ms2022-0304fig10-0547.pdf}
\includegraphics[width=9.0cm,angle=0]{ms2022-0304fig10-0548.pdf}
\vspace{-0.5cm}

\includegraphics[width=9.0cm,angle=0]{ms2022-0304fig10-0549.pdf}
\includegraphics[width=9.0cm,angle=0]{ms2022-0304fig10-0550.pdf}
\end{figure}
\clearpage

\begin{figure}
\includegraphics[width=9.0cm,angle=0]{ms2022-0304fig10-0551.pdf}
\includegraphics[width=9.0cm,angle=0]{ms2022-0304fig10-0552.pdf}
\vspace{-0.5cm}

\includegraphics[width=9.0cm,angle=0]{ms2022-0304fig10-0553.pdf}
\includegraphics[width=9.0cm,angle=0]{ms2022-0304fig10-0554.pdf}
\vspace{-0.5cm}

\includegraphics[width=9.0cm,angle=0]{ms2022-0304fig10-0555.pdf}
\includegraphics[width=9.0cm,angle=0]{ms2022-0304fig10-0556.pdf}
\vspace{-0.5cm}

\includegraphics[width=9.0cm,angle=0]{ms2022-0304fig10-0557.pdf}
\includegraphics[width=9.0cm,angle=0]{ms2022-0304fig10-0558.pdf}
\vspace{-0.5cm}

\includegraphics[width=9.0cm,angle=0]{ms2022-0304fig10-0559.pdf}
\includegraphics[width=9.0cm,angle=0]{ms2022-0304fig10-0560.pdf}
\end{figure}
\clearpage

\begin{figure}
\includegraphics[width=9.0cm,angle=0]{ms2022-0304fig10-0561.pdf}
\includegraphics[width=9.0cm,angle=0]{ms2022-0304fig10-0562.pdf}
\vspace{-0.5cm}

\includegraphics[width=9.0cm,angle=0]{ms2022-0304fig10-0563.pdf}
\includegraphics[width=9.0cm,angle=0]{ms2022-0304fig10-0564.pdf}
\vspace{-0.5cm}

\includegraphics[width=9.0cm,angle=0]{ms2022-0304fig10-0565.pdf}
\includegraphics[width=9.0cm,angle=0]{ms2022-0304fig10-0566.pdf}
\vspace{-0.5cm}

\includegraphics[width=9.0cm,angle=0]{ms2022-0304fig10-0567.pdf}
\includegraphics[width=9.0cm,angle=0]{ms2022-0304fig10-0568.pdf}
\vspace{-0.5cm}

\includegraphics[width=9.0cm,angle=0]{ms2022-0304fig10-0569.pdf}
\includegraphics[width=9.0cm,angle=0]{ms2022-0304fig10-0570.pdf}
\end{figure}
\clearpage

\begin{figure}
\includegraphics[width=9.0cm,angle=0]{ms2022-0304fig10-0571.pdf}
\includegraphics[width=9.0cm,angle=0]{ms2022-0304fig10-0572.pdf}
\vspace{-0.5cm}

\includegraphics[width=9.0cm,angle=0]{ms2022-0304fig10-0573.pdf}
\includegraphics[width=9.0cm,angle=0]{ms2022-0304fig10-0574.pdf}
\vspace{-0.5cm}

\includegraphics[width=9.0cm,angle=0]{ms2022-0304fig10-0575.pdf}
\includegraphics[width=9.0cm,angle=0]{ms2022-0304fig10-0576.pdf}
\vspace{-0.5cm}

\includegraphics[width=9.0cm,angle=0]{ms2022-0304fig10-0577.pdf}
\includegraphics[width=9.0cm,angle=0]{ms2022-0304fig10-0578.pdf}
\vspace{-0.5cm}

\includegraphics[width=9.0cm,angle=0]{ms2022-0304fig10-0579.pdf}
\includegraphics[width=9.0cm,angle=0]{ms2022-0304fig10-0580.pdf}
\end{figure}
\clearpage

\begin{figure}
\includegraphics[width=9.0cm,angle=0]{ms2022-0304fig10-0581.pdf}
\includegraphics[width=9.0cm,angle=0]{ms2022-0304fig10-0582.pdf}
\vspace{-0.5cm}

\includegraphics[width=9.0cm,angle=0]{ms2022-0304fig10-0583.pdf}
\includegraphics[width=9.0cm,angle=0]{ms2022-0304fig10-0584.pdf}
\vspace{-0.5cm}

\includegraphics[width=9.0cm,angle=0]{ms2022-0304fig10-0585.pdf}
\includegraphics[width=9.0cm,angle=0]{ms2022-0304fig10-0586.pdf}
\vspace{-0.5cm}

\includegraphics[width=9.0cm,angle=0]{ms2022-0304fig10-0587.pdf}
\includegraphics[width=9.0cm,angle=0]{ms2022-0304fig10-0588.pdf}
\vspace{-0.5cm}

\includegraphics[width=9.0cm,angle=0]{ms2022-0304fig10-0589.pdf}
\includegraphics[width=9.0cm,angle=0]{ms2022-0304fig10-0590.pdf}
\end{figure}
\clearpage

\begin{figure}
\includegraphics[width=9.0cm,angle=0]{ms2022-0304fig10-0591.pdf}
\includegraphics[width=9.0cm,angle=0]{ms2022-0304fig10-0592.pdf}
\vspace{-0.5cm}

\includegraphics[width=9.0cm,angle=0]{ms2022-0304fig10-0593.pdf}
\includegraphics[width=9.0cm,angle=0]{ms2022-0304fig10-0594.pdf}
\vspace{-0.5cm}

\includegraphics[width=9.0cm,angle=0]{ms2022-0304fig10-0595.pdf}
\includegraphics[width=9.0cm,angle=0]{ms2022-0304fig10-0596.pdf}
\vspace{-0.5cm}

\includegraphics[width=9.0cm,angle=0]{ms2022-0304fig10-0597.pdf}
\includegraphics[width=9.0cm,angle=0]{ms2022-0304fig10-0598.pdf}
\vspace{-0.5cm}

\includegraphics[width=9.0cm,angle=0]{ms2022-0304fig10-0599.pdf}
\includegraphics[width=9.0cm,angle=0]{ms2022-0304fig10-0600.pdf}
\end{figure}
\clearpage

\begin{figure}
\includegraphics[width=9.0cm,angle=0]{ms2022-0304fig10-0601.pdf}
\includegraphics[width=9.0cm,angle=0]{ms2022-0304fig10-0602.pdf}
\vspace{-0.5cm}

\includegraphics[width=9.0cm,angle=0]{ms2022-0304fig10-0603.pdf}
\includegraphics[width=9.0cm,angle=0]{ms2022-0304fig10-0604.pdf}
\vspace{-0.5cm}

\includegraphics[width=9.0cm,angle=0]{ms2022-0304fig10-0605.pdf}
\includegraphics[width=9.0cm,angle=0]{ms2022-0304fig10-0606.pdf}
\vspace{-0.5cm}

\includegraphics[width=9.0cm,angle=0]{ms2022-0304fig10-0607.pdf}
\includegraphics[width=9.0cm,angle=0]{ms2022-0304fig10-0608.pdf}
\vspace{-0.5cm}

\includegraphics[width=9.0cm,angle=0]{ms2022-0304fig10-0609.pdf}
\includegraphics[width=9.0cm,angle=0]{ms2022-0304fig10-0610.pdf}
\end{figure}
\clearpage

\begin{figure}
\includegraphics[width=9.0cm,angle=0]{ms2022-0304fig10-0611.pdf}
\includegraphics[width=9.0cm,angle=0]{ms2022-0304fig10-0612.pdf}
\vspace{-0.5cm}

\includegraphics[width=9.0cm,angle=0]{ms2022-0304fig10-0613.pdf}
\includegraphics[width=9.0cm,angle=0]{ms2022-0304fig10-0614.pdf}
\vspace{-0.5cm}

\includegraphics[width=9.0cm,angle=0]{ms2022-0304fig10-0615.pdf}
\includegraphics[width=9.0cm,angle=0]{ms2022-0304fig10-0616.pdf}
\vspace{-0.5cm}

\includegraphics[width=9.0cm,angle=0]{ms2022-0304fig10-0617.pdf}
\includegraphics[width=9.0cm,angle=0]{ms2022-0304fig10-0618.pdf}
\vspace{-0.5cm}

\includegraphics[width=9.0cm,angle=0]{ms2022-0304fig10-0619.pdf}
\includegraphics[width=9.0cm,angle=0]{ms2022-0304fig10-0620.pdf}
\end{figure}
\clearpage

\begin{figure}
\includegraphics[width=9.0cm,angle=0]{ms2022-0304fig10-0621.pdf}
\includegraphics[width=9.0cm,angle=0]{ms2022-0304fig10-0622.pdf}
\vspace{-0.5cm}

\includegraphics[width=9.0cm,angle=0]{ms2022-0304fig10-0623.pdf}
\includegraphics[width=9.0cm,angle=0]{ms2022-0304fig10-0624.pdf}
\vspace{-0.5cm}

\includegraphics[width=9.0cm,angle=0]{ms2022-0304fig10-0625.pdf}
\includegraphics[width=9.0cm,angle=0]{ms2022-0304fig10-0626.pdf}
\vspace{-0.5cm}

\includegraphics[width=9.0cm,angle=0]{ms2022-0304fig10-0627.pdf}
\includegraphics[width=9.0cm,angle=0]{ms2022-0304fig10-0628.pdf}
\vspace{-0.5cm}

\includegraphics[width=9.0cm,angle=0]{ms2022-0304fig10-0629.pdf}
\includegraphics[width=9.0cm,angle=0]{ms2022-0304fig10-0630.pdf}
\end{figure}
\clearpage

\begin{figure}
\includegraphics[width=9.0cm,angle=0]{ms2022-0304fig10-0631.pdf}
\includegraphics[width=9.0cm,angle=0]{ms2022-0304fig10-0632.pdf}
\vspace{-0.5cm}

\includegraphics[width=9.0cm,angle=0]{ms2022-0304fig10-0633.pdf}
\includegraphics[width=9.0cm,angle=0]{ms2022-0304fig10-0634.pdf}
\vspace{-0.5cm}

\includegraphics[width=9.0cm,angle=0]{ms2022-0304fig10-0635.pdf}
\includegraphics[width=9.0cm,angle=0]{ms2022-0304fig10-0636.pdf}
\vspace{-0.5cm}

\includegraphics[width=9.0cm,angle=0]{ms2022-0304fig10-0637.pdf}
\includegraphics[width=9.0cm,angle=0]{ms2022-0304fig10-0638.pdf}
\vspace{-0.5cm}

\includegraphics[width=9.0cm,angle=0]{ms2022-0304fig10-0639.pdf}
\includegraphics[width=9.0cm,angle=0]{ms2022-0304fig10-0640.pdf}
\end{figure}
\clearpage

\begin{figure}
\includegraphics[width=9.0cm,angle=0]{ms2022-0304fig10-0641.pdf}
\includegraphics[width=9.0cm,angle=0]{ms2022-0304fig10-0642.pdf}
\vspace{-0.5cm}

\includegraphics[width=9.0cm,angle=0]{ms2022-0304fig10-0643.pdf}
\includegraphics[width=9.0cm,angle=0]{ms2022-0304fig10-0644.pdf}
\vspace{-0.5cm}

\includegraphics[width=9.0cm,angle=0]{ms2022-0304fig10-0645.pdf}
\includegraphics[width=9.0cm,angle=0]{ms2022-0304fig10-0646.pdf}
\vspace{-0.5cm}

\includegraphics[width=9.0cm,angle=0]{ms2022-0304fig10-0647.pdf}
\includegraphics[width=9.0cm,angle=0]{ms2022-0304fig10-0648.pdf}
\vspace{-0.5cm}

\includegraphics[width=9.0cm,angle=0]{ms2022-0304fig10-0649.pdf}
\includegraphics[width=9.0cm,angle=0]{ms2022-0304fig10-0650.pdf}
\end{figure}
\clearpage

\begin{figure}
\includegraphics[width=9.0cm,angle=0]{ms2022-0304fig10-0651.pdf}
\includegraphics[width=9.0cm,angle=0]{ms2022-0304fig10-0652.pdf}
\vspace{-0.5cm}

\includegraphics[width=9.0cm,angle=0]{ms2022-0304fig10-0653.pdf}
\includegraphics[width=9.0cm,angle=0]{ms2022-0304fig10-0654.pdf}
\vspace{-0.5cm}

\includegraphics[width=9.0cm,angle=0]{ms2022-0304fig10-0655.pdf}
\includegraphics[width=9.0cm,angle=0]{ms2022-0304fig10-0656.pdf}
\vspace{-0.5cm}

\includegraphics[width=9.0cm,angle=0]{ms2022-0304fig10-0657.pdf}
\includegraphics[width=9.0cm,angle=0]{ms2022-0304fig10-0658.pdf}
\vspace{-0.5cm}

\includegraphics[width=9.0cm,angle=0]{ms2022-0304fig10-0659.pdf}
\includegraphics[width=9.0cm,angle=0]{ms2022-0304fig10-0660.pdf}
\end{figure}
\clearpage

\begin{figure}
\includegraphics[width=9.0cm,angle=0]{ms2022-0304fig10-0661.pdf}
\includegraphics[width=9.0cm,angle=0]{ms2022-0304fig10-0662.pdf}
\vspace{-0.5cm}

\includegraphics[width=9.0cm,angle=0]{ms2022-0304fig10-0663.pdf}
\includegraphics[width=9.0cm,angle=0]{ms2022-0304fig10-0664.pdf}
\vspace{-0.5cm}

\includegraphics[width=9.0cm,angle=0]{ms2022-0304fig10-0665.pdf}
\includegraphics[width=9.0cm,angle=0]{ms2022-0304fig10-0666.pdf}
\vspace{-0.5cm}

\includegraphics[width=9.0cm,angle=0]{ms2022-0304fig10-0667.pdf}
\includegraphics[width=9.0cm,angle=0]{ms2022-0304fig10-0668.pdf}
\vspace{-0.5cm}

\includegraphics[width=9.0cm,angle=0]{ms2022-0304fig10-0669.pdf}
\includegraphics[width=9.0cm,angle=0]{ms2022-0304fig10-0670.pdf}
\end{figure}
\clearpage

\begin{figure}
\includegraphics[width=9.0cm,angle=0]{ms2022-0304fig10-0671.pdf}
\includegraphics[width=9.0cm,angle=0]{ms2022-0304fig10-0672.pdf}
\vspace{-0.5cm}

\includegraphics[width=9.0cm,angle=0]{ms2022-0304fig10-0673.pdf}
\includegraphics[width=9.0cm,angle=0]{ms2022-0304fig10-0674.pdf}
\vspace{-0.5cm}

\includegraphics[width=9.0cm,angle=0]{ms2022-0304fig10-0675.pdf}
\includegraphics[width=9.0cm,angle=0]{ms2022-0304fig10-0676.pdf}
\vspace{-0.5cm}

\includegraphics[width=9.0cm,angle=0]{ms2022-0304fig10-0677.pdf}
\includegraphics[width=9.0cm,angle=0]{ms2022-0304fig10-0678.pdf}
\vspace{-0.5cm}

\includegraphics[width=9.0cm,angle=0]{ms2022-0304fig10-0679.pdf}
\includegraphics[width=9.0cm,angle=0]{ms2022-0304fig10-0680.pdf}
\end{figure}
\clearpage

\begin{figure}
\includegraphics[width=9.0cm,angle=0]{ms2022-0304fig10-0681.pdf}
\includegraphics[width=9.0cm,angle=0]{ms2022-0304fig10-0682.pdf}
\vspace{-0.5cm}

\includegraphics[width=9.0cm,angle=0]{ms2022-0304fig10-0683.pdf}
\includegraphics[width=9.0cm,angle=0]{ms2022-0304fig10-0684.pdf}
\vspace{-0.5cm}

\includegraphics[width=9.0cm,angle=0]{ms2022-0304fig10-0685.pdf}
\includegraphics[width=9.0cm,angle=0]{ms2022-0304fig10-0686.pdf}
\vspace{-0.5cm}

\includegraphics[width=9.0cm,angle=0]{ms2022-0304fig10-0687.pdf}
\includegraphics[width=9.0cm,angle=0]{ms2022-0304fig10-0688.pdf}
\vspace{-0.5cm}

\includegraphics[width=9.0cm,angle=0]{ms2022-0304fig10-0689.pdf}
\includegraphics[width=9.0cm,angle=0]{ms2022-0304fig10-0690.pdf}
\end{figure}
\clearpage

\begin{figure}
\includegraphics[width=9.0cm,angle=0]{ms2022-0304fig10-0691.pdf}
\includegraphics[width=9.0cm,angle=0]{ms2022-0304fig10-0692.pdf}
\vspace{-0.5cm}

\includegraphics[width=9.0cm,angle=0]{ms2022-0304fig10-0693.pdf}
\includegraphics[width=9.0cm,angle=0]{ms2022-0304fig10-0694.pdf}
\vspace{-0.5cm}

\includegraphics[width=9.0cm,angle=0]{ms2022-0304fig10-0695.pdf}
\includegraphics[width=9.0cm,angle=0]{ms2022-0304fig10-0696.pdf}
\vspace{-0.5cm}

\includegraphics[width=9.0cm,angle=0]{ms2022-0304fig10-0697.pdf}
\includegraphics[width=9.0cm,angle=0]{ms2022-0304fig10-0698.pdf}
\vspace{-0.5cm}

\includegraphics[width=9.0cm,angle=0]{ms2022-0304fig10-0699.pdf}
\includegraphics[width=9.0cm,angle=0]{ms2022-0304fig10-0700.pdf}
\end{figure}
\clearpage

\begin{figure}
\includegraphics[width=9.0cm,angle=0]{ms2022-0304fig10-0701.pdf}
\includegraphics[width=9.0cm,angle=0]{ms2022-0304fig10-0702.pdf}
\vspace{-0.5cm}

\includegraphics[width=9.0cm,angle=0]{ms2022-0304fig10-0703.pdf}
\includegraphics[width=9.0cm,angle=0]{ms2022-0304fig10-0704.pdf}
\vspace{-0.5cm}

\includegraphics[width=9.0cm,angle=0]{ms2022-0304fig10-0705.pdf}
\includegraphics[width=9.0cm,angle=0]{ms2022-0304fig10-0706.pdf}
\vspace{-0.5cm}

\includegraphics[width=9.0cm,angle=0]{ms2022-0304fig10-0707.pdf}
\includegraphics[width=9.0cm,angle=0]{ms2022-0304fig10-0708.pdf}
\vspace{-0.5cm}

\includegraphics[width=9.0cm,angle=0]{ms2022-0304fig10-0709.pdf}
\includegraphics[width=9.0cm,angle=0]{ms2022-0304fig10-0710.pdf}
\end{figure}
\clearpage

\begin{figure}
\includegraphics[width=9.0cm,angle=0]{ms2022-0304fig10-0711.pdf}
\includegraphics[width=9.0cm,angle=0]{ms2022-0304fig10-0712.pdf}
\vspace{-0.5cm}

\includegraphics[width=9.0cm,angle=0]{ms2022-0304fig10-0713.pdf}
\includegraphics[width=9.0cm,angle=0]{ms2022-0304fig10-0714.pdf}
\vspace{-0.5cm}

\includegraphics[width=9.0cm,angle=0]{ms2022-0304fig10-0715.pdf}
\includegraphics[width=9.0cm,angle=0]{ms2022-0304fig10-0716.pdf}
\vspace{-0.5cm}

\includegraphics[width=9.0cm,angle=0]{ms2022-0304fig10-0717.pdf}
\includegraphics[width=9.0cm,angle=0]{ms2022-0304fig10-0718.pdf}
\vspace{-0.5cm}

\includegraphics[width=9.0cm,angle=0]{ms2022-0304fig10-0719.pdf}
\includegraphics[width=9.0cm,angle=0]{ms2022-0304fig10-0720.pdf}
\end{figure}
\clearpage

\begin{figure}
\includegraphics[width=9.0cm,angle=0]{ms2022-0304fig10-0721.pdf}
\includegraphics[width=9.0cm,angle=0]{ms2022-0304fig10-0722.pdf}
\vspace{-0.5cm}

\includegraphics[width=9.0cm,angle=0]{ms2022-0304fig10-0723.pdf}
\includegraphics[width=9.0cm,angle=0]{ms2022-0304fig10-0724.pdf}
\vspace{-0.5cm}

\includegraphics[width=9.0cm,angle=0]{ms2022-0304fig10-0725.pdf}
\includegraphics[width=9.0cm,angle=0]{ms2022-0304fig10-0726.pdf}
\vspace{-0.5cm}

\includegraphics[width=9.0cm,angle=0]{ms2022-0304fig10-0727.pdf}
\includegraphics[width=9.0cm,angle=0]{ms2022-0304fig10-0728.pdf}
\vspace{-0.5cm}

\includegraphics[width=9.0cm,angle=0]{ms2022-0304fig10-0729.pdf}
\includegraphics[width=9.0cm,angle=0]{ms2022-0304fig10-0730.pdf}
\end{figure}
\clearpage

\begin{figure}
\includegraphics[width=9.0cm,angle=0]{ms2022-0304fig10-0731.pdf}
\includegraphics[width=9.0cm,angle=0]{ms2022-0304fig10-0732.pdf}
\vspace{-0.5cm}

\includegraphics[width=9.0cm,angle=0]{ms2022-0304fig10-0733.pdf}
\includegraphics[width=9.0cm,angle=0]{ms2022-0304fig10-0734.pdf}
\vspace{-0.5cm}

\includegraphics[width=9.0cm,angle=0]{ms2022-0304fig10-0735.pdf}
\includegraphics[width=9.0cm,angle=0]{ms2022-0304fig10-0736.pdf}
\vspace{-0.5cm}

\includegraphics[width=9.0cm,angle=0]{ms2022-0304fig10-0737.pdf}
\includegraphics[width=9.0cm,angle=0]{ms2022-0304fig10-0738.pdf}
\vspace{-0.5cm}

\includegraphics[width=9.0cm,angle=0]{ms2022-0304fig10-0739.pdf}
\includegraphics[width=9.0cm,angle=0]{ms2022-0304fig10-0740.pdf}
\end{figure}
\clearpage

\begin{figure}
\includegraphics[width=9.0cm,angle=0]{ms2022-0304fig10-0741.pdf}
\includegraphics[width=9.0cm,angle=0]{ms2022-0304fig10-0742.pdf}
\vspace{-0.5cm}

\includegraphics[width=9.0cm,angle=0]{ms2022-0304fig10-0743.pdf}
\includegraphics[width=9.0cm,angle=0]{ms2022-0304fig10-0744.pdf}
\vspace{-0.5cm}

\includegraphics[width=9.0cm,angle=0]{ms2022-0304fig10-0745.pdf}
\includegraphics[width=9.0cm,angle=0]{ms2022-0304fig10-0746.pdf}
\vspace{-0.5cm}

\includegraphics[width=9.0cm,angle=0]{ms2022-0304fig10-0747.pdf}
\includegraphics[width=9.0cm,angle=0]{ms2022-0304fig10-0748.pdf}
\vspace{-0.5cm}

\includegraphics[width=9.0cm,angle=0]{ms2022-0304fig10-0749.pdf}
\includegraphics[width=9.0cm,angle=0]{ms2022-0304fig10-0750.pdf}
\end{figure}
\clearpage

\begin{figure}
\includegraphics[width=9.0cm,angle=0]{ms2022-0304fig10-0751.pdf}
\includegraphics[width=9.0cm,angle=0]{ms2022-0304fig10-0752.pdf}
\vspace{-0.5cm}

\includegraphics[width=9.0cm,angle=0]{ms2022-0304fig10-0753.pdf}
\includegraphics[width=9.0cm,angle=0]{ms2022-0304fig10-0754.pdf}
\vspace{-0.5cm}

\includegraphics[width=9.0cm,angle=0]{ms2022-0304fig10-0755.pdf}
\includegraphics[width=9.0cm,angle=0]{ms2022-0304fig10-0756.pdf}
\vspace{-0.5cm}

\includegraphics[width=9.0cm,angle=0]{ms2022-0304fig10-0757.pdf}
\includegraphics[width=9.0cm,angle=0]{ms2022-0304fig10-0758.pdf}
\vspace{-0.5cm}

\includegraphics[width=9.0cm,angle=0]{ms2022-0304fig10-0759.pdf}
\includegraphics[width=9.0cm,angle=0]{ms2022-0304fig10-0760.pdf}
\end{figure}
\clearpage

\begin{figure}
\includegraphics[width=9.0cm,angle=0]{ms2022-0304fig10-0761.pdf}
\includegraphics[width=9.0cm,angle=0]{ms2022-0304fig10-0762.pdf}
\vspace{-0.5cm}

\includegraphics[width=9.0cm,angle=0]{ms2022-0304fig10-0763.pdf}
\includegraphics[width=9.0cm,angle=0]{ms2022-0304fig10-0764.pdf}
\vspace{-0.5cm}

\includegraphics[width=9.0cm,angle=0]{ms2022-0304fig10-0765.pdf}
\includegraphics[width=9.0cm,angle=0]{ms2022-0304fig10-0766.pdf}
\vspace{-0.5cm}

\includegraphics[width=9.0cm,angle=0]{ms2022-0304fig10-0767.pdf}
\includegraphics[width=9.0cm,angle=0]{ms2022-0304fig10-0768.pdf}
\vspace{-0.5cm}

\includegraphics[width=9.0cm,angle=0]{ms2022-0304fig10-0769.pdf}
\includegraphics[width=9.0cm,angle=0]{ms2022-0304fig10-0770.pdf}
\end{figure}
\clearpage

\begin{figure}
\includegraphics[width=9.0cm,angle=0]{ms2022-0304fig10-0771.pdf}
\includegraphics[width=9.0cm,angle=0]{ms2022-0304fig10-0772.pdf}
\vspace{-0.5cm}

\includegraphics[width=9.0cm,angle=0]{ms2022-0304fig10-0773.pdf}
\includegraphics[width=9.0cm,angle=0]{ms2022-0304fig10-0774.pdf}
\vspace{-0.5cm}

\includegraphics[width=9.0cm,angle=0]{ms2022-0304fig10-0775.pdf}
\includegraphics[width=9.0cm,angle=0]{ms2022-0304fig10-0776.pdf}
\vspace{-0.5cm}

\includegraphics[width=9.0cm,angle=0]{ms2022-0304fig10-0777.pdf}
\includegraphics[width=9.0cm,angle=0]{ms2022-0304fig10-0778.pdf}
\vspace{-0.5cm}

\includegraphics[width=9.0cm,angle=0]{ms2022-0304fig10-0779.pdf}
\includegraphics[width=9.0cm,angle=0]{ms2022-0304fig10-0780.pdf}
\end{figure}
\clearpage

\begin{figure}
\includegraphics[width=9.0cm,angle=0]{ms2022-0304fig10-0781.pdf}
\includegraphics[width=9.0cm,angle=0]{ms2022-0304fig10-0782.pdf}
\vspace{-0.5cm}

\includegraphics[width=9.0cm,angle=0]{ms2022-0304fig10-0783.pdf}
\includegraphics[width=9.0cm,angle=0]{ms2022-0304fig10-0784.pdf}
\vspace{-0.5cm}

\includegraphics[width=9.0cm,angle=0]{ms2022-0304fig10-0785.pdf}
\includegraphics[width=9.0cm,angle=0]{ms2022-0304fig10-0786.pdf}
\vspace{-0.5cm}

\includegraphics[width=9.0cm,angle=0]{ms2022-0304fig10-0787.pdf}
\includegraphics[width=9.0cm,angle=0]{ms2022-0304fig10-0788.pdf}
\vspace{-0.5cm}

\includegraphics[width=9.0cm,angle=0]{ms2022-0304fig10-0789.pdf}
\includegraphics[width=9.0cm,angle=0]{ms2022-0304fig10-0790.pdf}
\end{figure}
\clearpage

\begin{figure}
\includegraphics[width=9.0cm,angle=0]{ms2022-0304fig10-0791.pdf}
\includegraphics[width=9.0cm,angle=0]{ms2022-0304fig10-0792.pdf}
\vspace{-0.5cm}

\includegraphics[width=9.0cm,angle=0]{ms2022-0304fig10-0793.pdf}
\includegraphics[width=9.0cm,angle=0]{ms2022-0304fig10-0794.pdf}
\vspace{-0.5cm}

\includegraphics[width=9.0cm,angle=0]{ms2022-0304fig10-0795.pdf}
\includegraphics[width=9.0cm,angle=0]{ms2022-0304fig10-0796.pdf}
\vspace{-0.5cm}

\includegraphics[width=9.0cm,angle=0]{ms2022-0304fig10-0797.pdf}
\includegraphics[width=9.0cm,angle=0]{ms2022-0304fig10-0798.pdf}
\vspace{-0.5cm}

\includegraphics[width=9.0cm,angle=0]{ms2022-0304fig10-0799.pdf}
\includegraphics[width=9.0cm,angle=0]{ms2022-0304fig10-0800.pdf}
\end{figure}
\clearpage

\begin{figure}
\includegraphics[width=9.0cm,angle=0]{ms2022-0304fig10-0801.pdf}
\includegraphics[width=9.0cm,angle=0]{ms2022-0304fig10-0802.pdf}
\vspace{-0.5cm}

\includegraphics[width=9.0cm,angle=0]{ms2022-0304fig10-0803.pdf}
\includegraphics[width=9.0cm,angle=0]{ms2022-0304fig10-0804.pdf}
\vspace{-0.5cm}

\includegraphics[width=9.0cm,angle=0]{ms2022-0304fig10-0805.pdf}
\includegraphics[width=9.0cm,angle=0]{ms2022-0304fig10-0806.pdf}
\vspace{-0.5cm}

\includegraphics[width=9.0cm,angle=0]{ms2022-0304fig10-0807.pdf}
\includegraphics[width=9.0cm,angle=0]{ms2022-0304fig10-0808.pdf}
\vspace{-0.5cm}

\includegraphics[width=9.0cm,angle=0]{ms2022-0304fig10-0809.pdf}
\includegraphics[width=9.0cm,angle=0]{ms2022-0304fig10-0810.pdf}
\end{figure}
\clearpage

\begin{figure}
\includegraphics[width=9.0cm,angle=0]{ms2022-0304fig10-0811.pdf}
\includegraphics[width=9.0cm,angle=0]{ms2022-0304fig10-0812.pdf}
\vspace{-0.5cm}

\includegraphics[width=9.0cm,angle=0]{ms2022-0304fig10-0813.pdf}
\includegraphics[width=9.0cm,angle=0]{ms2022-0304fig10-0814.pdf}
\vspace{-0.5cm}

\includegraphics[width=9.0cm,angle=0]{ms2022-0304fig10-0815.pdf}
\includegraphics[width=9.0cm,angle=0]{ms2022-0304fig10-0816.pdf}
\vspace{-0.5cm}

\includegraphics[width=9.0cm,angle=0]{ms2022-0304fig10-0817.pdf}
\includegraphics[width=9.0cm,angle=0]{ms2022-0304fig10-0818.pdf}
\vspace{-0.5cm}

\includegraphics[width=9.0cm,angle=0]{ms2022-0304fig10-0819.pdf}
\includegraphics[width=9.0cm,angle=0]{ms2022-0304fig10-0820.pdf}
\end{figure}
\clearpage

\begin{figure}
\includegraphics[width=9.0cm,angle=0]{ms2022-0304fig10-0821.pdf}
\includegraphics[width=9.0cm,angle=0]{ms2022-0304fig10-0822.pdf}
\vspace{-0.5cm}

\includegraphics[width=9.0cm,angle=0]{ms2022-0304fig10-0823.pdf}
\includegraphics[width=9.0cm,angle=0]{ms2022-0304fig10-0824.pdf}
\vspace{-0.5cm}

\includegraphics[width=9.0cm,angle=0]{ms2022-0304fig10-0825.pdf}
\includegraphics[width=9.0cm,angle=0]{ms2022-0304fig10-0826.pdf}
\vspace{-0.5cm}

\includegraphics[width=9.0cm,angle=0]{ms2022-0304fig10-0827.pdf}
\includegraphics[width=9.0cm,angle=0]{ms2022-0304fig10-0828.pdf}
\vspace{-0.5cm}

\includegraphics[width=9.0cm,angle=0]{ms2022-0304fig10-0829.pdf}
\includegraphics[width=9.0cm,angle=0]{ms2022-0304fig10-0830.pdf}
\end{figure}
\clearpage

\begin{figure}
\includegraphics[width=9.0cm,angle=0]{ms2022-0304fig10-0831.pdf}
\includegraphics[width=9.0cm,angle=0]{ms2022-0304fig10-0832.pdf}
\vspace{-0.5cm}

\includegraphics[width=9.0cm,angle=0]{ms2022-0304fig10-0833.pdf}
\includegraphics[width=9.0cm,angle=0]{ms2022-0304fig10-0834.pdf}
\vspace{-0.5cm}

\includegraphics[width=9.0cm,angle=0]{ms2022-0304fig10-0835.pdf}
\includegraphics[width=9.0cm,angle=0]{ms2022-0304fig10-0836.pdf}
\vspace{-0.5cm}

\includegraphics[width=9.0cm,angle=0]{ms2022-0304fig10-0837.pdf}
\includegraphics[width=9.0cm,angle=0]{ms2022-0304fig10-0838.pdf}
\vspace{-0.5cm}

\includegraphics[width=9.0cm,angle=0]{ms2022-0304fig10-0839.pdf}
\includegraphics[width=9.0cm,angle=0]{ms2022-0304fig10-0840.pdf}
\end{figure}
\clearpage

\begin{figure}
\includegraphics[width=9.0cm,angle=0]{ms2022-0304fig10-0841.pdf}
\includegraphics[width=9.0cm,angle=0]{ms2022-0304fig10-0842.pdf}
\vspace{-0.5cm}

\includegraphics[width=9.0cm,angle=0]{ms2022-0304fig10-0843.pdf}
\includegraphics[width=9.0cm,angle=0]{ms2022-0304fig10-0844.pdf}
\vspace{-0.5cm}

\includegraphics[width=9.0cm,angle=0]{ms2022-0304fig10-0845.pdf}
\includegraphics[width=9.0cm,angle=0]{ms2022-0304fig10-0846.pdf}
\vspace{-0.5cm}

\includegraphics[width=9.0cm,angle=0]{ms2022-0304fig10-0847.pdf}
\includegraphics[width=9.0cm,angle=0]{ms2022-0304fig10-0848.pdf}
\vspace{-0.5cm}

\includegraphics[width=9.0cm,angle=0]{ms2022-0304fig10-0849.pdf}
\includegraphics[width=9.0cm,angle=0]{ms2022-0304fig10-0850.pdf}
\end{figure}
\clearpage

\begin{figure}
\includegraphics[width=9.0cm,angle=0]{ms2022-0304fig10-0851.pdf}
\includegraphics[width=9.0cm,angle=0]{ms2022-0304fig10-0852.pdf}
\vspace{-0.5cm}

\includegraphics[width=9.0cm,angle=0]{ms2022-0304fig10-0853.pdf}
\includegraphics[width=9.0cm,angle=0]{ms2022-0304fig10-0854.pdf}
\vspace{-0.5cm}

\includegraphics[width=9.0cm,angle=0]{ms2022-0304fig10-0855.pdf}
\includegraphics[width=9.0cm,angle=0]{ms2022-0304fig10-0856.pdf}
\vspace{-0.5cm}

\includegraphics[width=9.0cm,angle=0]{ms2022-0304fig10-0857.pdf}
\includegraphics[width=9.0cm,angle=0]{ms2022-0304fig10-0858.pdf}
\vspace{-0.5cm}

\includegraphics[width=9.0cm,angle=0]{ms2022-0304fig10-0859.pdf}
\includegraphics[width=9.0cm,angle=0]{ms2022-0304fig10-0860.pdf}
\end{figure}
\clearpage

\begin{figure}
\includegraphics[width=9.0cm,angle=0]{ms2022-0304fig10-0861.pdf}
\includegraphics[width=9.0cm,angle=0]{ms2022-0304fig10-0862.pdf}
\vspace{-0.5cm}

\includegraphics[width=9.0cm,angle=0]{ms2022-0304fig10-0863.pdf}
\includegraphics[width=9.0cm,angle=0]{ms2022-0304fig10-0864.pdf}
\vspace{-0.5cm}

\includegraphics[width=9.0cm,angle=0]{ms2022-0304fig10-0865.pdf}
\includegraphics[width=9.0cm,angle=0]{ms2022-0304fig10-0866.pdf}
\vspace{-0.5cm}

\includegraphics[width=9.0cm,angle=0]{ms2022-0304fig10-0867.pdf}
\includegraphics[width=9.0cm,angle=0]{ms2022-0304fig10-0868.pdf}
\vspace{-0.5cm}

\includegraphics[width=9.0cm,angle=0]{ms2022-0304fig10-0869.pdf}
\includegraphics[width=9.0cm,angle=0]{ms2022-0304fig10-0870.pdf}
\end{figure}
\clearpage

\begin{figure}
\includegraphics[width=9.0cm,angle=0]{ms2022-0304fig10-0871.pdf}
\includegraphics[width=9.0cm,angle=0]{ms2022-0304fig10-0872.pdf}
\vspace{-0.5cm}

\includegraphics[width=9.0cm,angle=0]{ms2022-0304fig10-0873.pdf}
\includegraphics[width=9.0cm,angle=0]{ms2022-0304fig10-0874.pdf}
\vspace{-0.5cm}

\includegraphics[width=9.0cm,angle=0]{ms2022-0304fig10-0875.pdf}
\includegraphics[width=9.0cm,angle=0]{ms2022-0304fig10-0876.pdf}
\vspace{-0.5cm}

\includegraphics[width=9.0cm,angle=0]{ms2022-0304fig10-0877.pdf}
\includegraphics[width=9.0cm,angle=0]{ms2022-0304fig10-0878.pdf}
\vspace{-0.5cm}

\includegraphics[width=9.0cm,angle=0]{ms2022-0304fig10-0879.pdf}
\includegraphics[width=9.0cm,angle=0]{ms2022-0304fig10-0880.pdf}
\end{figure}
\clearpage

\begin{figure}
\includegraphics[width=9.0cm,angle=0]{ms2022-0304fig10-0881.pdf}
\includegraphics[width=9.0cm,angle=0]{ms2022-0304fig10-0882.pdf}
\vspace{-0.5cm}

\includegraphics[width=9.0cm,angle=0]{ms2022-0304fig10-0883.pdf}
\includegraphics[width=9.0cm,angle=0]{ms2022-0304fig10-0884.pdf}
\vspace{-0.5cm}

\includegraphics[width=9.0cm,angle=0]{ms2022-0304fig10-0885.pdf}
\includegraphics[width=9.0cm,angle=0]{ms2022-0304fig10-0886.pdf}
\vspace{-0.5cm}

\includegraphics[width=9.0cm,angle=0]{ms2022-0304fig10-0887.pdf}
\includegraphics[width=9.0cm,angle=0]{ms2022-0304fig10-0888.pdf}
\vspace{-0.5cm}

\includegraphics[width=9.0cm,angle=0]{ms2022-0304fig10-0889.pdf}
\includegraphics[width=9.0cm,angle=0]{ms2022-0304fig10-0890.pdf}
\end{figure}
\clearpage

\begin{figure}
\includegraphics[width=9.0cm,angle=0]{ms2022-0304fig10-0891.pdf}
\includegraphics[width=9.0cm,angle=0]{ms2022-0304fig10-0892.pdf}
\vspace{-0.5cm}

\includegraphics[width=9.0cm,angle=0]{ms2022-0304fig10-0893.pdf}
\includegraphics[width=9.0cm,angle=0]{ms2022-0304fig10-0894.pdf}
\vspace{-0.5cm}

\includegraphics[width=9.0cm,angle=0]{ms2022-0304fig10-0895.pdf}
\includegraphics[width=9.0cm,angle=0]{ms2022-0304fig10-0896.pdf}
\vspace{-0.5cm}

\includegraphics[width=9.0cm,angle=0]{ms2022-0304fig10-0897.pdf}
\includegraphics[width=9.0cm,angle=0]{ms2022-0304fig10-0898.pdf}
\vspace{-0.5cm}

\includegraphics[width=9.0cm,angle=0]{ms2022-0304fig10-0899.pdf}
\includegraphics[width=9.0cm,angle=0]{ms2022-0304fig10-0900.pdf}
\end{figure}
\clearpage

\begin{figure}
\includegraphics[width=9.0cm,angle=0]{ms2022-0304fig10-0901.pdf}
\includegraphics[width=9.0cm,angle=0]{ms2022-0304fig10-0902.pdf}
\vspace{-0.5cm}

\includegraphics[width=9.0cm,angle=0]{ms2022-0304fig10-0903.pdf}
\includegraphics[width=9.0cm,angle=0]{ms2022-0304fig10-0904.pdf}
\vspace{-0.5cm}

\includegraphics[width=9.0cm,angle=0]{ms2022-0304fig10-0905.pdf}
\includegraphics[width=9.0cm,angle=0]{ms2022-0304fig10-0906.pdf}
\vspace{-0.5cm}

\includegraphics[width=9.0cm,angle=0]{ms2022-0304fig10-0907.pdf}
\includegraphics[width=9.0cm,angle=0]{ms2022-0304fig10-0908.pdf}
\vspace{-0.5cm}

\includegraphics[width=9.0cm,angle=0]{ms2022-0304fig10-0909.pdf}
\includegraphics[width=9.0cm,angle=0]{ms2022-0304fig10-0910.pdf}
\end{figure}
\clearpage

\begin{figure}
\includegraphics[width=9.0cm,angle=0]{ms2022-0304fig10-0911.pdf}
\includegraphics[width=9.0cm,angle=0]{ms2022-0304fig10-0912.pdf}
\vspace{-0.5cm}

\includegraphics[width=9.0cm,angle=0]{ms2022-0304fig10-0913.pdf}
\includegraphics[width=9.0cm,angle=0]{ms2022-0304fig10-0914.pdf}
\vspace{-0.5cm}

\includegraphics[width=9.0cm,angle=0]{ms2022-0304fig10-0915.pdf}
\includegraphics[width=9.0cm,angle=0]{ms2022-0304fig10-0916.pdf}
\vspace{-0.5cm}

\includegraphics[width=9.0cm,angle=0]{ms2022-0304fig10-0917.pdf}
\includegraphics[width=9.0cm,angle=0]{ms2022-0304fig10-0918.pdf}
\vspace{-0.5cm}

\includegraphics[width=9.0cm,angle=0]{ms2022-0304fig10-0919.pdf}
\includegraphics[width=9.0cm,angle=0]{ms2022-0304fig10-0920.pdf}
\end{figure}
\clearpage

\begin{figure}
\includegraphics[width=9.0cm,angle=0]{ms2022-0304fig10-0921.pdf}
\includegraphics[width=9.0cm,angle=0]{ms2022-0304fig10-0922.pdf}
\vspace{-0.5cm}

\includegraphics[width=9.0cm,angle=0]{ms2022-0304fig10-0923.pdf}
\includegraphics[width=9.0cm,angle=0]{ms2022-0304fig10-0924.pdf}
\vspace{-0.5cm}

\includegraphics[width=9.0cm,angle=0]{ms2022-0304fig10-0925.pdf}
\includegraphics[width=9.0cm,angle=0]{ms2022-0304fig10-0926.pdf}
\vspace{-0.5cm}

\includegraphics[width=9.0cm,angle=0]{ms2022-0304fig10-0927.pdf}
\includegraphics[width=9.0cm,angle=0]{ms2022-0304fig10-0928.pdf}
\vspace{-0.5cm}

\includegraphics[width=9.0cm,angle=0]{ms2022-0304fig10-0929.pdf}
\includegraphics[width=9.0cm,angle=0]{ms2022-0304fig10-0930.pdf}
\end{figure}
\clearpage

\begin{figure}
\includegraphics[width=9.0cm,angle=0]{ms2022-0304fig10-0931.pdf}
\includegraphics[width=9.0cm,angle=0]{ms2022-0304fig10-0932.pdf}
\vspace{-0.5cm}

\includegraphics[width=9.0cm,angle=0]{ms2022-0304fig10-0933.pdf}
\includegraphics[width=9.0cm,angle=0]{ms2022-0304fig10-0934.pdf}
\vspace{-0.5cm}

\includegraphics[width=9.0cm,angle=0]{ms2022-0304fig10-0935.pdf}
\includegraphics[width=9.0cm,angle=0]{ms2022-0304fig10-0936.pdf}
\vspace{-0.5cm}

\includegraphics[width=9.0cm,angle=0]{ms2022-0304fig10-0937.pdf}
\includegraphics[width=9.0cm,angle=0]{ms2022-0304fig10-0938.pdf}
\vspace{-0.5cm}

\includegraphics[width=9.0cm,angle=0]{ms2022-0304fig10-0939.pdf}
\includegraphics[width=9.0cm,angle=0]{ms2022-0304fig10-0940.pdf}
\end{figure}
\clearpage

\begin{figure}
\includegraphics[width=9.0cm,angle=0]{ms2022-0304fig10-0941.pdf}
\includegraphics[width=9.0cm,angle=0]{ms2022-0304fig10-0942.pdf}
\vspace{-0.5cm}

\includegraphics[width=9.0cm,angle=0]{ms2022-0304fig10-0943.pdf}
\includegraphics[width=9.0cm,angle=0]{ms2022-0304fig10-0944.pdf}
\vspace{-0.5cm}

\includegraphics[width=9.0cm,angle=0]{ms2022-0304fig10-0945.pdf}
\includegraphics[width=9.0cm,angle=0]{ms2022-0304fig10-0946.pdf}
\vspace{-0.5cm}

\includegraphics[width=9.0cm,angle=0]{ms2022-0304fig10-0947.pdf}
\includegraphics[width=9.0cm,angle=0]{ms2022-0304fig10-0948.pdf}
\vspace{-0.5cm}

\includegraphics[width=9.0cm,angle=0]{ms2022-0304fig10-0949.pdf}
\includegraphics[width=9.0cm,angle=0]{ms2022-0304fig10-0950.pdf}
\end{figure}
\clearpage

\begin{figure}
\includegraphics[width=9.0cm,angle=0]{ms2022-0304fig10-0951.pdf}
\includegraphics[width=9.0cm,angle=0]{ms2022-0304fig10-0952.pdf}
\vspace{-0.5cm}

\includegraphics[width=9.0cm,angle=0]{ms2022-0304fig10-0953.pdf}
\includegraphics[width=9.0cm,angle=0]{ms2022-0304fig10-0954.pdf}
\vspace{-0.5cm}

\includegraphics[width=9.0cm,angle=0]{ms2022-0304fig10-0955.pdf}
\includegraphics[width=9.0cm,angle=0]{ms2022-0304fig10-0956.pdf}
\vspace{-0.5cm}

\includegraphics[width=9.0cm,angle=0]{ms2022-0304fig10-0957.pdf}
\includegraphics[width=9.0cm,angle=0]{ms2022-0304fig10-0958.pdf}
\vspace{-0.5cm}

\includegraphics[width=9.0cm,angle=0]{ms2022-0304fig10-0959.pdf}
\includegraphics[width=9.0cm,angle=0]{ms2022-0304fig10-0960.pdf}
\end{figure}
\clearpage

\begin{figure}
\includegraphics[width=9.0cm,angle=0]{ms2022-0304fig10-0961.pdf}
\includegraphics[width=9.0cm,angle=0]{ms2022-0304fig10-0962.pdf}
\vspace{-0.5cm}

\includegraphics[width=9.0cm,angle=0]{ms2022-0304fig10-0963.pdf}
\includegraphics[width=9.0cm,angle=0]{ms2022-0304fig10-0964.pdf}
\vspace{-0.5cm}

\includegraphics[width=9.0cm,angle=0]{ms2022-0304fig10-0965.pdf}
\includegraphics[width=9.0cm,angle=0]{ms2022-0304fig10-0966.pdf}
\vspace{-0.5cm}

\includegraphics[width=9.0cm,angle=0]{ms2022-0304fig10-0967.pdf}
\includegraphics[width=9.0cm,angle=0]{ms2022-0304fig10-0968.pdf}
\vspace{-0.5cm}

\includegraphics[width=9.0cm,angle=0]{ms2022-0304fig10-0969.pdf}
\includegraphics[width=9.0cm,angle=0]{ms2022-0304fig10-0970.pdf}
\end{figure}
\clearpage

\begin{figure}
\includegraphics[width=9.0cm,angle=0]{ms2022-0304fig10-0971.pdf}
\includegraphics[width=9.0cm,angle=0]{ms2022-0304fig10-0972.pdf}
\vspace{-0.5cm}

\includegraphics[width=9.0cm,angle=0]{ms2022-0304fig10-0973.pdf}
\includegraphics[width=9.0cm,angle=0]{ms2022-0304fig10-0974.pdf}
\vspace{-0.5cm}

\includegraphics[width=9.0cm,angle=0]{ms2022-0304fig10-0975.pdf}
\includegraphics[width=9.0cm,angle=0]{ms2022-0304fig10-0976.pdf}
\vspace{-0.5cm}

\includegraphics[width=9.0cm,angle=0]{ms2022-0304fig10-0977.pdf}
\includegraphics[width=9.0cm,angle=0]{ms2022-0304fig10-0978.pdf}
\vspace{-0.5cm}

\includegraphics[width=9.0cm,angle=0]{ms2022-0304fig10-0979.pdf}
\includegraphics[width=9.0cm,angle=0]{ms2022-0304fig10-0980.pdf}
\end{figure}
\clearpage

\begin{figure}
\includegraphics[width=9.0cm,angle=0]{ms2022-0304fig10-0981.pdf}
\includegraphics[width=9.0cm,angle=0]{ms2022-0304fig10-0982.pdf}
\vspace{-0.5cm}

\includegraphics[width=9.0cm,angle=0]{ms2022-0304fig10-0983.pdf}
\includegraphics[width=9.0cm,angle=0]{ms2022-0304fig10-0984.pdf}
\vspace{-0.5cm}

\includegraphics[width=9.0cm,angle=0]{ms2022-0304fig10-0985.pdf}
\includegraphics[width=9.0cm,angle=0]{ms2022-0304fig10-0986.pdf}
\vspace{-0.5cm}

\includegraphics[width=9.0cm,angle=0]{ms2022-0304fig10-0987.pdf}
\includegraphics[width=9.0cm,angle=0]{ms2022-0304fig10-0988.pdf}
\vspace{-0.5cm}

\includegraphics[width=9.0cm,angle=0]{ms2022-0304fig10-0989.pdf}
\includegraphics[width=9.0cm,angle=0]{ms2022-0304fig10-0990.pdf}
\end{figure}
\clearpage

\begin{figure}
\includegraphics[width=9.0cm,angle=0]{ms2022-0304fig10-0991.pdf}
\includegraphics[width=9.0cm,angle=0]{ms2022-0304fig10-0992.pdf}
\vspace{-0.5cm}

\includegraphics[width=9.0cm,angle=0]{ms2022-0304fig10-0993.pdf}
\includegraphics[width=9.0cm,angle=0]{ms2022-0304fig10-0994.pdf}
\vspace{-0.5cm}

\includegraphics[width=9.0cm,angle=0]{ms2022-0304fig10-0995.pdf}
\includegraphics[width=9.0cm,angle=0]{ms2022-0304fig10-0996.pdf}
\vspace{-0.5cm}

\includegraphics[width=9.0cm,angle=0]{ms2022-0304fig10-0997.pdf}
\includegraphics[width=9.0cm,angle=0]{ms2022-0304fig10-0998.pdf}
\vspace{-0.5cm}

\includegraphics[width=9.0cm,angle=0]{ms2022-0304fig10-0999.pdf}
\includegraphics[width=9.0cm,angle=0]{ms2022-0304fig10-1000.pdf}
\end{figure}
\clearpage

\begin{figure}
\includegraphics[width=9.0cm,angle=0]{ms2022-0304fig10-1001.pdf}
\includegraphics[width=9.0cm,angle=0]{ms2022-0304fig10-1002.pdf}
\vspace{-0.5cm}

\includegraphics[width=9.0cm,angle=0]{ms2022-0304fig10-1003.pdf}
\includegraphics[width=9.0cm,angle=0]{ms2022-0304fig10-1004.pdf}
\vspace{-0.5cm}

\includegraphics[width=9.0cm,angle=0]{ms2022-0304fig10-1005.pdf}
\includegraphics[width=9.0cm,angle=0]{ms2022-0304fig10-1006.pdf}
\vspace{-0.5cm}

\includegraphics[width=9.0cm,angle=0]{ms2022-0304fig10-1007.pdf}
\includegraphics[width=9.0cm,angle=0]{ms2022-0304fig10-1008.pdf}
\vspace{-0.5cm}

\includegraphics[width=9.0cm,angle=0]{ms2022-0304fig10-1009.pdf}
\includegraphics[width=9.0cm,angle=0]{ms2022-0304fig10-1010.pdf}
\end{figure}
\clearpage

\begin{figure}
\includegraphics[width=9.0cm,angle=0]{ms2022-0304fig10-1011.pdf}
\includegraphics[width=9.0cm,angle=0]{ms2022-0304fig10-1012.pdf}
\vspace{-0.5cm}

\includegraphics[width=9.0cm,angle=0]{ms2022-0304fig10-1013.pdf}
\includegraphics[width=9.0cm,angle=0]{ms2022-0304fig10-1014.pdf}
\vspace{-0.5cm}

\includegraphics[width=9.0cm,angle=0]{ms2022-0304fig10-1015.pdf}
\includegraphics[width=9.0cm,angle=0]{ms2022-0304fig10-1016.pdf}
\vspace{-0.5cm}

\includegraphics[width=9.0cm,angle=0]{ms2022-0304fig10-1017.pdf}
\includegraphics[width=9.0cm,angle=0]{ms2022-0304fig10-1018.pdf}
\vspace{-0.5cm}

\includegraphics[width=9.0cm,angle=0]{ms2022-0304fig10-1019.pdf}
\includegraphics[width=9.0cm,angle=0]{ms2022-0304fig10-1020.pdf}
\end{figure}
\clearpage

\begin{figure}
\includegraphics[width=9.0cm,angle=0]{ms2022-0304fig10-1021.pdf}
\includegraphics[width=9.0cm,angle=0]{ms2022-0304fig10-1022.pdf}
\vspace{-0.5cm}

\includegraphics[width=9.0cm,angle=0]{ms2022-0304fig10-1023.pdf}
\includegraphics[width=9.0cm,angle=0]{ms2022-0304fig10-1024.pdf}
\vspace{-0.5cm}

\includegraphics[width=9.0cm,angle=0]{ms2022-0304fig10-1025.pdf}
\includegraphics[width=9.0cm,angle=0]{ms2022-0304fig10-1026.pdf}
\vspace{-0.5cm}

\includegraphics[width=9.0cm,angle=0]{ms2022-0304fig10-1027.pdf}
\includegraphics[width=9.0cm,angle=0]{ms2022-0304fig10-1028.pdf}
\vspace{-0.5cm}

\includegraphics[width=9.0cm,angle=0]{ms2022-0304fig10-1029.pdf}
\includegraphics[width=9.0cm,angle=0]{ms2022-0304fig10-1030.pdf}
\end{figure}
\clearpage

\begin{figure}
\includegraphics[width=9.0cm,angle=0]{ms2022-0304fig10-1031.pdf}
\includegraphics[width=9.0cm,angle=0]{ms2022-0304fig10-1032.pdf}
\vspace{-0.5cm}

\includegraphics[width=9.0cm,angle=0]{ms2022-0304fig10-1033.pdf}
\includegraphics[width=9.0cm,angle=0]{ms2022-0304fig10-1034.pdf}
\vspace{-0.5cm}

\includegraphics[width=9.0cm,angle=0]{ms2022-0304fig10-1035.pdf}
\includegraphics[width=9.0cm,angle=0]{ms2022-0304fig10-1036.pdf}
\vspace{-0.5cm}

\includegraphics[width=9.0cm,angle=0]{ms2022-0304fig10-1037.pdf}
\includegraphics[width=9.0cm,angle=0]{ms2022-0304fig10-1038.pdf}
\vspace{-0.5cm}

\includegraphics[width=9.0cm,angle=0]{ms2022-0304fig10-1039.pdf}
\includegraphics[width=9.0cm,angle=0]{ms2022-0304fig10-1040.pdf}
\end{figure}
\clearpage

\begin{figure}
\includegraphics[width=9.0cm,angle=0]{ms2022-0304fig10-1041.pdf}
\includegraphics[width=9.0cm,angle=0]{ms2022-0304fig10-1042.pdf}
\vspace{-0.5cm}

\includegraphics[width=9.0cm,angle=0]{ms2022-0304fig10-1043.pdf}
\includegraphics[width=9.0cm,angle=0]{ms2022-0304fig10-1044.pdf}
\vspace{-0.5cm}

\includegraphics[width=9.0cm,angle=0]{ms2022-0304fig10-1045.pdf}
\includegraphics[width=9.0cm,angle=0]{ms2022-0304fig10-1046.pdf}
\vspace{-0.5cm}

\includegraphics[width=9.0cm,angle=0]{ms2022-0304fig10-1047.pdf}
\includegraphics[width=9.0cm,angle=0]{ms2022-0304fig10-1048.pdf}
\vspace{-0.5cm}

\includegraphics[width=9.0cm,angle=0]{ms2022-0304fig10-1049.pdf}
\includegraphics[width=9.0cm,angle=0]{ms2022-0304fig10-1050.pdf}
\end{figure}
\clearpage

\begin{figure}
\includegraphics[width=9.0cm,angle=0]{ms2022-0304fig10-1051.pdf}
\includegraphics[width=9.0cm,angle=0]{ms2022-0304fig10-1052.pdf}
\vspace{-0.5cm}

\includegraphics[width=9.0cm,angle=0]{ms2022-0304fig10-1053.pdf}
\includegraphics[width=9.0cm,angle=0]{ms2022-0304fig10-1054.pdf}
\vspace{-0.5cm}

\includegraphics[width=9.0cm,angle=0]{ms2022-0304fig10-1055.pdf}
\includegraphics[width=9.0cm,angle=0]{ms2022-0304fig10-1056.pdf}
\vspace{-0.5cm}

\includegraphics[width=9.0cm,angle=0]{ms2022-0304fig10-1057.pdf}
\includegraphics[width=9.0cm,angle=0]{ms2022-0304fig10-1058.pdf}
\vspace{-0.5cm}

\includegraphics[width=9.0cm,angle=0]{ms2022-0304fig10-1059.pdf}
\includegraphics[width=9.0cm,angle=0]{ms2022-0304fig10-1060.pdf}
\end{figure}
\clearpage

\begin{figure}
\includegraphics[width=9.0cm,angle=0]{ms2022-0304fig10-1061.pdf}
\includegraphics[width=9.0cm,angle=0]{ms2022-0304fig10-1062.pdf}
\vspace{-0.5cm}

\includegraphics[width=9.0cm,angle=0]{ms2022-0304fig10-1063.pdf}
\includegraphics[width=9.0cm,angle=0]{ms2022-0304fig10-1064.pdf}
\vspace{-0.5cm}

\includegraphics[width=9.0cm,angle=0]{ms2022-0304fig10-1065.pdf}
\includegraphics[width=9.0cm,angle=0]{ms2022-0304fig10-1066.pdf}
\vspace{-0.5cm}

\includegraphics[width=9.0cm,angle=0]{ms2022-0304fig10-1067.pdf}
\includegraphics[width=9.0cm,angle=0]{ms2022-0304fig10-1068.pdf}
\vspace{-0.5cm}

\includegraphics[width=9.0cm,angle=0]{ms2022-0304fig10-1069.pdf}
\includegraphics[width=9.0cm,angle=0]{ms2022-0304fig10-1070.pdf}
\end{figure}
\clearpage

\begin{figure}
\includegraphics[width=9.0cm,angle=0]{ms2022-0304fig10-1071.pdf}
\includegraphics[width=9.0cm,angle=0]{ms2022-0304fig10-1072.pdf}
\vspace{-0.5cm}

\includegraphics[width=9.0cm,angle=0]{ms2022-0304fig10-1073.pdf}
\includegraphics[width=9.0cm,angle=0]{ms2022-0304fig10-1074.pdf}
\vspace{-0.5cm}

\includegraphics[width=9.0cm,angle=0]{ms2022-0304fig10-1075.pdf}
\includegraphics[width=9.0cm,angle=0]{ms2022-0304fig10-1076.pdf}
\vspace{-0.5cm}

\includegraphics[width=9.0cm,angle=0]{ms2022-0304fig10-1077.pdf}
\includegraphics[width=9.0cm,angle=0]{ms2022-0304fig10-1078.pdf}
\vspace{-0.5cm}

\includegraphics[width=9.0cm,angle=0]{ms2022-0304fig10-1079.pdf}
\includegraphics[width=9.0cm,angle=0]{ms2022-0304fig10-1080.pdf}
\end{figure}
\clearpage

\begin{figure}
\includegraphics[width=9.0cm,angle=0]{ms2022-0304fig10-1081.pdf}
\includegraphics[width=9.0cm,angle=0]{ms2022-0304fig10-1082.pdf}
\vspace{-0.5cm}

\includegraphics[width=9.0cm,angle=0]{ms2022-0304fig10-1083.pdf}
\includegraphics[width=9.0cm,angle=0]{ms2022-0304fig10-1084.pdf}
\vspace{-0.5cm}

\includegraphics[width=9.0cm,angle=0]{ms2022-0304fig10-1085.pdf}
\includegraphics[width=9.0cm,angle=0]{ms2022-0304fig10-1086.pdf}
\vspace{-0.5cm}

\includegraphics[width=9.0cm,angle=0]{ms2022-0304fig10-1087.pdf}
\includegraphics[width=9.0cm,angle=0]{ms2022-0304fig10-1088.pdf}
\vspace{-0.5cm}

\includegraphics[width=9.0cm,angle=0]{ms2022-0304fig10-1089.pdf}
\includegraphics[width=9.0cm,angle=0]{ms2022-0304fig10-1090.pdf}
\end{figure}
\clearpage

\begin{figure}
\includegraphics[width=9.0cm,angle=0]{ms2022-0304fig10-1091.pdf}
\includegraphics[width=9.0cm,angle=0]{ms2022-0304fig10-1092.pdf}
\vspace{-0.5cm}

\includegraphics[width=9.0cm,angle=0]{ms2022-0304fig10-1093.pdf}
\includegraphics[width=9.0cm,angle=0]{ms2022-0304fig10-1094.pdf}
\vspace{-0.5cm}

\includegraphics[width=9.0cm,angle=0]{ms2022-0304fig10-1095.pdf}
\includegraphics[width=9.0cm,angle=0]{ms2022-0304fig10-1096.pdf}
\vspace{-0.5cm}

\includegraphics[width=9.0cm,angle=0]{ms2022-0304fig10-1097.pdf}
\includegraphics[width=9.0cm,angle=0]{ms2022-0304fig10-1098.pdf}
\vspace{-0.5cm}

\includegraphics[width=9.0cm,angle=0]{ms2022-0304fig10-1099.pdf}
\includegraphics[width=9.0cm,angle=0]{ms2022-0304fig10-1100.pdf}
\end{figure}
\clearpage

\begin{figure}
\includegraphics[width=9.0cm,angle=0]{ms2022-0304fig10-1101.pdf}
\includegraphics[width=9.0cm,angle=0]{ms2022-0304fig10-1102.pdf}
\vspace{-0.5cm}

\includegraphics[width=9.0cm,angle=0]{ms2022-0304fig10-1103.pdf}
\includegraphics[width=9.0cm,angle=0]{ms2022-0304fig10-1104.pdf}
\vspace{-0.5cm}

\includegraphics[width=9.0cm,angle=0]{ms2022-0304fig10-1105.pdf}
\includegraphics[width=9.0cm,angle=0]{ms2022-0304fig10-1106.pdf}
\vspace{-0.5cm}

\includegraphics[width=9.0cm,angle=0]{ms2022-0304fig10-1107.pdf}
\includegraphics[width=9.0cm,angle=0]{ms2022-0304fig10-1108.pdf}
\vspace{-0.5cm}

\includegraphics[width=9.0cm,angle=0]{ms2022-0304fig10-1109.pdf}
\includegraphics[width=9.0cm,angle=0]{ms2022-0304fig10-1110.pdf}
\end{figure}
\clearpage

\begin{figure}
\includegraphics[width=9.0cm,angle=0]{ms2022-0304fig10-1111.pdf}
\includegraphics[width=9.0cm,angle=0]{ms2022-0304fig10-1112.pdf}
\vspace{-0.5cm}

\includegraphics[width=9.0cm,angle=0]{ms2022-0304fig10-1113.pdf}
\includegraphics[width=9.0cm,angle=0]{ms2022-0304fig10-1114.pdf}
\vspace{-0.5cm}

\includegraphics[width=9.0cm,angle=0]{ms2022-0304fig10-1115.pdf}
\includegraphics[width=9.0cm,angle=0]{ms2022-0304fig10-1116.pdf}
\vspace{-0.5cm}

\includegraphics[width=9.0cm,angle=0]{ms2022-0304fig10-1117.pdf}
\includegraphics[width=9.0cm,angle=0]{ms2022-0304fig10-1118.pdf}
\vspace{-0.5cm}

\includegraphics[width=9.0cm,angle=0]{ms2022-0304fig10-1119.pdf}
\includegraphics[width=9.0cm,angle=0]{ms2022-0304fig10-1120.pdf}
\end{figure}
\clearpage

\begin{figure}
\includegraphics[width=9.0cm,angle=0]{ms2022-0304fig10-1121.pdf}
\includegraphics[width=9.0cm,angle=0]{ms2022-0304fig10-1122.pdf}
\vspace{-0.5cm}

\includegraphics[width=9.0cm,angle=0]{ms2022-0304fig10-1123.pdf}
\includegraphics[width=9.0cm,angle=0]{ms2022-0304fig10-1124.pdf}
\vspace{-0.5cm}

\includegraphics[width=9.0cm,angle=0]{ms2022-0304fig10-1125.pdf}
\includegraphics[width=9.0cm,angle=0]{ms2022-0304fig10-1126.pdf}
\vspace{-0.5cm}

\includegraphics[width=9.0cm,angle=0]{ms2022-0304fig10-1127.pdf}
\includegraphics[width=9.0cm,angle=0]{ms2022-0304fig10-1128.pdf}
\vspace{-0.5cm}

\includegraphics[width=9.0cm,angle=0]{ms2022-0304fig10-1129.pdf}
\includegraphics[width=9.0cm,angle=0]{ms2022-0304fig10-1130.pdf}
\end{figure}
\clearpage

\begin{figure}
\includegraphics[width=9.0cm,angle=0]{ms2022-0304fig10-1131.pdf}
\includegraphics[width=9.0cm,angle=0]{ms2022-0304fig10-1132.pdf}
\vspace{-0.5cm}

\includegraphics[width=9.0cm,angle=0]{ms2022-0304fig10-1133.pdf}
\includegraphics[width=9.0cm,angle=0]{ms2022-0304fig10-1134.pdf}
\vspace{-0.5cm}

\includegraphics[width=9.0cm,angle=0]{ms2022-0304fig10-1135.pdf}
\includegraphics[width=9.0cm,angle=0]{ms2022-0304fig10-1136.pdf}
\vspace{-0.5cm}

\includegraphics[width=9.0cm,angle=0]{ms2022-0304fig10-1137.pdf}
\includegraphics[width=9.0cm,angle=0]{ms2022-0304fig10-1138.pdf}
\vspace{-0.5cm}

\includegraphics[width=9.0cm,angle=0]{ms2022-0304fig10-1139.pdf}
\includegraphics[width=9.0cm,angle=0]{ms2022-0304fig10-1140.pdf}
\end{figure}
\clearpage

\begin{figure}
\includegraphics[width=9.0cm,angle=0]{ms2022-0304fig10-1141.pdf}
\includegraphics[width=9.0cm,angle=0]{ms2022-0304fig10-1142.pdf}
\vspace{-0.5cm}

\includegraphics[width=9.0cm,angle=0]{ms2022-0304fig10-1143.pdf}
\includegraphics[width=9.0cm,angle=0]{ms2022-0304fig10-1144.pdf}
\vspace{-0.5cm}

\includegraphics[width=9.0cm,angle=0]{ms2022-0304fig10-1145.pdf}
\includegraphics[width=9.0cm,angle=0]{ms2022-0304fig10-1146.pdf}
\vspace{-0.5cm}

\includegraphics[width=9.0cm,angle=0]{ms2022-0304fig10-1147.pdf}
\includegraphics[width=9.0cm,angle=0]{ms2022-0304fig10-1148.pdf}
\vspace{-0.5cm}

\includegraphics[width=9.0cm,angle=0]{ms2022-0304fig10-1149.pdf}
\includegraphics[width=9.0cm,angle=0]{ms2022-0304fig10-1150.pdf}
\end{figure}
\clearpage

\begin{figure}
\includegraphics[width=9.0cm,angle=0]{ms2022-0304fig10-1151.pdf}
\includegraphics[width=9.0cm,angle=0]{ms2022-0304fig10-1152.pdf}
\vspace{-0.5cm}

\includegraphics[width=9.0cm,angle=0]{ms2022-0304fig10-1153.pdf}
\includegraphics[width=9.0cm,angle=0]{ms2022-0304fig10-1154.pdf}
\vspace{-0.5cm}

\includegraphics[width=9.0cm,angle=0]{ms2022-0304fig10-1155.pdf}
\includegraphics[width=9.0cm,angle=0]{ms2022-0304fig10-1156.pdf}
\vspace{-0.5cm}

\includegraphics[width=9.0cm,angle=0]{ms2022-0304fig10-1157.pdf}
\includegraphics[width=9.0cm,angle=0]{ms2022-0304fig10-1158.pdf}
\vspace{-0.5cm}

\includegraphics[width=9.0cm,angle=0]{ms2022-0304fig10-1159.pdf}
\includegraphics[width=9.0cm,angle=0]{ms2022-0304fig10-1160.pdf}
\end{figure}
\clearpage

\begin{figure}
\includegraphics[width=9.0cm,angle=0]{ms2022-0304fig10-1161.pdf}
\includegraphics[width=9.0cm,angle=0]{ms2022-0304fig10-1162.pdf}
\vspace{-0.5cm}

\includegraphics[width=9.0cm,angle=0]{ms2022-0304fig10-1163.pdf}
\includegraphics[width=9.0cm,angle=0]{ms2022-0304fig10-1164.pdf}
\vspace{-0.5cm}

\includegraphics[width=9.0cm,angle=0]{ms2022-0304fig10-1165.pdf}
\includegraphics[width=9.0cm,angle=0]{ms2022-0304fig10-1166.pdf}
\vspace{-0.5cm}

\includegraphics[width=9.0cm,angle=0]{ms2022-0304fig10-1167.pdf}
\includegraphics[width=9.0cm,angle=0]{ms2022-0304fig10-1168.pdf}
\vspace{-0.5cm}

\includegraphics[width=9.0cm,angle=0]{ms2022-0304fig10-1169.pdf}
\includegraphics[width=9.0cm,angle=0]{ms2022-0304fig10-1170.pdf}
\end{figure}
\clearpage

\begin{figure}
\includegraphics[width=9.0cm,angle=0]{ms2022-0304fig10-1171.pdf}
\includegraphics[width=9.0cm,angle=0]{ms2022-0304fig10-1172.pdf}
\vspace{-0.5cm}

\includegraphics[width=9.0cm,angle=0]{ms2022-0304fig10-1173.pdf}
\includegraphics[width=9.0cm,angle=0]{ms2022-0304fig10-1174.pdf}
\vspace{-0.5cm}

\includegraphics[width=9.0cm,angle=0]{ms2022-0304fig10-1175.pdf}
\includegraphics[width=9.0cm,angle=0]{ms2022-0304fig10-1176.pdf}
\vspace{-0.5cm}

\includegraphics[width=9.0cm,angle=0]{ms2022-0304fig10-1177.pdf}
\includegraphics[width=9.0cm,angle=0]{ms2022-0304fig10-1178.pdf}
\vspace{-0.5cm}

\includegraphics[width=9.0cm,angle=0]{ms2022-0304fig10-1179.pdf}
\includegraphics[width=9.0cm,angle=0]{ms2022-0304fig10-1180.pdf}
\end{figure}
\clearpage

\begin{figure}
\includegraphics[width=9.0cm,angle=0]{ms2022-0304fig10-1181.pdf}
\includegraphics[width=9.0cm,angle=0]{ms2022-0304fig10-1182.pdf}
\vspace{-0.5cm}

\includegraphics[width=9.0cm,angle=0]{ms2022-0304fig10-1183.pdf}
\includegraphics[width=9.0cm,angle=0]{ms2022-0304fig10-1184.pdf}
\vspace{-0.5cm}

\includegraphics[width=9.0cm,angle=0]{ms2022-0304fig10-1185.pdf}
\includegraphics[width=9.0cm,angle=0]{ms2022-0304fig10-1186.pdf}
\vspace{-0.5cm}

\includegraphics[width=9.0cm,angle=0]{ms2022-0304fig10-1187.pdf}
\includegraphics[width=9.0cm,angle=0]{ms2022-0304fig10-1188.pdf}
\vspace{-0.5cm}

\includegraphics[width=9.0cm,angle=0]{ms2022-0304fig10-1189.pdf}
\includegraphics[width=9.0cm,angle=0]{ms2022-0304fig10-1190.pdf}
\end{figure}
\clearpage

\begin{figure}
\includegraphics[width=9.0cm,angle=0]{ms2022-0304fig10-1191.pdf}
\includegraphics[width=9.0cm,angle=0]{ms2022-0304fig10-1192.pdf}
\vspace{-0.5cm}

\includegraphics[width=9.0cm,angle=0]{ms2022-0304fig10-1193.pdf}
\includegraphics[width=9.0cm,angle=0]{ms2022-0304fig10-1194.pdf}
\vspace{-0.5cm}

\includegraphics[width=9.0cm,angle=0]{ms2022-0304fig10-1195.pdf}
\includegraphics[width=9.0cm,angle=0]{ms2022-0304fig10-1196.pdf}
\vspace{-0.5cm}

\includegraphics[width=9.0cm,angle=0]{ms2022-0304fig10-1197.pdf}
\includegraphics[width=9.0cm,angle=0]{ms2022-0304fig10-1198.pdf}
\vspace{-0.5cm}

\includegraphics[width=9.0cm,angle=0]{ms2022-0304fig10-1199.pdf}
\includegraphics[width=9.0cm,angle=0]{ms2022-0304fig10-1200.pdf}
\end{figure}
\clearpage

\begin{figure}
\includegraphics[width=9.0cm,angle=0]{ms2022-0304fig10-1201.pdf}
\includegraphics[width=9.0cm,angle=0]{ms2022-0304fig10-1202.pdf}
\vspace{-0.5cm}

\includegraphics[width=9.0cm,angle=0]{ms2022-0304fig10-1203.pdf}
\includegraphics[width=9.0cm,angle=0]{ms2022-0304fig10-1204.pdf}
\vspace{-0.5cm}

\includegraphics[width=9.0cm,angle=0]{ms2022-0304fig10-1205.pdf}
\includegraphics[width=9.0cm,angle=0]{ms2022-0304fig10-1206.pdf}
\vspace{-0.5cm}

\includegraphics[width=9.0cm,angle=0]{ms2022-0304fig10-1207.pdf}
\includegraphics[width=9.0cm,angle=0]{ms2022-0304fig10-1208.pdf}
\vspace{-0.5cm}

\includegraphics[width=9.0cm,angle=0]{ms2022-0304fig10-1209.pdf}
\includegraphics[width=9.0cm,angle=0]{ms2022-0304fig10-1210.pdf}
\end{figure}
\clearpage

\begin{figure}
\includegraphics[width=9.0cm,angle=0]{ms2022-0304fig10-1211.pdf}
\includegraphics[width=9.0cm,angle=0]{ms2022-0304fig10-1212.pdf}
\vspace{-0.5cm}

\includegraphics[width=9.0cm,angle=0]{ms2022-0304fig10-1213.pdf}
\includegraphics[width=9.0cm,angle=0]{ms2022-0304fig10-1214.pdf}
\vspace{-0.5cm}

\includegraphics[width=9.0cm,angle=0]{ms2022-0304fig10-1215.pdf}
\includegraphics[width=9.0cm,angle=0]{ms2022-0304fig10-1216.pdf}
\vspace{-0.5cm}

\includegraphics[width=9.0cm,angle=0]{ms2022-0304fig10-1217.pdf}
\includegraphics[width=9.0cm,angle=0]{ms2022-0304fig10-1218.pdf}
\vspace{-0.5cm}

\includegraphics[width=9.0cm,angle=0]{ms2022-0304fig10-1219.pdf}
\includegraphics[width=9.0cm,angle=0]{ms2022-0304fig10-1220.pdf}
\end{figure}
\clearpage

\begin{figure}
\includegraphics[width=9.0cm,angle=0]{ms2022-0304fig10-1221.pdf}
\includegraphics[width=9.0cm,angle=0]{ms2022-0304fig10-1222.pdf}
\vspace{-0.5cm}

\includegraphics[width=9.0cm,angle=0]{ms2022-0304fig10-1223.pdf}
\includegraphics[width=9.0cm,angle=0]{ms2022-0304fig10-1224.pdf}
\vspace{-0.5cm}

\includegraphics[width=9.0cm,angle=0]{ms2022-0304fig10-1225.pdf}
\includegraphics[width=9.0cm,angle=0]{ms2022-0304fig10-1226.pdf}
\vspace{-0.5cm}

\includegraphics[width=9.0cm,angle=0]{ms2022-0304fig10-1227.pdf}
\includegraphics[width=9.0cm,angle=0]{ms2022-0304fig10-1228.pdf}
\vspace{-0.5cm}

\includegraphics[width=9.0cm,angle=0]{ms2022-0304fig10-1229.pdf}
\includegraphics[width=9.0cm,angle=0]{ms2022-0304fig10-1230.pdf}
\end{figure}
\clearpage

\begin{figure}
\includegraphics[width=9.0cm,angle=0]{ms2022-0304fig10-1231.pdf}
\includegraphics[width=9.0cm,angle=0]{ms2022-0304fig10-1232.pdf}
\vspace{-0.5cm}

\includegraphics[width=9.0cm,angle=0]{ms2022-0304fig10-1233.pdf}
\includegraphics[width=9.0cm,angle=0]{ms2022-0304fig10-1234.pdf}
\vspace{-0.5cm}

\includegraphics[width=9.0cm,angle=0]{ms2022-0304fig10-1235.pdf}
\includegraphics[width=9.0cm,angle=0]{ms2022-0304fig10-1236.pdf}
\vspace{-0.5cm}

\includegraphics[width=9.0cm,angle=0]{ms2022-0304fig10-1237.pdf}
\includegraphics[width=9.0cm,angle=0]{ms2022-0304fig10-1238.pdf}
\vspace{-0.5cm}

\includegraphics[width=9.0cm,angle=0]{ms2022-0304fig10-1239.pdf}
\includegraphics[width=9.0cm,angle=0]{ms2022-0304fig10-1240.pdf}
\end{figure}
\clearpage

\begin{figure}
\includegraphics[width=9.0cm,angle=0]{ms2022-0304fig10-1241.pdf}
\includegraphics[width=9.0cm,angle=0]{ms2022-0304fig10-1242.pdf}
\vspace{-0.5cm}

\includegraphics[width=9.0cm,angle=0]{ms2022-0304fig10-1243.pdf}
\includegraphics[width=9.0cm,angle=0]{ms2022-0304fig10-1244.pdf}
\vspace{-0.5cm}

\includegraphics[width=9.0cm,angle=0]{ms2022-0304fig10-1245.pdf}
\includegraphics[width=9.0cm,angle=0]{ms2022-0304fig10-1246.pdf}
\vspace{-0.5cm}

\includegraphics[width=9.0cm,angle=0]{ms2022-0304fig10-1247.pdf}
\includegraphics[width=9.0cm,angle=0]{ms2022-0304fig10-1248.pdf}
\vspace{-0.5cm}

\includegraphics[width=9.0cm,angle=0]{ms2022-0304fig10-1249.pdf}
\includegraphics[width=9.0cm,angle=0]{ms2022-0304fig10-1250.pdf}
\end{figure}
\clearpage

\begin{figure}
\includegraphics[width=9.0cm,angle=0]{ms2022-0304fig10-1251.pdf}
\includegraphics[width=9.0cm,angle=0]{ms2022-0304fig10-1252.pdf}
\vspace{-0.5cm}

\includegraphics[width=9.0cm,angle=0]{ms2022-0304fig10-1253.pdf}
\includegraphics[width=9.0cm,angle=0]{ms2022-0304fig10-1254.pdf}
\vspace{-0.5cm}

\includegraphics[width=9.0cm,angle=0]{ms2022-0304fig10-1255.pdf}
\includegraphics[width=9.0cm,angle=0]{ms2022-0304fig10-1256.pdf}
\vspace{-0.5cm}

\includegraphics[width=9.0cm,angle=0]{ms2022-0304fig10-1257.pdf}
\includegraphics[width=9.0cm,angle=0]{ms2022-0304fig10-1258.pdf}
\vspace{-0.5cm}

\includegraphics[width=9.0cm,angle=0]{ms2022-0304fig10-1259.pdf}
\includegraphics[width=9.0cm,angle=0]{ms2022-0304fig10-1260.pdf}
\end{figure}
\clearpage

\begin{figure}
\includegraphics[width=9.0cm,angle=0]{ms2022-0304fig10-1261.pdf}
\includegraphics[width=9.0cm,angle=0]{ms2022-0304fig10-1262.pdf}
\vspace{-0.5cm}

\includegraphics[width=9.0cm,angle=0]{ms2022-0304fig10-1263.pdf}
\includegraphics[width=9.0cm,angle=0]{ms2022-0304fig10-1264.pdf}
\vspace{-0.5cm}

\includegraphics[width=9.0cm,angle=0]{ms2022-0304fig10-1265.pdf}
\includegraphics[width=9.0cm,angle=0]{ms2022-0304fig10-1266.pdf}
\vspace{-0.5cm}

\includegraphics[width=9.0cm,angle=0]{ms2022-0304fig10-1267.pdf}
\includegraphics[width=9.0cm,angle=0]{ms2022-0304fig10-1268.pdf}
\vspace{-0.5cm}

\includegraphics[width=9.0cm,angle=0]{ms2022-0304fig10-1269.pdf}
\includegraphics[width=9.0cm,angle=0]{ms2022-0304fig10-1270.pdf}
\end{figure}
\clearpage

\begin{figure}
\includegraphics[width=9.0cm,angle=0]{ms2022-0304fig10-1271.pdf}
\includegraphics[width=9.0cm,angle=0]{ms2022-0304fig10-1272.pdf}
\vspace{-0.5cm}

\includegraphics[width=9.0cm,angle=0]{ms2022-0304fig10-1273.pdf}
\includegraphics[width=9.0cm,angle=0]{ms2022-0304fig10-1274.pdf}
\vspace{-0.5cm}

\includegraphics[width=9.0cm,angle=0]{ms2022-0304fig10-1275.pdf}
\includegraphics[width=9.0cm,angle=0]{ms2022-0304fig10-1276.pdf}
\vspace{-0.5cm}

\includegraphics[width=9.0cm,angle=0]{ms2022-0304fig10-1277.pdf}
\includegraphics[width=9.0cm,angle=0]{ms2022-0304fig10-1278.pdf}
\vspace{-0.5cm}

\includegraphics[width=9.0cm,angle=0]{ms2022-0304fig10-1279.pdf}
\includegraphics[width=9.0cm,angle=0]{ms2022-0304fig10-1280.pdf}
\end{figure}
\clearpage

\begin{figure}
\includegraphics[width=9.0cm,angle=0]{ms2022-0304fig10-1281.pdf}
\includegraphics[width=9.0cm,angle=0]{ms2022-0304fig10-1282.pdf}
\vspace{-0.5cm}

\includegraphics[width=9.0cm,angle=0]{ms2022-0304fig10-1283.pdf}
\includegraphics[width=9.0cm,angle=0]{ms2022-0304fig10-1284.pdf}
\vspace{-0.5cm}

\includegraphics[width=9.0cm,angle=0]{ms2022-0304fig10-1285.pdf}
\includegraphics[width=9.0cm,angle=0]{ms2022-0304fig10-1286.pdf}
\vspace{-0.5cm}

\includegraphics[width=9.0cm,angle=0]{ms2022-0304fig10-1287.pdf}
\includegraphics[width=9.0cm,angle=0]{ms2022-0304fig10-1288.pdf}
\vspace{-0.5cm}

\includegraphics[width=9.0cm,angle=0]{ms2022-0304fig10-1289.pdf}
\includegraphics[width=9.0cm,angle=0]{ms2022-0304fig10-1290.pdf}
\end{figure}
\clearpage

\begin{figure}
\includegraphics[width=9.0cm,angle=0]{ms2022-0304fig10-1291.pdf}
\includegraphics[width=9.0cm,angle=0]{ms2022-0304fig10-1292.pdf}
\vspace{-0.5cm}

\includegraphics[width=9.0cm,angle=0]{ms2022-0304fig10-1293.pdf}
\includegraphics[width=9.0cm,angle=0]{ms2022-0304fig10-1294.pdf}
\vspace{-0.5cm}

\includegraphics[width=9.0cm,angle=0]{ms2022-0304fig10-1295.pdf}
\includegraphics[width=9.0cm,angle=0]{ms2022-0304fig10-1296.pdf}
\vspace{-0.5cm}

\includegraphics[width=9.0cm,angle=0]{ms2022-0304fig10-1297.pdf}
\includegraphics[width=9.0cm,angle=0]{ms2022-0304fig10-1298.pdf}
\vspace{-0.5cm}

\includegraphics[width=9.0cm,angle=0]{ms2022-0304fig10-1299.pdf}
\includegraphics[width=9.0cm,angle=0]{ms2022-0304fig10-1300.pdf}
\end{figure}
\clearpage

\begin{figure}
\includegraphics[width=9.0cm,angle=0]{ms2022-0304fig10-1301.pdf}
\includegraphics[width=9.0cm,angle=0]{ms2022-0304fig10-1302.pdf}
\vspace{-0.5cm}

\includegraphics[width=9.0cm,angle=0]{ms2022-0304fig10-1303.pdf}
\includegraphics[width=9.0cm,angle=0]{ms2022-0304fig10-1304.pdf}
\vspace{-0.5cm}

\includegraphics[width=9.0cm,angle=0]{ms2022-0304fig10-1305.pdf}
\includegraphics[width=9.0cm,angle=0]{ms2022-0304fig10-1306.pdf}
\vspace{-0.5cm}

\includegraphics[width=9.0cm,angle=0]{ms2022-0304fig10-1307.pdf}
\includegraphics[width=9.0cm,angle=0]{ms2022-0304fig10-1308.pdf}
\vspace{-0.5cm}

\includegraphics[width=9.0cm,angle=0]{ms2022-0304fig10-1309.pdf}
\includegraphics[width=9.0cm,angle=0]{ms2022-0304fig10-1310.pdf}
\end{figure}
\clearpage

\begin{figure}
\includegraphics[width=9.0cm,angle=0]{ms2022-0304fig10-1311.pdf}
\includegraphics[width=9.0cm,angle=0]{ms2022-0304fig10-1312.pdf}
\vspace{-0.5cm}

\includegraphics[width=9.0cm,angle=0]{ms2022-0304fig10-1313.pdf}
\includegraphics[width=9.0cm,angle=0]{ms2022-0304fig10-1314.pdf}
\vspace{-0.5cm}

\includegraphics[width=9.0cm,angle=0]{ms2022-0304fig10-1315.pdf}
\includegraphics[width=9.0cm,angle=0]{ms2022-0304fig10-1316.pdf}
\vspace{-0.5cm}

\includegraphics[width=9.0cm,angle=0]{ms2022-0304fig10-1317.pdf}
\includegraphics[width=9.0cm,angle=0]{ms2022-0304fig10-1318.pdf}
\vspace{-0.5cm}

\includegraphics[width=9.0cm,angle=0]{ms2022-0304fig10-1319.pdf}
\includegraphics[width=9.0cm,angle=0]{ms2022-0304fig10-1320.pdf}
\end{figure}
\clearpage

\begin{figure}
\includegraphics[width=9.0cm,angle=0]{ms2022-0304fig10-1321.pdf}
\includegraphics[width=9.0cm,angle=0]{ms2022-0304fig10-1322.pdf}
\vspace{-0.5cm}

\includegraphics[width=9.0cm,angle=0]{ms2022-0304fig10-1323.pdf}
\includegraphics[width=9.0cm,angle=0]{ms2022-0304fig10-1324.pdf}
\vspace{-0.5cm}

\includegraphics[width=9.0cm,angle=0]{ms2022-0304fig10-1325.pdf}
\includegraphics[width=9.0cm,angle=0]{ms2022-0304fig10-1326.pdf}
\vspace{-0.5cm}

\includegraphics[width=9.0cm,angle=0]{ms2022-0304fig10-1327.pdf}
\includegraphics[width=9.0cm,angle=0]{ms2022-0304fig10-1328.pdf}
\vspace{-0.5cm}

\includegraphics[width=9.0cm,angle=0]{ms2022-0304fig10-1329.pdf}
\includegraphics[width=9.0cm,angle=0]{ms2022-0304fig10-1330.pdf}
\end{figure}
\clearpage

\begin{figure}
\includegraphics[width=9.0cm,angle=0]{ms2022-0304fig10-1331.pdf}
\includegraphics[width=9.0cm,angle=0]{ms2022-0304fig10-1332.pdf}
\vspace{-0.5cm}

\includegraphics[width=9.0cm,angle=0]{ms2022-0304fig10-1333.pdf}
\includegraphics[width=9.0cm,angle=0]{ms2022-0304fig10-1334.pdf}
\vspace{-0.5cm}

\includegraphics[width=9.0cm,angle=0]{ms2022-0304fig10-1335.pdf}
\includegraphics[width=9.0cm,angle=0]{ms2022-0304fig10-1336.pdf}
\vspace{-0.5cm}

\includegraphics[width=9.0cm,angle=0]{ms2022-0304fig10-1337.pdf}
\includegraphics[width=9.0cm,angle=0]{ms2022-0304fig10-1338.pdf}
\vspace{-0.5cm}

\includegraphics[width=9.0cm,angle=0]{ms2022-0304fig10-1339.pdf}
\includegraphics[width=9.0cm,angle=0]{ms2022-0304fig10-1340.pdf}
\end{figure}
\clearpage

\begin{figure}
\includegraphics[width=9.0cm,angle=0]{ms2022-0304fig10-1341.pdf}
\includegraphics[width=9.0cm,angle=0]{ms2022-0304fig10-1342.pdf}
\vspace{-0.5cm}

\includegraphics[width=9.0cm,angle=0]{ms2022-0304fig10-1343.pdf}
\includegraphics[width=9.0cm,angle=0]{ms2022-0304fig10-1344.pdf}
\vspace{-0.5cm}

\includegraphics[width=9.0cm,angle=0]{ms2022-0304fig10-1345.pdf}
\includegraphics[width=9.0cm,angle=0]{ms2022-0304fig10-1346.pdf}
\vspace{-0.5cm}

\includegraphics[width=9.0cm,angle=0]{ms2022-0304fig10-1347.pdf}
\includegraphics[width=9.0cm,angle=0]{ms2022-0304fig10-1348.pdf}
\vspace{-0.5cm}

\includegraphics[width=9.0cm,angle=0]{ms2022-0304fig10-1349.pdf}
\includegraphics[width=9.0cm,angle=0]{ms2022-0304fig10-1350.pdf}
\end{figure}
\clearpage

\begin{figure}
\includegraphics[width=9.0cm,angle=0]{ms2022-0304fig10-1351.pdf}
\includegraphics[width=9.0cm,angle=0]{ms2022-0304fig10-1352.pdf}
\vspace{-0.5cm}

\includegraphics[width=9.0cm,angle=0]{ms2022-0304fig10-1353.pdf}
\includegraphics[width=9.0cm,angle=0]{ms2022-0304fig10-1354.pdf}
\vspace{-0.5cm}

\includegraphics[width=9.0cm,angle=0]{ms2022-0304fig10-1355.pdf}
\includegraphics[width=9.0cm,angle=0]{ms2022-0304fig10-1356.pdf}
\vspace{-0.5cm}

\includegraphics[width=9.0cm,angle=0]{ms2022-0304fig10-1357.pdf}
\includegraphics[width=9.0cm,angle=0]{ms2022-0304fig10-1358.pdf}
\vspace{-0.5cm}

\includegraphics[width=9.0cm,angle=0]{ms2022-0304fig10-1359.pdf}
\includegraphics[width=9.0cm,angle=0]{ms2022-0304fig10-1360.pdf}
\end{figure}
\clearpage

\begin{figure}
\includegraphics[width=9.0cm,angle=0]{ms2022-0304fig10-1361.pdf}
\includegraphics[width=9.0cm,angle=0]{ms2022-0304fig10-1362.pdf}
\vspace{-0.5cm}

\includegraphics[width=9.0cm,angle=0]{ms2022-0304fig10-1363.pdf}
\includegraphics[width=9.0cm,angle=0]{ms2022-0304fig10-1364.pdf}
\vspace{-0.5cm}

\includegraphics[width=9.0cm,angle=0]{ms2022-0304fig10-1365.pdf}
\includegraphics[width=9.0cm,angle=0]{ms2022-0304fig10-1366.pdf}
\vspace{-0.5cm}

\includegraphics[width=9.0cm,angle=0]{ms2022-0304fig10-1367.pdf}
\includegraphics[width=9.0cm,angle=0]{ms2022-0304fig10-1368.pdf}
\vspace{-0.5cm}

\includegraphics[width=9.0cm,angle=0]{ms2022-0304fig10-1369.pdf}
\includegraphics[width=9.0cm,angle=0]{ms2022-0304fig10-1370.pdf}
\end{figure}
\clearpage

\begin{figure}
\includegraphics[width=9.0cm,angle=0]{ms2022-0304fig10-1371.pdf}
\includegraphics[width=9.0cm,angle=0]{ms2022-0304fig10-1372.pdf}
\vspace{-0.5cm}

\includegraphics[width=9.0cm,angle=0]{ms2022-0304fig10-1373.pdf}
\includegraphics[width=9.0cm,angle=0]{ms2022-0304fig10-1374.pdf}
\vspace{-0.5cm}

\includegraphics[width=9.0cm,angle=0]{ms2022-0304fig10-1375.pdf}
\includegraphics[width=9.0cm,angle=0]{ms2022-0304fig10-1376.pdf}
\vspace{-0.5cm}

\includegraphics[width=9.0cm,angle=0]{ms2022-0304fig10-1377.pdf}
\includegraphics[width=9.0cm,angle=0]{ms2022-0304fig10-1378.pdf}
\vspace{-0.5cm}

\includegraphics[width=9.0cm,angle=0]{ms2022-0304fig10-1379.pdf}
\includegraphics[width=9.0cm,angle=0]{ms2022-0304fig10-1380.pdf}
\end{figure}
\clearpage

\begin{figure}
\includegraphics[width=9.0cm,angle=0]{ms2022-0304fig10-1381.pdf}
\includegraphics[width=9.0cm,angle=0]{ms2022-0304fig10-1382.pdf}
\vspace{-0.5cm}

\includegraphics[width=9.0cm,angle=0]{ms2022-0304fig10-1383.pdf}
\includegraphics[width=9.0cm,angle=0]{ms2022-0304fig10-1384.pdf}
\vspace{-0.5cm}

\includegraphics[width=9.0cm,angle=0]{ms2022-0304fig10-1385.pdf}
\includegraphics[width=9.0cm,angle=0]{ms2022-0304fig10-1386.pdf}
\vspace{-0.5cm}

\includegraphics[width=9.0cm,angle=0]{ms2022-0304fig10-1387.pdf}
\includegraphics[width=9.0cm,angle=0]{ms2022-0304fig10-1388.pdf}
\vspace{-0.5cm}

\includegraphics[width=9.0cm,angle=0]{ms2022-0304fig10-1389.pdf}
\includegraphics[width=9.0cm,angle=0]{ms2022-0304fig10-1390.pdf}
\end{figure}
\clearpage

\begin{figure}
\includegraphics[width=9.0cm,angle=0]{ms2022-0304fig10-1391.pdf}
\includegraphics[width=9.0cm,angle=0]{ms2022-0304fig10-1392.pdf}
\vspace{-0.5cm}

\includegraphics[width=9.0cm,angle=0]{ms2022-0304fig10-1393.pdf}
\includegraphics[width=9.0cm,angle=0]{ms2022-0304fig10-1394.pdf}
\vspace{-0.5cm}

\includegraphics[width=9.0cm,angle=0]{ms2022-0304fig10-1395.pdf}
\includegraphics[width=9.0cm,angle=0]{ms2022-0304fig10-1396.pdf}
\vspace{-0.5cm}

\includegraphics[width=9.0cm,angle=0]{ms2022-0304fig10-1397.pdf}
\includegraphics[width=9.0cm,angle=0]{ms2022-0304fig10-1398.pdf}
\vspace{-0.5cm}

\includegraphics[width=9.0cm,angle=0]{ms2022-0304fig10-1399.pdf}
\includegraphics[width=9.0cm,angle=0]{ms2022-0304fig10-1400.pdf}
\end{figure}
\clearpage

\begin{figure}
\includegraphics[width=9.0cm,angle=0]{ms2022-0304fig10-1401.pdf}
\includegraphics[width=9.0cm,angle=0]{ms2022-0304fig10-1402.pdf}
\vspace{-0.5cm}

\includegraphics[width=9.0cm,angle=0]{ms2022-0304fig10-1403.pdf}
\includegraphics[width=9.0cm,angle=0]{ms2022-0304fig10-1404.pdf}
\vspace{-0.5cm}

\includegraphics[width=9.0cm,angle=0]{ms2022-0304fig10-1405.pdf}
\includegraphics[width=9.0cm,angle=0]{ms2022-0304fig10-1406.pdf}
\vspace{-0.5cm}

\includegraphics[width=9.0cm,angle=0]{ms2022-0304fig10-1407.pdf}
\includegraphics[width=9.0cm,angle=0]{ms2022-0304fig10-1408.pdf}
\vspace{-0.5cm}

\includegraphics[width=9.0cm,angle=0]{ms2022-0304fig10-1409.pdf}
\includegraphics[width=9.0cm,angle=0]{ms2022-0304fig10-1410.pdf}
\end{figure}
\clearpage

\begin{figure}
\includegraphics[width=9.0cm,angle=0]{ms2022-0304fig10-1411.pdf}
\includegraphics[width=9.0cm,angle=0]{ms2022-0304fig10-1412.pdf}
\vspace{-0.5cm}

\includegraphics[width=9.0cm,angle=0]{ms2022-0304fig10-1413.pdf}
\includegraphics[width=9.0cm,angle=0]{ms2022-0304fig10-1414.pdf}
\vspace{-0.5cm}

\includegraphics[width=9.0cm,angle=0]{ms2022-0304fig10-1415.pdf}
\includegraphics[width=9.0cm,angle=0]{ms2022-0304fig10-1416.pdf}
\vspace{-0.5cm}

\includegraphics[width=9.0cm,angle=0]{ms2022-0304fig10-1417.pdf}
\includegraphics[width=9.0cm,angle=0]{ms2022-0304fig10-1418.pdf}
\vspace{-0.5cm}

\includegraphics[width=9.0cm,angle=0]{ms2022-0304fig10-1419.pdf}
\includegraphics[width=9.0cm,angle=0]{ms2022-0304fig10-1420.pdf}
\end{figure}
\clearpage

\begin{figure}
\includegraphics[width=9.0cm,angle=0]{ms2022-0304fig10-1421.pdf}
\includegraphics[width=9.0cm,angle=0]{ms2022-0304fig10-1422.pdf}
\vspace{-0.5cm}

\includegraphics[width=9.0cm,angle=0]{ms2022-0304fig10-1423.pdf}
\includegraphics[width=9.0cm,angle=0]{ms2022-0304fig10-1424.pdf}
\vspace{-0.5cm}

\includegraphics[width=9.0cm,angle=0]{ms2022-0304fig10-1425.pdf}
\includegraphics[width=9.0cm,angle=0]{ms2022-0304fig10-1426.pdf}
\vspace{-0.5cm}

\includegraphics[width=9.0cm,angle=0]{ms2022-0304fig10-1427.pdf}
\includegraphics[width=9.0cm,angle=0]{ms2022-0304fig10-1428.pdf}
\vspace{-0.5cm}

\includegraphics[width=9.0cm,angle=0]{ms2022-0304fig10-1429.pdf}
\includegraphics[width=9.0cm,angle=0]{ms2022-0304fig10-1430.pdf}
\end{figure}
\clearpage

\begin{figure}
\includegraphics[width=9.0cm,angle=0]{ms2022-0304fig10-1431.pdf}
\includegraphics[width=9.0cm,angle=0]{ms2022-0304fig10-1432.pdf}
\vspace{-0.5cm}

\includegraphics[width=9.0cm,angle=0]{ms2022-0304fig10-1433.pdf}
\includegraphics[width=9.0cm,angle=0]{ms2022-0304fig10-1434.pdf}
\vspace{-0.5cm}

\includegraphics[width=9.0cm,angle=0]{ms2022-0304fig10-1435.pdf}
\includegraphics[width=9.0cm,angle=0]{ms2022-0304fig10-1436.pdf}
\vspace{-0.5cm}

\includegraphics[width=9.0cm,angle=0]{ms2022-0304fig10-1437.pdf}
\includegraphics[width=9.0cm,angle=0]{ms2022-0304fig10-1438.pdf}
\vspace{-0.5cm}

\includegraphics[width=9.0cm,angle=0]{ms2022-0304fig10-1439.pdf}
\includegraphics[width=9.0cm,angle=0]{ms2022-0304fig10-1440.pdf}
\end{figure}
\clearpage

\begin{figure}
\includegraphics[width=9.0cm,angle=0]{ms2022-0304fig10-1441.pdf}
\includegraphics[width=9.0cm,angle=0]{ms2022-0304fig10-1442.pdf}
\vspace{-0.5cm}

\includegraphics[width=9.0cm,angle=0]{ms2022-0304fig10-1443.pdf}
\includegraphics[width=9.0cm,angle=0]{ms2022-0304fig10-1444.pdf}
\vspace{-0.5cm}

\includegraphics[width=9.0cm,angle=0]{ms2022-0304fig10-1445.pdf}
\includegraphics[width=9.0cm,angle=0]{ms2022-0304fig10-1446.pdf}
\vspace{-0.5cm}

\includegraphics[width=9.0cm,angle=0]{ms2022-0304fig10-1447.pdf}
\includegraphics[width=9.0cm,angle=0]{ms2022-0304fig10-1448.pdf}
\vspace{-0.5cm}

\includegraphics[width=9.0cm,angle=0]{ms2022-0304fig10-1449.pdf}
\includegraphics[width=9.0cm,angle=0]{ms2022-0304fig10-1450.pdf}
\end{figure}
\clearpage

\begin{figure}
\includegraphics[width=9.0cm,angle=0]{ms2022-0304fig10-1451.pdf}
\includegraphics[width=9.0cm,angle=0]{ms2022-0304fig10-1452.pdf}
\vspace{-0.5cm}

\includegraphics[width=9.0cm,angle=0]{ms2022-0304fig10-1453.pdf}
\includegraphics[width=9.0cm,angle=0]{ms2022-0304fig10-1454.pdf}
\vspace{-0.5cm}

\includegraphics[width=9.0cm,angle=0]{ms2022-0304fig10-1455.pdf}
\includegraphics[width=9.0cm,angle=0]{ms2022-0304fig10-1456.pdf}
\vspace{-0.5cm}

\includegraphics[width=9.0cm,angle=0]{ms2022-0304fig10-1457.pdf}
\includegraphics[width=9.0cm,angle=0]{ms2022-0304fig10-1458.pdf}
\vspace{-0.5cm}

\includegraphics[width=9.0cm,angle=0]{ms2022-0304fig10-1459.pdf}
\includegraphics[width=9.0cm,angle=0]{ms2022-0304fig10-1460.pdf}
\end{figure}
\clearpage

\begin{figure}
\includegraphics[width=9.0cm,angle=0]{ms2022-0304fig10-1461.pdf}
\includegraphics[width=9.0cm,angle=0]{ms2022-0304fig10-1462.pdf}
\vspace{-0.5cm}

\includegraphics[width=9.0cm,angle=0]{ms2022-0304fig10-1463.pdf}
\includegraphics[width=9.0cm,angle=0]{ms2022-0304fig10-1464.pdf}
\vspace{-0.5cm}

\includegraphics[width=9.0cm,angle=0]{ms2022-0304fig10-1465.pdf}
\includegraphics[width=9.0cm,angle=0]{ms2022-0304fig10-1466.pdf}
\vspace{-0.5cm}

\includegraphics[width=9.0cm,angle=0]{ms2022-0304fig10-1467.pdf}
\includegraphics[width=9.0cm,angle=0]{ms2022-0304fig10-1468.pdf}
\vspace{-0.5cm}

\includegraphics[width=9.0cm,angle=0]{ms2022-0304fig10-1469.pdf}
\includegraphics[width=9.0cm,angle=0]{ms2022-0304fig10-1470.pdf}
\end{figure}
\clearpage

\begin{figure}
\includegraphics[width=9.0cm,angle=0]{ms2022-0304fig10-1471.pdf}
\includegraphics[width=9.0cm,angle=0]{ms2022-0304fig10-1472.pdf}
\vspace{-0.5cm}

\includegraphics[width=9.0cm,angle=0]{ms2022-0304fig10-1473.pdf}
\includegraphics[width=9.0cm,angle=0]{ms2022-0304fig10-1474.pdf}
\vspace{-0.5cm}

\includegraphics[width=9.0cm,angle=0]{ms2022-0304fig10-1475.pdf}
\includegraphics[width=9.0cm,angle=0]{ms2022-0304fig10-1476.pdf}
\vspace{-0.5cm}

\includegraphics[width=9.0cm,angle=0]{ms2022-0304fig10-1477.pdf}
\includegraphics[width=9.0cm,angle=0]{ms2022-0304fig10-1478.pdf}
\vspace{-0.5cm}

\includegraphics[width=9.0cm,angle=0]{ms2022-0304fig10-1479.pdf}
\includegraphics[width=9.0cm,angle=0]{ms2022-0304fig10-1480.pdf}
\end{figure}
\clearpage

\begin{figure}
\includegraphics[width=9.0cm,angle=0]{ms2022-0304fig10-1481.pdf}
\includegraphics[width=9.0cm,angle=0]{ms2022-0304fig10-1482.pdf}
\vspace{-0.5cm}

\includegraphics[width=9.0cm,angle=0]{ms2022-0304fig10-1483.pdf}
\includegraphics[width=9.0cm,angle=0]{ms2022-0304fig10-1484.pdf}
\vspace{-0.5cm}

\includegraphics[width=9.0cm,angle=0]{ms2022-0304fig10-1485.pdf}
\includegraphics[width=9.0cm,angle=0]{ms2022-0304fig10-1486.pdf}
\vspace{-0.5cm}

\includegraphics[width=9.0cm,angle=0]{ms2022-0304fig10-1487.pdf}
\includegraphics[width=9.0cm,angle=0]{ms2022-0304fig10-1488.pdf}
\vspace{-0.5cm}

\includegraphics[width=9.0cm,angle=0]{ms2022-0304fig10-1489.pdf}
\includegraphics[width=9.0cm,angle=0]{ms2022-0304fig10-1490.pdf}
\end{figure}
\clearpage

\begin{figure}
\includegraphics[width=9.0cm,angle=0]{ms2022-0304fig10-1491.pdf}
\includegraphics[width=9.0cm,angle=0]{ms2022-0304fig10-1492.pdf}
\vspace{-0.5cm}

\includegraphics[width=9.0cm,angle=0]{ms2022-0304fig10-1493.pdf}
\includegraphics[width=9.0cm,angle=0]{ms2022-0304fig10-1494.pdf}
\vspace{-0.5cm}

\includegraphics[width=9.0cm,angle=0]{ms2022-0304fig10-1495.pdf}
\includegraphics[width=9.0cm,angle=0]{ms2022-0304fig10-1496.pdf}
\vspace{-0.5cm}

\includegraphics[width=9.0cm,angle=0]{ms2022-0304fig10-1497.pdf}
\includegraphics[width=9.0cm,angle=0]{ms2022-0304fig10-1498.pdf}
\vspace{-0.5cm}

\includegraphics[width=9.0cm,angle=0]{ms2022-0304fig10-1499.pdf}
\includegraphics[width=9.0cm,angle=0]{ms2022-0304fig10-1500.pdf}
\end{figure}
\clearpage

\begin{figure}
\includegraphics[width=9.0cm,angle=0]{ms2022-0304fig10-1501.pdf}
\includegraphics[width=9.0cm,angle=0]{ms2022-0304fig10-1502.pdf}
\vspace{-0.5cm}

\includegraphics[width=9.0cm,angle=0]{ms2022-0304fig10-1503.pdf}
\includegraphics[width=9.0cm,angle=0]{ms2022-0304fig10-1504.pdf}
\vspace{-0.5cm}

\includegraphics[width=9.0cm,angle=0]{ms2022-0304fig10-1505.pdf}
\includegraphics[width=9.0cm,angle=0]{ms2022-0304fig10-1506.pdf}
\vspace{-0.5cm}

\includegraphics[width=9.0cm,angle=0]{ms2022-0304fig10-1507.pdf}
\includegraphics[width=9.0cm,angle=0]{ms2022-0304fig10-1508.pdf}
\vspace{-0.5cm}

\includegraphics[width=9.0cm,angle=0]{ms2022-0304fig10-1509.pdf}
\includegraphics[width=9.0cm,angle=0]{ms2022-0304fig10-1510.pdf}
\end{figure}
\clearpage

\begin{figure}
\includegraphics[width=9.0cm,angle=0]{ms2022-0304fig10-1511.pdf}
\includegraphics[width=9.0cm,angle=0]{ms2022-0304fig10-1512.pdf}
\vspace{-0.5cm}

\includegraphics[width=9.0cm,angle=0]{ms2022-0304fig10-1513.pdf}
\includegraphics[width=9.0cm,angle=0]{ms2022-0304fig10-1514.pdf}
\vspace{-0.5cm}

\includegraphics[width=9.0cm,angle=0]{ms2022-0304fig10-1515.pdf}
\includegraphics[width=9.0cm,angle=0]{ms2022-0304fig10-1516.pdf}
\vspace{-0.5cm}

\includegraphics[width=9.0cm,angle=0]{ms2022-0304fig10-1517.pdf}
\includegraphics[width=9.0cm,angle=0]{ms2022-0304fig10-1518.pdf}
\vspace{-0.5cm}

\includegraphics[width=9.0cm,angle=0]{ms2022-0304fig10-1519.pdf}
\includegraphics[width=9.0cm,angle=0]{ms2022-0304fig10-1520.pdf}
\end{figure}
\clearpage

\begin{figure}
\includegraphics[width=9.0cm,angle=0]{ms2022-0304fig10-1521.pdf}
\includegraphics[width=9.0cm,angle=0]{ms2022-0304fig10-1522.pdf}
\vspace{-0.5cm}

\includegraphics[width=9.0cm,angle=0]{ms2022-0304fig10-1523.pdf}
\includegraphics[width=9.0cm,angle=0]{ms2022-0304fig10-1524.pdf}
\vspace{-0.5cm}

\includegraphics[width=9.0cm,angle=0]{ms2022-0304fig10-1525.pdf}
\includegraphics[width=9.0cm,angle=0]{ms2022-0304fig10-1526.pdf}
\vspace{-0.5cm}

\includegraphics[width=9.0cm,angle=0]{ms2022-0304fig10-1527.pdf}
\includegraphics[width=9.0cm,angle=0]{ms2022-0304fig10-1528.pdf}
\vspace{-0.5cm}

\includegraphics[width=9.0cm,angle=0]{ms2022-0304fig10-1529.pdf}
\includegraphics[width=9.0cm,angle=0]{ms2022-0304fig10-1530.pdf}
\end{figure}
\clearpage

\begin{figure}
\includegraphics[width=9.0cm,angle=0]{ms2022-0304fig10-1531.pdf}
\includegraphics[width=9.0cm,angle=0]{ms2022-0304fig10-1532.pdf}
\vspace{-0.5cm}

\includegraphics[width=9.0cm,angle=0]{ms2022-0304fig10-1533.pdf}
\includegraphics[width=9.0cm,angle=0]{ms2022-0304fig10-1534.pdf}
\vspace{-0.5cm}

\includegraphics[width=9.0cm,angle=0]{ms2022-0304fig10-1535.pdf}
\includegraphics[width=9.0cm,angle=0]{ms2022-0304fig10-1536.pdf}
\vspace{-0.5cm}

\includegraphics[width=9.0cm,angle=0]{ms2022-0304fig10-1537.pdf}
\includegraphics[width=9.0cm,angle=0]{ms2022-0304fig10-1538.pdf}
\vspace{-0.5cm}

\includegraphics[width=9.0cm,angle=0]{ms2022-0304fig10-1539.pdf}
\includegraphics[width=9.0cm,angle=0]{ms2022-0304fig10-1540.pdf}
\end{figure}
\clearpage

\begin{figure}
\includegraphics[width=9.0cm,angle=0]{ms2022-0304fig10-1541.pdf}
\includegraphics[width=9.0cm,angle=0]{ms2022-0304fig10-1542.pdf}
\vspace{-0.5cm}

\includegraphics[width=9.0cm,angle=0]{ms2022-0304fig10-1543.pdf}
\includegraphics[width=9.0cm,angle=0]{ms2022-0304fig10-1544.pdf}
\vspace{-0.5cm}

\includegraphics[width=9.0cm,angle=0]{ms2022-0304fig10-1545.pdf}
\includegraphics[width=9.0cm,angle=0]{ms2022-0304fig10-1546.pdf}
\vspace{-0.5cm}

\includegraphics[width=9.0cm,angle=0]{ms2022-0304fig10-1547.pdf}
\includegraphics[width=9.0cm,angle=0]{ms2022-0304fig10-1548.pdf}
\vspace{-0.5cm}

\includegraphics[width=9.0cm,angle=0]{ms2022-0304fig10-1549.pdf}
\includegraphics[width=9.0cm,angle=0]{ms2022-0304fig10-1550.pdf}
\end{figure}
\clearpage

\begin{figure}
\includegraphics[width=9.0cm,angle=0]{ms2022-0304fig10-1551.pdf}
\includegraphics[width=9.0cm,angle=0]{ms2022-0304fig10-1552.pdf}
\vspace{-0.5cm}

\includegraphics[width=9.0cm,angle=0]{ms2022-0304fig10-1553.pdf}
\includegraphics[width=9.0cm,angle=0]{ms2022-0304fig10-1554.pdf}
\vspace{-0.5cm}

\includegraphics[width=9.0cm,angle=0]{ms2022-0304fig10-1555.pdf}
\includegraphics[width=9.0cm,angle=0]{ms2022-0304fig10-1556.pdf}
\vspace{-0.5cm}

\includegraphics[width=9.0cm,angle=0]{ms2022-0304fig10-1557.pdf}
\includegraphics[width=9.0cm,angle=0]{ms2022-0304fig10-1558.pdf}
\vspace{-0.5cm}

\includegraphics[width=9.0cm,angle=0]{ms2022-0304fig10-1559.pdf}
\includegraphics[width=9.0cm,angle=0]{ms2022-0304fig10-1560.pdf}
\end{figure}
\clearpage

\begin{figure}
\includegraphics[width=9.0cm,angle=0]{ms2022-0304fig10-1561.pdf}
\includegraphics[width=9.0cm,angle=0]{ms2022-0304fig10-1562.pdf}
\vspace{-0.5cm}

\includegraphics[width=9.0cm,angle=0]{ms2022-0304fig10-1563.pdf}
\includegraphics[width=9.0cm,angle=0]{ms2022-0304fig10-1564.pdf}
\vspace{-0.5cm}

\includegraphics[width=9.0cm,angle=0]{ms2022-0304fig10-1565.pdf}
\includegraphics[width=9.0cm,angle=0]{ms2022-0304fig10-1566.pdf}
\vspace{-0.5cm}

\includegraphics[width=9.0cm,angle=0]{ms2022-0304fig10-1567.pdf}
\includegraphics[width=9.0cm,angle=0]{ms2022-0304fig10-1568.pdf}
\vspace{-0.5cm}

\includegraphics[width=9.0cm,angle=0]{ms2022-0304fig10-1569.pdf}
\includegraphics[width=9.0cm,angle=0]{ms2022-0304fig10-1570.pdf}
\end{figure}
\clearpage

\begin{figure}
\includegraphics[width=9.0cm,angle=0]{ms2022-0304fig10-1571.pdf}
\includegraphics[width=9.0cm,angle=0]{ms2022-0304fig10-1572.pdf}
\vspace{-0.5cm}

\includegraphics[width=9.0cm,angle=0]{ms2022-0304fig10-1573.pdf}
\includegraphics[width=9.0cm,angle=0]{ms2022-0304fig10-1574.pdf}
\vspace{-0.5cm}

\includegraphics[width=9.0cm,angle=0]{ms2022-0304fig10-1575.pdf}
\includegraphics[width=9.0cm,angle=0]{ms2022-0304fig10-1576.pdf}
\vspace{-0.5cm}

\includegraphics[width=9.0cm,angle=0]{ms2022-0304fig10-1577.pdf}
\includegraphics[width=9.0cm,angle=0]{ms2022-0304fig10-1578.pdf}
\vspace{-0.5cm}

\includegraphics[width=9.0cm,angle=0]{ms2022-0304fig10-1579.pdf}
\includegraphics[width=9.0cm,angle=0]{ms2022-0304fig10-1580.pdf}
\end{figure}
\clearpage

\begin{figure}
\includegraphics[width=9.0cm,angle=0]{ms2022-0304fig10-1581.pdf}
\includegraphics[width=9.0cm,angle=0]{ms2022-0304fig10-1582.pdf}
\vspace{-0.5cm}

\includegraphics[width=9.0cm,angle=0]{ms2022-0304fig10-1583.pdf}
\includegraphics[width=9.0cm,angle=0]{ms2022-0304fig10-1584.pdf}
\vspace{-0.5cm}

\includegraphics[width=9.0cm,angle=0]{ms2022-0304fig10-1585.pdf}
\includegraphics[width=9.0cm,angle=0]{ms2022-0304fig10-1586.pdf}
\vspace{-0.5cm}

\includegraphics[width=9.0cm,angle=0]{ms2022-0304fig10-1587.pdf}
\includegraphics[width=9.0cm,angle=0]{ms2022-0304fig10-1588.pdf}
\vspace{-0.5cm}

\includegraphics[width=9.0cm,angle=0]{ms2022-0304fig10-1589.pdf}
\includegraphics[width=9.0cm,angle=0]{ms2022-0304fig10-1590.pdf}
\end{figure}
\clearpage

\begin{figure}
\includegraphics[width=9.0cm,angle=0]{ms2022-0304fig10-1591.pdf}
\includegraphics[width=9.0cm,angle=0]{ms2022-0304fig10-1592.pdf}
\vspace{-0.5cm}

\includegraphics[width=9.0cm,angle=0]{ms2022-0304fig10-1593.pdf}
\includegraphics[width=9.0cm,angle=0]{ms2022-0304fig10-1594.pdf}
\vspace{-0.5cm}

\includegraphics[width=9.0cm,angle=0]{ms2022-0304fig10-1595.pdf}
\includegraphics[width=9.0cm,angle=0]{ms2022-0304fig10-1596.pdf}
\vspace{-0.5cm}

\includegraphics[width=9.0cm,angle=0]{ms2022-0304fig10-1597.pdf}
\includegraphics[width=9.0cm,angle=0]{ms2022-0304fig10-1598.pdf}
\vspace{-0.5cm}

\includegraphics[width=9.0cm,angle=0]{ms2022-0304fig10-1599.pdf}
\includegraphics[width=9.0cm,angle=0]{ms2022-0304fig10-1600.pdf}
\end{figure}
\clearpage

\begin{figure}
\includegraphics[width=9.0cm,angle=0]{ms2022-0304fig10-1601.pdf}
\includegraphics[width=9.0cm,angle=0]{ms2022-0304fig10-1602.pdf}
\vspace{-0.5cm}

\includegraphics[width=9.0cm,angle=0]{ms2022-0304fig10-1603.pdf}
\includegraphics[width=9.0cm,angle=0]{ms2022-0304fig10-1604.pdf}
\vspace{-0.5cm}

\includegraphics[width=9.0cm,angle=0]{ms2022-0304fig10-1605.pdf}
\includegraphics[width=9.0cm,angle=0]{ms2022-0304fig10-1606.pdf}
\vspace{-0.5cm}

\includegraphics[width=9.0cm,angle=0]{ms2022-0304fig10-1607.pdf}
\includegraphics[width=9.0cm,angle=0]{ms2022-0304fig10-1608.pdf}
\vspace{-0.5cm}

\includegraphics[width=9.0cm,angle=0]{ms2022-0304fig10-1609.pdf}
\includegraphics[width=9.0cm,angle=0]{ms2022-0304fig10-1610.pdf}
\end{figure}
\clearpage

\begin{figure}
\includegraphics[width=9.0cm,angle=0]{ms2022-0304fig10-1611.pdf}
\includegraphics[width=9.0cm,angle=0]{ms2022-0304fig10-1612.pdf}
\vspace{-0.5cm}

\includegraphics[width=9.0cm,angle=0]{ms2022-0304fig10-1613.pdf}
\includegraphics[width=9.0cm,angle=0]{ms2022-0304fig10-1614.pdf}
\vspace{-0.5cm}

\includegraphics[width=9.0cm,angle=0]{ms2022-0304fig10-1615.pdf}
\includegraphics[width=9.0cm,angle=0]{ms2022-0304fig10-1616.pdf}
\vspace{-0.5cm}

\includegraphics[width=9.0cm,angle=0]{ms2022-0304fig10-1617.pdf}
\includegraphics[width=9.0cm,angle=0]{ms2022-0304fig10-1618.pdf}
\vspace{-0.5cm}

\includegraphics[width=9.0cm,angle=0]{ms2022-0304fig10-1619.pdf}
\includegraphics[width=9.0cm,angle=0]{ms2022-0304fig10-1620.pdf}
\end{figure}
\clearpage

\begin{figure}
\includegraphics[width=9.0cm,angle=0]{ms2022-0304fig10-1621.pdf}
\includegraphics[width=9.0cm,angle=0]{ms2022-0304fig10-1622.pdf}
\vspace{-0.5cm}

\includegraphics[width=9.0cm,angle=0]{ms2022-0304fig10-1623.pdf}
\includegraphics[width=9.0cm,angle=0]{ms2022-0304fig10-1624.pdf}
\vspace{-0.5cm}

\includegraphics[width=9.0cm,angle=0]{ms2022-0304fig10-1625.pdf}
\includegraphics[width=9.0cm,angle=0]{ms2022-0304fig10-1626.pdf}
\vspace{-0.5cm}

\includegraphics[width=9.0cm,angle=0]{ms2022-0304fig10-1627.pdf}
\includegraphics[width=9.0cm,angle=0]{ms2022-0304fig10-1628.pdf}
\vspace{-0.5cm}

\includegraphics[width=9.0cm,angle=0]{ms2022-0304fig10-1629.pdf}
\includegraphics[width=9.0cm,angle=0]{ms2022-0304fig10-1630.pdf}
\end{figure}
\clearpage

\begin{figure}
\includegraphics[width=9.0cm,angle=0]{ms2022-0304fig10-1631.pdf}
\includegraphics[width=9.0cm,angle=0]{ms2022-0304fig10-1632.pdf}
\vspace{-0.5cm}

\includegraphics[width=9.0cm,angle=0]{ms2022-0304fig10-1633.pdf}
\includegraphics[width=9.0cm,angle=0]{ms2022-0304fig10-1634.pdf}
\vspace{-0.5cm}

\includegraphics[width=9.0cm,angle=0]{ms2022-0304fig10-1635.pdf}
\includegraphics[width=9.0cm,angle=0]{ms2022-0304fig10-1636.pdf}
\vspace{-0.5cm}

\includegraphics[width=9.0cm,angle=0]{ms2022-0304fig10-1637.pdf}
\includegraphics[width=9.0cm,angle=0]{ms2022-0304fig10-1638.pdf}
\vspace{-0.5cm}

\includegraphics[width=9.0cm,angle=0]{ms2022-0304fig10-1639.pdf}
\includegraphics[width=9.0cm,angle=0]{ms2022-0304fig10-1640.pdf}
\end{figure}
\clearpage

\begin{figure}
\includegraphics[width=9.0cm,angle=0]{ms2022-0304fig10-1641.pdf}
\includegraphics[width=9.0cm,angle=0]{ms2022-0304fig10-1642.pdf}
\vspace{-0.5cm}

\includegraphics[width=9.0cm,angle=0]{ms2022-0304fig10-1643.pdf}
\includegraphics[width=9.0cm,angle=0]{ms2022-0304fig10-1644.pdf}
\vspace{-0.5cm}

\includegraphics[width=9.0cm,angle=0]{ms2022-0304fig10-1645.pdf}
\includegraphics[width=9.0cm,angle=0]{ms2022-0304fig10-1646.pdf}
\vspace{-0.5cm}

\includegraphics[width=9.0cm,angle=0]{ms2022-0304fig10-1647.pdf}
\includegraphics[width=9.0cm,angle=0]{ms2022-0304fig10-1648.pdf}
\vspace{-0.5cm}

\includegraphics[width=9.0cm,angle=0]{ms2022-0304fig10-1649.pdf}
\includegraphics[width=9.0cm,angle=0]{ms2022-0304fig10-1650.pdf}
\end{figure}
\clearpage

\begin{figure}
\includegraphics[width=9.0cm,angle=0]{ms2022-0304fig10-1651.pdf}
\includegraphics[width=9.0cm,angle=0]{ms2022-0304fig10-1652.pdf}
\vspace{-0.5cm}

\includegraphics[width=9.0cm,angle=0]{ms2022-0304fig10-1653.pdf}
\includegraphics[width=9.0cm,angle=0]{ms2022-0304fig10-1654.pdf}
\vspace{-0.5cm}

\includegraphics[width=9.0cm,angle=0]{ms2022-0304fig10-1655.pdf}
\includegraphics[width=9.0cm,angle=0]{ms2022-0304fig10-1656.pdf}
\vspace{-0.5cm}

\includegraphics[width=9.0cm,angle=0]{ms2022-0304fig10-1657.pdf}
\includegraphics[width=9.0cm,angle=0]{ms2022-0304fig10-1658.pdf}
\vspace{-0.5cm}

\includegraphics[width=9.0cm,angle=0]{ms2022-0304fig10-1659.pdf}
\includegraphics[width=9.0cm,angle=0]{ms2022-0304fig10-1660.pdf}
\end{figure}
\clearpage

\begin{figure}
\includegraphics[width=9.0cm,angle=0]{ms2022-0304fig10-1661.pdf}
\includegraphics[width=9.0cm,angle=0]{ms2022-0304fig10-1662.pdf}
\vspace{-0.5cm}

\includegraphics[width=9.0cm,angle=0]{ms2022-0304fig10-1663.pdf}
\includegraphics[width=9.0cm,angle=0]{ms2022-0304fig10-1664.pdf}
\vspace{-0.5cm}

\includegraphics[width=9.0cm,angle=0]{ms2022-0304fig10-1665.pdf}
\includegraphics[width=9.0cm,angle=0]{ms2022-0304fig10-1666.pdf}
\vspace{-0.5cm}

\includegraphics[width=9.0cm,angle=0]{ms2022-0304fig10-1667.pdf}
\includegraphics[width=9.0cm,angle=0]{ms2022-0304fig10-1668.pdf}
\vspace{-0.5cm}

\includegraphics[width=9.0cm,angle=0]{ms2022-0304fig10-1669.pdf}
\includegraphics[width=9.0cm,angle=0]{ms2022-0304fig10-1670.pdf}
\end{figure}
\clearpage

\begin{figure}
\includegraphics[width=9.0cm,angle=0]{ms2022-0304fig10-1671.pdf}
\includegraphics[width=9.0cm,angle=0]{ms2022-0304fig10-1672.pdf}
\vspace{-0.5cm}

\includegraphics[width=9.0cm,angle=0]{ms2022-0304fig10-1673.pdf}
\includegraphics[width=9.0cm,angle=0]{ms2022-0304fig10-1674.pdf}
\vspace{-0.5cm}

\includegraphics[width=9.0cm,angle=0]{ms2022-0304fig10-1675.pdf}
\includegraphics[width=9.0cm,angle=0]{ms2022-0304fig10-1676.pdf}
\vspace{-0.5cm}

\includegraphics[width=9.0cm,angle=0]{ms2022-0304fig10-1677.pdf}
\includegraphics[width=9.0cm,angle=0]{ms2022-0304fig10-1678.pdf}
\vspace{-0.5cm}

\includegraphics[width=9.0cm,angle=0]{ms2022-0304fig10-1679.pdf}
\includegraphics[width=9.0cm,angle=0]{ms2022-0304fig10-1680.pdf}
\end{figure}
\clearpage

\begin{figure}
\includegraphics[width=9.0cm,angle=0]{ms2022-0304fig10-1681.pdf}
\includegraphics[width=9.0cm,angle=0]{ms2022-0304fig10-1682.pdf}
\vspace{-0.5cm}

\includegraphics[width=9.0cm,angle=0]{ms2022-0304fig10-1683.pdf}
\includegraphics[width=9.0cm,angle=0]{ms2022-0304fig10-1684.pdf}
\vspace{-0.5cm}

\includegraphics[width=9.0cm,angle=0]{ms2022-0304fig10-1685.pdf}
\includegraphics[width=9.0cm,angle=0]{ms2022-0304fig10-1686.pdf}
\vspace{-0.5cm}

\includegraphics[width=9.0cm,angle=0]{ms2022-0304fig10-1687.pdf}
\includegraphics[width=9.0cm,angle=0]{ms2022-0304fig10-1688.pdf}
\vspace{-0.5cm}

\includegraphics[width=9.0cm,angle=0]{ms2022-0304fig10-1689.pdf}
\includegraphics[width=9.0cm,angle=0]{ms2022-0304fig10-1690.pdf}
\end{figure}
\clearpage

\begin{figure}
\includegraphics[width=9.0cm,angle=0]{ms2022-0304fig10-1691.pdf}
\includegraphics[width=9.0cm,angle=0]{ms2022-0304fig10-1692.pdf}
\vspace{-0.5cm}

\includegraphics[width=9.0cm,angle=0]{ms2022-0304fig10-1693.pdf}
\includegraphics[width=9.0cm,angle=0]{ms2022-0304fig10-1694.pdf}
\vspace{-0.5cm}

\includegraphics[width=9.0cm,angle=0]{ms2022-0304fig10-1695.pdf}
\includegraphics[width=9.0cm,angle=0]{ms2022-0304fig10-1696.pdf}
\vspace{-0.5cm}

\includegraphics[width=9.0cm,angle=0]{ms2022-0304fig10-1697.pdf}
\includegraphics[width=9.0cm,angle=0]{ms2022-0304fig10-1698.pdf}
\vspace{-0.5cm}

\includegraphics[width=9.0cm,angle=0]{ms2022-0304fig10-1699.pdf}
\includegraphics[width=9.0cm,angle=0]{ms2022-0304fig10-1700.pdf}
\end{figure}
\clearpage

\begin{figure}
\includegraphics[width=9.0cm,angle=0]{ms2022-0304fig10-1701.pdf}
\includegraphics[width=9.0cm,angle=0]{ms2022-0304fig10-1702.pdf}
\vspace{-0.5cm}

\includegraphics[width=9.0cm,angle=0]{ms2022-0304fig10-1703.pdf}
\includegraphics[width=9.0cm,angle=0]{ms2022-0304fig10-1704.pdf}
\vspace{-0.5cm}

\includegraphics[width=9.0cm,angle=0]{ms2022-0304fig10-1705.pdf}
\includegraphics[width=9.0cm,angle=0]{ms2022-0304fig10-1706.pdf}
\vspace{-0.5cm}

\includegraphics[width=9.0cm,angle=0]{ms2022-0304fig10-1707.pdf}
\includegraphics[width=9.0cm,angle=0]{ms2022-0304fig10-1708.pdf}
\vspace{-0.5cm}

\includegraphics[width=9.0cm,angle=0]{ms2022-0304fig10-1709.pdf}
\includegraphics[width=9.0cm,angle=0]{ms2022-0304fig10-1710.pdf}
\end{figure}
\clearpage

\begin{figure}
\includegraphics[width=9.0cm,angle=0]{ms2022-0304fig10-1711.pdf}
\includegraphics[width=9.0cm,angle=0]{ms2022-0304fig10-1712.pdf}
\vspace{-0.5cm}

\includegraphics[width=9.0cm,angle=0]{ms2022-0304fig10-1713.pdf}
\includegraphics[width=9.0cm,angle=0]{ms2022-0304fig10-1714.pdf}
\vspace{-0.5cm}

\includegraphics[width=9.0cm,angle=0]{ms2022-0304fig10-1715.pdf}
\includegraphics[width=9.0cm,angle=0]{ms2022-0304fig10-1716.pdf}
\vspace{-0.5cm}

\includegraphics[width=9.0cm,angle=0]{ms2022-0304fig10-1717.pdf}
\includegraphics[width=9.0cm,angle=0]{ms2022-0304fig10-1718.pdf}
\vspace{-0.5cm}

\includegraphics[width=9.0cm,angle=0]{ms2022-0304fig10-1719.pdf}
\includegraphics[width=9.0cm,angle=0]{ms2022-0304fig10-1720.pdf}
\end{figure}
\clearpage

\begin{figure}
\includegraphics[width=9.0cm,angle=0]{ms2022-0304fig10-1721.pdf}
\includegraphics[width=9.0cm,angle=0]{ms2022-0304fig10-1722.pdf}
\vspace{-0.5cm}

\includegraphics[width=9.0cm,angle=0]{ms2022-0304fig10-1723.pdf}
\includegraphics[width=9.0cm,angle=0]{ms2022-0304fig10-1724.pdf}
\vspace{-0.5cm}

\includegraphics[width=9.0cm,angle=0]{ms2022-0304fig10-1725.pdf}
\includegraphics[width=9.0cm,angle=0]{ms2022-0304fig10-1726.pdf}
\vspace{-0.5cm}

\includegraphics[width=9.0cm,angle=0]{ms2022-0304fig10-1727.pdf}
\includegraphics[width=9.0cm,angle=0]{ms2022-0304fig10-1728.pdf}
\vspace{-0.5cm}

\includegraphics[width=9.0cm,angle=0]{ms2022-0304fig10-1729.pdf}
\includegraphics[width=9.0cm,angle=0]{ms2022-0304fig10-1730.pdf}
\end{figure}
\clearpage

\begin{figure}
\includegraphics[width=9.0cm,angle=0]{ms2022-0304fig10-1731.pdf}
\includegraphics[width=9.0cm,angle=0]{ms2022-0304fig10-1732.pdf}
\vspace{-0.5cm}

\includegraphics[width=9.0cm,angle=0]{ms2022-0304fig10-1733.pdf}
\includegraphics[width=9.0cm,angle=0]{ms2022-0304fig10-1734.pdf}
\vspace{-0.5cm}

\includegraphics[width=9.0cm,angle=0]{ms2022-0304fig10-1735.pdf}
\includegraphics[width=9.0cm,angle=0]{ms2022-0304fig10-1736.pdf}
\vspace{-0.5cm}

\includegraphics[width=9.0cm,angle=0]{ms2022-0304fig10-1737.pdf}
\includegraphics[width=9.0cm,angle=0]{ms2022-0304fig10-1738.pdf}
\vspace{-0.5cm}

\includegraphics[width=9.0cm,angle=0]{ms2022-0304fig10-1739.pdf}
\includegraphics[width=9.0cm,angle=0]{ms2022-0304fig10-1740.pdf}
\end{figure}
\clearpage

\begin{figure}
\includegraphics[width=9.0cm,angle=0]{ms2022-0304fig10-1741.pdf}
\includegraphics[width=9.0cm,angle=0]{ms2022-0304fig10-1742.pdf}
\vspace{-0.5cm}

\includegraphics[width=9.0cm,angle=0]{ms2022-0304fig10-1743.pdf}
\includegraphics[width=9.0cm,angle=0]{ms2022-0304fig10-1744.pdf}
\vspace{-0.5cm}

\includegraphics[width=9.0cm,angle=0]{ms2022-0304fig10-1745.pdf}
\includegraphics[width=9.0cm,angle=0]{ms2022-0304fig10-1746.pdf}
\vspace{-0.5cm}

\includegraphics[width=9.0cm,angle=0]{ms2022-0304fig10-1747.pdf}
\includegraphics[width=9.0cm,angle=0]{ms2022-0304fig10-1748.pdf}
\vspace{-0.5cm}

\includegraphics[width=9.0cm,angle=0]{ms2022-0304fig10-1749.pdf}
\includegraphics[width=9.0cm,angle=0]{ms2022-0304fig10-1750.pdf}
\end{figure}
\clearpage

\begin{figure}
\includegraphics[width=9.0cm,angle=0]{ms2022-0304fig10-1751.pdf}
\includegraphics[width=9.0cm,angle=0]{ms2022-0304fig10-1752.pdf}
\vspace{-0.5cm}

\includegraphics[width=9.0cm,angle=0]{ms2022-0304fig10-1753.pdf}
\includegraphics[width=9.0cm,angle=0]{ms2022-0304fig10-1754.pdf}
\vspace{-0.5cm}

\includegraphics[width=9.0cm,angle=0]{ms2022-0304fig10-1755.pdf}
\includegraphics[width=9.0cm,angle=0]{ms2022-0304fig10-1756.pdf}
\vspace{-0.5cm}

\includegraphics[width=9.0cm,angle=0]{ms2022-0304fig10-1757.pdf}
\includegraphics[width=9.0cm,angle=0]{ms2022-0304fig10-1758.pdf}
\vspace{-0.5cm}

\includegraphics[width=9.0cm,angle=0]{ms2022-0304fig10-1759.pdf}
\includegraphics[width=9.0cm,angle=0]{ms2022-0304fig10-1760.pdf}
\end{figure}
\clearpage

\begin{figure}
\includegraphics[width=9.0cm,angle=0]{ms2022-0304fig10-1761.pdf}
\includegraphics[width=9.0cm,angle=0]{ms2022-0304fig10-1762.pdf}
\vspace{-0.5cm}

\includegraphics[width=9.0cm,angle=0]{ms2022-0304fig10-1763.pdf}
\includegraphics[width=9.0cm,angle=0]{ms2022-0304fig10-1764.pdf}
\vspace{-0.5cm}

\includegraphics[width=9.0cm,angle=0]{ms2022-0304fig10-1765.pdf}
\includegraphics[width=9.0cm,angle=0]{ms2022-0304fig10-1766.pdf}
\vspace{-0.5cm}

\includegraphics[width=9.0cm,angle=0]{ms2022-0304fig10-1767.pdf}
\includegraphics[width=9.0cm,angle=0]{ms2022-0304fig10-1768.pdf}
\vspace{-0.5cm}

\includegraphics[width=9.0cm,angle=0]{ms2022-0304fig10-1769.pdf}
\includegraphics[width=9.0cm,angle=0]{ms2022-0304fig10-1770.pdf}
\end{figure}
\clearpage

\begin{figure}
\includegraphics[width=9.0cm,angle=0]{ms2022-0304fig10-1771.pdf}
\includegraphics[width=9.0cm,angle=0]{ms2022-0304fig10-1772.pdf}
\vspace{-0.5cm}

\includegraphics[width=9.0cm,angle=0]{ms2022-0304fig10-1773.pdf}
\includegraphics[width=9.0cm,angle=0]{ms2022-0304fig10-1774.pdf}
\vspace{-0.5cm}

\includegraphics[width=9.0cm,angle=0]{ms2022-0304fig10-1775.pdf}
\includegraphics[width=9.0cm,angle=0]{ms2022-0304fig10-1776.pdf}
\vspace{-0.5cm}

\includegraphics[width=9.0cm,angle=0]{ms2022-0304fig10-1777.pdf}
\includegraphics[width=9.0cm,angle=0]{ms2022-0304fig10-1778.pdf}
\vspace{-0.5cm}

\includegraphics[width=9.0cm,angle=0]{ms2022-0304fig10-1779.pdf}
\includegraphics[width=9.0cm,angle=0]{ms2022-0304fig10-1780.pdf}
\end{figure}
\clearpage

\begin{figure}
\includegraphics[width=9.0cm,angle=0]{ms2022-0304fig10-1781.pdf}
\includegraphics[width=9.0cm,angle=0]{ms2022-0304fig10-1782.pdf}
\vspace{-0.5cm}

\includegraphics[width=9.0cm,angle=0]{ms2022-0304fig10-1783.pdf}
\includegraphics[width=9.0cm,angle=0]{ms2022-0304fig10-1784.pdf}
\vspace{-0.5cm}

\includegraphics[width=9.0cm,angle=0]{ms2022-0304fig10-1785.pdf}
\includegraphics[width=9.0cm,angle=0]{ms2022-0304fig10-1786.pdf}
\vspace{-0.5cm}

\includegraphics[width=9.0cm,angle=0]{ms2022-0304fig10-1787.pdf}
\includegraphics[width=9.0cm,angle=0]{ms2022-0304fig10-1788.pdf}
\vspace{-0.5cm}

\includegraphics[width=9.0cm,angle=0]{ms2022-0304fig10-1789.pdf}
\includegraphics[width=9.0cm,angle=0]{ms2022-0304fig10-1790.pdf}
\end{figure}
\clearpage

\begin{figure}
\includegraphics[width=9.0cm,angle=0]{ms2022-0304fig10-1791.pdf}
\includegraphics[width=9.0cm,angle=0]{ms2022-0304fig10-1792.pdf}
\vspace{-0.5cm}

\includegraphics[width=9.0cm,angle=0]{ms2022-0304fig10-1793.pdf}
\includegraphics[width=9.0cm,angle=0]{ms2022-0304fig10-1794.pdf}
\vspace{-0.5cm}

\includegraphics[width=9.0cm,angle=0]{ms2022-0304fig10-1795.pdf}
\includegraphics[width=9.0cm,angle=0]{ms2022-0304fig10-1796.pdf}
\vspace{-0.5cm}

\includegraphics[width=9.0cm,angle=0]{ms2022-0304fig10-1797.pdf}
\includegraphics[width=9.0cm,angle=0]{ms2022-0304fig10-1798.pdf}
\vspace{-0.5cm}

\includegraphics[width=9.0cm,angle=0]{ms2022-0304fig10-1799.pdf}
\includegraphics[width=9.0cm,angle=0]{ms2022-0304fig10-1800.pdf}
\end{figure}
\clearpage

\begin{figure}
\includegraphics[width=9.0cm,angle=0]{ms2022-0304fig10-1801.pdf}
\includegraphics[width=9.0cm,angle=0]{ms2022-0304fig10-1802.pdf}
\vspace{-0.5cm}

\includegraphics[width=9.0cm,angle=0]{ms2022-0304fig10-1803.pdf}
\includegraphics[width=9.0cm,angle=0]{ms2022-0304fig10-1804.pdf}
\vspace{-0.5cm}

\includegraphics[width=9.0cm,angle=0]{ms2022-0304fig10-1805.pdf}
\includegraphics[width=9.0cm,angle=0]{ms2022-0304fig10-1806.pdf}
\vspace{-0.5cm}

\includegraphics[width=9.0cm,angle=0]{ms2022-0304fig10-1807.pdf}
\includegraphics[width=9.0cm,angle=0]{ms2022-0304fig10-1808.pdf}
\vspace{-0.5cm}

\includegraphics[width=9.0cm,angle=0]{ms2022-0304fig10-1809.pdf}
\includegraphics[width=9.0cm,angle=0]{ms2022-0304fig10-1810.pdf}
\end{figure}
\clearpage

\begin{figure}
\includegraphics[width=9.0cm,angle=0]{ms2022-0304fig10-1811.pdf}
\includegraphics[width=9.0cm,angle=0]{ms2022-0304fig10-1812.pdf}
\vspace{-0.5cm}

\includegraphics[width=9.0cm,angle=0]{ms2022-0304fig10-1813.pdf}
\includegraphics[width=9.0cm,angle=0]{ms2022-0304fig10-1814.pdf}
\vspace{-0.5cm}

\includegraphics[width=9.0cm,angle=0]{ms2022-0304fig10-1815.pdf}
\includegraphics[width=9.0cm,angle=0]{ms2022-0304fig10-1816.pdf}
\vspace{-0.5cm}

\includegraphics[width=9.0cm,angle=0]{ms2022-0304fig10-1817.pdf}
\includegraphics[width=9.0cm,angle=0]{ms2022-0304fig10-1818.pdf}
\vspace{-0.5cm}

\includegraphics[width=9.0cm,angle=0]{ms2022-0304fig10-1819.pdf}
\includegraphics[width=9.0cm,angle=0]{ms2022-0304fig10-1820.pdf}
\end{figure}
\clearpage

\begin{figure}
\includegraphics[width=9.0cm,angle=0]{ms2022-0304fig10-1821.pdf}
\includegraphics[width=9.0cm,angle=0]{ms2022-0304fig10-1822.pdf}
\vspace{-0.5cm}

\includegraphics[width=9.0cm,angle=0]{ms2022-0304fig10-1823.pdf}
\includegraphics[width=9.0cm,angle=0]{ms2022-0304fig10-1824.pdf}
\vspace{-0.5cm}

\includegraphics[width=9.0cm,angle=0]{ms2022-0304fig10-1825.pdf}
\includegraphics[width=9.0cm,angle=0]{ms2022-0304fig10-1826.pdf}
\vspace{-0.5cm}

\includegraphics[width=9.0cm,angle=0]{ms2022-0304fig10-1827.pdf}
\includegraphics[width=9.0cm,angle=0]{ms2022-0304fig10-1828.pdf}
\vspace{-0.5cm}

\includegraphics[width=9.0cm,angle=0]{ms2022-0304fig10-1829.pdf}
\includegraphics[width=9.0cm,angle=0]{ms2022-0304fig10-1830.pdf}
\end{figure}
\clearpage

\begin{figure}
\includegraphics[width=9.0cm,angle=0]{ms2022-0304fig10-1831.pdf}
\includegraphics[width=9.0cm,angle=0]{ms2022-0304fig10-1832.pdf}
\vspace{-0.5cm}

\includegraphics[width=9.0cm,angle=0]{ms2022-0304fig10-1833.pdf}
\includegraphics[width=9.0cm,angle=0]{ms2022-0304fig10-1834.pdf}
\vspace{-0.5cm}

\includegraphics[width=9.0cm,angle=0]{ms2022-0304fig10-1835.pdf}
\includegraphics[width=9.0cm,angle=0]{ms2022-0304fig10-1836.pdf}
\vspace{-0.5cm}

\includegraphics[width=9.0cm,angle=0]{ms2022-0304fig10-1837.pdf}
\includegraphics[width=9.0cm,angle=0]{ms2022-0304fig10-1838.pdf}
\vspace{-0.5cm}

\includegraphics[width=9.0cm,angle=0]{ms2022-0304fig10-1839.pdf}
\includegraphics[width=9.0cm,angle=0]{ms2022-0304fig10-1840.pdf}
\end{figure}
\clearpage

\begin{figure}
\includegraphics[width=9.0cm,angle=0]{ms2022-0304fig10-1841.pdf}
\includegraphics[width=9.0cm,angle=0]{ms2022-0304fig10-1842.pdf}
\vspace{-0.5cm}

\includegraphics[width=9.0cm,angle=0]{ms2022-0304fig10-1843.pdf}
\includegraphics[width=9.0cm,angle=0]{ms2022-0304fig10-1844.pdf}
\vspace{-0.5cm}

\includegraphics[width=9.0cm,angle=0]{ms2022-0304fig10-1845.pdf}
\includegraphics[width=9.0cm,angle=0]{ms2022-0304fig10-1846.pdf}
\vspace{-0.5cm}

\includegraphics[width=9.0cm,angle=0]{ms2022-0304fig10-1847.pdf}
\includegraphics[width=9.0cm,angle=0]{ms2022-0304fig10-1848.pdf}
\vspace{-0.5cm}

\includegraphics[width=9.0cm,angle=0]{ms2022-0304fig10-1849.pdf}
\includegraphics[width=9.0cm,angle=0]{ms2022-0304fig10-1850.pdf}
\end{figure}
\clearpage

\begin{figure}
\includegraphics[width=9.0cm,angle=0]{ms2022-0304fig10-1851.pdf}
\includegraphics[width=9.0cm,angle=0]{ms2022-0304fig10-1852.pdf}
\vspace{-0.5cm}

\includegraphics[width=9.0cm,angle=0]{ms2022-0304fig10-1853.pdf}
\includegraphics[width=9.0cm,angle=0]{ms2022-0304fig10-1854.pdf}
\vspace{-0.5cm}

\includegraphics[width=9.0cm,angle=0]{ms2022-0304fig10-1855.pdf}
\includegraphics[width=9.0cm,angle=0]{ms2022-0304fig10-1856.pdf}
\vspace{-0.5cm}

\includegraphics[width=9.0cm,angle=0]{ms2022-0304fig10-1857.pdf}
\includegraphics[width=9.0cm,angle=0]{ms2022-0304fig10-1858.pdf}
\vspace{-0.5cm}

\includegraphics[width=9.0cm,angle=0]{ms2022-0304fig10-1859.pdf}
\includegraphics[width=9.0cm,angle=0]{ms2022-0304fig10-1860.pdf}
\end{figure}
\clearpage

\begin{figure}
\includegraphics[width=9.0cm,angle=0]{ms2022-0304fig10-1861.pdf}
\includegraphics[width=9.0cm,angle=0]{ms2022-0304fig10-1862.pdf}
\vspace{-0.5cm}

\includegraphics[width=9.0cm,angle=0]{ms2022-0304fig10-1863.pdf}
\includegraphics[width=9.0cm,angle=0]{ms2022-0304fig10-1864.pdf}
\vspace{-0.5cm}

\includegraphics[width=9.0cm,angle=0]{ms2022-0304fig10-1865.pdf}
\includegraphics[width=9.0cm,angle=0]{ms2022-0304fig10-1866.pdf}
\vspace{-0.5cm}

\includegraphics[width=9.0cm,angle=0]{ms2022-0304fig10-1867.pdf}
\includegraphics[width=9.0cm,angle=0]{ms2022-0304fig10-1868.pdf}
\vspace{-0.5cm}

\includegraphics[width=9.0cm,angle=0]{ms2022-0304fig10-1869.pdf}
\includegraphics[width=9.0cm,angle=0]{ms2022-0304fig10-1870.pdf}
\end{figure}
\clearpage

\begin{figure}
\includegraphics[width=9.0cm,angle=0]{ms2022-0304fig10-1871.pdf}
\includegraphics[width=9.0cm,angle=0]{ms2022-0304fig10-1872.pdf}
\vspace{-0.5cm}

\includegraphics[width=9.0cm,angle=0]{ms2022-0304fig10-1873.pdf}
\includegraphics[width=9.0cm,angle=0]{ms2022-0304fig10-1874.pdf}
\vspace{-0.5cm}

\includegraphics[width=9.0cm,angle=0]{ms2022-0304fig10-1875.pdf}
\includegraphics[width=9.0cm,angle=0]{ms2022-0304fig10-1876.pdf}
\vspace{-0.5cm}

\includegraphics[width=9.0cm,angle=0]{ms2022-0304fig10-1877.pdf}
\includegraphics[width=9.0cm,angle=0]{ms2022-0304fig10-1878.pdf}
\vspace{-0.5cm}

\includegraphics[width=9.0cm,angle=0]{ms2022-0304fig10-1879.pdf}
\includegraphics[width=9.0cm,angle=0]{ms2022-0304fig10-1880.pdf}
\end{figure}
\clearpage

\begin{figure}
\includegraphics[width=9.0cm,angle=0]{ms2022-0304fig10-1881.pdf}
\includegraphics[width=9.0cm,angle=0]{ms2022-0304fig10-1882.pdf}
\vspace{-0.5cm}

\includegraphics[width=9.0cm,angle=0]{ms2022-0304fig10-1883.pdf}
\includegraphics[width=9.0cm,angle=0]{ms2022-0304fig10-1884.pdf}
\vspace{-0.5cm}

\includegraphics[width=9.0cm,angle=0]{ms2022-0304fig10-1885.pdf}
\includegraphics[width=9.0cm,angle=0]{ms2022-0304fig10-1886.pdf}
\vspace{-0.5cm}

\includegraphics[width=9.0cm,angle=0]{ms2022-0304fig10-1887.pdf}
\includegraphics[width=9.0cm,angle=0]{ms2022-0304fig10-1888.pdf}
\vspace{-0.5cm}

\includegraphics[width=9.0cm,angle=0]{ms2022-0304fig10-1889.pdf}
\includegraphics[width=9.0cm,angle=0]{ms2022-0304fig10-1890.pdf}
\end{figure}
\clearpage

\begin{figure}
\includegraphics[width=9.0cm,angle=0]{ms2022-0304fig10-1891.pdf}
\includegraphics[width=9.0cm,angle=0]{ms2022-0304fig10-1892.pdf}
\vspace{-0.5cm}

\includegraphics[width=9.0cm,angle=0]{ms2022-0304fig10-1893.pdf}
\includegraphics[width=9.0cm,angle=0]{ms2022-0304fig10-1894.pdf}
\vspace{-0.5cm}

\includegraphics[width=9.0cm,angle=0]{ms2022-0304fig10-1895.pdf}
\includegraphics[width=9.0cm,angle=0]{ms2022-0304fig10-1896.pdf}
\vspace{-0.5cm}

\includegraphics[width=9.0cm,angle=0]{ms2022-0304fig10-1897.pdf}
\includegraphics[width=9.0cm,angle=0]{ms2022-0304fig10-1898.pdf}
\vspace{-0.5cm}

\includegraphics[width=9.0cm,angle=0]{ms2022-0304fig10-1899.pdf}
\includegraphics[width=9.0cm,angle=0]{ms2022-0304fig10-1900.pdf}
\end{figure}
\clearpage

\begin{figure}
\includegraphics[width=9.0cm,angle=0]{ms2022-0304fig10-1901.pdf}
\includegraphics[width=9.0cm,angle=0]{ms2022-0304fig10-1902.pdf}
\vspace{-0.5cm}

\includegraphics[width=9.0cm,angle=0]{ms2022-0304fig10-1903.pdf}
\includegraphics[width=9.0cm,angle=0]{ms2022-0304fig10-1904.pdf}
\vspace{-0.5cm}

\includegraphics[width=9.0cm,angle=0]{ms2022-0304fig10-1905.pdf}
\includegraphics[width=9.0cm,angle=0]{ms2022-0304fig10-1906.pdf}
\vspace{-0.5cm}

\includegraphics[width=9.0cm,angle=0]{ms2022-0304fig10-1907.pdf}
\includegraphics[width=9.0cm,angle=0]{ms2022-0304fig10-1908.pdf}
\vspace{-0.5cm}

\includegraphics[width=9.0cm,angle=0]{ms2022-0304fig10-1909.pdf}
\includegraphics[width=9.0cm,angle=0]{ms2022-0304fig10-1910.pdf}
\end{figure}
\clearpage

\begin{figure}
\includegraphics[width=9.0cm,angle=0]{ms2022-0304fig10-1911.pdf}
\includegraphics[width=9.0cm,angle=0]{ms2022-0304fig10-1912.pdf}
\vspace{-0.5cm}

\includegraphics[width=9.0cm,angle=0]{ms2022-0304fig10-1913.pdf}
\includegraphics[width=9.0cm,angle=0]{ms2022-0304fig10-1914.pdf}
\vspace{-0.5cm}

\includegraphics[width=9.0cm,angle=0]{ms2022-0304fig10-1915.pdf}
\includegraphics[width=9.0cm,angle=0]{ms2022-0304fig10-1916.pdf}
\vspace{-0.5cm}

\includegraphics[width=9.0cm,angle=0]{ms2022-0304fig10-1917.pdf}
\includegraphics[width=9.0cm,angle=0]{ms2022-0304fig10-1918.pdf}
\vspace{-0.5cm}

\includegraphics[width=9.0cm,angle=0]{ms2022-0304fig10-1919.pdf}
\includegraphics[width=9.0cm,angle=0]{ms2022-0304fig10-1920.pdf}
\end{figure}
\clearpage

\begin{figure}
\includegraphics[width=9.0cm,angle=0]{ms2022-0304fig10-1921.pdf}
\includegraphics[width=9.0cm,angle=0]{ms2022-0304fig10-1922.pdf}
\vspace{-0.5cm}

\includegraphics[width=9.0cm,angle=0]{ms2022-0304fig10-1923.pdf}
\includegraphics[width=9.0cm,angle=0]{ms2022-0304fig10-1924.pdf}
\vspace{-0.5cm}

\includegraphics[width=9.0cm,angle=0]{ms2022-0304fig10-1925.pdf}
\includegraphics[width=9.0cm,angle=0]{ms2022-0304fig10-1926.pdf}
\vspace{-0.5cm}

\includegraphics[width=9.0cm,angle=0]{ms2022-0304fig10-1927.pdf}
\includegraphics[width=9.0cm,angle=0]{ms2022-0304fig10-1928.pdf}
\vspace{-0.5cm}

\includegraphics[width=9.0cm,angle=0]{ms2022-0304fig10-1929.pdf}
\includegraphics[width=9.0cm,angle=0]{ms2022-0304fig10-1930.pdf}
\end{figure}
\clearpage

\begin{figure}
\includegraphics[width=9.0cm,angle=0]{ms2022-0304fig10-1931.pdf}
\includegraphics[width=9.0cm,angle=0]{ms2022-0304fig10-1932.pdf}
\vspace{-0.5cm}

\includegraphics[width=9.0cm,angle=0]{ms2022-0304fig10-1933.pdf}
\includegraphics[width=9.0cm,angle=0]{ms2022-0304fig10-1934.pdf}
\vspace{-0.5cm}

\includegraphics[width=9.0cm,angle=0]{ms2022-0304fig10-1935.pdf}
\includegraphics[width=9.0cm,angle=0]{ms2022-0304fig10-1936.pdf}
\vspace{-0.5cm}

\includegraphics[width=9.0cm,angle=0]{ms2022-0304fig10-1937.pdf}
\includegraphics[width=9.0cm,angle=0]{ms2022-0304fig10-1938.pdf}
\vspace{-0.5cm}

\includegraphics[width=9.0cm,angle=0]{ms2022-0304fig10-1939.pdf}
\includegraphics[width=9.0cm,angle=0]{ms2022-0304fig10-1940.pdf}
\end{figure}
\clearpage

\begin{figure}
\includegraphics[width=9.0cm,angle=0]{ms2022-0304fig10-1941.pdf}
\includegraphics[width=9.0cm,angle=0]{ms2022-0304fig10-1942.pdf}
\vspace{-0.5cm}

\includegraphics[width=9.0cm,angle=0]{ms2022-0304fig10-1943.pdf}
\includegraphics[width=9.0cm,angle=0]{ms2022-0304fig10-1944.pdf}
\vspace{-0.5cm}

\includegraphics[width=9.0cm,angle=0]{ms2022-0304fig10-1945.pdf}
\includegraphics[width=9.0cm,angle=0]{ms2022-0304fig10-1946.pdf}
\vspace{-0.5cm}

\includegraphics[width=9.0cm,angle=0]{ms2022-0304fig10-1947.pdf}
\includegraphics[width=9.0cm,angle=0]{ms2022-0304fig10-1948.pdf}
\vspace{-0.5cm}

\includegraphics[width=9.0cm,angle=0]{ms2022-0304fig10-1949.pdf}
\includegraphics[width=9.0cm,angle=0]{ms2022-0304fig10-1950.pdf}
\end{figure}
\clearpage

\begin{figure}
\includegraphics[width=9.0cm,angle=0]{ms2022-0304fig10-1951.pdf}
\includegraphics[width=9.0cm,angle=0]{ms2022-0304fig10-1952.pdf}
\vspace{-0.5cm}

\includegraphics[width=9.0cm,angle=0]{ms2022-0304fig10-1953.pdf}
\includegraphics[width=9.0cm,angle=0]{ms2022-0304fig10-1954.pdf}
\vspace{-0.5cm}

\includegraphics[width=9.0cm,angle=0]{ms2022-0304fig10-1955.pdf}
\includegraphics[width=9.0cm,angle=0]{ms2022-0304fig10-1956.pdf}
\vspace{-0.5cm}

\includegraphics[width=9.0cm,angle=0]{ms2022-0304fig10-1957.pdf}
\includegraphics[width=9.0cm,angle=0]{ms2022-0304fig10-1958.pdf}
\vspace{-0.5cm}

\includegraphics[width=9.0cm,angle=0]{ms2022-0304fig10-1959.pdf}
\includegraphics[width=9.0cm,angle=0]{ms2022-0304fig10-1960.pdf}
\end{figure}
\clearpage

\begin{figure}
\includegraphics[width=9.0cm,angle=0]{ms2022-0304fig10-1961.pdf}
\includegraphics[width=9.0cm,angle=0]{ms2022-0304fig10-1962.pdf}
\vspace{-0.5cm}

\includegraphics[width=9.0cm,angle=0]{ms2022-0304fig10-1963.pdf}
\includegraphics[width=9.0cm,angle=0]{ms2022-0304fig10-1964.pdf}
\vspace{-0.5cm}

\includegraphics[width=9.0cm,angle=0]{ms2022-0304fig10-1965.pdf}
\includegraphics[width=9.0cm,angle=0]{ms2022-0304fig10-1966.pdf}
\vspace{-0.5cm}

\includegraphics[width=9.0cm,angle=0]{ms2022-0304fig10-1967.pdf}
\includegraphics[width=9.0cm,angle=0]{ms2022-0304fig10-1968.pdf}
\vspace{-0.5cm}

\includegraphics[width=9.0cm,angle=0]{ms2022-0304fig10-1969.pdf}
\includegraphics[width=9.0cm,angle=0]{ms2022-0304fig10-1970.pdf}
\end{figure}
\clearpage

\begin{figure}
\includegraphics[width=9.0cm,angle=0]{ms2022-0304fig10-1971.pdf}
\includegraphics[width=9.0cm,angle=0]{ms2022-0304fig10-1972.pdf}
\vspace{-0.5cm}

\includegraphics[width=9.0cm,angle=0]{ms2022-0304fig10-1973.pdf}
\includegraphics[width=9.0cm,angle=0]{ms2022-0304fig10-1974.pdf}
\vspace{-0.5cm}

\includegraphics[width=9.0cm,angle=0]{ms2022-0304fig10-1975.pdf}
\includegraphics[width=9.0cm,angle=0]{ms2022-0304fig10-1976.pdf}
\vspace{-0.5cm}

\includegraphics[width=9.0cm,angle=0]{ms2022-0304fig10-1977.pdf}
\includegraphics[width=9.0cm,angle=0]{ms2022-0304fig10-1978.pdf}
\vspace{-0.5cm}

\includegraphics[width=9.0cm,angle=0]{ms2022-0304fig10-1979.pdf}
\includegraphics[width=9.0cm,angle=0]{ms2022-0304fig10-1980.pdf}
\end{figure}
\clearpage

\begin{figure}
\includegraphics[width=9.0cm,angle=0]{ms2022-0304fig10-1981.pdf}
\includegraphics[width=9.0cm,angle=0]{ms2022-0304fig10-1982.pdf}
\vspace{-0.5cm}

\includegraphics[width=9.0cm,angle=0]{ms2022-0304fig10-1983.pdf}
\includegraphics[width=9.0cm,angle=0]{ms2022-0304fig10-1984.pdf}
\vspace{-0.5cm}

\includegraphics[width=9.0cm,angle=0]{ms2022-0304fig10-1985.pdf}
\includegraphics[width=9.0cm,angle=0]{ms2022-0304fig10-1986.pdf}
\vspace{-0.5cm}

\includegraphics[width=9.0cm,angle=0]{ms2022-0304fig10-1987.pdf}
\includegraphics[width=9.0cm,angle=0]{ms2022-0304fig10-1988.pdf}
\vspace{-0.5cm}

\includegraphics[width=9.0cm,angle=0]{ms2022-0304fig10-1989.pdf}
\includegraphics[width=9.0cm,angle=0]{ms2022-0304fig10-1990.pdf}
\end{figure}
\clearpage

\begin{figure}
\includegraphics[width=9.0cm,angle=0]{ms2022-0304fig10-1991.pdf}
\includegraphics[width=9.0cm,angle=0]{ms2022-0304fig10-1992.pdf}
\vspace{-0.5cm}

\includegraphics[width=9.0cm,angle=0]{ms2022-0304fig10-1993.pdf}
\includegraphics[width=9.0cm,angle=0]{ms2022-0304fig10-1994.pdf}
\vspace{-0.5cm}

\includegraphics[width=9.0cm,angle=0]{ms2022-0304fig10-1995.pdf}
\includegraphics[width=9.0cm,angle=0]{ms2022-0304fig10-1996.pdf}
\vspace{-0.5cm}

\includegraphics[width=9.0cm,angle=0]{ms2022-0304fig10-1997.pdf}
\includegraphics[width=9.0cm,angle=0]{ms2022-0304fig10-1998.pdf}
\vspace{-0.5cm}

\includegraphics[width=9.0cm,angle=0]{ms2022-0304fig10-1999.pdf}
\includegraphics[width=9.0cm,angle=0]{ms2022-0304fig10-2000.pdf}
\end{figure}
\clearpage

\begin{figure}
\includegraphics[width=9.0cm,angle=0]{ms2022-0304fig10-2001.pdf}
\includegraphics[width=9.0cm,angle=0]{ms2022-0304fig10-2002.pdf}
\vspace{-0.5cm}

\includegraphics[width=9.0cm,angle=0]{ms2022-0304fig10-2003.pdf}
\includegraphics[width=9.0cm,angle=0]{ms2022-0304fig10-2004.pdf}
\vspace{-0.5cm}

\includegraphics[width=9.0cm,angle=0]{ms2022-0304fig10-2005.pdf}
\includegraphics[width=9.0cm,angle=0]{ms2022-0304fig10-2006.pdf}
\vspace{-0.5cm}

\includegraphics[width=9.0cm,angle=0]{ms2022-0304fig10-2007.pdf}
\includegraphics[width=9.0cm,angle=0]{ms2022-0304fig10-2008.pdf}
\vspace{-0.5cm}

\includegraphics[width=9.0cm,angle=0]{ms2022-0304fig10-2009.pdf}
\includegraphics[width=9.0cm,angle=0]{ms2022-0304fig10-2010.pdf}
\end{figure}
\clearpage

\begin{figure}
\includegraphics[width=9.0cm,angle=0]{ms2022-0304fig10-2011.pdf}
\includegraphics[width=9.0cm,angle=0]{ms2022-0304fig10-2012.pdf}
\vspace{-0.5cm}

\includegraphics[width=9.0cm,angle=0]{ms2022-0304fig10-2013.pdf}
\includegraphics[width=9.0cm,angle=0]{ms2022-0304fig10-2014.pdf}
\vspace{-0.5cm}

\includegraphics[width=9.0cm,angle=0]{ms2022-0304fig10-2015.pdf}
\includegraphics[width=9.0cm,angle=0]{ms2022-0304fig10-2016.pdf}
\vspace{-0.5cm}

\includegraphics[width=9.0cm,angle=0]{ms2022-0304fig10-2017.pdf}
\includegraphics[width=9.0cm,angle=0]{ms2022-0304fig10-2018.pdf}
\vspace{-0.5cm}

\includegraphics[width=9.0cm,angle=0]{ms2022-0304fig10-2019.pdf}
\includegraphics[width=9.0cm,angle=0]{ms2022-0304fig10-2020.pdf}
\end{figure}
\clearpage

\begin{figure}
\includegraphics[width=9.0cm,angle=0]{ms2022-0304fig10-2021.pdf}
\includegraphics[width=9.0cm,angle=0]{ms2022-0304fig10-2022.pdf}
\vspace{-0.5cm}

\includegraphics[width=9.0cm,angle=0]{ms2022-0304fig10-2023.pdf}
\includegraphics[width=9.0cm,angle=0]{ms2022-0304fig10-2024.pdf}
\vspace{-0.5cm}

\includegraphics[width=9.0cm,angle=0]{ms2022-0304fig10-2025.pdf}
\includegraphics[width=9.0cm,angle=0]{ms2022-0304fig10-2026.pdf}
\vspace{-0.5cm}

\includegraphics[width=9.0cm,angle=0]{ms2022-0304fig10-2027.pdf}
\includegraphics[width=9.0cm,angle=0]{ms2022-0304fig10-2028.pdf}
\vspace{-0.5cm}

\includegraphics[width=9.0cm,angle=0]{ms2022-0304fig10-2029.pdf}
\includegraphics[width=9.0cm,angle=0]{ms2022-0304fig10-2030.pdf}
\end{figure}
\clearpage

\begin{figure}
\includegraphics[width=9.0cm,angle=0]{ms2022-0304fig10-2031.pdf}
\includegraphics[width=9.0cm,angle=0]{ms2022-0304fig10-2032.pdf}
\vspace{-0.5cm}

\includegraphics[width=9.0cm,angle=0]{ms2022-0304fig10-2033.pdf}
\includegraphics[width=9.0cm,angle=0]{ms2022-0304fig10-2034.pdf}
\vspace{-0.5cm}

\includegraphics[width=9.0cm,angle=0]{ms2022-0304fig10-2035.pdf}
\includegraphics[width=9.0cm,angle=0]{ms2022-0304fig10-2036.pdf}
\vspace{-0.5cm}

\includegraphics[width=9.0cm,angle=0]{ms2022-0304fig10-2037.pdf}
\includegraphics[width=9.0cm,angle=0]{ms2022-0304fig10-2038.pdf}
\vspace{-0.5cm}

\includegraphics[width=9.0cm,angle=0]{ms2022-0304fig10-2039.pdf}
\includegraphics[width=9.0cm,angle=0]{ms2022-0304fig10-2040.pdf}
\end{figure}
\clearpage

\begin{figure}
\includegraphics[width=9.0cm,angle=0]{ms2022-0304fig10-2041.pdf}
\includegraphics[width=9.0cm,angle=0]{ms2022-0304fig10-2042.pdf}
\vspace{-0.5cm}

\includegraphics[width=9.0cm,angle=0]{ms2022-0304fig10-2043.pdf}
\includegraphics[width=9.0cm,angle=0]{ms2022-0304fig10-2044.pdf}
\vspace{-0.5cm}

\includegraphics[width=9.0cm,angle=0]{ms2022-0304fig10-2045.pdf}
\includegraphics[width=9.0cm,angle=0]{ms2022-0304fig10-2046.pdf}
\vspace{-0.5cm}

\includegraphics[width=9.0cm,angle=0]{ms2022-0304fig10-2047.pdf}
\includegraphics[width=9.0cm,angle=0]{ms2022-0304fig10-2048.pdf}
\vspace{-0.5cm}

\includegraphics[width=9.0cm,angle=0]{ms2022-0304fig10-2049.pdf}
\includegraphics[width=9.0cm,angle=0]{ms2022-0304fig10-2050.pdf}
\end{figure}
\clearpage

\begin{figure}
\includegraphics[width=9.0cm,angle=0]{ms2022-0304fig10-2051.pdf}
\includegraphics[width=9.0cm,angle=0]{ms2022-0304fig10-2052.pdf}
\vspace{-0.5cm}

\includegraphics[width=9.0cm,angle=0]{ms2022-0304fig10-2053.pdf}
\includegraphics[width=9.0cm,angle=0]{ms2022-0304fig10-2054.pdf}
\vspace{-0.5cm}

\includegraphics[width=9.0cm,angle=0]{ms2022-0304fig10-2055.pdf}
\includegraphics[width=9.0cm,angle=0]{ms2022-0304fig10-2056.pdf}
\vspace{-0.5cm}

\includegraphics[width=9.0cm,angle=0]{ms2022-0304fig10-2057.pdf}
\includegraphics[width=9.0cm,angle=0]{ms2022-0304fig10-2058.pdf}
\vspace{-0.5cm}

\includegraphics[width=9.0cm,angle=0]{ms2022-0304fig10-2059.pdf}
\includegraphics[width=9.0cm,angle=0]{ms2022-0304fig10-2060.pdf}
\end{figure}
\clearpage

\begin{figure}
\includegraphics[width=9.0cm,angle=0]{ms2022-0304fig10-2061.pdf}
\includegraphics[width=9.0cm,angle=0]{ms2022-0304fig10-2062.pdf}
\vspace{-0.5cm}

\includegraphics[width=9.0cm,angle=0]{ms2022-0304fig10-2063.pdf}
\includegraphics[width=9.0cm,angle=0]{ms2022-0304fig10-2064.pdf}
\vspace{-0.5cm}

\includegraphics[width=9.0cm,angle=0]{ms2022-0304fig10-2065.pdf}
\includegraphics[width=9.0cm,angle=0]{ms2022-0304fig10-2066.pdf}
\vspace{-0.5cm}

\includegraphics[width=9.0cm,angle=0]{ms2022-0304fig10-2067.pdf}
\includegraphics[width=9.0cm,angle=0]{ms2022-0304fig10-2068.pdf}
\vspace{-0.5cm}

\includegraphics[width=9.0cm,angle=0]{ms2022-0304fig10-2069.pdf}
\includegraphics[width=9.0cm,angle=0]{ms2022-0304fig10-2070.pdf}
\end{figure}
\clearpage

\begin{figure}
\includegraphics[width=9.0cm,angle=0]{ms2022-0304fig10-2071.pdf}
\includegraphics[width=9.0cm,angle=0]{ms2022-0304fig10-2072.pdf}
\vspace{-0.5cm}

\includegraphics[width=9.0cm,angle=0]{ms2022-0304fig10-2073.pdf}
\includegraphics[width=9.0cm,angle=0]{ms2022-0304fig10-2074.pdf}
\vspace{-0.5cm}

\includegraphics[width=9.0cm,angle=0]{ms2022-0304fig10-2075.pdf}
\includegraphics[width=9.0cm,angle=0]{ms2022-0304fig10-2076.pdf}
\vspace{-0.5cm}

\includegraphics[width=9.0cm,angle=0]{ms2022-0304fig10-2077.pdf}
\includegraphics[width=9.0cm,angle=0]{ms2022-0304fig10-2078.pdf}
\vspace{-0.5cm}

\includegraphics[width=9.0cm,angle=0]{ms2022-0304fig10-2079.pdf}
\includegraphics[width=9.0cm,angle=0]{ms2022-0304fig10-2080.pdf}
\end{figure}
\clearpage

\begin{figure}
\includegraphics[width=9.0cm,angle=0]{ms2022-0304fig10-2081.pdf}
\includegraphics[width=9.0cm,angle=0]{ms2022-0304fig10-2082.pdf}
\vspace{-0.5cm}

\includegraphics[width=9.0cm,angle=0]{ms2022-0304fig10-2083.pdf}
\includegraphics[width=9.0cm,angle=0]{ms2022-0304fig10-2084.pdf}
\vspace{-0.5cm}

\includegraphics[width=9.0cm,angle=0]{ms2022-0304fig10-2085.pdf}
\includegraphics[width=9.0cm,angle=0]{ms2022-0304fig10-2086.pdf}
\vspace{-0.5cm}

\includegraphics[width=9.0cm,angle=0]{ms2022-0304fig10-2087.pdf}
\includegraphics[width=9.0cm,angle=0]{ms2022-0304fig10-2088.pdf}
\vspace{-0.5cm}

\includegraphics[width=9.0cm,angle=0]{ms2022-0304fig10-2089.pdf}
\includegraphics[width=9.0cm,angle=0]{ms2022-0304fig10-2090.pdf}
\end{figure}
\clearpage

\begin{figure}
\includegraphics[width=9.0cm,angle=0]{ms2022-0304fig10-2091.pdf}
\includegraphics[width=9.0cm,angle=0]{ms2022-0304fig10-2092.pdf}
\vspace{-0.5cm}

\includegraphics[width=9.0cm,angle=0]{ms2022-0304fig10-2093.pdf}
\includegraphics[width=9.0cm,angle=0]{ms2022-0304fig10-2094.pdf}
\vspace{-0.5cm}

\includegraphics[width=9.0cm,angle=0]{ms2022-0304fig10-2095.pdf}
\includegraphics[width=9.0cm,angle=0]{ms2022-0304fig10-2096.pdf}
\vspace{-0.5cm}

\includegraphics[width=9.0cm,angle=0]{ms2022-0304fig10-2097.pdf}
\includegraphics[width=9.0cm,angle=0]{ms2022-0304fig10-2098.pdf}
\vspace{-0.5cm}

\includegraphics[width=9.0cm,angle=0]{ms2022-0304fig10-2099.pdf}
\includegraphics[width=9.0cm,angle=0]{ms2022-0304fig10-2100.pdf}
\end{figure}
\clearpage

\begin{figure}
\includegraphics[width=9.0cm,angle=0]{ms2022-0304fig10-2101.pdf}
\includegraphics[width=9.0cm,angle=0]{ms2022-0304fig10-2102.pdf}
\vspace{-0.5cm}

\includegraphics[width=9.0cm,angle=0]{ms2022-0304fig10-2103.pdf}
\includegraphics[width=9.0cm,angle=0]{ms2022-0304fig10-2104.pdf}
\vspace{-0.5cm}

\includegraphics[width=9.0cm,angle=0]{ms2022-0304fig10-2105.pdf}
\includegraphics[width=9.0cm,angle=0]{ms2022-0304fig10-2106.pdf}
\vspace{-0.5cm}

\includegraphics[width=9.0cm,angle=0]{ms2022-0304fig10-2107.pdf}
\includegraphics[width=9.0cm,angle=0]{ms2022-0304fig10-2108.pdf}
\vspace{-0.5cm}

\includegraphics[width=9.0cm,angle=0]{ms2022-0304fig10-2109.pdf}
\includegraphics[width=9.0cm,angle=0]{ms2022-0304fig10-2110.pdf}
\end{figure}
\clearpage

\begin{figure}
\includegraphics[width=9.0cm,angle=0]{ms2022-0304fig10-2111.pdf}
\includegraphics[width=9.0cm,angle=0]{ms2022-0304fig10-2112.pdf}
\vspace{-0.5cm}

\includegraphics[width=9.0cm,angle=0]{ms2022-0304fig10-2113.pdf}
\includegraphics[width=9.0cm,angle=0]{ms2022-0304fig10-2114.pdf}
\vspace{-0.5cm}

\includegraphics[width=9.0cm,angle=0]{ms2022-0304fig10-2115.pdf}
\includegraphics[width=9.0cm,angle=0]{ms2022-0304fig10-2116.pdf}
\vspace{-0.5cm}

\includegraphics[width=9.0cm,angle=0]{ms2022-0304fig10-2117.pdf}
\includegraphics[width=9.0cm,angle=0]{ms2022-0304fig10-2118.pdf}
\vspace{-0.5cm}

\includegraphics[width=9.0cm,angle=0]{ms2022-0304fig10-2119.pdf}
\includegraphics[width=9.0cm,angle=0]{ms2022-0304fig10-2120.pdf}
\end{figure}
\clearpage

\begin{figure}
\includegraphics[width=9.0cm,angle=0]{ms2022-0304fig10-2121.pdf}
\includegraphics[width=9.0cm,angle=0]{ms2022-0304fig10-2122.pdf}
\vspace{-0.5cm}

\includegraphics[width=9.0cm,angle=0]{ms2022-0304fig10-2123.pdf}
\includegraphics[width=9.0cm,angle=0]{ms2022-0304fig10-2124.pdf}
\vspace{-0.5cm}

\includegraphics[width=9.0cm,angle=0]{ms2022-0304fig10-2125.pdf}
\includegraphics[width=9.0cm,angle=0]{ms2022-0304fig10-2126.pdf}
\vspace{-0.5cm}

\includegraphics[width=9.0cm,angle=0]{ms2022-0304fig10-2127.pdf}
\includegraphics[width=9.0cm,angle=0]{ms2022-0304fig10-2128.pdf}
\vspace{-0.5cm}

\includegraphics[width=9.0cm,angle=0]{ms2022-0304fig10-2129.pdf}
\includegraphics[width=9.0cm,angle=0]{ms2022-0304fig10-2130.pdf}
\end{figure}
\clearpage

\begin{figure}
\includegraphics[width=9.0cm,angle=0]{ms2022-0304fig10-2131.pdf}
\includegraphics[width=9.0cm,angle=0]{ms2022-0304fig10-2132.pdf}
\vspace{-0.5cm}

\includegraphics[width=9.0cm,angle=0]{ms2022-0304fig10-2133.pdf}
\includegraphics[width=9.0cm,angle=0]{ms2022-0304fig10-2134.pdf}
\vspace{-0.5cm}

\includegraphics[width=9.0cm,angle=0]{ms2022-0304fig10-2135.pdf}
\includegraphics[width=9.0cm,angle=0]{ms2022-0304fig10-2136.pdf}
\vspace{-0.5cm}

\includegraphics[width=9.0cm,angle=0]{ms2022-0304fig10-2137.pdf}
\includegraphics[width=9.0cm,angle=0]{ms2022-0304fig10-2138.pdf}
\vspace{-0.5cm}

\includegraphics[width=9.0cm,angle=0]{ms2022-0304fig10-2139.pdf}
\includegraphics[width=9.0cm,angle=0]{ms2022-0304fig10-2140.pdf}
\end{figure}
\clearpage

\begin{figure}
\includegraphics[width=9.0cm,angle=0]{ms2022-0304fig10-2141.pdf}
\includegraphics[width=9.0cm,angle=0]{ms2022-0304fig10-2142.pdf}
\vspace{-0.5cm}

\includegraphics[width=9.0cm,angle=0]{ms2022-0304fig10-2143.pdf}
\includegraphics[width=9.0cm,angle=0]{ms2022-0304fig10-2144.pdf}
\vspace{-0.5cm}

\includegraphics[width=9.0cm,angle=0]{ms2022-0304fig10-2145.pdf}
\includegraphics[width=9.0cm,angle=0]{ms2022-0304fig10-2146.pdf}
\vspace{-0.5cm}

\includegraphics[width=9.0cm,angle=0]{ms2022-0304fig10-2147.pdf}
\includegraphics[width=9.0cm,angle=0]{ms2022-0304fig10-2148.pdf}
\vspace{-0.5cm}

\includegraphics[width=9.0cm,angle=0]{ms2022-0304fig10-2149.pdf}
\includegraphics[width=9.0cm,angle=0]{ms2022-0304fig10-2150.pdf}
\end{figure}
\clearpage

\begin{figure}
\includegraphics[width=9.0cm,angle=0]{ms2022-0304fig10-2151.pdf}
\includegraphics[width=9.0cm,angle=0]{ms2022-0304fig10-2152.pdf}
\vspace{-0.5cm}

\includegraphics[width=9.0cm,angle=0]{ms2022-0304fig10-2153.pdf}
\includegraphics[width=9.0cm,angle=0]{ms2022-0304fig10-2154.pdf}
\vspace{-0.5cm}

\includegraphics[width=9.0cm,angle=0]{ms2022-0304fig10-2155.pdf}
\includegraphics[width=9.0cm,angle=0]{ms2022-0304fig10-2156.pdf}
\vspace{-0.5cm}

\includegraphics[width=9.0cm,angle=0]{ms2022-0304fig10-2157.pdf}
\includegraphics[width=9.0cm,angle=0]{ms2022-0304fig10-2158.pdf}
\vspace{-0.5cm}

\includegraphics[width=9.0cm,angle=0]{ms2022-0304fig10-2159.pdf}
\includegraphics[width=9.0cm,angle=0]{ms2022-0304fig10-2160.pdf}
\end{figure}
\clearpage

\begin{figure}
\includegraphics[width=9.0cm,angle=0]{ms2022-0304fig10-2161.pdf}
\includegraphics[width=9.0cm,angle=0]{ms2022-0304fig10-2162.pdf}
\vspace{-0.5cm}

\includegraphics[width=9.0cm,angle=0]{ms2022-0304fig10-2163.pdf}
\includegraphics[width=9.0cm,angle=0]{ms2022-0304fig10-2164.pdf}
\vspace{-0.5cm}

\includegraphics[width=9.0cm,angle=0]{ms2022-0304fig10-2165.pdf}
\includegraphics[width=9.0cm,angle=0]{ms2022-0304fig10-2166.pdf}
\vspace{-0.5cm}

\includegraphics[width=9.0cm,angle=0]{ms2022-0304fig10-2167.pdf}
\includegraphics[width=9.0cm,angle=0]{ms2022-0304fig10-2168.pdf}
\vspace{-0.5cm}

\includegraphics[width=9.0cm,angle=0]{ms2022-0304fig10-2169.pdf}
\includegraphics[width=9.0cm,angle=0]{ms2022-0304fig10-2170.pdf}
\end{figure}
\clearpage

\begin{figure}
\includegraphics[width=9.0cm,angle=0]{ms2022-0304fig10-2171.pdf}
\includegraphics[width=9.0cm,angle=0]{ms2022-0304fig10-2172.pdf}
\vspace{-0.5cm}

\includegraphics[width=9.0cm,angle=0]{ms2022-0304fig10-2173.pdf}
\includegraphics[width=9.0cm,angle=0]{ms2022-0304fig10-2174.pdf}
\vspace{-0.5cm}

\includegraphics[width=9.0cm,angle=0]{ms2022-0304fig10-2175.pdf}
\includegraphics[width=9.0cm,angle=0]{ms2022-0304fig10-2176.pdf}
\vspace{-0.5cm}

\includegraphics[width=9.0cm,angle=0]{ms2022-0304fig10-2177.pdf}
\includegraphics[width=9.0cm,angle=0]{ms2022-0304fig10-2178.pdf}
\vspace{-0.5cm}

\includegraphics[width=9.0cm,angle=0]{ms2022-0304fig10-2179.pdf}
\includegraphics[width=9.0cm,angle=0]{ms2022-0304fig10-2180.pdf}
\end{figure}
\clearpage

\begin{figure}
\includegraphics[width=9.0cm,angle=0]{ms2022-0304fig10-2181.pdf}
\includegraphics[width=9.0cm,angle=0]{ms2022-0304fig10-2182.pdf}
\vspace{-0.5cm}

\includegraphics[width=9.0cm,angle=0]{ms2022-0304fig10-2183.pdf}
\includegraphics[width=9.0cm,angle=0]{ms2022-0304fig10-2184.pdf}
\vspace{-0.5cm}

\includegraphics[width=9.0cm,angle=0]{ms2022-0304fig10-2185.pdf}
\includegraphics[width=9.0cm,angle=0]{ms2022-0304fig10-2186.pdf}
\vspace{-0.5cm}

\includegraphics[width=9.0cm,angle=0]{ms2022-0304fig10-2187.pdf}
\includegraphics[width=9.0cm,angle=0]{ms2022-0304fig10-2188.pdf}
\vspace{-0.5cm}

\includegraphics[width=9.0cm,angle=0]{ms2022-0304fig10-2189.pdf}
\includegraphics[width=9.0cm,angle=0]{ms2022-0304fig10-2190.pdf}
\end{figure}
\clearpage

\begin{figure}
\includegraphics[width=9.0cm,angle=0]{ms2022-0304fig10-2191.pdf}
\includegraphics[width=9.0cm,angle=0]{ms2022-0304fig10-2192.pdf}
\vspace{-0.5cm}

\includegraphics[width=9.0cm,angle=0]{ms2022-0304fig10-2193.pdf}
\includegraphics[width=9.0cm,angle=0]{ms2022-0304fig10-2194.pdf}
\vspace{-0.5cm}

\includegraphics[width=9.0cm,angle=0]{ms2022-0304fig10-2195.pdf}
\includegraphics[width=9.0cm,angle=0]{ms2022-0304fig10-2196.pdf}
\vspace{-0.5cm}

\includegraphics[width=9.0cm,angle=0]{ms2022-0304fig10-2197.pdf}
\includegraphics[width=9.0cm,angle=0]{ms2022-0304fig10-2198.pdf}
\vspace{-0.5cm}

\includegraphics[width=9.0cm,angle=0]{ms2022-0304fig10-2199.pdf}
\includegraphics[width=9.0cm,angle=0]{ms2022-0304fig10-2200.pdf}
\end{figure}
\clearpage

\begin{figure}
\includegraphics[width=9.0cm,angle=0]{ms2022-0304fig10-2201.pdf}
\includegraphics[width=9.0cm,angle=0]{ms2022-0304fig10-2202.pdf}
\vspace{-0.5cm}

\includegraphics[width=9.0cm,angle=0]{ms2022-0304fig10-2203.pdf}
\includegraphics[width=9.0cm,angle=0]{ms2022-0304fig10-2204.pdf}
\vspace{-0.5cm}

\includegraphics[width=9.0cm,angle=0]{ms2022-0304fig10-2205.pdf}
\includegraphics[width=9.0cm,angle=0]{ms2022-0304fig10-2206.pdf}
\vspace{-0.5cm}

\includegraphics[width=9.0cm,angle=0]{ms2022-0304fig10-2207.pdf}
\includegraphics[width=9.0cm,angle=0]{ms2022-0304fig10-2208.pdf}
\vspace{-0.5cm}

\includegraphics[width=9.0cm,angle=0]{ms2022-0304fig10-2209.pdf}
\includegraphics[width=9.0cm,angle=0]{ms2022-0304fig10-2210.pdf}
\end{figure}
\clearpage

\begin{figure}
\includegraphics[width=9.0cm,angle=0]{ms2022-0304fig10-2211.pdf}
\includegraphics[width=9.0cm,angle=0]{ms2022-0304fig10-2212.pdf}
\vspace{-0.5cm}

\includegraphics[width=9.0cm,angle=0]{ms2022-0304fig10-2213.pdf}
\includegraphics[width=9.0cm,angle=0]{ms2022-0304fig10-2214.pdf}
\vspace{-0.5cm}

\includegraphics[width=9.0cm,angle=0]{ms2022-0304fig10-2215.pdf}
\includegraphics[width=9.0cm,angle=0]{ms2022-0304fig10-2216.pdf}
\vspace{-0.5cm}

\includegraphics[width=9.0cm,angle=0]{ms2022-0304fig10-2217.pdf}
\includegraphics[width=9.0cm,angle=0]{ms2022-0304fig10-2218.pdf}
\vspace{-0.5cm}

\includegraphics[width=9.0cm,angle=0]{ms2022-0304fig10-2219.pdf}
\includegraphics[width=9.0cm,angle=0]{ms2022-0304fig10-2220.pdf}
\end{figure}
\clearpage

\begin{figure}
\includegraphics[width=9.0cm,angle=0]{ms2022-0304fig10-2221.pdf}
\includegraphics[width=9.0cm,angle=0]{ms2022-0304fig10-2222.pdf}
\vspace{-0.5cm}

\includegraphics[width=9.0cm,angle=0]{ms2022-0304fig10-2223.pdf}
\includegraphics[width=9.0cm,angle=0]{ms2022-0304fig10-2224.pdf}
\vspace{-0.5cm}

\includegraphics[width=9.0cm,angle=0]{ms2022-0304fig10-2225.pdf}
\includegraphics[width=9.0cm,angle=0]{ms2022-0304fig10-2226.pdf}
\vspace{-0.5cm}

\includegraphics[width=9.0cm,angle=0]{ms2022-0304fig10-2227.pdf}
\includegraphics[width=9.0cm,angle=0]{ms2022-0304fig10-2228.pdf}
\vspace{-0.5cm}

\includegraphics[width=9.0cm,angle=0]{ms2022-0304fig10-2229.pdf}
\includegraphics[width=9.0cm,angle=0]{ms2022-0304fig10-2230.pdf}
\end{figure}
\clearpage

\begin{figure}
\includegraphics[width=9.0cm,angle=0]{ms2022-0304fig10-2231.pdf}
\includegraphics[width=9.0cm,angle=0]{ms2022-0304fig10-2232.pdf}
\vspace{-0.5cm}

\includegraphics[width=9.0cm,angle=0]{ms2022-0304fig10-2233.pdf}
\includegraphics[width=9.0cm,angle=0]{ms2022-0304fig10-2234.pdf}
\vspace{-0.5cm}

\includegraphics[width=9.0cm,angle=0]{ms2022-0304fig10-2235.pdf}
\includegraphics[width=9.0cm,angle=0]{ms2022-0304fig10-2236.pdf}
\vspace{-0.5cm}

\includegraphics[width=9.0cm,angle=0]{ms2022-0304fig10-2237.pdf}
\includegraphics[width=9.0cm,angle=0]{ms2022-0304fig10-2238.pdf}
\vspace{-0.5cm}

\includegraphics[width=9.0cm,angle=0]{ms2022-0304fig10-2239.pdf}
\includegraphics[width=9.0cm,angle=0]{ms2022-0304fig10-2240.pdf}
\end{figure}
\clearpage

\begin{figure}
\includegraphics[width=9.0cm,angle=0]{ms2022-0304fig10-2241.pdf}
\includegraphics[width=9.0cm,angle=0]{ms2022-0304fig10-2242.pdf}
\vspace{-0.5cm}

\includegraphics[width=9.0cm,angle=0]{ms2022-0304fig10-2243.pdf}
\includegraphics[width=9.0cm,angle=0]{ms2022-0304fig10-2244.pdf}
\vspace{-0.5cm}

\includegraphics[width=9.0cm,angle=0]{ms2022-0304fig10-2245.pdf}
\includegraphics[width=9.0cm,angle=0]{ms2022-0304fig10-2246.pdf}
\vspace{-0.5cm}

\includegraphics[width=9.0cm,angle=0]{ms2022-0304fig10-2247.pdf}
\includegraphics[width=9.0cm,angle=0]{ms2022-0304fig10-2248.pdf}
\vspace{-0.5cm}

\includegraphics[width=9.0cm,angle=0]{ms2022-0304fig10-2249.pdf}
\includegraphics[width=9.0cm,angle=0]{ms2022-0304fig10-2250.pdf}
\end{figure}
\clearpage

\begin{figure}
\includegraphics[width=9.0cm,angle=0]{ms2022-0304fig10-2251.pdf}
\includegraphics[width=9.0cm,angle=0]{ms2022-0304fig10-2252.pdf}
\vspace{-0.5cm}

\includegraphics[width=9.0cm,angle=0]{ms2022-0304fig10-2253.pdf}
\includegraphics[width=9.0cm,angle=0]{ms2022-0304fig10-2254.pdf}
\vspace{-0.5cm}

\includegraphics[width=9.0cm,angle=0]{ms2022-0304fig10-2255.pdf}
\includegraphics[width=9.0cm,angle=0]{ms2022-0304fig10-2256.pdf}
\vspace{-0.5cm}

\includegraphics[width=9.0cm,angle=0]{ms2022-0304fig10-2257.pdf}
\includegraphics[width=9.0cm,angle=0]{ms2022-0304fig10-2258.pdf}
\vspace{-0.5cm}

\includegraphics[width=9.0cm,angle=0]{ms2022-0304fig10-2259.pdf}
\includegraphics[width=9.0cm,angle=0]{ms2022-0304fig10-2260.pdf}
\end{figure}
\clearpage

\begin{figure}
\includegraphics[width=9.0cm,angle=0]{ms2022-0304fig10-2261.pdf}
\includegraphics[width=9.0cm,angle=0]{ms2022-0304fig10-2262.pdf}
\vspace{-0.5cm}

\includegraphics[width=9.0cm,angle=0]{ms2022-0304fig10-2263.pdf}
\includegraphics[width=9.0cm,angle=0]{ms2022-0304fig10-2264.pdf}
\vspace{-0.5cm}

\includegraphics[width=9.0cm,angle=0]{ms2022-0304fig10-2265.pdf}
\includegraphics[width=9.0cm,angle=0]{ms2022-0304fig10-2266.pdf}
\vspace{-0.5cm}

\includegraphics[width=9.0cm,angle=0]{ms2022-0304fig10-2267.pdf}
\includegraphics[width=9.0cm,angle=0]{ms2022-0304fig10-2268.pdf}
\vspace{-0.5cm}

\includegraphics[width=9.0cm,angle=0]{ms2022-0304fig10-2269.pdf}
\includegraphics[width=9.0cm,angle=0]{ms2022-0304fig10-2270.pdf}
\end{figure}
\clearpage

\begin{figure}
\includegraphics[width=9.0cm,angle=0]{ms2022-0304fig10-2271.pdf}
\includegraphics[width=9.0cm,angle=0]{ms2022-0304fig10-2272.pdf}
\vspace{-0.5cm}

\includegraphics[width=9.0cm,angle=0]{ms2022-0304fig10-2273.pdf}
\includegraphics[width=9.0cm,angle=0]{ms2022-0304fig10-2274.pdf}
\vspace{-0.5cm}

\includegraphics[width=9.0cm,angle=0]{ms2022-0304fig10-2275.pdf}
\includegraphics[width=9.0cm,angle=0]{ms2022-0304fig10-2276.pdf}
\vspace{-0.5cm}

\includegraphics[width=9.0cm,angle=0]{ms2022-0304fig10-2277.pdf}
\includegraphics[width=9.0cm,angle=0]{ms2022-0304fig10-2278.pdf}
\vspace{-0.5cm}

\includegraphics[width=9.0cm,angle=0]{ms2022-0304fig10-2279.pdf}
\includegraphics[width=9.0cm,angle=0]{ms2022-0304fig10-2280.pdf}
\end{figure}
\clearpage

\begin{figure}
\includegraphics[width=9.0cm,angle=0]{ms2022-0304fig10-2281.pdf}
\includegraphics[width=9.0cm,angle=0]{ms2022-0304fig10-2282.pdf}
\vspace{-0.5cm}

\includegraphics[width=9.0cm,angle=0]{ms2022-0304fig10-2283.pdf}
\includegraphics[width=9.0cm,angle=0]{ms2022-0304fig10-2284.pdf}
\vspace{-0.5cm}

\includegraphics[width=9.0cm,angle=0]{ms2022-0304fig10-2285.pdf}
\includegraphics[width=9.0cm,angle=0]{ms2022-0304fig10-2286.pdf}
\vspace{-0.5cm}

\includegraphics[width=9.0cm,angle=0]{ms2022-0304fig10-2287.pdf}
\includegraphics[width=9.0cm,angle=0]{ms2022-0304fig10-2288.pdf}
\vspace{-0.5cm}

\includegraphics[width=9.0cm,angle=0]{ms2022-0304fig10-2289.pdf}
\includegraphics[width=9.0cm,angle=0]{ms2022-0304fig10-2290.pdf}
\end{figure}
\clearpage

\begin{figure}
\includegraphics[width=9.0cm,angle=0]{ms2022-0304fig10-2291.pdf}
\includegraphics[width=9.0cm,angle=0]{ms2022-0304fig10-2292.pdf}
\vspace{-0.5cm}

\includegraphics[width=9.0cm,angle=0]{ms2022-0304fig10-2293.pdf}
\includegraphics[width=9.0cm,angle=0]{ms2022-0304fig10-2294.pdf}
\vspace{-0.5cm}

\includegraphics[width=9.0cm,angle=0]{ms2022-0304fig10-2295.pdf}
\includegraphics[width=9.0cm,angle=0]{ms2022-0304fig10-2296.pdf}
\vspace{-0.5cm}

\includegraphics[width=9.0cm,angle=0]{ms2022-0304fig10-2297.pdf}
\includegraphics[width=9.0cm,angle=0]{ms2022-0304fig10-2298.pdf}
\vspace{-0.5cm}

\includegraphics[width=9.0cm,angle=0]{ms2022-0304fig10-2299.pdf}
\includegraphics[width=9.0cm,angle=0]{ms2022-0304fig10-2300.pdf}
\end{figure}
\clearpage

\begin{figure}
\includegraphics[width=9.0cm,angle=0]{ms2022-0304fig10-2301.pdf}
\includegraphics[width=9.0cm,angle=0]{ms2022-0304fig10-2302.pdf}
\vspace{-0.5cm}

\includegraphics[width=9.0cm,angle=0]{ms2022-0304fig10-2303.pdf}
\includegraphics[width=9.0cm,angle=0]{ms2022-0304fig10-2304.pdf}
\vspace{-0.5cm}

\includegraphics[width=9.0cm,angle=0]{ms2022-0304fig10-2305.pdf}
\includegraphics[width=9.0cm,angle=0]{ms2022-0304fig10-2306.pdf}
\vspace{-0.5cm}

\includegraphics[width=9.0cm,angle=0]{ms2022-0304fig10-2307.pdf}
\includegraphics[width=9.0cm,angle=0]{ms2022-0304fig10-2308.pdf}
\vspace{-0.5cm}

\includegraphics[width=9.0cm,angle=0]{ms2022-0304fig10-2309.pdf}
\includegraphics[width=9.0cm,angle=0]{ms2022-0304fig10-2310.pdf}
\end{figure}
\clearpage

\begin{figure}
\includegraphics[width=9.0cm,angle=0]{ms2022-0304fig10-2311.pdf}
\includegraphics[width=9.0cm,angle=0]{ms2022-0304fig10-2312.pdf}
\vspace{-0.5cm}

\includegraphics[width=9.0cm,angle=0]{ms2022-0304fig10-2313.pdf}
\includegraphics[width=9.0cm,angle=0]{ms2022-0304fig10-2314.pdf}
\vspace{-0.5cm}

\includegraphics[width=9.0cm,angle=0]{ms2022-0304fig10-2315.pdf}
\includegraphics[width=9.0cm,angle=0]{ms2022-0304fig10-2316.pdf}
\vspace{-0.5cm}

\includegraphics[width=9.0cm,angle=0]{ms2022-0304fig10-2317.pdf}
\includegraphics[width=9.0cm,angle=0]{ms2022-0304fig10-2318.pdf}
\vspace{-0.5cm}

\includegraphics[width=9.0cm,angle=0]{ms2022-0304fig10-2319.pdf}
\includegraphics[width=9.0cm,angle=0]{ms2022-0304fig10-2320.pdf}
\end{figure}
\clearpage

\begin{figure}
\includegraphics[width=9.0cm,angle=0]{ms2022-0304fig10-2321.pdf}
\includegraphics[width=9.0cm,angle=0]{ms2022-0304fig10-2322.pdf}
\vspace{-0.5cm}

\includegraphics[width=9.0cm,angle=0]{ms2022-0304fig10-2323.pdf}
\includegraphics[width=9.0cm,angle=0]{ms2022-0304fig10-2324.pdf}
\vspace{-0.5cm}

\includegraphics[width=9.0cm,angle=0]{ms2022-0304fig10-2325.pdf}
\includegraphics[width=9.0cm,angle=0]{ms2022-0304fig10-2326.pdf}
\vspace{-0.5cm}

\includegraphics[width=9.0cm,angle=0]{ms2022-0304fig10-2327.pdf}
\includegraphics[width=9.0cm,angle=0]{ms2022-0304fig10-2328.pdf}
\vspace{-0.5cm}

\includegraphics[width=9.0cm,angle=0]{ms2022-0304fig10-2329.pdf}
\includegraphics[width=9.0cm,angle=0]{ms2022-0304fig10-2330.pdf}
\end{figure}
\clearpage

\begin{figure}
\includegraphics[width=9.0cm,angle=0]{ms2022-0304fig10-2331.pdf}
\includegraphics[width=9.0cm,angle=0]{ms2022-0304fig10-2332.pdf}
\vspace{-0.5cm}

\includegraphics[width=9.0cm,angle=0]{ms2022-0304fig10-2333.pdf}
\includegraphics[width=9.0cm,angle=0]{ms2022-0304fig10-2334.pdf}
\vspace{-0.5cm}

\includegraphics[width=9.0cm,angle=0]{ms2022-0304fig10-2335.pdf}
\includegraphics[width=9.0cm,angle=0]{ms2022-0304fig10-2336.pdf}
\vspace{-0.5cm}

\includegraphics[width=9.0cm,angle=0]{ms2022-0304fig10-2337.pdf}
\includegraphics[width=9.0cm,angle=0]{ms2022-0304fig10-2338.pdf}
\vspace{-0.5cm}

\includegraphics[width=9.0cm,angle=0]{ms2022-0304fig10-2339.pdf}
\includegraphics[width=9.0cm,angle=0]{ms2022-0304fig10-2340.pdf}
\end{figure}
\clearpage

\begin{figure}
\includegraphics[width=9.0cm,angle=0]{ms2022-0304fig10-2341.pdf}
\includegraphics[width=9.0cm,angle=0]{ms2022-0304fig10-2342.pdf}
\vspace{-0.5cm}

\includegraphics[width=9.0cm,angle=0]{ms2022-0304fig10-2343.pdf}
\includegraphics[width=9.0cm,angle=0]{ms2022-0304fig10-2344.pdf}
\vspace{-0.5cm}

\includegraphics[width=9.0cm,angle=0]{ms2022-0304fig10-2345.pdf}
\includegraphics[width=9.0cm,angle=0]{ms2022-0304fig10-2346.pdf}
\vspace{-0.5cm}

\includegraphics[width=9.0cm,angle=0]{ms2022-0304fig10-2347.pdf}
\includegraphics[width=9.0cm,angle=0]{ms2022-0304fig10-2348.pdf}
\vspace{-0.5cm}

\includegraphics[width=9.0cm,angle=0]{ms2022-0304fig10-2349.pdf}
\includegraphics[width=9.0cm,angle=0]{ms2022-0304fig10-2350.pdf}
\end{figure}
\clearpage

\begin{figure}
\includegraphics[width=9.0cm,angle=0]{ms2022-0304fig10-2351.pdf}
\includegraphics[width=9.0cm,angle=0]{ms2022-0304fig10-2352.pdf}
\vspace{-0.5cm}

\includegraphics[width=9.0cm,angle=0]{ms2022-0304fig10-2353.pdf}
\includegraphics[width=9.0cm,angle=0]{ms2022-0304fig10-2354.pdf}
\vspace{-0.5cm}

\includegraphics[width=9.0cm,angle=0]{ms2022-0304fig10-2355.pdf}
\includegraphics[width=9.0cm,angle=0]{ms2022-0304fig10-2356.pdf}
\vspace{-0.5cm}

\includegraphics[width=9.0cm,angle=0]{ms2022-0304fig10-2357.pdf}
\includegraphics[width=9.0cm,angle=0]{ms2022-0304fig10-2358.pdf}
\vspace{-0.5cm}

\includegraphics[width=9.0cm,angle=0]{ms2022-0304fig10-2359.pdf}
\includegraphics[width=9.0cm,angle=0]{ms2022-0304fig10-2360.pdf}
\end{figure}
\clearpage

\begin{figure}
\includegraphics[width=9.0cm,angle=0]{ms2022-0304fig10-2361.pdf}
\includegraphics[width=9.0cm,angle=0]{ms2022-0304fig10-2362.pdf}
\vspace{-0.5cm}

\includegraphics[width=9.0cm,angle=0]{ms2022-0304fig10-2363.pdf}
\includegraphics[width=9.0cm,angle=0]{ms2022-0304fig10-2364.pdf}
\vspace{-0.5cm}

\includegraphics[width=9.0cm,angle=0]{ms2022-0304fig10-2365.pdf}
\includegraphics[width=9.0cm,angle=0]{ms2022-0304fig10-2366.pdf}
\vspace{-0.5cm}

\includegraphics[width=9.0cm,angle=0]{ms2022-0304fig10-2367.pdf}
\includegraphics[width=9.0cm,angle=0]{ms2022-0304fig10-2368.pdf}
\vspace{-0.5cm}

\includegraphics[width=9.0cm,angle=0]{ms2022-0304fig10-2369.pdf}
\includegraphics[width=9.0cm,angle=0]{ms2022-0304fig10-2370.pdf}
\end{figure}
\clearpage

\begin{figure}
\includegraphics[width=9.0cm,angle=0]{ms2022-0304fig10-2371.pdf}
\includegraphics[width=9.0cm,angle=0]{ms2022-0304fig10-2372.pdf}
\vspace{-0.5cm}

\includegraphics[width=9.0cm,angle=0]{ms2022-0304fig10-2373.pdf}
\includegraphics[width=9.0cm,angle=0]{ms2022-0304fig10-2374.pdf}
\vspace{-0.5cm}

\includegraphics[width=9.0cm,angle=0]{ms2022-0304fig10-2375.pdf}
\includegraphics[width=9.0cm,angle=0]{ms2022-0304fig10-2376.pdf}
\vspace{-0.5cm}

\includegraphics[width=9.0cm,angle=0]{ms2022-0304fig10-2377.pdf}
\includegraphics[width=9.0cm,angle=0]{ms2022-0304fig10-2378.pdf}
\vspace{-0.5cm}

\includegraphics[width=9.0cm,angle=0]{ms2022-0304fig10-2379.pdf}
\includegraphics[width=9.0cm,angle=0]{ms2022-0304fig10-2380.pdf}
\end{figure}
\clearpage

\begin{figure}
\includegraphics[width=9.0cm,angle=0]{ms2022-0304fig10-2381.pdf}
\includegraphics[width=9.0cm,angle=0]{ms2022-0304fig10-2382.pdf}
\vspace{-0.5cm}

\includegraphics[width=9.0cm,angle=0]{ms2022-0304fig10-2383.pdf}
\includegraphics[width=9.0cm,angle=0]{ms2022-0304fig10-2384.pdf}
\vspace{-0.5cm}

\includegraphics[width=9.0cm,angle=0]{ms2022-0304fig10-2385.pdf}
\includegraphics[width=9.0cm,angle=0]{ms2022-0304fig10-2386.pdf}
\vspace{-0.5cm}

\includegraphics[width=9.0cm,angle=0]{ms2022-0304fig10-2387.pdf}
\includegraphics[width=9.0cm,angle=0]{ms2022-0304fig10-2388.pdf}
\vspace{-0.5cm}

\includegraphics[width=9.0cm,angle=0]{ms2022-0304fig10-2389.pdf}
\includegraphics[width=9.0cm,angle=0]{ms2022-0304fig10-2390.pdf}
\end{figure}
\clearpage

\begin{figure}
\includegraphics[width=9.0cm,angle=0]{ms2022-0304fig10-2391.pdf}
\includegraphics[width=9.0cm,angle=0]{ms2022-0304fig10-2392.pdf}
\vspace{-0.5cm}

\includegraphics[width=9.0cm,angle=0]{ms2022-0304fig10-2393.pdf}
\includegraphics[width=9.0cm,angle=0]{ms2022-0304fig10-2394.pdf}
\vspace{-0.5cm}

\includegraphics[width=9.0cm,angle=0]{ms2022-0304fig10-2395.pdf}
\includegraphics[width=9.0cm,angle=0]{ms2022-0304fig10-2396.pdf}
\vspace{-0.5cm}

\includegraphics[width=9.0cm,angle=0]{ms2022-0304fig10-2397.pdf}
\includegraphics[width=9.0cm,angle=0]{ms2022-0304fig10-2398.pdf}
\vspace{-0.5cm}

\includegraphics[width=9.0cm,angle=0]{ms2022-0304fig10-2399.pdf}
\includegraphics[width=9.0cm,angle=0]{ms2022-0304fig10-2400.pdf}
\end{figure}
\clearpage

\begin{figure}
\includegraphics[width=9.0cm,angle=0]{ms2022-0304fig10-2401.pdf}
\includegraphics[width=9.0cm,angle=0]{ms2022-0304fig10-2402.pdf}
\vspace{-0.5cm}

\includegraphics[width=9.0cm,angle=0]{ms2022-0304fig10-2403.pdf}
\includegraphics[width=9.0cm,angle=0]{ms2022-0304fig10-2404.pdf}
\vspace{-0.5cm}

\includegraphics[width=9.0cm,angle=0]{ms2022-0304fig10-2405.pdf}
\includegraphics[width=9.0cm,angle=0]{ms2022-0304fig10-2406.pdf}
\vspace{-0.5cm}

\includegraphics[width=9.0cm,angle=0]{ms2022-0304fig10-2407.pdf}
\includegraphics[width=9.0cm,angle=0]{ms2022-0304fig10-2408.pdf}
\vspace{-0.5cm}

\includegraphics[width=9.0cm,angle=0]{ms2022-0304fig10-2409.pdf}
\includegraphics[width=9.0cm,angle=0]{ms2022-0304fig10-2410.pdf}
\end{figure}
\clearpage

\begin{figure}
\includegraphics[width=9.0cm,angle=0]{ms2022-0304fig10-2411.pdf}
\includegraphics[width=9.0cm,angle=0]{ms2022-0304fig10-2412.pdf}
\vspace{-0.5cm}

\includegraphics[width=9.0cm,angle=0]{ms2022-0304fig10-2413.pdf}
\includegraphics[width=9.0cm,angle=0]{ms2022-0304fig10-2414.pdf}
\vspace{-0.5cm}

\includegraphics[width=9.0cm,angle=0]{ms2022-0304fig10-2415.pdf}
\includegraphics[width=9.0cm,angle=0]{ms2022-0304fig10-2416.pdf}
\vspace{-0.5cm}

\includegraphics[width=9.0cm,angle=0]{ms2022-0304fig10-2417.pdf}
\includegraphics[width=9.0cm,angle=0]{ms2022-0304fig10-2418.pdf}
\vspace{-0.5cm}

\includegraphics[width=9.0cm,angle=0]{ms2022-0304fig10-2419.pdf}
\includegraphics[width=9.0cm,angle=0]{ms2022-0304fig10-2420.pdf}
\end{figure}
\clearpage

\begin{figure}
\includegraphics[width=9.0cm,angle=0]{ms2022-0304fig10-2421.pdf}
\includegraphics[width=9.0cm,angle=0]{ms2022-0304fig10-2422.pdf}
\vspace{-0.5cm}

\includegraphics[width=9.0cm,angle=0]{ms2022-0304fig10-2423.pdf}
\includegraphics[width=9.0cm,angle=0]{ms2022-0304fig10-2424.pdf}
\vspace{-0.5cm}

\includegraphics[width=9.0cm,angle=0]{ms2022-0304fig10-2425.pdf}
\includegraphics[width=9.0cm,angle=0]{ms2022-0304fig10-2426.pdf}
\vspace{-0.5cm}

\includegraphics[width=9.0cm,angle=0]{ms2022-0304fig10-2427.pdf}
\includegraphics[width=9.0cm,angle=0]{ms2022-0304fig10-2428.pdf}
\vspace{-0.5cm}

\includegraphics[width=9.0cm,angle=0]{ms2022-0304fig10-2429.pdf}
\includegraphics[width=9.0cm,angle=0]{ms2022-0304fig10-2430.pdf}
\end{figure}
\clearpage

\begin{figure}
\includegraphics[width=9.0cm,angle=0]{ms2022-0304fig10-2431.pdf}
\includegraphics[width=9.0cm,angle=0]{ms2022-0304fig10-2432.pdf}
\vspace{-0.5cm}

\includegraphics[width=9.0cm,angle=0]{ms2022-0304fig10-2433.pdf}
\includegraphics[width=9.0cm,angle=0]{ms2022-0304fig10-2434.pdf}
\vspace{-0.5cm}

\includegraphics[width=9.0cm,angle=0]{ms2022-0304fig10-2435.pdf}
\includegraphics[width=9.0cm,angle=0]{ms2022-0304fig10-2436.pdf}
\vspace{-0.5cm}

\includegraphics[width=9.0cm,angle=0]{ms2022-0304fig10-2437.pdf}
\includegraphics[width=9.0cm,angle=0]{ms2022-0304fig10-2438.pdf}
\vspace{-0.5cm}

\includegraphics[width=9.0cm,angle=0]{ms2022-0304fig10-2439.pdf}
\includegraphics[width=9.0cm,angle=0]{ms2022-0304fig10-2440.pdf}
\end{figure}
\clearpage

\begin{figure}
\includegraphics[width=9.0cm,angle=0]{ms2022-0304fig10-2441.pdf}
\includegraphics[width=9.0cm,angle=0]{ms2022-0304fig10-2442.pdf}
\vspace{-0.5cm}

\includegraphics[width=9.0cm,angle=0]{ms2022-0304fig10-2443.pdf}
\includegraphics[width=9.0cm,angle=0]{ms2022-0304fig10-2444.pdf}
\vspace{-0.5cm}

\includegraphics[width=9.0cm,angle=0]{ms2022-0304fig10-2445.pdf}
\includegraphics[width=9.0cm,angle=0]{ms2022-0304fig10-2446.pdf}
\vspace{-0.5cm}

\includegraphics[width=9.0cm,angle=0]{ms2022-0304fig10-2447.pdf}
\includegraphics[width=9.0cm,angle=0]{ms2022-0304fig10-2448.pdf}
\vspace{-0.5cm}

\includegraphics[width=9.0cm,angle=0]{ms2022-0304fig10-2449.pdf}
\includegraphics[width=9.0cm,angle=0]{ms2022-0304fig10-2450.pdf}
\end{figure}
\clearpage

\begin{figure}
\includegraphics[width=9.0cm,angle=0]{ms2022-0304fig10-2451.pdf}
\includegraphics[width=9.0cm,angle=0]{ms2022-0304fig10-2452.pdf}
\vspace{-0.5cm}

\includegraphics[width=9.0cm,angle=0]{ms2022-0304fig10-2453.pdf}
\includegraphics[width=9.0cm,angle=0]{ms2022-0304fig10-2454.pdf}
\vspace{-0.5cm}

\includegraphics[width=9.0cm,angle=0]{ms2022-0304fig10-2455.pdf}
\includegraphics[width=9.0cm,angle=0]{ms2022-0304fig10-2456.pdf}
\vspace{-0.5cm}

\includegraphics[width=9.0cm,angle=0]{ms2022-0304fig10-2457.pdf}
\includegraphics[width=9.0cm,angle=0]{ms2022-0304fig10-2458.pdf}
\vspace{-0.5cm}

\includegraphics[width=9.0cm,angle=0]{ms2022-0304fig10-2459.pdf}
\includegraphics[width=9.0cm,angle=0]{ms2022-0304fig10-2460.pdf}
\end{figure}
\clearpage

\begin{figure}
\includegraphics[width=9.0cm,angle=0]{ms2022-0304fig10-2461.pdf}
\includegraphics[width=9.0cm,angle=0]{ms2022-0304fig10-2462.pdf}
\vspace{-0.5cm}

\includegraphics[width=9.0cm,angle=0]{ms2022-0304fig10-2463.pdf}
\includegraphics[width=9.0cm,angle=0]{ms2022-0304fig10-2464.pdf}
\vspace{-0.5cm}

\includegraphics[width=9.0cm,angle=0]{ms2022-0304fig10-2465.pdf}
\includegraphics[width=9.0cm,angle=0]{ms2022-0304fig10-2466.pdf}
\vspace{-0.5cm}

\includegraphics[width=9.0cm,angle=0]{ms2022-0304fig10-2467.pdf}
\includegraphics[width=9.0cm,angle=0]{ms2022-0304fig10-2468.pdf}
\vspace{-0.5cm}

\includegraphics[width=9.0cm,angle=0]{ms2022-0304fig10-2469.pdf}
\includegraphics[width=9.0cm,angle=0]{ms2022-0304fig10-2470.pdf}
\end{figure}
\clearpage

\begin{figure}
\includegraphics[width=9.0cm,angle=0]{ms2022-0304fig10-2471.pdf}
\includegraphics[width=9.0cm,angle=0]{ms2022-0304fig10-2472.pdf}
\vspace{-0.5cm}

\includegraphics[width=9.0cm,angle=0]{ms2022-0304fig10-2473.pdf}
\includegraphics[width=9.0cm,angle=0]{ms2022-0304fig10-2474.pdf}
\vspace{-0.5cm}

\includegraphics[width=9.0cm,angle=0]{ms2022-0304fig10-2475.pdf}
\includegraphics[width=9.0cm,angle=0]{ms2022-0304fig10-2476.pdf}
\vspace{-0.5cm}

\includegraphics[width=9.0cm,angle=0]{ms2022-0304fig10-2477.pdf}
\includegraphics[width=9.0cm,angle=0]{ms2022-0304fig10-2478.pdf}
\vspace{-0.5cm}

\includegraphics[width=9.0cm,angle=0]{ms2022-0304fig10-2479.pdf}
\includegraphics[width=9.0cm,angle=0]{ms2022-0304fig10-2480.pdf}
\end{figure}
\clearpage

\begin{figure}
\includegraphics[width=9.0cm,angle=0]{ms2022-0304fig10-2481.pdf}
\includegraphics[width=9.0cm,angle=0]{ms2022-0304fig10-2482.pdf}
\vspace{-0.5cm}

\includegraphics[width=9.0cm,angle=0]{ms2022-0304fig10-2483.pdf}
\includegraphics[width=9.0cm,angle=0]{ms2022-0304fig10-2484.pdf}
\vspace{-0.5cm}

\includegraphics[width=9.0cm,angle=0]{ms2022-0304fig10-2485.pdf}
\includegraphics[width=9.0cm,angle=0]{ms2022-0304fig10-2486.pdf}
\vspace{-0.5cm}

\includegraphics[width=9.0cm,angle=0]{ms2022-0304fig10-2487.pdf}
\includegraphics[width=9.0cm,angle=0]{ms2022-0304fig10-2488.pdf}
\vspace{-0.5cm}

\includegraphics[width=9.0cm,angle=0]{ms2022-0304fig10-2489.pdf}
\includegraphics[width=9.0cm,angle=0]{ms2022-0304fig10-2490.pdf}
\end{figure}
\clearpage

\begin{figure}
\includegraphics[width=9.0cm,angle=0]{ms2022-0304fig10-2491.pdf}
\includegraphics[width=9.0cm,angle=0]{ms2022-0304fig10-2492.pdf}
\vspace{-0.5cm}

\includegraphics[width=9.0cm,angle=0]{ms2022-0304fig10-2493.pdf}
\includegraphics[width=9.0cm,angle=0]{ms2022-0304fig10-2494.pdf}
\vspace{-0.5cm}

\includegraphics[width=9.0cm,angle=0]{ms2022-0304fig10-2495.pdf}
\includegraphics[width=9.0cm,angle=0]{ms2022-0304fig10-2496.pdf}
\vspace{-0.5cm}

\includegraphics[width=9.0cm,angle=0]{ms2022-0304fig10-2497.pdf}
\includegraphics[width=9.0cm,angle=0]{ms2022-0304fig10-2498.pdf}
\vspace{-0.5cm}

\includegraphics[width=9.0cm,angle=0]{ms2022-0304fig10-2499.pdf}
\includegraphics[width=9.0cm,angle=0]{ms2022-0304fig10-2500.pdf}
\end{figure}
\clearpage

\begin{figure}
\includegraphics[width=9.0cm,angle=0]{ms2022-0304fig10-2501.pdf}
\includegraphics[width=9.0cm,angle=0]{ms2022-0304fig10-2502.pdf}
\vspace{-0.5cm}

\includegraphics[width=9.0cm,angle=0]{ms2022-0304fig10-2503.pdf}
\includegraphics[width=9.0cm,angle=0]{ms2022-0304fig10-2504.pdf}
\vspace{-0.5cm}

\includegraphics[width=9.0cm,angle=0]{ms2022-0304fig10-2505.pdf}
\includegraphics[width=9.0cm,angle=0]{ms2022-0304fig10-2506.pdf}
\vspace{-0.5cm}

\includegraphics[width=9.0cm,angle=0]{ms2022-0304fig10-2507.pdf}
\includegraphics[width=9.0cm,angle=0]{ms2022-0304fig10-2508.pdf}
\vspace{-0.5cm}

\includegraphics[width=9.0cm,angle=0]{ms2022-0304fig10-2509.pdf}
\includegraphics[width=9.0cm,angle=0]{ms2022-0304fig10-2510.pdf}
\end{figure}
\clearpage

\begin{figure}
\includegraphics[width=9.0cm,angle=0]{ms2022-0304fig10-2511.pdf}
\includegraphics[width=9.0cm,angle=0]{ms2022-0304fig10-2512.pdf}
\vspace{-0.5cm}

\includegraphics[width=9.0cm,angle=0]{ms2022-0304fig10-2513.pdf}
\includegraphics[width=9.0cm,angle=0]{ms2022-0304fig10-2514.pdf}
\vspace{-0.5cm}

\includegraphics[width=9.0cm,angle=0]{ms2022-0304fig10-2515.pdf}
\includegraphics[width=9.0cm,angle=0]{ms2022-0304fig10-2516.pdf}
\vspace{-0.5cm}

\includegraphics[width=9.0cm,angle=0]{ms2022-0304fig10-2517.pdf}
\includegraphics[width=9.0cm,angle=0]{ms2022-0304fig10-2518.pdf}
\vspace{-0.5cm}

\includegraphics[width=9.0cm,angle=0]{ms2022-0304fig10-2519.pdf}
\includegraphics[width=9.0cm,angle=0]{ms2022-0304fig10-2520.pdf}
\end{figure}
\clearpage

\begin{figure}
\includegraphics[width=9.0cm,angle=0]{ms2022-0304fig10-2521.pdf}
\includegraphics[width=9.0cm,angle=0]{ms2022-0304fig10-2522.pdf}
\vspace{-0.5cm}

\includegraphics[width=9.0cm,angle=0]{ms2022-0304fig10-2523.pdf}
\includegraphics[width=9.0cm,angle=0]{ms2022-0304fig10-2524.pdf}
\vspace{-0.5cm}

\includegraphics[width=9.0cm,angle=0]{ms2022-0304fig10-2525.pdf}
\includegraphics[width=9.0cm,angle=0]{ms2022-0304fig10-2526.pdf}
\vspace{-0.5cm}

\includegraphics[width=9.0cm,angle=0]{ms2022-0304fig10-2527.pdf}
\includegraphics[width=9.0cm,angle=0]{ms2022-0304fig10-2528.pdf}
\vspace{-0.5cm}

\includegraphics[width=9.0cm,angle=0]{ms2022-0304fig10-2529.pdf}
\includegraphics[width=9.0cm,angle=0]{ms2022-0304fig10-2530.pdf}
\end{figure}
\clearpage

\begin{figure}
\includegraphics[width=9.0cm,angle=0]{ms2022-0304fig10-2531.pdf}
\includegraphics[width=9.0cm,angle=0]{ms2022-0304fig10-2532.pdf}
\vspace{-0.5cm}

\includegraphics[width=9.0cm,angle=0]{ms2022-0304fig10-2533.pdf}
\includegraphics[width=9.0cm,angle=0]{ms2022-0304fig10-2534.pdf}
\vspace{-0.5cm}

\includegraphics[width=9.0cm,angle=0]{ms2022-0304fig10-2535.pdf}
\includegraphics[width=9.0cm,angle=0]{ms2022-0304fig10-2536.pdf}
\vspace{-0.5cm}

\includegraphics[width=9.0cm,angle=0]{ms2022-0304fig10-2537.pdf}
\includegraphics[width=9.0cm,angle=0]{ms2022-0304fig10-2538.pdf}
\vspace{-0.5cm}

\includegraphics[width=9.0cm,angle=0]{ms2022-0304fig10-2539.pdf}
\includegraphics[width=9.0cm,angle=0]{ms2022-0304fig10-2540.pdf}
\end{figure}
\clearpage

\begin{figure}
\includegraphics[width=9.0cm,angle=0]{ms2022-0304fig10-2541.pdf}
\includegraphics[width=9.0cm,angle=0]{ms2022-0304fig10-2542.pdf}
\vspace{-0.5cm}

\includegraphics[width=9.0cm,angle=0]{ms2022-0304fig10-2543.pdf}
\includegraphics[width=9.0cm,angle=0]{ms2022-0304fig10-2544.pdf}
\vspace{-0.5cm}

\includegraphics[width=9.0cm,angle=0]{ms2022-0304fig10-2545.pdf}
\includegraphics[width=9.0cm,angle=0]{ms2022-0304fig10-2546.pdf}
\vspace{-0.5cm}

\includegraphics[width=9.0cm,angle=0]{ms2022-0304fig10-2547.pdf}
\includegraphics[width=9.0cm,angle=0]{ms2022-0304fig10-2548.pdf}
\vspace{-0.5cm}

\includegraphics[width=9.0cm,angle=0]{ms2022-0304fig10-2549.pdf}
\includegraphics[width=9.0cm,angle=0]{ms2022-0304fig10-2550.pdf}
\end{figure}
\clearpage

\begin{figure}
\includegraphics[width=9.0cm,angle=0]{ms2022-0304fig10-2551.pdf}
\includegraphics[width=9.0cm,angle=0]{ms2022-0304fig10-2552.pdf}
\vspace{-0.5cm}

\includegraphics[width=9.0cm,angle=0]{ms2022-0304fig10-2553.pdf}
\includegraphics[width=9.0cm,angle=0]{ms2022-0304fig10-2554.pdf}
\vspace{-0.5cm}

\includegraphics[width=9.0cm,angle=0]{ms2022-0304fig10-2555.pdf}
\includegraphics[width=9.0cm,angle=0]{ms2022-0304fig10-2556.pdf}
\vspace{-0.5cm}

\includegraphics[width=9.0cm,angle=0]{ms2022-0304fig10-2557.pdf}
\includegraphics[width=9.0cm,angle=0]{ms2022-0304fig10-2558.pdf}
\vspace{-0.5cm}

\includegraphics[width=9.0cm,angle=0]{ms2022-0304fig10-2559.pdf}
\includegraphics[width=9.0cm,angle=0]{ms2022-0304fig10-2560.pdf}
\end{figure}
\clearpage

\begin{figure}
\includegraphics[width=9.0cm,angle=0]{ms2022-0304fig10-2561.pdf}
\includegraphics[width=9.0cm,angle=0]{ms2022-0304fig10-2562.pdf}
\vspace{-0.5cm}

\includegraphics[width=9.0cm,angle=0]{ms2022-0304fig10-2563.pdf}
\includegraphics[width=9.0cm,angle=0]{ms2022-0304fig10-2564.pdf}
\vspace{-0.5cm}

\includegraphics[width=9.0cm,angle=0]{ms2022-0304fig10-2565.pdf}
\includegraphics[width=9.0cm,angle=0]{ms2022-0304fig10-2566.pdf}
\vspace{-0.5cm}

\includegraphics[width=9.0cm,angle=0]{ms2022-0304fig10-2567.pdf}
\includegraphics[width=9.0cm,angle=0]{ms2022-0304fig10-2568.pdf}
\vspace{-0.5cm}

\includegraphics[width=9.0cm,angle=0]{ms2022-0304fig10-2569.pdf}
\includegraphics[width=9.0cm,angle=0]{ms2022-0304fig10-2570.pdf}
\end{figure}
\clearpage

\begin{figure}
\includegraphics[width=9.0cm,angle=0]{ms2022-0304fig10-2571.pdf}
\includegraphics[width=9.0cm,angle=0]{ms2022-0304fig10-2572.pdf}
\vspace{-0.5cm}

\includegraphics[width=9.0cm,angle=0]{ms2022-0304fig10-2573.pdf}
\includegraphics[width=9.0cm,angle=0]{ms2022-0304fig10-2574.pdf}
\vspace{-0.5cm}

\includegraphics[width=9.0cm,angle=0]{ms2022-0304fig10-2575.pdf}
\includegraphics[width=9.0cm,angle=0]{ms2022-0304fig10-2576.pdf}
\vspace{-0.5cm}

\includegraphics[width=9.0cm,angle=0]{ms2022-0304fig10-2577.pdf}
\includegraphics[width=9.0cm,angle=0]{ms2022-0304fig10-2578.pdf}
\vspace{-0.5cm}

\includegraphics[width=9.0cm,angle=0]{ms2022-0304fig10-2579.pdf}
\includegraphics[width=9.0cm,angle=0]{ms2022-0304fig10-2580.pdf}
\end{figure}
\clearpage

\begin{figure}
\includegraphics[width=9.0cm,angle=0]{ms2022-0304fig10-2581.pdf}
\includegraphics[width=9.0cm,angle=0]{ms2022-0304fig10-2582.pdf}
\vspace{-0.5cm}

\includegraphics[width=9.0cm,angle=0]{ms2022-0304fig10-2583.pdf}
\includegraphics[width=9.0cm,angle=0]{ms2022-0304fig10-2584.pdf}
\vspace{-0.5cm}

\includegraphics[width=9.0cm,angle=0]{ms2022-0304fig10-2585.pdf}
\includegraphics[width=9.0cm,angle=0]{ms2022-0304fig10-2586.pdf}
\vspace{-0.5cm}

\includegraphics[width=9.0cm,angle=0]{ms2022-0304fig10-2587.pdf}
\includegraphics[width=9.0cm,angle=0]{ms2022-0304fig10-2588.pdf}
\vspace{-0.5cm}

\includegraphics[width=9.0cm,angle=0]{ms2022-0304fig10-2589.pdf}
\includegraphics[width=9.0cm,angle=0]{ms2022-0304fig10-2590.pdf}
\end{figure}
\clearpage

\begin{figure}
\includegraphics[width=9.0cm,angle=0]{ms2022-0304fig10-2591.pdf}
\includegraphics[width=9.0cm,angle=0]{ms2022-0304fig10-2592.pdf}
\vspace{-0.5cm}

\includegraphics[width=9.0cm,angle=0]{ms2022-0304fig10-2593.pdf}
\includegraphics[width=9.0cm,angle=0]{ms2022-0304fig10-2594.pdf}
\vspace{-0.5cm}

\includegraphics[width=9.0cm,angle=0]{ms2022-0304fig10-2595.pdf}
\includegraphics[width=9.0cm,angle=0]{ms2022-0304fig10-2596.pdf}
\vspace{-0.5cm}

\includegraphics[width=9.0cm,angle=0]{ms2022-0304fig10-2597.pdf}
\includegraphics[width=9.0cm,angle=0]{ms2022-0304fig10-2598.pdf}
\vspace{-0.5cm}

\includegraphics[width=9.0cm,angle=0]{ms2022-0304fig10-2599.pdf}
\includegraphics[width=9.0cm,angle=0]{ms2022-0304fig10-2600.pdf}
\end{figure}
\clearpage

\begin{figure}
\includegraphics[width=9.0cm,angle=0]{ms2022-0304fig10-2601.pdf}
\includegraphics[width=9.0cm,angle=0]{ms2022-0304fig10-2602.pdf}
\vspace{-0.5cm}

\includegraphics[width=9.0cm,angle=0]{ms2022-0304fig10-2603.pdf}
\includegraphics[width=9.0cm,angle=0]{ms2022-0304fig10-2604.pdf}
\vspace{-0.5cm}

\includegraphics[width=9.0cm,angle=0]{ms2022-0304fig10-2605.pdf}
\includegraphics[width=9.0cm,angle=0]{ms2022-0304fig10-2606.pdf}
\vspace{-0.5cm}

\includegraphics[width=9.0cm,angle=0]{ms2022-0304fig10-2607.pdf}
\includegraphics[width=9.0cm,angle=0]{ms2022-0304fig10-2608.pdf}
\vspace{-0.5cm}

\includegraphics[width=9.0cm,angle=0]{ms2022-0304fig10-2609.pdf}
\includegraphics[width=9.0cm,angle=0]{ms2022-0304fig10-2610.pdf}
\end{figure}
\clearpage

\begin{figure}
\includegraphics[width=9.0cm,angle=0]{ms2022-0304fig10-2611.pdf}
\includegraphics[width=9.0cm,angle=0]{ms2022-0304fig10-2612.pdf}
\vspace{-0.5cm}

\includegraphics[width=9.0cm,angle=0]{ms2022-0304fig10-2613.pdf}
\includegraphics[width=9.0cm,angle=0]{ms2022-0304fig10-2614.pdf}
\vspace{-0.5cm}

\includegraphics[width=9.0cm,angle=0]{ms2022-0304fig10-2615.pdf}
\includegraphics[width=9.0cm,angle=0]{ms2022-0304fig10-2616.pdf}
\vspace{-0.5cm}

\includegraphics[width=9.0cm,angle=0]{ms2022-0304fig10-2617.pdf}
\includegraphics[width=9.0cm,angle=0]{ms2022-0304fig10-2618.pdf}
\vspace{-0.5cm}

\includegraphics[width=9.0cm,angle=0]{ms2022-0304fig10-2619.pdf}
\includegraphics[width=9.0cm,angle=0]{ms2022-0304fig10-2620.pdf}
\end{figure}
\clearpage

\begin{figure}
\includegraphics[width=9.0cm,angle=0]{ms2022-0304fig10-2621.pdf}
\includegraphics[width=9.0cm,angle=0]{ms2022-0304fig10-2622.pdf}
\vspace{-0.5cm}

\includegraphics[width=9.0cm,angle=0]{ms2022-0304fig10-2623.pdf}
\includegraphics[width=9.0cm,angle=0]{ms2022-0304fig10-2624.pdf}
\vspace{-0.5cm}

\includegraphics[width=9.0cm,angle=0]{ms2022-0304fig10-2625.pdf}
\includegraphics[width=9.0cm,angle=0]{ms2022-0304fig10-2626.pdf}
\vspace{-0.5cm}

\includegraphics[width=9.0cm,angle=0]{ms2022-0304fig10-2627.pdf}
\includegraphics[width=9.0cm,angle=0]{ms2022-0304fig10-2628.pdf}
\vspace{-0.5cm}

\includegraphics[width=9.0cm,angle=0]{ms2022-0304fig10-2629.pdf}
\includegraphics[width=9.0cm,angle=0]{ms2022-0304fig10-2630.pdf}
\end{figure}
\clearpage

\begin{figure}
\includegraphics[width=9.0cm,angle=0]{ms2022-0304fig10-2631.pdf}
\includegraphics[width=9.0cm,angle=0]{ms2022-0304fig10-2632.pdf}
\vspace{-0.5cm}

\includegraphics[width=9.0cm,angle=0]{ms2022-0304fig10-2633.pdf}
\includegraphics[width=9.0cm,angle=0]{ms2022-0304fig10-2634.pdf}
\vspace{-0.5cm}

\includegraphics[width=9.0cm,angle=0]{ms2022-0304fig10-2635.pdf}
\includegraphics[width=9.0cm,angle=0]{ms2022-0304fig10-2636.pdf}
\vspace{-0.5cm}

\includegraphics[width=9.0cm,angle=0]{ms2022-0304fig10-2637.pdf}
\includegraphics[width=9.0cm,angle=0]{ms2022-0304fig10-2638.pdf}
\vspace{-0.5cm}

\includegraphics[width=9.0cm,angle=0]{ms2022-0304fig10-2639.pdf}
\includegraphics[width=9.0cm,angle=0]{ms2022-0304fig10-2640.pdf}
\end{figure}
\clearpage

\begin{figure}
\includegraphics[width=9.0cm,angle=0]{ms2022-0304fig10-2641.pdf}
\includegraphics[width=9.0cm,angle=0]{ms2022-0304fig10-2642.pdf}
\vspace{-0.5cm}

\includegraphics[width=9.0cm,angle=0]{ms2022-0304fig10-2643.pdf}
\includegraphics[width=9.0cm,angle=0]{ms2022-0304fig10-2644.pdf}
\vspace{-0.5cm}

\includegraphics[width=9.0cm,angle=0]{ms2022-0304fig10-2645.pdf}
\includegraphics[width=9.0cm,angle=0]{ms2022-0304fig10-2646.pdf}
\vspace{-0.5cm}

\includegraphics[width=9.0cm,angle=0]{ms2022-0304fig10-2647.pdf}
\includegraphics[width=9.0cm,angle=0]{ms2022-0304fig10-2648.pdf}
\vspace{-0.5cm}

\includegraphics[width=9.0cm,angle=0]{ms2022-0304fig10-2649.pdf}
\includegraphics[width=9.0cm,angle=0]{ms2022-0304fig10-2650.pdf}
\end{figure}
\clearpage

\begin{figure}
\includegraphics[width=9.0cm,angle=0]{ms2022-0304fig10-2651.pdf}
\includegraphics[width=9.0cm,angle=0]{ms2022-0304fig10-2652.pdf}
\vspace{-0.5cm}

\includegraphics[width=9.0cm,angle=0]{ms2022-0304fig10-2653.pdf}
\includegraphics[width=9.0cm,angle=0]{ms2022-0304fig10-2654.pdf}
\vspace{-0.5cm}

\includegraphics[width=9.0cm,angle=0]{ms2022-0304fig10-2655.pdf}
\includegraphics[width=9.0cm,angle=0]{ms2022-0304fig10-2656.pdf}
\vspace{-0.5cm}

\includegraphics[width=9.0cm,angle=0]{ms2022-0304fig10-2657.pdf}
\includegraphics[width=9.0cm,angle=0]{ms2022-0304fig10-2658.pdf}
\vspace{-0.5cm}

\includegraphics[width=9.0cm,angle=0]{ms2022-0304fig10-2659.pdf}
\includegraphics[width=9.0cm,angle=0]{ms2022-0304fig10-2660.pdf}
\end{figure}
\clearpage

\begin{figure}
\includegraphics[width=9.0cm,angle=0]{ms2022-0304fig10-2661.pdf}
\includegraphics[width=9.0cm,angle=0]{ms2022-0304fig10-2662.pdf}
\vspace{-0.5cm}

\includegraphics[width=9.0cm,angle=0]{ms2022-0304fig10-2663.pdf}
\includegraphics[width=9.0cm,angle=0]{ms2022-0304fig10-2664.pdf}
\vspace{-0.5cm}

\includegraphics[width=9.0cm,angle=0]{ms2022-0304fig10-2665.pdf}
\includegraphics[width=9.0cm,angle=0]{ms2022-0304fig10-2666.pdf}
\vspace{-0.5cm}

\includegraphics[width=9.0cm,angle=0]{ms2022-0304fig10-2667.pdf}
\includegraphics[width=9.0cm,angle=0]{ms2022-0304fig10-2668.pdf}
\vspace{-0.5cm}

\includegraphics[width=9.0cm,angle=0]{ms2022-0304fig10-2669.pdf}
\includegraphics[width=9.0cm,angle=0]{ms2022-0304fig10-2670.pdf}
\end{figure}
\clearpage

\begin{figure}
\includegraphics[width=9.0cm,angle=0]{ms2022-0304fig10-2671.pdf}
\includegraphics[width=9.0cm,angle=0]{ms2022-0304fig10-2672.pdf}
\vspace{-0.5cm}

\includegraphics[width=9.0cm,angle=0]{ms2022-0304fig10-2673.pdf}
\includegraphics[width=9.0cm,angle=0]{ms2022-0304fig10-2674.pdf}
\vspace{-0.5cm}

\includegraphics[width=9.0cm,angle=0]{ms2022-0304fig10-2675.pdf}
\includegraphics[width=9.0cm,angle=0]{ms2022-0304fig10-2676.pdf}
\vspace{-0.5cm}

\includegraphics[width=9.0cm,angle=0]{ms2022-0304fig10-2677.pdf}
\includegraphics[width=9.0cm,angle=0]{ms2022-0304fig10-2678.pdf}
\vspace{-0.5cm}

\includegraphics[width=9.0cm,angle=0]{ms2022-0304fig10-2679.pdf}
\includegraphics[width=9.0cm,angle=0]{ms2022-0304fig10-2680.pdf}
\end{figure}
\clearpage

\begin{figure}
\includegraphics[width=9.0cm,angle=0]{ms2022-0304fig10-2681.pdf}
\includegraphics[width=9.0cm,angle=0]{ms2022-0304fig10-2682.pdf}
\vspace{-0.5cm}

\includegraphics[width=9.0cm,angle=0]{ms2022-0304fig10-2683.pdf}
\includegraphics[width=9.0cm,angle=0]{ms2022-0304fig10-2684.pdf}
\vspace{-0.5cm}

\includegraphics[width=9.0cm,angle=0]{ms2022-0304fig10-2685.pdf}
\includegraphics[width=9.0cm,angle=0]{ms2022-0304fig10-2686.pdf}
\vspace{-0.5cm}

\includegraphics[width=9.0cm,angle=0]{ms2022-0304fig10-2687.pdf}
\includegraphics[width=9.0cm,angle=0]{ms2022-0304fig10-2688.pdf}
\vspace{-0.5cm}

\includegraphics[width=9.0cm,angle=0]{ms2022-0304fig10-2689.pdf}
\includegraphics[width=9.0cm,angle=0]{ms2022-0304fig10-2690.pdf}
\end{figure}
\clearpage

\begin{figure}
\includegraphics[width=9.0cm,angle=0]{ms2022-0304fig10-2691.pdf}
\includegraphics[width=9.0cm,angle=0]{ms2022-0304fig10-2692.pdf}
\vspace{-0.5cm}

\includegraphics[width=9.0cm,angle=0]{ms2022-0304fig10-2693.pdf}
\includegraphics[width=9.0cm,angle=0]{ms2022-0304fig10-2694.pdf}
\vspace{-0.5cm}

\includegraphics[width=9.0cm,angle=0]{ms2022-0304fig10-2695.pdf}
\includegraphics[width=9.0cm,angle=0]{ms2022-0304fig10-2696.pdf}
\vspace{-0.5cm}

\includegraphics[width=9.0cm,angle=0]{ms2022-0304fig10-2697.pdf}
\includegraphics[width=9.0cm,angle=0]{ms2022-0304fig10-2698.pdf}
\vspace{-0.5cm}

\includegraphics[width=9.0cm,angle=0]{ms2022-0304fig10-2699.pdf}
\includegraphics[width=9.0cm,angle=0]{ms2022-0304fig10-2700.pdf}
\end{figure}
\clearpage

\begin{figure}
\includegraphics[width=9.0cm,angle=0]{ms2022-0304fig10-2701.pdf}
\includegraphics[width=9.0cm,angle=0]{ms2022-0304fig10-2702.pdf}
\vspace{-0.5cm}

\includegraphics[width=9.0cm,angle=0]{ms2022-0304fig10-2703.pdf}
\includegraphics[width=9.0cm,angle=0]{ms2022-0304fig10-2704.pdf}
\vspace{-0.5cm}

\includegraphics[width=9.0cm,angle=0]{ms2022-0304fig10-2705.pdf}
\includegraphics[width=9.0cm,angle=0]{ms2022-0304fig10-2706.pdf}
\vspace{-0.5cm}

\includegraphics[width=9.0cm,angle=0]{ms2022-0304fig10-2707.pdf}
\includegraphics[width=9.0cm,angle=0]{ms2022-0304fig10-2708.pdf}
\vspace{-0.5cm}

\includegraphics[width=9.0cm,angle=0]{ms2022-0304fig10-2709.pdf}
\includegraphics[width=9.0cm,angle=0]{ms2022-0304fig10-2710.pdf}
\end{figure}
\clearpage

\begin{figure}
\includegraphics[width=9.0cm,angle=0]{ms2022-0304fig10-2711.pdf}
\includegraphics[width=9.0cm,angle=0]{ms2022-0304fig10-2712.pdf}
\vspace{-0.5cm}

\includegraphics[width=9.0cm,angle=0]{ms2022-0304fig10-2713.pdf}
\includegraphics[width=9.0cm,angle=0]{ms2022-0304fig10-2714.pdf}
\vspace{-0.5cm}

\includegraphics[width=9.0cm,angle=0]{ms2022-0304fig10-2715.pdf}
\includegraphics[width=9.0cm,angle=0]{ms2022-0304fig10-2716.pdf}
\vspace{-0.5cm}

\includegraphics[width=9.0cm,angle=0]{ms2022-0304fig10-2717.pdf}
\includegraphics[width=9.0cm,angle=0]{ms2022-0304fig10-2718.pdf}
\vspace{-0.5cm}

\includegraphics[width=9.0cm,angle=0]{ms2022-0304fig10-2719.pdf}
\includegraphics[width=9.0cm,angle=0]{ms2022-0304fig10-2720.pdf}
\end{figure}
\clearpage

\begin{figure}
\includegraphics[width=9.0cm,angle=0]{ms2022-0304fig10-2721.pdf}
\includegraphics[width=9.0cm,angle=0]{ms2022-0304fig10-2722.pdf}
\vspace{-0.5cm}

\includegraphics[width=9.0cm,angle=0]{ms2022-0304fig10-2723.pdf}
\includegraphics[width=9.0cm,angle=0]{ms2022-0304fig10-2724.pdf}
\vspace{-0.5cm}

\includegraphics[width=9.0cm,angle=0]{ms2022-0304fig10-2725.pdf}
\includegraphics[width=9.0cm,angle=0]{ms2022-0304fig10-2726.pdf}
\vspace{-0.5cm}

\includegraphics[width=9.0cm,angle=0]{ms2022-0304fig10-2727.pdf}
\includegraphics[width=9.0cm,angle=0]{ms2022-0304fig10-2728.pdf}
\vspace{-0.5cm}

\includegraphics[width=9.0cm,angle=0]{ms2022-0304fig10-2729.pdf}
\includegraphics[width=9.0cm,angle=0]{ms2022-0304fig10-2730.pdf}
\end{figure}
\clearpage

\begin{figure}
\includegraphics[width=9.0cm,angle=0]{ms2022-0304fig10-2731.pdf}
\includegraphics[width=9.0cm,angle=0]{ms2022-0304fig10-2732.pdf}
\vspace{-0.5cm}

\includegraphics[width=9.0cm,angle=0]{ms2022-0304fig10-2733.pdf}
\includegraphics[width=9.0cm,angle=0]{ms2022-0304fig10-2734.pdf}
\vspace{-0.5cm}

\includegraphics[width=9.0cm,angle=0]{ms2022-0304fig10-2735.pdf}
\includegraphics[width=9.0cm,angle=0]{ms2022-0304fig10-2736.pdf}
\vspace{-0.5cm}

\includegraphics[width=9.0cm,angle=0]{ms2022-0304fig10-2737.pdf}
\includegraphics[width=9.0cm,angle=0]{ms2022-0304fig10-2738.pdf}
\vspace{-0.5cm}

\includegraphics[width=9.0cm,angle=0]{ms2022-0304fig10-2739.pdf}
\includegraphics[width=9.0cm,angle=0]{ms2022-0304fig10-2740.pdf}
\end{figure}
\clearpage

\begin{figure}
\includegraphics[width=9.0cm,angle=0]{ms2022-0304fig10-2741.pdf}
\includegraphics[width=9.0cm,angle=0]{ms2022-0304fig10-2742.pdf}
\vspace{-0.5cm}

\includegraphics[width=9.0cm,angle=0]{ms2022-0304fig10-2743.pdf}
\includegraphics[width=9.0cm,angle=0]{ms2022-0304fig10-2744.pdf}
\vspace{-0.5cm}

\includegraphics[width=9.0cm,angle=0]{ms2022-0304fig10-2745.pdf}
\includegraphics[width=9.0cm,angle=0]{ms2022-0304fig10-2746.pdf}
\vspace{-0.5cm}

\includegraphics[width=9.0cm,angle=0]{ms2022-0304fig10-2747.pdf}
\includegraphics[width=9.0cm,angle=0]{ms2022-0304fig10-2748.pdf}
\vspace{-0.5cm}

\includegraphics[width=9.0cm,angle=0]{ms2022-0304fig10-2749.pdf}
\includegraphics[width=9.0cm,angle=0]{ms2022-0304fig10-2750.pdf}
\end{figure}
\clearpage

\begin{figure}
\includegraphics[width=9.0cm,angle=0]{ms2022-0304fig10-2751.pdf}
\includegraphics[width=9.0cm,angle=0]{ms2022-0304fig10-2752.pdf}
\vspace{-0.5cm}

\includegraphics[width=9.0cm,angle=0]{ms2022-0304fig10-2753.pdf}
\includegraphics[width=9.0cm,angle=0]{ms2022-0304fig10-2754.pdf}
\vspace{-0.5cm}

\includegraphics[width=9.0cm,angle=0]{ms2022-0304fig10-2755.pdf}
\includegraphics[width=9.0cm,angle=0]{ms2022-0304fig10-2756.pdf}
\vspace{-0.5cm}

\includegraphics[width=9.0cm,angle=0]{ms2022-0304fig10-2757.pdf}
\includegraphics[width=9.0cm,angle=0]{ms2022-0304fig10-2758.pdf}
\vspace{-0.5cm}

\includegraphics[width=9.0cm,angle=0]{ms2022-0304fig10-2759.pdf}
\includegraphics[width=9.0cm,angle=0]{ms2022-0304fig10-2760.pdf}
\end{figure}
\clearpage

\begin{figure}
\includegraphics[width=9.0cm,angle=0]{ms2022-0304fig10-2761.pdf}
\includegraphics[width=9.0cm,angle=0]{ms2022-0304fig10-2762.pdf}
\vspace{-0.5cm}

\includegraphics[width=9.0cm,angle=0]{ms2022-0304fig10-2763.pdf}
\includegraphics[width=9.0cm,angle=0]{ms2022-0304fig10-2764.pdf}
\vspace{-0.5cm}

\includegraphics[width=9.0cm,angle=0]{ms2022-0304fig10-2765.pdf}
\includegraphics[width=9.0cm,angle=0]{ms2022-0304fig10-2766.pdf}
\vspace{-0.5cm}

\includegraphics[width=9.0cm,angle=0]{ms2022-0304fig10-2767.pdf}
\includegraphics[width=9.0cm,angle=0]{ms2022-0304fig10-2768.pdf}
\vspace{-0.5cm}

\includegraphics[width=9.0cm,angle=0]{ms2022-0304fig10-2769.pdf}
\includegraphics[width=9.0cm,angle=0]{ms2022-0304fig10-2770.pdf}
\end{figure}
\clearpage

\begin{figure}
\includegraphics[width=9.0cm,angle=0]{ms2022-0304fig10-2771.pdf}
\includegraphics[width=9.0cm,angle=0]{ms2022-0304fig10-2772.pdf}
\vspace{-0.5cm}

\includegraphics[width=9.0cm,angle=0]{ms2022-0304fig10-2773.pdf}
\includegraphics[width=9.0cm,angle=0]{ms2022-0304fig10-2774.pdf}
\vspace{-0.5cm}

\includegraphics[width=9.0cm,angle=0]{ms2022-0304fig10-2775.pdf}
\includegraphics[width=9.0cm,angle=0]{ms2022-0304fig10-2776.pdf}
\vspace{-0.5cm}

\includegraphics[width=9.0cm,angle=0]{ms2022-0304fig10-2777.pdf}
\includegraphics[width=9.0cm,angle=0]{ms2022-0304fig10-2778.pdf}
\vspace{-0.5cm}

\includegraphics[width=9.0cm,angle=0]{ms2022-0304fig10-2779.pdf}
\includegraphics[width=9.0cm,angle=0]{ms2022-0304fig10-2780.pdf}
\end{figure}
\clearpage

\begin{figure}
\includegraphics[width=9.0cm,angle=0]{ms2022-0304fig10-2781.pdf}
\includegraphics[width=9.0cm,angle=0]{ms2022-0304fig10-2782.pdf}
\vspace{-0.5cm}

\includegraphics[width=9.0cm,angle=0]{ms2022-0304fig10-2783.pdf}
\includegraphics[width=9.0cm,angle=0]{ms2022-0304fig10-2784.pdf}
\vspace{-0.5cm}

\includegraphics[width=9.0cm,angle=0]{ms2022-0304fig10-2785.pdf}
\includegraphics[width=9.0cm,angle=0]{ms2022-0304fig10-2786.pdf}
\vspace{-0.5cm}

\includegraphics[width=9.0cm,angle=0]{ms2022-0304fig10-2787.pdf}
\includegraphics[width=9.0cm,angle=0]{ms2022-0304fig10-2788.pdf}
\vspace{-0.5cm}

\includegraphics[width=9.0cm,angle=0]{ms2022-0304fig10-2789.pdf}
\includegraphics[width=9.0cm,angle=0]{ms2022-0304fig10-2790.pdf}
\end{figure}
\clearpage

\begin{figure}
\includegraphics[width=9.0cm,angle=0]{ms2022-0304fig10-2791.pdf}
\includegraphics[width=9.0cm,angle=0]{ms2022-0304fig10-2792.pdf}
\vspace{-0.5cm}

\includegraphics[width=9.0cm,angle=0]{ms2022-0304fig10-2793.pdf}
\includegraphics[width=9.0cm,angle=0]{ms2022-0304fig10-2794.pdf}
\vspace{-0.5cm}

\includegraphics[width=9.0cm,angle=0]{ms2022-0304fig10-2795.pdf}
\includegraphics[width=9.0cm,angle=0]{ms2022-0304fig10-2796.pdf}
\vspace{-0.5cm}

\includegraphics[width=9.0cm,angle=0]{ms2022-0304fig10-2797.pdf}
\includegraphics[width=9.0cm,angle=0]{ms2022-0304fig10-2798.pdf}
\vspace{-0.5cm}

\includegraphics[width=9.0cm,angle=0]{ms2022-0304fig10-2799.pdf}
\includegraphics[width=9.0cm,angle=0]{ms2022-0304fig10-2800.pdf}
\end{figure}
\clearpage

\begin{figure}
\includegraphics[width=9.0cm,angle=0]{ms2022-0304fig10-2801.pdf}
\includegraphics[width=9.0cm,angle=0]{ms2022-0304fig10-2802.pdf}
\vspace{-0.5cm}

\includegraphics[width=9.0cm,angle=0]{ms2022-0304fig10-2803.pdf}
\includegraphics[width=9.0cm,angle=0]{ms2022-0304fig10-2804.pdf}
\vspace{-0.5cm}

\includegraphics[width=9.0cm,angle=0]{ms2022-0304fig10-2805.pdf}
\includegraphics[width=9.0cm,angle=0]{ms2022-0304fig10-2806.pdf}
\vspace{-0.5cm}

\includegraphics[width=9.0cm,angle=0]{ms2022-0304fig10-2807.pdf}
\includegraphics[width=9.0cm,angle=0]{ms2022-0304fig10-2808.pdf}
\vspace{-0.5cm}

\includegraphics[width=9.0cm,angle=0]{ms2022-0304fig10-2809.pdf}
\includegraphics[width=9.0cm,angle=0]{ms2022-0304fig10-2810.pdf}
\end{figure}
\clearpage

\begin{figure}
\includegraphics[width=9.0cm,angle=0]{ms2022-0304fig10-2811.pdf}
\includegraphics[width=9.0cm,angle=0]{ms2022-0304fig10-2812.pdf}
\vspace{-0.5cm}

\includegraphics[width=9.0cm,angle=0]{ms2022-0304fig10-2813.pdf}
\includegraphics[width=9.0cm,angle=0]{ms2022-0304fig10-2814.pdf}
\vspace{-0.5cm}

\includegraphics[width=9.0cm,angle=0]{ms2022-0304fig10-2815.pdf}
\includegraphics[width=9.0cm,angle=0]{ms2022-0304fig10-2816.pdf}
\vspace{-0.5cm}

\includegraphics[width=9.0cm,angle=0]{ms2022-0304fig10-2817.pdf}
\includegraphics[width=9.0cm,angle=0]{ms2022-0304fig10-2818.pdf}
\vspace{-0.5cm}

\includegraphics[width=9.0cm,angle=0]{ms2022-0304fig10-2819.pdf}
\includegraphics[width=9.0cm,angle=0]{ms2022-0304fig10-2820.pdf}
\end{figure}
\clearpage

\begin{figure}
\includegraphics[width=9.0cm,angle=0]{ms2022-0304fig10-2821.pdf}
\includegraphics[width=9.0cm,angle=0]{ms2022-0304fig10-2822.pdf}
\vspace{-0.5cm}

\includegraphics[width=9.0cm,angle=0]{ms2022-0304fig10-2823.pdf}
\includegraphics[width=9.0cm,angle=0]{ms2022-0304fig10-2824.pdf}
\vspace{-0.5cm}

\includegraphics[width=9.0cm,angle=0]{ms2022-0304fig10-2825.pdf}
\includegraphics[width=9.0cm,angle=0]{ms2022-0304fig10-2826.pdf}
\vspace{-0.5cm}

\includegraphics[width=9.0cm,angle=0]{ms2022-0304fig10-2827.pdf}
\includegraphics[width=9.0cm,angle=0]{ms2022-0304fig10-2828.pdf}
\vspace{-0.5cm}

\includegraphics[width=9.0cm,angle=0]{ms2022-0304fig10-2829.pdf}
\includegraphics[width=9.0cm,angle=0]{ms2022-0304fig10-2830.pdf}
\end{figure}
\clearpage

\begin{figure}
\includegraphics[width=9.0cm,angle=0]{ms2022-0304fig10-2831.pdf}
\includegraphics[width=9.0cm,angle=0]{ms2022-0304fig10-2832.pdf}
\vspace{-0.5cm}

\includegraphics[width=9.0cm,angle=0]{ms2022-0304fig10-2833.pdf}
\includegraphics[width=9.0cm,angle=0]{ms2022-0304fig10-2834.pdf}
\vspace{-0.5cm}

\includegraphics[width=9.0cm,angle=0]{ms2022-0304fig10-2835.pdf}
\includegraphics[width=9.0cm,angle=0]{ms2022-0304fig10-2836.pdf}
\vspace{-0.5cm}

\includegraphics[width=9.0cm,angle=0]{ms2022-0304fig10-2837.pdf}
\includegraphics[width=9.0cm,angle=0]{ms2022-0304fig10-2838.pdf}
\vspace{-0.5cm}

\includegraphics[width=9.0cm,angle=0]{ms2022-0304fig10-2839.pdf}
\includegraphics[width=9.0cm,angle=0]{ms2022-0304fig10-2840.pdf}
\end{figure}
\clearpage

\begin{figure}
\includegraphics[width=9.0cm,angle=0]{ms2022-0304fig10-2841.pdf}
\includegraphics[width=9.0cm,angle=0]{ms2022-0304fig10-2842.pdf}
\vspace{-0.5cm}

\includegraphics[width=9.0cm,angle=0]{ms2022-0304fig10-2843.pdf}
\includegraphics[width=9.0cm,angle=0]{ms2022-0304fig10-2844.pdf}
\vspace{-0.5cm}

\includegraphics[width=9.0cm,angle=0]{ms2022-0304fig10-2845.pdf}
\includegraphics[width=9.0cm,angle=0]{ms2022-0304fig10-2846.pdf}
\vspace{-0.5cm}

\includegraphics[width=9.0cm,angle=0]{ms2022-0304fig10-2847.pdf}
\includegraphics[width=9.0cm,angle=0]{ms2022-0304fig10-2848.pdf}
\vspace{-0.5cm}

\includegraphics[width=9.0cm,angle=0]{ms2022-0304fig10-2849.pdf}
\includegraphics[width=9.0cm,angle=0]{ms2022-0304fig10-2850.pdf}
\end{figure}
\clearpage

\begin{figure}
\includegraphics[width=9.0cm,angle=0]{ms2022-0304fig10-2851.pdf}
\includegraphics[width=9.0cm,angle=0]{ms2022-0304fig10-2852.pdf}
\vspace{-0.5cm}

\includegraphics[width=9.0cm,angle=0]{ms2022-0304fig10-2853.pdf}
\includegraphics[width=9.0cm,angle=0]{ms2022-0304fig10-2854.pdf}
\vspace{-0.5cm}

\includegraphics[width=9.0cm,angle=0]{ms2022-0304fig10-2855.pdf}
\includegraphics[width=9.0cm,angle=0]{ms2022-0304fig10-2856.pdf}
\vspace{-0.5cm}

\includegraphics[width=9.0cm,angle=0]{ms2022-0304fig10-2857.pdf}
\includegraphics[width=9.0cm,angle=0]{ms2022-0304fig10-2858.pdf}
\vspace{-0.5cm}

\includegraphics[width=9.0cm,angle=0]{ms2022-0304fig10-2859.pdf}
\includegraphics[width=9.0cm,angle=0]{ms2022-0304fig10-2860.pdf}
\end{figure}
\clearpage

\begin{figure}
\includegraphics[width=9.0cm,angle=0]{ms2022-0304fig10-2861.pdf}
\includegraphics[width=9.0cm,angle=0]{ms2022-0304fig10-2862.pdf}
\vspace{-0.5cm}

\includegraphics[width=9.0cm,angle=0]{ms2022-0304fig10-2863.pdf}
\includegraphics[width=9.0cm,angle=0]{ms2022-0304fig10-2864.pdf}
\vspace{-0.5cm}

\includegraphics[width=9.0cm,angle=0]{ms2022-0304fig10-2865.pdf}
\includegraphics[width=9.0cm,angle=0]{ms2022-0304fig10-2866.pdf}
\vspace{-0.5cm}

\includegraphics[width=9.0cm,angle=0]{ms2022-0304fig10-2867.pdf}
\includegraphics[width=9.0cm,angle=0]{ms2022-0304fig10-2868.pdf}
\vspace{-0.5cm}

\includegraphics[width=9.0cm,angle=0]{ms2022-0304fig10-2869.pdf}
\includegraphics[width=9.0cm,angle=0]{ms2022-0304fig10-2870.pdf}
\end{figure}
\clearpage

\begin{figure}
\includegraphics[width=9.0cm,angle=0]{ms2022-0304fig10-2871.pdf}
\includegraphics[width=9.0cm,angle=0]{ms2022-0304fig10-2872.pdf}
\vspace{-0.5cm}

\includegraphics[width=9.0cm,angle=0]{ms2022-0304fig10-2873.pdf}
\includegraphics[width=9.0cm,angle=0]{ms2022-0304fig10-2874.pdf}
\vspace{-0.5cm}

\includegraphics[width=9.0cm,angle=0]{ms2022-0304fig10-2875.pdf}
\includegraphics[width=9.0cm,angle=0]{ms2022-0304fig10-2876.pdf}
\vspace{-0.5cm}

\includegraphics[width=9.0cm,angle=0]{ms2022-0304fig10-2877.pdf}
\includegraphics[width=9.0cm,angle=0]{ms2022-0304fig10-2878.pdf}
\vspace{-0.5cm}

\includegraphics[width=9.0cm,angle=0]{ms2022-0304fig10-2879.pdf}
\includegraphics[width=9.0cm,angle=0]{ms2022-0304fig10-2880.pdf}
\end{figure}
\clearpage

\begin{figure}
\includegraphics[width=9.0cm,angle=0]{ms2022-0304fig10-2881.pdf}
\includegraphics[width=9.0cm,angle=0]{ms2022-0304fig10-2882.pdf}
\vspace{-0.5cm}

\includegraphics[width=9.0cm,angle=0]{ms2022-0304fig10-2883.pdf}
\includegraphics[width=9.0cm,angle=0]{ms2022-0304fig10-2884.pdf}
\vspace{-0.5cm}

\includegraphics[width=9.0cm,angle=0]{ms2022-0304fig10-2885.pdf}
\includegraphics[width=9.0cm,angle=0]{ms2022-0304fig10-2886.pdf}
\vspace{-0.5cm}

\includegraphics[width=9.0cm,angle=0]{ms2022-0304fig10-2887.pdf}
\includegraphics[width=9.0cm,angle=0]{ms2022-0304fig10-2888.pdf}
\vspace{-0.5cm}

\includegraphics[width=9.0cm,angle=0]{ms2022-0304fig10-2889.pdf}
\includegraphics[width=9.0cm,angle=0]{ms2022-0304fig10-2890.pdf}
\end{figure}
\clearpage

\begin{figure}
\includegraphics[width=9.0cm,angle=0]{ms2022-0304fig10-2891.pdf}
\includegraphics[width=9.0cm,angle=0]{ms2022-0304fig10-2892.pdf}
\vspace{-0.5cm}

\includegraphics[width=9.0cm,angle=0]{ms2022-0304fig10-2893.pdf}
\includegraphics[width=9.0cm,angle=0]{ms2022-0304fig10-2894.pdf}
\vspace{-0.5cm}

\includegraphics[width=9.0cm,angle=0]{ms2022-0304fig10-2895.pdf}
\includegraphics[width=9.0cm,angle=0]{ms2022-0304fig10-2896.pdf}
\vspace{-0.5cm}

\includegraphics[width=9.0cm,angle=0]{ms2022-0304fig10-2897.pdf}
\includegraphics[width=9.0cm,angle=0]{ms2022-0304fig10-2898.pdf}
\vspace{-0.5cm}

\includegraphics[width=9.0cm,angle=0]{ms2022-0304fig10-2899.pdf}
\includegraphics[width=9.0cm,angle=0]{ms2022-0304fig10-2900.pdf}
\end{figure}
\clearpage

\begin{figure}
\includegraphics[width=9.0cm,angle=0]{ms2022-0304fig10-2901.pdf}
\includegraphics[width=9.0cm,angle=0]{ms2022-0304fig10-2902.pdf}
\vspace{-0.5cm}

\includegraphics[width=9.0cm,angle=0]{ms2022-0304fig10-2903.pdf}
\includegraphics[width=9.0cm,angle=0]{ms2022-0304fig10-2904.pdf}
\vspace{-0.5cm}

\includegraphics[width=9.0cm,angle=0]{ms2022-0304fig10-2905.pdf}
\includegraphics[width=9.0cm,angle=0]{ms2022-0304fig10-2906.pdf}
\vspace{-0.5cm}

\includegraphics[width=9.0cm,angle=0]{ms2022-0304fig10-2907.pdf}
\includegraphics[width=9.0cm,angle=0]{ms2022-0304fig10-2908.pdf}
\vspace{-0.5cm}

\includegraphics[width=9.0cm,angle=0]{ms2022-0304fig10-2909.pdf}
\includegraphics[width=9.0cm,angle=0]{ms2022-0304fig10-2910.pdf}
\end{figure}
\clearpage

\begin{figure}
\includegraphics[width=9.0cm,angle=0]{ms2022-0304fig10-2911.pdf}
\includegraphics[width=9.0cm,angle=0]{ms2022-0304fig10-2912.pdf}
\vspace{-0.5cm}

\includegraphics[width=9.0cm,angle=0]{ms2022-0304fig10-2913.pdf}
\includegraphics[width=9.0cm,angle=0]{ms2022-0304fig10-2914.pdf}
\vspace{-0.5cm}

\includegraphics[width=9.0cm,angle=0]{ms2022-0304fig10-2915.pdf}
\includegraphics[width=9.0cm,angle=0]{ms2022-0304fig10-2916.pdf}
\vspace{-0.5cm}

\includegraphics[width=9.0cm,angle=0]{ms2022-0304fig10-2917.pdf}
\includegraphics[width=9.0cm,angle=0]{ms2022-0304fig10-2918.pdf}
\vspace{-0.5cm}

\includegraphics[width=9.0cm,angle=0]{ms2022-0304fig10-2919.pdf}
\includegraphics[width=9.0cm,angle=0]{ms2022-0304fig10-2920.pdf}
\end{figure}
\clearpage

\begin{figure}
\includegraphics[width=9.0cm,angle=0]{ms2022-0304fig10-2921.pdf}
\includegraphics[width=9.0cm,angle=0]{ms2022-0304fig10-2922.pdf}
\vspace{-0.5cm}

\includegraphics[width=9.0cm,angle=0]{ms2022-0304fig10-2923.pdf}
\includegraphics[width=9.0cm,angle=0]{ms2022-0304fig10-2924.pdf}
\vspace{-0.5cm}

\includegraphics[width=9.0cm,angle=0]{ms2022-0304fig10-2925.pdf}
\includegraphics[width=9.0cm,angle=0]{ms2022-0304fig10-2926.pdf}
\vspace{-0.5cm}

\includegraphics[width=9.0cm,angle=0]{ms2022-0304fig10-2927.pdf}
\includegraphics[width=9.0cm,angle=0]{ms2022-0304fig10-2928.pdf}
\vspace{-0.5cm}

\includegraphics[width=9.0cm,angle=0]{ms2022-0304fig10-2929.pdf}
\includegraphics[width=9.0cm,angle=0]{ms2022-0304fig10-2930.pdf}
\end{figure}
\clearpage

\begin{figure}
\includegraphics[width=9.0cm,angle=0]{ms2022-0304fig10-2931.pdf}
\includegraphics[width=9.0cm,angle=0]{ms2022-0304fig10-2932.pdf}
\vspace{-0.5cm}

\includegraphics[width=9.0cm,angle=0]{ms2022-0304fig10-2933.pdf}
\includegraphics[width=9.0cm,angle=0]{ms2022-0304fig10-2934.pdf}
\vspace{-0.5cm}

\includegraphics[width=9.0cm,angle=0]{ms2022-0304fig10-2935.pdf}
\includegraphics[width=9.0cm,angle=0]{ms2022-0304fig10-2936.pdf}
\vspace{-0.5cm}

\includegraphics[width=9.0cm,angle=0]{ms2022-0304fig10-2937.pdf}
\includegraphics[width=9.0cm,angle=0]{ms2022-0304fig10-2938.pdf}
\vspace{-0.5cm}

\includegraphics[width=9.0cm,angle=0]{ms2022-0304fig10-2939.pdf}
\includegraphics[width=9.0cm,angle=0]{ms2022-0304fig10-2940.pdf}
\end{figure}
\clearpage

\begin{figure}
\includegraphics[width=9.0cm,angle=0]{ms2022-0304fig10-2941.pdf}
\includegraphics[width=9.0cm,angle=0]{ms2022-0304fig10-2942.pdf}
\vspace{-0.5cm}

\includegraphics[width=9.0cm,angle=0]{ms2022-0304fig10-2943.pdf}
\includegraphics[width=9.0cm,angle=0]{ms2022-0304fig10-2944.pdf}
\vspace{-0.5cm}

\includegraphics[width=9.0cm,angle=0]{ms2022-0304fig10-2945.pdf}
\includegraphics[width=9.0cm,angle=0]{ms2022-0304fig10-2946.pdf}
\vspace{-0.5cm}

\includegraphics[width=9.0cm,angle=0]{ms2022-0304fig10-2947.pdf}
\includegraphics[width=9.0cm,angle=0]{ms2022-0304fig10-2948.pdf}
\vspace{-0.5cm}

\includegraphics[width=9.0cm,angle=0]{ms2022-0304fig10-2949.pdf}
\includegraphics[width=9.0cm,angle=0]{ms2022-0304fig10-2950.pdf}
\end{figure}
\clearpage

\begin{figure}
\includegraphics[width=9.0cm,angle=0]{ms2022-0304fig10-2951.pdf}
\includegraphics[width=9.0cm,angle=0]{ms2022-0304fig10-2952.pdf}
\vspace{-0.5cm}

\includegraphics[width=9.0cm,angle=0]{ms2022-0304fig10-2953.pdf}
\includegraphics[width=9.0cm,angle=0]{ms2022-0304fig10-2954.pdf}
\vspace{-0.5cm}

\includegraphics[width=9.0cm,angle=0]{ms2022-0304fig10-2955.pdf}
\includegraphics[width=9.0cm,angle=0]{ms2022-0304fig10-2956.pdf}
\vspace{-0.5cm}

\includegraphics[width=9.0cm,angle=0]{ms2022-0304fig10-2957.pdf}
\includegraphics[width=9.0cm,angle=0]{ms2022-0304fig10-2958.pdf}
\vspace{-0.5cm}

\includegraphics[width=9.0cm,angle=0]{ms2022-0304fig10-2959.pdf}
\includegraphics[width=9.0cm,angle=0]{ms2022-0304fig10-2960.pdf}
\end{figure}
\clearpage

\begin{figure}
\includegraphics[width=9.0cm,angle=0]{ms2022-0304fig10-2961.pdf}
\includegraphics[width=9.0cm,angle=0]{ms2022-0304fig10-2962.pdf}
\vspace{-0.5cm}

\includegraphics[width=9.0cm,angle=0]{ms2022-0304fig10-2963.pdf}
\includegraphics[width=9.0cm,angle=0]{ms2022-0304fig10-2964.pdf}
\vspace{-0.5cm}

\includegraphics[width=9.0cm,angle=0]{ms2022-0304fig10-2965.pdf}
\includegraphics[width=9.0cm,angle=0]{ms2022-0304fig10-2966.pdf}
\vspace{-0.5cm}

\includegraphics[width=9.0cm,angle=0]{ms2022-0304fig10-2967.pdf}
\includegraphics[width=9.0cm,angle=0]{ms2022-0304fig10-2968.pdf}
\vspace{-0.5cm}

\includegraphics[width=9.0cm,angle=0]{ms2022-0304fig10-2969.pdf}
\includegraphics[width=9.0cm,angle=0]{ms2022-0304fig10-2970.pdf}
\end{figure}
\clearpage

\begin{figure}
\includegraphics[width=9.0cm,angle=0]{ms2022-0304fig10-2971.pdf}
\includegraphics[width=9.0cm,angle=0]{ms2022-0304fig10-2972.pdf}
\vspace{-0.5cm}

\includegraphics[width=9.0cm,angle=0]{ms2022-0304fig10-2973.pdf}
\includegraphics[width=9.0cm,angle=0]{ms2022-0304fig10-2974.pdf}
\vspace{-0.5cm}

\includegraphics[width=9.0cm,angle=0]{ms2022-0304fig10-2975.pdf}
\includegraphics[width=9.0cm,angle=0]{ms2022-0304fig10-2976.pdf}
\vspace{-0.5cm}

\includegraphics[width=9.0cm,angle=0]{ms2022-0304fig10-2977.pdf}
\includegraphics[width=9.0cm,angle=0]{ms2022-0304fig10-2978.pdf}
\vspace{-0.5cm}

\includegraphics[width=9.0cm,angle=0]{ms2022-0304fig10-2979.pdf}
\includegraphics[width=9.0cm,angle=0]{ms2022-0304fig10-2980.pdf}
\end{figure}
\clearpage

\begin{figure}
\includegraphics[width=9.0cm,angle=0]{ms2022-0304fig10-2981.pdf}
\includegraphics[width=9.0cm,angle=0]{ms2022-0304fig10-2982.pdf}
\vspace{-0.5cm}

\includegraphics[width=9.0cm,angle=0]{ms2022-0304fig10-2983.pdf}
\includegraphics[width=9.0cm,angle=0]{ms2022-0304fig10-2984.pdf}
\vspace{-0.5cm}

\includegraphics[width=9.0cm,angle=0]{ms2022-0304fig10-2985.pdf}
\includegraphics[width=9.0cm,angle=0]{ms2022-0304fig10-2986.pdf}
\vspace{-0.5cm}

\includegraphics[width=9.0cm,angle=0]{ms2022-0304fig10-2987.pdf}
\includegraphics[width=9.0cm,angle=0]{ms2022-0304fig10-2988.pdf}
\vspace{-0.5cm}

\includegraphics[width=9.0cm,angle=0]{ms2022-0304fig10-2989.pdf}
\includegraphics[width=9.0cm,angle=0]{ms2022-0304fig10-2990.pdf}
\end{figure}
\clearpage

\begin{figure}
\includegraphics[width=9.0cm,angle=0]{ms2022-0304fig10-2991.pdf}
\includegraphics[width=9.0cm,angle=0]{ms2022-0304fig10-2992.pdf}
\vspace{-0.5cm}

\includegraphics[width=9.0cm,angle=0]{ms2022-0304fig10-2993.pdf}
\includegraphics[width=9.0cm,angle=0]{ms2022-0304fig10-2994.pdf}
\vspace{-0.5cm}

\includegraphics[width=9.0cm,angle=0]{ms2022-0304fig10-2995.pdf}
\includegraphics[width=9.0cm,angle=0]{ms2022-0304fig10-2996.pdf}
\vspace{-0.5cm}

\includegraphics[width=9.0cm,angle=0]{ms2022-0304fig10-2997.pdf}
\includegraphics[width=9.0cm,angle=0]{ms2022-0304fig10-2998.pdf}
\vspace{-0.5cm}

\includegraphics[width=9.0cm,angle=0]{ms2022-0304fig10-2999.pdf}
\includegraphics[width=9.0cm,angle=0]{ms2022-0304fig10-3000.pdf}
\end{figure}
\clearpage

\begin{figure}
\includegraphics[width=9.0cm,angle=0]{ms2022-0304fig10-3001.pdf}
\includegraphics[width=9.0cm,angle=0]{ms2022-0304fig10-3002.pdf}
\vspace{-0.5cm}

\includegraphics[width=9.0cm,angle=0]{ms2022-0304fig10-3003.pdf}
\includegraphics[width=9.0cm,angle=0]{ms2022-0304fig10-3004.pdf}
\vspace{-0.5cm}

\includegraphics[width=9.0cm,angle=0]{ms2022-0304fig10-3005.pdf}
\includegraphics[width=9.0cm,angle=0]{ms2022-0304fig10-3006.pdf}
\vspace{-0.5cm}

\includegraphics[width=9.0cm,angle=0]{ms2022-0304fig10-3007.pdf}
\includegraphics[width=9.0cm,angle=0]{ms2022-0304fig10-3008.pdf}
\vspace{-0.5cm}

\includegraphics[width=9.0cm,angle=0]{ms2022-0304fig10-3009.pdf}
\includegraphics[width=9.0cm,angle=0]{ms2022-0304fig10-3010.pdf}
\end{figure}
\clearpage

\begin{figure}
\includegraphics[width=9.0cm,angle=0]{ms2022-0304fig10-3011.pdf}
\includegraphics[width=9.0cm,angle=0]{ms2022-0304fig10-3012.pdf}
\vspace{-0.5cm}

\includegraphics[width=9.0cm,angle=0]{ms2022-0304fig10-3013.pdf}
\includegraphics[width=9.0cm,angle=0]{ms2022-0304fig10-3014.pdf}
\vspace{-0.5cm}

\includegraphics[width=9.0cm,angle=0]{ms2022-0304fig10-3015.pdf}
\includegraphics[width=9.0cm,angle=0]{ms2022-0304fig10-3016.pdf}
\vspace{-0.5cm}

\includegraphics[width=9.0cm,angle=0]{ms2022-0304fig10-3017.pdf}
\includegraphics[width=9.0cm,angle=0]{ms2022-0304fig10-3018.pdf}
\vspace{-0.5cm}

\includegraphics[width=9.0cm,angle=0]{ms2022-0304fig10-3019.pdf}
\includegraphics[width=9.0cm,angle=0]{ms2022-0304fig10-3020.pdf}
\end{figure}
\clearpage

\begin{figure}
\includegraphics[width=9.0cm,angle=0]{ms2022-0304fig10-3021.pdf}
\includegraphics[width=9.0cm,angle=0]{ms2022-0304fig10-3022.pdf}
\vspace{-0.5cm}

\includegraphics[width=9.0cm,angle=0]{ms2022-0304fig10-3023.pdf}
\includegraphics[width=9.0cm,angle=0]{ms2022-0304fig10-3024.pdf}
\vspace{-0.5cm}

\includegraphics[width=9.0cm,angle=0]{ms2022-0304fig10-3025.pdf}
\includegraphics[width=9.0cm,angle=0]{ms2022-0304fig10-3026.pdf}
\vspace{-0.5cm}

\includegraphics[width=9.0cm,angle=0]{ms2022-0304fig10-3027.pdf}
\includegraphics[width=9.0cm,angle=0]{ms2022-0304fig10-3028.pdf}
\vspace{-0.5cm}

\includegraphics[width=9.0cm,angle=0]{ms2022-0304fig10-3029.pdf}
\includegraphics[width=9.0cm,angle=0]{ms2022-0304fig10-3030.pdf}
\end{figure}
\clearpage

\begin{figure}
\includegraphics[width=9.0cm,angle=0]{ms2022-0304fig10-3031.pdf}
\includegraphics[width=9.0cm,angle=0]{ms2022-0304fig10-3032.pdf}
\vspace{-0.5cm}

\includegraphics[width=9.0cm,angle=0]{ms2022-0304fig10-3033.pdf}
\includegraphics[width=9.0cm,angle=0]{ms2022-0304fig10-3034.pdf}
\vspace{-0.5cm}

\includegraphics[width=9.0cm,angle=0]{ms2022-0304fig10-3035.pdf}
\includegraphics[width=9.0cm,angle=0]{ms2022-0304fig10-3036.pdf}
\vspace{-0.5cm}

\includegraphics[width=9.0cm,angle=0]{ms2022-0304fig10-3037.pdf}
\includegraphics[width=9.0cm,angle=0]{ms2022-0304fig10-3038.pdf}
\vspace{-0.5cm}

\includegraphics[width=9.0cm,angle=0]{ms2022-0304fig10-3039.pdf}
\includegraphics[width=9.0cm,angle=0]{ms2022-0304fig10-3040.pdf}
\end{figure}
\clearpage

\begin{figure}
\includegraphics[width=9.0cm,angle=0]{ms2022-0304fig10-3041.pdf}
\includegraphics[width=9.0cm,angle=0]{ms2022-0304fig10-3042.pdf}
\vspace{-0.5cm}

\includegraphics[width=9.0cm,angle=0]{ms2022-0304fig10-3043.pdf}
\includegraphics[width=9.0cm,angle=0]{ms2022-0304fig10-3044.pdf}
\vspace{-0.5cm}

\includegraphics[width=9.0cm,angle=0]{ms2022-0304fig10-3045.pdf}
\includegraphics[width=9.0cm,angle=0]{ms2022-0304fig10-3046.pdf}
\vspace{-0.5cm}

\includegraphics[width=9.0cm,angle=0]{ms2022-0304fig10-3047.pdf}
\includegraphics[width=9.0cm,angle=0]{ms2022-0304fig10-3048.pdf}
\vspace{-0.5cm}

\includegraphics[width=9.0cm,angle=0]{ms2022-0304fig10-3049.pdf}
\includegraphics[width=9.0cm,angle=0]{ms2022-0304fig10-3050.pdf}
\end{figure}
\clearpage

\begin{figure}
\includegraphics[width=9.0cm,angle=0]{ms2022-0304fig10-3051.pdf}
\includegraphics[width=9.0cm,angle=0]{ms2022-0304fig10-3052.pdf}
\vspace{-0.5cm}

\includegraphics[width=9.0cm,angle=0]{ms2022-0304fig10-3053.pdf}
\includegraphics[width=9.0cm,angle=0]{ms2022-0304fig10-3054.pdf}
\vspace{-0.5cm}

\includegraphics[width=9.0cm,angle=0]{ms2022-0304fig10-3055.pdf}
\includegraphics[width=9.0cm,angle=0]{ms2022-0304fig10-3056.pdf}
\vspace{-0.5cm}

\includegraphics[width=9.0cm,angle=0]{ms2022-0304fig10-3057.pdf}
\includegraphics[width=9.0cm,angle=0]{ms2022-0304fig10-3058.pdf}
\vspace{-0.5cm}

\includegraphics[width=9.0cm,angle=0]{ms2022-0304fig10-3059.pdf}
\includegraphics[width=9.0cm,angle=0]{ms2022-0304fig10-3060.pdf}
\end{figure}
\clearpage

\begin{figure}
\includegraphics[width=9.0cm,angle=0]{ms2022-0304fig10-3061.pdf}
\includegraphics[width=9.0cm,angle=0]{ms2022-0304fig10-3062.pdf}
\vspace{-0.5cm}

\includegraphics[width=9.0cm,angle=0]{ms2022-0304fig10-3063.pdf}
\includegraphics[width=9.0cm,angle=0]{ms2022-0304fig10-3064.pdf}
\vspace{-0.5cm}

\includegraphics[width=9.0cm,angle=0]{ms2022-0304fig10-3065.pdf}
\includegraphics[width=9.0cm,angle=0]{ms2022-0304fig10-3066.pdf}
\vspace{-0.5cm}

\includegraphics[width=9.0cm,angle=0]{ms2022-0304fig10-3067.pdf}
\includegraphics[width=9.0cm,angle=0]{ms2022-0304fig10-3068.pdf}
\vspace{-0.5cm}

\includegraphics[width=9.0cm,angle=0]{ms2022-0304fig10-3069.pdf}
\includegraphics[width=9.0cm,angle=0]{ms2022-0304fig10-3070.pdf}
\end{figure}
\clearpage

\begin{figure}
\includegraphics[width=9.0cm,angle=0]{ms2022-0304fig10-3071.pdf}
\includegraphics[width=9.0cm,angle=0]{ms2022-0304fig10-3072.pdf}
\vspace{-0.5cm}

\includegraphics[width=9.0cm,angle=0]{ms2022-0304fig10-3073.pdf}
\includegraphics[width=9.0cm,angle=0]{ms2022-0304fig10-3074.pdf}
\vspace{-0.5cm}

\includegraphics[width=9.0cm,angle=0]{ms2022-0304fig10-3075.pdf}
\includegraphics[width=9.0cm,angle=0]{ms2022-0304fig10-3076.pdf}
\vspace{-0.5cm}

\includegraphics[width=9.0cm,angle=0]{ms2022-0304fig10-3077.pdf}
\includegraphics[width=9.0cm,angle=0]{ms2022-0304fig10-3078.pdf}
\vspace{-0.5cm}

\includegraphics[width=9.0cm,angle=0]{ms2022-0304fig10-3079.pdf}
\includegraphics[width=9.0cm,angle=0]{ms2022-0304fig10-3080.pdf}
\end{figure}
\clearpage

\begin{figure}
\includegraphics[width=9.0cm,angle=0]{ms2022-0304fig10-3081.pdf}
\includegraphics[width=9.0cm,angle=0]{ms2022-0304fig10-3082.pdf}
\vspace{-0.5cm}

\includegraphics[width=9.0cm,angle=0]{ms2022-0304fig10-3083.pdf}
\includegraphics[width=9.0cm,angle=0]{ms2022-0304fig10-3084.pdf}
\vspace{-0.5cm}

\includegraphics[width=9.0cm,angle=0]{ms2022-0304fig10-3085.pdf}
\includegraphics[width=9.0cm,angle=0]{ms2022-0304fig10-3086.pdf}
\vspace{-0.5cm}

\includegraphics[width=9.0cm,angle=0]{ms2022-0304fig10-3087.pdf}
\includegraphics[width=9.0cm,angle=0]{ms2022-0304fig10-3088.pdf}
\vspace{-0.5cm}

\includegraphics[width=9.0cm,angle=0]{ms2022-0304fig10-3089.pdf}
\includegraphics[width=9.0cm,angle=0]{ms2022-0304fig10-3090.pdf}
\end{figure}
\clearpage

\begin{figure}
\includegraphics[width=9.0cm,angle=0]{ms2022-0304fig10-3091.pdf}
\includegraphics[width=9.0cm,angle=0]{ms2022-0304fig10-3092.pdf}
\vspace{-0.5cm}

\includegraphics[width=9.0cm,angle=0]{ms2022-0304fig10-3093.pdf}
\includegraphics[width=9.0cm,angle=0]{ms2022-0304fig10-3094.pdf}
\vspace{-0.5cm}

\includegraphics[width=9.0cm,angle=0]{ms2022-0304fig10-3095.pdf}
\includegraphics[width=9.0cm,angle=0]{ms2022-0304fig10-3096.pdf}
\vspace{-0.5cm}

\includegraphics[width=9.0cm,angle=0]{ms2022-0304fig10-3097.pdf}
\includegraphics[width=9.0cm,angle=0]{ms2022-0304fig10-3098.pdf}
\vspace{-0.5cm}

\includegraphics[width=9.0cm,angle=0]{ms2022-0304fig10-3099.pdf}
\includegraphics[width=9.0cm,angle=0]{ms2022-0304fig10-3100.pdf}
\end{figure}
\clearpage

\begin{figure}
\includegraphics[width=9.0cm,angle=0]{ms2022-0304fig10-3101.pdf}
\includegraphics[width=9.0cm,angle=0]{ms2022-0304fig10-3102.pdf}
\vspace{-0.5cm}

\includegraphics[width=9.0cm,angle=0]{ms2022-0304fig10-3103.pdf}
\includegraphics[width=9.0cm,angle=0]{ms2022-0304fig10-3104.pdf}
\vspace{-0.5cm}

\includegraphics[width=9.0cm,angle=0]{ms2022-0304fig10-3105.pdf}
\includegraphics[width=9.0cm,angle=0]{ms2022-0304fig10-3106.pdf}
\vspace{-0.5cm}

\includegraphics[width=9.0cm,angle=0]{ms2022-0304fig10-3107.pdf}
\includegraphics[width=9.0cm,angle=0]{ms2022-0304fig10-3108.pdf}
\vspace{-0.5cm}

\includegraphics[width=9.0cm,angle=0]{ms2022-0304fig10-3109.pdf}
\includegraphics[width=9.0cm,angle=0]{ms2022-0304fig10-3110.pdf}
\end{figure}
\clearpage

\begin{figure}
\includegraphics[width=9.0cm,angle=0]{ms2022-0304fig10-3111.pdf}
\includegraphics[width=9.0cm,angle=0]{ms2022-0304fig10-3112.pdf}
\vspace{-0.5cm}

\includegraphics[width=9.0cm,angle=0]{ms2022-0304fig10-3113.pdf}
\includegraphics[width=9.0cm,angle=0]{ms2022-0304fig10-3114.pdf}
\vspace{-0.5cm}

\includegraphics[width=9.0cm,angle=0]{ms2022-0304fig10-3115.pdf}
\includegraphics[width=9.0cm,angle=0]{ms2022-0304fig10-3116.pdf}
\vspace{-0.5cm}

\includegraphics[width=9.0cm,angle=0]{ms2022-0304fig10-3117.pdf}
\includegraphics[width=9.0cm,angle=0]{ms2022-0304fig10-3118.pdf}
\vspace{-0.5cm}

\includegraphics[width=9.0cm,angle=0]{ms2022-0304fig10-3119.pdf}
\includegraphics[width=9.0cm,angle=0]{ms2022-0304fig10-3120.pdf}
\end{figure}
\clearpage

\begin{figure}
\includegraphics[width=9.0cm,angle=0]{ms2022-0304fig10-3121.pdf}
\includegraphics[width=9.0cm,angle=0]{ms2022-0304fig10-3122.pdf}
\vspace{-0.5cm}

\includegraphics[width=9.0cm,angle=0]{ms2022-0304fig10-3123.pdf}
\includegraphics[width=9.0cm,angle=0]{ms2022-0304fig10-3124.pdf}
\vspace{-0.5cm}

\includegraphics[width=9.0cm,angle=0]{ms2022-0304fig10-3125.pdf}
\includegraphics[width=9.0cm,angle=0]{ms2022-0304fig10-3126.pdf}
\vspace{-0.5cm}

\includegraphics[width=9.0cm,angle=0]{ms2022-0304fig10-3127.pdf}
\includegraphics[width=9.0cm,angle=0]{ms2022-0304fig10-3128.pdf}
\vspace{-0.5cm}

\includegraphics[width=9.0cm,angle=0]{ms2022-0304fig10-3129.pdf}
\includegraphics[width=9.0cm,angle=0]{ms2022-0304fig10-3130.pdf}
\end{figure}
\clearpage

\begin{figure}
\includegraphics[width=9.0cm,angle=0]{ms2022-0304fig10-3131.pdf}
\includegraphics[width=9.0cm,angle=0]{ms2022-0304fig10-3132.pdf}
\vspace{-0.5cm}

\includegraphics[width=9.0cm,angle=0]{ms2022-0304fig10-3133.pdf}
\includegraphics[width=9.0cm,angle=0]{ms2022-0304fig10-3134.pdf}
\vspace{-0.5cm}

\includegraphics[width=9.0cm,angle=0]{ms2022-0304fig10-3135.pdf}
\includegraphics[width=9.0cm,angle=0]{ms2022-0304fig10-3136.pdf}
\vspace{-0.5cm}

\includegraphics[width=9.0cm,angle=0]{ms2022-0304fig10-3137.pdf}
\includegraphics[width=9.0cm,angle=0]{ms2022-0304fig10-3138.pdf}
\vspace{-0.5cm}

\includegraphics[width=9.0cm,angle=0]{ms2022-0304fig10-3139.pdf}
\includegraphics[width=9.0cm,angle=0]{ms2022-0304fig10-3140.pdf}
\end{figure}
\clearpage

\begin{figure}
\includegraphics[width=9.0cm,angle=0]{ms2022-0304fig10-3141.pdf}
\includegraphics[width=9.0cm,angle=0]{ms2022-0304fig10-3142.pdf}
\vspace{-0.5cm}

\includegraphics[width=9.0cm,angle=0]{ms2022-0304fig10-3143.pdf}
\includegraphics[width=9.0cm,angle=0]{ms2022-0304fig10-3144.pdf}
\vspace{-0.5cm}

\includegraphics[width=9.0cm,angle=0]{ms2022-0304fig10-3145.pdf}
\includegraphics[width=9.0cm,angle=0]{ms2022-0304fig10-3146.pdf}
\vspace{-0.5cm}

\includegraphics[width=9.0cm,angle=0]{ms2022-0304fig10-3147.pdf}
\includegraphics[width=9.0cm,angle=0]{ms2022-0304fig10-3148.pdf}
\vspace{-0.5cm}

\includegraphics[width=9.0cm,angle=0]{ms2022-0304fig10-3149.pdf}
\includegraphics[width=9.0cm,angle=0]{ms2022-0304fig10-3150.pdf}
\end{figure}
\clearpage

\begin{figure}
\includegraphics[width=9.0cm,angle=0]{ms2022-0304fig10-3151.pdf}
\includegraphics[width=9.0cm,angle=0]{ms2022-0304fig10-3152.pdf}
\vspace{-0.5cm}

\includegraphics[width=9.0cm,angle=0]{ms2022-0304fig10-3153.pdf}
\includegraphics[width=9.0cm,angle=0]{ms2022-0304fig10-3154.pdf}
\vspace{-0.5cm}

\includegraphics[width=9.0cm,angle=0]{ms2022-0304fig10-3155.pdf}
\includegraphics[width=9.0cm,angle=0]{ms2022-0304fig10-3156.pdf}
\vspace{-0.5cm}

\includegraphics[width=9.0cm,angle=0]{ms2022-0304fig10-3157.pdf}
\includegraphics[width=9.0cm,angle=0]{ms2022-0304fig10-3158.pdf}
\vspace{-0.5cm}

\includegraphics[width=9.0cm,angle=0]{ms2022-0304fig10-3159.pdf}
\includegraphics[width=9.0cm,angle=0]{ms2022-0304fig10-3160.pdf}
\end{figure}
\clearpage

\begin{figure}
\includegraphics[width=9.0cm,angle=0]{ms2022-0304fig10-3161.pdf}
\includegraphics[width=9.0cm,angle=0]{ms2022-0304fig10-3162.pdf}
\vspace{-0.5cm}

\includegraphics[width=9.0cm,angle=0]{ms2022-0304fig10-3163.pdf}
\includegraphics[width=9.0cm,angle=0]{ms2022-0304fig10-3164.pdf}
\vspace{-0.5cm}

\includegraphics[width=9.0cm,angle=0]{ms2022-0304fig10-3165.pdf}
\includegraphics[width=9.0cm,angle=0]{ms2022-0304fig10-3166.pdf}
\vspace{-0.5cm}

\includegraphics[width=9.0cm,angle=0]{ms2022-0304fig10-3167.pdf}
\includegraphics[width=9.0cm,angle=0]{ms2022-0304fig10-3168.pdf}
\vspace{-0.5cm}

\includegraphics[width=9.0cm,angle=0]{ms2022-0304fig10-3169.pdf}
\includegraphics[width=9.0cm,angle=0]{ms2022-0304fig10-3170.pdf}
\end{figure}
\clearpage

\begin{figure}
\includegraphics[width=9.0cm,angle=0]{ms2022-0304fig10-3171.pdf}
\includegraphics[width=9.0cm,angle=0]{ms2022-0304fig10-3172.pdf}
\vspace{-0.5cm}

\includegraphics[width=9.0cm,angle=0]{ms2022-0304fig10-3173.pdf}
\includegraphics[width=9.0cm,angle=0]{ms2022-0304fig10-3174.pdf}
\vspace{-0.5cm}

\includegraphics[width=9.0cm,angle=0]{ms2022-0304fig10-3175.pdf}
\includegraphics[width=9.0cm,angle=0]{ms2022-0304fig10-3176.pdf}
\vspace{-0.5cm}

\includegraphics[width=9.0cm,angle=0]{ms2022-0304fig10-3177.pdf}
\includegraphics[width=9.0cm,angle=0]{ms2022-0304fig10-3178.pdf}
\vspace{-0.5cm}

\includegraphics[width=9.0cm,angle=0]{ms2022-0304fig10-3179.pdf}
\includegraphics[width=9.0cm,angle=0]{ms2022-0304fig10-3180.pdf}
\end{figure}
\clearpage

\begin{figure}
\includegraphics[width=9.0cm,angle=0]{ms2022-0304fig10-3181.pdf}
\includegraphics[width=9.0cm,angle=0]{ms2022-0304fig10-3182.pdf}
\vspace{-0.5cm}

\includegraphics[width=9.0cm,angle=0]{ms2022-0304fig10-3183.pdf}
\includegraphics[width=9.0cm,angle=0]{ms2022-0304fig10-3184.pdf}
\vspace{-0.5cm}

\includegraphics[width=9.0cm,angle=0]{ms2022-0304fig10-3185.pdf}
\includegraphics[width=9.0cm,angle=0]{ms2022-0304fig10-3186.pdf}
\vspace{-0.5cm}

\includegraphics[width=9.0cm,angle=0]{ms2022-0304fig10-3187.pdf}
\includegraphics[width=9.0cm,angle=0]{ms2022-0304fig10-3188.pdf}
\vspace{-0.5cm}

\includegraphics[width=9.0cm,angle=0]{ms2022-0304fig10-3189.pdf}
\includegraphics[width=9.0cm,angle=0]{ms2022-0304fig10-3190.pdf}
\end{figure}
\clearpage

\begin{figure}
\includegraphics[width=9.0cm,angle=0]{ms2022-0304fig10-3191.pdf}
\includegraphics[width=9.0cm,angle=0]{ms2022-0304fig10-3192.pdf}
\vspace{-0.5cm}

\includegraphics[width=9.0cm,angle=0]{ms2022-0304fig10-3193.pdf}
\includegraphics[width=9.0cm,angle=0]{ms2022-0304fig10-3194.pdf}
\vspace{-0.5cm}

\includegraphics[width=9.0cm,angle=0]{ms2022-0304fig10-3195.pdf}
\includegraphics[width=9.0cm,angle=0]{ms2022-0304fig10-3196.pdf}
\vspace{-0.5cm}

\includegraphics[width=9.0cm,angle=0]{ms2022-0304fig10-3197.pdf}
\includegraphics[width=9.0cm,angle=0]{ms2022-0304fig10-3198.pdf}
\vspace{-0.5cm}

\includegraphics[width=9.0cm,angle=0]{ms2022-0304fig10-3199.pdf}
\includegraphics[width=9.0cm,angle=0]{ms2022-0304fig10-3200.pdf}
\end{figure}
\clearpage

\begin{figure}
\includegraphics[width=9.0cm,angle=0]{ms2022-0304fig10-3201.pdf}
\includegraphics[width=9.0cm,angle=0]{ms2022-0304fig10-3202.pdf}
\vspace{-0.5cm}

\includegraphics[width=9.0cm,angle=0]{ms2022-0304fig10-3203.pdf}
\includegraphics[width=9.0cm,angle=0]{ms2022-0304fig10-3204.pdf}
\vspace{-0.5cm}

\includegraphics[width=9.0cm,angle=0]{ms2022-0304fig10-3205.pdf}
\includegraphics[width=9.0cm,angle=0]{ms2022-0304fig10-3206.pdf}
\vspace{-0.5cm}

\includegraphics[width=9.0cm,angle=0]{ms2022-0304fig10-3207.pdf}
\includegraphics[width=9.0cm,angle=0]{ms2022-0304fig10-3208.pdf}
\vspace{-0.5cm}

\includegraphics[width=9.0cm,angle=0]{ms2022-0304fig10-3209.pdf}
\includegraphics[width=9.0cm,angle=0]{ms2022-0304fig10-3210.pdf}
\end{figure}
\clearpage

\begin{figure}
\includegraphics[width=9.0cm,angle=0]{ms2022-0304fig10-3211.pdf}
\includegraphics[width=9.0cm,angle=0]{ms2022-0304fig10-3212.pdf}
\vspace{-0.5cm}

\includegraphics[width=9.0cm,angle=0]{ms2022-0304fig10-3213.pdf}
\includegraphics[width=9.0cm,angle=0]{ms2022-0304fig10-3214.pdf}
\vspace{-0.5cm}

\includegraphics[width=9.0cm,angle=0]{ms2022-0304fig10-3215.pdf}
\includegraphics[width=9.0cm,angle=0]{ms2022-0304fig10-3216.pdf}
\vspace{-0.5cm}

\includegraphics[width=9.0cm,angle=0]{ms2022-0304fig10-3217.pdf}
\includegraphics[width=9.0cm,angle=0]{ms2022-0304fig10-3218.pdf}
\vspace{-0.5cm}

\includegraphics[width=9.0cm,angle=0]{ms2022-0304fig10-3219.pdf}
\includegraphics[width=9.0cm,angle=0]{ms2022-0304fig10-3220.pdf}
\end{figure}
\clearpage

\begin{figure}
\includegraphics[width=9.0cm,angle=0]{ms2022-0304fig10-3221.pdf}
\includegraphics[width=9.0cm,angle=0]{ms2022-0304fig10-3222.pdf}
\vspace{-0.5cm}

\includegraphics[width=9.0cm,angle=0]{ms2022-0304fig10-3223.pdf}
\includegraphics[width=9.0cm,angle=0]{ms2022-0304fig10-3224.pdf}
\vspace{-0.5cm}

\includegraphics[width=9.0cm,angle=0]{ms2022-0304fig10-3225.pdf}
\includegraphics[width=9.0cm,angle=0]{ms2022-0304fig10-3226.pdf}
\vspace{-0.5cm}

\includegraphics[width=9.0cm,angle=0]{ms2022-0304fig10-3227.pdf}
\includegraphics[width=9.0cm,angle=0]{ms2022-0304fig10-3228.pdf}
\vspace{-0.5cm}

\includegraphics[width=9.0cm,angle=0]{ms2022-0304fig10-3229.pdf}
\includegraphics[width=9.0cm,angle=0]{ms2022-0304fig10-3230.pdf}
\end{figure}
\clearpage

\begin{figure}
\includegraphics[width=9.0cm,angle=0]{ms2022-0304fig10-3231.pdf}
\includegraphics[width=9.0cm,angle=0]{ms2022-0304fig10-3232.pdf}
\vspace{-0.5cm}

\includegraphics[width=9.0cm,angle=0]{ms2022-0304fig10-3233.pdf}
\includegraphics[width=9.0cm,angle=0]{ms2022-0304fig10-3234.pdf}
\vspace{-0.5cm}

\includegraphics[width=9.0cm,angle=0]{ms2022-0304fig10-3235.pdf}
\includegraphics[width=9.0cm,angle=0]{ms2022-0304fig10-3236.pdf}
\vspace{-0.5cm}

\includegraphics[width=9.0cm,angle=0]{ms2022-0304fig10-3237.pdf}
\includegraphics[width=9.0cm,angle=0]{ms2022-0304fig10-3238.pdf}
\vspace{-0.5cm}

\includegraphics[width=9.0cm,angle=0]{ms2022-0304fig10-3239.pdf}
\includegraphics[width=9.0cm,angle=0]{ms2022-0304fig10-3240.pdf}
\end{figure}
\clearpage

\begin{figure}
\includegraphics[width=9.0cm,angle=0]{ms2022-0304fig10-3241.pdf}
\includegraphics[width=9.0cm,angle=0]{ms2022-0304fig10-3242.pdf}
\vspace{-0.5cm}

\includegraphics[width=9.0cm,angle=0]{ms2022-0304fig10-3243.pdf}
\includegraphics[width=9.0cm,angle=0]{ms2022-0304fig10-3244.pdf}
\vspace{-0.5cm}

\includegraphics[width=9.0cm,angle=0]{ms2022-0304fig10-3245.pdf}
\includegraphics[width=9.0cm,angle=0]{ms2022-0304fig10-3246.pdf}
\vspace{-0.5cm}

\includegraphics[width=9.0cm,angle=0]{ms2022-0304fig10-3247.pdf}
\includegraphics[width=9.0cm,angle=0]{ms2022-0304fig10-3248.pdf}
\vspace{-0.5cm}

\includegraphics[width=9.0cm,angle=0]{ms2022-0304fig10-3249.pdf}
\includegraphics[width=9.0cm,angle=0]{ms2022-0304fig10-3250.pdf}
\end{figure}
\clearpage

\begin{figure}
\includegraphics[width=9.0cm,angle=0]{ms2022-0304fig10-3251.pdf}
\includegraphics[width=9.0cm,angle=0]{ms2022-0304fig10-3252.pdf}
\vspace{-0.5cm}

\includegraphics[width=9.0cm,angle=0]{ms2022-0304fig10-3253.pdf}
\includegraphics[width=9.0cm,angle=0]{ms2022-0304fig10-3254.pdf}
\vspace{-0.5cm}

\includegraphics[width=9.0cm,angle=0]{ms2022-0304fig10-3255.pdf}
\includegraphics[width=9.0cm,angle=0]{ms2022-0304fig10-3256.pdf}
\vspace{-0.5cm}

\includegraphics[width=9.0cm,angle=0]{ms2022-0304fig10-3257.pdf}
\includegraphics[width=9.0cm,angle=0]{ms2022-0304fig10-3258.pdf}
\vspace{-0.5cm}

\includegraphics[width=9.0cm,angle=0]{ms2022-0304fig10-3259.pdf}
\includegraphics[width=9.0cm,angle=0]{ms2022-0304fig10-3260.pdf}
\end{figure}
\clearpage

\begin{figure}
\includegraphics[width=9.0cm,angle=0]{ms2022-0304fig10-3261.pdf}
\includegraphics[width=9.0cm,angle=0]{ms2022-0304fig10-3262.pdf}
\vspace{-0.5cm}

\includegraphics[width=9.0cm,angle=0]{ms2022-0304fig10-3263.pdf}
\includegraphics[width=9.0cm,angle=0]{ms2022-0304fig10-3264.pdf}
\vspace{-0.5cm}

\includegraphics[width=9.0cm,angle=0]{ms2022-0304fig10-3265.pdf}
\includegraphics[width=9.0cm,angle=0]{ms2022-0304fig10-3266.pdf}
\vspace{-0.5cm}

\includegraphics[width=9.0cm,angle=0]{ms2022-0304fig10-3267.pdf}
\includegraphics[width=9.0cm,angle=0]{ms2022-0304fig10-3268.pdf}
\vspace{-0.5cm}

\includegraphics[width=9.0cm,angle=0]{ms2022-0304fig10-3269.pdf}
\includegraphics[width=9.0cm,angle=0]{ms2022-0304fig10-3270.pdf}
\end{figure}
\clearpage

\begin{figure}
\includegraphics[width=9.0cm,angle=0]{ms2022-0304fig10-3271.pdf}
\includegraphics[width=9.0cm,angle=0]{ms2022-0304fig10-3272.pdf}
\vspace{-0.5cm}

\includegraphics[width=9.0cm,angle=0]{ms2022-0304fig10-3273.pdf}
\includegraphics[width=9.0cm,angle=0]{ms2022-0304fig10-3274.pdf}
\vspace{-0.5cm}

\includegraphics[width=9.0cm,angle=0]{ms2022-0304fig10-3275.pdf}
\includegraphics[width=9.0cm,angle=0]{ms2022-0304fig10-3276.pdf}
\vspace{-0.5cm}

\includegraphics[width=9.0cm,angle=0]{ms2022-0304fig10-3277.pdf}
\includegraphics[width=9.0cm,angle=0]{ms2022-0304fig10-3278.pdf}
\vspace{-0.5cm}

\includegraphics[width=9.0cm,angle=0]{ms2022-0304fig10-3279.pdf}
\includegraphics[width=9.0cm,angle=0]{ms2022-0304fig10-3280.pdf}
\end{figure}
\clearpage

\begin{figure}
\includegraphics[width=9.0cm,angle=0]{ms2022-0304fig10-3281.pdf}
\includegraphics[width=9.0cm,angle=0]{ms2022-0304fig10-3282.pdf}
\vspace{-0.5cm}

\includegraphics[width=9.0cm,angle=0]{ms2022-0304fig10-3283.pdf}
\includegraphics[width=9.0cm,angle=0]{ms2022-0304fig10-3284.pdf}
\vspace{-0.5cm}

\includegraphics[width=9.0cm,angle=0]{ms2022-0304fig10-3285.pdf}
\includegraphics[width=9.0cm,angle=0]{ms2022-0304fig10-3286.pdf}
\vspace{-0.5cm}

\includegraphics[width=9.0cm,angle=0]{ms2022-0304fig10-3287.pdf}
\includegraphics[width=9.0cm,angle=0]{ms2022-0304fig10-3288.pdf}
\vspace{-0.5cm}

\includegraphics[width=9.0cm,angle=0]{ms2022-0304fig10-3289.pdf}
\includegraphics[width=9.0cm,angle=0]{ms2022-0304fig10-3290.pdf}
\end{figure}
\clearpage

\begin{figure}
\includegraphics[width=9.0cm,angle=0]{ms2022-0304fig10-3291.pdf}
\includegraphics[width=9.0cm,angle=0]{ms2022-0304fig10-3292.pdf}
\vspace{-0.5cm}

\includegraphics[width=9.0cm,angle=0]{ms2022-0304fig10-3293.pdf}
\includegraphics[width=9.0cm,angle=0]{ms2022-0304fig10-3294.pdf}
\vspace{-0.5cm}

\includegraphics[width=9.0cm,angle=0]{ms2022-0304fig10-3295.pdf}
\includegraphics[width=9.0cm,angle=0]{ms2022-0304fig10-3296.pdf}
\vspace{-0.5cm}

\includegraphics[width=9.0cm,angle=0]{ms2022-0304fig10-3297.pdf}
\includegraphics[width=9.0cm,angle=0]{ms2022-0304fig10-3298.pdf}
\vspace{-0.5cm}

\includegraphics[width=9.0cm,angle=0]{ms2022-0304fig10-3299.pdf}
\includegraphics[width=9.0cm,angle=0]{ms2022-0304fig10-3300.pdf}
\end{figure}
\clearpage

\begin{figure}
\includegraphics[width=9.0cm,angle=0]{ms2022-0304fig10-3301.pdf}
\includegraphics[width=9.0cm,angle=0]{ms2022-0304fig10-3302.pdf}
\vspace{-0.5cm}

\includegraphics[width=9.0cm,angle=0]{ms2022-0304fig10-3303.pdf}
\includegraphics[width=9.0cm,angle=0]{ms2022-0304fig10-3304.pdf}
\vspace{-0.5cm}

\includegraphics[width=9.0cm,angle=0]{ms2022-0304fig10-3305.pdf}
\includegraphics[width=9.0cm,angle=0]{ms2022-0304fig10-3306.pdf}
\vspace{-0.5cm}

\includegraphics[width=9.0cm,angle=0]{ms2022-0304fig10-3307.pdf}
\includegraphics[width=9.0cm,angle=0]{ms2022-0304fig10-3308.pdf}
\vspace{-0.5cm}

\includegraphics[width=9.0cm,angle=0]{ms2022-0304fig10-3309.pdf}
\includegraphics[width=9.0cm,angle=0]{ms2022-0304fig10-3310.pdf}
\end{figure}
\clearpage

\begin{figure}
\includegraphics[width=9.0cm,angle=0]{ms2022-0304fig10-3311.pdf}
\includegraphics[width=9.0cm,angle=0]{ms2022-0304fig10-3312.pdf}
\vspace{-0.5cm}

\includegraphics[width=9.0cm,angle=0]{ms2022-0304fig10-3313.pdf}
\includegraphics[width=9.0cm,angle=0]{ms2022-0304fig10-3314.pdf}
\vspace{-0.5cm}

\includegraphics[width=9.0cm,angle=0]{ms2022-0304fig10-3315.pdf}
\includegraphics[width=9.0cm,angle=0]{ms2022-0304fig10-3316.pdf}
\vspace{-0.5cm}

\includegraphics[width=9.0cm,angle=0]{ms2022-0304fig10-3317.pdf}
\includegraphics[width=9.0cm,angle=0]{ms2022-0304fig10-3318.pdf}
\vspace{-0.5cm}

\includegraphics[width=9.0cm,angle=0]{ms2022-0304fig10-3319.pdf}
\includegraphics[width=9.0cm,angle=0]{ms2022-0304fig10-3320.pdf}
\end{figure}
\clearpage

\begin{figure}
\includegraphics[width=9.0cm,angle=0]{ms2022-0304fig10-3321.pdf}
\includegraphics[width=9.0cm,angle=0]{ms2022-0304fig10-3322.pdf}
\vspace{-0.5cm}

\includegraphics[width=9.0cm,angle=0]{ms2022-0304fig10-3323.pdf}
\includegraphics[width=9.0cm,angle=0]{ms2022-0304fig10-3324.pdf}
\vspace{-0.5cm}

\includegraphics[width=9.0cm,angle=0]{ms2022-0304fig10-3325.pdf}
\includegraphics[width=9.0cm,angle=0]{ms2022-0304fig10-3326.pdf}
\vspace{-0.5cm}

\includegraphics[width=9.0cm,angle=0]{ms2022-0304fig10-3327.pdf}
\includegraphics[width=9.0cm,angle=0]{ms2022-0304fig10-3328.pdf}
\vspace{-0.5cm}

\includegraphics[width=9.0cm,angle=0]{ms2022-0304fig10-3329.pdf}
\includegraphics[width=9.0cm,angle=0]{ms2022-0304fig10-3330.pdf}
\end{figure}
\clearpage

\begin{figure}
\includegraphics[width=9.0cm,angle=0]{ms2022-0304fig10-3331.pdf}
\includegraphics[width=9.0cm,angle=0]{ms2022-0304fig10-3332.pdf}
\vspace{-0.5cm}

\includegraphics[width=9.0cm,angle=0]{ms2022-0304fig10-3333.pdf}
\includegraphics[width=9.0cm,angle=0]{ms2022-0304fig10-3334.pdf}
\vspace{-0.5cm}

\includegraphics[width=9.0cm,angle=0]{ms2022-0304fig10-3335.pdf}
\includegraphics[width=9.0cm,angle=0]{ms2022-0304fig10-3336.pdf}
\vspace{-0.5cm}

\includegraphics[width=9.0cm,angle=0]{ms2022-0304fig10-3337.pdf}
\includegraphics[width=9.0cm,angle=0]{ms2022-0304fig10-3338.pdf}
\vspace{-0.5cm}

\includegraphics[width=9.0cm,angle=0]{ms2022-0304fig10-3339.pdf}
\includegraphics[width=9.0cm,angle=0]{ms2022-0304fig10-3340.pdf}
\end{figure}
\clearpage

\begin{figure}
\includegraphics[width=9.0cm,angle=0]{ms2022-0304fig10-3341.pdf}
\includegraphics[width=9.0cm,angle=0]{ms2022-0304fig10-3342.pdf}
\vspace{-0.5cm}

\includegraphics[width=9.0cm,angle=0]{ms2022-0304fig10-3343.pdf}
\includegraphics[width=9.0cm,angle=0]{ms2022-0304fig10-3344.pdf}
\vspace{-0.5cm}

\includegraphics[width=9.0cm,angle=0]{ms2022-0304fig10-3345.pdf}
\includegraphics[width=9.0cm,angle=0]{ms2022-0304fig10-3346.pdf}
\vspace{-0.5cm}

\includegraphics[width=9.0cm,angle=0]{ms2022-0304fig10-3347.pdf}
\includegraphics[width=9.0cm,angle=0]{ms2022-0304fig10-3348.pdf}
\vspace{-0.5cm}

\includegraphics[width=9.0cm,angle=0]{ms2022-0304fig10-3349.pdf}
\includegraphics[width=9.0cm,angle=0]{ms2022-0304fig10-3350.pdf}
\end{figure}
\clearpage

\begin{figure}
\includegraphics[width=9.0cm,angle=0]{ms2022-0304fig10-3351.pdf}
\includegraphics[width=9.0cm,angle=0]{ms2022-0304fig10-3352.pdf}
\vspace{-0.5cm}

\includegraphics[width=9.0cm,angle=0]{ms2022-0304fig10-3353.pdf}
\includegraphics[width=9.0cm,angle=0]{ms2022-0304fig10-3354.pdf}
\vspace{-0.5cm}

\includegraphics[width=9.0cm,angle=0]{ms2022-0304fig10-3355.pdf}
\includegraphics[width=9.0cm,angle=0]{ms2022-0304fig10-3356.pdf}
\vspace{-0.5cm}

\includegraphics[width=9.0cm,angle=0]{ms2022-0304fig10-3357.pdf}
\includegraphics[width=9.0cm,angle=0]{ms2022-0304fig10-3358.pdf}
\vspace{-0.5cm}

\includegraphics[width=9.0cm,angle=0]{ms2022-0304fig10-3359.pdf}
\includegraphics[width=9.0cm,angle=0]{ms2022-0304fig10-3360.pdf}
\end{figure}
\clearpage

\begin{figure}
\includegraphics[width=9.0cm,angle=0]{ms2022-0304fig10-3361.pdf}
\includegraphics[width=9.0cm,angle=0]{ms2022-0304fig10-3362.pdf}
\vspace{-0.5cm}

\includegraphics[width=9.0cm,angle=0]{ms2022-0304fig10-3363.pdf}
\includegraphics[width=9.0cm,angle=0]{ms2022-0304fig10-3364.pdf}
\vspace{-0.5cm}

\includegraphics[width=9.0cm,angle=0]{ms2022-0304fig10-3365.pdf}
\includegraphics[width=9.0cm,angle=0]{ms2022-0304fig10-3366.pdf}
\vspace{-0.5cm}

\includegraphics[width=9.0cm,angle=0]{ms2022-0304fig10-3367.pdf}
\includegraphics[width=9.0cm,angle=0]{ms2022-0304fig10-3368.pdf}
\vspace{-0.5cm}

\includegraphics[width=9.0cm,angle=0]{ms2022-0304fig10-3369.pdf}
\includegraphics[width=9.0cm,angle=0]{ms2022-0304fig10-3370.pdf}
\end{figure}
\clearpage

\begin{figure}
\includegraphics[width=9.0cm,angle=0]{ms2022-0304fig10-3371.pdf}
\includegraphics[width=9.0cm,angle=0]{ms2022-0304fig10-3372.pdf}
\vspace{-0.5cm}

\includegraphics[width=9.0cm,angle=0]{ms2022-0304fig10-3373.pdf}
\includegraphics[width=9.0cm,angle=0]{ms2022-0304fig10-3374.pdf}
\vspace{-0.5cm}

\includegraphics[width=9.0cm,angle=0]{ms2022-0304fig10-3375.pdf}
\includegraphics[width=9.0cm,angle=0]{ms2022-0304fig10-3376.pdf}
\vspace{-0.5cm}

\includegraphics[width=9.0cm,angle=0]{ms2022-0304fig10-3377.pdf}
\includegraphics[width=9.0cm,angle=0]{ms2022-0304fig10-3378.pdf}
\vspace{-0.5cm}

\includegraphics[width=9.0cm,angle=0]{ms2022-0304fig10-3379.pdf}
\includegraphics[width=9.0cm,angle=0]{ms2022-0304fig10-3380.pdf}
\end{figure}
\clearpage

\begin{figure}
\includegraphics[width=9.0cm,angle=0]{ms2022-0304fig10-3381.pdf}
\includegraphics[width=9.0cm,angle=0]{ms2022-0304fig10-3382.pdf}
\vspace{-0.5cm}

\includegraphics[width=9.0cm,angle=0]{ms2022-0304fig10-3383.pdf}
\includegraphics[width=9.0cm,angle=0]{ms2022-0304fig10-3384.pdf}
\vspace{-0.5cm}

\includegraphics[width=9.0cm,angle=0]{ms2022-0304fig10-3385.pdf}
\includegraphics[width=9.0cm,angle=0]{ms2022-0304fig10-3386.pdf}
\vspace{-0.5cm}

\includegraphics[width=9.0cm,angle=0]{ms2022-0304fig10-3387.pdf}
\includegraphics[width=9.0cm,angle=0]{ms2022-0304fig10-3388.pdf}
\vspace{-0.5cm}

\includegraphics[width=9.0cm,angle=0]{ms2022-0304fig10-3389.pdf}
\includegraphics[width=9.0cm,angle=0]{ms2022-0304fig10-3390.pdf}
\end{figure}
\clearpage

\begin{figure}
\includegraphics[width=9.0cm,angle=0]{ms2022-0304fig10-3391.pdf}
\includegraphics[width=9.0cm,angle=0]{ms2022-0304fig10-3392.pdf}
\vspace{-0.5cm}

\includegraphics[width=9.0cm,angle=0]{ms2022-0304fig10-3393.pdf}
\includegraphics[width=9.0cm,angle=0]{ms2022-0304fig10-3394.pdf}
\vspace{-0.5cm}

\includegraphics[width=9.0cm,angle=0]{ms2022-0304fig10-3395.pdf}
\includegraphics[width=9.0cm,angle=0]{ms2022-0304fig10-3396.pdf}
\vspace{-0.5cm}

\includegraphics[width=9.0cm,angle=0]{ms2022-0304fig10-3397.pdf}
\includegraphics[width=9.0cm,angle=0]{ms2022-0304fig10-3398.pdf}
\vspace{-0.5cm}

\includegraphics[width=9.0cm,angle=0]{ms2022-0304fig10-3399.pdf}
\includegraphics[width=9.0cm,angle=0]{ms2022-0304fig10-3400.pdf}
\end{figure}
\clearpage

\begin{figure}
\includegraphics[width=9.0cm,angle=0]{ms2022-0304fig10-3401.pdf}
\includegraphics[width=9.0cm,angle=0]{ms2022-0304fig10-3402.pdf}
\vspace{-0.5cm}

\includegraphics[width=9.0cm,angle=0]{ms2022-0304fig10-3403.pdf}
\includegraphics[width=9.0cm,angle=0]{ms2022-0304fig10-3404.pdf}
\vspace{-0.5cm}

\includegraphics[width=9.0cm,angle=0]{ms2022-0304fig10-3405.pdf}
\includegraphics[width=9.0cm,angle=0]{ms2022-0304fig10-3406.pdf}
\vspace{-0.5cm}

\includegraphics[width=9.0cm,angle=0]{ms2022-0304fig10-3407.pdf}
\includegraphics[width=9.0cm,angle=0]{ms2022-0304fig10-3408.pdf}
\vspace{-0.5cm}

\includegraphics[width=9.0cm,angle=0]{ms2022-0304fig10-3409.pdf}
\includegraphics[width=9.0cm,angle=0]{ms2022-0304fig10-3410.pdf}
\end{figure}
\clearpage

\begin{figure}
\includegraphics[width=9.0cm,angle=0]{ms2022-0304fig10-3411.pdf}
\includegraphics[width=9.0cm,angle=0]{ms2022-0304fig10-3412.pdf}
\vspace{-0.5cm}

\includegraphics[width=9.0cm,angle=0]{ms2022-0304fig10-3413.pdf}
\includegraphics[width=9.0cm,angle=0]{ms2022-0304fig10-3414.pdf}
\vspace{-0.5cm}

\includegraphics[width=9.0cm,angle=0]{ms2022-0304fig10-3415.pdf}
\includegraphics[width=9.0cm,angle=0]{ms2022-0304fig10-3416.pdf}
\vspace{-0.5cm}

\includegraphics[width=9.0cm,angle=0]{ms2022-0304fig10-3417.pdf}
\includegraphics[width=9.0cm,angle=0]{ms2022-0304fig10-3418.pdf}
\vspace{-0.5cm}

\includegraphics[width=9.0cm,angle=0]{ms2022-0304fig10-3419.pdf}
\includegraphics[width=9.0cm,angle=0]{ms2022-0304fig10-3420.pdf}
\end{figure}
\clearpage

\begin{figure}
\includegraphics[width=9.0cm,angle=0]{ms2022-0304fig10-3421.pdf}
\includegraphics[width=9.0cm,angle=0]{ms2022-0304fig10-3422.pdf}
\vspace{-0.5cm}

\includegraphics[width=9.0cm,angle=0]{ms2022-0304fig10-3423.pdf}
\includegraphics[width=9.0cm,angle=0]{ms2022-0304fig10-3424.pdf}
\vspace{-0.5cm}

\includegraphics[width=9.0cm,angle=0]{ms2022-0304fig10-3425.pdf}
\includegraphics[width=9.0cm,angle=0]{ms2022-0304fig10-3426.pdf}
\vspace{-0.5cm}

\includegraphics[width=9.0cm,angle=0]{ms2022-0304fig10-3427.pdf}
\includegraphics[width=9.0cm,angle=0]{ms2022-0304fig10-3428.pdf}
\vspace{-0.5cm}

\includegraphics[width=9.0cm,angle=0]{ms2022-0304fig10-3429.pdf}
\includegraphics[width=9.0cm,angle=0]{ms2022-0304fig10-3430.pdf}
\end{figure}
\clearpage

\begin{figure}
\includegraphics[width=9.0cm,angle=0]{ms2022-0304fig10-3431.pdf}
\includegraphics[width=9.0cm,angle=0]{ms2022-0304fig10-3432.pdf}
\vspace{-0.5cm}

\includegraphics[width=9.0cm,angle=0]{ms2022-0304fig10-3433.pdf}
\includegraphics[width=9.0cm,angle=0]{ms2022-0304fig10-3434.pdf}
\vspace{-0.5cm}

\includegraphics[width=9.0cm,angle=0]{ms2022-0304fig10-3435.pdf}
\includegraphics[width=9.0cm,angle=0]{ms2022-0304fig10-3436.pdf}
\vspace{-0.5cm}

\includegraphics[width=9.0cm,angle=0]{ms2022-0304fig10-3437.pdf}
\includegraphics[width=9.0cm,angle=0]{ms2022-0304fig10-3438.pdf}
\vspace{-0.5cm}

\includegraphics[width=9.0cm,angle=0]{ms2022-0304fig10-3439.pdf}
\includegraphics[width=9.0cm,angle=0]{ms2022-0304fig10-3440.pdf}
\end{figure}
\clearpage

\begin{figure}
\includegraphics[width=9.0cm,angle=0]{ms2022-0304fig10-3441.pdf}
\includegraphics[width=9.0cm,angle=0]{ms2022-0304fig10-3442.pdf}
\vspace{-0.5cm}

\includegraphics[width=9.0cm,angle=0]{ms2022-0304fig10-3443.pdf}
\includegraphics[width=9.0cm,angle=0]{ms2022-0304fig10-3444.pdf}
\vspace{-0.5cm}

\includegraphics[width=9.0cm,angle=0]{ms2022-0304fig10-3445.pdf}
\includegraphics[width=9.0cm,angle=0]{ms2022-0304fig10-3446.pdf}
\vspace{-0.5cm}

\includegraphics[width=9.0cm,angle=0]{ms2022-0304fig10-3447.pdf}
\includegraphics[width=9.0cm,angle=0]{ms2022-0304fig10-3448.pdf}
\vspace{-0.5cm}

\includegraphics[width=9.0cm,angle=0]{ms2022-0304fig10-3449.pdf}
\includegraphics[width=9.0cm,angle=0]{ms2022-0304fig10-3450.pdf}
\end{figure}
\clearpage

\begin{figure}
\includegraphics[width=9.0cm,angle=0]{ms2022-0304fig10-3451.pdf}
\includegraphics[width=9.0cm,angle=0]{ms2022-0304fig10-3452.pdf}
\vspace{-0.5cm}

\includegraphics[width=9.0cm,angle=0]{ms2022-0304fig10-3453.pdf}
\includegraphics[width=9.0cm,angle=0]{ms2022-0304fig10-3454.pdf}
\vspace{-0.5cm}

\includegraphics[width=9.0cm,angle=0]{ms2022-0304fig10-3455.pdf}
\includegraphics[width=9.0cm,angle=0]{ms2022-0304fig10-3456.pdf}
\vspace{-0.5cm}

\includegraphics[width=9.0cm,angle=0]{ms2022-0304fig10-3457.pdf}
\includegraphics[width=9.0cm,angle=0]{ms2022-0304fig10-3458.pdf}
\vspace{-0.5cm}

\includegraphics[width=9.0cm,angle=0]{ms2022-0304fig10-3459.pdf}
\includegraphics[width=9.0cm,angle=0]{ms2022-0304fig10-3460.pdf}
\end{figure}
\clearpage

\begin{figure}
\includegraphics[width=9.0cm,angle=0]{ms2022-0304fig10-3461.pdf}
\includegraphics[width=9.0cm,angle=0]{ms2022-0304fig10-3462.pdf}
\vspace{-0.5cm}

\includegraphics[width=9.0cm,angle=0]{ms2022-0304fig10-3463.pdf}
\includegraphics[width=9.0cm,angle=0]{ms2022-0304fig10-3464.pdf}
\vspace{-0.5cm}

\includegraphics[width=9.0cm,angle=0]{ms2022-0304fig10-3465.pdf}
\includegraphics[width=9.0cm,angle=0]{ms2022-0304fig10-3466.pdf}
\vspace{-0.5cm}

\includegraphics[width=9.0cm,angle=0]{ms2022-0304fig10-3467.pdf}
\includegraphics[width=9.0cm,angle=0]{ms2022-0304fig10-3468.pdf}
\vspace{-0.5cm}

\includegraphics[width=9.0cm,angle=0]{ms2022-0304fig10-3469.pdf}
\includegraphics[width=9.0cm,angle=0]{ms2022-0304fig10-3470.pdf}
\end{figure}
\clearpage

\begin{figure}
\includegraphics[width=9.0cm,angle=0]{ms2022-0304fig10-3471.pdf}
\includegraphics[width=9.0cm,angle=0]{ms2022-0304fig10-3472.pdf}
\vspace{-0.5cm}

\includegraphics[width=9.0cm,angle=0]{ms2022-0304fig10-3473.pdf}
\includegraphics[width=9.0cm,angle=0]{ms2022-0304fig10-3474.pdf}
\vspace{-0.5cm}

\includegraphics[width=9.0cm,angle=0]{ms2022-0304fig10-3475.pdf}
\includegraphics[width=9.0cm,angle=0]{ms2022-0304fig10-3476.pdf}
\vspace{-0.5cm}

\includegraphics[width=9.0cm,angle=0]{ms2022-0304fig10-3477.pdf}
\includegraphics[width=9.0cm,angle=0]{ms2022-0304fig10-3478.pdf}
\vspace{-0.5cm}

\includegraphics[width=9.0cm,angle=0]{ms2022-0304fig10-3479.pdf}
\includegraphics[width=9.0cm,angle=0]{ms2022-0304fig10-3480.pdf}
\end{figure}
\clearpage

\begin{figure}
\includegraphics[width=9.0cm,angle=0]{ms2022-0304fig10-3481.pdf}
\includegraphics[width=9.0cm,angle=0]{ms2022-0304fig10-3482.pdf}
\vspace{-0.5cm}

\includegraphics[width=9.0cm,angle=0]{ms2022-0304fig10-3483.pdf}
\includegraphics[width=9.0cm,angle=0]{ms2022-0304fig10-3484.pdf}
\vspace{-0.5cm}

\includegraphics[width=9.0cm,angle=0]{ms2022-0304fig10-3485.pdf}
\includegraphics[width=9.0cm,angle=0]{ms2022-0304fig10-3486.pdf}
\vspace{-0.5cm}

\includegraphics[width=9.0cm,angle=0]{ms2022-0304fig10-3487.pdf}
\includegraphics[width=9.0cm,angle=0]{ms2022-0304fig10-3488.pdf}
\vspace{-0.5cm}

\includegraphics[width=9.0cm,angle=0]{ms2022-0304fig10-3489.pdf}
\includegraphics[width=9.0cm,angle=0]{ms2022-0304fig10-3490.pdf}
\end{figure}
\clearpage

\begin{figure}
\includegraphics[width=9.0cm,angle=0]{ms2022-0304fig10-3491.pdf}
\includegraphics[width=9.0cm,angle=0]{ms2022-0304fig10-3492.pdf}
\vspace{-0.5cm}

\includegraphics[width=9.0cm,angle=0]{ms2022-0304fig10-3493.pdf}
\includegraphics[width=9.0cm,angle=0]{ms2022-0304fig10-3494.pdf}
\vspace{-0.5cm}

\includegraphics[width=9.0cm,angle=0]{ms2022-0304fig10-3495.pdf}
\includegraphics[width=9.0cm,angle=0]{ms2022-0304fig10-3496.pdf}
\vspace{-0.5cm}

\includegraphics[width=9.0cm,angle=0]{ms2022-0304fig10-3497.pdf}
\includegraphics[width=9.0cm,angle=0]{ms2022-0304fig10-3498.pdf}
\vspace{-0.5cm}

\includegraphics[width=9.0cm,angle=0]{ms2022-0304fig10-3499.pdf}
\includegraphics[width=9.0cm,angle=0]{ms2022-0304fig10-3500.pdf}
\end{figure}
\clearpage

\begin{figure}
\includegraphics[width=9.0cm,angle=0]{ms2022-0304fig10-3501.pdf}
\includegraphics[width=9.0cm,angle=0]{ms2022-0304fig10-3502.pdf}
\vspace{-0.5cm}

\includegraphics[width=9.0cm,angle=0]{ms2022-0304fig10-3503.pdf}
\includegraphics[width=9.0cm,angle=0]{ms2022-0304fig10-3504.pdf}
\vspace{-0.5cm}

\includegraphics[width=9.0cm,angle=0]{ms2022-0304fig10-3505.pdf}
\includegraphics[width=9.0cm,angle=0]{ms2022-0304fig10-3506.pdf}
\vspace{-0.5cm}

\includegraphics[width=9.0cm,angle=0]{ms2022-0304fig10-3507.pdf}
\includegraphics[width=9.0cm,angle=0]{ms2022-0304fig10-3508.pdf}
\vspace{-0.5cm}

\includegraphics[width=9.0cm,angle=0]{ms2022-0304fig10-3509.pdf}
\includegraphics[width=9.0cm,angle=0]{ms2022-0304fig10-3510.pdf}
\end{figure}
\clearpage

\begin{figure}
\includegraphics[width=9.0cm,angle=0]{ms2022-0304fig10-3511.pdf}
\includegraphics[width=9.0cm,angle=0]{ms2022-0304fig10-3512.pdf}
\vspace{-0.5cm}

\includegraphics[width=9.0cm,angle=0]{ms2022-0304fig10-3513.pdf}
\includegraphics[width=9.0cm,angle=0]{ms2022-0304fig10-3514.pdf}
\vspace{-0.5cm}

\includegraphics[width=9.0cm,angle=0]{ms2022-0304fig10-3515.pdf}
\includegraphics[width=9.0cm,angle=0]{ms2022-0304fig10-3516.pdf}
\vspace{-0.5cm}

\includegraphics[width=9.0cm,angle=0]{ms2022-0304fig10-3517.pdf}
\includegraphics[width=9.0cm,angle=0]{ms2022-0304fig10-3518.pdf}
\vspace{-0.5cm}

\includegraphics[width=9.0cm,angle=0]{ms2022-0304fig10-3519.pdf}
\includegraphics[width=9.0cm,angle=0]{ms2022-0304fig10-3520.pdf}
\end{figure}
\clearpage

\begin{figure}
\includegraphics[width=9.0cm,angle=0]{ms2022-0304fig10-3521.pdf}
\includegraphics[width=9.0cm,angle=0]{ms2022-0304fig10-3522.pdf}
\vspace{-0.5cm}

\includegraphics[width=9.0cm,angle=0]{ms2022-0304fig10-3523.pdf}
\includegraphics[width=9.0cm,angle=0]{ms2022-0304fig10-3524.pdf}
\vspace{-0.5cm}

\includegraphics[width=9.0cm,angle=0]{ms2022-0304fig10-3525.pdf}
\includegraphics[width=9.0cm,angle=0]{ms2022-0304fig10-3526.pdf}
\vspace{-0.5cm}

\includegraphics[width=9.0cm,angle=0]{ms2022-0304fig10-3527.pdf}
\includegraphics[width=9.0cm,angle=0]{ms2022-0304fig10-3528.pdf}
\vspace{-0.5cm}

\includegraphics[width=9.0cm,angle=0]{ms2022-0304fig10-3529.pdf}
\includegraphics[width=9.0cm,angle=0]{ms2022-0304fig10-3530.pdf}
\end{figure}
\clearpage

\begin{figure}
\includegraphics[width=9.0cm,angle=0]{ms2022-0304fig10-3531.pdf}
\includegraphics[width=9.0cm,angle=0]{ms2022-0304fig10-3532.pdf}
\vspace{-0.5cm}

\includegraphics[width=9.0cm,angle=0]{ms2022-0304fig10-3533.pdf}
\end{figure}

 
\bibliographystyle{raa}
\bibliography{reference.bib}
\end{document}
