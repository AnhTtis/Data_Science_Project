\section{Experiments on datasets derived from the US Census}
In this section we apply our approach to more datasets derived from the US census database. We discuss what are the differences on three (plus the one in the main body of the paper) prediction tasks based on the US census data: ACS Income (main body), ACS Travel Time, ACS Employment and ACS Mobility.

We follow the same methodology as in the experimental section. Where we divide a dataset into three equal splits $\{\Dd{tr},\Dd{val},\Dd{te} \} \subseteq \D$ and select our protected attribute, $Z$, to be a feature that indicates the ethnicity of an individual. We train our model $f_\beta$ on $\{X^{tr},Y^{tr}\}$ 
and, the \enquote{Demographic Parity Inspector} $g_\psi$ on $\{\Ss(f_\beta,X^{val}),Z^{val}\}$. Both methods are evaluated on $\{X^{te},Z^{te},y^{te}\}$. For the type of models, $f_\beta$, as we are in tabular data we focus on we choose $f_\theta$ to be a \texttt{xgboost}\cite{xgboost}  that achieve state-of-the-art model performance~\cite{grinsztajn2022why,DBLP:journals/corr/abs-2101-02118,BorisovNNtabular},and for the inspector $g_\psi$ a logistic regression. The final explanations are given by the coefficients of the logistic regression.


\subsection{ACS Employment}
The goal of this task is to predict whether an individual, is employed. 
In this section we are going to do a comparison within different protected attribute groups for CA14.



\begin{figure*}[ht]
\centering
\includegraphics[width=.49\textwidth]{images/detector_auc_ACSEmployment.png}\hfill
\includegraphics[width=.49\textwidth]{images/feature_importance_ACSEmployment.png}
\caption{In the left figure, comparison of the performance of \textit{Demographic Parity Inspector}, over the state of CA14 for the ACS Employment dataset. For this dataset most of different protected attributes comparison have different AUC distributions in terms of mean and variance. The group that receives the most comparison is Other-Black. This difference in the model behaviour is due to features such as \enquote{Education} and \enquote{Citizenship}. 
Interestingly, features such as \enquote{difficulties on the hearing or seeing}, do not play a role in the OOD model behaviour.}
\label{fig:xai.employment}
\end{figure*}
Compared to other prediction tasks based on the US census dataset, the measured domographic parity violation for this task  is smallest. The AUC of the \enquote{Demographic Parity Inspector} is ranging from $0.55$ to $0.70$. For Asian-Black demographic parity violation we see that there is a lot of variation of the AUC, indicating that the method achieves different values on the bootstrapping folds. 
If we look for the features driving the demographic parity gap, we see particularly high values when comparing \enquote{Asian} and \enquote{Black} populations, and for features \enquote{Citizenship} and \enquote{Employment}. 
On average, the most important features across all group comparisons are also \enquote{Education} and \enquote{Area}.


\subsection{ACS Travel Time}

The goal of this task is to predict whether an individual has a commute to work that is longer than 20 minutes. The threshold of 20 minutes was chosen as it is the US-wide median travel time to work based on 2018 data



\begin{figure*}[ht]
\centering
\includegraphics[width=.49\textwidth]{images/detector_auc_ACSTravelTime.png}\hfill
\includegraphics[width=.49\textwidth]{images/feature_importance_ACSTravelTime.png}
\caption{In the left figure, comparison of the performance of \textit{Demographic Parity Inspector}, in different pair-wise ethnicities for the ACS Travel Time dataset. In the right image, the features that are more probable to be originating the demographic parity violation (higher implies more probable).}
\label{fig:xai.traveltime}
\end{figure*}

In general, the values of the AUC of the \enquote{Demographic Parity Inspector} are in the range of $0.50$ to $0.60$. 
By inspecting the features we can see that our proposed methods highlight different demographic parity violation drivers depending on the pair-wise comparison made. 
In general, the feature \enquote{Education}, \enquote{Citizenship} and \enquote{Area} are the features with the highest difference. Even though for Asian-Black pairwise comparison \enquote{Employment} is also one of the most relevant features.

\subsection{ACS Mobility}

The goal of this task is to predict whether an individual had the same residential address one year ago, only including individuals between the ages of 18 and 35. This filtering increases the difficulty of the prediction task, as the base rate of staying at the same address is above $90\%$ for the general population~\cite{ding2021retiring}.

\begin{figure*}[ht]
\centering
\includegraphics[width=.49\textwidth]{images/detector_auc_ACSMobility.png}\hfill
\includegraphics[width=.49\textwidth]{images/feature_importance_ACSMobility.png}
\caption{In the left figure, comparison of the performance of \textit{Demographic Parity Inspector}, in different pair-wise ethnicities for the ACS Mobility dataset. In the right image, the features that are more probable to be originating the demographic parity violation (higher implies more probable).}
\label{fig:xai.mobility}
\end{figure*}

In general, the values of the AUC of the \enquote{Demographic Parity Inspector} are in the range of $0.55$ to $0.80$. 
By inspecting the features we can see that our proposed methods highlight different demographic parity violation to be the source of the demographic parity violation depending on the ethnic pair-wise comparison selected. 
In general the feature \enquote{Ancestry}, i.e. "ancestors' lives with details like where they lived, who they lived with, and what they did for a living", plays a high relevance when predicting demographic parity violation in the mobility prediction task. 
In other prediction tasks such as when predicting employment (cf Figure \ref{fig:xai.employment}), even if the input data distribution across the protected groups is similar, the weight that the \enquote{Demographic Parity Inspector} assigns is distinct, this is due to our method measuring demographic parity violations with respect to the model not just with respect to the data.

