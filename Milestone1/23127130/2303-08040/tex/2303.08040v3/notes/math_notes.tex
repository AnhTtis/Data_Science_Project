\newpage
\section{Notes on Math}
\todo[inline]{Remove the notes before submitting}
There are two ways to formalize the math of your experiments and you need to decide for one
\subsection{The Random Variables Way}
This way models the random variables and each sample is a realization of these.
This approach focuses on more on the theoretical model behind the experiments.

\paragraph{Basic Notion.}
Each random variable is given by an upper case character and might contain an additional distribution name, e.g. $X^{ood} \sim  N(0,\begin{bmatrix}1 & c \\c & 1 \end{bmatrix})$ \todo{I suggest to keep the distribution name always as upper index instead of mixing upper and lower position}.

Different feature dimensions are written as lower index, e.g. $X=(X_1,X_2)$ or 
\[X^{ood}=(X^{ood}_1,X^{ood}_2) \sim  N(0,\begin{bmatrix}1 & c \\c & 1 \end{bmatrix})\]

Consequently, the target variable would be defined as: $Y=X_1\cdot X_2 + \epsilon$ with $\epsilon \sim N(0,0.1)$.

An equivalent notion would be $Y\sim N(X_1\cdot X_2, 0.1)$

\paragraph{Samples and Sample Size.}
For the experiments (especially with statistical tests), it is important to know the sample size. Just use sentences like \emph{For the experiment, we draw 1000 realizations/samples of $X$ and 500 realizations of $X^{ood}$}.

If you need a concrete sample (in many cases you can just use the random variable), use lower case characters and a sentence like: \emph{Let $x=(x_1,x_2)$ be a realization of $X$, ...} \todo{DO NOT write something like $x\in X$}


\subsection{The Data-Centeres Way}
This way models the concrete data and is much closer to the concrete implementation.

\paragraph{Basic Notion.}
Matrices are written as upper case characters and vectors as lower case characters. Again, the name should always be at the same position (suggestion:upper index).
The connection between data and distribution is given in textual form:
\emph{$X=(x_1,x_2)$ is a $n\times 2$ matrix of $n=1000$ samples drawn from $N(0,\begin{bmatrix}1 & c \\c & 1 \end{bmatrix})$}\todo{DO NOT use $\in$ or $\sim$. Curve brackets $\{\ldots\}$ indicate sets. I would not use them.}