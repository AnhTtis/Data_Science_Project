\documentclass{article}


% if you need to pass options to natbib, use, e.g.:
%     \PassOptionsToPackage{numbers, compress}{natbib}
% before loading neurips_2022


% ready for submission
%\usepackage{iclr2024_conference}
\usepackage[preprint]{neurips_2022}


% to compile a preprint version, e.g., for submission to arXiv, add add the
% [preprint] option:
%     \usepackage[preprint]{neurips_2022}


% to compile a camera-ready version, add the [final] option, e.g.:
%     \usepackage[final]{neurips_2022}


% to avoid loading the natbib package, add option nonatbib:
%    \usepackage[nonatbib]{neurips_2022}


\usepackage[utf8]{inputenc} % allow utf-8 input
\usepackage[T1]{fontenc}    % use 8-bit T1 fonts
\usepackage[hidelinks]{hyperref}       % hyperlinks
\hypersetup{%
  colorlinks=true,% hyperlinks will be coloured
  linkcolor=blue,% hyperlink text will be green
  citecolor=blue,
}
\usepackage{url}            % simple URL typesetting
\usepackage{booktabs}       % professional-quality tables
\usepackage{amsfonts}       % blackboard math symbols
\usepackage{nicefrac}       % compact symbols for 1/2, etc.
\usepackage{microtype}      % microtypography
\usepackage{xcolor}         % colors

%%%%%%%%%%%%%
%% MY PACKAGES
%%%%%%%
\usepackage{csquotes}
\usepackage{amsmath}
\usepackage{amsthm}
\usepackage{graphicx} 
\usepackage{comment}
\usepackage{pifont}% http://ctan.org/pkg/pifont
%\usepackage{todonotes}
\newcommand{\cmark}{\ding{51}}%
\newcommand{\xmark}{\ding{55}}%
\newcommand{\steffen}[1]{{\color{red}Steffen: #1}}
\newcommand{\carlos}[1]{{\color{purple}Carlos: #1}}
\newcommand{\antonio}[1]{{\color{green}Antonio: #1}}
% SR
\newcommand{\ind}{\mathrel{{\scalebox{1.07}{$\perp\mkern-10mu\perp$}}}}
\newcommand{\nind}{\mathrel{{\scalebox{1.07}{$\not\perp\mkern-10mu\perp$}}}}


%\newcommand{\steffen}[1]{{}}
%\newcommand{\carlos}[1]{{}}
% LAURA
\newcommand{\eti}[1]{{\scshape equal treatment inspector}}


%%%%%%%%%%%%%%%%%%%%%%%%%%%%%%%%
% THEOREMS
%%%%%%%%%%%%%%%%%%%%%%%%%%%%%%%%%
\theoremstyle{plain}
\newtheorem{theorem}{Theorem}[section]
\newtheorem{proposition}[theorem]{Proposition}
\newtheorem{lemma}[theorem]{Lemma}
\newtheorem{corollary}[theorem]{Corollary}
\theoremstyle{definition}
\newtheorem{definition}[theorem]{Definition}
\newtheorem{problem}{Problem}
\newtheorem{assumption}[theorem]{Assumption}
\theoremstyle{remark}
\newtheorem{remark}[theorem]{Remark}
\theoremstyle{definition}
\newtheorem{example}{Example}[section]
% My math
\newcommand{\bbox}{\text{bbox}}
\newcommand{\alphapck}{\alpha_\bbox}
\newcommand{\kcycle}{\text{k-CyPCK}}
\newcommand{\cycle}{\text{-CyPCK}}

\newcommand{\I}{\mathbf{I}}
\newcommand{\Ia}{\I^\text{a}}
\newcommand{\Ib}{\I^\text{b}}
\newcommand{\Iatob}{\I^\text{a $\rightarrow$ b}}
\newcommand{\F}{\mathbf{F}}
\newcommand{\Fa}{\F^\text{a}}
\newcommand{\Fb}{\F^\text{b}}
\newcommand{\f}{\mathbf{f}}
\newcommand{\fa}{\f^\text{a}}
\newcommand{\fb}{\f^\text{b}}
\newcommand{\p}{\mathbf{p}}
\newcommand{\pa}{\p^\text{a}}
\newcommand{\pb}{\p^\text{b}}
\newcommand{\A}{\boldsymbol{\Phi}_\text{align}}
\newcommand{\G}{\mathbf{G}}
\newcommand{\C}{\mathbf{C}}
\newcommand{\Ca}{\C^\text{a}}
\newcommand{\Cb}{\C^\text{b}}
\newcommand{\cc}{\mathbf{c}}
\newcommand{\cca}{\cc^\text{a}}
\newcommand{\ccb}{\cc^\text{b}}
\newcommand{\Irec}{\I_\text{Recon}}
\newcommand{\M}{\mathbf{M}}
\newcommand{\Mrec}{\M_\text{Recon}}
\newcommand{\loss}{\mathcal{L}}
\newcommand{\T}{\mathcal{T}}
\newcommand{\W}{\mathcal{W}}
\newcommand{\Id}{\mathcal{I}}


\title{Beyond Demographic Parity:\\ Redefining Equal Treatment}
% Measuring Fairness using  Explanation Distributions
% Political Philosophy Notions via Explanations Distributions
% Political Philosophy Notions via Shapley Distributions
% Political Philosophy Notions via Feature Attributions Distributions



% The \author macro works with any number of authors. There are two commands
% used to separate the names and addresses of multiple authors: \And and \AND.
%
% Using \And between authors leaves it to LaTeX to determine where to break the
% lines. Using \AND forces a line break at that point. So, if LaTeX puts 3 of 4
% authors names on the first line, and the last on the second line, try using
% \AND instead of \And before the third author name.


\author{%
  Carlos Mougan \\
 University of Southampton\\
 United Kingdom\\
\And
Laura State \\
Scuola Normale Superiore\\
 University of Pisa, Italy\\
\And
Antonio Ferrara\\
Gesis\\
RWTH Aachen University, Germany
\And
Salvatore Ruggieri\\
University of Pisa\\
Italy\\
\And
Steffen Staab\\
University of Southampton\\
University of Stuttgart, Germany
}


\begin{document}


\maketitle


\begin{abstract}

Liberalism-oriented political philosophy reasons that all individuals should be treated equally independently of their protected characteristics.
Related work in machine learning has translated the concept of \emph{equal treatment} into terms of \emph{equal outcome} and measured it as \emph{demographic parity} (also called \emph{statistical parity}).
Our analysis reveals that the two concepts of equal outcome and equal treatment diverge; therefore, demographic parity does not faithfully represent the notion of \emph{equal treatment}.
We propose a new formalization for equal treatment by (i) considering the influence of feature values on predictions, such as computed by Shapley values decomposing predictions across its features, 
(ii) defining distributions of explanations, and (iii) comparing explanation distributions between populations with different protected characteristics. We show the theoretical properties of our notion of equal treatment and devise a classifier two-sample test based on the AUC of an equal treatment inspector. We study our formalization of equal treatment on synthetic and natural data. We release \texttt{explanationspace}, an open-source Python package with methods and tutorials.
\end{abstract}

\section{Introduction}
\label{sec:introduction}
% \begin{itemize}
%     % Diffusion of FL
%     \item {\st{Diffusion of FL}}
%     % Security threats to FL
%     \item {\st{Security threats to FL with particular focus on model poisoning}}
%     % Limitations of existing countermeasures
%     \item {\st{Current countermeasures (e.g., KRUM) and their limitations}}
%     % Proposed method and its advantages
%     \item {\st{Intuitive description of the proposed method and its difference (i.e., advantages) w.r.t. state of the art}}
%     % Main contributions
%     \item {\st{Summary of the main contributions of this work}}
%     % Paper's structure and organization
%     \item {\st{Paper's structure and organization}}
% \end{itemize}

% Diffusion of FL
Recently, {\em federated learning} (FL) has emerged as the leading paradigm for training distributed, large-scale, and privacy-preserving machine learning (ML) systems~\cite{mcmahan2017googleai,mcmahan2017aistats}. 
The core idea of FL is to allow multiple edge clients to collaboratively train a shared, global model without disclosing their local private training data.
%Specifically, an FL system consists of a central server and many edge clients; 
A typical FL round involves the following steps: {\em(i)} the server randomly picks some clients and sends them the current, global model; {\em(ii)} each selected client locally trains its model with its own private data; then, it sends the resulting local model to the server;\footnote{Whenever we refer to global/local model, we mean global/local model {\em parameters}.} {\em(iii)} the server updates the global model by computing an \emph{aggregation function}, usually the average (FedAvg), on the local models received from clients.
% \begin{enumerate}
%     \item[{\em(i)}] the server sends the current, global model to the clients and appoints some of them for training;
%     \item[{\em(ii)}] each selected client locally trains its copy of the global model with its own private data; then, it sends the resulting local model back to the server;\footnote{Whenever we refer to global/local model, we mean global/local model {\em parameters}.}
%     \item[{\em(iii)}] the server updates the global model by computing an \emph{aggregation function} on the local models received from clients (by default, the average, also referred to as FedAvg~\cite{mcmahan2017aistats}).
% \end{enumerate}
This process goes on until the global model converges. %(e.g., after a certain number of rounds or other similar stopping criteria).
%\\
% The advantages of FL over the traditional, centralized learning paradigm are undoubtedly clear in terms of flexibility/scalability (clients can join/disconnect from the FL network dynamically), network communications (only model weights\footnote{We will use \textit{parameters} and \textit{weights} interchangeably.} are exchanged between clients and server), and privacy (each client's private training data is kept local at the client's end and not uploaded to the server).
\\
% Security threats to FL
%However, the growing adoption of FL also raises security concerns~\cite{costa2022covert}, particularly about its confidentiality, integrity, and availability.
Although its advantages over standard ML, FL also raises security concerns~\cite{costa2022covert}. %, particularly about its confidentiality, integrity, and availability~\cite{costa2022covert}.
% OLD, LONG VERSION
% Indeed, some work deals with privacy leakage that may expose the local data of some clients~\cite{melis2019sp}. 
% A large body of work, instead, investigates attacks that usually aim to detriment the predictive accuracy of the learned global model. For instance, \emph{data poisoning} attacks achieve this goal by letting an adversary pollute the training set of some corrupt FL clients with maliciously crafted examples~\cite{jagielski2018sp}.
% Similarly, in \emph{model poisoning} the attacker attempts to tweak the global model weights~\cite{bhagoji2019pmlr} by directly perturbing the local model's weights of some infected FL clients before these are sent to the central server for aggregation, usually via so-called Byzantine attacks. 
% It turns out that Byzantine model poisoning attacks severely impact standard FedAvg; therefore, more robust aggregation functions must be designed to make FL systems secure.
Here, we focus on \emph{untargeted model poisoning} attacks~\cite{bhagoji2019pmlr}, where an adversary attempts to tweak the global model weights %\footnote{We will use the terms \textit{parameters} and \textit{weights} interchangeably.} 
by directly perturbing the local model's parameters of some infected clients before these are sent to the central server for aggregation.
In doing so, the adversary aims to jeopardize the global model \textit{indiscriminately} at inference time.
Such model poisoning attacks severely impact standard FedAvg; therefore, more robust aggregation functions must be designed to secure FL systems.
\\
% In this paper, we focus on designing a novel robust aggregation scheme at the server's end to contrast the effect of Byzantine model poisoning attacks.
%
% Current countermeasures and their limitations
%Several countermeasures have been proposed in the literature to combat model poisoning attacks on FL systems.
% Some methods use simple statistics more robust than plain average to smooth the impact of malicious updates (e.g., Trimmed Mean and FedMedian~\cite{yin2018icml}). 
% Other defenses implement outlier detection techniques to discard malicious updates from the aggregation performed at the server's end. Those are either based on heuristics (e.g., Krum/Multi-Krum~\cite{blanchard2017nips} and Bulyan~\cite{mhamdi2018pmlr}) or data-driven approaches (e.g., K-means clustering~\cite{shen2016acm} or DnC via spectral analysis~\cite{shejwalkar2021ndss}). 
% Finally, some strategies rely on a centralized ``source of trust'' to spot potential malicious updates (e.g., FLTrust~\cite{cao2020fltrust}).
% Several countermeasures have been proposed in the literature to combat model poisoning attacks on FL systems, i.e., to discard possible malicious local updates from the aggregation performed at the server's end. 
% These techniques range from simple statistics more robust than plain average (e.g., Trimmed Mean and FedMedian~\cite{yin2018icml}) to outlier detection heuristics (e.g., Krum/Multi-Krum~\cite{blanchard2017nips} and Bulyan~\cite{mhamdi2018pmlr}) or data-driven approaches (e.g., spectral analysis via K-means clustering~\cite{shen2016acm} or spectral analysis), or methods based on ``source of trust'' (e.g., FLTrust~\cite{cao2020fltrust}).
% OLD, LONG VERSION
%Several countermeasures have been proposed in the literature to combat Byzantine model poisoning attacks on FL systems.
% Descriptive statistics
% For example, Trimmed Mean and FedMedian aggregate local model updates using more robust statistics than standard average~\cite{yin2018icml}.
%
% % Heuristics for outlier detection
% Many existing Byzantine-resilient strategies implement some outlier detection heuristics to discard the model updates sent by potentially malicious clients from the input of the aggregation function.
% One of the most popular heuristics is Krum~\cite{blanchard2017nips}.
% This strategy tries to mitigate the impact of Byzantine attacks by selecting as a global model the local model with the smallest sum of Euclidean distances to {\em all} the other local models.
% Although powerful, Krum requires the server to know (or, at least, estimate) the number of malicious FL clients upfront, which is generally impossible in a realistic attack scenario. %
% Moreover, Krum may become ineffective for complex, high-dimensional model parameter spaces due to the curse of dimensionality.
% Bulyan~\cite{mhamdi2018pmlr} tries to overcome this issue by combining Krum with a variant of Trimmed Mean.
% % Data-driven outlier detection
% Other strategies use data-driven outlier detection techniques -- e.g., via K-means clustering~\cite{shen2016acm} -- to spot potential malicious local model updates. 
% %For instance, Shen et al. propose to cluster local model updates with K-means and thus identify outliers.
%
% % Other techniques
% As far as the server is concerned, any local model received can be from a potential malicious client. 
% FLTrust~\cite{cao2020fltrust} assumes the server acts as a client, i.e., trains a local model on an additional {\em trustworthy} dataset at the server's end and compares it against all the local models from other clients. 
% This way, the server can rely on some ``source of trust'' when discarding potentially malicious clients.
%\\
% Limitations of existing Byzantine-resilient strategies
Unfortunately, existing defense mechanisms either rely on simple heuristics (e.g., Trimmed Mean and FedMedian by~\cite{yin2018icml}) or need strong and unrealistic assumptions to work effectively (e.g., foreknowledge or estimation of the number of malicious clients in the FL system, as for Krum/Multi-Krum~\cite{blanchard2017nips} and Bulyan~\cite{mhamdi2018pmlr}, which, however, cannot exceed a fixed threshold).
Furthermore, outlier detection methods using K-means clustering~\cite{shen2016acm} or spectral analysis like DnC~\cite{shejwalkar2021ndss} do not directly consider the temporal evolution of local model updates received.
Finally, strategies like FLTrust~\cite{cao2020fltrust} require the server to collect its own dataset and act as a proper client, thereby altering the standard FL protocol.
\\
% OLD, LONG VERSION
% Overall, existing Byzantine-resilient strategies are either simple heuristics (e.g., FedMedian) or, if they are more complex, they rely on strong and unrealistic assumptions to work effectively (e.g., knowing the number of malicious clients in the FL system in advance, as for Krum and alike).
% Furthermore, data-driven outlier detection methods do not consider the temporary evolution of local model updates received (e.g., K-means clustering). 
% Finally, strategies like FLTrust requires the server to collect its own dataset and act as a proper client, thereby altering the standard FL protocol.
%
% Description of the proposed method
This work introduces a novel pre-aggregation \textit{filter} robust to untargeted model poisoning attacks. Notably, this filter $(i)$ operates without requiring prior knowledge or constraints on the number of malicious clients and $(ii)$ inherently integrates temporal dependencies. 
The FL server can employ this filter as a preprocessing step before applying \textit{any} aggregation function, be it standard like FedAvg or robust like Krum or Bulyan.
Specifically, we formulate the problem of identifying corrupted updates as a multidimensional (i.e., matrix-valued) time series anomaly detection task. 
The key idea is that legitimate local updates, resulting from well-calibrated iterative procedures like stochastic gradient descent (SGD) with an appropriate learning rate, show \textit{higher predictability} compared to malicious updates. This hypothesis stems from the fact that the sequence of gradients (thus, model parameters) observed during legitimate training exhibit regular patterns, as validated in Section~\ref{subsec:intuition}. %until convergence. 
%This regularity may be more pronounced for smooth convex loss functions, but it can still be captured within an appropriate time window, even for more complex and convoluted loss surfaces. 
%We provide evidence of this claim in Appendix~B, where we show that the average mutual information (i.e., ``predictability''), calculated over pairs of legitimate model updates sent at different FL rounds, is significantly higher than the corresponding computation for a malicious client.
\\
Inspired by the matrix autoregressive (MAR) framework for multidimensional time series forecasting~\cite{chen2021je}, we propose the FLANDERS ({\em \textbf{F}ederated \textbf{L}earning meets \textbf{AN}omaly \textbf{DE}tection for a \textbf{R}obust and \textbf{S}ecure}) filter.
The main advantages of FLANDERS over existing strategies like FLDetector~\cite{zhao2020multivariate} are its resilience to large-scale attacks, where $50\%$ or more FL participants are hostile, and the capability of working under realistic non-iid scenarios.
We attribute such a capability to two key factors: $(i)$ FLANDERS works without knowing a priori the ratio of corrupted clients, and $(ii)$ it embodies temporal dependencies between intra- and inter-client updates, quickly recognizing local model drifts caused by evil players. Below, we summarize our main contributions:

\begin{itemize}
\item[{\em(i)}]
We provide empirical evidence that the sequence of models sent by legitimate clients is more predictable than those of malicious participants performing untargeted model poisoning attacks.
\\
\item[{\em(ii)}] 
We introduce FLANDERS, the first pre-aggregation filter for FL robust to untargeted model poisoning based on multidimensional time series anomaly detection.
\\
\item[{\em(iii)}] 
We integrate FLANDERS into Flower,\footnote{\scriptsize{\url{https://flower.dev/}}} a popular FL simulation framework for reproducibility.
\\
\item[{\em(iv)}] 
We show that FLANDERS improves the robustness of the existing aggregation methods under multiple settings: different datasets, client's data distribution (non-iid), models, and attack scenarios.
\\
\item[{\em(v)}] 
We publicly release all the implementation code of FLANDERS along with our experiments.\footnote{\scriptsize{\url{https://anonymous.4open.science/r/flanders_exp-7EEB}}}
\end{itemize}

% Paper's structure and organization
The remainder of the paper is structured as follows. %some related work and the current state-of-the-art solutions to security issues that FL entails. 
Section~\ref{sec:background} covers background and preliminaries. 
In Section~\ref{sec:related}, we discuss related work.
Section~\ref{sec:problem} and Section~\ref{sec:method} describe the problem formulation and the method proposed. % to tackle it. 
Section~\ref{sec:experiments} gathers experimental results. %, and Section~\ref{sec:limitations} discusses some limitations of this work.
Finally, we conclude in Section~\ref{sec:conclusion}.
 %discusses the limitations of this work and draws future research directions.
%reports conclusions and draws perspectives for future research directions.

%%%%%%% OLD %%%%%%%
%to overcome the resilience of Byzantine failures in distributed Stochastic Gradient Descent computations. 
% The strength of Krum is its time complexity, which is linear in the gradient dimension. 
% However, the robustness of the approach is guaranteed for gradient-based learning applications only when the majority of the clients are not compromised. 
% Besides, the aggregation mechanism of Krum, as well as that of similar methods, is robust from a coarse-grained perspective and does not provide solutions to errors and perturbations that may occur at inference time.
%A related approach to~\cite{blanchard2017nips} is the work of Su et al.~\cite{su2016dc}. Here, the authors propose an iterated approximate agreement to tackle a multi-layer scenario attacked by Byzantine agents. 
%However, the method works efficiently on the sole discrete context and it is inapplicable to continuous state environments.
%\gabri{Maybe, we should just talk about the main limitations of existing countermeasures without digging into their details (or, we can just mention Krum as this is the most popular one). I will move the description of all these methods to the Related Work section.}
\section{Related Work}
\label{sec:relatedwork}

%%%%%%%%%%%%%%%%%%%%%%%%%% Outline %%%%%%%%%%%%%%%%%%%%%%%%%%%%%%%%%%%%%
%(1) Evasion Attacks
%(1.1) Surveys on evasion attacks and their relation to data properties - Michael
%(1.2) Individual papers that study non-data related reasons behind evasion attacks - Michael
%(1.3) Techniques related to evasion attacks and defenses (new) - Gabby
%(2) Non-Evasion Attacks (new), and - ???
%(3) Effects of training data on standard generalization - done 
%
%
%
%(1) Evasion Attacks
%(1.1) A number of surveys review literature on evasion attacks. - Michael
%Most of them do not focus specifically on properties of data but also discuss attack and defense mechanisms, non-data-related reasons for adversarial vulnarability, and  more. ~\jr{cite 4}.
%Yet, they these surveys mention data and its relation to evasion attacks. Specifically \jr{what they say about data.}
%The most close to ours is concurrent work by XXX + concrete facts that we have and they don't.
%
%(1.2) individual papers that study non-data related reasons behind evasion attacks, - Michael
%Literature identifies multiple reasons for adversarial vulnerability, in particular, for evasion attacks. 
%These include data-related properties extensively discussed in this survey, as well as reasons related to the models 		   themselves, computations resources, and feature representations. We discuss these below. 
%
%\jr{the rest is from the paper (non-data related reasons for adversarial vulnerability), with sections potentially renamed.}
%
%{\bf Model.}
%
%{\bf Computational Resources.}
%
%{\bf Robustness of Features.}
%
%(1.3) Techniques Related to Evasion Attacks and Defenses (new) - Gabby
%A number of works focus on techniques for generating evasion attacks, countermeasures against these attacks, 
%and defining the notion of the attack itself.   
%
%{\bf Attacks and Defense.}
%Here are the 5 remaining surveys + 1 additional paper for the reviewer.
%
%{\bf Adversarial Examples.}
%2 surveys lines 13 and 14 + 1 additional paper for the reviewer.
%
%(2) Non-Evasion Attacks (new) 
%Need to say that there are other type of attacks, define them, cite surveys (Bo's survey, maybe something else). 
%Only one work explicitly focus on effects of data. 
%
%
%(3) Effects of training data on standard generalization (done)

%%%%%%%%%%%%%%%%%%%%%%%%% Outline %%%%%%%%%%%%%%%%%%%%%%%%%%%%%%%%%%%%%


\revreplace{
We divide related work into three categories:
(1) surveys on adversarial robustness and its relation to data properties,
(2) surveys that discuss the influence of data properties on standard generalization, and
(3) individual papers that study non-data-related reasons for adversarial vulnerability.\\
}
{
This survey investigates properties of training data in the context of model robustness under evasion attacks. 
We start the discussion of related work by reviewing other surveys that focus on evasion attacks and 
include some discussion about data (Section~\ref{sec:relatedwork-surveys-data}).  
We then discuss non-data related reasons behind evasion attacks (Section~\ref{sec:relatedwork-not-data}),
as well as techniques related to evasion attacks and defenses (Section~\ref{sec:relatedwork-attacks}). 
Finally, we discuss data-related concerns for non-evasion attacks (Section~\ref{sec:relatedwork-poisoning}) and
the effects of training data on standard generalization (Section~\ref{sec:relatedwork-standard}).
}

%\vspace{-0.1in}
\subsection{Surveys on Evasion Attacks that Discuss Data}
\label{sec:relatedwork-surveys-data}
Numerous existing surveys 
\revreplace{focus on attack and defense techniques for adversarial robustness. 
%~\cite{Biggio:Roli:PR:2018,
%Rosenberg:Shabtai:Elovici:Rokach:CSUR:2021,
%Li:Li:Ye:Xu:CSUR:2021,
%Maiorca:Biggio:Giorgio:CSUR:2019,
%Demetrio:Coull:Biggio:Lagorio:Armando:Roli:ACMTPS:2021,
%Liu:Tantithamthavorn:Li:Liu:CSUR:2022,
%Liu:Nogueria:Fernandes:Kantarci:IEEECST:2022,
%Akhtar:Mian:IEEEAccess:2018,
%Akhtar:Mian:Kardan:Shah:IEEEAccess:2021,
%Serban:Poll:Visser:CSUR:2020,
%Machado:Silva:Goldschmidt:CSUR:2021,
%Zhang:Sheng:Alhazmi:Li:ACMTIST:2020}.
Only a few of these works mention the relationship between adversarial robustness and properties of the underlying data.} 
{review the literature on evasion attacks.
Most of these works do not focus specifically on properties of data but discuss attack and defense mechanisms, non-data-related reasons for adversarial vulnerability, 
and the different threat models. 
Only a few of these works mention data-related reasons for the existence of adversarial examples~\cite{Serban:Poll:Visser:CSUR:2020, Machado:Silva:Goldschmidt:CSUR:2021, Akhtar:Mian:Kardan:Shah:IEEEAccess:2021, Akhtar:Mian:IEEEAccess:2018}.
}
Specifically, Serban et al.~\cite{Serban:Poll:Visser:CSUR:2020} observe that adversarial vulnerability can be caused by an insufficient training sample size %~\cite{Schmidt:Santurkar:Tsipras:Talwar:Madry:NeurIPS:2018}
and high data dimensionality. %~\cite{Gilmer:Metz:Faghri:Schoenholz:Raghu:Wattenberg:Goodfellow:ICLR:2018}.
Similarly, Machado et al.~\cite{Machado:Silva:Goldschmidt:CSUR:2021} mention that the lack of sufficient training data, high dimensionality, 
and high concentration contribute to adversarial vulnerability.
\revadd{
Akhtar et al.~\cite{Akhtar:Mian:IEEEAccess:2018, Akhtar:Mian:Kardan:Shah:IEEEAccess:2021} also mention high dimensionality, along with other non-data-related reasons, 
as a source of adversarial examples.}

\revadd{A concurrent work by Han et al.~\cite{Han:Lin:Shen:Wang:Guan:CSUR:2023} (published at the end of April 2023) 
studies the origins of adversarial vulnerability in deep learning w.r.t. the model, data, and other perspectives.
The authors mention high dimensionality, distributions with high concentration, a small number of output classes, data imbalance, and the perceptual difference in image frequencies as potential sources of adversarial examples.
However, as (a) the focus of that survey is not on data-related properties in particular, 
(b) its paper search was conducted in 2021, and 
(c) it focuses on deep learning models only, 
our work was able to identify more than 50 additional relevant papers which focus on other types of models, 
e.g., non-parametric and linear classifiers, 
and/or discuss additional types of data-related properties, 
such as, types of distribution, class density, separation, and label quality.}
\revreplace{Yet, none of these surveys explicitly collect and analyze work that focuses on the effects of data properties
on adversarial robustness.}
{In summary, by explicitly focusing on the effects of data properties on evasion attacks in our survey, 
we are able to provide a more complete and detailed discussion on this topic, not covered in prior surveys.}

\vspace{-0.05in}
\subsection{Non-data-related Reasons Behind Evasion Attacks}
\label{sec:relatedwork-not-data}

%\vspace{-0.1in}
%\subsection{Non-data Related Reasons for Adversarial Vulnerability}

There has been a variety of hypotheses regarding the reasons behind adversarial vulnerability of ML systems, particularly for evasion attacks.
%\revreplace{
%In addition to the data used for training,  adversarial robustness could also depend on the choice of the model architecture,
%the training procedure, and the interplay between data and the learning algorithm, i.e., correspondence between the complexity of a model to that of the data.
%This section summarizes the key hypotheses regarding these aspects.
%%The hypotheses reviewed in this section are complementary to the potential influence from the data.
%}
These include data-related properties extensively discussed in this survey, as well as reasons related to the models themselves, 
computational resources, and feature learning procedures. We discuss these below.

%\jr{there is a lot of undefined terminology and jargon in this section.}

\vspace{0.02in}
\noindent
\textbf{Model.}
When Szegedy et al.~\cite{Szegedy:Zaremba:Sutskever:Bruna:Erhan:Goodfellow:Fergus:ICLR:2014} first discovered adversarial examples for visual models, they suspected that the high non-linearity of DNNs resulted in low probability `pockets' of adversarial examples in the learned representation manifold.
They hypothesize that while these pockets can be found through attack algorithms, the samples residing in these pockets have different distributions compared to normal samples and are thus subsequently harder to find when randomly sampling from the input space.
Instead, Goodfellow et al.~\cite{Goodfellow:Shlens:Szegedy:ICLR:2015} hypothesize that
the linearity from activation functions, like ReLU and sigmoid found in high-dimensional neural networks, induce vulnerability towards adversarial perturbations.
To support their claim, they present the attack method FGSM that exploits the linearity of the target classifier.
Fawzi et al.~\cite{Fawzi:Fawzi:Frossard:ICMLWorkshop:2015} also argue against the hypothesis of high non-linearity as the cause for adversarial examples.
They show that all classifiers are susceptible to adversarial attacks and claim that it is the low flexibility of the classifier compared to the complexity of the classification task that results in vulnerability.
The lack of consensus on the primary causes of model vulnerability invites more studies on this topic.

Singla et al.~\cite{Singla:Ge:Basri:Jacobs:NeurIPS:2021} show that enforcing invariance to circular shifts (e.g., rotation) in neural networks induces decision boundaries with a smaller margin than normal, fully connected networks,
which, in turn, reduces the adversarial robustness of the model.
Moosavi{-}Dezfooli et al.~\cite{Moosavi-Dezfooli:Fawzi:Fawzi:Frossard:Soatto:ICLR:2018} introduce universal,
input-agnostic perturbations to mislead the classifier and hypothesize that the vulnerability of a multi-class classifier to such perturbations is related to the shape of its decision boundaries, e.g.,
linear classifiers with decision boundaries that are parallel to each other and
nonlinear classifier with decision boundaries that are curved in a similar way
tend to be less robust as
perturbations in one direction can change the prediction label for a different class.

Tanay and Griffin~\cite{Tanay:Griffin:ArXiv:2016} conjecture that the decision boundary learned by the classifier being too close to (or `tilted towards') the data manifold instead of being perpendicular to it,
results in small perturbations being sufficient to move samples across the decision boundary for misclassification.
%data manifold refers to the underlying structure that the data exhibit

\vspace{0.02in}
\noindent
\textbf{Computational Resources.}
Bubeck et al.~\cite{Bubeck:Lee:Price:Razenshteyn:ICML:2019} use computational hardness theory to show that the time complexity for learning a robust model is exponential to the size of input data and thus is computationally intractable.
Hence, they attribute adversarial vulnerability to computational limitations of current learning algorithms.
Degwekar et al.~\cite{Degwekar:Nakkiran:Vaikuntanathan:COLT:2019} further extend this work and also show the impossibility of efficiently training robust classifiers.

%\subsubsection{Ineffective Learning Perspective}
\vspace{0.02in}
\noindent
\textbf{Feature Learning.}
Ilyas et al.~\cite{Ilyas:Santurkar:Tsipras:Engstrom:Tran:Madry:NeurIPS:2019} show that adversarial vulnerability can be a consequence of a model exploiting well-generalizing but non-robust features,
i.e., features that are spurious and sometimes incomprehensible to humans;
when constraining the model to use robust features, the adversarial robustness increases together with the
interpretability of the learned features.
However, Tsipras et al.~\cite{Tsipras:Santurkar:Engstrom:Turner:Madry:ICLR:2019} note that, as the features for achieving high accuracy may be different from the ones for achieving high robustness, robustness may be at odds with standard accuracy.
%
%\jr{why is it called Ineffective learning when it is about features.}\gx{I put it under ineffective learning as in this case, the model learns/decides the features for generalization, and when given the correct objective, the model in fact, can learn more robust features, so I think the underlying reason is objective we gave for the model didn't guide the model to learn the right features}
%
Instead of seeing adversarial vulnerability as a product of classifiers being overly sensitive to changes in spurious features, Jacobsen et al.~\cite{Jacobsen:Behrmann:Zemel:Bethge:ICLR:2019} hypothesize that classifiers can rather be
overly insensitive to relevant semantic information, e.g., images with drastically different content can share similar latent representations.
The authors introduce a new type of adversarial examples that exploit such insensitivity, where the content of images is altered without changing the resulting prediction label.
%As both insensitivity to semantic content and sensitivity to spurious changes can simultaneously exist in models,
%more investigation into how to define proper objectives for models to effectively distinguish the relevant information is needed.

While all these works propose possible reasons for adversarial vulnerabilities, they are orthogonal to our survey, which focuses particularly on the influence of training data.

\vspace{-0.05in}
\revadd{
\subsection{Evasion Attacks and Defenses}
\label{sec:relatedwork-attacks}
A number of works focus on techniques for generating evasion attacks, countermeasures against these attacks, 
and defining the notion of the attack itself.

%\jr{need to include~\cite{Biggio:Roli:PR:2018,
%Rosenberg:Shabtai:Elovici:Rokach:CSUR:2021,
%Li:Li:Ye:Xu:CSUR:2021,
%Maiorca:Biggio:Giorgio:CSUR:2019,
%Demetrio:Coull:Biggio:Lagorio:Armando:Roli:ACMTPS:2021,
%Liu:Tantithamthavorn:Li:Liu:CSUR:2022,
%Liu:Nogueria:Fernandes:Kantarci:IEEECST:2022,
%Zhang:Sheng:Alhazmi:Li:ACMTIST:2020} x and one more survey.}
%\js{\cite{Biggio:Roli:PR:2018, Rosenberg:Shabtai:Elovici:Rokach:CSUR:2021} moved to Adversarial Examples.
%\cite{Rosenberg:Shabtai:Elovici:Rokach:CSUR:2021,
%Li:Li:Ye:Xu:CSUR:2021,
%Maiorca:Biggio:Giorgio:CSUR:2019, Liu:Tantithamthavorn:Li:Liu:CSUR:2022,
%Liu:Nogueria:Fernandes:Kantarci:IEEECST:2022,
%Zhang:Sheng:Alhazmi:Li:ACMTIST:2020, Demetrio:Coull:Biggio:Lagorio:Armando:Roli:ACMTPS:2021} in Attacks and Defense. \cite{Sun:Dou:Yang:Zhang:Wang:Philip:He:Li:TKDE:2022} was the "one more survey" and is also in Attacks and Defenses.}

\vspace{0.02in}
\noindent
{\bf Attacks and Defense.}
Several works~\cite{Liu:Tantithamthavorn:Li:Liu:CSUR:2022,Liu:Nogueria:Fernandes:Kantarci:IEEECST:2022,Sun:Dou:Yang:Zhang:Wang:Philip:He:Li:TKDE:2022, Demetrio:Coull:Biggio:Lagorio:Armando:Roli:ACMTPS:2021} survey adversarial attacks and defenses, observing that most work focuses on computer vision and NLP domains. 
Zhang et al.~\cite{Zhang:Sheng:Alhazmi:Li:ACMTIST:2020}, 
Rosenberg et al.~\cite{Rosenberg:Shabtai:Elovici:Rokach:CSUR:2021},
Li et al.~\cite{Li:Li:Ye:Xu:CSUR:2021}, and 
Maiorca et al.~\cite{Maiorca:Biggio:Giorgio:CSUR:2019}, 
survey attacks and defenses in the NLP domain, cybersecurity domain for networks, Android malware, and PDF malware, respectively. 
These works identify a similar trend of new attacks constantly bypassing defenses, which gives rise to new defenses being proposed, only to be broken again (a.k.a. the `cat and mouse race' or the `arms race'). 
They also observe that research in this field studies attacks / defenses at a feature-level, which restricts 
the practicality of the developed techniques by the feasibility of perturbing the corresponding features in real life. 

%practical attacks are quite difficult and require some basic knowledge about the model or training data such as the feature set or model architecture. 
%Zhang et al.~\cite{Zhang:Sheng:Alhazmi:Li:ACMTIST:2020}, who study adversarial attacks and defenses in the NLP domain,  
%also find that there are obstacles to generating attacks in real-time. 
%For instance, methods that iteratively use gradients to create adversarial examples can be time-consuming, while one-time approaches may fail to produce potent adversarial examples.
%Several works~\cite{Liu:Tantithamthavorn:Li:Liu:CSUR:2022,Liu:Nogueria:Fernandes:Kantarci:IEEECST:2022,Sun:Dou:Yang:Zhang:Wang:Philip:He:Li:TKDE:2022, Demetrio:Coull:Biggio:Lagorio:Armando:Roli:ACMTPS:2021} 
%discuss how most new attacks and defenses are explored in computer vision and NLP, prior to other fields.


%our survey finds the state of the art w.r.t. data properties
%our survey finds that dimensionality is bad ...
%
%%%Here are the 5 remaining surveys + 1 additional paper for the reviewer.
%Numerous surveys have explored the landscape of adversarial evasion attacks and defenses. 
%For instance, Akhtar et al.~\cite{Akhtar:Mian:IEEEAccess:2018, Akhtar:Mian:Kardan:Shah:IEEEAccess:2021} survey the literature on adversarial robustness of deep learning models from Computer Vision field.
%They review popular attacks on visual models, and provided a categorization of existing defense techniques based on the components it modify in the visual model system \gx{Check}.
%
%Rosenberg et al.~\cite{Rosenberg:Shabtai:Elovici:Rokach:ACMComputingSurvey:2021}, Li et al. ~\cite{Li:Li:Ye:Xu:ACMComputingSurvey:2021} and Demetrio et al.~\cite{Demetrio:Coull:Biggio:Lagorio:Armando:Roli:ACMTPS:2021} review the literature on evasion attacks for cyber-security fields. 
%Li et al. proposed a partial order scheme to compare key attacks and defenses techniques for malware detection in Windows, Android, and PDF domains. 
%
%Zhang et al.~\cite{Zhang:Sheng:Alhazmi:Li:ACMTIST:2020} review the literature on adversarial attacks on deep-learning models for textual classification.
%They pointed out the intrinsic differences between Computer Vision and Natural Language Processing fields that pose challenges to directly apply attacks proposed for Visual models to NLP models and identified the strategies proposed that overcomes the barriers.
%The challenges they identified for creating realistic attacks in NLP fields are from a domain characteristics perspective (e.g., definition of imperceptible perturbations, measurement of the semantic changes),  we differ from them by trying to understand the adversarial robustness of machine learning from the characteristics of underlying data. 
%
%Attack and Defenses for wireless and Mobile systems~\cite{Liu:Nogueria:Fernandes:Kantarci:IEEECST:2022}
%
%

More recent research, not included in the surveys above, has also started investigating the 
susceptibility of newer models to adversarial evasion attacks. 
For example, several studies~\cite{Wang:Pan:Hu:Duan:Pan:IJSWIS:2022,Yin:Lin:Sun:Wei:Chen:TIFS:2023, 
Shi:Han:Tan:Kuang:NeurIPS:2022, Wang:Xie:Microsoft:ChatGPT:ArXiv:2023} proposed attack techniques against contemporary models, 
such as Graph Neural Networks, Generative Pre-training Transformers (GPT), and Vision Transformers. 
These studies showed that adversarial examples persist even for the newer models, some of which are 
trained with large volumes of data. 
As all these works focus on attack and defense mechanisms rather than 
the effects of data on adversarial robustness, our work extends and complements this research.
}

\revadd{
\vspace{0.02in}
\noindent
{\bf Adversarial Examples.}
%2 surveys lines 13 and 14 + 1 additional paper for the reviewer.
Adversarial examples are inputs constructed by perturbing a correctly classified sample in a way that makes the change imperceptible to a human. % but causes the model to misclassify the sample.
However, as `imperceptible to a human' is hard to define, existing research on adversarial examples approximates imperceptibility with a small perturbation measured through $L_p$ norms.
A line of research~\cite{Gilmer:Adams:Goodfellow:Anderson:Dahl:ArXiv:2018,Sharif:Bauer:Reiter:CVPRW:2018,Fezza:Bakhti:Hamidouche:Deforges:QoMEX:2019, Mezher:Deng:Karam:EUVIP:2022} 
investigates the validity of this assumption. 
This work shows that perturbations generated by $L_p$ norms do not entirely align with human perceptions, 
i.e., some changes with a small $L_p$ norm can be apparent to humans. 
In addition, adversarial examples with the minimum $L_p$ perturbation may be less effective and transferable than 
higher perturbation~\cite{Biggio:Roli:PR:2018,Rosenberg:Shabtai:Elovici:Rokach:CSUR:2021}. 
Hence, a number of approaches explore metrics for imperceptibility 
in computer vision and NLP domains~\cite{Fezza:Bakhti:Hamidouche:Deforges:QoMEX:2019,Mezher:Deng:Karam:EUVIP:2022, Zhang:Sheng:Alhazmi:Li:ACMTIST:2020}. 
Yet another issue with $L_p$ norms is that they cannot be used reliably in domains other than images. 
For example, in the case of software/malware, simply generating adversarial examples with $L_p$ norms 
may result in feature representations that are not possible in 
the problem space~\cite{Rosenberg:Shabtai:Elovici:Rokach:CSUR:2021,Pierazzi:Pendlebury:Cortellazz:Cavallaro:2020}. 

While all these works focus on the properties of adversarial examples, 
they are orthogonal to the topic of our survey, as we rather focus on how properties of the training data 
affect the success of adversarial examples.
}

%Gilmer et al.~\cite{Gilmer:Adams:Goodfellow:Anderson:Dahl:ArXiv:2018} argue that, while constraining the perturbations by sufficiently small $L_p$ norms can generate indistinguishable samples for most inputs, the actual imperceptibility of the changes depends on the input sample. 
%Several individual studies~\cite{Sharif:Bauer:Reiter:CVPRW:2018,Fezza:Bakhti:Hamidouche:Deforges:QoMEX:2019, Mezher:Deng:Karam:EUVIP:2022} find faults with using $L_p$ norms to generate adversarial examples. They show that the changes measured by $L_p$ norm, does not entirely align with human perceptions, i.e., some changes with a small $L_p$ norm appear apparent to humans. 
%In some domains adversarial examples do not need to be imperceptible but rather semantically preserving. 
%For example, in the case of Android malware~\cite{Rosenberg:Shabtai:Elovici:Rokach:CSUR:2021}, adversarial examples are small perturbations which fool a model while preserving the semantics of the sample, 
%i.e., a malware stays malicious even after the perturbation. 
%This highlights another problem with $L_p$ norm based adversarial examples as Dong et al.~\cite{Dong:Liu:Shang:NeurIPS:2022} show that the semantics of a sample change during adversarial training. 
%Hence, there is a need for metrics to measure the size of perturbations that is imperceptible or semantically preserving.
%Fezza et al.~\cite{Fezza:Bakhti:Hamidouche:Deforges:QoMEX:2019} and Mezher et al.~\cite{Mezher:Deng:Karam:EUVIP:2022} propose to use objective metrics for image quality to approximate the imperceptibility in the computer vision domain.
%Zhang et al.~\cite{Zhang:Sheng:Alhazmi:Li:ACMTIST:2020}, focusing on providing such a metric for Natural Language Processing.
%Vadillo et al.~\cite{Vadillo:Santana:CS:2022} also highlight conducted subject studies to evaluate the noticeability of audio adversarial examples.

%Even in computer vision, adversarial examples are not always imperceptible. For example, Machado et al.~\cite{Machado:Silva:Goldschmidt:CSUR:2021} find that visible perturbations such as adversarial patch~\cite{Brown:Mane:Roy:Abadi:Gilmer:ArXiv:2017}, and graffiti on stop signs~\cite{Eykholt:Evtimov:Fernandes:Li:Rahmati:Xiao:Prakash:Kohno:Song:CVPR:2018} are also considered adversarial examples in research.

%The aforementioned research examines the work on defining and creating adversarial examples, demonstrating the insufficiency of using conventional $L_p$ norms to evaluate the imperceptibility and semantics between clean and adversarial examples. 

\vspace{-0.1in}
\revadd{
\subsection{Non-Evasion Attacks}
\label{sec:relatedwork-poisoning}
Similar to evasion attacks, data poisoning and backdoor attacks aim to compromise model accuracy. 
However, they achieve it by tampering the training data to create deceptive model decision boundaries. 
%Data poisoning attacks involve modifying the training data to create deceptive decision boundaries, either to manipulate the prediction outcomes of a specific input or the entire model.
%Meanwhile, Backdoor attacks are a form of poisoning attacks where the attacker inject tempered training data with triggers 
% and then activates the attack by showing the trigger pattern at inference time.
In addition, backdoor attacks also require perturbing the test instance to result in a misclassification. 
This is achieved by introducing manipulated training data with triggers that can be activated during the testing phase.

Goldblum et al.~\cite{Goldblum:Tsipras:Xie:Chen:Schwarzchild:song:Madry:Li:Goldstein:TPAMI:2022} and Cinà et al.~\cite{Cina:Grosse:Demontis:Sebastiano:Zellinger:Moser:Oprea:Biggio:Pelillo:Roli:CSUR:2023} 
review recent literature on attack methodologies and countermeasures for both poisoning and backdoor attacks.
Both of these surveys found that existing research made overly-optimistic assumptions when designing / validating attack techniques, e.g., assuming the knowledge of a large portion of training data. 
They advocate for researchers to test proposed methods in more realistic situations to better assess the potential threats. 
Furthermore, they encourage exploration of the relationship between poisoning attacks and evasion attacks. 
This could lead to the creation of attacks that produce less noticeable poisoning examples, 
or defensive strategies that can safeguard models against both backdoor and evasion attacks.
%Their survey catalogs and systematizes the threats in the dataset creation process, and discuss the open problems that benefits the understanding of dataset security. 

In addition to undermining model accuracy, 
adversarial attacks also aim at breaching the privacy and confidentiality of training data. 
In particular, membership inference attacks~\cite{Shokri:Stronati:Song:Shmatikov:SP:2017} attempt to determine whether a specific data point was part of the training set used to train the model.
Hu et al.~\cite{Hu:Salcic:Sun:Dobbie:Yu:Zhang:CSUR:2022} present a comprehensive survey of existing research efforts on membership inference attacks. 
They find that, similar to evasion attacks, the membership inference attack success rate decreases as 
%the training data better represents the whole data distribution, i.e., 
the number of training samples increases.
%and model stealing attacks~\cite{Oliynyk:Mayer:Rauber:CSUR:2023} are designed to breach the privacy of training data and machine learning models. 
However, all these attacks are orthogonal to our survey, as we focus on adversarial evasion attacks.

%Li et al. ~\cite{Li:Jiang:Li:Xia:TNNLS:2022} 
%provide the first survey that focuses on backdoor attacks and identified common scenarios in which backdoor attack happen in real life. 
%Furthermore, they proposed a systematic taxonomy for backdoor attacks and defenses for researchers and practitioners to identify the characteristics and limitations of each method. 

%Wang et al.~\cite{Wang:Ma:Wang:Hu:Qin:Ren:CSUR:2022} and Tian et al.~\cite{Tian:Cui:Liang:Yu:CSUR:2022} argue federated learning~\cite{McMahan:Moore:Ramage:Hampson:Arcas:AISTATS:2017} 
%creates new venue for poisoning attack, and survey recent literature on poisoning attacks for both standard and federated learning scenarios. 
%They present a unified framework to categorize both data poisoning and model poisoning attacks, and compared the defense techniques proposed for each of the learning framework, analyzed their advantages and disadvantages.
}

\vspace{-0.1in}
\subsection{Effects of Training Data on Standard Generalization}
\label{sec:relatedwork-standard}
A number of surveys investigate the influence of data properties on standard
rather than robust generalization.
One of the earliest is probably the work of Raudys and Jain~\cite{Raudys:Jain:TPAMI:1991},
who review studies related to the influence of sample size on binary classifiers, showing that
a limited sample size usually leads to sub-optimal generalization.
%With the development of deep learning and the ever-increasing need for larger training datasets,
%a variety of data augmentation techniques have been proposed.
Bansal et al.~\cite{Bansal:Sharma:Kathuria:CSUR:2021} and
Bayer et al.~\cite{Bayer:Kaufhold:Reuter:CSUR:2022} also survey papers addressing the data scarcity problem,
focusing in particular on the recent advancements in data augmentation techniques in the fields of computer vision, security, and text classification.
Their results show that augmentation techniques %exist for various application domain and
can help improve a model's generalization by reducing the problem of model overfitting.
%They evaluate the effectiveness of such techniques in improving the accuracy of machine learning models.

%Limited sample size is also one of the culprit behind poor robust generalization~\cite{Schmidt:Santurkar:Tsipras:Talwar:Madry:NeurIPS:2018}, we collected a number of researches characterize the sample complexity for robust generalization or propose data augmentation techniques to fill in the sample complexity gap.

Label noise is another aspect of data that influences both standard and robust generalization.
Most works on this topic find that the presence of noisy labels increases the need for a greater number of training samples and may result in unnecessarily complex decision boundaries~\cite{Frenay:Verleysen:TNNLS:2014,Song:Kim:Park:Shin:Lee:TNNLS:2022}.
For example, Fr\'{e}nay and Verleysen~\cite{Frenay:Verleysen:TNNLS:2014} show
that overfitting to label noise greatly degrades a model's standard generalization;
the same effect has been observed in the case of robust generalization~\cite{Sanyal:Dokania:Kanade:Torr:ICLR:2021}.
Song et al.~\cite{Song:Kim:Park:Shin:Lee:TNNLS:2022} survey the impact of label noise in deep learning, arguing
that the presence of noisy labels is a more serious concern for deep models as they contain a larger number of parameters which makes them prone to overfitting to the noise in training data.
%They also point out the connection between adversarial poisoning attacks and noisy labels as
%the countermeasures for both share the goal of learning noise-resilient representations.
They mention that adversarial defense techniques, e.g., adversarial training, are effective against label noise~\cite{Zhu:Zhang:Han:Liu:Niu:Yang:Kankanhalli:Sugiyama:ArXiv:2021, Fatras:Damodaran:Lobry:Flamary:Tuia:Courty:TPAMI:2022}
but do not discuss how label noise influences a deep learning model's robustness under attacks.

Lorena et al.~\cite{Lorena:Garcia:Lehmann:Souto:Ho:CSUR:2020} identify a collection of 26 quantitative metrics that measure data complexity with respect to
(1) ambiguity of classes, i.e., whether the classes can be clearly distinguished with the given features,
(2) sparsity and dimensionality of data, 
%i.e., whether enough information are provided to learn confident decision boundaries, and
(3) complexity of boundary separating the classes, i.e., whether more intricate functions are required to describe the decision boundaries.
The authors also discuss how these metrics help estimate the difficulty of performing classification on a given dataset.
Similar to our survey, the authors show that high dimensionality and small separation between classes hinder standard generalization.
However, the relationship of some of the metrics reviewed by these authors, e.g.,
%faction of borderline points (i.e., a measure for the complexity of the required decision boundary) and
%the fraction of hyperspheres covering data (i.e.,
the number of non-intersecting spheres needed to enclose all data points of a class,
to robust generalization is not studied, according to our survey.

%Moreover, the effect of XXX on standard generalization needs future investigation as well (that is if we found something they do not have).

%Knowing the characteristics of a dataset according to these perspectives can assist researchers and practitioners to select optimal learning algorithms~\cite{Ho:Basu:TPAMI:2002}.

He and Garcia~\cite{He:Garcia:TKDE:2009} focus on the imbalance learning problem. %~--
%the disproportion in the number of samples belonging to each class in a given dataset.
The authors found that most standard algorithms %are designed with the assumption of a balanced class distribution.
%These algorithms
fail to reliably represent the characteristics of the imbalanced data and result in unfavorable performance across classes.
Furthermore, L\'{o}pez et al.~\cite{Lopez:Fernandez:Garcia:Palade:Herrera:InfSci:2013} discuss six intrinsic data characteristics that potentially complicate learning from imbalanced data:
low density, sample overlap between classes, noisy data, borderline instances,
dataset shift between training and testing distributions, and
small disjuncts, i.e., disperse small clusters of samples from a single class.
Their analysis concludes that while all these ``unfavorable'' data characteristics further complicate the data imbalance
issues, data overlap between classes is probably one of the most harmful.
To follow up on this point, Santos et al.~\cite{Santos:Henriques:Pedro:Japkowicz:Fernandez:Soares:Wilk:Santos:AIR:2022}
focus on the joint effect of data imbalance and class overlap on model generalization.
The negative impact of data imbalance, low separation, and noisy data on robust generalization was also discussed in our survey.
Yet, the compounding effect of these factors, as well as the effect of other properties,
on robust generalization needs future investigation.

Recently, Yang et al.~\cite{Yang:Jiang:Song:Guo:IJCV:2022} summarized relevant studies focusing on
long-tailed distributions in the field of Computer Vision.
% and categorize the main methods for alleviating the issues caused by long-tailed distribution.
%They present quantitative metrics for measuring data imbalance and .
This survey also includes work on the influence of long-tail distributions on a model's adversarial robustness~\cite{Wu:Liu:Huang:Wang:Lin:CVPR:2021}, which is covered in our survey.
%which is included in our survey,
The authors advocate for more research on adapting long-tailed-based approaches for standard generalization to improve robust generalization.

Finally, Moreno-Torres et al.~\cite{MorenoTorres:Raeder:Rodrigues:Chawla:Herrera:PR:2012} present a unifying framework to categorize existing definitions of dataset shift~-- the case where the joint distribution of inputs and outputs differs between training and testing data.
While ML models are normally trained under the premise that testing data has a similar distribution to the training data,
in reality, the observed data distribution may be different from the historical data that the model is trained on.
Such difference can substantially compromise the quality of model predictions.
The authors analyze the possible causes for dataset shift, e.g., malicious software that evolves over time, and
review the techniques dealing with dataset shift.
They characterize adversarial attacks as one form of dataset shift, where adversaries adaptively
change test instances to create a distribution that differs from training data.
%All works discussed in our survey assumed similar distribution on training and testing data, treating adversarial attacks as the only dataset shift in the problem setup.
%However, in real applications, the underlying data distribution itself can be non-stationary, and the characterize the influence of the dataset shift between training and testing data on the adversarial robustness is yet to be investigated.

\revadd{Overall, despite the similarities with our work, literature discussed in this section focuses on standard generalization while our survey discusses 
the effect of data on robust generalization.}

%More works use the connection between adversarial attacks and distributional shift to analyze the effect of adversaries on generalization performance~\cite{Tu:Zhang:Tao:NeurIPS:2019}.
%However, we do not discuss them in detail, as they focus more on models instead of data.
%\jr{How is that relevant to data properties section?} \gx{This can be removed, as it an individual work we filtered}

\vspace{-0.1in}
\subsection{Summary}
\revadd{
Our survey is the first to explicitly focus on properties of training data in the context of model robustness under evasion attacks.
Numerous other surveys on evasion attacks discuss attack and defense mechanisms, non-data-related reasons for adversarial vulnerability, and the different threat models. 
We identified only five surveys that considered data-related reasons for evasion attacks. 
However, as these surveys are older and do not focus on data in particular, our work provides a more extensive
and comprehensive view on this topic. 
By including more than 50 papers not covered in prior work, we were able to 
identify additional relevant properties, practical suggestions, and future research directions in this area. 

Additional work studies non-data-related reasons for evasion attacks, as well as non-evasion attacks, 
such as poisoning and backdoor. 
Yet another body of literature examines how data properties affect standard generalization. These works show that 
some of the properties discussed in our survey, such as 
the number of samples, dimensionality, and label quality, also affect clean accuracy. 
There are also additional data properties that are covered exclusively by these or by our work. 
Studying the interplay between data properties for clean and robust accuracy is an interesting research direction, 
which could be facilitated by our work. 
However, all these current works are orthogonal and complementary to ours.
}

%\ad{
%The related work of our survey can be categorized into four key topics: 
%The first topic examines data for other adversarial attacks, this include the research that investigates the link between the data characteristics and model's resilience against poisoning attacks as well as the studies that explore data poisoning and backdoor attacks and their countermeasures. \jr{same issues as before: this is meta-summary, we need a concrete summary.}
%These studies complement our survey as they highlight the threats directly aimed at data, thus emphasizing the importance of secure data collection. 
%The second topic focuses on the relationship between various properties of training data and model's standard generalization ability. 
%This body of work suggests that data traits such as number of samples, dimensionality, label quality also influence model's ability to generalize in standard classification. \jr{this looks more concrete!}
%
%The third strand of research concerns adversarial evasion attacks. 
%The work in this area encompasses the research frontier in evasion attacks and the countermeasures. 
%Due to the large volume of work in this area, there are numerous surveys that gives more detail on the advancement. 
%\jr{meta-summary again}
%In addition to attacks and defenses, one relevant line of work investigates the alignment of the conventional similarity metrics used for adversarial examples and human perception, showing the need for supplementary metrics. \jr{why important?}
%These studies \jr{which "these studies"?} collectively present an extensive overview of other types of work conducted on adversarial robustness.
%The last category of work proposes alternative explanations for model vulnerability to adversarial examples.
%These studies presented hypothesis showing the characteristics of machine learning models, e.g., nonlinearity, invariance to rotational shift etc, induces susceptibility to attacks, as well as limited computational resources and non-robust feature representations. \jr{all text based on previous related work looks somewhat concrete; the new additions should be at least at the same level, or better.}
%These studies supplement our work, offering a broader perspective of potential factors affecting model's robust generalization ability. }
%


\section{Method}
\label{s:method}

We consider the 3D euclidean space $\Real^3$ with points $p=(x,y,z)\in\Real^3$. We discretize the unit cube $\gC=[0,1]^3$ with a 3D voxel grid $\gG=\set{p_I}$, with nodes $p_I$ indexed by $I=(i,j,k)$, $i,j,k\in [n]=\set{1,\ldots,n}$, \ie, $p_I=(x_{ijk},y_{ijk},z_{ijk})$. We denote by $h=n^{-1}$, and by $N=n^3$ the total number of nodes.   
We represent our reconstructed surface as a zero level of a scalar function $f$ defined over the cube $\gC$. $f$ is defined by prescribing its values at the grid's nodes $f_I\in\Real$ and trilinear interpolating in each voxel. We will denote by $f(p)$ the interpolated value at point $p$. 

Given an input point cloud consisting of $m$ points $q_k\in\Real^3$ with or without (unit norm) normals $n_k\in \Real^3$, $k\in [m]$, our goal is to compute $f$ so that its zero level set approximates the unknown surface, \ie, 
\begin{equation}
    \gS = \set{p\in\gC \ \vert \ f(p)=0}.
\end{equation}
Our approach to compute $f$ is to minimize a loss function of the form
\begin{equation}
    \gL = \gL_{\text{data}} + \gL_{\text{prior}}
\end{equation}
where 
\begin{equation}\label{e:loss_data}
    \gL_{\text{data}} = \frac{\lambda_{\text{p}}}{m}\sum_{k=1}^m \abs{f(q_k)}^2 + \frac{\lambda_{\text{n}}}{m}\sum_{k=1}^m \norm{\nabla f(q_k) - n_k}^2
\end{equation}
where $\norm{\cdot}$ is the standard euclidean norm in $\Real^3$, $\nabla f(p) \in \Real^3$ is the gradient of $f$ sampled at point $p$. Note that $\nabla f$ is defined in interior of voxels, which is generically where the input points $q_k$ resides. $\gL_{\text{data}}$ is the standard data loss encouraging the zero level to pass through the input points $q_k$, and its normals (defined by gradients of $f$) to coincide with input normals $n_k$. 

The prior, $\gL_{\text{prior}}$, is the main contribution of this work, where we combine two novel losses,
\begin{equation}
    \gL_{\text{prior}} = \lambda_{\text{v}} \gL_{\text{viscosity}} + \lambda_{\text{c}} \gL_{\text{coarea}}
\end{equation}
Intuitively, the viscosity loss optimizes for a smooth Signed Distance Function (SDF) solutions, avoiding auxiliary bad minima of the Eikonal equation, while the coarea loss strives to minimize the area of the zero level surface. Our loss has $4$ hyper-parameters $\lambda_{\text{p}},\lambda_{\text{n}},\lambda_{\text{v}},\lambda_{\text{c}}$. We provide more details on these priors next. 


\subsection{Viscosity Loss}\label{ss:viscosity_loss}
The goal of the viscosity loss is to make $f$ approximate an SDF over $\gC$. Given boundary conditions asking $f$ to vanish on some closed compact surface $\gS$, the SDF solves the Eikonal equation PDE, \ie, $\norm{\nabla f(p)}=1$, in a certain well defined sense (viscosity). This motivated some previous work to directly optimize the Eikonal loss \citep{gropp2020implicit,sitzmann2020implicit}
\begin{equation}\label{e:loss_eikonal}
    \gL_{\text{eikonal}} = \int_\gC \Big (\norm{\nabla f(p)}-1\Big )^2 dp
\end{equation}
\begin{wrapfigure}[14]{r}{0.28\textwidth}\vspace{-15pt}
  \begin{center}
    \includegraphics[width=0.25\textwidth]{figs/illustrations/eikonl_1d.png}
  \end{center}
  \caption{Two global minimizers of the Eikonal loss over a grid in 1D. Top solution is not an SDF. }\label{fig:eikonal_1d}
\end{wrapfigure}
Unfortunately, the Eikonal loss has many undesirable minima which are not SDFs. Figure \ref{fig:eikonal_1d} shows a 1D example: both depicted solutions (denoted $f$) vanish at the input points $q_1,q_2$ (black points) and globally minimize the Eikonal loss over the grid (grid points are shown in blue). The INR works mentioned above use neural networks for representing $f$ which injects an inductive bias avoiding these bad minima, however on grids, minimizing \eqref{e:loss_eikonal} cannot avoid these solutions. See, \eg, middle column in Figure \ref{fig:teaser}. 

More classical Eikonal solvers do work with grids however use mostly fast marching or sweeping methods \citep{osher1988fronts,sethian1996fast,zhao2005fast,chacon2012fast}. Namely, use a special discretization of the Eikonal equation favoring the viscosity  solution of the Eikonal \cite{rouy1992viscosity}, and update node values according to a moving front \cite{sethian1996fast}. Since this discretization is up-wind (will only propagate values in one direction) and requires choosing the maximal among its solution, its success in adaptation to a loss is not clear. 

We use a different approach to build a loss favoring SDF solutions over grids motivated by the vanishing viscosity method \cite{crandall1983viscosity}. Namely, adding to the Eikonal PDE a small perturbation of the Laplacian of $f$ (denoted by $\Delta f$), \ie, $\norm{\nabla f(p)}-1 - \eps\Delta f(p)=0$, makes the PDE semi-linear elliptic \citep{calder2018lecture}, and hence with suitable boundary conditions it is uniquely solvable inside $\gS$ with a smooth solution, approaching the viscosity positive distance function to the boundary as $\eps\too 0$. Similarly, for $1-\norm{\nabla f(p)} - \eps \Delta f(p)=0$ the solution approaches the negative distance function inside the domain. 
Motivated by the vanishing viscosity principle we suggest the following viscosity loss:
\begin{equation}\label{e:loss_viscosity_eikonal}
\gL_{\text{viscosity}} = \int_\gC \Big((\norm{\nabla f (p)}-1)\mathrm{sign}(f(p)) - \eps \Delta f(p)\Big)^2 dp
\end{equation}
We discretize this loss over the grid $\gG$ by replacing the first order derivatives and second order derivatives with symmetric finite  differences, \ie,
\begin{align*}
    D_x f_I=D_x f_{i,j,k} = \frac{f_{i+1,j,k}-f_{i-1,j,k}}{2h}, \quad D^2_x f_I = D^2_x f_{i,j,k}=\frac{f_{i+1,j,k}-2f_{i,j,k}+f_{i-1,j,k}}{h^2}
\end{align*}
and similarly for $D_y$ and $D_z$. We use these discrete operators to approximate the gradient $\widehat{\nabla} f(p_I) = (D_x f_I, D_y f_I, D_z f_I)$ and Laplacian $\widehat{\Delta}f(p_I) = D_x^2f_I + D_y^2 f_I + D_z^2 f_I$. The discretized viscosity loss now takes the form
\begin{equation}
    \widehat{\gL}_{\text{viscosity}} = \frac{1}{N}\sum_{I} \Big((\|\widehat{\nabla} f (p_I)\|-1)\mathrm{sign}(f(p_I)) - \eps \widehat{\Delta} f(p_I)\Big)^2
\end{equation}



\subsection{Coarea loss}\label{ss:coarea_loss}
The coarea loss is approximating the area of the zero level set, and therefore incorporating it in the optimization pushes the reconstructed surface to be economic in area. 

First, similarly to  \citep{yariv2021volume} we use the centered Laplace CDF
\begin{equation}
   \Psi\beta(s)= \begin{cases}
   \frac{1}{2}\exp\parr{\frac{s}{\beta}} & s\leq 0 \\ 1-\frac{1}{2}\exp\parr{-\frac{s}{\beta}} & s\geq  0
   \end{cases}
\end{equation} to transform the SDF $f$ to a smooth approximation of the indicator function:
\begin{equation}
    \chi_\beta(p)=\Psi\beta (-f(p))
\end{equation}
As $\beta\too 0$, $\chi_\beta$ converges to an indicator function leading to $1$ inside $\gS$ and $0$ outside. The coarea loss is defined as 
\begin{equation}
    \gL_{\text{coarea}} = \int_\gC \norm{\nabla \chi_\beta (p)} dp
\end{equation}
To understand why this loss approximates the area of $\gS$ we can use the coarea formula \citep{rindler2018calculus}:
\begin{equation}\label{e:coarea}
    \int_\gC \norm{\nabla \chi_\beta(p)}dp = \int_{-\infty}^{\infty} \mathrm{area}(\chi_\beta^{-1}(s))ds,
\end{equation}
where $\chi_\beta^{-1}(s)=\set{p\ \vert \ \chi_\beta(p)=s}$ is the preimage of the value $s$. Since $\chi_x(p)\in [0,1]$ the r.h.s.~integral can be restricted to the interval $[0,1]$, and therefore the coarea loss averages the area of the level sets of $\chi_\beta$. Next,  $$\chi_\beta^{-1}(s)= \set{p\ \vert \ \Psi\beta (-f(p)) = s } = \{p\ \vert \ f(p) = -\Psi\beta^{-1} (s) \} = f^{-1}(-\Psi\beta^{-1} (s)),$$
\begin{wrapfigure}[11]{r}{0.28\textwidth}\vspace{-20pt}
  \begin{center}
  \includegraphics[width=0.25\textwidth]{figs/semi.png}
  \end{center}
  \caption{Reconstruction of a semisphere point cloud (white dots) without (left) and with (right) coarea loss. }\label{fig:coarea_semisphere}
\end{wrapfigure}

which shows that the level set $s\in (0,1)$ of $\chi_\beta$ is the level set $-\Psi\beta^{-1}(s)$ of the SDF $f$. As $\beta\too 0$, $-\Psi\beta^{-1}(s)\too 0$ for all $s\in (0,1)$ (and uniformly in $(\eps,1-\eps)$ for fixed $\eps>0$). Therefore the average of the level set area (\ie, the r.h.s.~of \eqref{e:coarea}) converges to the area of $f^{-1}(0)=\gS$. Figure \ref{fig:teaser} (right) shows how removing the coarea loss introduces an extraneous zero level set, and hence results in an undesired surface part. Figure \ref{fig:coarea_semisphere} shows a comparison of a reconstruction of semisphere with and without coarea. In the experiments section we provide more ablation tests with the coarea and viscosity losses.

To discretize the coarea loss we let $w_I$ denote the centers of grid's voxels, and note that $\nabla \chi_\beta(w_I) = \Phi_\beta(-f(w_I))\nabla f(w_I)$, where 
\begin{equation*}
    \Phi_\beta(s) = \frac{1}{2\beta}\exp\parr{\frac{\abs{s}}{\beta}}
\end{equation*}
is the PDF of the Laplace distribution, and $\nabla f(w_I)$ is computed as a linear combination of the voxel's corner values $f_{I_1},\ldots,f_{I_8}$, see more details in the Appendix. We end up with the discretized loss:
\begin{equation}
    \widehat{\gL}_{\text{coarea}} = \frac{1}{N}\sum_{I}\Phi_\beta(-f(w_I))\norm{\nabla f(w_I)}
\end{equation}
This loss is usually incorporated with a small hyper-parameter $\lambda_{\text{c}}$ with the purpose of eliminating redundant surface parts.


% !TEX root = ./CauchyCombination.tex
\section{Multiple Hypothesis Testing} \label{secPrelims}

This section introduces the notation on multiple hypothesis testing and the benchmark procedures for addressing the multiple testing problem. 

\subsection{Setting}
Let $H_{i}$ denote the $i^{\text{th}}$ null hypothesis of interest, with $i=1,...,d$,  
and $d$ being the total number of individual hypotheses. To test the $d$ hypotheses, we can use the associated vector of test statistics $\bm{X}=(X_{1},X_{2},\ldots,X_{d})^{^{\prime }}$, one for each hypothesis being tested, or the corresponding raw $p$-values $p_{1},\ldots ,p_{d}$. The test statistics can be independent or % corrected.
correlated. 

%In some cases, like in Section \ref{secApplDriftBurst}, the test statistics are constructed from rolling windows and are extremely serially correlated. 

The first task is to test the global null hypothesis. Let $\mathcal{H}_{0}$ be the collection of null hypotheses of interest. 
The strategy of a classical global test is to abandon the multiplicity issue altogether and replace multiple tests with the global null hypothesis that all elementary hypotheses are true.  The alternative is that at least one elementary hypothesis is false. For example, in high-frequency financial econometrics,  we often need to monitor the presence of certain events (e.g., jumps or drift bursts) within a fixed time period (e.g., within a day). The global null is that there is no occurrence of such an event at all (e.g., 
% in Example 2, none of the stocks has a significant alpha
in Example 1, there is no drift burst within the day or in Example 2, none of the stocks has a significant alpha).
The goal is to get $\alpha$-level control under this global null, i.e., $P_{\mathcal{H}_0} [\text{reject}\, \mathcal{H}_0] \leq \alpha$. The test is conservative when $P_{\mathcal{H}_0} [\text{reject}\, \mathcal{H}_0]$ is strictly less than the theoretical upper bound $\alpha$ and ideal when it is equal to  $\alpha$. 

When, by any  test, the global null $\mathcal{H}_0$ is rejected, the second task is to identify which of the elementary hypotheses $H_{i}$ should be rejected. The set of true hypotheses $\mathcal{T}$, the set of false hypotheses $\mathcal{F}$ and the set of rejected hypotheses $\mathcal{R}$ are defined as: 
\begin{align}
	\begin{split}  \label{eqLocalHypothesis}
		\mathcal{T} &= \{ H_{i}\in \mathcal{H}_0: H_{i} \, \text{is true}\}, \\
		\mathcal{F} &= \{ H_{i}\in \mathcal{H}_0: H_{i} \, \text{is false}\}, \text{and} \\
		\mathcal{R} &= \{H_{i}\in \mathcal{H}_0: H_{i} \, \text{is rejected}\}.
	\end{split}%
\end{align}
The set of true and false hypotheses are unknown. We choose a set of hypotheses to reject. 
on the basis of our data. 
The set of discoveries $\mathcal{R}$ 
should coincide with the set of false hypotheses $\mathcal{F}$ as much as possible.

The goal of various multiple testing corrections is to control the familywise error rate (FWER), defined as the probability of at least one false
rejection in the family, $P[\mathcal{T} \cap \mathcal{R} \neq \varnothing]$,
while retaining the reasonable power in detecting false hypotheses. We want procedures for which the FWER is less than or equal to the upper bound $\alpha$ and ideally as close as possible to the upper bound. We focus on strong control of the FWER, meaning that some of the hypotheses we are testing can be false ($\mathcal{F} \neq \varnothing$), as opposed to the weak FWER control where all hypotheses of interest are true, i.e., $\mathcal{H}_0=\mathcal{T}$.

The probability of falsely rejecting a single hypothesis that is true (i.e., false positive or Type I error) is usually controlled at a nominal $\alpha$-level. However, when the number of tested hypotheses is large, the problem of multiplicity arises: the probability of having at least one false positive conclusion rises well above $\alpha$ if the Type I error of each individual test is controlled at the $\alpha$-level. Numerous controlling procedures have been proposed to deal with this problem. 
We review two classes of controlling procedures: one based on statistical inequalities (Section \ref{ssecOrderdPvals}) and one based on the maximum of the test statistics (Section \ref{ssecMaxTest}).


\subsection{Procedures based on statistical inequalities}
\label{ssecOrderdPvals}

Let us denote by $0 < p_{(1)}\leq p_{(2)}\leq \ldots\leq p_{(d)} < 1$ the set of $d$  ordered (in ascending order) raw $p$-values and $H_{(1)},H_{(2)},\ldots,H_{(d)}$ their corresponding null hypotheses. A single-stage method uses the same rejection 
criterion for all individual hypotheses, like the conservative Bonferroni threshold, while a multi-stage method examines the ordered $p$-values sequentially and adjusts the rejection criterion for each of the individual tests  \citep[e.g.,][]{holm1979simple,hochberg1988sharper,hommel1988stagewise}. 

The Bonferroni method rejects the elementary null hypothesis $H_{(i)}$ if 
$p_{(i)}\leq\alpha/d$ 
for $i=1,\ldots,d$. \citet{holm1979simple} and \citet{hochberg1988sharper} use the same critical values $ \alpha / (d - i + 1)$ depending on the rank of the $p$-value, but reject differently depending on whether they ``step up" or ``step down". The terminologies (``step up" or ``step down") were originally formulated in terms of test statistics which can be confusing when discussing $p$-values.  \citet{holm1979simple} proposes a step-down method that ``steps up" from the smallest $p$-value to the largest one. It is a pessimistic approach: it scans forward and stops as soon as a $p$-value fails to clear its threshold. \citet{hochberg1988sharper} suggests a step-up method that ``steps down" from the largest $p$-value to the	smallest one. It is an optimistic approach: it scans backward and stops as soon as a $p$-value succeeds in clearing its threshold. By construction, Hochberg's procedure will reject as many hypotheses as Holm's procedure. 

\cite{hommel1988stagewise} proposes a more complicated procedure which applies  \citet{simes1986improved}' global test to the $p$-value subset $\left\{ p_{\left(k\right) }\right\} _{ k = i }^{d}$, instead of relying  only on $p_{\left(i\right)}$ to draw inference on $H_{(i)}$ and thus borrows power across hypotheses. 
Hommel's procedure is shown to have higher power than Hochberg's method \citep{hommel1989}. We refer to Appendix \ref{AppOrderedPVals} for more details on the practical implementation of these procedures.


Bonferroni and \citet{holm1979simple}'s method are based on the first-order Bonferroni inequality, which states that given any set of events, the probability of their union is smaller than or equal to the sum of their probabilities. 
Under the null hypothesis, 
the probability that there is at least one hypothesis $H_{(i)}$  for which its raw $p$-value $p_{(i)} \leq \alpha / d$ 
% is not greater than $\alpha$ 
is bounded by $\alpha$: 
\begin{align}  \label{eqIneqPval}
	\Pr\left( \min_{i} p_{(i)} \leq \frac{\alpha}{d} \right) 
	= \Pr\left(\bigcup_{i =
		1}^{d} \left\{ p_{(i)} \leq \frac{\alpha}{d} \right\} \right) 
	&\leq
	\sum^{d}_{i = 1} \Pr \left( p_{(i)} \leq \frac{\alpha}{d}\right) \leq d
	\frac{\alpha}{d} \leq \alpha.
\end{align}
The Bonferroni \eqref{eqIneqPval} inequality 
makes no specific assumption on the dependence between the $p$-values, but protects against the so-called ``worst-case", in which all events are independent and the rejection regions are disjoint (the right half of Equation \eqref{eqIneqPval}) . 
The inequality becomes an equality when all test statistics are independent, and a strict inequality when the hypotheses are dependent. 
In other words, the Bonferroni correction is  conservative when the $p$-values are correlated. 

The methods of \cite{hochberg1988sharper} and \cite{hommel1988stagewise} are based on  \cite{simes1986improved}'s inequality. If a set of hypotheses $H_{(1)}, ...,H_{(d)}$ are all true, the probability of the joint event is: 
\begin{equation}  \label{eqSimes}
	\Pr\left( p_{\left( i\right) }> \frac{i\alpha}{d}, \text{ for all } i=1,\ldots
	,d\right) \geq 1-\alpha.
\end{equation}
\citet{simes1986improved}' inequality was developed for independent uniform $p$-values, and it is applicable for a large family of multivariate distributions. The simulations of \citet{simes1986improved} do show, however, that the test is very conservative for highly correlated multivariate normal statistics, but less so than the classical Bonferroni correction. 



\subsection{Procedures based on the maximum of test statistics}
\label{ssecMaxTest}

Another class of controlling procedures uses the maximum in a group of test statistics: $X_m = \max_{i} \abs{X_{i}}$, with $i = 1, \ldots, d$, to set a stringent critical value. The same critical value can be used for each elementary hypothesis and will control the familywise error rate. In particular, when the individual test statistics are independent and follow the standard normal distribution under the null, the maximum of the test statistics follows a Gumbel distribution when $d$ is large. Quantiles of the Gumbel distribution were used as critical values of the individual tests as a multiple testing correction when, for example, conducting jump tests in high-frequency asset returns \citep[][]{lee2007jumps}. Unfortunately, if the sequence of test statistics exhibits strong correlation, the number of tests severely overstates the effective number of independent copies in a given sample, which makes the Gumbel critical values too conservative \citep[see e.g.,][]{christensen2018drift}. We refer to Appendix \ref{AppOrderedPVals} for more details on the Gumbel distribution. 

Resampling-based methods account for the dependence structure that is specific to the considered dataset, leading to less conservative testing outcomes than the Gumbel-based methods and the inequality-based procedures. Depending on the empirical problem of interest, the resampling can be carried out by bootstrap, permutation, simulation, or randomization \citep[see e.g.,][for detailed discussions on resampling methods and testing procedures]{white2000,romano2005exact,romano2005stepwise,lehmann2005testing}. We refer to Section \ref{secApplDriftBurst} for an example of the resampling-based approach for the drift burst test. 


\section{Cauchy Combination Tests}
\label{secSeqCauchy}

In this section, we first review the global Cauchy combination (GCC) test of \citet{liu2020cauchy} and present our sequential Cauchy combination (SCC) test. While the global test tests the global null hypothesis $\mathcal{H}_0 = \bigcap_{i=1}^{d} H_{i}$ against the alternative hypothesis that at least one of the elementary null hypotheses is false, the sequential test aims at identifying the violations of the elementary null hypotheses while controlling the global error rate. 

\subsection{Global Cauchy combination test}
\label{sec:CC}

The GCC test statistic is constructed from the raw $p$-values of the test statistics $X_i$, which follow a uniform distribution between $0$ and $1$ under the  null hypothesis. The idea of this test is first to transform the uniformly distributed $p$-values into standard Cauchy variates using the formula $\tan \{(0.5-p_{i})\pi \}$ and then construct a new test statistic as the weighted sum of these transformed $p$-values. The new test statistic is denoted by $\tilde{T}$ and defined as: 
\begin{equation}
	\label{eqCauchyStatistic}
	{\normalsize \tilde{T}=\sum_{i=1}^{d}w_{i}\tan \{(0.5-p_{i})\pi \},} 
\end{equation}
in which the $w_{i}$'s are non-negative weights summing to 1. Throughout the paper, the weights $w_{i}$ are set to $1/d$ for $i=1,\ldots,d$ as in \citet{liu2020cauchy}. 

When the raw and hence the transformed $p$-values are independent (resp. perfectly correlated), % under the null hypothesis, 
the new test statistic $\tilde{T}$ is a linear combination of independent (resp. perfectly correlated) Cauchy variates and therefore follows a standard Cauchy distribution because the family of Cauchy densities is closed under convolutions. Although the correlation structure can affect the null distribution of $\tilde{T}$ in the case of general dependence, \citet{liu2020cauchy} show that the impact on the tail is very limited because of the heaviness of the Cauchy tail. Specifically, they prove that: 
\begin{equation}
	\lim_{h\rightarrow \infty }\frac{\Pr\left( \tilde{T}>h\right) }{\Pr\left(
		C>h\right) }=1,  \label{eq:tail}
\end{equation}
in which $C$ is a standard Cauchy random variable, under the null hypothesis $\mathcal{H}_0$ and Assumption \ref{ass1} which requires the test statistics to follow a bivariate zero mean normal distribution.
\begin{assumption}
	\label{ass1} (1) The original test statistics $(X_{i},X_{j})$, for any $1\leq
	i<j\leq d$, follow a bivariate normal distribution; (2) $E\left( \bm{X}%
	\right) =0$, with $\bm{X}=(X_{1},X_{2},\ldots,X_{d})^{^{\prime }}$. 
\end{assumption}
The bivariate normal requirement of Assumption \ref{ass1} is a condition weaker than joint normality, making the procedure applicable for high-dimensional settings. When the dimension $d$ increases at a certain rate with the sample size, the test statistics $\bm{X}$ may not jointly converge to a multivariate normal distribution due to its slower rate of convergence \citep[see][and references therein]{liu2020cauchy} and thus a joint normality assumption is not realistic for those settings. In contrast, the bivariate normality assumption is much weaker and more realistic. There are, of course, applications for which the test statistics are not normally distributed. Through simulations, \citet[][]{liu2020cauchy} show the Cauchy approximation is still accurate when the normality assumption is violated and follows a multivariate Student-$t$ distribution (with 4 degrees of freedom) instead. 
We refer to Section \ref{secApplFan} for a showcase example in finance with test statistics being Student-$t$ distributed. 

The result in  \eqref{eq:tail} suggests that, under the null hypothesis $\mathcal{H}_0$, the tail of the Cauchy combination test statistic is approximately Cauchy under arbitrary dependence structures, so that a $p$-value of the Cauchy combination test, denoted 
$\widetilde{p}$, can  be calculated from the standard Cauchy distribution. Suppose that we observe $\tilde{T}=t_{0}$, then: 
\begin{equation}
	\label{eqCauchyPval}
	\widetilde{p}=\frac{1}{2}-\frac{\arctan t_{0}}{\pi }. 
\end{equation}

Using the GCC $p$-values, the tail result in \eqref{eq:tail} can be equivalently stated as the actual size converging to the nominal size $\alpha$ as the significance level tends to zero:  % \textit{i.e.}, 
\begin{equation}
	\lim_{\alpha\rightarrow 0 }\frac{\Pr\left( \widetilde{p} \leq \alpha\right) }{\alpha}=1, \label{eq:wFWER}
\end{equation} 
The approximation should be particularly accurate for small $\alpha$'s, which are of particular interest in large-scale problems as in Examples 1 and 2. The simulations in \citeauthor{liu2020cauchy} show that when the significance level is moderately small ($\alpha = 10^{-1}, 10^{-2},10^{-3},10^{-4},10^{-5}$), the $p$-value calculation is accurate:  the ratio of the empirical size to the significance level is close to 1 for different types of correlations. 
Put differently, the GCC test achieves the weak familywise error rate control as the empirical size is very close to the nominal size $\alpha$ regardless of the correlation structure. 


Figure \ref{figCauchyPvalsAR} illustrates the fact that while the dependence between the individual test statistics $X_i$ can affect the null distribution of the GCC test statistic, the impact of the dependence is marginal on the tail. We simulate a vector of $d$ test statistics  $\bm{X}$ from a $d$-variate normal distribution with correlation matrix $\bm{\Sigma}$, \textit{i.e.}, $N_d(\bm{0}, \bm{\Sigma})$ with $\bm{\Sigma} = (\sigma_{ij})$ and $d = 300$. The diagonal elements $\sigma_{ii}=1$ for all $i=1,\ldots,d$ and the off-diagonal elements $\sigma_{ij} = \theta^{\abs{i-j}}$ for $i \neq j$, with $\theta = 0.2, 0.4, 0.6,$ $0.8, 0.90, 0.95$. The simulation is repeated $10^7$ times. For each draw, we calculate the GCC test statistic \eqref{eqCauchyStatistic} and its corresponding $p$-value \eqref{eqCauchyPval}. The histogram of the $10^7$ GCC $p$-values is displayed in Figure \ref{figCauchyPvalsAR}. For a low level of autocorrelation (i.e., $\theta=0.2$), the distribution of the $p$-values is close to a uniform distribution. When the level of autocorrelation is higher, there is a pothole in the middle and a bump at the end of the histogram, but whatever the strength of the autoregressive parameter, the percentage of the GCC $p$-values in the first bin is always around $5$\% as is ensured by the limit result in \eqref{eq:wFWER}. 


\begin{figure}[p]
	\caption{The impact of dependence on the tail of the GCC test statistic}
	\label{figCauchyPvalsAR}
	\centering
	
	\par
	
	\subfloat[${\theta} = 0.2$ ]{{\includegraphics[width=.40\textwidth,angle =
			-90,scale=0.70]{Sample_pval_dim_300_rho_0.2_alpha0.05.eps} }} 
	\subfloat[$\theta = 0.4$ ]{{\includegraphics[width=.40\textwidth,angle =
			-90,scale=0.70]{Sample_pval_dim_300_rho_0.4_alpha0.05.eps} }} 
	
	\vspace{0.4cm}
	
	\subfloat[$\theta = 0.6$ ]{{\includegraphics[width=.40\textwidth,angle =
			-90,scale=0.70]{Sample_pval_dim_300_rho_0.6_alpha0.05.eps} }} 
	\subfloat[$\theta = 0.8$ ]{{\includegraphics[width=.40\textwidth,angle =
			-90,scale=0.70]{Sample_pval_dim_300_rho_0.8_alpha0.05.eps} }} 
	
	\vspace{0.4cm}	
	
	\subfloat[$\theta = 0.90$]{{\includegraphics[width=.40\textwidth,angle =
			-90,scale=0.70]{Sample_pval_dim_300_rho_0.9_alpha0.05.eps} }} 
	% 
	\subfloat[$\theta = 0.95$]{{\includegraphics[width=.40\textwidth,angle =
			-90,scale=0.70]{Sample_pval_dim_300_rho_0.95_alpha0.05.eps} }} 
	
	\par
	\begin{minipage}{1.0\linewidth}
		\begin{tablenotes}
			\small
			\item {
				\medskip
				Note: We plot histograms of GCC $p$-values \eqref{eqCauchyPval} for various correlation strengths. The individual test statistics are drawn from a $d$-variate normal distribution $N_d(\bm{0}, \bm{\Sigma})$ with $\bm{\Sigma}= (\sigma_{ij})$ and $d=300$. The diagonal elements of the covariance matrix $\sigma_{ii}=1$ for all $i=1,\ldots,d$ and the off-diagonal elements  $\sigma_{ij} = \theta^{\vert i-j \vert}$ for $i\neq j$, with $\theta = 0.2, 0.4, 0.6,$ $0.8, 0.90, 0.95$. We compute the GCC $p$-value from the test statistic sequence. The simulation is repeated $10^7$ times. The simulated GCC $p$-values are sorted into bins with the bin edges being a sequence of edges from 0 to 1 with a width of  0.05. 				Each bin includes the right edge (right-closed) but does not include the left edge (left-open). We highlight the first bin in black and we
				also add a text note with the probability of $p$-values being in the first bin. }
		\end{tablenotes}
	\end{minipage}
\end{figure}

Interestingly,  \citet{liu2020cauchy} show that the tail property \eqref{eq:tail}  also holds when the number of hypotheses $d$ diverges to infinity at a rate of $o\left(h^{\eta }\right) \ $with $0<\eta <1/2$ and the following additional assumption is satisfied.
\begin{assumption}
	\label{ass2} Let $\mathbf{\Sigma }=corr\left( \bm{X}\right) $. 
	(1) The	largest eigenvalue of the correlation matrix $\lambda _{\max}\left( \mathbf{%
		\Sigma }\right) \leq C_0$ for some constant $C_0>0$; 
	(2) $\max_{1\leq i<j\leq
		d}\left\{ \sigma _{i,j}^{2}\right\} \leq \sigma _{\max }^{2}<1$ for some
	constant $0<\sigma _{\max }^{2}<1$, where $\sigma _{i,j}$ is the $\left(
	i,j\right) $ element of $\mathbf{\Sigma }$.
\end{assumption}
The additional assumptions on the correlation matrix are mild and standard in high dimensional settings and are general enough to incorporate a large class of tests. 


\subsection{Sequential Cauchy Combination Test}

The main contribution of this paper is the sequential Cauchy combination (SCC) test, which unravels the GCC test to make statements on the elementary hypotheses. The raw $p$-values are sorted in ascending order so that  $p_{(1)}\leq p_{(2)}\leq \ldots \leq p_{(d)}$, which is standard for step-down and step-up sequential procedures (see Section \ref{ssecOrderdPvals}). For the inference on hypothesis $H_{\left( i\right) }$ we compute a Cauchy combination test statistic ${\normalsize \tilde{T}}_{\left( i\right) }$ from a subset of $p$-values, running from $p_{(i)}$ to $p_{(d)}$ as:  
\begin{equation}
{\normalsize \ \tilde{T}%
		_{\left( i\right) }=\sum_{j=i}^{d}w_{j}\tan \{(0.5-p_{(j)})\pi \}
	}.
	\label{eq:CC_mt}
\end{equation}
The corresponding $p$-value is: $$ \widetilde{p}_{(i)}=\frac{1}{2}-\frac{\arctan \tilde{T}_{\left(i\right) }}{\pi }.$$ We reject the null hypothesis $H_{(i)}$ if  $\widetilde{p}_{(i)}\leq\alpha$. Like the step-up procedure of  \citet{hommel1988stagewise}, the SCC test also borrows power across hypotheses: the test statistic $\tilde{T}_{(i)}$ is computed from the raw $p$-values associated with $\mathcal{H}_0^{(i)}=\bigcap_{j=i}^{d} H_{(j)}$.


\subsubsection*{Theoretical Properties}
The SCC testing procedure can be viewed as a sequential rejection procedure. Let $\mathcal{R}^{(s)}$ be the collection of rejected hypothesis after step $s$, with $s=\left\{1,2,\ldots,d\right\}$. The hypothesis of interest and decision rules in each step  are illustrated in Table \ref{tabDecisionRule}.
\begin{table}[H]
	\caption{Decision rule in the sequential Cauchy combination test}
	\label{tabDecisionRule}
	\centering
	\begin{tabular}{p{1.cm}p{3.8cm}p{10.5cm}}
		\hline
		Step & Hypothesis & Decision\\ 
		$s=1$ & $\mathcal{H}_0^{\left( d\right) }=H_{(d)}$ & 
		If $\widetilde{p}_{(d)}\leq\alpha$ then reject $\mathcal{H}_0^{\left( d\right) }$ and include $H_{(d)}$ in $\mathcal{R}^{(1)}$
		\\
		$s=2$ & $\mathcal{H}_0^{\left( d-1\right) }=\bigcap_{j=d-1}^d H_{(j)}$  
		& 
		If $\widetilde{p}_{(d-1)}\leq\alpha$ then reject $\mathcal{H}_0^{\left( d-1\right) }$  and include $H_{(d-1)}$ in $\mathcal{R}^{(2)}$  
		\\
		$\ldots$  & $\ldots$  & $\ldots$  \\ 
		$s=d$ & $\mathcal{H}_0^{\left( 1\right) }=\bigcap_{j=1}^d H_{(j)}$ & 
		If $\widetilde{p}_{(1)}\leq\alpha$ then reject $\mathcal{H}_0^{\left( 1\right) }$ and include $H_{(1)}$ in $\mathcal{R}^{(d)}$ \\
		\hline
	\end{tabular}
\end{table}
Let $\mathcal{N}\left(\mathcal{R}^{(s)}\right)$ be the successor function, representing hypotheses to be rejected in the next step given that $\mathcal{R}^{(s)}$ has been rejected. For the SCC test, the successor function is defined as: 
\[
\mathcal{N}\left(\mathcal{R}^{(s)}\right)=\left\{H_{(d-s)} :  \widetilde{p}_{(d-s)} \leq \alpha_{\mathcal{R}^{(s)}}=\alpha\right\}.
\]
The cut-off value is fixed (i.e., $\alpha_{\mathcal{R}^{(s)}}=\alpha$) instead of depending on the rejection set $\mathcal{R}^{(s)}$ like in many other sequential procedures. According to the sequential rejection principle of \cite{goeman2010sequential},  the SCC test  achieves a strong family-wise error rate control if the following two conditions are satisfied. 
\begin{condition}[Monotonicity]
	For every $\mathcal{R}^{(s)}\subseteq \mathcal{R}^{(l)} \subset \mathcal{H}_{0}$, 
	\[
	\mathcal{N}(\mathcal{R}^{(s)}) \subseteq \mathcal{N}(\mathcal{R}^{(l)}) \cup \mathcal{R}^{(l)}
	\]
	almost surely. 
\end{condition}
% By construction, 
The transformed $p$-values of the SCC test are monotonic by construction, with $\widetilde{p}_{(d)}$ being the largest for the smallest set of global null hypotheses $\mathcal{H}_0^{(d)} = H_{(d)}$ and $\widetilde{p}_{(1)}$ being the smallest for the largest set of global nulls $\mathcal{H}_0^{(1)} = \bigcap_{j=1}^{d} H_{(j)}$ (see Figure \ref{figSequentialCauchyIllustration}(e) for an illustration of the monotonic $p$-values). Note that the largest set of global null hypotheses has the same null specification as the GCC test \eqref{eqCauchyStatistic}. It follows that that $\widetilde{p}_{(s)}\geq \widetilde{p}_{(l)}$. Since the cut-off value is fixed, the monotonicity condition of the successor function is satisfied.

\begin{condition}[Single-step condition] \label{SS} 
	When $\mathcal{H}_{0}^{(i)} =\mathcal{T}$, 
	$\Pr\left( \widetilde{p}_{(i)} \leq \alpha\right) \leq \alpha. $
\end{condition}
Condition \ref{SS} requires FWER control of the underlying test of SCC (i.e., the Cauchy combination test) at the ``critical case" in which all hypotheses of interest are true: $\mathcal{H}_{0}^{(i)} =\mathcal{T}$. The condition can be rewritten as $\Pr{\mathcal{N}(\mathcal{F})\subseteq \mathcal{F}} \geq 1-\alpha$ and has been shown to be satisfied by \cite{liu2020cauchy}. In fact,  when $\alpha$ is very small, the familywise false rejection probability of the GCC test under the null is not only bounded by $\alpha$ but also approaches the nominal size $\alpha$, as stated in \eqref{eq:wFWER}, which implies that it is less conservative than tests based on statistical inequalities or tests which impose independence in the presence of correlation. 

The theorem below follows directly from \cite[Theorem 1]{goeman2010sequential} for general sequential rejection procedures, so that Type I control in the critical case is sufficient for overall familywise error control of the sequential procedure. 
\begin{theorem}\label{thm}
	The SCC testing procedure satisfies both the monotonicity and the single-step condition and achieves the strong FWER control:  
	\[
	\lim_{\alpha\rightarrow 0}\Pr\left\{\mathcal{R}^{(d)} \subseteq \mathcal{F} \right\} \geq 1-\alpha, 
	\]
	under Assumption \ref{ass1} if $d$ is fixed and under Assumptions \ref{ass1} and \ref{ass2} if $d\rightarrow \infty$.
\end{theorem}


\subsubsection*{An Illustration}

A more prescriptive description of the SCC testing procedure is as follows: 
\begin{enumerate}
	
	\item Obtain raw $p$-values $p_1, p_2,\ldots, p_d$ corresponding to the null hypotheses $H_{1}, H_{2},\ldots, H_{d} $;%
	
	\item Order the raw $p$-values in ascending order, 	$p_{(1)},p_{(2)},\ldots,p_{(d)}$, with corresponding null ordered hypotheses $H_{(1)},H_{(2)},\ldots,H_{(d)}$;
	
	\item Calculate the SCC test statistic $\tilde T_{(i)}$ and the transformed Cauchy $p$-values $\widetilde{p}_{(i)}$ from a subset of the ordered $p$-values $\left\{p_{(j)}\right\} _{j=i}^{d}$ using \eqref{eq:CC_mt} for $i=1,\ldots,d$;
	
	\item Obtain the rejection set $\mathcal{R}=\left\{H_{\left(i\right)} : \widetilde{p}_{(i)}\leq \alpha\right\}$. 
\end{enumerate}

Figure \ref{figSequentialCauchyIllustration} illustrates the sequential Cauchy combination procedure on a simulated sequence of test statistics. The top row shows the simulated test statistics and their corresponding $p$-values, of which some hypotheses are under the null (grey dots) and some are under the alternative (black dots). The data-generating process is the same as that in Figure \ref{figCauchyPvalsAR} with $\theta=0.9$ and $d=100$. We add constant signals for $5$ out of 100 hypotheses,  with a signal strength equal to $\pm2.806$. The sign of the signal is the same as the sign of the test statistic under the null, such that the signal always amplifies the magnitude of the test statistic. 
The GCC test rejects the global null at $\alpha = 5\%$ for this sequence of $p$-values, which tells us there is at least one signal in the sequence.  
%The estimated first-order autocorrelation of the simulated test statistics is equal to $0.7910$ under the null and is equal to $0.4987$ under the alternative. 

\begin{figure}[p]
	\caption{Rejection procedure of the sequential Cauchy Combination test}
	\label{figSequentialCauchyIllustration}\centering

	\par
	
	\subfloat[Raw test statistics]{{\includegraphics[width=.31\textwidth,angle =
			-90]{1_tstat_d_100_rho_0.9_signal_5_5} }} 
	\subfloat[Raw
	$p$-values]{{\includegraphics[width=.31\textwidth,angle = -90]{2_pvals_d_100_rho_0.9_signal_5_5} }}
	
	\vspace{0.4cm}	
	
	\subfloat[Ordered raw $p$-values]{{\includegraphics[width=.31\textwidth,angle =
			-90]{3_spvals_d_100_rho_0.9_signal_5_5} }} 
	
	\vspace{0.4cm}	
	
	\subfloat[SCC test statistics
	]{{\includegraphics[width=.31\textwidth,angle = -90]{4_ctstats_d_100_rho_0.9_signal_5_5}}}
	\subfloat[SCC 
	$p$-values]{{\includegraphics[width=.31\textwidth,angle = -90]{4_cpvals_d_100_rho_0.9_signal_5_5}}}
	
	
	\begin{minipage}{1.0\linewidth}
		\begin{tablenotes}
			\small
			\item {
				\medskip
				Note: We illustrate the mechanics of the SCC procedure on a simulated test statistic sequence with sparse signals. The top row shows raw test statistics and $p$-values of which some hypotheses are under the null and some are under the alternative. The test statistics are simulated from $N_{d}(\bm{0},\bm{\Sigma})$ as in Figure \ref{figCauchyPvalsAR}. We set $d=100$, $\theta=0.9$ and add $5\%$ signals. The strength of the signal is $\pm2.806$, with its sign identical to that of the test statistic under the null. The horizon line in panel (e) is the 5\% significance level.
			}
		\end{tablenotes}
	\end{minipage}
\end{figure}

The SCC test can tell us which individual $p$-values trigger the rejection of the GCC test. The middle row plots the raw $p$-values in ascending order and the bottom row plots its sequential Cauchy combination test statistics and $p$-values. Specifically, the bottom right panel shows that the SCC $p$-values $\widetilde{p}_{(i)}$ decrease as $i$ moves from $d$ to $1$. In this example, the SCC test rejects three out of the five alternative hypotheses and does not reject under the null hypothesis. The rejections correspond to the 4$^\text{th}$, 29$^\text{th}$ and  46$^\text{th}$ hypotheses in the top row. Note that the smallest SCC $p$-value corresponds to the $p$-value of the GCC test of \citet{liu2020cauchy}, which performs the test on the largest set of hypotheses. 
We present in section~\ref{ssec:faces} an application of PnP-HVAE on face images, using a pretrained state-of-the-art hierarchical VAE. 
Next, we study the application of our framework to natural images. To that end, we introduce  in section~\ref{ssec:patchVDVAE}  a patch hierachical VAE architecture, that is able to model natural images of different resolutions. In section~\ref{ssec:app_nat}, we provide deblurring, super-resolution and inpainting experiments to demonstrate the relevance of the proposed method.

Additional results are presented in Appendix~\ref{app:add}. All experiments can be reproduced using the code available at \url{https://github.com/jprost76/PnP-HVAE}.



\subsection{Face Image restoration (FFHQ)}\label{ssec:faces}
We first demonstrate the effectiveness of PnP-HVAE on highly structured data, by performing face image restoration.
Latent variable generative models can accurately model structured images such as face images \cite{karras2019style,vahdat2020nvae,child2021very,kingma2018glow}, and then be used to produce high quality restoration of such data. 
In our experiments, we use the VDVAE model of~\cite{child2021very}, pre-trained on the FFHQ dataset~\cite{karras2019style}, as our hierarchical VAE prior.
VDVAE has $L=66$ latent variable groups in its hierarchy and generates images at resolution $256\times256$.

We compare PnP-HVAE with the intermediate layer optimization algorithm (ILO)~\cite{daras2021intermediate} that is based on a different class of generative models than HVAE. ILO is a GAN inversion method which optimizes the image latent code along with the intermediate layer representation of a StyleGAN to generate an image consistent with a degraded observation.
We use the official implementation of ILO, along with a StyleGAN2 model~\cite{karras2020analyzing, stylegan2pytorch}, that was trained for 550k iterations on images of resolution $256\times256$ from FFHQ.  
As VDVAE and StyleGAN models are not trained on the same train-test split of FFHQ, we chose to evaluate the methods on a subset of 100 images from the CelebA dataset~\cite{liu2018large}. 
For super-resolution, the degradation model corresponds to the application of a gaussian low-pass filter followed by a $\times 4$ sub-sampling, and the addition of a gaussian white noise with $\sigma=3$.
For the deblurring, we considered motion blur and  gaussian kernels, both with a noise level $\sigma=8$. %

We provide quantitative comparisons in table~\ref{table:comp_ILO}, along with a visual comparison of the results in figure~\ref{fig:face_restoration}.
PnP-HVAE has the best  PSNR and SSIM results for all the considered restoration tasks, while ILO provides better results  for the perceptual distance.
By jointly optimizing the image and its latent variable, PnP-HVAE provides  results that are both realistic and consistent with the degraded observation.
On the other hand,  ILO  only optimizes on an extended latent space. This method generates  sharp and realistic images with better LPIPS scores,   
but the results lack  of consistency with respect to the observation, which explains the overall lower PSNR performance. 






\subsection{PatchVDVAE: a HVAE for natural images}\label{ssec:patchVDVAE}
Available generative models in the literature operate on images of  fixed resolutions and
are either restrained to datasets of limited diversity, or even to registered face images~\cite{kingma2018glow,child2021very, vahdat2020nvae, karras2019style}, or requiring additional class information~\cite{brock2018large, dhariwal2021diffusion, song2020score, luhman2022optimizing}.
Fitting an unconditional model on natural images appears to be a more difficult task, as their resolution can change, and their content is highly diverse.
The complexity of the problem can be reduced by learning a prior model on patches of reduced dimension. 
For image restoration problems, the patch model can be reused on images of higher dimensions~\cite{zoran2011learning,prost2021learning,altekruger2022patchnr}. When the model is a full CNN, the prior on the set of the  patches can  be computed efficiently by applying the network on the full image~\cite{prost2021learning}.

We thus introduce  patchVDVAE, a fully convolutional hierarchical VAE.
Contrary to existing HVAE models whose resolution is constrained by the constant tensor at the input of the top-down block, patchVDVAE can generate images of different resolutions by controlling the dimension of the input latent. 
This amounts to defining a prior on patches whose dimension corresponds to the receptive field of the VAE. A similar model is used for image denoising in~\cite{prakash2021interpretable}.

 
For PatchVDVAE architecture, we use the same bottom-up and top-down blocks as VDVAE~\cite{child2021very}, and replace the constant trainable input in the first top-down block by a latent variable, to make the model fully convolutional (details on the  architecture are given in Appendix~\ref{app:details}). 
The training dataset is composed of $128\times 128$ patches extracted from a combination of DIV2K~\cite{agustsson2017ntire} and Flickr2K~\cite{Lim_2017_CVPR_workshops} datasets.
We perform data augmentation by extracting  patches at $3$ resolutions: HR-images and $\times 2$ and $\times 4$ downscaled images. 
The model is trained for $7.10^5$ iterations with a batch size of $64$. Following the recommendation of~\cite{hazami2022efficient}, we use Adamax optimizer with an exponential moving average and gradient smoothing of the variance.
We set the decoder model to be a gaussian with diagonal covariance, as in~\cite{luhman2022optimizing}.
PatchVDVAE is fully convolutional and can generate images of dimension that are multiples of $64$ as illustrated by
figure~\ref{fig:vdvae}.

\newlength{\patchwidth}
\setlength{\patchwidth}{0.135\columnwidth}
\begin{figure}[!ht]
    \centering
    \begin{subfigure}[t]{.34\columnwidth}\hspace{0.1cm}
        \setlength{\tabcolsep}{0.02pt}
\renewcommand{\arraystretch}{0}
        \begin{tabular}{*{2}{p{1.03\patchwidth}}}
            \includegraphics[width=\patchwidth]{figures_arxiv/patchVDVAE/samples/generated/64x64/setup-5-image-0018.png} &
            \includegraphics[width=\patchwidth]{figures_arxiv/patchVDVAE/samples/generated/64x64/setup-5-image-0016.png} \\
            \includegraphics[width=\patchwidth]{figures_arxiv/patchVDVAE/samples/generated/64x64/setup-5-image-0008.png} &
            \includegraphics[width=\patchwidth]{figures_arxiv/patchVDVAE/samples/generated/64x64/setup-5-image-0019.png}   
        \end{tabular}
    \end{subfigure}\hspace{-0.15cm}
    \begin{subfigure}[t]{.64\columnwidth}
\begin{tabular}{cc}\vspace{-0.1cm}
\includegraphics[width=2\patchwidth]{figures_arxiv/patchVDVAE/samples/generated/256x256/setup-2-image-0009.png}&
        \includegraphics[width=2\patchwidth]{figures_arxiv/patchVDVAE/samples/generated/256x256/setup-2-image-0002.png}\end{tabular}

    \end{subfigure}
    \caption{\label{fig:vdvae} Left: $64\times64$ patches samples from our patchVDVAE model trained on patches from natural images.
    Right: PatchVDVAE is fully convolutional and it can generate images of higher resolution (here: $128\times128$).\vspace{-0.2cm}}
\end{figure}

\subsection{Natural images restoration}\label{ssec:app_nat}
We  evaluate PnP-HVAE on natural image restoration.
For each task, we report the average value of the PSNR, the SSIM, and the LPIPS metrics on $20$ images from the test set of the BSD dataset~\cite{MartinFTM01}.\\


\noindent
{\bf Image deblurring.}
In the experiments, we consider $2$ gaussian kernels and $2$ motion blur kernels from~\cite{levin2009understanding}, with $3$ different noise levels 
$\sigma \in \{2.55, 7.65, 12.75\}$.
As a baseline we consider  EPLL~\cite{zoran2011learning}, which learns a prior on image patches with a gaussian mixture model.
We also compare PnP-HVAE  with PnP-MMO and GS-PnP, $2$ competing convergent Plug-and-Play methods based on CNN denoisers.
PnP-MMO~\cite{pesquet2021learning} restricts the denoiser to be contraction in order to guarantee the convergence of the PnP forward-backard algorithm. GS-PnP~\cite{hurault2022gradient} considers a gradient step denoiser and reaches state-of-the-art performances of non converging methods~\cite{zhang2021plug}.
We set the temperature $\tau$  in our method as $0.95$, $0.8$ and $0.6$ for noise levels $2.55$, $7.65$ and $12.75$ respectively, and we let it run for a maximum of $50$ iterations. 
For the three compared methods we use the official implementations and pre-trained models provided by the respective authors. 
Details on the choice of hyperparameters for the concurrent methods are provided in the Appendix~\ref{app:details}
Figure~\ref{fig:deblurring_bsd} illustrates that our method provides correct deblurring results. 

According to table~\ref{tab:deb}, the performance of PnP-HVAE is between those of EPLL and GS-PnP and it outperforms PnP-MMO for large noise levels.\\

\begin{table}
\begin{center}\footnotesize
    \begin{tabular}{>{\centering}m{.3cm}*{5}{c}}
    $\sigma$ &Method & PSNR$\uparrow$ & SSIM$\uparrow$ & LPIPS$\downarrow$  \\ 
    \hline
    \multirow{4}{*}{\vcell{$2.55$}}
    & PnP-HVAE & $27.75$ & $0.79$ & $0.31$\\
    & GS-PNP \cite{hurault2022gradient} & $\mathbf{29.59}$ & $\mathbf{0.84}$ & $\mathbf{0.22}$\\
    & EPLL \cite{zoran2011learning} & $26.49$ & $0.71$ & $0.36$\\ 
    & PnP-MMO \cite{pesquet2021learning} & $\underbar{29.50}$ & $\underbar{0.83}$ & $\underbar{0.20}$ \\ \hline
    \multirow{4}{*}{\vcell{$7.65$}}
    & PnP-HVAE & $\underbar{26.36}$ & $\underbar{0.72}$ & $\underbar{0.40}$\\
    & GS-PNP \cite{hurault2022gradient} & $\mathbf{27.33}$ & $\mathbf{0.77}$ & $\mathbf{0.31}$\\
    & EPLL \cite{zoran2011learning} & $24.04$ & $0.66$ & $0.45$ \\ 
    & PnP-MMO \cite{pesquet2021learning} & $25.34$ & $0.69$ & $0.34$\\
    \hline
    \multirow{4}{*}{\vcell{$12.75$}}
    & PnP-HVAE & $\underbar{25.12}$ & $\mathbf{0.73}$ & $\underbar{0.47}$\\
    & GS-PNP \cite{hurault2022gradient} & $\mathbf{26.32}$ & $\mathbf{0.73}$ & $\mathbf{0.37}$\\
    & EPLL \cite{zoran2011learning} & $23.28$ & $0.61$ & $0.51$ \\ 
    & PnP-MMO \cite{pesquet2021learning} & $22.42$ & $0.53$& $0.54$ \\
    \hline
    &\vspace*{-.3cm}\\
            \multicolumn{2}{c}{Blur and motion kernels}& \multicolumn{3}{c}{
        \includegraphics*[scale=1]{figures_arxiv/kernels/4.png}\;\includegraphics*[scale=1]{figures_arxiv/kernels/7.png}\;\includegraphics*[scale=1]{figures_arxiv/kernels/9.png}\;\includegraphics*[scale=1]{figures_arxiv/kernels/11.png}} 
    \end{tabular}
        \caption{\label{tab:deb}Comparison  of PnP-HVAE  and other restoration methods on deblurring. Results are averaged on $4$ kernels.\vspace{-0.2cm}}% on image deblurring.}
    \end{center}
\end{table}

\begin{figure}
    
    \begin{subfigure}[h]{\linewidth}
        \centering
        \includegraphics*[width=\columnwidth]{figures_arxiv/deb_s255_k7.pdf}\vspace{-0.1cm}
        \caption{Gaussian blur, $\sigma=2.55$}
    \end{subfigure}
    \begin{subfigure}[h]{\linewidth}
        \centering
        \includegraphics*[width=\columnwidth]{figures_arxiv/deb_s765_k11.pdf}\vspace{-0.1cm}
        \caption{Motion blur, $\sigma=7.65$}
    \end{subfigure}\vspace*{-0.1cm}
    \caption{\label{fig:deblurring_bsd} Natural image deblurring\vspace{-0.1cm}}
\end{figure}

\noindent {\bf Effect of the temperature.}
PnP-HVAE gives control on the temperature of the prior over the latent space.
In figure~\ref{fig:temp_effect}, we illustrate that reducing the temperature increases the strength of the regularization prior. In this example the tuning $\tau=0.7$ produces the best performance.\\
\begin{figure}[!ht]
   
    \includegraphics[width=\columnwidth]{figures_arxiv/demo_temp.pdf}\vspace{-0.15cm}
    \caption{ \label{fig:temp_effect} Effect of the temperature in PnP-VAE on a deblurring problem, with $\sigma=7.65$.\vspace{-0.15cm}}
\end{figure}


\noindent
{\bf Image inpainting.}
Next we consider the task of noisy image inpainting. 
We compose a test-set of 10 images from the validation set of BSD~\cite{MartinFTM01} and we create masks
  by occluding diverse objects of small size in the images. 
A gaussian white noise with $\sigma=3$ is added to the images.
As a comparaison, we still consider GS-PnP and EPLL.
For PnP-HVAE, the temperature is set to $\tau=0.6$, and the algorithm is run for a maximum of $200$ iterations, unless the residual $||\x_{k+1}-\x_k||$ is on a plateau.
We provide on Table~\ref{tab:inpainting_bsd} the distortion metrics with the ground truth, as well as a visual
\begin{table}



\begin{center}
    \begin{tabular}{cccc}
        & PSNR$\uparrow$ & SSIM$\uparrow$ &LPIPS$\downarrow$ \\\hline
        PnP-HVAE  & $\mathbf{29.54}$ & $\mathbf{0.93}$ & $\mathbf{0.06}$\\
        GS-PNP & $28.52$ & $\mathbf{0.93}$ & $0.09$\\
        EPLL & $\underline{29.16}$ & $\mathbf{0.93}$ & $\mathbf{0.06}$\\
    \end{tabular}
    \caption{\label{tab:inpainting_bsd}Quantitative evaluation for inpainting on BSD.}
    \end{center}
\end{table}
comparison on figure~\ref{fig:inpainting_bsd}. 
With its hierarchical structure,  PnP-HVAE outperforms the compared methods. \vspace{0.05cm}



\begin{figure}[!h]
    \includegraphics[width=\columnwidth]{figures_arxiv/demo_inp_bsd2.pdf}\vspace{-0.1cm}
    \caption{\label{fig:inpainting_bsd}Natural image inpainting\vspace{-0.3cm}}
\end{figure}











\section{Conclusions}
We consider the phase-extraction problem, and we showed that, given a unitary $U = e^{i\pi H}$ and its inverse $U^{\dag}$, we could implement a block-encoding of $\phi(H)$ for some smooth function $\phi(x)$. The word `smooth' here means existence and continuity of the derivatives: the higher the number of continuous derivatives that a function has, the faster its Fourier sum (and thus the Laurent polynomial on the eigenphases) uniformly converges to that function. We are confident this can have many more applications beyond what is shown in this work. It is also worth remarking that Jackson showed that the convergence rate of a Fourier series is almost-optimal, in the sense that no trigonometric (or, equivalently, complex exponential) series can approximate the desired function faster, up to that $\log d$ factor~\cite[p.\ 21]{jacksonTheoryApproximation1930a}. Also remember that `smoothing' a function, i.e., replacing its derivative with a continuous function, does not give faster convergence for free in general, as its derivative will become steep in the points where we smooth out discontinuities, and this translates to a high Lipschitz constant: a~clear example is given by Eq.~\ref{eq:lipschitz-constant-recurrence-solution}, but in that case, fortunately, nothing depends on the size of the input $N$, and thus does not influence the asymptotic query complexity of Algorithm~\ref{alg:prop-sampling-qsp}, although the constant factor can become large even for $p = 20$. From a theoretical point of view, this work shows that, for any $\eta > 0$, there is an algorithm with query complexity 
$$\Tilde{\bigO}\left(\frac{1}{\bar{c}^{\frac{1}{2} + \eta}} \frac{1}{\epsilon^\eta} \right)$$
solving the proportional-sampling problem. This statement seems to suggest there exists an algorithm which directly solves the problem with $\eta = 0$, and an open question would be to find such algorithm.


It is also interesting to remark that Theorems~\ref{thm:haah-construction},~\ref{thm:haah-completion} indeed allow the construction for any $\phi$, even complex-valued, provided that its absolute value is reciprocal.

One could think that, in Section~\ref{sec:prop-sampling}, instead of using the linear function in the phase-extraction subroutine, we could approximate the square root and then apply the transformation directly on $e^{i \pi c(x)}$. However, in the case of proportional sampling this would be inconvenient, as the derivative of the square root function has a discontinuity with an infinite jump around 0, and we could not choose a constant $\delta$ if we had values of the oracle that are too close to $0$.


\bibliography{references}
\bibliographystyle{iclr2024_conference}


\newpage
\appendix
\tableofcontents
\section{Definition and Properties of Shapley values}\label{app:foundation.xai}

Explainability has become an important concept in legal and ethical guidelines for data processing and machine learning applications ~\citep{selbst2018intuitive}. A wide variety of methods have been developed, aiming to account for the decision of algorithmic systems ~\citep{DBLP:journals/csur/GuidottiMRTGP19,DBLP:conf/fat/MittelstadtRW19,DBLP:journals/inffus/ArrietaRSBTBGGM20}. 
One of the most popular approaches to explainability in machine learning is Shapley values.

Shapley values are used to attribute relevance to features according to how the model relies on them ~\citep{DBLP:journals/natmi/LundbergECDPNKH20,DBLP:conf/nips/LundbergL17,DBLP:conf/ijcai/RozemberczkiWBY22}. Shapley values are a coalition game theory concept that aims to allocate the surplus generated by the grand coalition in a game to each of its players~\citep{shapley1997value}. 

For set of players $N = \{1, \ldots, p\}$, and a value function $\val:2^N \to \mathbb{R}$, the Shapley value $\mathcal{S}_j$ of the $j$'th player is defined as the average marginal contribution of player $j$ in all possibles coalitions of players:
\begin{small}
\[
\mathcal{S}_j = \sum_{T\subseteq N\setminus \{j\}} \frac{|T|!(p-|T|-1)!}{p!}(\val(T\cup \{j\}) - \val(T))
\]
\end{small}
%
In the context of machine learning models, players correspond to features $X_1, \ldots, X_p$, and the contribution of the feature $X_j$ is with reference to the prediction of a model $f$ for an instance $x^{\star}$ to be explained. Thus, we write $\mathcal{S}(f, x^{\star})_j$ for the Shapley value of feature $X_j$ in the prediction $f(x^{\star})$. We denote by $\mathcal{S}(f, x^{\star})$ the vector of Shapely values $(\mathcal{S}(f, x^{\star})_1, \ldots, \mathcal{S}(f, x^{\star})_p)$. 

There are two variants for the term $\val(T)$~\citep{DBLP:journals/ai/AasJL21,DBLP:journals/corr/abs-2006-16234,Zern2023Interventional}: the \textit{observational} and the \textit{interventional}. When using the observational conditional expectation, we consider the expected value of $f$ over the joint distribution of all features conditioned to fix features in $T$ to the values they have in $x^{\star}$:
\begin{equation}\label{def:val:obs}
\val(T) = E[f(x^{\star}_T, X_{N\setminus T})|X_T=x^{\star}_T]
\end{equation}
where $f(x^{\star}_T, X_{N\setminus T})$ denotes that features in $T$ are fixed to their values in $x^{\star}$, and features not in $T$ are random variables over the joint distribution of features.
Opposed, the interventional conditional expectation considers the expected value of $f$ over the marginal distribution of features not in $T$: % and with features in $T$ being fixed to the values they have in $x^{\star}$:
\begin{equation}\label{def:val:int}
\val(T) = E[f(x^{\star}_T, X_{N\setminus T})]
\end{equation}
In the interventional variant, the marginal distribution is unaffected by the knowledge that $X_T=x^{\star}_T$. In general, the estimation of (\ref{def:val:obs}) is difficult, and some implementations (e.g., SHAP) actually consider (\ref{def:val:int}) as the default one. In the case of decision tree models, TreeSHAP offers both possibilities.


The Shapley value framework is the only feature attribution method that satisfies the properties of efficiency, symmetry, uninformativeness and additivity~\citep{molnar2019,shapley1997value,winter2002shapley,aumann1974cooperative}.  We recall next the key properties of efficiency and uninformativeness:

\paragraph{Efficiency.} Feature contributions add up to the difference of prediction for $x^{\star}$ and the expected value of $f$:
\begin{gather}\label{eq:eff}
    \sum_{j \in N} \Ss(f, x^{\star})_j = f(x^{\star}) - E[f(X)])
\end{gather}

The following property only holds for the interventional variant (e.g., for SHAP values), but not for the observational variant.
\paragraph{Uninformativeness.}
A feature $X_j$ that does not change the predicted value (i.e., for all $x, x'_j$: $f(x_{N\setminus \{j\}}, x_j) = f(x_{N\setminus \{j\}}, x'_j)$) has a Shapley value of zero, i.e., $\Ss(f, x^{\star})_j = 0$.
 

In the case of a linear model $f_\beta(x) = \beta_0 + \sum_j \beta_j \cdot x_j$, the SHAP values turns out to be $\Ss(f, x^{\star})_i = \beta_i(x^{\star}_i-\mu_i)$ where $\mu_i = E[X_i]$. For the observational case, this holds only if the features are independent \citep{DBLP:journals/ai/AasJL21}.

\section{Detailed Related Work}\label{app:relatedWork}

This section provides an in-depth review of the related theoretical work that informs our research. We contextualize our contribution within the broader field of explainable AI and fairness auditing. We discuss the use of fairness measures such as demographic parity, as well as explainability techniques like Shapley values and counterfactual explanations.

\subsection{Fairness Notions: Paper Blind Reviews Use Case}\label{app:fair.notions}

%\steffen{explain with formulas}\carlos{I have cross-reference the eq. of the main body}\steffen{The weakness of this use case is that one may argue that the prior probability of a paper to succeed is the same for Germany and UK. I think it is easier to argue the case if the prior probability between different groups is different. This is the case for university admissions. Or for COMPAS dataset.}\steffen{there is the assumption in that notion that the two countries indeed have the same performance --- which may not be the case}\carlos{There is no assumption, we are not saying which one is better. We could change for country A and country B, or institution A and institution B}

To illustrate the difference between equal opportunity, equal outcomes, and equal treatment, based on the previously discussed framework, we consider the example of conference papers' blind reviews and focus on the protected attribute of the country of origin of the paper's author, comparing Germany and the United Kingdom.


For \emph{equal opportunity}, we quantify fairness by the true positive rate (cf. Definition~\ref{def:eo}). In words, it is the acceptance ratio given that the quality of the paper is high. Achieving equal opportunity will imply that these ratios are similar between the two countries. In blind reviews, the purpose is to evaluate the paper's quality and the research's merit without being influenced by factors such as the author's identity, affiliations, background or country. If we were to enforce equal opportunity in this use case, we would aim for similar true positive rates for submissions from different countries. However, this approach could lead to unintended consequences, such as unintentionally favouring, reverse discrimination, overcorrection or quotas of affirmative action towards certain countries. 


For \emph{equal outcomes}, we require that the distribution of acceptance rates is similar, independently of the quality of the paper (cf. Definition~\ref{def:dp}). Note that the outcomes can have similar rates due to random chance, even if there is a country bias in the acceptance procedure.

For \emph{equal treatment}, we require that the contributions of the features used to make a decision on paper's acceptance has similar distributions (cf. Definition~\ref{def:et}). Equality of treatment through blindness is more desirable than equal opportunity or equal outcomes because it ensures that all submissions are evaluated solely on the basis of their quality, without any bias or discrimination towards any particular country. By achieving equality of treatment through blindness, we can promote fairness and objectivity in the review process and ensure that all papers have an equal chance to be evaluated on their merits.


In comparing our introduced measure of \emph{equal treatment} with \emph{equal outcomes} (or demographic or statistical parity, used as synonymous), we note that the latter looks at the distributions of predictions and measures their similarity. Equal treatment goes a step further by evaluating whether the contribution of features to the decision, is similar. Our definition of \emph{equal treatment} implies the notion of \emph{equal outcome}, but the converse is not necessarily true, as we showed in Section \ref{sec:stat.independence}.


\subsection{Measuring Fairness}

%\enquote{Audits are tools for interrogating complex processes, often to determine whether they comply with company policy, industry standards or regulations}~\citep{DBLP:conf/fat/RajiSWMGHSTB20}. Algorithmic fairness audits are closely linked to audits studies as understood in the social sciences, with a strong emphasis on social justice~\citep{DBLP:conf/eaamo/VecchioneLB21}. A recent survey on public algorithmic audit identified four categories of \enquote{problematic machine behaviour} that can be unveiled by audit studies: discrimination, distortion, exploitation and misjudgement \citep{DBLP:journals/pacmhci/Bandy21,liu2012enterprise}. This taxonomy is highly helpful when putting forward auditing studies and also relevant for this work: our \enquote{Equal Treatment Inspector} can be seen as a tool to help understand discrimination in machine learning models, thus, falling into the first category.


Selecting a measure to compare fairness between two sensitive groups has been a highly discussed topic, where results such as~\citep{DBLP:journals/bigdata/Chouldechova17,DBLP:conf/nips/HardtPNS16,DBLP:conf/innovations/KleinbergMR17}, have highlighted the impossibility to satisfy simultaneously three type of fairness measures: demographic parity~\citep{DBLP:conf/innovations/DworkHPRZ12}, equalized odds~\citep{DBLP:conf/nips/HardtPNS16}, and predictive parity~\citep{DBLP:conf/kdd/Corbett-DaviesP17,DBLP:journals/corr/abs-2102-08453,wachter2020bias}. 


%Within the fairness mitigation context~\citep{DBLP:journals/corr/EdwardsS15} formulates an adversarial learning problem of the model and data to remove statistical parity from images. A limitation is that their work is limited to images and neural networks. Our work focuses on tabular data and leverages Shapley value theory to provide equal treatment auditing guarantees. 

Previous work has relied on measuring and calculating demographic parity on the model predictions~\citep{DBLP:conf/fat/RajiSWMGHSTB20,DBLP:conf/icml/KearnsNRW18}, or on the input data~\citep{DBLP:journals/cviu/FabbrizziPNK22,DBLP:conf/fat/YangQ0DR20,DBLP:conf/emnlp/ZhaoWYOC17}.  In this work, we perform equal treatment measures on the explanation distribution, which measures that each feature contributes equally to the prediction, which differs from the previous notions. 

In this work, we focus on  Equal Treatment (ET), as this fairness metric does not require a ground truth target variable, allowing for our method to work in its absence~\citep{DBLP:conf/aies/AkaBBGM21}, and under distribution shift conditions~\citep{DBLP:journals/corr/abs-2210-12369} where model performance metrics are not feasible to calculate~\citep{DBLP:conf/icml/GargBKL21,DBLP:conf/iclr/GargBLNS22,DBLP:conf/aaai/MouganN23}. Demographic Parity requires independence of the model's output from the protected features, written $f_\theta(X) \perp Z$, while Equal Treatment requires independence accross the feature attributions of the model $\Ss(f_\theta(X),X) \perp Z$.

\subsection{Explainability and fair supervised learning}\label{app:relatedwork.Lundberg}


%\subsection{Formal Comparison with Previous Work on Explaining Quantitative Fairness Measures via Shapley Values}

The intersection of fairness and explainable AI has been an active topic in recent years. The work most close to our approach is~\cite{lundberg2020explaining} where Shapley values are aimed at testing for demographic parity.
%Our work continues in-depth on this research line by formalizing the explanation distribution, deriving theoretical guarantees, and proposing novel methodology. 
%Specifically, our approach allows for comparison across different protected groups. We also introduce the concept of accountability, which refers to the ability of the algorithm to provide insights into the sources of \emph{equal treatment} violation of the model. Additionally, we evaluate our method on multiple datasets and synthetic examples, demonstrating its effectiveness in identifying and mitigating equal treatment disparities in machine learning models.
%In this subsection, we formally compare our approach to the prior work of Lundberg et al. \cite{lundberg2020explaining} and the related SHAP Python package documentation. Lundberg et al. addressed the concept of demographic parity (DP) using SHAP value estimation. 
This concise workshop paper emphasizes the importance of \enquote{decomposing a fairness metric among each of a model’s inputs to reveal which input features may be driving any observed fairness disparities}.
%
In terms of statistical independence, the approach can be rephrased as decomposing $f_\theta(X) \perp Z$ by examining $S(f_\theta,X)_i \perp Z$ for $i \in [1, p]$. 
Actually, the paper limits to consider difference in means, namely testing for $E[\Ss(f_\theta,X)_i|Z=1] \neq E[\Ss(f_\theta,X)_i|Z=0]$. Our approach goes beyond this, as we consider different distributions, and introduce the ET fairness notion for that. On the contrary, \cite{lundberg2020explaining} claims a decomposition method specific of DP. % Instead, it attempts to decompose the DP fairness of the model's output into the DP of its input components. In doing that, however, the
However, the decomposition method proposed is not sufficient nor necessary to prove DP, as showed next.

\begin{lemma}\label{lemma:lundberg}
$f_\theta(X) \perp Z$ is neither implied by nor it implies ($\Ss(f_\theta,X)_i \perp Z$ for $i \in [1, p]$).
\end{lemma}
\begin{proof}
Consider $f_\theta(X_1,X_2) = X_1 - X_2$ with $X_1,X_2 \sim \texttt{Ber}(0.5)$ and $Z=1$ if $X_1=X_2$, and $Z=0$ otherwise. Hence $Z\sim \texttt{Ber}(0.5)$. We have $\Ss(f_\theta,X_1) = X_1 \perp Z$ and $\Ss(f,X_2) = -X_2 \perp Z$. However, $f_\theta(X_1,X_2) = X_1-X_2$ does not satisfy $f_\theta(X_1,X_2) \perp Z$, e.g., $P(Z=0|f_\theta(X_1,X_2)=0) = P(Z=0|X_1-X_2=0) = 1$. Example \ref{ex42} illustrates a case where $f_\theta(X) \perp Z$ yet $\Ss(f_\theta, X)_1$ and $\Ss(f_\theta,X)_2$ are not independent of $Z$. 
\end{proof}

Our approach to ET considers the independence of the \textit{multivariate} distribution of $\Ss(f,X)$ with respect to $Z$, rather than the independence of each marginal distribution $\Ss(f_\theta, X)_i \perp Z$. With such a definition, we obtain a sufficient condition for DP, as shown in Lemma \ref{lemma:inc}.


~\cite{DBLP:conf/ssci/StevensDVV20} presents an approach based on adapting the Shapley value function to explain model unfairness. They also introduce a new meta-algorithm  that considers the problem of learning an additive perturbation to an existing model in order to impose fairness. 
In our work, we do not adopt the Shapley value function. Instead, we use the theoretical Shapley properties to provide fairness auditing guarantees. Our \enquote{Equal Treatment Inspector} is not perturbation-based but uses Shapley values to project the model to the explanation distribution, and then measures \emph{un-equal treatment}. It also allows us to pinpoint what are the specific features driving this violation.
 
 
\cite{DBLP:conf/fat/GrabowiczPM22} present a post-processing method based on Shapley values aiming to detect and nullify the influence of a protected attribute on the output of the system. For this, they assume there are direct causal links from the data to the protected attribute and that there are no measured confounders. Our work does not use causal graphs but exploits the theoretical properties of the Shapley values to obtain fairness model auditing guarantees.

A few works have researched fairness using other explainability techniques such as counterfactual explanations \citep{DBLP:conf/nips/KusnerLRS17,DBLP:conf/cogmi/ManerbaG21,DBLP:conf/aies/MutluYG22}.
We don't focus on counterfactual explanations but on feature attribution methods that allow us to measure unequal feature contribution to the prediction. Further work can be envisioned by applying explainable AI techniques to the \enquote{Equal Treatment Inspector} or constructing the explanation distribution out of other techniques.

\subsection{Classifier Two-Sample Test (C2ST)}
The use of classifiers as a statistical tests of independence $W \perp Z$ for a binary $Z$ has been previously explored in the literature \citep{DBLP:conf/iclr/Lopez-PazO17}. The approach relies on testing accuracy of a classifier trained to distinguish $Z=1$ (positives) from $Z=0$ (negatives) given $W=w$. 
In the null hypothesis that the distributions of positives and negatives are the same, no classifier is better than a random answer with accuracy $\nicefrac{1}{2}$. This assumes equal proportion of instances of the two distributions in the training and test set. 
Our approach builds on this idea, but it considers testing the AUC instead of the accuracy. Thus, we remove the assumption of equal proportions\footnote{For unequal proportions, one can consider the accuracy of the majority class, but this still make the requirement to know the true proportion of positives and negatives.}. We also show in Section \ref{app:stat.independence.exp} that using AUC may achieve a better power than using accuracy.

\cite{DBLP:conf/icml/LiuXL0GS20} propose a kernel-based approach to two-sample
tests classification. 
Alike work has also been used in Kaggle competitions under the name of \enquote{Adversarial Validation}~\citep{kaggleAdversarial,howtowinKaggle}, a technique which aims to detect which features are distinct between train and leaderboard datasets to avoid possible leaderboard shakes. 


\cite{DBLP:journals/corr/EdwardsS15} focuses on removing statistical parity from images by
using an adversary that tries to predict the relevant sensitive variable from the model representation and censoring the learning of the representation of the model and data on images and neural networks. While methods for images or text data are often developed specifically for neural networks and cannot be directly applied to traditional machine learning techniques, we focus on tabular data where techniques such as gradient boosting decision trees achieve state-of-the-art model performance \citep{DBLP:conf/nips/GrinsztajnOV22,DBLP:journals/corr/abs-2101-02118,BorisovNNtabular}. Furthermore, our model and data projection into the explanation distributions leverages Shapley value theory to provide fairness auditing guarantees. In this sense, our work can be viewed as an extension of their work, both in theoretical and practical applications.




\section{True to the Model or True to the Data?}
The \enquote{Equal Treatment Inspector} proposed in this work relies on the explanation distributions that satisfies efficiency and uninformative theoretical properties. 
We have used the Shapley values as an explainable AI method that satisfies these properties. 
A variety of (current) papers discusses the application of Shapley values, for feature attribution in machine learning models~\cite{DBLP:journals/kais/StrumbeljK14,DBLP:journals/natmi/LundbergECDPNKH20,DBLP:conf/nips/LundbergL17,lundberg2018explainable}. 
However, the correct way to connect a model to a coalitional game, which is the central concept of Shapley values, is a source of controversy, with two main approaches $(i)$ an interventional \cite{DBLP:journals/ai/AasJL21,DBLP:conf/nips/FryeRF20,Zern2023Interventional} or $(ii)$ an observational formulation of the conditional expectation (cf. Section \ref{sec:relatedWork}) \cite{DBLP:conf/icml/SundararajanN20,DBLP:conf/sp/DattaSZ16,DBLP:journals/corr/abs-1911-00467}.

In the following experiment, we compare what are the differences between estimating the Shapley values using one or the other approach.  We benchmark this experiment on the four prediction tasks based on the US census data~\cite{DBLP:conf/nips/DingHMS21} and using the \enquote{Equal Treatment Inspector}, where both the model $f_\theta(X)$ and $g_\psi(\Ss(f_\theta,X))$ are linear models. We will calculate the Shapley values using the SHAP linear explainer. \footnote{\url{https://shap.readthedocs.io/en/latest/generated/shap.explainers.Linear.html}}

The comparison depends on a feature perturbation hyperparameter: whether the approach to compute the SHAP values is either \textit{interventional} or \textit{correlation dependent}. The interventional SHAP values break the dependence structure between features in the model to uncover how the model would behave if the inputs are changed (as it was an intervention). 
This option is said to stay \enquote{true to the model} meaning it will only give allocation credit to the features that are actually used by the model.

On the other hand, the full conditional approximation of the SHAP values respects the correlations of the input features. If the model depends on one input that is correlated with another input, then both get some credit for the model’s behaviour. 
This option is said to say \enquote{true to the data}, meaning that it only considers how the model would behave when respecting the correlations in the input data~\cite{DBLP:journals/corr/abs-2006-16234}.

In our case, we will measure the difference between the two approaches by looking at the linear coefficients of the model $g_\psi$ and comparing the performance of predicting protected attributes, for this case only between White-Other.

% Please add the following required packages to your document preamble:
% \usepackage[table,xcdraw]{xcolor}
% If you use beamer only pass "xcolor=table" option, i.e. \documentclass[xcolor=table]{beamer}
\begin{table}[ht]
\centering
\caption{AUC comparison of the \enquote{Explanation Shift Detector} between estimating the Shapley values between the interventional and the correlation-dependent approaches for the four prediction tasks based on the US census dataset~\cite{DBLP:conf/nips/DingHMS21}. The $\%$ character represents the relative difference. The performance differences are negligible.}\label{tab:t2mt2d.auc}
\begin{tabular}{l|llc}
           & Interventional                                                          & Correlation & \%           \\ \hline
Income      & 0.736438                                                                & 0.736439              & 1.1e-06 \\
Employment  & 0.747923                                                                & 0.747923              & 4.44e-07 \\
Mobility    & 0.690734                                                                & 0.690735              & 8.2e-07 \\
Travel Time & 0.790512 & 0.790512              & 3.0e-07
\end{tabular}
\end{table}

\begin{table}[ht]
\caption{Linear regression coefficients comparison of the \carlos{Check that this errors is not more consistent}\enquote{Explanation Shift Detector} between estimating the Shapley values between the interventional and the correlation-dependent approaches for one of the US census based prediction tasks (ACS Income). The $\%$ character represents the relative difference. The coefficients show negligible differences between the calculation methods}\label{tab:t2mt2d.coefficients}
\centering
\begin{tabular}{l|rrr}

                & \multicolumn{1}{l}{Interventional} & \multicolumn{1}{l}{Correlation} & \multicolumn{1}{c}{\%} \\ \hline
Marital         & 0.348170                           & 0.348190                        & 2.0e-05                 \\
Worked Hours    & 0.103258                           & -0.103254                       & 3.5e-06                 \\
Class of worker & 0.579126                           & 0.579119                        & 6.6e-06                 \\
Sex             & 0.003494                           & 0.003497                        & 3.4e-06                 \\
Occupation      & 0.195736                           & 0.195744                        & 8.2e-06                 \\
Age             & -0.018958                          & -0.018954                       & 4.2e-06                 \\
Education       & -0.006840                          & -0.006840                       & 5.9e-07                 \\
Relationship    & 0.034209                           & 0.034212                        & 2.5e-06                

\end{tabular}
\end{table}


In Table \ref{tab:t2mt2d.auc} and Table \ref{tab:t2mt2d.coefficients} we can see the comparison of the effects of using the aforementioned approaches to learn our proposed method, the \enquote{Explanation Shift Detector}. 
Even though the two approaches differ theoretically, the differences become negligible when explaining the protected characteristic, i.e. when providing the linear regression coefficients.
\section{Experiments on datasets derived from the US Census}
In this section we apply our approach to more datasets derived from the US census database. We discuss what are the differences on three (plus the one in the main body of the paper) prediction tasks based on the US census data: ACS Income (main body), ACS Travel Time, ACS Employment and ACS Mobility.

We follow the same methodology as in the experimental section. Where we divide a dataset into three equal splits $\{\Dd{tr},\Dd{val},\Dd{te} \} \subseteq \D$ and select our protected attribute, $Z$, to be a feature that indicates the ethnicity of an individual. We train our model $f_\beta$ on $\{X^{tr},Y^{tr}\}$ 
and, the \enquote{Demographic Parity Inspector} $g_\psi$ on $\{\Ss(f_\beta,X^{val}),Z^{val}\}$. Both methods are evaluated on $\{X^{te},Z^{te},y^{te}\}$. For the type of models, $f_\beta$, as we are in tabular data we focus on we choose $f_\theta$ to be a \texttt{xgboost}\cite{xgboost}  that achieve state-of-the-art model performance~\cite{grinsztajn2022why,DBLP:journals/corr/abs-2101-02118,BorisovNNtabular},and for the inspector $g_\psi$ a logistic regression. The final explanations are given by the coefficients of the logistic regression.


\subsection{ACS Employment}
The goal of this task is to predict whether an individual, is employed. 
In this section we are going to do a comparison within different protected attribute groups for CA14.



\begin{figure*}[ht]
\centering
\includegraphics[width=.49\textwidth]{images/detector_auc_ACSEmployment.png}\hfill
\includegraphics[width=.49\textwidth]{images/feature_importance_ACSEmployment.png}
\caption{In the left figure, comparison of the performance of \textit{Demographic Parity Inspector}, over the state of CA14 for the ACS Employment dataset. For this dataset most of different protected attributes comparison have different AUC distributions in terms of mean and variance. The group that receives the most comparison is Other-Black. This difference in the model behaviour is due to features such as \enquote{Education} and \enquote{Citizenship}. 
Interestingly, features such as \enquote{difficulties on the hearing or seeing}, do not play a role in the OOD model behaviour.}
\label{fig:xai.employment}
\end{figure*}
Compared to other prediction tasks based on the US census dataset, the measured domographic parity violation for this task  is smallest. The AUC of the \enquote{Demographic Parity Inspector} is ranging from $0.55$ to $0.70$. For Asian-Black demographic parity violation we see that there is a lot of variation of the AUC, indicating that the method achieves different values on the bootstrapping folds. 
If we look for the features driving the demographic parity gap, we see particularly high values when comparing \enquote{Asian} and \enquote{Black} populations, and for features \enquote{Citizenship} and \enquote{Employment}. 
On average, the most important features across all group comparisons are also \enquote{Education} and \enquote{Area}.


\subsection{ACS Travel Time}

The goal of this task is to predict whether an individual has a commute to work that is longer than 20 minutes. The threshold of 20 minutes was chosen as it is the US-wide median travel time to work based on 2018 data



\begin{figure*}[ht]
\centering
\includegraphics[width=.49\textwidth]{images/detector_auc_ACSTravelTime.png}\hfill
\includegraphics[width=.49\textwidth]{images/feature_importance_ACSTravelTime.png}
\caption{In the left figure, comparison of the performance of \textit{Demographic Parity Inspector}, in different pair-wise ethnicities for the ACS Travel Time dataset. In the right image, the features that are more probable to be originating the demographic parity violation (higher implies more probable).}
\label{fig:xai.traveltime}
\end{figure*}

In general, the values of the AUC of the \enquote{Demographic Parity Inspector} are in the range of $0.50$ to $0.60$. 
By inspecting the features we can see that our proposed methods highlight different demographic parity violation drivers depending on the pair-wise comparison made. 
In general, the feature \enquote{Education}, \enquote{Citizenship} and \enquote{Area} are the features with the highest difference. Even though for Asian-Black pairwise comparison \enquote{Employment} is also one of the most relevant features.

\subsection{ACS Mobility}

The goal of this task is to predict whether an individual had the same residential address one year ago, only including individuals between the ages of 18 and 35. This filtering increases the difficulty of the prediction task, as the base rate of staying at the same address is above $90\%$ for the general population~\cite{ding2021retiring}.

\begin{figure*}[ht]
\centering
\includegraphics[width=.49\textwidth]{images/detector_auc_ACSMobility.png}\hfill
\includegraphics[width=.49\textwidth]{images/feature_importance_ACSMobility.png}
\caption{In the left figure, comparison of the performance of \textit{Demographic Parity Inspector}, in different pair-wise ethnicities for the ACS Mobility dataset. In the right image, the features that are more probable to be originating the demographic parity violation (higher implies more probable).}
\label{fig:xai.mobility}
\end{figure*}

In general, the values of the AUC of the \enquote{Demographic Parity Inspector} are in the range of $0.55$ to $0.80$. 
By inspecting the features we can see that our proposed methods highlight different demographic parity violation to be the source of the demographic parity violation depending on the ethnic pair-wise comparison selected. 
In general the feature \enquote{Ancestry}, i.e. "ancestors' lives with details like where they lived, who they lived with, and what they did for a living", plays a high relevance when predicting demographic parity violation in the mobility prediction task. 
In other prediction tasks such as when predicting employment (cf Figure \ref{fig:xai.employment}), even if the input data distribution across the protected groups is similar, the weight that the \enquote{Demographic Parity Inspector} assigns is distinct, this is due to our method measuring demographic parity violations with respect to the model not just with respect to the data.


\section{Experiments on real data}
In this section, we extend the prediction task of the main body of the paper.  The methodology used follows the same structure, we start by creating a distribution shift by training the model $f_\theta$ in California in 2014 and evaluating it in the rest of the states in 2018, creating a geopolitical and temporal shift. The model $g_\theta$ is trained each time on each state using only the $X^{ood}$ in the absence of the label, and its performance is evaluated by a 50/50 random train-test split. As models we use a gradient boosting decision tree\cite{xgboost,catboost} as estimator $f_\theta$, approximating the Shapley values by TreeExplainer \cite{lundberg2020local2global}, and using logistic regression for the \textit{Explanation Shift Detector}.

\subsection{ACS Employment}
The goal of this task is to predict whether an individual, between the ages of 16 and 90, is employed. For this prediction task the AUC of all the other states (except PR18) falls below $0.60$, indicating not high OOD explanations. For the most OOD state, PR18, the \enquote{Explanation Shift Detector} finds that the model has shifted due to features such as Citizenship or Military Service.

\begin{figure*}[ht]
\centering
\includegraphics[width=.49\textwidth]{images/AUC_OOD_ACSEmployment.png}\hfill
\includegraphics[width=.49\textwidth]{images/feature_importance_ACSEmployment.png}
\caption{In the left figure, comparison of the performance of \textit{Explanation Shift Detector}, in different states for the ACS Employment dataset. For this dataset, most of the states have the same OOD detection AUC, except for PR18. This difference in the model behaviour is due to features such as Citizenship and Military Service. Features such as difficulties in hearing or seeing, do not play a role in the OOD model behaviour.}
\label{fig:xai.employment}
\end{figure*}

\subsection{ACS Income}
 The goal of this task is to predict whether an individual's income is above $\$50,000$, only includes individuals above the age of 16, and report an income of at least $\$100$. This dataset can serve as a comparable replacement to the UCI Adult dataset.

 For this prediction task the results are different from the previous two cases, the state with the highest OOD score is $KS18$, with the \enquote{Explanation Shift Detector} highlighting features as Place of Birth, Race or Working Hours Per Week. The closest state to ID is CA18, where there is only a temporal shift without any geospatial distribution shift.
\begin{figure*}[ht]
\centering
\includegraphics[width=.49\textwidth]{images/AUC_OOD_ACSIncome.png}\hfill
\includegraphics[width=.49\textwidth]{images/feature_importance_ACSIncome.png}
\caption{In the left figure, comparison of the performance of \textit{Explanation Shift Detector}, in different states for the ACS Income prediction task.  In the left figure, we can see how the state with the highest OOD AUC detection is KS18 and not PR18 as in other prediction tasks, this difference with respect to the other prediction task can be attributed to \enquote{Place of Birth}, whose feature attributions the model finds to be more different than in CA14.}
\label{fig:xai.income}
\end{figure*}


\subsection{ACS Mobility}

The goal of this task is to predict whether an individual had the same residential address one year ago, only including individuals between the ages of 18 and 35. The goal of this filtering is to increase the prediction task difficulty, staying at the same address base rate is above $90\%$ for the population~\cite{ding2021retiring}.

The results of this experiment present a similar behaviour as the ACS Income prediction task (cf. Section \ref{fig:xai.income}), where the in-land states of the US are in an AUC range of $0.55-0.70$ and is the state of PR18 who achieves a higher OOD AUC. The features driving this behaviour are Citizenship for PR18 and Ancestry(Census record of your ancestors' lives with details like where they lived, who they lived with, and what they did for a living) for the other states. 

\begin{figure*}[ht]
\centering
\includegraphics[width=.49\textwidth]{images/AUC_OOD_ACSMobility.png}\hfill
\includegraphics[width=.49\textwidth]{images/feature_importance_ACSMobility.png}
\caption{In the left figure, comparison of the performance of \textit{Explanation Shift Detector}, in different states for the ACS Mobility dataset. Except for PR18, all the other states fall below an AUC of OOD detection of $0.70$. If we look at the features driving this difference is due to the Citizenship and the Ancestry relationship. For the other states protected social attributes such as Race or Marital status play an important role.}
\label{fig:xai.mobility}
\end{figure*}

\clearpage
\section{LIME as an Alternative to Shapley Values}\label{app:LIME}

The definition of ET (Def. \ref{def:et}) is parametric in the explanation function. We used Shapley values for their theoretical advantages (see Appendix~\ref{app:foundation.xai}).
Another widely used feature attribution technique is %that satisfies the aforementioned properties (efficiency and uninformative features Section~\ref{app:foundation.xai}) and can be used to create the explanation distributions is 
LIME (Local Interpretable Model-Agnostic Explanations). The intuition behind LIME is to create a local linear model that approximates the behavior of the original model in a small neighbourhood of the instance to explain~\citep{ribeiro2016why,ribeiro2016modelagnostic}, whose mathematical intuition is very similar to the Taylor/Maclaurin series. %In this work, we have proposed distributions of Shapley values to measure the philosophical liberal-political oriented notion of  Equal Treatment. 
This section discusses the differences in our approach when adopting LIME instead of the SHAP implementation of Shapley values. First of all, LIME has certain drawbacks:

\begin{itemize}
    \item \textbf{Computationally Expensive:} Its current implementation is more computationally expensive than current SHAP implementations such as TreeSHAP~\citep{DBLP:journals/natmi/LundbergECDPNKH20}, Data SHAP \citep{DBLP:conf/aistats/KwonRZ21,DBLP:conf/icml/GhorbaniZ19}, or Local and Connected SHAP~\citep{DBLP:conf/iclr/ChenSWJ19}. This problem is exacerbated when producing explanations for multiple instances (as in our case). In fact, LIME requires sampling data and fitting a linear model, which is a computationally more expensive approach than the aforementioned model-specific approaches to SHAP. A comparison of the execution  time is reported in the next sub-section.
    
    \item \textbf{Local Neighborhood:} The randomness in the calculation of local neighbourhoods can lead to instability of the LIME explanations. %Slight variations of this explanation hyperparameter can lead to different local explanations. 
    Works including \cite{DBLP:conf/aies/SlackHJSL20,alvarez2018towards,adebayo2018sanity} highlight that several types of feature attributions explanations, including LIME, can vary greatly.
    
    \item \textbf{Dimensionality:} LIME requires, as a hyperparameter, the number of features to use for the local linear model. For our method, all the features in the explanation distribution should be used. However, linear models suffer from the curse of dimensionality. In our experiments, this is not apparent, since our synthetic and real datasets are low-dimensional.  
    %This creates a dimensionality problem because, for our method to work, the explanation distributions have to be from the exact same dimensions as the input data. Reducing the number of features to be explained might improve the computational burden. Particularly on the experiments on synthetic data of this section it is not since the number of features is limited to three.
\end{itemize}

\begin{figure}[ht]
\centering
\includegraphics[width=.49\textwidth]{images/fairAuditSyntheticCaseALIME.pdf}\hfill
\includegraphics[width=.49\textwidth]{images/fairAuditSyntheticCaseBLIME.pdf}
\caption{AUC of the ET inspect using SHAP vs using LIME.}
\label{fig:lime}
\end{figure}

Figure \ref{fig:lime} compares the AUC of the ET inspector using SHAP and LIME as explanation functions over the synthetic dataset of Section \ref{subsec:exp.synthetic}. In both scenarios (indirect case and uninformative case), the two approaches have similar results. In both cases, however, the stability of using SHAP is better than using LIME.

%The left plot shows the sensitivity under the Indirect Discrimination Case, using the synthetic data experimental setup of Section~\ref{subsec:exp.synthetic} and figure Figure~\ref{fig:fairSyn} of the main body of the paper, both distributions capture this equal treatment violation; LIME is less sensitive than SHAP. The right plot shows the sensitivity to detect uninformative discrimination. The performance for measuring equal treatment of the model  is similar between the two types of distributions, but LIME suffers from two drawbacks: its theoretical properties rely on the definition of a local neighborhood, which can lead to unstable explanations (false positives or false negatives on explanation shift detection), and its computational runtime required is much higher than that of SHAP (see experiments below).

\subsection{Runtime}

We conduct an analysis of the runtimes of generating the explanation distributions using TreeShap vs LIME. %The experiments were run on a server with 4 vCPUs and 32 GB of RAM. 
We adopt\texttt{shap} version $0.41.0$ and \texttt{lime} version $0.2.0.1$ as software packages. In order to define the local neighborhood for both methods in this example, we used all the data provided as background data. The model $f_{\theta}$ is set to \texttt{xgboost}. As data we produce a randon generated data matrix, of varying dimensions. When varying the number of samples, we use 5 features, and when varying the number of features, we use$1000$ samples.
%
Figure \ref{fig:computational} shows the elapsed  time for generating explanation distributions with varying numbers of samples and columns. 

The runtime required to generate explanation distributions using LIME is considerably greater than using SHAP. %This is due to the fact that LIME requires training a local model for each instance of the input data to be explained, which can be computationally expensive. In contrast, SHAP relies on heuristic approximations to estimate the feature attribution with no need to train a model for each instance. 
The difference becomes more pronounced as the number of samples and features increases.
%
%We note that the computational burden of generating the explanation distributions can be further reduced by limiting the number of features to be explained, as this reduces the dimensionality of the explanation distributions. However, this will inhibit the quality of the explanation shift detection as it won't be able to detect changes in the distribution shift that impact the model on those features.
%
%Given the current state-of-the-art of software packages we have used, SHAP values due to lower runtime required and that theoretical guarantees hold with the implementations. 
%Further work can be envisioned on developing novel methods and software to select the best approach between SHAP, LIME and possibly other approaches for a dataset at hand.



\begin{figure}[ht]
\centering
\includegraphics[width=.49\textwidth]{images/computational_samples.pdf}\hfill
\includegraphics[width=.49\textwidth]{images/computational_features.pdf}
\caption{Elapsed time for generating explanation distributions using SHAP and LIME with different numbers of samples (left) and features (right) on synthetically generated datasets. Note that the y-scale is logarithmic.} %The experiments were run on a server with 4 vCPUs and 32 GB of RAM. The runtime required to create explanation distributions with LIME is far greater than SHAP for a gradient-boosting decision tree.}
\label{fig:computational}
\end{figure}

\end{document}