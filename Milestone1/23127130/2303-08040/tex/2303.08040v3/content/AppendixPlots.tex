\section{Experiments on datasets derived from the US Census}\label{app:extra.experiments}

In the main body of the paper, we considered the  ACS Income dataset. Here, we experiment with additional datasets derived from the US census database \citep{DBLP:conf/nips/DingHMS21}: ACS Travel Time, ACS Employment and ACS Mobility. We compare fairness of the prediction tasks for pairs of protected attribute groups over the California 2014 district data.

%
We follow the same methodology as in the experimental Section \ref{sec:experiments}. 
%
%Where we divide a dataset into three equal splits $\{\Dd{tr},\Dd{val},\Dd{te} \} \subseteq \D$ and select our protected attribute, $Z$, to be a feature that indicates the ethnicity of an individual. We train our model $f_\beta$ on $\{X^{tr},Y^{tr}\}$ 
%and, the \enquote{Equal Treatment Inspector} $g_\psi$ on $\{\Ss(f_\beta,X^{val}),Z^{val}\}$. Both methods are evaluated on $\{X^{te},Z^{te},y^{te}\}$. For the type of models, 
The choice of \texttt{xgboost} \citep{DBLP:conf/kdd/ChenG16} for the model
$f_\beta$ is motivated as it achieves state-of-the-art 
performance~\cite{DBLP:conf/nips/GrinsztajnOV22,DBLP:journals/corr/abs-2101-02118,BorisovNNtabular}. The choice of logistic regression for the inspector $g_\psi$ is motivated by its direct interpretability.

\subsection{ACS Employment}
The goal of this task is to predict whether an individual, is employed. Figure \ref{fig:xai.employment} shows a low
DP violation, compared to the other prediction tasks based on the US census dataset. The AUC of the \enquote{Equal Treatment Inspector} is ranging from $0.55$ to $0.70$. For Asian vs Black un-equal treatment we see that there significant variation of the AUC, indicating that the method achieves different values on the bootstrapping folds. 
Looking at the features driving the ET violation, we see particularly high values when comparing \enquote{Asian} and \enquote{Black} populations, and for features \enquote{Citizenship} and \enquote{Employment}. 
On average, the most important features across all group comparisons are also \enquote{Education} and \enquote{Area}. Interestingly, features such as \enquote{difficulties on the hearing or seeing}, do not play a role.



\begin{figure*}[ht]
\centering
\includegraphics[width=.49\textwidth]{images/detector_auc_ACSEmployment.pdf}\hfill
\includegraphics[width=.49\textwidth]{images/feature_importance_ACSEmployment.pdf}
\caption{Left: AUC of the inspector for ET and DP, over the district of California 2014 for the ACS Employment dataset. Right: contribution of features to the ET inspector performance.}
\label{fig:xai.employment}
\end{figure*}




\subsection{ACS Travel Time}

The goal of this task is to predict whether an individual has a commute to work that is longer than 20 minutes. The threshold of 20 minutes was chosen as it is the US-wide median travel time to work based on 2018 data. Figure \ref{fig:xai.traveltime} shows an AUC for the ET inspector in the range of $0.50$ to $0.60$. 
By looking at the features, they highlight different ET drivers depending on the pair-wise comparison made. 
In general, the feature \enquote{Education}, \enquote{Citizenship} and \enquote{Area} are the those with the highest difference. Even though for Asian-Black pairwise comparison \enquote{Employment} is also one of the most relevant features.

\begin{figure*}[ht]
\centering
\includegraphics[width=.49\textwidth]{images/detector_auc_ACSTravelTime.pdf}\hfill
\includegraphics[width=.49\textwidth]{images/feature_importance_ACSTravelTime.pdf}
\caption{Left: AUC of the inspector for ET and DP, over the district of California 2014 for the ACS Travel Time dataset. Right: contribution of features to the ET inspector performance.}
\label{fig:xai.traveltime}
\end{figure*}



\subsection{ACS Mobility}

The goal of this task is to predict whether an individual had the same residential address one year ago, only including individuals between the ages of 18 and 35. This filtering increases the difficulty of the prediction task, as the base rate of staying at the same address is above $90\%$ for the general population~\citep{DBLP:conf/nips/DingHMS21}. Figure \ref{fig:xai.mobility} show an AUC of the ET inspector in the range of $0.55$ to $0.80$. 
By looking at the features, they highlight different  source of the ET violation depending on the group pair-wise comparison. 
In general the feature \enquote{Ancestry}, i.e. ``ancestors' lives with details like where they lived, who they lived with, and what they did for a living", plays a high relevance when predicting ET violation. 
%In other prediction tasks such as when predicting employment (cf Figure \ref{fig:xai.employment}), even if the input data distribution across the protected groups is similar, the weight that the \enquote{Equal Treatment Inspector} assigns is distinct, this is due to our method measuring demographic parity violations with respect to the model not just with respect to the data.

\begin{figure*}[ht]
\centering
\includegraphics[width=.49\textwidth]{images/detector_auc_ACSMobility.pdf}\hfill
\includegraphics[width=.49\textwidth]{images/feature_importance_ACSMobility.pdf}
\caption{Left: AUC of the inspector for ET and DP, over the district of California 2014 for the ACS Mobility dataset. Right: contribution of features to the ET inspector performance.}
\label{fig:xai.mobility}
\end{figure*}



