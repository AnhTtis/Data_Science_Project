\documentclass{article}
\usepackage[utf8]{inputenc}
\usepackage{hyperref}
\usepackage{amsmath}
\usepackage{natbib}

\title{tempCarlos}
\author{Salvatore Ruggieri}
\date{November 2022}

\begin{document}

%\maketitle

%\section{Introduction}

%Assume regression problems.

\subsection*{Preliminaries}

Attributes $N = \{1, \ldots, n\}$.\\

\noindent Shapley value of $j \in N$: 
\[ \phi_j(x^{\star}) = \sum_{S \subseteq N\setminus\{j\}} \frac{|S|! (n-|S|-1)!}{n!} (v(S \cup \{j\}, x^{\star}) - v(S, x^{\star}))\] 
where $$v(S, x^{\star}) = E[f(x)|x_S = x^{\star}_S] = \int f(x) p(x|x_S = x^{\star}_S) dx$$

\noindent Efficiency property: $f(x^{\star}) = \phi_0 + \sum_j \phi_j(x^{\star})$ where $\phi_0 = E[f(x)]$.\\

(c) Recall that $A \perp B$ iff $P(A\leq a, B \leq b) = P(A\leq a) \cdot P(B \leq b)$ for all $a, b$\\


\subsection*{Results: proofs and counter-examples}


%(1) $\phi_k(X) \equiv 0 \Rightarrow f(X) \perp \phi_k(X)$?\\
%This is true. Because $\phi_k(x) \leq b$ is always true ($b \geq 0$) or always false ($b < 0$) for any $x$\\

We denote by $\mathbf{X}$ a collection of attributes (random variables) $X_1, \ldots, X_n$. Without any loss of generality, we assume that $E[X_i] = 0$.\\

A first result is useful when auditing a linear model (with unknown coefficients) over \textit{independent} attributes.\\

\textbf{Lemma 1.} Let $f(x) = \beta_0 + \sum_j \beta_j \cdot x_j$ be a linear model over $\mathbf{X}$. If the attributes in $\mathbf{X}$ are independent, then for any $k \in N$: $\phi_k(X) \equiv 0 \Leftrightarrow f(X) \perp X_k$.\\
\textbf{Proof.}
For linear models over independent attributes, it turns out \cite{DBLP:journals/ai/AasJL21} that
$\phi_k(x) = \beta_k \cdot (x_k - E[X_k]) = \beta_k \cdot x_k$ (since we assume $E[X_k] = 0$). Thus, $\phi_k(x) \equiv 0$ \footnote{Since $\phi_k(x) = \beta_k \cdot (x_k - E[X_k])$, this is equivalent to have$\phi_k(x) = \phi_k(x') = 0$  for $x, x'$ such that $x_k \neq x'_k$.} iff $\beta_k = 0$. By independence of $\mathbf{X}$, this is equivalent to $f(X) \perp X_k$.

\mbox{}\hfill QED\\

The result does not extend to the case of dependent attributes. We show this next.
%
Let us consider a simple linear model $f(x_1, x_2) = \beta_0 + \beta_1 \cdot x_1 + \beta_2 \cdot x_2$ with $X_1, X_2$ dependent. We calculate:

$$v(\emptyset, x^{\star}) = E[f(x)] = \beta_0 + \beta_1 \cdot E[X_1] + \beta_2 \cdot E[X_2] = \beta_0$$
$$v(\{1\}, x^{\star}) = E[f(x)|x_1 =x^{\star}_1] = \beta_0 + \beta_1 \cdot x^{\star}_1 + \beta_2 \cdot E[X_2|x_1 =x^{\star}_1]$$
$$v(\{2\}, x^{\star}) = E[f(x)|x_2 =x^{\star}_2] = \beta_0 + \beta_1 \cdot E[X_1|x_2 =x^{\star}_2] + \beta_2 \cdot x^{\star}_2$$
$$v(\{1, 2\}, x^{\star}) = \beta_0 + \beta_1 \cdot x^{\star}_1 + \beta_2 \cdot x^{\star}_2$$

\noindent and then:

$$\phi_1(x^{\star}) = \frac{\beta_2}{2} E[X_2|x_1 =x^{\star}_1] + \beta_1 \cdot x^{\star}_1  - \frac{\beta_1}{2} \cdot E[X_1|x_2 =x^{\star}_2]$$

$$\phi_2(x^{\star}) = \frac{\beta_1}{2} E[X_1|x_2 =x^{\star}_2] + \beta_2 \cdot x^{\star}_2  - \frac{\beta_2}{2} \cdot E[X_2|x_1 =x^{\star}_1]$$
%
%
%$$\phi_2(x^{\star}) = \frac{\beta_1}{2} (E[X_1|x_2 =x^{\star}_2] - E[X_1]) + \beta_2 \cdot (x^{\star}_2 - E[X_2])  + \frac{\beta_2}{2} \cdot (E[X_2] - E[X_2|x_1 =x^{\star}_1])$$

% If $\beta_1 = \beta_2 = 0$ then both $\phi_k(X) \equiv 0$ and $f(X) \perp X_k$ trivially hold.

Consider $\beta_1 \neq 0$ and $\beta_2 = 0$. We have $\phi_2(X) \equiv 0$ iff $E[X_1|x_2 =x^{\star}_2] = 0$ for all $x^{\star}_2$. Consider a uniform distribution with $P(X_1=v) = 1/4$ for $v=1, -1, 2, -2$, and $X_2=X_1^2$. We have $E[X_1|x_2 =v] = E[X_1] = 0$. Thus $\phi_2(x^{\star}) \equiv 0$. However, $f(X) = \beta_0 + \beta_1 \cdot X_1 \not \perp X_2$.

\medskip

We observe that:
$$\phi_2(x^{\star}) - \phi_1(x^{\star}) = \beta_2 \cdot x^{\star}_2 - \beta_1 \cdot x^{\star}_1$$
This and the efficiency property:
$$\phi_1(x^{\star}) + \phi_2(x^{\star}) = f(x^{\star}) - \phi_0 = \beta_0 - \phi_0 + \beta_1 \cdot x^{\star}_1 + \beta_2 \cdot x^{\star}_2$$


\bibliographystyle{plain}
\bibliography{references.bib}

\end{document}
