% \begin{abstract}
Creating multiple-choice questions to assess reading comprehension of a given article involves generating question-answer pairs (QAPs) and adequate distractors. 
%We tackle this challenge by dividing it into two tasks: question generation and distractor generation. 
We present two methods to tackle the challenge of QAP generations:
(1) A deep-learning-based end-to-end question generation system based on T5 Transformer with Preprocessing and Postprocessing Pipelines (TP3).
We use the finetuned T5 model for our downstream task of question generation and improve accuracy using a combination of various NLP tools and algorithms in preprocessing and postprocessing to select appropriate answers and filter undesirable questions.
(2) A sequence-learning-based scheme to generate adequate QAPs via meta-sequence representations of sentences.
A meta-sequence is a sequence of vectors comprising semantic and syntactic tags. we devise a scheme called MetaQA to learn meta sequences from training data to form pairs of a meta sequence for a declarative sentence and a corresponding interrogative sentence.
% A trained MetaQA model converts a given declarative sentence to a meta sequence, finds a matched MD, and uses the corresponding MIs and the input sentence to generate QAPs. 
The TP3 works well on unseen data, which is complemented by MetaQA. 
Both methods can generate well-formed and grammatically correct questions.
Moreover, we present a novel approach to automatically generate adequate distractors for a given QAP. The method is a combination of part-of-speech tagging, named-entity tagging, semantic-role labeling, regular expressions, domain knowledge bases, word embeddings, word edit distance, WordNet, and other algorithms. 
% \end{abstract}