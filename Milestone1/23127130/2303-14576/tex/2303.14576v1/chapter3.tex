\chapter{Deep-learning-based Question Generation}

\section{Description of TP3}\label{sec:finetuning}

We describe how we train and finetune a pretrained T5 transformer for our downstream task of question generation and use a combination of various NLP tools and algorithms to build the preprocessing and postprocessing pipelines for generating QAPs. %the QG system.


%More details can be referred from the original paper \cite{raffel2020exploring}.

%\subsection{Selecting a dataset to finetune T5}

%To use the T5 model for our downstream task of question generation, we finetune the T5 model on QAP datasets.

There are a number of public QAP datasets available for fine-tuning T5, including RACE \cite{lai2017large}, CoQA \cite{reddy2019coqa}, and SQuAD \cite{Rajpurkar_2016}. 
%
RACE is a large-scale dataset collected from Gaokao English examinations over the years, where Gaokao is the national college entrance examinations held once every year in mainland China. It consists of
 more than 28,000 passages and nearly 100,000 questions, including cloze questions.
CoQA is a conversational-style question-answer dataset. It contains a series of interconnected questions and answers %that appear 
in conversations. % about a text. 
SQuAD is a reading comprehension dataset, consisting of more than 100,000 QAPs %and corresponding answers 
posted by crowdworkers on a set of Wikipedia articles.

Among these datasets, %Based on question types, and the 
SQuAD is more commonly used in the question generation research. We use SQuAD to finetune pretrained T5 models. 
%
%The T5 model is tasked with generating a relevant question for a given answer and a context for that answer.
For each QAP and corresponding context extracted from the SQuAD training dataset, we concatenate the answer and the context with markings in the format of $\langle answer\rangle answer\_text \langle context\rangle context\_text$ as input, with the question as the target, where the context is the entire article for the QAP in SQuAD. 
%for calculating loss during training, 
We then set the maximum input length to 512 and the target length to 128 to avoid infinite loops and repetitions of target outputs. We feed the concatenated text input and question target into a pretrained T5 model for fine-tuning and use AdamW \cite{loshchilov2018decoupled} as an optimizer with various learning rates to obtain a better model.

%We discuss the experiment results in Section \ref{sec:evaluations}
%large auto lr: 0.0019054607179632484
%base auto lr: 0.0004365158322401656
%We'd like to explore a large number of different learning rates. 
To explore various learning rates, we first use automatic evaluation methods to narrow down a smaller range of the learning rates and then use human judges to determine the best learning rate.
In particular, we first finetune the base model with a learning rate of $1.905 \times 10^{-3}$ and the large model with a learning rate of $4.365 \times 10^{-4}$. The learning rates are calculated %selected 
using the Cyclical Learning Rates (CLR) method \cite{7926641}, which is used to find automatically the best global learning rate. % and schedule for the global learning rates.
%, and it is used in many well-known deep learning frameworks, such as PyTorch and TensorFlow.
Evaluated by human judges, we found that %the generated  are fair but not ideal as our expected. We also noticed that 
the best learning rate calculated by CLR is always larger than the actual best learning rate in our experiments.

We then finetune T5-Base and T5-Large with dynamic learning rates from the learning rate calculated by CLR with a reduced learning rate for each epoch. For example, we finetune T5-Base starting from a learning rate of $1.905 \times 10^{-3}$ and multiply the previous learning rate by 0.5 for the current epoch until % multiplication in each epoch decreased to 
the learning rate of $1.86 \times 10^{-6}$ is reached. Likewise, we finetune T5-Large in the same way starting from $4.365 \times 10^{-4}$ until
%,  with 0.5 multiplication in each epoch decreased to 
the learning rate of $1.364 \times 10^{-5}$ is reached. However, the generated results are still below expectations. 

We therefore proceed to 
%Since the learning rate calculated by CLR is not reliable, we then tried to 
finetune the models with various learning rates we choose. In particular,
%According to previous models training experience, 
we first finetune T5-Base with a constant learning rate from $10^{-3}$ to $10^{-4}$ with a $2.5 \times 10^{-4}$ decrement for each model, and from $10^{-4}$ to $10^{-5}$ with a $2.5 \times 10^{-5}$ decrement for each epoch. Likewise, we  
finetune T5-Large with a learning rate from $10^{-4}$ to $10^{-5}$ with a $2 \times 10^{-5}$ decrement for each epoch, and from $10^{-5}$ to $10^{-6}$ with a $2 \times 10^{-6}$ decrement for each epoch. 


%The following metrics are used to automatically evaluate the performance of question generation from the T5 models:
%
%BLEU scores measure the quality of text that has been translated by a machine from one natural language to another using n-grams. We used a cumulative 4-gram BLEU score (B4) as an evaluation metric.
%
%ROUGE-L uses statistics based on the Longest Common Subsequence (LCS) to evaluate recall by how many words in reference sentences are used in predicted sentences.
%
%METEOR is a precision-based metric for evaluating machine-translation out- put.
%
%BERTScore is a contextual embedding based metric for ...

Evaluated using BLEU \cite{papineni-etal-2002-bleu}, ROUGE \cite{lin-2004-rouge}, METEOR \cite{banerjee-lavie-2005-meteor} and BERTScore \cite{bert-score}, we find that the learning rates ranging from $10^{-4}$ to $10^{-5}$ for T5-Base and the learning rates ranging from $10^{-5}$ to $10^{-6}$ for T5-Large perform better.
%large model performed better results.
%
%After human evaluation, 
%Specifically, 
%we find that T5-Base with a learning rate of $3 \times 10^{-5}$ and T5-Large with a learning rate of $6 \times 10^{-6}$ produce the best results, and so we use these models to generate questions. 
Moreover, as expected,
the overall performance of T5-Large is better than T5-Base. 

Tables \ref{tab:auto-evaluation-base} and \ref{tab:auto-evaluation-large}
depict the measurement results for T5-Base and T5-Large, respectively, where R1, R2, RL, and RLsum stand for, respectively,
Rouge-1, Rouge-2, Rouge-L, and Rouge-Lsum. %The average is taken over all measures after multiplying the BERTScore scores by 100. 
The boldfaced number indicates the best in its column. It is evident that T5-Base with the
learning rate of $3 \times 10^{-5}$ and
T5-Large with the learning rate of $8 \times 10^{-6}$ produce the best results.
For convenience, we refer to these two finetuned models as T5-Base-SQuAD$_1$ and T5-Large-SQuAD$_1$
to distinguish them with the existing T5-Base-SQuAD model.
We sometimes also denote T5-Base-SQuAD$_1$ as T5-SQUAD$_1$ when there is no
confusion of what size of the dataset is used to pretrain T5.
% and so we use it to generate questions. 
%picked these models for our QG system.

\begin{table}[h]
\footnotesize
\centering
\caption{Automatic Evaluation of T5-Base-SQuAD$_1$}
\begin{tabular}{l|c|c|c|c|c|c|c|c}
\hline
\textbf{Learning Rate} & \textbf{BLEU}  & \textbf{R1}    & \textbf{R2}    & \textbf{RL}    & \textbf{RLsum} & \textbf{METEOR} & \textbf{BERTScore} & \textbf{Average} \\ \hline
5e-5              & 20.01 & 50.71 & 28.38 & 46.59 & 46.61 & 45.46  & 51.51 & 41.32     \\ \hline
3e-5              & \textbf{22.63} & \textbf{54.90} & \textbf{32.22} & \textbf{50.97} & \textbf{50.99} & \textbf{48.98}  & \textbf{55.82} & \textbf{45.22}     \\ \hline
2.5e-5            & 22.50 & 54.36 & 31.93 & 50.49 & 50.50 & 48.64  & 55.61 & 44.86     \\ \hline
1e-5              & 20.17 & 50.46 & 28.38 & 46.79 & 46.81 & 44.97  & 51.82 & 41.34     \\ \hline
Dynamic %1.2e-4 $\rightarrow$ 3e-5 
& 20.57 & 51.88 & 28.99 & 47.67 & 47.68 & 47.38  & 53.34 & 42.50     \\ \hline
%Dynamic automatic      & 4.74  & 22.49 & 7.54  & 19.04 & 19.05 & 19.84  & 13.94 & 15.23     \\ \hline
\end{tabular}
\label{tab:auto-evaluation-base}
\end{table}

\begin{table}[h]
\footnotesize
\centering
\caption{Automatic Evaluation on T5-Large-SQuAD$_1$}
\begin{tabular}{l|c|c|c|c|c|c|c|c}
\hline
\textbf{Learning Rate} & \textbf{BLEU}  & \textbf{R1}    & \textbf{R2}    & \textbf{RL}    & \textbf{RLsum} & \textbf{METEOR} & \textbf{BERTScore} & \textbf{Average} \\ \hline
3e-5                    & 23.01 & 54.49 & 31.92 & 50.51 & 50.51 & 50.00  & 56.19 & 45.23    \\ \hline
1e-5                    & 23.66 & 51.88 & 32.88 & 51.43 & 51.42 & 50.53  & 56.65 & 45.50   \\ \hline
8e-6                    & 23.83 & \textbf{55.48} & \textbf{33.08} & \textbf{51.58} & \textbf{51.58} & 50.61  & \textbf{56.94} & \textbf{46.15}   \\ \hline
6e-6                    & \textbf{23.84} & 55.24 & 32.91 & 51.35 & 51.35 & \textbf{50.70}  & 56.57 & 45.99   \\ \hline
Dynamic                 & 20.86 & 52.00 & 29.46 & 48.03 & 48.03 & 47.68  & 53.85 & 42.84   \\ \hline
\end{tabular}
\label{tab:auto-evaluation-large}
\end{table}

\section{Processing Pipelines} %Candidate Answers selecting and filtering}
\label{sec:P3}
%preprocessing

The processing pipelines consist of preprocessing to select appropriate answers, question generation, and postprocessing to filter undesirable questions (see Fig. \ref{fig:2}).

\begin{figure}[h]
  \centering
  \includegraphics[width= \linewidth]{TP3Architecture}
  \caption{TP3 Architecture}
  \label{fig:2}
%   \Description{}
\end{figure}


\subsection{Preprocessing}

We observe that how to choose an answer would affect the quality of a question generated for the answer.
%however, not all answers can generate a satisfying question, 
%In general, T5 models work well on unseen data. and generates well-formed and grammatically correct questions from the given answer and context. There are, however, a few types of questions that could lead to generating incoherent questions. For example, asking about the type of verbs or asking about the clauses that express reasons or purposes may generate incoherent QAPs. 
We use a combination of NLP tools and algorithms to construct a preprocessing pipeline for
selecting appropriate answers as follows: % and filter undesirable questions. 
%
%In preprocessing stage, we select candidate answers by following 3 steps:
\begin{enumerate}
\item \textit{Remove unsuitable sentences}. We
first remove all interrogative and imperative sentences from the given article. We may do so 
by, for instance, simply removing any sentence that begins with a WH word or a verb and any sentence that ends with a question mark.
We then use semantic-role labeling \cite{Shi2019SimpleBM} to analyze sentences and remove those
that do not have any one of the following semantic-role tags: %ach sentence must consist of 3 semantic roles: 
subject, verb, and object. % we filtered out if any sentence doesn't satisfy this requirement. 
For each remaining sentence, if the total number of words contained in it, excluding stop words, is less than 4, then remove this sentence.
We then label the remaining sentences as \textit{suitable} sentences.

\item \textit{Remove candidate answers with inappropriate semantic-role labels}.
%The basic idea of candidate answers extraction is that, 
%We'd like to extract 
Nouns and phrasal nouns are candidate answers. But not any noun or phrasal noun would be suitable to be an answer.
We'd want a candidate answer to associate with a specific meaning. 
%In particular, we'd select candidate answer
%with a semantic-role label of subject, object, or manner. 
%To do so, 
%we first remove candidate answers with inappropriate semantic-role labels.
%We'd also like to extract 
%semantic roles with nouns. % as the main word in the role.
%To achieve this, 
Specifically, if a noun in a suitable sentence is identified as a name entity \cite{Peters2017SemisupervisedST}
or has a semantic-role label in the set of $\{$ARG, TMP, LOC, MNR, CAU, DIR$\}$, then keep it
as a candidate answer and remove the rest,
where ARG represents subject or object,
TMP represents time, LOC represents location, MNR represents manner, CAU represents cause, and DIR represents direction.
%
%with the following semantic-role labels as candidate answers and remove: ARG, TMP, LOC, MNR, and CAU as candidate answers, as well as words with named-entity labels \cite{Peters2017SemisupervisedST}. 
If a few candidate nouns occur consecutively, we treat the sequence of these nouns as a candidate answer phrase.

For example, in the sentence ``The engineers at the Massachusetts Institute of Technology (MIT) have taken it a step further changing the actual composition of plants in order to get them to perform diverse, even unusual functions", the phrase ``Massachusetts Institute of Technology" is recognized as a named entity, without a semantic-role label. Thus, it should not be selected as an answer. If it is selected, then the following QAP
(``Where is MIT located", ``Massachusetts Institute of Technology") will be generated, which is inadequate. 

\item \textit{Remove or prune answers with inadequate POS tags}.
Using semantic-role labels to identify what nouns to keep does not always work.
%works well most of the times, but
%there are situations that it would produce inadequate QAPs. 
%words are labeled wrong or labels redundant words. 
For example, the phrasal noun ``This widget" in the sentence "This widget is more technologically advanced now"
has a semantic-role label of ARG1 (subject), which leads to the generation of the following question: 
``What widget is more technologically advanced now?" It is evident that this QAP is inadequate even though it is
grammatically correct. 
Note that ``This" has
a POS (part-of-speech) tag of PDT (predeterminer).
%This example indicates that semantic-role labeling alone is insufficient and POS tagging should also be used.
%is not necessary for the answer and it is labeled as predeterminer (PDT) by POS tagging, if the model generates question according to this answer phrase, it may generate inappropriate question "What widget is more technologically advanced now?" with answer "This widget".
For another example, while the word ``now" in the sentence 
has a semantic-role label of TMP (time), its POS tag is RB (adverb).
In general, we remove nouns with a POS tag in $\{$RB, RP, CC, DT, IN, MD, PDT, PRP, WP, WDT, WRB$\}$ 
or prune words with such a POS tag at either end of a phrasal noun.
%
%should be removed. In general,
%, it should not be used as answer for question generation.
%we remove such answers or keep the noun component in a phrase by pruning the words at either end 
%with %in an answer phrase according to 
%the following POS tags: 
%RB, RP, CC, DT, IN, MD, PDT, PRP, WP, WDT, WRB. 
After this treatment, the candidate answer ``now" is removed and the candidate answer phrase ``This widget" is pruned to ``widget". For this answer and the input sentence, the following question is generated: ``What is more technologically advanced now?" Evidently this question is more adequate.

\item \textit{Remove common answers}. %%%needs work
%Through experiments, % and generalization, 
We observe that certain candidate answers, such as ``anyone", ``people", and ``stuff", 
would often lead to generation of inadequate questions. 
%For example, "anyone", "people", "stuff" etc. 
Such words tend to be common words that should be removed. 
We do so by looking up the probabilities of 1-grams from the
%language model of the 
Google Books Ngram Dataset \cite{doi:10.1126/science.1199644}. If the probability of a noun word is greater 
than 0.15\%, we remove its candidacy. Likewise, we may also treat noun phrases by looking up the probabilities $n$-grams for $n > 1$, but doing so would incur much more processing time.

\item \textit{Filtering answers appearing in clauses}.
We observe that a candidate answer appearing in the latter part of a clause would often lead to a generation of an
inadequate QAP. Such candidate answers would appear at lower levels in a dependency tree. We use the following procedure to identify such candidate answers: For each remaining sentence $s$, we first generate its dependency tree \cite{varga2010wlv}. Let $h_s$ be the height of the tree. Suppose that a candidate answer $a$ appears in a clause contained in $s$. If $a$ is a single noun, let its height in the tree be $h_a$. If $a$ is a phrasal noun, 
let the average height of the heights of the words contained in $a$ be $h_a$. If $h_a \geq \tfrac{2}{3}h_s$, then remove $a$.

Take the following sentence as an example: ``While I tend to buy a lot of books, these three were given to me as gifts, which might add to the meaning I attach to them." In 
this sentences, the following noun ``gifts” and phrasal nouns ``a lot of books" and ``the meaning I attach to them"are labeled as object.
However, T5
% is labeled as object, where 
%``a lot of books" and ``gifts" both reference the same semantic-role label of object and the underlying transformer may not be able to
resolves multiple objects poorly, 
and if we choose ``the meaning I attach to them" as an answer, T5 will generate the following question: 
%answer phrase "the meaning I attach to them" is labeled as object in " which" clause in the sentence , since  and the model sometime can't resolve multiple objects appropriately, the phrase may leads to generate an inadequate question 
``What did the gifts add to the books", which is inadequate. Since this phrasal noun appears in a clause and at a lower level of the dependency tree, it is removed from being selected as a candidate answer.


%and its height is at least 2/3 of the height of the tree, then remove it.
%is higher than that of the verb node, 
%and if it higher than 75\% of the dependency tree height, we filter the it out, 
%If the answer is a phrasal noun, we take the average height of each word in the phrase as its height. 

\item \textit{Removing redundant answers}.
%We remove duplicate answers.
% by Named Entity labeling and semantic-role labeling, 
%If a candidate answer is contained in a longer answer phrase, we remove the shorter one.
If a candidate answer word or phrase is contained in another candidate answer phrase and appear in the same sentence,
we extract from the dependency tree of the sentence the subtree $T_s$ for the shorter candidate phrase and subtree $T_l$ for the longer candidate phrase, then $T_s$ is also a subtree of $T_l$. If $T_s$ and $T_l$ share the same root, %is same as the root part of the $h_l$, we say 
then the shorter candidate answer is more syntactically important than the longer one, and so we remove the longer candidate answer. Otherwise, remove the shorter candidate answer.
 
Take the sentence ``The longest track and field event at the Summer Olympics is the 50-kilometer race walk, which is about five miles longer than the marathon" as an example.  
The shorter phrase ``Summer Olympics" is recognized as a named entity, 
%and a longer phrase "The longest track and field event at the Summer Olympics" is labeled as subject by semantic role.
which leads to the generation of the following inadequate QAP: (``What is the longest track and field event", ``Summer Olympics). 
On the other hand,
the longer phrase ``The longest track and field event at the Summer Olympics" is labeled as subject for its semantic role, which leads to the generation of the following adequate QAP: (``What is the 50-kilometer race walk", ``The longest track and field event at the Summer Olympics").
Since the root word for the longer phrase is ``event" that is not contained in the shorter phrase, so the shorter phrase is removed to avoid generating the inadequate QAP.


\end{enumerate}

\subsection{Question generation}

After extracting all candidate answers from the preprocessing pipeline, for each answer extracted, we use three adjacent sentences as the context, with the middle sentence containing the answer, and concatenate the answer and the context with marks into the following format as input to a fine-turned T5 model: $\langle answer\rangle$answer\_text$\langle context\rangle$context\_text, to generate candidate questions.
We note that the greedy search in the decoder of the T5 model does not guarantee the optimal result, we use beam search with 3 beams to select the word sequences with the top 3 probabilities from the probability distribution and acquire 3 candidate questions.
We then concatenate each candidate question with the corresponding answer as a new sentence and generate
an embedding vector representation for it using %map the sentence to %dense vector space 
%sentence embeddings by utilizing 
the pretrained  RoBERTa-Large model \cite{Liu2019RoBERTaAR,reimers-2019-sentence-bert}, and select the most semantically similar question to the context as the final target question.

\subsection{Postprocessing}

Recall that in the preprocessing pipeline, %the answers selecting and filtering stage, 
we have removed inappropriate candidate answers.
% that would lead to generating appropriate questions.
However, some of the remaining answers may still lead to generating inappropriate questions. 
Thus, in the postprocessing pipeline, we proceed to remove inadequate questions as follows:
%by following 3 steps to tackle this problem:
\begin{enumerate}
\item \textit{Remove questions that contain the answers}.
Remove a question if the corresponding answer or the main body of the answer is contained in the question.
%\hl{Changes start here}
If the answer includes a clause, we extract the main body of the answer as follows:
Parse the answer to constituency tree \cite{Joshi2018ExtendingAP} and remove the subtree rooted with a subordinate clause label SBAR, the remaining part of the phrase is the main body of the answer.

For example, in the sentence ``The first, which I take to reading every spring is Ernest Hemningway's A Moveable Feast", ``The first, which I take to reading every spring" is labeled as subject. Using it as a candidate answer  generates an inadequate question for the answer ``What is the first book I reread?" Note that the phrase ``The first" can be extracted as the main body of the answer, which is contained in the question. Thus, this QAP is removed.
%the generated question can be removed.
%\hl{Changes end here}

\item \textit{Remove short questions}.
If the generated question, after removing stop words, consists of only one word, then remove the question.
%we filter this question. 
For example, ``What is it?" and ``Who is she?" will be removed because after removing stop words,
the former becomes ``What" and the latter becomes ``Who". On the other hand, ``Where is Boston?" will remain.

\item \textit{Remove unsuitable questions}.
Recall that we generate the question from the adjacent three sentences in the article, with the middle sentence containing the answer. However, the middle sentence may not be the only sentence containing the answer. In other words,
the first or the last sentences may also contain the answer. 
Assuming that all three sentences contain the answer, our finetuned T5 transformer may generate a question based on the first sentence or the last sentence. If the first sentence or the last sentence is not a suitable sentence
we labeled in the preprocessing pipeline, 
%our generatable sentence list which we described in the preprocessing stage, 
the question being generated may be in appropriate. 
We'd want to make sure that the question is generated for a suitable sentence.
For this purpose, we first identify which sentence the question is generated for. In particular,
%it may cause generating inappropriate questions.
let $s_i$ for $i=1,2,3$ be the 3 sentences and $(q,a)$ be the question generated for answer $a$.
Let $QA$ denote the union of the set of words in $q$ and the set of words in $a$.
Likewise, let $S_i$ be the set of words in $s_i$. If $QA \cap S_i$ is the largest among the other two
intersections, then $q$ is likely generated from $s_i$ for $a$. If $s_i$ is not suitable, then
remove $q$.

Note that we may also consider word sequences in addition to word sets. For example, we may consider longest common subsequences or longest common substrings when comparing two word sequences. But in our experiments, they don't seem to 
add extra benefits.
%a word sequence that concatenate the word sequence in the generated question 
%and the word sequence in the answer, and form three other word sequences each of which corresponds to an input sentence.
%
%in the order they appear and the answer into a word sequence, and place words in each sentence in the order they appear into another word sequence.
%We then count the number of words in the intersection of two word sequences and the number of longest common subsequence for question answer with each sentence. If the sentence has more intersection and more longest common subsequence with the question and answer, we think the question is generated from that sentence, and if the sentence is not in the generatable sentence list, we filter out the corresponding generated question.

%(4) Prune question,  parser question to cp tree,  if question contains clause (tree node contains SBAR), and the question longer than 10 words, remove the leaves followed by when, where, which. 
\end{enumerate}




\section{Running Samples}
Suppose that the following sentences are given as a article:

Returning to a book you've read many times can feel like drinks with an old friend. There's a welcome familiarity - but also sometimes a slight suspicion that time has changed you both, and thus the relationship. But books don't change, people do. And that's what makes the act of rereading so rich and transformative.
The beauty of rereading lies in the idea that our bond with the work is based on our present mental register. It's true, the older I get, the more I feel time has wings. But with reading, it's all about the present. It's about the now and what one contributes to the now, because reading is a give and take between author and reader. Each has to pull their own weight.
There are three books I reread annually. The first, which I take to reading every spring is Ernest Hemningway's A Moveable Feast. Published in 1964, it's his classic memoir of 1920s Paris. The language is almost intoxicating, an aging writer looking back on an ambitious yet simpler time. Another is Annie Dillard's Holy the Firm, her poetic 1975 ramble about everything and nothing. The third book is Julio Cortazar's Save Twilight: Selected Poems, because poetry. And because Cortazar. 
While I tend to buy a lot of books, these three were given to me as gifts, which might add to the meaning I attach to them. But I imagine that, while money is indeed wonderful and necessary, rereading an author's work is the highest currency a reader can pay them. The best books are the ones that open further as time passes. But remember, it's you that has to grow and read and reread in order to better understand your friends.


The following QAPs are generated by TP3:

\begin{enumerate}
\item Question: Who wrote Holy the Firm?

Answer: Annie Dillard

\item Question: Who wrote A Moveable Feast?

Answer: Ernest Hemningway

\item Question: What is the first book I reread every spring?

Answer: Ernest Hemningway's A Moveable Feast

\item Question: Which book by Annie Dillard is a 1975 ramble about everything and nothing?

Answer: Holy the Firm

\item Question: What is the name of the book I reread every year?

Answer: Julio Cortazar's Save Twilight

\item Question: What Ernest Hemingway book do I reread every spring?

Answer:  A Moveable Feast

\item Question: What is intoxicating about Ernest Hemingway's A Moveable Feast?

Answer: The language

\item Question: How many books do I tend to buy?

Answer: a lot of books

\item Question: Who is Ernest Hemningway?

Answer: an aging writer

\item Question: Rereading an author's work is what do I imagine a reader can pay them?

Answer: the highest currency

\end{enumerate}



\section{Evaluations}\label{sec:evaluations}

To evaluate the quality of QAPs generated by TP3-Base and TP3-Large, we use the standard automatic evaluation metrics as well as
human judgments.

\subsection{Automatic evaluations}

We first compare T5-SQuAD$_1$ with the exiting QG models 
with the standard automatic evaluation metrics as before:
BLEU, % \cite{10.3115/1073083.1073135}, 
ROUGE-1 (R1), ROUGE-2 (R2), ROUGE-L (RL), ROUGE-LSum (RLsum), % \cite{lin-2004-rouge}, 
METEOR (MTR), % \cite{banerjee-lavie-2005-meteor}) to automatically evaluate our finetuned models, those n-grams-based metrics was used to evaluate the syntactic reconstruction ability of the models. Besides the n-grams-based metrics, we also used BERTScore \cite{bert-score} as semantic-based metric to evaluate the semantic reconstruction ability of the models.
and BERTScore (BScore). 
Since most existing QG models are based on pretrained transformers with the base dataset, we will
compare T5-Base-SQuAD$_1$ with the existing QG models. 
%The results are shown on
%Table \ref{tab:auto-evaluation}.



\begin{table}[h]
\footnotesize
\centering
\caption{Automatic evaluation results}
\begin{tabular}{l|c|c|c|c|c|c|c|c|c}
\hline
\textbf{Model}  & \textbf{Size}  & \textbf{BLEU} & \textbf{R1}  & \textbf{R2} & \textbf{RL} & \textbf{RLsum}          & \textbf{MTR}    & \textbf{BScore}     &\textbf{Average}  \\ \hline
ProphetNet  & Large & 22.88          & 51.37          & 29.48          & 47.11          & 47.09          & 41.46          & 49.31     & 41.24   \\ \hline
BART-hl     & Base  & 21.13          & 51.88          & 29.43          & 48.00          & 48.01          & 40.23          & 54.33     & 41.86    \\ \hline
BART-SQuAD  & Base  & 22.09          & 52.75          & 30.56          & 48.79          & 48.78          & 41.39          & 54.86     & 42.75    \\ \hline
T5-hl       & Base  & 23.19          & 53.52          & 31.22          & 49.40          & 49.40          & 42.68          & 55.48     & 43.56    \\ \hline
T5-SQuAD    & Base  & \textbf{23.74} & 54.12          & 31.84          & 49.82          & 49.81          & 43.63          & 55.68     & 44.09    \\ \hline
MixQG$_1$       & Base  & 23.53          & 54.39          & 32.06          & 50.05          & 50.02          & 43.83          & 55.66     & 44.22    \\ \hline
MixQG$_2$       & Base  & 23.74          & 54.28          & 32.23          & 50.35          & 50.34          & 43.91          & 55.71     & 44.37    \\ \hline
MixQG-SQuAD & Base  & 23.46          & 54.48          & 32.18          & 50.14          & 50.10          & 44.15          & \textbf{55.82} & 44.33\\ \hline
T5-SQuAD$_1$ & Base  & 22.62          & \textbf{54.87} & \textbf{32.20} & \textbf{50.99} & \textbf{50.98} & \textbf{48.98} & \textbf{55.82} & \textbf{45.21}  \\ \hline
\end{tabular}
\label{tab:auto-evaluation}
\end{table}

Table \ref{tab:auto-evaluation} shows automatic evaluation comparison results with 
%ProphetNet \cite{qi2020prophetnet}, BART-HL and BART-SQuAD \cite{lewis-etal-2020-bart}; T5-HL and T5 \cite{raffel2020exploring}; and MixQG$_1$, MixQG$_2$, and MixQG-SQuAD \cite{murakhovska2021mixqg}. All but ProphetNet are pretrained on the base dataset and then finetuned on the SQuAD dataset. 
%\hl{ Changes start here: }
ProphetNet \cite{qi2020prophetnet}, BART \cite{lewis-etal-2020-bart}, T5 \cite{raffel2020exploring} and MixQG \cite{murakhovska2021mixqg}. 
BART-SQuAD, T5-SQuAD, and MixQG-SQuAD are corresponding models finetuned on the SQuAD dataset.
BART-hl and T5-hl are augmented models using the ``highlight" encoding scheme introduced by Chan and Fan \cite{chan2019recurrent}.
%\hl{ Changes end here}

The results of MixQG$_1$ were presented in the original paper \cite{murakhovska2021mixqg}, and the results of MixQG$_2$ were computed by us using the pretrained model posted on HuggingFace (https://huggingface.co/Salesforce/mixqg-base). %The result of MetaQG was our improved model.
The results show that, except BLEU, T5-SQuAD$_1$ outperforms all other models on the ROUGE and METEOR metrics,  produces the same BERTScore score as that of MixQG-SQuAD. 
%T5-SQuAD$_1$'s BLEU score is slightly lower than MixQG because some of the lengths of the generated questions are shorter which results in Brevity Penalty in the BLEU algorithm. 
 Overall, T5-SQuAD$_1$ performs better than all the models in comparison.

\subsection{Manual evaluations of TP3}

A number of publications (e.g., see \cite{callison-burch-etal-2006-evaluating,liu-etal-2016-evaluate,nema-khapra-2018-towards}) have shown that the aforementioned automatic evaluation metrics based on n-gram similarities do not always 
correlate well with human judgments about the answerability of a question. 
Thus, we'd also need to use human experts to evaluate the qualities of QAPs generated
by TP3. We do so 
%We then manually evaluated our base and large models 
on the Gaokao-EN dataset consisting of 85 articles, where each article contains 15 to 20 sentences. We chose Gaokao-EN because expert evaluations are provided to us from a project we work on.
%TP3 generates a total of at least 1,270 QAPs. 
Table \ref{tab:manually-evaluation} depicts the evaluation results. Title abbreviations are explained below,
where the numbers in boldface are the best in the corresponding columns:
\begin{enumerate}
\item \textbf{Total} means the total number of QAPs generated by TP3. 
\item \textbf{ADQT} means the total number of adequate QAPs. These QAPs can be directly used without any modification. 
\item \textbf{ACPT} means the total number of
QAPs where the question, while semantically correct, contains a minor English issue that can be corrected with a minor effort. For example,
a question may simply be missing a word or a phrase at the end. Such QAPs may be deemed acceptable. 
\item \textbf{UA} means unacceptable QAPs. 
\item \textbf{ADQT-R} means the ratio of the adequate QAPs over all the QAPs being generated.
\item \textbf{ACPT-R} means the ratio of the adequate and acceptable QAPs over all the QAPs being generated.
%\item \textbf{Better} means the QAPs among the adequate QAPs under the current model that are better than the adequate QAPs generated under different models with respect to the same answers.
\end{enumerate}

\begin{table}[h]
\centering
\caption{Manual evaluation results for TP3-Base and TP3-Large over Gaokao-EN}
\begin{tabular}{l|c|c|c|c|c|c|c}
\hline
\textbf{TP3} &	\textbf{Learning Rate} &	\textbf{Total} & \textbf{ADQT} & \textbf{ACPT}	 & \textbf{UA} 
	& \textbf{ADQT-R}	& \textbf{ACPT-R} %& \textbf{Better} 	
	\\ \hline
Base		           & 3e-5 & \textbf{1290}& 1145& \textbf{63}& 82& 88.76& 93.64 %& 20
\\ \hline
\multirow{5}{*}{Large} & 3e-5 & 1287& 1162& 49& 76& 90.29& 94.09 %& 19
\\\cline{2-8}
   					   & 1e-5 & 1271& 1166& 39& 76& 91.74& 94.81 %& 16
   					   \\\cline{2-8}
   					   & 8e-6 & 1270& 1162& 39& 69& 91.50& 94.57 %& 27
   					   \\\cline{2-8}
   					   & 6e-6 & 1273& \textbf{1172}& 45& \textbf{56}& \textbf{92.07}& \textbf{95.60} %& \textbf{36}
   					   \\\cline{2-8}
   				 & Dynamic & 1288& 1116& 51& 121& 86.65& 90.61 %& 22
   				 \\\hline
\end{tabular}
\label{tab:manually-evaluation}
\end{table}