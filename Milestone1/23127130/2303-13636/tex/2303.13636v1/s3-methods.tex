\section{Learning-Oriented Efficient HR Estimation with PPG Data}
\label{sec:method}

In this section, we present the methodology for learning-oriented efficient HR estimation that combines signal processing and ML models using PPG data.
% Firstly, the monitoring system is introduced and we show each system module. 
% More specifically, for HR estimation, we present our method to predict HR using the ISPC dataset and our collected data.

% There are three approaches to calculating HR. 
% The conventional approach is calculating HR based on the ECG RR intervals.
% The second approach is to obtain HR directly from the PPG signal.
% Likewise, we concentrate on the third approach which exploits ML to estimate HR based on PPG HR.
% The ML algorithm takes the PPG HR obtained in the second approach as features and ECG HR in the first approach as labels, trying to estimate HR with higher accuracy.

\subsection{PPG-based HR monitoring system}\label{sec:ppg_system}

%In this section, we present the details of our monitoring system.
%The monitoring system architecture is shown in Fig.\ref{fig:system}.
\begin{figure}
  \centering
  \includegraphics[width=1\linewidth]{pic/hr-only-systemv2.pdf}
  \caption{The architecture of the PPG-based HR monitoring system.}
  \label{fig:system}
\end{figure}

Fig.\ref{fig:system} shows the architecture of our HR monitoring system, which has four stages.
In Stage 1, data collection modules collect the PPG signals. Here, a PPG sensor is attached to the subject's fingertip, which outputs red and infrared light signals. In Stage 2, the signal processing module reduces noises in PPG data and generates rough estimations of HR for every second.
%, ECG HR for the next stage. 
In Stage 3, an ML module takes a sequence of rough HRs to get more accurate HR estimations.
Finally, a report module reports the estimated HR readings to users. %, or saves them to the cloud for further analysis. 
The rest of this section provides a detailed description of Stages 2 and 3.

\subsection{HR Estimation}

\begin{figure}
  \centering
  \includegraphics[width=1\linewidth]{pic/hr-only-flowv2.pdf}
  \caption{Major steps of the PPG-based HR monitoring system.}
  \label{fig:stage2}
\end{figure}

Figure~\ref{fig:stage2} provides the detailed processing steps of our HR monitoring system, which are described as follows.

\subsubsection{Signal Processing based HR Estimation (Stage 2)}\label{sec:stage2}
% \paragraph{PPG data collection}
The input data for Stage 2 are time-series signals generated by PPG in the PPG data collection stage, which contains the red/infrared light signals reflected by blood under the skin. 
% Let [$s_1$, $s_2$, $s_3$, ..., $s_{25}$] denotes a sequence of light signals. 
We employed a sampling rate of 25 Hz, i.e., 25 signal samples every second.

\paragraph{Initial HR extraction (Stage 2.1)}
The processing in stage 2.1 takes the PPG light signals from the PPG data collection stage, finds the local peaks within the light sequence, and then calculated by counting the number of peaks to estimate HR~\cite{ppg_allen2007photoplethysmography}. Currently, our signal processing algorithm converts 25 signals in a second to 4 HR readings for that second (denoted as initial PPG-HR).
%a script is used to extract PPG HR. 
%Based on the principle of PPG sensor works for obtaining heart rate, 
%The output of step 2.1 is the calculated PPG HR, as shown in 2.2.
%\subsubsection{Initial HR outlier removal (step 2.3 and step 2.4)}
%to PPG HR to make it more reliable, stable, and less noisy. 

\paragraph{Z-score outlier filter (Stage 2.2)}
%The input for the Z-score filter is the PPG HR data in step 2.2, which was derived by the HR calculation algorithm based on the PPG raw signal data, and has a sample rate of \TODO{4Hz}. 
Due to motion artifacts, there may be large noises in PPG signals, leading to outliers in the initial HR estimations from stage 2.1. To eliminate these outliers, a Z-score outlier filter is applied.

The Z-score filter is a popular method used to find outliers. Given a data point, its z-score represents the distance (i.e., deviation) between its value and the mean of all data points, measured by multiples of standard deviation ($std$)~\cite{zscore1_mendenhall2016statistics, zscore2_spiegel2018schaum}. For example, if a data point's z-score is 3, then the difference between this data point and the mean is $3\times std$.
Therefore, if the z-score (i.e., distance) is larger than a threshold, then the data point can be viewed as an outlier. 
Here, we set the threshold z-score to 3 following common practice~\cite{zscore3_zhang2011illustration, zscore4_vysochanskij1980justification}. 

However, we cannot simply remove the outliers because HR is a time series that will be utilized as features by ML models. Here, deleting any values will possibly lose the time-related information. Therefore, we choose to revise the outliers' value to be the average of their two surrounding HR readings.

\paragraph{Smoothing (Stage 2.3)}
Even after the Z-score filter, there still could be abrupt peak or valley HR readings that are overly higher or lower than their surrounding HRs due to PPG signal noises,
Therefore, to further reduce the PPG HR fluctuation, we applied additional data smoothing. 

To smooth, we first take an average of the four HR readings within a second to convert them into one HR per second. This smoothing removes fluctuations of the HRs within a second. It also reduces the number of HRs to be input to the ML models in Stage 3.

Moreover, the HR reading of a person in general does not change abruptly in one second and returns back in the next second. That is, an HR reading should not be significantly different from the HRs before and after it. Hence, it is possible to smooth the HR based on a specific upper and lower boundary to further restrict the PPG HR fluctuation. In our experiment, we set the boundary to 5\%. That is, an HR reading can only be less than or equal to 5\% above or below its predecessor.
For HR readings that are more than 5\% higher (or lower) than the previous value, their values are changed to be 5\% higher (or lower) than the previous value. This 5\% boundary is determined based on the fluctuation range observed from the more reliable ECG data.

\subsubsection{ML-based HR Estimation (Stage 3)}\label{sec:ml_estimation}
% \paragraph{ML-based HR Estimation}
% Finally, 
The rough HR estimations generated by signal processing (denoted as PPG-HR) in Stage 2 are passed to an ML model to generate a more accurate HR estimation. As our PPG sensor sampling rate was set at 25Hz for power consideration to preserve energy, the PPG-HRs from these samples may still contain large errors. The ML model is particularly trained to reduce errors due to the low sampling frequency. 

More specifically, the ML model takes a sequence of the last $k$ PPG-HR estimations to generate the current HR reading. Intuitively, the ML model uses these $k$ PPG-HRs to assess the potential errors in them to produce a more accurate HR reading. We evaluated different values for $k$ in our experiments. We also evaluated different types of ML models, including DT, RF, KNN, SVM, and MLP. These evaluation results are reported in Section~\ref{sec:Evaluation}.


% The conventional method to estimate HR from PPG is using the signal processing method. 
% Recently, ML is found to be a promising method to estimate HR from PPG because it is powerful in summarising patterns from big data.
% As shown in section \ref{sec:related_work}, for both existing signal processing-based work and existing ML-based work, the input is the PPG light signal. 
% Besides, the signal processing methods usually need accelerometer signals to better remove the motion artifacts in the PPG signals to make sure they can estimate HR more accurately.
% Similarly, some ML-based models also include accelerometer signals as input features. 
% However, the input feature size could be very large when it includes both PPG light signals and accelerometer signals, which could result in a heavy and complex algorithm/model.
% Moreover, complicated signal processing method usually requires a higher sampling rate which will cause a resource burden to the wearable device and edge device.
% In this work, we combine signal processing with ML - the input for the learning model is not a light signal but processed PPG HR.
% In this way, the ML can ease the burden of the signal processing algorithm, as the result the processing algorithm can be simpler, the PPG sensor sampling rate can be smaller for wearable device resource consumption, and the ML model size can be smaller because the input dimension is smaller.

% In this section, we present the HR estimation method we used for our collected data, and since the ISPC dataset is a popular dataset to evaluate the HR monitoring system, we also consider the ISPC dataset in our HR estimation.
% Firstly, we present how we collect our data and process it in stage 1 and stage 2.
% For the ISPC dataset, there is no collection stage.
% And since there are many great existing signal processing methods proposed for the ISPC dataset, we simplify the method in \cite{zhang2014troika} and use it as the processing method in stage 2.
% The ML module in stage 3 is similar to both our data and the ISPC data, hence we present the ML module at the end of this section.

% \subsubsection{Our Data Processing}

% The major purpose of stage 2 is to provide quality data for ML models.
% Therefore, we extract PPG HR from PPG light signal, remove noise and outliers, and feed the processed PPG HR into the ML model.
% The details is shown in Fig.\ref{fig:stage2}. 


% Firstly, the PPG device and ECG device recorded the PPG raw data and ECG raw data simultaneously and respectively in stage 1. 
% Once the recording is done, the PPG sensor will export the raw data that contains the red light and infrared light time-domain sequence as shown in step a1. 
% And the ECG will generate a bin file that contains the ECG signals as shown in step b1.

% Then, at the beginning of stage 2, a script is utilized to extract PPG HR as shown in step 2.1.
% Based on the theory of how the PPG sensor works for obtaining heart rate \cite{ppg_allen2007photoplethysmography}, the script will take the PPG light signal from step a1, find the local peaks during the light sequence, and then the PPG HR can be calculated by counting the number of peaks in 1 minute. 
% The output of step 2.1 is the calculated PPG HR, as shown in 2.2.
% Besides, as shown in step b2.1, the associate ECG analysis software will extract the RR interval from the ECG bin file in step b1, and generate the ECG RR in step b2.2.

% After that, to eliminate the HR outlier, a Z-score outlier filter (step 2.3) and a smoothing algorithm (step 2.4) are applied to PPG HR to make it more reliable, stable, and less noisy. 
% Similarly, we preprocess the ECG RR intervals to eliminate the outliers using the Z-score outlier filter (step b2.3) and calculate ECG HR based on the processed ECG RR interval (step b2.4). 
% Next, we normalize the processed PPG HR and sent it to the ML model in stage 3 along with the ECG HR as shown in steps 2.5 and b2.4.
% Finally, the ML model in step 3.1 will be trained and make predictions, and we obtain the estimation results in step 3.2.
% Details of the Z-score outlier filter and the smoothing algorithm are as follows.

% \paragraph{Z-score outlier filter}

% The input for the Z-score filter is the PPG HR data in step 2.2, which was derived by the HR calculation algorithm based on the PPG raw signal data, and has a sample rate of 4Hz. 
% The Z-score filter is a popular method used to find outliers.
% For a certain sample, its z-score absolute value represents the distance between its value and the mean of all samples' values in the units of standard deviation \cite{zscore1_mendenhall2016statistics, zscore2_spiegel2018schaum}.
% If the distance is too large, we consider it to be an outlier.
% Therefore, we can locate outliers after we set a distance threshold, which usually is set to 3 \cite{zscore3_zhang2011illustration, zscore4_vysochanskij1980justification}. 

% After locating the outliers, we cannot just remove them because HR is a time-domain series data and simply deleting any values will possibly lose the time-related information.
% Hence, in this paper, we revise the outliers' value to be the average value of their surrounding samples.

% \paragraph{Smoothing}

% After the Z-score filter, the PPG HR sample rate is still 4Hz, which means there are four HR readings per second.
% To pair the PPG HR to the ECG HR, which has 1 reading per second, we designed a smoothing algorithm as shown in step 2.4 to reduce the PPG HR sampling rate to be the same as ECG data and in the meantime further reduce the PPG HR fluctuation.

% The smoothing algorithm consists of two parts. 
% The first part is to discard a specific number of maximum and minimum values in a subset of data and take the average of the remaining samples and use that average value as the reading.
% In our experiment, we choose the 4 samples in the same second as a subset and do not discard any value because 4 is already a small number and we want to keep the processing algorithm to be simple to make sure it can run fast on the resource-limited edge device.
% But the subset ranges can be extended to longer time horizons if desired.
% After this, there will be one PPG HR reading every second.

% The second part is to adjust the data value based on a specific upper and lower boundary to further restrict the PPG HR fluctuation.
% The limitation can be a percentage or an absolute difference.
% And the boundary threshold can be determined by the fluctuation of ECG data, or by a given value. 
% In our experiment, we set the boundary to 5\%, this means a sample can only be less than or equal to 5\% above or below its predecessor.
% For outliers that are more than 5\% higher or lower than the previous value, we change their value to be 5\% higher or lower than the previous value.

\subsection{Data Sets and Model Training}
\subsubsection{Our Data Set and Model Training}
As discussed previously, the features of our ML models are $k$ last PPG-HR estimations, i.e., the PPG-HR estimations from the last $k$ seconds given one PPG-HR per second. The labels (i.e., ground truth HR) are obtained by ECG. During data collection, our subject wears the PPG monitoring system (described in Section~\ref{sec:ppg_system} and ECG electrodes at the same time for data recording. Hence, the ECG can provide ground truth HR for corresponding PPG signals and PPG-HR. The subject engaged in three scenarios: sitting, sleeping, and conducting daily activities. The daily activity includes office working, walking, drinking water, etc. Each recording lasted for 2 hours.

The collected data sets are then partitioned into training and testing data sets, with a split of 80\% and 20\%, respectively. All ML models also went through random search based hyperparameter tuning~\cite{bergstra2012random} to find the best model.

\subsubsection{ISPC Data Sets and Model Training}
To show that the accuracy benefit from our combined signal processing and ML method is generic, we also applied our method to the popular ISPC dataset~\cite{zhang2014troika}. This dataset provides raw PPG signals at 125 Hz along with ground truth HR readings. 
% Hence, it can be directly applied to ML model training.

Note that, since ISPC data sets used a different PPG sensor and different sampling rate, our signal processing steps cannot be applied to it. Therefore, we applied the signal processing steps in the original paper of the ISPC data set~\cite{zhang2014troika}. Moreover, because the source code of the ISPC data set's signal processing is not publicly available, and the potential changes of the dataset after its publication, the HR estimation errors from our reproduced signal processing are not completely the same as those reported in the original paper.

% \subsubsection{ISPC Data Processing}

% The experiment flow is similar to ISPC data except that the data processing method is different.
% Since our data are collected under different settings, for example, our subjects engaged in light activity, while ISPC subjects engaged in vigorous exercise thus there are more motion artifacts in ISPC data, the processing algorithm for our data cannot be applied to ISPC data. 
% Therefore, we designed a different stage 2 processing algorithm for the ISPC data.

% % There is a more intensive activity in ISPC data, hence more motion artifacts that need to be removed.
% Many existing signal processing methods were proposed for the ISPC dataset and worked well, but they tend to be complex.
% In this work, since we are combining the signal processing method with ML, we can let the ML model share the burden and thus simplify the signal processing algorithm.
% Inspired by the TROIKA \cite{zhang2014troika}, we designed a similar but simpler algorithm to process PPG signals and obtain PPG HR. 
% The algorithm contains four steps: bandpass, decomposition, reconstruction, and verification. 
% The bandpass and decomposition steps are similar to TROIKA. 
% The main difference between our ISPC data processing algorithm and TROIKA is in the reconstruction and verification phase.

% \paragraph{Reconstruction}

% In TROIKA, they proposed and performed the temporal difference algorithm and FOCUSS algorithm. 
% However, in our work, we just identify the motion artifacts by applying the periodogram function to the decomposed signals and removing the noise.

% \paragraph{Verification}

% In TROIKA, they implemented 2 rules for the verification. 
% The first rule is used to prevent a large change in the estimated HR values in 2 successive time windows. 
% The second rule is to prevent losing tracking HR over a long time. 
% However, in our work, we only use the first rule because the second rule requires a lot of extra memory and computation.

% \subsubsection{ML method for HR}

% Once the ECG data and PPG data are possessed, the ML model in stage 3 can be trained and tested with them. 
% The input data for the ML model contains many rows, and each row is an instance of the ML algorithm. 
% Each row contains the data to predict HR for one second, which includes several PPG HR for that second and before, and one ECG HR for that second.
% For each instance, the PPG HR is the features, and the ECG HR is the label.
% Here, the size of the features, i.e. the number of historical PPG HR used, is a parameter that we can tune. 
% In section \ref{sec:Evaluation}, we investigate the effect of the different numbers of historical PPG HR on estimation accuracy.

% The output of the ML model is the predicted HR for that second. 
% We can evaluate the estimation accuracy by comparing the predicted HR with the label.
% In the experiment, features, and labels will be split into a training set and a testing set, and different ML algorithms will be trained on the training set and evaluated with the testing set.

