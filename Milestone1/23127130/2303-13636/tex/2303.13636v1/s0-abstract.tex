\begin{abstract}

Recent studies showed that Photoplethysmography (PPG) sensors embedded in wearable devices can estimate heart rate (HR) with high accuracy. However, despite of prior research efforts, applying PPG sensor based HR estimation to embedded devices still faces challenges due to the energy-intensive high-frequency PPG sampling and the resource-intensive machine-learning models. In this work, we aim to explore HR estimation techniques that are more suitable for lower-power and resource-constrained embedded devices. More specifically, we seek to design techniques that could provide high-accuracy HR estimation with low-frequency PPG sampling, small model size, and fast inference time. First, we show that by combining signal processing and ML, it is possible to reduce the PPG sampling frequency from 125 Hz to only 25 Hz while providing higher HR estimation accuracy. This combination also helps to reduce the ML model feature size, leading to smaller models. Additionally, we present a comprehensive analysis on different ML models and feature sizes to compare their accuracy, model size, and inference time. The models explored include Decision Tree (DT), Random Forest (RF), K-nearest neighbor (KNN), Support vector machines (SVM), and Multi-layer perceptron (MLP). Experiments were conducted using both a widely-utilized dataset and our self-collected dataset. The experimental results show that our method by combining signal processing and ML had only $5\%$ error for HR estimation using  low-frequency PPG data. Moreover, our analysis showed that DT models with 10 to 20 input features usually have good accuracy, while are several magnitude smaller in model sizes and faster in inference time. 

%we investigate their effectiveness in predicting HR using PPG sensing data. To deploy the models in resource-constrained wearable devices, we also study the trade-off between prediction accuracy and the size of resulting models as well as the inference time. In addition to a widely-utilized ISPC dataset, we devised a device with an off-the-shelf PPG sensor to obtain a high-fidelity PPG dataset with a tunable setting. The evaluation results show that, for HR predictions, based on both ISPC and our customized datasets, all the considered learning models can accurately predict with  {\em mean average prediction error (MAPE)} being around $5\%$, which is comparable with the results reported in the prior work. Moreover, the DT models have the smallest size for different settings, which can be $2$ to $4$ magnitude smaller than other models. 



% Photoplethysmography (PPG) is a widely deployed sensor in wearable devices which used to observe HR and can also be exploited to monitor HRV. 
% However, the accuracy of PPG is suffering from motion artifacts. 
% Besides, a system that simultaneously monitors HR and HRV in real time is a challenge. 
% In this paper, we design a real time HR and HRV monitor framework that use PPG and machine learning to monitor HR and HRV simultaneously in real-time. 
% We install the PPG on fingertips, and apply machine learning to improve the PPG HR/HRV accuracy.
% We also installed a three-lead ECG as ground truth.
% Compared with the ground truth, the system achieve xxx.
% Therefore, we proved the effectiveness of the proposed system and the feasibility of applying machine learning in real time HR and HRV monitoring. 


\end{abstract}
