\section{Introduction}

\eat{HR are critical health vital signs and it is necessary to monitor them for different purposes continuously; 
the traditional approach to obtaining HR is to adopt ECG devices, which are hard to operate and inconvenient to utilize; 
Several wearable devices (such as chest straps, and smartwatches) have been designed to monitor HR using PPG sensors. However, most such devices do not provide an interface to access data for further analysis. 
In this work, with the objective of developing a low-power and sustainable wearable device to monitor HR using PPG sensors, we study various ML models and data processing methods to obtain accurate HR from PPG sensor data. Explain data processing methodology and ML models, and summary of results; 
list of contributions; paper organization
}

Heart rate (HR) is an important vital sign for the cardiovascular system and has been widely used as a biomarker for diagnostic and early prognostic of several diseases such as hypertension and heart failure~\cite{fox2007resting}. Besides the critical condition monitoring in the hospital setting, many applications also depend on continuously measured HR, such as fitness tracking, bio-metric identification, and frailty detection~\cite{biswas2019cornet,biswas2019heart,eskandari2022frailty}. Therefore, it is desirable to have a real-time HR monitoring system, which can conveniently provide accurate data in an effective manner to support such applications. 

The traditional and reliable approach to continuously monitoring HR is to utilize electrocardiogram (ECG) devices, which are generally expensive and inconvenient to deploy for outpatients or other users to operate in a continuous manner. Besides ECG, Photoplethysmography (PPG) sensors, when applied to the surface of the skin, like fingers, wrist, or earlobe, can utilize light signal to monitor changes in blood flow, which can be exploited to derive HR~\cite{castaneda2018review}. Given its low cost and convenience, PPG sensors have been widely utilized as an inexpensive alternative to monitor HR in wearable embedded devices (e.g., smartwatches), which have limited energy and computing resources. However, as PPG is susceptible to motion artifacts (MA), existing PPG-based work that removes MA to detect HR typically has two limitations.

%To make sure the monitoring system is easy to use in everyday life and to avoid transmitting sensitive health data over the Internet, there is a need to implement it in an edge device rather and utilize detection hardware that is portable and easy to install.


% To make sure the monitoring systems is easy to use in everyday life and protect user data privacy, there is a need to implement it in an edge device with detection hardware that is portable and easy to install.
% where a certain number of electrodes are placed at proper spots on the human torso to produce an electrocardiogram that shows the heart's electrical activity over time~\cite{wittenberg2020evaluation}. 
% For patients who need intensive care in hospitals, such ECG devices can provide accurate and real-time HR monitoring. 
% However, they are generally expensive and the cable-connected electrodes make their deployment inconvenient for outpatients or other users to operate in a continuous manner.

First, the classical method for extracting HR from PPG data is based on signal processing, which usually requires a relatively high sampling frequency (hundreds of Hz) and a complex process to remove MA noises to achieve high accuracy. This high sampling frequency may incur high energy consumption, preventing these signal processing techniques from being applied to energy and resource-constrained devices. 
% to obtain satisfactory estimation accuracy, and therefore not suitable for edge devices.
For example, a widely utilized data set for HR estimation is the IEEE Signal Processing Cup (ISPC) data set, which was proposed by Zhang et al., along with a signal processing-based algorithm that includes signal decomposition, sparse signal reconstruction, and spectrum peak tracking, to extract HR from PPG data~\cite{zhang2014troika}.
The ISPC dataset contains PPG signals sampled at a high frequency of 125 Hz, which was widely adopted by other studies~\cite{zhang2015photoplethysmography,bashar2019machine,puranik2019heart,chang2021deepheart}.
However, Bhowmik et al. has reported that the PPG sensor on smartwatch sampling at 100 Hz consumed significantly more energy than 25 Hz~\cite{bhowmik2017novel}.
%Bhowmik et al. came up with an algorithm to detect peaks in smartwatch PPG signals and found that keeping the PPG sampling rate at 100 Hz consumed significantly more energy than 25 Hz, and hence chose 25 Hz for their experiments\cite{bhowmik2017novel}. 
Furthermore, certain signal processing techniques may require intricate algorithms to achieve high accuracy, which further worsens energy consumption~\cite{tobola2015sampling}. 
%and a relatively higher PPG sampling rate, which is energy consuming, thus it is difficult for resource-constrained edge devices to support continuous HR monitoring for a long time.
% Therefore, the signal processing method is not suitable for edge devices.
% Based on the PPG sensing data, signal processing has been the classical approach to deriving various health-related information (including respiratory rate~\cite{johansson2003neural}, HR~\cite{zhang2014troika}). 

Second, with the advancement of machine learning (ML), several studies exploited various machine learning models to remove MA noises and estimate HR.
However, ML models, especially neural networks, usually are computationally intensive, limiting their application to embedded devices with limited resources. 
%In most existing works that use ML to predict HR from PPG signals, they used PPG signals as learning features, which result in large feature dimensions and possibly large model sizes
For example, Wittenberg et al. used complex deep learning models including convolutional neural network (CNN) and Gated recurrent unit (GRU) for PPG peaks detection \cite{wittenberg2020evaluation}. 
There were also studies that employed ML models such as K-means, Random Forest (RF), and Bayesian learning algorithm~\cite{bashar2019machine,alqaraawi2016heart}. 
Moreover, existing studies employ ML models~\cite{everson2019biotranslator,biswas2019cornet,xu2019deep} usually employed a large number of features (i.e., PPG signals) to achieve high accuracy, which led to large models that may not fit in small embedded devices.
To enable ML-based HR monitoring in a resource-constrained embedded device, there needs research to investigate feature dimension reduction to allow smaller ML models. There also needs exploration to determine the type of ML models that can provide accurate HR readings with less resource usage. 
%of an ML model is needed to ensure that it can run smoothly on resource-constrained devices such as embedded devices or edge devices.
%It also is necessary to evaluate the ML model efficiency on these resource-limited devices to find the recommended ML models for the monitoring system.
%However, for HR monitoring, previous works have not compared the accuracy and model size between different ML models, which cannot provide sufficient insights on which ML model is suitable for edge devices. 
%Some prior works used deep learning models, which usually are more resource-intensive than simple ML models. 

%Therefore, a systematic evaluation of different ML models is in demand to find the best ML model that achieves a good balance between accuracy and model size.

% Recently, with the advancement of machine learning (ML), there have been several studies that exploited various machine learning models (such as Bayesian learning~\cite{alqaraawi2016heart}, RF~\cite{bashar2019machine}, and neural network~\cite{wittenberg2020evaluation,everson2019biotranslator}) to estimate HR.
% For example, Bashar et al. proposed a K-means clustering and RF-based HR estimation algorithm \cite{bashar2019machine}.
% Alqaraawi et al. used a Bayesian learning algorithm to detect the PPG peaks and concluded that Bayesian learning is helpful and improved peak detection performance \cite{alqaraawi2016heart}.
% However, ML algorithms usually are computationally intensive, while PPG is usually used in a wearable device, which is resource-limited.
% Therefore, it is necessary to evaluate the ML model efficiency on these resource-limited devices to ensure that they can run smoothly on devices with limited resources.
% For HR monitoring, previous work has not compared the accuracy and model size between different ML models, which cannot provide sufficient insights on which ML model is suitable for edge devices. 
% Some of them used a simple ML model. Some of them used ML models, but they did not evaluate the model size. 

% Furthermore, there is also a lack of a dataset for ML-based HR estimation study.
% As HR is measured within a certain time window, a long experiment duration is essential to acquire enough data for developing and evaluating ML-based HR estimation techniques. 

%Giving consideration to that, the ISPC dataset is too short for HR estimation since each recording is only about 5 minutes long.


\eat{
As an alternative, Photoplethysmography (PPG) sensor is an inexpensive optical measurement device that has been widely used in commercial wearable devices such as smartwatches to monitor HR \cite{castaneda2018review}. 
But PPG is suffering from motion artifacts (MA) and many other factors which degrade the signal quality and hinder the HR accuracy \cite{fine2021sources}. 
Besides, most commercial HR monitoring devices do not provide the interface to access data for further analysis.
Moreover, the frequency of HR results provided by most commercial devices is relatively low - it may take a few seconds or even minutes to update a data reading.
But many applications require more timely data updates.
For example, for driving fatigue detection, since a lot of things can happen in a minute when driving, the data may need to be updated every second or faster.
In this work, to provide an accurate, real-time, low-cost, and sustainable system to monitor HR, we design an intelligent real-time HR monitoring system that uses a PPG sensor powered by machine learning (ML). 
}

%Therefore, in this paper, to design an ML-based real-time HR monitoring system that utilizes a PPG sensor on a resource-constrained device, we devise a monitoring device with an off-the-shelf PPG sensor for customized data collection and investigate its performance in accuracy and efficiency of various ML models for HR prediction. 
% To the best of our knowledge, all existing works apply PPG light signals as ML features in HR prediction, which may result in a large feature dimension and thus a large model.
In this paper, we report the design of our HR monitoring solution using an off-the-shelf PPG sensor. The system is specially designed for resource-constrained devices and addresses the above limitations. To address the limitations, we combine signal processing and ML methods. That is, we first applied signal processing to generate rough HR estimations. Then these rough HRs are passed through a smaller ML model to generate more accurate HR estimations.
%to exploit the strengths of both, improve HR prediction accuracy and mitigate their weaknesses.
On one hand, applying the ML model to signal-processing-generated HRs allowed us to sample PPG signals at only 25 Hz while achieving higher accuracy because ML models can be trained specifically to improve accuracy and remove MA with low-frequency samples. On the other hand, applying signal processing before ML eliminates the need for the ML model to directly take large numbers of PPG signals as inputs, reducing feature sizes and model sizes. 
%We use the signal processed PPG HR as ML features for HR prediction to relax the high-frequency requirement - with PPG sampling rate at 25 Hz, reduce the complexity of signal processing algorithm, and in the meantime, greatly reduce the ML feature dimension to facilitate smaller ML model size.

To address the ML model type issue of the second limitation, we compare the accuracy, model size, and inference time of five different ML models, including Decision Tree (DT), Random Forest (RF), K-nearest neighbor (KNN), Support Vector Machines (SVM) and Multi-layer Perceptron (MLP), and provide insights on which are more suitable for the resource-constrained device. Moreover, because existing data sets are usually too short for extensive evaluation, we also collected a new data set for HR study which is long enough for HR ML model training and testing with different features.

Our experimental evaluations show that the system can achieve less than 5\% mean average prediction errors (MAPE) for HR estimation with a PPG sampling rate of only 25 Hz. Moreover, our exploration results show that Decision Tree (DT) models usually could provide accurate estimation with a smaller model size of about 10 KB and a shorter inference time of less than 3 microseconds ($\mu$s).

The contributions of this paper include:
% In this work, we combine signal processing and ML - using signal processed PPG HR as ML features for HR estimation, to reduce the feature dimension and contribute to smaller model size.
% Through combining signal processing with ML, we explore the advantage of both of them to provide high estimation accuracy and relieve their disadvantage of them, the high energy consumption for signal processing and the high memory requirement of ML.

% \eat{
%The main challenge of the system is to design a suitable ML model that can predict HR accurately from the noisy raw PPG data in a real-time manner.
%Therefore, there are two major components in the system, one is a PPG sensor component that collects raw PPG data continuously, and another is an edge device that focuses on processing the data and provides accurate HR in real-time.
% The PPG sensor is mounted on a fingertip and connected to the edge device. 
% We also deployed an ECG device to obtain HR as the ground truth.
%On the edge device, 4 modules are cooperating to provide accurate and real-time readings: firstly, a data collection module to collect the PPG reading, then a data preprocessing module to reduce noises, after that, an ML module to estimate HR, and finally, a report module to report HR to users, or save them to cloud for further analysis. 

%The key modules in terms of HR accuracy are the data preprocessing module and ML module.
%The less noise contained in the data entering the ML model, the better the model estimation result is.
%Therefore, for the data preprocessing module, we need to design an efficient preprocessing algorithm to reduce noises and generate better input for the ML model.
%As for the ML module, we need to find an efficient ML algorithm to accurately estimate HR.
%There are existing works that utilize ML in HR monitoring, but this work has a few differences from them.
% Firstly, to the best of our knowledge, all existing works apply PPG light signals as ML features in HR estimation, while in this work, we use processed PPG HR as ML features for HR estimation.
% The advantage of doing this is that we can apply our system to further improve commercial device results. 
% Since most of the commercial devices do not provide PPG light signals to users, the existing methods are not able to be applied to improve them.
% Secondly, for HR, in addition to our collected data, we also used the ISPC dataset, and for the ISPC dataset, we combine both signal processing and ML in HR estimation, while other work either uses a signal processing method only or uses a simple preprocess before sending the PPG data to the learning model.

% Another challenge of the system is to keep the system small and reliable to make sure that it can run smoothly with few resources.
% The edge device could be any smart device such as a smartphone or smart home gateway. 
% Since the resources on the edge device are limited, the ML model should not consume too many resources. 
% }

 
\begin{itemize}
\item Novel HR estimation methodologies that combined both signal processing and ML models, allowing high accuracy HR estimations with low-frequency PPG signals, fewer ML features, and smaller ML models. 
%We are the first that combines signal processing with ML, using PPG HR as ML features instead of PPG light signal, which achieves higher accuracy for HR estimation. This combination also allowed
%, MAPE less than 5\% for HR, and contributes to a
%smaller feature dimension, which leads to smaller ML models. 

\item A systematic analysis of different ML model types and feature sizes to study their impact on the accuracy in HR estimation. This analysis showed that all considered ML models can provide about 5\% MAPE for both the ISPC dataset and our collected dataset. 
%when features dimension is around 20 for HR.
% We compared several machine learning models, including DT, RF, KNN, SVM, and MLP for accurate real-time HR monitoring. We evaluate their estimation accuracy and memory consumption in HR estimation.

\item Comprehensive evaluations of the model size and inference time for different ML model types and configurations to study their suitability for the resource-constrained HR monitoring environment. This analysis found that DT can provide a good balance between accuracy, model size, and inference time.
% \item A comprehensive dataset to study HR estimation with PPG signals, which contain different levels of activity intensity with each last for more than 2 hours. 
%We devise a real-time HR monitoring system powered by PPG sensors and ML, and collect data in different settings.
\end{itemize}

The rest of the paper is structured as follows.
Section~\ref{sec:related_work} discusses the related work. 
The methodology and implementation are detailed in Section~\ref{sec:method}. 
Evaluation results are presented in Section~\ref{sec:Evaluation} and the conclusions are drawn in Section~\ref{sec:Conclusion}.
