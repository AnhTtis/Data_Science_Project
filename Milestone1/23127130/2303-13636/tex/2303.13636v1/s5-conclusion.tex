\section{Conclusion}
\label{sec:Conclusion}

% The results from our preliminary study based on the limited PPG and ECG data validate the feasibility of the proposed system, which can accurately obtain HR based on PPG sensor data in real-time powered by ML and signal processing. 
% In the experiment, we compared the estimation accuracy and model size for different ML algorithms and find DT, RF, and MLP are more suitable than SVM and KNN. 
% For HR estimation, we can achieve MAPE lower than 5\%.

While Photoplethysmography (PPG) sensors could provide accurate HR estimations, applying these sensors to embedded devices is still challenging due to the high-frequency PPG sampling, which has high power consumption, and the complex machine-learning models, which are too computational-intensive and large for small embedded devices. 
%In this work, we aim to explore HR estimation techniques that are more suitable for lower-power and resource-constrained embedded devices. More specifically, we seek to design techniques that could provide high-accuracy HR estimation with low-frequency PPG sampling, small model size, and fast inference time. 
In this paper, we showed that combining signal processing and ML could significantly reduce PPG sampling frequency while providing high HR estimation accuracy. The experimental results showed that our method that combined signal processing and ML had only $5\%$ error for HR estimation using  low-frequency PPG data. This combination also reduces the ML model's feature size, leading to smaller models.
Additionally, we also conducted a comprehensive analysis of different ML models and feature sizes to compare their accuracy, model size, and inference time. Our analysis showed that DT models with 10 to 20 features usually have good accuracy while being several magnitude smaller in model sizes and faster in inference time. 



% In the future, we plan to collect more data from different subjects to further validate the results.
% To further improve the monitor system accuracy, we plan to add an accelerometer to help diminish MA.
% Furthermore, to further validate and investigate the efficiency of the monitoring system, we will deploy the system on resource-limited microcontroller-based hardware.
