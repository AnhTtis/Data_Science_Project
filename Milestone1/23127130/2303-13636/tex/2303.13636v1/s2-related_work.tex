\section{Background and Closely Related Work}
\label{sec:related_work}

In this section, we present the background and closely related work on HR estimation.
% The traditional way to obtain HR in a clinic is using an electrocardiogram (ECG), hence we first illustrate how to estimate HR from ECG.
% Then, since ECG is difficult to implement in daily life, and a great alternative for ECG is PPG, we also explore the conventional signal processing method to calculate HR from PPG signal.
% Lastly, we present some popular ML regression algorithms that can be utilized for HR estimation, and introduce some existing works that employ ML to estimate HR.

\subsection{HR from ECG}

The traditional medical device to measure HR is ECG.
ECG records heart activity utilizing electrodes placed at certain skin spots on the human body and produces an electrocardiogram, which is a graph that shows the heart's electrical activity over time.
%By analyzing electrocardiograms, doctors can measure how the heart is functioning. 
An electrocardiogram contains the QRS complexes information, which is the most important waveform in an electrocardiogram that shows the spread of a stimulus through the ventricles~\cite{dohare2014efficient,GOLDBERGER201811}. 
RR intervals can be derived from the QRS complex, which in turn, gives HR. 
More specifically, because the RR interval is the interval between heartbeats, the reciprocal of the RR interval is the HR~\cite{LANFRANCHI2011226}.
Although ECG can produce accurate HR, attaching electrodes to the human body makes it inconvenient to use.
% ECG can export the RR intervals, which is the elapsed time between two consecutive R waves of the QRS complex.
% And then, we can adopt the RR intervals to calculate HR.
%\subsubsection{HR from ECG}
%Usually, ECG devices will come with software that processes the electrocardiogram and exports the RR intervals. 
% Therefore, we can easily calculate the HR based on the reported RR intervals obtained. 
%ECG can export RR intervals accurately. 
%Consequently, in this work, we apply the HR generated from ECG as labels for HR learning models.


\subsection{HR from PPG - Signal Processing}
% Since ECG needs to be connected by wires to electrodes that are attached to the body, it is inconvenient to install and operate in daily life. 
% On the contrary, PPG is easy to install, and that is one of the reasons why PPG is a great substitute for ECG.
% Another reason that PPG is more suitable than ECG for the HR monitor system is that PPG is already widely integrated into many wearable devices such as chest straps, smart rings, and smartwatches.

Due to their convenient usage and small sizes, Photoplethysmography (PPG) sensors have become a popular replacement for ECG in HR monitoring.
PPG uses light signals to monitor blood flow.
Heartbeats cause periodical changes in the blood flow, which cause periodical changes in the reflected light received by the PPG sensor.
Hence, periodical PPG light wave changes can be exploited to derive the heartbeat.
% There are two types of PPG, reflection, and transmission, the difference is whether the light source and the detector are on the same side or on different sides.

The main issue with PPG sensors is the noise in the signal, which is usually the result of motion artifacts (MA).
% The common method to remove MA to calculate HR is signal processing, which is discussed in the following paragraphs. %In this section, some existing works on signal processing methods to obtain HR from PPG are discussed. 
The common method to remove MA and calculate HR is signal processing, which typically tracks the peaks of the PPG signals. Zhang et al. proposed an algorithm that consists of signal decomposition, sparse signal reconstruction, and spectrum peak tracking to extract HR from PPG signals in intensive physical activities environment~\cite{zhang2014troika}.
Their dataset, denoted as the IEEE Signal Processing Cup (ISPC) 2015 dataset, has been widely utilized for evaluating HR monitoring solutions. 
It includes PPG sensor data, accelerometer sensor data, and ECG data. 
%Different from our work, 
In this ISPC dataset, the PPG sensor is installed on the wrist and the ground truth is a one-channel ECG.
In contrast, we applied PPG on fingertips and utilized a three-lead ECG to get ground truth HR. Moreover, each ISPC data recording lasts for only a few minutes, whereas each trace in our data lasts for about two hours.

\subsection{HR from PPG - Machine Learning}

% In this section, we introduce existing works that use ML to estimate HR.
Recently, ML has been found to be a promising method to remove MA from PPG signals and estimate HR.
Prior studies utilized various ML algorithms to estimate HR. 
For example, Bashar et al. employed K-means clustering and Random Forest (RF) for HR estimation~\cite{bashar2019machine}.
The raw PPG signal was pre-processed with a 2nd order bandpass filter.
Then the K-means clustering algorithm was used to identify noisy data and Random Forest regression was used to predict HR based on PPG and acceleration data.
They used features extracted from PPG signals and also examined adding features from accelerometers.
%a different set of features for the ML algorithms such as PPG signal with or without acceleration signal.
Puranik and Morales estimated HR in real-time with an adaptive Neural Network filter and a post-processing smoothing and median filter~\cite{puranik2019heart}. 
Chang et al. came up with DeepHeart, an HR estimation approach that combines deep learning and spectrum analysis~\cite{chang2021deepheart}.
The raw PPG signal was pre-processed by a three-order Butterworth bandpass filter. 
Then the PPG signal was sent to an ensemble model to remove noise.
The ensemble model contains several deep learning models with convolutional layers.
%They evaluated the proposed method with leave-one-record-out cross-validation using the ISPC dataset.
Biswas et al. proposed CorNET, a convolutional neural network with long short-term memory (LSTM) to estimate HR and perform biometric identification based on PPG signals~\cite{biswas2019cornet}.
The raw PPG signal was pre-processed with a band-pass 4th order Butterworth filter and a normalizer.
This method was estimated with leave-one-window-out validation on the ISPC dataset.
%They also evaluated the proposed method with 2 subjects and achieves an MAE of 5.93±4.63 and 8.54±7.63.
Rocha et al. designed Binary CorNET, a binarized CorNET to estimate HR 
%with resources-constrained hardware
~\cite{rocha2020binary}.
%The Binary CorNET achieved a small circuit footprint and low clock frequency.
All the above works used the ISPC dataset in their evaluation, which shows the popularity of the ISPC dataset.
However, as pointed out in \cite{reiss2018ppg}, the ISPC dataset is insufficient for deep learning approaches since the available data for different activities is short and the total number of samples is limited. The ISPC dataset also used a high (125 Hz) sampling frequency, which incurs high energy usage.

Reiss et al. presented a CNN architecture for PPG-based HR estimation \cite{reiss2018ppg}. They first preprocessed the PPG sensor data with FFT and z-normalization, and then train and evaluate the CNN model using leave-one-session-out (LOSO) cross-validation with the ISPC dataset.
In the experiment, they compared the proposed method with three classical signal processing methods and concluded that the performance of the CNN-based HR estimation is comparable with classical methods.
Panwar et al. designed a convolutional neural network with LSTM to estimate HR and blood pressure based on PPG signals \cite{panwar2020pp}. They evaluated the model on the MIMIC dataset \cite{saeed2011multiparameter} and showed the effectiveness of the neural model. 
The MIMIC dataset is collected from patients, while in this work, we focus on HR for healthy subjects.

Unlike prior studies that applied signal processing only for data pre-processing, our solution used signal processing first to generate a set of HR estimations, which are then further processed by ML models to improve estimation accuracy. By combining signal processing based and ML-based HR estimation, we can reduce both PPG sampling frequency and ML feature size while retaining high accuracy.
