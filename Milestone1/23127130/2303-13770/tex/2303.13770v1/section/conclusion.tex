In this paper, we conduct a large-scale empirical study to investigate the performance of existing works on reentrancy detection for smart contracts. We collect 230,548 contracts from Etherscan and select five well-known or recent tools that can locate the possibly defective functions with reentrancy issues in a contract. By manually examining 21,212 reentrant contracts detected using these tools, we obtain 34 true positive contracts with reentrancy and 21,178 false positive contracts without reentrancy. We also analyze the causes of the true and false positives. Using the two sets of contracts with different causes, we evaluate the tools. Two additional tests are conducted to evaluate the tools on a number of contracts with recent reentrancy attacks and to verify whether the detected reentrancy issues can be identified by the Ethereum's official IDE, Remix. The results show that the tools have a extremely high false positive rate of more than 99.8\% and that the tools can only detect reentrancy related to \emph{call.value()}, 58.8\% of which can be detected by Remix. Based on the results, we suggest that researchers turn to discovering and detecting new reentrancy issues in spite of the classical and simple patterns related to \emph{call.value()}. In future work, we plan to improve existing reentrancy detection tools by reducing false positives based on our summarized causes, e.g., adding rule-based filters to the tools by automating some of the patterns described in Section~\ref{subsec: FPofTools}. Moreover, we will study the reentrancy issues reported in recent attacked contracts.