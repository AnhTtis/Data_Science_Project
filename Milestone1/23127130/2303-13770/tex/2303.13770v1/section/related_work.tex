\subsection{Reentrancy Detection Tools}
Reentrancy is one of the most notorious vulnerabilities, first discovered in 2016~\cite{luu2016making}, and continually discussed since. Due to the significance of this security issue, many detection tools have been proposed to prevent smart contracts from being attacked, including dynamic testing tools and static analyzers. For example, ContractFuzzer~\cite{jiang2018contractfuzzer}, sFuzz~\cite{nguyen2020sfuzz}, and Smartian~\cite{choi2021smartian} use dynamic testing~\cite{hanson1993testing} technologies to trigger reentrancy issues. ContractFuzzer is a fuzzing tool, which develops test oracles~\cite{baresi2001test} for the reentrancy vulnerability. Due to its inefficiency, Nguyen et al. propose an adaptive strategy in sFuzz to guide fuzzer toward executing unreached paths. However, these tools pay little attention to the characteristics of smart contracts. Choi et al. claim that some paths can only be reached by critical transaction sequences and thus propose Smartian. 

Oyente~\cite{luu2016making}, Securify~\cite{tsankov2018securify}, Slither~\cite{feist2019slither}, and Zeus~\cite{kalra2018zeus} use static analysis technologies to discover reentrancy vulnerability. In 2016, Luu et al. propose Oyente which addresses the problem of reentrancy detection. Oyente constructs the control flow graph of contracts and identifies reentrancy issues by risky patterns~\cite{luu2016making}. Due to the lack of context information, Oyente is imprecise. Tsankov et al. propose Securify to extract semantic information from smart contracts and perform analysis to check the existence of predefined patterns. Furthermore, Feist et al. propose Slither to perform automatic analysis~\cite{feist2019slither}. Bose et al. put forward Slither to handle state-inconsistency bugs (e.g., reentrancy) and achieve a better performance~\cite{kalra2018zeus}.

\subsection{Benchmark in Smart Contracts}
A benchmark is a foundation for comparing different methods~\cite{perazzi2016benchmark}. However, only a few benchmarks have been proposed in the area of smart contracts. \textit{VeriSmartBench}~\cite{verismartbench} is one of these benchmarks, which contains examples for all CVE cases. Unfortunately, this repository contains very few examples of reentrancy issues. \textit{SolidiFi-benchmark}~\cite{durieux2020empirical} is another benchmark which contains 31 examples with reentrancy issues, but many of them are artificial and toy examples.

Efforts have been made in other empirical studies to find different aspects of insight. Durieux et al.~\cite{durieux2020empirical} conduct a large-scale evaluation of automated analysis tools. Rather than focusing on specific vulnerabilities, they pay close attention to the effectiveness and efficiency of distinct tools. In addition, Perez et al.~\cite{perez2021smart} focus on evaluating whether vulnerable contracts have been exploited. They reveal that only a small proportion of smart contract vulnerabilities have been profited from. Xue et al.~\cite{xue2020cross} reveal five \textit{path protective techniques} which are relevant with the false alarm of reentrancy detection. Yet, these approaches only cover a portion of Solidity versions and cannot be used to draw a comprehensive conclusion.