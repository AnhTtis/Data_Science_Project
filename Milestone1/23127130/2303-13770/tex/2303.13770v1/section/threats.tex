\textbf{Threats to internal validity} relate to the errors in the implementation of the reentrancy detection tools and the bias of participants in the manual examination of the reentrant contracts detected by the tools and the cause analysis of true (\emph{resp.} false) positive contracts with (\emph{resp.} without) reentrancy. We implement the tools using their Docker images or Dockerfiles released at GitHub. For the manual examination task, we perform two rounds to reduce the heavy workload of 21,212 contracts. In the first round, we recruit 48 participants willing to participate in our task and ensure that they have relatively sufficient knowledge of Solidity and reentrancy to perform the task, adopting a four-stage process (Section~\ref{par_rec}). In the second round, we ask two PhDs with experience in Solidity and reentrancy to review the examination results of the first round and re-examine the contracts with low-quality results. For the cause analysis task, we ask both PhDs to collaboratively analyze the causes of true and false positive contracts. To avoid the subjective bias of a single person, the manual examination and cause analysis tasks for each contract are performed by two participants (including the two PhDs). It should be pointed out that the participants might miss some possible reentrancy issues in contracts that they do not know or have not been trained with.

\textbf{Threats to external validity} relate to the generalizability of the results. The objective of this study is to investigate the capability of existing works on reentrancy detection for smart contracts. We collect a large dataset of 230,548 contracts with Solidity code from Etherscan and select five well-known or recent reentrancy detection tools according to some key criteria for practical use (Section~\ref{tool_sel}). We exclude some popular tools as they either cannot be applied to Solidity code or cannot locate the possibly defective functions in contracts. We acknowledge that our results obtained using five tools cannot reflect all existing works on reentrancy detection. However, the results can help researchers gain a good understanding of the limitations of existing works and motivate researchers to address new reentrancy issues in spite of those related to \emph{call.value()}.