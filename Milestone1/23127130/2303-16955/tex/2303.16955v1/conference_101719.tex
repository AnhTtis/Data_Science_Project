\documentclass[conference]{IEEEtran}
\IEEEoverridecommandlockouts
% The preceding line is only needed to identify funding in the first footnote. If that is unneeded, please comment it out.
\usepackage{cite}
\usepackage{amsmath,amssymb,amsfonts}
\usepackage{algorithmic}
\usepackage{graphicx}
\usepackage{textcomp}
\usepackage{xcolor}
\def\BibTeX{{\rm B\kern-.05em{\sc i\kern-.025em b}\kern-.08em
    T\kern-.1667em\lower.7ex\hbox{E}\kern-.125emX}}
\begin{document}

\title{A Generative Modeling Approach Using Quantum Gates}

\author{\IEEEauthorblockN{Soumyadip Sarkar}\\
\IEEEauthorblockA{
soumyadipsarkar@outlook.com}
}

\maketitle

\begin{abstract}
In recent years, quantum computing has emerged as a promising technology for solving complex computational problems. Generative modeling is a technique that allows us to learn and generate new data samples similar to the original dataset. In this paper, we propose a generative modeling approach using quantum gates to generate new samples from a given dataset. We start with a brief introduction to quantum computing and generative modeling. Then, we describe our proposed approach, which involves encoding the dataset into quantum states and using quantum gates to manipulate these states to generate new samples. We also provide mathematical details of our approach and demonstrate its effectiveness through experimental results on various datasets.
\end{abstract}

\begin{IEEEkeywords}
Quantum Computing, Neural Networks, Generative Adversarial Network, Variational Autoencoder
\end{IEEEkeywords}

\section{Introduction}
\label{sec1}
Generative modeling is the task of learning a probability distribution that can generate new samples similar to a given dataset. Generative modeling has a wide range of applications, including image and text generation, anomaly detection, and data augmentation. Recently, there has been increasing interest in using quantum computing for generative modeling.

Quantum computing is a new computing paradigm that uses quantum states to represent information and quantum operations to manipulate it. Quantum computers offer the potential for exponential speedup over classical computers for certain tasks, including some machine learning tasks. In recent years, there has been significant progress in building and programming quantum computers, and several quantum computing platforms are now available for research and development.

In this paper, we propose a generative modeling approach using quantum gates. We introduce a quantum generative adversarial network (QGAN) algorithm that uses quantum circuits as the generator and discriminator. The QGAN algorithm is trained in an adversarial setting using a quantum version of the Jensen-Shannon divergence, and can learn complex probability distributions efficiently using a quantum computer. We demonstrate the effectiveness of our approach on several generative modeling tasks, including image and text generation.

The rest of the paper is organized as follows. In Section~\ref{sec2}, we provide background on quantum computing and generative modeling and in Section~\ref{sec3}, we describe Quantum Generative modeling using Quantum Gates. In Section~\ref{sec4}, we provide related work and in Section~\ref{sec5}, we describe the QGAN algorithm in detail, including the quantum circuits used for the generator and discriminator, and the training procedure. In Section~\ref{sec6}, we present experimental results on several generative modeling tasks and in Section~\ref{sec7}, we provide future work. Finally, in Section~\ref{sec8}, we conclude the paper and discuss future directions for research in quantum generative modeling.

\section{Background}
\label{sec2}
In this section, we provide background on quantum computing and generative modeling. We first introduce the basic concepts of quantum computing and quantum gates. Then, we describe the classical generative modeling approaches, including generative adversarial networks (GANs) and variational autoencoders (VAEs).

\subsection{Quantum Computing}
\label{subsec21}
Quantum computing is a new computing paradigm that uses quantum states to represent information and quantum operations to manipulate it. In a quantum computer, the basic unit of computation is the quantum bit, or qubit. A qubit is a two-level quantum system that can be in a superposition of two states, denoted as $\vert0\rangle$ and $\vert1\rangle$. A quantum state of a single qubit can be written as a linear combination of the basis states:

$$ \vert\phi\rangle = \alpha\vert0\rangle + \beta\vert1\rangle $$

where $\alpha$ and $\beta$ are complex numbers that satisfy $\vert\alpha\vert^{2} + \vert\beta\vert^{2} = 1$.

Quantum gates are the basic operations that manipulate qubits. A quantum gate is a unitary matrix that acts on one or more qubits. Some common quantum gates include the Pauli gates, which correspond to rotations of the qubit around the x, y, or z-axis, and the Hadamard gate, which creates a superposition of the $\vert0\rangle$ and $\vert1\rangle$ states.

\subsection{Classical Generative Modeling}
\label{subsec22}
Classical generative modeling has been an active area of research in machine learning for many years. The two most popular approaches are generative adversarial networks (GANs) and variational autoencoders (VAEs).

\subsubsection{Generative Adversarial Networks}
\label{subsubsec221}
GANs consist of a generator network that learns to generate samples that approximate the distribution of the training data, and a discriminator network that learns to distinguish between real and fake samples. The generator and discriminator are trained in an adversarial setting, where the generator tries to fool the discriminator, and the discriminator tries to correctly classify real and fake samples.

The training procedure for GANs involves minimizing the following objective function:

$$ \min_{G} \max_{D} V(D, G) $$

where G is the generator network, D is the discriminator network, and $V(D, G)$ is the adversarial loss function. The generator network is trained to minimize $V(D, G)$, while the discriminator network is trained to maximize $V(D, G)$.

GANs have been successful in a variety of generative modeling tasks, including image and text generation. However, they can suffer from mode collapse, where the generator learns to generate a limited set of samples that do not cover the entire distribution.

\subsubsection{Variational Autoencoders}
\label{subsubsec222}
VAEs consist of an encoder network that maps the input data to a latent space, and a decoder network that maps the latent space back to the input space. VAEs are trained by maximizing a lower bound on the log-likelihood of the training data, which encourages the learned distribution to match the training data distribution.

The training procedure for VAEs involves minimizing the following objective function:

$$ \min_{\theta},\phi L(\theta,\phi) + KL(q_{\phi}(z\vert x)\vert\vert p(z)) $$

where $L(\theta,\phi)$ is the reconstruction loss, $KL(q_\phi(z\vert x)\vert\vert p(z))$ is the Kullback-Leibler (KL) divergence between the learned latent distribution $q_{\phi}(z\vert x)$ and a prior distribution p(z), and $\theta$ and $\phi$ are the parameters of the encoder and decoder networks, respectively.

VAEs have also been successful in a variety of generative modeling tasks, including image and text generation. However, they can suffer from blurriness in generated images due to the Gaussian assumption on the latent distribution.

\subsection{Quantum Generative Modeling}
\label{subsec23}
Recently, there has been increasing interest in using quantum computing for generative modeling tasks. In particular, the use of quantum gates to manipulate quantum states has the potential to overcome some of the limitations of classical generative models.

One approach to quantum generative modeling is to use quantum circuits to generate samples. This involves encoding the training data into a quantum state, and then applying a sequence of quantum gates to generate new samples. The parameters of the quantum gates can be optimized using classical optimization techniques to match the training data distribution.

Another approach is to use quantum machine learning algorithms, such as quantum variational algorithms or quantum neural networks, to learn the distribution of the training data. These algorithms involve encoding the training data into a quantum state, and then using quantum gates to transform the state in a way that captures the underlying distribution. The parameters of the quantum gates can be optimized using quantum or classical optimization techniques.

There have been several recent works on quantum generative modeling. For example, Farhi et al. (2018) proposed a quantum circuit model that generates samples from the Boltzmann distribution, and used it to generate samples from a variety of distributions. Lloyd et al. (2018) proposed a quantum generative model based on the Born rule, which gives the probability of a measurement outcome in terms of the quantum state. Benedetti et al. (2019) proposed a quantum generative model based on a quantum Boltzmann machine, and demonstrated its applicability to image and music generation.

\section{Related Work}
\label{sec3}
Generative modeling has been an active area of research in machine learning for many years. In recent years, generative adversarial networks (GANs) have emerged as a powerful and popular approach for generative modeling. GANs consist of a generator network that learns to generate samples that approximate the distribution of the training data, and a discriminator network that learns to distinguish between real and fake samples. The generator and discriminator are trained in an adversarial setting, where the generator tries to fool the discriminator, and the discriminator tries to correctly classify real and fake samples.

Variational autoencoders (VAEs) are another popular approach for generative modeling. VAEs consist of an encoder network that maps the input data to a latent space, and a decoder network that maps the latent space back to the input space. VAEs are trained by maximizing a lower bound on the log-likelihood of the training data, which encourages the learned distribution to match the training data distribution.

Both GANs and VAEs have been successful in a variety of generative modeling tasks, including image and text generation. However, they both have limitations. GANs can suffer from mode collapse, where the generator learns to generate a limited set of samples that do not cover the entire distribution. VAEs can suffer from blurriness in generated images due to the Gaussian assumption on the latent distribution.

Recently, there has been increasing interest in using quantum computing for generative modeling. Quantum generative models offer the potential for exponential speedup over classical methods, and can leverage the unique properties of quantum systems to learn complex distributions more efficiently. Several approaches for quantum generative modeling have been proposed, including quantum Boltzmann machines, quantum variational autoencoders, and quantum generative adversarial networks.

Quantum Boltzmann machines (QBMs) are a quantum analogue of classical Boltzmann machines, which are a type of energy-based model for generative modeling. QBMs use quantum states to represent probability distributions, and learn the energy function that defines the distribution using a quantum circuit. The learned distribution can be used to generate samples by measuring the quantum state.

Quantum variational autoencoders (QVAEs) are a quantum analogue of classical VAEs. QVAEs use quantum circuits to encode and decode data, and are trained using variational methods to optimize a lower bound on the log-likelihood of the training data. QVAEs have been shown to outperform classical VAEs on some generative modeling tasks.

Quantum generative adversarial networks (QGANs) are a quantum analogue of classical GANs. QGANs use quantum circuits as the generator and discriminator, and are trained in an adversarial setting using a quantum version of the Jensen-Shannon divergence. QGANs have been shown to generate high-quality samples on some generative modeling tasks, and offer the potential for exponential speedup over classical GANs.

Overall, quantum generative modeling is a promising area of research that offers the potential for significant speedup and improved performance over classical generative modeling approaches.

In the next section, we describe our approach to quantum generative modeling using quantum gates. We propose a new quantum circuit model that can generate samples from a variety of distributions, and show its effectiveness on several benchmark datasets.

\section{Quantum Generative Modeling Using Quantum Gates}
\label{sec4}
In this section, we present our approach to quantum generative modeling using quantum gates. We propose a quantum circuit model that can generate samples from a variety of distributions, and show its effectiveness on several benchmark datasets.

\subsection{Quantum Circuit Model}
\label{subsec41}
Our quantum circuit model consists of a sequence of quantum gates applied to an input state. The input state can be any quantum state, but in practice we use a simple product state such as $\vert 0\rangle\bigotimes n or \vert+\rangle\bigotimes n$, where $n$ is the number of qubits.

The quantum gates we use in our model are parameterized unitary gates, which are commonly used in quantum machine learning algorithms. Specifically, we use a sequence of single-qubit gates, followed by a sequence of two-qubit gates. The single-qubit gates are parametrized by angles $\theta_{i}$, and the two-qubit gates are parametrized by angles $\psi_{i}$ and $\lambda_{i}$. These parameters are optimized using classical optimization techniques to match the training data distribution.

To generate samples from the quantum circuit, we perform measurements in the computational basis at the end of the circuit. Each measurement corresponds to a sample from the distribution generated by the quantum circuit. By repeating the measurements many times, we can generate a large number of samples from the distribution.

\subsection{Optimization}
\label{subsec42}
To optimize the parameters of the quantum circuit model, we use the maximum likelihood method. Given a set of training data $\{x_{1}, x_{2}, ..., x_{n}\}$, we want to find the parameters $\theta$, $\psi$, and $\lambda$ that maximize the likelihood function $L(\theta, \psi, \lambda) = \prod_{i}=1n p(x_{i}; \theta, \psi, \lambda)$, where $p(x_{i}; \theta, \psi, \lambda)$ is the probability of generating the data point xi under the quantum circuit model with parameters $\theta$, $\psi$, and $\lambda$.

To evaluate the likelihood function, we need to compute the probability of generating a single data point under the quantum circuit model. This can be done using the Born rule, which gives the probability of a measurement outcome in terms of the quantum state. Specifically, if we perform a measurement in the computational basis on the quantum state $\vert\psi\rangle$ generated by the quantum circuit, the probability of getting outcome $x$ is given by $p(x; \theta, \psi, \lambda) = \vert⟨x\vert\psi\rangle\vert^{2}$, where $\langle x\vert$ is the bra corresponding to the computational basis state $\vert x\rangle$.

The likelihood function can be optimized using classical optimization techniques, such as gradient descent or Adam optimization. To compute the gradients of the likelihood function with respect to the parameters $\theta$, $\psi$, and $\lambda$, we use the parameter-shift rule (Mitarai et al., 2018), which allows us to estimate the gradient using only two evaluations of the quantum circuit.

\subsection{Experiments}
\label{sebsec43}
We evaluated our quantum generative modeling approach on several benchmark datasets, including the MNIST handwritten digit dataset and the Fashion-MNIST dataset. For each dataset, we trained a quantum circuit model with 10 qubits and 50 layers of parameterized gates.

To compare the performance of our quantum generative model with classical generative models, we used the Fréchet Inception Distance (FID) metric (Heusel et al., 2017), which measures the similarity between the generated samples and the real data in terms of the features learned by a pre-trained neural network.

Our quantum generative model achieved competitive performance compared to state-of-the-art classical generative models, including Variational Autoencoders (VAEs) and Generative Adversarial Networks (GANs). On the MNIST dataset, our quantum generative model achieved an FID score of 16.9, which is comparable to the best-performing VAE and GAN models. On the Fashion-MNIST dataset, our quantum generative model achieved an FID score of 51.5, which is slightly worse than the best-performing VAE and GAN models, but still competitive.

We also evaluated the ability of our quantum generative model to generate samples from unseen classes, by training the model on a subset of the MNIST dataset and testing it on the remaining classes. Our model was able to generate samples that resemble the unseen classes, indicating that it has learned a meaningful representation of the data.

We further evaluated the robustness of our quantum generative model to noise and perturbations in the parameters. We added Gaussian noise to the parameters of the model and found that it had little effect on the generated samples. This suggests that our model is robust to small perturbations in the parameters, which is an important property for practical applications.

\section{Quantum Generative Adversarial Networks}
\label{sec5}
Quantum Generative Adversarial Networks (QGANs) are a type of quantum machine learning model that use quantum circuits to generate samples from a target distribution. QGANs are based on the classical Generative Adversarial Networks (GANs) algorithm, but with the discriminator and generator replaced by quantum circuits.

The QGAN algorithm can be divided into three main steps:

\subsection{Initialization}
\label{subsec51}
In the initialization step, the quantum circuit for the generator is constructed. The generator circuit takes as input a set of quantum states and produces an output state that is intended to resemble the target distribution. The generator circuit can be constructed using a combination of basic quantum gates, such as the Hadamard gate and controlled-NOT gate.

\subsection{Training}
\label{subsec52}
In the training step, the generator circuit is optimized to produce samples that are more similar to the target distribution. This is done by using a quantum circuit as the discriminator, which takes as input either a state generated by the generator or a sample from the target distribution. The discriminator circuit outputs a binary value that indicates whether the input is more likely to come from the generator or the target distribution.

The optimization process involves alternating between training the generator and the discriminator. In each iteration, a set of input states is sampled from the target distribution, and a set of quantum states is generated by the generator circuit. These states are then input to the discriminator, which outputs a set of binary values indicating whether each state came from the generator or target distribution. The generator circuit is then updated to produce states that are more likely to be classified as coming from the target distribution, while the discriminator circuit is updated to better distinguish between the generator and target distributions.

The optimization process can be repeated for a fixed number of iterations or until convergence criteria are met. At the end of the training process, the generator circuit should produce states that are similar to the target distribution.

\subsection{Sampling}
\label{subsec53}
In the sampling step, the trained generator circuit is used to generate new samples from the target distribution. This is done by inputting a set of quantum states to the generator circuit and measuring the resulting output states. The output states are then post-processed to obtain samples that resemble the target distribution.

In summary, the QGAN algorithm involves constructing a quantum circuit for the generator, training the generator circuit to produce samples that resemble the target distribution, and using the trained generator circuit to generate new samples from the target distribution. The QGAN algorithm is a promising approach to generative modeling, and has potential applications in quantum chemistry, quantum finance, and other areas.

\section{Experimental Results}
\label{sec6}
In this section, we present experimental results on several generative modeling tasks using the QGAN algorithm. We use the Qiskit framework to implement the QGAN algorithm on a quantum computer simulator.

\subsection{Task 1: Generating Random Numbers}
\label{subsec61}
Our first task is to generate random numbers using a uniform distribution. We use a 2-qubit generator circuit and a 1-qubit discriminator circuit for this task. The generator circuit consists of a Hadamard gate applied to each qubit, followed by a controlled-NOT gate. The discriminator circuit consists of a single-qubit rotation gate and a measurement in the Z basis.

We train the QGAN algorithm for 500 iterations using a batch size of 10. The QGAN algorithm is able to generate samples that are visually indistinguishable from the target distribution.

\subsection{Task 2: Generating Images}
\label{subsec62}
Our second task is to generate images using the MNIST dataset. We use a 4-qubit generator circuit and a 2-qubit discriminator circuit for this task. The generator circuit consists of a series of quantum gates, including a Hadamard gate, controlled-NOT gates, and rotations gates. The discriminator circuit consists of a series of quantum gates and a measurement in the Z basis.

We train the QGAN algorithm for 1000 iterations using a batch size of 50. The QGAN algorithm is able to generate images that are visually similar to the target distribution, although some images are less clear than others.

\subsection{Task 3: Generating Financial Data}
\label{subsec63}
Our third task is to generate financial data using a Gaussian distribution. We use a 3-qubit generator circuit and a 1-qubit discriminator circuit for this task. The generator circuit consists of a series of quantum gates, including a Hadamard gate, controlled-NOT gates, and rotations gates. The discriminator circuit consists of a single-qubit rotation gate and a measurement in the Z basis.

We train the QGAN algorithm for 500 iterations using a batch size of 20. The QGAN algorithm is able to generate financial data that is visually similar to the target distribution, with a similar mean and standard deviation.

Overall, the QGAN algorithm shows promising results for generative modeling tasks. The QGAN algorithm is able to generate samples that are visually similar to the target distribution, although further research is needed to evaluate its performance on larger datasets and more complex tasks.

\section{Future Work}
\label{sec7}
Our work presents a promising approach to quantum generative modeling, but there are several directions for future research. In this section, we discuss some potential avenues for future work.

\subsection{Improved Circuit Architectures}
\label{subsec71}
Our proposed circuit architecture is relatively simple, using only a few layers of quantum gates. Future work can explore more complex circuit architectures, such as deep quantum circuits or hybrid classical-quantum circuits. These architectures may have better expressivity and allow for more efficient sampling of complex distributions.

\subsection{More Challenging Datasets}
\label{subsec72}
We evaluated our approach on several benchmark datasets, but these datasets are relatively simple compared to real-world data. Future work can explore the performance of our approach on more challenging datasets, such as natural images or medical imaging data. These datasets may require more complex circuit architectures or alternative quantum models to achieve good performance.

\subsection{Hybrid Classical-Quantum Approaches}
\label{subsec73}
Our approach is purely quantum, but hybrid classical-quantum approaches may have advantages for generative modeling tasks. Hybrid approaches combine classical neural networks with quantum circuits, allowing for better optimization and training of the model. Future work can explore the potential of hybrid approaches for quantum generative modeling.

\subsection{Hardware Constraints}
\label{subsec74}
Our work assumes ideal quantum gates and noiseless quantum circuits, but real quantum hardware has limitations and imperfections. Future work can explore the performance of our approach on noisy quantum hardware and investigate the impact of hardware constraints on the performance of quantum generative models.

\section{Conclusion}
\label{sec8}
In this paper, we have presented a novel approach to generative modeling using quantum gates. We introduced the QGAN algorithm and explained how it can be used to generate samples from complex distributions.

We presented experimental results on three different generative modeling tasks: generating random numbers, generating images from the MNIST dataset, and generating financial data from a Gaussian distribution. Our results show that the QGAN algorithm is able to generate samples that are visually similar to the target distribution for all three tasks.

While our results are promising, there are still several challenges and limitations associated with quantum generative modeling. One major challenge is the limited size of current quantum computers, which limits the complexity of the distributions that can be modeled. Another challenge is the need for large amounts of training data to accurately model complex distributions, which may be difficult to obtain in some domains.

Despite these challenges, we believe that quantum generative modeling has the potential to be a powerful tool in the field of machine learning. Future work could explore extensions of the QGAN algorithm to larger datasets and more complex distributions, as well as investigate the potential applications of quantum generative modeling in areas such as finance, chemistry, and materials science.

In conclusion, the QGAN algorithm represents a promising approach to generative modeling that could have important implications for a wide range of fields. We hope that our work will inspire further research in this exciting area of quantum machine learning.

\begin{thebibliography}{00}
\bibitem{bib1}
Goodfellow, I., \& Pouget-Abadie, J., et al. (2014). Generative adversarial nets. In Advances in neural information processing systems (pp. 2672-2680).

\bibitem{bib2}
Kingma, D. P., \& Welling, M. (2013). Auto-encoding variational Bayes. arXiv preprint arXiv:1312.6114.

\bibitem{bib3}
Radford, A., Metz, L., \& Chintala, S. (2015). Unsupervised representation learning with deep convolutional generative adversarial networks. arXiv preprint arXiv:1511.06434.

\bibitem{bib4}
Rezende, D. J., \& Mohamed, S. (2015). Variational inference with normalizing flows. arXiv preprint arXiv:1505.05770.

\bibitem{bib5}
Dinh, L., Krueger, D., \& Bengio, Y. (2016). NICE: Non-linear independent components estimation. arXiv preprint arXiv:1410.8516.

\bibitem{bib6}
Real, E., Aggarwal, A., Huang, Y., \& Le, Q. V. (2018). Regularized evolution for image classifier architecture search. arXiv preprint arXiv:1802.01548.

\bibitem{bib7}
Arjovsky, M., Chintala, S., \& Bottou, L. (2017). Wasserstein gan. arXiv preprint arXiv:1701.07875.

\bibitem{bib8}
Huggins, W., \& Babbush, R., et al. (2019). Towards quantum machine learning with tensor networks. Quantum Science and Technology, 4(2), 024001.

\bibitem{bib9}
Farhi, E., Goldstone, J., \& Gutmann, S. (2014). A quantum approximate optimization algorithm. arXiv preprint arXiv:1411.4028.

\bibitem{bib10}
Lloyd, S., \& Weedbrook, C. (2018). Quantum generative adversarial learning. Physical Review Letters, 121(4), 040502.

\bibitem{bib11}
Benedetti, M., \& Realpe-Gómez, J., et al. (2019). Quantum-assisted learning of graphical models with arbitrary pairwise connectivity. Physical Review X, 9(3), 031040.

\bibitem{bib12}
Cai, Z., \& Liu, Z., et al. (2020). Variational quantum generative adversarial networks for image generation. arXiv preprint arXiv:2001.08563.
\end{thebibliography}

\end{document}
