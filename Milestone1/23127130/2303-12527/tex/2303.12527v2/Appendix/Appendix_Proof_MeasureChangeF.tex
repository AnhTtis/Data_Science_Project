\subsection{Proof of  \Cref{prop:ZisMartingale}}\label{app:Proof_MeasureChangeF}
Inspired by \cite{Benth2019}, we prove 
that $\mathbb{E}_{{\mathbb{P}}}[Z^{\mathbb{P}\widetilde{\mathbb{Q}}}(\tau_1,\tau_1,\tau_2)]=1$  and expand their Theorem 3.5 to a geometric setting with stochastic volatility in order to address settings as in \cite{Kemper2022}.
For the scope of the proof, we consider the swap price $F$ from  \Cref{lem:SwapPGeom} characterized by  stochastic volatility of the form
\begin{align}
\sigma (t,\tau) \sqrt{\nu (t)}\;,
\end{align} 
where $\sigma(t,\tau)$ is deterministic and $\nu$ is the stochastic volatility that is modeled as a Cox-Ingersoll-Ross process evolving as
\begin{align}
d\nu (t) = \kappa_\nu \left(\theta_\nu -\nu (t)\right)dt + \sigma_\nu \sqrt{\nu (t)}dB^\mathbb{P}_t\;, \label{eq:CIR}
\end{align}
for $\nu(0)=\nu_0>0$, where $B^\mathbb{P}$ and $\widetilde{J}^{\mathbb{P}}$ are independent of each other and $B^\mathbb{P}$ and $W^{\mathbb{P}}$ are correlated.
In particular, we assume a correlation structure $d\langle W^{\mathbb{P}},B^{\mathbb{P}} \rangle_t = \rho dt$ where $\rho \in(-1,1)$ such that we can rewrite $B^{\mathbb{P}}=\rho W ^{\mathbb{P}}+\sqrt{1-\rho^2}\bar{B}^{\mathbb{P}}$ for $\bar{B} ^{\mathbb{P}}\independent W^{\mathbb{P}}$. 
Moreover, we assume that $\kappa _\nu,\theta_\nu ,\sigma_\nu>0$ satisfy the extended Feller condition, i.e., $\sigma_\nu^2<\kappa _\nu \theta_\nu $, to ensure that $\mathbb{E}_{\mathbb{P}} [\nu ^{-2}(t)]$ is bounded on the entire trading time $t\in[0,\tau_1]$ (see \cite{Dereich}, cf.\ Chapter 3).
Note, that the extended Feller implies the classical Feller condition (see \cite{KaratzasShreve1991}, cf.\ Chapter 5) ensuring that the volatility stays positive.\\

We proceed in the following steps:
\begin{enumerate}
\item Derivation of a new risk-neutral measure $\widetilde{\mathbb{Q}}^n$ through a stopping time $\hat{\tau}_n$.
\item Proof that $\mathbb{E}_{\widetilde{\mathbb{Q}}}[Z^{\mathbb{P}\widetilde{\mathbb{Q}}}(\tau_1,\tau_1,\tau_2)]$ is lower boundend, i.e., $$\mathbb{E}_{\widetilde{\mathbb{Q}}}\left[Z^{\mathbb{P}\widetilde{\mathbb{Q}}}(\tau_1,\tau_1,\tau_2)\right]\geq
1-\frac1n \mathbb{E}_{\widetilde{\mathbb{Q}}^n}\left[\sup_s\ln F (s,\tau_1,\tau_2)\right]-\frac1n \mathbb{E}_{\widetilde{\mathbb{Q}}^n}\left[\sup_s\nu ^{-1}(s)\right]-\frac1n \mathbb{E}_{\widetilde{\mathbb{Q}}^n}\left[\sup_s\nu (s)\right]\;.$$
\item Proof that there exist upper boundaries for $\mathbb{E}_{\widetilde{\mathbb{Q}}^n}[\sup_s\ln F (s,\tau_1,\tau_2)]$, $\mathbb{E}_{\widetilde{\mathbb{Q}}^n}[\sup_s\nu ^{-1}(s)]$, and $\mathbb{E}_{\widetilde{\mathbb{Q}}^n}[\sup_s\nu (s)]$,  that are independent of $n$.\\
\end{enumerate}

\textbf{1. Derivation of $\widetilde{\mathbb{Q}}^n$.}
Similar to \cite{Benth2019}, we set $g(z):=(1+z)\log(1+z)-z$ and define the predictable compensator of 
$\frac12\langle H^c,H^c \rangle + \sum_{t\leq \cdot} g(\Delta H(t))$ by
\begin{align*}
C(t,\tau_1,\tau_2):=\frac12 \int_{0}^{t}  \pi_1 ^{\mathbb{P}\widetilde{\mathbb{Q}}}(s,\tau_1,\tau_2)^2 + \pi_\nu ^{\mathbb{P}\widetilde{\mathbb{Q}}}(s,\tau_1,\tau_2)^2
ds + \int_{0}^{t}\int_{\mathbb{R}}g(\pi_2^{\mathbb{P}\widetilde{\mathbb{Q}}}(s,\tau_1,\tau_2)z)\ell^{{\mathbb{P}}}(dz)ds\;, 
\end{align*}
where $H$ from \Eqref{eq:H} now embraces stochastic volatility such that  %$H(t,\tau_1,\tau_2):=H_1 (t,\tau_1,\tau_2)+H_2(t,\tau_1,\tau_2)+H_\nu(t,\tau_1,\tau_2)$ with
\begin{align*}
H(t,\tau_1,\tau_2):=\int_{0}^{t} \pi_1 ^{\mathbb{P}\widetilde{\mathbb{Q}}}(s,\tau_1,\tau_2)dW ^{\mathbb{P}}_s+\int_{0}^{t}\int_{\mathbb{R}} \pi_2^{\mathbb{P}\widetilde{\mathbb{Q}}}(s,\tau_1,\tau_2)z\widetilde{N}^{\mathbb{P}}(ds,dz)+ \int_{0}^{t} \pi_\nu ^{\mathbb{P}\widetilde{\mathbb{Q}}}(s,\tau_1,\tau_2)d\bar{B} ^{\mathbb{P}}_s\;.
%H_1 (t,\tau_1,\tau_2):=& \int_{0}^{t} \pi_1 ^{\mathbb{P}\widetilde{\mathbb{Q}}}(s,\tau_1,\tau_2)dW ^{\mathbb{P}}(s)\;, \\
%H_2(t,\tau_1,\tau_2):=& \int_{0}^{t}\int_{\mathbb{R}} \pi_2^{\mathbb{P}\widetilde{\mathbb{Q}}}(s,\tau_1,\tau_2)z\widetilde{N}^{\mathbb{P}}(ds,dz)\;,\\
%H_\nu(t,\tau_1,\tau_2):=& \int_{0}^{t} \pi_\nu ^{\mathbb{P}\widetilde{\mathbb{Q}}}(s,\tau_1,\tau_2)d\bar{B} ^{\mathbb{P}}(s)\;.
\end{align*}
Note that this stochastic volatility setting covers a three-dimensional market price of risk $\pi:=(\pi_1,\pi_2,\pi_\nu)$ for all independent random parts $W^{\mathbb{P}},\widetilde{J}^{\mathbb{P}},\bar{B}^{\mathbb{P}}$.
As we are in an incomplete setting, we choose the market price of volatility risk, $\pi_\nu$, such that the market price of risk admits the same structure as in the Heston model, i.e., $\rho  \pi_1^{\mathbb{P}\widetilde{\mathbb{Q}}} +\sqrt{1-\rho^2}\pi_\nu^{\mathbb{P}\widetilde{\mathbb{Q}}}=\frac{\delta_\nu}{\sigma_\nu }\sqrt{\nu (t)}$ (see \cite{Heston}). Now let us define a sequence of stopping times
\begin{align}
\hat{\tau}_n:=\inf \big\{ t\in[0,\tau_1]: \vert \ln F(t,\tau_1,\tau_2)\vert\geq n, \text{ or } \vert \nu^{-1}(t)\vert \geq n,  \text{ or } \vert \nu(t)\vert \geq n \big\}\;, \label{eq:StoppingTime}
\end{align}
and observe that for every $n\in\mathbb{N}$, the stopped process $C(t\wedge\hat{\tau}_n,\tau_1,\tau_2)$ is bounded.
Hence, by \cite{LepingleMEmin} (cf.\ Theorem III.1), we know that $Z^{\mathbb{P}\widetilde{\mathbb{Q}}}(t\wedge\hat{\tau}_n,\tau_1,\tau_2)$ is a uniformly integrable martingale such that we can define the probability measure $\widetilde{\mathbb{Q}}^n$ by
\begin{align}
\frac{d\widetilde{\mathbb{Q}}^n}{d\mathbb{P}} := Z^{\mathbb{P}\widetilde{\mathbb{Q}}}(\tau_1\wedge\hat{\tau}_n,\tau_1,\tau_2)\;.\label{eq:MeasureChangeQn}
\end{align}

\textbf{2. Proof of lower boundary of $\mathbb{E}_{{\mathbb{P}}}[Z^{\mathbb{P}\widetilde{\mathbb{Q}}}(\tau_1)]$.} 
First, $Z^{\mathbb{P}\widetilde{\mathbb{Q}}}$ is a positive local martingale by the assumption that $\pi_2^{\mathbb{P}\widetilde{\mathbb{Q}}}(t,\tau_1,\tau_2)\geq -1$ for all $t\in[0,\tau_1]$. Hence, it is a supermartingale, so that
we know the upper boundary   for $\tau_1\geq 0$:
\begin{align*}
\mathbb{E}_{\mathbb{P}}[Z^{\mathbb{P}\widetilde{\mathbb{Q}}}(\tau_1,\tau_1,\tau_2)] \leq \mathbb{E}_{\mathbb{P}}[Z^{\mathbb{P}\widetilde{\mathbb{Q}}}(0,\tau_1,\tau_2)] =1\;.
\end{align*}
Next, we consider the lower boundary, following \cite{Benth2019}:
\begin{align*}
\mathbb{E}_{\mathbb{P}}[Z^{\mathbb{P}\widetilde{\mathbb{Q}}}(\tau_1,\tau_1,\tau_2)] \geq
\mathbb{E}_{\mathbb{P}}[Z^{\mathbb{P}\widetilde{\mathbb{Q}}}(\tau_1,\tau_1,\tau_2) \mathbbm{1}_{\hat{\tau}_n > \tau_1}] 
=\mathbb{E}_{\mathbb{P}}[Z^{\mathbb{P}\widetilde{\mathbb{Q}}}(\tau_1\wedge \hat{\tau}_n,\tau_1,\tau_2) \mathbbm{1}_{\hat{\tau}_n > \tau_1}]= \widetilde{\mathbb{Q}}^n[\hat{\tau}_n > \tau_1]\;,
\end{align*}
where the last equality follows from the change of measure defined in Step 1 (see \Eqref{eq:MeasureChangeQn}). By definition of the stopping time $\hat{\tau}_n$ (see \Eqref{eq:StoppingTime}), we deduce
\begin{align*}
\mathbb{E}_{\mathbb{P}}[Z^{\mathbb{P}\widetilde{\mathbb{Q}}}(\tau_1,\tau_1,\tau_2)] \geq & 1- \widetilde{\mathbb{Q}}^n[\hat{\tau}_n \leq \tau_1]\\ \geq & 1- \left(\widetilde{\mathbb{Q}}^n\left[\sup_{s\in[0,\tau_1]}\ln F (s,\tau_1,\tau_2)\geq n\right]+\widetilde{\mathbb{Q}}^n\left[\sup_{s\in[0,\tau_1]}\nu ^{-1}(s)\geq n\right] +\widetilde{\mathbb{Q}}^n\left[\sup_{s\in[0,\tau_1]}\nu (s)\geq n\right] \right)\\
\geq & 1- \frac1n  \left(\mathbb{E}_{\widetilde{\mathbb{Q}}^n}\left[\sup_{s\in[0,\tau_1]}\ln F (s,\tau_1,\tau_2)\right]+\mathbb{E}_{\widetilde{\mathbb{Q}}^n}\left[\sup_{s\in[0,\tau_1]}\nu ^{-1}(s)\right]+\mathbb{E}_{\widetilde{\mathbb{Q}}^n}\left[\sup_{s\in[0,\tau_1]}\nu (s)\right] \right)\;,
\end{align*}
where the last inequality follows from Markov's inequality.
If we show that the expectations on the right hand side have upper boundaries that are independent of $n\in\mathbb{N}$, then  $\mathbb{E}_{\mathbb{P}}[Z^{\mathbb{P}\widetilde{\mathbb{Q}}}(\tau_1,\tau_1,\tau_2)]=1$, which is addressed in the third step.\\

\textbf{3. Proof of upper boundaries.}
In order to identify upper boundaries under the measure $\widetilde{\mathbb{Q}}^n$ defined in \Eqref{eq:MeasureChangeQn}, we need to derive the dynamics of $\ln F$, $\nu ^{-1}$, and $\nu$  under $\widetilde{\mathbb{Q}}^n$. 
We apply Girsanov's theorem (see \cite{OksendalSulem}, cf.\ Theorem 1.35) to Equations~\eqref{eq:Y2tilde} and \eqref{eq:CIR}, where 
\begin{align*}
W^{\widetilde{\mathbb{Q}}^n}_t &= W^{\mathbb{P}}_t + \int_{0}^{t} \Pi_1 ^{\mathbb{P}\widetilde{\mathbb{Q}}}(s,\tau_1,\tau_2)\mathbbm{1}_{[0,\hat{\tau}_n]}(s)ds\;,\\
B^{\widetilde{\mathbb{Q}}^n}_t &= B^{\mathbb{P}}_t + \int_{0}^{t} \frac{\delta_\nu }{\sigma_\nu }\sqrt{\nu (s)}\mathbbm{1}_{[0,\hat{\tau}_n]}(s)ds\;,
\end{align*}
are correlated standard Brownian motions under $\widetilde{\mathbb{Q}}^n$ and 
\begin{align*}
\widetilde{N}^{\widetilde{\mathbb{Q}}^n}(dt,dz)=\widetilde{N}^{\mathbb{P}}(dt,dz) + \Pi_2^{\mathbb{P}\widetilde{\mathbb{Q}}}(t,\tau_1,\tau_2)\mathbbm{1}_{[0,\hat{\tau}_n]}(t)\ell^{\mathbb{P}}(dz)dt\;,
\end{align*}
is the $\widetilde{\mathbb{Q}}^n$-compensated Poisson random measure.
Moreover, by Ito's formula, we find 
\begin{align*}
d\nu ^{-1}(t) = \nu ^{-1}(t)\left( \kappa_\nu+\delta_\nu \mathbbm{1}_{[0,\hat{\tau}_n]}(t)-\nu ^{-1}(t)(\kappa_\nu \theta_\nu -\sigma_\nu^2)\right)dt 
- \sigma_\nu \nu ^{-\frac32}(t)dB ^{\widetilde{\mathbb{Q}}^n}_t\;.
\end{align*}
Hence, we can show 
\begingroup
\allowdisplaybreaks
\begin{align*}
&~\mathbb{E}_{\widetilde{\mathbb{Q}}^n}\left[\sup_{s\in[0,\tau_1]}\vert \nu ^{-1}(s) \vert\right]\\ 
\overset{(\star)}{\leq}&~ 
\frac{1}{\nu_0} + \mathbb{E}_{\widetilde{\mathbb{Q}}^n}\left[\sup_{s\in[0,\tau_1]}\int_{0}^{s}\nu ^{-1}(t)\left( \kappa_\nu+\delta_\nu \mathbbm{1}_{[0,\hat{\tau}_n]}(t)-\nu ^{-1}(t)(\kappa_\nu \theta_\nu -\sigma_\nu^2)\right)dt \right]+\mathbb{E}_{\widetilde{\mathbb{Q}}^n}\left[\sup_{s\in[0,\tau_1]}\int_{0}^{s}\sigma_\nu\nu ^{-\frac32}(t) dB ^{\widetilde{\mathbb{Q}}^n}_t\right]\\
\overset{(\star\star)}{\leq}&~ 
\frac{1}{\nu_0} + 
(\kappa+\No{\delta_\nu})\mathbb{E}_{\widetilde{\mathbb{Q}}^n}\left[\int_{0}^{\tau_1}\nu ^{-1}(t)dt\right]+(\kappa_\nu\theta_\nu -\sigma_\nu^2)\mathbb{E}_{\widetilde{\mathbb{Q}}^n}\left[\int_{0}^{\tau_1}\nu ^{-2}(t)dt\right]+\sigma_\nu n^{-\frac32}\mathbb{E}_{\widetilde{\mathbb{Q}}^n}\left[\sup_{s\in[0,\tau_1]}\int_{0}^{s}dB ^{\widetilde{\mathbb{Q}}^n}_t\right]\\
%
\overset{(\star\star\star)}{=}&~
\frac{1}{\nu_0} + 
(\kappa+\No{\delta_\nu})\int_{0}^{\tau_1}\mathbb{E}_{\widetilde{\mathbb{Q}}^n}\left[\nu ^{-1}(t)\right]dt+(\kappa_\nu\theta_\nu -\sigma_\nu^2)\int_{0}^{\tau_1}\mathbb{E}_{\widetilde{\mathbb{Q}}^n}\left[\nu ^{-2}(t)\right]dt\;.
\end{align*}
\endgroup
Inequality~$(\star)$ follows from the integral representation of $\nu ^{-1}$ and the triangle inequality. Inequality~$(\star\star)$ results from the fact, that the extended Feller condition is satisfied (i.e., $\sigma_\nu^2<\kappa_\nu\theta_\nu $) and that  $\nu^{-1}\leq n$ under~$\widetilde{\mathbb{Q}}^n$. Since both processes $\nu ^{-1}$ and $\nu ^{-2}$ are positive, the supremum disappears in the first two cases and the upper boundary is used. 
Equality~$(\star\star\star)$ is reached by stochastic Fubini to the first two integrals and the last term disappears.
From \cite{Dereich} (cf.\ Chapter 3), we know that the expectations of the inverse and the inverse quadratic stochastic volatility, $\mathbb{E}_{\widetilde{\mathbb{Q}}^n}\left[\nu ^{-1}(t)\right]$ and $\mathbb{E}_{\widetilde{\mathbb{Q}}^n}\left[\nu ^{-2}(t)\right]$, can be characterized explicitly and are bounded independently of $n$, as long as the extended Feller condition $\sigma_\nu^2 < \kappa_\nu \theta_\nu $ is satisfied. Hence, $\mathbb{E}_{\widetilde{\mathbb{Q}}^n}\left[\sup_{s\in[0,\tau_1]}\vert \nu^{-1}(s) \vert\right]\leq c_1 \independent n$.

Moreover, we can show that $\vert \nu  \vert^2$ is uniformly integrable:
{\small
\begingroup
\allowdisplaybreaks
\begin{align*}
&~\mathbb{E}_{\widetilde{\mathbb{Q}}^n}\left[\sup_{s\in[0,\tau_1]} \vert \nu (s) \vert^2\right]
=
\mathbb{E}_{\widetilde{\mathbb{Q}}^n}\left[\sup_{s\in[0,\tau_1]}
\left(\nu_0+\int_{0}^{s} \kappa_\nu\theta_\nu  -(\kappa_\nu +\delta_\nu  \mathbbm{1}_{[0,\hat{\tau}_n]}(t))\nu (t)dt + \int_{0}^{s}\sigma_\nu \sqrt{\nu (t)}dB ^{\widetilde{\mathbb{Q}}^n}(t)\right)^2\right]\\
\overset{(\star)}{\leq}&~
4\Bigg( v_0^2+ 
\mathbb{E}_{\widetilde{\mathbb{Q}}^n}\left[\sup_{s\in[0,\tau_1]}
\left(\int_{0}^{s} \kappa_\nu\theta_\nu dt\right)^2
+\sup_{s\in[0,\tau_1]}
\left(\int_{0}^{s} (\kappa_\nu +\delta_\nu  \mathbbm{1}_{[0,\hat{\tau}_n]}(t))\nu (t)dt\right)^2%\\&
+\sup_{s\in[0,\tau_1]}
\left(\int_{0}^{s}\sigma_\nu \sqrt{\nu (t)}dB ^{\widetilde{\mathbb{Q}}^n}_t\right)^2\right]
\Bigg)\\
\overset{(\star\star)}{\leq}&~
4\Bigg( v_0^2+ 
4\mathbb{E}_{\widetilde{\mathbb{Q}}^n}\left[
\left(\int_{0}^{\tau_1} \kappa_\nu\theta_\nu dt\right)^2\right]
+4\mathbb{E}_{\widetilde{\mathbb{Q}}^n}\left[
\left(\int_{0}^{\tau_1} (\kappa_\nu +\delta_\nu  \mathbbm{1}_{[0,\hat{\tau}_n]}(t))\nu (t)dt\right)^2\right]
+4\mathbb{E}_{\widetilde{\mathbb{Q}}^n}\left[
\left(\int_{0}^{\tau_1}\sigma_\nu \sqrt{\nu (t)}dB ^{\widetilde{\mathbb{Q}}^n}_t\right)^2\right]
\Bigg)\\
\overset{(\star\star\star)}{\leq}&~
4\Bigg( v_0^2+ 
4\tau_1\int_{0}^{\tau_1} \kappa_\nu^2\theta_\nu^2dt
+4\tau_1 (\kappa_\nu +\No{\delta_\nu })^2\int_{0}^{\tau_1}\mathbb{E}_{\widetilde{\mathbb{Q}}^n}\left[ \sup_{s\in[0,t]} \nu (s)^2\right]dt
+4\sigma_\nu^2\mathbb{E}_{\widetilde{\mathbb{Q}}^n}\left[
\int_{0}^{\tau_1} \nu (t)dt\right]\;,
\Bigg)
\end{align*}
\endgroup
}
where the first equality represents the integral version of $\nu$. Inequality~$(\star)$ results from the Cauchy-Schwartz inequality to the sum
and an application of the triangle inequality. 
We apply Doob's inequality to all expectations in Inequality~$(\star\star)$. 
In Inequality~$(\star\star\star)$, we apply the Cauchy-Schwartz inequality to the first and second integral and apply Ito's isometry to the last summand.  
We finish with the stochastic Fubini to the  second integral while making the integrand even bigger. 
Note that for the last summand, we have $\mathbb{E}_{\widetilde{\mathbb{Q}}^n}\left[
\int_{0}^{\tau_1} \nu (t)dt\right]\leq \tilde{c}_{\nu}\independent n$ since we can find explicit expressions in \cite{cont_tankov2003financial} (cf.\ Chapter 15).
Setting $c_{\nu}:=4v_0^2+ 
16\tau_1^2 \kappa_\nu^2\theta_\nu ^2
+16\sigma_\nu^2\tilde{c}_{\nu}$, then, by Gronwall, we receive
$
\mathbb{E}_{\widetilde{\mathbb{Q}}^n}\left[\sup_{s\in[0,\tau_1]} \vert \nu (s) \vert^2\right] \leq c_{\nu} e^{16(\kappa_\nu +\No{\delta_\nu })^2\tau_1^2 }=:c_2 \independent n$.
%Hence, $\mathbb{E}_{\widetilde{\mathbb{Q}}^n}\left[\sup_{s\in[0,\tau_1]}\vert \nu(s) \vert\right]\leq c_2\independent n$.


Next, we show that $\vert \ln F \vert^2$ is uniformly integrable:
%\begingroup
%\allowdisplaybreaks
%\begin{align*}
%&\mathbb{E}_{\widetilde{\mathbb{Q}}^n}\left[\sup_{s\in[0,\tau_1]} \vert \ln F_1(s,\tau_1,\tau_2) \vert^2\right]\\
%=& \mathbb{E}_{\widetilde{\mathbb{Q}}^n}\Bigg[\sup_{s\in[0,\tau_1]} \Big(\ln F_1(0,\tau_1,\tau_2)+ \int_{0}^{s}  \left(1-\mathbbm{1}_{[0,\hat{\tau}_n]}(t)\right)\left(\mathbb{E} [s_1(t,U)\mu_1(t,U)]- \kappa_1(t)\ln F_1(t,\tau_1,\tau_2)\right)dt\\&
%-\int_{0}^{s}\frac12\mathbb{E} [\sigma_1(t,U)s_1(t,U)]^2\mathbbm{1}_{[0,\hat{\tau}_n]}(t)dt
%+\int_{0}^{s}\mathbb{E} [\sigma_1(t,U)s_1(t,U)]dW_1^{\widetilde{\mathbb{Q}}^n}(t) \Big)^2\Bigg]\\
%%
%\overset{(\star)}{\leq}&
%5\Big(\ln F_1(0,\tau_1,\tau_2)^2 + \mathbb{E}_{\widetilde{\mathbb{Q}}^n}\left[
%\sup_{s\in[0,\tau_1]}  \left(\int_{0}^{s}\left(1-\mathbbm{1}_{[0,\hat{\tau}_n]}(t)\right)\mathbb{E} [s_1(t,U)\mu_1(t,U)]dt\right)^2\right]
%+
%\mathbb{E}_{\widetilde{\mathbb{Q}}^n}\left[
%\sup_{s\in[0,\tau_1]} \left(\int_{0}^{s}
%\frac12\mathbb{E} [\sigma_1(t,U)s_1(t,U)]^2\mathbbm{1}_{[0,\hat{\tau}_n]}(t)dt\right)^2\right]
%\\&+
%\mathbb{E}_{\widetilde{\mathbb{Q}}^n}\left[\sup_{s\in[0,\tau_1]}\left(\int_{0}^{s}\left(1-\mathbbm{1}_{[0,\hat{\tau}_n]}(t)\right) \kappa_1(t)\ln F_1(t,\tau_1,\tau_2)dt\right)^2\right]+
%\mathbb{E}_{\widetilde{\mathbb{Q}}^n}\left[\sup_{s\in[0,\tau_1]}\left(\int_{0}^{s}\mathbb{E} [\sigma_1(t,U)s_1(t,U)]dW_1^{\widetilde{\mathbb{Q}}^n}(t) \right)^2\right]
%\Big)
%\\
%\overset{(\star\star)}{\leq}&
%5\Big(\ln F_1(0,\tau_1,\tau_2)^2 + 4\mathbb{E}_{\widetilde{\mathbb{Q}}^n}\left[
% \left(\int_{0}^{\tau_1}\left(1-\mathbbm{1}_{[0,\hat{\tau}_n]}(t)\right)\mathbb{E} [s_1(t,U)\mu_1(t,U)]dt\right)^2\right]
% +
% 4\mathbb{E}_{\widetilde{\mathbb{Q}}^n}\left[
% \left(\int_{0}^{\tau_1}
% \frac12\mathbb{E} [\sigma_1(t,U)s_1(t,U)]^2\mathbbm{1}_{[0,\hat{\tau}_n]}(t)dt\right)^2\right]
%\\&+
%4\mathbb{E}_{\widetilde{\mathbb{Q}}^n}\left[\left(\int_{0}^{\tau_1}\left(1-\mathbbm{1}_{[0,\hat{\tau}_n]}(t)\right) \kappa_1(t)\ln F_1(t,\tau_1,\tau_2)dt\right)^2\right]+
%4\mathbb{E}_{\widetilde{\mathbb{Q}}^n}\left[\left(\int_{0}^{\tau_1}\mathbb{E} [\sigma_1(t,U)s_1(t,U)]dW_1^{\widetilde{\mathbb{Q}}^n}(t) \right)^2\right]
%\Big)\\
%\overset{(\star\star\star)}{\leq}&
%5\Big(\ln F_1(0,\tau_1,\tau_2)^2 + 
%4\tau_1\int_{0}^{\tau_1}\mathbb{E} [s_1(t,U)\mu_1(t,U)]^2dt
%+\tau_1\mathbb{E}_{\widetilde{\mathbb{Q}}^n}\left[\int_{0}^{\tau_1}\mathbb{E} [\sigma_1(t,U)s_1(t,U)]^4dt\right]\\&
%+4\int_{0}^{\tau_1} \kappa_1(t)^2dt
%\mathbb{E}_{\widetilde{\mathbb{Q}}^n}\left[\int_{0}^{\tau_1}\ln F_1(t,\tau_1,\tau_2)^2dt\right]+
%4\mathbb{E}_{\widetilde{\mathbb{Q}}^n}\left[\left(\int_{0}^{\tau_1}\mathbb{E} [\sigma_1(t,U)s_1(t,U)]dW_1^{\widetilde{\mathbb{Q}}^n}(t) \right)^2\right]
%\Big)\\
%\overset{}{\leq}&
%5\Big(\ln F_1(0,\tau_1,\tau_2)^2 + 
%4\tau_1\int_{0}^{\tau_1}\mathbb{E} [s_1(t,U)\mu_1(t,U)]^2dt
%+\tau_1\mathbb{E}_{\widetilde{\mathbb{Q}}^n}\left[\int_{0}^{\tau_1}\mathbb{E} [\sigma_1(t,U)s_1(t,U)]^4dt\right]\\&
%+4\int_{0}^{\tau_1} \kappa_1(t)^2dt
%\int_{0}^{\tau_1}\mathbb{E}_{\widetilde{\mathbb{Q}}^n}\left[\sup_{s\in[0,t]}\ln F_1(s,\tau_1,\tau_2)^2\right]dt+
%4\mathbb{E}_{\widetilde{\mathbb{Q}}^n}\left[\int_{0}^{\tau_1}\mathbb{E} [\sigma_1(t,U)s_1(t,U)]^2dt \right]
%\Big)\;.
%%
%%2\bar{v_\sigma}^2\mathbb{E}_{\widetilde{\mathbb{Q}}^n}\left[\int_{0}^{\tau_1}\nu(t)dt \right]\Big)\;.
%\end{align*}
%\endgroup
%The first equality represents the integral version of $\ln F_1$. Inequality $(\star)$ results from the Cauchy-Schwartz inequality to the sum
%and an application of the triangle inequality. 
%We apply Doob's inequality to all expectations in Inequality $(\star\star)$. 
%In Inequality $(\star\star\star)$, we apply the Cauchy-Schwartz inequality to the first three integrals.  
%We finish with Ito's isometry to the last summand and applying stochastic Fubini to the $\ln F_1$-term while making the integrand even bigger. 
%By the previous considerations, we know that the last summand $\mathbb{E}_{\widetilde{\mathbb{Q}}^n}\left[\int_{0}^{\tau_1}\mathbb{E} [\sigma_1(t,U)s_1(t,U)]^2dt \right]\leq 4\sigma_1^2\bar{s}_1^2\tilde{c}_{\nu}$ is bounded independently of  $n$. Moreover, 
%\begin{align*}
%\mathbb{E}_{\widetilde{\mathbb{Q}}^n}\left[\int_{0}^{\tau_1}\mathbb{E} [\sigma_1(t,U)s_1(t,U)]^4dt\right] \leq \sigma \bar{s}_1\mathbb{E}_{\widetilde{\mathbb{Q}}^n}\left[\int_{0}^{\tau_1}\nu_1(t)^2dt\right]\leq
%\sigma \bar{s}_1\int_{0}^{\tau_1}\mathbb{E}_{\widetilde{\mathbb{Q}}^n}\left[\sup_{s\in[0,t]}\nu_1(s)^2\right]dt \leq\sigma \bar{s}_1 c_{\nu_1}\tau_1\;.
%\end{align*}
%
%Setting $c_{Y_1}:=5\ln F_1(0,\tau_1,\tau_2)^2 + 
%20\tau_1\int_{0}^{\tau_1}\mathbb{E} [s_1(t,U)\mu_1(t,U)]^2dt
%+\sigma \bar{s}_1 c_{\nu_1} \tau_1^2
%+   20 \sigma_1^2\bar{s}_1^2\tilde{c}_{\nu}$, then, by Gronwall
%\begin{align*}
%\mathbb{E}_{\mathbb{P}^n}\left[\sup_{s\in[0,\tau_1]} \vert \ln F_1(s,\tau_1,\tau_2) \vert^2\right] \leq c_{Y_1} e^{16\int_{0}^{\tau_1} \kappa_1(t)^2dt } =:c_3\independent n\;.
%\end{align*}
%
%For $\vert \ln F_2 \vert^2$ we proceed analogously:
{\small
\begingroup
\allowdisplaybreaks
\begin{align*}
&~\mathbb{E}_{\widetilde{\mathbb{Q}}^n}\left[\sup_{s\in[0,\tau_1]} \vert \ln F(s,\tau_1,\tau_2) \vert^2\right]\\
=&~ 
\mathbb{E}_{\widetilde{\mathbb{Q}}^n}\Bigg[\sup_{s\in[0,\tau_1]} \Big(\ln F(0,\tau_1,\tau_2)+ \int_{0}^{s}  \left(1-\mathbbm{1}_{[0,\hat{\tau}_n]}(t)\right)\left(\mathbb{E} [ \mu(t,U)]- \kappa(t)\ln F(t,\tau_1,\tau_2)\right)dt
-\int_{0}^{s}\frac12\mathbb{E} [\sigma (t,U) ]^2\nu (t)\mathbbm{1}_{[0,\hat{\tau}_n]}(t)dt\\&~
+\int_{0}^{s}\mathbb{E} [\sigma (t,U) ]\sqrt{\nu (t)}dW^{\widetilde{\mathbb{Q}}^n}_t
+\int_{0}^{s}\mathbb{E} [ \eta(t,U)]d\widetilde{J}^{\widetilde{\mathbb{Q}}^n}_t\\&~
-\int_{0}^{s}\mathbb{E} [ \eta(t,U)] \left(1- \frac{\mathbb{E} [ \eta(t,U)]\int_{\mathbb{R}}z\ell^{\mathbb{P}}(dz)}{\int_{\mathbb{R}}
	e^{\mathbb{E} [ \eta(t,U)]z}-1
	\ell^{\mathbb{P}}(dz)}\right)\mathbbm{1}_{[0,\hat{\tau}_n]}(t)\int_{\mathbb{R}}z  \ell^{\mathbb{P}}(dz)dt
\Big)^2\Bigg]\\
%
\overset{(\star)}{\leq}&~
7\Bigg(\ln F(0,\tau_1,\tau_2)^2 + \mathbb{E}_{\widetilde{\mathbb{Q}}^n}\left[
\sup_{s\in[0,\tau_1]}  \left(\int_{0}^{s}\left(1-\mathbbm{1}_{[0,\hat{\tau}_n]}(t)\right)\mathbb{E} [ \mu(t,U)]dt\right)^2\right]
\\&~
+
\mathbb{E}_{\widetilde{\mathbb{Q}}^n}\left[
\sup_{s\in[0,\tau_1]} \left(\int_{0}^{s}
\frac12\mathbb{E} [\sigma (t,U) ]^2\nu(t)\mathbbm{1}_{[0,\hat{\tau}_n]}(t)dt\right)^2\right]
+
\mathbb{E}_{\widetilde{\mathbb{Q}}^n}\left[\sup_{s\in[0,\tau_1]}\left(\int_{0}^{s}\left(1-\mathbbm{1}_{[0,\hat{\tau}_n]}(t)\right) \kappa(t)\ln F(t,\tau_1,\tau_2)dt\right)^2\right]\\&~
+
\mathbb{E}_{\widetilde{\mathbb{Q}}^n}\left[\sup_{s\in[0,\tau_1]}\left(\int_{0}^{s}\mathbb{E} [\sigma (t,U) ]\sqrt{\nu(t)}dW^{\widetilde{\mathbb{Q}}^n}_t \right)^2\right]
+\mathbb{E}_{\widetilde{\mathbb{Q}}^n}\left[\sup_{s\in[0,\tau_1]}\left(\int_{0}^{s}\mathbb{E} [ \eta(t,U)]d\widetilde{J}^{\widetilde{\mathbb{Q}}^n}_t
\right)^2\right]\\&~
+\mathbb{E}_{\widetilde{\mathbb{Q}}^n}\left[\sup_{s\in[0,\tau_1]}\left(\int_{0}^{s}\mathbb{E} [ \eta(t,U)] \left(1- \frac{\mathbb{E} [ \eta(t,U)]\int_{\mathbb{R}}z\ell^{\mathbb{P}}(dz)}{\int_{\mathbb{R}}
	e^{\mathbb{E} [ \eta(t,U)]z}-1
	\ell^{\mathbb{P}}(dz)}\right)\mathbbm{1}_{[0,\hat{\tau}_n]}(t)\int_{\mathbb{R}}z  \ell^{\mathbb{P}}(dz)dt
\right)^2\right]
\Bigg)
\\
\overset{(\star\star)}{\leq}&~
7\Bigg(\ln F(0,\tau_1,\tau_2)^2 + 4\mathbb{E}_{\widetilde{\mathbb{Q}}^n}\left[
 \left(\int_{0}^{\tau_1}\left(1-\mathbbm{1}_{[0,\hat{\tau}_n]}(t)\right)\mathbb{E} [ \mu(t,U)]dt\right)^2\right]
\\&~
+
4\mathbb{E}_{\widetilde{\mathbb{Q}}^n}\left[
\left(\int_{0}^{\tau_1}
\frac12\mathbb{E} [\sigma (t,U) ]^2\nu(t)\mathbbm{1}_{[0,\hat{\tau}_n]}(t)dt\right)^2\right]
+
4\mathbb{E}_{\widetilde{\mathbb{Q}}^n}\left[\left(\int_{0}^{\tau_1}\left(1-\mathbbm{1}_{[0,\hat{\tau}_n]}(t)\right) \kappa(t)\ln F(t,\tau_1,\tau_2)dt\right)^2\right]\\&~
+
4\mathbb{E}_{\widetilde{\mathbb{Q}}^n}\left[\left(\int_{0}^{\tau_1}\mathbb{E} [\sigma (t,U) ]\sqrt{\nu(t)}dW^{\widetilde{\mathbb{Q}}^n}_t \right)^2\right]
+4\mathbb{E}_{\widetilde{\mathbb{Q}}^n}\left[\left(\int_{0}^{\tau_1}\mathbb{E} [ \eta(t,U)]d\widetilde{J}^{\widetilde{\mathbb{Q}}^n}_t
\right)^2\right]\\&~
+4\mathbb{E}_{\widetilde{\mathbb{Q}}^n}\left[\left(\int_{0}^{\tau_1}\mathbb{E} [ \eta(t,U)] \left(1- \frac{\mathbb{E} [ \eta(t,U)]\int_{\mathbb{R}}z\ell^{\mathbb{P}}(dz)}{\int_{\mathbb{R}}
	e^{\mathbb{E} [ \eta(t,U)]z}-1
	\ell^{\mathbb{P}}(dz)}\right)\mathbbm{1}_{[0,\hat{\tau}_n]}(t)\int_{\mathbb{R}}z  \ell^{\mathbb{P}}(dz)dt
\right)^2\right]
\Bigg)
\\
\overset{(\star\star\star)}{\leq}&~
7\Bigg(\ln F(0,\tau_1,\tau_2)^2 + 4\tau_1\int_{0}^{\tau_1}\mathbb{E} [ \mu(t,U)]^2dt
\\&~
+
4\int_{0}^{\tau_1}\mathbb{E} [\sigma (t,U) ]^4dt~\mathbb{E}_{\widetilde{\mathbb{Q}}^n}\left[ \int_{0}^{\tau_1}\nu(t)^2dt\right]
+
4\int_{0}^{\tau_1} \kappa(t)^2dt~\mathbb{E}_{\widetilde{\mathbb{Q}}^n}\left[\int_{0}^{\tau_1}\ln F(t,\tau_1,\tau_2)^2dt\right]\\&~
+
4\mathbb{E}_{\widetilde{\mathbb{Q}}^n}\left[\left(\int_{0}^{\tau_1}\mathbb{E} [\sigma (t,U) ]\sqrt{\nu(t)}dW^{\widetilde{\mathbb{Q}}^n}_t \right)^2\right]
+4\mathbb{E}_{\widetilde{\mathbb{Q}}^n}\left[\left(\int_{0}^{\tau_1}\mathbb{E} [ \eta(t,U)]d\widetilde{J}^{\widetilde{\mathbb{Q}}^n}_t
\right)^2\right]\\&~
+4\int_{0}^{\tau_1}\mathbb{E} [ \eta(t,U)]^2dt
\mathbb{E}_{\widetilde{\mathbb{Q}}^n}\left[\int_{0}^{\tau_1}\left(1-\mathbb{E} [ \eta(t,U)]^2 \frac{\left(\int_{\mathbb{R}}z\ell^{\mathbb{P}}(dz)\right)^2}{\left(\int_{\mathbb{R}}
	e^{\mathbb{E} [ \eta(t,U)]z}-1
	\ell^{\mathbb{P}}(dz)\right)^2}\right)\left(\int_{\mathbb{R}}z \ell^{\mathbb{P}}(dz)\right)^2dt\right]
\Bigg)
\\
\overset{}{\leq}&~
7\Bigg(\ln F(0,\tau_1,\tau_2)^2 + 4\tau_1\int_{0}^{\tau_1}\mathbb{E} [ \mu(t,U)]^2dt
+
4c_2 \tau_1 \int_{0}^{\tau_1}\mathbb{E} [\sigma (t,U) ]^4dt\\&~
+
4\int_{0}^{\tau_1} \kappa(t)^2dt~\mathbb{E}_{\widetilde{\mathbb{Q}}^n}\left[\int_{0}^{\tau_1}\sup_{s\in[0,t]}\ln F(s,\tau_1,\tau_2)^2dt\right]
+
4\sqrt{\int_{0}^{\tau_1}\mathbb{E} [\sigma (t,U) ]^4dt} \sqrt{\tau_1 c_2}
+4\int_{0}^{\tau_1}\mathbb{E} [ \eta(t,U)]^2\int_{\mathbb{R}}z^2\ell^{\widetilde{\mathbb{Q}}^n}(dz)dt\\&~
+4\int_{0}^{\tau_1}\mathbb{E} [ \eta(t,U)]^2dt\int_{0}^{\tau_1}\left(\left(\int_{\mathbb{R}}z \ell^{\mathbb{P}}(dz)\right)^2+\mathbb{E} [ \eta(t,U)]^2 \frac{\left(\int_{\mathbb{R}}z\ell^{\mathbb{P}}(dz)\right)^4}{\left(\int_{\mathbb{R}}
	e^{\mathbb{E} [ \eta(t,U)]z}-1
	\ell^{\mathbb{P}}(dz)\right)^2}\right)dt
\Bigg)\;,\\
=:&~c_Y +28\int_{0}^{\tau_1} \kappa(t)^2dt~\mathbb{E}_{\widetilde{\mathbb{Q}}^n}\left[\int_{0}^{\tau_1}\sup_{s\in[0,t]}\ln F(s,\tau_1,\tau_2)^2dt\right]\;.
\end{align*}
\endgroup
}
The first equality represents the integral version of $\ln F$. Inequality~$(\star)$ results from the Cauchy-Schwartz inequality to the sum
and an application of the triangle inequality. 
We apply Doob's inequality to all expectations in Inequality~$(\star\star)$. 
In Inequality~$(\star\star\star)$, we apply the Cauchy-Schwartz inequality to the first three integrals.  
We finish with Itô-Lévy Isometry (see \cite{OksendalSulem}, cf.\ Theorem 1.17) to the last summand and an application of the stochastic Fubini theorem  to the fourth summand (including $\ln F$) while making the integrand even bigger. 
By the previous considerations, we know that  $\mathbb{E}_{\widetilde{\mathbb{Q}}^n}\left[\int_{0}^{\tau_1}\mathbb{E} [\sigma (t,U) ]^2\nu(t)dt \right]\leq\sqrt{\int_{0}^{\tau_1}\mathbb{E} [\sigma (t,U) ]^4dt}\mathbb{E}_{\widetilde{\mathbb{Q}}^n}\left[\sqrt{\int_{0}^{\tau_1}\nu(t)^2dt}\right]\leq \sqrt{\int_{0}^{\tau_1}\mathbb{E} [\sigma (t,U) ]^4dt} \sqrt{\tau_1 c_2}$ is bounded independently of  $n$ and that
$\int_{0}^{\tau_1}\mathbb{E} [\sigma (t,U) ]^4dt~\mathbb{E}_{\widetilde{\mathbb{Q}}^n}\left[ \int_{0}^{\tau_1}\nu(t)^2dt\right] \leq c_2 \tau_1 \int_{0}^{\tau_1}\mathbb{E} [\sigma (t,U) ]^4dt$ is independent of $n$.
By the choice of $c_Y$, an application of Gronwall's inequality yields
$
\mathbb{E}_{\mathbb{P}^n}\left[\sup_{s\in[0,\tau_1]} \vert \ln F(s,\tau_1,\tau_2) \vert^2\right] \leq c_{Y} e^{28\int_{0}^{\tau_1} \kappa(t)^2dt } =:c_3\independent n\;,
$
such that we have shown, that $Z^{\mathbb{P}\widetilde{\mathbb{Q}}}$ is indeed a true martingale.
\newline