%\documentclass[prl,epsfig,psfig,twocolumn,superscriptaddress,floatfix]{revtex4-1}
\documentclass[epsfig,psfig,aps,twocolumn,prl,showpacs]{revtex4-1}
\usepackage{amsfonts,amsmath,bbm,wasysym}
\usepackage{amsfonts,dsfont}
%\usepackage{amssymb}
%\usepackage[utf8]{inputenc} 
%\usepackage[T1]{fontenc}
\usepackage[english]{babel}
\usepackage{graphicx}
\usepackage{epsfig}
\usepackage{relsize}
\usepackage[colorlinks=true,citecolor=red,urlcolor=blue]{hyperref}
\usepackage{soul}
\newcommand{\sgn}{\operatorname{sgn}} 
 \usepackage{bm}
 \def\vb#1{{\bm#1}}
\def\v#1{\mathbf{#1}}
 %\usepackage{bbm}
 %\newcommand{\beq}{\begin{eqnarray}}
 

%\usepackage{epsf}
%%%%%%%%%%%%%%%%%%%%%%%%%%%%
%-----------------------------------
%------------------------------------------------------
\begin{document}

\title{Twisted bilayer graphene reveals its flat bands under spin pumping
}
\author{Sonia Haddad$^{1,2}$}
\email{sonia.haddad@fst.utm.tn}
\author{Takeo Kato$^{2}$}
\author{Lassaad Mandhour$^{1}$}
\affiliation{
$^1$Laboratoire de Physique de la Mati\`ere Condens\'ee, Facult\'e des Sciences de Tunis, Universit\'e Tunis El Manar, Campus Universitaire 1060 Tunis, Tunisia\\
$^2$ Institute for Solid State Physics, University of Tokyo, Kashiwa, Chiba 277-8581, Japan
}
\date{\today}

\begin{abstract}
%
The salient property of the electronic band structure of twisted bilayer graphene (TBG), at the so-called magic angle (MA), is the emergence of flat bands around the charge neutrality point. These bands are associated with the observed superconducting phases and the correlated insulating states occurring at the MA.
Scanning tunneling microscopy combined with angle resolved photoemission spectroscopy are usually used to visualize the flatness of the band structure of the TGB at the MA.
Here, we theoretically argue that spin pumping (SP) provides a direct probe of the flat bands of TBG and an accurate determination of the MA. 
We consider a junction separating a ferromagnetic insulator and an heterostructure of TBG adjacent to a monolayer of a transition metal dichalcogenide.
We show that the Gilbert damping of the ferromagnetic resonance experiment through this junction depends on the twist angle of TBG, and shows a sharp drop at the MA. Our results open the way to a twist switchable spintronics in twisted van der Waals heterostructures.
\end{abstract}

\maketitle

{\it Introduction. -- } Stacking two graphene layers with a relative twist angle $\theta$ results into a moir\'e superstructure which is found to host, in the vicinity of the so-called magic angle (MA) $\theta_M\sim1.1^{\circ}$, unconventional superconductivity and strongly correlated insulating states~\cite{Herrero1,Herrero2,Yank}. There is a general consensus that such strong electronic correlations originate from the moiré flat bands emerging at the MA around the charge neutrality point~\cite{Volovik,Senthil,Wu,Roy,Bernevig,Efetov,Young,Herrero3}.
The tantalizing signature of the flat bands have been experimentally demonstrated by probing the corresponding peaks of the density of states using transport~\cite{Herrero2,Pablo18,Dean,Efetov19,Dean19}, electronic compressibility measurements~\cite{Pablo19,Pablo20}, scanning tunneling microscopy (STM) and spectroscopy (STS)~\cite{Eva,Kerelsky,Yazdani19,Nadj19,Yazdani20,Nadj21,Eva2,Yazdani22}. The direct evidence of these flat bands has been reported by angle resolved photoemission spectroscopy (ARPES) measurements combined to different imaging techniques~\cite{Utama,Efetov21,Sato}.
However, spectroscopic measurements on magic-angle TBG raise many technical challenges related to the need of an accurate control of the twist angle, and the necessity to have non-encapsulated samples which can degrade in air.  \cite{Efetov21}.


Here we propose a non-invasive method to probe the flat bands of the TBG and accurately determine the MA. 
This method is based on spin pumping (SP) induced by ferromagnetic resonance (FMR)~\cite{Bauer1,Bauer2,SP-book,Hellman}, where the increase in the FMR linewidth, given by the Gilbert damping (GD) coefficient, provides insight into the spin excitations of the nonmagnetic (NM) material adjacent to the ferromagnet \cite{Qiu2016,Yang2018,Han2020}.
SP is expected to be efficient if the NM has high spin–orbit coupling (SOC) strength~\cite{Hait}. \newline
In our work, we consider spin injection from a ferromagnetic insulator (FI) into a TBG aligned on a monolayer of a transition metal dichalcogenides (TMD) which are considered as good substrate candidates to induce relatively strong SOC in graphene and TBG~\cite{Castro,Morpu15,Morpu16,Bock16,Casa,Shi,Wees,Eroms,Eroms2,Makk,BouchitaPRL,BLG,Omar,Valen,Roche,Zaletel,Wang,Bouchiat19,David,Zaletel20,Bouchiat21,Lin,Alex,Bhowmick}.

% \textcolor{blue}{Recent experimental studies revealed that SOC, induced in TBG adjacent to $\mathrm{WSe}_2$, gives rise to ordered phases even at non-integer moiré band fillings~\cite{Bhowmick}. It was also found that it stabilizes superconductivity at a twist angle ($\theta\sim0.8^{\circ}$) smaller than the MA~\cite{Alex} and it transforms, at a given filling factors, the correlated insulating state of TBG into a ferromagnetic phase~\cite{Lin}}.\
We theoretically study a planar junction of a FI and a TBG adjacent to $\mathrm{WSe}_2$ (TBG/$\mathrm{WSe}_2$) as depicted in figure~\ref{junction}. 
We consider the case where a microwave of a frequency $\Omega$ is applied to this junction, and focus on the role of the twist angle on the FMR linewidth of the FI.
We start by deriving the the continuum model of the heterostructure TBG/$\mathrm{WSe}_2$. Using a second-order perturbative theory, with respect to the interfacial exchange interaction~\cite{Ohnuma,Matsuo18,Kato19,Kato20,Matsuo-JP,Matsuo20,Yama}, we determine the twist angle dependence of the correction to the GD coefficient. \\

\begin{figure}[hpbt] 
\begin{center}
\includegraphics[width=0.9\columnwidth]{GD-structure.eps}
\end{center}
\caption{Schematic representation of the junction between a ferromagnetic insulator (FI) and an heterostructure of TBG adjacent to a monolayer of $\mathrm{WSe}_2$. The labels (1) and (2) denote the graphene layers of TBG represented by the red and the blue lines. The red arrow indicates the spin orientation of the FI characterized by an average spin,  written in the coordinate frame of the FI magnetization as $\langle\mathbf{S}_{FI} \rangle=\left(S_0,0,0\right)$.}
\label{junction}
\end{figure}
%
%
{\it Continuum model. -- } In TBG with a twist angle $\theta$, the Hamiltonian of a graphene layer $l$ ($l=1,2$), rotated at an angle $\theta_l$, is 
\begin{eqnarray}
h_{l,\text{rot}}(\mathbf{k})=e^{i\frac{\theta_l}2\sigma_z} h_{l}(\mathbf{k})e^{-i\frac{\theta_l}2\sigma_z}
\label{hl}
\end{eqnarray}
where $\theta_2=-\theta_1=\frac{\theta}2$, $h_{l}(\mathbf{k})$ is the monolayer Hamiltonian. $h_{1}(\mathbf{k})$ reduces to the graphene Hamiltonian given, in the continuum limit, by $h_{1}(\mathbf{k})=-\hbar v_F \mathbf{k}\cdot\mathbf{\sigma}^{\ast}$, where $v_F$ is the Fermi velocity, $\xi$ is the valley index, $\mathbf{\sigma}^{\ast}=\left(\xi \sigma_x,\sigma_y\right)$ and $\sigma_i$ ($i=x,y,z$) are the sublattice-Pauli matrices.\
The layer (2), in contact with $\mathrm{WSe}_2$ monolayer, is descried by the Hamiltonian~\cite{Alex}
\begin{eqnarray}
h_{2}(\mathbf{k})=h_{1}(\mathbf{k})+h_{\text{SOC}}+\frac{m}2\sigma_z
\label{h2}
\end{eqnarray}
where $h_{\text{SOC}}$ is given by
\begin{eqnarray}
h_{\text{SOC}}=\frac{\lambda_I}2\xi s_z+\frac{\lambda_R}2\left(\xi \sigma_x s_y-\sigma_y s_x\right)+\frac{\lambda_{KM}}2\xi \sigma_z s_z,
\label{hSOC}
\end{eqnarray}
$s_i$ ($i=x,y,z$) are the spin-Pauli matrices, $\lambda_I$, $\lambda_R$ and $\lambda_{KM}$ correspond, respectively, to the Ising, Rashba and Kane-Mele SOC parameters~\cite{Alex}. The variation ranges of these parameters are $\lambda_I\sim 1-5 \;\mathrm {meV}$, $\lambda_R\sim 1-15 \;\mathrm {meV}$, while $\lambda_{KM}$ is expected to be small~\cite{SOC1,SOC2,SOC3,SOC4,SOC5,SOC6,SOC7}. The last term in Eq.~\ref{h2} is due to the inversion symmetry breaking induced by the TMD layer. Hereafter, we neglect this term regarding the small value of $m$ compared to the SOC parameters~\cite{Alex}.\newline
As in the case of TBG~\cite{Mc11}, the low-energy Hamiltonian of TBG/$\mathrm{WSe}_2$ reduces, at the valley $\xi$, to

\begin{eqnarray}
H_{\xi,SOC}(\mathbf{k})= 
\begin{pmatrix}
h_{1}(\mathbf{k}) & T_1 & T_2 & T_3\\
T^{\dagger}_1 & h_{2,1}(\mathbf{k}) & 0&0\\
T^{\dagger}_2 & 0& h_{2,2}(\mathbf{k}) & 0\\
T^{\dagger}_3 & 0& 0& h_{2,3}(\mathbf{k})\\
\end{pmatrix}.
\label{HTBG}
\end{eqnarray}
$H_{\xi,\text{SOC}}(\mathbf{k})$ is written in the basis 
$\Psi=\left(\psi_0(\mathbf{k}),\psi_1(\mathbf{k}),\psi_2(\mathbf{k}),\psi_3(\mathbf{k})\right)$ constructed on the four-component spin-sublattice spinor $\psi_0(\mathbf{k})$ and $\psi_j(\mathbf{k})$, ($j=1,2,3$) corresponding, respectively, to layer $(1)$ and layer $(2)$ (see sections I and II of the supplemental material~\cite{supp}). The momentum $\mathbf{k}$ is measured relatively to the Dirac point $\mathbf{K}_{1\xi}$ of layer (1). In Eq.~\ref{HTBG}, $h_{2,j}\left(\mathbf{k}\right)=h_2\left(\mathbf{k}+\mathbf{q}_{j\xi}\right)$, ($j=1,2,3$) where $\mathbf{q}_{j\xi}$ are the vectors connecting $\mathbf{K}_{1\xi}$ to its three neighboring Dirac points $\mathbf{K}_{2\xi}$ of layer (2) in the moiré Brillouin zone (mBZ)~\cite{Mc11}, and are given by
$\mathbf{q}_{1\xi}=\mathbf{K}_{1\xi}-\mathbf{K}_{2\xi}$, 
 $\mathbf{q}_{2\xi}=\mathbf{q}_{1\xi}+\xi\mathbf{G}^M_1$, $\mathbf{q}_{3\xi}=\mathbf{q}_{1\xi}+\xi\left(\mathbf{G}^M_1+\mathbf{G}^M_2\right)$,
where $\left(\mathbf{G}^M_1,\mathbf{G}^M_2\right)$ is the mBZ basis (see section I of the supplemental material~\cite{supp}).\newline
We consider AA stacked moiré domains of TBG, where the flat bands are expected to emerge. The $T_j$ are the spin-independent interlayer coupling matrices given, in the unrelaxed TBG, by 
$T_1= w\left(\mathbb{I}_{\sigma}+\sigma_x\right)$, 
$T_2=w \left(\mathbb{I}_{\sigma}-\frac 12\sigma_x+\xi \frac {\sqrt{3}}2\sigma_y\right)$ and $T_3=w\left( \mathbb{I}_{\sigma}-\frac 12\sigma_x-\xi\frac {\sqrt{3}}2\sigma_y\right)$~\cite{supp}, where $w\sim 118\, \mathrm{meV}$~\cite{w} is the interlayer tunneling amplitude and $\mathbb{I}_{\sigma}$ is the identity matrix acting on the sublattice indices.\

Using the perturbative approach of Ref.~\cite{Mc11}, we derive, from Eq.~\ref{HTBG}, the effective low-energy Hamiltonian $H_{\xi,\text{SOC}}^{(1)}(\mathbf{k})$ of TBG/$\mathrm{WSe_2}$ (see section II of the supplemental material~\cite{supp}). To the leading order in $\mathbf{k}$, $H_{\xi,\text{SOC}}^{(1)}(\mathbf{k})$ reads as~\cite{supp}
\begin{widetext}
\begin{eqnarray}
&& H_{\xi,\text{SOC}}^{(1)}(\mathbf{k})=\frac{\langle\Psi|H_{\xi,\text{SOC}}|\Psi\rangle}{\langle\Psi|\Psi\rangle}=
 \psi^{\dagger}_0 \left[h_{\text{eff}}\left(\mathbf{k}\right) + h_{\text{eff}}^{\text{SOC}} \right]\psi_0,
 \label{H1}\\
 &&h_{\text{eff}}\left(\mathbf{k}\right)=-\frac{\hbar v}{\langle\Psi|\Psi\rangle}
 \left\{ k_x\left[ \left(1-3\alpha^2\right) \xi \sigma_x \mathbb{I}_{s}
 -\frac{3\alpha^2}{\hbar v q_0}\left( \xi \lambda_I \sigma_y s_z+\lambda_R\left(\xi \sigma_ys_y-\sigma_x s_x\right)
 \right)\right]\right.\nonumber\\
 &&\hspace{2.8cm}\left.+k_y \left[ \left(1-3\alpha^2\right) \sigma_y \mathbb{I}_{s}
 -\frac{3\alpha^2}{\hbar v q_0}\left( - \lambda_I \sigma_x s_z+\lambda_R\left(\sigma_xs_y+\xi \sigma_y s_x\right)
 \right)\right]
 \right\},
 \label{heff}\\
 &&h_{\text{eff}}^{\text{SOC}}=\frac{3\alpha^2}{\langle\Psi|\Psi\rangle}\left[ \xi \lambda_Is_z\mathbb{I}_{\sigma}+\frac {\lambda_R}2\left( s_x\sigma_y-\xi s_y\sigma_x\right) \right],
 \label{heff_SOC}
\end{eqnarray}
\end{widetext}
$\langle\Psi|\Psi\rangle\sim 1+6\alpha^2$, $\alpha=\frac{w}{\hbar v_F q_0}$, $q_0=|\mathbf{q}_{j\xi}|=\frac{4\pi}{3a}\theta$, $a$ is the graphene lattice constant and
 $\sigma_i$, ($i=x,y,z$) act now on the band indices $\sigma=\pm$ of the eigenergies of $H_{\xi,\text{SOC}}^{(1)}$ denoted $E_{\sigma,\pm}$ and given, to the leading orders in $\mathbf{k}$ and $\frac{\lambda_{I,R}}{\hbar v_Fq_0}$, by
\begin{eqnarray}
E(\mathbf{k}))_{\sigma,\pm}= \frac {\sigma}{\langle \Psi |\Psi\rangle }
\sqrt{f_1(\mathbf{k})\pm 6 \alpha^2\sqrt{f_2(\mathbf{k})}}
\label{energy_H1}
\end{eqnarray}
\begin{eqnarray}
&&f_1(\mathbf{k})=(\hbar v)^2\left(1-3\alpha^2\right)^2||{\mathbf{k}}||^2+\frac 92\alpha^4\left(2\lambda_I^2+\lambda_R^2\right)\nonumber\\
&&f_2(\mathbf{k})= (\hbar v)^2\left(1-3\alpha^2\right)^2||{\mathbf{k}}||^2\left(\lambda_I^2+\frac 14\lambda_R^2\right)+ \frac9{16} \alpha^4 \lambda_R^4.\nonumber
\end{eqnarray}
Equation~\ref{heff_SOC} shows that the SOC parameters $\lambda_I$ and $\lambda_R$ are renormalized by the moiré structure of TBG to 
\begin{eqnarray}
 \tilde{\lambda}_I=\frac{6\alpha^2}{\langle\Psi|\Psi\rangle}\lambda_I,\;
 \tilde{\lambda}_R=\frac{3\alpha^2}{\langle\Psi|\Psi\rangle}\lambda_R.
 \label{SOCeff}
\end{eqnarray}
%
The expression of $H_{\xi,\text{SOC}}^{(1)}$ (Eq.~\ref{H1}) can be taken as a starting point to unveil the role of SOC in the emergence of the stable superconducting phase observed, at $\theta\sim 0.8^{\circ}$, in TBG adjacent to $\mathrm{WSe}_2$~\cite{Alex}.\

To derive the GD, one needs to determine the magnon self-energy of the FI, resulting from the interfacial exchange interaction between the FI and the heterostructure TBG/$\mathrm{WSe}_2$.\\

{\it Gilbert damping. -- } In the absence of a junction, the magnon Green function of the FI is defined as~\cite{Yama} 
\begin{eqnarray}
G_0\left(\mathbf{q},i\omega_n\right)=\frac{2S_0/\hbar}{i\omega_n-\omega_{\mathbf{q}}+i\alpha_G \omega_n}
\end{eqnarray}
where $\omega_n=2\pi n/\hbar \beta$ are the Matsubara frequencies for bosons, $S_0$ is the amplitude of the average spin per site, $\alpha_G\sim 10^{-4}-10^{-3}$ is the GD strength.
In the presence of the interfacial coupling, the magnon Green function is given by the Dyson equation (see section IV of the supplemental material)
\begin{eqnarray}
G_0\left(\mathbf{q},i\omega_n\right)=\frac{1}{G_0(\mathbf{q},i\omega_n)^{-1}-\Sigma(\mathbf{q},i\omega_n)},
\label{G}
\end{eqnarray}
where $\Sigma\left(\mathbf{q},i\omega_n\right)$ is the magnon self-energy resulting from the interfacial exchange interaction.
For the sake of simplicity, we neglect the real part of $\Sigma\left(\mathbf{q},i\omega_n\right)$ related to the shift of the FMR.\
We also assume that the interface is clean which turns out to consider that,
in the FMR experiment, the microwave irradiation induces a uniform spin precession, which limits the magnon self-energy to the processes with $\mathbf{q}=0$~\cite{Funato}.\

In the second order perturbation with respect to the interfacial exchange interaction $T_\mathbf{q}$, the magnon self-energy (Eq.~\ref{G}) is written as~\cite{Yama}
%\begin{widetext}
\begin{eqnarray}
\Sigma (\mathbf{q},i\omega_n)&=&\frac{|T_\mathbf{q}|^2}{4\beta}\sum_{\mathbf{k},i\omega_m}
\mathrm{Tr}\left[ \sigma_s^{x^{\prime},-}\; \hat{g} (\mathbf{k},i\omega_m)\right.\nonumber\\
&\times&\left.\sigma_s^{x^{\prime},+}\;\hat{g} (\mathbf{k}+\mathbf{q},i\omega_m+i\omega_n)
\right].
\label{self}
\end{eqnarray}
%\end{widetext}
$\sigma_s^{x^{\prime},\pm}$ are the electronic spin ladder operators written in the coordinate system $\left(x^{\prime},y^{\prime},z^{\prime}\right)$ of the FI magnetization characterized by an average spin $\langle\mathbf{S}_{FI}  \rangle=\left(S_0,0,0\right)$ (see section IV of the supplemental material~\cite{supp}).
$\hat{g}(\mathbf{k},i\omega_m)$ is the electronic Matsubara Green function given by 
\begin{eqnarray}
\hat{g}(\mathbf{k},i\omega_n)=\left[ i\omega_n \mathbb{I}- H_{\text{SOC}}^{(1)}(\mathbf{k})\right]^{-1}
\end{eqnarray}
where $\omega_n=(2n+1)\pi/\hbar \beta$ are the fermionic Matsubara frequencies.\

In the basis of the spin-band four-component spinor $\Psi=\left( \psi_{+,\uparrow},\psi_{+,\downarrow},\psi_{-,\uparrow},\psi_{-,\downarrow}\right)$~\cite{Alex}, $\hat{g}(\mathbf{k},i\omega_n)$ reads as
\begin{eqnarray}
\hat{g}(\mathbf{k},i\omega_n)=\hat{g}_0(\mathbf{k},i\omega_n)\mathbb{I}_s+
\mathbf{\hat{g}}(\mathbf{k},i\omega_n)\cdot \mathbf{s}
\end{eqnarray}
where $\mathbf{s}=(s_x,s_y,s_z) $ are the spin-Pauli matrices, $\mathbf{\hat{g}}=(\hat{g}_x,\hat{g}_y,\hat{g}_z)$, $\hat{g}_0$ and $\hat{g}_i$ ($i=x,y,z$) are expressed, to the leading order in the SOC, in terms of the band-Pauli matrices $\sigma_i$ (see section III of the supplemental material~\cite{supp}).\

Using the analytical continuation $i\omega_n\rightarrow \omega+i\delta$ in Eq.~\ref{G}, we obtain the magnon retarded Green function
\begin{eqnarray}
&&G_0^R\left(\mathbf{q},i\omega_n\right)=\frac{2S_0/\hbar}{\omega-\omega_{\mathbf{q=0}}+i\left(\alpha_G+\delta\alpha_G\right)\omega},\\
&&\delta\alpha_G\left(\omega\right)\equiv -\frac{2S_0}{\hbar \omega}\mathrm{Im}\Sigma^R\left(\mathbf{q=0},\omega\right)
\label{GD-exp}
\end{eqnarray}
$\delta\alpha_G$ is the correction to the GD due to the interfacial exchange interaction between the FI and the TBG/$\mathrm{WSe_2}$. In general, $\delta\alpha_G$ depends on $\omega$. However, since the ferromagnetic peak, given by $\mathrm{Im}G_0^R$, is sharp enough, namely $\alpha_G+\delta\alpha_G\ll 1$, one can replace $\omega$ by the FMR frequency $\Omega$
\begin{eqnarray}
 \delta\alpha_G\sim -\frac{2S_0}{\hbar \Omega}\mathrm{Im}\Sigma^R\left(\mathbf{q=0},\Omega\right)
 \label{GDfinal}
\end{eqnarray}
Carrying out the summation over $\omega_m$ in Eq.~\ref{self}, we obtain analytical expressions of the magnon self-energy (see section IV of the supplemental material~\cite{supp}). The sum over the electronic states $\mathbf{k}=\left(k,\varphi_{\mathbf{k}}\right)$ runs over the states included within a cutoff, $k_c=q_0/2$, on the momentum amplitude $k$, where the Dirac cone description of the monolayer Hamiltonian is expected to hold.\

In the Following, we discuss the behavior of the normalized GD coefficient 
\begin{eqnarray}
 \delta\alpha_G/\alpha_G^0=\left(\frac{\lambda}{\hbar\Omega} \right)^2 \tilde{\Sigma}\left(\mathbf{q}=\mathbf{0},\Omega\right),
 \label{alpha_R}
\end{eqnarray}
where $\delta\alpha_G$ is given by Eq.~\ref{GDfinal},
$\tilde{\Sigma}$ is a dimensionless function depending on the twist angle $\theta$, temperature $T$, the chemical potential $\mu$ and the orientation of the FI magnetization, $\alpha_G^0=2S_0\left(\frac{|T_\mathbf{0}|}{\lambda}\right)^2$ and $\lambda=\frac{\lambda_I+\lambda_R}2$ is the average SOC (for details, see section IV of the supplemental material~\cite{supp}).\\

%%%%%%%%%%%%%%%%%%%%%%%%
%%%%%%%%%%%%%%%%%%%%%%%%
%%%%%%%%%%%%%%% discussion%%%%%%%%%%%%%%%%%%%%
%
{\it Discussion. -- } In figure~\ref{GD2}, we plot $\delta\alpha_G/\alpha_G^0$ (Eq.\ref{alpha_R}), as a function of the twist angle $\theta$, for the undoped TBG, at different temperatures and for a fixed FMR energy $\hbar \Omega=0.06\,\mathrm{meV}$ which corresponds to the yttrium iron garnet. The SOC parameters are $\lambda_I=3 \,\mathrm{meV}$ and $\lambda_I=4\, \mathrm{meV}$ as in Ref.\onlinecite{Alex}.\\

\begin{figure}[hpbt] 
\begin{center}
\includegraphics[width=0.8\columnwidth]{GD-theta.eps}
\end{center}
\caption{Normalized GD, $\delta\alpha_G/\alpha_G^0$ (Eq.\ref{alpha_R}), as a function of the twist angle at different temperature ranges. Calculations are done for $\lambda_I=3\;\mathrm{meV}$, $\lambda_R=4\;\mathrm{meV}$, $\mu=0$ and for a FMR energy $\hbar\Omega=0.06\;\mathrm{meV}$. }
\label{GD2}
\end{figure}
Figure~\ref{GD2} shows that, regardless of the temperature range, $\delta\alpha_G$ increases by decreasing $\theta$ but drops sharply at the MA where it exhibits a relatively small peak, which is smeared out at low temperature.\

Putting aside its drop at the MA, the enhancement of $\delta\alpha_G$, by decreasing $\theta$, results from the dependence of the magnon self-energy (Eq.~\ref{self}) on the effective SOC (Eq.~\ref{SOCeff}) which increase as $\frac1{\theta^2}$ by decreasing $\theta$ (see section II of the supplemental material~\cite{supp}).\

%
To understand the behavior of $\delta\alpha_G$ at the MA one needs to go back to the band structure, $E_{\sigma,\pm}(\mathbf{k})$ (Eq.~\ref{energy_H1}), of the continuum Hamiltonian of TBG/$\mathrm{WSe_2}$, which is depicted in Fig.~\ref{band} at different twist angles. The arrows indicate the out-of-plane electronic spin projection $\langle s_z\rangle$ which has been numerically calculated, in Ref.\onlinecite{Alex}, within the perturbative approach~\cite{Mc11}, and which can be deduced from Eq.~\ref{heff_SOC}.\

\begin{widetext}
\begin{figure*}[hpbt] 
\begin{center}
$
\begin{array}{cccc}
\includegraphics[width=0.5\columnwidth]{energy-plot-theta=0,5-MA}
\includegraphics[width=0.5\columnwidth]{energy-plot-MA-.eps}
\includegraphics[width=0.5\columnwidth]{energy-plot-MA+.eps}
\includegraphics[width=0.5\columnwidth]{energy-plot-theta=1,2-MA}
\end{array}
$
\end{center}
\caption{Band structure of TBG/$\mathrm{WSe_2}$ in the continuum limit (Eq.~\ref{energy_H1}) at (a) $\theta=0.5^{\circ}$, (b) $\theta=\theta^-_M=1.043^{\circ}$, (c) $\theta=\theta^+_M=1.058^{\circ}$ and $\theta=1.2^{\circ}$. The dashed lines represent the bands at the MA ($\theta_M=1.05^{\circ}$). The red (blue) arrows correspond to the out-of-plane electronic spin projection $\langle s_z\rangle=+1$ ($\langle s_z\rangle=-1$)~\cite{Alex}.}
\label{band}
\end{figure*}
\end{widetext}

Away from the MA, the band dispersion gets larger as $\theta$ decreases and, in particular, the separation between bands with opposite $\langle s_z\rangle$, involved in the SP process, increases. This behavior is due to the angle dependence of the effective Fermi velocity of TBG $v^{\ast}\sim v_F\frac{1-3\alpha^2}{1+6\alpha^2}$ (see sections I and II of the supplemental Material~\cite{supp}).\newline
However, around the MA at $\theta^-_M$ and $\theta^+_M$, the bands $E_{+,-}$ and $E_{-,-}$ get closer, compared to those at $\theta_M$.\newline
% 
The expression of the GD (Eq.~\ref{alpha_R}) includes transitions between bands with opposite $\langle s_z\rangle$ (see section IV of the supplemental material~\cite{supp}). These transitions depend on the statistical weight $\Delta f(E)=f(E_{\langle s_z\rangle})-f(E_{-\langle s_z\rangle})$ where $f(x)$ is the Fermi-Dirac function and $E_{\langle s_z\rangle}$ is the energy band with a spin orientation $\langle s_z\rangle$. \

\begin{figure}[hpbt] 
\begin{center}
\includegraphics[width=0.8\columnwidth]{GD-interp.eps}
\end{center}
\caption{Schematic representation of the band structure $E_{\sigma,\pm}$ (Eq.~\ref{energy_H1}) and the Fermi-Dirac distribution $f(E)$. The bands in dashed and green lines correspond, respectively, to the MA and to an angle $\theta$ far from the MA. The red (blue) arrows represent the projection of the out-of-plane spin projection $\langle S_z\rangle=+1$ ($\langle S_z\rangle=-1$).
Around the MA, the bands are almost flat and the statistical weights $\Delta f(E)$, corresponding to the transitions between 
$E_{-,+}\rightarrow E_{+,+}$ and $E_{-,-}\rightarrow E_{+,-}$, are small compared to the case of the twist angle away from the MA, where the band dispersion is larger. As a consequence, the correction to the Gilbert damping $\delta\alpha_G$, which depends on $\Delta f(E)$, is reduced at the MA (for details, see section IV of the supplemental material).
}
\label{FD}
\end{figure}
In Fig.~\ref{FD}, we plot a schematic representation of the band structure of the continuum model (Eq.\ref{energy_H1}) and the Fermi-Dirac distribution $f(E)$ at a given temperature $T$. The dashed (green) line corresponds to a twist angle $\theta$ close to (far from) the MA (for details, see section IV of the supplemental material). 
The band dispersion gets larger as $\theta$ moves away from the MA (Fig.~\ref{band}) and
the separation between the bands with opposite $\langle S_z\rangle$ increases.
As a consequence, the corresponding statistical weight $\Delta f(E)$ is enhanced compared to the case around the MA. 
This behavior explains the drop of the GD at the MA.\

For twist angles around the MA ($\theta^+_M$ and $\theta^-_M$), the separating between the bands $E_{-,-}$ and $ E_{+,-}$ is smaller compared to the case at the MA (Fig.~\ref{band}), which reduces the statistical weight $\Delta f(E)$. This behavior gives rise to the small peak at the MA (Fig.~\ref{GD2}), which vanishes at low temperature ($k_BT<\lambda$) where bands around the MA have the same statistical weight $\Delta f(E)= 1$.\


It comes out that the twist angle dependence of $\delta\alpha_G$ is a direct probe of the emergence of the flat bands in the TBG. On the other hand, the temperature dependence of the fine structure around the MA provides, not only, an accurate measurement of the MA but also an estimation of the SOC induced in TBG adjacent to a monolayer of TMD.\newline  
  
%%%%%%%%%%%%%
{\it Conclusion. -- } To conclude, we have proposed an experiment to probe the flat bands of TBG and to measure accurately its MA. The experiment is based on a spin pumping measurement through a junction separating a FI and a TBG adjacent to a monolayer of $\mathrm{WSe_2}$.
We first derived the continuum model of TBG with SOC, which constitutes a first step to develop an analytical understanding of the emergence of a stable superconducting state at small twist angle observed in TBG in proximity to $\mathrm{WSe_2}$~\cite{Alex}.
Our results show that, the twist angle dependence of the GD correction, $\delta\alpha_G$ exhibits a drop at the MA with a temperature-dependent fine structure. The latter provides an accurate determination of the MA and an estimation of the SOC induced in TBG by its proximity to the TMD layer.
Our work opens the gate to a twist tunable spintronics in twisted layered heterostructures.
\\


{\it Acknowledgment. -- } 
 We thank Prof. Mamoru Matsuo for stimulating discussions.
S. H. acknowledges the kind hospitality of ISSP where this work was carried out. S. H. was supported by the ISSP visiting professor programme.



%%%%%%%%%%%%%%%%%%%%%%%%%%%%%%%%%%%%%%%%%%%%%%%%%%%%%%%%%%%%%%%%%%%
\begin{thebibliography} {200}

\bibitem{Herrero1} Y. Cao, V. Fatemi, S. Fang, K. Watanabe, T. Taniguchi, E. Kaxiras, and P. Jarillo-Herrero, Nature, {\bf 556}, 43 (2018).

\bibitem{Herrero2} Y. Cao, V. Fatemi, A. Demir, S. Fang, S. L. Tomarken, J. Y. Luo, J. D. Sanchez-Yamagishi, K. Watanabe, T. Taniguchi, E. Kaxiras, R. C. Ashoori, and P. Jarillo-Herrero, Nature, {\bf 556}, 80 (2018).

\bibitem{Yank} M. Yankowitz, S. Chen, H. Polshyn, Y. Zhang, K. Watanabe, T. Taniguchi, D. Graf, A. F. Young, C. R. Dean, Science {\bf 363}, eaav1910 (2019).

\bibitem{Volovik} N. B. Kopnin, T. T. Heikkila, and G. E. Volovik, Phys. Rev. B {\bf 83}, 220503(R) (2011).

\bibitem{Senthil} H. C. Po, L. Zou, A. Vishwanath, and T. Senthil, Phys. Rev. X {\bf 8}, 031089 (2018).

\bibitem{Wu} F. Wu, A. H. MacDonald, and I. Martin, Phys. Rev. Lett. {\bf 121}, 257001 (2018).

\bibitem{Roy} B. Roy and V. Juricic, Phys. Rev. B {\bf 99}, 121407(R) (2019).

\bibitem{Bernevig} B. Lian, Z. Wang, and B. A. Bernevig, Phys. Rev. Lett. {\bf 122}, 257002 (2019).

\bibitem{Efetov} P. Stepanov, I. Das, X. Lu, A. Fahimniya, K. Watanabe, T. Taniguchi, F. H. L. Koppens, J. Lischner, L. Levitov and D. K. Efetov, Nature {\bf 583}, 375 (2020).

\bibitem{Young} Y. Saito, J. Ge, K. Watanabe, T. Taniguchi, A. F. Young, Nature Physics {\bf 16}, 926 (2020).

\bibitem{Herrero3} Y. Cao, D. Rodan-Legrain, J. M. Park, F. N. Yuan, K. Watanabe, T. Taniguchi, R. M. Fernandes, L. Fu, P. Jarillo-Herrero, Science, {\bf 372}, 264 (2021).

\bibitem{Pablo18} Y. Cao, V. Fatemi, S. Fang, K. Watanabe, T. Taniguchi, E. Kaxiras and P. Jarillo-Herrero, Nature {\bf 556}, 43 (2018).


\bibitem{Dean} M. Yankowitz, S. Chen, H. Polshyn, Y. Zhang, K.
Watanabe, T. Taniguchi, D. Graf, A. F. Young, and C. R.
Dean, Science {\bf 363}, 1059 (2019).

\bibitem{Efetov19} X. Lu, P. Stepanov, W. Yang, M. Xie, M. A. Aamir, I. Das, C. Urgell, K. Watanabe, T. Taniguchi, G. Zhang, A.
Bachtold, A. H. MacDonald, and D. K. Efetov, Nature
{\bf 574}, 653 (2019).

\bibitem{Dean19} H. Polshyn, M. Yankowitz, S. Chen, Y. Zhang, K.
Watanabe, T. Taniguchi, C. R. Dean, and A. F. Young,
Nat. Phys. {bf 15}, 1011 (2019).

\bibitem{Pablo20} Y. Cao, D. Chowdhury, D. Rodan-Legrain, O.
Rubies-Bigorda, K. Watanabe, T. Taniguchi, T. Senthil,
and P. Jarillo-Herrero, Phys. Rev. Lett. {\bf 124}, 076801
(2020).

\bibitem{Pablo19} S. L. Tomarken, Y. Cao, A. Demir, K. Watanabe, T. Taniguchi, P. Jarillo-Herrero, and R. C. Ashoori
Phys. Rev. Lett. {\bf 123}, 046601 (2019).

\bibitem{Eva} G. Li, A. Luican, J. M. B. Lopes dos Santos, A. H. Castro Neto, A. Reina, J. Kong and E. Y. Andrei, Nature Physics, {\bf 6}, 109 (2010).

\bibitem{Kerelsky} A. Kerelsky, L. J. McGilly, D. M. Kennes, L. Xian, M. Yankowitz, S. Chen, K. Watanabe,
T. Taniguchi, J. Hone, C. Dean, A. Rubio and A. N. Pasupathy, Nature, {572}, 95 (2019).

\bibitem{Yazdani19} Y. Xie, B. Lian, B. Jäck, X. Liu, C.-L. Chiu, K. Watanabe, T. Taniguchi, B. A. Bernevig, and A. Yazdani, Nature
{\bf 572}, 101 (2019).

\bibitem{Nadj19} Y. Choi, J. Kemmer, Y. Peng, A. Thomson, H. Arora, R. Polski, Y. Zhang, H. Ren, J. Alicea, G. Refael, F. von
Oppen, K. Watanabe, T. Taniguchi, and S. Nadj-Perge,
Nat. Phys. {\bf 15}, 1174 (2019).

\bibitem{Yazdani20} D. Wong, K. P. Nuckolls, M. Oh, B. Lian, Y. Xie, S. Jeon, K. Watanabe, T. Taniguchi, B. A. Bernevig, and A.
Yazdani, Nature {\bf 582}, 198 (2020).

\bibitem{Nadj21}Y. Choi, H. Kim, Y. Peng, A. Thomson, C. Lewandowski, R. Polski, Y. Zhang, H. S. Arora, K. Watanabe, T.
Taniguchi, J. Alicea, and S. Nadj-Perge, Nature
{\bf 589}, 536 (2021).

\bibitem{Eva2} N. Tilak, X. Lai, S. Wu, Z. Zhang, M. Xu, R. de Almeida Ribeiro, P. C. Canfield and E. Y. Andrei, Nat Commun {bf 12}, 4180 (2021). 

\bibitem{Yazdani22} D. C\u{a}lug\u{a}ru, N. Regnault, M. Oh, K. P. Nuckolls, D. Wong, R. L. Lee, A. Yazdani, O. Vafek, and B. A. Bernevig, Phys. Rev. Lett. {\bf 129}, 117602 (2022).

\bibitem{Utama} M. I. B. Utama, R. J. Koch, K. Lee, {\it et al.} Nat. Phys. {\bf 17}, 184 (2021).

\bibitem{Efetov21} S. Lisi, X. Lu, T. Benschop, {\it et al.}, Nat. Phys. {bf 17}, 189 (2021).

\bibitem{Sato} K. Sato, N. Hayashi, T. Ito,{\it et al.}, Commun Mater {\bf 2}, 117 (2021).

%%%%%%%%%%%%%%%% SP%%%%%%%%%%
\bibitem{Bauer1} Y. Tserkovnyak, A. Brataas, and G. E. W. Bauer, Phys. Rev. Lett. {\bf 88}, 117601 (2002).

\bibitem{Bauer2} Y. Tserkovnyak, A. Brataas, G. E. W. Bauer, and B. I. Halperin, Rev. Mod. Phys. {\bf 77}, 1375 (2005).

\bibitem{SP-book} M. Sadamichi and others (eds.), Spin Current, 1st edn, Series on Semiconductor Science and Technology  (Oxford, 2012; online edn, Oxford Academic, 17 Dec. 2013).

\bibitem{Hellman} F. Hellman, A. Hoffmann, Y. Tserkovnyak, G. S. D. Beach, E. E. Fullerton, C. Leighton, A. H. MacDonald, D. C. Ralph, D. A. Arena, H. A. D\"urr, P. Fischer, J. Grollier, J. P. Heremans, T. Jungwirth, A. V. Kimel, B. Koopmans, I. N. Krivorotov, S. J. May, A. K. Petford-Long, J. M. Rondinelli, N. Samarth, I. K. Schuller, A. N. Slavin, M. D. Stiles, O. Tchernyshyov, A. Thiaville, and B. L. Zink, Rev. Mod. Phys. {\bf 89}, 025006 (2017).

\bibitem{Qiu2016} Z. Qiu, J. Li, D. Hou, E. Arenholz, A. T. N'Diaye, A. Tan, K.-i. Uchida, K. Sato, S. Okamoto, Y. Tserkovnyak, Z. Q. Qiu, and E. Saitoh, Nat. Commun. {\bf 7}, 12670 (2016).

\bibitem{Yang2018} F. Yang and P. C. Hammel, J. Phys. D Appl. Phys. {\bf 51}, 253001 (2018).

\bibitem{Han2020} W. Han, S. Maekawa, and X. Xie, Nat. Mater. {\bf 19}, 139 (2020).

\bibitem{Hait} S. Hait {\it et al.}, ACS Appl. Mater. Interfaces {\bf 14}, 37182 (2022).

%%%%%%%%%%%%%%%%%% Gr/TMD %%%%%%%%%%%%%%%%%%%
\bibitem{Castro} A. Avsar, J. Y. Tan, T. Taychatanapat, J. Balakrishnan, G. K. W.
Koon, Y. Yeo, J. Lahiri, A. Carvalho, A. S. Rodin, E. C. T.
O’Farrell, G. Eda, A. H. Castro Neto, and B. Özyilmaz, Nat.
Commun. {\bf 5}, 4875 (2014).

\bibitem{Morpu15} Z. Wang, D.-K. Ki, H. Chen, H. Berger, A. H. MacDonald, and A. F. Morpurgo, Nat. Commun. {\bf 6}, 8339 (2015).

\bibitem{Morpu16} Z. Wang, D.-K. Ki, J. Y. Khoo, D. Mauro, H. Berger, L. S. Levitov, and A. F. Morpurgo, Phys. Rev. X {\bf 6}, 041020 (2016).

\bibitem{Bock16} B. Yang, M.-F. Tu, J. Kim, Y. Wu, H. Wang, J. Alicea, R. Wu, M. Bockrath, and J. Shi, 2D Mater. {\bf 3}, 031012 (2016).

\bibitem{Casa} W. Yan, O. Txoperena, R. Llopis, H. Dery, L. E. Hueso, and F. Casanova, Nat. Commun. {\bf 7}, 13372 (2016).


\bibitem{Shi} B. Yang, M. Lohmann, D. Barroso, I. Liao, Z. Lin, Y. Liu, L. Bartels, K. Watanabe, T. Taniguchi, and J. Shi, Phys. Rev. B {\bf 96}, 041409(R) (2017).

\bibitem{Wees} T. S. Ghiasi, J. Ingla-Aynés, A. A. Kaverzin, and B. J. van Wees, Nano Lett. {\bf 17}, 7528 (2017).

\bibitem{Eroms} A. Dankert and S. P. Dash, Nat. Commun. 8, 16093 (2017).
\bibitem{Eroms2} T. V\"{o}lkl, T. Rockinger, M. Drienovsky, K. Watanabe, T. Taniguchi, D. Weiss, and J. Eroms, Phys. Rev. B {\bf 96}, 125405 (2017).

\bibitem{Makk} S. Zihlmann, A. W. Cummings, J. H. Garcia, M.  Kedves, K. Watanabe, T. Taniguchi, C. Sch\"{o}nenberger, and P. Makk, Phys. Rev. B {\bf 97}, 075434 (2018).


\bibitem{BouchitaPRL} T. Wakamura, F. Reale, P. Palczynski, S. Guéron, C. Mattevi, and H. Bouchiat, Phys. Rev. Lett. {\bf 120}, 106802 (2018).

\bibitem{BLG} J. C. Leutenantsmeyer, J. Ingla-Aynés, J. Fabian, and B. J. van Wees, Phys. Rev. Lett. {\bf 121}, 127702 (2018).


\bibitem{Omar} S. Omar and B. J. van Wees, Phys. Rev. B {\bf 97}, 045414 (2018).


\bibitem{Valen} L. A. Ben\'{\i}tez, J. F. Sierra, W. S. Torres, A. Arrighi, F. Bonell, M. V. Costache, and S. O. Valenzuela, Nat. Phys. {\bf 14}, 303 (2018).

\bibitem{Roche} C. K. Safeer, J. Ingla-Aynés, F. Herling, J. H. Garcia, M. Vila, N. Ontoso, M. R. Calvo, S. Roche, L. E. Hueso, and F. Casanova, Nano Lett. {\bf 19}, 1074 (2019).


\bibitem{Zaletel} J. O. Island, X. Cui, C. Lewandowski, J. Y. Khoo, E. M. Spanton, H. Zhou, D. Rhodes, J. C. Hone, T. Taniguchi, K. Watanabe, L. S. Levitov, M. P. Zaletel and A. F. Young, Nature {\bf 571}, 85 (2019).

\bibitem{Wang} D. Wang, S. Che, G. Cao, R. Lyu, K. Watanabe, T. Taniguchi, C. Ning Lau, and M. Bockrath, Nano Lett. {\bf 19}, 7028 (2019).

\bibitem{Bouchiat19} T. Wakamura, F. Reale, P. Palczynski, M. Q. Zhao, A. T. C. Johnson, S. Guéron, C. Mattevi, A. Ouerghi, and H. Bouchiat, Phys. Rev. B {\bf 99}, 245402 (2019).

\bibitem{David} A. David, P. Rakyta, A. Korm\'{a}nyos, and G. Burkard
Phys. Rev. B {bf 100}, 085412(2019).

\bibitem{Zaletel20} T. Wang, N. Bultinck, and M. P. Zaletel
Phys. Rev. B {\bf 102}, 235146 (2020).

\bibitem{Bouchiat21} For a rieview see, T. Wakamura, S. Gu\'eron and H. Bouchiat, Comptes Rendus. Physique, {\bf 22}, 145 (2021).


%%%%%%%%%%%%%%%%%%%%%%%%%%%%

\bibitem{Alex} H. S. Arora, R. Polski, Y. Zhang, A. Thomson,
Y. Choi, H. Kim, Z. Lin, I. Z. Wilson, X. Xu,
J. -H. Chu, K. Watanabe, T. Taniguchi, J. Alicea and S. Nadj-Perge, Nature {\bf 583} 379 (2020).

\bibitem{Lin} J.-X. Lin, Y.-H. Zhang, E. Morissette, Z. Wang, S. Liu, D. Rhodes, K. Watanabe, T. Taniguchi, J. Hone, J. I. A. Li, Science {\bf 375}, 437 (2022).

\bibitem{Bhowmick} S. Bhowmik, B. Ghawri, Y. Park, D. Lee, S. Datta, R. Soni, K. Watanabe, T. Taniguchi, A. Ghosh, J. Jung, U. Chandni, arXiv:2211.01251.

\bibitem{Ohnuma} Y. Ohnuma, H. Adachi, E. Saitoh, and S. Maekawa
Phys. Rev. B {\bf 89}, 174417 (2014).

\bibitem{Matsuo18} M. Matsuo, Y. Ohnuma, T. Kato, and S. Maekawa, Phys. Rev. Lett. {\bf 120}, 037201 (2018).

\bibitem{Kato19} T. Kato, Y. Ohnuma, M. Matsuo, J. Rech, T. Jonckheere, and T. Martin, Phys. Rev. B {\bf 99}, 144411 (2019).

\bibitem{Kato20} T. Kato, Y. Ohnuma, and M. Matsuo, Phys. Rev. B {\bf 102}, 094437 (2020).

\bibitem{Matsuo-JP} Y. Ominato and M. Matsuo, J. Phys. Soc. Jpn. {\bf 89}, 053704 (2020).

\bibitem{Matsuo20} Y. Ominato, J. Fujimoto, and M. Matsuo, Phys. Rev. Lett. 124, 166803 (2020).

\bibitem{Yama} M. Yama, M. Tatsuno, T. Kato, and M. Matsuo
Phys. Rev. B {\bf 104}, 054410 (2021).


\bibitem{SOC1} M. Gmitra, and J. Fabian, Phys. Rev. B {\bf 92}, 155403 (2015).

\bibitem{SOC2} M. Gmitra, D. Kochan, P. Hogl, and J. Fabian, Phys. Rev. B {\bf 93}, 155104 (2016).

\bibitem{SOC3} S. Zihlmann, {\it et al.}, Phys. Rev. B {\bf 97}, 075434 (2018).

\bibitem{SOC4} J. O. Island, {\it et al.}, Nature {\bf 571}, 85 (2019).

\bibitem{SOC5} D. Wang, {\it et al.}, Nano Lett. {\bf 19}, 7028 (2019).

\bibitem{SOC6} Z. Wang {\it et al.}, Phys. Rev. X {\bf 6}, 041020 (2016).

\bibitem{SOC7} A. W. Cummings, J. H. Garcia, J. Fabian, J. and S. Roche, Phys. Rev. Lett. {\bf 119}, 206601 (2017).

\bibitem{Mc11} R. Bistritzer and A. H. MacDonald, Proc. Natl. Acad. Sci.U.S.A., {\bf 108}, 12233 (2011).

\bibitem{supp} For details, see the Supplemental Material.

\bibitem{w} G. Cantele, D.Alf{\`e}, F. Conte, V. Cataudella, D. Ninno, and P. Lucignano, Phys. Rev. Research {\bf 2}, 043127 (2020).

\bibitem{Funato}T. Funato, T. Kato, and M. Matsuo,
Phys. Rev. B {\bf 106}, 144418 (2022).

%%%%%%%%%%% Ref supp Mater%%%%%%%%%%%

\bibitem{Koshino18} M. Koshino, N. F. Q. Yuan, T. Koretsune, M. Ochi, K. Kuroki, and L. Fu, Phys. Rev. X {\bf 8}, 031087 (2018).



\bibitem{Falko19} D. A. Ruiz-Tijerina and V. I. Fal'ko, Phys. Rev. B {\bf 99}, 125424 (2019).


\bibitem{Zhang} S. Zhang, A. Song, L. Chen, C. Jiang, C. Chen, L. Gao, Y. Hou, L. Liu, T. Ma, H. Wang, X.-Q. Feng and Q. Li, Sc. Adv.  eabc5555 {\bf 6} (2020).

\bibitem{Gadelha} A. C. Gadelha, D. A. A. Ohlberg, C. Rabelo, {\it et al.}, Nature {\bf 590}, 405 (2021).

\bibitem{Bi} Z. Bi, N. F. Q. Yuan and L. Fu, Phys. Rev. B {\bf 100}, 035448 (2019).



\end{thebibliography}

%%%%%%%%%%%%%%%%%%%%%%%%%%%%%%%%%%%%%%%
\clearpage

\widetext
\begin{center}
\textbf{\large Supplemental material for Twisted bilayer graphene reveals its flat bands under spin pumping}
\end{center}
\renewcommand{\thefigure}{S.\arabic{figure}}
\renewcommand{\thesection}{S.\arabic{section}}
\renewcommand{\theequation}{S.\arabic{equation}}
\setcounter{figure}{0}
\setcounter{equation}{0}
\setcounter{section}{0}
\section{I. Derivation of the low-energy Hamiltonian of TBG without SOC}
\label{TBG}

We start by a brief overview of the perturbative approach proposed by Bistritzer and MacDonald~\cite{Mc11} to derive the continuum model of TBG. We consider a TBG where the two layers $l=1,2$ are rotated oppositely $\theta_2=-\theta_1=\frac{\theta}2$. The Hamiltonian of a graphene layer $l$ rotated at an angle $\theta_l$ is
\begin{eqnarray}
h_{l,rot}(\mathbf{k})=e^{i\frac{\theta_l}2\sigma_z} h_{l}(\mathbf{k})e^{-i\frac{\theta_l}2\sigma_z}
\label{hl-supp}
\end{eqnarray}
where $h_{l}(\mathbf{k})$ is the Hamiltonian of the unrotated layer $(l)$ given in the continuum limit, by 
\begin{eqnarray}
h_{l}(\mathbf{k})=-\hbar v_F \mathbf{k}\cdot\mathbf{\sigma}^{\ast},
\label{hgr}
\end{eqnarray}
where the momentum $\mathbf{k}$ is written relatively to $\mathbf{K}_{1,\xi}$,
$v_F$ is the Fermi velocity, $\xi$ is the valley index, $\mathbf{\sigma}^{\ast}=\left(\xi \sigma_x,\sigma_y\right)$ and $\sigma_i$ ($i=x,y,z$) are the sublattice-Pauli matrices.\

The leading contribution of the interlayer tunneling can be limited to three nearest hopping processes in the momentum space connecting states $|\mathbf {k}\rangle_1$, around the Dirac point $\mathbf{K}_{1,\xi}$ of layer $(1)$, to the states $|\mathbf{k+q_{j\xi}}\rangle_2$ around $\mathbf{K}_{2,\xi}$, the Dirac point of layer $(2)$. The $\mathbf{q_{j\xi}}$ vectors are given by~\cite{Mc11}
\begin{eqnarray}
 &&\mathbf{q}_{1\xi}=\xi k_{\theta}\left(0,1\right),\; 
 \mathbf{q}_{2\xi}=\mathbf{q}_{1\xi}+\xi\mathbf{G}^M_1=
 \xi k_{\theta}\left(-\frac{\sqrt{3}}2,-\frac12\right),\nonumber\\
 &&\mathbf{q_3}=\mathbf{q}_{1\xi}+\xi\left(\mathbf{G}^M_1+\mathbf{G}^M_2\right)=
 \xi k_{\theta}\left(\frac{\sqrt{3}}2,-\frac12\right),\nonumber\\
 \label{q0}
\end{eqnarray}
where $k_{\theta}=2k_D\sin\frac{\theta}2\sim \theta k_D$ and $k_D=|\mathbf{K}_{1,\xi}|=|\mathbf{K}_{2,\xi}|=\frac {4\pi}{3a}$, $a$ being the graphene lattice parameter. The $\left(\mathbf{G}^M_1,\mathbf{G}^M_2\right)$ is the moiré BZ basis given by $\mathbf{G}^M_i=\mathcal{R}_t^T\mathbf{G}_i$, $\mathbf{G}_i$ are the lattice basis vectors of the monolayer reciprocal lattice $\mathbf{G}_1=\frac{2\pi}a \left(1,-1/\sqrt{3}\right)$ and $\mathbf{G}_2=\frac{2\pi}a \left(0,2/\sqrt{3}\right)$.
$\mathcal{R}_t$ is the rotation tensor written, in the sublattice basis, at a small twist angle as
\begin{eqnarray}
R(\theta)= 
\begin{pmatrix}
0 & -\theta \\
\theta & 0 
\end{pmatrix}.
\end{eqnarray}
In the basis $\left\{|\mathbf {k}\rangle_1,|\mathbf{k+q_{j,\xi}}\rangle_2 \right\}$, the Hamiltonian, at the valley $\xi$, reads as~\cite{Mc11}
\begin{eqnarray}
H(\mathbf{k})= 
\begin{pmatrix}
h_1(\mathbf{k}) & T_1 & T_2 & T_3\\
T^{\dagger}_1 & h_{2,1}(\mathbf{k}) & 0&0\\
T^{\dagger}_2 & 0& h_{2,2}(\mathbf{k}) & 0\\
T^{\dagger}_3 & 0& 0& h_{2,3}(\mathbf{k})\\
\end{pmatrix},\nonumber\\
\label{HBL0}
\end{eqnarray}
For the relaxed TBG the $T_j$ matrices are given by~\cite{Koshino18}
\begin{eqnarray}
T_1= 
\begin{pmatrix}
w & w^{\prime}\\
w^{\prime} & w^{\prime\prime}\
\end{pmatrix},
T_2= e^{i\xi\mathbf{G}^M_1\cdot\mathbf{r}}
\begin{pmatrix}
w & w^{\prime}e^{-i\xi\Phi}\\
w^{\prime}e^{i\xi\Phi}& w^{\prime\prime}\\
\end{pmatrix},
T_3= e^{i\xi\left(\mathbf{G}^M_1+\mathbf{G}^M_2\right)\cdot\mathbf{r}}
\begin{pmatrix}
w & w^{\prime}e^{i\xi\Phi}\\
w^{\prime}e^{-i\xi\Phi}& w^{\prime\prime}\
\end{pmatrix},
\label{T}
\end{eqnarray}
Here $h_1(\mathbf{k})$ is given by Eq.~\ref{hgr} and $h_{2,j}(\mathbf{k})\equiv h_2(\mathbf{k}+\mathbf{q}_{j\xi})=h_1(\mathbf{k}+\mathbf{q}_{j\xi})$, where the momentum $\mathbf{k}$ is written relatively to $\mathbf{K}_{1,\xi}$.
$\Phi=\frac{2\pi}3$ and $\mathbf{r}$ is the shortest inplane shifts between carbon atoms of the two layers~\cite{Falko19}. 
We consider the AA stacking domains hosting the flat bands of TGB~\cite{Zhang,Gadelha}. In this configuration, each A atom of the top layer is directly located above an A atom of the bottom layer and $\mathbf{r}= \mathbf{0}$~\cite{Falko19}.
The parameters $w$, $w^{\prime}$ and $w^{\prime\prime}$ are the tunneling amplitudes which have the same value $w=w^{\prime}=w^{\prime\prime}\sim 118\; \mathrm{meV}$ in the rigid TBG~\cite{Bi}.
In the TBG relaxed lattice, these amplitudes are no more equal $w\sim w^{\prime\prime} \sim 90$ meV and $w^{\prime}=117$ meV.
In the present work,  we do not consider the lattice relaxation effect, since the SOC parameters $\lambda_I,\; \lambda_R\sim 2\, \mathrm{meV} $ are small compared to the difference between the interlayer amplitudes $\Delta w=w-w^{\prime}\sim 20\, \mathrm{meV} $.
In the unrelaxed AA domains, the $T_j$ matrices can be written as
\begin{eqnarray}
 T_1= w\left(\mathbb{I}_{\sigma}+\sigma_x\right),
 T_2=w \left(\mathbb{I}_{\sigma}-\frac 12\sigma_x+\xi \frac {\sqrt{3}}2\sigma_y\right),
 T_3=w\left( \mathbb{I}_{\sigma}-\frac 12\sigma_x-\xi\frac {\sqrt{3}}2\sigma_y\right).
 \label{Tj}
\end{eqnarray}
Here, $\mathbb{I}_{\sigma}$ is the identity matrix acting on the sublattice indices.\

Considering, in Eq.~\ref{HBL0}, the $\mathbf{k}$ dependent term  as a perturbation, the effective Hamiltonian can be written, to the leading order in $\mathbf{k}$, as
\begin{eqnarray}
 H^{(1)}\left(\mathbf{k}\right)=\frac{\langle\Psi|H(\mathbf{k})|\Psi\rangle}{\langle\Psi|\Psi\rangle},
 \label{H10}
\end{eqnarray}
where $\Psi=\left(\psi_0(\mathbf{k}),\psi_1(\mathbf{k}),\psi_2(\mathbf{k}),\psi_3(\mathbf{k})\right)$ is the zero energy eigenstate of $H\left(\mathbf{k}=\mathbf{0}\right)$.  $\Psi$ is constructed on the two-component sublattice spinor $\psi_0(\mathbf{k})$ ($\psi_j(\mathbf{k})$) of layer $1$ (layer $2$) taken at the momentum $\mathbf{k}$ ($\mathbf{k}+\mathbf{q}_{j\xi}$) around the Dirac point $\mathbf{K}_{1,\xi}$ ($\mathbf{K}_{2,\xi}$) at the valley $\xi$. $\psi_0$ is the zero energy eigenstate of $h_1$.
Then, the $\Psi$ components satisfy
\begin{eqnarray}
 h_1\psi_0+\sum_j T_j\psi_j=0,\,\mathrm{and}\; T^{\dagger}_j\psi_0+h_j\psi_j=0, \;\mathrm{with}\; h_1\psi_0=0,
\end{eqnarray}
where $h_j\equiv h_2(\mathbf{q}_{j,\xi})$. Then
\begin{eqnarray}
 \psi_j=-h^{-1}_j T^{\dagger}_j\psi_0, \mathrm{and}\; \sum_jT_jh^{-1}_j T^{\dagger}_j=0.
 \label{psij}
\end{eqnarray}
To the leading order in $\mathbf{k}$, $H^{(1)}\left(\mathbf{k}\right)$, takes the following form
\begin{eqnarray}
 H^{(1)}\left(\mathbf{k}\right)=\frac{\langle\Psi|H(\mathbf{k})|\Psi\rangle}{\langle\Psi|\Psi\rangle}=\frac1{\langle\Psi|\Psi\rangle}
 \left[\psi^{\dagger}_0h_1\left(\mathbf{k}\right) \psi_0+ \psi^{\dagger}_0\sum_jT_j h^{-1}_j h_j\left(\mathbf{k}\right)h^{-1}_j T^{\dagger}_j \psi_0\right],
\label{psiHpsi}
\end{eqnarray}
\begin{eqnarray}
 H^{(1)}\left(\mathbf{k}\right)=-\hbar  v^{\ast} \psi^{\dagger}_0\, \mathbf{k}\cdot\mathbf{\sigma^{\ast}}\,\psi_0,
 \label{H1eff}
\end{eqnarray}
Eq.~\ref{H1eff} is obtained by neglecting $\theta$ in $h_j\left(\mathbf{k}\right)$, which turns out to take $\mathbf{q}_{j,\xi}=\mathbf{0}$ in $h_j\left(\mathbf{k}\right)$~\cite{Mc11}.\newline
$v^{\ast}$ is the effective velocity of the energy band of TBG around the zero energy which vanishes at the MA $\theta_m$, and is given by~\cite{Mc11}
\begin{eqnarray}
 v^{\ast}=v_F\frac{1-3\alpha^2}{1+6\alpha^2}
\end{eqnarray}
where $\alpha=\frac{w}{\hbar v_F q_{0}}$, $q_{0}=|\mathbf{q}_{j\xi}|\sim \frac{4\pi}{3a}\theta$. \newline
In our numerical calculations, we take  $w=118\; \mathrm{meV}$ and $\hbar v_F/a\sim 2.68\; \mathrm{eV}$ which corresponds to $\theta_m=1.05^{\circ}$ for the first MA~\cite{Mc11}.


%%%%%%%%%%%%%%%%%%%%%%%%%%%%%%

\section{II. Derivation of the low-energy Hamiltonian of TBG with SOC}
\label{TBG-SOC}
We now consider the heterostructure consisting of TBG adjacent to a monolayer of $\mathrm{WSe_2}$ as shown in Fig.1 of the main text, where we denote the graphene layer in contact with the TMD by layer (2). This layer is subject to a SOC induced by proximity effect by the TMD, and the corresponding Hamiltonian can be written as 
\begin{eqnarray}
h_{2}(\mathbf{k})=h_{1}(\mathbf{k})+h_{\text{SOC}}+\frac{m}2\sigma_z
\label{h2-supp}
\end{eqnarray}
where $h_{\text{SOC}}$ is given by Eq.~3 of the main text.\\
 
To derive the continuum model of TBG/$\mathrm{WSe}_2$, we follow the perturbative approach of Ref.~\cite{Mc11} presented in the previous section. Now, the basis $\Psi=\left(\psi_0(\mathbf{k}),\psi_1(\mathbf{k}),\psi_2(\mathbf{k}),\psi_3(\mathbf{k})\right)$ is constructed on the four-component spin-sublattice spinor $\psi_0(\mathbf{k})$ and $\psi_j(\mathbf{k})$, ($j=1,2,3$) corresponding, respectively, to layer $(1)$ and layer $(2)$. 
$\psi_0(\mathbf{k})$ is written as 
$\psi_0(\mathbf{k})^T=\left(\psi_{0,A\uparrow},\psi_{0,A\downarrow},\psi_{0,B\uparrow},\psi_{0,B\downarrow}\right)$. In this basis, the Hamiltonian of TBG/$\mathrm{WSe}_2$ takes the form
\begin{eqnarray}
H_{\xi,\text{SOC}}(\mathbf{k})= 
\begin{pmatrix}
h_{1}(\mathbf{k}) & T_1 & T_2 & T_3\\
T^{\dagger}_1 & h_{2,1}(\mathbf{k}) & 0&0\\
T^{\dagger}_2 & 0& h_{2,2}(\mathbf{k}) & 0\\
T^{\dagger}_3 & 0& 0& h_{2,3}(\mathbf{k})\\
\end{pmatrix}.
\label{HTBG-supp}
\end{eqnarray}
The momentum $\mathbf{k}$ is measured relatively to the Dirac point $\mathbf{K}_{1\xi}$ of layer (1), $h_{2,j}\left(\mathbf{k}\right)=h_2\left(\mathbf{k}+\mathbf{q}_{j\xi}\right)$, ($j=1,2,3$) and $h_2\left(\mathbf{k}\right)$ includes now the SOC terms (Eq.~\ref{h2-supp}).\newline
We consider AA stacked moiré domains with spin independent interlayer hopping. The $T_j$ matrices are written as the tensor product of those given by Eq.~\ref{Tj}, with the $2\times2$ identity spin-matrix $\mathbb{I}_s$.\\

Regarding the small values of the SOC, we assume that $H_{\xi,\text{SOC}}(\mathbf{k})$ has a zero eigenenergy and the corresponding eigenstate $\Psi$ satisfies the condition given by Eq.~\ref{psij}.\

Following the same procedure as in the previous section, we derive from Eq.~\ref{psiHpsi} the effective low energy Hamiltonian $H_{\xi,\text{SOC}}^{(1)}(\mathbf{k})$ of TBG/$\mathrm{WSe_2}$ by substituting $h_j\left(\mathbf{k}\right)$ by the Hamiltonian of layer (2), rotated at $\theta/2$ and including SOC as 
\begin{eqnarray}
 h_{\text{2,rot}}(\mathbf{k})=-\hbar \mathbf{k}\cdot\mathbf{\sigma}^{\ast}\mathbb{I}_s +\frac{\lambda_I}2\xi s_z+\frac{\lambda_R}2\left(\xi \sigma_x s_y-\sigma_y s_x\right)+\frac{\lambda_{KM}}2\xi \sigma_z s_z-\frac{\lambda_R}2\theta\left(\xi \sigma_y s_y+\sigma_x s_x\right),
\label{h2rot}
\end{eqnarray}
Hereafter, we neglect the Kane and Mele term whose contribution, to the leading order in $\mathbf{k}$, is found to vanish. We also disregard the last term in Eq.~\ref{h2rot}, which results into a higher order correction in $\theta$.\\

To the first order in the SOC coupling, we obtain the continuum model of TBG/$\mathrm{WSe_2}$ described by the Hamiltonian $H_{\xi,\text{SOC}}^{(1)}(\mathbf{k})$ given by Eq.~\ref{H1} in the main text.
This Hamiltonian contain a SOC term ($h_{\text{eff}}^{\text{SOC}}$) with renormalized Ising and Rashba interactions
\begin{eqnarray}
 \tilde{\lambda}_I=\frac{6\alpha^2}{\langle\Psi|\Psi\rangle}\lambda_I,\;
 \tilde{\lambda}_R=\frac{3\alpha^2}{\langle\Psi|\Psi\rangle}\lambda_R, \; \mathrm{and}\;
 \langle\Psi|\Psi\rangle\sim 1+ 6\alpha^2
 \label{SOCeff-supp}
\end{eqnarray}
$\tilde{\lambda}_I$ and $\tilde{\lambda}_R$ are enhanced by decreasing the twist angle from the MA.\\

\begin{figure}[hpbt] 
\begin{center}
\includegraphics[width=0.5\columnwidth]{energy-plot.eps}
\end{center}
\caption{Energy bands of TBG/$\mathrm{WSe}_2$ around zero energy as function of the dimensionless momentum amplitude $k/q_0$ at different twist angles. The bands are represented up to the cutoff $k_c=q_0/2$. Calculations are done for $\lambda_I=3\;\mathrm{meV}$ and $\lambda_R=4\;\mathrm{meV}$~\cite{Alex}. }
\label{energy}
\end{figure}

To the leading order in $\mathbf{k}$, the four eigenergies of the Hamiltonian $H_{\xi,\text{SOC}}^{(1)}$ (Eq.~\ref{H1} of the main text), denoted $E(\mathbf{k})_{\sigma,\pm}$, are given by
\begin{eqnarray}
E(\mathbf{k}))_{\sigma,\pm}= \frac {\sigma}{\langle \Psi |\Psi\rangle }
\sqrt{f_1(\mathbf{k}))\pm 6 \alpha^2\sqrt{f_2(\mathbf{k}))}},
\label{band-supp}
\end{eqnarray}
\begin{eqnarray}
&&f_1(\mathbf{k}))=(\hbar v)^2\left(1-3\alpha^2\right)^2||{\mathbf{k}}||^2+\frac 92\alpha^4\left(2\lambda_I^2+\lambda_R^2\right),\nonumber\\
&&f_2(\mathbf{k}))= (\hbar v)^2\left(1-3\alpha^2\right)^2||{\mathbf{k}}||^2\left(\lambda_I^2+\frac 14\lambda_R^2\right)+ \frac9{16} \alpha^4 \lambda_R^4,
\end{eqnarray}
where $\sigma=\pm$ is the band index. $E(\mathbf{k})_{\sigma,\pm}$ are depicted in Fig.~\ref{energy} at different twist angles.\newline

At the Dirac point, the eigenenergies reduce to
\begin{eqnarray}
E_{\pm,+}=\frac  {\pm 3\alpha^2}{\langle \Psi |\Psi\rangle } \sqrt{\lambda_I^2+\lambda_R^2}\;,
E_{\pm,-}=\frac  {\pm 3\alpha^2}{\langle \Psi |\Psi\rangle } \lambda_I.
\end{eqnarray}

\section{Electronic Green function}\label{Green-electronic}

The Matsubara Green function associated to the effective Hamiltonian $H_{\xi,\text{SOC}}^{(1)}$ (Eq.~\ref{H1}) is

\begin{eqnarray}
\hat{g}(\mathbf{k},i\omega_n)=\left[ i\omega_n \mathbb{I}_{S}\mathbb{I}_{\sigma} - H_{\xi,\text{SOC}}(\mathbf{k})\right]^{-1},
\end{eqnarray}
where $\mathbb{I}_{S}$ and $\mathbb{I}_{\sigma}$ are the $2\times 2$ spin and band identity matrices, respectively.
$\hat{g}(\mathbf{k},i\omega_n)$ can be expressed as
\begin{eqnarray}
\hat{g}(\mathbf{k},i\omega_n)= \hat{g}_0(\mathbf{k},i\omega_n)\mathbb{I}_{s}+ \mathbf{\hat{g}}\cdot\mathbf{s}.
\label{green-elec}
\end{eqnarray}
$\mathbf{s}=(s_x,s_y,s_z)$ are spin-Pauli matrices, $\hat{g}_0(\mathbf{k},i\omega_n)$ and the components $\hat{g}_i$ ($i=x,y,z$) of $\mathbf{\hat{g}}$ are written as
\begin{eqnarray}
&&\hat{g}_0(\mathbf{k},i\omega_n)=A_0(\mathbf{k},i\omega_n)+C_0(\mathbf{k},i\omega_n),\,
\hat{g}_x(\mathbf{k},i\omega_n)=B_x(\mathbf{k},i\omega_n)+D_x(\mathbf{k},i\omega_n),\,\nonumber\\
&&g_y(\mathbf{k},i\omega_n)=B_y(\mathbf{k},i\omega_n)+D_y(\mathbf{k},i\omega_n),\,
\hat{g}_z(\mathbf{k},i\omega_n)=A_z(\mathbf{k},i\omega_n)+C_z(\mathbf{k},i\omega_n).
\label{gg}
\end{eqnarray}

The $A,B,C$ and $D$ operators functions are written in terms of the band-Pauli matrices $\sigma_{x,y,z}$ and the corresponding identity matrix $\mathbb{I}_{\sigma}$
\begin{eqnarray}
A_i(\mathbf{k},i\omega_n)=A_{i1}(\mathbf{k},i\omega_n) \mathbb{I}_{\sigma}+A_{iz}(\mathbf{k},i\omega_n)\sigma_z  \nonumber\\
C_{i}(\mathbf{k},i\omega_n)=C_{ix}(\mathbf{k},i\omega_n) \sigma_x +C_{iy}(\mathbf{k},i\omega_n)\sigma_y  \nonumber\\
B_{j}(\mathbf{k},i\omega_n)=B_{j1}(\mathbf{k},i\omega_n) \mathbb{I}_{\sigma}+B_{jz}(\mathbf{k},i\omega_n)\sigma_z  \nonumber\\
D_j(\mathbf{k},i\omega_n)=D_{jx}(\mathbf{k},i\omega_n) \sigma_x +D_{jy}(\mathbf{k},i\omega_n)\sigma_y
\label{abcd}
\end{eqnarray}
here $j=x,y$ and $i=0,z$. \\

In the limit of small SOC couplings $\lambda_R,\lambda_I \ll \hbar v q_0$, A, B, C, and D become
%\textcolor{red} {in the A,B,C D expressions change the $\lambda$ to $\lambda/2$}
\begin{eqnarray}
A_{01}(\mathbf{k},i\omega_m)&&= \frac {Y_{00}-E_1^2}{2(E_3^2-E_1^2)} 
\left[\frac 1 {i\hbar \omega_m-E_1}+\frac 1 {i\hbar \omega_m-E_2}\right]
+\frac {E_3^2-Y_{00}}{2(E_3^2-E_1^2)} 
\left[\frac 1 {i\hbar \omega_m-E_3}+\frac 1 {i\hbar \omega_m-E_4}\right]
\label{A01}\\
&& Y_{00}= \frac{(\hbar v)^2 (1-3\alpha^2)^2}{\langle\Psi|\Psi\rangle^2}||\mathbf{k}||^2
+\frac 12\left(\frac{3\alpha^2\lambda_R} {\langle\Psi|\Psi\rangle}\right)^2+
\left(\frac {3\alpha^2 \lambda_I}{\langle\Psi|\Psi\rangle}\right)^2\nonumber\\
A_{0z}(\mathbf{k},i\omega_m)&&=\frac {Y_0}{2\left(E_1^2-E_3^2 \right)}
\left\{ 
\frac 1{E_1}\left(\frac 1 {i\hbar \omega_m-E_1}-\frac 1 {i\hbar \omega_m-E_2}\right)
-\frac 1{E_3}\left(\frac 1 {i\hbar \omega_m-E_3}-\frac 1 {i\hbar \omega_m-E_4}\right)\right\}\\
&&Y_0=-\frac 32\left( \frac{3\alpha^2\lambda_R} {\langle\Psi|\Psi\rangle}\right)^2 \frac {\alpha^2 \lambda_I}{\langle\Psi|\Psi\rangle}\nonumber
\end{eqnarray}
\begin{eqnarray}
C_{0x}(\mathbf{k},i\omega_m)&&=
\frac 1{2\left(E_1^2-E_3^2 \right)} 
\left\{
Y_2\left[\frac 1{E_1}\left(\frac 1 {i\hbar \omega_m-E_1}-\frac 1 {i\hbar \omega_m-E_2}\right)-\frac 1{E_3} \left(\frac 1 {i\hbar \omega_m-E_3}-\frac 1 {i\hbar \omega_m-E_4}\right)\right]\right.\nonumber\\
&&\left.
-Y_1\left[E_1\left(\frac 1 {i\hbar \omega_m-E_1}-\frac 1 {i\hbar \omega_m-E_2}\right)-E_3\left(\frac 1 {i\hbar \omega_m-E_3}-\frac 1 {i\hbar \omega_m-E_4}\right)\right]\right\} \xi k_x\\
&& Y_1= \frac{\hbar v (1-3\alpha^2)} {\langle\Psi|\Psi\rangle},\;
Y_2=- Y_1  \left(\frac {3\alpha^2 \lambda_I}{\langle\Psi|\Psi\rangle} \right)^2 \nonumber
\end{eqnarray}
\begin{eqnarray}
C_{0y}(\mathbf{k},i\omega_m)&&=
\frac 1{2\left(E_1^2-E_3^2 \right)} 
\left\{
Y_2\left[\frac 1{E_1}\left(\frac 1 {i\hbar \omega_m-E_1}-\frac 1 {i\hbar \omega_m-E_2}\right)-\frac 1{E_3} \left(\frac 1 {i\hbar \omega_m-E_3}-\frac 1 {i\hbar \omega_m-E_4}\right)\right]\right.\nonumber\\
&&\left.
-Y_1\left[\left(E_1\frac 1 {i\hbar \omega_m-E_1}-\frac 1 {i\hbar \omega_m-E_2}\right)-E_3\left(\frac 1 {i\hbar \omega_m-E_3}-\frac 1 {i\hbar \omega_m-E_4}\right)\right]\right\} k_y
\end{eqnarray}
\begin{eqnarray}
A_{z1}(\mathbf{k},i\omega_m)&&=\xi
\frac 1{2\left(E_1^2-E_3^2 \right)} 
\left\{
Y_4\left[\frac 1{E_1}\left(\frac 1 {i\hbar \omega_m-E_1}-\frac 1 {i\hbar \omega_m-E_2}\right)-\frac 1{E_3} \left(\frac 1 {i\hbar \omega_m-E_3}-\frac 1 {i\hbar \omega_m-E_4}\right)\right]\right.\nonumber\\
&&\left.
-Y_3\left[E_1\left(\frac 1 {i\hbar \omega_m-E_1}-\frac 1 {i\hbar \omega_m-E_2}\right)-E_3\left(\frac 1 {i\hbar \omega_m-E_3}-\frac 1 {i\hbar \omega_m-E_4}\right)\right]\right\} \\
&& Y_3=- \frac {3\alpha^2 \lambda_I}{\langle\Psi|\Psi\rangle},\;
Y_4=-Y_3 \left[\frac{(\hbar v)^2}{\langle\Psi|\Psi\rangle^2} (1-3\alpha^2)^2 ||\mathbf{k}||^2- \frac12 \left(\frac{3\alpha^2 \lambda_R}{\langle\Psi|\Psi\rangle}\right)^2 -\left(\frac {3\alpha^2 \lambda_I}{\langle\Psi|\Psi\rangle} \right)^2  \right]\nonumber
\end{eqnarray}
\begin{eqnarray}
A_{zz}(\mathbf{k},i\omega_m)&&=\xi\frac {Y_5}{2\left(E_3^2-E_1^2 \right)}\left\{
\frac 1 {i\hbar \omega_m-E_1}+\frac 1 {i\hbar \omega_m-E_2}
-\frac 1 {i\hbar \omega_m-E_3}-\frac 1 {i\hbar \omega_m-E_4}\right\}\\
&&Y_5= \frac 12\left( \frac{\lambda_R} {\langle\Psi|\Psi\rangle}\right)^2\nonumber
\end{eqnarray}
\begin{eqnarray}
C_{zx}(\mathbf{k},i\omega_m)&&=\frac {Y_6}{2\left(E_3^2-E_1^2 \right)}\left\{
\frac 1 {i\hbar \omega_m-E_1}+\frac 1 {i\hbar \omega_m-E_2}
-\frac 1 {i\hbar \omega_m-E_3}-\frac 1 {i\hbar \omega_m-E_4}\right\} k_x\\
&&Y_6=  \hbar v (1-3\alpha^2)
\frac {6\alpha^2 \lambda_I}{\langle\Psi|\Psi\rangle^2}\nonumber
\end{eqnarray}
\begin{eqnarray}
C_{zy}(\mathbf{k},i\omega_m)&&=\frac {Y_6}{2\left(E_3^2-E_1^2 \right)}\left\{
\frac 1 {i\hbar \omega_m-E_1}+\frac 1 {i\hbar \omega_m-E_2}
-\frac 1 {i\hbar \omega_m-E_3}-\frac 1 {i\hbar \omega_m-E_4}\right\}\xi k_y\nonumber\\
\end{eqnarray}
\begin{eqnarray}
B_{x1}(\mathbf{k},i\omega_m)&&=\frac {Y_7}{2\left(E_3^2-E_1^2 \right)}\left\{
\frac 1 {i\hbar \omega_m-E_1}+\frac 1 {i\hbar \omega_m-E_2}
-\frac 1 {i\hbar \omega_m-E_3}-\frac 1 {i\hbar \omega_m-E_4}\right\} k_y\\
&&Y_7=  \hbar v (1-3\alpha^2)
\frac {3\alpha^2 \lambda_R}{\langle\Psi|\Psi\rangle^2}\nonumber
\end{eqnarray}
\begin{eqnarray}
B_{xz}(\mathbf{k},i\omega_m)&&=
Y_8\frac 1 {2\left(E_1^2-E_3^2 \right)}\,k_y \times\nonumber\\
&&\left\{\frac 1{E_1}\left[\frac 1 {i\hbar \omega_m-E_1}-\frac 1 {i\hbar \omega_m-E_2}\right]
+\frac 1 {E_3}\left[\frac 1 {i\hbar \omega_m-E_3}-\frac 1 {i\hbar \omega_m-E_4}\right]\right\} \\
&&Y_8=- \hbar v (1-3\alpha^2)
\frac {3\alpha^2 \lambda_R}{\langle\Psi|\Psi\rangle^2} \frac {3\alpha^2 \lambda_I}{\langle\Psi|\Psi\rangle}\nonumber
\end{eqnarray}
\begin{eqnarray}
D_{xx}(\mathbf{k},i\omega_m)&&=Y_9\frac 1 {2\left(E_1^2-E_3^2 \right)}\,\xi k_x k_y\times\nonumber\\
&&\left\{\frac 1{E_1}\left[\frac 1 {i\hbar \omega_m-E_1}-\frac 1 {i\hbar \omega_m-E_2}\right]
-\frac 1 {E_3}\left[\frac 1 {i\hbar \omega_m-E_3}-\frac 1 {i\hbar \omega_m-E_4}\right]\right\} \\
&&Y_9= (\hbar v)^2 (1-3\alpha^2)^2
\frac {3\alpha^2 \lambda_R}{\langle\Psi|\Psi\rangle} \nonumber
\end{eqnarray}
\begin{eqnarray}
D_{xy}(\mathbf{k},i\omega_m)=&&
\frac 1{2\left(E_1^2-E_3^2 \right)} 
\left\{
Y_{11}\left[\frac 1{E_1}\left(\frac 1 {i\hbar \omega_m-E_1}-\frac 1 {i\hbar \omega_m-E_2}\right)-\frac 1{E_3} \left(\frac 1 {i\hbar \omega_m-E_3}-\frac 1 {i\hbar \omega_m-E_4}\right)\right]\right.\nonumber\\
&&\left.
-Y_{10}\left[E_1\left(\frac 1 {i\hbar \omega_m-E_1}-\frac 1 {i\hbar \omega_m-E_2}\right)-E_3\left(\frac 1 {i\hbar \omega_m-E_3}-\frac 1 {i\hbar \omega_m-E_4}\right)\right]\right\} \\
&& Y_{10}=-\frac {3\alpha^2 \lambda_R}{2\langle\Psi|\Psi\rangle},\;
Y_{11}=Y_{10} \left[(\hbar v)^2 (1-3\alpha^2)^2 (k_x^2-k_y^2)+ \left(\frac {3\alpha^2 \lambda_I}{\langle\Psi|\Psi\rangle} \right)^2  \right]\nonumber
\end{eqnarray}
\begin{eqnarray}
&&B_{y1}(\mathbf{k},i\omega_m)=-Y_{7} \frac 1 {2\left(E_3^3-E_1^2 \right)}\left\{ 
\frac 1 {i\hbar \omega_m-E_1}+\frac 1 {i\hbar \omega_m-E_2}
-\frac 1 {i\hbar \omega_m-E_3}-\frac 1 {i\hbar \omega_m-E_4}\right\} k_x
\end{eqnarray}

\begin{eqnarray}
&&B_{yz}(\mathbf{k},i\omega_m)=-Y_8\frac 1 {2\left(E_1^2-E_3^2 \right)}\,k_x
\left\{\frac 1{E_1}
\left[\frac 1 {i\hbar \omega_m-E_1}-\frac 1 {i\hbar \omega_m-E_2}\right]
-\frac 1{E_3}\left[\frac 1 {i\hbar \omega_m-E_3}-\frac 1 {i\hbar \omega_m-E_4}\right]\right\} 
\end{eqnarray}
\begin{eqnarray}
D_{yx}(\mathbf{k},i\omega_m)&=&\xi
\frac 1{2\left(E_1^2-E_3^2 \right)} 
\left\{
Y_{12}\left[\frac 1{E_1}\left(\frac 1 {i\hbar \omega_m-E_1}-\frac 1 {i\hbar \omega_m-E_2}\right)-\frac 1{E_3} \left(\frac 1 {i\hbar \omega_m-E_3}-\frac 1 {i\hbar \omega_m-E_4}\right)\right]\right.\nonumber\\
&&\left.
+Y_{10}\left[E_1\left(\frac 1 {i\hbar \omega_m-E_1}-\frac 1 {i\hbar \omega_m-E_2}\right)-E_3\left(\frac 1 {i\hbar \omega_m-E_3}-\frac 1 {i\hbar \omega_m-E_4}\right)\right]\right\} \\
&&Y_{12}= \frac{3\alpha^2 \lambda_R}{3\langle\Psi|\Psi\rangle} \left[-(\hbar v)^2 (1-3\alpha^2)^2 (k_x^2-k_y^2)+ \left(\frac {3\alpha^2 \lambda_I}{\langle\Psi|\Psi\rangle} \right)^2  \right]\nonumber
\end{eqnarray}
\begin{eqnarray}
D_{yy}(\mathbf{k},i\omega_m)&&=-Y_9\frac 1 {\left(E_1^2-E_3^2 \right)}k_x k_y
\left\{\frac1{E_1}
\left[\frac 1 {i\hbar \omega_m-E_1}-\frac 1 {i\hbar \omega_m+E_1}\right]
-\frac1{E_3}\left[\frac 1 {i\hbar \omega_m-E_3}-\frac 1 {i\hbar \omega_m-E_4}\right]\right\}
\label{Dyy}
\end{eqnarray}


\section{III. Magnon Green function and Gilbert damping}\label{Green-mag}

\subsection{Interfacial exchange coupling between a ferro-
magnetic insulator (FI) and a TBG}

We consider the Hamiltonian of the ferromagnetic insulator (FI) in the independent magnon approximation as
\begin{align}
H_{\rm FI} = \sum_{\bm k} \hbar \omega_{\bm k} b_{\bm k}^\dagger b_{\bm k},
\end{align}
where $\hbar \omega_{\bm k}\simeq {\cal D}|{\bm k}|^2 + \hbar \gamma h_{\rm dc}$ is a dispersion of magnons, ${\cal D}$ is a spin stiffness, $\gamma$ is the gyromagnetic ratio, $h_{\rm dc}$ is a static magnetic field.
In the spin pumping setup, only the static part associated to $\bm{k= 0}$ is relevant.
Considering only the uniform spin precession, the Hamiltonian of the FI can be simply written as
\begin{align}
H_{\rm FI} = \hbar \omega_{\bm 0} b_{\bm 0}^\dagger b_{\bm 0},  
\end{align}
where $b_{\bm q}$ is the Fourier transformation of the site representation $b_i$ defined as
\begin{align}
b_i &= \frac{1}{\sqrt{N_{\rm FI}}} \sum_{\bm q}
e^{i{\bm q}\cdot {\bm r}_i} b_{\bm q}
\simeq \frac{1}{\sqrt{N_{\rm FI}}} b_{\bm 0},\label{bapp1} \\
b_i^\dagger &= \frac{1}{\sqrt{N_{\rm FI}}} \sum_{\bm q}
e^{-i{\bm q}\cdot {\bm r}_i} b^\dagger_{\bm q} \simeq \frac{1}{\sqrt{N_{\rm FI}}} b^\dagger_{\bm 0},
\label{bapp2}
\end{align}
where $N_{\rm FI}$ is the number of unit cells in the FI.\newline

We consider the (retarded) magnon Green function as
\begin{align}
G^R({\bm q},\omega) &= \int dt \, G^R({\bm q},t) e^{i\omega t}, \\
G^R({\bm q},t) &= -\frac{i}{\hbar} \theta(t) \langle [S^+_{\bm q}(t),S^-_{\bm q}(0)] \rangle \nonumber \\
&= -\frac{2iS_0}{\hbar} \theta(t) \langle [b_{\bm q}(t),b_{\bm q}^\dagger(0)] \rangle.
\end{align}
In the absence of the junction, the magnon Green function is
\begin{align}
G_0^R({\bm q},\omega) = \frac{2S_0/\hbar}{\omega - \omega_{\bm q}+i\delta} .
\end{align}
Usually, we introduce spin relaxation of bulk FI as
\begin{align}
G_0^R({\bm q},\omega) = \frac{2S_0/\hbar}{\omega - \omega_{\bm q}+i\alpha_{\rm G}\omega} ,
\end{align}
where $\alpha_{\rm G}$ is a dimensionless strength of the Gilbert damping, which is of order of $10^{-4}$--$10^{-3}$.
We note that the FMR line shape is proportional to ${\rm Im}\, G_0^R({\bm q}={\bm 0},\omega)$.\newline
In the presence of the interfacial coupling and for a uniform spin precession, the magnon Green function is given by the Dyson equation as
\begin{align}
G_0^R({\bm {q=0}},\omega) = \frac{2S_0/\hbar}{\omega - \omega_{\bm q=0}+i\alpha_{\rm G}\omega-(2S_0/\hbar) \Sigma^R(\bm {q=0},\omega)} ,
\end{align}
where $\Sigma^R(\omega)$ is the self-energy.
Although the real part of $\Sigma^R(\omega)$ is related to the shift of the ferromagnetic resonance, we neglect it for simplicity.
Then, the magnon Green function is rewritten as
\begin{align}
G_0^R({\bm 0},\omega) &= \frac{2S_0/\hbar}{\omega - \omega_{\bm q=0}+i(\alpha_{\rm G}+\delta \alpha_{\rm G})}\omega, \\
\delta \alpha_{\rm G}(\omega) &= - \frac{2S_0}{\hbar \omega} \rm{Im}\Sigma^R(\bm {q=0},\omega).
\end{align}
We note that $\delta \alpha_{\rm G}(\omega)$ depends on $\omega$ in general.
However, since the ferromagnetic resonance peak is sharp enough ($\alpha_{\rm G}+\delta \alpha_{\rm G}\ll 1$), we can replace $\omega$ with $\omega_{\rm 0}=\Omega$ (the peak position of the ferromagnetic resonance):
\begin{align}
\delta \alpha_{\rm G} &\simeq - \frac{2S_0}{\hbar \Omega} \Sigma^R(\bm {q=0},\Omega).
\end{align}

The Hamiltonian of the interfacial coupling is given as
\begin{align}
H_{\rm int} = \sum_{\langle i,j \rangle} T_{ij} (S^+_i s^-_j + {\rm h.c.}).
\end{align}
Here, $S^\pm_i$ is a spin ladder operator of the FI and is described by magnon annihilation/creation operators ($b_i$ and $b_i^\dagger$) as
\begin{align}
S^+_i = \sqrt{2S_0} b_i, \quad S^-_i = \sqrt{2S_0} b_i^\dagger,
\end{align}
where $S_0$ is an amplitude of the localized spin in the FI.
$s^\pm_j$ is a spin ladder operator of electrons in two-dimensional electron systems (twisted bilayer graphene) and is described by the electron annihilation/creation operators ($c_{j\sigma}$ and $c_{j\sigma}^\dagger$) as
\begin{align}
s^+_j = c_{j\uparrow}^\dagger c_{j\downarrow}, \quad
s^-_j = c_{j\downarrow}^\dagger c_{j\uparrow}.
\end{align}
We define the Fourier transformation as
\begin{align}
c_{j\sigma} &= \frac{1}{\sqrt{N}} \sum_{\bm k}
e^{i{\bm k}\cdot {\bm r}_j} c_{{\bm k}\sigma}, \label{fourier1} \\
c^\dagger_{j\sigma} &= \frac{1}{\sqrt{N} } \sum_{\bm k}
e^{-i{\bm k}\cdot {\bm r}_j} c^\dagger_{{\bm k}\sigma}, \label{fourier2}
\end{align}
where $N$ is the number of unit cells and ${\bm r}_j$ is the position of the site $j$ in TBG.
Then, we obtain
\begin{align}
s^+_j &= \frac{1}{N} \sum_{{\bm k},{\bm k}'} e^{-i{\bm k}\cdot {\bm r}_j+i{\bm k}'\cdot {\bm r}_j} c_{{\bm k}\uparrow}^\dagger c_{{\bm k}'\downarrow}, \\
s^-_j &= \frac{1}{N} \sum_{{\bm k},{\bm k}'} e^{-i{\bm k}\cdot {\bm r}_j+i{\bm k}'\cdot {\bm r}_j} c_{{\bm k}\downarrow}^\dagger c_{{\bm k}'\uparrow}.
\end{align}
We define the Fourier transformation of $s^\pm_j$ as
\begin{align}
s^+_j &= \frac{1}{N} \sum_{\bm q} e^{i{\bm q}\cdot {\bm r}_j} s^+_{\bm q},  \\
s^-_j &= \frac{1}{N} \sum_{\bm q} e^{i{\bm q}\cdot {\bm r}_j} s^-_{\bm q} .
\end{align}
From $s^-_j=(s^+_j)^\dagger$, we obtain the relation 
\begin{align}
s^-_{\bm q} = (s^+_{-{\bm q}})^\dagger .
\end{align}
Note that $(s^-_{\bm q})^\dagger \ne s^+_{{\bm q}}$.
The inverse Fourier transformation is given as
\begin{align}
s^+_{\bm q} &= \sum_{j} e^{-i{\bm q}\cdot {\bm r}_j} s^+_j , \\
s^-_{\bm q} &= (s^+_{-{\bm q}})
= \sum_{j} e^{-i{\bm q}\cdot {\bm r}_j} s^-_j . 
\end{align}
Especially for ${\bm q}={\bm 0}$, we obtain
\begin{align}
s^+_{\bm 0} = \sum_{j} s^+_j , 
\quad
s^-_{\bm 0} = \sum_{j} s^-_j , 
\end{align}
Using Eqs.~(\ref{fourier1}) and (\ref{fourier2}), we obtain
\begin{align}
s^+_{\bm q} &= \frac{1}{N}
\sum_{j} e^{-i{\bm q}\cdot {\bm r}_j} \sum_{{\bm k},{\bm k}'} e^{-i{\bm k}\cdot {\bm r}_j+i{\bm k}'\cdot {\bm r}_j} c_{{\bm k}\uparrow}^\dagger c_{{\bm k}'\downarrow} \nonumber \\
&=\sum_{{\bm k}}c_{{\bm k}\uparrow}^\dagger c_{{\bm k}+{\bm q} \downarrow} , \\
s^-_{\bm q} &= \frac{1}{N}
\sum_{j} e^{-i{\bm q}\cdot {\bm r}_j} \sum_{{\bm k},{\bm k}'} e^{-i{\bm k}\cdot {\bm r}_j+i{\bm k}'\cdot {\bm r}_j} c_{{\bm k}\downarrow}^\dagger c_{{\bm k}'\uparrow} \nonumber \\
&=\sum_{{\bm k}}c_{{\bm k}\downarrow}^\dagger c_{{\bm k}+{\bm q} \uparrow} .
\end{align}

For a clean interface, we can set $T_{ij}=T$.
Then, using Eqs.~(\ref{bapp1}) and (\ref{bapp2}), the Hamiltonian of the interface is written as
\begin{align}
H_{\rm int} &= \frac{T\sqrt{2S_0}}{\sqrt{N_{\rm FI}}} \sum_{\langle i,j\rangle} (b_{\bm 0} s_j^- +
b_{\bm 0}^\dagger s_j^+) \nonumber \\
&\simeq \frac{T\sqrt{2S_0}}{\sqrt{N_{\rm FI}}} \left[ b_{\bm 0} \Bigl(\sum_j s_j^-\Bigr) +
b_{\bm 0}^\dagger \Bigl(\sum_j s_j^+\Bigr) \right] \nonumber \\
&= \sqrt{2S_0} b_{\bm 0} \tilde{s}^- + \sqrt{2S_0} b_{\bm 0} \tilde{s}^+,
\end{align}
where $\tilde{s}^{\pm}$ is defined as
\begin{align}
\tilde{s}^- &= \frac{T}{\sqrt{N_{\rm FI}}} s_{\bm 0}^- , \\
\tilde{s}^+ &= \frac{T}{\sqrt{N_{\rm FI}}} s_{\bm 0}^+ .
\end{align}



By the second-order perturbation, the self-energy of the magnon at ${\bm q}={\bm 0}$ is calculated as
\begin{align}
\Sigma^R(\omega) &= \int dt \, \Sigma^R(t) e^{i\omega t} , \\
\Sigma^R(t) &= -\frac{i}{\hbar}\theta(t) \langle [\tilde{s}^+(t),\tilde{s}^-(0)] \rangle.
\end{align}
The self-energy can be related to a retarded component of the dynamic spin susceptibility {\it per unit area} as
\begin{align}
\Sigma^R(\omega) &= -\frac{T^2N}{N_{\rm FI}} \chi({\bm 0},\omega), \\
\chi({\bm q},\omega) &= \int dt \, \chi({\bm q},t) e^{i\omega t} , \\
\chi({\bm q},t) &= \frac{i}{N\hbar}\theta(t) \langle [s_{-{\bm q}}^+(t),s_{\bm q}^-(0)] \rangle.
\end{align}
$\chi({\bm q},t)$ is calculated for one-band of TBG without spin-orbit interaction as
\begin{align}
\chi({\bm q},t) = \frac{1}{N} \sum_{\bm q}
\frac{f(\xi_{
\bm k})-f(\xi_{{\bm k}+{\bm q}})}
{\hbar \omega + i\delta + \xi_{{\bm k}+{\bm q}}-\xi_{\bm k}},
\end{align}
where $\xi_{\bm k}=\epsilon_{\bm k}-\mu$, $\epsilon_{\bm k}$ is a dispersion of electrons, $\mu$ is a chemical potential.
This is just a Lindhard function.\newline
We note that $\chi({\bm q},t)$ is independent of the system size (area).
For systems with spin-orbit interaction, we have to extend the Lindhard function into the spin-dependent one.

Then, the enhancement of the Gilbert damping is written as
\begin{align}
\delta \alpha_{\rm G} &= - \frac{2S_0}{\hbar \Omega} \, {\rm Im} \Sigma^R(\rm {q=0},\Omega) \nonumber \\
&= \frac{2S_0T^2N}{N_{\rm FI}\hbar \omega_{\bm 0}} \chi(\rm {q=0},\Omega).
\end{align}
We note that the number of the unit cell of twisted bilayer graphene is written as $N=S/A$ where $S$ is a area of the junction and $A$ is an area of a unit cell of twisted bilayer graphene.
We also note that the number of the unit cell of the FI is written as $N_{\rm FI}=Sd/a^3$ where $d$ is a thickness of the FI, $a$ is a lattice constant of the FI.
Using these parameters, we obtain
\begin{align}
\delta \alpha_{\rm G} &= \frac{2S_0T^2a^3}{Ad\hbar \omega_{\bm 0}} \chi(\rm {q=0},\Omega).
\end{align}
We note that $\delta \alpha_{\rm G}$ is proportional to $1/d$ in consistent with experimental results.

\begin{table}[htp]
 \caption{Experimental parameters.}
 \label{table:Parameters}
 \centering
  \begin{tabular}{lll}
   \hline
   Microwave frequency & $\omega_{\bm 0}$ & $1\, {\rm GHz}$ \\
   Amplitude of spins of FI & $S_0$ & 10 \\
   Lattice constant of FI & $a$ & $12.376\,$\AA \\
   Thickness of FI & $d$ & $\ge 10\, {\rm nm}$ \\
   Interfacial exchange coupling & $J$ & $\sim 1\, {\rm K}$ (not known) \\
   \hline
  \end{tabular}
\end{table}

If YIG (Yttrium Iron Garnet) is chosen as the ferromagnet insulator, the parameter is given in the Table.

\subsection{Electronic spins in the FI magnetization frame}

Regarding the dependence of the electronic Hamiltonian (Eq.~5 of the main text) on $s_z$, one should consider a $3D$ FI magnetization as in Ref.~\cite{Funato}.  The average spin vector is along the orthoradial spherical vector $\langle \mathbf{S}_{FI}\rangle= \langle \mathbf{S}_{FI}\rangle \,\mathbf{u}_{x^{\prime}}$. The radial vector $\mathbf{u}_{z^{\prime}}$ forms an angle $\theta_m$ with the $z$ axis perpendicular to the interface. The third axis $y^{\prime}$ is in the $(xoy)$ plane and its unit vector is the orthoradial inplane vector $\mathbf{u}_{y^{\prime}}=-\sin\Phi_m \;\mathbf{u}_x+\cos\Phi_m\; \mathbf{u}_y $ as shown in Fig.\ref{axis}.\

\begin{figure}[hpbt] 
\begin{center}
\includegraphics[width=0.2\columnwidth]{SP-axis.eps}
\end{center}
\caption{magnetization-fixed coordinate frame $\left(x^{\prime},y^{\prime},z^{\prime}\right)$ with respect to the Laboratory frame $\left(x,y,z\right)$.}
\label{axis}
\end{figure}

In the FI spin frame ($x^{\prime},y^{\prime},z^{\prime}$), the components of the electronic spin operators are given by:
\begin{eqnarray}
&&s^{x^{\prime}}_{\mathbf{k}}=\mathbf{s}_{\mathbf{k}}\cdot \mathbf{u}_{x^{\prime}}=
\cos \theta_m\cos\Phi_m \,s^x_{\mathbf{k}}+\cos\theta_m\sin\Phi_m\, s^y_{\mathbf{k}}-\sin \theta_m \,s^z_{\mathbf{k}}\nonumber\\
&&s^{y^{\prime}}_{\mathbf{k}}=\mathbf{s}_{\mathbf{k}}\cdot \mathbf{u}_{y^{\prime}}=
-\sin \Phi_m\, s^x_{\mathbf{k}}+\cos\Phi_m\, s^y_{\mathbf{k}}\nonumber\\
&&s^{z^{\prime}}_{\mathbf{k}}=\mathbf{s}_{\mathbf{k}}\cdot \mathbf{u}_{z^{\prime}}=
\sin \theta_m\cos\Phi_m \,s^x_{\mathbf{k}}+\sin\theta_m\sin\Phi_m\, s^y_{\mathbf{k}}+\cos \theta_m \,s^z_{\mathbf{k}}\nonumber\\
\end{eqnarray}
We define the ladder electronic spin operators as
\begin{eqnarray}
s^{x^{\prime},\pm}_{\mathbf{k}}&=&s^{y^{\prime}}_{\pm\mathbf{k}}\pm i s^{z^{\prime}}_{\pm\mathbf{k}}
=\frac 12 \sum_{\sigma,\sigma^{\prime},\mathbf{k}^{\prime}} c^{\dagger}_{\mathbf{k}^{\prime},\sigma} \left(\sigma_s^{x^{\prime},\pm} \right)_{\sigma,\sigma'}
c^{\dagger}_{\mathbf{k}^{\prime}\pm \mathbf{k},\sigma'}
\end{eqnarray}
where $s^{x^{\prime},\pm}= s_x \left( -\sin\Phi_m\pm i \sin \theta_m\cos\Phi_m\right)
+s_y\left(\cos\Phi_m\pm i \sin \theta_m\sin\Phi_m \right)\pm i \cos \theta_m s_z$.

\subsection{Magnon self-energy}\label{mag}

In the second order perturbation with respect to the interfacial exchange interaction $T_\mathbf{q}$  the magnon self-energy is given by~\cite{Yama}
\begin{eqnarray}
\Sigma (\mathbf{q},i\omega_n)=\frac{|T_\mathbf{q}|^2}{4\beta}\sum_{\mathbf{k},i\omega_m}
\text{Tr}\left[ \sigma_s^{x^{\prime},-} \;\hat{g} (\mathbf{k},i\omega_m)\sigma_s^{x^{\prime},+}\;\hat{g} (\mathbf{k}+\mathbf{q},i\omega_m+i\omega_n)
\right]
\label{self-supp}
\end{eqnarray}
where $\hat{g} (\mathbf{k},i\omega_m)$ is the electronic Green function given by Eq.~\ref{green-elec}.\

The trace term is of the form:
\begin{eqnarray}
\text{Tr}\left[ \mathbf{a}^{\ast}\cdot\mathbf{\sigma}_s \left(\hat{g}_0+\hat{\mathbf{g}}\cdot\mathbf{\sigma}_s\right)
\mathbf{a}\cdot\mathbf{\sigma}_s 
\left(\hat{g}^{\prime}_0+\hat{\mathbf{g}}^{\prime}\cdot\mathbf{\sigma}_s\right)
\right]
\label{Trace}
\end{eqnarray}
where the vector $\mathbf{a}=\left(-\sin \Phi_m+i\sin\theta_m\cos\Phi_m,\cos \Phi_m+i\sin\theta_m\sin\Phi_m,i\cos \theta_m \right)$ is written in the laboratory frame $(x,y,z)$.\

We set $\hat{g}=\hat{g} (\mathbf{k},i\omega_m)$ and $\hat{g}^{\prime}=\hat{g} (\mathbf{k},i\omega_m+i \omega_n)$.
Taking into account the operator character of $\hat{g}$ one could use the identity
\begin{eqnarray}
\left( \mathbf{a}\cdot \mathbf{\sigma}_s\right)\left( \mathbf{b}\cdot \mathbf{\sigma}_s\right)=
\left( \mathbf{a}\cdot \mathbf{b}\right) \mathbb{I} +i \left( \mathbf{a}\times \mathbf{b}\right)\cdot \mathbf{\sigma}_s
\end{eqnarray}
Given the expressions of $\hat{g}$ and $\hat{g}^{\prime}$ in Eq.~\ref{gg}, the trace term (Eq.~\ref{Trace}) reduces to:
\begin{eqnarray}
\text{Tr}\left[ \mathbf{a}^{\ast}\cdot\mathbf{\sigma}_s \left(\hat{g}_0+\hat{\mathbf{g}}\cdot\mathbf{\sigma}_s\right)
\mathbf{a}\cdot\mathbf{\sigma}_s 
\left(\hat{g}^{\prime}_0+\hat{\mathbf{g}}^{\prime}\cdot\mathbf{\sigma}_s\right)
\right]= \sum_{i=0,1,2} F_i(\mathbf{k},i\omega_m,i\omega_n)
\label{Trace2}
\end{eqnarray}
where
\begin{eqnarray}
&& F_0(\mathbf{k},i\omega_m,i\omega_n)=4\left(A_{01}A'_{01}+A_{0z}A'_{0z}+C_{0x}C'_{0x}+C_{0y}C'_{0y}\right)\nonumber\\
%
&&F_{1}(\mathbf{k},i\omega_m,i\omega_n)=-2\left\{
\cos \theta_m\cos \Phi_m\right.\nonumber\\ 
&&\left(A_{01}B'_{x1}-B_{x1}A'_{01}+A_{0z}B'_{xz}-A'_{0z}B_{xz}
+C_{0x}D'_{xx}-C'_{0x}D_{xx}+C_{0y}D'_{xy}-C'_{0y}D_{xy}
\right)\nonumber\\
&&+\cos \theta_m\sin \Phi_m \left(A_{01}B'_{y1}-B_{y1}A'_{01}+A_{0z}B'_{yz}-A'_{0z}B_{yz}
+C_{0x}D'_{yx}-C'_{0x}D_{yx}+C_{0y}D'_{yy}-C'_{0y}D_{yy}
\right)\nonumber\\
&&\left.-\sin \theta_m\left(A_{01}A'_{z1}-A'_{01}A_{z1}+A_{0z}A'_{zz}-A'_{0z}A_{zz}+
C_{0x}C'_{zx}-C'_{0x}C_{zx}+C_{0y}C'_{zy}-C'_{zy}C_{0y}
\right)
\right\}\nonumber\\
%
&&F_{2}(\mathbf{k},i\omega_m,i\omega_n)=
-2\cos^2 \theta_m\cos^2 \Phi_m
\left(B_{x1}B'_{x1}+B_{xz}B'_{xz}+D_{xx}D'_{xx}+D_{xy}D'_{xy}
\right)\nonumber\\
&&-2\cos^2 \theta_m\sin^2 \Phi_m
\left(B_{y1}B'_{y1}+B_{yz}B'_{yz}+D_{yx}D'_{yx}+D_{yy}D'_{yy}
\right)\nonumber\\
&&-2\sin^2\theta_m
\left(A_{z1}A'_{z1}+A_{zz}A'_{zz}+C_{zx}C'_{zx}+C_{zy}C'_{zy}
\right)\nonumber\\
&&-\cos^2 \theta_m\sin 2 \Phi_m
\left(B_{x1}B'_{y1}+B_{xz}B'_{yz}+D_{xx}D'_{yx}+D_{xy}D'_{yy}
+B'_{x1}B_{y1}+B'_{xz}B_{yz}+D'_{xx}D_{yx}+D'_{xy}D_{yy}
\right)\nonumber\\
&&+\cos\Phi_m \sin 2 \theta_m
\left(B_{x1}A'_{z1}+B_{xz}A'_{zz}+D_{xx}C'_{zx}+D_{xy}C'_{zy}
B'_{x1}A_{z1}+B'_{xz}A_{zz}+D'_{xx}C_{zx}+D'_{xy}C_{zy}
\right)\nonumber\\
&&+\sin\Phi_m \sin 2 \theta_m
\left(B_{y1}A'_{z1}+B_{yz}A'_{zz}+D_{yx}C'_{zx}+D_{yy}C'_{zy}
B'_{y1}A_{z1}+B'_{yz}A_{zz}+D'_{yx}C_{zx}+D'_{yy}C_{zy}
\right)
\end{eqnarray}
The terms with a prime are expressed in terms of $i\omega_n'=i\omega_n+i\omega_m$.\

Regarding the $\mathbf{k}$ dependence of the $A,B,C$ and $D$ operators (Eqs.~\ref{A01}-~\ref{Dyy}), only $F_0$, the last term in $F_1$ and the three first terms in $F_2$ give non-vanishing contributions after summing over $\mathbf{k}$ in Eq.~\ref{self}.\

On the other hand, the terms between parentheses in the first and second line in $F_2$ expression give the same contribution.
As a result, the GD is found to be independent of the azimutal angle $\Phi_m$, which expresses isotropy of the electronic band structure $E_{\sigma,\mp}$ (Eq.\ref{band-supp}). However, the GD depends on the out-of-plane orientation of the FI magnetization via the angle $\theta_m$.\\

According to Eq.~\ref{self-supp}, the terms to calculate are of the form 

\begin{eqnarray}
\sum_{\mathbf{k}} F(\mathbf{k})\sum_{\omega_m}\frac 1 {i\hbar \omega_m-E_i} \frac 1 {i\hbar \omega_m^{\prime}-E_j},
\label{sum}
\end{eqnarray}
 where $\omega_m^{\prime}=\omega_m+\omega_n$ and $F(\mathbf{k})$ is a function of $\mathbf{k}=\left(k,\varphi_{\mathbf{k}}\right)$.\
 
The summation over $\omega_m$ in Eq.~\ref{sum} can be written as
\begin{eqnarray}
\sum_{\omega_m}\frac 1 {i\hbar \omega_m-E_i} \frac 1 {i\hbar \omega_m^{\prime}-E_j}
&&= \frac 1 {i\hbar\omega_n- (E_j-E_i)}\sum_{\omega_m}\left[\frac 1 {i\hbar \omega_m-E_i}- \frac 1 {i\hbar \omega_m^{\prime}-E_j} \right]\nonumber\\
&&=-\frac 1 {i\hbar\omega_n- (E_j-E_i)} \frac 1{k_BT}\int_c \frac{dz}{2\pi i} h(z) f(z)
\end{eqnarray}
where $h(z)=\frac 1 {z-E_i}- \frac 1 {z+i\hbar \omega_n-E_j}$, $f(z)$ is the Fermi-Dirac function, $C$ is clockwise contour around the poles $z=i\hbar \omega_m$.\

Equation~\ref{sum} reduces, then, to
\begin{eqnarray}
\sum_{\omega_m}\frac 1 {i\hbar \omega_m-E_i} \frac 1 {i\hbar \omega_m^{\prime}-E_j}
=\frac {f(E_j)-f(E_i)} {i\hbar\omega_n- (E_j-E_i)}  
\end{eqnarray}
Taking the analytic continuation $ i\hbar\omega_n=\hbar\omega +i\eta$, Eq.~\ref{sum} becomes
\begin{eqnarray}
\lim_{\eta \to 0}\sum_{\mathbf{k}} F(\mathbf{k})  \eta\; \frac {f(E_j)-f(E_i)} {(\hbar\omega - E_j+E_i)^2+\eta^2}=\lim_{\eta \to 0}\sum_{\mathbf{k}}F(\mathbf{k}) (f(E_j)-f(E_i))
L(\hbar \omega-(E_j-E_i)),
\label{lor}
\end{eqnarray}
$L(x)=\frac {\eta}{x^2+\eta^2}$ being the Lorentzian function.\
The sum over $\mathbf{k}=(k,\varphi_{\mathbf{k}})$ in Eq.~\ref{sum} reduces to $\frac A{(2\pi)^2}\int_0^{k_c} k dk\int_0^{2\pi} d\varphi_{\mathbf{k}}$, where $A$ is the moiré superlattice area, $k_c$ is a cutoff on the momentum amplitude $k$, below which the continuum model for the monolayer is justified. We take $k_c=q_0/2$, where $q_0=\frac{4\pi}{3 a}\theta$ is the separation between the Dirac points $\mathbf{K}_{1,\xi}$ and $\mathbf{K}_{2,\xi}$ of, respectively, layer (1) and layer (2) at a given monolayer valley $\xi$.

\subsection{Gilbert damping correction}

For a uniform spin precession, the Gilbert damping correction $\delta\alpha_G$, at the FMR frequency $\Omega$, can be expressed as~\cite{Yama}
\begin{eqnarray}
\delta\alpha_G=-\frac {2S_0}{\hbar \Omega} \mathrm{Im} \Sigma(\mathbf{q=0},\Omega)
\end{eqnarray}
The imaginary part of the magnon self-energy is of the form $\mathrm{Im} \Sigma(\mathbf{q=0},\Omega)=\frac{T_0^2}{\hbar \Omega} \tilde{\Sigma}(\mathbf{q=0},\Omega)$, where $\tilde{\Sigma}(\mathbf{q=0},\Omega)$ is a dimensionless integral.
Introducing the average SOC $\lambda=\frac12(\lambda_I+\lambda_R)$, $\delta\alpha_G$ can be written as
\begin{eqnarray}
\delta\alpha_G=-\alpha_G^0 \left(\frac{\lambda}{\hbar \Omega}\right)^2\tilde{\Sigma}(\mathbf{q=0},\Omega)
\end{eqnarray}
where $\alpha_G^0=2S_0\frac{T_0^2}{\lambda^2}$.
 
In Fig.~\ref{GD1-supp}, we plot the normalized Gilbert damping correction $\delta\alpha_G/\alpha_G^0$ as a function of the twist angle $\theta$ and the FMR energy $\hbar\Omega$ at different temperatures. The bottom panels are a zoomed representation around the MA.
Fig.~\ref{GD1-supp} shows that, the GD increases by decreasing the twist angle and sharply drops at the MA, regardless of the temperature range and the FMR frequency .\
 
At high temperature ($k_BT>\lambda$), the GD exhibits, around the MA, a fine structure characterized by a peak which disappears at low temperature.
The origin of this peak is, as discussed in the main text, due to the dispersion of the energy bands of TBG with the SOC and their corresponding thermal weights.


%%%%%%%%%%%%%%%%%%%%%%%%%%%%%%%%%%%%%%%%%%%%%%%%%%%%%%%%%%%

\begin{figure*}[hpbt]
\begin{center}
\includegraphics[width=0.8\columnwidth]{theta-omega-T.eps}
\includegraphics[width=0.8\columnwidth]{theta-omega-T-MA.eps}
\end{center}
\caption{Color plot of the normalized Gilbert damping correction $\delta\alpha_G/\alpha_G^0$ as a function of the twist angle $\theta$ and the FMR energy $\hbar \Omega$ at $k_BT=0.1\;\mathrm{meV} $ ((a) and (d)), $k_BT=1\;\mathrm{meV} $ ((b) and (e)) and $k_BT=25\;\mathrm{meV} $ ((c) and (f)). The bottom panels show the behavior of the GD around the MA.
Calculations are done for $\mu=0$, $\lambda_I=3\;\mathrm{meV}$ and $\lambda_R=4\,\mathrm{meV}$.}
\label{GD1-supp}
\end{figure*}

According to Eq.\ref{lor}, the GD depends on the thermal weight $\Delta f(E)=f(E_{\langle S_z\rangle})-f(E_{-\langle S_z\rangle})$ (see main text).
In Fig.~\ref{FD-supp} we plot $\Delta f(E)$ corresponding to the transitions between $E_{-,+}\rightarrow E_{+,+}$ and $E_{-,-}\rightarrow E_{+,-}$ in the case of the undoped system.\

Figures~\ref{FD-supp} (a) and (b) show that, at high temperature ($k_BT>\lambda_I,\lambda_R$), $\Delta f(E)$ increases as the band dispersion gets larger and reaches its minimal value at the MA. This behavior explains the drop of the GD at the MA and its enhancement at small twist angles.\

In figure~\ref{FD-supp} (c), we plot $\Delta f(E)$ around the MA, at relatively high thermal energy compared to the SOC, where the GD exhibits a peak at the MA (Fig.~2 of the main text).
In this case, $\Delta f(E)$ is maximal at the MA and decreases at the angles $\theta^+_M$ and $\theta^-_M$ close to the MA.
This feature results from the decrease of the energy separation between $E_{-,-}$ and $E_{+,-}$ at $\theta^+_M$ and $\theta^-_M$, compared to that at $\theta_M$ (Fig.~3 of the main text). \
At low temperature, and around the MA, one gets $\Delta f(E)=1$ for the transitions  $E_{-,-}\rightarrow E_{+,-}$ and $E_{-,+}\rightarrow E_{+,+}$. As a consequence, the GD behavior is now only dependent on the effective Fermi velocity $v^{ast}$ which vanishes at the MA. As a consequence, the small peak of the GD, emerging at the MA at relatively high temperature, disappears.\\

\begin{figure*}[hpbt] 
\begin{center}
$
\begin{array}{ccc}
\includegraphics[width=0.3\columnwidth]{diff-FD-E1-E2-kbT.eps}
\includegraphics[width=0.3\columnwidth]{diff-FD-E3-E4-kbT.eps}
\includegraphics[width=0.3\columnwidth]{diff-FD-MA-zoom-kbT.eps}
\end{array}
$
\end{center}
\caption{Statistical weight $\Delta f(E)$ corresponding to the transitions between (a) $E_{-,+}\rightarrow E_{+,+}$ and (b) $E_{-,-}\rightarrow E_{+,-}$ at different temperatures. The dots in (a) ((b) and (c)) represent the energy $E_{+,+}$ ($E_{+,-}$) at the MA and the arrows mark the limit of the band $E_{+,+}$ ($E_{+,-}$) at the indicated twist angle. In (c), $\Delta f(E)$ is shown around the MA for the transition between $E_{-,-}\rightarrow E_{+,-}$. Calculations are done for the SOC $\lambda_I=3\;\mathrm{meV}$, $\lambda_R=4\;\mathrm{meV}$~\cite{Alex} and in the undoped TBG ($\mu=0$).}
\label{FD-supp}
\end{figure*}

Figure \ref{GD-mu} shows the behavior of the normalized GD correction $\delta\alpha_G/\alpha_G^0$ as function of the chemical potential $\mu$ at $k_BT=25\,\mathrm{meV}$ and for the FMR energy $\hbar \Omega=0.06\,\mathrm{meV}$. The decrease of $\delta\alpha_G/$ is a consequence of the thermal weight.\\

\begin{figure*}[hpbt] 
\begin{center}
\includegraphics[width=0.5\columnwidth]{GD-mu-kBT=25.eps}
\end{center}
\caption{Normalized GD correction $\delta\alpha_G/\alpha_G^0$ as function of the chemical potential $\mu$ at $k_BT=25\,\mathrm{meV}$. The upper limit of is $mu_c=\hbar v_F k_c$ corresponding to the momentum cutoff $k_c=\frac{q_0}2$. Calculations are done for the SOC $\lambda_I=3\;\mathrm{meV}$, $\lambda_R=4\;\mathrm{meV}$~\cite{Alex}, $k_BT=25\,\mathrm{meV}$ and for the FMR energy $\hbar \Omega=0.06\,\mathrm{meV}$.}
\label{GD-mu}
\end{figure*}

%%%%%%%%%%%%%%%%%%%%%%%%%%%%%%%%%%%%%%%%%%%%%%%%%%%%%%%%%%%%%
\end{document}


