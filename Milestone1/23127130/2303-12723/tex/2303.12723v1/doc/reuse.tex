\section{Mask Reuse With Shift Calibration}\label{shift}

\subsection{Mask Reusability}
We assume that repeating patterns can share mask for efficiency if the query design pattern $\mathbf{P}$ has a matched design pattern $\mathbf{P'}$ stored in the library with the same shape. However, \Cref{fig:slicing} has shown that when the whole design is sliced into small patterns, it is inevitable to find pattern location shift $(\Delta x, \Delta y)$ between $\mathbf{P}$ and $\mathbf{P'}$. In real lithography and OPC flow, if no surrounding factors affect lithography, the printed wafer image patch must have no distortion but only an identical shift to the design pattern, as visualized in \Cref{fig:shift}. Therefore, if we want to reuse the mask, the first requirement is to make sure that location shift during lithography will not cause any geometric distortion.
% \enspace As mentioned in \Cref{intro}, repeatitive patterns can share mask for efficiency. If the query design pattern $\mathbf{P}$ has a matched design pattern $\mathbf{P'}$ stored in the library wtih same shape, We simply reuse the stored mask $\mathbf{M}_{\mathbf{P'}}$ of the matched reference pattern. An pattern location shift $(\Delta x, \Delta y)$ between $\mathbf{P}$ and $\mathbf{P'}$ is inevitable when the whole design is sliced into small patterns. The optimized masks after OPC shall hold same geometric shape but also with location shift $(\Delta x, \Delta y)$. Thus, we can simple correct the mas $\mathbf{M}_{\mathbf{P'}}$ with shift $(-\Delta x, -\Delta y)$ as the initial mask $\mathbf{M}_{\mathbf{P}}$.  

We mathematically prove the location shift remains unchanged before and after the lithography in order to show the feasibility of the mask shift calibration approach, we denote the Hopkins diffraction model through lithography in \Cref{litho_model} as $Litho(\cdot)$ and the location shift as $\delta_{\Delta x, \Delta y}(\cdot)$, we show that:

\begin{mytheorem}[Shift Equivariance]
    \label{shift_theorem}
    Given pattern $\mathbf{P}$ and mask $\mathbf{M}_{\mathbf{P}}$ where
    \begin{equation}
        \mathbf{P} = Litho(\mathbf{M}_{\mathbf{P}}) .
    \end{equation}
     The following statement always holds: 
    \begin{equation}
        \label{litho_equivariance}
        \delta_{\Delta x, \Delta y}(\mathbf{P}) = Litho(\delta_{\Delta x, \Delta y}(\mathbf{M}_{\mathbf{P}})).
    \end{equation}
\end{mytheorem}
\begin{proof}
    For any position (x,y) on pattern $\mathbf{P}$:
    \begin{equation}
        \begin{aligned}
            &\delta_{\Delta x, \Delta y}(\mathbf{P}(x,y)) = \mathbf{P}(x + \Delta x,y + \Delta y), \\
            &= \sum_{k = 1}^{N^2} w_{k} \left| h_{k}(x + \Delta x,y + \Delta y) \otimes \mathbf{M}_{\mathbf{P}}(x + \Delta x,y + \Delta y) \right|^{2}, \\
            &= \sum_{k = 1}^{N^2} w_{k} | \sum_{i=1}^{N} \sum_{j=1}^{N} h_{k}(i,j) \mathbf{M}_{\mathbf{P}}(x + \Delta x + i - \frac{N}{2}, y + \Delta y + j - \frac{N}{2})|^{2}, \\[4pt]
            &= \sum_{k = 1}^{N^2} w_{k} | \sum_{i=1}^{N} \sum_{j=1}^{N} h_{k}(i,j) \mathbf{M}_{\mathbf{P}}(x + i - \frac{N}{2}+ \Delta x, y + j - \frac{N}{2} + \Delta y)|^{2}, \\[4pt]
            & = \sum_{k = 1}^{N^2} w_{k} | \sum_{i=1}^{N} \sum_{j=1}^{N} h_{k}(i,j) \delta_{\Delta x, \Delta y}(\mathbf{M}_{\mathbf{P}}(x + i - \frac{N}{2}, y + j - \frac{N}{2}))|^{2}, \\[4pt]
            & = \sum_{k = 1}^{N^2} w_{k} \left| h_{k}(x,y) \otimes \delta_{\Delta x, \Delta y}(\mathbf{M}_{\mathbf{P}}(x,y)) \right|^{2}, \\[4 pt]
            & = Litho(\delta_{\Delta x, \Delta y}(\mathbf{M}_{\mathbf{P}}(x,y))) .
        \end{aligned}
    \end{equation}
    Then \Cref{litho_equivariance} is proved.
\end{proof}
\begin{figure}
    \includegraphics[width=0.95\linewidth]{figs/shift} 
    \caption{Printed wafer image must share identical location shift to design pattern with no geometric shape distortion.}
    \label{fig:shift}
\end{figure}

Since mask shift will only result in a printing shift after lithography, repeating patterns in design can share OPC-optimized masks with a simple shift correction. We pick matched mask $\mathbf{M}_{\mathbf{P'}}$ stored in pattern library and add correction $(-\Delta x, -\Delta y)$ to acquire the initial mask $\mathbf{M}_{\mathbf{P}}$ for $\mathbf{P}$.

\subsection{Pattern Shift Calibration}
We calculate the shift by computing the pixel-level similarity of two patterns  $\mathbf{P}$ and $\mathbf{P'}$. The pixel-wise \textbf{cross-correlation} of $\mathbf{P}$ and $\mathbf{P'}$ reflects the pixel-wise similarity where the pixel of highest response value on the correlation map is the position shift of center point $(x_{ctr}, y_{ctr})$. The cross-correlation computation of two large 2-D pattern is time-comsuming. The calculation process of cross-correlation equal to convolution of $\mathbf{P}$ and $Rotate(\mathbf{P'})$, where $Rotate(\cdot)$ denotes rotation of $180^{\circ}$:
\begin{equation}
    CrossCorr(\mathbf{P}, \mathbf{P'}) = Conv(\mathbf{P}, Rotate(\mathbf{P'})),
\end{equation}
We replace the calculation with Convolution and accelerate the computation with Fast Fourier Transform (FFT) \cite{vasilache2014fast}. The pattern shift can be calculated with:
\begin{equation}
    \begin{aligned}
        x^{*}, y^{*} &= \argmax_{x,y}\  Conv\_FFT (\mathbf{P}, Rotate(\mathbf{P'})), \\
        \Delta x &= x^{*} - x_{ctr},\  \Delta y = y^{*} - y_{ctr},
    \end{aligned}
\end{equation}
And the initial mask is corrected with:
\begin{equation}
    \mathbf{M}_{\mathbf{P}} = \delta_{-\Delta x, -\Delta y}(\mathbf{M}_{\mathbf{P'}}).
\end{equation}
In real application, we send calibrated mask into lithography model to verify the mask and into ILT solver for one or two further iterations if necessary, just in case of any noise caused by shift calibration operation.
We use the same pattern size as \cite{yang2019gan} $2048 \times 2048$. With our implementation, the shift calculation time is less than 0.25s on CPU.  
