\section{Introduction}\label{intro}
\label{sec:intro}
Over the past few decades, VLSI technology node has been continuously shrinking, resulting non-neglectable lithography proximity effect, which affects the real manufacturability \cite{pan2013design}. Resolution enhancement techniques (RETs) are utilized to improve the printability in the lithography process. Optical Proximity Correction (OPC) is one of the widely used RETs to optimize mask printability by compensating for the diffraction effect in the lithography process. 

OPC approaches can be categorized into: 
\begin{enumerate*}
    \item rule-based OPC \cite{park2000efficient}, 
    \item model-based OPC \cite{kuang2015robust, su2016fast, matsunawa2015optical}, 
    \item inverse lithography technique (ILT)-based OPC \cite{poonawala2007mask, ma2020unified, yu2021gpu} and 
    \item machine learning (ML)-based OPC \cite{yang2019gan, jiang2019fast, geng2019sraf, chen2021damo}.
\end{enumerate*}
% rule-based OPC~\cite{park2000efficient}, model-based OPC~\cite{kuang2015robust, su2016fast, matsunawa2015optical}, inverse lithography technique (ILT)-based OPC~\cite{poonawala2007mask, ma2020unified, yu2021gpu} and machine learning (ML)-based OPC~\cite{yang2019gan, jiang2019fast, geng2019sraf, chen2021damo}. 
Rule-based methods solve the problem heuristically, which is simple and fast but only suitable for less aggressive designs. 
Model-based OPCs mathematically model the lithography process and move/shift the edge fractures accordingly, ensuring mask fidelity but restricted by the solutions space in advanced technology nodes. 
ILT-based methods solve the inverse problem of the imaging system through optimizing an objective function, which is the most performant analytical method to tackle the OPC problem. 
%\cite{poonawala2007mask} firstly proposed a pixel-based mask representation for ILT-based mask update.  
%ILT-based methods iteratively update mask by gradient descent of error between the wafer image and the target, converging to an analytically optimized mask. 
%\cite{yu2021gpu} proposed a level set method as boundary evolution to update the mask to reduce the number of iterations. 
%However, the multiple rounds of lithography simulation and gradient backpropagation are still time-consuming, especially when the design scale gets larger. 
As recent years witnessed the rapid development of machine learning algorithms and hardware,
ML-based OPCs have shown remarkable speed-up in the OPC flows and are prevailing in design for manufacturing (DFM) academia. \cite{yang2019gan} and \cite{chen2021develset} use deep learning models for initial mask generation to reduce the iterations number of ILT.
\cite{jiang2020neural} uses a deep learning model to simulate the conventional ILT correction process.
A lethal drawback is that machine learning model is a data-driven black box. Such methods are not guaranteed to work for some critical patterns. 
In summary, no approach is flawless and shows absolute superiority over others. Patterns with different complexity require different approaches.
% \todo{summarize: different approaches are suitable for designs of different complexity.}

%Moreover, once the lithography model or the design technology node changes, a well-trained DL model is not guaranteed to provide a valid mask anymore. Apart from that, sufficient training data is also required once the manufacturing settings change. 

\begin{figure}[tb!]
    \raggedleft
    \includegraphics[width=\linewidth]{figs/intro} 
    \caption{Visualization of a real design layer. Two discoveries motivated our OPC framework design: 1. Patterns scattered unevenly along the design layout with different complexity. We denote complicated patterns as critical and simple patterns as non-critical. 2. Patterns have large ratio of repetition on a full layout. }
    \label{fig:whole design}
\end{figure}

% Different approaches have certain pros and cons regarding their methodology characteristics. Rule-based methods solve the problem heuristically, which is simple and fast; however not robust enough and only suitable for less aggressive designs. Despite their effectiveness in generating high fidelity mask, model-based methods are restricted by the solutions space and becoming more challenged by advanced technology nodes. 
% ILT is the most performant analytical method to tackle the OPC problem under white box testing scenario. Such pixelated mask optimization enables better contour fidelity. However, the multiple rounds of lithography simulation and gradient backpropagation demands heavy time consumption, especially when the design scale gets larger. ML-based OPC methods are prevailing in design for manufacturing (DFM) academia. Especially deep learning has shown excellent performance on some contest benchmarks \cite{banerjee2013iccad} in terms of speed and accuracy. Nevertheless, a lethal drawback is that machine learning model is a data-driven black box. Such methods are not guaranteed to work for some critical patterns. Moreover, once the lithography model or the design technology node changes, a well-trained DL model is not guaranteed to provide a valid mask anymore. Apart from that, sufficient data for training is also required once the manufacturing settings change. 
%Previous OPC solutions intended to tackle the problem with technique innovations but only focused on mask optimization of design patterns with a fixed size (e.g., $2048\times 2048$).
% Inspecting the pattern distribution on a real design, we discovered that patterns are placed on a full design layout with both diversity and similarity characteristics.
Achieving desired OPC results with high efficiency on real designs require a systematic analysis on pattern distribution and complexity. 
By inspecting a real design, we come up with a few properties that can be leveraged to assist the analysis. 
Firstly, we notice that there exists a \emph{diversity} of pattern density, which indicates that some regions are dense while some regions are sparse, as visualized in \Cref{fig:whole design}, which implicates different kinds of OPC solutions are required. 
%\cite{chen2021damo} has extended the OPC problem to full-scale design by applying a clustering step to slice the layout. However, no approach has considered the full design characteristics.  
%We investigated into real full design layer and discovered some useful global features that can help us to improve the efficiency and performance of the complete OPC process.
%As visualized in \Cref{fig:whole design}, we notice that a full design layer possesses different densities at different sub-regions. 
Moreover, as we take a closer observation of each sub-region, there also exists a \emph{similarity} of the pattern distribution in different sub-regions. 
Many patterns are repeatedly placed on the full design layer. These patterns have similar geometric shapes and are placed at different locations. Such pattern repetition enables us to utilize their shared geometric characteristics, which motivates the idea that the OPC solution of a pattern can be reused in a similar pattern for efficiency improvement.  
Motivated by these observations, we propose a self-adaptive framework, namely AdaOPC, for conducting OPC on real designs.

Firstly, AdaOPC is equipped with pattern analysis that can classify a sub-region as critical or non-critical, such that the proper OPC solver can be selected. 
%Different approaches have certain pros and cons regarding their methodology characteristics. 
%We present a flexible self-adaptive OPC framework for mask optimization in a real scenario for a whole design layout. 
More specifically, for densely scattered sub-regions, not only the diffraction effect but also the optical interference of incident light caused by neighboring components will jointly affect the final printed image. 
Such complex patterns for OPC are regarded as critical and more suitable for robust yet rigorous numerical optimization methods for higher manufacturability. 
In contrast, sub-regions with sparsely scattered patterns are simpler, and the mask optimization process is more suitable for machine learning models to learn with superior inference speed.  
% More importantly, in a real OPC scenario with a new design and even possibly a new lithography engine, patterns and optimized masks generated by robust methods can be the dataset to train the machine learning model to adapt to new settings. Apart from that, Our framework is flexible by maintaining an extensible solver pool. More advanced techniques with superior performance accuracy or speed are applicable to be taken in as a candidate solution. 

Secondly, given that many patterns are repetitive on the design layer with the same geometric shape, as shown in \Cref{fig:whole design},  we investigate the feasibility of reusing the optimized masks of repeating patterns to avoid redundant iterations of OPC. 
There are three obstacles standing in the way of our idea:
\begin{enumerate*}
    \item As shown in \Cref{fig:slicing}, slicing the large design layout into small patterns inevitably result in a location shift of patterns with the same geometric shape. Whether and how can an optimized mask with location shift be reused?
    \item Given the query pattern, how to match a same pattern from a large number of stored ones accurately within an acceptable time? 
    \item How to measure the geometric similarity of patterns with location shift?
\end{enumerate*}
In response to the mentioned three questions, we build a \textbf{dynamic pattern library} with online updating to store and reuse repeating patterns and optimized masks by constructing a dynamic hierarchical graph.
We mathematically prove the shift equivariance property of the lithography process to show the feasibility of mask reuse by calculating the shift of design pattern and calibrating the mask. 
%We also propose a fast shift calculation approach by comparing the pixel-wise similarity of matching patterns with an efficient cross-correlation computation. 
A graph-based approximation nearest neighbor search for pattern matching within a short query time. 
%As the geometric similarity of patterns cannot be directly measured because of the library's pattern complexity and large size, we embed sliced patterns into multi-dimensional vectors with deep metric learning and supervised contrastive loss. The similarity is measured by distance metrics in the embedding space.
We summarize the contributions of this paper as follows:
\begin{itemize}
    \item We propose a self-adaptive workflow for flexible OPC solver selection.
    \item We prove the feasibility of mask resue to speed up the OPC process for real design patterns and provide an efficient mask shift calibration method in practice.
    \item We generate design patterns embedding by supervised contrastive learning for similarity measurement and pattern matching.
    \item We construct a dynamic pattern library using a hierarchical graph with online update along with a greedy graph-based nearest neighbor search for fast matching.
    \item With experiments on different pattern cases from a real design layout, we proved our framework can reduce over 90\% runtime while still preserving the optimal OPC performance.
\end{itemize}
\begin{figure}
    \centering
    \includegraphics[width=0.9\linewidth]{figs/slicing}
    \caption{Slicing repeating full layout inevitably causes some location shift on repeating patterns.}
    \label{fig:slicing}
\end{figure}
