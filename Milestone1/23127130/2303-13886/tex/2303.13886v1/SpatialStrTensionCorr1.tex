
\documentclass[12pt]{article}

\usepackage{graphicx}
\usepackage{amsmath}
\usepackage[cp866]{inputenc}
\usepackage[english]{babel}

\sloppy
\textwidth 16cm
\textheight 23cm
\hoffset=-1.5cm
\voffset= -3cm

\title{The spatial string tension  from the Field Correlator Method}
\author{ N. O. Agasian, Z. V. Khaidukov and  Yu. A. Simonov  \\
NRC ``Kurchatov Institute'', Moscow, Russia}
% \date{}

\newcommand{\beq}{\begin{eqnarray}}
 \newcommand{\eeq}{\end{eqnarray}}
\newcommand{\be}{\begin{equation}}
 \newcommand{\ee}{\end{equation}}

 \def\la{\mathrel{\mathpalette\fun <}}
\def\ga{\mathrel{\mathpalette\fun >}}
\def\fun#1#2{\lower3.6pt\vbox{\baselineskip0pt\lineskip.9pt
\ialign{$\mathsurround=0pt#1\hfil ##\hfil$\crcr#2\crcr\sim\crcr}}}
\newcommand{\veX}{\mbox{\boldmath${\rm X}$}}
\newcommand{{\SD}}{\rm SD}
\newcommand{\pp}{\prime\prime}
\newcommand{{\Mc}}{\mathcal{M}}
\newcommand{\veY}{\mbox{\boldmath${\rm Y}$}}
\newcommand{\vex}{\mbox{\boldmath${\rm x}$}}
%\newcommand{\vexi}{\mbox{\boldmath${\rm \xi}$}}
\newcommand{\vey}{\mbox{\boldmath${\rm y}$}}
\newcommand{\ver}{\mbox{\boldmath${\rm r}$}}
\newcommand{\vesig}{\mbox{\boldmath${\rm \sigma}$}}
\newcommand{\vedelta}{\mbox{\boldmath${\rm \delta}$}}
\newcommand{\veP}{\mbox{\boldmath${\rm P}$}}
\newcommand{\vep}{\mbox{\boldmath${\rm p}$}}
\newcommand{\veq}{\mbox{\boldmath${\rm q}$}}
\newcommand{\veQ}{\mbox{\boldmath${\rm Q}$}}
\newcommand{\vez}{\mbox{\boldmath${\rm z}$}}
\newcommand{\veS}{\mbox{\boldmath${\rm S}$}}
\newcommand{\veL}{\mbox{\boldmath${\rm L}$}}
 \newcommand{\veA}{\mbox{\boldmath${\rm A}$}}
\newcommand{\veR}{\mbox{\boldmath${\rm R}$}}
\newcommand{\ves}{\mbox{\boldmath${\rm s}$}}
\newcommand{\vek}{\mbox{\boldmath${\rm k}$}}
\newcommand{\ven}{\mbox{\boldmath${\rm n}$}}
\newcommand{\veu}{\mbox{\boldmath${\rm u}$}}
\newcommand{\vev}{\mbox{\boldmath${\rm v}$}}
\newcommand{\veh}{\mbox{\boldmath${\rm h}$}}
\newcommand{\vew}{\mbox{\boldmath${\rm w}$}}
\newcommand{\verho}{\mbox{\boldmath${\rm \rho}$}}
\newcommand{\vexi}{\mbox{\boldmath${\rm \xi}$}}
\newcommand{\veta}{\mbox{\boldmath${\rm \eta}$}}
\newcommand{\veB}{\mbox{\boldmath${\rm B}$}}
\newcommand{\veH}{\mbox{\boldmath${\rm H}$}}
\newcommand{\veE}{\mbox{\boldmath${\rm E}$}}
\newcommand{\veJ}{\mbox{\boldmath${\rm J}$}}
\newcommand{\veal}{\mbox{\boldmath${\rm \alpha}$}}
\newcommand{\vepi}{\mbox{\boldmath${\rm \pi}$}}
\newcommand{\vemu}{\mbox{\boldmath${\rm \mu}$}}
\newcommand{\vegam}{\mbox{\boldmath${\rm \gamma}$}}
\newcommand{\vepar}{\mbox{\boldmath${\rm \partial}$}}
\newcommand{\llan}{\langle\langle}
\newcommand{\rran}{\rangle\rangle}
\newcommand{\lan}{\langle}
\newcommand{\ran}{\rangle}


\begin{document}
%\baselineskip 11.5pt
\maketitle
\begin{abstract}

The phenomenon of the almost linear growth of the square root of spatial string tension $\sqrt{\sigma_s(T)}= c_{\sigma} g^2 T$  was found both in lattice and in theory, based on the Field Correlator Method (FCM). In the latter the string tension (both spatial and colorelectric) is expressed as an integral of the two gluon Green's function calculated with the same string tension: $\sigma= \int G^{(2g)}_{\sigma}$. This relation allows to check the selfconsistency of the theory. However at nonzero
temperature $T$ in the two-gluon Green's function in the space-like region appear terms which create in $\sigma_s$ quadratic in T behavior.  We calculate below in the paper the $\sigma_s$  numerically in the whole temperature region $T_c< T < 5 T_c$ using the FCM method and compare the results with lattice data finding a good agreement. This justifies the use of the FCM in the space-like region and in high T thermodynamics without extra parameters.

\end{abstract}

\section{Introduction}

The phenomenon of confinement in QCD was explained in the framework of the Field Correlator Method (FCM) \cite{1,2,3,4,5,5*},
both qualitatively and quanitatively, via the vacuum field correlators  of the
colorelectric (CE) and the colormagnetic (CM) fields $E_i^a,H_i^a$, and at the temperature
$T=0$ the behavior of all physical quantities is expressed  via the only nonperturbative
parameter -- the string tension, $\sigma_E=\sigma_H= \sigma$.
The resulting field correlators $D^E(x),D^H(x)$ are the so-called gluelump Green's functions,
which define all confining QCD dynamics. These gluelumps have been calculated in good agreement
between the FCM \cite{6} and the lattice data \cite{7}, while $D^E(x),D^H(x)$ were also studied
in detail at $T>0$ on the lattice \cite{5}. It was found that
the situation is drastically changing at $T>0$, where both $\sigma_E(T)$ and $\sigma_H(T)= \sigma_s(T)$
have  different behavior. Namely  $\sigma_E(T)$
disappears above $T=T_c$, while in contrast to that $\sqrt{\sigma_s(T)}$
grows almost linearly at large $T$, as it was
found on the lattice \cite{4,7*,8,9} and supported by the studies
in the framework of the FCM \cite{10,10*,11,12,12*,12**,18,15,14}.
This dependence contains the coefficient, denoted as $c_{\sigma}$  \cite{8,9},

\be
\sqrt{\sigma_s(T)}= c_{\sigma} g^2(T) T,
\label{1}
\ee
defined numerically in the lattice calculations \cite{8,9} in the case  $N_c=3, N_f=0$ as
\be
c_{\sigma}= 0.566\pm 0.013.
\label{2}
\ee
For the $SU(2)$ theory a similar lattice calculation \cite{13,13*} has given a smaller value,
\be
c_{\sigma}^2= (0.136 \pm 0.011),~~ c_{\sigma}= 0.369\pm 0.015.~~
\label{3}
\ee

On the theoretical side the linear temperature growth of $\sigma_s(T)$ was derived in the framework of FCM \cite{10,11,12}, where the basic notion is the colormagnetic (CM) field correlator
$D^H(z)$, which can be expressed via the  gluelump Green's function $G(x.y|z)$ and in this case the (two-gluon) gluelump  is the string triangle, created by two gluons and the adjoint spectator point. The exact theory of these gluelumps was given in \cite{6,10,11} and will be discussed in the next section in the case of the CM correlators. Note that the gluelump Green's
functions define the field correlators in the self-consistent way, without fitting  parameters.  In the case of the CM field  the resulting $\sigma_s(T)$ has exactly the same form as in (\ref{1})
\cite{12} with $c_{\sigma}$, expressed via the gluelump Green's function integral $G(0,0|z)$,
\be
c_{\sigma}^2= \frac{N_c^2 -1}{4} \int d^2 z G(0,0|z).
\label{4}
\ee
In the FCM the numerical values of $c_{\sigma}$ were  calculated in \cite{12*}
using the oscillator basis with the subsequent  correction coefficient for the
linear spatial confinement (see Appendix A2 of the present paper and \cite{12*})
which demonstrate a close agreement with the lattice data of \cite{8,9}.
However in the full FCM expression for the spatial string tension
the term in (\ref{1}) is only a fast growing part  of the whole expression which was hitherto not known.

It is the purpose of the present paper to derive the total expression of the spatial string tension including the linear in $T$ part and to calculate the numerical value of $\sigma_s(T)$ in the whole region of $T$ and compare it with the lattice data.
As will be seen , the results, obtained within the FCM,   provide the values
of $c_{\sigma}$ in the same ballpark with lattice data with high accuracy.
In the next section we discuss the general expression for $\sigma_s(T)$ in terms of the gluelump Green's function  and in the
section 3 we formulate its final form, which is discussed and compared with the lattice data in the section 4. The 3 Appendices
give the explicit form of the renormalized coupling constant $g^2(T)$ (Appendix A1), the calculation of the coefficient $c_{\sigma}$ (Appendix A2) and finally the details of the expression of the string tension $\sigma_s(T)$ via the gluelump Green's function
(Appendix A3).

\section{The spatial string tension in the FCM}

In Introduction we have mentioned that the spatial Green's function, which defines the $\sigma_s$, is the Green's function of two gluons and a straight gluonic Wilson line (the parallel transporter), which automatically in the confining vacuum are all connected by three confining strings, and the string tension is proportional to the integral of this Green's function in the $3d$ space, where one of three space coordinates can be taken as an evolution parameter (the Euclidean ``time"). To calculate $\sigma_s(T)$ one can start with the technic, developed in \cite{10,11,12} for
$D^E(z), D^H(z)$, which allows to express it via the two-gluon  Green's function: $G^{(2g)}_{4d} (z) =  G_{4d}^{(g)}\otimes G_{4d}^{(g)}$, where two gluons
interact nonperturbatively, and later we shall neglect  the spin interactions in the first approximation. (See Appendix A3
for the details of derivation)

\be
(-D^2)^{-1}_{xy}=\left\lan x\left|\int^{\infty}_0 dt
e^{tD^2(B)}\right|y\right\ran=
\int^{\infty}_0dt(Dz)^w_{xy}e^{-K}\Phi(x,y), \label{6}
\ee where
\be
K=\frac{1}{4}\int^s_0d\tau \left(\frac{dz_\mu}{d\tau}
\right)^2, ~~\Phi(x,y)=P \exp ig\int^x_yB_{\mu}dz_{\mu},\label{7}
\ee and a winding path measure is \be (Dz)^w_{xy}=\lim_{N\to
\infty}\prod^{N}_{m=1}\frac{d^4\zeta(m)}{(4\pi\varepsilon)^2}
\sum^{+\infty}_{n=-\infty}
\int\frac{d^4p}{(2\pi)^4}e^{ip(\sum\zeta(m)-(x-y)-n\beta\delta_{\mu
4})}. \label{8} \ee

The starting point  for the gluon propagator $G^{(g)}_{4d}$ is the integration in the 4-th direction in (\ref{6}) with the exponent $K_4 = \frac14 \int^s_0 d\tau \left( \frac{d z_4}{d\tau}\right)^2,$ which gives for the spatial loop with $x_4=y_4$,
\be
J_4 \equiv \int (Dz_4)_{x_4x_4} e^{-K_4} = \sum^{+\infty}_{n=-\infty} \frac{1}{2 \sqrt{\pi s}} e^{-\frac{(n\beta)^2}{4s}}
%=
%\frac{1}{2\sqrt{\pi s}} \left( 1+ \sum_{n=\pm 1,\pm 2}
%e^{-\frac{(n\beta)^2}{4s}}\right).
\label{9}
\ee
%The  second term in (\ref{9}) at large $T\gg \frac{1}{2\sqrt{s}}$ yields
%$2\sqrt{\pi s} T$,    which gives two different limiting values,
%\be
%J_4 =\frac{1}{2\sqrt{\pi s}} for T=0, J_4 = T, for T \gg (2\sqrt{\pi s})^(-1).
%\label{10}
%\ee
One can notice that the sum in (\ref{9}) is a known function
\be
\sum^{+\infty}_{n=-\infty}e^{-\frac{n^2}{4sT^2}}\equiv\vartheta_3(q),~~~q=e^{-\frac{1}{4sT^2}},
\label{10a}
\ee
where the function $\vartheta_3(q)$ is defined as
\be
\vartheta_3(q)=\sum^{+\infty}_{n=-\infty}q^{n^2}=1+2q+2q^4+O(q^9)
\label{10b}
\ee
Then starting from low temperature there is an expansion
$$
J_4=\frac{1}{2\sqrt{\pi s}}\sum^{+\infty}_{n=-\infty}e^{-\frac{n^2}{4sT^2}}
\equiv\frac{1}{2\sqrt{\pi s}}\vartheta_3(e^{-\frac{1}{4sT^2}})
$$
\be
=\frac{1}{2\sqrt{\pi s}}(1+2e^{-\frac{1}{4sT^2}}+O(e^{-\frac{1}{sT^2}}))
\label{10c}
\ee
To find the asymptotics at high T one can use the relation
\be
\sum^{+\infty}_{n=-\infty}e^{-\frac{\beta^2n^2}{4s}}=
\frac{2\sqrt{\pi s}}{\beta} \sum^{+\infty}_{n=-\infty} e^{-\frac{4\pi^2n^2}{\beta^2}s}
\label{10d}
\ee
As a result at large T one obtains an equality
$$
J_4=T\sum^{+\infty}_{n=-\infty}e^{-4\pi^2sT^2n^2}\equiv T\vartheta_3(e^{-4\pi^2sT^2})
$$
\be
=T(1+2e^{-4\pi^2sT^2}+O(e^{-16\pi^2sT^2}))
\label{10e}
\ee
As a result one can use $J_4$ at an arbitrary T in the form
\be
J_4(s,T)\equiv\frac{1}{2\sqrt{\pi s}}\vartheta_3(e^{-\frac{1}{4sT^2}})
\label{10f}
\ee
In this way starting from the low T one obtains an exact expression for $J_4(T)$ valid in the whole range of $T$.
One could approximate this behavior  as a sum of linear and constant term implying a soft transition from
$T=0$ case to the linear in $T$ behavior however this approximation fails numerically (see Appendix 3 for details) and one should take into account the sharp transition at some intermediate point $T^*$ from the regime $T=0$ to the large $T$ behavior
which is given by (\ref{10f}). As a result
we express the $J_4$ as follows
\be
J_4= \frac{1}{2\sqrt{\pi s}} \left( \sum^{+\infty}_{n=-\infty}e^{-\frac{(n\beta)^2}{4s}}\right)=
\frac{1}{2\sqrt{\pi s}} f(\sqrt{s}T).
\label{11}
\ee
At this point we turn to the general form of the field correlator $D^H(z)$ with the aim to express the string tension via the
factors $f(x)$. One has
\be
D^H(z) = \frac{g^4(N^2_c-1) }{2} \lan G^{(2g)} (z,T) \ran,
\label{12}
\ee
where $G^{(2g)}(z,T)$ is the gluelump Green's function
\be
G^{(2g)}(z,T)= \frac{z}{8\pi} \int \frac{d\omega_1}{\omega_1^{3/2}} \frac{d\omega_2}{\omega_2^{3/2}} D^{3}r_1 D^{3}r_2
\exp{(-K_1-K_2-V(\ver_1,\ver_2)z)}
\label{13}
\ee
As a result one obtains $\sigma_s(T)$ in the following form
$$
\sigma_s(T)= \frac{g^4(N_c^2-1)}{4}\int d^2z z/(8\pi) \int d\omega_1 d\omega_2 (\omega_1\omega_2)^{-3/2}
$$
\be
\times \sum_{n=0,1,} |\psi_n(0,0)|^2 \exp(-M_n(\omega_1,\omega_2)z) f(\sqrt{z/2\omega_1}T)f(\sqrt{z/2\omega_2}T),
\label{14}
\ee
The integrals in (\ref{14}) without factors $f(cT)$ do not contain the temperature dependent factors, and
one can see in (\ref{14}) the only T-dependent factors $g^4(T)$ and $f(\sqrt{z/2\omega_1}T)$ which define the dependence of $\sigma_s(T)$. Therefore one can write $\sigma_s(T)$ in the following form
\be
\sigma_s(T)={\rm const} g^4(T) <f^2(\sqrt{z/(2\omega)}T)>= {\rm const} g^4(T) f^2(<\sqrt{z/2\omega)}>T)
 \label{15} \ee
  The appearance of $g^4(T)$ which is decreasing with $T$ as $(\ln T)^{-2}$ defines the $T$
  dependence of $\sigma_s(T)$ to be lower than $T^2$, thus confirming the behavior of $\sigma_s(T)$
  in the lattice data of \cite{8}, where the data were fitted
   as $\sigma_s(T)= {\rm const} g^4(T) T^2$ . However this fit fails for $T<2T_c$
   claiming the necessity of another factor in (\ref{15}).
   Correspondingly we are writing the resulting equation for the $\sigma_s(T)$ denoting the average value of
   $\sqrt{z/(2\omega)}T$ as $\rho T/T_c$. As a result one obtains the equation for the string tension
\be
\sigma_s(T)={\rm const}{g^4(T)} f^2(<w>), w= \rho T/T_c.
\label{16}
\ee
In the next sections we shall demonstrate that this new form with the well defined factor $f(<w>)$
 describes the whole region of $T>T_c$ with a good accuracy.

\section{General expression for the spatial string tension vs lattice data}

Using (\ref{10a}) $J_4(s,T) = \frac{1}{2\sqrt{\pi s}}f(w), w=2\sqrt{s}T$ one can write
$f(w)= \sum^{+\infty}_{n=-\infty}e^{-\frac{n^2}{w^2}}\equiv\vartheta_3(q),~~~q=e^{-\frac{1}{w^2}}$
 Which we express as $w^2=\frac{\rho^2 T^2}{T_c^2}$. Correspondingly the $f(<w>)$ acquires the form
\be
f(<w>) = F(T/T_c)= \vartheta_3(e^{-\frac{T_c^2}{(\rho T)^2}})
\label{17} \ee
 The numerical analysis of the data \cite{8} allows to
reproduce well the data with the equation of the form \be
\sigma_s(T)=\sigma_s(T_c) \frac{g^4(T) F^2(T/T_c)}{g^4(T_c)F^2(1)}
\label{18} \ee

Analysis of the lattice data from the (\ref{18}) is shown in Fig 1 , where for $g^4(T)$  in the Appendix 1
the explicit value of the $L_\sigma= 0.104 $ as in the \cite{8} was used while in $f(<w>)$ in (\ref{17})
the value $\rho= 3   $. The Fig 1 demonstrates a good agreement between the lattice data and our equation (\ref{18}),
including the region $ T< 2.5 T_c$ where the lattice fit $T^2 g^4(T)$ starts to disagree with numerical data.

\begin{figure}%[h]
\center{\includegraphics[width=0.7\linewidth]{Fit.eps}}
    \caption{Spatial string tension $\sigma_s (T)/\sigma$ for SU(3)
gauge theory as function of $T/T_c$.  The lattice data are from Refs.\cite{8}. $T_{c}$=270 MeV}
    \label{FiG1}
\end{figure}


\section{Discussion of results and Conclusions}



in our paper in the framework of the FCM  we have discussed the spatial confinement mechanism for $T>T_c$, which is defined by the colormagnetic field correlators $D^H (Z)$ and the spatial string tension $\sigma_s(T)$. It was shown that the resulting physical picture is rather specific and strongly connected with the transverse motion
of the colored objects, since purely longitudinal motion in this region is associated only with perturbative QCD interaction, which  decreases at large T. Therefore the nonperturbative effects are strongly $T$ --
dependent because  the transverse motion is generated by temperature.
In contrast with the colorelectric string tension $\sigma_E$, which is the basic independent QCD parameter and is computed in FCM self-consistently via itself, the $\sigma_s$  can be computed within the theory and it is very important for the theory to test in this way its self-consistency. Here we have provided  this test within  our theory of confinement, based on the FCM, and for  $\sigma_s$ and $c_{\sigma}$ we found
 a good agreement with lattice data, providing in this way good arguments in favor of its self-consistency
Even more important  role of the spatial string tension may be  in the high $T$ thermodynamics where
in the framework of FCM it provides the basic nonperturbative contribution to the pressure and other observables \cite{18} in good agreement with the lattice data.

\section{Acknowledgments.}
The work of Khaidukov.Z.V was supported by the Russian Science Foundation 21-12-00237.
\section*{ Appendix A1. Two-loop expression for $g^{-2} (t)$}


 \setcounter{equation}{0} \def\theequation{A1.\arabic{equation}}


This expression has the standard form  in $SU(3)$ as a function of $t= \frac{T}{T_c}$ :
\be
g^{-2}(t)= c_0 \ln \frac{t}{L_{\sigma}} + c_1 \ln\left(2 \ln\frac{t}{L_{\sigma}}\right),
\label{A1.1}
\ee
where
\be
c_0= \frac{11}{8\pi^2}, c_1= \frac{51}{88\pi^2}.
\label{A1.2}
\ee
Here $L_\sigma= \frac{\Lambda_\sigma}{T_c}= 0.104 \pm 0.009$ \cite{8,9}. In the last sections we ar also using $L_\sigma=
0.0916$.






\section*{Appendix A2. Numerical calculation of $c^2_{\sigma}$ in the lowest approximations}



 \setcounter{equation}{0} \def\theequation{A2.\arabic{equation}}

We calculate here two lowest eigenvalues and eigenfunctions of the Hamiltonian, which enter
 in the expression for the $c^2_{\sigma}$ in (\ref{16}). This Hamiltonian without spin-dependent terms can be written in the equivalent oscillator form,
 \be
 H= \frac{\omega^2_1 + \vep^2_1}{2\omega_1} + \frac{\omega^2_2 + \vep^2_2}{2\omega_2} + \frac{\sigma^2\ver^2_1}
{2 \nu_1} + \frac{\sigma^2 \ver^2_2}{2 \nu_2} + \frac{\sigma^2 (\ver_1- \ver_2)^2}{2 \nu_3} + \frac{\nu_1 + \nu_2 + \nu_3}{2}.
\label{A2.1}
\ee
Here we have used the property $\sigma |\ver|= \min\left(\frac{\sigma^2 \ver^2 + \nu^2}{2 \nu}\right)$, so that
minimizing the eigenvalues of (\ref{A2.1}) in the variables $\nu_i$, we obtain the eigenvalues and eigenfunctions of the Hamiltonian with a good accuracy $\sim 5\%$.
Then for $\omega_1 = \omega_2 = \omega$ and $\nu_1= \nu_2=\nu$ one obtains the lowest eigenvalue for n=0,

\be
 M_0= \omega + \frac{\sigma}{\sqrt{\omega\nu}}\left(1 + \sqrt{\frac{\nu_3 +2\nu}{\nu_3}}\right) + \frac{2\nu+\nu_3}{2}.
\label{A2.2}
\ee
The conditions of minima $\frac{\partial M_0}{\partial z_i}= 0$ with $z_i= \omega_i,\nu_i$ yields the
final result with notation $\omega_i(0),\nu_i(0)$ for the extremal values.

\be
 \omega_1(0)= \omega_2(0)= 1.29 \sqrt \sigma,~ \nu_0= 0.79 \sqrt \sigma,~ \nu_3(0)= 1.25 \nu_0,~
 \min (M_0)=4.95 \sqrt \sigma.
 \label{A2.3}
 \ee
From the oscillator wave functions it easy to get the factor $|\psi(0,0)|^2= 1.61 \sigma^2$ and
to calculate the integrals over $d\omega_1 d\omega_2$ in (\ref{16}), expanding $M_0(\omega_1,\omega_2)$ near the stationary points in (\ref{A2.3}) up to the second order in $\omega_i-\omega_i(0)$ and denoting the second derivative of $M_0$ as $M"(\omega_0)$. Then for the integral in (\ref{17})
one has
\be
\int d^2 w G(w)= \frac{2\pi |\psi(0,0)|^2}{M^2_0 \omega^3_0 M"(\omega_0)}.
\label{A2.4}
\ee
Inserting the stationary values from (\ref{A2.3}) and the second derivative at the stationary point
$M"(\omega_0)= 0.51 \sigma^{-1/2}$, one finally obtains

\be
 \int d^2 w G(w)= \frac{2\pi 0.228 |\psi|^2}{\sigma M^2_0}= 0.093,~
c^2_{\sigma} = \frac{N^2_c-1}{4} \int d^2 w G(w) = 0.186 ~(N_c=3).\label{A2.5}
\ee
In a similar way one can calculate the contribution of the $n=2$ term in the (\ref{18}), which yields approximately
$c^2_{\sigma}(n=2)= 0.019$ and for the sum of two terms with $n=0,2$   $c^2_{\sigma} = 0.205$ for $N_c=3$ which is the lower bound. However as will be seen below the most important corrections appear when one estimates the accuracy of the replacement
of the original linear confinement Hamiltonian (\ref{18}) by the oscillator Hamiltonian (\ref{A2.1}). To get an idea of this effect we can estimate the ratio of the integral in (\ref{A2.5}) which we denote as $I_{\rm osc}$ and the corresponding integral
for the real (linear) interaction $I_{ \rm lin}$. To simplify matter we replace $|\psi(0)|^2$ and $M_0$ of the gluelump system by the  simple two gluon system connected by the linear or oscillator interaction and write approximately
\be
R= \frac{I_{\rm lin}}{I_{\rm osc}} \approx \frac{|\psi_{\rm lin}(0)|^2 M^2_{\rm osc}}{|\psi_{\rm osc}(0)|^2 M^2_{\rm lin}}.
\label{A2.6} \ee
For the two-gluon system with linear confining interaction the spectrum and wave functions are well known \cite{19,20,21}:
\be
M_n= 4 \sqrt{\sigma} (\frac{a(n)}{3})^{3/4},~ |\psi_{\rm lin}(0)|^2= \frac {\sigma M_0}{16 \pi},
\label{A2.7} \ee
where for the ground state $n=0$ $a(0)= 2.338$.
Inserting for the linear and oscillator potentials the resulting values $M_{\rm lin}= 3.31 \sqrt {\sigma}$, $ M_{\rm osc}= 3.59 \sqrt {\sigma}$ and
$ |\psi_{\rm lin}(0)|^2= 0.065 \sigma^{3/2}$, $ |\psi_{\rm osc}(0)|^2= 0.043 \sigma^{3/2}$, one obtains the approximate ratio which estimates  the effect of the replacement by oscillator interaction
\be
R=  \frac{c^2_{\sigma}({\rm lin})}{c^2_{\sigma}({\rm osc})} \approx 1.82;~ c_{\sigma}({\rm lin}) \approx  1.35 c_{\sigma}({\rm osc}).
\label{A2.8} \ee
As a result using the $n=0$  oscillator value of $c^2_{\sigma}$ in (\ref{A2.5}) we obtain  the linear confinement coefficient $c_{\sigma} \approx 0.582 $ which agrees well with the lattice value $ 0.566 \pm 0.013$ from \cite{8,9}.
A more accurate calculation of the $c_{\sigma}({\rm lin})$ is possible with the solution of the linear integral equations for the
gluelump Green's functions as it was done in \cite{6} for the gluelump masses.





\section*{Appendix A3. Calculation of $\sigma_s(T)$ via gluelump Green's functions}

 \setcounter{equation}{0} \def\theequation{A3.\arabic{equation}}

We can write $J_4$ in (\ref{9}) in the limits of large $T$  and $T=0$ as $J_4=T$ and $J_4 =\frac{1}{2\sqrt{\pi s}}$ respectively.
As  a result the $4d$ gluon propagator at large temperature $T$ reduces to the linear in $T$ the $3d$ propagator,while at $T=0$ one has $G_{4d}^{(g)}(z,0)$  propagator.
\be
G_{4d}^{(g)}(z,T) = T G_{3d}^{(g)}(z) ,\label{A3.1}
\ee
 In \cite{12*} was considered this term in
(\ref{A3.1}), assuming the limit of large $T$. Substituting these terms in  the  general expression for $D^H(z)$, one has

\be
D^H(z) = \frac{g^4(N^2_c-1) }{2} \lan G^{(2g)}_{4d} (z,T) \ran,
\label{A3.2}
\ee
and $G^{(2g)}(z,T)$ is formed from the product of two one-gluon Green's functions $G^{(g)}(z,T)$
where both gluons are connected by the adjoint strings,which we denote by the sign $<...>$.
Taking into account the relation (\ref{4}) one obtains
\be
G_{4d}^{(2g)}(z,T)=<G_{4d}^{(g)}(z,T) G_{4d}^{(g)}(z,T)>= T^2 G_{3d}^{(2g)}(z),\label{A3.3}
\ee
Here the two-gluon Greens functions can be written in the form
\be
G_{3d}^{(2g)}(z)= \frac{z}{8\pi} \int \frac{d\omega_1}{\omega_1^{3/2}} \frac{d\omega_2}{\omega_2^{3/2}} D^{3}r_1 D^{3}r_2
\exp({-K_1-K_2-V(\ver_1,\ver_2)z}).
\label{A3.4}
\ee
Finally taking into account  the relations \cite{10,11}
\be
\sigma_s(T)= \frac{1}{2} \int d^2z D^{H}(z,T)= \frac{g^4(N^2_c-1)}{4} \int d^2z G_{4d}^{(2g)}(z,T),
\label{A3.5}
\ee
one obtains the final form which will is calculated and discussed above at large T.

\be
\sigma_s(T)= g^4 c_{\sigma}^2 T^2,\label{A3.6}
\ee
At the same time one can use  (\ref{A3.5}) to get the explicit relations for the $T=0$ limit of the spatial string tension
\be
\sigma_s(0)= \frac{g^4(N^2_c-1)}{4} \int d^2z G^{(2g)}_{4d} (z)
\label{A3.7}
\ee
In terms of the gluelump phenomenology, studied in \cite{6,7}, the expression (\ref{A3.7}) is called the two-gluon gluelump, which was computed on the lattice \cite{7} and analytically in
\cite{6}. In our case of large $T$ limit and the $T=0$ case we are interested both in the $3d$ version and $4d$ versions of the corresponding Green's function. Choosing in $4d$ the $x_3\equiv t$ axis as the
Euclidean time, we proceed, as in \cite{7}, exploiting the path integral technic \cite{10,11,12}, which yields
\be
G^{(2g)}_{4d} (x-y) = \frac{t}{8\pi} \int^\infty_0 \frac{d\omega_1}{\omega_1^{3/2}} \int^\infty_0
\frac{d\omega_2}{\omega_2^{3/2}} (D^3r_1)_{xy}(D^3r_2)_{xy}
e^{-K_1(\omega_1)-K_2(\omega_2)-Vt}, \label{A3.8}
\ee
where $V$ includes the spatial confining interaction between the three objects: gluon 1, gluon 2, and  the fixed straight line of the parallel transporter, which makes all construction
gauge invariant (see \cite{6,15} for details). In (\ref{A3.8}) $t=|x-y|\equiv |w|;$ .
Constructing in the exponent of (\ref{A3.8}) the three-body Hamiltonian in the $3d$ spatial coordinates, in the standard
Fock-Feynman-Schwinger \cite{st} procedure one has
\be
H(\omega_1, \omega_2) = \frac{ \omega_1^2+ \vep^2_1}{2\omega_1}+  \frac{
\omega_2^2+ \vep^2_2}{2\omega_2}+ V(\ver_1, \ver_2), \label{A3.9}
\ee
one can rewrite (\ref{A3.8}) as follows (see \cite{12}),
 \be
 G^{(2g)}_{4d} (t) = \frac{t}{8\pi} \int^\infty_0 \frac{d\omega_1}{\omega_1^{3/2}} \int^\infty_0
\frac{d\omega_2}{\omega_2^{3/2}} \sum^\infty_{n=0} |\psi_n (0,0)|^2 e^{-M_n
(\omega_1,\omega_2) t}.\label{A3.10}
\ee
This equation has the universal form in $3d$,$4d$ and $3d,4d$ dimensions, the resulting expressions for $G^{(2g)}$  differ
in the values of $M_n$ and values and dimensions of $|\psi_n(0,0)|^2$.
Here $\Psi_n(0,0)\equiv \Psi_n (\vez_1,\vez_2)|_{\vez_1=\vez_2=0}$, and $M_n$  is the eigenvalue of $H(\omega_1,
\omega_2)$. The latter was studied in \cite{6} in three spatial coordinates. For our purpose here we only mention that $ G^{(2g)}_{3d}  (z)$ has the dimension
of the mass squared and therefore the integral (\ref{4}) defining $\sigma_s$ are  dimensionless. Hence, one obtains $\sqrt{\sigma_s (T)} = g^2 T  c_{\sigma}$, as it was stated in (\ref{1}), where
 \be
 c^2_\sigma = \frac{(N_c^2-1)}{4} \int d^2 w \lan G_{3d}^{(2g)} (w)\ran. \label{A3.11}
\ee

and the relativistic eigenvalues $M_n(\omega_1,\omega_2)$ and the eigenfunctions $\psi_n(\ver_1,\ver_2)$
are defined via the gluelump Green's function .
The contribution of the $G^{(2g)}_{3d}$ was found via $c_{\sigma}$ in \cite{12*} , the contribution of the term
$G^{(2g)}_{4d}$ is the standard $\sigma_s(T=0)= \sigma_E(T=0)= 0.18 GeV^2$ .



 computed in FCM self-consistently via itself, the $\sigma_s$  can be computed within the theory and it is very important for the theory to test in this way its self-consistency. Here we have provided  this test within  our theory of confinement, based on the FCM, and for  $\sigma_s$ and $c_{\sigma}$ we found
 a good agreement with lattice data, providing in this way good arguments in favor of its self-consistency
Even more important  role of the spatial string tension may be  in the high $T$ thermodynamics where
in the framework of FCM it provides the basic nonperturbative contribution to the pressure and other observables \cite{18} in good agreement with the lattice data.













\begin{thebibliography}{99}

\bibitem{1}
H. G. Dosch and Yu.A.Simonov, Phys. Lett. {\bf  B 205}, 339 (1988).

\bibitem{2}
Yu. A. Simonov, Phys. Usp. {\bf 166}, 337 (1996), arXiv: hep-ph/9709344; D. S. Kuzmenko, V. I. Shevchenko, and Yu. A. Simonov, Phys. Usp. {\bf 47}, 1 (2004),
arXiv: hep-ph/0310190.

\bibitem{3}
A. Di Giacomo, H. G. Dosch, V. I. Shevchenko, and Yu. A. Simonov, Phys. Rept. {\bf 372}, 319 (2002), arXiv: hep-ph/0007223.
\bibitem{4}
Yu. A. Simonov, Phys. Rev. {\bf D 99}, 056012 (2019), arXiv: 1804.08946.

\bibitem{5}
 M. D'Elia, A. Di Giacomo, and E. Meggiolaro, Phys. Rev. {\bf D 67}, 114504 (2003), arXiv: hep-lat/0205018.

\bibitem{5*}
Yu. A. Simonov, The colormagnetic confinement in QCD, arXiv: 2203.07850 [hep-ph].

\bibitem{6}
Yu. A. Simonov, Nucl. Phys., {\bf B 592}, 350 (2001), arXiv: hep-ph/0003114.

\bibitem{7}
I. Jorycz and C. Michael, Nucl. Phys. {\bf B 302}, 448 (1988); N. Campbell, I. Joricz, and C. Michael,
Phys. Lett. {\bf B 167}, 91 (1986).

\bibitem{7*}
C. Borgs, Nucl. Phys. {\bf B 261}, 451 (1985); E. Manousakis and J. Polonyi, Phys. Rev. Lett. {\bf 58}, 847 (1987).

\bibitem{8}
G. Boyd et al., Nucl. Phys. {\bf B 469}, 419 (1996), arXiv: hep-lat/9602007.

\bibitem{9}
F. Karsch, E. Laermann, and M. Lutgemeier, Phys. Lett. {\bf B 346}, 94 (1995), arXiv: hep-lat/9411020.

\bibitem{10}
Yu. A. Simonov, Phys. Atom. Nucl, {\bf 58}, 339 (1995); N. O. Agasian, JETP Lett. {\bf 57}, 208 (1993);
JETP Lett. {\bf 71}, 43 (2000); Phys. Lett.  {\bf B 519}, 71 (2001), [arXiv:hep-ph/0104014].

\bibitem{10*}
N.~O.~Agasian,
%``Thermal gluomagnetic vacuum of SU(N) gauge theory,''
Phys. Lett. B \textbf{562},  257 (2003), arXiv:hep-ph/0303127.

\bibitem{11}
Yu. A. Simonov, Phys. Atom. Nucl. {\bf 69}, 528 (2006), arXiv: hep-ph/0501182; Yu. A. Simonov and V. I. Shevchenko, Adv. High Energy Phys. {\bf 2009}, 873051 (2009),
arXiv: 0902.1405 [hep-ph].

\bibitem{12}
Yu. A. Simonov, Phys. Rev. {\bf D 96}, 096002 (2017), arXiv: 1605.07060.

\bibitem{12*}
Yu.A.Simonov, The spatial string tension and the nonperturbative Debye mass from the Field Correlator Method, arXiv:2206.14489.

\bibitem{12**}
N.~O.~Agasian and I.~A.~Shushpanov,
%``The Quark and gluon condensates and low-energy QCD theorems in a magnetic field,''
Phys. Lett. B \textbf{472}, 143 (2000), [arXiv:hep-ph/9911254];
JHEP \textbf{10}, 006 (2001), [arXiv:hep-ph/0107128].

\bibitem{18}
N. O. Agasian, M. S. Lukashov, and Yu. A. Simonov, Eur. Phys. J. {\bf A 53}, 138 (2017), arXiv: 1701.07959;
Mod. Phys. Lett. A \textbf{31}, no.37, 1650222 (2016), [arXiv:1610.01472 [hep-lat]].

\bibitem{15}
N. O. Agasian and Yu. A. Simonov, Phys. Lett. {\bf B 639}, 82 (2006), arXiv:hep-ph/0604004.

\bibitem{14}
E. L. Gubankova and Yu. A. Simonov, Phys. Lett. {\bf B 360}, 93 (1995), arXiv:hep-ph/9507254.


\bibitem{13}
G. S. Bali et al., Phys. Rev.  {\bf 71}, 3059 (1993); hep-lat/9306024.

\bibitem{13*}
M. Teper, Phys. Lett. {\bf B 311}, 223 (1993).


%\bibitem{16}
%O. Kaczmarek and F. Zantow, Contribution to: Workshop on Extreme QCD, 108-112, arXiv: hep-lat/0512031.

%\bibitem{17}
%O. Kaczmarek and F. Zantow, Phys. Rev. {\bf D 71}, 114510 (2005), arXiv: hep-lat/0503017.


\bibitem{19}
Dan La Course and M. G. Olsson, Phys. Rev. {\bf D 39}, 2751 (1989).

\bibitem{20}
W. Lucha, F. F. Schoeberl and D. Gromes, Phys. Rep. {\bf 200}, 127 (1991).

\bibitem{21}
Yu. A. Simonov, Phys. Lett. {\bf B 226}, 151 (1989); Yu. A. Simonov, QCD and Theory of Hadrons, in: ``QCD: Perturbative or
Nonperturbative." Interscience, Singapore, 2000; arXiv: hep-ph/9911237.

\bibitem{st}
Yu. A. Simonov and J.A.Tjon, Ann.Phys. {\bf 300}, 54 (2002).

\end{thebibliography}

\end{document}
