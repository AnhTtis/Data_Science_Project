
\documentclass[12pt]{article}

\usepackage{graphicx}
\usepackage{amsmath}
\usepackage[cp866]{inputenc}
\usepackage[english]{babel}
\usepackage{xcolor}
\usepackage{epsfig}
\usepackage{floatrow}
\usepackage{amssymb,amsmath,amsfonts,amsthm,graphicx,psfrag}
\usepackage{indentfirst}
\usepackage{hyperref}
\usepackage[title,titletoc]{appendix}
\usepackage{graphicx}
\usepackage{amsfonts}
\usepackage{bm}
\usepackage{verbatim}
\usepackage{epsfig}
\graphicspath{{./pictures/}}
\usepackage{authblk}
\usepackage{cite}
\usepackage[title,titletoc]{appendix}
\usepackage{slashed}
\usepackage{amsmath}

		\newcommand{\be}{\begin{eqnarray}}
\newcommand{\ee}{\end{eqnarray}}
\newcommand{\nn}{\nonumber}
\newcommand{\ba}{\begin{aligned}}
\newcommand{\ea}{\end{aligned}}
\usepackage{authblk}

\sloppy
\textwidth 16cm
\textheight 23cm
\hoffset=-1.5cm
\voffset= -3cm

\title{The spatial string tension  from the Field Correlator Method}
\author{ N. O. Agasian$^{+}$, Z. V. Khaidukov$^{+,a}$ and  Yu. A. Simonov$^{+}$  \\
$^{+}$NRC ``Kurchatov Institute'', Moscow, Russia \\
$^{a}$ Moscow Institute of Physics and Technology, 9 Institutskiy per., Dolgoprudny, Moscow region, 141701, Russia}
% \date{}

\newcommand{\beq}{\begin{eqnarray}}
 \newcommand{\eeq}{\end{eqnarray}}


 \def\la{\mathrel{\mathpalette\fun <}}
\def\ga{\mathrel{\mathpalette\fun >}}
\def\fun#1#2{\lower3.6pt\vbox{\baselineskip0pt\lineskip.9pt
\ialign{$\mathsurround=0pt#1\hfil ##\hfil$\crcr#2\crcr\sim\crcr}}}
\newcommand{\veX}{\mbox{\boldmath${\rm X}$}}
\newcommand{{\SD}}{\rm SD}
\newcommand{\pp}{\prime\prime}
\newcommand{{\Mc}}{\mathcal{M}}
\newcommand{\veY}{\mbox{\boldmath${\rm Y}$}}
\newcommand{\vex}{\mbox{\boldmath${\rm x}$}}
%\newcommand{\vexi}{\mbox{\boldmath${\rm \xi}$}}
\newcommand{\vey}{\mbox{\boldmath${\rm y}$}}
\newcommand{\ver}{\mbox{\boldmath${\rm r}$}}
\newcommand{\vesig}{\mbox{\boldmath${\rm \sigma}$}}
\newcommand{\vedelta}{\mbox{\boldmath${\rm \delta}$}}
\newcommand{\veP}{\mbox{\boldmath${\rm P}$}}
\newcommand{\vep}{\mbox{\boldmath${\rm p}$}}
\newcommand{\veq}{\mbox{\boldmath${\rm q}$}}
\newcommand{\veQ}{\mbox{\boldmath${\rm Q}$}}
\newcommand{\vez}{\mbox{\boldmath${\rm z}$}}
\newcommand{\veS}{\mbox{\boldmath${\rm S}$}}
\newcommand{\veL}{\mbox{\boldmath${\rm L}$}}
 \newcommand{\veA}{\mbox{\boldmath${\rm A}$}}
\newcommand{\veR}{\mbox{\boldmath${\rm R}$}}
\newcommand{\ves}{\mbox{\boldmath${\rm s}$}}
\newcommand{\vek}{\mbox{\boldmath${\rm k}$}}
\newcommand{\ven}{\mbox{\boldmath${\rm n}$}}
\newcommand{\veu}{\mbox{\boldmath${\rm u}$}}
\newcommand{\vev}{\mbox{\boldmath${\rm v}$}}
\newcommand{\veh}{\mbox{\boldmath${\rm h}$}}
\newcommand{\vew}{\mbox{\boldmath${\rm w}$}}
\newcommand{\verho}{\mbox{\boldmath${\rm \rho}$}}
\newcommand{\vexi}{\mbox{\boldmath${\rm \xi}$}}
\newcommand{\veta}{\mbox{\boldmath${\rm \eta}$}}
\newcommand{\veB}{\mbox{\boldmath${\rm B}$}}
\newcommand{\veH}{\mbox{\boldmath${\rm H}$}}
\newcommand{\veE}{\mbox{\boldmath${\rm E}$}}
\newcommand{\veJ}{\mbox{\boldmath${\rm J}$}}
\newcommand{\veal}{\mbox{\boldmath${\rm \alpha}$}}
\newcommand{\vepi}{\mbox{\boldmath${\rm \pi}$}}
\newcommand{\vemu}{\mbox{\boldmath${\rm \mu}$}}
\newcommand{\vegam}{\mbox{\boldmath${\rm \gamma}$}}
\newcommand{\vepar}{\mbox{\boldmath${\rm \partial}$}}
\newcommand{\llan}{\langle\langle}
\newcommand{\rran}{\rangle\rangle}
\newcommand{\lan}{\langle}
\newcommand{\ran}{\rangle}


\begin{document}
%\baselineskip 11.5pt
\maketitle
\begin{abstract}

The behaviour of the spatial string tension $\sigma_s(T)$ as a function of the temperature $T$ is found in the framework of the Field Correlator Method
(FCM). Here the string tension is calculated using the gluon-gluon  Green's function where gluons are interacting via the same spatial string tension interaction. The resulting selfconsistent T-dependence was obtained without extra parameters in the region
$T_c < T < 5 T_c$ using the formalism of elliptic $\theta_3$ functions,
demonstrating a good agreement with available lattice data.


\end{abstract}

\section{Introduction}
The confinement in QCD is a basic phenomenon which ensures more than 90 percent of the visible mass in the Universe and makes the world such as we see it. At zero temperature
the theory of confinement in QCD was formulated in the framework of the Field Correlator Method (FCM) \cite{1,2,3,4,5,5*},
 via the vacuum field correlators  of the
colorelectric (CE) and the colormagnetic (CM) fields $E_i^a,H_i^a$, and at the temperature
$T=0$ the behaviour of all physical quantities is expressed  via the basic nonperturbative
parameter -- the string tension, which differs in the light-like areas $\sigma_E$ and space-like areas
$\sigma_H$  but coincides at the zero temperature T,  $\sigma_E(T=0)=\sigma_H(T=0)= \sigma$.
The CE and CM field correlators are defined as bilocal vacuum averages of the CE and CM fields
\be
D^{(E,H)}(x-y)= 1/N_c <tr((E_i,H_i)(x)\Phi(x,y)(E_i,H_i)(y))>  \label{1} \ee
The field correlators $D^E(x),D^H(x)$ define all confining QCD dynamics and in particular the string tensions and $\Phi(x,y)$ is the Wilson line that is connected points x and y.
\be
\sigma_E= 1/2 \int (d^2z)_{i4} D^{E}(z), \sigma_H= 1/2 \int (d^2z)_{ik} D^{H}(z) \label{2} \ee
These correlators  have been calculated in good agreement
between the FCM \cite{6} and the lattice data \cite{7}, while $D^E(x),D^H(x)$ were also studied
in detail at $T>0$ on the lattice \cite{5}. The most interesting point is that while both $\sigma_E$ and $\sigma^H$ coincide at $T=0$,it was found that
 at $T>0$ both $\sigma_E(T)$ and $\sigma_H(T)= \sigma_s(T)$
have  different behaviour. Namely  $\sigma_E(T)$ displays a spectacular drop before $T=T_c$ and
disappears above $T=T_c$, while in contrast to that $\sqrt{\sigma_s(T)}$
grows almost quadratically at large $T$, as it was
found on the lattice \cite{4,7*,8,9} and supported by the studies
in the framework of the FCM \cite{10,10*,11,12,12*,12**,18,15,14}.
Indeed, it was found in \cite{8,9} that the dominant part of the spatial string tension $\sigma_s(T)$ grows quadratically at large $T$
\be
\sigma_s(T)= (c_{\sigma})^2 g^4(T) T^2  \label{3} \ee

where $c_{\sigma}$ was

defined numerically in the lattice calculations \cite{8,9} in the case  $N_c=3, N_f=0$ as
\be
c_{\sigma}= 0.566\pm 0.013.  \label{4} \ee

 On the theoretical side the quadratic growth of the $\sigma_s(T)$ was derived in the framework of FCM
 \cite{6,10,11,12} and the  value of $c_{\sigma}$ was found in the  \cite{12*}
 in a good agreement with the lattice data of \cite{8,9}.

However in the full FCM expression for the spatial string tension
the term in (\ref{3}) is only a fast growing part  of the whole expression which was hitherto not known.

It is the purpose of the present paper to derive the total expression of the spatial string tension including the linear in $T$ part and to calculate the numerical value of $\sigma_s(T)$  in the  region of $[T_{c}..\infty]$\footnote{In FCM $\sigma_{s}$ below $T_{c}$ almost coincides with the  vacuum $\sigma_E$ value } and compare it with the lattice data.
As will be seen , the results, obtained within the FCM,   provide the values
of $\sigma_s (T)$ in a good agreement with the lattice data.
In the next section we discuss the general expression for $\sigma_s(T)$ in terms of the field correlators (gluelump Green's functions)  and in the
section 3 we formulate its final form    and obtain  the expression of the string tension $\sigma_s(T)$ via the field correlator, which is discussed and compared with the lattice data in the section 4.The renormalized  coupling constant $g^2(T)$ is  given in Appendix A1, the detailed connection of the spatial string tension with the gluelump Green's function in Appendix A2.

\section{The spatial string tension in the FCM}

As one can see in \cite{1,2} the field correlator $D^E(x,y),D^H(x,y)$ in the nonabelian case due to the
relation e.g $F_{\mu\nu}= \partial_{\mu}A_{\nu}- \partial_{\nu}A_{\mu} -ig [A_{\mu}A_{\nu}]$ can be represented as a sum of terms where the one or two gluons propagate along the Wilson line $\Phi(x,y)$ and as explained in \cite{11} in the nonperturbative vacuum all these gluon propagators and the Wilson line are connected by the adjoint strings.

This gluon-Wilson line construction was studied also on the lattice \cite{11*} and called there the "gluelump". As was found in \cite{1,2,3,4,5,6,11,12,12*} the string tension
is defined by the double-gluelump propagator with two gluons and the Wilson line all connected by adjoint strings with the string tension $\sigma_a= 9/4 \sigma_f$. In this way the calculation of the string tension is a selfconsistent process which we describe below for the spatial string tension.
The spatial string tension is proportional to the integral of this two-gluon (double gluelump) Green's function in the $3d$ space, where one of three space coordinates can be taken as an evolution parameter (the Euclidean ``time").Using the technic, developed in \cite{10,11,12} for
$D^E(z), D^H(z)$, which allows to express it via the two-gluon  Green's function: $G^{(2g)}_{4d} (z) =  G_{4d}^{(g)}\otimes G_{4d}^{(g)}$, where two gluons and the Wilson line (the "parallel transporter")
interact nonperturbatively, and  we  neglect  the spin interactions in the first approximation.
We proceed as in \cite{12,12*}
\be
(-D^2)^{-1}_{xy}=\left\lan x\left|\int^{\infty}_0 dt
e^{tD^2(B)}\right|y\right\ran=
\int^{\infty}_0dt(Dz)^w_{xy}e^{-K}\Phi(x,y), \label{6}
\ee where
\be
K=\frac{1}{4}\int^s_0d\tau \left(\frac{dz_\mu}{d\tau}
\right)^2, ~~\Phi(x,y)=P \exp ig\int^x_yB_{\mu}dz_{\mu},\label{7}
\ee and a winding path measure is \be (Dz)^w_{xy}=\lim_{N\to
\infty}\prod^{N}_{m=1}\frac{d^4\zeta(m)}{(4\pi\varepsilon)^2}
\sum^{+\infty}_{n=-\infty}
\int\frac{d^4p}{(2\pi)^4}e^{ip(\sum\zeta(m)-(x-y)-n\beta\delta_{\mu
4})}. \label{8} \ee

The important point for the resulting $T$ dependence of the string tension is the integration in the gluon propagator $G^{(g)}_{4d}$  in the 4-th direction in (\ref{6}) with the exponent $K_4 = \frac14 \int^s_0 d\tau \left( \frac{d z_4}{d\tau}\right)^2,$ which gives for the spatial string tension  with $x_4=y_4$, and for the temporal string tension with the nonzero $x_4-y_4$  completely different behaviour, namely for the $\sigma_s$ case
\be
J_4 \equiv \int (Dz_4)_{x_4x_4} e^{-K_4} = \sum^{+\infty}_{n=-\infty} \frac{1}{2 \sqrt{\pi s}} e^{-\frac{(n\beta)^2}{4s}}
%=
%\frac{1}{2\sqrt{\pi s}} \left( 1+ \sum_{n=\pm 1,\pm 2}
%e^{-\frac{(n\beta)^2}{4s}}\right).
\label{9}
\ee
%The  second term in (\ref{9}) at large $T\gg \frac{1}{2\sqrt{s}}$ yields
%$2\sqrt{\pi s} T$,    which gives two different limiting values,
%\be
%J_4 =\frac{1}{2\sqrt{\pi s}} for T=0, J_4 = T, for T \gg (2\sqrt{\pi s})^(-1).
%\label{10}
%\ee
One can notice that the sum in (\ref{9}) is a known function
\be
\sum^{+\infty}_{n=-\infty}e^{-\frac{n^2}{4sT^2}}\equiv\vartheta_3(q),~~~q=e^{-\frac{1}{4sT^2}},
\label{10a}
\ee
where the function $\vartheta_3(q)$ is defined as
\be
\vartheta_3(q)=\sum^{+\infty}_{n=-\infty}q^{n^2}=1+2q+2q^4+O(q^9)
\label{10b}
\ee
Then starting from low temperature there is an expansion
$$
J_4=\frac{1}{2\sqrt{\pi s}}\sum^{+\infty}_{n=-\infty}e^{-\frac{n^2}{4sT^2}}
\equiv\frac{1}{2\sqrt{\pi s}}\vartheta_3(e^{-\frac{1}{4sT^2}})
$$
\be
=\frac{1}{2\sqrt{\pi s}}(1+2e^{-\frac{1}{4sT^2}}+O(e^{-\frac{1}{sT^2}}))
\label{10c}
\ee
To find the asymptotics at high T one can use the relation
\be
\sum^{+\infty}_{n=-\infty}e^{-\frac{\beta^2n^2}{4s}}=
\frac{2\sqrt{\pi s}}{\beta} \sum^{+\infty}_{n=-\infty} e^{-\frac{4\pi^2n^2}{\beta^2}s}
\label{10d}
\ee
As a result at large T one obtains an equality
$$
J_4=T\sum^{+\infty}_{n=-\infty}e^{-4\pi^2sT^2n^2}\equiv T\vartheta_3(e^{-4\pi^2sT^2})
$$
\be
=T(1+2e^{-4\pi^2sT^2}+O(e^{-16\pi^2sT^2}))
\label{10e} \ee

Here we are using the elliptic $\vartheta_{3}(q)$ functions defined as in (\ref{10b})

Their behaviour as a function of $q$ is given in the Fig.\ref{FiG1}
\begin{figure}%[h]
\center{\includegraphics[width=0.7\linewidth]{Nu.eps}}
    \caption{the elliptic $\vartheta_{3}(q)$ as a function of $q$}
    \label{FiG1}
\end{figure}
As a result one can use $J_{4}$ at an arbitrary T in the form
\be
J_4(s,T)\equiv\frac{1}{2\sqrt{\pi s}}\vartheta_3(e^{-\frac{1}{4sT^2}})
\label{10f}
\ee
In this way starting from the low T one obtains an exact expression for $J_4(T)$ valid in the whole range of $T$.
One could approximate this behaviour  as a sum of linear and constant term implying a soft transition from
$T=0$ case to the linear in $T$ behaviour however this approximation fails numerically  and actually one
observes the sharp transition at some intermediate point $T^*$ from the regime $T=0$ to the large $T$ behaviour
as it is given by (\ref{10f}). As a result
we express the $J_4$ as follows
\be
J_4= \frac{1}{2\sqrt{\pi s}} \left( \sum^{+\infty}_{n=-\infty}e^{-\frac{(n\beta)^2}{4s}}\right)=
\frac{1}{2\sqrt{\pi s}} f(\sqrt{s}T).
\label{11}
\ee
At this point we turn to the general form of the field correlator $D^H(z)$ with the aim to express the string tension via the
factors $f(x)$. One has
\be
D^H(z) = \frac{g^4(N^2_c-1) }{2} \lan G^{(2g)} (z,T) \ran,
\label{12}
\ee
where $G^{(2g)}(z,T)$ is the gluelump Green's function
\be
G^{(2g)}(z,T)= \frac{z}{8\pi} \int \frac{d\omega_1}{\omega_1^{3/2}} \frac{d\omega_2}{\omega_2^{3/2}} D^{3}r_1 D^{3}r_2
\exp{(-K_1-K_2-V(\ver_1,\ver_2)z)}
\label{13}
\ee
As a result one obtains $\sigma_s(T)$ in the following form
$$
\sigma_s(T)= \frac{g^4(N_c^2-1)}{4}\int d^2z z/(8\pi) \int d\omega_1 d\omega_2 (\omega_1\omega_2)^{-3/2}
$$
\be
\times \sum_{n=0,1,} |\psi_n(0,0)|^2 \exp(-M_n(\omega_1,\omega_2)z) f(\sqrt{z/2\omega_1}T)f(\sqrt{z/2\omega_2}T),
\label{14}
\ee
The integrals in (\ref{14}) without factors $f(cT)$ do not contain the temperature dependent factors, and
one can see in (\ref{14}) the only T-dependent factors $g^4(T)$ and $f(\sqrt{z/2\omega_1}T)$ which define the dependence of $\sigma_s(T)$. Therefore one can write $\sigma_s(T)$ in the following form
\be
\sigma_s(T)={\rm const} g^4(T) <f^2(\sqrt{z/(2\omega)}T)>= {\rm const} g^4(T) f^2(<\sqrt{z/2\omega)}>T)
 \label{15} \ee
  The appearance of $g^4(T)$ which is decreasing with $T$ as $(\ln T)^{-2}$ defines the $T$
  dependence of $\sigma_s(T)$ to be lower than $T^2$, thus confirming the behaviour of $\sigma_s(T)$
  in the lattice data of \cite{8}, where the data were fitted
   as $\sigma_s(T)= {\rm const} g^4(T) T^2$ . However this fit fails for $T<2T_c$
   claiming the necessity of another factor in (\ref{15}).
   Correspondingly we are writing the resulting equation for the $\sigma_s(T)$ denoting the average value of
   $\sqrt{z/(2\omega)}T$ as $\rho T/T_c$. As a result one obtains the equation for the string tension
\be
\sigma_s(T)={\rm const}{g^4(T)} f^2(<w>), w= \rho T/T_c.
\label{16}
\ee
In the next sections we shall demonstrate that this new form with the well defined factor $f(<w>)$
 describes the whole region of $T>T_c$ with a good accuracy.

\section{General expression for the spatial string tension vs lattice data}

Using (\ref{10a}) $J_4(s,T) = \frac{1}{2\sqrt{\pi s}}f(w), w=2\sqrt{s}T$ one can write
$f(w)= \sum^{+\infty}_{n=-\infty}e^{-\frac{n^2}{w^2}}\equiv\vartheta_3(q),~~~q=e^{-\frac{1}{w^2}}$
with $w^2=\frac{\rho^2 T^2}{T_c^2}$. Correspondingly the $f(<w>)$ acquires the form
\be
f(<w>) = F(T/T_c)= \vartheta_3(e^{-\frac{T_c^2}{(\rho T)^2}})
\label{17} \ee
 The numerical analysis of the data \cite{8} allows to
reproduce well the data with the equation of the form \be
\sigma_s(T)=\sigma_s(T_c) \frac{g^4(T) F^2(T/T_c)}{g^4(T_c)F^2(1)}
\label{18} \ee

Analysis of the lattice data from the (\ref{18}) is shown in Fig 2 , where for $g^4(T)$  in the Appendix 1
the explicit value of the $L_\sigma= 0.104 $ as in the \cite{8} was used while in $f(<w>)$ in (\ref{17})
the value $\rho= 3   $. The Fig 2 demonstrates a good agreement between the lattice data and our equation (\ref{18}),
including the region $ T< 2.5 T_c$ where the lattice fit $T^2 g^4(T)$ starts to disagree with numerical data.

\begin{figure}%[h]
\center{\includegraphics[width=0.7\linewidth]{Fit.eps}}
    \caption{Spatial string tension $\sigma_s (T)/\sigma$ for SU(3)
gauge theory as function of $T/T_c$.  The lattice data are from Refs.\cite{8}. $T_{c}$=270 MeV}
    \label{FiG2}
\end{figure}


\section{Discussion of results and Conclusions}

The main purpose of our work was the construction of the detailed mechanism of the spatial confinement
in the whole region of the temperature from $T=T_c$ to asymptotically large temperatures. As it can be seen in the Fig 2 our resulting curve for the spatial string tension is in a good agreement with the
accurate lattice data in the whole measured region $T_c< T < 5 T_c$ where our result does not contain
fitting parameters except for the standard evolution form for the coupling constant depending on the temperature $T$ given in Appendix 1, which has also allowed the authors of \cite{8} to get agreement
with their data in the asymptotic region. To describe also the region of smaller $T$ we have used the formalism
of the elliptic $\vartheta_3(z)$ functions which describe well the sharp transition of the gluon propagator in the two-gluon (gluelump) Green's function from the constant to the linear behaviour. One should stress that the formalism presented in the paper is standard for the temperature dependence of any Green's functions developing in the spacial or time-like continuum with inclusion of interaction via the
field correlators \cite{10,11}. The inclusion of the temperature via the Matsubara-type formalism with the T-dependent factors $I(x_4-y_4,T)= \exp(-ip_4(x_4-y_4 -n/T))$ in the gluon Green's functions is shown in (\ref{8}). For the space-like correlators with $x_4=y_4$  the use of the Poisson summation formula
$1/(2\pi) \sum_n \exp(ip_4 n\beta)= \sum_k \delta(p-4 \beta -2\pi k), \beta=1/T$, brings about an additional factor of $T$. This finally leads to the $T^2$ dependence of the leading term in the $\sigma_s(T)$ as in the (\ref{3}).
  On the contrary for the time-like Green's functions with the nonzero $x_4-y_4$
the $T$ dependence is dictated by the corresponding mass parameters and  for the $\sigma_E(T)$ the situation is even more dramatic since it drops to zero(deconfinement) at $T=T_c$ approximately as $(1-(T/T_c)^4)^{1/2}$ \cite{22}. One can wonder why these two phenomena -spatial (colormagnetic) confinement and the colorelectric confinement are so different and hence disconnected and as follows from the lattice data (see Fig 9,10 in \cite{5}) the CE gluon condensate $<G_2^E (T)>$ and the CM condensate $<G_2^M (T)>$ being equal at $T=0$ behave also in a similarly different manner with the growing $T$ ?
The answer  lies in the different active regions of these phenomena- the space-like continuum for CM and
the time-like continuum for the CE confinement which have a little dynamical intersection as space-like
and time-like surfaces, which is evident in the FCM  and is an additional argument in favor of its selfconsistency. As it is we have found  a good agreement of our FCM approach for CM string tension with lattice data \cite{8} in this paper as well as good agreement of all our CE calculations with the corresponding lattice  and experimental data \cite{2,3,4,5,6,10,11,12,12*} including the latest CE calculations of the deconfining process  \cite{22}. Turning back to the CM physics it was found within
our approach that
even more important  role of the spatial string tension may be  in the high $T$ thermodynamics where
in the framework of FCM it provides the basic nonperturbative contribution to the pressure and other observables,see e.g. \cite{18}, in good agreement with the lattice data and solving as in \cite{12 } the old "Linde problem" which precludes pure perturbative thermodynamic calculations of interacting systems at large temperature. Another interesting development of this method is the dynamical theory of the QCD  systems
in the external magnetic field where the FCM yields all results in a good agreement with lattice data without any parameters, see e.g. \cite{ 23,24 }. In this way the FCM plays an important role in the
development of the present QCD theory.

\section{Acknowledgments.}
The work of Khaidukov.Z.V was supported by the Russian Science Foundation 21-12-00237.
\section*{ Appendix A1. Two-loop expression for $g^{-2} (t)$}


 \setcounter{equation}{0} \def\theequation{A1.\arabic{equation}}
From the point of view of  FCM  the running coupling enters naturally in the formalism   qualitatively in the same way as in the standard theory  \cite{25}.
 One can decompose the  non-abelian gauge field into  perturbative and non-perturbative parts $A_{\mu}=B_{\mu}+a_{\mu}$.  At this step one can  treat $B_{\mu}$ as an external gauge field and at the next step  integrate out perturbative fields in path integral. This procedure in UV domain produces the running coupling constant and it can be used for calculation of  beta-function for example in three loops   \cite{CR}. As a result one can use the two-loop beta-function calculations on the lattice in SU(3) gluodynamics \cite{8} where the  expression has the standard form as a function of $t= \frac{T}{T_c}$ :
\be
g^{-2}(t)= c_0 \ln \frac{t}{L_{\sigma}} + c_1 \ln\left(2 \ln\frac{t}{L_{\sigma}}\right),
\label{A1.1}
\ee
where
\be
c_0= \frac{11}{8\pi^2}, c_1= \frac{51}{88\pi^2}.
\label{A1.2}
\ee
Here $L_\sigma= \frac{\Lambda_\sigma}{T_c}= 0.104 \pm 0.009$ as in \cite{8,9}. All other parameters that we used for the calculations involving $g(t)$ are the same as in  \cite{8}.

\section*{Appendix A2. Calculation of $\sigma_s(T)$ via gluelump Green's functions}

 \setcounter{equation}{0} \def\theequation{A2.\arabic{equation}}

We can write $J_4$ in \ref{9} in the limits of large $T$  and $T=0$ as \be
J_4 =\frac{1}{2\sqrt{\pi s}},  T=0 \ee \be J_4 = T,  T \gg (2\sqrt{\pi s})^{-1}.
\label{10}
\ee.
As  a result the $4d$ gluon propagator at large temperature $T$ reduces to the linear in $T$ the $3d$ propagator and temperature independent part   $K_{3d}(z)$
\be
G_{4d}^{(g)}(z,T) = T G_{3d}^{(g)}(z)+K_{3d}(z) ,\label{A2.1}
\ee
 Substituting the 4d gluon propagator in  the  general expression for $D^H(z)$, one has
\be
D^H(z) = \frac{g^4(N^2_c-1) }{2} < G^{(2g)}_{4d} (z,T)>,\label{A2.2}
\ee
 $G^{(2g)}(z,T)$ is formed from the product of two one-gluon Green's functions $G^{(g)}(z,T)$ and the fixed gluon Wilson line (parallel transporter) where both gluons and the Wilson line are connected by the adjoint strings, we denote the whole construction by the sign $<...>$.

In terms of the gluelump phenomenology, studied in \cite{6}, the expression (\ref{A2.2}) is called the two-gluon gluelump, which was computed on the lattice and analytically in
\cite{6}. Choosing in $4d$ the $x_3\equiv t$ axis as the
Euclidean time, we proceed exploiting the path integral technic \cite{10,11,12}, which yields
\be
G^{(2g)}_{4d} (x-y) = \frac{t}{8\pi} \int^\infty_0 \frac{d\omega_1}{\omega_1^{3/2}} \int^\infty_0
\frac{d\omega_2}{\omega_2^{3/2}} (D^3r_1)_{xy}(D^3r_2)_{xy}
e^{-K_1(\omega_1)-K_2(\omega_2)-Vt}, \label{A2.3}
\ee
where $V$ includes the spatial confining interaction between the three objects: gluon 1, gluon 2, and  the fixed straight line of the parallel transporter, which makes all construction
gauge invariant (see \cite{6,15} for details). In (\ref{A2.3}) $t=|x-y|\equiv |w|;$ .
Constructing in the exponent of (\ref{A2.3}) the three-body Hamiltonian in the $3d$ spatial coordinates, in the standard
FCM procedure \cite{11} one has
\be
H(\omega_1, \omega_2) = \frac{ \omega_1^2+ \vep^2_1}{2\omega_1}+  \frac{
\omega_2^2+ \vep^2_2}{2\omega_2}+ V(\ver_1, \ver_2), \label{A2.4}
\ee
one can rewrite (\ref{A2.3}) as follows (see \cite{12}),
 \be
 G^{(2g)}_{4d} (t) = \frac{t}{8\pi} \int^\infty_0 \frac{d\omega_1}{\omega_1^{3/2}} \int^\infty_0
\frac{d\omega_2}{\omega_2^{3/2}} \sum^\infty_{n=0} |\psi_n (0,0)|^2 e^{-M_n
(\omega_1,\omega_2) t}.\label{A2.5}
\ee
This equation has the universal form in $3d$ and $4d$ dimensions, the resulting expressions for $G^{(2g)}$  differ
in the values of $M_n$ and values and dimensions of $|\psi_n(0,0)|^2$.
Here $\Psi_n(0,0)\equiv \Psi_n (\vez_1,\vez_2)|_{\vez_1=\vez_2=0}$, and $M_n$  is the eigenvalue of $H(\omega_1,
\omega_2)$. The latter was studied in \cite{6} in three spatial coordinates. For our purpose here we only mention that $ G^{(2g)}_{3d}  (z)$ has the dimension
of the mass squared and therefore the integral (\ref{4}) defining $\sigma_s$ is  dimensionless.
For the further analysis of this expression see \cite{12*}.






\begin{thebibliography}{99}

\bibitem{1}
H. G. Dosch and Yu.A.Simonov, Phys. Lett. {\bf  B 205}, 339 (1988).

\bibitem{2}
Yu. A. Simonov, Phys. Usp. {\bf 166}, 337 (1996), arXiv: hep-ph/9709344; D. S. Kuzmenko, V. I. Shevchenko, and Yu. A. Simonov, Phys. Usp. {\bf 47}, 1 (2004),
arXiv: hep-ph/0310190.

\bibitem{3}
A. Di Giacomo, H. G. Dosch, V. I. Shevchenko, and Yu. A. Simonov, Phys. Rept. {\bf 372}, 319 (2002), arXiv: hep-ph/0007223.
\bibitem{4}
Yu. A. Simonov, Phys. Rev. {\bf D 99}, 056012 (2019), arXiv: 1804.08946.

\bibitem{5}
 M. D'Elia, A. Di Giacomo, and E. Meggiolaro, Phys. Rev. {\bf D 67}, 114504 (2003), arXiv: hep-lat/0205018.

\bibitem{5*}
Yu. A. Simonov, The colormagnetic confinement in QCD, arXiv: 2203.07850 [hep-ph].

\bibitem{6}
Yu. A. Simonov, Nucl. Phys., {\bf B 592}, 350 (2001), arXiv: hep-ph/0003114.


\bibitem{7*}
C. Borgs, Nucl. Phys. {\bf B 261}, 451 (1985); E. Manousakis and J. Polonyi, Phys. Rev. Lett. {\bf 58}, 847 (1987).

\bibitem{8}
G. Boyd et al., Nucl. Phys. {\bf B 469}, 419 (1996), arXiv: hep-lat/9602007.

\bibitem{9}
F. Karsch, E. Laermann, and M. Lutgemeier, Phys. Lett. {\bf B 346}, 94 (1995), arXiv: hep-lat/9411020.

\bibitem{10}
Yu. A. Simonov, Phys. Atom. Nucl, {\bf 58}, 339 (1995); N. O. Agasian, JETP Lett. {\bf 57}, 208 (1993);
JETP Lett. {\bf 71}, 43 (2000); Phys. Lett.  {\bf B 519}, 71 (2001), [arXiv:hep-ph/0104014].

\bibitem{10*}
N.~O.~Agasian,
%``Thermal gluomagnetic vacuum of SU(N) gauge theory,''
Phys. Lett. B \textbf{562},  257 (2003), arXiv:hep-ph/0303127.

\bibitem{11}
Yu. A. Simonov,    arXiv:hep-ph/9311216
Yu. A. Simonov, Phys. Atom. Nucl. {\bf 69}, 528 (2006), arXiv: hep-ph/0501182; Yu. A. Simonov and V. I. Shevchenko, Adv. High Energy Phys. {\bf 2009}, 873051 (2009),
arXiv: 0902.1405 [hep-ph].

\bibitem{11*}
I. H. Jorisz and C. Michael, Nucl. Phys. {\bf B 302}, 448 (1988) ,
M. Foster and C. Michael, Phys. Rev. {\bf D 59}, 094509 (1999)



\bibitem{12}
Yu. A. Simonov, Phys. Rev. {\bf D 96}, 096002 (2017), arXiv: 1605.07060.

\bibitem{12*}
Yu.A.Simonov, The spatial string tension and the nonperturbative Debye mass from the Field Correlator Method, Phys. Atom. Nucl. {\bf 85}, 304 (2022), arXiv:2206.14489.



\bibitem{12**}
N.~O.~Agasian and I.~A.~Shushpanov,
%``The Quark and gluon condensates and low-energy QCD theorems in a magnetic field,''
Phys. Lett. B \textbf{472}, 143 (2000), [arXiv:hep-ph/9911254];
JHEP \textbf{10}, 006 (2001), [arXiv:hep-ph/0107128].

\bibitem{18}
N. O. Agasian, M. S. Lukashov, and Yu. A. Simonov, Eur. Phys. J. {\bf A 53}, 138 (2017), arXiv: 1701.07959;
Mod. Phys. Lett. A \textbf{31}, no.37, 1650222 (2016), [arXiv:1610.01472 [hep-lat]].

\bibitem{15}
N. O. Agasian and Yu. A. Simonov, Phys. Lett. {\bf B 639}, 82 (2006), arXiv:hep-ph/0604004.

\bibitem{14}
E. L. Gubankova and Yu. A. Simonov, Phys. Lett. {\bf B 360}, 93 (1995), arXiv:hep-ph/9507254.


\bibitem{13}
G. S. Bali et al., Phys. Rev.  {\bf 71}, 3059 (1993); hep-lat/9306024.

\bibitem{13*}
M. Teper, Phys. Lett. {\bf B 311}, 223 (1993).


%\bibitem{16}
%O. Kaczmarek and F. Zantow, Contribution to: Workshop on Extreme QCD, 108-112, arXiv: hep-lat/0512031.

%\bibitem{17}
%O. Kaczmarek and F. Zantow, Phys. Rev. {\bf D 71}, 114510 (2005), arXiv: hep-lat/0503017.


\bibitem{19}
Dan La Course and M. G. Olsson, Phys. Rev. {\bf D 39}, 2751 (1989).

\bibitem{20}
W. Lucha, F. F. Schoeberl and D. Gromes, Phys. Rep. {\bf 200}, 127 (1991).

\bibitem{21}
Yu. A. Simonov, Phys. Lett. {\bf B 226}, 151 (1989); Yu. A. Simonov, QCD and Theory of Hadrons, in: ``QCD: Perturbative or
Nonperturbative." Interscience, Singapore, 2000; arXiv: hep-ph/9911237.
\bibitem{22}
M. S. Lukashov and Yu. A. Simonov, arXiv:2305.00558

\bibitem{23}M.A.Andreichikov and Yu.A.Simonov, EPJC (78) 420 (2018),arXiv:1712.02925
\bibitem{24}V.D. Orlovsky and Yu.A.Simonov,Phys.Rev.D89,07434(2014), arXiv:1311.1087
\bibitem{25} M.Peskin, D.Schroder,"'An Introduction To Quantum Field Theory"', CRC Press 2019
\bibitem{CR}J.-P. Boernsen, Anton E. M. van de Ven,Nucl.Phys. B657 (2003) 257-303,arXiv:hep-th/0211246





\end{thebibliography}

\end{document}

\section*{Appendix A2. Numerical calculation of $c^2_{\sigma}$ in the lowest approximations}



 \setcounter{equation}{0} \def\theequation{A2.\arabic{equation}}

We calculate here two lowest eigenvalues and eigenfunctions of the Hamiltonian, which enter
 in the expression for the $c^2_{\sigma}$ in (\ref{16}). This Hamiltonian without spin-dependent terms can be written in the equivalent oscillator form,
 \be
 H= \frac{\omega^2_1 + \vep^2_1}{2\omega_1} + \frac{\omega^2_2 + \vep^2_2}{2\omega_2} + \frac{\sigma^2\ver^2_1}
{2 \nu_1} + \frac{\sigma^2 \ver^2_2}{2 \nu_2} + \frac{\sigma^2 (\ver_1- \ver_2)^2}{2 \nu_3} + \frac{\nu_1 + \nu_2 + \nu_3}{2}.
\label{A2.1}
\ee
Here we have used the property $\sigma |\ver|= \min\left(\frac{\sigma^2 \ver^2 + \nu^2}{2 \nu}\right)$, so that
minimizing the eigenvalues of (\ref{A2.1}) in the variables $\nu_i$, we obtain the eigenvalues and eigenfunctions of the Hamiltonian with a good accuracy $\sim 5\%$.
Then for $\omega_1 = \omega_2 = \omega$ and $\nu_1= \nu_2=\nu$ one obtains the lowest eigenvalue for n=0,

\be
 M_0= \omega + \frac{\sigma}{\sqrt{\omega\nu}}\left(1 + \sqrt{\frac{\nu_3 +2\nu}{\nu_3}}\right) + \frac{2\nu+\nu_3}{2}.
\label{A2.2}
\ee
The conditions of minima $\frac{\partial M_0}{\partial z_i}= 0$ with $z_i= \omega_i,\nu_i$ yields the
final result with notation $\omega_i(0),\nu_i(0)$ for the extremal values.

\be
 \omega_1(0)= \omega_2(0)= 1.29 \sqrt \sigma,~ \nu_0= 0.79 \sqrt \sigma,~ \nu_3(0)= 1.25 \nu_0,~
 \min (M_0)=4.95 \sqrt \sigma.
 \label{A2.3}
 \ee
From the oscillator wave functions it easy to get the factor $|\psi(0,0)|^2= 1.61 \sigma^2$ and
to calculate the integrals over $d\omega_1 d\omega_2$ in (\ref{16}), expanding $M_0(\omega_1,\omega_2)$ near the stationary points in (\ref{A2.3}) up to the second order in $\omega_i-\omega_i(0)$ and denoting the second derivative of $M_0$ as $M"(\omega_0)$. Then for the integral in (\ref{17})
one has
\be
\int d^2 w G(w)= \frac{2\pi |\psi(0,0)|^2}{M^2_0 \omega^3_0 M"(\omega_0)}.
\label{A2.4}
\ee
Inserting the stationary values from (\ref{A2.3}) and the second derivative at the stationary point
$M"(\omega_0)= 0.51 \sigma^{-1/2}$, one finally obtains

\be
 \int d^2 w G(w)= \frac{2\pi 0.228 |\psi|^2}{\sigma M^2_0}= 0.093,~
c^2_{\sigma} = \frac{N^2_c-1}{4} \int d^2 w G(w) = 0.186 ~(N_c=3).\label{A2.5}
\ee
In a similar way one can calculate the contribution of the $n=2$ term in the (\ref{18}), which yields approximately
$c^2_{\sigma}(n=2)= 0.019$ and for the sum of two terms with $n=0,2$   $c^2_{\sigma} = 0.205$ for $N_c=3$ which is the lower bound. However as will be seen below the most important corrections appear when one estimates the accuracy of the replacement
of the original linear confinement Hamiltonian (\ref{18}) by the oscillator Hamiltonian (\ref{A2.1}). To get an idea of this effect we can estimate the ratio of the integral in (\ref{A2.5}) which we denote as $I_{\rm osc}$ and the corresponding integral
for the real (linear) interaction $I_{ \rm lin}$. To simplify matter we replace $|\psi(0)|^2$ and $M_0$ of the gluelump system by the  simple two gluon system connected by the linear or oscillator interaction and write approximately
\be
R= \frac{I_{\rm lin}}{I_{\rm osc}} \approx \frac{|\psi_{\rm lin}(0)|^2 M^2_{\rm osc}}{|\psi_{\rm osc}(0)|^2 M^2_{\rm lin}}.
\label{A2.6} \ee
For the two-gluon system with linear confining interaction the spectrum and wave functions are well known \cite{19,20,21}:
\be
M_n= 4 \sqrt{\sigma} (\frac{a(n)}{3})^{3/4},~ |\psi_{\rm lin}(0)|^2= \frac {\sigma M_0}{16 \pi},
\label{A2.7} \ee
where for the ground state $n=0$ $a(0)= 2.338$.
Inserting for the linear and oscillator potentials the resulting values $M_{\rm lin}= 3.31 \sqrt {\sigma}$, $ M_{\rm osc}= 3.59 \sqrt {\sigma}$ and
$ |\psi_{\rm lin}(0)|^2= 0.065 \sigma^{3/2}$, $ |\psi_{\rm osc}(0)|^2= 0.043 \sigma^{3/2}$, one obtains the approximate ratio which estimates  the effect of the replacement by oscillator interaction
\be
R=  \frac{c^2_{\sigma}({\rm lin})}{c^2_{\sigma}({\rm osc})} \approx 1.82;~ c_{\sigma}({\rm lin}) \approx  1.35 c_{\sigma}({\rm osc}).
\label{A2.8} \ee
As a result using the $n=0$  oscillator value of $c^2_{\sigma}$ in (\ref{A2.5}) we obtain  the linear confinement coefficient $c_{\sigma} \approx 0.582 $ which agrees well with the lattice value $ 0.566 \pm 0.013$ from \cite{8,9}.
A more accurate calculation of the $c_{\sigma}({\rm lin})$ is possible with the solution of the linear integral equations for the
gluelump Green's functions as it was done in \cite{6} for the gluelump masses.
