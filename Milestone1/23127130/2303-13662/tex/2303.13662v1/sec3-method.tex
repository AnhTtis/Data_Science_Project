%%%%%%%%%%%%%%%%%%%%%%%%%%%%%%%%%%%% SETUP  %%%%%%%%%%%%%%%%%%%%%%%%%%%%%%%%%%%%
\Section{Method}
\label{sec:method}
\vspace{\segsep}

We formally introduce the new learning framework, \textit{FAS with separability and alignment} (dubbed \textbf{SA-FAS}). The goal is to produce a feature space with two critical properties: 
(1) \textit{Separability}: We encourage samples from different domains and from different classes to be well-separated; 
(2) \textit{Alignment}: Live-to-spoof transition\footnote{The transition can be considered as a path in the high-D manifold. }  is aligned in the same direction for all domains.  
These two properties work jointly: 
\textit{separability} ensures the awareness of domain variance in the feature space; \textit{alignment} encourages the domain variance to be invariant to its live-vs-spoof hyperplane. 
%

This section is structured as follows:
Sec.~\ref{sec:setup} describes the problem setup, followed by the algorithm design of separability (Sec.~\ref{sec:sep}) and alignment (Sec.~\ref{sec:align}).
Finally, Sec.~\ref{sec:train_step} summarizes the training and inference processes.

\begin{figure*}[t]
    \small\centering
    \includegraphics[width=0.95\textwidth]{figs/method.pdf}
    \caption{\small 
        {\bf Optimization objectives:}
        Illustration of feature space optimized by different objectives: (a) ERM, (b) ERM+SupCon, (c) SA-FAS (ours). 
        Circle/cross denotes live/spoof label; different colors indicate different domains. 
        A UMAP visualization for real data is provided in Appendix (Fig.~\ref{fig:method-sup}) to support the feature distribution shown in the toy example.
        %
    }
    \label{fig:method}
\end{figure*}

%\protect\footnotemark
%\footnotetext{}

%

\SubSection{Problem Setup}
\label{sec:setup}
We start by defining the setting of the cross-domain FAS problem. We denote by $\mathcal{X}=\mathbb{R}^d$  the input space and 
$\mathcal{Y}=\{0 \text{ (live)}, 1 \text{ (spoof)}\}$ 
the output space. A learner is given access to a set of training data from $E$ domains $\mathcal{E} = \{e^{(1)}, e^{(2)}, .., e^{(E)}\}$ and is evaluated on test domain $e^*$. Let $e_i$ as the domain label for the $i$-th sample, we denote $\mathcal{D}=\{(\*x_i,y_i, e_i)\}_{i=1}^N$ drawn from an unknown joint data distribution $\mathcal{P}$ defined on $\mathcal{X} \times \mathcal{Y} \times \mathcal{E}$. 
Cross-domain FAS is a special binary classification problem to distinguish live and spoof faces from an unseen domain. The goal is to define a decision function:
\begin{align}
    f: \mathbf{x} \rightarrow \{0 \text{ (live)}, 1 \text{ (spoof)}\}, \nonumber
\end{align}
which classifies whether a sample $\*x$ from a new domain $e^*$ is live or spoof. 

In our network architecture, function $f$ consists of two components: (1) a deep neural network encoder $\phi:\mathcal{X} \rightarrow \mathbb{R}^m$ that maps the input $\bx$ to a $l_2$-normalized feature embedding $\*z = \phi(\bx)$; 
%
(2) a classifier (via a weight vector) $\beta:\mathbb{R}^m \rightarrow \mathbb{R}$ that maps the $m$-dimensional embedding $\*z$ to a scalar value, where a binary cross-entropy loss can be applied after using a sigmoid function. Because the true distribution of live/spoof data is unknown, the optimization commonly relies on an Empirical Risk Minimization (ERM).

\noindent \textbf{Remark on the terminology}: $\beta$ can be considered as a norm vector of the hyperplane separating live and spoof samples. In the remaining part of the paper, when we use ``\textbf{live-vs-spoof hyperplane}'' or ``\textbf{hyperplane}'', it has the same meaning as $\beta$. Note, ``live-to-spoof transition'' is an abstract procedure in the image space, while ``live-vs-spoof hyperplane'' refers to a concrete classifier in the feature space. 

\Paragraph{Preliminary on Empirical Risk Minimization (ERM):}
ERM principle~\cite{vapnik1991principles} is a ubiquitous strategy that merges data from all training domains and learns a predictor that minimizes an averaged training error.
Specifically,
\begin{align}
    & \mathcal{L}_{\textit{ERM}} = \min _{\phi, \beta} \frac{1}{|\mathcal{E}|} \sum_{e \in \mathcal{E}} \mathcal{R}^{e}(\phi, \beta), 
\label{eq:erm}
\end{align}
where the empirical risk function $\mathcal{R}^{e}(\phi, \beta)$ for a given environment $e$ is defined by:
$$
\mathcal{R}^{e}(\phi, \beta) \triangleq \mathbb{E}_{(\*x_i, y_i, e_i=e) \sim \mathcal{D}} \ell\left(f(\*x_i;\phi, \beta), y_i\right).
$$
Common choices of the loss function $\ell(\cdot,\cdot)$ include cross-entropy loss~\cite{jia2020ssdg} and $L_1$ regression loss~\cite{liu2018learning, george2019deep}.

However, if samples from different domains are mixed together, 
ERM can utilize the easiest difference (image resolution, blurriness, camera setting) to differentiate live \vs.~spoof. 
Such a classifier will undesirably leverage spurious correlations to make  live/spoof predictions~\cite{arjovsky2019irm}. Therefore, the naive strategy can hurt the generalization of the unseen domain. % 
As shown in Fig.~\ref{fig:method}(a), ERM tends to fit all training data together and fails to learn a domain-invariant classifier with the mixed feature space. 
%

%%%%%%%%%%%%%%%%%%%%%%%%%%%%%%%%%%%% METHOD  %%%%%%%%%%%%%%%%%%%%%%%%%%%%%%%%%%%%
%
\SubSection{Separability}
\label{sec:sep}

%
We characterize the domain separability as supervised contrastive learning (dubbed SupCon)~\cite{2020supcon}, one of the latest developments for visual representation learning. Unlike other contrastive learning methods~\cite{chen2020simclr, chen2021exploring} that treat the augmented samples as a single class, SupCon aims to learn a representation space that gathers samples with the same labels while repelling samples from different ones. It naturally suits the need for the cross-domain FAS setting, since we treat samples with the same domain and with the same live/spoof label to form a cluster. 
%


%\textbf{Training objective} 
Given a training mini-batch $\{\*x_i, y_i, e_i\}_{i=1}^b$, we augment~\cite{2020supcon} the mini-batch as $\{\tilde{\*x}_i, \tilde{y}_i, \tilde{e}_i\}_{i=1}^{2b}$, using two random augmentations $\tilde{\bx}_{2i}$ and $\tilde{\*x}_{2i-1}$ of inputs $\bx_i$, with $\tilde{y}_{2i-1} \!=\! \tilde{y}_{2i} \!=\! y_i$, $\tilde{e}_{2i-1} \!=\! \tilde{e}_{2i} \!=\! e_i$. These images are fed into the network, yielding $L_2$-normalized embeddings $\{\*z_i\}_{i=1}^{2b}$. The per-batch SupCon loss (separability loss) is defined as:
\begin{equation}
\mathcal{L}_\textit{sep}=\sum_{i=1}^{2b} \frac{-1}{|S(i)|} \sum_{j \in S(i)} \log \frac{\exp \left(\*z_{i} \cdot \*z_{j} / \tau\right)}{\sum_{t=1, t\neq i}^{2b}\exp \left(\*z_{i} \cdot \*z_{t} / \tau\right)},
\label{eq:supcon}
\end{equation}
where $\tau$ is a temperature parameter, $i$ is the index of a sample typically called the \textit{anchor}, $S(i) \!=\! \{j \!\in\!\{1,\ldots,2b\} \!:\! j\neq i, \tilde{y}_j = \tilde{y}_i, \tilde{e}_j = \tilde{e}_i\}$ is the index set of \emph{positive samples} 
%consisting of the indices of the augmented sample and other 
that have the same live/spoof labels and belong to the same domain as the anchor $i$, and $|S(i)|$ is its cardinality. 
All the other samples in the mini-batch are referred to as \textit{negative samples}. 
%
Since positive samples are pulled together and negative samples are pushed apart, SupCon in Fig.~\ref{fig:method}(b) is capable of providing more distinguishable feature clusters for different domains and liveness classes, compared to a typical feature space learned by a vanilla ERM in Fig.~\ref{fig:method}(a). 
%
\SubSection{Alignment}
\label{sec:align}

% 
Fig.~\ref{fig:method}(b) also shows that separability alone is not sufficient for improving domain generalization. 
The separated feature clusters can be located in any place in the feature space, and hence the domain-wise optimal hyperplane remains \textbf{variant}.
In this case, the global classifier can still undesirably incorporate the spurious correlation as the deciding factor as we show in Fig.~\ref{fig:teaser}.
%  
To tackle this, we naturally investigate the following problem: 

\textit{How do we regularize a global live-vs-spoof hyperplane to align with domain-wise live-vs-spoof hyperplanes?}

We propose to formulate this problem as Invariant Risk Minimization (IRM)~\cite{arjovsky2019irm}, which aims to jointly optimize the feature space $\phi$ and the global live-vs-spoof hyperplane $\beta$, where $\beta$ is also optimal for each domain, shown in Fig.~\ref{fig:method}(c). 

%   

\begin{figure*}[t]
    \small\centering
    \includegraphics[width=\textwidth]{figs/illustration.pdf}
    \vspace{-4ex}
    \caption{\small 
    {\bf PG-IRM optimization process:}
    An illustration of why a vanilla IRM can suffer from an infeasible solution (a), and how the proposed PG-IRM algorithm jointly updates the feature space and multiple hyperplanes towards convergence (b)-(d). }
    \label{fig:illustration}
\end{figure*}

% 
\Paragraph{Preliminary on Invariant Risk Minimization (IRM):}
Specifically, the IRM objective can be formulated as the following constrained optimization problem:

\begin{align}
    & \min _{\phi, \beta^{*}} \frac{1}{|\mathcal{E}|} \sum_{e \in \mathcal{E}} \mathcal{R}^{e}(\phi, \beta^{*}) \rightarrow \mathcal{L}_{\textit{IRM}}  \label{eq:irm_target} \\ 
    & \textit{s.t.} \quad \beta^{*} \in \underset{\beta}{\arg\min} \mathcal{R}^{e}(\phi, \beta), \forall e \in \mathcal{E}.
\label{eq:irm_constrain}
\end{align}

%

Compared to the ERM~\eqref{eq:erm}, IRM enforces an additional constraint \eqref{eq:irm_constrain} to learn the domain-invariant hyperplanes. 
Specifically, if we define the domain-wise optimal hyperplane as 
%
$\beta_e \in {\arg\min  }_{\beta} \mathcal{R}^{e}(\phi, \beta)$.
A sufficient condition for constraint \eqref{eq:irm_constrain} to hold is $\beta_{e^{(1)}}=...=\beta_{e^{(E)}} = \beta^*$, which requires consistency between the globally optimal hyperplane and the domain-wise optimal hyperplanes.
However, IRM is known to be hard to solve~\cite{kamath2021doesirm,rosenfeld2020riskirm} due to the bi-level optimization nature of objective~(\ref{eq:irm_target}) and constraint~(\ref{eq:irm_constrain}). 

%

%\paragraph{PG-IRM objective} 
\Paragraph{Projected Gradient Optimization for IRM (PG-IRM):} 
%
We leverage Projected Gradient (PG) algorithm~\cite{nocedal1999numerical} to solve the non-trivial optimization objective~\eqref{eq:irm_target}, termed as PG-IRM. 
In PG-IRM, we propose to optimize multiple hyperplanes and converge them into a globally one via projected gradient. In Appendix~\ref{sec:proof_s1}, we provide detailed proof of PG-IRM objective being \textbf{equivalent} to IRM. Formally, the objective is rewritten as:
\begin{theorem}
\label{th:ourirm_objective}
\textbf{(PG-IRM objective)} For all $\alpha \!\in\! (0,1) $, the IRM objective is equivalent to the following objective: 

\begin{align}
    & \min _{\phi, \beta_{e^{(1)}}, ..., \beta_{e^{(E)}}} \frac{1}{|\mathcal{E}|} \sum_{e \in \mathcal{E}} \mathcal{R}^{e}(\phi, \beta_e) \rightarrow \mathcal{L}_{\textit{align}} \\ 
    &\text { s.t. }  \forall e \in \mathcal{E}, \exists \beta_e \in \Omega_e(\phi), \beta_e \in \Upsilon_{\alpha}(\beta_e), \nonumber 
    % 
\end{align}
where the parametric constrained set for each environment is simplified as 
$ \Omega_e(\phi) = \underset{\beta}{\arg \min } \mathcal{R}^{e}(\phi, \beta),$
and we define the \textbf{$\alpha$-adjacency set}: 
\begin{align}
    \Upsilon_{\alpha}(\beta_e) = \{\upsilon |& \underset{e' \in \mathcal{E} \backslash e}{\max}  \ \underset{\beta_{e'} \in \Omega_{e'}(\phi)}{\min}\|\upsilon - \beta_{e'}\|_2 \\
    & \le \alpha \underset{e' \in \mathcal{E} \backslash e}{\max}  \ \underset{\beta_{e'} \in \Omega_{e'}(\phi)}{\min}\|\beta_e - \beta_{e'}\|_2\}
    \label{eq:gamma_set}
\end{align}
\figvspace
\end{theorem}
Fig.~\ref{fig:illustration} shows the intuition of  the optimization process. 
%
For a $3$-domain case, PG-IRM starts with a shared feature space $\phi$ and $3$ separate hyperplanes $\beta_{e^{(1)}}$, $\beta_{e^{(2)}}$, $\beta_{e^{(3)}}$ for each domain (Fig.~\ref{fig:illustration}(b)). 
After each projected gradient descent, the hyperplanes move closer with feature space jointly updated (Fig.~\ref{fig:illustration}(c)). 
Upon convergence, $\beta_{e^{(1)}}, \beta_{e^{(2)}}, \beta_{e^{(3)}}$ become nearly identical (Fig.~\ref{fig:illustration}(d)), satisfying the IRM constraint $\beta^* \!=\! \beta_{e^{(1)}} \!=\! \beta_{e^{(2)}} \!=\! \beta_{e^{(3)}}$ for {\it all} domains.
%The above explanation delivers the following 
We provide two main insights of our PG-IRM  algorithm (see more details in Appendix~\ref{sec:proof}):
\begin{compactenum}
    \item \textbf{Optimizing multiple  hyperplanes:} 
    Compared to the conventional IRM that optimizes a single hyperplane, it is easier to converge for PG-IRM that optimizes multiple hyperplanes (\ie, one for each domain).
    Shown in Fig.~\ref{fig:illustration}(a-b), for the same feature space from the intermediate optimization stage, the solution $\beta^*$ to conventional IRM may not exist and the optimization has to be terminated.
    In contrast, $\beta_{e^{(1)}}, ..., \beta_{e^{(E)}}$ \textbf{always} exists (Fig.~\ref{fig:illustration}(b)) which makes solving for multiple hyperplanes more viable.
    %
    \item \textbf{Pushing hyperplanes to be closer:}
    To align $\beta_{e^{(1)}}$, $\beta_{e^{(2)}}$ and $\beta_{e^{(3)}}$,
    %
    PG-IRM updates domain-wise hyperplanes by interpolating with other hyperplanes. It can be mathematically considered as projecting the parameters of a hyperplane into the $\alpha$-adjacency set $\Upsilon_{\alpha}(\beta_e)$ as we illustrated in Fig.~\ref{fig:proj}.

    \begin{figure}[t!]
    \small\centering
    \includegraphics[width=.4\textwidth]{figs/projection.pdf}
    \caption{\small 
        {\bf Euclidean projection:}
        Illustration of Euclidean projection (solid black dot) to the $\alpha$-adjacency set $\Upsilon_\alpha\left(\beta_e\right)$. 
        Detailed proof and steps are provided in Alg.~\ref{alg:proj_grad_appendix} in Appendix~\ref{sec:proof_s2}. 
    }
    \figvspace
    \label{fig:proj}
\end{figure}

    \textbf{Remark (why PG is not applicable to IRM):}
    The PG algorithm can be infeasible for the conventional IRM, as the solution set for \eqref{eq:irm_constrain} can be \textbf{empty} and is thus non-projectable.
    Our PG-IRM objective in Eq.~\eqref{eq:gamma_set} contains a non-empty $\alpha$-adjacency set $\Upsilon_{\alpha}(\beta_e)$, and guarantees being projectable by simple linear interpolation. 

    %
\end{compactenum}


%\Paragraph{Intuition.}  For the reader's convenience, 


% \begin{algorithm}[t]
\begin{algorithmic}\State Initialize $\phi, \beta_{e^{(1)}}, ... , \beta_{e^{(E)}}$, learning rate $\gamma$, alignment parameter $\alpha$, alignment starting epoch $T_a$.
\For{$\text{t in 0, 1, ..., }$}
\State Run forward pass and calculate the gradient.
\For{$e \in \mathcal{E}$}
    
    \State $\tilde{\beta}^{t+1}_e = \beta^{t}_e - \gamma \nabla_{\beta^{t}_e} \mathcal{L}_{\textit{PG-IRM}}$
    \State $\alpha' := 1 - \mathbf{1}_{t > T_a} (1 - \alpha)$ 
    \State select $\beta^{t}_{\bar{e}}$ with $\bar{e}  = \underset{e' \in \mathcal{E}  \backslash e }{\text{argmax}} \|\tilde{\beta}^{t+1}_e - \beta^{t}_{e'}\|_2$
    \State $\beta^{t+1}_e = \alpha'
    \tilde{\beta}^{t+1}_e + (1 - \alpha')  \beta^t_{\bar{e}}$
\EndFor
\State Update $\phi^{t+1} = \phi^{t} - \gamma \nabla_{\phi^t} \mathcal{L}_{\textit{PG-IRM}}$.
\EndFor
\end{algorithmic}
\caption{\small PG-IRM}
\label{alg:proj_grad_appendix}
\end{algorithm}

% 



\algnewcommand{\ElseIIf}[1]{\algorithmicelse\ #1}
\begin{algorithm}[t]
\caption{ \texttt{FateZero} Algorithm}
\label{alg:real_image_editing}
\begin{algorithmic}

\State \textbf{Input:} 
\\ 
- $z_0$: \text{Latent code from source video}
\\
- $p_{src}$: \text{Source text prompt for input video}
\\
- $p_{edit}$: \text{Target text prompt for edition}
\\
\State {\bf Hyperparameters:} 
\\
 - ${t}_c$: Last timestep of the cross attention fusion
 \\
 - ${t}_s$: Last timestep of the self attention blending
 \\
 - $\tau$: Threshold for blending mask

\\
 \State {\bf Output:} 
 \\
 - $\hat{z}_0$: \text{Final edited latent code}

\\ \\
 $\triangleright$ DDIM for inversion latents and attention maps

% \xiaodong{check $p_{edit}$ or $p_{src}$}

\For{$t = 1,2,...,T$}
    \State $\epsilon_t, c_{t}^{\text{src}}, s_{t}^{\text{src}} \gets  \epsilon_\theta(z_t, t, p_{src})$
    % \State $z_{t+1}=\Call{Inversion\_step}{z_t, \hat{\epsilon}, t}$
    \State $z_{t} = \sqrt{\alpha_{t}} \; \frac{z_{t-1} - \sqrt{1-\alpha_{t-1}}\epsilon_t}{\sqrt{\alpha_{t-1}}}+ \sqrt{1-\alpha_{t}}\epsilon_t$
\EndFor

\\
\State $\triangleright$ Denoising the inverted latents with attention fusion  


\For{$t = T, (T-1),...,1$}
    % \State $\_\_, M_{t}^{\text{edit}} \gets \epsilon_\theta(z_t, t, p_{\text{edit}})$
    
    \State $\text{Edited\_index} = (p_{src} \text{ !=\ \ } \ p_{edit})$
    % \State $M_{\text{cross}} = (p_{src} \text{ !=\ \ } \ p_{edit})$
    \State $\triangleright$ Cross-attention mask is from the edited index~\cite{p2p}
    \State $M_{\text{cross}}[\text{Edited\_index}] = 1$
    \State $\triangleright$ Self-attention blending mask is from cross-attention.
    \State $M_{\text{self}} =  (c_{t}^{\text{src}}[\text{Edited\_index}] > \tau)$
    \State $\hat{\epsilon_t} \gets \Call{Att-Fusion}{\varepsilon_\theta, z_t, t, p_{\text{edit}}, M_{\text{edit}}, M_{\text{self}}, c_{t}^{\text{src}}, s_{t}^{\text{src}}}$
    % \State $z_{t-1}=\Call{Denoising\_step}{z_t, \hat{\epsilon}, t}$
    \State $z_{t-1} = \sqrt{\alpha_{t-1}} \; \frac{z_t - \sqrt{1-\alpha_t}\hat{\epsilon_t}}{\sqrt{\alpha_t}}+ \sqrt{1-\alpha_{t-1}}\hat{\epsilon_t}$
\EndFor



\State $\triangleright$ Fuse the inversion and editing attention of all $B$ blocks.
\State $\triangleright$ We only show the operation of attention and omit the feed-forward, residual convolution layer for simplicity.
% \Function{Att-Fusion}{$\theta, z_t, t, p_{\text{edit}}, c_{t}^{\text{src}}, s_{t}^{\text{src}}$}
\Function{Att-Fusion}{$\varepsilon_\theta, z_t, t, p_{\text{edit}}, M_{\text{cross}}, M_{\text{self}}, c_{t}^{\text{src}}, s_{t}^{\text{src}}$}
\For{$i = 1...B$}
    \State $s_{t}^{\text{edit}} = \text{Softmax}(W_i^Q(z_{t})W_i^K(z_{t})/\sqrt{d_i} )$
    \State $s_{t}^{\text{fused}} = \Call{Self-Blending}{s_{t}^{\text{edit}}, s_{t}^{\text{src}}, M_{\text{self}}, c_{t}^{\text{src}}, t}$
    \State $z_{t} \ \ \ \ = W_i^V(z_{t})\cdot s_{t}^{\text{fused}}$
    \State $c_{t}^{\text{edit}} = \text{Softmax}(W_i^Q(z_{t})W_i^K(p_{edit})/\sqrt{d_i} )$
    \State $c_{t}^{\text{fused}} = \Call{Cross-Fusion}{c_{t}^{\text{edit}}, c_{t}^{\text{src}}, M_{\text{edit}}, t}$
    \State $z_{t} \ \ \ \ = W_i^V(p_{\text{edit}})\cdot c_{t}^{\text{fused}}$
        
\EndFor
\State \Return $z_t$
\EndFunction

\end{algorithmic}
\end{algorithm}


\begin{algorithm}[t]
\caption{Attention Fusion and Blending Algorithm}
\label{alg:attention_fusion}
\begin{algorithmic}

\\
\State $\triangleright$ Cross-attention fusion using the difference mask between source and editing prompt following prompt-to-prompt.
\Function{Cross-Fusion}{$c_{t}^{\text{edit}}, c_{t}^{\text{src}}, M_{\text{edit}}, t$}
\If{$t > t_c $} 
\State \Return $ M_{\text{cross}} \cdot c_{t}^{\text{edit}}  + (1-M_{\text{cross}}) \cdot c_{t}^{\text{src}}$
% \\
\Else \State \Return $c_{t}^{\text{edit}}$
% \Else  1
% \ElseIf{$3$} $4$
\EndIf
\EndFunction
\\
\State $\triangleright$ Self-attention blending with cross attention.
\Function{Slef-Blending}{$s_{t}^{\text{edit}}, s_{t}^{\text{src}}, c_{t}^{\text{src}}, M_{\text{self}}, t$}
\If{$t > t_s $} 
\State \Return $ M_{\text{self}} \cdot s_{t}^{\text{edit}}  + (1-M_{\text{self}}) \cdot s_{t}^{\text{src}}$
% \\
\Else \State \Return $s_{t}^{\text{edit}}$
% \Else  1
% \ElseIf{$3$} $4$
\EndIf
\EndFunction




\end{algorithmic}
\end{algorithm}




% 
\begin{table*}[t]
    \begin{minipage}[c]{0.72\textwidth}
        
        \small\centering
        \scalebox{0.85}{
            \begin{tabular}{rrrrrrrrr} \toprule
            \multicolumn{1}{c}{\multirow{2}{*}{\textbf{Method ($\%$)}}} & \multicolumn{2}{c}{\textbf{OCI$\rightarrow$M}} & \multicolumn{2}{c}{\textbf{OMI$\rightarrow$C}} & \multicolumn{2}{c}{\textbf{OCM$\rightarrow$I}} & \multicolumn{2}{c}{\textbf{ICM$\rightarrow$O}} \\
            \multicolumn{1}{c}{} & \multicolumn{1}{c}{\textbf{HTER }$\downarrow$} & \multicolumn{1}{c}{\textbf{AUC} $\uparrow$} & \multicolumn{1}{c}{\textbf{HTER  $\downarrow$}} & \multicolumn{1}{c}{\textbf{AUC $\uparrow$}} & \multicolumn{1}{c}{\textbf{HTER}  $\downarrow$} & \multicolumn{1}{c}{\textbf{AUC} $\uparrow$} & \multicolumn{1}{c}{\textbf{HTER}  $\downarrow$} & \multicolumn{1}{c}{\textbf{AUC}  $\uparrow$}
             \\ \midrule
            MMD-AAE~\cite{li2018domain} & 27.08 & 83.19 & 44.59 & 58.29 & 31.58 & 75.18 & 40.98 & 63.08 \\
            MADDG~\cite{shao2019multi} & 17.69 & 88.06 & 24.50 & 84.51 & 22.19 & 84.99 & 27.98 & 80.02 \\
            SSDG-M~\cite{jia2020ssdg} & 16.67 & 90.47 & 23.11 & 85.45 & 18.21 & 94.61 & 25.17 & 81.83 \\
            DR-MD-Net~\cite{wang2020cross} & 17.02 & 90.10 & 19.68 & 87.43 & 20.87 & 86.72 & 25.02 & 81.47 \\
            RFMeta~\cite{shao2020regularized} & 13.89 & 93.98 & 20.27 & 88.16 & 17.30 & 90.48 & 16.45 & 91.16 \\
            NAS-FAS~\cite{yu2020fas} & 19.53 & 88.63 & 16.54 & 90.18 & 14.51 & 93.84 & 13.80 & 93.43 \\
            D2AM~\cite{chen2021generalizable} & 12.70 & 95.66 & 20.98 & 85.58 & 15.43 & 91.22 & 15.27 & 90.87 \\
            SDA~\cite{wang2021self} & 15.40 & 91.80 & 24.50 & 84.40 & 15.60 & 90.10 & 23.10 & 84.30 \\
            DRDG~\cite{liu2021dual} & 12.43 & 95.81 & 19.05 & 88.79 & 15.56 & 91.79 & 15.63 & 91.75 \\
            ANRL~\cite{liu2021adaptive} & 10.83 & 96.75 & 17.83 & 89.26 & 16.03 & 91.04 & 15.67 & 91.90 \\
            SSAN-M~\cite{wang2022ssan} & 10.42 & 94.76 & 16.47 & 90.81 & 14.00 & 94.58 & 19.51 & 88.17 \\
            SSDG-R~\cite{jia2020ssdg} & 7.38 & 97.17 & 10.44 & 95.94 & 11.71 & 96.59 & 15.61 & 91.54 \\
            SSAN-R~\cite{wang2022ssan} & 6.67 & \textbf{98.75} & 10.00 & \textbf{96.67} & 8.88 & 96.79 & 13.72 & 93.63 \\
            PatchNet~\cite{wang2022patchnet} & 7.10 & 98.46 & 11.33 & 94.58 & 13.40 & 95.67 & 11.82 & 95.07 \\
            % Ours (ImageNet) & 6.67 & 97 & 9.67 & 93.84 & 10.15 & 96.16 & 11.64 & 94.92 \\
            \methodname (Ours) & \textbf{5.95} & 96.55 & \textbf{8.78} & 95.37 & \textbf{6.58}  & \textbf{97.54} & \textbf{10.00} & \textbf{96.23}  \\ \bottomrule
            \end{tabular}
        }
    \end{minipage}
    \hfill
    \begin{minipage}[t]{0.28\textwidth}
        \vspace*{-40pt}
        \caption{
            \small 
            {\bf Comparisons with SoTA methods:} 
            Cross-domain face anti-spoofing is evaluated among four popular benchmark datasets: CASIA (\textbf{C}), Idiap Replay (\textbf{I}), MSU-MFSD (\textbf{M}), and Oulu-NPU (\textbf{O}). 
            Methods are compared at their best performance following the commonly used evaluation process \cite{jia2020ssdg}. 
            $\uparrow$ indicates larger values are better, and $\downarrow$ indicates smaller values are better.
        }
        \label{tab:best}
    \end{minipage}
    \vspace{-1ex}
\end{table*}


\begin{table*}[t]
\small \centering
\scalebox{0.81}{
\begin{tabular}{lllll} \toprule
\multicolumn{1}{c}{\multirow{2}{*}{\textbf{Method ($\%$)}}} & \multicolumn{1}{c}{\textbf{OCI$\rightarrow$M}} & \multicolumn{1}{c}{\textbf{OMI$\rightarrow$C}} & \multicolumn{1}{c}{\textbf{OCM$\rightarrow$I}} & \multicolumn{1}{c}{\textbf{ICM$\rightarrow$O}} \\
& \multicolumn{1}{c}{\textbf{HTER}$\downarrow$ /\textbf{AUC}$\uparrow$/\textbf{TPR95}$\uparrow$}  & \multicolumn{1}{c}{\textbf{HTER}$\downarrow$ /\textbf{AUC}$\uparrow$/\textbf{TPR95}$\uparrow$}  & \multicolumn{1}{c}{\textbf{HTER}$\downarrow$ /\textbf{AUC}$\uparrow$/\textbf{TPR95}$\uparrow$}  &
\multicolumn{1}{c}{\textbf{HTER}$\downarrow$ /\textbf{AUC}$\uparrow$/\textbf{TPR95}$\uparrow$}
 \\ \midrule
SSDG-R~\cite{jia2020ssdg} & 14.65 $ ^{{1.21}} $ / 91.93	$ ^{{1.35}} $  / 53.68  $ ^{{2.56}} $
& 28.76	$ ^{{0.89}} $ / 80.91	$ ^{{1.10}} $ / 41.47 $ ^{{2.68}} $ 
& 22.84	$ ^{{1.14}} $ / 78.67	$ ^{{1.31}} $  / 50.80  $ ^{{5.95}} $
& 15.83	$ ^{{1.29}} $ / 92.13	$ ^{{0.96}} $ / 66.54 $ ^{{4.00}} $ \\
SSAN-R~\cite{wang2022ssan} & 21.79 $ ^{{3.68}} $ /  84.06   $ ^{{3.78}} $  / 51.91  $ ^{{4.28}} $
&  26.44   $ ^{{2.91}} $ /  78.84   $ ^{{2.83}} $ / 45.36 $ ^{{4.29}} $ 
&  35.39   $ ^{{8.04}} $ /  70.13   $ ^{{9.03}} $  / 64.00  $ ^{{2.70}} $
&  25.72   $ ^{{3.74}} $ /  79.37   $ ^{{4.69}} $ / 36.75  $ ^{{5.19}} $ \\
PatchNet~\cite{wang2022patchnet} & 25.92   $ ^{{1.13}} $ /  83.43   $ ^{{0.87}} $ / 38.75 $ ^{{8.31}} $
&  36.26   $ ^{{1.98}} $ /  71.38   $ ^{{1.89}} $ / 19.22 $ ^{{3.85}} $ 
&  29.75   $ ^{{2.76}} $ /  80.53   $ ^{{1.35}} $  / 54.25  $ ^{{2.18}} $
&  23.49   $ ^{{1.80}} $ / 84.62   $ ^{{1.92}} $ / 39.39  $ ^{{6.83}} $ \\ \midrule 
SA-FAS (Ours) 
& \textbf{14.36}	$ ^{{1.10}} $ / \textbf{92.06}	$ ^{{0.53}} $ / \textbf{55.71}	$ ^{{4.82}} $ 
& \textbf{19.40}	$ ^{{0.66}} $ / \textbf{
88.69}	$ ^{{0.67}} $ / \textbf{50.53}	$ ^{{3.60}} $ 
& \textbf{11.48}	$ ^{{1.10}} $ / \textbf{
95.74}	$ ^{{0.55}} $ / \textbf{77.05}	$ ^{{3.26}} $ 
& \textbf{11.29}	$ ^{{0.32}} $ / \textbf{95.23}	$ ^{{0.24}} $ / \textbf{73.38} $ ^{{1.64}} $ \\
\bottomrule
\end{tabular}}
\figvspace
\caption{\small 
{\bf Evaluation upon convergence:}
Evaluation of cross-domain face anti-spoofing among CASIA (\textbf{C}), Idiap Replay (\textbf{I}), MSU-MFSD
(\textbf{M}), and Oulu-NPU (\textbf{O}) databases. Methods are compared at their mean/std performance based on the last 10 epochs. 
%
}
\label{tab:mean}
\vspace{-1.5ex}
\end{table*}







\SubSection{Training and inference}
\label{sec:train_step}

\Paragraph{Overall losses}
Considering the contrastive loss Eqn.~\eqref{eq:supcon}, the overall objective (dubbed as SA-FAS) can be written as: 

\begin{align}
    & \min _{\phi, \beta_{e^{(1)}}, ..., \beta_{e^{(E)}}} \mathcal{L}_{\textit{\textit{align}}} + \lambda \mathcal{L}_{\textit{sep}} \quad \rightarrow \mathcal{L}_{\textit{all}}  
    \label{eq:irm_constrain_overall} \\
    &\text { s.t. }  \forall e \in \mathcal{E}, \exists \beta_e \in \Omega_e(\phi), \beta_e \in \Upsilon_{\alpha}(\beta_e), \nonumber
    %\figvspace
\end{align}
where $\lambda$ is the coefficient for the loss term. The overall training pipeline is provided in Alg.~\ref{alg:main}.
% 

\Paragraph{Inference}
At the inference stage, we use the mean hyperplane from $\beta_{e^{(1)}}, ..., \beta_{e^{(E)}}$ to get the final score. 
% 
Specifically, the output is given by $$f(\*x) = \mathbb{E}_{e\in\mathcal{E}}[\beta_{e}^T\phi(\*x)].$$
Note that upon convergence, the cosine distance between any two of $\beta_{e^{(1)}}, ..., \beta_{e^{(E)}}$ is very close to 1, \ie, $\beta_{e^{(1)}} \!\approx\! ... \!\approx\! \beta_{e^{(E)}}$. 
This observation is verified in Appendix~\ref{sec:cosine_curve}, with an ablation (converged angles \vs different $\alpha$) in Appendix~\ref{sec:sensitivity}. 



% 