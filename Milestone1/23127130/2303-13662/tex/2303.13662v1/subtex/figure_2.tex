%\iffalse
\begin{figure}[t]
\small\centering
\includegraphics[width=0.45\textwidth]{figs/teaser.pdf}
\caption{\small
{\bf Cross-domain FAS:}
(a) Common FAS solutions aim to remove domain-specific signals and mix domains in one cluster. However, we empirically show domain-specific signals still exists in the feature space, and model might pick domain-specific signals as spurious correlation\protect\footnotemark~for classification.
(b) Our SA-FAS aims to retain domain signal. Specifically, we train a feature space with two critical properties: (1) \textbf{Separability}: Samples from different domains and
live/spoof classes are well-separated; (2) \textbf{Alignment}: Live-to-spoof transitions are aligned in the same direction for all domains. With these two properties, our method keeps the domain-specific signals invariant to the decision boundary.}
\label{fig:teaser}
\figvspace
\end{figure}

\footnotetext{In statistics, spurious correlation is a mathematical relationship in which multiple events or variables are associated but not causally related.}

\iffalse
\begin{figure*}[t]
\small\centering
\includegraphics[width=1\textwidth]{figs/teaser.pdf}
\caption{\small(a) Common FAS solutions aim to remove domain-specific signals and mix domains in one cluster. However, we empirically show domain-specific signals still exists in the feature space, and model might even pick some domain-specific signals as spurious correlation for decision boundary; (b) Our SA-FAS aims to retain domain signal. By separating data from different domains and aligning each domain's spoofing degradation, our method can keep the domain-specific signals invariant to the decision boundary.}
\label{fig:teaser}
\figvspace
\end{figure*}
\fi