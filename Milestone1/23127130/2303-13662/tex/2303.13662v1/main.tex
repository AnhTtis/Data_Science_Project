% CVPR 2022 Paper Template
% based on the CVPR template provided by Ming-Ming Cheng (https://github.com/MCG-NKU/CVPR_Template)
% modified and extended by Stefan Roth (stefan.roth@NOSPAMtu-darmstadt.de)

\documentclass[10pt,twocolumn,letterpaper]{article}

%%%%%%%%% PAPER TYPE  - PLEASE UPDATE FOR FINAL VERSION
%\usepackage[review]{cvpr}      % To produce the REVIEW version
%\usepackage{cvpr}              % To produce the CAMERA-READY version
\usepackage[pagenumbers]{cvpr} % To force page numbers, e.g. for an arXiv version

% Include other packages here, before hyperref.
\usepackage{graphicx}
\usepackage{amsmath}
\usepackage{amssymb}
\usepackage{amsthm}
\usepackage{cite}
\usepackage{algorithm}
\usepackage{algpseudocode}
\usepackage{booktabs}
\usepackage{amsfonts}       
\usepackage{multirow}
\usepackage{tabularx}

\makeatletter
\newcommand{\multiline}[1]{%
    \begin{tabularx}{\dimexpr\linewidth-\ALG@thistlm}[t]{@{}X@{}}
        #1
    \end{tabularx}
}

\newtheorem{theorem}{Theorem}
\newtheorem{lemma}[theorem]{Lemma}
\newtheorem*{theorem*}{Theorem}

\usepackage[dvipsnames]{xcolor}
\makeatletter
\@namedef{ver@everyshi.sty}{}
\makeatother
\usepackage{tikz}
\usetikzlibrary{backgrounds}
\usetikzlibrary{arrows,shapes}
\usetikzlibrary{tikzmark}
\usetikzlibrary{calc}

\usepackage{mathtools, nccmath}
\usepackage{wrapfig}
\usepackage{comment}

% To generate dummy text
\usepackage{blindtext}


%%%%%%%%%%%%%%%%%%%%%%%%%%%%%%%%% COLORFUL LATEX %%%%%%%%%%%%%%%%%%%%%%%%%%%%%%%%%
% for custom commands
\usepackage{xspace}

% table alignment
\usepackage{array}
\usepackage{ragged2e}
\newcolumntype{P}[1]{>{\RaggedRight\hspace{0pt}}p{#1}}
\newcolumntype{X}[1]{>{\RaggedRight\hspace*{0pt}}p{#1}}

% color box
\usepackage{tcolorbox}
\usepackage{paralist,tabularx}


% for tikz
\usepackage{tikz}
%\usetikzlibrary{trees}
\usetikzlibrary{arrows,shapes,positioning,shadows,trees,mindmap}
% \usepackage{forest}
\usepackage[edges]{forest}
\usetikzlibrary{arrows.meta}
\colorlet{linecol}{black!75}
\usepackage{xkcdcolors} % xkcd colors
% for colorful equation
\usepackage{tikz}
\usetikzlibrary{backgrounds}
\usetikzlibrary{arrows,shapes}
\usetikzlibrary{tikzmark}
\usetikzlibrary{calc}
% Commands for Highlighting text -- non tikz method
\newcommand{\highlight}[2]{\colorbox{#1!17}{$\displaystyle #2$}}
\newcommand{\highlightdark}[2]{\colorbox{#1!47}{$\displaystyle #2$}}
% my custom colors for shading
\colorlet{mhpurple}{Plum!80}
% Commands for Highlighting text -- non tikz method
\renewcommand{\highlight}[2]{\colorbox{#1!17}{#2}}
\renewcommand{\highlightdark}[2]{\colorbox{#1!47}{#2}}

% Some math definitions
\newcommand{\lap}{\mathrm{Lap}}
\newcommand{\pr}{\mathrm{Pr}}
\newcommand{\Tset}{\mathcal{T}}
\newcommand{\Dset}{\mathcal{D}}
\newcommand{\Rbound}{\widetilde{\mathcal{R}}}


\def\*#1{\mathbf{#1}}
\newcommand{\bx}{\mathbf{x}}
\newcommand{\bw}{\mathbf{w}}
\newcommand{\methodname}{SA-FAS }

\newcommand\yiyou[1]{\textcolor{magenta}{[Yiyou: #1]}}

% It is strongly recommended to use hyperref, especially for the review version.
% hyperref with option pagebackref eases the reviewers' job.
% Please disable hyperref *only* if you encounter grave issues, e.g. with the
% file validation for the camera-ready version.
%
% If you comment hyperref and then uncomment it, you should delete
% ReviewTempalte.aux before re-running LaTeX.
% (Or just hit 'q' on the first LaTeX run, let it finish, and you
%  should be clear).
\usepackage[pagebackref,breaklinks,colorlinks,citecolor=blue]{hyperref}


% Support for easy cross-referencing
\usepackage[capitalize]{cleveref}
\crefname{section}{Sec.}{Secs.}
\Crefname{section}{Section}{Sections}
\Crefname{table}{Table}{Tables}
\crefname{table}{Tab.}{Tabs.}


\newcommand{\yaojie}[1]{{\leavevmode\color[rgb]{1,0,1}[Yaojie: #1]}}
\newcommand{\vincent}[1]{{\leavevmode\color[rgb]{1,0,0}Vincent: #1}}

%%%%%%%%% PAPER ID  - PLEASE UPDATE
\def\cvprPaperID{3950} % *** Enter the CVPR Paper ID here
\def\confName{CVPR}
\def\confYear{2023}

\begin{document}

%%%%%%%%% TITLE - PLEASE UPDATE
\title{Rethinking Domain Generalization for Face Anti-spoofing: \\ Separability and Alignment}

\author{\\
Yiyou Sun$^{2}$\thanks{This work was done during Yiyou Sun’s internship at Google.}, Yaojie Liu$^1$, Xiaoming Liu$^{1,3}$, Yixuan Li$^2$, Wen-Sheng Chu$^1$\\
$^1$Google Research, $^2$University of Wisconsin-Madison, $^3$Michigan State University\\
{\tt\small $^1$\{yaojieliu,xiaomingl,wschu\}@google.com, $^2$\{sunyiyou,sharonli\}@cs.wisc.edu, $^3$liuxm@cse.msu.edu}
% For a paper whose authors are all at the same institution,
% omit the following lines up until the closing ``}''.
% Additional authors and addresses can be added with ``\and'',
% just like the second author.
% To save space, use either the email address or home page, not both
}
\maketitle

\def\eqnvspace{{\vspace{-2mm}}}
\def\tabvspace{{\vspace{-1mm}}}
\def\figvspace{{\vspace{-3mm}}}
\newcommand{\norm}[1]{\left\lVert#1\right\rVert}

\newcommand{\Paragraph}[1]{\vspace{1mm} \noindent \textbf{#1} \hspace{0mm}}
\newcommand{\Section}[1]{\vspace{-2mm} \section{#1} \vspace{-1mm}}
\newcommand{\SubSection}[1]{\vspace{-1mm} \subsection{#1} \vspace{-1mm}}
\newcommand{\SubSubSection}[1]{\vspace{-1mm} \subsubsection{#1} \vspace{-1mm}}

\newlength\segsep
\setlength{\segsep}{-0.5ex}


%%%%%%%%%%%%%%%%%%%%%%%%%%%%%%%%%%%% ABSTRACT  %%%%%%%%%%%%%%%%%%%%%%%%%%%%%%%%%%%%
\begin{abstract}
This work studies the generalization issue of face anti-spoofing (FAS) models on domain gaps, such as image resolution, blurriness and sensor variations. Most prior works regard domain-specific signals as a negative impact, and apply metric learning or adversarial losses to remove them from feature representation. 
Though learning a domain-invariant feature space is viable for the training data, we show that the feature shift still exists in an unseen test domain, which backfires on the generalizability of the classifier. 
In this work, instead of constructing a domain-invariant feature space, we encourage domain separability while aligning the live-to-spoof transition (i.e., the trajectory from live to spoof) to be the same for all domains. 
We formulate this FAS strategy of separability and alignment (SA-FAS) as a problem of invariant risk minimization (IRM), and learn domain-variant feature representation but domain-invariant classifier. We demonstrate the effectiveness of SA-FAS on challenging cross-domain FAS datasets and establish state-of-the-art performance. Code is available at \small{\url{https://github.com/sunyiyou/SAFAS}}.
\end{abstract}

%%%%%%%%% BODY TEXT

\Section{Introduction}
\label{sec:intro}
\vspace{\segsep}

% FAS
Face recognition (FR)~\cite{deng2019arcface} has achieved remarkable success and has been widely employed in mobile access control and electronic payments. Despite the promise, FR systems still suffer from presentation attacks (PAs), including print attacks, digital replay, and 3D masks. As a result, face anti-spoofing (FAS) has been an important topic for almost two decades~\cite{yang2014learn,liu2019deep,wang2022patchnet,atoum2017face,liu2018learning,yu2020face, kim2019basn}. 

% \begin{figure}[htb]
    \small\centering
    \includegraphics[width=0.4\textwidth]{figs/problem.pdf}
    \figvspace
    \caption{\small Training data examples for cross-domain face anti-spoofing.}
    \label{fig:prob}
\end{figure}
%\iffalse
\begin{figure}[t]
\small\centering
\includegraphics[width=0.45\textwidth]{figs/teaser.pdf}
\caption{\small
{\bf Cross-domain FAS:}
(a) Common FAS solutions aim to remove domain-specific signals and mix domains in one cluster. However, we empirically show domain-specific signals still exists in the feature space, and model might pick domain-specific signals as spurious correlation\protect\footnotemark~for classification.
(b) Our SA-FAS aims to retain domain signal. Specifically, we train a feature space with two critical properties: (1) \textbf{Separability}: Samples from different domains and
live/spoof classes are well-separated; (2) \textbf{Alignment}: Live-to-spoof transitions are aligned in the same direction for all domains. With these two properties, our method keeps the domain-specific signals invariant to the decision boundary.}
\label{fig:teaser}
\figvspace
\end{figure}

\footnotetext{In statistics, spurious correlation is a mathematical relationship in which multiple events or variables are associated but not causally related.}

\iffalse
\begin{figure*}[t]
\small\centering
\includegraphics[width=1\textwidth]{figs/teaser.pdf}
\caption{\small(a) Common FAS solutions aim to remove domain-specific signals and mix domains in one cluster. However, we empirically show domain-specific signals still exists in the feature space, and model might even pick some domain-specific signals as spurious correlation for decision boundary; (b) Our SA-FAS aims to retain domain signal. By separating data from different domains and aligning each domain's spoofing degradation, our method can keep the domain-specific signals invariant to the decision boundary.}
\label{fig:teaser}
\figvspace
\end{figure*}
\fi

% From intra-domain to cross-domain
In early systems like building access and border control with limited variations (\eg, lighting and poses), simple methods~\cite{boulkenafet2015face,freitas2012lbp,li2016original} have exhibited promise. These algorithms are designed for the closed-world setting, where the camera and environment are assumed to be the same between train and test. This assumption, however, rarely holds for in-the-wild applications, \eg, mobile face unlock and sensor-invariant ID verification.
Face images in those FAS cases may be acquired from wider angles, complex scenes, and different devices, where it is hard for training data to cover all the variations.
These differences between training and test data are termed domain gaps  and the FAS solutions to tackle the domain gaps are termed cross-domain FAS.


% Cross-domain FAS
Learning domain-invariant representation is the main approach in generic domain generalization~\cite{wang2022generalizing}, and has soon been widely applied to cross-domain FAS~\cite{wang2019improving,shao2019multi,jia2020ssdg,liu2021dual,liu2021adaptive,wang2022ssan}.
Those methods consider domain-specific signals as a confounding factor for model generalization, and hence aim to remove domain discrepancy from the feature representation partially or entirely. 
Adversarial training is commonly applied so that upon convergence the domain discriminator cannot distinguish which domain the features come from.
In addition, some methods apply metric learning to further regularize the feature space, \eg,~triplet loss~\cite{wang2019improving}, dual-force triplet loss~\cite{shao2019multi}, and single-side triplet loss~\cite{jia2020ssdg}.

% Cross-domain FAS Challenges
% 
There are two crucial issues that limit the generalization ability of these methods~\cite{wang2019improving,shao2019multi,jia2020ssdg,liu2021dual,liu2021adaptive,wang2022ssan} with domain-invariant feature losses. First, 
%
these methods posit a strong assumption that the feature space is perfectly domain-invariant after removing the domain-specific signals from training data. 
%
However, this assumption is unrealistic due to the limited size and domain variants of the training data, on which the loss might easily overfit during training. As shown in Fig.~\ref{fig:cmp_da_umap}, the test distribution is more expanded compared to the training one, and the spatial relation between live and spoof has largely deviated from the learned classifier.
% 
Second, feature space becomes ambiguous when domains are mixed together. Note that the domain can carry information on certain image resolutions, blurriness and sensor patterns.
If features from different domains are collapsed together~\cite{papyan2020prevalence}, the live/spoof classifier will undesirably leverage spurious correlations to make the live/spoof predictions as shown in Fig.~\ref{fig:teaser} (a), \eg, comparing live from low-resolution domains to spoof from high-resolution ones. Such a classifier will unlikely generalize to a test domain when the correlation does not exist.

%
 
% Our solution
In this work, we rethink feature learning for cross-domain FAS. Instead of constructing a domain-invariant feature space, we aim to find a generalized classifier while explicitly maintaining domain-specific signals in the representation. Our strategy can be summarized by the following two properties:
\begin{itemize}
    \tabvspace\item \textbf{Separability:} We encourage features from different domains and live/spoof classes to be separated which facilitates maintaining the domain signal.
    %which facilitates the learning of a domain-invariant decision boundary as shown in Fig.~\ref{fig:teaser} (right bottom). 
    According to~\cite{ben2006analysis}, representations with well-disentangled domain variation and task-relevant features are more general and transferable to different domains. 
    \tabvspace\item \textbf{Alignment:} Inspired by~\cite{jourabloo2018face}, we regard spoofing as the process of transition. For similar PA types\footnote{This work focuses on print and replay attacks.}, the transition process would be similar, regardless of environments and sensor variations. With this assumption, we regularize the live-to-spoof transition to be aligned in the same direction for all domains. 
    % 
\end{itemize}
We refer to this new learning framework as \textit{FAS with separability and alignment} (dubbed \textbf{SA-FAS}), shown in Fig.~\ref{fig:teaser} (b).
To tackle the separability, we leverage Supervised Contrastive Learning (SupCon)~\cite{2020supcon} to learn representations that force samples from the same domain and the same live/spoof labels to form a compact cluster. 
To achieve the alignment, we devise a novel Projected Gradient optimization strategy based on Invariant Risk Minimization (PG-IRM) to regularize the %live/spoof decision factors 
live-to-spoof transition invariant to the domain variance. 
%
With normalization, the feature space is naturally divided into two symmetric half-spaces: one for live and one for spoof (see Fig.~\ref{fig:umap}).
%
Domain variations will manifest inside the half-spaces but %be parallel 
have minimal impact to the live/spoof classifier. 
%

We summarize our contributions as three-fold:
\begin{compactitem}
    \item 
    We offer a new perspective for cross-domain FAS.
    Instead of removing the domain signal, we propose to maintain it and design the feature space based on separability and alignment;
    
    \item We first systematically exploit the domain-variant representation learning by combining contrastive learning and effectively optimizing invariant risk minimization (IRM) through the projected gradient algorithm for cross-domain FAS;
    
    \item We achieve state-of-the-art performance on widely-used cross-domain FAS benchmark, and provide in-depth analysis and insights on how separability and alignment lead to the performance boost.
\end{compactitem}

\Section{Related Work}
\label{sec:related}
\vspace{\segsep}

\Paragraph{Face Anti-Spoofing} 
Face anti-spoofing attracts growing attention in several thriving directions.
%
Early works exploit spontaneous human behaviors (\eg, eye blinking, head motion)~\cite{kollreider2007real, pan2007eyeblink} or predefined movements (\eg, head-turning, expression changes)~\cite{chetty2010biometric}. %These behavior-based strategy requires user interaction and are vulnerable to the video replaying attacks. 
%(2) Another line of approaches evolve into modeling material properties (\ie, texture). 
Later, hand-crafted features are utilized to describe spoof patterns, \eg, LBP~\cite{boulkenafet2015face, freitas2012lbp}, HoG ~\cite{freitas2012lbp,yang2013face}
and SIFT~\cite{patel2016secure} features. 
%, and train a live/spoof classifier using support vector machines or linear discriminant analysis. 
Recently, deep neural networks have been applied to face anti-spoofing.
There are classification-based methods~\cite{yang2014learn,liu2019deep,wang2022patchnet}, regression-based methods~\cite{atoum2017face,liu2018learning,yu2020face, kim2019basn}, and generative models~\cite{jourabloo2018face,liu2020disentangling,wang2022ssan,liu2022spoof}. 
In addition, the vision transformer also shows promising performance in tackling FAS~\cite{george2021effectiveness,huang2022adaptive}.
%and achieved state-of-the-art performance than conventional methods~\cite{yang2014learn,feng2016integration,li2016original,patel2016cross}.

% domain generalization
\Paragraph{Cross-domain FAS} 
 Recently, several works explore learning FAS models from multiple domains that generalize to unseen ones. 
Some methods~\cite{zhou2022generative, li2018unsupervised, wang2020unsupervised, guo2022multi,unified-detection-of-digital-and-physical-face-attacks} require data from the target domain to adapt the model (\ie, domain adaptation), while others~\cite{shao2019multi,kim2021suppressing,saha2020domain,jia2020ssdg,wang2022ssan,noise-modeling-synthesis-and-classification-for-generic-object-anti-spoofing} learn shared features based on adversarial training and triplet loss (\ie, domain generalization).
%
A few methods~\cite{shao2020regularized,chen2021generalizable,wang2021self} explore meta-learning to simulate the domain shift at training time. 
%
Most previous works regard the domain-specific signals as a negative impact. Contrastively, our paper first systematically exploits the explicit usage of domain-specific signals by invariant risk minimization in cross-domain FAS.

\Paragraph{Domain-invariant Classifier}
Learning a domain-invariant classifier has always been the focus of machine learning for decades~\cite{van2018cpc,chen2020simclr,caron2020swav,he2019moco} and is also one of the keys to the success of domain generalization. Along this line, kernel-based methods~\cite{blanchard2021domain,muandet2013domain,grubinger2015domain,gan2016learning,li2018domain,ghifary2016scatter} propose to learn a domain-invariant kernel from the training data. Domain adversarial learning~\cite{li2018domain,ganin2015unsupervised,ganin2016domain,gong2019dlow,li2018deep,shao2019multi,mahfujur2019correlation,wang2022ssan,jia2020ssdg} adversarially trains the generator and discriminator while the generator is trained to fool the discriminator to learn domain invariant feature representations. Recently, Invariant Risk Minimization (IRM) and its variants~\cite{arjovsky2019irm,ahuja2021ibirm,krueger2021rex,mahfujur2019correlation,mitrovic2020representation,choe2020empirical,sonar2021invariant} seek to directly enforce the optimal classifier on top of the representation space to be the same across all domains. However, IRM is known to be hard to optimize and can fail in non-linear optimization~\cite{kamath2021doesirm,rosenfeld2020riskirm}. In this paper, we propose an equivalent objective (PG-IRM) which is easier to optimize and achieve strong performance. % as shown in Sec.~\ref{sec:exp}.


%\paragraph{Invariant Risk Minimization (IRM)} 
%%%%%%%%%%%%%%%%%%%%%%%%%%%%%%%%%%%% SETUP  %%%%%%%%%%%%%%%%%%%%%%%%%%%%%%%%%%%%
\Section{Method}
\label{sec:method}
\vspace{\segsep}

We formally introduce the new learning framework, \textit{FAS with separability and alignment} (dubbed \textbf{SA-FAS}). The goal is to produce a feature space with two critical properties: 
(1) \textit{Separability}: We encourage samples from different domains and from different classes to be well-separated; 
(2) \textit{Alignment}: Live-to-spoof transition\footnote{The transition can be considered as a path in the high-D manifold. }  is aligned in the same direction for all domains.  
These two properties work jointly: 
\textit{separability} ensures the awareness of domain variance in the feature space; \textit{alignment} encourages the domain variance to be invariant to its live-vs-spoof hyperplane. 
%

This section is structured as follows:
Sec.~\ref{sec:setup} describes the problem setup, followed by the algorithm design of separability (Sec.~\ref{sec:sep}) and alignment (Sec.~\ref{sec:align}).
Finally, Sec.~\ref{sec:train_step} summarizes the training and inference processes.

\begin{figure*}[t]
    \small\centering
    \includegraphics[width=0.95\textwidth]{figs/method.pdf}
    \caption{\small 
        {\bf Optimization objectives:}
        Illustration of feature space optimized by different objectives: (a) ERM, (b) ERM+SupCon, (c) SA-FAS (ours). 
        Circle/cross denotes live/spoof label; different colors indicate different domains. 
        A UMAP visualization for real data is provided in Appendix (Fig.~\ref{fig:method-sup}) to support the feature distribution shown in the toy example.
        %
    }
    \label{fig:method}
\end{figure*}

%\protect\footnotemark
%\footnotetext{}

%

\SubSection{Problem Setup}
\label{sec:setup}
We start by defining the setting of the cross-domain FAS problem. We denote by $\mathcal{X}=\mathbb{R}^d$  the input space and 
$\mathcal{Y}=\{0 \text{ (live)}, 1 \text{ (spoof)}\}$ 
the output space. A learner is given access to a set of training data from $E$ domains $\mathcal{E} = \{e^{(1)}, e^{(2)}, .., e^{(E)}\}$ and is evaluated on test domain $e^*$. Let $e_i$ as the domain label for the $i$-th sample, we denote $\mathcal{D}=\{(\*x_i,y_i, e_i)\}_{i=1}^N$ drawn from an unknown joint data distribution $\mathcal{P}$ defined on $\mathcal{X} \times \mathcal{Y} \times \mathcal{E}$. 
Cross-domain FAS is a special binary classification problem to distinguish live and spoof faces from an unseen domain. The goal is to define a decision function:
\begin{align}
    f: \mathbf{x} \rightarrow \{0 \text{ (live)}, 1 \text{ (spoof)}\}, \nonumber
\end{align}
which classifies whether a sample $\*x$ from a new domain $e^*$ is live or spoof. 

In our network architecture, function $f$ consists of two components: (1) a deep neural network encoder $\phi:\mathcal{X} \rightarrow \mathbb{R}^m$ that maps the input $\bx$ to a $l_2$-normalized feature embedding $\*z = \phi(\bx)$; 
%
(2) a classifier (via a weight vector) $\beta:\mathbb{R}^m \rightarrow \mathbb{R}$ that maps the $m$-dimensional embedding $\*z$ to a scalar value, where a binary cross-entropy loss can be applied after using a sigmoid function. Because the true distribution of live/spoof data is unknown, the optimization commonly relies on an Empirical Risk Minimization (ERM).

\noindent \textbf{Remark on the terminology}: $\beta$ can be considered as a norm vector of the hyperplane separating live and spoof samples. In the remaining part of the paper, when we use ``\textbf{live-vs-spoof hyperplane}'' or ``\textbf{hyperplane}'', it has the same meaning as $\beta$. Note, ``live-to-spoof transition'' is an abstract procedure in the image space, while ``live-vs-spoof hyperplane'' refers to a concrete classifier in the feature space. 

\Paragraph{Preliminary on Empirical Risk Minimization (ERM):}
ERM principle~\cite{vapnik1991principles} is a ubiquitous strategy that merges data from all training domains and learns a predictor that minimizes an averaged training error.
Specifically,
\begin{align}
    & \mathcal{L}_{\textit{ERM}} = \min _{\phi, \beta} \frac{1}{|\mathcal{E}|} \sum_{e \in \mathcal{E}} \mathcal{R}^{e}(\phi, \beta), 
\label{eq:erm}
\end{align}
where the empirical risk function $\mathcal{R}^{e}(\phi, \beta)$ for a given environment $e$ is defined by:
$$
\mathcal{R}^{e}(\phi, \beta) \triangleq \mathbb{E}_{(\*x_i, y_i, e_i=e) \sim \mathcal{D}} \ell\left(f(\*x_i;\phi, \beta), y_i\right).
$$
Common choices of the loss function $\ell(\cdot,\cdot)$ include cross-entropy loss~\cite{jia2020ssdg} and $L_1$ regression loss~\cite{liu2018learning, george2019deep}.

However, if samples from different domains are mixed together, 
ERM can utilize the easiest difference (image resolution, blurriness, camera setting) to differentiate live \vs.~spoof. 
Such a classifier will undesirably leverage spurious correlations to make  live/spoof predictions~\cite{arjovsky2019irm}. Therefore, the naive strategy can hurt the generalization of the unseen domain. % 
As shown in Fig.~\ref{fig:method}(a), ERM tends to fit all training data together and fails to learn a domain-invariant classifier with the mixed feature space. 
%

%%%%%%%%%%%%%%%%%%%%%%%%%%%%%%%%%%%% METHOD  %%%%%%%%%%%%%%%%%%%%%%%%%%%%%%%%%%%%
%
\SubSection{Separability}
\label{sec:sep}

%
We characterize the domain separability as supervised contrastive learning (dubbed SupCon)~\cite{2020supcon}, one of the latest developments for visual representation learning. Unlike other contrastive learning methods~\cite{chen2020simclr, chen2021exploring} that treat the augmented samples as a single class, SupCon aims to learn a representation space that gathers samples with the same labels while repelling samples from different ones. It naturally suits the need for the cross-domain FAS setting, since we treat samples with the same domain and with the same live/spoof label to form a cluster. 
%


%\textbf{Training objective} 
Given a training mini-batch $\{\*x_i, y_i, e_i\}_{i=1}^b$, we augment~\cite{2020supcon} the mini-batch as $\{\tilde{\*x}_i, \tilde{y}_i, \tilde{e}_i\}_{i=1}^{2b}$, using two random augmentations $\tilde{\bx}_{2i}$ and $\tilde{\*x}_{2i-1}$ of inputs $\bx_i$, with $\tilde{y}_{2i-1} \!=\! \tilde{y}_{2i} \!=\! y_i$, $\tilde{e}_{2i-1} \!=\! \tilde{e}_{2i} \!=\! e_i$. These images are fed into the network, yielding $L_2$-normalized embeddings $\{\*z_i\}_{i=1}^{2b}$. The per-batch SupCon loss (separability loss) is defined as:
\begin{equation}
\mathcal{L}_\textit{sep}=\sum_{i=1}^{2b} \frac{-1}{|S(i)|} \sum_{j \in S(i)} \log \frac{\exp \left(\*z_{i} \cdot \*z_{j} / \tau\right)}{\sum_{t=1, t\neq i}^{2b}\exp \left(\*z_{i} \cdot \*z_{t} / \tau\right)},
\label{eq:supcon}
\end{equation}
where $\tau$ is a temperature parameter, $i$ is the index of a sample typically called the \textit{anchor}, $S(i) \!=\! \{j \!\in\!\{1,\ldots,2b\} \!:\! j\neq i, \tilde{y}_j = \tilde{y}_i, \tilde{e}_j = \tilde{e}_i\}$ is the index set of \emph{positive samples} 
%consisting of the indices of the augmented sample and other 
that have the same live/spoof labels and belong to the same domain as the anchor $i$, and $|S(i)|$ is its cardinality. 
All the other samples in the mini-batch are referred to as \textit{negative samples}. 
%
Since positive samples are pulled together and negative samples are pushed apart, SupCon in Fig.~\ref{fig:method}(b) is capable of providing more distinguishable feature clusters for different domains and liveness classes, compared to a typical feature space learned by a vanilla ERM in Fig.~\ref{fig:method}(a). 
%
\SubSection{Alignment}
\label{sec:align}

% 
Fig.~\ref{fig:method}(b) also shows that separability alone is not sufficient for improving domain generalization. 
The separated feature clusters can be located in any place in the feature space, and hence the domain-wise optimal hyperplane remains \textbf{variant}.
In this case, the global classifier can still undesirably incorporate the spurious correlation as the deciding factor as we show in Fig.~\ref{fig:teaser}.
%  
To tackle this, we naturally investigate the following problem: 

\textit{How do we regularize a global live-vs-spoof hyperplane to align with domain-wise live-vs-spoof hyperplanes?}

We propose to formulate this problem as Invariant Risk Minimization (IRM)~\cite{arjovsky2019irm}, which aims to jointly optimize the feature space $\phi$ and the global live-vs-spoof hyperplane $\beta$, where $\beta$ is also optimal for each domain, shown in Fig.~\ref{fig:method}(c). 

%   

\begin{figure*}[t]
    \small\centering
    \includegraphics[width=\textwidth]{figs/illustration.pdf}
    \vspace{-4ex}
    \caption{\small 
    {\bf PG-IRM optimization process:}
    An illustration of why a vanilla IRM can suffer from an infeasible solution (a), and how the proposed PG-IRM algorithm jointly updates the feature space and multiple hyperplanes towards convergence (b)-(d). }
    \label{fig:illustration}
\end{figure*}

% 
\Paragraph{Preliminary on Invariant Risk Minimization (IRM):}
Specifically, the IRM objective can be formulated as the following constrained optimization problem:

\begin{align}
    & \min _{\phi, \beta^{*}} \frac{1}{|\mathcal{E}|} \sum_{e \in \mathcal{E}} \mathcal{R}^{e}(\phi, \beta^{*}) \rightarrow \mathcal{L}_{\textit{IRM}}  \label{eq:irm_target} \\ 
    & \textit{s.t.} \quad \beta^{*} \in \underset{\beta}{\arg\min} \mathcal{R}^{e}(\phi, \beta), \forall e \in \mathcal{E}.
\label{eq:irm_constrain}
\end{align}

%

Compared to the ERM~\eqref{eq:erm}, IRM enforces an additional constraint \eqref{eq:irm_constrain} to learn the domain-invariant hyperplanes. 
Specifically, if we define the domain-wise optimal hyperplane as 
%
$\beta_e \in {\arg\min  }_{\beta} \mathcal{R}^{e}(\phi, \beta)$.
A sufficient condition for constraint \eqref{eq:irm_constrain} to hold is $\beta_{e^{(1)}}=...=\beta_{e^{(E)}} = \beta^*$, which requires consistency between the globally optimal hyperplane and the domain-wise optimal hyperplanes.
However, IRM is known to be hard to solve~\cite{kamath2021doesirm,rosenfeld2020riskirm} due to the bi-level optimization nature of objective~(\ref{eq:irm_target}) and constraint~(\ref{eq:irm_constrain}). 

%

%\paragraph{PG-IRM objective} 
\Paragraph{Projected Gradient Optimization for IRM (PG-IRM):} 
%
We leverage Projected Gradient (PG) algorithm~\cite{nocedal1999numerical} to solve the non-trivial optimization objective~\eqref{eq:irm_target}, termed as PG-IRM. 
In PG-IRM, we propose to optimize multiple hyperplanes and converge them into a globally one via projected gradient. In Appendix~\ref{sec:proof_s1}, we provide detailed proof of PG-IRM objective being \textbf{equivalent} to IRM. Formally, the objective is rewritten as:
\begin{theorem}
\label{th:ourirm_objective}
\textbf{(PG-IRM objective)} For all $\alpha \!\in\! (0,1) $, the IRM objective is equivalent to the following objective: 

\begin{align}
    & \min _{\phi, \beta_{e^{(1)}}, ..., \beta_{e^{(E)}}} \frac{1}{|\mathcal{E}|} \sum_{e \in \mathcal{E}} \mathcal{R}^{e}(\phi, \beta_e) \rightarrow \mathcal{L}_{\textit{align}} \\ 
    &\text { s.t. }  \forall e \in \mathcal{E}, \exists \beta_e \in \Omega_e(\phi), \beta_e \in \Upsilon_{\alpha}(\beta_e), \nonumber 
    % 
\end{align}
where the parametric constrained set for each environment is simplified as 
$ \Omega_e(\phi) = \underset{\beta}{\arg \min } \mathcal{R}^{e}(\phi, \beta),$
and we define the \textbf{$\alpha$-adjacency set}: 
\begin{align}
    \Upsilon_{\alpha}(\beta_e) = \{\upsilon |& \underset{e' \in \mathcal{E} \backslash e}{\max}  \ \underset{\beta_{e'} \in \Omega_{e'}(\phi)}{\min}\|\upsilon - \beta_{e'}\|_2 \\
    & \le \alpha \underset{e' \in \mathcal{E} \backslash e}{\max}  \ \underset{\beta_{e'} \in \Omega_{e'}(\phi)}{\min}\|\beta_e - \beta_{e'}\|_2\}
    \label{eq:gamma_set}
\end{align}
\figvspace
\end{theorem}
Fig.~\ref{fig:illustration} shows the intuition of  the optimization process. 
%
For a $3$-domain case, PG-IRM starts with a shared feature space $\phi$ and $3$ separate hyperplanes $\beta_{e^{(1)}}$, $\beta_{e^{(2)}}$, $\beta_{e^{(3)}}$ for each domain (Fig.~\ref{fig:illustration}(b)). 
After each projected gradient descent, the hyperplanes move closer with feature space jointly updated (Fig.~\ref{fig:illustration}(c)). 
Upon convergence, $\beta_{e^{(1)}}, \beta_{e^{(2)}}, \beta_{e^{(3)}}$ become nearly identical (Fig.~\ref{fig:illustration}(d)), satisfying the IRM constraint $\beta^* \!=\! \beta_{e^{(1)}} \!=\! \beta_{e^{(2)}} \!=\! \beta_{e^{(3)}}$ for {\it all} domains.
%The above explanation delivers the following 
We provide two main insights of our PG-IRM  algorithm (see more details in Appendix~\ref{sec:proof}):
\begin{compactenum}
    \item \textbf{Optimizing multiple  hyperplanes:} 
    Compared to the conventional IRM that optimizes a single hyperplane, it is easier to converge for PG-IRM that optimizes multiple hyperplanes (\ie, one for each domain).
    Shown in Fig.~\ref{fig:illustration}(a-b), for the same feature space from the intermediate optimization stage, the solution $\beta^*$ to conventional IRM may not exist and the optimization has to be terminated.
    In contrast, $\beta_{e^{(1)}}, ..., \beta_{e^{(E)}}$ \textbf{always} exists (Fig.~\ref{fig:illustration}(b)) which makes solving for multiple hyperplanes more viable.
    %
    \item \textbf{Pushing hyperplanes to be closer:}
    To align $\beta_{e^{(1)}}$, $\beta_{e^{(2)}}$ and $\beta_{e^{(3)}}$,
    %
    PG-IRM updates domain-wise hyperplanes by interpolating with other hyperplanes. It can be mathematically considered as projecting the parameters of a hyperplane into the $\alpha$-adjacency set $\Upsilon_{\alpha}(\beta_e)$ as we illustrated in Fig.~\ref{fig:proj}.

    \begin{figure}[t!]
    \small\centering
    \includegraphics[width=.4\textwidth]{figs/projection.pdf}
    \caption{\small 
        {\bf Euclidean projection:}
        Illustration of Euclidean projection (solid black dot) to the $\alpha$-adjacency set $\Upsilon_\alpha\left(\beta_e\right)$. 
        Detailed proof and steps are provided in Alg.~\ref{alg:proj_grad_appendix} in Appendix~\ref{sec:proof_s2}. 
    }
    \figvspace
    \label{fig:proj}
\end{figure}

    \textbf{Remark (why PG is not applicable to IRM):}
    The PG algorithm can be infeasible for the conventional IRM, as the solution set for \eqref{eq:irm_constrain} can be \textbf{empty} and is thus non-projectable.
    Our PG-IRM objective in Eq.~\eqref{eq:gamma_set} contains a non-empty $\alpha$-adjacency set $\Upsilon_{\alpha}(\beta_e)$, and guarantees being projectable by simple linear interpolation. 

    %
\end{compactenum}


%\Paragraph{Intuition.}  For the reader's convenience, 


% \begin{algorithm}[t]
\begin{algorithmic}\State Initialize $\phi, \beta_{e^{(1)}}, ... , \beta_{e^{(E)}}$, learning rate $\gamma$, alignment parameter $\alpha$, alignment starting epoch $T_a$.
\For{$\text{t in 0, 1, ..., }$}
\State Run forward pass and calculate the gradient.
\For{$e \in \mathcal{E}$}
    
    \State $\tilde{\beta}^{t+1}_e = \beta^{t}_e - \gamma \nabla_{\beta^{t}_e} \mathcal{L}_{\textit{PG-IRM}}$
    \State $\alpha' := 1 - \mathbf{1}_{t > T_a} (1 - \alpha)$ 
    \State select $\beta^{t}_{\bar{e}}$ with $\bar{e}  = \underset{e' \in \mathcal{E}  \backslash e }{\text{argmax}} \|\tilde{\beta}^{t+1}_e - \beta^{t}_{e'}\|_2$
    \State $\beta^{t+1}_e = \alpha'
    \tilde{\beta}^{t+1}_e + (1 - \alpha')  \beta^t_{\bar{e}}$
\EndFor
\State Update $\phi^{t+1} = \phi^{t} - \gamma \nabla_{\phi^t} \mathcal{L}_{\textit{PG-IRM}}$.
\EndFor
\end{algorithmic}
\caption{\small PG-IRM}
\label{alg:proj_grad_appendix}
\end{algorithm}

% 



\algnewcommand{\ElseIIf}[1]{\algorithmicelse\ #1}
\begin{algorithm}[t]
\caption{ \texttt{FateZero} Algorithm}
\label{alg:real_image_editing}
\begin{algorithmic}

\State \textbf{Input:} 
\\ 
- $z_0$: \text{Latent code from source video}
\\
- $p_{src}$: \text{Source text prompt for input video}
\\
- $p_{edit}$: \text{Target text prompt for edition}
\\
\State {\bf Hyperparameters:} 
\\
 - ${t}_c$: Last timestep of the cross attention fusion
 \\
 - ${t}_s$: Last timestep of the self attention blending
 \\
 - $\tau$: Threshold for blending mask

\\
 \State {\bf Output:} 
 \\
 - $\hat{z}_0$: \text{Final edited latent code}

\\ \\
 $\triangleright$ DDIM for inversion latents and attention maps

% \xiaodong{check $p_{edit}$ or $p_{src}$}

\For{$t = 1,2,...,T$}
    \State $\epsilon_t, c_{t}^{\text{src}}, s_{t}^{\text{src}} \gets  \epsilon_\theta(z_t, t, p_{src})$
    % \State $z_{t+1}=\Call{Inversion\_step}{z_t, \hat{\epsilon}, t}$
    \State $z_{t} = \sqrt{\alpha_{t}} \; \frac{z_{t-1} - \sqrt{1-\alpha_{t-1}}\epsilon_t}{\sqrt{\alpha_{t-1}}}+ \sqrt{1-\alpha_{t}}\epsilon_t$
\EndFor

\\
\State $\triangleright$ Denoising the inverted latents with attention fusion  


\For{$t = T, (T-1),...,1$}
    % \State $\_\_, M_{t}^{\text{edit}} \gets \epsilon_\theta(z_t, t, p_{\text{edit}})$
    
    \State $\text{Edited\_index} = (p_{src} \text{ !=\ \ } \ p_{edit})$
    % \State $M_{\text{cross}} = (p_{src} \text{ !=\ \ } \ p_{edit})$
    \State $\triangleright$ Cross-attention mask is from the edited index~\cite{p2p}
    \State $M_{\text{cross}}[\text{Edited\_index}] = 1$
    \State $\triangleright$ Self-attention blending mask is from cross-attention.
    \State $M_{\text{self}} =  (c_{t}^{\text{src}}[\text{Edited\_index}] > \tau)$
    \State $\hat{\epsilon_t} \gets \Call{Att-Fusion}{\varepsilon_\theta, z_t, t, p_{\text{edit}}, M_{\text{edit}}, M_{\text{self}}, c_{t}^{\text{src}}, s_{t}^{\text{src}}}$
    % \State $z_{t-1}=\Call{Denoising\_step}{z_t, \hat{\epsilon}, t}$
    \State $z_{t-1} = \sqrt{\alpha_{t-1}} \; \frac{z_t - \sqrt{1-\alpha_t}\hat{\epsilon_t}}{\sqrt{\alpha_t}}+ \sqrt{1-\alpha_{t-1}}\hat{\epsilon_t}$
\EndFor



\State $\triangleright$ Fuse the inversion and editing attention of all $B$ blocks.
\State $\triangleright$ We only show the operation of attention and omit the feed-forward, residual convolution layer for simplicity.
% \Function{Att-Fusion}{$\theta, z_t, t, p_{\text{edit}}, c_{t}^{\text{src}}, s_{t}^{\text{src}}$}
\Function{Att-Fusion}{$\varepsilon_\theta, z_t, t, p_{\text{edit}}, M_{\text{cross}}, M_{\text{self}}, c_{t}^{\text{src}}, s_{t}^{\text{src}}$}
\For{$i = 1...B$}
    \State $s_{t}^{\text{edit}} = \text{Softmax}(W_i^Q(z_{t})W_i^K(z_{t})/\sqrt{d_i} )$
    \State $s_{t}^{\text{fused}} = \Call{Self-Blending}{s_{t}^{\text{edit}}, s_{t}^{\text{src}}, M_{\text{self}}, c_{t}^{\text{src}}, t}$
    \State $z_{t} \ \ \ \ = W_i^V(z_{t})\cdot s_{t}^{\text{fused}}$
    \State $c_{t}^{\text{edit}} = \text{Softmax}(W_i^Q(z_{t})W_i^K(p_{edit})/\sqrt{d_i} )$
    \State $c_{t}^{\text{fused}} = \Call{Cross-Fusion}{c_{t}^{\text{edit}}, c_{t}^{\text{src}}, M_{\text{edit}}, t}$
    \State $z_{t} \ \ \ \ = W_i^V(p_{\text{edit}})\cdot c_{t}^{\text{fused}}$
        
\EndFor
\State \Return $z_t$
\EndFunction

\end{algorithmic}
\end{algorithm}


\begin{algorithm}[t]
\caption{Attention Fusion and Blending Algorithm}
\label{alg:attention_fusion}
\begin{algorithmic}

\\
\State $\triangleright$ Cross-attention fusion using the difference mask between source and editing prompt following prompt-to-prompt.
\Function{Cross-Fusion}{$c_{t}^{\text{edit}}, c_{t}^{\text{src}}, M_{\text{edit}}, t$}
\If{$t > t_c $} 
\State \Return $ M_{\text{cross}} \cdot c_{t}^{\text{edit}}  + (1-M_{\text{cross}}) \cdot c_{t}^{\text{src}}$
% \\
\Else \State \Return $c_{t}^{\text{edit}}$
% \Else  1
% \ElseIf{$3$} $4$
\EndIf
\EndFunction
\\
\State $\triangleright$ Self-attention blending with cross attention.
\Function{Slef-Blending}{$s_{t}^{\text{edit}}, s_{t}^{\text{src}}, c_{t}^{\text{src}}, M_{\text{self}}, t$}
\If{$t > t_s $} 
\State \Return $ M_{\text{self}} \cdot s_{t}^{\text{edit}}  + (1-M_{\text{self}}) \cdot s_{t}^{\text{src}}$
% \\
\Else \State \Return $s_{t}^{\text{edit}}$
% \Else  1
% \ElseIf{$3$} $4$
\EndIf
\EndFunction




\end{algorithmic}
\end{algorithm}




% 
\begin{table*}[t]
    \begin{minipage}[c]{0.72\textwidth}
        
        \small\centering
        \scalebox{0.85}{
            \begin{tabular}{rrrrrrrrr} \toprule
            \multicolumn{1}{c}{\multirow{2}{*}{\textbf{Method ($\%$)}}} & \multicolumn{2}{c}{\textbf{OCI$\rightarrow$M}} & \multicolumn{2}{c}{\textbf{OMI$\rightarrow$C}} & \multicolumn{2}{c}{\textbf{OCM$\rightarrow$I}} & \multicolumn{2}{c}{\textbf{ICM$\rightarrow$O}} \\
            \multicolumn{1}{c}{} & \multicolumn{1}{c}{\textbf{HTER }$\downarrow$} & \multicolumn{1}{c}{\textbf{AUC} $\uparrow$} & \multicolumn{1}{c}{\textbf{HTER  $\downarrow$}} & \multicolumn{1}{c}{\textbf{AUC $\uparrow$}} & \multicolumn{1}{c}{\textbf{HTER}  $\downarrow$} & \multicolumn{1}{c}{\textbf{AUC} $\uparrow$} & \multicolumn{1}{c}{\textbf{HTER}  $\downarrow$} & \multicolumn{1}{c}{\textbf{AUC}  $\uparrow$}
             \\ \midrule
            MMD-AAE~\cite{li2018domain} & 27.08 & 83.19 & 44.59 & 58.29 & 31.58 & 75.18 & 40.98 & 63.08 \\
            MADDG~\cite{shao2019multi} & 17.69 & 88.06 & 24.50 & 84.51 & 22.19 & 84.99 & 27.98 & 80.02 \\
            SSDG-M~\cite{jia2020ssdg} & 16.67 & 90.47 & 23.11 & 85.45 & 18.21 & 94.61 & 25.17 & 81.83 \\
            DR-MD-Net~\cite{wang2020cross} & 17.02 & 90.10 & 19.68 & 87.43 & 20.87 & 86.72 & 25.02 & 81.47 \\
            RFMeta~\cite{shao2020regularized} & 13.89 & 93.98 & 20.27 & 88.16 & 17.30 & 90.48 & 16.45 & 91.16 \\
            NAS-FAS~\cite{yu2020fas} & 19.53 & 88.63 & 16.54 & 90.18 & 14.51 & 93.84 & 13.80 & 93.43 \\
            D2AM~\cite{chen2021generalizable} & 12.70 & 95.66 & 20.98 & 85.58 & 15.43 & 91.22 & 15.27 & 90.87 \\
            SDA~\cite{wang2021self} & 15.40 & 91.80 & 24.50 & 84.40 & 15.60 & 90.10 & 23.10 & 84.30 \\
            DRDG~\cite{liu2021dual} & 12.43 & 95.81 & 19.05 & 88.79 & 15.56 & 91.79 & 15.63 & 91.75 \\
            ANRL~\cite{liu2021adaptive} & 10.83 & 96.75 & 17.83 & 89.26 & 16.03 & 91.04 & 15.67 & 91.90 \\
            SSAN-M~\cite{wang2022ssan} & 10.42 & 94.76 & 16.47 & 90.81 & 14.00 & 94.58 & 19.51 & 88.17 \\
            SSDG-R~\cite{jia2020ssdg} & 7.38 & 97.17 & 10.44 & 95.94 & 11.71 & 96.59 & 15.61 & 91.54 \\
            SSAN-R~\cite{wang2022ssan} & 6.67 & \textbf{98.75} & 10.00 & \textbf{96.67} & 8.88 & 96.79 & 13.72 & 93.63 \\
            PatchNet~\cite{wang2022patchnet} & 7.10 & 98.46 & 11.33 & 94.58 & 13.40 & 95.67 & 11.82 & 95.07 \\
            % Ours (ImageNet) & 6.67 & 97 & 9.67 & 93.84 & 10.15 & 96.16 & 11.64 & 94.92 \\
            \methodname (Ours) & \textbf{5.95} & 96.55 & \textbf{8.78} & 95.37 & \textbf{6.58}  & \textbf{97.54} & \textbf{10.00} & \textbf{96.23}  \\ \bottomrule
            \end{tabular}
        }
    \end{minipage}
    \hfill
    \begin{minipage}[t]{0.28\textwidth}
        \vspace*{-40pt}
        \caption{
            \small 
            {\bf Comparisons with SoTA methods:} 
            Cross-domain face anti-spoofing is evaluated among four popular benchmark datasets: CASIA (\textbf{C}), Idiap Replay (\textbf{I}), MSU-MFSD (\textbf{M}), and Oulu-NPU (\textbf{O}). 
            Methods are compared at their best performance following the commonly used evaluation process \cite{jia2020ssdg}. 
            $\uparrow$ indicates larger values are better, and $\downarrow$ indicates smaller values are better.
        }
        \label{tab:best}
    \end{minipage}
    \vspace{-1ex}
\end{table*}


\begin{table*}[t]
\small \centering
\scalebox{0.81}{
\begin{tabular}{lllll} \toprule
\multicolumn{1}{c}{\multirow{2}{*}{\textbf{Method ($\%$)}}} & \multicolumn{1}{c}{\textbf{OCI$\rightarrow$M}} & \multicolumn{1}{c}{\textbf{OMI$\rightarrow$C}} & \multicolumn{1}{c}{\textbf{OCM$\rightarrow$I}} & \multicolumn{1}{c}{\textbf{ICM$\rightarrow$O}} \\
& \multicolumn{1}{c}{\textbf{HTER}$\downarrow$ /\textbf{AUC}$\uparrow$/\textbf{TPR95}$\uparrow$}  & \multicolumn{1}{c}{\textbf{HTER}$\downarrow$ /\textbf{AUC}$\uparrow$/\textbf{TPR95}$\uparrow$}  & \multicolumn{1}{c}{\textbf{HTER}$\downarrow$ /\textbf{AUC}$\uparrow$/\textbf{TPR95}$\uparrow$}  &
\multicolumn{1}{c}{\textbf{HTER}$\downarrow$ /\textbf{AUC}$\uparrow$/\textbf{TPR95}$\uparrow$}
 \\ \midrule
SSDG-R~\cite{jia2020ssdg} & 14.65 $ ^{{1.21}} $ / 91.93	$ ^{{1.35}} $  / 53.68  $ ^{{2.56}} $
& 28.76	$ ^{{0.89}} $ / 80.91	$ ^{{1.10}} $ / 41.47 $ ^{{2.68}} $ 
& 22.84	$ ^{{1.14}} $ / 78.67	$ ^{{1.31}} $  / 50.80  $ ^{{5.95}} $
& 15.83	$ ^{{1.29}} $ / 92.13	$ ^{{0.96}} $ / 66.54 $ ^{{4.00}} $ \\
SSAN-R~\cite{wang2022ssan} & 21.79 $ ^{{3.68}} $ /  84.06   $ ^{{3.78}} $  / 51.91  $ ^{{4.28}} $
&  26.44   $ ^{{2.91}} $ /  78.84   $ ^{{2.83}} $ / 45.36 $ ^{{4.29}} $ 
&  35.39   $ ^{{8.04}} $ /  70.13   $ ^{{9.03}} $  / 64.00  $ ^{{2.70}} $
&  25.72   $ ^{{3.74}} $ /  79.37   $ ^{{4.69}} $ / 36.75  $ ^{{5.19}} $ \\
PatchNet~\cite{wang2022patchnet} & 25.92   $ ^{{1.13}} $ /  83.43   $ ^{{0.87}} $ / 38.75 $ ^{{8.31}} $
&  36.26   $ ^{{1.98}} $ /  71.38   $ ^{{1.89}} $ / 19.22 $ ^{{3.85}} $ 
&  29.75   $ ^{{2.76}} $ /  80.53   $ ^{{1.35}} $  / 54.25  $ ^{{2.18}} $
&  23.49   $ ^{{1.80}} $ / 84.62   $ ^{{1.92}} $ / 39.39  $ ^{{6.83}} $ \\ \midrule 
SA-FAS (Ours) 
& \textbf{14.36}	$ ^{{1.10}} $ / \textbf{92.06}	$ ^{{0.53}} $ / \textbf{55.71}	$ ^{{4.82}} $ 
& \textbf{19.40}	$ ^{{0.66}} $ / \textbf{
88.69}	$ ^{{0.67}} $ / \textbf{50.53}	$ ^{{3.60}} $ 
& \textbf{11.48}	$ ^{{1.10}} $ / \textbf{
95.74}	$ ^{{0.55}} $ / \textbf{77.05}	$ ^{{3.26}} $ 
& \textbf{11.29}	$ ^{{0.32}} $ / \textbf{95.23}	$ ^{{0.24}} $ / \textbf{73.38} $ ^{{1.64}} $ \\
\bottomrule
\end{tabular}}
\figvspace
\caption{\small 
{\bf Evaluation upon convergence:}
Evaluation of cross-domain face anti-spoofing among CASIA (\textbf{C}), Idiap Replay (\textbf{I}), MSU-MFSD
(\textbf{M}), and Oulu-NPU (\textbf{O}) databases. Methods are compared at their mean/std performance based on the last 10 epochs. 
%
}
\label{tab:mean}
\vspace{-1.5ex}
\end{table*}







\SubSection{Training and inference}
\label{sec:train_step}

\Paragraph{Overall losses}
Considering the contrastive loss Eqn.~\eqref{eq:supcon}, the overall objective (dubbed as SA-FAS) can be written as: 

\begin{align}
    & \min _{\phi, \beta_{e^{(1)}}, ..., \beta_{e^{(E)}}} \mathcal{L}_{\textit{\textit{align}}} + \lambda \mathcal{L}_{\textit{sep}} \quad \rightarrow \mathcal{L}_{\textit{all}}  
    \label{eq:irm_constrain_overall} \\
    &\text { s.t. }  \forall e \in \mathcal{E}, \exists \beta_e \in \Omega_e(\phi), \beta_e \in \Upsilon_{\alpha}(\beta_e), \nonumber
    %\figvspace
\end{align}
where $\lambda$ is the coefficient for the loss term. The overall training pipeline is provided in Alg.~\ref{alg:main}.
% 

\Paragraph{Inference}
At the inference stage, we use the mean hyperplane from $\beta_{e^{(1)}}, ..., \beta_{e^{(E)}}$ to get the final score. 
% 
Specifically, the output is given by $$f(\*x) = \mathbb{E}_{e\in\mathcal{E}}[\beta_{e}^T\phi(\*x)].$$
Note that upon convergence, the cosine distance between any two of $\beta_{e^{(1)}}, ..., \beta_{e^{(E)}}$ is very close to 1, \ie, $\beta_{e^{(1)}} \!\approx\! ... \!\approx\! \beta_{e^{(E)}}$. 
This observation is verified in Appendix~\ref{sec:cosine_curve}, with an ablation (converged angles \vs different $\alpha$) in Appendix~\ref{sec:sensitivity}. 



% 
\Section{Experiments}
\label{sec:exp}

\SubSection{Experimental setups}


\Paragraph{Datasets and protocols} 
We  evaluate on four widely used  datasets: \texttt{Oulu-NPU} (\textbf{O})~\cite{boulkenafet2017oulu}, \texttt{CASIA} (\textbf{C})~\cite{zhang2012casia}, \texttt{Idiap Replay attack} (\textbf{I})~\cite{chingovska2012replay}, and \texttt{MSU-MFSD} (\textbf{M})~\cite{wen2015msu}. 
Following  prior works, we treat each dataset as one domain and apply the leave-one-out test protocol to evaluate their cross-domain generalization. 
Specifically, we refer \textbf{OCI$\rightarrow$M} to be the protocol that trains on \texttt{Oulu-NPU}, \texttt{CASIA}, \texttt{Idiap Replay attack} and tests on  \texttt{MSU-MFSD}. \textbf{OMI$\rightarrow$C}, \textbf{OCM$\rightarrow$I} and \textbf{ICM$\rightarrow$O} are defined in a similar fashion. 

% 
\Paragraph{Implementation details} 
The input images are cropped using MTCNN~\cite{zhang2016mtcnn} and resized to 256$\times$256. 
%
For fair comparisons with SoTA methods \cite{jia2020ssdg,wang2022ssan,wang2022patchnet}, we use the same ResNet-18 backbone.
We train the network with SGD optimizer and an initial learning rate of 5e\text{-}3, which is decayed by 2 at epoch 40 and 80 and the total training epoch is 100 in most set-ups\footnote{Due to the smaller training data size of \textbf{ICM}, we let the  \textbf{ICM$\rightarrow$O} to train for 300 epochs and decay at epoch 120 and 240. }.
We set the weight decay as 5e\text{-}4 and the batch size as 96 for each training domain. 
For \methodname hyperparameters, we set $\alpha\!=\!0.995$, $\lambda\!=\! 0.1$ and $T_a \!=\! 20$. 
%

\Paragraph{Evaluation metrics} We evaluate the model performance using three standard metrics: Half Total Error Rate
(HTER), Area Under Curve (AUC), and True Positive Rate (TPR95) at a False Positive Rate
(FPR) 5\%. 
While HTER and AUC assess the theoretical performance, TPR at a certain FPR is adept at
reflecting how well the model performs in practice. 
% 

\SubSection{Cross-domain performance}

Tab.~\ref{tab:best} summarizes our comparison with an extensive collection of recent studies, including SoTA methods: \texttt{PatchNet}~\cite{wang2022patchnet}, \texttt{SSAN}~\cite{wang2022ssan} and \texttt{SSDG}~\cite{jia2020ssdg}.
%
\methodname outperforms the rivals by a significant margin on cross-domain FAS benchmarks.
In particular, we improve upon the best baseline \cite{wang2022ssan} by 2.30\% in HTER in the setting \textbf{OCM$\rightarrow$I}, which is more than \textbf{25\%} improvement.


\Paragraph{Comparison upon convergence} 
Note that the performance in Tab.~\ref{tab:best} follows the convention in \cite{jia2020ssdg}, which is reported on the training snapshot (\eg, epoch 16) with the lowest test error. 
While this setting may manifest the best performance from the model, the results can significantly fluctuate on the test set and hard to reflect the generalization performance when a test set is unavailable (shown in Appendix~\ref{sec:whyfairsetting}).
To provide a more fair setting, we propose to report the average performance from the \textbf{last 10 epochs} upon convergence.
In our case, the stopping criterion is either (1) the binary classification loss for live/spoof is smaller than 1e\text{-}3 for consecutive 10 epochs, or (2) the epoch number reaches max limit, whichever comes first.

%

In Tab.~\ref{tab:mean}, we compare with SoTA methods in this setting, and provide three key observations: 
(1) The numbers are way worse than the ones in Tab.~\ref{tab:best} across all methods, indicating the best model selected by conventional lowest test errors \cite{jia2020ssdg} has large randomness.
This also shows that cross-domain FAS is far less-solved than expected.
%
(2) The standard deviation in Tab.~\ref{tab:mean} denotes how stable each method performs.
Most methods can converge to a relatively stable status, while methods with an adversarial loss (\eg, \cite{wang2022ssan}) have a relatively larger standard deviation, indicating adversarial loss might trigger more unstable training.
(3) In our setting, \methodname still largely outperforms SoTA \cite{wang2022patchnet,wang2022ssan,jia2020ssdg}, which further validates the superiority of our method.
Our method is also the most stable compared to SoTA, with the smallest standard deviation.
We proceed by analyzing why traditional methods are less favorable in cross-domain FAS and why our methods perform better.

%

\Section{Ablation and Discussion}
\begin{table}[t]
\small \centering
\scalebox{0.85}{
\begin{tabular}{llll} \toprule
\multicolumn{1}{c}{\multirow{2}{*}{\textbf{Method}}} & \multicolumn{3}{c}{\textbf{Average}} \\
& \multicolumn{1}{c}{\textbf{HTER}$\downarrow$} & \multicolumn{1}{c}{\textbf{AUC}$\uparrow$} & \multicolumn{1}{c}{\textbf{TPR95}$\uparrow$}
 \\   \midrule
 SimCLR~\cite{chen2020simclr}  & 22.53 $ ^{1.31} $ & 84.42 $ ^{1.04} $ & 51.14 $ ^{3.44} $  \\
SimSiam~\cite{chen2021exploring}  & 18.89 $ ^{0.97} $ & 89.93 $ ^{0.80} $ & 56.62 $ ^{2.88} $  \\
  Triplet~\cite{schroff2015facenet}  & 18.75 $ ^{2.31} $ & 88.11 $ ^{2.30} $ & 50.53 $ ^{8.76} $  \\
SupCon (SSDG)~\cite{jia2020ssdg}  & 17.91 $ ^{1.05} $ & 90.10 $ ^{0.68} $ & 61.98 $ ^{2.87} $  \\
SupCon~\cite{2020supcon}  & 17.03 $ ^{1.73} $ & 90.68 $ ^{1.29} $ & 56.72 $ ^{5.06} $  \\
 \midrule 
ERM   & 17.22 $ ^{1.26} $ & 90.21 $ ^{1.38} $ & 58.62 $ ^{3.77} $  \\
DANN~\cite{ganin2016dann}  & 17.93 $ ^{1.02} $ & 90.66 $ ^{0.56} $ & 58.66 $ ^{3.14} $  \\
  IRM-v1~\cite{arjovsky2019irm}   & 17.41 $ ^{0.77} $ & 91.16 $ ^{0.52} $ & 60.98 $ ^{2.10} $  \\
 VREx~\cite{krueger2021out}   & 25.02 $ ^{1.92} $ & 80.65 $ ^{2.20} $ & 45.12 $ ^{3.78} $  \\
IB-IRM~\cite{ahuja2021ibirm}   & 17.57 $ ^{0.74} $ & 91.71 $ ^{0.51} $ & 62.16 $ ^{2.35} $  \\ 
PG-IRM (Ours)   & 15.58 $ ^{0.96} $ & 92.03 $ ^{0.62} $ & 63.31 $ ^{2.59} $  \\  \midrule
SA-FAS (Ours)  & \textbf{14.25} $ ^{0.79} $ & \textbf{92.93} $ ^{0.49} $ & \textbf{64.16} $ ^{3.33} $  \\
\bottomrule
\end{tabular}}
\figvspace
\caption{\small 
{\bf Ablation study:}
The averaged performance is computed over all four cross-domain settings.}
\label{tab:abl_main}
\vspace{-2ex}
\end{table}




\SubSection{Effectiveness of loss components}  
Our overall objective function Eq.~\eqref{eq:irm_constrain_overall} consists of two parts: (a) Separability loss ($\mathcal{L}_{sep}$) for feature space; and (b) Alignment loss ($\mathcal{L}_{align}$) for regularizing the classifier. We ablate the contribution of each component in Tab.~\ref{tab:abl_main}. 

\Paragraph{Separability loss}
%
We consider the two most common strategies used in the contrastive learning community (\ie, \texttt{SimCLR}~\cite{chen2020simclr}, \texttt{SimSiam}~\cite{chen2020simclr}) and one in face recognition (\ie, \texttt{Triplet loss}~\cite{schroff2015facenet}). We also provide the comparison of SupCon with the clustering policy used in SSDG~\cite{jia2020ssdg}\footnote{SSDG assumes live samples in all domains form one cluster, and spoof samples in each domain respectively form the other three clusters.}. 
%
All losses are directly applied on the penultimate layer's feature $\phi(\*x)$ with the same hyper-parameters, and all final classifications are supervised by ERM.
We observe that the SupCon loss used in our framework outperforms other rivals.
This validates the effectiveness of the domain-wise separable feature space for cross-domain FAS.

%
\Paragraph{Alignment loss}
IRM objective is known to be hard to optimize. 
Other than the proposed PG-IRM, existing works  \texttt{IRM-v1}~\cite{arjovsky2019irm}, \texttt{IB-IRM}~\cite{ahuja2021ibirm}, and \texttt{VRex}~\cite{krueger2021rex} alternatively consider a Lagrangian form:
\begin{equation}
\min _{\phi, \beta^{*}} \frac{1}{|\mathcal{E}|} \sum_{e \in \mathcal{E}}\left[\mathcal{R}^{e}(\phi, \beta^{*})+\lambda\left\|\nabla_{\beta^{*}} \mathcal{R}^{e}(\phi, \beta^{*})\right\|_{2}^{2}\right].
\label{eq:irm_langran}
\end{equation}
In Tab.~\ref{tab:abl_main}, we compare PG-IRM with the baseline ERM as well as other IRM alternatives, and our method shows a better overall performance. This shows further evidence that the Lagrangian penalty term can be ineffective, especially in the non-linear case \cite{rosenfeld2020riskirm, kamath2021doesirm}. In comparison, PG-IRM optimizes the IRM objective directly with Projected Gradient, which clearly distinguishes it from existing methods. 

%
Overall, the ablation studies suggest all components in our framework are indispensable to enhancing the generalization ability of cross-domain spoof detection.
\begin{figure}[t]
    \small\centering
    \includegraphics[width=0.5\textwidth]{figs/corr_all.pdf}
    \vspace{-4ex}
    \caption{\small 
    {\bf Correlation of performance and SA-FAS:}:
    Correlation between the test performance AUC and two properties measure ($S_{align}/S_{sep}$). Each dot represents one snapshot during the training stage in all four cross-domain settings. We provide separate figures for each setting in Appendix (Fig.~\ref{fig:more_corr}).}
    \label{fig:corr_all}
    \vspace{-0.5ex}
\end{figure}
\begin{figure*}[t]
    \small\centering
    \includegraphics[width=1.0\textwidth]{figs/umap.pdf}
    \vspace{-5ex}
    \caption{\small 
    {\bf Feature learned in different domains}:
    UMAP~\cite{mcinnes2018umap} projection of the penultimate layer of ResNet-18 trained with \methodname in the cross-test setting of face anti-spoofing datasets. 
    The dotted line shows the decision boundary derived from training samples in 2D space.}
    \figvspace
    \label{fig:umap}
\end{figure*}
\begin{figure}[t]
    \small\centering
    \includegraphics[width=.46\textwidth]{figs/cmp_da_umap.pdf}
    \figvspace
    \caption{\small
    {\bf Features of DANN \vs~SSDG:}
    UMAP \cite{mcinnes2018umap} visualization of the penultimate layer of ResNet-18 trained with DANN \cite{ganin2016dann} and SSDG \cite{jia2020ssdg} in the cross-test setting of OMI$\rightarrow$C.}
    \vspace{-0.3cm}
    \label{fig:cmp_da_umap}
\end{figure}


\SubSection{Separability and alignment analysis}  

SA-FAS aims to produce a feature space with two critical properties: Separability and Alignment. 
In this section, we empirically investigate if these two properties can lead to a better generalization performance. Specifically, we provide two corresponding measures based on the learned classifiers and the extracted feature vector $\*z$ of samples from the \textbf{test} domain. 
We define the separability score as:
$$S_{sep} =  1 - \cos(\mathbb{E}_{spoof}[\*z], \mathbb{E}_{live}[\*z]), $$
where we measure the cosine angle between the center of live/spoof features. 
A separated feature space naturally leads to a small cosine value and thus a larger $S_{sep}$ score. 
For the alignment score, we define:
\vspace{-2ex}
% 
\vspace{\baselineskip}
\begin{equation}
    S_{align} = \mathbb{E}_{e \in \mathcal{E}}[ \cos(\beta_e, \tikzmarknode{oracle}{\highlight{blue}{$\mathbb{E}_{spoof}[\*z] - \mathbb{E}_{live}[\*z])$}})],
\end{equation}
where the trajectory from the center of spoof to live is treated as an \textit{oracle} vector, which we measure how close it is with the norm vector of $\beta$ of the learned hyperplane.
\begin{tikzpicture}[overlay,remember picture,>=stealth,nodes={align=left,inner ysep=1pt},<-]
    \path (oracle.north) ++ (0,0.5em) node[anchor=south west,color=blue!67] (scalep){\textit{oracle vector}};
    \draw [color=blue!87](oracle.north) |- ([xshift=-0.3ex,color=blue]scalep.south east);
\end{tikzpicture}


With the measure of two properties, we show their correlation to the generalization performance (\ie, AUC) in Fig.~\ref{fig:corr_all}.
We see that $S_{sep}$ and $S_{align}$ are positively related to their test performance. 
It validates that these two properties are beneficial for a domain-invariant classifier. 
Specifically, Fig.~\ref{fig:corr_all}(a) compares the setting with and without PG-IRM. Using PG-IRM leads to a higher alignment score and AUC, which verifies that PG-IRM can better align the live-vs-spoof hyperplanes for the unseen domain and improve the generalization ability. Similarly, Fig.~\ref{fig:corr_all}(b) compares the setting with and without SupCon. The results validate that SupCon can lead to better separability in the feature space which benefits the classification. 

\SubSection{UMAP visualization}  
% 
Fig.~\ref{fig:umap} first provides UMAP~\cite{mcinnes2018umap} visualization of \methodname feature space from the penultimate layer. We see that the hyperplane between live samples and spoof samples is consistent across different training domains and also transferable to unseen test domains. 
For instance, in the setting of \texttt{OMI$\rightarrow$C}, 
the test live samples in blue circles can be separated from the test spoof samples in blue crosses by the hyperplane.  Another interesting finding is that some \texttt{CASIA} samples in blue are closer to \texttt{OULU} with \textbf{high resolution} and some are closer to \texttt{MSU} or \texttt{REPLAY} with \textbf{low resolution}, which reflects the fact that \texttt{CASIA} is a mixed dataset with both low and high resolution images. 
These findings validate that the domain gap (resolution) manifests in a way that is invariant to the live-vs-spoof hyperplane. 

%
Beyond numerical and visual results, the superiority of domain-variant feature space can also be validated by theoretical support. Specifically, the estimated error bound for binary
classification in domain generalization~\cite{blanchard2021domain} becomes larger if $(M, n)$ is replaced with $(1, Mn)$, where $M$ is the domain number and $n$ is the training set size per domain. 
It indicates that separately training datasets from different domains is better than pooling them into one mixed dataset.

% 
\Paragraph{DANN~\cite{ganin2016dann} and SSDG~\cite{jia2020ssdg} visualization} 
We also compare the feature space of methods that aim to remove domain-specific signals from its feature representation.
%
DANN~\cite{ganin2016dann} leverages the adversarial loss to encourage the backbone to provide a domain-invariant feature. 
%
Fig.~\ref{fig:cmp_da_umap}(a) shows that the domain gap yet still broadly exists, especially for the test data from an unseen domain, which backfires on the generalizability of the classifier.
%
Similarly, SSDG~\cite{jia2020ssdg} learns a partial domain-invariant feature space where all live samples are clustered in one group while spoof samples are kept to be domain-dispersed. 
Although the degradation direction aligns better between train and test, compared to DANN, the domain gap still exists for live training samples as shown in Fig.~\ref{fig:cmp_da_umap}(b). 
%
These findings further validate the necessity of regularizing the live-vs-spoof hyperplanes to be consistent across different domains.

% 



\Section{Conclusion}

This paper provides a new learning framework \methodname that learns domain-variant features but domain-invariant decision boundaries for cross-domain FAS. Our framework is naturally motivated, which facilitates invariant decision boundaries and learning distinguishable representations. We provide important theoretical insights that IRM objectives can be equivalently optimized by the PG with an alternative objective. Experiments show that \methodname can notably improve performance compared to the current best methods, establishing state-of-the-art. We also discuss the limitation of our work in Appendix~\ref{sec:limit}. 
We hope this paper will inspire more future works in incorporating domain-specific signals in FAS feature representation, and also extending this idea to broader domain generalization tasks.

\newpage

%%%%%%%%% REFERENCES

{\small
\bibliographystyle{ieee_fullname}
\bibliography{egbib}
}

%%%%%%%%% REFERENCES

\section{Appendix for Proofs}

\paragraph{Proof of Theorem \ref{thm:main}.}

\begin{proof}
\label{proof:main}
Our proof has two steps. In Step 1, we will show that SimCLR is equivalent to minimizing the cross entropy loss defined in Eqn.~(\ref{eqn:cross-entropy}). 
In Step 2, we will show  that minimizing the cross-entropy loss 
is equivalent to spectral clustering on $\bfpi$. 
Combining the two steps together, we have proved our theorem. 

\textbf{Step 1: } SimCLR is equivalent to minimizing the cross entropy loss.

The cross-entropy loss takes expectation over 
$\bfW_\bfX\sim \mathbb{P}(\cdot ; \bfpi)$, 
which means $\bfW_\bfX$ has exactly one non-zero entry in each row $i$. By Lemma~\ref{lem:multinomial}, we know every row $i$ of $\bfW_\bfX$ is independent of other rows. Moreover, 
$\bfW_{\bfX,i}\sim \mathcal{M}(1, \bfpi_i/\sum_j \bfpi_{i,j})=\mathcal{M}(1, \bfpi_i)$, because $\bfpi_i$ itself is a probability distribution.
Similarly, we know $\bfW_\bfZ$ also has the row-independent property by sampling over $\mathbb{P}(\cdot;\bfK_\bfZ)$.
Therefore, by Lemma~\ref{lem:cross_split}, we know Eqn.~(\ref{eqn:cross-entropy}) is equivalent to:
\[
 -\sum_{i=1}^n \mathbb{E}_{\bfW_{\bfX,i}}[\log \mathbb{P}(\bfW_{\bfZ,i}=\bfW_{\bfX,i};\bfK_\bfZ)],
\]

This expression takes expectation over $\bfW_{\bfX,i}$ for the given row $i$. Notice that 
$\bfW_{\bfX,i}$ has exactly one non-zero entry, which equals $1$ (same for $\bfW_{\bfZ,i}$). 
As a result
we expand the above expression to be:
\begin{equation}
 -\sum_{i=1}^n \sum_{j\neq i} \Pr(\bfW_{\bfX,i,j}=1)\log \Pr(\bfW_{\bfZ,i,j}=1).
\label{eqn:detailed-expansion}    
\end{equation}


By Lemma~\ref{lem:multinomial}, $\Pr(\bfW_{\bfZ,i,j}=1)=\bfK_{\bfZ,i,j}/\|\bfK_{\bfZ,i}\|_1$ for $j\neq i$. Recall that $\bfK_\bfZ=(k(\bfZ_i-\bfZ_j))_{(i,j)\in[n]^2}$, which means 
$\bfK_{\bfZ,i,j}/\|\bfK_{\bfZ,i}\|_1=\frac{\exp(-\|\bfZ_i-\bfZ_j\|^2/{2\tau})}{\sum_{k\neq i}
\exp(-\|\bfZ_i-\bfZ_k\|^2/{2\tau})
}$ for $j\neq i$, when $k$ is the Gaussian kernel with variance $\tau$. 

Notice that $\bfZ_i=f(\bfX_i)$, so we know
\begin{equation}
-\log \Pr(\bfW_{\bfZ,i,j}=1)=
-\log \frac{\exp(-\|f(\bfX_i)-f(\bfX_j)\|^2/{2\tau})}{\sum_{k\neq i}
\exp(-\|f(\bfX_i)-f(\bfX_k)\|^2/{2\tau}),
}
\label{eqn:infonce-equivalence}    
\end{equation}


The right hand side is exactly the InfoNCE loss defined in Eqn.~(\ref{eqn:infonce}).
Inserting Eqn.~(\ref{eqn:infonce-equivalence}) into Eqn.~(\ref{eqn:detailed-expansion}), we get the SimCLR algorithm, which first samples augmentation pairs $(i,j)$ with $\Pr(\bfW_{\bfX,i,j}=1)$ for each row $i$, and then optimize the InfoNCE loss. 

\textbf{Step 2: } minimizing the cross entropy loss 
is equivalent to spectral clustering on $\bfpi$.


By Lemma~\ref{lem:convert_to_spectral}, we may further convert the loss to 
\begin{equation}
\label{eqn:main-theorem-repul-attr}
\min_{\bfZ}
-\sum_{(i,j)\in [n]^2} \mathbf{P}_{i,j}
\log k (\bfZ_i-\bfZ_j)+\log \mathbf{R}(\bfZ).
\end{equation}
Since $k$ is the Gaussian kernel, this reduces to \[
\min_\bfZ \mathrm{tr}(\bfZ^\top \mathbf{L}(\bfpi) \bfZ)
+\log \mathbf{R}(\bfZ),
\]

where we use the fact that $\mathbb{E}_{\bfW_\bfX\sim \mathbb{P}(\cdot; \bfpi)}[\mathbf{L}(\bfW_\bfX)]
=\mathbf{L}(\bfpi)
$, because the Laplacian operator is linear and $
\mathbb{E}_{\bfW_\bfX\sim \mathbb{P}(\cdot; \bfpi)}(\bfW_\bfX)=\bfpi
$.
\end{proof}

\paragraph{Proof of Theorem \ref{thm:clip}.}
\begin{proof}
Since $\bfW_\bfX\sim \mathbb{P}(\cdot;\bfpi_{\mathbf{A}, \mathbf{B}})$, we know 
$\bfW_\bfX$ has exactly one non-zero entry in each row, denoting the pair that got sampled. 
A notable difference compared to the previous proof is we now have $n_\mathcal{A}+n_\mathcal{B}$ objects in our graph. CLIP deals with this by taking a mini-batch of size $2N$, 
such that $n_\mathcal{A}=n_\mathcal{B}=N$, and adding the $2N$ InfoNCE losses together. We label the objects in $\mathcal{A}$ as $[n_\mathcal{A}]$, and the objects in $\mathcal{B}$ as $\{n_\mathcal{A}+1, \cdots, n_\mathcal{A}+n_\mathcal{B}\}$. 

Notice that $\bfpi_{\mathbf{A}, \mathbf{B}}$ is a bipartite graph, so the edges of objects in $\mathcal{A}$ will only connect to object in $\mathcal{B}$ and vice versa. We can define the similarity matrix in $\cZ$ as $\bfK_\bfZ$, 
where $\bfK_\bfZ(i, j+n_\mathcal{A})=\bfK_\bfZ(j+n_\mathcal{A},i)= k(\bfZ_i-\bfZ_j)$ for $i\in [n_\mathcal{A}], j\in [n_\mathcal{B}]$, and otherwise we set $\bfK_\bfZ(i,j)=0$. 
The rest is same as the previous proof. 
\end{proof}

\paragraph{Proof of Theorem \ref{thm:exponential}.}

\begin{proof}
\label{proof:exponential}
Since the objective function consists of a linear term combined with an entropy regularization, which is a strongly concave function, the maximization problem is a convex optimization problem. Owing to the implicit constraints provided by the entropy function, the problem is equivalent to having only the equality constraint. We then introduce the Lagrangian multiplier $\lambda$ and obtain the following relaxed problem:

$$
\widetilde{E}(\boldsymbol{\alpha})=\psi_{1}-\sum_{i=1}^n \alpha_{i} \psi_{i}+\tau \sum_{i=1}^n \alpha_{i}\log \alpha_{i}+\lambda\left(\boldsymbol{\alpha}^{\top} \mathbf{1}_n-1\right).
$$

As the relaxed problem is unconstrained, taking the derivative with respect to $\alpha_{i}$ yields

$$
\frac{\partial \widetilde{E}(\boldsymbol{\alpha})}{\partial \alpha_{i}}=-\psi_{i}+\tau\left(\log \alpha_{i}+\alpha_{i} \frac{1}{\alpha_{i}}\right)+\lambda=0.
$$

Solving the above equation implies that $\alpha_{i}$ takes the form
$
\alpha_{i}=\exp \left(\frac{1}{\tau} \psi_{i}\right) \exp \left(\frac{-\lambda}{\tau}-1\right).
$ Since $\alpha_{i}$ lies on the probability simplex, the optimal $\alpha_{i}$ is explicitly given by
$
\alpha^{*}_{i}=\frac{\exp \left(\frac{1}{\tau} \psi_{i}\right)}{\sum_{i^{\prime}=1}^n \exp \left(\frac{1}{\tau} \psi_{i^{\prime}}\right)} .
$ Substituting the optimal point into the objective function, we obtain
$$
\begin{aligned}
E\left(\boldsymbol{\alpha}^*\right)  &=\psi_1-\sum_{i=1}^n \frac{\exp \left(\frac{1}{\tau} \psi_{i}\right)}{\sum_{i^{\prime}=1}^n \exp \left(\frac{1}{\tau} \psi_{i^{\prime}}\right)} \psi_{i}+\tau \sum_{i=1}^n \frac{\exp \left(\frac{1}{\tau} \psi_{i}\right)}{\sum_{i^{\prime}=1}^n \exp \left(\frac{1}{\tau} \psi_{i^{\prime}}\right)}\log \frac{\exp \left(\frac{1}{\tau} \psi_{i}\right)}{\sum_{i^{\prime}=1}^n \exp \left(\frac{1}{\tau} \psi_{i^{\prime}}\right)} \\
& =\psi_1 - \tau \log \left(\sum_{i=1}^n \exp \left(\frac{1}{\tau} \psi_{i}\right)\right).
\end{aligned}
$$
Thus, the Lagrangian dual function is given by
\begin{equation*}
-E\left(\boldsymbol{\alpha}^*\right)= -\tau \log \frac{\exp \left(\frac{1}{\tau} \psi_{1}\right)}{\sum_{i=1}^n \exp \left(\frac{1}{\tau} \psi_{i}\right)}.\qedhere
\end{equation*}
\end{proof}



\section{More on Experiments} \label{section: experiment_details}

\paragraph{CIFAR-10 and CIFAR-100} CIFAR-10 ~\citep{krizhevsky2009learning} and CIFAR-100 ~\citep{krizhevsky2009learning} are well-known classic image classification datasets. Both CIFAR-10 and CIFAR-100 contain a total of 60k $32 \times 32$ labeled images of different classes, with 50k for training and 10k for testing. CIFAR-10 is similar to CIFAR-100, except there are 10 different classes in CIFAR-10 and 100 classes in CIFAR-100.

\paragraph{TinyImageNet} TinyImageNet ~\citep{le2015tiny} is a subset of ImageNet ~\citep{deng2009imagenet}. There are 200 different object classes in TinyImageNet, with 500 training images, 50 validation images, and 50 test images for each class. All the images in TinyImageNet are colored and labeled with a size of $64 \times 64$.

\textbf{Pseudo-code.} Algorithm \ref{alg:Training Procedure} presents the pseudo-code for our empirical training procedure.

\begin{algorithm}[!htbp]
\caption{Training Procedure}
\label{alg:Training Procedure}
\begin{algorithmic}[1]
\REQUIRE trainable encoder network $f$, batch size $N$, augmentation strategy \textit{aug}, loss function $L$ with hyperparameters \textit{args}
\FOR {sampled minibatch ${x_i}_{i=1}^N$}
\FORALL{$i \in { 1, ..., N }$}
\STATE draw two augmentations $t_i = \textit{aug}\left(x_i\right) $, $t_i' = \textit{aug}\left(x_i\right) $
\STATE $z_i = f\left(t_i\right)$, $z_i' = f\left(t_i'\right)$
\ENDFOR
\STATE compute loss $\mathcal{L} = L(N, z, z', \textit{args})$
\STATE update encoder network $f$ to minimize $\mathcal{L}$
\ENDFOR
\STATE \textbf{Return} encoder network $f$
\end{algorithmic}
\end{algorithm}

We also provide the pseudo-code for our core loss function used in the training procedure in Algorithm \ref{alg:Core loss}. The pseudo-code is almost identical to SimCLR's loss function, with the exception of an extra parameter $\gamma$.

\begin{algorithm}[!htbp]
\caption{Core loss function $\mathcal{C}$}
\label{alg:Core loss}
\begin{algorithmic}[1]
\REQUIRE batch size $N$, two encoded minibatches $z_1, z_2$, $\gamma$, temperature $\tau$
\STATE $z = \textit{concat}\left(z_1, z_2\right)$
\FOR {$i \in {1, ..., 2N }, j \in {1, ..., 2N}$ }
\STATE $s_{i,j} = \Vert z_i - z_j \Vert_2^{\gamma}$
\ENDFOR
\STATE \textbf{define} $l(i, j)$ \textbf{as} $l(i, j) = - \log \frac{exp\left(s_{i,j}/\tau \right)}{\sum_{k=1}^{2N} \mathbf{1}{[k \ne i]} exp\left(s{i, j} / \tau \right)} $
\STATE \textbf{Return} $\frac{1}{2N} \sum_{k=1}^N\left[l(i, i+N) + l(i+N, i)\right]$
\end{algorithmic}
\end{algorithm}

Utilizing the core loss function $\mathcal{C}$, we can define all kernel loss functions used in our experiments in Table \ref{table: loss definition}. For all $z_i \in z$ with even dimensions $n$, we define $z_{L_i} = z_i\left[0:n/2\right]$ and $z_{R_i} = z_i\left[n/2:n\right]$.

\begin{table}[ht]
\centering
\begin{tabular}{{@{}l|l@{}}}
Kernel  &  Loss function \\ \midrule
Laplacian & $\mathcal{C}\left(N, z, z', \gamma=1, \tau\right)$\\ \midrule
Sum       & $\lambda * \mathcal{C}\left(N, z, z', \gamma=1, \tau_1\right) + (1-\lambda) * \mathcal{C}\left(N, z, z', \gamma=2, \tau_2\right)$  \\ \midrule
Concatenation Sum&$\lambda * \mathcal{C}\left(N, z_L, z'_L, \gamma=1, \tau_1\right) + (1-\lambda) * \mathcal{C}\left(N, z_R, z'_R, \gamma=2, \tau_2\right)$\\ \midrule
$\gamma = 0.5$ & $\mathcal{C}\left(N, z, z', \gamma=0.5, \tau\right)$          \\ 

\end{tabular}

\caption{Definition of kernel loss functions in our experiments}
\label {table: loss definition}
\end{table}

\textbf{Baselines.} We reproduce the SimCLR algorithm using PyTorch Lightning~\citep{PytorchLightning}.

\textbf{Encoder details.}
The encoder $f$ consists of a backbone network and a projection network. We employ ResNet50~\citep{ResNet} as the backbone and a 2-layer MLP (connected by a batch normalization~\citep{ioffe2015batch} layer and a ReLU \cite{nair2010rectified} layer) with hidden dimensions 2048 and output dimensions 128 (or 256 in the concatenation kernel case).

\textbf{Encoder hyperparameter tuning.}
For each encoder training case, we randomly sample 500 hyperparameter groups (sample details are shown in Table \ref{table: Hyperparameter sample}) and train these samples simultaneously using Ray Tune ~\citep{RayTune}, with the ASHA scheduler~\citep{li2018massively}. Ultimately, the hyperparameter group that maximizes the online validation accuracy (integrated in PyTorch Lightning) within 5000 validation steps is chosen for the given encoder training case.

\begin{table}[ht]
\centering

\begin{tabular}{@{}l|l|l@{}}
\midrule
Hyperparameter  & Sample Range & Sample Strategy \\ \midrule
start learning rate & $\left[10^{-2}, 10\right]$ & log uniform \\ \midrule
$\lambda$       & $\left[0, 1\right]$ & uniform \\ \midrule
$\tau$, $\tau_1$, $\tau_2$ & $\left[0, 1\right]$ & log uniform \\ \midrule
\end{tabular}

\caption{Hyperparameters sample strategy}
\label {table: Hyperparameter sample}
\end{table}

\textbf{Encoder training.} 
We train each encoder using the LARS optimizer~\citep{LARSOptimizer}, LambdaLR Scheduler in PyTorch, momentum 0.9, weight decay $10^{-6}$, batch size 256, and the aforementioned hyperparameters for 400 epochs on a single A-100 GPU.

\textbf{Image transformation.} The image transformation strategy, including augmentation, is identical to the default transformation strategy provided by PyTorch Lightning.

\textbf{Linear evaluation.}
The linear head is trained using the SGD optimizer with a cosine learning rate scheduler, batch size 64, and weight decay $10^{-6}$ for 100 epochs. The learning rate starts at $0.3$ and ends at $0$.

\textbf{Moco Experiments.} We also tested our method based on MoCo~\citep{he2019moco}. The results are summarized in Table \ref{tab:results-moco}. Here we choose ResNet18~\citep{ResNet} as the backbone and set a temperature of $0.1$ as default. For our simple sum kernel, we set $\lambda=0.8$. The results show that our method outperforms the original MoCo method.

\begin{table}[thb]
\centering
\caption{MoCo Experiment Results on CIFAR-10 and CIFAR-100.}
\label{tab:results-moco}
\resizebox{\textwidth}{!}{%
\begin{tabular}{@{}c|ccc|ccc@{}}
\toprule
\multirow{3}{*}{Method} & \multicolumn{3}{c|}{CIFAR-10} & \multicolumn{3}{c}{CIFAR-100} \\ \cmidrule(lr){2-4} \cmidrule(lr){5-7} 
                        & 200 epochs & 400 epochs    & 1000 epochs   & 200 epochs & 400 epochs & 1000 epochs         \\ \midrule
MoCo (repro.)         & $76.41 \pm 0.12$    & $80.01 \pm 0.15$          & $84.45 \pm 0.08$    & $\mathbf{47.02 \pm 0.11}$ & $52.50 \pm 0.07$ & $57.62 \pm 0.15$            \\
\midrule
Laplacian Kernel        & ${78.09 \pm 0.10}$    & $\mathbf{83.85 \pm 0.09}$          & $\mathbf{88.34 \pm 0.16}$    & $46.12 \pm 0.22$   & $53.44 \pm 0.17$ & $59.10 \pm 0.14$        \\
Simple Sum Kernel & $\mathbf{78.12 \pm 0.15}$   & $83.23 \pm 0.18$ & $87.50 \pm 0.20$ & $46.65 \pm 0.06$ & $\mathbf{53.62 \pm 0.19}$ & $\mathbf{59.83 \pm 0.12}$\\
\bottomrule
\end{tabular}
}
\end{table}



\section{More Experiments on Synthetic Data}


Consider a scenario with $n$ clusters, each containing $k$ vertices. Let the probability of vertices $u$ and $v$ from the same cluster belonging to $\bfpi$ be $p$. Conversely, for vertices $u$ and $v$ from different clusters, let the probability of belonging to $\pi$ be $q$. We generate the graph $\bfpi$ randomly, based on $p$ and $q$. We experiment with values of $k=100$ and $n=6$ for ease of visualization, embedding all points in a two-dimensional space. Each vertex's initial position originates from a normal distribution. In each iteration, we sample a subgraph of $\bfpi$ uniformly, ensuring each vertex has an out-degree of $1$. We then optimize the corresponding vectors using InfoNCE loss with an SGD optimizer and iterate until convergence. Our experimental setup consists of an SGD learning rate of $1$, an InfoNCE loss temperature of $0.5$, and a batch size of $50$. We evaluate two scenarios with different $p$ and $q$ values: $p=1$, $q=0$, and $p=0.75$, $q=0.2$. The results of these experiments are visualized in Figure \ref{fig:vis-spectral-cluster}. The obtained embeddings exhibit the hallmark pattern of spectral clustering of graph $\bfpi$.

\begin{figure}[!tb]
\centering
\subfigure{
\includegraphics[width=1\textwidth]{Figures/cluster_pi.png}
\label{fig:vis-cluster}
}
\subfigure{
\includegraphics[width=1\textwidth]{Figures/noised_cluster_pi.png}
\label{fig:vis-noised-cluster}
}
\caption{Visualizations of the optimization process using InfoNCE Loss on the vectors corresponding to $\bfpi$. Points of identical color belong to the same cluster within $\bfpi$. To showcase the internal structure of $\bfpi$, we randomly select 10 vertices from each cluster to display the edge distribution of $\bfpi$.}
\label{fig:vis-spectral-cluster}
\end{figure}



\end{document}
