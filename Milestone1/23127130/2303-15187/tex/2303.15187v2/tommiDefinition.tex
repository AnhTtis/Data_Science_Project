




\linespread{1}







\usepackage[usenames,dvipsnames]{xcolor}
 
 
\usepackage{epsfig} %
\usepackage{graphics} %
\usepackage{amsmath} %
\usepackage{amssymb}  %
\usepackage{cite}	
\usepackage{graphicx}
\usepackage{tabularx}
\usepackage{lipsum}
\usepackage{multirow}
\usepackage{psfrag}
\usepackage{float}
\usepackage{placeins}


\usepackage{algorithm}
\usepackage{algorithmicx}
\usepackage[]{algpseudocode}


\usepackage{subfloat}
\usepackage{subfig}
\usepackage{caption}

\usepackage{soul}

\usepackage[colorlinks = true,
            linkcolor = black,
            urlcolor  = black,
            citecolor = black,
            anchorcolor = black]{hyperref}



\usepackage{bm}
\usepackage{microtype}
\usepackage{wasysym} %
\usepackage{adjustbox}
\usepackage{lipsum}

\let\labelindent\relax %
\usepackage{enumitem}  
\setlist[description]{font=\normalfont\itshape}



\usepackage{siunitx}
\DeclareSIUnit\gauss{G}

\usepackage{eurosym}




\usepackage{pgfplots}
\usetikzlibrary{shapes,arrows,fit,calc}
\usepackage[dvipsnames]{xcolor}
\usepackage{pgfplotstable}
\usetikzlibrary{math} %
\usepackage{tikz}
\pgfplotsset{compat=newest} 
\pgfplotsset{plot coordinates/math parser=false} 
\newlength\figureheight 
\newlength\figurewidth 





\newcommand\T{\rule{0pt}{4mm}}
\newcommand\B{\rule[-2mm]{0pt}{0pt}}
\renewcommand*\arraystretch{1.5}


\usepackage{array}
\newcolumntype{L}[1]{>{\raggedright\let\newline\\\arraybackslash\hspace{0pt}}m{#1}}
\newcolumntype{C}[1]{>{\centering\let\newline\\\arraybackslash\hspace{0pt}}m{#1}}
\newcolumntype{R}[1]{>{\raggedleft\let\newline\\\arraybackslash\hspace{0pt}}m{#1}}
\newcolumntype{M}{>{\centering\let\newline\\\arraybackslash\hspace{0pt}}X} %
\newcolumntype{s}{>{\hsize=.4\hsize}X}
\renewcommand\tabularxcolumn[1]{m{#1}}



\makeatletter
\newcommand{\thickhline}{%
    \noalign {\ifnum 0=`}\fi \hrule height 1pt
    \futurelet \reserved@a \@xhline
}
\newcolumntype{"}{@{\hskip\tabcolsep\vrule width 1pt\hskip\tabcolsep}}
\makeatother


\makeatletter
\makeatother

\algnewcommand\algorithmicinput{\textbf{Initialization:}}
\algnewcommand\init{\item[\algorithmicinput]}

\algnewcommand\algorithmicalib{\textbf{Calibration:}}
\algnewcommand\calib{\item[\algorithmicalib]}


\algnewcommand\algorithmicevol{\textbf{Monitoring:}}
\algnewcommand\evol{\item[\algorithmicevol]}


\algdef{SE}[SUBALG]{Indent}{EndIndent}{}{\algorithmicend\ }%
\algtext*{Indent}
\algtext*{EndIndent}



\DeclareMathOperator{\atantwo}{atan2}
\DeclareMathOperator{\arctantwo}{arctan2}
\DeclareMathOperator*{\argmin}{argmin}

\newcommand*{\MyPath}{.}%

\newcommand{\ie}{{i.e.},~}
\newcommand{\iec}{{i.e.},~}
\newcommand{\ied}{{i.e.}:~}
\newcommand{\eg}{{e.g.}~}
\newcommand{\egc}{{e.g.},~}

\newcommand{\eal}{\textit{et al.}~}
\newcommand{\cfr}{cfr.~}

\newcommand{\fig}{Fig.~}
\newcommand{\figs}{Figs.~}

\newcommand{\tabb}{Tab.~}


\newcommand{\sect}{Sect.~}
\newcommand{\chap}{Chapter~}
\newcommand{\app}{Appendix~}


\newcommand{\dof}{DoF~}
\newcommand{\dofs}{DoFs~}

\newcommand{\phd}{Ph.D}

\newcommand{\TL}[1]{{\color{olive} #1}}
\newcommand{\Tnote}[1]{{\color{blue}\textbf{[~!~]}\footnote{\color{blue}TL: #1}}}
\newcommand{\TLnote}[1]{{\color{blue}\textbf{[~!~]}\footnote{\color{blue}TL: #1}}}

\newcommand{\TLc}[1]{{\color{red}\st{#1}}}
\newcommand{\TLcc}[1]{{\color{red}\textbf{DEL}: #1}}
\newcommand{\TLun}[1]{{\setulcolor{red}\ul{#1}}}

\newcommand{\pg}[1]{\paragraph{\textbf{#1}}\mbox{}}


\definecolor{mycolor1}{rgb}{0.00000,0.44700,0.74100}%
\definecolor{mycolor2}{rgb}{0.85000,0.32500,0.09800}%
\definecolor{mycolor3}{rgb}{0.92900,0.69400,0.12500}
\definecolor{mycolor4}{rgb}{0.290,0.624,0.4}

\definecolor{acrobatYellow}{rgb}{1,1,0}
\definecolor{lightgray}{rgb}{0.83, 0.83, 0.83}
\sethlcolor{lightgray}

\newcommand{\orcid}[1]{\href{https://orcid.org/#1}{\includegraphics[scale=0.5]{img/orcid_16x16.png}}}



