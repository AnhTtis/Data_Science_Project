%%%%%%%%%%%%%%%%%%%%%%%%%%%%%%%%%%%%%%%%%%%%%%%%%%%%%%%%%%%%%%%%%%%%%%%%%%%%
%% Author template for INFORMS Journal on Computing (ijoc)
%% Mirko Janc, Ph.D., INFORMS, mirko.janc@informs.org
%% ver. 0.95, December 2010
%%%%%%%%%%%%%%%%%%%%%%%%%%%%%%%%%%%%%%%%%%%%%%%%%%%%%%%%%%%%%%%%%%%%%%%%%%%%
%\documentclass[ijoc,blindrev]{informs3}
\documentclass[ijoc,nonblindrev]{informs3} % current default for manuscript submission

%%\OneAndAHalfSpacedXI
% \OneAndAHalfSpacedXII % current default line spacing
\DoubleSpacedXII
%%\DoubleSpacedXI

% If hyperref is used, dvi-to-ps driver of choice must be declared as
%   an additional option to the \documentclass. For example
%\documentclass[dvips,ijoc]{informs3}      % if dvips is used
%\documentclass[dvipsone,ijoc]{informs3}   % if dvipsone is used, etc.

% Private macros here (check that there is no clash with the style)

% Natbib setup for author-year style
\usepackage{natbib}
 \bibpunct[, ]{(}{)}{,}{a}{}{,}%
 \def\bibfont{\small}%
 \def\bibsep{\smallskipamount}%
 \def\bibhang{24pt}%
 \def\newblock{\ }%
 \def\BIBand{and}%

\usepackage{enumitem}
 \usepackage{multirow}
 % \usepackage{subcaption}

%% Setup of theorem styles. Outcomment only one. 
%% Preferred default is the first option.
\TheoremsNumberedThrough     % Preferred (Theorem 1, Lemma 1, Theorem 2)
%\TheoremsNumberedByChapter  % (Theorem 1.1, Lema 1.1, Theorem 1.2)

%% Setup of the equation numbering system. Outcomment only one.
%% Preferred default is the first option.
\EquationsNumberedThrough    % Default: (1), (2), ...
%\EquationsNumberedBySection % (1.1), (1.2), ...

% In the reviewing and copyediting stage enter the manuscript number.
%\MANUSCRIPTNO{} % When the article is logged in and DOI assigned to it,
                 %   this manuscript number is no longer necessary

%%%%%%%%%%%%%%%%
\begin{document}
%%%%%%%%%%%%%%%%

% Outcomment only when entries are known. Otherwise leave as is and 
%   default values will be used.
%\setcounter{page}{1}
%\VOLUME{00}%
%\NO{0}%
%\MONTH{Xxxxx}% (month or a similar seasonal id)
%\YEAR{0000}% e.g., 2005
%\FIRSTPAGE{000}%
%\LASTPAGE{000}%
%\SHORTYEAR{00}% shortened year (two-digit)
%\ISSUE{0000} %
%\LONGFIRSTPAGE{0001} %
%\DOI{10.1287/xxxx.0000.0000}%

% Author's names for the running heads
% Sample depending on the number of authors;
\RUNAUTHOR{Zhang}
% \RUNAUTHOR{Jones and Wilson}
% \RUNAUTHOR{Jones, Miller, and Wilson}
% \RUNAUTHOR{Jones et al.} % for four or more authors
% Enter authors following the given pattern:
%\RUNAUTHOR{}

% Title or shortened title suitable for running heads. Sample:
\RUNTITLE{Integer Programming Approaches for Risk-Adjustable DRCCs}
% Enter the (shortened) title:
%\RUNTITLE{}

% Full title. Sample:
% \TITLE{Bundling Information Goods of Decreasing Value}
% Enter the full title:
\TITLE{Integer Programming Approaches for
Distributionally Robust Chance Constraints
with Adjustable Risks                                      
 }
% 
% Block of authors and their affiliations starts here:
% NOTE: Authors with same affiliation, if the order of authors allows, 
%   should be entered in ONE field, separated by a comma. 
%   \EMAIL field can be repeated if more than one author
\ARTICLEAUTHORS{%
\AUTHOR{Yiling Zhang}
\AFF{Industrial and Systems Engineering, University of Minnesota, \EMAIL{yiling@umn.edu}}
% Enter all authors
} % end of the block

\ABSTRACT{%
 % Enter your abstract
We study distributionally robust chance constrained programs (DRCCPs)  with individual chance constraints and random right-hand sides.
The DRCCPs treat the risk tolerances associated with the distributionally robust chance constraints (DRCCs) as decision variables to trade off between the system cost and risk of violations by penalizing the risk tolerances in the objective function.
We consider two types of Wasserstein ambiguity sets: one with finite support and one with a continuum of realizations.
By exploring the hidden discrete structures, we develop mixed integer programming reformulations under the two types of ambiguity sets to determine the optimal risk tolerance for the chance constraint. Valid inequalities are derived to strengthen the formulations.
We test instances with transportation problems of diverse sizes  and a demand response management problem.

% Aggregation of heating, ventilation, and air conditioning (HVAC) loads can provide reserves to absorb volatile renewable energy, especially solar photo-voltaic (PV) generation. In this work, we decide HVAC control schedules under uncertain PV generation, using a distributionally robust chance-constrained (DRCC) building load control model under Wasserstein ambiguity sets. We derive strengthened mixed integer linear programming (MILP) reformulations for DRCC problems with right-hand side uncertainty. To achieve tradeoff between the operational risk and costs, we propose an adjustable chance-constrained variant. MILP and conic reformulations are derived. Using real-world data, we conduct computational studies to demonstrate the efficiency of the solution approaches and the effectiveness of the solutions.

}%

% Sample 
%\KEYWORDS{deterministic inventory theory; infinite linear programming duality; 
%  existence of optimal policies; semi-Markov decision process; cyclic schedule}

% Fill in data. If unknown, outcomment the field
\KEYWORDS{Distributionally robust optimization, Chance-constrained programming, Wasserstein metric, Mixed-integer programming, Reliability, Adjustable risk}
\HISTORY{}

\maketitle
%%%%%%%%%%%%%%%%%%%%%%%%%%%%%%%%%%%%%%%%%%%%%%%%%%%%%%%%%%%%%%%%%%%%%%

% Samples of sectioning (and labeling) in IJOC
% NOTE: (1) \section and \subsection do NOT end with a period
%       (2) \subsubsection and lower need end punctuation
%       (3) capitalization is as shown (title style).
%
	\graphicspath{{figures/}}
 \section{Introduction}
In many planning and operational problems, chance constraints are often used for ensuring the quality of service (QoS) or system reliability. For example, 
chance constraints can be used to restrict the risk of under-utilizing renewable energy in  power systems \citep[e.g.,][]{ma2019distributionally,zhang2022building}, to constrain the risk of loss in portfolio optimization \citep[e.g.,][]{lejeune2016multi}, and to impose the probability of satisfying demand in humanitarian relief networks \citep[e.g.,][]{elcci2018chance}.
% 
% in power systems \citep[e.g.,][]{ma2019distributionally,zhang2022building},  chance constraints can be used to restrict the risk that renewable energy is under-utilized. 
% In portfolio optimization \citep[e.g.,][]{lejeune2016multi}, chance constraints can be used to constrain loss risks. In humanitarian relief \citep[e.g.,][]{elcci2018chance}, chance constraints .
In particular, with a predetermined risk tolerance $\alpha\in[0,1]$, a generic chance constraint is formulated in the following form.
% 
 \begin{equation*}\label{eq:cc}
    \mathbb P_{f}(T(\xi)x\ge q(\xi)) \ge 1-\alpha, 
\end{equation*}
 where $x\in\mathbb R^d$ and the probability of violating the constraint $T(\xi)x\le q(\xi)$ is no more than $\alpha$ with
 a random vector $\xi\in\mathbb R^l$ following distribution distribution $f$. 
The technology matrix is obtained using a function $T:\mathbb R^l \mapsto\mathbb R^{m\times d}$ and the right-hand side is a function $q:\mathbb R^l\mapsto \mathbb R^m$.

 
 When an accurate estimate of the underlying distribution $f$ is not accessible, \emph{distributionally robust optimization} (DRO) provides tools to accommodate incomplete distributional information. Instead of assuming a known underlying distribution, DRO considers a prescribed set $\mathcal D$ of probability distributions, termed as an \emph{ambiguity set}. The distributionally robust variant of chance constraint \eqref{eq:cc} is as follows.
\begin{equation*}\label{eq:drcc}
      \inf_{f\in\mathcal D}\mathbb P_{f}(T(\xi)x\ge q(\xi)) \ge 1-\alpha. 
\end{equation*}
In the 
 distributionally robust chance constraint (DRCC), $\alpha$ represents the worst-case probability of violating constraints $T(\xi)x\ge q(\xi)$ with respect to the ambiguity set $\mathcal D$.
 
 % enforces the probability of violating constraints $T(\xi)x\ge q(\xi)$ no more than $\alpha$  for every distribution within the prescribed ambiguity set $\mathcal D$. 

 In many system planning and operational problems,
a higher value of the probability $1-\alpha$ can lead to potentially better customer satisfaction and/or a lower probability of unfavorable events.
However, a too-large $1-\alpha$  may lead to problem infeasibility and requires additional resources and operational costs \citep[e.g.,][]{ma2019distributionally}.
% 
% However, additional resources and operational costs are usually needed to achieve a high reliability $1-\alpha$.
% For example, SOME EXAMPLES (zhang's work on healthcare operations, portfolio optimization, humanitarian).
% 
% 
To find a proper balance between the cost and reliability objectives, alternatively, in this paper, we consider DRCC problems with an adjustable risk, where
the risk tolerance $\alpha$ is treated as a  variable. 



\subsection{Problem Formulation}
\label{sec:formulaiton}
In particular, with a variable risk tolerance $\alpha$, we consider 
% \begin{equation}\label{eq:adjust-drcc}
%     z_0: = \min_{x\in\mathcal X,\alpha} \left\{ c^\top x + g(\alpha): \ \eqref{eq:drcc}, \ \alpha \in (0,\overline{\alpha}] \right\},
% \end{equation}
% 
\begin{subequations}\label{eq:adjust-drcc}
    \begin{eqnarray}
        z_0: = \min_{x\in\mathcal X,\alpha} && c^\top x + g(\alpha)\\
        \mbox{s.t.} && \inf_{f\in\mathcal D}\mathbb P_{f}(Tx\ge \xi) \ge 1-\alpha \label{eq:drcc}\\
        && \alpha \in [0,\overline{\alpha}], 
    \end{eqnarray}
\end{subequations}
% \begin{equation}\label{eq:drcc}
%       \inf_{f\in\mathcal D}\mathbb P_{f}(Tx\ge \xi) \ge 1-\alpha,
% \end{equation}
% where
% 
 where the technology matrix $T\in\mathbb R^{m\times d}$ and 
 a random right-hand side (RHS) vector $\xi\in\mathbb R^m$.
 % $\alpha$ denotes the worst-case probability of violating the constraint $Tx\le\xi$ with a technology matrix $T\in\mathbb R^{m\times d}$ and 
 % a random right-hand side vector $\xi\in\mathbb R^m$. 
 The risk tolerance $\alpha$ is upper bounded by a parameter $\overline{\alpha}< 1$. 
 The parameter $\bar \alpha$ is predetermined and can be viewed as the most risk of unfavorable events that the decision maker is willing to take. The objective trades off between the system cost $c^\top x$ with $c\in\mathbb R^d$ and the (penalty) cost of allowed violation risk $g(\alpha): \ [0,\bar{\alpha}]\rightarrow \mathbb R_0^+$. The risk cost function $g(\alpha)$ is assumed monotonically increasing in $\alpha$. In this paper, we can assume a linear risk cost function $g(\alpha) = p\alpha$, which, however, is not required in the proposed models and methods.
 % For simplicity, we consider a linear risk cost function $g(\alpha) = p\alpha$.
 In the following, we focus on  individual chance constraints, i.e., $m=1$ with the random RHS $\xi$ following a univariate distribution.  
 % In this case, ADD STH TO EXPLAIN WHY to CONSIDER INDIVIDUAL CHANCE CONSTRAINT.

Another motivation for focusing on   (individual) chance constraints with adjustable risks is that they can provide approximation schemes for joint chance constraints. 
In general, joint chance constraints are significantly harder than individual chance constraints. 
The Bonferroni approximation replaces a joint chance constraint with $m$ individual chance constraints and requires the sum of individual-chance-constraint risk tolerances to be upper bounded by the risk tolerance of the joint chance constraint. Optimizing those risk tolerances of individual chance constraints potentially leads to better approximations \citep[see, e.g.,][]{prekopa2003probabilistic}.

Particularly, in the risk-adjustable DRCC \eqref{eq:drcc}, we consider a Wasserstein ambiguity set $\mathcal D$ constructed as follows.
 Given a series of  $N$ historical data samples $\{\xi^n\}_{n=1}^N$ drawn from  $\mathbb R$, the empirical distribution is constructed as $\mathbb P_0(\tilde \xi = \xi^n) = 1/N, \ n=1,\ldots,N$.
For a positive radius $\epsilon > 0$, the Wasserstein ambiguity set defines a ball around a reference distribution (e.g., the empirical distribution) in the space of probability distributions as follows:
\begin{equation}
    \mathcal D: = \left\{f: \ \mathbb P_f(\tilde \xi \in \mathbb R)=1, \ W(\mathbb P_f,\mathbb P_0) \le \epsilon \right\}.
\end{equation}
The Wasserstein distance is defined as
\begin{equation}
    W(\mathbb P_f, \mathbb P_0) := \inf_{\mathbb Q\sim (\mathbb P_1,\mathbb P_2)}\mathbb E_{\mathbb Q} \left[\|\tilde \xi_1 - \tilde \xi_2 \|_p \right],
\end{equation}
where $\tilde \xi_1$ and $\tilde \xi_2$ are random variables following distribution $\mathbb P_1$ and $\mathbb P_2$, $\mathbb Q\sim (\mathbb P_1, \mathbb P_2)$ denotes a joint distribution of $\tilde \xi_1$ and $\tilde \xi_2$ with marginals $\mathbb P_1$ and $\mathbb P_2$, and $\|\cdot\|_p$ denotes the $p$-norm.

Throughout this paper, we focus on the risk-adjustable DRCC model \eqref{eq:adjust-drcc} with two uncertainty types: 

% (A1) Only the right-hand side vector $\xi$ is random;
\begin{enumerate}[label={A\arabic*}]

    \item \emph{Finite distribution:} The random vector $\xi$ has a finite support. The mass probability of each atom is unknown and allowed to vary.
\label{assump:finite}
% a finite distribution.

\item \emph{Continuous distribution:} The random variable $\xi$ has a continuum (infinite number) of realizations. \label{assump:continuous}
\end{enumerate}


In the rest of the paper, without loss of generality, we assume that the samples are in a non-increasing order: $\xi^1 \ge \xi^2\ge \cdots \ge \xi^N$. 
To exclude trivial special cases, throughout the rest of the paper, we assume that $\epsilon>0$ and $\alpha\in (0,1)$.

The remainder of the paper is organized as follows. Section \ref{sec:lit} reviews the prior work related to risk-adjustable chance constraints and DRCCs. Section \ref{sec:prelim} presents preliminary results regarding individual DRCC with RHS uncertainties. Section \ref{sec:risk-adjustable_form} utilizes hidden discrete structures to derive mixed integer programming reformulations along with valid inequalities to strengthen the mixed integer programming reformulations under the two distributional assumptions \ref{assump:finite} and \ref{assump:continuous}. Section \ref{sec:comp} demonstrates the computational efficacy of the proposed approaches for solving a transportation problem with diverse problem sizes and a demand response management problem. Finally, we draw conclusions in Section \ref{sec:conclusions}.

\subsection{Literature review}
\label{sec:lit}
The idea of using variable risk tolerances can be dated back to \citet{evers1967new}, where they trade off between the cost of charge materials and the probability of not meeting specifications in metal melting furnace operations. It later has wide applications including facility sizing \citep{rengarajan2009estimating}, flexible ramping capacity \citep{wang2018adjustable}, power dispatch \citep{qiu2016data,ma2019distributionally}, portfolio optimization \citep{lejeune2016multi}, humanitarian relief network design \citep{elcci2018chance}, and inventory control problem \citep{rengarajan2013convex}. 

\citet{rengarajan2009estimating, rengarajan2013convex} perform Pareto analyses to seek an efficient frontier for a trade-off between the total investment cost and the probability of disruptions that cause undesirable events. In particular, they require to solve a series of chance-constrained programs for a large number of risk-level $\alpha$ choices. Unlike \citet{rengarajan2009estimating, rengarajan2013convex}, another stream of research treats the risk tolerance $\alpha$ as a decision variable and develops nonparametric approaches to trade off the cost and reliability. 
With only right-hand side uncertainty, \citet{shen2014using} develops a mixed integer linear programming (MILP) reformulation
for individual chance constraints with only RHS uncertainty ($m=1$ in the chance constraint \eqref{eq:cc}) under discrete distributions. Along the same line, \citet{elcci2018chance} propose an alternative MILP reformulation for the same setting using knapsack inequalities.
In the context of joint chance constraints ($m>1$), \citet{lejeune2016multi} use Boolean modeling framework to develop exact reformulations for the case with RHS uncertainty and inner approximations for the case with left-hand side uncertainty. 
All these studies assume known underlying (discrete) probability distributions. In recent work, \citet{zhang2022building}  focus on the distributionally robust variants of the risk-adjustable chance constraints under ambiguity sets with moment constraints and Wasserstein metrics, respectively.  They consider an individual DRCC with RHS uncertainty. For the moment-based ambiguity set, they develop two second-order cone programs (SOCPs) with $\alpha$ in different ranges; for the Wasserstein ambiguity set, they propose an exact MILP reformulation. 
In particular, their results for the Wasserstein ambiguity set require decision variables to be all pure binary to facilitate the linearization of bilinear terms.  Without assuming pure binary decisions, in this paper, we will develop integer approaches for solving the DRCCs with adjustable risks under Wasserstein metrics.


% In this paper, we focus on the WASSERSTEIN METRICS. SHOW AMBIGUITY SETS.
% Recently, there is growing interest in ambiguity sets in the form of a ball  in the space of distributions with respect to the Wasserstein or optimal transport distance, first proposed in PFLUG AND WOZABAL 2007. 

% Both Chen and XIe show an equivalent mixed integer programming formulation to the DRCC. 



% \subsection{Main Contributions}
\paragraph{\textbf{Main Contributions:}}

The main contributions of the paper are three-fold. First, by exploiting the (hidden) discrete structures of the individual DRCC with random RHS, we provide tractable mixed-integer reformulations for risk-adjustable DRCCs using the Wasserstein ambiguity set.
% 
% using the Wasserstein ambiguity set, we equivalently reformulate the risk-adjustable DRCCs as new tractable mixed-integer reformulations  by exploiting the (hidden) discrete structures of the individual DRCC with random RHS. 
Specifically, a MILP reformulation is proposed under the finite distribution assumption, and a mixed-integer second-order cone programming (MISOCP) reformulation is derived under the continuous distribution assumption.
% Under the finite distribution assumption \ref{assump:finite}, relying on the finite restriction of the worst-case Value-at-Risk, we provide a MILP reformulation. Under the continuous distribution assumption \ref{assump:continuous}, a mixed-integer second-order cone programming (MISOCP) reformulation is derived based the order of the samples.
Second, we strengthen the proposed mixed-integer reformulations by deriving valid inequalities by exploring the mixing set structure of the MILP reformulation and submodularity in the MISOCP reformulation.  Third,  extensive numerical studies are conducted to demonstrate the computational efficacy of the proposed solution approaches. 

 
\section{Preliminary Results for DRCC with a Known Risk Tolerance $\alpha$}%with RHS uncertainty}
\label{sec:prelim}


\begin{proposition}[Adapted from Theorem 2 in \citet{chen2022data}]
    For a given risk tolerance $\alpha$, the DRCC \eqref{eq:drcc} is equivalent to 
    \begin{equation}\label{eq:drcc-dist}
         \frac{1}{N}\sum_{n=1}^{\alpha N} \left(Tx-\xi^n\right)^+\ge \epsilon,
    \end{equation}
    % \begin{equation}
    %     \frac{\epsilon}{\alpha} + \text{CVaR}_{1-\alpha} \left[ -\left(Tx-\xi\right)^+ \right] \le 0,
    % \end{equation}
    where $(a)^+ = \max \{a,0\}$ and the summation on the left-hand side is a partial sum for fractional $\alpha N$: $\sum_{n=1}^{\alpha N} k_n = \sum_{n=1}^{\lfloor \alpha N \rfloor}k_n + (\alpha N - \lfloor \alpha N \rfloor) k_{\lfloor\alpha N \rfloor+1}$. 
\end{proposition}


The equivalent formulation \eqref{eq:drcc-dist} has a \emph{water-filling} interpretation as illustrated in Figure \ref{fig:water-filling}. The height of patch $n$ is given by $\xi^n$ and the width is given by $1/N$. The region with a width of $\alpha$ is flooded to a level $t_\alpha$ which uses a total amount of water equal to $\epsilon$. Then the reformulation of DR chance constraint \eqref{eq:drcc-dist} is equivalent to a linear inequality $Tx\ge t_\alpha$. The water level $t_\alpha$ represents the worst-case value-at-risk (VaR):
% 
\begin{equation}\label{eq:t-def}t_\alpha := \inf_v\left\{v: \ \inf_{f\in\mathcal D}\mathbb P(v\ge \xi) \ge 1-\alpha \right\}  = \min_v\left\{ v: \ \frac{1}{N}\sum_{n=1}^{\alpha N} (v-\xi^n)^+\ge \epsilon \right\}.\end{equation}%=\inf_{f\in\mathcal D}\text{VaR}_{1-\alpha} (\xi) $.



\begin{figure}[htb]
	\centering
  \includegraphics[width=0.48\linewidth]{illustration}
  \caption{Illustration of the water-filling interpretation for the partial-sum inequality \eqref{eq:drcc-dist}.}
  \label{fig:water-filling}
\end{figure}

Let $j^*$ be the largest index such that when  the amount of water that fills the region of width $\alpha$ to the level $\xi^{j^*}$ is no less than $\epsilon$. That is, \begin{equation}\label{eq:dist_j}j^*: = \max\left\{j\in\{1,\ldots,N\}: \ \xi^j \ge \frac{N\epsilon + \sum_{n=j+1}^{\alpha N}\xi^n }{N\alpha - j} \right\}.\end{equation}
For example, in Figure \ref{fig:water-filling}, if the water is filled up to the level as $\xi^1$ or $\xi^2$, the amount of water exceeds $\epsilon$. In this example, $j^* = 2$.
% $j^* = 2$ and it is easy to see that the amount of water will exceed $\epsilon$ if it is filled up to the level of $\xi^2$. 
We note that such index $j^*$ may not always exist, i.e., the  problem  \eqref{eq:dist_j} can be infeasible. This happens if the amount of water is strictly less than $\epsilon$ even when the water level reaches $\xi^1$.
% when the water level is $\xi^1$ and the amount of water that fills the region of width $\alpha$ is less than $\epsilon$. 
In this case, one can keep increasing the water level until the amount of water equals $\epsilon$ and let the worst-case VaR $t_\alpha$ equal the water level. Otherwise, the worst-case VaR can be obtained using the  propositions below.
% or until reaching an upper bound of $\xi$. In the following two propositions, we assume such index $j^*$ exists, or equivalently $t\le \xi^1$.

\begin{proposition}[Adapted from Theorem 2 in \citet{ji2021data}]
    When Assumption \ref{assump:finite} holds where the random vector $\xi$ has a finite support with unknown mass probability, the worst-case VaR $t^d_\alpha = \xi^{j^*}.$  
    \label{prop:finite_dis_var}
\end{proposition}

\begin{proposition}[Adapted from Theorem 8 in \citet{ji2021data}]
    When Assumption \ref{assump:continuous} holds where the random vector $\xi$ has a continuum of realizations, the worst-case VaR $$t^c_\alpha = \frac{N\epsilon + \sum_{n=j^*+1}^{\alpha N}\xi^n }{N\alpha - j^*}.$$% where $$j^* = \max\left\{j: \ \xi^j \ge \frac{N\epsilon + \sum_{n=j+1}^{\alpha N}\xi^n }{N\alpha - j} \right\}.$$
    \label{prop:conti_dis_var}
\end{proposition}

It is easy to verify that $t_\alpha^d = \xi^{j^*} \ge t_\alpha^c$ given the definition of the critical index $j^*$ in \eqref{eq:dist_j}.





\section{Risk-Adjustable DRCC}
\label{sec:risk-adjustable_form}
In this section, we develop tractable mixed-integer reformulations for the risk-adjustable DRCC \eqref{eq:adjust-drcc} under the two assumptions of finite distribution \ref{assump:finite} and continuous distribution \ref{assump:continuous}.
First, in Section \ref{sec:relation_finite_continuous}, we provide the relation of the optimal values under the two distribution assumptions. In Section \ref{sec:dominance}, we derive solution dominance under different allowed risk tolerance which later assists to develop tractable reformulations. Then, we 
develop tractable mixed-integer reformulations and valid inequalities under the finite and continuous distribution assumptions in Sections \ref{sec:finite_dist} and \ref{sec:continuous}, respectively.
\subsection{Relation of the Optimal Values under the Two Distribution Assumptions}
\label{sec:relation_finite_continuous}

Consider a function $t(\alpha)$ which maps the risk tolerance $\alpha$ to its corresponding worst-case VaR. If the function is known, the DRCC \eqref{eq:drcc} is equivalent to a linear constraint of $x$.
% If the worst-case VaR $t_\alpha$ is known for any $\alpha \in [0,\bar\alpha]$, which is a function of $\alpha$, 
Thus, the risk-adjustable DRCC problem \eqref{eq:adjust-drcc} is rewritten as follows.
\begin{equation}\label{eq:linear_equ}
    z(t(\alpha)):=\min_{x\in\mathcal X, \alpha\in [0,\bar\alpha]}\left\{ c^\top x+ g(\alpha): \ Tx\ge t(\alpha) \right\},
\end{equation}
where the optimal value depends on the choice of function $t(\alpha)$. Under Assumption \ref{assump:finite} of the finite distribution, let $t^d(\alpha)$ be the worst-case VaR function and  the optimal value of \eqref{eq:linear_equ} be $z^d: = z(t^d(\alpha))$. Similarly, under Assumption \ref{assump:continuous} of the continuum realizations, let $t^c(\alpha)$ be the worst-case VaR function and the optimal value of \eqref{eq:linear_equ} be $z^c: = z(t^c(\alpha))$. 
The next proposition presents the relation between the optimal values with finite and continuous distributions.
% 
% When assuming Assumption A2 of the finite distribution
\begin{proposition}\label{prop:obj}
The risk-adjustable DRCC problem \eqref{eq:adjust-drcc} under the continuous distribution assumption \ref{assump:continuous} yields an optimal value no more than that under the finite distribution assumption \ref{assump:finite}, i.e., $z^d\ge z^c$.
% The solution yielded by the risk-adjustable DRCC problem under the continuous distribution assumption 
% yields more conservative solution    
\end{proposition}
The proposition is an immediate result from the fact that, for a given $\alpha$, $t_\alpha^d\ge t_\alpha^c$.% The proof is thus omitted.
% \begin{proof}{Proof of Proposition \ref{prop:obj}:}
%     Given an optimal solution $(\hat x, \hat \alpha)$ obtained from the risk-adjustable DRCC under the finite support assumption, the solution is feasible for the continuum-realization case. To see this, we need to show $T\hat x\ge t^c(\hat\alpha)$. This is true because $t^d(\hat\alpha) \ge t^c(\hat\alpha)$ according to Propositions \ref{prop:finite_dis_var} and \ref{prop:conti_dis_var}. The proof thus is complete.
%     % we denote the corresponding worst-case VaR to $\bar\alpha$ as $\bar t^d_\alpha$ and $\bar t^d_\alpha$ belongs to the discrete set, i.e., $\bar t^d_\alpha \in \{\xi^1,\ldots,\xi^N\}$. According to Propositions \ref{prop:finite_dis_var} and \ref{prop:conti_dis_var}, 
% \end{proof}


\subsection{Dominance of Risk Tolerance}
\label{sec:dominance}
The chance constrained programming literature \citep[see, e.g.,][]{prekopa1990dual,dentcheva2000concavity,ruszczynski2002probabilistic,prekopa2003probabilistic} defines the concept of \emph{non-dominated points}, or the so-called $p$-efficient points, where $p$ refers to $1-\alpha$ in this paper. 

% Lemma \ref{lemma:VaR_rev} implies a generalization of the concept of \emph{non-dominated points}, or the so-called $p$-efficient points in chance constrained programming literature (CITE RELATED PAPERS), where $p$ refers to $1-\alpha$ in this paper. 

\begin{definition}[$p$-efficient point] \citep{prekopa2003probabilistic,dentcheva2000concavity}
    Let $p\in(0,1)$. A point $v\in\mathbb R^m$ is a $p$-efficient point of the probability distribution $f$, $\mathbb P_f(v) \ge p$ and there is no $w\le v, \ w\neq v$ such that $\mathbb P_f(w) \ge p$.
\end{definition}
 The concept can be extended to the DRO variant in the following definition.
\begin{definition}[Distributionally Robust $p$-efficient point]
    Let $p\in(0,1)$. A point $v\in\mathbb R^m$ is a distributionally robust $p$-efficient point of the ambiguity set $\mathcal D$, $\inf_{f\in\mathcal D}\mathbb P_f(v) \ge p$ and there is no $w\le v, \ w\neq v$ such that $\inf_{f\in\mathcal D}\mathbb P_f(w) \ge p$.
\end{definition}

In the case of individual chance constraint with an uncertain RHS, under the empirical distribution of $\{\xi^n\}_{n=1}^N$, the $(1-\alpha)$-efficient point is the $(1-\alpha)$-quantile (or $(1-\alpha)$-VaR) of the empirical distribution. The distributionally robust $(1-\alpha)$-efficient point coincides with the worst-case VaR $t_\alpha$ (obtained by assuming either the finite distribution or the continuous distribution), which is greater than the $(1-\alpha)$-quantile of the reference distribution in the Wasserstein ball $\mathcal D$. Similar to the $(1-\alpha)$-quantile, the worst-case VaR is nonincreasing in the risk tolerance, which is formally stated in the following proposition.
\begin{proposition}\label{prop:VaR}
        Given $0\le\alpha_1< \alpha_2\le\bar\alpha$, let $t_{\alpha_1}$ and $t_{\alpha_2}$ be the worst-case VaRs associated with $\alpha_1$  and $\alpha_2$, respectively. Then, $t_{\alpha_1} \ge t_{\alpha_2}$.
\end{proposition}
% \begin{lemma}\label{lemma:VaR_rev}
%     Let $t_{\alpha_1}$ and $t_{\alpha_2}$ be the worst-case VaRs associated with $\alpha_1$  and $\alpha_2$, respectively. If $t_{\alpha_1} \ge t_{\alpha_2}$, then $\alpha_1\le \alpha_2$.
% \end{lemma}
The proof can be easily derived based on the water-filling interpretation in Section \ref{sec:prelim} and  is omitted for brevity.

% Under the finite distribution assumption, Proposition \ref{prop:finite_dis_var} and 

% to read papers of p-efficient points


\subsection{Finite Distribution}
\label{sec:finite_dist}


% Lemma \ref{lemma:VaR} shows that the worst-case value-at-risk is nondecreasing in $\alpha$. 

According to Propositions \ref{prop:finite_dis_var} and \ref{prop:conti_dis_var}, the worst-case VaR $t_\alpha^d$ (if exists) under the finite distribution assumption is the smallest $\xi^j$ which is no less than the worst-case VaR $t_\alpha^c$ under the continuous distribution assumption. That is, $t_\alpha^d = \min_{j\in \{1,\ldots,N\}} \left\{\xi^j:\  \xi^j\ge t_\alpha^c \right\}$.
% Under Assumption \ref{assump:finite} of finite distributions, for a given $\alpha$, if the corresponding worst-case VaR $t_\alpha^c$ exists, the worst-case $(1-\alpha)$-efficient point is the smallest scenario $\xi^j$ which is greater than $t_\alpha^c$, i.e., $t_\alpha^d$ equals to the smallest scenario $\xi^j$. 
We thus have the following result, which has already been anticipated in Proposition \ref{prop:finite_dis_var}.
\begin{corollary}\label{coro:finite_dist_lp}
    Under the finite distribution assumption \ref{assump:finite} , for any risk tolerance $\alpha\in(0,1)$ such that $\xi^j \ge t_\alpha^c> \xi^{j+1} $ for some $j\in \{1,\ldots,N\}$, the DRCC $ \inf_{f\in\mathcal D}\mathbb P_{f}(Tx\ge \xi) \ge 1-\alpha$ is equivalent to a linear constraint:
    $$Tx\ge \xi^j.$$
    % Problem \eqref{eq:adjust-drcc}
    % \label{thm:discrete_level}
\end{corollary}
% 
Given any fixed $\alpha_1,\alpha_2\in (0,1)$ such that $\xi^j \ge t_{\alpha_1}^c\ge t_{\alpha_2}^c > \xi^{j+1}$, the DRCCs of the two risk tolerances yield the same linear reformulation $Tx\ge \xi^j$ under the finite distribution assumption. Thus, in the risk-adjustable DRCC problem \eqref{eq:adjust-drcc}, it suffices to strengthen
$\alpha\in(0,\bar\alpha]$ by restricting it to the risk tolerances $\alpha$ such that
the corresponding worst-case VaR $t_\alpha^d\in(0,1)$ belongs to a discrete set:
$$t_\alpha^d\in \{\xi^1,\ldots, \xi^N\}.$$



For each sample $\xi^n, \ n=1,\ldots,N$ in the discrete set, a risk tolerance $\alpha_n$, which achieves the worst-case VaR at $t^d_{\alpha_n} = \xi^n$, can be obtained using a bisection search method as the flooded area is non-decreasing in the risk tolerance (see the water-filling interpretation in Section \ref{eq:adjust-drcc}). %Before providing a tractable formulation for the risk-adjustable DRCC problem, let us briefly discuss how the optimal value changes as more samples are included. Consider two sample sets $\{\xi^1,\ldots,\xi^{N_1}\}$ and $\{\xi^1,\ldots,\xi^{N_2}\}$ such that $\{\xi^1,\ldots,\xi^{N_1}\} \subset \{\xi^1,\ldots,\xi^{N_2}\}$.
% 
% 
We note that when $\xi^n$ is too small, the corresponding risk tolerance may not exist.
In this section, we assume that  the corresponding risk tolerances exist for the first $N^\prime\le N$ largest  samples $\left\{ \xi^n \right\}_{n=1}^{N^\prime}$ and denote their corresponding  risk tolerances by $\alpha_n, \ n=1,\ldots,N^\prime$. 
\begin{theorem}\label{thm:mip_discrete}
Under the finite distribution assumption, the risk-adjustable DRCC problem \eqref{eq:adjust-drcc} is equivalent to the following MILP formulation.
    \begin{subequations}\label{eq:mip_discrete}
\begin{eqnarray}
\label{eq:mip_discrete_obj} z^d = \min_{x\in\mathcal X, y} && c^\top x + \sum_{n=1}^{N^\prime-1}\Delta_n y_n + \Delta_{N^\prime}  \\
\label{eq:mip_discrete_constr1}\mbox{s.t.} && Tx \ge \xi^n - M_n(1-y_n), \ n=1,\ldots,N^\prime-1\\
% \label{eq:mip_discrete_y_order}&& y_{n+1} \ge y_{n}, \ n=1,\ldots,N^\prime -1\\
\label{eq:mip_discrete_alpha}&& \sum_{n=1}^{N^\prime-1}(\alpha_{n}-\alpha_{n+1})y_{n} + \alpha_{N^\prime} \in (0,\bar{\alpha}]\\
\label{eq:mip_discrete_binary}&& y_n\in\{0,1\}, \ n=1,\ldots,N^\prime-1,
\end{eqnarray}
\end{subequations}
where $\Delta_{n}:=g(\alpha_{n}) - g(\alpha_{n+1})\le 0, \ n=1,\ldots,N^\prime-1$, $\Delta_{N^\prime} := g(\alpha_{N^\prime})$ and $M_n$ is a big-M constant. 
\end{theorem}
\begin{proof}{Proof of Theorem \ref{thm:mip_discrete}:}
To see the equivalence, we need to show (1) $z^d\le z_0$ and (2) $z^d \ge z_0$. Recall that $z_0$ is the optimal value of the risk-adjustable DRCC problem \eqref{eq:adjust-drcc}.
\begin{enumerate}
    \item[(1)] $z^d\le z_0$: Given an optimal solution $( x_0, \alpha_0)$ to the risk-adjustable DRCC problem \eqref{eq:adjust-drcc}, we will construct a feasible solution to MILP \eqref{eq:mip_discrete}.  Let $\bar y_n = 1$ if $T x_0 \ge \xi^n$ and $\bar y_n=0$ otherwise, for $ n=1,\ldots,N^\prime$. Then the solution $(x_0,\bar y_{n}, n=1,\ldots,N^\prime)$ satisfy constraints \eqref{eq:mip_discrete_constr1} and \eqref{eq:mip_discrete_binary}.
    % It is obvious that constraints \eqref{eq:mip_discrete_y_order} are satisfied by $\bar y_n, \ n=1,\ldots,N^\prime$ given that the samples of $\xi$ are sorted in non-increasing order. 
    
    Let $j^*$ be the smallest index such that $T x_0 \ge \xi^{j^*}$. We will show that $\alpha_0 = \alpha_{j^*}$.
When $j^* = 1$, $t_{\alpha_0}^d = \xi^1$ and $\alpha_0 = \alpha_1$. When $j^* \ge 2$, we prove by contradiction by assuming two cases (i) $t_{\alpha_0}^d > \xi^{j^*}$ and (ii) $t_{\alpha_0}^d < \xi^{j^*}$. In the first case, $t_{\alpha_0}^d \le \xi^{j^*-1}$. According to Proposition \ref{prop:VaR}, $\alpha_0 > \alpha_{j^*-1}$. Then, $(x_0,\alpha_{j^*-1})$ is feasible to the risk-adjustable DRCC \eqref{eq:adjust-drcc} with a smaller objective value than $z_0$ as the function $g(\alpha)$ is increasing in $\alpha$. In the second case, $\alpha_0 \ge \alpha_{j^*}$ due to Proposition \ref{prop:VaR} and $(x_0,\alpha_{j^*})$ is a feasible solution with a smaller objective than $z_0$ in the risk-adjustable DRCC \eqref{eq:adjust-drcc}. Both cases result in a contradiction to the fact that $z_0$ is the optimal value of the risk-adjustable DRCC \eqref{eq:adjust-drcc}.
    
  %   Then, we must have $t_{\alpha_0} = \xi^{j^*}$. If $\xi^{j^*}< t^d_{\alpha_0}$, $t^d_{\alpha_0}<\xi^{j^*-1}$. 
    
  %   Then, $t_{\alpha_0}^d \ge \xi^{j^*}$ and, moreover,  $t_{\alpha_0}^d = \xi^{j^*}$, i.e., $\alpha_0 = \alpha_{j^*}$.
  %   % $\alpha_0$ is a risk such that $t_{\alpha_0}^d = \xi^{j^*}$, i.e., $\alpha_0 = \alpha_{j^*}$. 
  %   Otherwise, $\xi^{j^*-1}>t^d_{{\alpha_0}} > \xi^{j^*}$. 
  %   % Denote the corresponding risk level for the worst-case VaR $\xi^{j^*}$ as $\tilde\alpha$.
  %   According to Proposition \ref{prop:VaR}, $\alpha_0 > \alpha_{j^*}$.
  % Then, $( x_0,\alpha_0)$ is not optimal to  the risk-adjustable DRCC \eqref{eq:adjust-drcc}.  Becuase the solution $(x_0,\alpha_{j^*})$ is also feasible to \eqref{eq:adjust-drcc} but its objective value is smaller than $z_0$ due to the increasing  function $g(\alpha)$.

  Since $\alpha_0  = \alpha_{j^*}$ and $\alpha_0 \in (0,\bar\alpha]$,
  $\alpha_{j^*} = \sum_{n=2}^{N^\prime}(\alpha_{n-1}-\alpha_n)\bar{y}_{n-1} + \bar{y}_{N^\prime}\alpha_{N^\prime}\in (0,\bar\alpha]$ satisfies constraint \eqref{eq:mip_discrete_alpha}. Solution $(x_0,\bar y)$ is feasible to \eqref{eq:mip_discrete} with $c^\top x_0 + g(\alpha_{j^*}) = z_0$. Thus, $z^d \le z_0$.
    % Consider the worst-case VaR of $\bar\alpha$ such that $\xi^{j^*-1}>t^d_{\bar{\alpha}} > \xi^{j^*}$.
    \item[(2)] $z^d\ge z_0$: Given an optimal solution $(\hat x,\hat y)$ to problem \eqref{eq:mip_discrete}, we construct a feasible solution to the risk-adjustable DRCC problem \eqref{eq:adjust-drcc}. Denote $j$ the smallest index such that $T\hat x \ge \xi^j$, or, equivalently, the smallest index such that $\hat y_j = 1$.
    Let $\hat\alpha = \alpha_j$. It is easy to see that $(\hat x,\hat \alpha)$ is feasible to the risk-adjustable DRCC problem \eqref{eq:adjust-drcc} and its objective value is $c^\top \hat x + g(\alpha_j) = z^d$. So $z^d \ge z_0$.
\end{enumerate}
Combining the two statements above completes the proof.
\end{proof}
\begin{remark}
    The big-M constant $M_n$ is no less than $\xi^n - \xi^{N^\prime}$.
\end{remark}
% \begin{remark}
    % \textcolor{blue}{According to the proof above, for a given solution $\bar x$ to problem \eqref{eq:mip_discrete}, there exists a threshold index $j^*$ such that $T\bar x \ge \xi^{i}$, for any $i\ge j^*$, and $T\bar x < \xi^i$, for any $i<j^*$.} 
    Recall that the non-increasing order of samples: $\xi^1\ge\xi^2\ge\cdots\ge \xi^N$.
     Thus, given an optimal solution $\bar x$ to MILP \eqref{eq:mip_discrete}, there exists a threshold index $j^*$ such that $T\bar x \ge \xi^{i}$, for any $i\ge j^*$, and $T\bar x < \xi^i$, for any $i<j^*$. 
     As the objective coefficient  $\Delta_n, \ n=1,\ldots,N^\prime -1$ in \eqref{eq:mip_discrete_obj} are non-positive, in the optimal solution $(\bar x,\bar y)$, we have $\bar y_n = 1$, for $i\ge j^*$, and $\bar y_n=0$ for $i>j^*$.
     % Thus, the corresponding $\bar y$ binary variables are as $\bar y_n = 1$, for $i\ge j^*$, and $\bar y_n=0$ for $i>j^*$. 
     By exploiting this solution structure, the next proposition presents how 
    the MILP formulation \eqref{eq:mip_discrete} can be strengthened.
% \end{remark}
% Under the finite distribution assumption, Theorem \ref{thm:discrete_level} implies a 
\begin{proposition}\label{prop:finite_strengthening}
~
    \begin{itemize}
        \item[i.] 
The following inequalities are valid for the MILP \eqref{eq:mip_discrete}:        
            \begin{equation}\label{eq:mip_discrete_y_order}
            y_{n+1} \ge y_{n}, \ n=1,\ldots,N^\prime -1.\end{equation}
\item[ii.] The \emph{strengthened star inequality} \citep{luedtke2010integer} is valid for the MILP \eqref{eq:mip_discrete}:
\begin{equation}
\label{eq:mip_discrete_constr_star}Tx\ge \xi^{N^\prime} + \sum_{n=1}^{N^\prime - 1}(\xi^n - \xi^{n+1})y_n.
\end{equation}
    \end{itemize}
\end{proposition}
\begin{proof}{Proof of Proposition \ref{prop:finite_strengthening}:}
    The valid inequalities \eqref{eq:mip_discrete_y_order} follow from the discussion above.
    To see the second statement of the extended star inequalities, we introduce binary variable $z_n = 1-y_n, \ n=1,\ldots,N^\prime-1.$ Without loss of generality, we assume that $\xi^n\ge0$. Constraints \eqref{eq:mip_discrete_constr1} and \eqref{eq:mip_discrete_alpha} lead us to consider a \emph{mixing set} \citep{atamturk2000mixed,gunluk2001mixing, luedtke2010integer}:
    \begin{equation}
        P = \left\{ (t,z)\in\mathbb R_+ \times \{0,1\}^{N^\prime - 1}: \ \sum_{n=1}^{N^\prime-1}(\alpha_{n}-\alpha_{n+1})y_{n} + \alpha_{N^\prime}\le \bar\alpha,  \ t+ z_n \xi^n \ge \xi^n, \ n=1,\ldots,N^\prime \right\}
    \end{equation}
    where $t = Tx$. According to Theorem 2 in \citet{luedtke2010integer}, constraint  \eqref{eq:mip_discrete_constr_star} is face-defining for $\text{conv(P)}$. The proof is complete.
\end{proof}
\begin{remark}
    When the distribution is known, a similar formulation for the stochastic chance-constrained problem can also be derived based on the Sample Average Approximation \citep{luedtke2008sample}. In this case, let $\alpha_n$ be the allowed risk tolerance when the VaR equals $\xi^n$ and $\alpha_n = n/N$.
    % the risk-at value $\alpha_n$  for $\xi^n$ equals to $n/N$.
    The detailed MILP formulation for the stochastic chance-constrained formulation can be found in Appendix  \ref{apx:cc}. We note that the MILP formulation in Appendix  \ref{apx:cc} can be viewed as a hybrid of those in \citet{shen2014using,elcci2018chance}.
\end{remark}
% \textcolor{blue}{Give a proof.}
%     The proofs are omitted as i) the first set of valid inequalities \eqref{eq:mip_discrete_y_order} follows from the analysis before the proposition, and ii) 
%     the mixing inequality \eqref{eq:mip_discrete_constr_star} is the \emph{strengthend star inequalities} by \citet{luedtke2010integer}. 
%     \textcolor{blue}{SAY STH related to facet-defining}
%     \textcolor{blue}{Together with inequalities \eqref{eq:mip_discrete_y_order},} 
%     The inequality \eqref{eq:mip_discrete_constr_star} is facet

%     chance constrained,  MILP,  Appendix \ref{apx:cc}
    
        



\subsection{Continuous Distribution}
\label{sec:continuous}

Unlike the case with finite distributions, under the continuous distribution assumption \ref{assump:continuous}, the worst-case VaR cannot be restricted to a discrete set.

For a given risk tolerance $\alpha$, 
constraint \eqref{eq:drcc-dist} is equivalent to
\begin{equation}\label{eq:soc}
(\alpha N-j)(Tx-\xi^{k+1}) - \sum_{i=j+1}^k(\xi^i-\xi^{k+1}) \ge N\epsilon 
\end{equation}
% the summation in the left-hand side of \eqref{eq:drcc-dist} equals to $1/N\left[(\alpha N-j)(Tx-\xi^{k+1}) - \sum_{i=j+1}^k(\xi^i-\xi^{k+1})\right]$,
where $k = \lfloor \alpha N \rfloor$ and
$j$ is the smallest index such that $Tx - \xi^{j+1} \ge 0$. For instance, in Figure \ref{fig:water-filling}, $j=2$ and $k=5$. When the risk tolerance $\alpha$ is not known, we introduce a binary variable $o_{jk}\in \{0,1\}$ to indicate if $j$ and $k$ are the two critical indices. Denote $\xi^0$ be an upper bound of $\xi$. 
We consider a mild assumption: 
\begin{enumerate}[label={A3}] %[label={A\arabic*}]
    \item For an optimal solution $\hat x$, $T\hat x\ge \xi^N$. That is, the optimal solution is restricted by the smallest realization of $\xi$.  \label{assump:x_stric}
\end{enumerate}
 
\begin{theorem}\label{thm:mip_continuous}
    Under the continuous distribution assumption \ref{assump:continuous} and Assumption \ref{assump:x_stric}, the risk-adjustable DRCC problem is equivalent to the following mixed 0-1 conic formulation.
    \begin{subequations}\label{eq:mip-continuous}
        \begin{eqnarray}
          z^c=  \min_{x\in\mathcal X, o,\alpha,u,w}&& c^\top x + g(\alpha)\\
            \label{eq:mip-continuous_conic}\mbox{s.t.} && uw \ge \sum_{j=0}^{N-1}\sum_{k=j}^{N-1}o_{jk}\sum_{i=j+1}^k(\xi^i-\xi^{k+1}) + N\epsilon \\
            \label{eq:mip-continuous_conic1}&& u\le \alpha N-\sum_{j=0}^{N-1}\sum_{k=j}^{N-1}jo_{jk}\\
\label{eq:mip-continuous_conic2}            && w \le Tx - \sum_{j=0}^{N-1}\sum_{k=j}^{N-1} \xi^{k+1}o_{jk}\\
\label{eq:mip-continuous_Tx}      && \sum_{j=0}^{N-1}\sum_{k=j}^{N-1} \xi^{j}o_{jk} \ge Tx \ge \sum_{j=0}^{N-1}\sum_{k=j}^{N-1} \xi^{j+1}o_{jk}\\
\label{eq:mip-continuous_alphaN}      && \sum_{j=0}^{N-1} \sum_{k=j}^{N-1} (k+1) o_{jk} \ge \alpha N \ge \sum_{j=0}^{N-1} \sum_{k=j}^{N-1} k o_{jk}\\
            \label{eq:mip-continuous_sos1}&& \sum_{j=0}^{N-1}\sum_{k=j}^{N-1} o_{jk}=1\\
            && \alpha \in (0,\bar{\alpha}]\\
            && w\ge 0, \ u\ge 0\\
            && o_{jk}\in \{0,1\}, \ 0\le j\le k \le N-1.
        \end{eqnarray}
    \end{subequations}
\end{theorem}
\begin{proof}{Proof of Theorem \ref{thm:mip_continuous}:}
To establish the equivalence, we first show that $z^c\le z_0$ by constructing a feasible solution to problem \eqref{eq:mip-continuous} given an optimal solution to the risk-adjustable DRCC problem \eqref{eq:adjust-drcc}. Let $(x_0,\alpha_0)$ be an optimal solution to \eqref{eq:adjust-drcc}. Denote $k^* = \lfloor \alpha_0 N \rfloor$ and $j^*$ as the smallest index such that $Tx_0 - \xi^{j^*+1}\ge 0$. Let $\bar o_{j^*k^*} = 1$, $\bar o_{jk}=0, \ j\neq j^*, \ k\neq k^*, \ 0\le j\le k\le N-1$, $\bar u = \alpha_0N+1-j^*$, and $\bar w=Tx_0 - \xi^{k^*+1}$. 
It is easy to verify that $(x_0,\bar o, \alpha_0,\bar u,\bar w)$ is a feasible solution and its objective value equals $z_0$. Thus, $z^c\le z_0$.

To see the opposite direction
$z^c \ge z_0$, consider an optimal solution $(\hat x,\hat o,\hat \alpha,\hat u,\hat w)$ to problem \eqref{eq:mip-continuous}. Since $\hat o$ is feasible, there exists $\hat o_{\hat{j}\hat{k}}$ such that $\hat o_{\hat{j}\hat{k}}=1$ and $\hat o_{jk}=0, \ j\neq\hat j,\ k\neq \hat k$.
Combining constraints \eqref{eq:mip-continuous_conic}--\eqref{eq:mip-continuous_conic2}, we obtain 
\begin{equation}\label{eq:mip-continuous_conic_eq}
(\hat\alpha N+1-\hat j)(T\hat x-\xi^{\hat k+1}) - \sum_{i=\hat j+1}^{\hat k}(\xi^i-\xi^{\hat k+1})\ge N\epsilon.\end{equation} 
Constraints \eqref{eq:mip-continuous_Tx} and \eqref{eq:mip-continuous_alphaN} are equivalent to  
\begin{equation}\label{eq:mip-continuous_Tx_alphaN}
\xi^{\hat j }\ge T\hat x \ge \xi^{\hat j +1} \text{ and } \hat k+1\ge \alpha N \ge \hat k,    
\end{equation}
respectively.
Constraints \eqref{eq:mip-continuous_conic_eq} and \eqref{eq:mip-continuous_Tx_alphaN} imply that $(\hat x,\hat \alpha)$ satisfies the DR chance constraint \eqref{eq:drcc}. Thus, $(\hat x,\hat \alpha)$ is feasible to the risk-adjustable DRCC problem \eqref{eq:adjust-drcc} and $z^c\ge z_0$ as expected.
\end{proof}
\begin{remark}
The mixed 0-1 conic reformulation \eqref{eq:mip-continuous} consists of $(N^2-N)/2$ (additional) binary variables and two continuous variables.
When the decision $x\in\mathcal X\subset \{0,1\}^d$ is restricted to binary variables, under the continuous distribution assumption, \citet{zhang2022building} propose a MILP formulation (details are in Appendix \ref{apx:milp}) by linearizing bilinear terms in the quadratic constraint \eqref{eq:soc} using McCormick inequalities \citep[see, e.g.,][]{mccormick1976computability}. 
In addition to $(N^2-N)/2$  binary variables as those in the conic reformulatio \eqref{eq:mip-continuous},  the linearization  introduces $(N^2-N)(2d+1)$ continuous variables, where $d$ is the dimension of $x$. The MILP reformulation usually does not scale well when the problem size grows, partly due to the weaker relaxations caused by the big-M type constraints, and also due to a larger number of added variables and constraints. We will later show the computational comparison in Section \ref{sec:comp_continuous}.

    
\end{remark}


% \begin{remark}
    In the mixed 0-1 conic reformulation \eqref{eq:mip-continuous}, there is a rotated conic quadratic mixed 0-1 constraint \eqref{eq:mip-continuous_conic}. 
    Although the resulting mixed-integer conic reformulation can be directly solved by optimization solvers, mixed 0-1 conic programs are often time-consuming to solve, mainly due to the binary restrictions.
    % 
    % Due to the difficulties for optimization solvers to handle the nonlinear convex relaxations,  mixed integer conic programs are often time-consuming to solve.
    % the {mixed integer conic formulation \eqref{eq:mip-continuous}} may fail to adequately process the root node in larger-sized instances. 
    In the following, we will develop valid inequalities for the mixed 0-1 conic reformulation \eqref{eq:mip-continuous} to help accelerate the branch-and-cut algorithm for solving \eqref{eq:mip-continuous}. Specifically, we explore the \emph{submodularity} structure of constraint \eqref{eq:mip-continuous_conic} as follows.
    % and a branch-and-cut algorithm to solve \eqref{eq:mip-continuous} efficiently.
% \end{remark}

We first note that constraint \eqref{eq:mip-continuous_conic} can be rewritten in the following form 
\begin{equation}
\label{eq:mip-continuous_conic_general}
    \sigma + \sum_{j=0}^{N-1}\sum_{k=j}^{N-1}d_{jk}o_{jk} \le uw,
\end{equation}
where $\sigma = N\epsilon>0$ and $d_{jk} = \sum_{i=j}^k(\xi^i-\xi^{k+1})\ge 0, \ j=0,\ldots,N-1, \ k=j,\ldots,N-1$.
% 
By introducing auxiliary variable $\tau\ge 0$, constraint \eqref{eq:mip-continuous_conic_general} is equivalent to 
\begin{subequations}
\label{eq:mip-continuous_conic_general_soc}
\begin{eqnarray}
\label{eq:mip-continuous_conic_general_soc1}
    &&\sqrt{\sigma + \sum_{j=0}^{N-1}\sum_{k=j}^{N-1}d_{jk}o_{jk}} \le \tau\\
  \label{eq:mip-continuous_conic_general_soc2}  && \sqrt{\tau^2 + (w-u)^2} \le w+u.
\end{eqnarray}
\end{subequations}
The two inequalities \eqref{eq:mip-continuous_conic_general_soc1}--\eqref{eq:mip-continuous_conic_general_soc2} above are two second-order conic (SOC) constraints.   In particular, %the first constraint \eqref{eq:mip-continuous_conic_general_soc1} involves 0-1 variables, 
 the convex hull of the first constraint \eqref{eq:mip-continuous_conic_general_soc1} can be fully described utilizing extended polymatroid inequalities as the left-hand side of constraint \eqref{eq:mip-continuous_conic_general_soc1} is a \emph{submodular} function \citep[see, e.g.,][]{atamturk2008polymatroids,atamturk2020submodularity}.
\begin{definition}[Submodular Function] Define the collection of set $[(N^2-N)/2]$'s subsets $\mathcal C:= \{S: \ \forall S\subset [(N^2-N)/2] \}$. 
Given a set function $g$: $\mathcal C\rightarrow \mathbb R$, g is submodular if and only if
\begin{equation*}
    g\left(S \cup  \{j\}\right) - g(S) \ge g\left(R \cup \{j\} \right) - g\left( R\right),
\end{equation*}
for all subsets $S\subset R \subset \mathcal C$ and all elements $j\in \mathcal C\backslash R.$
\end{definition}
We use $g(S)$ and $g(o)$ interchangeably, where $o\in\{0,1 \}^{(N^2-N)/2}$ denotes the indicating vector of $S\subset\mathcal C$, i.e., $o_s=1$ if $s\in S$ and $o_s=0$ otherwise. The left-hand side of constraint \eqref{eq:mip-continuous_conic_general_soc1}, $h(o) :=\sqrt{\sigma + \sum_{j=0}^{N-1}\sum_{k=j}^{N-1}d_{jk}o_{jk}} $ is a submodular function, where $o$ is a one dimensional vector consisting of $o_{jk}, \ 0\le j\le k\le N-1$. 

\begin{definition}[Extended Polymatroid]
For a submodular function $g(S)$, the polyhedron
$$EP_g = \left\{\pi\in\mathbb R^{(N^2-N)/2}: \ \pi(S)\le g(S), \ \forall S\subset \mathcal C \right\}$$
is called an extended polymatroid associated with $g$, where $\pi(S) = \sum_{i\in S}\pi_i$.
\end{definition}

For submodular function $h$, 
linear inequality 
\begin{equation}
    \pi^\top o \le z\label{eq:extended_polymatroid_ineq}
\end{equation}
is valid for the convex hull of the epigraph of $h$, i.e., conv$\{(o,z)\in\{0,1\}^{(N^2-N)/2}\times\mathbb R: \ z\ge h(o) \}$, if and only if $\pi$ is in the extended polymatroid, i.e., $\pi\in EP_h$ \citep[see][]{atamturk2008polymatroids}. The inequality \eqref{eq:extended_polymatroid_ineq} is called \emph{extended polymatroid inequality}. 

Although it suffices to only impose the extended polymatroid inequality at the extreme points of the extended polymatroid $EP_h$, there are an exponential number of them. Instead of adding all of them to the formulation  \eqref{eq:mip-continuous}, one can add them as needed in a branch-and-cut algorithm. Moreover,  the separation of the valid inequality \eqref{eq:extended_polymatroid_ineq} can be done efficiently using a $O(n\log{n})$ time greedy algorithm as follows.
% Given a solution $(\hat o,\hat z)\in [0,1]^{N^2}\times \mathbb R_+$, 
% The extreme points of the extended polymatrioid $EP_h$ can be separated efficiently in an $O(n\log{n})$ time greedy algorithm as follows. 
Given a solution $(\hat o,\hat z)\in [0,1]^{(N^2-N)/2}\times\mathbb R_+$, one can obtain a permutation $\{(1),\ldots,(N^2)\}$ such that the elements of $o$ are sorted in a non-increasing order, $o_{(1)}\ge \ldots \ge o_{(N^2-N)/2}$. Let $S_{(i)} := \{(1),\ldots, (i) \}, \ i=1,\ldots, (N^2-N)/2$. Calculate $\hat\pi_{(1)} = h(S_{(1)})$ and  $\hat\pi_{(i)} = h(S_{(i)}) - h(S_{(i-1)}), \ i=2,\ldots, (N^2-N)/2$.  If $\hat \pi^\top o\le \hat z$, the current solution $(\hat o,\hat z)$ is optimal; otherwise, generate a valid inequality $\hat \pi^\top o \le z$.


% for $\pi\in EP_g$, inequality 
% \begin{equation}
%     \pi^\top o \le z
% \end{equation}
% is called an extended polymatroid inequality. The extended polymatroid inequality is valid for the epigraph of $g$, i.e., $\{(o,z)\in\{0,1\}^{N^2}\times\mathbb R: \ z\ge g(o) \}$.
% Consider the set 
% $$H = \left\{(o,z)\in \{0,1\}^n \right\}$$
% Submodular functions exhibit a natural diminishing-returns property: the marginal gain of incorporating any additional element $j$ is non-increasing in the subset $S$. 

% \subsection{Benchemark Bisection Search Method}

% The worst-case VaR $t^d_\alpha$ for the discretely distributed random vector $\xi$ can be obtained using the worst-case VaR $t^c_\alpha$ of the continuous realizations. That is, $t^d_\alpha = \min_{j=1,\ldots,N}\{\xi^j: \ \xi^j \ge t^c_\alpha\}$.  To find the worst-case VaR $t^c_\alpha$ of the continuous case, one can use a bisection search method \citep[see][]{ji2021data}  given that the flooded region is monotonically increasing as the water level increases. Alternatively, our prior work \citet{zhang2022building}  introduces a mixed-integer linear programming (MILP) approach to solve for $t_\alpha$. The benefits of employing the MILP approach over the bisection search method are (1) by utilizing powerful state-of-the-art MILP solvers, the MILP model can be solved efficiently especially when the risk tolerance $\alpha$ is small; and (2) the MILP model can be extended to providing tractable reformulation for the risk-adjustable DR chance constraint under discrete distributions as in Section XXXX.

% % solving the following mixed integer linear programming (MILP) problem returns  $t^d = v^*$, where $v^*$ is the optimal solution.% of \eqref{eq:t-milp}.
% % The benefit of introducing the MILP model is two-fold: first, 
% The MILP problem, whose optimal solution $v^*$ gives value to $t_\alpha$, is formulated as follows.%
% % 
% \begin{subequations}\label{eq:t-milp}
%     \begin{eqnarray}
%        t_\alpha= \min && v\\
%         \mbox{s.t.}&& a_n \le v-\xi^n + M_n^1(1-y_n), \ n=1,\ldots,\lceil \alpha N \rceil\\
%         && a_n \le  M_n^2y_n, \ n=1,\ldots,\lceil \alpha N \rceil\\
%         && \sum_{n=1}^{\alpha N} a_n \ge N\epsilon\\
%         && y_n\in\{0,1\}, \ n=1,\ldots,\lceil \alpha N\rceil,
%     \end{eqnarray}
% \end{subequations}
% where $M_n^1$ and $M_n^2$ are  positive and sufficiently large constants, e.g., $M_n^1 = \xi^n - \xi^{\lceil \alpha N \rceil}$ and $M_n^2 = \xi^1 -\xi^n$. To see the equivalence between the optimal value of \eqref{eq:t-milp} and $t_\alpha$, consider a solution $\hat v$, (1) $a_n = \hat v-\xi^n, \ y_n = 1$ if $\hat v - \xi^n \ge 0 $, and (2) $a_n = 0, \ y_n=0$ if $ \hat v- \xi^n < 0$. Combining the two cases, $a_n=(v-\xi^n)^+$ and the equivalence is established. %Then the equivalence is estabilished between \eqref{eq:t-milp} and the second minimization problem in \eqref{eq:t-def} .
% % $k=\lfloor \alpha N\rfloor.$



\section{Computational Study}
\label{sec:comp}

In the computational study, we demonstrate the computational effectiveness of the proposed mixed integer programming formulations (with both discrete and continuous distributions) on instances of a DRCC counterpart of the transportation problem with random demand \citep{luedtke2010integer,elcci2018chance}. 
% When the decisions are restricted to binary under the continuous distribution assumption, \citet{zhang2022building} propose a MILP formulation (details are in Appendix \ref{apx:milp}) by linearizing bilinear terms in the quadratic constraint \eqref{eq:soc} using McCormick inequalities \citep[see, e.g.,][]{mccormick1976computability}. Thus, 
For continuous distributions, 
 we also compute instances of demand response management using building load where the decisions are pure binary to compare the alternative MILP (which can be found in Appendix \ref{apx:milp}) proposed in \citet{zhang2022building} and our proposed mixed 0-1 conic reformulation.
 % and compare them using the MILP-based formulation (which is given in Appendix \ref{apx:milp}) proposed in \citet{zhang2022building}.
In Section \ref{sec:comp_setup}, we describe the instance setup of the transportation problem and the demand response management problem. There are mainly two parts of results: (1) the computational performance (with CPU time, optimality gap, etc) of the different risk-adjustable DRCC models
in Section \ref{sec:cpu_gaps}, and (2) the solution details given by the models in Section \ref{sec:comp_sol}. In particular, Section \ref{sec:cpu_gaps} demonstrates the computational efficacy of the proposed mixed integer formulations and valid inequalities. Section \ref{sec:comp_sol} shows that the risk-adjustable DRCC following the finite distribution assumption \ref{assump:finite} provides the highest objective values compared to  the risk-adjustable DRCC under the continuous distribution assumption \ref{assump:continuous} and the stochastic chance-constrained counterpart (which is presented in Appendix  \ref{apx:cc}).

% we report the computational performance (with CPU time, optimality gap, etc) of the different risk-adjustable DRCC models.

% In Section XXX, blablabla..... blablabla...

% \ref{sec:comp_setup}: setup
% \ref{sec:cpu_gaps}: cpu time and gaps
% \ref{sec:comp_sol}: solution details, opt val, allowed risk tolerance

\subsection{Computational Setup}
\label{sec:comp_setup}
\noindent\paragraph{Transportation problem:}
There are $I$ suppliers  and $D$ customers. The suppliers have limited capacity $M_i, \ i=1,\ldots, I$. There occurs a transportation cost $c_{ij}$ for shipping one unit from supplier $i$ to customer $j$. The customer demands $\tilde \xi_j , \ j=1,\ldots,D$ are random. Let $f_j$ denote the distribution of $\tilde \xi_j$ and $\mathcal D_j$ be the Wasserstein ambiguity set regarding the distribution $f_j$. % of $\tilde \xi_j$.
 With a penalty cost $p$ of risk tolerance $\alpha_j$ for every customer $j$, the risk-adjustable DRCC transportation problem is formulated as follows.
\begin{equation}
    \min_{x\in\mathcal X,\alpha} \left\{\sum_{i=1}^I\sum_{j=1}^D c_{ij}x_{ij} +   p\sum_{j=1}^D \alpha_j: \inf_{f_j\in\mathcal D_j}\mathbb P_{f_j}(\sum_{i=1}^I x_{ij} \ge \tilde{\xi}_j) \ge 1-\alpha_j, \ 0\le \alpha_j\le \bar{\alpha}, \ j=1,\ldots,D\right\},
\end{equation}
where $\mathcal X:= \{x\in\mathbb R_+^{I\times D}: \ \sum_{j=1}^D x_{ij}\le M_i, \ i=1,\ldots,I\}$. %In the objective function, $p$ is a penalty cost of risk violation.
Following \citet{elcci2018chance}, the risk threshold's upper bound $\bar\alpha$ is set to 0.3 and $p$ is set to $10^6$. To break symmetry, a random perturbation is added to the penalty cost $p$ following a uniform distribution on the interval $[0,100]$ for every $\alpha_j, \ j=1,\ldots,D$. The radius of the Wasserstein ball $\mathcal D_j$ is 0.05.
For other parameters (i.e., $c_{ij}, M_i$ and samples of $\tilde\xi_j$), we use the data sets with $I=40$ suppliers in \citet{luedtke2010integer} with equal probabilities for all samples.

\noindent\paragraph{Building load control problem:} There is an aggregate HVAC (i.e., heating, ventilation, and air conditioning) load of $n$ buildings to absorb random local solar photovoltaic (PV) generation $\tilde P^\text{PV}_t$ over $T$ time periods throughout the day. %Solar PV generation generally cannot adjust its output with high frequent fluctuations. 
Let $\mathcal D_t$ be the Wasserstein ambiguity regarding the distribution $f$ of PV generation $\tilde P^\text{PV}_t$ during period $t$.
% For the building load control problem, there are $n$ buildings equipped with HVAC (i.e., heating, ventilation, and air conditioning) systems. Consider a planning horizon of $T$ time periods. 
For each time period $t$, we solve the following risk-adjustable DRCC formulation for deciding   the room temperature $x_{t,\ell}$  and  HVAC ON/OFF decision $u_{t,\ell}$ of building $\ell$. 
\begin{equation}\label{eq:blc}
    \min_{(x,u)\in\mathcal X_t}\left\{ c_\text{sys}\sum_{\ell=1}^n |x_{t,\ell}-x_\text{ref}| + c_\text{switch}\sum_{\ell=1}^n u_{t,\ell} + p\alpha_t: \ \inf_{f\in\mathcal D_t}\mathbb P_f \left(\sum_{\ell=1}^n P_\ell u_{t,\ell}\ge \tilde P_t^\text{PV} \right)\ge 1-\alpha_t, \ 0\le\alpha_t\le\bar\alpha \right\},
\end{equation}
where $\mathcal X_t = \left\{(x_t,u_{t})\in\mathbb R^n\times\{0,1\}^n: \  x_{t,\ell} = A_\ell x_{t-1,\ell} + B_\ell u_{t,\ell} + G_\ell v_\ell, \ x_\text{min}\le x_{t,\ell}\le x_\text{max},  \ \ell = 1,\ldots,n \right\}$.
The objective minimizes the cost of (1) the user's discomfort (indicated by the room temperature deviation from the set-point $x_\text{ref}$), (2) switching cycles, and (3) risk violation of PV tracking. The DRCC  ensures that PV generation is absorbed by the HCAC fleet with probability $1-\alpha_t$. The indoor temperature $x_t$ and binary ON/OFF decision $u_t$ need to satisfy the constraints of temperature comfort band and thermal dynamics in the feasible set $\mathcal X_t$. The parameters $A_\ell,B_\ell,G_\ell$ are obtained from the building's thermal resistances, thermal capacity, and cooling capacity. Parameter $v_\ell$ is a given system disturbance. 
We use all the parameters and data following \citet{zhang2022building}. 
In particular, the radius of the Wasserstein ball $\mathcal D_t$ is 0.02. 
To solve the ON/OFF decisions for a planning horizon of $T=53$ periods throughout the day, one needs to sequentially solve $53$ problems in the form of
\eqref{eq:blc}, one for each period. 



The computations are conducted on 
 a Windows 10 Pro machine with Intel(R) Core(TM) i7-8700 CPU3.20 GHz and 16 GB memory.
All models and algorithms are implemented
in Python 3.7.6 using Gurobi 10.0.1. 
The Gurobi default settings are used for optimizing all integer formulations except for the mixed integer conic formulation \eqref{eq:mip-continuous}, for which the Gurobi parameter \texttt{MIPFocus} is set to 3.
When implementing the branch-and-cut algorithm, we add the violated extended polymatroid inequalities using Gurobi \texttt{callback} class by \texttt{Model.cbLazy()} for integer solutions. For all the nodes in the branch-and-bound tree, we generate violated cuts at each node as long as any exists. The optimality gap tolerance is default as $10^{-4}$. The time limit  is set to 1800 seconds for computing the transportation problem instances and 100 seconds for solving the building load control problem in one period.
%. We also set the threshold for identifying violated cuts as 10−4, and set the time limit for computing each instance as 3600 seconds.




\subsection{CPU and Optimality Gaps}
\label{sec:cpu_gaps}
% We solve the MILP and MISOCP reformulations with instances of the transportation problem. We also 

Under the finite distribution assumption \ref{assump:finite}, we solve the MILP \eqref{eq:mip_discrete} with and without valid inequalities in Proposition \ref{prop:finite_strengthening}. 
% 
Under the continuous distribution assumption \ref{assump:continuous}, the mixed 0-1 conic formulation \eqref{eq:mip-continuous} can be rewritten as a mixed 0-1 second-order cone programming (SOCP) formulation if constraint  \eqref{eq:mip-continuous_conic_general} is replaced by constraints \eqref{eq:mip-continuous_conic_general_soc}. We solve the mixed 0-1 SOCP reformulation with and without valid the extended polymatroid inequalities. With only binary decisions, we also compare the mixed 0-1 SOCP reformulation with the alternative MILP reformulation in Appendix \ref{apx:milp}. Our valid inequalities significantly reduce the solution time of directly solving the mixed integer models in Gurobi.
% 
% Under the continuous distribution assumption \ref{assump:continuous}, by replacing
% we solve 
% 
% We solve the MILP and MISOCP reformulations with and without the strengthening techniques in Proposition \ref{prop:finite_strengthening}. The strengthening techniques significantly reduce the solution time of solving the MILP and MISCOP models in Gurobi. 
The details are presented as follows.

\subsubsection{Finite distributions}

We first optimize transportation problem instances with the finite distribution model using the 
 MILP reformulation with and without strengthening techniques proposed in Proposition \ref{prop:finite_strengthening}. 
 % 
 % The strengthening techniques significantly reduce the solution time of solving the MILP models in Gurobi. The details are presented as follows in 
 Table \ref{tab:discrete_cpu} presents the CPU time (in seconds), 
 ``\textbf{Opt. Gap}'' as the optimality gap, and  `\textbf{`Node}'' as the total number of branching nodes.
 The CPU time includes the preprocessing time $t_\text{BS}$ for calculating the violation risk $\alpha_n$ corresponding to every sample $\xi^n, \ n=1,\ldots,N$ using the bisection search method, and the time $t_\text{MILP}$ for solving the MILP reformulation \eqref{eq:mip_discrete} using Gurobi. 
 In Table \ref{tab:discrete_cpu}, we solve the transportation problem with $J\in \{100,200\}$ customer demands with $N = \{50,100,200,1000,2000,3000\}$ samples. For each $(J,N)$ setting, five instances are solved. Table \ref{tab:discrete_cpu} presents the average CPU times, the average optimality gaps, and the average number of branching nodes.  Details of each instance can be found in Appendix \ref{apx:finite_cput}.
 
 In Table \ref{tab:discrete_cpu}, with valid inequalities proposed in Proposition \ref{prop:finite_strengthening}, all the instances are solved optimally at the root node within the time limit (thus optimality gap is zero and omitted). Whereas, if being solved without the valid inequalities, the instances of more samples $(N\ge 1000)$ cannot be solved within the 18000-second time limit and ends with an  optimality gap up to 5.47\%. For larger-sized problems, solving the MILP  \eqref{eq:mip_discrete} with valid inequalities is much faster than solving the MILP  \eqref{eq:mip_discrete} directly due to the strength of the strengthened star inequality \eqref{eq:mip_discrete_constr_star}. With the valid inequalities, most of the CPU time spends on preprocessing ($t_\text{BS}$). %\textcolor{blue}{comment on the strong formulation}
  
\begin{table}[htbp]
  \centering
  \caption{Comparison of CPU time (in seconds) and optimality gaps of finite distributions}
    \resizebox{.7\textwidth}{!}{
  \begin{tabular}{@{\extracolsep{4pt}}ccrrrrrrrrr@{}}
%\begin{tabular}{rrrrrrrrrrr}
	\hline
	\multicolumn{1}{c}{\multirow{2}[4]{*}{Demand}} & \multicolumn{1}{c}{\multirow{2}[4]{*}{$N$}} & \multicolumn{4}{c}{MILP + Valid Ineq.} &       & \multicolumn{4}{c}{MILP} \\
	\cline{3-6}\cline{7-11}          &       & \multicolumn{1}{c}{$t_\text{BS}$} & \multicolumn{1}{c}{$t_\text{MILP}$} & \multicolumn{1}{c}{Time} & \multicolumn{1}{c}{Node} & \multicolumn{1}{c}{$t_\text{BS}$} & \multicolumn{1}{c}{$t_\text{MILP}$} & \multicolumn{1}{c}{Time} & \multicolumn{1}{c}{Opt. Gap} & \multicolumn{1}{c}{Node} \\
	\hline
	100   & 50    & 0.00  & 0.04  & 0.04  & 1     & 0.01  & 0.31  & 0.31  & N/A   & 1 \\
	100   & 100   & 0.02  & 0.07  & 0.09  & 1     & 0.02  & 1.02  & 1.04  & N/A   & 158 \\
	100   & 200   & 0.07  & 0.09  & 0.15  & 1     & 0.06  & 6.45  & 6.52  & N/A   & 2018 \\
	100   & 1000  & 1.26  & 0.33  & 1.60  & 1     & {1.21} & {LIMIT} & {LIMIT} & {0.09\%} & {33454} \\
	100   & 2000  & 4.67  & 0.72  & 5.39  & 1     & {4.64} & {LIMIT} & {LIMIT} & {0.65\%} & {9022} \\
	200   & 2000  & 9.18  & 1.60  & 10.79 & 1     & {9.39} & {LIMIT} & {LIMIT} & {2.58\%} & {6678} \\
	200   & 3000  & 20.37 & 2.38  & 22.75 & 1     & {20.59} & {LIMIT} & {LIMIT} & {5.47\%} & {6083} \\
	\hline
\end{tabular}
    }
  \label{tab:discrete_cpu}%
\end{table}%

\subsubsection{Continuous distributions}
\label{sec:comp_continuous}
We first focus on the computational performance of solving the building load control problem with the binary decision. We use the proposed mixed 0-1 SOCP formulation (``\textbf{MISOCP}'') and the alternative MILP formulation (``\textbf{MILP-Binary}'') from \citet{zhang2022building}. 
In particular, the MISOCP is obtained by replacing the rotated conic constraint \eqref{eq:mip-continuous_conic} with  the SOC constraints \eqref{eq:mip-continuous_conic_general_soc1}-\eqref{eq:mip-continuous_conic_general_soc2}. 
The left-hand side function $h(o)$ of \eqref{eq:mip-continuous_conic_general_soc1} is submodular and thus the extended polymatroid inequalities \eqref{eq:extended_polymatroid_ineq} is added in a branch-and-cut (``\textbf{B\&C}'') algorithm when being violated.  

Table \ref{tab:continuous_milp_socp} reports, for each instance, the total CPU time of solving the building load control problem \eqref{eq:blc} for all 53 periods. If for any period, the problem cannot be solved within the time limit, we report ``\textbf{\#LIMIT}'' as the number of periods which cannot be solved, and ``\textbf{Avg. Gap}'' as their average optimality gap. 
% Due to the stronger relaxations and fewer added variables and constraints 
Owing to its stronger relaxations and fewer variables, the MISOCP (B\&C) quickly solves all the instances, with an average of only 1.2 seconds per instance.
The optimality gaps are all zeros and thus not reported in the table. In contrast, MILP-Binary fails to be solved within the 100-second time limit for each period, with an average of 17 periods not solved to optimal. 

% Solving the MISCOP using the branch-and-cut algorithm is much faster and takes only about 1 second \textcolor{blue}{(due to the strong relaxation of MISOCP)}.

%  However, due to the robust relaxation of the MISOCP, its resolution via the B&C algorithm is significantly faster, with an average of only 1 second per instance.

 

% Table generated by Excel2LaTeX from sheet 'continuous'


% Table generated by Excel2LaTeX from sheet 'BLC-binary'
\begin{table}[htbp]
  \centering
  \caption{Comparison of CPU time (in seconds) and optimality gaps of continuous distributions with binary variables}

  \resizebox{.7\textwidth}{!}{
      \begin{tabular}{@{\extracolsep{4pt}}cccccccc@{}}
% 
  % 
    % \begin{tabular}{lccccccc}
    \hline
          & \multicolumn{4}{c}{MILP-Binary}      & \multicolumn{3}{c}{MISOCP (B\&C)} \\
\cline{2-5} \cline{6-8}    Instance & \multicolumn{1}{l}{Time} & \multicolumn{1}{l}{\#LIMIT} & \multicolumn{1}{l}{Avg. Node} & \multicolumn{1}{l}{Avg. Gap} & \multicolumn{1}{l}{Time} & \multicolumn{1}{l}{\#LIMIT} & \multicolumn{1}{l}{Avg. Node} \\
    \hline
    1     & 2359.28 & 21    & 314880 & 0.34\% & 1.26  & 0     & 1 \\
    2     & 2124.01 & 18    & 282644 & 0.40\% & 1.32  & 0     & 11 \\
    3     & 2165.93 & 19    & 205147 & 0.59\% & 1.30  & 0     & 1 \\
    4     & 2293.98 & 19    & 286217 & 0.62\% & 1.50  & 0     & 14 \\
    5     & 1827.85 & 14    & 202231 & 0.55\% & 1.22  & 0     & 1 \\
    6     & 2206.03 & 18    & 246280 & 0.63\% & 1.20  & 0     & 1 \\
    7     & 2190.73 & 20    & 262080 & 0.62\% & 1.23  & 0     & 18 \\
    8     & 1893.50 & 15    & 223771 & 0.33\% & 1.00  & 0     & 1 \\
    9     & 1720.97 & 13    & 211767 & 0.63\% & 1.04  & 0     & 1 \\
    10    & 1899.43 & 16    & 220385 & 0.47\% & 1.41  & 0     & 60 \\
    \hline
    avg.  & 2068.17 & 17    & 245540 & 0.52\% & 1.25  & 0     & 11 \\
    \hline
    \end{tabular}%
    }
  \label{tab:continuous_milp_socp}%
\end{table}%

Next, we compare the branch-and-cut algorithm using the extended polymatroid inequalities (in column ``\textbf{B\&C}'') with directly solving the MISOCP \eqref{eq:mip-continuous} (in column ``\textbf{No Cuts}'') on the transportation problem instances. If any instance cannot be solved within the 1800-second time limit, we report the average optimality gap and the number of unsolved instances in parentheses. 
In Table \ref{tab:cputime_continuous_socp}, the branch-and-cut algorithm solves the MISOCP faster than directly solving it in Gurobi.
\begin{table}[htbp]
  \centering
  \caption{Comparison of CPU time (in seconds) and optimality gaps of continuous distributions using MISOCP}
  \resizebox{.7\textwidth}{!}{
      \begin{tabular}{@{\extracolsep{4pt}}clcrrrrrr@{}}
    % {lllrrrrrr}%{@{\extracolsep{\fill}}*{6}{c}}
     % \begin{tabular*}{\textwidth}{@{\extracolsep{\fill}}*{6}{c}}
\hline
       \multirow{2}[4]{*}{Demand} & \multirow{2}[4]{*}{$N$} & \multirow{2}[4]{*}{Instance} & \multicolumn{3}{c}{No Cuts} & \multicolumn{3}{c}{B\&C} \\
\cline{4-6}     \cline{7-9}    
&         &       & \multicolumn{1}{c}{Time} & \multicolumn{1}{c}{Opt. Gap} & \multicolumn{1}{c}{Node} & \multicolumn{1}{c}{Time} & \multicolumn{1}{c}{Opt. Gap} & \multicolumn{1}{c}{Node} \\
    
    \hline
    100   & 50    & a     & 92.33 & N/A   & 9186  & 7.75  & N/A   & 1 \\
    100   & 50    & b     & 66.09 & N/A   & 16896 & 10.17 & N/A   & 6 \\
    100   & 50    & c     & 80.81 & N/A   & 11245 & 10.49 & N/A   & 335 \\
    100   & 50    & d     & 38.65 & N/A   & 6526  & 15.13 & N/A   & 995 \\
    100   & 50    & e     & 67.36 & N/A   & 6565  & 8.31  & N/A   & 1 \\
    \hline
          &       & avg.  & 69.05 & N/A   & 10084 & 10.37 & N/A   & 268 \\
    \hline
    100   & 100   & a     & 64.15 & N/A   & 1     & 40.33 & N/A   & 1 \\
    100   & 100   & b     & 86.54 & N/A   & 46    & 39.79 & N/A   & 1 \\
    100   & 100   & c     & 367.28 & N/A   & 8467  & 29.57 & N/A   & 1 \\
    100   & 100   & d     & 46.37 & N/A   & 1     & 17.59 & N/A   & 1 \\
    100   & 100   & e     & 47.24 & N/A   & 1     & 29.49 & N/A   & 1 \\
    \hline
          &       & avg.  & 122.32 & N/A   & 1703  & 31.36 & N/A   & 1 \\
    \hline
    100   & 200   & a     & 1455.53 & N/A   & 1743  & 442.66 & N/A   & 1246 \\
    100   & 200   & b     & LIMIT & 0.12\% & 7211  & 434.15 & N/A   & 3 \\
    100   & 200   & c     & LIMIT & 0.32\% & 10108 & LIMIT & 0.18\% & 116520 \\
    100   & 200   & d     & LIMIT & 0.08\% & 664   & 366.55 & N/A   & 662 \\
    100   & 200   & e     & LIMIT & 0.34\% & 28139 & 619.82 & N/A   & 878 \\
    \hline
          &       & avg.  & 1731.98 & 0.21\% (4) & 9573  & 732.73 & 0.18\% (1) & 23862 \\
    \hline
    \end{tabular}%
    }
  \label{tab:cputime_continuous_socp}%
\end{table}%


\subsection{Solution Details of Models with Finite and Continuous Distributions }
\label{sec:comp_sol}
In this section, we focus on the solution details of the transportation problem, which are obtained by solving the risk-adjustable DRCC models (assuming finite distributions (``\textbf{Finite}'') and continuous distributions (``\textbf{Continuous}'')), as well as the risk-adjustable stochastic chance-constrained model (``\textbf{Stochastic}''). The detailed formulation of the stochastic chance-constrained model is available in Appendix \ref{apx:cc}.
In Section \ref{sec:cpu_gaps}, we observe that 
the branch-and-cut algorithm does not scale as efficiently as the MILP \eqref{eq:mip_discrete}  assuming finite distributions, particularly when the sample size increases. In this section, the solution details  suggest that the MISOCP model assuming continuous distributions can be effectively approximated by the MILP model \eqref{eq:mip_discrete} for larger sample sizes.
% The solution details suggest approximating  the DRCC model with continuous distributions with the finite distribution model. 
The details are presented below.


\subsubsection{Optimal objective Values}

We compare the optimal objective values obtained from solving the three models: Finite, Continuous, and Stochastic. In Table \ref{tab:obj_continuous_vs_discrete},  the relative difference (in columns ``\textbf{Diff.}'') is calculated as the relative gap with the Finite model. The positive relative differences of the Continuous models are as indicated by Proposition \ref{prop:obj}.
All the relative differences  for both Continuous and Stochastic models are positive, which indicates the conservatism of the Finite model compared to the other two models. Furthermore, the differences between the Continuous and the Finite models decrease as the sample size grows larger. For instance, with a sample size $N=50$, the average  difference  between the Finite and Continuous models is 7.5\%, which reduces to 1.7\% when $N=200$. This observation implies that solving the Finite model as a conservative approximation of the Continuous model becomes more suitable when the sample size is large and the  MISOCP for the Continuous model is time-consuming to solve. 
% Table generated by Excel2LaTeX from sheet 'compare_disc_cont'
\begin{table}[htbp]
  \centering
   \caption{Comparison of objective costs}
   \resizebox{.65\textwidth}{!}{
    \begin{tabular}{@{\extracolsep{4pt}}cccrrrrr@{}}
    % \begin{tabular}{lllrrrrr}
     \hline
    \multicolumn{1}{c}{\multirow{2}[4]{*}{Demand}} & \multicolumn{1}{c}{\multirow{2}[4]{*}{$N$}} & \multicolumn{1}{c}{\multirow{2}[4]{*}{Instance}} & \multicolumn{1}{c}{Finite} & \multicolumn{2}{c}{Continuous} & \multicolumn{2}{c}{Stochastic} \\
\cline{4-4} \cline{5-6} \cline{7-8}          
% \cline{4-8}
&       &       & \multicolumn{1}{c}{Obj.} & \multicolumn{1}{c}{Obj.} & \multicolumn{1}{c}{Diff.} & \multicolumn{1}{c}{Obj.} & \multicolumn{1}{c}{Diff.} \\
      \hline
    100   & 50    & a     & 37891812 & 34882415 & 7.9\% & 35877509 & 5.3\% \\
    100   & 50    & b     & 39600119 & 36583112 & 7.6\% & 37580027 & 5.1\% \\
    100   & 50    & c     & 40591537 & 37591717 & 7.4\% & 38591438 & 4.9\% \\
    100   & 50    & d     & 39992224 & 36977377 & 7.5\% & 37972322 & 5.1\% \\
    100   & 50    & e     & 41872481 & 38851630 & 7.2\% & 39851708 & 4.8\% \\
     \hline
          &       & avg.  & 39989635 & 36977250 & 7.5\% & 37974601 & 5.0\% \\
     \hline
    100   & 100   & a     & 36300456 & 34797106 & 4.1\% & 35236038 & 2.9\% \\
    100   & 100   & b     & 38370855 & 36931801 & 3.8\% & 37356067 & 2.6\% \\
    100   & 100   & c     & 39353292 & 37849666 & 3.8\% & 38297139 & 2.7\% \\
    100   & 100   & d     & 38786542 & 37271588 & 3.9\% & 37715295 & 2.8\% \\
    100   & 100   & e     & 40741978 & 39244431 & 3.7\% & 39710167 & 2.5\% \\
      \hline
          &       & avg.  & 38710625 & 37218919 & 3.9\% & 37662941 & 2.7\% \\
     \hline
    100   & 200   & a     & 35767904 & 35150000 & 1.7\% & 35202312 & 1.6\% \\
    100   & 200   & b     & 37895369 & 37270088 & 1.7\% & 37318954 & 1.5\% \\
    100   & 200   & c     & 38919823 & 38266132 & 1.7\% & 38322395 & 1.5\% \\
    100   & 200   & d     & 38302041 & 37643158 & 1.7\% & 37668983 & 1.7\% \\
    100   & 200   & e     & 40113429 & 39463734 & 1.6\% & 39516533 & 1.5\% \\
    \hline
          &       & avg.  & 38199713 & 37558622 & 1.7\% & 37605836 & 1.6\% \\
    \hline
    \end{tabular}%
    }
  \label{tab:obj_continuous_vs_discrete}
\end{table}%

\subsubsection{Allowed risk tolerance}
In this section, we look into the risk tolerance allowed by solving the three models. Recall that the transportation problem imposes a chance constraint for each demand location and with $D=100$ customers, there are 100 allowed risk tolerances $\alpha_j, \ j=1,\ldots,100$. Figures \ref{fig:awesome_image1}-\ref{fig:awesome_image3} show the distributions of the risk tolerances obtained by solving the Finite, Continuous, and Stochastic models with sample size $N=\{50,100,200\}$. The Stochastic model assigns $\alpha$'s to smaller values than the two DRCC models. Additionally, as the sample size increases, there is more  overlap between the distributions obtained from solving the Finite and Continuous models, which supports approximating the Continuous model with the Finite model when the sample size is large. 

\begin{figure}[!htb]
\minipage{0.32\textwidth}
  \includegraphics[width=\linewidth]{hist_Demand-100-50a.png}
  \caption{ $N=50$}\label{fig:awesome_image1}
\endminipage\hfill
\minipage{0.32\textwidth}
  \includegraphics[width=\linewidth]{hist_Demand-100-100a.png}
  \caption{$N=100$}\label{fig:awesome_image2}
\endminipage\hfill
\minipage{0.32\textwidth}%
  \includegraphics[width=\linewidth]{hist_Demand-100-200a.png}
  \caption{$N=200$}\label{fig:awesome_image3}
\endminipage
\end{figure}

% \begin{figure}
%     \centering
%     \subfigure{\includegraphics[width=0.32\textwidth]{hist_Demand-100-50a.png}} 
%     \subfigure{\includegraphics[width=0.32\textwidth]{hist_Demand-100-100a.png}} 
%     \subfigure{\includegraphics[width=0.32\textwidth]{hist_Demand-100-200a.png}}
%     \caption{(a) blah (b) blah (c) blah (d) blah}
%     \label{fig:foobar}
% \end{figure}

% Table generated by Excel2LaTeX from sheet 'compare_disc_cont'
% \begin{table}[htbp]
%   \centering
%   \caption{Comparison of objective costs under discrete and continuous distributions}
%    \resizebox{\textwidth}{!}{
%     \begin{tabular}{@{\extracolsep{4pt}}lllrrrrrrrrr@{}}
%     % {lllrrrrrrrrr}
%     \hline
%     \multirow{2}[4]{*}{Demand} & \multirow{2}[4]{*}{S} & \multirow{2}[4]{*}{Instance}      & \multicolumn{3}{c}{Continuous} & \multicolumn{3}{c}{Discrete} & \multicolumn{3}{c}{Gaps} \\
% \cline{4-6} \cline{7-9} \cline{10-12}          &       &       & \multicolumn{1}{c}{Total Cost} & \multicolumn{1}{c}{Decision} & \multicolumn{1}{c}{ Violation} & \multicolumn{1}{c}{Total Cost} & \multicolumn{1}{c}{Decision} & \multicolumn{1}{c}{ Violation} & \multicolumn{1}{c}{Total Cost} & \multicolumn{1}{c}{Decision} & \multicolumn{1}{c}{ Violation} \\
%     \hline
%        100   & 50    & a     & 34882415 & 31905322 & 2977093 & 37891812 & 32531500 & 5360312 & 8.6\% & 2.0\% & 80.1\% \\
%     100   & 50    & b     & 36583112 & 33656564 & 2926548 & 39600119 & 34219873 & 5380246 & 8.2\% & 1.7\% & 83.8\% \\
%     100   & 50    & c     & 37591717 & 34602247 & 2989470 & 40591537 & 35051263 & 5540274 & 8.0\% & 1.3\% & 85.3\% \\
%     100   & 50    & d     & 36977377 & 33930056 & 3047321 & 39992224 & 34431942 & 5560282 & 8.2\% & 1.5\% & 82.5\% \\
%     100   & 50    & e     & 38851630 & 35964200 & 2887430 & 41872481 & 36572203 & 5300278 & 7.8\% & 1.7\% & 83.6\% \\
%     \hline
%           &       & avg.  & 36977250 & 34011678 & 2965572 & 39989635 & 34561356 & 5428279 & 8.2\% & 1.6\% & 83.0\% \\
%     \hline
%     100   & 100   & a     & 34797106 & 32626586 & 2170520 & 36300456 & 32790252 & 3510204 & 4.3\% & 0.5\% & 61.7\% \\
%     100   & 100   & b     & 36931801 & 34740459 & 2191343 & 38370855 & 34850691 & 3520164 & 3.9\% & 0.3\% & 60.6\% \\
%     100   & 100   & c     & 37849666 & 35619964 & 2229702 & 39353292 & 35743118 & 3610174 & 4.0\% & 0.3\% & 61.9\% \\
%     100   & 100   & d     & 37271588 & 34583748 & 2687839 & 38786542 & 34656340 & 4130202 & 4.1\% & 0.2\% & 53.7\% \\
%     100   & 100   & e     & 39244431 & 36889432 & 2355000 & 40741978 & 37001772 & 3740206 & 3.8\% & 0.3\% & 58.8\% \\
%     \hline
%           &       & avg.  & 37218919 & 34892038 & 2326881 & 38710625 & 35008435 & 3702190 & 4.0\% & 0.3\% & 59.4\% \\
%     \hline
%     100   & 200   & a     & 35150000 & 33059138 & 2090862 & 35767904 & 32992740 & 2775164 & 1.8\% & -0.2\% & 32.7\% \\
%     100   & 200   & b     & 37270088 & 35201348 & 2068739 & 37895369 & 35050241 & 2845128 & 1.7\% & -0.4\% & 37.5\% \\
%     100   & 200   & c     & 38266132 & 35968438 & 2297694 & 38919823 & 35904676 & 3015147 & 1.7\% & -0.2\% & 31.2\% \\
%     100   & 200   & d     & 37643158 & 35034015 & 2609143 & 38302041 & 35031868 & 3270173 & 1.8\% & 0.0\% & 25.3\% \\
%     100   & 200   & e     & 39463734 & 37331740 & 2131994 & 40113429 & 37248274 & 2865155 & 1.6\% & -0.2\% & 34.4\% \\
%     \hline
%           &       & avg.  & 37558622 & 35318936 & 2239687 & 38199713 & 35245560 & 2954154 & 1.7\% & -0.2\% & 32.2\% \\
%  \hline
%     \end{tabular}%
%     }
%   \label{tab:obj_continuous_vs_discrete}%
% \end{table}%


\section{Conclusions}
\label{sec:conclusions}

In this paper, we investigated distributionally robust individual chance-constrained problems with a data-driven Wasserstein ambiguity set, where the uncertainty only affects the right-hand side and the risk tolerance is considered as a decision variable. The goal of the risk-adjustable DRCC is to trade-off between system costs and risk violation costs via penalizing the risk tolerance in the objective function.
We considered two types of Wasserstein ambiguity sets with finite and continuous distributions. We provided a MILP reformulation of the risk-adjustable DRCC problem with finite distributions and a MISOCP reformulation for the continuous distribution case. Moreover, we derived valid inequalities for both reformulations. Via extensive numerical studies, we demonstrated that our valid inequalities accelerate solving the risk-adjustable DRCC models when compared to optimization solvers. 
Although the MISOCP reformulation does not scale well with larger  sample size, the MILP reformulation can be used as an approximation of the MISOCP reformulation.

%\subsection{Duality and the Classical EOQ Problem.}\label{class-EOQ} %% 1.1.
%\subsection{Outline.}\label{outline1} %% 1.2.
%\subsubsection{Cyclic Schedules for the General Deterministic SMDP.}
%  \label{cyclic-schedules} %% 1.2.1
%\section{Problem Description.}\label{problemdescription} %% 2.

% Text of your paper here


% Acknowledgments here
% \ACKNOWLEDGMENT{%
% % Enter the text of acknowledgments here
% }% Leave this (end of acknowledgment)


% Appendix here
% Options are (1) APPENDIX (with or without general title) or 
%             (2) APPENDICES (if it has more than one unrelated sections)
% Outcomment the appropriate case if necessary
%
% \begin{APPENDIX}{The MILP-based Reformulation: Binary Variables}
\bibliographystyle{informs2014}
\bibliography{YilingRef}

\newpage
\begin{APPENDICES}

\section{Stochastic Chance Constrained Problem: MILP Reformulation}
\label{apx:cc}
     % 
    Let $N^\prime = \lceil\bar\alpha N\rceil$. \begin{subequations}\label{eq:mip_sp}
\begin{eqnarray}
 \min_{x\in\mathcal X, y} && c^\top x + \sum_{n=1}^{N^\prime-1}\Delta_n y_n + \Delta_{N^\prime}  \\
\label{eq:sp_constr1}\mbox{s.t.} && Tx \ge \xi^{N^\prime} + \sum_{n=1}^{N^\prime - 1}(\xi^n - \xi^{n+1}) y_n\\
% && Tx \ge \xi^{N^\prime} + M_n(1-y_n), \ n=1,\ldots,N^\prime\\
\label{eq:sp_y_order}&& y_{n+1} \ge y_{n}, \ n=1,\ldots,N^\prime -2\\
\label{eq:sp_alpha}&& \frac{1}{N} \left( N^\prime-1- \sum_{n=1}^{N^\prime-1}y_{n} \right) \in (0,\bar{\alpha}]\\
&& y_n\in\{0,1\}, \ n=1,\ldots,N^\prime - 1,
\end{eqnarray}
\end{subequations}
where $\Delta_n:=g(n/N) - g((n+1)/N), \ n=1,\ldots,N^\prime-1$, and $\Delta_{N^\prime} := g(N^\prime /N)$.% and $M_n$ is a big-M constant. 


\section{Alternative MILP Reformulation for Risk-adjustable DRCC with Binary Variables}
\label{apx:milp}
When all the decision variables are pure binary, i.e., $\mathcal X\subset \{0,1 \}^d$, \citet{zhang2022building} developed a MILP reformulation. The reformulation uses a binary variable $o_{jk}$ to identify the critical indices $j$ and $k$ following the similar idea as in Section \ref{sec:continuous}. 
% 
\begin{equation}\label{eq:soc-milp}
\sum_{j=0}^{N-1}\sum_{k=j}^{N-1}o_{jk}\left[(\alpha N-j)(Tx-\xi^{k+1}) - \sum_{i=j+1}^k(\xi^i-\xi^{k+1}) \right]\ge N\epsilon 
\end{equation}
% 
% 
% When the risk tolerance $\alpha$ in constraint \eqref{eq:soc} is treated as a variable, 
There are   bilinear terms $o_{jk}x_\ell, \alpha o_{jk}$ and trilinear term $o_{jk}\alpha x_\ell$ in constraint \eqref{eq:soc-milp}. When the decisions $x_\ell, \ \ell=1,\ldots,d$ are pure binary, they can all be linearized using McCormick inequalities \citep{mccormick1976computability}. The alternative MILP reformulation is as follows.
% 
\begin{subequations}\label{eq:milp-continuous}
        \begin{eqnarray}
            \min_{x\in\mathcal X, o,\alpha,u,w}&& c^\top x + g(\alpha)\\
            \mbox{s.t.} && \sum_{j=0}^{N-1}\sum_{k=j}^{N-1}\left[ o_{jk}\sum_{i=j+1}^{k} \left( \xi^{k+1}-\xi^i \right) + N T(\delta_{jk} - j\tau_{jk}) - \xi^{k+1}(\varepsilon_{jk} - j o_{jk}) \right] \ge N\epsilon \\
%             \label{eq:milp-continuous_conic}\mbox{s.t.} && uw \ge \sum_{j=0}^{N-1}\sum_{k=j}^{N-1}o_{jk}\sum_{i=j+1}^k(\xi^i-\xi^{k+1}) + N\epsilon \\
%             \label{eq:milp-continuous_conic1}&& u\le \alpha N-\sum_{j=0}^{N-1}\sum_{k=j}^{N-1}jo_{jk}\\
% \label{eq:milp-continuous_conic2}            && w \le Tx - \sum_{j=0}^{N-1}\sum_{k=j}^{N-1} \xi^{k+1}o_{jk}\\
&& \sum_{j=0}^{N-1}\sum_{k=j}^{N-1} \xi^{j}o_{jk} \ge Tx \ge \sum_{j=0}^{N-1}\sum_{k=j}^{N-1} \xi^{j+1}o_{jk}\\
&& \sum_{j=0}^{N-1} \sum_{k=j}^{N-1} (k+1) o_{jk} \ge \alpha N \ge \sum_{j=0}^{N-1} \sum_{k=j}^{N-1} k o_{jk}\\
            \label{eq:milp-continuous_sos1}&& \sum_{j=0}^{N-1}\sum_{k=j}^{N-1} o_{jk}=1\\
            && \alpha \in (0,\bar{\alpha}]\\
            &&  \varepsilon_{jk}\le o_{jk}, \ \varepsilon_{jk}\le \alpha, \ \varepsilon_{jk}\ge \alpha + o_{jk} -1, \ \varepsilon_{jk} \ge 0, \ 0\le j\le k\le N-1 \\
            &&  \delta_{\ell jk}\le \varepsilon_{jk}, \ \delta_{\ell jk}\le x_\ell, \ \delta_{\ell jk}\ge \varepsilon_{jk} + x_{\ell} -1, \ \delta_{\ell jk} \ge 0, \ 0\le j\le k\le N-1, \ 1 \le \ell \le d \\
            &&  \tau_{\ell jk}\le o_{jk}, \ \tau_{\ell jk}\le x_\ell, \ \tau_{\ell jk}\ge o_{jk} + x_{\ell} -1, \ \tau_{\ell jk} \ge 0, \ 0\le j\le k\le N-1, \ 1 \le \ell \le d \\
            && o_{jk}\in \{0,1\}, \ 0\le j\le k \le N-1.
        \end{eqnarray}
    \end{subequations}

\section{More results for CPU time and Optimality Gaps with Finite Distributions}
\label{apx:finite_cput}
\begin{table}[htbp]
  \centering
  \caption{Comparison of CPU time (in seconds) and optimality gaps of finite distributions}
    \resizebox{.65\textwidth}{!}{
  \begin{tabular}{@{\extracolsep{4pt}}cccrrrrrrrrr@{}}
%\begin{tabular}{rrrrrrrrrrr}
	\hline
	\multicolumn{1}{c}{\multirow{2}[4]{*}{Demand}} & \multicolumn{1}{c}{\multirow{2}[4]{*}{$N$}} & \multicolumn{1}{c}{\multirow{2}[4]{*}{Instance}} & \multicolumn{4}{c}{MILP + Valid Ineq.} &       & \multicolumn{4}{c}{MILP} \\
	\cline{4-7}\cline{8-12}  &        &       & \multicolumn{1}{c}{$t_\text{BS}$} & \multicolumn{1}{c}{$t_\text{MILP}$} & \multicolumn{1}{c}{Time} & \multicolumn{1}{c}{Node} & \multicolumn{1}{c}{$t_\text{BS}$} & \multicolumn{1}{c}{$t_\text{MILP}$} & \multicolumn{1}{c}{Time} & \multicolumn{1}{c}{Opt. Gap} & \multicolumn{1}{c}{Node} \\	
    \hline
    100   & 50    & a     & 0.02  & 0.05  & 0.06  & 1     & 0.01  & 0.33  & 0.34  & N/A   & 1 \\
    100   & 50    & b     & 0.00  & 0.05  & 0.05  & 1     & 0.01  & 0.25  & 0.26  & N/A   & 1 \\
    100   & 50    & c     & 0.00  & 0.03  & 0.03  & 1     & 0.01  & 0.31  & 0.32  & N/A   & 1 \\
    100   & 50    & d     & 0.00  & 0.03  & 0.03  & 1     & 0.00  & 0.38  & 0.38  & N/A   & 1 \\
    100   & 50    & e     & 0.00  & 0.05  & 0.05  & 1     & 0.00  & 0.27  & 0.27  & N/A   & 1 \\
\hline
          &       & avg.  & 0.00  & 0.04  & 0.04  & 1     & 0.01  & 0.31  & 0.31  & N/A   & 1 \\
\hline
    100   & 100   & a     & 0.02  & 0.06  & 0.08  & 1     & 0.02  & 0.91  & 0.93  & N/A   & 85 \\
    100   & 100   & b     & 0.02  & 0.06  & 0.08  & 1     & 0.02  & 0.84  & 0.87  & N/A   & 1 \\
    100   & 100   & c     & 0.02  & 0.08  & 0.09  & 1     & 0.02  & 1.06  & 1.08  & N/A   & 223 \\
    100   & 100   & d     & 0.02  & 0.07  & 0.09  & 1     & 0.03  & 1.37  & 1.40  & N/A   & 480 \\
    100   & 100   & e     & 0.02  & 0.08  & 0.10  & 1     & 0.02  & 0.91  & 0.93  & N/A   & 1 \\
 \hline
          &       & avg.  & 0.02  & 0.07  & 0.09  & 1     & 0.02  & 1.02  & 1.04  & N/A   & 158 \\
\hline
    100   & 200   & a     & 0.06  & 0.09  & 0.15  & 1     & 0.07  & 4.73  & 4.81  & N/A   & 1420 \\
    100   & 200   & b     & 0.06  & 0.09  & 0.14  & 1     & 0.06  & 5.41  & 5.47  & N/A   & 1705 \\
    100   & 200   & c     & 0.06  & 0.08  & 0.14  & 1     & 0.06  & 6.87  & 6.94  & N/A   & 2605 \\
    100   & 200   & d     & 0.08  & 0.08  & 0.16  & 1     & 0.06  & 8.21  & 8.28  & N/A   & 2559 \\
    100   & 200   & e     & 0.06  & 0.09  & 0.16  & 1     & 0.06  & 7.04  & 7.11  & N/A   & 1800 \\
\hline
          &       & avg.  & 0.07  & 0.09  & 0.15  & 1     & 0.06  & 6.45  & 6.52  & N/A   & 2018 \\
\hline
    100   & 1000  & a     & 1.20  & 0.33  & 1.54  & 1     & 1.20  & LIMIT & LIMIT & 0.06\% & 55681 \\
    100   & 1000  & b     & 1.23  & 0.35  & 1.58  & 1     & 1.19  & LIMIT & LIMIT & 0.05\% & 32960 \\
    100   & 1000  & c     & 1.23  & 0.32  & 1.55  & 1     & 1.20  & LIMIT & LIMIT & 0.06\% & 25663 \\
    100   & 1000  & d     & 1.47  & 0.31  & 1.78  & 1     & 1.23  & LIMIT & LIMIT & 0.23\% & 26593 \\
    100   & 1000  & e     & 1.19  & 0.35  & 1.54  & 1     & 1.21  & LIMIT & LIMIT & 0.05\% & 26371 \\
\hline
          &       & avg.  & 1.26  & 0.33  & 1.60  & 1     & 1.21  & LIMIT & LIMIT & 0.09\% & 33454 \\
\hline
    100   & 2000  & a     & 4.62  & 0.74  & 5.36  & 1     & 4.69  & LIMIT & LIMIT & 0.70\% & 7529 \\
    100   & 2000  & b     & 4.70  & 0.75  & 5.45  & 1     & 4.70  & LIMIT & LIMIT & 0.59\% & 9567 \\
    100   & 2000  & c     & 4.64  & 0.69  & 5.33  & 1     & 4.58  & LIMIT & LIMIT & 0.67\% & 8824 \\
    100   & 2000  & d     & 4.72  & 0.70  & 5.42  & 1     & 4.61  & LIMIT & LIMIT & 0.64\% & 11390 \\
    100   & 2000  & e     & 4.68  & 0.70  & 5.38  & 1     & 4.62  & LIMIT & LIMIT & 0.67\% & 7799 \\
\hline
          &       & avg.  & 4.67  & 0.72  & 5.39  & 1     & 4.64  & LIMIT & LIMIT & 0.65\% & 9022 \\
\hline
    200   & 2000  & a     & 9.12  & 1.59  & 10.71 & 1     & 9.27  & LIMIT & LIMIT & 1.80\% & 6840 \\
    200   & 2000  & b     & 9.25  & 1.61  & 10.86 & 1     & 9.34  & LIMIT & LIMIT & 1.86\% & 6835 \\
    200   & 2000  & c     & 9.19  & 1.58  & 10.77 & 1     & 9.43  & LIMIT & LIMIT & 3.98\% & 6615 \\
    200   & 2000  & d     & 9.26  & 1.67  & 10.93 & 1     & 9.48  & LIMIT & LIMIT & 3.24\% & 6479 \\
    200   & 2000  & e     & 9.08  & 1.58  & 10.66 & 1     & 9.45  & LIMIT & LIMIT & 2.00\% & 6619 \\
\hline
          &       & avg.  & 9.18  & 1.60  & 10.79 & 1     & 9.39  & LIMIT & LIMIT & 2.58\% & 6678 \\
\hline
    200   & 3000  & a     & 20.20 & 2.16  & 22.36 & 1     & 20.88 & LIMIT & LIMIT & 5.09\% & 6549 \\
    200   & 3000  & b     & 20.08 & 2.58  & 22.66 & 1     & 20.22 & LIMIT & LIMIT & 5.50\% & 6510 \\
    200   & 3000  & c     & 20.45 & 2.39  & 22.84 & 1     & 20.54 & LIMIT & LIMIT & 4.81\% & 6574 \\
    200   & 3000  & d     & 20.73 & 2.36  & 23.09 & 1     & 20.81 & LIMIT & LIMIT & 6.00\% & 4232 \\
    200   & 3000  & e     & 20.39 & 2.39  & 22.78 & 1     & 20.51 & LIMIT & LIMIT & 5.96\% & 6552 \\
\hline
          &       & avg.  & 20.37 & 2.38  & 22.75 & 1     & 20.59 & LIMIT & LIMIT & 5.47\% & 6083 \\
   
	\hline
\end{tabular}
    }
  \label{tab:discrete_cpu_details}%
\end{table}%


\end{APPENDICES}
%
%   or 
%
% \begin{APPENDICES}
% \section{<Title of Section A>}
% \section{<Title of Section B>}
% etc
% \end{APPENDICES}


% References here (outcomment the appropriate case) 

% CASE 1: BiBTeX used to constantly update the references 
%   (while the paper is being written).
%\bibliographystyle{informs2014} % outcomment this and next line in Case 1
%\bibliography{<your bib file(s)>} % if more than one, comma separated

% CASE 2: BiBTeX used to generate mypaper.bbl (to be further fine tuned)
%\documentclass[review]{elsarticle}

\usepackage{hyperref}
\usepackage{amsmath,amssymb,amsfonts}
\usepackage{algorithmic}
\usepackage{ntheorem}
\usepackage{caption}
\usepackage{graphicx}
\usepackage{textcomp}
\usepackage{indentfirst}
\usepackage{float}
\usepackage{bm}
\usepackage{tagging}
\usepackage{amsfonts,amssymb}
\usepackage{amsmath}
\usepackage{tablists}
\usepackage{subfigure}
\usepackage{ragged2e} 
\usepackage{booktabs,makecell, multirow, tabularx}
\usepackage{stfloats}
\usepackage{color}
\usepackage{appendix}
\DeclareMathAlphabet\mathbfcal{OMS}{cmsy}{b}{n}

\journal{Journal of \LaTeX\ Templates}

%%%%%%%%%%%%%%%%%%%%%%%
%% Elsevier bibliography styles
%%%%%%%%%%%%%%%%%%%%%%%
%% To change the style, put a % in front of the second line of the current style and
%% remove the % from the second line of the style you would like to use.
%%%%%%%%%%%%%%%%%%%%%%%

%% Numbered
%\bibliographystyle{model1-num-names}

%% Numbered without titles
%\bibliographystyle{model1a-num-names}

%% Harvard
%\bibliographystyle{model2-names.bst}\biboptions{authoryear}

%% Vancouver numbered
%\usepackage{numcompress}\bibliographystyle{model3-num-names}

%% Vancouver name/year
%\usepackage{numcompress}\bibliographystyle{model4-names}\biboptions{authoryear}

%% APA style
%\bibliographystyle{model5-names}\biboptions{authoryear}

%% AMA style
%\usepackage{numcompress}\bibliographystyle{model6-num-names}

%% `Elsevier LaTeX' style
\bibliographystyle{elsarticle-num}
%%%%%%%%%%%%%%%%%%%%%%%

\begin{document}
\captionsetup[figure]{labelfont={bf},labelformat={default},labelsep=period,name={Fig.}}

\begin{frontmatter}

\title{The Graph feature fusion technique for speaker recognition based on wav2vec2.0 framework \tnoteref{mytitlenote} }
\tnotetext[mytitlenote]{This work has been supported by National Natural Science Foundations of China (No.62071242)}

%% Group authors per affiliation:
\author{Zirui~Ge, Haiyan~Guo, Tingting~Wang, Zhen~Yang\corref{mycorrespondingauthor} }
\address{School of Communication and Information Engineering, Nanjing University of Posts and Telecommunications, Nanjing 2100023, China}
\fntext[myfootnote]{E-mail address: yangz@njupt.edu.cn (Z. Yang)}
%%1019010430@njupt.edu.cn (Z. Ge), guohy@njupt.edu.cn (H.Guo), 2018010215@njupt.edu.cn (T. Wang), 

%% or include affiliations in footnotes:
%\author[mymainaddress,mysecondaryaddress]{Elsevier Inc}
%\ead[url]{www.elsevier.com}
%%ead{ }
%\author[mysecondaryaddress]{Global Customer Service\corref{mycorrespondingauthor}}
\cortext[mycorrespondingauthor]{Corresponding author}
%\ead{ This work has been supported by National Natural Science Foundations of China (No.61671252, No.61271335, No.61901229), the Natural Science Research of}

%\address[mymainaddress]{1600 John F Kennedy Boulevard, Philadelphia}
%\address[mysecondaryaddress]{360 Park Avenue South, New York}

\begin{abstract}
Pre-trained wav2vec2.0 model has been proved its effectiveness for speaker recognition. However, current feature processing methods are focusing on classical pooling on the output features of the pre-trained wav2vec2.0 model, such as mean pooling, max pooling etc. That methods take the features as the independent and irrelevant units, ignoring the inter-relationship among all the features, and do not take the features as an overall representation of a speaker. Gated Recurrent Unit (GRU), as a feature fusion method, can also be considered as a complicated pooling technique, mainly focuses on the temporal information, which may show poor performance in some situations that the main information is not on the temporal dimension. In this paper, we investigate the graph neural network (GNN) as a backend processing module based on wav2vec2.0 framework to provide a solution for the mentioned matters. The GNN takes all the output features as the graph signal data and extracts the related graph structure information of features for speaker recognition. Specifically, we first give a simple proof that the GNN feature fusion method can outperform than the mean, max, random pooling methods and so on theoretically. Then, we model the output features of wav2vec2.0 as the vertices of a graph, and construct the graph adjacency matrix by graph attention network (GAT). Finally, we follow the message passing neural network (MPNN) to design our message function, vertex update function and readout function to transform the speaker features into the graph features. The experiments show our performance can provide a relative improvement compared to the baseline methods. Code is available at xxx.
\end{abstract}

\begin{keyword}
speaker recognition, wav2vec2.0, graph neural network, pooling, feature fusion
\end{keyword}
\end{frontmatter}


\section{Introduction}
\par Automatic speaker recognition (ASR) is the task of authenticating the claimed identity using the speaker’s voiceprint. As a means of using bio-metrics, ASR has attracted considerable attention from many researchers due to its accessibility and uniqueness. With the development of deep neural network, speaker recognition models based on deep neural networks are being more complicated, and need more larger quantities of labeled training data.
\par However, producing large and high quality labeled data is hard and expensive, and only learning from the labeled samples also seems to be inconsistent with the process of language acquisition of the infants, i.e., self-learning from listening and watching, supervise learning from training and testing with instructors. Self-supervision learning on unlabeled data and fine-tuning on the pre-trained models is similar to the mentioned two stages and have been proved successful for natural language processing such as BERT \cite{Ref1}, GPT-3 \cite{Ref2} \emph{etc}.
\par In the field of speech signal processing, wav2vec2.0 \cite{Ref3} also applies to the two stages learning process. Wav2vec2.0 shows an excellent performance on speech recognition, and it first learns the speech representation from the unlabeled speech audio dataset and fine-tune the pre-trained weights on the labeled data. There are mainly four modules in wav2vec2.0 framework, i.e., a multi-layer convolution feature encoder, a Transformer group module, the quantization module and contrastive loss. More specifically, wav2vec2.0 first encodes the raw audio signal into latent speech representations via the multi-layer convolution feature encoder. Then, the masked latent speech representations are fed into the Transformer module group to capture the contextualized representations from the entire sequence. Meanwhile, the quantization module converts the unmasked latent speech representation into its discrete version via product quantization. Finally, the discrete representation of quantization module and the output of the Transformer group are put into the contrastive loss to identify the true quantized latent speech representation. Wav2vec2.0 framework has achieved 1.8/3.3 word error rate (WER) on the clean/other test sets using all labeled data of Librispeech dataset. When using more less labeled data, wav2vec2.0 still outperforms the state of the art at that time. The excellent performance shows that the phonemic constructions are well learned during the pre-training and the downstream modules can finish their tasks via fine-tuning the pre-training weights. 
\par Pre-trained wav2vec2.0 as the upstream model also works well in speaker recognition. The authors in \cite{Ref4} first applied the wav2vec2.0 to multi-task learning, i.e., speaker recognition and language identification task, and investigated wav2vec2.0 as the audio encoder to extract the speaker and language features \cite{Ref4}. The work of multi-task learning in \cite{Ref4} first demonstrated the effectiveness of wav2vec2.0 on the speaker recognition and language identification task. At the same time, the authors in \cite{Ref5} took pre-trained wav2vec2.0 models to implement the speech emotion recognition task, and propose to weight the output of several layers from the pre-trained model using trainable weights which are learned jointly with the downstream model. The authors in \cite{Ref6} applied wav2vec2.0 framework to speaker recognition task, and investigated effectiveness of different pooling methods. \cite{Ref6} further proposed the first\&cls pooling method that inserts a "start token" (all values are +1) in the input sequence of encoder, and selected the first output token as the speaker embedding. These literatures completed different speech related tasks based on wav2vec2.0 framework via using mean, max, mean\&std pooling methods etc. Though these classical methods have been proven their effectiveness, they only focus on some simple information of features, for example, the mean and mean\&max pooling mainly consider the distribution of features, and max pooling mainly focuses on the "texture information". Besides, these methods consider the output features as many independent elements in the regular (Euclidean) space. They do not view these output features as an entirety feature of a speaker identity and do not consider more complicate relationship among these features. Though GRU as a complicate feature fusion method can focus the temporal information, it may be uncompetitive when meet the features lacking the temporal information. For example, these features have been processed by some other models that can also extract the temporal information, such as the Transformer module \cite{Ref7}. Therefore, when features require to be viewed as an entirety or not merely extracted the temporal information, a new data structure or signal processing technique require to be introduced.
\par Some literatures \cite{Ref8,Ref9,Ref10} have shown that the speech signals can be reformulated as a graph signal and processed using graph signal processing theory (GSP) \cite{Ref25,Ref26} in the irregular space, i.e., graph domain, and a better performance can be obtained compared to regular space. In above literatures, speech features are considered as an entirety, i.e., a graph signal, not merely a set of different independent features. Graph neural networks (GNN) as a nonlinear form of GSP that corporate the advantages of the graph structure and deep neural networks \cite{Ref11,Ref12} have shown the excellent performance on image classification \cite{Ref13,Ref14,Ref15} which is mainly processed in the regular (Euclidean) space previously. Thus, exploring the combination of GNN and speaker recognition is well-founded.
\par The authors in \cite{Ref16} first proposed the concept of GNN and extended the neural networks for processing the data resided in graph domains. With the development of GNN, there are some major variants of GNN in the world including the message passing neural network (MPNN) \cite{Ref17}, graph attention network (GAT) etc. MPNN contains two phases, a message passing phase and a readout phase. The message passing phase mainly include hidden states updating and graph vertex aggregating, and the readout phase is aim to obtain the whole graph representation using the readout function. GAT incorporates the self-attention mechanism \cite{Ref7} into the propagation step and obtains new node features via weighting neighborhood node features using attention coefficients.
\par \cite{Ref19,Ref20} first applied GNN as the backend feature fusion method of ResNet and RawNet2 model to speaker recognition task. \cite{Ref20} uses the GAT mechanism to design an undirected graph with asymmetric weight matrix to depict the relationship between different graph vertex pairs, and take the graph U-Net architecture to obtain the final graph representation. The GNN related framework in [19] takes a pair of utterances as the enrollment and the test utterance to train their similarity score. \cite{Ref21} also took the same GNN architecture in \cite{Ref19} to implement speaker anti-spoofing task. The both GNN backend models obtain the improved performance. However, that mentioned utterance-pair classification format is inferior to the single-utterance classification for speaker recognition based on wav2vec2.0 framework \cite{Ref6}. Besides, both works do not talk about the relationship between the GNN fusion method and classical fusion method such as mean, max, random, etc., and why the GNN based feature fusion models can outperform the compared GRU model.
\par Under this background, we first discuss the relationship between GNN or GSP and classical pooling methods, such as mean, max, random, etc., and show that why GNN based feature fusion models can outperform the mean, max, random pooling methods theoretically. Then we propose GNN as the downstream processing framework to explore the output features of wav2vec2.0 framework in non-Euclidean space. Specifically, we model the set of output features as a graph signal, i.e., each feature is considered as a graph vertex’s feature, and the number of graph vertices is equal to the number of output features. Then we use the graph attention mechanism to construct the symmetric weight matrix to capture the similarity between different vertex pairs. The attention weight matrix is applied to our designed message passing graph neural network to obtain the graph embedding which is also the final speaker embedding. When the downstream processing framework is constructed, we take the wav2vec2.0 as the upstream audio feature extractor and fine-tune its pre-training weights. We implement our experiments on the VoxCeleb datasets \cite{Ref20, Ref21}, and the experiments show that proposed graph pooling method can obtain better performance than the classical feature pooling methods. We also explain why GNN can obtain the top performance compared to other feature fusion methods. To the best of our knowledge, this study is the first to apply a GNN to wav2vec2.0 framework for the speaker recognition task.

\section{Related Work}

\par In this section, we first review the pre-training of the wav2vec 2.0 and how to apply the pre-training to downstream. Then, we introduce graph message passing neural network.
\subsection{Wav2vec 2.0 pretraining and fine-tuning}
\par The main body of the model consists a CNN based feature encoder, a Transformer-based context network, a quantization module and a contrastive loss. The feature encoder consists of 7 blocks and each block contains a temporal convolution with 512 channels with respective kernel sizes of (10, 3, 3, 3, 3, 2, 2) and stride (5, 2, 2, 2, 2, 2, 2) followed by a layer normalization and a GELU activation function \cite{Ref21}. As the Figure.1 left depicted, the CNN feature extractor takes as input raw audio $X$ and and outputs latent speech representations $Z$. Then, the latent speech representations are projected into a new dimension, before fed to the following modules.
\par The context network contains 12 Transformer blocks and a residual 2-layer feed forward network with 3072 and 768 units. The relative positional embeddings instead of fixed positional embeddings are first added to the masked speech representations, before the masked representations are input to the context network. Transformer then contextualizes the masked representations and finally generates context representations $C$. The outputs of 12 Transformer blocks and projected speech representation are considered as hidden features.
\par The quantization module is to discretize the output of the feature encoder to a finite set of speech representations via product quantization, where the product quantization is aim to choosing quantized representations from multiple codebooks and concatenating them. The number of codebooks is equal to 2 and there are 320 elements and respective size is 128 in each codebook. The gumbel softmax function \cite{Ref24} is also used to enable choosing discrete codebook entries in a fully differentiable way.
\par The objective function is the weighted sum of the contrastive loss and diversity loss. The contrastive loss requires to identify the true quantized latent speech representation for a masked time step within a set of distractors. The diversity loss is designed to encourage using the codebook entries equally. 
\par Fig.1 right shows the fine-tuning stage, the models are identical to the training stage, except the quantization modules and extra output layers. We take a speech audio with 48000 samples as the input, and show the output shape of different modules. In the fine-tuning stage, how to select the hidden feature has an important effect on downstream tasks \cite{Ref5}, therefore we consider two approaches to select hidden feature, one way is taking the output of last Transformer block, the other is weighting all the hidden features.

\begin{figure}[t]
	{	\centering
		\includegraphics[width=5.2in]{finetuning2.eps}\\}
	\caption{An overview of the pre-training and fine-tuning. The right subfigure shows the output shape of different model for a given input.}\label{fig1}
\end{figure}

\subsection{Message Passing Neural Networks }
\par Let $G=(V,E,A)$ be a graph and $X\in \mathbb{R} ^{F\times N}$ be input vertex features, where $V=\left\{ v_1,...,v_N \right\} $ is the set of $N=|V|$ vertices, $F$ is the dimension of one input feature. $\mathcal{E} =\left\{ e_{i,j} \right\} _{i,j\in \mathcal{V}}$ is the set of edges between vertices, such that $e_{i,j}=1$ if there is a link from node $j$ to node $i$, otherwise $e_{i,j}=0$. $A$ is the $N\times N$ weight adjacency matrix and its entries are the edge weights $a_{i,j}$, for  $i, j=1,...,N$.
\par The MPNN contains two phases, a message passing phase and a readout phase. The message phase runs for $T$ time steps and contains two functions that are message function $M_t$ and vertex update function $U_t$. The hidden state $h_t$ of each vertex at the $t$ time step can be written as:
$$
m_{v}^{t+1}=\sum_{w\in N(v)}{M_t}\left( h_{v}^{t},h_{w}^{t},e_{vw} \right),\eqno(1) \label{eq1}
$$
$$
h_{v}^{t+1}=U_t\left( h_{v}^{t},m_{v}^{t+1} \right) ,\eqno(2) \label{eq2}
$$
where $N(v)$ denotes the neighbors of vertex $v$ in graph $G$. The readout phase computes the final graph feature vector via the following readout function $R$
$$
\hat{y}=R\left( \left\{ h_{v}^{T}\mid v\in G \right\} \right) ,\eqno(3) \label{eq3}
$$
The message function $M_t$, vertex update function $U_t$ and readout function $R$ are all learned differential functions. 

\section{METHODOLOGY}
\label{}
\par In this section, we first show that some classical pooling methods can be represented by GNN or GSP, and introduce the graph neural network as the feature pooling method. The proposed module is located after the context network and takes as input the hidden states of wav2vec2.0. There are three components in our proposed model: 1) the graph attention layer; 2) the message and vertex update function; and 3) the readout function. Specifically, we first show how to obtain the graph weight adjacency matrix from the graph attention layer. Then we show how to aggregate neighbor vertices information in term of one vertex, and update the status of every vertex. Finally, we propose a readout function to obtain the whole graph feature embedding. The overall scheme is illustrated in Figure 2.
\subsection{Reformulating some classical pooling methods with GNN}
\par We reformulate some pooling methods using GSP and GNN. Specifically, we use the GSP to reformulate the linear pooling, including mean, random, first, middle, last pooling method, and use the GNN to reformulate the nonlinear pooling, i.e., the max pooling method.
\par In the graph signal processing theory, the weight adjacency matrix $A$ is also is also named as the shift operator, and the notion of graph shift operator is defined as a local operation that replaces a graph signal feature with the linear combination of features at the neighbors of that vertex \cite{Ref27}. The graph shift operation can be expressed as 
$$
Y=AX^T, \eqno(4) \label{eq4}
$$
where $X\in \mathbb{R} ^{D\times N}$.
\par We take the above graph shift operator to reformulate the mean and random pooling methods. When 
$$
A=\left[\begin{array}{ccc}
1 / N & \cdots & 1 / N \\
\vdots & \ddots & \vdots \\
1 / N & \cdots & 1 / N
\end{array}\right], \eqno(5) \label{eq5}
$$

the expression $A^T$ can be considered as the mean pooling method, and for 

$$
A=\left[\begin{array}{ccc}
0 & 1 & 0 \\
\vdots &\vdots & \vdots \\
0 & 1 & 0
\end{array}\right], \eqno(6) \label{eq6}
$$
the expression $A^T$ can be considered as the random pooling method. When the random pooling method selects the $i$th feature, the entries of the $i$th column in $A$ in are all 1. When the $i$ in random pooling is always 1, $N$, or $\lfloor {{N} /{2}} \rfloor $, the pooling method can be considered as the first, last, and middle pooling.
\par The max pooling method as a nonlinear operation is impossible to be reformulated by the linear operation. We take the MLP as an approximation for the max pooling method, thanks to the universal approximation theorem \cite{Ref28}, and it can be expressed as 
$$
Y=MLP\left( AX^T \right)  , \eqno(7) \label{eq7}
$$
\par Therefore, some classical pooling methods can be reformulated with GSP or GNN, and we may obtain a better performance theoretically, if we directly use GSP or GNN as our pooling methods.
\subsection{Graph attention layer}
\par We first formulate a graph using the output features of wav2vec2.0 framework. Specially, each output feature is considered as a vertex of a graph. Due to these feature does not have an obvious graph structure, we need to specify the edge set and weight adjacency matrix. For this model, we argue that every vertex pair has an edge to link each other, such that the entries $e_{i,j}$ in edge set $\mathcal{E} $ are all 1. Hence set $\mathcal{E} $ means that the entire output features are interacted each other. Next, we show how to obtain the weight adjacency matrix from the graph attention mechanism. Specially, we consider the weights in the adjacency matrix as the degree of similarity between pairs of vertices. There are two main approaches to graph attention mechanism. One approach leverages the explicit attention mechanism to obtain the attention weights such as the cosine similarity between different vertex pairs \cite{Ref29}. The other approach does not rely on any prior information and leverages complete parameter learning to gain attention weight \cite{Ref30}. In this study, we take the first approach as our graph attention layer. Compared to \cite{Ref30} used in \cite{Ref20}, the symmetric weight adjacency matrix learned in \cite{Ref29} can save half the computation in the graph attention layer. The GAT process can be described as:
$$
b_i=Wx_i, \eqno(8) \label{eq8}
$$
$$
a\left( i,j \right) =\frac{\exp \left( \beta \cos \left( b_i,b_j \right) \right)}{\sum\nolimits_{k=1}^N{\exp \left( \beta \cos \left( b_i,b_j \right) \right)}}, \eqno(9) \label{eq9}
$$
where $x_i\in \mathbb{R} ^F$ is the vertex feature, and $W\in \mathbb{R} ^{F'\times F}$ is the projection matrix, $\beta $ is the learnable parameter. In the GAT layer, the vertex feature is first projected into $F'$ dimensional space via multiplying $W$. Then the attention score is obtained by equation (9). 
\subsection{Message and vertex update function}
\par In this study, we formulate the message function as 
$$
M_t=AH_{t-1}^{\top}, \eqno(10) \label{eq10}
$$
where $H_t\in \mathbb{R} ^{F'\times N}$ is the set of the hidden states $h$, $M_t$ is the aggregated states set, and the subscript $t$ is the $t$th massage aggregation, $\top $ is the transpose symbol. (10) can also be considered as the one order graph shift. When $t=1$, $H_{t-1}$ is the projected graph vertex features set $B=\left\{ b_i \right\} ,i=1,...,N$. 
\par The vertex update function is defined as the nonlinear transformation of aggregated representation:
$$
H_t=\sigma \left( LN\left( MLP_t\left( M_t \right) \right) \right) , \eqno(11) \label{eq11}
$$
where $\sigma \left( \cdot \right) $ is the non-linear activation, $LN\left( \cdot \right) $ is the layer normalization function, $MLP\left( \cdot \right) $ is the multi-layer perceptron (MLP) with a set of learnable parameters.
\subsection{Readout function }
When all the hidden states are sufficiently updated, they are aggregated to a graph-level representation for the speaker voice print feature, based on which the final prediction is produced. We define the readout function as:
$$
H^T=MLP_{\theta}\left( H^{T-1} \right) \odot sigmoid\left( MLP_{\varphi}\left( H^{T-1} \right) \right) , \eqno(12) \label{eq12}
$$
$$
h_{\mathcal{G}}=\frac{1}{|\mathcal{V} |}\sum_{i=0}^T{\sum_{v\in \mathcal{V}}{h_{v}^{T}}}+\,\,\mathrm{Maxpooling} \left( H^T \right) , \eqno(13) \label{eq13}
$$
where $\odot $ is the element-wise multiplication. In the equation (13), $\frac{1}{|\mathcal{V} |}\sum_{i=0}^{T-1}{\sum_{v\in \mathcal{V}}{h_{v}^{T}}}$ denotes the residual connection in GNNs. In the residual connection, we extract the distribution information of early representation as the supplement for the final representation. The MPNN framework is shown in Figure 2.

\begin{figure}[t]
	{	\centering
		\includegraphics[width=5.2in]{graph-aggregation2.eps}\\}
	\caption{The MPNN framework.}\label{fig2}
\end{figure}




\section{EXPERIMENTS}

\begin{figure}[t]
	{	\centering
		\includegraphics[width=5.2in]{feature_weights.eps}\\}
	\caption{The weights of different hidden features.}\label{fig2}
\end{figure}
\begin{figure}[t]
	{	\centering
		\includegraphics[width=4.3in]{weight2.eps}\\}
	\caption{The weight matrices of different speakers.}\label{fig3}
\end{figure}
\subsection{Dataset}
All experiments are conducted on the VoxCeleb1\&2 datasets \cite{Ref22, Ref23}. VoxCeleb2 development set contains over a million utterances from 5994 celebrates from the YouTobe, and the average duration of a signal speaker is about 7.2 seconds. The reported performance in terms of equal error rates (EERs) is evaluated on extended (vox1-o, vox1-e, vox1-h) test sets from the VoxCeleb12. The pretrained weights1 used in the experiments with the wav2vec2.0 framework are released on Hugging-Face \cite{Ref31}.
\subsection{Model description and implementation details}
\par We conduct all experiments using the PyTorch framework, on a 3090 GPU, and take the cosine similarity as the back-end performance evaluation tool.
\par \textbf{Baselines.} In the experiments, we take some released works \cite{Ref4,Ref6,Ref32} for speaker recognition based on wav2vec2.0 framework and feature fusion methods as our baselines. The authors in \cite{Ref6} considered different pooling methods, and they are mean, max, mean\&std, quantile, first\&cls, middle, last, first, random. In this study, we use new symbols to represent them, i.e., w2v2-mean, w2v2-max, w2v2-mean\&std, w2v2-quantile, w2v2-first\&cls, w2v2-middle, w2v2-last, w2v2-first, w2v2-random. The mean pooling is also used in \cite{Ref4} for speaker recognition task, which is restamp as w2v2-mean. Graph U-Net and GRU are also considered as the baseline which are used as a pooling method for speaker recognition in \cite{Ref20,Ref32}. 
\par \textbf{Proposed method.} In the graph attention layer, we set $F'=F$, do not change the dimension of input features, and just project the input features into another space. In the vertex update and readout function, we set $T=2$, the number of hidden layer of ${MLP_t}$, $t=1,2$, $MLP_{\theta}$ and $MLP_{\varphi}$ is 1, and the dimension of hidden layer is 1024 for all the MLPs. 
\par \textbf{Implementation details.} In each experiment, we set batch size is 48, and every sample’s duration is 3 seconds sampling from the audio files. In Table II, we take the pretraining weights of w2v2-mean as the initial weights of wav2vec2.0 for the other models to accelerate the convergence process of the models. We train 30 epochs for w2v2-mean, and train 5 epochs for other models. The optimizer is Adam \cite{Ref33} with a OneCycle learning rate schedule [34], and the loss function is angular additive softmax (AAM) loss function \cite{Ref35,Ref36}. We also propose a thin version of our model, i.e., the   is removed in this version.
\par Similar to the [5], we also weight all the hidden features as:
$$
x=\frac{\sum_{i=1}^{13}{w_i}x_i}{\sum_{i=1}^{13}{w_i}}, \eqno(14) \label{eq14}
$$
where the trainable weights $w_i$ are initialized with 1.0, and the $x_i$ is the output feature of different hidden modules, where $x_1$ s the projected speaker representation, and $x_2$, $\cdots$, $x_{13}$ is the 12 Transformer modules’ output features successively.


\subsection{Results}

\par Table I shows the paraments of different module in each model. From the Table I. we can obtain that our method can reduce the paraments compared to the GRU, but still possesses more paraments than graph U-Net pooling method.
% Please add the following required packages to your document preamble:
% \usepackage{multirow}
\begin{table}[]
\caption{The number of parameter in different pooling methods.}
\begin{tabular}{llll}
            & wav2vec2 & pooling & loss\_fn \\
our         & 94.4 M   & 6.9 M   & 4.6 M    \\
our/thin    & 94.4 M   & 5.3 M   & 4.6 M    \\
Graph U-Net & 94.4 M   & 3.7 M   & 4.6 M    \\
GRU         & 94.4 M   & 7.9 M   & 6.1 M   
\end{tabular}
\end{table}
\begin{table}[]
\caption{The performance of different pooling methods.}
\begin{tabular}{lllll}

                             &                & voxceleb1 & voxceleb-e & voxceleb-h \\
\multirow{2}{*}{our}         & all\_features  & 1.79      & 1.75       & 3.2        \\
                             & final\_feature & 2.29      & 2.33       & 3.9        \\
\multirow{2}{*}{our/thin}    & all\_features  & 1.83      & 1.88       & 3.22       \\
                             & final\_feature & 1.96      & 1.84       & 3.36       \\
\multirow{2}{*}{Graph U-Net} & all\_features  & 2.07      & 2.2        & 3.99       \\
                             & final\_feature & 1.9       & 1.84       & 3.36       \\
\multirow{2}{*}{GRU}         & all\_features  & 2.1       & 2.07       & 3.92       \\
                             & final\_feature & 2.2       & 2.17       & 4.05       \\
\multirow{2}{*}{max}         & all\_features  & 2.09      & 2.03       & 3.65       \\
                             & final\_feature & 2.04      & 2.03       & 3.65       \\
\multirow{2}{*}{mean}        & all\_features  & 2.14      & 1.9        & 3.89       \\
                             & final\_feature & 1.89      & 1.85       & 3.42      

\end{tabular}
\end{table}

 
\par Table II shows the performance for different pooling methods based on the wav2vec2.0 framework so that the benefits of proposed methods are assessed. From the Table II, we can obtain that our method can provide a better performance compared to other fusion methods. Our thin model provides a comparable performance and reduce about 23\% parameters compared to the original version. 
\par We also notice that if the only final feature is used as the output feature, the performance of our models and GRU will be degraded, especially in our original model. However, these phenomena are not obvious in other models, and some models even show the opposite results, such as mean, max, graph U-net model. These models all has less parameters than our models and GRU. The number of parameters can determine the upper limit of the expressive power of a model. We believe that the additional information of weighted features beyond the capacity of these models, the extra information may confuse the backend classifier and degrade the performance.
\par Figure 2 shows different weights of each hidden feature, we notice that the final output feature possesses the largest weight compared to other output features in all the models, which is different from the \cite{Ref5}. In the \cite{Ref5}, the larger weights are clustered in the middle features, which means that the emotion features are different from the identity features of a speaker and that is consistent with common sense.
\par It’s worth noting that the GRU based backend classifier does not show the competitive performance compared to other methods. That’s an interesting phenomenon. GRU fusion technique has shown an excellent performance in many models, including speech recognition \cite{Ref37,Ref38,Ref39}, speech emotion recognition \cite{Ref40,Ref41}, speech enhancement \cite{Ref42,Ref43}, speaker recognition \cite{Ref32,Ref44}, etc. We notice that most of these models put the GRU modules right behind convolution modules. That’s reasonable and effective, the convolution modules extract latent speech representation and GRU modules further fused the information in temporal dimension. However, in the wav2vec2.0 framework, the extracted latent representations from the convolution modules have been fed to the Transformer modules, which can also focus the temporal information and has been proved to be excellent at this \cite{Ref7}. Therefore, it is difficult for the GRU modules to extract extra temporal information from the output features of Transformer modules. That is the main reason for GRU possesses the largest number of parameters, but do not show the competitive performance.
\par Mean and max fusion methods mainly focus on the statistical information of output features and the representative features, respectively. These two methods both provide extra information in non-temporal dimension. That is the reason why these two methods can outperform than GRU.
\par GNN uses graph message passing and vertex updating mechanism to obtain the graph structure information in the non-Euclidean space. In Figure 4, each subfigure shows the graph adjacency matrix in different utterances. From the Figure 4, we can obtain that each center feature has high weights with its adjacent features, which means that our GNN can extract the temporal information. Besides, it’s worth noting that the high similarity has no obvious distance property, i.e., the high weights also appear in some features that are far away from the center features. That means that GNN can not only fuse its adjacency features, but also some important features even in a long distance.  Compared to mean, middle, first and random fusion method that allocate equal weight to each feature and the maximum weight to a random feature, our method can treat different features with different weights. In subsection 3.1, we have proved that the mean, max and random fusion method are the special case of GNN and Figure 3 further concretes this experimentally. 

\section{CONCLUSION}
\par In this paper, we mainly take the wav2vec2.0 framework as the speaker feature extractor to apply to speaker recognition task and then investigate the graph neural network as the backend processing tool to aggregate the speaker features. Specifically, we first show that our motivation is reasonable by proving some classical pooling methods can be expressed in the GSP form or GNN form. Then, we obtain the graph structure by GAT and we follow the MPNN framework to design our GNN including message function, vertex update function and readout function. Finally, we evaluate our model on Voxceleb1 dataset, the experiments show the GNN can obtain better performance than the classical pooling method, GRU feature fusion method and the other GNN applied to the speaker recognition task. In experiment results, we show why our proposed method can obtain the best performance compared to other methods.


\bibliography{mybibfile}

\end{document}
 % outcomment this line in Case 2


\end{document}


