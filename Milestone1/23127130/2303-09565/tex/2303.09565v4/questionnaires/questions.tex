\documentclass{report}
\usepackage[a4paper, total={7.5in, 10in}]{geometry}
\usepackage[T1]{fontenc}
\usepackage[polish]{babel}
\usepackage[utf8]{inputenc}
\usepackage{graphicx}
\usepackage{enumitem}

  
\newlist{myprose}{itemize}{1}
\setlist[myprose,1]{
  leftmargin=-10pt,
  label=(\alph*),
  align=left,
  itemindent=\dimexpr\parindent+\labelwidth+\labelsep\relax}

\begin{document}

\begin{center}
    \huge Questionnaire for suitability and applicability assessment of Simulation-Physical Systems Development Procedure.
\end{center}  
\textbf{Participants}:
\begin{enumerate}
    \item Maciej Groszyk 
     \item Michał Walęcki  
     \item Łukasz Dudek 
     \item Jacek Paszek 
     \item Dawid Seredyński 
     \item Maciej Więgierek 
     \item Kamila Lis 
     \item Jan Kaniuka 
     \item Stanislau Stankevich 
     \item Mateusz Zembroń 
     \item Daniel Giełdowski 
     \item Adam Krawczyk 
     \item Tomasz Indeka 
     \item Maciej Bogusz 
     \item Jakub Sikora 
     \item Jakub Ostrysz
     \item Maciek Święch
     \item Gabriel Brzeziński
    \item Adrian Brzozowski 
    \item Grzegorz Bojarczuk 
    \item Dominik Belter?
    \item Mateusz Cholewiński?
    \item Vitalii Kutia?
    
\end{enumerate}
\textbf{Questions}:
\begin{enumerate}
    \item Your experience questions:
    \begin{enumerate}
        \item How long do you develop robotic or other Cyber-Physical Systems? (months)
        \item How long do you manage technology development teams? (months)
        \item What is/are your technology domain? Choose one or more from: 
        \begin{itemize}
            \item robotics, 
            \item automotive, 
            \item avionics, 
            \item another cyber-physical system, 
            \item production automation, 
            \item information systems, 
            \item AI
            \item other: 
            \begin{itemize}
                \item write your domains:
            \end{itemize}
        \end{itemize}
        \item How many setups had the most complicated system you developed, or how big was the product family of the system?       
        \item Were you familiar with the Digital Twin concept before the presentation?
        \item In which phase of the system life cycle have you utilized simulation technology, including digital twin? 
        
        Choose one or more from:
        \begin{itemize}
            \item analysis
            \item design,
            \item development,
            \item testing,
            \item implementation,
            \item maintenance,
            \item evaluation,
            \item other: 
            \begin{itemize}
                \item write life cycle phases:
            \end{itemize}
        \end{itemize}  
        \item Are you a member of a science or engineering organisation? Choose one or more from: Write your biggest success in the systems engineering
        \begin{itemize}
            \item IEEE, 
            \item INCOSE
            \item Other: which one?
        \end{itemize}
\item Did you receive a certificate or graduate a university in the field of systems engineering or related? Yes (which) / No
\item Write your biggest success in the systems engineering domain.

            \item How valuable is it for you to join simulation and physical models? (1-5) 
    \end{enumerate}
    \item SPSysML questions:
    \begin{enumerate}
        \item Is the requirement-based decomposition applicable to your technology domain? (true/false) If not, please write why.
        \item How sure are you that the requirement-based decomposition strengthens traceability between system components and the requirements? (0-100\%)
   \item How sure are you that the requirement-based decomposition would help you in comprehensiveness analysis and verification? (0-100\%) 
            \item If you would be developing a cyber-physical system including simulation components, which of the following parts of the methodology would you apply? Write why or why not.
            \begin{itemize}
            \item  SPSys Modelling Language as a modelling language to specify and share the system design
                 \begin{itemize}
                     \item (please write why)
                     \item (please write why not)
                 \end{itemize}
            
                 \item requirement-based decomposition 
                 \begin{itemize}
                     \item (please write why)
                     \item (please write why not)
                 \end{itemize}
                \item agent-based architecture 
                 \begin{itemize}
                     \item (please write why)
                     \item (please write why not)
                 \end{itemize}
                \item simulation-physical integrity factors evaluation for the design evaluation 
                 \begin{itemize}
                     \item (please write why)
                     \item (please write why not)
                 \end{itemize}
                \item systematic procedure (based on the simulation-implementation factor) for planning and analysing components development order 
                 \begin{itemize}
                     \item (please write why)
                     \item (please write why not)
                 \end{itemize}
            \end{itemize}
            \item Do you agree that:
            \begin{itemize}
                \item \textbf{Controller integrity factor} is a~significant evaluation factor for simulation-physical systems? (yes/no) 
                \item \textbf{Driver generalisation factor} is a~significant evaluation factor for simulation-physical systems? (yes/no) 
                \item \textbf{Digital Twin coverage} is a~significant evaluation factor for simulation-physical systems? (yes/no) 
                \item \textbf{Mirror integrity factor} is a~significant evaluation factor for simulation-physical systems? (yes/no) 
            \end{itemize}
            \item If you were a designer of a complex cyber-physical system project, how profitable would it be to you to measure:
            \begin{itemize}
                \item the \textbf{Controller integrity factor} and change the system design to control it (enter your probability 0-100\%)
                \item the \textbf{Driver generalisation factor} and change the system design to control it (enter your probability 0-100\%)
                \item the \textbf{Digital Twin coverage} and control it while developing the system (enter your probability 0-100\%)
                \item the \textbf{Mirror integrity factor} and control it while developing the system (enter your probability 0-100\%)
            \end{itemize}
     
        \item How clear (1-5) are the rules for improving the system based on the proposed design evaluation factors?
        
           Clarity scale:
            \begin{itemize}
                \item[1] -- It is completely fuzzy. I cannot even comment on it. (Please tell us what aspects are clear,  if any)
                \item[2] -- I get the general concept of it. I can give abstract comments and relate it roughly to my knowledge. (Please tell us what was not deep enough presented)
                \item[3] -- I understand most of it but can give hesitant comments. (Please tell us what is fuzzy to you)
                \item[4] -- I feel confident to comment on most of it. (Please tell us what is not clear to you)
                \item[5] -- I feel confident giving precise comments on all its aspects.
            \end{itemize}

        \item Based on your experience and knowledge, assess the probability of the integrity improvement between simulation and physical embodiments in subsequent design iterations by applying the methodology and:
        \begin{itemize}
                \item the \textbf{Controller integrity factor} and change the system design to control it (enter your probability 0-100\%)
                \item the \textbf{Driver generalisation factor} and change the system design to control it (enter your probability 0-100\%)
                \item the \textbf{Digital Twin coverage} and control it while developing the system (enter your probability 0-100\%)
                \item the \textbf{Mirror integrity factor} and control it while developing the system (enter your probability 0-100\%)
        \end{itemize}
        
        \item Based on your experience and knowledge, does the method comply with the V model?
        \begin{itemize}
            \item No, (Tell us why)
            \item Yes, (Tell us why)
            \item I don't know the V-model,
            \item other: (please write your answer)
        \end{itemize}

        \item Do you see profits in joining simulation and physical models as SPSysML proposes?
        \begin{itemize}
            \item No, (Tell us why)
            \item Yes, (Tell us what profits do you see)
            \item other: (please write your answer)
        \end{itemize}
        \item If you were a complex cyber-physical system project designer, how would SPSysML and SPSysDP help you in change analysis? Choose one or more from:
        \begin{itemize}
            \item component change propagation analysis,
            \item promote wider simulation-based change verification,
            \item promote wider simulation-based change validation,
            \item tracing change from a requirement to the system components' attributes,
            \item communicating change impact,
            \item other: (please write your answer)
            
        \end{itemize}Write your answer

        \item If you were a complex cyber-physical system project designer, what features of SPSysML would help you maintain the system?
        \begin{itemize}
            \item SPSys Modelling Language as a modelling language to specify changes and share the system design
            \item requirement-based decomposition
            \item other: (please write your answer)
            \item none
        \end{itemize}
        \item If you were a complex cyber-physical system project designer, how sure are you that SPSysML would help you maintain the system (0-100\%)?
        \item What difficulties do you see in the application of:
        \begin{itemize}
            \item SPSysML as a~language for system specification and presentation,
            \item SPSysDP as a~development procedure,
            \item the proposed design evaluation factors as design assessment markers,
            \item the requirement-based decomposition in finding systems' setups, their components and simulation-physical classes of these components,
            \item the proposed requirement model in tracing component requirements,
        \end{itemize}
        \item What is the SPSysML and SPSysDP usability level for each Zachman framework cell Fig.~\ref{fig:zachman}? Assign your rating (1-5) to each matrix cell. (1: does not apply, 2: applies with no profit, 3: I have no opinion, 4: applies with some profit, 5: applies with clear profit)
        \begin{figure}[tbh]
            \centering
            \includegraphics[width=\linewidth]{questionnaires/image.png}
            \caption{Zachman framework}
            \label{fig:zachman}
        \end{figure}
        \item Assess against the following qualities of great models how SPSysML and SPSysDP affect the quality of the resulting models (1: decrease significantly, 2: seem to decrease, 3: not affect, 4: seem to increase, 5: increase significantly):
        {\small \begin{itemize}[leftmargin=-10mm]
        \item DECISIONS
        \begin{itemize}
            \item \textbf{Linked to Decision Support:} denotes the fundamental quality of models that seamlessly integrate into decision-making frameworks. Exceptional models within this paradigm serve as invaluable tools for navigating scenarios where numerous parameters must be carefully balanced. They excel in elucidating the precise manner in which their outputs inform and guide decision-making processes.
            \item \textbf{Model Credibility:} signifies the degree to which decision-makers trust the results produced by a model. The credibility of a model is pivotal, as it directly impacts the decision-making process. When models lack credibility, decision-makers are hesitant to base their decisions on their outputs, potentially leading to squandered time and resources or even jeopardizing the project's success. Building model credibility encompasses various strategies, including the establishment of rigorous standards and processes to evaluate and validate the model's performance.
            
        \end{itemize}
        \item SCOPE
        \begin{itemize}
            \item \textbf{Clear Scope:} entails defining the extent of system modeling required for a project. This involves determining which system or subsystem will be modeled, ensuring alignment with project objectives. For instance, in automotive design, it could mean broad modeling of all engine components initially, followed by narrower modeling focusing on individual components with stable interfaces.
            \item \textbf{Verification \& Validation With Models:} refers to the process of utilizing models to verify and validate products, processes, or businesses. A quality systems engineering model should explicitly demonstrate why modeling is the preferred route for verification and validation tasks. However, it's essential to note that relying solely on a single model for verification and validation is discouraged.
        \end{itemize}
        \item USABILITY
        \begin{itemize}
            \item \textbf{Understandable and Well-Organized:} entails clarity regarding where and how additional components can be integrated into the model. Following a structured Model Development Process enhances modularity and organization, fostering greater comprehensibility.
            \item \textbf{Analyzable and Traceable:} this quality denotes models that can be readily interrogated and offer clear insight into the factors influencing their outputs. They enable easy identification of the specific variables or sections of the model that contribute to the results.
            \item \textbf{Data Extrapolation:} refers to the capability of models to operate within predefined boundaries of data, conditions, physics, and assumptions. Great models explicitly delineate their validity range, distinguishing where they are applicable and where they are not.
        \end{itemize}
        \item QUALITIES
        \begin{itemize}
            \item \textbf{Complete Relative to Scope and Intended Purpose:} signifies that the model comprehensively encompasses all pertinent physics or dynamics within its defined scope and purpose. 
            \item \textbf{Internally Consistent:} denotes a model that maintains coherence without direct contradictions throughout its components. For instance, assumptions such as the gravity constant remain consistent across all sections of the model.
            \item \textbf{Verifiable:} signifies that the model's outputs can be verified to meet the modeling requirements and align with calibration data, enhancing its credibility for decision-making purposes.
            \item \textbf{Validation:} entails ensuring that the model aligns with and satisfies customer needs and expectations. For descriptive models, this involves effective presentation of information, often through layered or unlayered approaches. For analytical models, validation requires demonstrating their efficacy in enhancing decision-making processes as intended.
        \end{itemize}
        \item IMPLEMENTATION
        \begin{itemize}
            \item \textbf{Model Fidelity:} refers to ensuring that the model possesses the correct level of detail in relation to the decision being made and the design phase. Excessive fidelity can complicate evaluation and waste resources, while insufficient fidelity can lead to flawed decisions or unwarranted confidence. Selecting the appropriate fidelity level hinges on the system requirements and operational parameters.
            \item \textbf{Elegant:} denotes a model crafted with a balance of simplicity and effectiveness, avoiding unnecessary complexity. For instance, an elegant model minimizes redundancy, such as by storing and reusing computed results rather than recalculating them repeatedly from the same data.
            \item \textbf{Well Formed for Optimization:} refers to the construction of a model to enable optimization if required. It entails ensuring that the model provides pertinent optimization information, such as gradients or convexity, to facilitate efficient optimization processes.
            \item \textbf{Avoid Optimizing on a Black Box:} advises against optimizing models that operate as "black boxes" with obscure or inaccessible internal features. Such optimization routines may perform inadequately when applied to black box models. Instead, optimization processes should leverage the explicit structure and features of the model whenever feasible.
        \end{itemize}
        \item INTERCHANGE
        \begin{itemize}
            \item \textbf{Availability of Interfaces:} signifies that great models offer readily accessible interfaces to interact with underlying data and outputs and components.
            \item \textbf{Reusable:} implies designing models to be applicable across various systems or scenarios beyond their initial creation context. Achieving this involves adopting a modular model structure and avoiding hardcoded parameters. While model reuse can expedite product development and reduce costs, it may also introduce risks if applied beyond the validated range of applicability.
        \end{itemize}
        \end{myprose}
        }
    \end{enumerate}
        \item Methodology presentation quality:

    Clarity scale:
            \begin{itemize}
                \item[1] -- It is completely fuzzy. I cannot even comment on it. (Please tell us what aspects are clear, if any)
                \item[2] -- I get the general concept of it. I can give abstract comments and relate it roughly to my knowledge. (Please tell us what was not deep enough presented)
                \item[3] -- I understand most of it but can give hesitant comments on it, (Please tell us what is fuzzy to you)
                \item[4] -- I feel confident to give comments on most of it, (Please tell us what is not clear to you)
                \item[5] -- I feel confident in giving precise comments on all its aspects.
            \end{itemize}
    
    \begin{enumerate}
        \item How clear is the motivation of Simulation-Physical System (SPSys) to you (1-5)?
        \item How clear is the SPSys idea to you (1-5)?
        \item How clear is the requirement-based decomposition to you (1-5)?
        \item How clear is the motivation of SPSys Development Procedure to you (1-5)?
        \item How clear is the idea of SPSys Development Procedure to you (1-5)?
        \item How clear are the steps of the SPSys Development Procedure to you (1-5)?
        \item How clear is the idea of evaluation factors to you (1-5)?
        \item How clear are the relationships of SPSysML and SPSysDP with the following tools?
        \begin{itemize}
            \item V-model (1-5), 
            \item Zachman framework (1-5), 
            \item SysML (1-5).
        \end{itemize}
    \end{enumerate}
\end{enumerate}
\end{document}