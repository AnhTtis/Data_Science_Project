\documentclass[10pt,twocolumn,letterpaper]{article}

\usepackage{iccv}
\usepackage{times}
\usepackage{epsfig}
\usepackage{graphicx}
\usepackage{amsmath}
\usepackage{amssymb}
\usepackage{bm}
\usepackage{subfigure}
\usepackage{booktabs}
% Include other packages here, before hyperref.

% If you comment hyperref and then uncomment it, you should delete
% egpaper.aux before re-running latex.  (Or just hit 'q' on the first latex
% run, let it finish, and you should be clear).
% \usepackage[breaklinks=true,bookmarks=false]{hyperref}
\usepackage[pagebackref=true,breaklinks=true,letterpaper=true,colorlinks,bookmarks=false]{hyperref}

\iccvfinalcopy % *** Uncomment this line for the final submission

\def\iccvPaperID{10968} % *** Enter the ICCV Paper ID here
\def\httilde{\mbox{\tt\raisebox{-.5ex}{\symbol{126}}}}

% Pages are numbered in submission mode, and unnumbered in camera-ready
\ificcvfinal\pagestyle{empty}\fi

\begin{document}

%%%%%%%%% TITLE
% \title{Target-dominant-based Diffusion model for unsupervised domain adaptation}
% \title{Generate Target-dominant samples for unsupervised domain adaptation}
% \title{Target-dominant Diffusion generative model for unsupervised domain adaptation}
% \title{Target Diffusion: Generate Target Samples for Unsupervised Domain Adaptation}
\title{Diffusion-based Target Sampler for Unsupervised Domain Adaptation}

\author{Yulong Zhang$^1$\thanks{Equal contribution.} \quad Shuhao Chen$^2$\footnotemark[1] \quad Yu Zhang$^{2}$\thanks{Corresponding author.} \quad Jiangang Lu$^1$\\
% College of Control Science and Engineering,
$^1$Zhejiang University\\
$^2$Southern University of Science and Technology\\
\!\!\!{\tt\small zhangylcse@zju.edu.cn, 12232388@mail.sustech.edu.cn, yu.zhang.ust@gmail.com, lujg@zju.edu.cn}
% For a paper whose authors are all at the same institution,
% omit the following lines up until the closing ``}''.
% Additional authors and addresses can be added with ``\and'',
% just like the second author.
% To save space, use either the email address or home page, not both
% \and
% Shuhao Chen\\
% Southern University of Science and Technology\\
% First line of institution2 address\\
% {\tt\small 12232388@mail.sustech.edu.cn}
}

\maketitle
% Remove page # from the first page of camera-ready.
\ificcvfinal\thispagestyle{empty}\fi

%%%%%%%%% ABSTRACT
\begin{abstract}
Limited transferability hinders the performance of deep learning models when applied to new application scenarios.
Recently, unsupervised domain adaptation (UDA) has achieved significant progress in addressing this issue via learning domain-invariant features.
However, large domain shifts and the sample scarcity in the target domain make existing UDA methods achieve suboptimal performance.
To alleviate these issues, we propose a plug-and-play Diffusion-based Target Sampler (DTS) to generate high fidelity and diversity pseudo target samples.
By introducing class-conditional information, the labels of the generated target samples can be controlled.
The generated samples can well simulate the data distribution of the target domain and help existing UDA methods transfer from the source domain to the target domain more easily, thus improving the transfer performance.
Extensive experiments on various benchmarks demonstrate that the performance of existing UDA methods can be greatly improved through the proposed DTS method.
The code will be released soon.
\end{abstract}

%%%%%%%%% BODY TEXT
\section{Introduction}
With the development of data collection and computing facilities, deep learning models have achieved remarkable advances in a variety of applications for their powerful representation learning capabilities~\cite{wang2023energyinspired, Shi_2021_ICCV, subramanian2022generalization}.
However, when domain shifts occur, the well-trained models in a source domain will suffer significant performance degradation in a target domain.
In this case, large-scale target samples need to be recollected and labeled, which is time-consuming and expensive.


Unsupervised Domain Adaptation (UDA) \cite{yang2020transfer} leverages transferable knowledge from the source domain and applies it to an unlabeled target domain to improve the reusability of existing models and data \cite{sun2022prior, goyal2022test, jiang2022transferability}.
To alleviate domain shifts, existing UDA methods mainly learn a domain-invariant feature representation explicitly or implicitly.
% Specifically, the explicit methods minimize the distribution discrepancy between the source and target domains through some distance measurement techniques
Specifically, some UDA methods explicitly  minimize the distribution discrepancy based on distance measures between the source and target domains \cite{ouyang2021maximum, shen2018wasserstein, NEURIPS2021_ae0909a3}.
Inspired by generative adversarial networks (GAN) \cite{goodfellow2014generative}, a domain discriminator is introduced to bridge the source and target domains implicitly through adversarial training \cite{zhang2022spectral, chen2022reusing, rangwani2022closer}.
However, under the UDA setting where the domain shift occurs, the limited or even scarce target samples cannot represent the data distribution in the target domain accurately, which limits the performance of existing UDA methods.
In other words, direct domain alignments based on source samples and scarce unlabeled target samples will lead to suboptimal transfer effects.

% To address the abovementioned problems, this paper aims to generate target-dominant samples which are similar to the target domain samples.
% To alleviate the abovementioned difficulties, the purpose of this paper is to generate target-dominant samples that obey the target domain distribution.
To address the aforementioned limitations, in this paper, we propose a Diffusion-based Target Sampler (DTS) for UDA.  The proposed DTS is to generate pseudo target samples that could follow the target distribution.
In this way, target samples can be augmented with pseudo target samples and they could improve the performance of UDA models.
%Compared with the original target domain, the feature space of the target domain can be more accurately described by the generated samples.
%The performance of existing UDA models can be improved with the help of the generated target samples.
% To this end, the diffusion model has recently been shown to generate high quality images, and we introduce it to generate high quality samples.
% To this end, diffusion model has recently proved to achieve excellent performance in the field of image generation, and we introduce it to generate high-quality samples.
Specifically, the DTS method adopts the Diffusion Probabilistic Model (DPM) \cite{ho2020denoising, dhariwal2021diffusion}, which has recently been proposed to achieve superior performance in image generation, to generate high-quality target samples.
Based on pseudo labels given by existing UDA methods for target samples, the DPM is trained conditionally to achieve class-conditional generation, and hence the pseudo labels of generated samples can be naturally determined.
% And then, the generated target-dominant samples and the original source domain samples are used as the intermediate domain, which is closer to the target domain and reduces the difficulty of domain adaptation.
Then generated target samples and original source samples are combined as the augmented source domain, where original source samples are used to suppress the effect of noisy labels of generated target samples.
% which is closer to the target domain and reduces the difficulty of domain adaptation.
% which makes the distribution of the augmented source domain closer to the target domain and reduces the difficulty of domain adaptation.
In this manner, the distribution of the augmented source domain is closer to the target domain, which reduces the difficulty of domain adaptation (DA).
% Note that the proposed framework is a plug-and-play method that can be embedded into any existing UDA methods and improve their performance, which has rarely been investigated in the literature.
Note that the proposed DTS framework is a plug-and-play module that can be embedded into any existing UDA methods to improve their transfer performance, which has rarely been investigated in the literature.
We combine the proposed DTS framework with state-of-the-art UDA methods and conduct extensive experiments on three UDA benchmark datasets to demonstrate the superiority of the proposed DTS framework.

The contributions of this paper are three-fold.
\begin{itemize}
% \item We propose a novel target-dominant diffusion generative framework that generate high fidelity and diversity samples that are similar to the target domain.
\item  We propose a novel DTS method that directly generates high fidelity and diversity samples that are likely to follow the distribution of the target domain. To the best of our knowledge, this is the first application of DPMs in UDA.
\item The proposed plug-and-play DTS framework can be embedded into any existing UDA method to improve their transfer performance.
\item Extensive experiments on UDA benchmark datasets  demonstrate the effectiveness of the proposed DTS framework. The augmented source domain consisting of source samples and generated target samples can help reduce the domain discrepancy.
% the proposed method can effectively alleviate the domain shift. The effectiveness of this method is verified by comparison with the state-of-the-art methods.
\end{itemize}

\section{Related Work} \label{section:RW}
\subsection{Unsupervised Domain Adaptation}
% Unsupervised Domain Adaptation defines -> UDA methods -> some problem of UDA methods -> someone use generate model to deal with this problem -> however, the methods above don't consider some problems -> In this paper, we use DPM to deal with this problem.
UDA methods \cite{zhao2020review, long2015learning, ganin2016domain} extract knowledge from the labeled source domain to facilitate learning in the unlabeled target domain.
As the distribution of the target domain differs from that of the source domain, various methods are proposed to reduce the domain discrepancy and they can be mainly classified into two categories: discrepancy-based methods and adversarial-based methods.

Discrepancy-based methods learn a feature extractor to minimize the distribution discrepancy between the source and target domains.
For example, \cite{long2015learning, zhu2020deep} try to minimize the maximum mean discrepancy (MMD) \cite{gretton2012kernel} across domains, while MCC \cite{jin2020minimum} minimizes the class confusion loss without explicit domain alignment.
Margin disparity discrepancy (MDD) \cite{MDD} is to perform domain alignment with analyses in terms of generalization bounds.
\cite{shen2018wasserstein} uses the Wasserstein distance for distributional alignment to achieve better transfer effects.

On the other hand, adversarial-based methods align distributions across domains through adversarial training \cite{goodfellow2014explaining}.
For example, DANN \cite{ganin2016domain} adversarially learns a domain discriminator to distinguish samples in two domains and thus enables the feature extractor to confuse the domain discriminator. % while guaranteeing feature representation capability
CDAN \cite{long2018conditional} injects class-specific information into the discriminator to facilitate the alignment of multi-modal distributions.
SDAT \cite{rangwani2022closer} uses sharpness-aware minimization \cite{foret2020sharpness} to seek a flat minimize for better generalization, while ELS \cite{zhang2023free} improves over SDAT by introducing label smoothing to domain labels and achieves better transfer performance.

Adversarial training can also be used to generate samples to bridge the source and target domains.
CyCADA \cite{hoffman2018cycada} adopts GAN for image-to-image translation via the cycle consistency loss \cite{zhu2017unpaired} and semantic consistency loss.
Specifically, it directly changes the style of source images to that of the target domain to obtain a better classifier for the target domain.
\cite{yang2020bi} uses two cross-domain generators to synthesize data of each domain  conditioned on the other, and learns two task-specific classifiers.
\cite{gao2022middlegan} bridges the source and target domains by generating the intermediate domain which is similar to both domains.
% The abovementioned GAN-based approaches aim to generate images through adversarial training, which is unstable in training and requires careful hyperparameters and regularizers selection process to achieve convergence.
% Furthermore, GAN is difficult to cover the entire data distribution \cite{mescheder2018training}.
% Different from them, the proposed DTS method is easy to implement as a plug-and-play framework.
Compared with those methods that generate an intermediate domain to interpolate between the distributions of the source and target domains, the proposed DTS framework directly generates pseudo target samples that obey the target distribution.
Instead of using adversarial training strategies, the proposed DTS framework is based on the DPM, which has better generation capabilities and is easier to converge in the training process \cite{dhariwal2021diffusion}.
%the proposed DTS method is easier to implement as a plug-and-play framework.
Moreover, the generated categories and the number of generated samples can be flexibly controlled by DTS, which is more advantageous than the UDA methods based on image-to-image translation.
% to make the generated image similar to both the source domain and target domain.
% And DTS directly generates target samples without image-to-image translation, which makes the category and number of samples to be flexibly controlled.

\subsection{Diffusion Probabilistic Model}
% Denoising Diffusion Probalilistic Models (DDPM)
% \subsection{Denoising Diffusion Probalilistic Model}
% DDPM  -> conditional -> Accelarate
% Denoising Diffusion Probalilistic Models (DDPM)
DPM \cite{sohl2015deep} has achieved great success in various generative tasks in recent years.
Acting as the Markov chain, DPM includes a forward diffusion process and a reverse denoising process, which convert the original data into Gaussian noise and deduce the Gaussian noise into the original data gradually.
Specifically, the forward process gradually adds Gaussian noise to the input image $\bm{x}_0$ until the image is completely converted to a random noise $\bm{x}_T$,
%The image of the later time step is obtained by adding noise to the image of the previous time step,
i.e., $q\left( {{\bm{x}_t}| {{\bm{x}_{t - 1}}}} \right) = \mathcal{N} \left( {\sqrt {1 - {\beta _t}} {\bm{x}_{t - 1}},{\beta _t}\bf{I} } \right)$, where $\mathcal{N}(\cdot,\cdot)$ denotes a normal distribution with mean and variance specified by arguments, and ${\beta _t}$ increases gradually according to a variance schedule.
The noise obtained in the forward process can be regarded as a series of labels which the model ${\epsilon _\theta }\left( {{\bm{x}_t},t} \right)$ is trained to predict in the reverse process, where $\theta$ denotes the parameters of the model.
In practice, the reverse process can be formulated as optimizing a variational upper bound on the negative log likelihood, which can be formalized as the Kullback-Leibler (KL) divergence \cite{ho2020denoising, song2020denoising, nichol2021improved,huang2021variational}.
After the training process, the model can gradually denoise a random noise $\bm{x}_T$ to a generated image $\bm{x}_0$ according to ${\bm{x}_{t - 1}} = \frac{1}{{\sqrt {{\alpha _t}} }}\left( {{\bm{x}_t} - \frac{{1 - {\alpha _t}}}{{\sqrt {1 - {{\bar \alpha }_t}} }}{{\rm{\epsilon}}_\theta }\left( {{\bm{x}_t},t} \right)} \right) + {\sigma _t}z$, where $z \sim \mathcal{N}\left( {0,\bf{I}} \right)$, ${\alpha _t} = 1 - {\beta _t}$, ${{\bar \alpha }_t} = \prod\nolimits_{s = 0}^t {{\alpha _s}} $, and ${\sigma _t} = \frac{{1 - {{\bar \alpha }_{t - 1}}}}{{1 - {{\bar \alpha }_t}}}{\beta _t}$.

% Since the target domain is unlabeled, we hope that the label of the generated target samples can be controlled.
% In order to make the diffusion model class-conditional, the label information can be injected into the DPM in the same way as the timestep $t$ through the label embedding $v_i$ \cite{nichol2021improved}.
% The new embedding $emb$ can be obtained by the sum of the time embedding $e_i$ and label embedding $v_i$.
% Then the $emb$ will be fed into residual blocks of the model ${\epsilon _\theta }\left( {{x_t},t} \right)$ to achieve class-conditional generation.
% \cite{dhariwal2021diffusion} incorporated a classifier model to guide the diffusion toward the class $y$ during training.
% \cite{ho2022classifier} trained a conditional diffusion model together with an unconditional diffusion model instead of a second classifier model.

\begin{figure*}[t]
\centering
\includegraphics[width=6.5in]{diffusion2.eps}
% \caption{An overview of the proposed DTS framework. With the pretrained target classifier, we train the CDPM in the target domain. Then the pseudo target domain $D_g$ can be obtained through sampling. The final classifier can be trained by transferring from the augmented source domain ${D_{s \cup g}}$ to the target domain$D_t$.}
\caption{An overview of the proposed DTS framework. In step 1, we pretrain the target classifier with some existing UDA methods. In step 2, with the pretrained classifier, we obtain the pseudo labels of target domain samples to train CDPM $\epsilon _\theta$. In step 3, the generated target domain $\mathcal{D}_g$ can be obtained through sampling the pretrained CDPM $\epsilon _\theta^{*}$ in step 2. Finally, the final classifier is trained by transferring from the augmented source domain ${\mathcal{D}_{\hat{s}}}$ to the target domain $\mathcal{D}_t$.}
\label{model}
\end{figure*}

Although DPM has obtained superior performance in image generation, it still has the problem of slow sampling speed \cite{nichol2021improved, lu2022dpmsolver} due to thousands of denoising steps required to generate a sample of high quality, which greatly hinders the application of DPM \cite{ho2020denoising, lu2022dpm, yang2022diffusion}.
% To solve the issue of slow sampling speed of the original DPMs,
% To address this issue, \cite{ho2020denoising} presented denoising diffusion implicit model (DDIM) to achieve 10$\times$ to 50$\times$ sampling acceleration through non-Markov diffusion processes.
To address this issue, denoising diffusion implicit model (DDIM) \cite{ho2020denoising} achieves 10$\times$ to 50$\times$ sampling acceleration through non-Markov diffusion processes.
%  generalizes DPMs via a class of non-Markovian which can make the generative process deterministic.
% DPM-Solver \cite{lu2022dpmsolver} view DPMs as solving the corresponding diffusion ordinary different equations (ODEs), reducing the denoise process to 10-20 steps.
DPM-Solver \cite{lu2022dpmsolver} formulates DPMs as diffusion ordinary differential equations to reduce the number of denoising steps from 1000 to about 10.
Moreover, \cite{lu2022dpm} proposes DPM-Solver++ to alleviate the problem of unstable guided sampling in DPM-Solver.
In this paper, to generate class-conditional samples, we use conditional diffusion probabilistic model (CDPM) \cite{nichol2021improved} in the DTS framework. % to generate high fidelity samples.
For fast sampling, we use DPM-Solver++ to accelerate the sampling process.


\section{The DTS Method}

In UDA, we are given a labeled source domain ${\mathcal{D}_s} = \{ (\bm{x}_i^{sd},y_i^{sd})\} _{i = 1}^{{N_s}}$ and an unlabeled target domain ${\mathcal{D}_t} = \{\bm{x}_j^{td}\} _{j = 1}^{{N_t}}$.
We aim to explore useful information in $\mathcal{D}_s$ to help the prediction in $\mathcal{D}_t$. Usually $N_s$ is assumed to be much larger than $N_t$ and $N_t$ is not so large, making $\mathcal{D}_t$ not enough to model the entire data distribution of the target domain. To alleviate this problem, the proposed DTS method is to generate more target samples via DPM.
%The objective of UDA is to explore domain-invariant features and apply the knowledge of the source domain to the target domain for label prediction.
%Our goal is to generate pseudo target samples to help existing DA methods learn domain-invariant feature representations.
% And the proposed framework with diffusion-based target sampler will be discussed in detail.
% And the proposed DTS framework will be discussed in detail
% And the overall DTS framework is shown in Figure~\ref{model}, which can be divided into three steps: pretraining the classifier, training CDPM, and training the final classifier.

As illustrated in Figure~\ref{model}, the overall DTS framework can be divided into three steps.
In step 1, we use some existing UDA methods to obtain the target classifier. In step 2, pseudo labels of target samples are assigned by the pretrained classifier in step 1 and target samples with pseudo labels are used to train the CDPM $\epsilon_\theta$.
In step 3, the pretrained CDPM $\epsilon_\theta^*$ is adopted to generate target samples and those generated target samples are combined with the original source samples as the augmented source domain ${\mathcal{D}_{\hat{s}}}$. The final classifier can be obtained by transferring from the augmented source domain ${\mathcal{D}_{\hat{s}}}$ to the target domain $\mathcal{D}_t$.
% And these samples are combined with the original source domain and transfer to the target domain to obtain the final classifier.
In the following sections, we will discuss each step in detail.


%-------------------------------------------------------------------------
\subsection{Pretraining a Target Classifier}
The labels of target samples are not available under the UDA setting.
Therefore, to obtain pseudo labels of target samples, the proposed DTS framework adopts an existing UDA model to pretrain the target classifier.
Here step 1 starts by training a UDA model, whose objective function is formulated as
\begin{align}
\label{eqn1}
\!\!\!\!{f^*} \!\!=\!  \mathop {\arg \min }_{f}& \frac{1}{{{N_s}}}\!\!\sum\limits_{i = 1}^{{N_s}} \!\mathcal{L}_c (\! {f( {{\bm{x}_i^{sd}}} ),{y_i^{sd}}}\! )
\!\!+\!\!\lambda {{R(T(\mathcal{D}_s),\!T(\mathcal{D}_t)\!)}},\!\!\!
\end{align}
where $T$ denotes a feature transformation network on the source and target domains, $f(\cdot)$ denotes the classifier,
% $y_i^{sd}$ is the ground truth label of $x_i^{sd}$.
$\mathcal{L}_c(\cdot,\cdot)$ denotes the task loss in the source domain such as the cross-entropy loss, and $R(\cdot,\cdot)$ denotes a transfer regularization term such as the discrepancy-based loss \cite{ouyang2021maximum, shen2018wasserstein, NEURIPS2021_ae0909a3} and adversarial-based loss \cite{zhang2022spectral, chen2022reusing, rangwani2022closer}. %and reconstruction-based loss.
% According to (\ref{eqn1}), we can obtain the pretrained classifier in the target domain.
After solving problem (\ref{eqn1}), we can obtain the pretrained classifier $f^*$.% which could provide pseudo target labels $\{{\bar y}^{td}_j\}$ for step 2.

% \begin{equation}
% \begin{split}
% \begin{array}{l}
% \label{eqn3}
% {g^*} = \displaystyle \mathop {\arg \min }\limits_{g} \frac{1}{{{N_{\hat{s}}}}}\sum\limits_{i = 1}^{{N_{\hat{s}}}} \mathcal{L} \left( {g\left( {{\bm{x}_i^{{\hat{s}d}}}} \right),{y_i^{\hat{s}d}}} \right) \\
% \ \ \ \ \ \ \ \ \ +\lambda {{R(T(\mathcal{D}_{\hat{s}}),T(\mathcal{D}_t))}},
% \end{array}
% \end{split}
% \end{equation}

% \subsection{Training and Sampling of Diffusion Models}
\subsection{Training CDPM}
% Instead of using implicit generative models,
In step 2, we aim to simulate the target distribution via $P_\theta (\bm{x}^{td})$ with parameter $\theta$ to generate samples similar to the target domain.
% we can maximize $logP_\theta (x^{td})$ as with most generation models \cite{su2018variational}.
Recall that it is not easy to transform noise to structured data, but it is much easier to convert structured data to noise.
In particular, we can use a forward process to convert an original image $\bm{x}_0^{td}$ into a noise $\bm{x}_T^{td}$ through $T$ time steps of a stochastic encoder $q\left( {{\bm{x}_t^{td}}| {{\bm{x}_{t - 1}^{td}}}} \right)$. Then we learn a reverse process to undo this process with a decoder ${{p_\theta }\left( {{\bm{x}_{t - 1}^{td}}| {{\bm{x}^{td}_t}}} \right)}$.
To achieve this, the mainstream method is to maximize the $\ln P_\theta (\bm{x}^{td})$ \cite{su2018variational}.
To construct an upper bound of the log likelihood, we add a KL divergence to the negative log likelihood function, where for brevity the superscript $td$ of $\bm{x}^{td}$ is omitted in the following formulas, as
% \begin{equation}
% \begin{split}
% \begin{array}{l}
% \label{eqn4}
% \begin{array}{c}
% \log {p_\theta }\left( {{\bm{x}_0}} \right) = \log \left[ {\int {q\left( {{x_{1:T}}\left| {{x_0}} \right.} \right)\frac{{{p_\theta }\left( {{x_{0:T}}} \right)}}{{q\left( {{x_{1:T}}\left| {{x_0}} \right.} \right)}}} d{x_{1:T}}} \right]\\
% \ge \int {q\left( {{x_{1:T}}\left| {{x_0}} \right.} \right)\log \left( {p\left( {{x_T}} \right)\prod\limits_{t = 1}^T {\frac{{{p_\theta }\left( {{x_{t - 1}}\left| {{x_t}} \right.} \right)}}{{q\left( {{x_t}\left| {{x_{t - 1}}} \right.} \right)}}} } \right)} \\
% = E\left[ {\log p\left( {{x_T}} \right) + \sum\limits_{t = 1}^T {\log \frac{{{p_\theta }\left( {{x_{t - 1}}\left| {{x_t}} \right.} \right)}}{{q\left( {{x_t}\left| {{x_{t - 1}}} \right.} \right)}}} } \right] = {L_{\textrm{VLB}}}
% \end{array},
% \end{array}
% \end{split}
% \end{equation}
% \begin{equation}
% \begin{split}
% \begin{array}{l}
% \label{eqn5}
% % \begin{array}{c}
% - \log {p_\theta }\left( {{\bm{x}_0}} \right) \le  - \log {p_\theta }\left( {{\bm{x}_0}} \right) + {D_{KL}}\left( {q\left( {{\bm{x}_{1:T}}\left| {{\bm{x}_0}} \right.} \right)} \right)\\
% =  - \log {p_\theta }\left( {{\bm{x}_0}} \right) + {\mathbb{E}_{q\left( {{\bm{x}_{1:T}}\left| {{\bm{x}_0}} \right.} \right)}}\left[ {\log \frac{{q\left( {{\bm{x}_{1:T}}\left| {{\bm{x}_0}} \right.} \right){p_\theta }\left( {{\bm{x}_0}} \right)}}{{{p_\theta }\left( {{\bm{x}_{0:T}}} \right)}}} \right]\\
% = {\mathbb{E}_{q\left( {{\bm{x}_{1:T}}\left| {{\bm{x}_0}} \right.} \right)}}\left[ {\log \frac{{q\left( {{\bm{x}_{1:T}}\left| {{\bm{x}_0}} \right.} \right)}}{{{p_\theta }\left( {{\bm{x}_{0:T}}} \right)}}} \right] \buildrel \Delta \over = {L_{\textrm{VLB}}},
% % \end{array}
% \end{array}
% \end{split}
% \end{equation}
\begin{align}
- \log {p_\theta }\left( {{\bm{x}_0}} \right) &\le  - \log {p_\theta }\left( {{\bm{x}_0}} \right) + {D_{KL}}\left( {q\left( {{\bm{x}_{1:T}}| {{\bm{x}_0}}} \right)} \right) \nonumber\\
% =  - \log {p_\theta }\left( {{\bm{x}_0}} \right) + {\mathbb{E}_{q\left( {{\bm{x}_{1:T}}\left| {{\bm{x}_0}} \right.} \right)}}\left[ {\log \frac{{q\left( {{\bm{x}_{1:T}}\left| {{\bm{x}_0}} \right.} \right){p_\theta }\left( {{\bm{x}_0}} \right)}}{{{p_\theta }\left( {{\bm{x}_{0:T}}} \right)}}} \right] \nonumber\\
&= {\mathbb{E}_{q\left( {{\bm{x}_{1:T}}| {{\bm{x}_0}}} \right)}}\left[ {\log \frac{{q\left( {{\bm{x}_{1:T}}|{{\bm{x}_0}}} \right)}}{{{p_\theta }\left( {{\bm{x}_{0:T}}} \right)}}} \right] \nonumber\\
&\buildrel \Delta \over = {L_{\textrm{VLB}}},\label{eqn5}
\end{align}
where $\bm{x}_0$ denotes the original target samples, and $D_{KL}(\cdot)$ denotes the KL divergence.
% By the Markov property and Bayes' rule, we simplify the variational lower bound $L_{VLB}$:
Built upon the Markov property and the Bayes rule, $L_{\textrm{VLB}}$ can be simplified as
% \begin{equation}
% \begin{split}
% \begin{array}{l}
% \label{eqn6}
% {L_{\textrm{VLB}}} = {\mathbb{E}_{q\left( {{\bm{x}_{1:T}}\left| {{\bm{x}_0}} \right.} \right)}}\left[ {\log \frac{{\prod\limits_{t = 1}^T {q\left( {{\bm{x}_t}\left| {{\bm{x}_{t - 1}}} \right.} \right)} }}{{{p_\theta }\left( {{\bm{x}_T}} \right)\prod\limits_{t = 1}^T {{p_\theta }\left( {{\bm{x}_{t - 1}}\left| {{\bm{x}_t}} \right.} \right)} }}} \right]\\
% = {\mathbb{E}_{q\left( {{\bm{x}_{1:T}}\left| {{\bm{x}_0}} \right.} \right)}}\left[ { - \log {p_\theta }\left( {{\bm{x}_T}} \right) + \sum\limits_{t = 1}^T {\log \frac{{q\left( {{\bm{x}_t}\left| {{\bm{x}_{t - 1}}} \right.} \right)}}{{{p_\theta }\left( {{\bm{x}_{t - 1}}\left| {{\bm{x}_t}} \right.} \right)}}} } \right]\\
% = {\mathbb{E}_{q\left( {{\bm{x}_{1:T}}\left| {{\bm{x}_0}} \right.} \right)}} [ \log \frac{{q\left( {{\bm{x}_T}\left| {{\bm{x}_0}} \right.} \right)}}{{{p_\theta }\left( {{\bm{x}_T}} \right)}} + \sum\limits_{t = 2}^T {\log \frac{{q\left( {{\bm{x}_{t - 1}}\left| {{\bm{x}_t},{\bm{x}_0}} \right.} \right)}}{{{p_\theta }\left( {{\bm{x}_{t - 1}}\left| {{\bm{x}_t}} \right.} \right)}}}\\
% \ \ \ \  - \log {p_\theta }\left( {{\bm{x}_0}\left| {{\bm{x}_1}} \right.} \right) ]\\
% = {\mathbb{E}_{q\left( {{\bm{x}_{1:T}}\left| {{\bm{x}_0}} \right.} \right)}}[{\underbrace {{D_{KL}}\left( {q\left( {{\bm{x}_T}\left| {{\bm{x}_0}} \right.} \right)\left\| {{p_\theta }\left( {{\bm{x}_T}} \right)} \right.} \right)}_{{L_T}}}\\
% \ \ \ \ + {\sum\limits_{t = 2}^T {\underbrace {{D_{KL}}\left( {q\left( {{\bm{x}_{t - 1}}\left| {{\bm{x}_t},{\bm{x}_0}} \right.} \right)\left\| {{p_\theta }\left( {{\bm{x}_{t - 1}}\left| {{\bm{x}_t}} \right.} \right)} \right.} \right)}_{{L_{t - 1}}}} }\\
% \ \ \ \ {\underbrace { - \log {p_\theta }\left( {{\bm{x}_0}\left| {{\bm{x}_1}} \right.} \right)}_{{L_0}}}],
% \end{array}
% \end{split}
% \end{equation}
\begin{align}
{L_{\textrm{VLB}}}& = {\mathbb{E}_{q\left( {{\bm{x}_{1:T}}|{{\bm{x}_0}}} \right)}} {\log \frac{{\prod\limits_{t = 1}^T {q\left( {{\bm{x}_t}| {{\bm{x}_{t - 1}}}} \right)} }}{{{p_\theta }\left( {{\bm{x}_T}} \right)\prod\limits_{t = 1}^T {{p_\theta }\left( {{\bm{x}_{t - 1}}| {{\bm{x}_t}}} \right)} }}}  \nonumber\\
% &= {\mathbb{E}_{q\left( {{\bm{x}_{1:T}}\left| {{\bm{x}_0}} \right.} \right)}}\left[ { - \log {p_\theta }\left( {{\bm{x}_T}} \right) + \sum\limits_{t = 1}^T {\log \frac{{q\left( {{\bm{x}_t}\left| {{\bm{x}_{t - 1}}} \right.} \right)}}{{{p_\theta }\left( {{\bm{x}_{t - 1}}\left| {{\bm{x}_t}} \right.} \right)}}} } \right] \nonumber\\
% &= {\mathbb{E}_{q\left( {{\bm{x}_{1:T}}\left| {{\bm{x}_0}} \right.} \right)}} [ \log \frac{{q\left( {{\bm{x}_T}\left| {{\bm{x}_0}} \right.} \right)}}{{{p_\theta }\left( {{\bm{x}_T}} \right)}} + \sum\limits_{t = 2}^T {\log \frac{{q\left( {{\bm{x}_{t - 1}}\left| {{\bm{x}_t},{\bm{x}_0}} \right.} \right)}}{{{p_\theta }\left( {{\bm{x}_{t - 1}}\left| {{\bm{x}_t}} \right.} \right)}}} \nonumber\\
% &\ \ \ \  - \log {p_\theta }\left( {{\bm{x}_0}\left| {{\bm{x}_1}} \right.} \right) ] \nonumber\\
&= {\mathbb{E}_{q\left( {{\bm{x}_{1:T}}|{{\bm{x}_0}}} \right)}}[{\underbrace {{D_{KL}}\left( {q\left( {{\bm{x}_T}|{{\bm{x}_0}}} \right) \| \;{{p_\theta }\left( {{\bm{x}_T}} \right)}} \right)}_{{L_T}}} \nonumber\\
&\ \ \ \ + {\sum\limits_{t = 2}^T {\underbrace {{D_{KL}}\left( {q\left( {{\bm{x}_{t - 1}}| {{\bm{x}_t},{\bm{x}_0}}} \right)\|\;{{p_\theta }\left( {{\bm{x}_{t - 1}}|{{\bm{x}_t}}} \right)}} \right)}_{{L_{t - 1}}}} }\nonumber \\
&\ \ \ \ \ {\underbrace { - \log {p_\theta }\left( {{\bm{x}_0}|{{\bm{x}_1}}} \right)}_{{L_0}}}],\label{eqn6}
\end{align}
where $L_T$ and $L_0$ terms can be easily computed  \cite{ho2020denoising}. $L_{t-1}$ can be simplified as
% \begin{equation}
% \begin{split}
% \begin{array}{l}
% \label{eqn7}
% {L_{t - 1}} = {E_q}\left[ {\frac{1}{{2\sigma _t^2}}{{\left\| {{{\tilde \mu }_t}\left( {{\bm{x}_t},{\bm{x}_0}} \right) - {\mu _\theta }\left( {{\bm{x}_t},t} \right)} \right\|}^2}} \right] + C \\
% {L_{t - 1}} - C = {\mathbb{E}_{\epsilon \sim \mathcal{N}\left( {0,\bf{I}} \right)}}\left[ {{\lambda _t}{{\left\| {\epsilon  - {\epsilon _\theta }\left( {{\bm{x}_t},t} \right)} \right\|}^2}} \right],
% \end{array}
% \end{split}
% \end{equation}
\begin{align}
%{L_{t - 1}} &= {\mathbb{E}_q}\left[ {\frac{1}{{2\sigma _t^2}}{{\left\| {{{\tilde \mu }_t}\left( {{\bm{x}_t},{\bm{x}_0}} \right) - {\mu _\theta }\left( {{\bm{x}_t},t} \right)} \right\|}^2}} \right] \nonumber\\
{L_{t - 1}} &= {\mathbb{E}_{\epsilon \sim \mathcal{N}\left( {0,\bf{I}} \right)}}\left[ {{\lambda _t}{{\left\| {\epsilon  - {\epsilon _\theta }\left( {{\bm{x}_t},t} \right)} \right\|}^2}} \right]+C',\label{eqn7}
\end{align}
where $C'$ denotes a constant independent of $\theta$, ${\lambda _t} = \beta _t^2/2\sigma _t^2{\alpha _t}\left( {1 - {{\bar \alpha }_t}} \right)$, and ${\bm{x}_t} = \sqrt {{{\bar \alpha }_t}} \bm{x}_0 + \sqrt {1 - {{\bar \alpha }_t}} \epsilon$.
Based on that, we can train a noise prediction function ${\epsilon _\theta }\left( {{\bm{x}_t},t} \right)$ to simulate the target distribution.

% Considering the superior image generation capability of DPM and its easy implementation, we train DPM in the target domain to simulate target distribution in step 2.
% In order to control the categories of the generated images, the label information is injected into the DPM in the same way as the timestep $t$ through a label embedding. And we call this conditional DDPM (CDPM).
% In order to achieve class-conditional generation, we use the pretrained classifier to obtain the pseudo labels of target domain samples.
Though the target domain is unlabeled, we expect that labels of generated target samples can be controlled.
To make the DPM class-conditional, the pseudo label ${\bar y}^{td}$ assigned by the pretrained classifier in step 1 can be injected into the DPM in the same way as the time step $t$ through the label embedding $e_l=\mathrm{emb}({\bar y}^{td})$, where $\mathrm{emb}(\cdot)$ denotes the embedding function.
% The new embedding $emb$ can be obtained by the sum of the time embedding $e_i$ and label embedding $v_i$. \cite{nichol2021improved}
The entire embedding can be obtained by the sum of the time embedding %$e_t$
and label embedding.% $e_l$.
% Then the $emb$ will be fed into residual blocks of the model ${\epsilon _\theta }\left( {{x_t}, v_i, t} \right)$ to achieve class-conditional generation.
Then the embedding will be fed into residual blocks of the model ${{{\rm{\epsilon}}_\theta }\left( {\bm{x}_t^{td},{{\bar y}^{td}},t} \right)}$ to achieve class-conditional generation.



% However, the labels of the target domain are not available in UDA settings.
% In this case, we propose to adopt an existing UDA method to obtain the pseudo labels ${{\hat y}^t}$ in the target domain.
%Meanwhile, the pretrained target classifier in step 1 is used to obtain the pseudo labels ${{\bar y}^{td}}$ in the target domain.
% In the following experiment part, we will demonstrate that CDPM is robust to these pseudo labels containing noise.
With the pretrained DPM on ImageNet \cite{nichol2021improved}, the model can converge faster in the target domain while reducing the required target domain samples.
To train CDPM, we use the gradient descent operation to minimize the following loss as
\begin{equation}
\begin{split}
\begin{array}{l}
\label{eqn4}
\mathcal{L}_d={\mathbb{E}_{{\epsilon \sim \mathcal{N}\left( {0,\bf{I}} \right)}}} {\left\| {\epsilon  - {\epsilon _\theta }\left( {{\bm{x}_t^{td}}, {{\bar y}^{td}}, t} \right)} \right\|^2},
\end{array}
\end{split}
\end{equation}
where $\theta$ denotes the parameters of the model $\epsilon _\theta$, $t$ is sampled from a uniform distribution $\mathrm{Uniform}\left( {\left\{ {1,...,T} \right\}} \right)$,
% $\epsilon \sim \mathcal{N}\left( {0,\bf{I}} \right)$,
and ${\bm{x}_t^{td}} = \sqrt {{{\bar \alpha }_t}} \bm{x}_0^{td} + \sqrt {1 - {{\bar \alpha }_t}} \epsilon$.
When the loss $\mathcal{L}_d$ converges, we get an optimal CDPM $\epsilon _\theta^*$.


\subsection{Sampling and Training Final Classifier}
After the training of CDPM, we generate pseudo target samples with pseudo labels ${\mathcal{D}_g} = \{ ({\bm{x}}_k^g,\bar y_k^g)\} _{k = 1}^{N_g}$, where $N_g$ denotes the number of the generated target samples.
The distribution of $\mathcal{D}_g$ could be similar to that of the real target domain, thus alleviating the problem of scarce and unlabeled target samples.
%Moreover, the categories and the number of generated samples can be easily controlled, which is more advantageous than the UDA methods based on image-to-image translation.
% And it is more direct than the methods based on generating intermediate domains.
% which alleviates the problem of scarce and unlabeled samples in the target domain.
Due to the slow sampling speed of DPMs, we adopt the DPM-Solver++ \cite{lu2022dpm} to greatly speed up the guided sampling process of CDPM, and so the sampling process is very efficient.
Specifically, given the initial noise $\bm{x}_T$ and time steps $\left\{ {{t_i}} \right\}_{i = 0}^M$,
% The multistep second-order solver for diffusion ODEs can be obtained:
the multi-step second-order solver for CDPM can be obtained as
% \begin{equation}
% \begin{split}
% \begin{array}{l}
% \label{eqn2}
% {W_i} \displaystyle \leftarrow \left( {1 + \frac{1}{{2{r_i}}}} \right){\epsilon_\theta }\left( {{{\bm{x}}_{{t_{i - 1}}}}, l, {t_{i - 1}}} \right) \\
% \ \ \ \ \ \ \ \ \ \ \ \displaystyle - \frac{{{h_i}}}{{2{h_{i - 1}}}}{\epsilon_\theta }\left( {{{\bm{x}}_{{t_{i - 2}}}}, l, {t_{i - 2}}} \right)\\
% {{\bm{x}}_{{t_i}}} \displaystyle \leftarrow \frac{{{\sigma _{{t_i}}}}}{{{\sigma _{{t_{i - 1}}}}}}{{\bm{x}}_{{t_{i - 1}}}} - {\alpha _{{t_i}}}\left( {{e^{ - {h_i}}} - 1} \right){W_i},
% \end{array}
% \end{split}
% \end{equation}
\begin{align}
{W_i} \displaystyle &\!\leftarrow\!\! \left(\!{1\! +\!\frac{1}{{2{r_i}}}} \!\right)\!{\epsilon_\theta \!}\left( {{{\bm{x}}_{{t_{i - 1}}}}, l, {t_{i - 1}}} \right) \!-\! \frac{{{h_i}}}{{2{h_{i - 1}}}}{\epsilon_\theta }\!\left( {{{\bm{x}}_{{t_{i - 2}}}}, l, {t_{i - 2}}} \right)\nonumber\\
{{\bm{x}}_{{t_i}}} \displaystyle &\!\leftarrow\! \frac{{{\sigma _{{t_i}}}}}{{{\sigma _{{t_{i - 1}}}}}}{{\bm{x}}_{{t_{i - 1}}}} - {\alpha _{{t_i}}}\left( {{e^{ - {h_i}}} - 1} \right){W_i},\label{eqn2}
\end{align}
where ${h_i} = {\lambda _{{t_i}}} - {\lambda _{{t_{i - 1}}}}$ for $i=1,...,M$, ${\lambda _t} = \log \left( {\frac{{{\alpha _t}}}{{{\sigma _t}}}} \right)$, $M$ denotes the required time steps, and $l$ denotes the category we want to generate.

Considering that noisy pseudo labels may be contained in the generated target domain $\mathcal{D}_g$, we combine the original source domain $\mathcal{D}_s$ and $\mathcal{D}_g$ together as the augmented source domain ${\mathcal{D}_{\hat{s}}}$, and use some existing UDA method to transfer from the augmented source domain ${\mathcal{D}_{\hat{s}}}$ to the target domain $\mathcal{D}_t$ by solving the following objective function as
\begin{align}
\label{eqn3}
\!\!\!\! {g^*}\!\!\!=\! \displaystyle \mathop {\arg \min \! }\limits_{g} \frac{1}{{{N_{\hat{s}}}}}\!\!\sum\limits_{i = 1}^{{N_{\hat{s}}}} \!\mathcal{L}_c\! \left( {g\!\left( {{\bm{x}_i^{{\hat{s}d}}}} \right)\!,\!{y_i^{\hat{s}d}}} \right)
 \!\!+\!\!\lambda {{R(T(\mathcal{D}_{\hat{s}}),\!T(\mathcal{D}_t)\!)}},\!\!\!\!
\end{align}
% \begin{align}
% \label{eqn1}
% \!\!\!\!\!{f^*} \!\!=\!\!  \mathop {\arg \min }_{f}& \frac{1}{{{N_s}}}\!\!\sum\limits_{i = 1}^{{N_s}} \!\mathcal{L}_c (\! {f( {{\bm{x}_i^{sd}}} ),{y_i^{sd}}}\! )
% \!\!+\!\!\lambda {{R(T(\mathcal{D}_s),\!T(\mathcal{D}_t))}},\!\!\!
% \end{align}
where $N_{\hat{s}}$ denotes the number of samples in the augmented source domain.
After optimizing problem (\ref{eqn3}), the final target classifier $g^*$ can be obtained for the target domain.

\subsection{Discussions}
Traditional discrepancy-based and adversarial-based UDA methods focus on learning domain-invariant feature representations based on the source and target data.
% However, the target domain data is scarce, and the transfer performance upper bound of the DA model is limited by the existing data.
However, the target data is limited and cannot represent the real target distribution.
The transfer performance of UDA models is usually limited by target samples.
Most of the existing generation-based UDA methods are based on image-to-image translation across domains \cite{hoffman2018cycada, yang2020bi} or generating data in an intermediate domain  \cite{cui2020gradually, na2021fixbi, gao2022middlegan}.
Those methods need to design complex mechanisms to maintain the detailed and semantic information of generated images.
% and are prone to failure in complex scenarios.
% The image-to-image translation-based methods need to design a complex mechanism to maintain the details and semantic information of the generated images, which is tend to fail in complex scenarios.
% Furthermore, using image-to-image translation-based methods is limited to the scale of source and target datasets, and cannot generate an arbitrary number of samples.
% Furthermore, these methods are limited to the scale of source and target datasets, and cannot produce extra samples.
% The DA method based on generating intermediate domain is prone to problems when training image pairs are from different categories.
% On the other hand, to produce intermediate domain images of the same category, the generative model needs to be trained separately for each category \cite{gao2022middlegan}, which is time-consuming when there are a large number of categories.
Besides, the aforementioned GAN-based methods are unstable in adversarial training and require careful hyperparameter and regularizer selection to achieve the convergence \cite{dhariwal2021diffusion, murphy2023probabilistic}.


% 基于GAN的方法很难采用预训练模型,会造成训练的不稳定。就需要大量的训练样本,这在样本匮乏的目标域中是不现实的


Different from those GAN-based methods, the proposed diffusion-based DTS framework directly generates pseudo target samples that could obey the target distribution without adversarial training.
\cite{nichol2021improved} has shown that DPMs are better at covering the modes of a distribution than GANs, which well meets the needs of target data generation here.
And the category and the number of samples generated can be flexibly controlled.
Moreover, the easy-to-implement plug-and-play DTS framework can be embedded into any UDA method to improve the transfer performance as shown in our experiments.

% \subsection{Theoretical Analysis}
% In this section, we demonstrate the effectiveness of the proposed DTS framework with theory.
Next, we discuss the validity of the proposed DTS framework through the UDA theory \cite{ben2010theory}.
	
\emph{Proposition 1 \cite{ben2010theory}:} For a hypothesis space $\mathcal{H}$, we have
\begin{equation} \label{eqn-thrm}
r_t\left( h \right)\leq  r_s\left( h \right)+\frac{1}{2}{d_{\mathcal{H} \Delta\mathcal{H}} }\left( {\mathcal{S} ,\mathcal{T}} \right)+C,
\end{equation}
where $r_s\left( h \right)$ and $r_t\left( h \right)$ denote the expected risks in the source and target domain, respectively, ${d_{\mathcal{H} \Delta\mathcal{H}} }\left( {\mathcal{S},\mathcal{T}} \right)$ denotes the $\mathcal{H} \Delta \mathcal{H}$-distance between the source and target distributions $\mathcal{S}$ and $\mathcal{T}$, and
% w.r.t. a hypothesis set $\mathcal{H}$.
$C$ denotes a negligible term \cite{ganin2015unsupervised, long2015learning}.
In DTS, with generated target samples, we can have
% \label{eqn-thrm1}
% \begin{equation}
% \small{
\begin{small}
\begin{align}
\label{eqn-thrm1}
{r_t}(\hat{h}) \leq& {r_{\hat{s}}}( {\hat h} ) + \frac{1}{2}{d_{{\rm{{\cal H}}}\Delta {\rm{{\cal H}}}}}( {\hat {\rm{{\cal S}}},{\rm{{\cal T}}}} ) + C \nonumber \\
=& {r_{\hat s}}( {\hat h} ) + {r_{\hat s}}\left( h \right) - {r_{\hat s}}\left( h \right) + \frac{1}{2}{d_{{\rm{{\cal H}}}\Delta {\rm{{\cal H}}}}}( {\hat {\rm{{\cal S}}},{\rm{{\cal T}}}} ) + C \nonumber \\
\leq & {r_{\hat s}}\left( h \right) + \frac{1}{2}{d_{{\rm{{\cal H}}}\Delta {\rm{{\cal H}}}}}( {\hat {\rm{{\cal S}}},{\rm{{\cal T}}}} ) + C \nonumber \\
=& {r_s}\left( h \right) + {r_g}\left( h \right) + \frac{1}{2}{d_{{\rm{{\cal H}}}\Delta {\rm{{\cal H}}}}}( {\hat {\rm{{\cal S}}},{\rm{{\cal T}}}} ) + C \nonumber \\ \nonumber
\leq& {r_s}\left( h \right) + {r_g}\left( h \right) + \frac{1}{2}\big[ \alpha {d_{{\rm{{\cal H}}}\Delta {\rm{{\cal H}}}}}( {{\rm{{\cal S}}},{\rm{{\cal T}}}} ) \\ \nonumber
 & + \left( {1 - \alpha } \right){d_{{\rm{{\cal H}}}\Delta {\rm{{\cal H}}}}}( {{{\rm{{\cal S}}}_g},{\rm{{\cal T}}}} ) \big] + C\\ \nonumber
=& {r_s}\left( h \right) + \frac{1}{2}{d_{{\rm{{\cal H}}}\Delta {\rm{{\cal H}}}}}\left( {{\rm{{\cal S}}},{\rm{{\cal T}}}} \right)  + C +\! {r_g\!}\left( h \right)\\
& + \frac{1}{2}(1 - \alpha )\left[ {{d_{{\rm{{\cal H}}}\Delta {\rm{{\cal H}}}}}\left( {{{\rm{{\cal S}}}_g},{\rm{{\cal T}}}} \right) \!-\! {d_{{\rm{{\cal H}}}\Delta {\rm{{\cal H}}}}}\left( {{\rm{{\cal S}}},{\rm{{\cal T}}}} \right)} \right],
\end{align}
\vspace{-0.3cm}
\end{small}

\noindent
where $\hat {\rm{{\cal S}}}$ and ${{\rm{{\cal S}}}_g}$ denote the distribution of the augmented source domain ${\mathcal{D}_{\hat{s}}}$ and the generated domain $\mathcal{D}_g$, respectively,
${r_g}(h)$ denotes the expected risk in the generated domain $\mathcal{D}_g$,
$\alpha$ is a constant in $[0,1]$, and
$\hat{h} = \mathop{\arg\min}\limits_{h \in \mathcal{H}} r_{\hat{s}}(h)$ represents the classifier trained in the augmented source domain ${\mathcal{D}_{\hat{s}}}$.
Note that ${r_s}\left( h \right) + \frac{1}{2}{d_{{\rm{{\cal H}}}\Delta {\rm{{\cal H}}}}}\left( {{\rm{{\cal S}}},{\rm{{\cal T}}}} \right)+C$ is the same as the right-hand side of \emph{Proposition 1}.
As $\mathcal{S}_g$ is generated to approximate $\mathcal{T}$, it is expected that $\mathcal{S}_g$ has a lower discrepancy to $\mathcal{T}$ than $\mathcal{S}$,
i.e., $ {{d_{{\rm{{\cal H}}}\Delta {\rm{{\cal H}}}}}\left( {{{\rm{{\cal S}}}_g},{\rm{{\cal T}}}} \right) < {d_{{\rm{{\cal H}}}\Delta {\rm{{\cal H}}}}}\left( {{\rm{{\cal S}}},{\rm{{\cal T}}}} \right)}$.
When ${r_g}\left( h \right)$ is small, ${r_t}({\hat h})$ will have a lower upper-bound when compared with $r_t (h)$, which could explain the superior performance of the DTS method from the theoretical perspective.

\section{Experiments}

In this section, we empirically evaluate the proposed DTS framework.
%on various DA benchmarks, including Office-31 \cite{saenko2010adapting}, Office-Home \cite{venkateswara2017deep} and VisDA-2017 \cite{peng2017visda}.
% The produced pseudo target images are shown to demonstrate
% And the quantitative experimental results demonstrate that the proposed framework exceeds to the state-of-the-art UDA method over 5\%.
%The quantitative experimental results demonstrate that the proposed framework can bring significant performance improvement to the state-of-the-art UDA methods.
%And the qualitative experimental results show the generated high fidelity and diversity pseudo target samples.
\subsection{Experimental Settings}

\noindent\textbf{Datasets.} Experiments are conducted on three benchmark datasets, including Office-31 \cite{saenko2010adapting}, Office-Home \cite{venkateswara2017deep}, and VisDA-2017 \cite{peng2017visda}.
The \textbf{Office-31} dataset contains 4,110 images in 31 categories of three distinct domains: Amazon (A), DSLR (D), and Webcam (W).
%  from which six different transfer tasks can be built.
From these three domains, we build six different transfer tasks, i.e., A$\rightarrow$W, D$\rightarrow$W, W$\rightarrow$D,  A$\rightarrow$D, D$\rightarrow$A, W$\rightarrow$A.
The \textbf{Office-Home} dataset contains 15,500 images in total from 65 categories of four image domains: Art (Ar), Clipart (Cl), Product (Pr), and Real-World (Rw). On this dataset, we build 12 transfer tasks in the four domains.
%The large number of categories makes the transfer tasks on Office-Home more challenging than on Office-31.
% \textbf{VisDA-2017} is a simulation to real world benchmark dataset for DA with over 280K images across 12 categories.
The \textbf{VisDA-2017} dataset is a large-scale synthetic-to-real  benchmark dataset for UDA with 12 categories.
It contains more than 150k images in the source domain and 50k images in the target domain.





\noindent\textbf{Implementation details.}
% In the diffusion part, we follow the same hyperparameter settings and pretrained diffusion model provided by \cite{nichol2021improved}.
We use the pretrained CDPM on ImageNet provided by \cite{nichol2021improved}. $\epsilon _\theta$ adopts a U-Net model \cite{ronneberger2015u}.
The number of target samples generated for each category in DTS is set to 200, 200, and 2000 on the Office-31, Office-Home, and VisDA datasets, respectively.
The resolution of generated target samples is 256$\times$256.
The DPM-Solver++ is used to ensure stable class-conditional sampling.
% We use the pretrained DDPM model provided by \cite{nichol2021improved}.
For a fair comparison, the same backbone is used for all the methods on one dataset.
Specifically, ResNet-50 \cite{he2016deep} is used on the Office-31 and Office-Home datasets, and ResNet-101 is used on the VisDA-2017 dataset.
We use mini-batch stochastic gradient descent (SGD) optimizer with a momentum of 0.9 and the same learning rate annealing strategy as \cite{ganin2016domain}.
All experiments are implemented on a NVIDIA V100 GPU.
% For each UDA task, the number of samples produced is determined empirically, so that the final classifier can achieve the best performance in the target domain.
% 这里要加一个每个方法的简短介绍
% \begin{figure}[t]
% \begin{center}
% \fbox{\rule{0pt}{2in} \rule{0.9\linewidth}{0pt}}
%    %\includegraphics[width=0.8\linewidth]{egfigure.eps}
% \end{center}
%    \caption{Example of a caption.  It is set in Roman so mathematics
%    (always set in Roman: $B \sin A = A \sin B$) may be included without an
%    ugly clash.}
% \label{fig:long}
% \label{fig:onecol}
% \end{figure}



\noindent{\bf Baselines.} We compare with state-of-the-art UDA methods, including
DANN \cite{ganin2016domain}, AFN \cite{AFN}, CDAN \cite{long2018conditional}, MDD \cite{MDD}, SDAT \cite{rangwani2022closer}, MCC \cite{jin2020minimum}, and ELS \cite{zhang2023free}.
In particular, we combine the proposed DTS with MCC and ELS to evaluate its effectiveness.
%All of the above methods have been introduced in Section~\ref{section:RW}.
We also compare with ERM \cite{vapnik1999nature}, which trains a classifier on the source domain and applies it directly to the target domain without domain alignment.



\subsection{Experimental Results}
In this section, we analyze results on the three benchmark datasets.
The DTS framework is easily embedded into state-of-the-art UDA models such as MCC \cite{jin2020minimum} and ELS \cite{zhang2023free}, i.e., MCC+DTS and ELS+DTS.

\noindent\textbf{Office-31.}
According to results shown in Table \ref{office31} for the Office-31 dataset, we can see that the proposed DTS method outperforms baseline methods by a large margin.
Specifically, MCC+DTS achieves an average accuracy of 90.6\%, and the proposed DTS framework brings a  performance improvement of 0.8\% compared with the original MCC method.
ELS+DTS achieves an average accuracy of 91.0\%, which outperforms the original ELS method by 0.6\%.
% Besides, it can be observed that the proposed framework obtains 2.5\% and 2.1\% performance improvement in the most difficult transfer tasks, i.e., W$\rightarrow$A.
Moreover, for the most difficult transfer task W$\rightarrow$A, where the difficulty of a transfer task is measured via the average classification accuracy of all the models in comparison, the proposed DTS framework achieves performance improvements of 2.5\% and 2.1\% over MCC and ELS, respectively.
Therefore, DTS can help improve the performance of UDA methods.
%Therefore, DTS can reduce the difficulty of transfer learning and greatly improve the performance of DA algorithms by producing class-conditional images that obey the target distribution.

\begin{table}[!t]\small
\centering
% \renewcommand{\arraystretch}{1.1}
\caption{Accuracy comparison (\%) on the Office-31 dataset with ResNet-50 as the backbone. $\uparrow$ denotes the accuracy improvement brought by the DTS framework over the corresponding baseline (i.e., MCC or ELS). The best performance of each task is marked in bold.}
%\vskip .5in
\label{office31}
\setlength{\tabcolsep}{1.5mm}{
\resizebox{\columnwidth}{!}{
\begin{tabular}{ccccccc @{\hskip 0.2in} c}
\toprule Method & A$\rightarrow$W        & D$\rightarrow$W        & W$\rightarrow$D        & A$\rightarrow$D        & D$\rightarrow$A        & W$\rightarrow$A        & Average \\
\midrule
ERM \cite{vapnik1999nature}  & 77.0 & 96.6 & 99.2 & 82.8 & 64.1 & 64.1 & 80.3 \\
DANN \cite{ganin2016domain} & 89.3 & 98.0 & \textbf{100.0} & 83.5 & 74.0 & 74.3 & 86.5 \\
AFN \cite{AFN}  & 91.3 & 98.7 & \textbf{100.0} & 95.6 & 72.1 & 70.7 & 88.1 \\
CDAN \cite{long2018conditional} & 92.6 & 98.5 & \textbf{100.0}  & 92.2 & 75.7 & 73.1 & 88.7 \\
% MCC \cite{jin2020minimum}  & 94.1 & 98.4 & 99.8  & 95.6 & 75.5 & 74.2 & 89.6 \\
MDD \cite{MDD}  & 94.6 & 98.5 & \textbf{100.0}  & 94.0 & 75.7 & 73.9 & 89.4 \\
SDAT \cite{rangwani2022closer} & 88.8 & \textbf{99.0} & \textbf{100.0} & 93.8 & 76.3 & 72.8 & 88.4 \\
\midrule
MCC \cite{jin2020minimum}  & 93.7 & 98.4 & 99.6  & 95.2 & 76.3 & 75.5 & 89.8 \\
MCC+DTS  & \textbf{94.7} & 98.4 & 99.6  & 95.4 & 77.8 & \textbf{78.0} & 90.6 \\
$\uparrow $  & 1.0 & 0.0 & 0.0 & 0.2 & 1.5 & 2.5 & 0.8 \\
\midrule
ELS \cite{zhang2023free}  & 94.3 & 98.9 & \textbf{100.0} & 95.6 & 78.5 & 75.0 & 90.4 \\
% DSAN      & 92.54$\pm$0.32 & 98.70$\pm$0.06 & 100.00$\pm$0   & 90.16$\pm$0.85 & 74.00$\pm$0.32 & 70.60$\pm$1.10 & 87.67   \\
ELS+DTS      & 94.5 & \textbf{99.0} & \textbf{100.0} & \textbf{96.0} & \textbf{79.3} & 77.1 & \textbf{91.0} \\
$\uparrow $  & 0.2 & 0.1 & 0.0 & 0.4 & 0.8 & 2.1 & 0.6 \\
\bottomrule
\end{tabular}}}
\end{table}


\begin{table*}[!tbph]\small
\centering
% \renewcommand{\arraystretch}{1.1}
\caption{Accuracy comparison (\%) on the Office-Home dataset with ResNet-50 as the backbone. $\uparrow$ denotes the accuracy improvement brought by the DTS framework  over the corresponding baseline (i.e., ELS). The best performance of each task is marked in bold.}
%\vskip .5in
\label{officehome}
\setlength{\tabcolsep}{1mm}{
\begin{tabular}{ccccccccccccc @{\hskip 0.05in} c}
\toprule Method & Ar$\rightarrow$Cl   & Ar$\rightarrow$Pr  & Ar$\rightarrow$Rw & Cl$\rightarrow$Ar & Cl$\rightarrow$Pr & Cl$\rightarrow$Rw & Pr$\rightarrow$Ar & Pr$\rightarrow$Cl & Pr$\rightarrow$Rw & Rw$\rightarrow$Ar & Rw$\rightarrow$Cl & Rw$\rightarrow$Pr & Average \\
\midrule
ERM \cite{vapnik1999nature}  & 44.2 & 67.2 & 74.6 & 53.3 & 61.8 & 64.4 & 51.9 & 38.8 & 73.0 & 64.3 & 43.6 & 75.4 & 59.4  \\
DANN \cite{ganin2016domain} & 52.1 & 63.3 & 73.5 & 56.9 & 67.4 & 67.8 & 57.8 & 54.8 & 78.7 & 71.4 & 60.6 & 80.0 & 65.4   \\
% MCC \cite{jin2020minimum}  & 58.4 & 79.6 & 83.0  & 67.5 & 77.0 & 78.5 & 66.6 & 54.8 & 81.8 & 74.4 & 61.4 & 85.6 & 72.4 \\
AFN \cite{AFN} & 52.6 & 72.6 & 76.9 & 64.9 & 71.5 & 73.0 & 63.7 & 51.5 & 77.8 & 72.1 & 57.7 & 82.3 & 68.0 \\
CDAN \cite{long2018conditional} & 53.9 & 72.4 & 78.6 & 61.9 & 70.8 & 72.3 & 63.5 & 55.6 & 80.9 & 75.0 & 61.5 & 83.2 & 69.1 \\
MDD \cite{MDD} & 55.7 & 75.9 & 79.1 & 63.0 & 73.7 & 74.1 & 63.0 & 55.6 & 79.8 & 73.8 & 61.4 & 84.0 & 69.9 \\
SDAT \cite{rangwani2022closer} & 58.4 & 77.9 & 81.6 & 66.2 & 76.8 & 76.4 & 62.2 & 56.2 & 82.6 & 75.7 & 62.0 & 85.3 & 71.8 \\
\midrule
MCC  & 56.2 & 79.5 & 82.5 & 68.1 & 76.5 & 78.0 & 66.9 & 54.9 & 82.1 & 73.6 & 61.9 & 85.5 & 72.1   \\
MCC+DTS      & 56.7 & \textbf{80.8} & \textbf{83.8} & \textbf{69.6} & 79.5 & \textbf{81.4} & \textbf{67.7} & 57.1 & \textbf{83.9} & 73.6 & 62.1 & \textbf{86.4} & 73.5\\
$\uparrow $      & 0.4 & 1.3 & 1.3 & 1.5 & 3.0 & 3.4 & 0.8 & 2.2 & 1.8 & 0.0  & 0.3 & 0.9 & 1.4   \\
\midrule
ELS \cite{zhang2023free}  & 57.2 & 77.2 & 82.0 & 66.5 & 77.2 & 76.7 & 62.3 & 56.3& 82.2 &  75.6 & 63.9 & 85.4 & 71.9  \\
% DSAN      & 92.54$\pm$0.32 & 98.70$\pm$0.06 & 100.00$\pm$0   & 90.16$\pm$0.85 & 74.00$\pm$0.32 & 70.60$\pm$1.10 & 87.67   \\
ELS+DTS      & \textbf{59.9} & 78.8 & 83.2 & 67.8 & \textbf{81.1} & 80.6 & 63.9 & \textbf{57.9} & 83.2 & \textbf{75.9} & \textbf{64.2} & 86.1 & \textbf{73.6}\\
% $\uparrow $      &4.72 & 2.07 & 1.46 & 1.95 & 5.05 & 5.08 & 2.57 & 2.84 & 1.22 & 0.40 & 0.47 & 0.82 & 2.36 \\
$\uparrow $      & 2.7 & 1.6 & 1.2 & 1.3 & 3.9 & 3.9 & 1.6 & 1.6 & 1.0 & 0.3 & 0.3 & 0.7 & 1.7 \\
\bottomrule
\end{tabular}}
\end{table*}



\begin{table*}[!tbph]\small
\centering
% \renewcommand{\arraystretch}{1.1}
\caption{Accuracy comparison (\%) on the VisDA-2017 dataset with ResNet-101 as the backbone. $\uparrow$ denotes the accuracy improvement brought by the DTS framework  over the corresponding baseline (i.e., MCC or ELS). The best performance of each task is marked in bold.}
%\vskip .5in
\label{visda}
\setlength{\tabcolsep}{1.5mm}{
\begin{tabular}{ccccccccccccc @{\hskip 0.2in} c}
\toprule Method & aero & bicycle & bus & car & horse & knife & motor & person & plant & skate & train & truck & mean \\
\midrule
ERM \cite{vapnik1999nature}  & 86.9 & 20.8 & 58.4 & 74.3 & 76.3 & 15.4 & 85.8 & 17.5 & 82.5 & 29.2 & 80.2 & 5.5	& 52.7  \\
DANN \cite{ganin2016domain} & 94.3 & 76.1 & 83.6 & 44.4 & 86.3 & 92.8 & 87.4 & 78.4 & 89.0 & 88.3 & 87.5 & 46.1 & 79.5   \\
AFN \cite{AFN} & 92.2 & 54.8 & 82.8 & 72.2 & 90.8 & 74.4 & 92.3 & 70.9 & \textbf{94.7} & 56.4 & \textbf{89.1} & 25.3	& 74.6 \\
CDAN \cite{long2018conditional} & 95.0 & 75.9 & \textbf{85.7} & 57.0 & 90.9 & 95.9 & 89.9 & 75.6 & 85.8 & 89.5 & 88.5 & 41.1	&80.9 \\
MDD \cite{MDD} & 89.3 & 70.2 & 82.3 & 66.7 & 92.6 & 94.4 & \textbf{93.3} & 78.9 & 92.6 & 88.0 & 82.3 & 49.0	& 81.6 \\
SDAT \cite{rangwani2022closer} & 94.7 & 85.1 & 73.2 & 67.0 & 93.0 & 94.7 & 89.2 & 80.7 & 90.8 & 93.0 &  84.1 & 56.0 & 83.5  \\
\midrule
MCC \cite{jin2020minimum}  & 95.1 & 84.9 & 73.0  & 69.9 & 93.2 & 95.6 & 87.6 & 79.0 & 90.4 & 89.9 & 84.9 & 55.7 & 83.3 \\
MCC+DTS   & 96.4  & \textbf{87.2} & 82.2 & \textbf{78.2} & 94.6 & \textbf{96.8} & 88.7 & \textbf{82.6} & 92.4 & \textbf{93.8} & 86.6 & 57.8 & \textbf{86.4} \\
$\uparrow $  & 1.3 & 2.3 & 9.2 & 8.3 & 1.4 & 1.2 & 1.1 & 3.6 & 2.0 & 3.9 & 1.7 & 2.1 & 3.1 \\
\midrule
ELS \cite{zhang2023free} & 95.3 & 84.9 & 75.2 & 66.3 & 93.0 & 93.8 & 88.4 & 79.4 & 90.4 & 92.5 & 83.8 & 57.7 & 83.4 \\
% DSAN      & 92.54$\pm$0.32 & 98.70$\pm$0.06 & 100.00$\pm$0   & 90.16$\pm$0.85 & 74.00$\pm$0.32 & 70.60$\pm$1.10 & 87.67   \\
ELS+DTS   & \textbf{96.6}  & 86.1 & 82.2 & 68.8 & \textbf{95.4} & 96.3 & 90.8 & 82.5 & 92.7 & 93.3 &87.9 & \textbf{60.5} & 86.1 \\
$\uparrow $  & 1.3 & 1.2 & 7.0 & 2.5 & 2.4 & 2.5 & 2.4 & 3.1 & 2.3 & 0.8 & 4.1 & 2.8 & 2.7 \\
\bottomrule
\end{tabular}}
% \vspace{-0.3cm}
\end{table*}




\noindent\textbf{Office-Home.} According to results shown in Table \ref{officehome} for the Office-Home dataset,
% In this case, 12 transfer tasks are built from 4 distinct domains.
% Our DTS framework brings great performance improvements in all 12 transfer tasks.
we can see that the proposed DTS framework brings performance improvement for most transfer tasks, which demonstrates its effectiveness.
Specifically, ELS+DTS achieves an average accuracy of 73.6\%, which improves the average accuracy of ELS by 1.7\%.
Especially, in transfer tasks Cl$\rightarrow$Pr and Cl$\rightarrow$Rw, DTS brings significant performance improvements (i.e., 3.9\%) over ELS.
% With the proposed DTS framework,
Moreover, MCC achieves an average accuracy improvement of 1.4\% over the original MCC method, which further demonstrates the effectiveness of the DTS method.
Overall, the proposed DTS framework has shown its effectiveness on this dataset.
% Due to page limit, experiments of MCC+DTS on the Office-Home dataset are put in the appendix in the supplementary material.


\noindent\textbf{VisDA-2017.} The results on the VisDA-2017 dataset are shown in Table \ref{visda}.
% It can be seen that MCC+DTS achieves an average classification accuracy of 86.4\%, which outperforms baseline UDA methods with a large margin of over 3.1\%.
It can be seen that MCC+DTS achieves an average classification accuracy of 86.4\%, which outperforms the original MCC by a large margin of 3.1\%.
ELS+DTS achieves an average classification accuracy of 86.1\%, which outperforms the origin ELS by 2.7\%. Furthermore, the performance of all categories has been improved.
In particular, DTS obtains a significant improvement of 9.2\% on the `bus' category when compared with MCC.
Moreover, even the smallest accuracy improvement of MCC+DTS over MCC for each class is 1.1\%, which further demonstrates the superiority of the proposed DTS framework.
% With our DTS framework, the average performance of ELS can be improved by 3.2\%.









% \begin{table*}[!t]
% \small
% \centering
% % \color{black}
% % \renewcommand{\arraystretch}{1.1}
% \caption{Ablation studies (\%) on Office-31 for UDA (with ResNet-50 backbone). ELS is our baseline model. No generation means that the target domain (the pseudo labels are given by the pretrained classifier) and the original source domain are combined and transfered to the target domain without using the generative model. No original source domain means that the produced pseudo target samples are transfer to the target domain directly without combining with the original source domain. Unlike the proposed DTS framework of training from scratch from the augmented source domain to the target domain, finetune the pretrained UDA model means that the pretrained classifier is used as the parameter initialization.}
% \vskip .5in
% \label{ablation}
% \setlength{\tabcolsep}{1.5mm}{
% \begin{tabular}{ccccccc @{\hskip 0.2in} c}
% \toprule
% Ablation of ELS+DTS & A$\rightarrow$W  & D$\rightarrow$W   & W$\rightarrow$D  & A$\rightarrow$D   & D$\rightarrow$A   & W$\rightarrow$A        & Average \\
% % \hline
% \midrule
% ELS \cite{zhang2023free}  & 94.3 & 98.9 & 100.0 & 95.6 & 78.5 & 75.0 & 90.4 \\
% ELS+DTS (w/o generation)  & 94.5 & 98.9 & 100.0 & 95.6 & 78.6 & 75.0 & 90.4 \\
% ELS+DTS (w/o original source domain) & 94.3 & 98.9 & 100.0 & 95.8 & 79.3 & 76.7 & 90.8 \\
% ELS+DTS (Finetune the pretrained classifier) & 94.6 & 98.9 & 100.0  & 95.6 & 79.5 & 76.8 & 90.9 \\
% % MCC \cite{jin2020minimum}  & 94.1 & 98.4 & 99.8  & 95.6 & 75.5 & 74.2 & 89.6 \\
% ELS+DTS (Train from scratch) & 94.5 & 99.0 & 100.0 & 96.0 & 79.3 & 77.1 & 91.0 \\
% % \hline
% \bottomrule
% \end{tabular}}
% \end{table*}


\begin{figure*}[t]
\centering
\includegraphics[width=\textwidth]{generate2.eps}
% \caption{An overview of the proposed DTS framework. With the pretrained target classifier, we train the CDPM in the target domain. Then the pseudo target domain $D_g$ can be obtained through sampling. The final classifier can be trained by transferring from the augmented source domain ${D_{s \cup g}}$ to the target domain$D_t$.}
\caption{Real and generated images for the transfer task Ar$\rightarrow$Cl on the Office-Home dataset. (a) and (b) show the real source and target images, respectively, and (c) shows the generated target samples.}
\label{generate1}
\end{figure*}

\begin{figure*}[t]
\centering
\includegraphics[width=\textwidth]{generate.eps}%[width=6.5in]
\caption{Real and generated images for the transfer task A$\rightarrow$W on the Office-31 dataset. The first row shows the source samples. The second row shows the target samples.  The third row shows the generated target samples.}
\label{generate}
\end{figure*}

\begin{figure*}[t]
\centering
\includegraphics[width=\textwidth]{generate3.eps}
\caption{Real and generated images on the VisDA dataset. (a) and (b) show the real source and target images, respectively, and (c) shows the generated target samples.}
\label{visda_generate}
\end{figure*}

\begin{table*}[!t]\small
\centering
% \renewcommand{\arraystretch}{1.1}
\caption{Ablation studies (in terms of the classification accuracy (\%)) on the Office-Home dataset for UDA with ResNet-50 as the backbone. ELS is the baseline method. The last line is our method. The best performance of each task is marked in bold.}
%\vskip .1in
\label{ablation}
\setlength{\tabcolsep}{1mm}{
\resizebox{\textwidth}{!}{
\begin{tabular}{ccccccccccccc @{\hskip 0.05in} c}
\toprule Ablation of ELS+DTS & Ar$\rightarrow$Cl   & Ar$\rightarrow$Pr  & Ar$\rightarrow$Rw & Cl$\rightarrow$Ar & Cl$\rightarrow$Pr & Cl$\rightarrow$Rw & Pr$\rightarrow$Ar & Pr$\rightarrow$Cl & Pr$\rightarrow$Rw & Rw$\rightarrow$Ar & Rw$\rightarrow$Cl & Rw$\rightarrow$Pr & Average \\
\midrule
ELS \cite{zhang2023free}  & 57.2 & 77.2 & 82.0 & 66.5 & 77.2 & 76.7 & 62.3 & 56.3& 82.2 &  75.6 & 63.9 & 85.4 & 71.9  \\
ELS+DTS (w/o generation) & 58.6 & 77.6 & 81.2  & 66.2 & 76.4 & 76.0 & 61.9 & 56.2 & 82.6 & 75.2 & 62.7 & 84.9 & 71.6 \\
ELS+DTS (w/o original source domain) &57.5 &77.5	&82.4 &66.8&80.1	&79.6	&62.4	&54.2	&81.6	& 72.9 & 60.8	&84.8	&71.7 \\									
ELS+DTS (Finetune the pretrained classifier) & 59.1 & 78.0 & 83.0  &\textbf{68.1} &79.8 & 79.3 & 62.7 & 57.5 & 82.8 & 75.8 & \textbf{64.2} & \textbf{86.4} & 73.1 \\
ELS+DTS (Train from scratch)    & \textbf{59.9} & \textbf{78.8} & \textbf{83.2} & 67.8 & \textbf{81.1} & \textbf{80.6} & \textbf{63.9} & \textbf{57.9} & \textbf{83.2} & \textbf{75.9} & \textbf{64.2} & 86.1 & \textbf{73.6}\\
% $\uparrow $      &4.72 & 2.07 & 1.46 & 1.95 & 5.05 & 5.08 & 2.57 & 2.84 & 1.22 & 0.40 & 0.47 & 0.82 & 2.36 \\
% $\uparrow $      & 4.7 & 2.1 & 1.5 & 2.0 & 5.1 & 5.1 & 2.6 & 2.8 & 1.2 & 0.4 & 0.5 & 0.8 & 2.4 \\
\bottomrule
\end{tabular}}}
% \vspace{-0.2cm}
\end{table*}


\subsection{Ablation Studies}
To analyze the effectiveness of the proposed DTS method in detail, ablation experiments are conducted on the Office-Home dataset.
`ELS+DTS (w/o generation)' means that the target samples with pseudo labels given by the classifier trained by ELS and the original source domain are combined to train the final classifier without using the generative model.
`ELS+DTS (w/o original source domain)' means that  generated target samples with pseudo labels are treated as a new source domain and then transferred  to the target domain via the ELS directly without the original source domain.
Different from the proposed DTS framework that trains a UDA model from scratch to transfer the augmented source domain to the target domain, `ELS+DTS (Finetune the pretrained classifier)' means that the pretrained classifier in step 1 is used for parameter initialization.

% As shown in Table \ref{ablation}, to demonstrate our



% As shown in Table \ref{ablation}, we can see that
%if we directly apply the pretrained target classifier to obtain pseudo labels of target samples, and combine these samples with source samples to train  to transfer to the target domain without the generation process (ELS+DTS (w/o generation)), the performance of ELS will decline.
% ELS+DTS (w/o generation) performs worse on average than ELS.
% One reason is that wrong pseudo labels will bring noise to the classifier, which leads to the negative transfer.

% Therefore, we use the CDPM to produce class-conditional target samples and try to mitigate the impact of the wrong pseudo labels
% Therefore, it is reasonable to introduce CDPM trained with these pseudo labels to generate class-conditional pseudo target samples and try to mitigate the impact of noise labels.
% Therefore, we introduce CDPM trained with these pseudo labels to generate class-conditional pseudo target samples and try to mitigate the impact of noise labels.
% And the experiments show that most of the categories of the generated target samples are correct.
% which brings significant performance improvement to UDA model.

As shown in Table \ref{ablation}, we can see that if we directly transfer from generated target samples to the target domain as ELS+DTS (w/o original source domain) did, a small degree of negative transfer (i.e., 0.2\% in terms of average classification accuracy) still occurs. This happens due to noisy pseudo labels of generated target samples.
Hence, the proposed DTS framework combines the original source domain and the generated pseudo target domain as the augmented source domain and transfers it to the target domain.
On one hand, correct source information could help alleviate label noises of generated samples.
On the other hand, generated target samples can inject more useful information about the target domain into the source domain to reduce the domain discrepancy and improve the transfer effects.
ELS+DTS (w/o generation) performs worse on average than ELS+DTS (Train from scratch), which further demonstrates that generated target samples contain useful information of the target domain.

In the proposed DTS framework, we train a UDA model from scratch in step 3 to transfer from the augmented source domain to the target domain without using the pretrained classifier in step 1.
According to Table \ref{ablation}, we can see that training from scratch performs slightly better with an average improvement of 0.5\% than finetuning the pretrained classifier.
%However, in some transfer tasks, the performance of finetuning the pretrained classifier is better, e.g., Cl$\rightarrow$Ar and Rw$\rightarrow$Pr.
% The reason may be that the classifier trained on the existing data limits the extended data
As a result, finetuning the pretrained classifier can converge faster and reduce the training time but with slightly lower transfer performance on average.
% Users can choose between precision and time complexity according to the actual situation.
Hence, to trade off between the accuracy and time complexity, we can choose to finetune the pretrained classifier or train the UDA model from scratch according to the actual situation.



\subsection{Analyses on Generated Results}
To demonstrate the fidelity and diversity of generated samples, we show in Figure~\ref{generate1} generative results of the transfer task Ar$\rightarrow$Cl on the Office-Home dataset.
% It can be seen that the generated samples have high fidelity and diversity, so the distribution of the target domain can be well simulated.
It can be seen that the style of generated samples is similar to real target samples.
Moreover, the generated samples have high fidelity and diversity, so they can well simulate the distribution of the target domain.
Figure~\ref{generate} shows the generated target samples in the transfer task A$\rightarrow$W on the Office-31 dataset.
%The images in the first and second rows are the original images of the source domain and the target domain, respectively.
%The generated specified category of images is shown in the third row.
We can see that generated samples are of high quality and similar to real target samples, and that the categories of generated images can be well controlled. % to obtain the label information of them naturally.
As the size of the Office-31 dataset is the smallest among three datasets, high fidelity generated target samples demonstrate the effectiveness of the proposed DTS method.
% To demonstrate the fidelity and diversity of generated samples, we show in
According to generative target samples on the VisDA dataset as shown in Figure~\ref{visda_generate},
% It can be seen in these results that the proposed DTS can well simulate the distribution of the target domain.
we can see that the proposed DTS method can well simulate the distribution of the target domain while maintaining the diversity of generated target samples.

Figure~\ref{tSNE} visualizes feature representations in the fully-connected layer of the final classifier on two transfer tasks of the Office-31 dataset via the t-SNE method \cite{van2008visualizing}.
It can be observed that generated target samples are very close to real target samples and hence they can help model the data distribution of the target domain as real target samples are limited.
On the other hand, due to the incorporation of generated target samples which can well approximate the data distribution of the target domain, the augmented source domain could have a smaller domain discrepancy to the target domain than that between the original source and target domains, and this makes the transfer tasks easier.
Hence, the performance of existing UDA methods can be improved.
%In contrast, the  can represent the target distribution better.
% With these produced samples, the transfer tasks become easier.
%On the other hand, because CDPM is trained based on the pseudo labels in the target domain, some generated class-conditional samples are misclassified.
%To reduce the impact of these misclassified samples, we combine the original source domain and the generated pseudo target domain as the augmented source domain, and then transfer it to the target domain.
%In the following ablation experiment, we will further explain the reason for this operation.

\begin{figure}[t]
\centering
\subfigure[A$\rightarrow$W]{
    \includegraphics[width=3.7cm]{A2W.eps}
    \label{x1}
}
\quad
\subfigure[D$\rightarrow$W]{
    \includegraphics[width=3.7cm]{D2W.eps}
    \label{x2}
}
    \caption{Feature visualization via t-SNE on the Office-31 dataset. Red and Blue points denote the generated target samples and original target samples, respectively. Best viewed in color.}
\label{tSNE}
% \vspace{-0.45cm}
\end{figure}

\begin{table}[!t]\small
\centering
% \color{black}
% \renewcommand{\arraystretch}{1.1}
\caption{$\mathcal{A}$-distance across domains over transfer tasks on the Office-31 dataset.}
%\vskip .5in
\label{Adistance}
\setlength{\tabcolsep}{1.5mm}{
\begin{tabular}{c @{\hskip 0.3in} cccc}
\toprule $\mathcal{A}$-distance & A$\rightarrow$W  & A$\rightarrow$D   & D$\rightarrow$A   & W$\rightarrow$A  \\
\midrule
$\mathcal{D}_s$ and $\mathcal{D}_t$ & 1.92 & 1.87 & 1.84 & 1.89 \\
$\mathcal{D}_g$ and $\mathcal{D}_t$ & 1.61 & 1.64 & 1.57 & 1.55 \\
$\mathcal{D}_{\hat{s}}$ and $\mathcal{D}_t$ & 1.71 & 1.78 & 1.59 & 1.64 \\
\bottomrule
\end{tabular}}
% \vspace{-0.35cm}
\end{table}




To further demonstrate the effectiveness of the proposed DTS framework, we introduce the $\mathcal{A}$-distance to measure the distribution discrepancy on the Office-31 dataset.
The $\mathcal{A}$-distance is defined as ${d_{\rm{\mathcal A}}}(p,q) = 2\left( {1 - 2\nu} \right)$, where $\nu$ denotes the error of a linear domain discriminator to distinguish source and target samples.
According to the results shown in Table \ref{Adistance}, we can see that the distribution of generated target samples is closer to that of the target domain than the original source domain.
Similar observations hold between the augmented source domain and the target domain.
Therefore, it is easier for UDA methods to transfer from the augmented source domain to the target domain, which further verifies the effectiveness of the DTS framework.
% \vspace{0.4cm}

\subsection{Sensitivity Analysis}
\label{sec:sen_analysis}

In this section, we conduct sensitivity analysis to analyze the effect of the number of generated target samples for each category to the transfer performance on three datasets.
The average results of ELS+DTS are shown in Figure~\ref{Sensitivity}.
% and the detailed results for each transfer task are shown in the appendix of the supplementary material.
Note that when the number of generated target samples is 0, ELS+DTS degenerates to the original ELS.
According to results, we can see that increasing the number of generated samples improves the transfer performance, which proves the effectiveness of the proposed DTS method.
On the three datasets, we find that when the number of generated samples increases to a certain extent, %there will be an inflection point.
% After this point, there will be no significant change in the performance of additional samples.
%After this point,
the transfer performance tends to stabilize.
Therefore, we need to select an appropriate number of generated target samples to balance the transfer performance and the time consumption of generating target samples.
In this paper, the number of generated target samples for each category is set to 200, 200, and 2000 by default on the Office-31, Office-Home, and VisDA datasets, respectively, to achieve a good balance.


Detailed results for each transfer task in the sensitivity analysis are shown in Table \ref{office31_number}, \ref{officehome_number}, and \ref{visda_number}.
% According to results, we can see that increasing the number of generated samples improves the transfer performance, which proves the effectiveness of the proposed DTS method.
According to results, we can see that increasing the number of generated target samples improves the transfer performance, which demonstrates the effectiveness of the proposed DTS method.
For small-scale datasets such as Office-31 and Office-Home, only a few samples (e.g., 50-400 per category) need to be generated to improve the transfer performance, while for large-scale datasets such as VisDA, more samples (e.g., 2,000-4,000 per category) need to be generated to achieve stable performance improvement.


\begin{figure}[t]
\centering
\subfigure{
    \includegraphics[width=3.95cm]{Office_number.eps}
    \label{x3}
}
% \quad
\subfigure{
    \includegraphics[width=3.95cm]{VisDA_number.eps}
    \label{x4}
}
    \caption{Sensitivity analysis on the number of generated target samples for each category on Office-31, Office-Home, and VisDA.}
\label{Sensitivity}
% \vspace{-0.45cm}
\end{figure}


\begin{table}[!tbph]\small
\centering
% \renewcommand{\arraystretch}{1.1}
\caption{Accuracy (\%) when varying the number of generated samples per category on the Office-31 dataset with ResNet-50 as the backbone.}
%\vskip .5in
\label{office31_number}
\setlength{\tabcolsep}{1.5mm}{
\resizebox{\columnwidth}{!}{
\begin{tabular}{ccccccc @{\hskip 0.2in} c}
\toprule Method & A$\rightarrow$W        & D$\rightarrow$W        & W$\rightarrow$D        & A$\rightarrow$D        & D$\rightarrow$A        & W$\rightarrow$A        & Average \\
\midrule
0  & 94.3 & 98.9 & 100.0 & 95.6 & 78.5 & 75.0 & 90.4 \\
50  & 94.2 & 99.0 & 100.0 & 96.2 & 78.8 & 76.6 & 90.8  \\
100  & 94.3 & 99.0 & 100.0 & 96.0 & 79.6 & 77.3 & 91.0  \\
200  & 94.5 & 99.0 & 100.0 & 96.0 & 79.3 & 77.1 & 91.0  \\
300  & 94.7 & 98.9 & 100.0 & 96.0 & 79.7 & 77.1 & 91.1 \\
400  & 94.5 & 98.9 & 100.0 & 96.0 & 79.9 & 76.8 & 91.0  \\
\bottomrule
\end{tabular}}}
\end{table}


\begin{table*}[!tbph]\small
\centering
% \renewcommand{\arraystretch}{1.1}
\caption{Accuracy (\%) when varying the number of generated samples per category on the Office-Home dataset with ResNet-50 as the backbone.}
%\vskip .5in
\label{officehome_number}
\setlength{\tabcolsep}{1mm}{
\begin{tabular}{ccccccccccccc @{\hskip 0.05in} c}
\toprule Number & Ar$\rightarrow$Cl   & Ar$\rightarrow$Pr  & Ar$\rightarrow$Rw & Cl$\rightarrow$Ar & Cl$\rightarrow$Pr & Cl$\rightarrow$Rw & Pr$\rightarrow$Ar & Pr$\rightarrow$Cl & Pr$\rightarrow$Rw & Rw$\rightarrow$Ar & Rw$\rightarrow$Cl & Rw$\rightarrow$Pr & Average \\
\midrule
0  & 57.2 & 77.2 & 82.0 & 66.5 & 77.2 & 76.7 & 62.3 & 56.3& 82.2 &  75.6 & 63.9 & 85.4 & 71.9  \\
50  & 59.7 & 78.4 & 83.0 & 68.6 & 81.5 & 79.8 & 64.7 & 57.4 & 83.5 & 75.4 & 64.7 & 86.5 & 73.6  \\
100  & 59.7 & 78.4 & 83.0 & 68.6 & 81.5 & 79.8 & 64.7 & 57.4 & 83.5 & 75.4 & 64.7 & 86.5 & 73.6  \\
200  & 59.9 & 78.8 & 83.2 & 67.8 & 81.1 & 80.6 & 63.9 & 57.9 & 83.2 & 75.9 & 64.2 & 86.1 & 73.6    \\
300  & 60.3 & 78.9 & 83.4 & 69.2 & 81.6 & 80.7 & 64.6 & 57.1 & 83.5 & 75.4 & 64.9 & 86.5 & 73.8  \\
400  & 60.6 & 79.1 & 83.5 & 68.9 & 81.7 & 80.9 & 63.7 & 58.0 & 83.7 & 75.2 & 64.7 & 86.4 & 73.8   \\
\bottomrule
\end{tabular}}
\end{table*}

\begin{table*}[!tbph]\small
\centering
% \renewcommand{\arraystretch}{1.1}
\caption{Accuracy (\%) when varying the number of generated samples per category on the VisDA-2017 dataset with ResNet-101 as the backbone. }
%\vskip .5in
\label{visda_number}
\setlength{\tabcolsep}{1.5mm}{
\begin{tabular}{ccccccccccccc @{\hskip 0.2in} c}
\toprule Number & aero & bicycle & bus & car & horse & knife & motor & person & plant & skate & train & truck & mean \\
\midrule
0 & 95.3 & 84.9 & 75.2 & 66.3 & 93.0 & 93.8 & 88.4 & 79.4 & 90.4 & 92.5 & 83.8 & 57.7 & 83.4 \\
500  & 95.9 & 83.9 & 80.2 & 69.9 & 94.1 & 93.0 & 90.3 & 81.5 & 91.0 & 94.3 & 86.4 & 56.5 & 84.7 \\
1000 & 96.1 & 82.5 & 81.6 & 72.3 & 94.9 & 97.3 & 90.4 & 80.8 & 92.9 & 92.4 & 85.8 & 56.9 & 85.3 \\
2000 & 96.6 & 86.1 & 82.2 & 68.8 & 95.4 & 96.3 & 90.8 & 82.5 & 92.7 & 93.3 & 87.9 & 60.5 & 86.1 \\
3000 & 96.5 & 85.2 & 83.0 & 74.7 &95.0 & 97.3 & 89.5 & 82.1 & 93.2 & 93.5 & 87.0 & 54.7 & 86.0 \\
4000 & 96.1 & 84.1 & 83.2 & 73.5 & 93.4 & 97.1 & 90.7 & 83.0 & 93.8 & 93.6 & 87.8 & 59.5 & 86.3 \\
\bottomrule
\end{tabular}}
\end{table*}

\section{Conclusion}
In this paper, we propose the DTS method for UDA. The DTS method is to generate high-quality samples for the target domain via CDPM. The generated target samples with pseudo labels are incorporated into source samples to form an augmented source domain, which could have a smaller domain discrepancy to the target domain than that between the original source and target domains.
In this way, the DTS method could help improve the transfer performance.
The DTS method is easy to implement and can be embedded into any UDA algorithm.
Extensive experiments demonstrate the effectiveness of the proposed DTS method.
In our future work, we are interested in applying the DTS method to other settings in transfer learning.

%The generated target samples are similar to the target samples.
% Based on our framework, the number and category of the generated samples can be flexibly controlled to alleviate the issue of scarce and unlabeled samples in the target domain.
% The generated samples are combined with the original source samples as the augmented source domain.
% And then the augmented source domain is transferred to the target domain, which reduces the difficulty of domain adaptation.
% This framework is easy to implement and can be embedded into any UDA algorithm to improve its transfer performance.
% Compared with various state-of-the-art UDA methods, our method achieves superior performance on three benchmarks.
% the effectiveness and robustness of our method can be verified.
% Qualitative experiments show that the proposed method can generate high-quality and diverse pseudo target samples.

\section*{Acknowledgements}

This work is supported by NSFC key grant 62136005, NSFC general grant 62076118, and Shenzhen fundamental research program JCYJ20210324105000003.

\bibliographystyle{ieee_fullname}
\bibliography{DTS}

% \clearpage

% \par
% \begin{Large}
% \begin{center}
% \onecolumn
% \appendix{\textbf{Appendix}}
% \end{center}
% \end{Large}

% \title{Appendix}
% \renewcommand{\appendixname}{Appendix~\Alph{section}}

% \section{More Generated Results}





% \section{More Experimental Results}
% \section{Detailed Results for Sensitivity Analysis}
% \subsection{MCC+DTS on the Office-Home Dataset}

% The comparison results between MCC and MCC+DTS on the Office-Home dataset with ResNet-50 as the backbone are shown in Table \ref{officehome_mcc}.
% With the proposed DTS framework, MCC achieves significant performance improvement in most transfer tasks, with an average accuracy improvement of 1.4\% over the original MCC method, which further demonstrates the effectiveness of the DTS method.

% \subsection{Detailed Results for Sensitivity Analysis}







% \begin{table*}[!tbph]\small
% \centering
% % \renewcommand{\arraystretch}{1.1}
% \caption{Accuracy (\%) on the Office-Home dataset with ResNet-50 as the backbone. $\uparrow$ denotes the accuracy improvement brought by the DTS framework  over the MCC method. }
% %\vskip .5in
% \label{officehome_mcc}
% \setlength{\tabcolsep}{1mm}{
% \begin{tabular}{ccccccccccccc @{\hskip 0.05in} c}
% \toprule Method & Ar$\rightarrow$Cl   & Ar$\rightarrow$Pr  & Ar$\rightarrow$Rw & Cl$\rightarrow$Ar & Cl$\rightarrow$Pr & Cl$\rightarrow$Rw & Pr$\rightarrow$Ar & Pr$\rightarrow$Cl & Pr$\rightarrow$Rw & Rw$\rightarrow$Ar & Rw$\rightarrow$Cl & Rw$\rightarrow$Pr & Average \\
% \midrule
% MCC  & 56.2 & 79.5 & 82.5 & 68.1 & 76.5 & 78.0 & 66.9 & 54.9 & 82.1 & 73.6 & 61.9 & 85.5 & 72.1   \\
% MCC+DTS      & 56.7 & 80.8 & 83.8 & 69.6 & 79.5 & 81.4 & 67.7 & 57.1 & 83.9 & 73.3 & 62.1 & 86.4 & 73.5\\
% $\uparrow $      & 0.4 & 1.3 & 1.3 & 1.5 & 3.0 & 3.4 & 0.8 & 2.2 & 1.8 & -0.3 & 0.3 & 0.9 & 1.4   \\
% \bottomrule
% \end{tabular}}
% \end{table*}

\end{document}
