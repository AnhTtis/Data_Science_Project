\section{Introduction}
\label{sec:introduction}

% Safety is important
Road safety is an important key performance indicator in transportation. 
In addition to the suffering of people as a consequence of crashes in traffic, these crashes cause enormous societal and economic losses.
As a result, road safety research is an important research topic.
For example, in 2018\footnote{At the time of writing, more recent results were not yet available.}, there were over 6.7 million crashes in the U.S.A.\ \autocite{nhtsa2020summary}, which is about 1.3 crashes per 1 million vehicle kilometers driven.
These crashes in 2018 led to 2.7 million injured people and 37 thousand fatalities \autocite{nhtsa2020summary}.
Furthermore, apart from these societal losses, the economic costs of all crashes in the U.S.A.\ in 2018 was 242 billion dollars \autocite{nhtsa2020summary}.
Similarly, the \textcite{eu2020roadsafety} reported over 22 thousand fatalities in 2019.

% Quantify safety with surrogate metrics
Road safety can be expressed in terms of injuries, fatalities, or crashes per kilometer of driving, but ``that is a slow, reactive process'' \autocite{arun2021systematic}.
Furthermore, ``crashes are rare events and historical crash data does not capture near crashes that are also critical for improving safety'' \autocite{wang2021review}.
An alternative for expressing road safety that does not rely on historical crash data is the use of safety indicators that directly measure the safety risk in traffic conflicts \autocite{tarko2018surrogate, arun2021systematic, wang2021review}.
%Particularly challenging is that ``there is no yet consensus on what `driving safely' means'' \autocite{tejada2020safe}.
Traffic conflicts are far more frequent than traffic crashes and the frequency of traffic conflicts can be used to predict the frequency of crashes \autocite{davis2011outline, tarko2018estimating}.
To define traffic conflicts, thresholds on so-called \acp{ssm} are used, where \acp{ssm} characterize the risk of a crash or harm given an initial condition \autocite{arun2021systematic}.
% Examples
\acp{ssm} vary from measures that estimate the remaining time until a crash, such as the well-known \ac{ttc} \autocite{hayward1972near}, to metrics that estimate the probability that a human driver cannot avoid a crash, see, e.g., \autocite{wang2014evaluation}.

% Common approach
\Acp{ssm} typically rely on assumptions of what drivers or systems controlling the vehicles of interest are capable of doing and how their future trajectories --- given an initial condition --- will develop. 
For example, \ac{ttc} \autocite{hayward1972near}, the ratio of the distance toward and the speed difference with an approaching object, is computed by assuming a constant relative velocity. 
As a result of these assumptions, \acp{ssm} are only applicable in certain types of scenarios.
For example, \ac{ttc} is only applicable when approaching an object.
More complex \acp{ssm} consider, e.g., a human model that can react to a risky situation by braking \autocite{wang2014evaluation} or the uncertainty over the future ambient traffic state \autocite{mullakkal2020probabilistic}.
Regardless of the complexity of these models, however, these \acp{ssm} consider neither the specific capabilities of the driver or of the system controlling the vehicle, nor the local context for predicting the future of the vehicle's environment.  

% Our proposal
This paper presents the \ac{ourmethod} method, which is a data-driven approach for deriving \acp{ssm} that are not limited to certain types of scenarios.
Because the method is not bound to certain predetermined assumptions about driver behavior, the derived \acp{ssm} can be adapted to the situations in which they are applied. 
In addition, to avoid relying on predetermined assumptions on how the ambient traffic evolves over time, the \ac{ourmethod} method includes a data-driven approach for modeling the variations of the trajectories of the ambient traffic.
Monte Carlo simulations are employed to predict the safety risk given these variations.
To enable the real-time evaluation of the derived \acp{ssm}, we use the \ac{nw} kernel estimator \autocite{wasserman2006nonparametric} for local regression.
% Advantages of our method
The \ac{ourmethod} method has the following characteristics:
\begin{itemize}
	\item The derived \acp{ssm} give a probability that a specified event, e.g., a crash or a near miss will happen in the near future, e.g., within the next 10 seconds, given an initial state and the foreseen evolutions of traffic participant trajectories. 
	Since a traffic conflict can be defined as the probability of an unsuccessful evasion in a traffic interaction, according to \textcite{davis2011outline}, a probability is easier to interpret than, e.g., a value ranging from 0 to infinity such as the \ac{ttc}.
	
	\item Next to deriving new \acp{ssm}, i.e., new ways to estimate the probability of an event such as a crash, it is possible to reproduce already existing measures that provide a probability. 
	Therefore, the \ac{ourmethod} method can be seen as a generalization for deriving such existing \acp{ssm}.
	
	\item A driver behavior model can be used.
	It is also possible to use a model of \iac{ads}, such that the derived \ac{ssm} estimates the safety risk if this \ac{ads} controls the vehicle.
	
	\item Because a data-driven approach is adopted, the derived \ac{ssm} adapts to the recorded data. 
	In this way, it is possible to adapt the \ac{ssm} to, e.g., the local traffic behavior provided that this local traffic behavior is captured by the recorded data.
	
	\item The \ac{ourmethod} method is not limited to one type of scenario.
\end{itemize}

% Explain case study.
We illustrate the \ac{ourmethod} method and its benefits by means of a case study. 
The case study demonstrates that when using the \ac{ourmethod} method with the assumptions of the \ac{ssm} of \textcite{wang2014evaluation}, both the \ac{ssm} derived by the \ac{ourmethod} method and the latter yield the same result.
The case study continues with evaluating the crash risk of three longitudinal traffic conflicts which are a priori known to be, respectively, safe (i.e., no crash possible), moderately safe, and unsafe (i.e., a crash occurs), based on vehicle kinematics. 
We evaluate the risk of each of the scenarios using the \ac{ssm} by \textcite{wang2014evaluation} and \iac{ssm} derived by the \ac{ourmethod} method, based on data from the \ac{ngsim} \autocite{alexiadis2004next}. 
Moreover, since a comparison between these measures is not directly possible in general scenarios, a method to benchmark \acp{ssm} using expected risk trends is introduced in the case study.

% Structure
This article is organized as follows.
\cref{sec:literature review} provides an overview of \acp{ssm} described in the literature.
The proposed \ac{ourmethod} method is presented in \cref{sec:method}.
In \cref{sec:case study}, we illustrate the method in a case study.
Some implications of this work are discussed in \cref{sec:discussion}.
The article is concluded in \cref{sec:conclusions}.
