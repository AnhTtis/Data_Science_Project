\section{Discussion}
\label{sec:discussion}

% Discuss assumptions -> still needed, different assumptions lead to different SSMs
Typically, \acp{ssm} rely on assumptions regarding the behavior of traffic participants. 
An advantage of the presented \ac{ourmethod} method for deriving \acp{ssm} is that the \ac{ourmethod} method is not bound to certain predetermined assumptions. 
We want to stress, however, that when using the \ac{ourmethod} method for deriving \iac{ssm}, a set of assumptions is still needed.
In fact, multiple \acp{ssm} can be derived by using the \ac{ourmethod} method with different sets of assumptions. 
As a result, the \ac{ourmethod} method can be used to derive multiple \acp{ssm} that are applicable in various types of scenarios, e.g., ranging from vehicle-following scenarios to scenarios at intersections. 
Note that although the \ac{ourmethod} method is applicable in various types of scenarios, the current case study focuses on longitudinal traffic conflicts.
In a future work, we will present the application of the \ac{ourmethod} method for deriving \acp{ssm} for lateral traffic conflicts.

% Data-driven approach can also lead to bias
The \ac{ourmethod} method uses data to adapt the \acp{ssm} to, e.g., the local traffic behavior. 
More specifically, the data are used to predict the possible future situations ($\situationfuture$) given an initial situation ($\situationinitial$).
This can be an advantage because the data can be used to rely less on assumptions as to how the future develops given an initial situation. 
To fully benefit from this approach, the data should satisfy a few conditions.
First, the recorded data need to represent the actual traffic behavior in which the \acp{ssm} are applied. 
Second, we need enough data to estimate $\densitycond{\situationfuture}{\situationinitial}$.
In \autocite{degelder2019completeness}, a metric is presented that can be used to determine whether enough data have been collected to estimate $\densitycond{\situationfuture}{\situationinitial}$ accurately.

% If no data -> assume a p(y|x) or make assumptions such that p(y|x) is not needed
The \ac{ourmethod} method can still be applied in case no data are available.
The first alternative is to use existing knowledge to determine an estimate of $\densitycond{\situationfuture}{\situationinitial}$ instead of estimating $\densitycond{\situationfuture}{\situationinitial}$ on the basis of data. 
For example, statistics or literature on driving behavior of traffic participants may be used.
The second alternative is to use assumptions on how the future develops given an initial situation $\situationinitial$.
For example, when assuming that the speed of the leading vehicle in \cref{sec:wang stamatiadis replicate} remains constant, it is not needed to estimate $\densitycond{\situationfuture}{\situationinitial}$.
Note that a combination is also possible.
For example, estimate $\densitycond{\situationfuture}{\situationinitial}$ based on data in case $\situationinitial$ is well represented in the data, but define $\densitycond{\situationfuture}{\situationinitial}$ on the basis of existing knowledge and/or assumptions for the cases where $\situationinitial$ is underrepresented in the data.
\cstart A third alternative is to use other methods for predicting the trajectories of the other traffic participants, e.g., using hidden Markov models \autocite{laugier2011probabilistic}, sequence similarity methods \autocite{saunier2007probabilistic}, Gaussian mixture models \autocite{wiest2012probabilistic}, or long short-term memory networks \autocite{deo2018multi}. 
For an overview of trajectory prediction models for vehicles and pedestrians, see \autocite{lefevre2014survey} and \autocite{rudenko2020human}, respectively. \cend

Note that the \ac{ourmethod} method is used to derive \acp{ssm} that predict the probability of a specific event, such as a crash, i.e., the derived \acp{ssm} can be used as a measure of proximity of the specified event.
However, the \ac{ourmethod} method is not used to measure the severity of an interaction, i.e., the extent of harm in case the interaction leads to a crash.
For measuring the severity of an interaction, typically energy-based \acp{ssm} are used \autocite{wang2021review}.
So, if there is a need to also have an indicator of the severity of an interaction, an energy-based \ac{ssm}, e.g., see \autocite{ozbay2008derivation, alhajyaseen2015integration, laureshyn2017search, mullakkal2020probabilistic}, may be considered alongside \iac{ssm} derived using the \ac{ourmethod} method.

% In examples -> only look at collisions. If focus is on "safe driving", other "events" can be considered.
We have illustrated the \ac{ourmethod} method through different derived \acp{ssm} in the case study.
The derived \acp{ssm} estimate the probability of a crash with a leading vehicle under different assumptions.
Because of the focus on crashes, the resulting \acp{ssm} may still be low given an initial situation that is generally considered to be unsafe. 
For example, the \ac{ssm} described in \cref{sec:ngsim metric} gives a crash probability of approximately \SI{14}{\percent} when approaching a leading vehicle that is driving at a constant speed of $\speedleadsymbol=\SI{12}{\meter\per\second}$ ($\accelerationleadsymbol=\SI{0}{\meter\per\second\squared}$) with a speed of $\speedegosymbol=\SI{25}{\meter\per\second}$ and a gap of $\gapsymbol=\SI{20}{\meter}$ (see left heat map in \cref{fig:heatmaps}).
In this initial situation, the \ac{thw} is only $\gapsymbol/\speedegosymbol=\SI{0.8}{\second}$ and the \ac{ttc} is only $\gapsymbol/(\speedegosymbol-\speedleadsymbol)=\SI{1.5}{\second}$, whereas \iac{thw} of less than \SI{1}{\second} or \iac{ttc} of less than \SI{1.5}{\second} is considered unsafe \autocite{vogel2003comparison}.
In order to put more emphasis on such unsafe situations, different events --- instead of crashes --- can be considered.
For example, we can derive \iac{ssm} that estimates the probability that the \ac{ttc} is below \SI{1}{\second} within the next 5 seconds.
More research is needed to investigate whether such \acp{ssm} can be of practical use, e.g., for evaluating whether a driver is actively pursuing large safety margins.

% Choice of grid points: more=better, but more=slower. Choose smart: around changes
A few choices have to be made when using the \ac{ourmethod} method for deriving \acp{ssm}.
One such a choice is the set of initial situations $\{\situationinitialinstance{1},\ldots,\situationinitialinstance{\numberofdesignpoints}\}$ for which the probability $\probabilitycond{\collision}{\situationinitial}$ is estimated.
Generally speaking, for larger $\numberofdesignpoints$, the approximation of $\probabilitycond{\collision}{\situationinitial}$ in \cref{eq:nadaraya watson} improves.
One disadvantage, however, is that more simulation runs are required when $\numberofdesignpoints$ is larger, but because these simulation runs are performed offline, this problem might be solved by, e.g., parallel computing resources. 
Another disadvantage is that the computational cost of the approximation in \cref{eq:nadaraya watson} scales linearly with $\numberofdesignpoints$.
Especially when using this approximation for real-time evaluation of the \ac{ssm}, this can be a bottleneck.
One solution to this is to not use all $\numberofdesignpoints$ initial situations for evaluating \cref{eq:nadaraya watson}.
The intuition is as follows: since \cref{eq:nadaraya watson} uses local regression, an initial situation $\situationinitialinstance{\situationindexdesign}$ can be removed from the set $\{\situationinitialinstance{1},\ldots,\situationinitialinstance{\numberofdesignpoints}\}$ if all neighboring data points give (approximately) the same probability of the event $\collision$, i.e., $|\probabilityestcond{\collision}{\situationinitialinstance{\situationindex}}-\probabilityestcond{\collision}{\situationinitialinstance{\situationindexdesign}}|$ is below a threshold for all $\situationinitialinstance{\situationindex}$, $\situationindex\ne\situationindexdesign$ for which $\normtwo{\situationinitialinstance{\situationindex}-\situationinitialinstance{\situationindexdesign}}$ is below another threshold (assuming that $\probabilityestcond{\collision}{\situationinitial}$ is sufficiently smooth).
For example, the \ac{ssm} that is shown in \cref{fig:heatmaps}, only a few initial situations are required in the upper right region of the heat maps, since the estimated probability is always lower than 0.1.

% Choice of epsilon: lower=better, but lower=slower. Tradeoff, but since it is offline, it may be fine if it takes long.
Another choice is the threshold $\simulationthreshold$ that controls the number of simulation runs ($\numberofsimulations$) that are used to estimate $\probabilitycond{\collision}{\situationinitial}$.
According to \cref{eq:condition stop simulations}, $\numberofsimulations$ is increased until the variance of the estimation error is below $\simulationthreshold$, i.e., $\variance{\probabilitycond{\collision}{\situationinitial}-\probabilityestcond{\collision}{\situationinitial}}<\simulationthreshold$.
Therefore, a lower $\simulationthreshold$ generally results in more accurate estimations of the probability, as illustrated in \cref{fig:ws comparison}.
The downside, however, is that for a lower $\simulationthreshold$, more offline simulation runs are required. 
Although a good choice of $\simulationthreshold$ remains a topic of research, based on experience, we advice to use a maximum threshold of $\simulationthreshold=0.1$ and lower values if the computational resources allow for this.

% Multiple actors: compute probability of collision per actor.
In the examples presented in \cref{sec:case study}, we have considered the leading vehicle as the only traffic participant other than the ego vehicle.
The \ac{ourmethod} method can be applied in scenarios with multiple traffic participants other than the ego vehicle.
However, the number of parameters ($\situationinitialdim$, i.e., the size of $\situationinitial$) then becomes larger.
As a result, two problems may arise.
First, as $\numberofdesignpoints$ grows exponentially with $\situationinitialdim$, so does the number of simulation runs.
Second, even if these simulation runs can be performed, the regression using \cref{eq:nadaraya watson} becomes slow due to the large $\numberofdesignpoints$.
To overcome these problems, \iac{ssm} can be computed for each traffic participant independently.
For example, let $\numberoftrafficparticipants$ denote the number of traffic participants other than the ego vehicle.
With $\trafficparticipantindex\in\{1,\ldots,\numberoftrafficparticipants\}$, let $\collision_{\trafficparticipantindex}$ denote the event of colliding with the $\trafficparticipantindex$-th traffic participant and let $\situationinitialtp{\trafficparticipantindex}$ denote the initial situation considering the $\trafficparticipantindex$-th traffic participant.
Under the assumption that $\probabilitycond{\collision_{\trafficparticipantindex}}{\situationinitialtp{\trafficparticipantindex}}$ is independent of $\situationinitialtp{\trafficparticipantindexb}$ for all $\trafficparticipantindex\ne\trafficparticipantindexb$, we can calculate the probability of colliding with one or more traffic participants using
\begin{equation}
	\label{eq:combine traffic participants}
	1 - \prod_{\trafficparticipantindex=1}^{\numberoftrafficparticipants}
	\left( 1 - \probabilitycond{\collision_{\trafficparticipantindexb}}{\situationinitialtp{\trafficparticipantindexb}} \right).
\end{equation}
For example, consider a scenario with multiple crossing pedestrians. 
Using the \ac{ourmethod} method, we can derive \iac{ssm} that estimates the probability of colliding with a pedestrian.
Then, after evaluating this \ac{ssm} for each pedestrian, the probability of colliding with one or more pedestrians can be calculated using \cref{eq:combine traffic participants} without the need for \iac{ssm} that considers multiple pedestrians.

% Verifying the probability
In the case study, we have shown how to analyze \iac{ssm} both qualitatively, using heat maps and testing the \ac{ssm} in different scenarios, and quantitatively by benchmarking the \ac{ssm} with expected risk trends \autocite{mullakkal2017comparative}.
Since the \acp{ssm} derived using the \ac{ourmethod} method provide a probability, it is also possible to verify the estimated probability by comparing it with real data. 
This requires, however, an extensive data set that would allow for estimating the probability of the event $\collision$, e.g., a crash, in the near future given a certain situation a vehicle is in. 
It remains a topic for future work to use such a data set to verify the \acp{ssm} derived using the \ac{ourmethod} method.

% Limitations
% 1. x must be small
% 2. KDE doesn't work well with large d
% 3. A lot of simulations may be required
% 4. No single best SSM 
% 5. Response of ego vehicle must be assumed
\cstart The \ac{ourmethod} method is a novel approach for deriving probabilistic \acp{ssm} for risk evaluation. 
Some limitations, however, may hamper its use for real-world applications.
First, as described earlier, if the set of initial situations $\{\situationinitialinstance{1},\ldots,\situationinitialinstance{\numberofdesignpoints}\}$ is too large, real-time calculation of the \ac{ssm} may be difficult. 
As a consequence, the dimension of the vector describing the initial situation, $\situationinitial$, cannot be too large, meaning that the initial situation needs to be encoded into a limited set of numbers.
Second, \ac{kde} does not work well for large dimensions.
Note, however, that we have provided some options for reducing the dimensionality, and, if reducing the dimensionality further is not an good option, the \ac{ourmethod} method can also be applied when other methods are used for the probability density estimation.
Third, many simulations may be required. 
It helps that the simulations can be conducted offline, but it may still be challenging to conduct the simulations in a reasonable time window.
Fourth, although we claim that the \ac{ourmethod} method can be used to derive multiple \acp{ssm} for different type of scenarios, we cannot claim that the derived \acp{ssm} are more valid than others \acp{ssm}, nor can we derive \iac{ssm} that is valid for all types of scenarios.
Lastly, for the simulations, the response of the ego vehicle must be assumed.
This may not coincide with the ego vehicle response in reality.
In a future work, in case of a human driver, this may be tackled by considering the state of the human driver, such as whether the eyes are on the road and/or towards a conflicting traffic participant, as part of the state vector that is used to describe the initial situation. \cend
