\section{Literature review}
\label{sec:literature review}

Risk in the context of traffic safety is often defined as the probability of a crash occurring \autocite{hakkert2002uses}.
Most \acp{ssm} are derived under specific assumptions of the expected behavior of the  driving participants under a specific  driving scenario.
Several \acp{ssm} have been developed under such assumptions with the goal of quantifying the risk involved in driving in traffic on the road \autocite{minderhoud2001extended, ozbay2008derivation, cunto2009simulated, laureshyn2010evaluation}.
In general, the risk is quantified in terms of the proximity between two traffic agents in time and/or space, the ability to perform evasive actions like braking or swerving, or the magnitude of such actions \autocite{shi2018key,zheng2020modeling}. 
In a potential crash situation, the proximity indicator is close to zero while the magnitude of evasive action is close to the limits of the driver and the vehicle \autocite{zheng2020modeling}. 
The above clustering of \acp{ssm} in terms of time, space, and evasive action is common in the literature; so our literature review follows this pattern of clustering \acp{ssm}.
We focus on the most commonly used measures in each cluster and their underlying assumptions.
\cstart Additionally, we discuss some well-known \ac{ssm}-based metrics, i.e., metrics that are derived from other \acp{ssm}. \cend

The most common \acp{ssm} are time-based. 
A popular time-based proximity indicator is the \ac{ttc}, which is an estimate of the remaining time until two vehicles collide and is defined as the time remaining until two vehicles collide if they would continue on the same course and speed \autocite{hayward1972near}.
The assumption for the \ac{ttc} is that the relative speed and course will remain the same. 
In addition, the \ac{ttc} is only relevant when two objects are approaching each other. 
These assumptions make it difficult to use it  for various driving scenarios.
Several other time-based \acp{ssm} have been derived from or based on the \ac{ttc}. 
Notable among those are:
\begin{itemize}
    \item the time-exposed \ac{ttc}, which measures the amount of time the \ac{ttc} is below a certain threshold \autocite{minderhoud2001extended};
    \item the \ac{tit}, which calculates the total area in \iac{ttc} versus time diagram where the \ac{ttc} is below a certain threshold  \autocite{minderhoud2001extended};
    \item the \ac{mttc}, which is able to calculate the \ac{ttc} for cases where vehicles do not keep a constant speed and the follower is slower than the leader \autocite{ozbay2008derivation}; \cstart and
    \item the \ac{ttcd}, which calculates the \ac{ttc} in case the leader is decelerating with a constant deceleration \autocite{xie2019use}. \cend
\end{itemize}
For the \ac{mttc} \cstart and \ac{ttcd}\cend, the relative speed is not assumed to be constant, but new assumptions on the acceleration and speed of the objects are introduced. 

Other time-based proximity indicators include \ac{pet}, which measures the ``time between the moment that a vehicle leaves the area of potential collision, i.e., the area in which the paths of the two vehicles intersect, and the other vehicle arrives in the same area'' \autocite{mahmud2017application} and \ac{thw}.
\ac{pet} can only be calculated when the collision area of the two participants is known. 
This assumption makes it mostly useful for scenarios with obvious crossing conflicts like intersections.

For distance-based proximity indicators, the \ac{picud} measures the remaining distance between vehicles during an emergency stop \autocite{iida2001traffic, uno2003objective} and the \ac{psd} measures the remaining distance to the potential point of collision divided by the minimum acceptable stopping distance \autocite{allen1978analysis, guido2011comparing, mahmud2017application}. 
These two measures assume that the vehicles will apply the maximum deceleration during emergency situations. 
This makes them suitable for emergency situations for which these assumption will most likely hold. 
For non-critical situations, however, the deceleration that the drivers will apply, may vary.
%Other similar distance-based \acp{ssm} are the ``margin to collision'' and ``difference of space distance and stopping distance'', which are computed similar to the \ac{psd} and the \ac{picud}, respectively \autocite{kitajima2009estimation, okamura2011impact}. 
More recently, a distance-based measure that assumes ``correct'' driving behavior has been proposed \autocite{shalev2017formal}. 
This measure calculates the minimum safety distance between a follower and its leader, such that no crash occurs if the leader vehicle brakes with a specified deceleration and the follower brakes after a specified reaction time with another specified deceleration. 
Based on the definition of this measure, it is not suitable for driving situations where the driver does not follow the description of ``correct'' driving given above.

In terms of indicators relating to performing evasive actions, the \ac{drac} is the most widely used. 
The \ac{drac} is calculated as the ratio of the difference in speed between a following vehicle and a leading vehicle and their closing time \autocite{almqvist1991use, mahmud2017application}. 
Another indicator is the \ac{cpi}, which calculates the probability that a vehicle's \ac{drac} will exceed its \ac{madr} in a given time interval \autocite{cunto2009simulated}. 
The \ac{drac} is not a risk measure on its own, if it is not compared with the braking capacity.
This is a limitation and this is why the \ac{cpi} measure has been developed.
Both \ac{drac} and \ac{cpi} are mostly suitable for a car-following situation and are not suitable for lateral movements \autocite{mahmud2017application}.

\cstart\textcite{wang2014evaluation} and \textcite{xie2019use} propose \iac{ssm}-based metric, i.e., they derive a probability using the \ac{ttc} and \ac{ttcd}, respectively, and some assumptions. 
In particular, \textcite{wang2014evaluation} assume distributions of the vehicle braking capability and the driver's reaction time, while \textcite{xie2019use} assume a distribution of the deceleration rate of the leader.
Although these probabilities are suitable for various car-following situations, lane-change conflicts, and crossing conflicts, they are limited because they use the \ac{ttc} and \ac{ttcd}, respectively, in their calculations; so these probabilities are undefined when the \ac{ttc} and \ac{ttcd}, respectively, are undefined.

\textcite{saunier2008probabilistic} derive a probability based on the \acp{ttc} that are estimated using hypothetical trajectories of the different traffic participants. 
Similar to the approach we present in \cref{sec:method}, the hypothetical trajectories of the traffic participants are based on a data-driven model \autocite{saunier2007probabilistic}. 
In \autocite{saunier2008probabilistic}, also the probabilities of the different hypothetical trajectories are considered. 
In order to keep the computation of the metric feasible in real time, the number of hypothetical trajectories to be considered is limited.
\textcite{altendorfer2021new} also calculate the probability of a collision. 
They provide a general framework that can consider any arbitrary prediction model of the trajectories of the traffic participants.
To ensure real-time computations, their example considers Gaussian distributions and simplified geometries of the traffic participants, such that they can use numerical integration instead of Monte Carlo simulations. \cend

\textcite{shi2018key} use indicators like \ac{tit}, \ac{cpi}, and \ac{psd} to measure the effectiveness of risk indicators for predicting crashes. 
The idea is to use a combination of indicators and thresholds on the indicators to predict whether an interaction may become a crash.
This results in new indicators, but they inherit the union of the assumptions of the other indicators.
\textcite{mullakkal2020probabilistic} propose a probabilistic driving risk field.
The method derives the risk a vehicle is exposed to using a kinematic approach with the inclusion of uncertainty in the vehicle's future state. 
\textcite{mullakkal2020probabilistic} define this for an encounter between the ego vehicle and a road obstacle, such as other vehicles or objects. 
This research shares similar ideas with our proposed method of risk estimation, but \textcite{mullakkal2020probabilistic} do not use a data-driven approach to derive the \ac{ssm}. 
Furthermore, the future state of the vehicle is estimated with a fixed distribution (i.e., a normal distribution). 
This limits the application in scenarios where the data may have an entirely different distribution.

To estimate crash probabilities based on existing \acp{ssm}, the probabilistic approach using the \ac{evt} has been applied successfully \autocite{songchitruksa2006extreme, tarko2012use, wang2021review}.
For example, based on a specific \ac{ttc} value, \ac{evt} can be used to predict the probability of a crash. 
Using \ac{evt}, the crash probability is estimated by assuming the generalized extreme value distribution and fitting the parameters of the distribution using either the ``block maxima'' approach or the ``peak over'' approach \autocite{wang2021review}.
It is also possible to combine multiple \acp{ssm} using the \ac{evt}.
The advantage of \ac{evt} is that \ac{evt} provides probabilities that are directly linked to historical data and that these probabilities have been used successfully to predict the frequency of crashes \autocite{songchitruksa2006extreme, aasljung2017using}.
Disadvantages of \ac{evt} are that it might inherit the assumptions of the \acp{ssm} that it uses to estimate the crash probability and that it assumes a fixed distribution of the extreme events, which is only justified if a lot of data is used.
Furthermore, as the estimated crash probability is solely based on the fitted distribution, it does not consider potential changes to the driver's behavior (model).
