\subsection{Related Work}
The information elicitation problem and its implications for decision making have been one of the most popular areas in economics. The seminal work by \cite{savage1971elicitation} characterizes the truthful elicitation with the design of proper scoring rules. More recently, several work have studied the optimization problem of scoring rules for the principal's different objectives~\cite{CY-21,li2022optimization,10.1145/3490486.3538261, neyman2021binary}. Closer to our problem is the model studied by \citet{10.1145/3490486.3538261,oesterheld2020minimum}, as we both explore the connection between information acquisition and contract design. However, our model is strictly more general in that we assume the agent may have multi-level efforts, and the information state may be affected by the agent's action. In addition, we primarily focus on the learning aspects, when the principal has limited knowledge of the game parameters. 

More generally, our problem is related to a family of online learning problems under the principal-agent framework, which includes special cases such as security game, information design, auction design, contract design and etc. However, to our best knowledge, there is no previous work that considered any similar learning problem for our model of information acquisition. We also remark that, in many of these existing works, e.g.,~\citet{balcan2015commitment,guo2022no,castiglioni2020online,wu2022sequential}, the learner is assumed to have sufficient knowledge about the other strategic player(s), and her uncertainty is regarding the environment or her own utility. Assumptions of such kind can significantly simplify the problem into the standard online learning problems, once the learner can almost predict the best response of other player(s). Notably, our work does not make any of such assumption and the most challenging part of our learning algorithm design is indeed to ensure the desired agent response under uncertainty. 
We defer more discussions on the related work to \Cref{sec:related-work-full}.