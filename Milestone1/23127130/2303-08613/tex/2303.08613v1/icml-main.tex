%%%%%%%% ICML 2023 EXAMPLE LATEX SUBMISSION FILE %%%%%%%%%%%%%%%%%
\pdfoutput=0
\documentclass{article}

% Recommended, but optional, packages for figures and better typesetting:
\usepackage{microtype}
\usepackage{graphicx}
\usepackage{subfigure}
\usepackage{booktabs} % for professional tables

% hyperref makes hyperlinks in the resulting PDF.
% If your build breaks (sometimes temporarily if a hyperlink spans a page)
% please comment out the following usepackage line and replace
% \usepackage{icml2023} with \usepackage[nohyperref]{icml2023} above.
\usepackage{hyperref}


% Attempt to make hyperref and algorithmic work together better:
\newcommand{\theHalgorithm}{\arabic{algorithm}}

% Use the following line for the initial blind version submitted for review:
\usepackage{icml2023}

% If accepted, instead use the following line for the camera-ready submission:
% \usepackage[accepted]{icml2023}

% For theorems and such
\usepackage{amsmath}
\usepackage{amssymb}
\usepackage{mathtools}
\usepackage{amsthm}

% if you use cleveref..
\usepackage[capitalize,noabbrev]{cleveref}

% \documentclass[11pt]{article}
% \usepackage[utf8]{inputenc}
% \pdfoutput=1
% \documentclass[11pt]{article}
%\documentclass{article}
% \usepackage[OT1]{fontenc}
% \renewcommand{\baselinestretch}{1.2}

% \usepackage{times}
% \usepackage{amsmath,amsthm, amssymb}
% \usepackage[ruled,vlined]{algorithm2e}
% \SetKwComment{Comment}{/* }{ */}
% \usepackage{fullpage}
\usepackage{color}
\usepackage{natbib}
% \renewcommand{\nu}{\ensuremath{\mathbf{n}(\mathbf{u})}\xspace}  % the normal vector at pixel location \V{u}
\newcommand{\pu}{\ensuremath{\mathbf{p}(\mathbf{u})}\xspace}   % the 3d point correspoinding the pixel \V{u}
\newcommand{\du}{\ensuremath{d(\mathbf{u})}\xspace}  
\newcommand{\zu}{\ensuremath{z(\mathbf{u})}\xspace}
\newcommand{\eu}{\ensuremath{\mathbf{e}(\mathbf{u})}\xspace}
\newcommand{\up}{\ensuremath{\V{u}_{\V{p}}}\xspace}
\newcommand{\tup}{\ensuremath{\tilde{\V{u}}_{\V{p}}}\xspace}

\newcommand{\oz}{\ensuremath{\Omega_z}\xspace}  
\newcommand{\on}{\ensuremath{\Omega_n}\xspace}
\newcommand{\Nu}{\ensuremath{\mathcal{N}(\V{u})}\xspace}

\renewcommand{\ni}{normal integration\xspace}
\newcommand{\NI}{Normal Integration\xspace}
\newcommand{\dpe}{discrete Poisson's equation\xspace}
\newcommand{\Dpe}{Discrete Poisson's equation\xspace}


\newcommand{\z}{\ensuremath{\V{z}}\xspace}
\newcommand{\zs}{\ensuremath{\V{z}^*}\xspace}
\newcommand{\rz}{\ensuremath{\red{\V{z}}}\xspace}
\newcommand{\zt}{\ensuremath{\V{z}_{t}}\xspace}
\newcommand{\zto}{\ensuremath{\V{z}_{t+1}}\xspace}
\newcommand{\R}{\ensuremath{\mathbb{R}}\xspace}
\newcommand{\fz}{\ensuremath{f(\V{z})}\xspace}

\newcommand{\rt}{\ensuremath{\V{r}_{t}}\xspace}
\newcommand{\rto}{\ensuremath{\V{r}_{t+1}}\xspace}


\newcommand{\dup}{\ensuremath{\V{D}_u^{+}}\xspace}
\newcommand{\dun}{\ensuremath{\V{D}_u^{-}}\xspace}
\newcommand{\dvp}{\ensuremath{\V{D}_v^{+}}\xspace}
\newcommand{\dvn}{\ensuremath{\V{D}_v^{-}}\xspace}
\newcommand{\nx}{\ensuremath{\V{n}_x}\xspace}
\newcommand{\ny}{\ensuremath{\V{n}_y}\xspace}
\newcommand{\nz}{\ensuremath{\V{n}_z}\xspace}
\newcommand{\Nz}{\ensuremath{\V{N}_z}\xspace}

\newcommand{\ft}{\ensuremath{F(\red{\V{z}};\V{z}_t)}\xspace}
\newcommand{\ftt}{\ensuremath{F(\V{z}_t;\V{z}_t)}\xspace}
\newcommand{\fto}{\ensuremath{F(\V{z}_{t+1};\V{z}_t)}\xspace}

\newcommand{\dpu}{\ensuremath{\partial_u \V{p}}\xspace}
\newcommand{\dpv}{\ensuremath{\partial_v \V{p}}\xspace}

\renewcommand{\u}{\ensuremath{\V{u}}\xspace}
\newcommand{\dzdu}{\ensuremath{\partial_u z}\xspace}
\newcommand{\dzdv}{\ensuremath{\partial_v z}\xspace}
\newcommand{\dztdu}{\ensuremath{\partial_u \tilde{z}}\xspace}
\newcommand{\dztdv}{\ensuremath{\partial_v \tilde{z}}\xspace}
\newcommand{\dzpdu}{\ensuremath{\partial_{u}^{+} z}\xspace}
\newcommand{\dzpdv}{\ensuremath{\partial_{v}^{+} z}\xspace}
\newcommand{\dzndu}{\ensuremath{\partial_{u}^{-} z}\xspace}
\newcommand{\dzndv}{\ensuremath{\partial_{v}^{-} z}\xspace}

\newcommand{\dzpduv}{\ensuremath{\partial_{\{u,v\}}^{+} z}\xspace}
\newcommand{\dznduv}{\ensuremath{\partial_{\{u,v\}}^{-} z}\xspace}
\newcommand{\dzduv}{\ensuremath{\partial_{\{u,v\}} z}\xspace}

\newcommand{\dupz}{\ensuremath{\Delta_{u}^{+} z}\xspace}
\newcommand{\dunz}{\ensuremath{\Delta_{u}^{-} z}\xspace}
\newcommand{\dvpz}{\ensuremath{\Delta_{v}^{+} z}\xspace}
\newcommand{\dvnz}{\ensuremath{\Delta_{v}^{-} z}\xspace}

\newcommand{\nuv}{\ensuremath{\V{n}(u,v)}\xspace}
\newcommand{\zuv}{\ensuremath{z(u,v)}\xspace}
\newcommand{\puv}{\ensuremath{\V{p}(u,v)}\xspace}

\newcommand{\halfpi}{\ensuremath{\pm {\pi \over 2}}\xspace}


\newcommand{\curve}{\ensuremath{\mathbb{S}}\xspace}
\newcommand{\zenith}{zenith\xspace}
\newcommand{\surface}{\ensuremath{\mathcal{M}}\xspace}
\newcommand{\visibility}{\ensuremath{\Phi_{i}}\xspace}
\newcommand{\point}{\ensuremath{\V{x}}\xspace}
\newcommand{\normal}{\ensuremath{\V{n}}\xspace}
\newcommand{\tangent}{\ensuremath{\V{t}}\xspace}
\newcommand{\cameraNum}{\ensuremath{C}\xspace}
\newcommand{\cameraCenter}{\ensuremath{\V{o}_{i}}\xspace}
\newcommand{\viewDirection}{\ensuremath{\V{v}}\xspace}
\newcommand{\batchsize}{\ensuremath{P}\xspace}
\newcommand{\mask}{\ensuremath{O}\xspace}
\newcommand{\projectedTangentVector}{projected tangent vector\xspace}
\newcommand{\projectedTangentVectors}{projected tangent vectors\xspace}
\newcommand{\stackedTangentVectors}{\ensuremath{\V{T}(\point)}\xspace}
\newcommand{\diligentmv}{\mbox{DiLiGenT-MV}\xspace}
\newcommand{\diligent}{DiLiGenT}
\newcommand{\loss}{\mathcal{L}\xspace}
\newcommand{\opticalAxis}{\ensuremath{\V{e}_{z}\xspace}}
\newcommand{\opticalAxisViewI}{\ensuremath{\V{e}_{z_{i}}}\xspace}
\newcommand{\opticalAxisMatrix}{\ensuremath{\V{C}}\xspace}
\newcommand{\ms}{Mumford-Shah integrator\xspace}
\newcommand{\made}{MADE\xspace}

\newcommand{\pandora}{\mbox{PANDORA}\xspace}
\newcommand{\psnerf}{\mbox{PS-NeRF}\xspace}
\newcommand{\sdps}{\mbox{SDPS}\xspace}
\newcommand{\uanet}{\mbox{UA-MVPS}\xspace}
\newcommand{\rmvps}{\mbox{R-MVPS}\xspace}
\newcommand{\bmvps}{\mbox{B-MVPS}\xspace}
\newcommand{\volsdf}{\mbox{VolSDF}\xspace}
\newcommand{\unisurf}{\mbox{UNISURF}\xspace}


\newcommand{\mvas}{MVAS\xspace}

\newcommand{\tsc}{\mbox{TSC}\xspace}

\newcommand{\pointOne}{\ensuremath{\point_1}\xspace}
\newcommand{\pointTwo}{\ensuremath{\point_2}\xspace}
\newcommand{\pointsetOne}{\ensuremath{\chi_{1}}\xspace}
\newcommand{\pointsetTwo}{\ensuremath{\chi_{2}}\xspace}
\newcommand{\fscoreThreshold}{\ensuremath{\tau}\xspace}
\newcommand{\chamferDist}{\ensuremath{d(\pointsetOne, \pointsetTwo)}\xspace}
\newcommand{\precision}{\ensuremath{\mathcal{P}}\xspace}
\newcommand{\recall}{\ensuremath{\mathcal{R}}\xspace}
\newcommand{\fscore}{\ensuremath{\mathcal{F}}\xspace}

\newcommand{\phaseangle}{\ensuremath{\hat{\phi}}\xspace}
\newcommand{\azimuthangle}{\ensuremath{\phi}\xspace}

\newcommand{\colorbar}[3]{
\begin{tabular}[t]{@{}l@{}l@{}}
	\includegraphics[height=#1\linewidth,width=0.5em]{colorbar.pdf} & 
	\begin{tabular}[b]{@{}l}
		#2 \\ [#3pt]
		$0$
	\end{tabular}
\end{tabular}
}


\usepackage{smile}




\newcommand{\todo}[1]{{\color{red}  [\text{TODO:} #1]}}
\newcommand{\BR}[1]{\mathrm{BR}\rbr{#1}}
\def\vol{\mathrm{Vol}}
\def\where{\quad\text{where}\quad}
\def\LP{\text{LP}}
\def\OptLP{\text{Opt-LP}}
\def\BS{\text{BS}}

\icmltitlerunning{Learning to   Incentivize  Information Acquisition}
\def\baselinestretch{0.99}
\begin{document}

\twocolumn[
\icmltitle{Learning to Incentivize  Information Acquisition: \\ Proper Scoring Rules Meet Principal-Agent Model}]
%\title{{\huge Final Report}\\ {Learning to Optimally Acquire Information} }

\icmlsetsymbol{equal}{*}

\begin{icmlauthorlist}
\icmlauthor{Siyu Chen}{}
\icmlauthor{Jibang Wu}{}
\icmlauthor{Yifan Wu}{}
%\icmlauthor{}{sch}
%\icmlauthor{}{sch}
\end{icmlauthorlist}

\icmlaffiliation{yyy}{Department of XXX, University of YYY, Location, Country}
\icmlaffiliation{comp}{Company Name, Location, Country}
\icmlaffiliation{sch}{School of ZZZ, Institute of WWW, Location, Country}

\icmlkeywords{Scoring Rules, Acquire Information, Online Learning}


\begin{abstract}
We study the incentivized information acquisition problem, where a principal hires an agent to gather information on her behalf.
Such a problem is modeled as a Stackelberg game between the principal and the agent, where the principal announces a scoring rule that specifies the payment, and then the agent then chooses an effort level that maximizes her own profit and reports the information.
We study the online setting of such a problem from the principal's perspective, i.e., designing the optimal scoring rule by repeatedly interacting with the strategic agent. We design a provably sample efficient algorithm that tailors the UCB algorithm \citep{auer2002finite} to our model, which achieves a sublinear $  T^{2/3}$-regret after $T$ iterations. 
Our algorithm features a delicate estimation procedure for the optimal profit of the principal, and a conservative correction scheme that ensures the desired agent's actions are incentivized. Furthermore, a  key feature of our regret bound is that it is independent of the number of states of the environment.  
\end{abstract}

\section{Introduction}

The increasing complexity of source code poses a key challenge to the reliability of large-scale software systems. Software bugs in these systems can lead to safety issues~\cite{bug_safety} for users around the world as well as cause non-negligible financial losses~\cite{bug_loss}. As such, developers have to spend a large amount of time and effort on bug fixing. Consequently, \aprfull (\apr), designed to automatically generate patches to fix software bugs, has attracted wide attention from both academia and industry~\cite{long2016prophet, legoues2012genprog, long2015spr, lou2020can, tufano2018empstudy}. 


To achieve \apr, one popular approach is known as Generate-and-Validate (G\&V)~\cite{qi2015gv, ghanbari2019prapr, lou2020can, le2016hdrepair, legoues2012genprog, wen2018capgen, hua2018sketchfix, martinez2016astor, koyuncu2020fixminder, liu2019tbar, liu2019avatar}, which is typically based on the following pipeline: First, fault localization techniques~\cite{wong2016fl, abreu2007ochiai, zhang2013injecting, papadakis2015metallaxis, li2019deepfl, li2017transforming} are applied to determine the suspicious locations in programs where bugs are likely to exist. Then, the buggy locations are used by the \apr tools to generate a list of patches that replace buggy lines with correct lines. Afterward, each patch is validated against the original test suite to identify any \emph{plausible patches} (i.e., passing all tests in the test suite). Finally, to determine the \emph{correct patches}, developers examine the list of plausible patches to see if any of them can correctly fix the bug. 

Traditional \apr tools can mainly be categorized into heuristic-based~\cite{legoues2012genprog, le2016hdrepair, wen2018capgen}, constraint-based~\cite{mechtaev2016angelix, le2017s3, demacro2014nopol, long2015spr} and \template~\cite{ghanbari2019prapr, hua2018sketchfix, martinez2016astor, liu2019tbar, liu2019avatar}. Among these traditional tools, \template \apr tools~\cite{ghanbari2019prapr, liu2019tbar, benton2020effectiveness} have been able to achieve state-of-the-art results. \Template \apr tools typically leverage pre-defined templates (e.g., adding a nullness check) for bug fixing. However, since these fix templates are typically handcrafted, the number and types of bugs they are able to fix can be limited. 



To address the limitations of traditional \apr, researchers have proposed various \learning \apr tools~\cite{li2020dlfix, chen2018sequencer, jiang2021cure, lutellier2020coconut, zhu2021recoder, ye2022rewardrepair} based on the \nmtfull (\nmt) architecture~\cite{sutskever2014mt} where the input is the buggy code snippets and the goal is to translate the buggy code snippets into a fixed version. To accomplish this, \learning \apr tools require supervised training datasets with pairs of both buggy and fixed code snippets in order to learn how to perform this translation step. These training data are usually obtained by mining historical bug fixes using heuristics/keywords~\cite{dallmeier2007benchmark}, which can be imprecise for identifying bug-fixing commits; even the actual bug-fixing commits can include irrelevant code changes, leading to further pollution in the dataset~\cite{xia2022alpharepair}.
% 
Moreover, it can be hard for such \apr tools to generalize and fix bug types unseen during training. 



To better leverage recent advances in \plmfull{s} (\plm{s}), researchers~\cite{xia2022alpharepair, xia2023repairstudy, kolak2022patch, prenner2021codexws} have directly applied \plm{s} to generate patches without bug-fixing datasets. These \llm-based \apr tools work by either directly generating a complete code function~\cite{prenner2021codexws, xia2023repairstudy} or predict/infill the correct code snippet given its surrounding context~\cite{xia2022alpharepair, xia2023repairstudy}. By directly using \llm{s} that are pre-trained on billions of open-source code snippets, \llm-based \apr tools can achieve state-of-the-art performance on many repair datasets~\cite{xia2022alpharepair}. 


% 
%
%

Traditional \apr tools have long used the insight of the \emph{plastic surgery hypothesis}~\cite{barr2014plastic} where it states that the code ingredients to fix a bug already exist within the same project. Traditional \apr tools have manually designed pattern-~\cite{ghanbari2019prapr, saha2017elixir} or heuristic-based~\cite{jiang2018simfix, legoues2012genprog} approaches to finding and using such relevant code ingredients to generate fixes for bugs. However, the plastic surgery hypothesis has been largely ignored in \llm-based \apr. In fact, \llm provides a unique opportunity to fully automate the plastic surgery hypothesis idea via fine-tuning (learning project-specific information via model updates from the buggy project) and prompting (directly providing relevant code ingredients to the model), and make it directly applicable to different languages (since the \llm{s} are typically multi-lingual).%
Moreover, despite the intensive manual efforts involved, traditional \apr tools still cannot fully leverage project-specific information due to large search space for leveraging/composing existing code ingredients. In contrast, the project-specific information can effectively leveraged by \llm{s} due to their power in code understanding/vectorization, e.g., even partial/imprecise information may still guide \llm{s} in correct patch generation!
 To this end, we ask the question: \emph{How useful is the plastic surgery hypothesis in the era of \plm{s}}?








\mypara{Our Work.} To answer the question, we present \ourtech{\xspace} -- a \llm-based approach that automatically utilizes the plastic surgery hypothesis by systematically combining multiple fine-tuning and prompting strategies for \apr. \ourtech fine-tunes \plm{s} using two novel domain-specific training strategies: \textbf{\epfinetune} -- we fine-tune using the original buggy project by aggressively masking out a high percentage of tokens, which allows \plm to learn project-specific code tokens and programming styles; and \textbf{\rofinetune} -- which only masks out a single continuous code sequence per training sample, allowing the model to get used to the final \csapr task of predicting a single continuous code sequence. Furthermore, we directly leverage the ability for \plm{s} to understand natural language instructions and introduce a novel prompting strategy, \textbf{\idprompting}, which uses information retrieval and static analysis to obtain a list of relevant identifiers for the buggy lines. While such relevant identifiers are critical for fixing some difficult bugs, they may not be seen by the \llm during inference due to limited context window size. Through the use of prompting, we directly tell the model to use these extracted identifiers (relevant code ingredients) to generate the correct code. Finally, to perform repair, we combine all four model variants (including the base model, both fine-tuned models and the base model with prompting) for the final repair.





While our insight of leveraging the plastic surgery hypothesis for \llm-based \apr is generalizable across different types of \plm{s}, to implement \ourtech, we choose a recent \plm{\xspace}, \ctfive~\cite{wang2021codet5}, which is pre-trained on millions of open-source code snippets. \ctfive is an encoder-decoder model trained using \mspfull (\msp) objective where a percentage of tokens are masked out and each continuous masked token sequence is referred to as a masked span. Also, although we only extract relevant identifiers from the current buggy project (since this paper focuses on the plastic surgery hypothesis), our work can be easily extended to obtain other code information (such as relevant statements or functions) from other sources, such as  the massive pre-training corpora~\cite{husain2020codesearchnet} or historical bug-fixing datasets~\cite{jiang2019infer}, which can provide more coding knowledge for \llm{s}. Besides, although we mainly focus on using traditional string comparison algorithms for information retrieval in this paper, these techniques can be easily replaced by other frequency-based retrieval~\cite{robertson2009probabilistic} and neural search (or embedding-based search)~\cite{reimers2019sentence}.
  In summary, this paper makes the following contributions:


%


\begin{itemize}[noitemsep, leftmargin=*, topsep=0pt]
    \item \textbf{Dimension.} This paper is the first to revisit the important plastic surgery hypothesis in the era of \llm{s}. It opens up a new dimension for \llm-based \apr to incorporate previously neglected information from the buggy project itself to boost \apr performance. Furthermore, it demonstrates the promising future of retrieval-based prompting for modern \llm-based \apr.
    \item \textbf{Implementation.} We implement \ourtech based on the recent \ctfive model. We augment the model using two novel fine-tuning strategies: \epfinetune and \rofinetune, along with a novel prompting strategy based on information retrieval and static analysis: \idprompting. We combine the patches generated by all four models together and perform patch ranking to speed up \apr.% 
    \item \textbf{Evaluation Study.} We conduct an extensive evaluation against state-of-the-art \apr tools. On the widely studied \dfj 1.2 and 2.0 datasets~\cite{just2014dfj}, \ourtech is able to achieve the new state-of-the-art results of 89 and 44 correct bug fixes (15 and 8 more than best baseline) respectively.  Furthermore, we perform a broad ablation study to justify our design. \ourtech demonstrates for the first time that the plastic surgery hypothesis can substantially boost \llm-based \apr and advance state-of-the-art \apr, while being fully automated and general. Moreover, even partial/imprecise code ingredients may still effectively guide \llm{s} for \apr!
\end{itemize}


\section{Proposed Framework: {\ourmodel}}
\label{model}


In this section, we introduce a novel self-supervised co-training framework {\ourmodel}.
The proposed framework is illustrated in Figure~\ref{fig:intro_model} and works in three phases.
Phase one automatically generates two sets of pseudo labels.
We use a combination of off-the-shelf pre-trained POS and NER taggers, knowledge graph, and GPT-2 scorer for generating the first set of pseudo labels automatically without any hand-crafted rules for matching the slot values.
The other set of pseudo labels is acquired through a zero-shot slot filling model~\cite{liu2020coach}, trained on the out-of-domain dataset.
It is critical to emphasize that both sets of labels are noisy and incomplete which poses serious challenges to training effective models for the task of open-domain slot filling.
Phase two fine-tunes the pre-trained BERT to the slot filling task that effectively transfers the knowledge from the pre-trained language model~(LM) to overcome the issue of label incompleteness to some extent. 
Further, we employ the early stopping technique to minimize the noise in the labels.
The output of this phase is two BERT models that can generate soft labels for self-supervision during co-training in phase three.
Phase three leverages the fine-tuned models and further trains them in an iterative fashion.
Specifically, the proposed peer training approach facilitates high-confidence soft label selection for the other peer to perform training. This phase progressively reduces the noise in the labels and enables effective model fitting. 



\subsection{Phase One: Automatic Label Generation}
To acquire the first set of labels, we perform the following steps.
First of all, off-the-shelf trained POS and NER taggers are used to predict initial estimates of the slot values irrespective of the slot types. Then, the type information of the slot values is queried from the KG and the slot value is tagged for the most appropriate slot in the target domain.
This approach, however, produces low recall. 
To expand the candidate slot values, we generate n-grams of the natural language text and employ a partial matching scheme to query the KG for type information (e.g., \myspecial{Jason} \myspecial{Aldean} = \myspecial{American} \myspecial{singer}) of the n-grams if the entry exists.
This process generates multiple overlapping hypotheses about the slot values.
We replace a span of text that corresponds to a slot value by its type information and a GPT-2 based scorer (see Section~\ref{sec:nlpmodels}) is used to select the best candidate based on the fluency of the text.
Naturally, if a token (or span of tokens) is replaced by its type, the sentence should score higher as compared to the case where an inappropriate substitution is performed. 
We select the best hypothesis if the score is greater than the threshold.
Intuitively, the candidate selection threshold can automatically be searched based on a small validation set from the target domain, making the label generation process fully automatic. 
The other set of noisy labels is acquired by the zero-shot slot filling model~\cite{liu2020coach} that has been trained using an out-of-domain dataset. It is important to highlight that the zero-shot slot filling model does not require any labeled in-domain training example. 
To summarize the automatic label generation phase, both sets of labels are acquired in a fully automatic fashion without any hand-crafting.


In contrast to previous work in weak supervision~\cite{ren2015clustype,he2017autoentity,fries2017swellshark,giannakopoulos2017unsupervised} that obtains a single set of noisy labels and then propose techniques to overcome the challenge of fitting an effective model to the noisy labels, we acquire two sets of complementary labels.
The choice of these two sets of labels is guided by the intuition that they should be complementary and the models trained on these sets of labels should be able to share complementary information with the other to improve the performance in the later phases of the framework.
Essentially, the first set of labels carries information from external knowledge sources, whereas the labels generated through the pre-trained zero-shot slot filling model capture how the slot values are mentioned in other domains.
%
To further elaborate on the motivation and our process for the first set of labels (i.e., labels using KG and other NLP models), the pre-trained LMs have been shown to have a great deal of knowledge~\cite{petroni2019language}, thus should be capable of generating automatic labels with no need of external KG. 
To the best of our knowledge, there exists no work that shows that accurate token-level automatic labeling (e.g., slot filling task) is possible with pre-trained LMs. 
Moreover, such approaches would require heavy prompting in each new target domain, whereas our label generation process is fully automatic and only relies on the readily-available pre-trained NLP models and external KG.

\subsection{Phase Two: LM-assisted Weak Supervision}
Since we do not have access to dataset $\{(\mathbf{X}_n,\mathbf{Y}_n)\}_{n=1}^N$ with true ground-truth labels.
We use pseudo labels generated in phase one, $\{(\mathbf{X}_n,\mathbf{D}_n)\}_{n=1}^N$, to learn 
$f_{m,c}(\cdot; \cdot)$ that outputs the probability of the $m$-th token to take on class $c$. 
We learn $f_{m,c}(\cdot; \cdot)$ by minimizing the following loss over the noisy dataset $\{(\mathbf{X}_n,\mathbf{D}_n)\}_{n=1}^N$: 
$$
\hat\theta = \argmin_{\theta}\frac{1}{N}\sum_{n=1}^{N} \ell(\mathbf{D}_n, f(\mathbf{X}_{n}; \theta)),
\label{eq:stage1}
$$
where $\ell(\mathbf{D}_n, f(\mathbf{X}_{n}; \theta)) = \frac{1}{M} \sum_{m=1}^{M} -\log{f_{m,d_{n, m}}(\mathbf{X}_{n}; \theta)}$. 
We employ the pre-trained multilingual BERT with token-level classification head that uses Adam optimizer \cite{kingma2014adam,Liu2019} with early stopping and multiple random initializations. 


Since slot filling task is similar to the MLM training objective of the BERT, we employ pre-trained BERT as the backbone model.
That is, MLM's goal is to predict the masked tokens using bidirectional contexts. Similarly, slot filling tries to predict the label for a token leveraging both left and right contexts simultaneously, which makes the pre-trained BERT an ideal model of choice that greatly facilitates minimizing incomplete labels.
It is important to highlight that our automatically generated labels are not only incomplete but also potentially wrong.
The training strategies employed in this phase minimize the noise in the label to some extent. 
Specifically, early stopping can provide a strong regularization and would not let the model overfit to the noisy labels, especially wrong labels. 
Moreover, early stopping does not let the model forget the knowledge in the pre-trained model.
Similarly, multiple random initializations enforce robustness. 
Since the model is fine-tuned on the noisy labels, averaging the predictions of multiple models for each token ensures that wrong labels end up with low probabilities and true labels consistently achieve high probabilities.
Using the above-mentioned strategies, we train two slot filling models, which we call the peers. The peer one is trained on the first set of pseudo labels that were generated using POS and NER taggers, KG, and the GPT-2 scorer in phase one. Similarly, peer two is trained using the predictions of the zero-shot slot filling model~\cite{liu2020coach}.
Both models have the same architecture and follow the same training procedures.

\begin{table*}[t!]
\centering
\caption{Dataset statistics.}
\vspace{-7pt}
\label{tab:dataset}
\begin{tabular}{lccccc}
\toprule
\textbf{Dataset}  & \textbf{Dataset Size} & \textbf{Vocab. Size} & \textbf{Avg. Length} & \textbf{\# of Domains} & \textbf{\# of Slots} \\ \hline
\textbf{SGD}      & 188K                  & 33.6K                & 13.8                 & 20                     & 240                  \\
\textbf{MultiWoZ} & 67.4K                 & 10.5K                & 13.3                 & 8                      & 61 \\
\bottomrule
\end{tabular}
\vspace{-7pt}
\end{table*}

\subsection{Phase Three: Self-supervised Co-training}
We introduce an iterative peer training algorithm where both peers generate high-confidence soft labels for training the other peer in the next iteration. 
Theoretically, these peers can be anything, but in this work, 
we explore two of the most promising directions that have shown the promise to minimize the need for manual labeling for the task: zero-shot learning and distant supervision.
This phase uses a self-supervised co-training scheme to exploit the patterns of slot values from other domains through the labels generated by the zero-shot filling model (i.e., peer two)~\cite{liu2020coach} as well as utilize the knowledge in external KGs and pre-trained models via labels provided by the peer one.
Specifically, we initialize the peers trained in phase two and use their pseudo labels to kick-start training in this phase.
Specifically, peer one $f_{m,c}(\cdot; \theta_{\textrm{p1}})$ would generate labels $\{\tilde{\mathbf{Y}}^{(t)}_n = [\tilde{y}_{n,1}^{(t)}, ..., \tilde{y}_{n,m}^{(t)}]\}_{n=1}^{N}$ for peer two $f_{m,c}(\cdot; \theta_{\textrm{p2}})$ at the $t$-th iteration by:
$$
\tilde{y}_{n,m}^{(t)} = \argmax_{c}{f_{m,c}(\mathbf{X}_n; \theta_{\textrm{p1}}^{(t)})}. 
\label{eq:pseudo}
$$

Based on these labels, the peer two can be fine-tuned by: 
$$
\hat\theta_{\textrm{p2}}^{(t+1)} = \argmin_{\theta}\frac{1}{N}\sum_{n=1}^N \ell(\tilde{\mathbf{Y}}_n^{(t)}, f(\mathbf{X}_{n}; \theta)).
\label{eq:self_train1}
$$

Similarly, peer two $f_{m,c}(\cdot; \theta_{\textrm{p2}})$ would generate pseudo labels for peer one $f_{m,c}(\cdot; \theta_{\textrm{p1}})$ that are used to fine-tune peer one. 
We also notice that it is beneficial to stop early during this phase as well, to improve the model fitting and gradually reduce the noise associated with the automatically generated labels.
Since pseudo labels are refined gradually in an iterative way, both peers can benefit from the knowledge contained within the labels of the other while avoiding overfitting.
Furthermore, as an alternative to pseudo labels, we also generate soft labels that are used for confidence re-weighting. 
The high-confidence soft label selection strategy enables better model fitting and efficient learning via better quality of the automatic labels.
Specifically, for the given $m$-th token in the $n$-th training example, the probability for all classes $C$ is $[f_{m,1}(\mathbf{X}_n;\theta),...,f_{m,C}(\mathbf{X}_n;\theta)]$. 
Following ~\cite{xie2016unsupervised}, at $t$-th iteration, peer one generates soft labels, $\{\mathbf{S}_n^{(t)} = [\mathbf{s}_{n,m}^{(t)}]_{m=1}^M \}_{n=1}^N$, as given below:
$$
\mathbf{s}_{n,m}^{(t)} = [s_{n,m,c}^{(t)}]_{c=1}^{C} = \Bigg[  \frac{f_{m,c}^2(\mathbf{X}_n;\theta_{\textrm{peer1}}^{(t)})/p_{c}}{\sum_{c'=1}^C f_{m,c'}^2(\mathbf{X}_n;\theta_{\textrm{peer1}}^{(t)})/p_{c'}}\Bigg]_{c=1}^{C}
\label{eq:soft}
$$ 
where $p_{c} = \sum_{n=1}^N \sum_{m=1}^M f_{m,c}(\mathbf{X}_n;\theta_{\textrm{p1}}^{(t)})$ computes the frequency of the tokens for the $c$-th class. 
Then, peer two $f(\cdot; \theta_{\textrm{p2}}^{(t+1)})$ is fine-tuned by:
$$
\theta_{\textrm{p2}}^{(t+1)} = \argmin_{\theta} \frac{1}{N} \sum_{n=1}^{N} \ell_{\rm KL}(\mathbf{S}_n^{(t)}, f(\mathbf{X}_{n}; \theta)),
$$
where $\ell_{\rm KL}(\cdot,\cdot)$ is the KL-divergence-based loss:
$$
\ell_{\rm KL}(\mathbf{S}_n^{(t)}, f(\mathbf{X}_{n}; \theta))=\frac{1}{M}\sum_{m=1}^M\sum_{c=1}^C - s_{n,m,c}^{(t)} \log f_{m,c}(\mathbf{X}_{n}; \theta).
\label{eq:klloss}
$$

Moreover, we also investigate selecting tokens that have high confidence. 
For instance, we pick high-confidence tokens from the $m$-th input example at the $t$-th iteration by  
$
H^{(t)}_n = \{m : \max_{c} s_{n,m,c}^{(t)} > \epsilon \},
$
where $\epsilon\in [0,1]$ is a threshold that can be searched based on a small validation set. 
Then, peer two $f(\cdot; \theta_{\textrm{p2}}^{(t+1)})$ is fine-tuned by:
$$
\theta_{\textrm{p2}}^{(t+1)} %&= \argmin_{\theta} \frac{1}{N} \sum_{n=1}^{N} \ell_{\rm S-KL}(\bS_n^{(t)}, f(\bX_{n}; \theta)) \\
= \argmin_{\theta} \frac{1}{N|H^{(t)}_n|}\sum_{n=1}^{N} \sum_{m\in H^{(t)}_n}\sum_{c=1}^C - s_{n,m,c}^{(t)} \log f_{m,c}(\mathbf{X}_{n}; \theta).
$$

This phase improves the robustness to effectively fit the model for tokens with high confidence. 
Both peers keep sharing information and their confidence by producing soft labels for their counterparts until they approximate to the true labels while employing early stopping and scheduled learning rates.
It is important to remind that phase three is the most important phase that progressively reduces noise from the labels to a great extent and enables superior performance for the task of open-domain slot filling.

% \normalem
\begin{algorithm}[t!]
\small{
\SetAlgoLined
\SetKwInOut{Input}{Input}
\SetKwInOut{Output}{Output}
\Indmm \Indmm
\Input{Backbone network $f$, projection head $g$, two prediction heads $h_1$, $h_2$, sample generator $q_\phi$ and $p_\theta$, and the training set $\mathcal{D}$}
\Output{Trained backbone network $f$}
\Indpp \Indpp
\ForEach{batch $\mathcal{B}^s \subset \mathcal{D}$}
{
    $L_{\text{fair-CL}}, L_{\text{self-KD}} \gets 0, 0 $ \\
    $\mathcal{B}^{s}_{\text{pert}} \gets \text{TabMix}(\mathcal{B}^s, \text{Sample}(\mathcal{D} \setminus \mathcal{B}^s))$  \tcp{(Eq.~\ref{eq:tabmix})}
    $\mathcal{B}^{s'}_{\text{cnt}} \gets \text{Convert}(\mathcal{B}^s, q_\phi, p_\theta)$ \tcp{(Sec. 3.2)}
    \ForEach{$\mathbf{x}^s \in \mathcal{B}_s$, $\mathbf{x}^s_{\text{pert}} \in \mathcal{B}^{s}_{\text{pert}}$, and $\mathbf{x}^{s'}_{\text{cnt}} \in \mathcal{B}^{s'}_{\text{cnt}}$}
    {   
        $\mathcal{X}^s_{-} = \{ \mathbf{x} | \mathbf{x} \in \mathcal{B}_s\setminus\{\mathbf{x}^s\} \}$ \\
         \tcp{Fairness-aware contrastive loss (Eq.~\ref{eq:contrast_loss})}
         $L_{\text{fair-CL}} \gets L_{\text{fair-CL}} + L_{\text{gen-c}} (\mathbf{x}^s, \mathbf{x}^{s'}_{\text{cnt}}, \mathcal{X}^s_{-})$ \\ \vspace{2mm}
         \tcp{Self-knowledge distillation loss (Eq.~\ref{eq:self-kd})}
         $\mathbf{p}_{\text{student}} \gets h_2 \circ g \circ f(\mathbf{x}^s_{\text{pert}})$ \\
         $\mathbf{z}_{\text{teacher}} \gets (g \circ f(\mathbf{x}^s)).\text{detach()}$ \\
         $L_{\text{self-KD}} \gets L_{\text{self-KD}} - ({\mathbf{p}_{\text{student}} \over || \mathbf{p}_{\text{student}} ||_2} \cdot {\mathbf{z}_{\text{teacher}}  \over || \mathbf{z}_{\text{teacher}} ||_2})$  \\
    }
    $L_{\text{total}} = {1 \over |\mathcal{B}_s|}(L_{\text{fair-CL}} +  L_{\text{self-KD}})$  \\
    Update weights via back-propagation
}
\caption{Overall training procedure (for one epoch)}}
\label{algo:algorithm}
\end{algorithm}
% \ULforem
\section{Theoretical Results}
% \subsection{Model Assumptions}
% \begin{assumption}[Regularity of the information structure and the scoring rule]\label{asp:reg}
%     We assume that,
% \begin{itemize}
%     \item[(i)] (convexity) $\sS$ is a star-shaped convex function class, meaning that if $S_1, S_2\in\sS$, we have $\alpha S_1 + (1-\alpha) S_2 \in\sS$ for any $0\le \alpha\le 1$ and $\lambda S\in\sS$ for any $\lambda\in[0, 1)$ if $S\in\sS$. Moreover, we assume that $S(\sigma, \omega)=u(a^*(\sigma), \omega)\in\sS$.
%     % \item[(ii)] (compactness) Let $v_S = \sbr{\EE S(\Sigma_i, \omega)}_{i\in[K]}$ be the vector of expected scoring under each signal and define $\cV=\cbr{v_S\given \forall S\in\sS}$. We assume that $\cV$ is a compact subset of $\RR^K$.
%     % \item[(iii)] (realizibility) Let $\cB_\epsilon(c)=\cbr{z\in\RR^K\given \nbr{z-c}_\infty\le \epsilon}$ denotes the $\epsilon$-ball centered at $c$ in $\RR^K$. We assume that $\cB_\epsilon(c)\in\Int\cV$, where $c$ is the cost vector. 
%     \item[(ii)] (decaying marginal information gain) Suppose that this $K$ signal are sorted in the increasing order of the cost. Define $u_k = \EE_{\omega\sim\sigma, \sigma\sim q_k}\sbr{u(a^*(\sigma), \omega)}$. We assume the marginal information gain is strictly decaying, i.e., there exists a $\epsilon>0$
%     \begin{gather*}
%         \frac{u_{K}-u_{K-1}}{c_{K}-c_{K-1}} >\epsilon, \quad \text{and}\quad
%         \log\frac{u_k-u_{k-1}}{c_k-c_{k-1}} - \log{\frac{u_{k+1}-u_k}{c_{k+1}-c_k}} >\log(1+\epsilon), \quad \forall k=1,\cdots, K-1.
%     \end{gather*}
%     Moreover, we assume that $c_k - c_{k-1} >\delta_c$ for $c=1, \cdots, K$.
% \end{itemize}
% \end{assumption}
% The convexity of $\sS$ enables us to deploy the \textit{linear scoring rule}.

% The decaying marginal information gain property indicates that all the signals are \textit{linearly-implementable}.
% \begin{lemma}[Linear implementable]
% Under Assumption \ref{asp:reg}, all the signals are linearly-implementable \citep{dutting2019simple}. 
% \end{lemma}
% Linear implementable means that using the linear scoring rule $S(\sigma, \omega)=\lambda u(a^*(\sigma), \omega)$ where $\lambda\in[0, 1/\epsilon]$, all the signals can be induced. 
% We remark that the compactness assumption is satisfied if and only if there exists $S_1, \cdots, S_K\in\sS$ such that $(v_{S_i})_{i\in[K]}$ are linear independent.
% Such an assumption characterizes the richness of the scoring rule class as well as the nondegeneration of the signal structure since it implies that the prior over the posteriors of each signal should also be linear independent.
% Such a property is essential for stimulating all the possible actions from the agent using the scoring rule. 
% For brevity, we define 
% \begin{align*}
%     S(\sigma) = \EE_{\omega\sim\sigma}\sbr{S(\sigma, \omega)}, \quad
%     v_S^{(i)} = \EE\sbr{S(\Sigma_i, \omega)}, \quad
%     u^{(i)} = \EE\sbr{u(a(\Sigma_i), \omega)},
% \end{align*}
% where $a^*(\sigma)=\argmax_{a\in\cA} \EE_{\omega\sim\sigma}[u(a, \omega)]$.
% In the sequel, we consider $\supp(\sigma)$ to be a finite set with cardinality $M$.
% The following results may be useful.

% We assume $u$ function is globally bounded, and the scoring rule class $\sS$ is also globally bounded.

% \subsection{Algorithm Outline}
% We propose a two-step online algorithm for solving the scoring rule learning problem. 
% \paragraph{Step \RNum{1}:}(Induce the $K$ signals) In the beginning, we deploy linear contract $S(\sigma, \omega)=\lambda u(a^*(\sigma), \omega)$ with a binary search on $\lambda\in [0, 1/\epsilon]$.
% The target is to find $\tilde S_0, \cdots, \tilde S_K$ such that when deploying $\tilde S_k$, the agent is guaranteed to respond with $\Sigma_k$.
% \begin{lemma}\label{lem:linear-binary-search}
% This binary search needs at most $2K\log_2(1/4\epsilon)$ interventions with the agent. Moreover, we can find $\tilde S_0, \cdots, \tilde S_K$ such that $(v_{\tilde S_k, k} - c_k) - (v_{\tilde S_k, i}-c_i) > \epsilon\delta_c/2(1+\epsilon)$ for any $i\neq k$.
% \end{lemma}
% We remark that if we have some prior knowledge that enables us to induce the $K$ signals, we can drop the decaying marginal information gain assumption and go directly to the next step.







% % However, the problem is that the agent might not respond with signal $\Sigma_{k_t^*}$ as we have expected. We next address the problem by adopting a interpolation scoring rule.

% % \subparagraph{Interacting in the $t$-th loop.} We deploy scoring rule 
% \begin{align*}
%     S_t = \alpha_t\tilde S_{k_t^*} + \rbr{1-\alpha_t} S_t^*
% \end{align*}
% in the $t$-th loop. The following lemma explains why we should use a interpolation of these two scoring rule. 

% The above lemma characterizes the probability that the agent fails to respond $\Sigma_{k_t^*}$ as we have expected. Since $S_t^*$ may be over-optimistic about the fact that the agent should respond with $\Sigma_{k_t^*}$, we move the scoring rule towards the original $\tilde S_{k_t^*}$ (under this scoring rule, the agent is known to respond with $\Sigma_{k_t^*}$) in order to ensure a high probability that the agent will respond with signal $\Sigma_{k_t^*}$.
% In the sequel, we let $\alpha_t$ to be a fixed decaying sequence, e.g., $\alpha_t=(K-1) t^{-1/3}\land 1$.



In this subsection, we provide the learning result for the OSRL-UCB algorithm.
% \subsection{Learning Result}
\begin{theorem}[Regret for OSRL-UCB]\label{thm:regret}
    Under Assumption \ref{asp:oracle} on the action-informed oracle, with $\alpha_t = \min\{K t^{-1/3}, 1\}$, the   OSRL-UCB algorithm \ref{alg:UCB} has regret
\begin{align*}
    \Reg(T) = \tilde\cO\bigl ( (B_S+B_u)B_S^2\varepsilon^{-2} \cdot K^2 C_\cO\cdot T^{2/3}\big) , 
\end{align*}
where $B_S$ and $B_u$ bound the magnitudes of the scoring rule and the utility function, respectively, $\varepsilon$ is the marginal profit gain given by the action informed oracle, $K$ is the number of the agent's actions, and $C_\cO$ is the cardinality of the observation set.
\vspace{-10pt}
\begin{proof}
We defer the detailed proof to \S\ref{prof: main theorem}. 
% At a high level, it hinges on the following regret decomposition,
% \begin{align*}
%  \Reg^\pi(T) & = \sum_{t=1}^T\underbrace{\EE\sbr{\ind(k_t=k_t^*)\subopt(k_t, S_t)}}_{\displaystyle{A_t}}\\ 
%  & + \sum_{t=1}^T\underbrace{\EE\sbr{\ind(k_t\neq k_t^*)\subopt(k_t, S_t)}}_{\displaystyle{B_t}}.    
% \end{align*}
\end{proof}
\end{theorem}

Here, we use $\tilde \cO$ to omit logarithmic factors.
The regret depends quadratically on the agent's action number and linearly on the cardinality of the observation set.
Notably, the regret is independent of the size of the hidden state $|\Omega|$. 
In addition, we achieve a $\cO(T^{2/3})$ sublinear rate of regret in terms of the principal's accumulative profit for eliciting information under the scoring rule framework. Such a result is achieved with a mild assumption of an action-informed oracle that provides a set of scoring rules with marginal profit gain for the agent that induce all the agent's actions. 
We do not assume the learner to have sufficient knowledge about the other strategic player(s) in contrast to many existing works \citep{balcan2015commitment,guo2022no,wu2022sequential}. 
In addition, we only assume the principal to have knowledge of her utility and can observe the agent's action choice. For discussion of the these two assumptions, we refer the readers to the footnote in \S\ref{sec:model}.

For the action-informed oracle with a set of foreknown scoring rules, these foreknown scoring rules do not need to be optimal in each section $\cV_k$.
They can even be obtained through random sampling from $\beta$-strongly proper scoring rules  (See \Cref{exp:random sampling}) for general setting, or discovered in a linear scoring rule class (See \Cref{exp:linear contract}) if the marginal information gain is strictly decaying.
We also give the following corollaries that characterize the regret combined with the effort to find an action-informed oracle.
\begin{corollary}[Regret with oracle in \Cref{exp:random sampling}]\label{cor:random sampling}
Let $\tilde\cV_k=\{S\in\cS_\beta\given g(k, S)\ge g(k', S)+\kappa, \forall k'\neq k, k'\in[K]\}$ where $\cS_\beta\in\cS$ is the class of $\beta$-strongly proper scoring rules, and suppose $\vol(\tilde\cV_k)\ge \eta \Vol(\cS_\beta)$ for $k\in[K]$.
Running the oracle acquisition process in Example \ref{exp:random sampling} for $T^\gamma$ rounds before deploying the OSRL-UCB algorithm for $T-T^\gamma$ rounds, the online regret is bounded by
% \vspace{-2mm}
\begin{align*}
    \Reg(T) &= \tilde\cO\big( (d_2 d_1^2 \beta)^{-2} \cdot KM\cdot T^{2/3}\big) 
    % \nend&\qquad
    +\cO(KT\exp(-T^\gamma\eta/M) + T^\gamma),
\end{align*}
where $d_1=\min_{i\neq j, \forall (i,j)\in[M]}\nbr{\sigma_i-\sigma_j}_\infty$, $d_2=\min_{k'\neq k, (k, k')\in[K]}\max_{i\in[M]}\sbr{q_k(i)-q_{k'}(i)}$, and $M$ is the cardinality of $\Sigma$.
\end{corollary}
% \vspace{-2mm}
And also we characterize the regret for the action-informed oracle obtained by linear scoring rule.
\begin{corollary}[Regret with oracle in \Cref{exp:linear contract}]\label{cor:linear contract}
Suppose the model assumption that the marginal information
gain is strictly decaying in \Cref{exp:linear contract} holds.
By running the oracle acquisition process in \Cref{exp:linear contract} for $\cO(K\log_2(\varepsilon^{-1}))$ rounds and the OSRL-UCB algorithm for the remaining rounds, the online regret is bounded by
\begin{align*}
    \Reg(T) &= \tilde\cO\big(\varepsilon^{-2} \cdot K^2 C_\cO\cdot T^{2/3}\big)+ \cO(K\log_2(\varepsilon^{-1})), 
\end{align*}
where $\varepsilon=\epsilon u_1/4b^2$, and $\epsilon, u_1, b$ are constants defined in \Cref{exp:linear contract}.
\end{corollary}

Corollaries \Cref{cor:random sampling} and \Cref{cor:linear contract} both provide regret bound without any using of oracle. Specifically, \Cref{cor:random sampling} considers a more general framework under the assumption of lower bounded action section volume while \Cref{cor:linear contract} assumes marginal information decay, which is commonly seen in real world practice. Specifically, \Cref{cor:random sampling} shows that by random sampling for $T^\gamma$ rounds where $0<\gamma<2/3$, it suffices for the principal to have $\tilde \cO(T^{2/3})$ regret. In addition, $\gamma$ can be significantly small since the second term diminishes exponentially on $T^\gamma$. 
In addition, \Cref{cor:linear contract} shows that running constant number of additional rounds in the oracle acquisition process does not deteriorate the regret bound.

Following the discussion in \citet{jin2018q}, our algorithm also has PAC guarantee as the following.
\begin{corollary}[PAC guarantee]
For every $\zeta>0$, the OSRL-UCB algorithm with action informed oracle finds a $\zeta$-optimal scoring rule using $\tilde\cO(\varepsilon^{-6} K^6C_\cO^3\zeta^{-3})$ samples.
\end{corollary}

%In general, we remark that the regret bound in \Cref{thm:regret} improves over the $\cO(T^{3/4})$ regret of \citet{camara2020mechanisms}, though their model for a stage game covers ours as special case. 
%Meanwhile, 
\vspace{-1mm}
\citet{zhu2022sample} provides an $\cO(T^{2/3})$ regret lower bound for the online learning problem towards the optimal contract. Despite that standard contract design is a special case of our model, their setting is different from ours in that the agent may have possibly infinitely many actions. So the regret lower bound of our problem remains an open question. To close this gap, we believe the key question remains to be answered is whether the decision boundary of $\cV_{k^*}$ can be determined efficiently. That is, even if the best action $k^*$ is known, the learner is still unable to solve the optimal scoring rule from $\OptLP_{k^*}$ without enough knowledge about $\cV_{k^*}$. For now, we are able to construct a class of instances where such boundary of $\cV_{k^*}$ can determine with binary search and thus avoid the costly learning of $q_k$ for every $k\in[K]$.  However, it still remains unclear if these efficient search techniques can possibly be generalized to arbitrary instances --- a definitive answer should close up the regret lower and upper bound of this problem. We leave this intriguing direction to  future work.
% --- Is it necessary to learn $q_k$ for every $k\in[K]$ to determine the $\cV_{k^*}$ and solve $\OptLP_{k^*}$?  
%  According to Lemma \ref{lem:mistake}, the total rounds it takes for the agent to align with the principal's expected response is $n_{k, t}=\cO(t^{2/3})$ for all $k\in[K]$. Therefore, the regret from agent misalignment is  $\cO(K T^{2/3})$. On the other hand, since we shrink $S_t^*$ to $\tilde S_{k_t^*}$ by $\alpha_t$, the regret from suboptimality adds up to $\cO(\sum_{t=1}^T\alpha_t) = \cO(T^{2/3})$. This suggests that our regret analysis of Algorithm \ref{alg:UCB} is tight. 
% Combining these two parts of regret gives the result.



% We will be keeping the current scoring rule $S$ and a target scoring rule $\hat S$ generated by Algorithm \ref{alg:Opt} during the update. The logic under Algorithm \ref{alg:pseudo} is that with the current knowledge of the signal structure, we try to find the target scoring rule $\hat S$ that makes the payment just compensate for the cost of each known signal. Now, we move the current scoring rule to the target scoring rule by means of binary search on the segment connecting $S$ and $\hat S$. If this goes on well, we are just done. Otherwise, there must be a signal that we haven't identified yet and the agent will switch to that signal at some step of the binary search. Thus, we can expand our observed signal set and do this all again. 

% The guarantee is that if the agent actions just the same as what we have expected under $\hat S$ (choosing a signal from the current observed signal set), such a target scoring rule $\hat S$ provides a upper bound of the principal's best earnings. Since we assume that there are at most $K$ available signals, the expansion of the observed signal set will occur only for a finite number of times.
% So finally the optimal is reached. 

% \todo{Current algorithm is simplified by assuming that the estimation of the belief in Algorithm \ref{alg:game} is exact and ignoring a $2^{-N}$ error of $\lambda$ in Algorithm \ref{alg:BS}.}

% \SetKwComment{Comment}{/* }{ */}
% \begin{algorithm}
% \caption{(Play the game) $\hat q = \mathrm{Game}(S, T)$\;
% Description: Input the scoring rule and the total number of rounds. The algorithm plays the game for $T$ rounds and collects the report $\sigma^t$ from the agent and output the estimated prior of $\sigma^t$, which we assume to be within some belief class $\sQ\subseteq\Delta(\Delta(\Omega)).$ 
% }\label{alg:game}
% \KwData{$S, T, \sQ$\Comment*[r]{$\sQ$ is a given model class}}
% \KwResult{$\hat q$}
% Play with scoring rule $S$ for $T$ rounds and collect $[\sigma^t]_{t\in[T]}$\;
% % $\hat u=\sum_{t\in[T]}\EE_{\omega\sim\sigma^t }\sbr{u(a(\sigma^t), \omega)}/T$\;
% % $\hat v = \sum_{t\in[T]}\EE_{\omega\sim\sigma^t }\sbr{S(\sigma^t, \omega)}/T$\;
% $\hat q = \min_{q\in \sQ} \kl\rbr{\sum_{t\in T}\ind(\sigma = \sigma^t)/T\,\|\, q}$\;
% \end{algorithm}

% \begin{algorithm}
% \caption{(Binary Search) $(\lambda, q_\lambda) = \mathrm{BS}(S_0, S_1, N, T, \delta_q, Q)$\;
% Description: Input initial scoring rule $S_0$, target scoring rule $S_1$, the maximal times of binary searches $N$, the total round number $T$ played during each search, a threshold $\delta_q$ for categorification within the available belief set $Q$.
% The algorithm does binary search along the path connecting two scoring rules and tries to identify the largest $\lambda$ where the agent first switches its policy. If the agent doesn't switch its policy, then the algorithm will not terminate (meaning that the algorithm has found the optimal scoring rule. )}\label{alg:BS}
% \KwData{$S_0, S_1, N, T, \delta_q, Q$}
% \KwResult{$\lambda, q_\lambda$}
% $\lambda_{\min}\leftarrow 0,\quad \lambda_{\max}\leftarrow 1$\;
% $n\leftarrow 0$\;
% \While{$n<N$}{
% $\lambda = \rbr{\lambda_{\min}+\lambda_{\max}}/2$\;
% $S=\lambda S_0 + (1-\lambda) S_1$\;
% $q_\lambda\leftarrow \mathrm{Game}(S, T)$\;
% \eIf{$\min_{q\in Q}\kl\rbr{q_\lambda \,\|\, q}>\delta_q$}{
%     $\lambda_{\min} \leftarrow \lambda$\Comment*[r]{Unknown behavior}
% }
% {
%     $\lambda_{\max}\leftarrow \lambda$\Comment*[r]{Known behavior}
% }
% $n\leftarrow n+1$\;
% }
% \end{algorithm}

% \begin{algorithm}
% \caption{(Scoring Rule Optimization) $\hat S = \mathrm{Opt}(C, Q, \beta_c)$\;
% Description: Input the relative cost set $C$, the belief set $Q$, and also a hyper-parameter $\beta_c$. 
% The algorithm tries to minimize the principal's payment by designing a scoring rule that minimizes $b_S^{(1)}$ (since we can show that the first signal found by the algorithm always has the largest gain, i.e., $g^{(1)}=u^{(1)}-c^{(1)}$ is the largest. Therefore, when the payment equals the cost for all known signals, the first signal is favourable by the principal). 
% % Moreover, the objective is subject to a penalty for $v_S$ deviating from the relative cost $C$, which means we tempts to make the scoring rule aligned with the cost. 
% }\label{alg:Opt}
% \KwData{$C, Q, \beta_c, \sS$}
% \KwResult{$\hat S$}
% \vspace{-20pt}
% \begin{align*}
%     \hat S = \argmin_{S\in\sS} \cbr{v_S^{(1)} + \beta_c\min_{\eta\in\RR}\nbr{v_S-C-\eta \ind}_\infty},\qquad \text{where}\quad
%     v_S \leftarrow \sbr{\EE_{\sigma\sim q}S(\sigma) }_{q\in Q};
% \end{align*}
% \end{algorithm}

% \begin{algorithm}
% \caption{Pseudo Algorithm for Learning the $\epsilon$-Optimal Scoring Rule
% }\label{alg:pseudo}
% \KwData{$u, M, T, \beta_c, \delta_q, \sS$}
% \KwResult{$\hat S$}
% $C\leftarrow \cbr{0}, \quad Q\leftarrow \emptyset$\Comment*[r]{Initialize the relative cost set}
% $S=u(a(\sigma), \omega)$\;
% $(u, v, q)\leftarrow\mathrm{Game}(S, T)$\Comment*[r]{Run the game with scoring $u$}
% $Q\leftarrow \cbr{q}$\Comment*[r]{Initialize the signal belief set}
% $m\leftarrow 0$\;
% \While{$m < M$}{
% $\hat S \leftarrow \mathrm{Opt}(C, Q, \beta_c)$\Comment*[r]{Optimize $\hat S$ under current $C, Q$}
% $(\lambda, q) = \mathrm{BS}(S, \hat S, N, T, \delta_q, Q)$\Comment*[r]{Binary search for the first switch}
% $S\leftarrow \lambda S + (1-\lambda)\hat S$\;
% $Q\leftarrow Q\cup\cbr{q}$\Comment*[r]{Add a new signal's belief}
% $C\leftarrow C\cup\cbr{\EE_{\sigma\sim q}\sbr{ S(\sigma)} - \frac{1}{\abr{Q}}\sum_{q'\in Q}\EE_{\sigma\sim q'}\sbr{S(\sigma)} + \frac{1}{\abr{C}}\sum_{c\in C} c}$\;
% $m\leftarrow m+1$\;
% }
% \end{algorithm}


% \section{Future Work}
% In this section, we would like to point out several directions for the future development of this class project. 

% \subsection{Finer-grained Analysis of Regret }
% Despite an improvement over prior work \cite{camara2020mechanisms}, the current $O(T^{2/3})$ regret bound is still worse than $O(T^{1/2})$ regret bound in the standard bandit learning problems. We conjecture that there exists a large class of lower bound instances that require $\Omega(T^{2/3})$ regret. That is, in order to have $o(T^{2/3})$ regret, the suboptimal signals shall be induced with $o(T^{2/3})$ times. This however means that the suboptimal signals are estimated with error at least $\omega(T^{-1/3})$. Then, it is difficult to construct a $T^{-1/2}$-optimal scoring rule to induce the optimal signal after the exploration phase. 
% Meanwhile, we believe there are certain interesting special cases where $O(T^{1/2})$ regret bound is attainable. For example, when the signal structures are known, but only the utility function is unknown. 


% \subsection{Extension to Markovian Environment}
% We would also like to study the information acquisition problem under the Markovian environment, where both the agent and the principal may have long term objectives. For example, the agent may have incentives to share less information at the beginning to gain more manipulating power to extract more surplus in later steps of the process.  
\vspace{-2mm}
\section{Conclusion}
\vspace{-1mm}
We study the problem of incentivizing information acquisition through proper scoring rules under the principal-agent framework with information asymmetry. We propose the OSRL-UCB algorithm and show that with a mild oracle assumption, it  achieves  a $\cO(K^2 C_\cO T^{2/3})$ sublinear regret. Future direction includes establishing regret lower bound  and extensions to the contextual and dynamic settings.

\newpage 
\bibliographystyle{icml2023}

\bibliography{ref.bib}

\newpage
\appendix
\onecolumn
\section{Related Work}
\label{sec:relatedwork}

%%%%%%%%%%%%%%%%%%%%%%%%%% Outline %%%%%%%%%%%%%%%%%%%%%%%%%%%%%%%%%%%%%
%(1) Evasion Attacks
%(1.1) Surveys on evasion attacks and their relation to data properties - Michael
%(1.2) Individual papers that study non-data related reasons behind evasion attacks - Michael
%(1.3) Techniques related to evasion attacks and defenses (new) - Gabby
%(2) Non-Evasion Attacks (new), and - ???
%(3) Effects of training data on standard generalization - done 
%
%
%
%(1) Evasion Attacks
%(1.1) A number of surveys review literature on evasion attacks. - Michael
%Most of them do not focus specifically on properties of data but also discuss attack and defense mechanisms, non-data-related reasons for adversarial vulnarability, and  more. ~\jr{cite 4}.
%Yet, they these surveys mention data and its relation to evasion attacks. Specifically \jr{what they say about data.}
%The most close to ours is concurrent work by XXX + concrete facts that we have and they don't.
%
%(1.2) individual papers that study non-data related reasons behind evasion attacks, - Michael
%Literature identifies multiple reasons for adversarial vulnerability, in particular, for evasion attacks. 
%These include data-related properties extensively discussed in this survey, as well as reasons related to the models 		   themselves, computations resources, and feature representations. We discuss these below. 
%
%\jr{the rest is from the paper (non-data related reasons for adversarial vulnerability), with sections potentially renamed.}
%
%{\bf Model.}
%
%{\bf Computational Resources.}
%
%{\bf Robustness of Features.}
%
%(1.3) Techniques Related to Evasion Attacks and Defenses (new) - Gabby
%A number of works focus on techniques for generating evasion attacks, countermeasures against these attacks, 
%and defining the notion of the attack itself.   
%
%{\bf Attacks and Defense.}
%Here are the 5 remaining surveys + 1 additional paper for the reviewer.
%
%{\bf Adversarial Examples.}
%2 surveys lines 13 and 14 + 1 additional paper for the reviewer.
%
%(2) Non-Evasion Attacks (new) 
%Need to say that there are other type of attacks, define them, cite surveys (Bo's survey, maybe something else). 
%Only one work explicitly focus on effects of data. 
%
%
%(3) Effects of training data on standard generalization (done)

%%%%%%%%%%%%%%%%%%%%%%%%% Outline %%%%%%%%%%%%%%%%%%%%%%%%%%%%%%%%%%%%%


\revreplace{
We divide related work into three categories:
(1) surveys on adversarial robustness and its relation to data properties,
(2) surveys that discuss the influence of data properties on standard generalization, and
(3) individual papers that study non-data-related reasons for adversarial vulnerability.\\
}
{
This survey investigates properties of training data in the context of model robustness under evasion attacks. 
We start the discussion of related work by reviewing other surveys that focus on evasion attacks and 
include some discussion about data (Section~\ref{sec:relatedwork-surveys-data}).  
We then discuss non-data related reasons behind evasion attacks (Section~\ref{sec:relatedwork-not-data}),
as well as techniques related to evasion attacks and defenses (Section~\ref{sec:relatedwork-attacks}). 
Finally, we discuss data-related concerns for non-evasion attacks (Section~\ref{sec:relatedwork-poisoning}) and
the effects of training data on standard generalization (Section~\ref{sec:relatedwork-standard}).
}

%\vspace{-0.1in}
\subsection{Surveys on Evasion Attacks that Discuss Data}
\label{sec:relatedwork-surveys-data}
Numerous existing surveys 
\revreplace{focus on attack and defense techniques for adversarial robustness. 
%~\cite{Biggio:Roli:PR:2018,
%Rosenberg:Shabtai:Elovici:Rokach:CSUR:2021,
%Li:Li:Ye:Xu:CSUR:2021,
%Maiorca:Biggio:Giorgio:CSUR:2019,
%Demetrio:Coull:Biggio:Lagorio:Armando:Roli:ACMTPS:2021,
%Liu:Tantithamthavorn:Li:Liu:CSUR:2022,
%Liu:Nogueria:Fernandes:Kantarci:IEEECST:2022,
%Akhtar:Mian:IEEEAccess:2018,
%Akhtar:Mian:Kardan:Shah:IEEEAccess:2021,
%Serban:Poll:Visser:CSUR:2020,
%Machado:Silva:Goldschmidt:CSUR:2021,
%Zhang:Sheng:Alhazmi:Li:ACMTIST:2020}.
Only a few of these works mention the relationship between adversarial robustness and properties of the underlying data.} 
{review the literature on evasion attacks.
Most of these works do not focus specifically on properties of data but discuss attack and defense mechanisms, non-data-related reasons for adversarial vulnerability, 
and the different threat models. 
Only a few of these works mention data-related reasons for the existence of adversarial examples~\cite{Serban:Poll:Visser:CSUR:2020, Machado:Silva:Goldschmidt:CSUR:2021, Akhtar:Mian:Kardan:Shah:IEEEAccess:2021, Akhtar:Mian:IEEEAccess:2018}.
}
Specifically, Serban et al.~\cite{Serban:Poll:Visser:CSUR:2020} observe that adversarial vulnerability can be caused by an insufficient training sample size %~\cite{Schmidt:Santurkar:Tsipras:Talwar:Madry:NeurIPS:2018}
and high data dimensionality. %~\cite{Gilmer:Metz:Faghri:Schoenholz:Raghu:Wattenberg:Goodfellow:ICLR:2018}.
Similarly, Machado et al.~\cite{Machado:Silva:Goldschmidt:CSUR:2021} mention that the lack of sufficient training data, high dimensionality, 
and high concentration contribute to adversarial vulnerability.
\revadd{
Akhtar et al.~\cite{Akhtar:Mian:IEEEAccess:2018, Akhtar:Mian:Kardan:Shah:IEEEAccess:2021} also mention high dimensionality, along with other non-data-related reasons, 
as a source of adversarial examples.}

\revadd{A concurrent work by Han et al.~\cite{Han:Lin:Shen:Wang:Guan:CSUR:2023} (published at the end of April 2023) 
studies the origins of adversarial vulnerability in deep learning w.r.t. the model, data, and other perspectives.
The authors mention high dimensionality, distributions with high concentration, a small number of output classes, data imbalance, and the perceptual difference in image frequencies as potential sources of adversarial examples.
However, as (a) the focus of that survey is not on data-related properties in particular, 
(b) its paper search was conducted in 2021, and 
(c) it focuses on deep learning models only, 
our work was able to identify more than 50 additional relevant papers which focus on other types of models, 
e.g., non-parametric and linear classifiers, 
and/or discuss additional types of data-related properties, 
such as, types of distribution, class density, separation, and label quality.}
\revreplace{Yet, none of these surveys explicitly collect and analyze work that focuses on the effects of data properties
on adversarial robustness.}
{In summary, by explicitly focusing on the effects of data properties on evasion attacks in our survey, 
we are able to provide a more complete and detailed discussion on this topic, not covered in prior surveys.}

\vspace{-0.05in}
\subsection{Non-data-related Reasons Behind Evasion Attacks}
\label{sec:relatedwork-not-data}

%\vspace{-0.1in}
%\subsection{Non-data Related Reasons for Adversarial Vulnerability}

There has been a variety of hypotheses regarding the reasons behind adversarial vulnerability of ML systems, particularly for evasion attacks.
%\revreplace{
%In addition to the data used for training,  adversarial robustness could also depend on the choice of the model architecture,
%the training procedure, and the interplay between data and the learning algorithm, i.e., correspondence between the complexity of a model to that of the data.
%This section summarizes the key hypotheses regarding these aspects.
%%The hypotheses reviewed in this section are complementary to the potential influence from the data.
%}
These include data-related properties extensively discussed in this survey, as well as reasons related to the models themselves, 
computational resources, and feature learning procedures. We discuss these below.

%\jr{there is a lot of undefined terminology and jargon in this section.}

\vspace{0.02in}
\noindent
\textbf{Model.}
When Szegedy et al.~\cite{Szegedy:Zaremba:Sutskever:Bruna:Erhan:Goodfellow:Fergus:ICLR:2014} first discovered adversarial examples for visual models, they suspected that the high non-linearity of DNNs resulted in low probability `pockets' of adversarial examples in the learned representation manifold.
They hypothesize that while these pockets can be found through attack algorithms, the samples residing in these pockets have different distributions compared to normal samples and are thus subsequently harder to find when randomly sampling from the input space.
Instead, Goodfellow et al.~\cite{Goodfellow:Shlens:Szegedy:ICLR:2015} hypothesize that
the linearity from activation functions, like ReLU and sigmoid found in high-dimensional neural networks, induce vulnerability towards adversarial perturbations.
To support their claim, they present the attack method FGSM that exploits the linearity of the target classifier.
Fawzi et al.~\cite{Fawzi:Fawzi:Frossard:ICMLWorkshop:2015} also argue against the hypothesis of high non-linearity as the cause for adversarial examples.
They show that all classifiers are susceptible to adversarial attacks and claim that it is the low flexibility of the classifier compared to the complexity of the classification task that results in vulnerability.
The lack of consensus on the primary causes of model vulnerability invites more studies on this topic.

Singla et al.~\cite{Singla:Ge:Basri:Jacobs:NeurIPS:2021} show that enforcing invariance to circular shifts (e.g., rotation) in neural networks induces decision boundaries with a smaller margin than normal, fully connected networks,
which, in turn, reduces the adversarial robustness of the model.
Moosavi{-}Dezfooli et al.~\cite{Moosavi-Dezfooli:Fawzi:Fawzi:Frossard:Soatto:ICLR:2018} introduce universal,
input-agnostic perturbations to mislead the classifier and hypothesize that the vulnerability of a multi-class classifier to such perturbations is related to the shape of its decision boundaries, e.g.,
linear classifiers with decision boundaries that are parallel to each other and
nonlinear classifier with decision boundaries that are curved in a similar way
tend to be less robust as
perturbations in one direction can change the prediction label for a different class.

Tanay and Griffin~\cite{Tanay:Griffin:ArXiv:2016} conjecture that the decision boundary learned by the classifier being too close to (or `tilted towards') the data manifold instead of being perpendicular to it,
results in small perturbations being sufficient to move samples across the decision boundary for misclassification.
%data manifold refers to the underlying structure that the data exhibit

\vspace{0.02in}
\noindent
\textbf{Computational Resources.}
Bubeck et al.~\cite{Bubeck:Lee:Price:Razenshteyn:ICML:2019} use computational hardness theory to show that the time complexity for learning a robust model is exponential to the size of input data and thus is computationally intractable.
Hence, they attribute adversarial vulnerability to computational limitations of current learning algorithms.
Degwekar et al.~\cite{Degwekar:Nakkiran:Vaikuntanathan:COLT:2019} further extend this work and also show the impossibility of efficiently training robust classifiers.

%\subsubsection{Ineffective Learning Perspective}
\vspace{0.02in}
\noindent
\textbf{Feature Learning.}
Ilyas et al.~\cite{Ilyas:Santurkar:Tsipras:Engstrom:Tran:Madry:NeurIPS:2019} show that adversarial vulnerability can be a consequence of a model exploiting well-generalizing but non-robust features,
i.e., features that are spurious and sometimes incomprehensible to humans;
when constraining the model to use robust features, the adversarial robustness increases together with the
interpretability of the learned features.
However, Tsipras et al.~\cite{Tsipras:Santurkar:Engstrom:Turner:Madry:ICLR:2019} note that, as the features for achieving high accuracy may be different from the ones for achieving high robustness, robustness may be at odds with standard accuracy.
%
%\jr{why is it called Ineffective learning when it is about features.}\gx{I put it under ineffective learning as in this case, the model learns/decides the features for generalization, and when given the correct objective, the model in fact, can learn more robust features, so I think the underlying reason is objective we gave for the model didn't guide the model to learn the right features}
%
Instead of seeing adversarial vulnerability as a product of classifiers being overly sensitive to changes in spurious features, Jacobsen et al.~\cite{Jacobsen:Behrmann:Zemel:Bethge:ICLR:2019} hypothesize that classifiers can rather be
overly insensitive to relevant semantic information, e.g., images with drastically different content can share similar latent representations.
The authors introduce a new type of adversarial examples that exploit such insensitivity, where the content of images is altered without changing the resulting prediction label.
%As both insensitivity to semantic content and sensitivity to spurious changes can simultaneously exist in models,
%more investigation into how to define proper objectives for models to effectively distinguish the relevant information is needed.

While all these works propose possible reasons for adversarial vulnerabilities, they are orthogonal to our survey, which focuses particularly on the influence of training data.

\vspace{-0.05in}
\revadd{
\subsection{Evasion Attacks and Defenses}
\label{sec:relatedwork-attacks}
A number of works focus on techniques for generating evasion attacks, countermeasures against these attacks, 
and defining the notion of the attack itself.

%\jr{need to include~\cite{Biggio:Roli:PR:2018,
%Rosenberg:Shabtai:Elovici:Rokach:CSUR:2021,
%Li:Li:Ye:Xu:CSUR:2021,
%Maiorca:Biggio:Giorgio:CSUR:2019,
%Demetrio:Coull:Biggio:Lagorio:Armando:Roli:ACMTPS:2021,
%Liu:Tantithamthavorn:Li:Liu:CSUR:2022,
%Liu:Nogueria:Fernandes:Kantarci:IEEECST:2022,
%Zhang:Sheng:Alhazmi:Li:ACMTIST:2020} x and one more survey.}
%\js{\cite{Biggio:Roli:PR:2018, Rosenberg:Shabtai:Elovici:Rokach:CSUR:2021} moved to Adversarial Examples.
%\cite{Rosenberg:Shabtai:Elovici:Rokach:CSUR:2021,
%Li:Li:Ye:Xu:CSUR:2021,
%Maiorca:Biggio:Giorgio:CSUR:2019, Liu:Tantithamthavorn:Li:Liu:CSUR:2022,
%Liu:Nogueria:Fernandes:Kantarci:IEEECST:2022,
%Zhang:Sheng:Alhazmi:Li:ACMTIST:2020, Demetrio:Coull:Biggio:Lagorio:Armando:Roli:ACMTPS:2021} in Attacks and Defense. \cite{Sun:Dou:Yang:Zhang:Wang:Philip:He:Li:TKDE:2022} was the "one more survey" and is also in Attacks and Defenses.}

\vspace{0.02in}
\noindent
{\bf Attacks and Defense.}
Several works~\cite{Liu:Tantithamthavorn:Li:Liu:CSUR:2022,Liu:Nogueria:Fernandes:Kantarci:IEEECST:2022,Sun:Dou:Yang:Zhang:Wang:Philip:He:Li:TKDE:2022, Demetrio:Coull:Biggio:Lagorio:Armando:Roli:ACMTPS:2021} survey adversarial attacks and defenses, observing that most work focuses on computer vision and NLP domains. 
Zhang et al.~\cite{Zhang:Sheng:Alhazmi:Li:ACMTIST:2020}, 
Rosenberg et al.~\cite{Rosenberg:Shabtai:Elovici:Rokach:CSUR:2021},
Li et al.~\cite{Li:Li:Ye:Xu:CSUR:2021}, and 
Maiorca et al.~\cite{Maiorca:Biggio:Giorgio:CSUR:2019}, 
survey attacks and defenses in the NLP domain, cybersecurity domain for networks, Android malware, and PDF malware, respectively. 
These works identify a similar trend of new attacks constantly bypassing defenses, which gives rise to new defenses being proposed, only to be broken again (a.k.a. the `cat and mouse race' or the `arms race'). 
They also observe that research in this field studies attacks / defenses at a feature-level, which restricts 
the practicality of the developed techniques by the feasibility of perturbing the corresponding features in real life. 

%practical attacks are quite difficult and require some basic knowledge about the model or training data such as the feature set or model architecture. 
%Zhang et al.~\cite{Zhang:Sheng:Alhazmi:Li:ACMTIST:2020}, who study adversarial attacks and defenses in the NLP domain,  
%also find that there are obstacles to generating attacks in real-time. 
%For instance, methods that iteratively use gradients to create adversarial examples can be time-consuming, while one-time approaches may fail to produce potent adversarial examples.
%Several works~\cite{Liu:Tantithamthavorn:Li:Liu:CSUR:2022,Liu:Nogueria:Fernandes:Kantarci:IEEECST:2022,Sun:Dou:Yang:Zhang:Wang:Philip:He:Li:TKDE:2022, Demetrio:Coull:Biggio:Lagorio:Armando:Roli:ACMTPS:2021} 
%discuss how most new attacks and defenses are explored in computer vision and NLP, prior to other fields.


%our survey finds the state of the art w.r.t. data properties
%our survey finds that dimensionality is bad ...
%
%%%Here are the 5 remaining surveys + 1 additional paper for the reviewer.
%Numerous surveys have explored the landscape of adversarial evasion attacks and defenses. 
%For instance, Akhtar et al.~\cite{Akhtar:Mian:IEEEAccess:2018, Akhtar:Mian:Kardan:Shah:IEEEAccess:2021} survey the literature on adversarial robustness of deep learning models from Computer Vision field.
%They review popular attacks on visual models, and provided a categorization of existing defense techniques based on the components it modify in the visual model system \gx{Check}.
%
%Rosenberg et al.~\cite{Rosenberg:Shabtai:Elovici:Rokach:ACMComputingSurvey:2021}, Li et al. ~\cite{Li:Li:Ye:Xu:ACMComputingSurvey:2021} and Demetrio et al.~\cite{Demetrio:Coull:Biggio:Lagorio:Armando:Roli:ACMTPS:2021} review the literature on evasion attacks for cyber-security fields. 
%Li et al. proposed a partial order scheme to compare key attacks and defenses techniques for malware detection in Windows, Android, and PDF domains. 
%
%Zhang et al.~\cite{Zhang:Sheng:Alhazmi:Li:ACMTIST:2020} review the literature on adversarial attacks on deep-learning models for textual classification.
%They pointed out the intrinsic differences between Computer Vision and Natural Language Processing fields that pose challenges to directly apply attacks proposed for Visual models to NLP models and identified the strategies proposed that overcomes the barriers.
%The challenges they identified for creating realistic attacks in NLP fields are from a domain characteristics perspective (e.g., definition of imperceptible perturbations, measurement of the semantic changes),  we differ from them by trying to understand the adversarial robustness of machine learning from the characteristics of underlying data. 
%
%Attack and Defenses for wireless and Mobile systems~\cite{Liu:Nogueria:Fernandes:Kantarci:IEEECST:2022}
%
%

More recent research, not included in the surveys above, has also started investigating the 
susceptibility of newer models to adversarial evasion attacks. 
For example, several studies~\cite{Wang:Pan:Hu:Duan:Pan:IJSWIS:2022,Yin:Lin:Sun:Wei:Chen:TIFS:2023, 
Shi:Han:Tan:Kuang:NeurIPS:2022, Wang:Xie:Microsoft:ChatGPT:ArXiv:2023} proposed attack techniques against contemporary models, 
such as Graph Neural Networks, Generative Pre-training Transformers (GPT), and Vision Transformers. 
These studies showed that adversarial examples persist even for the newer models, some of which are 
trained with large volumes of data. 
As all these works focus on attack and defense mechanisms rather than 
the effects of data on adversarial robustness, our work extends and complements this research.
}

\revadd{
\vspace{0.02in}
\noindent
{\bf Adversarial Examples.}
%2 surveys lines 13 and 14 + 1 additional paper for the reviewer.
Adversarial examples are inputs constructed by perturbing a correctly classified sample in a way that makes the change imperceptible to a human. % but causes the model to misclassify the sample.
However, as `imperceptible to a human' is hard to define, existing research on adversarial examples approximates imperceptibility with a small perturbation measured through $L_p$ norms.
A line of research~\cite{Gilmer:Adams:Goodfellow:Anderson:Dahl:ArXiv:2018,Sharif:Bauer:Reiter:CVPRW:2018,Fezza:Bakhti:Hamidouche:Deforges:QoMEX:2019, Mezher:Deng:Karam:EUVIP:2022} 
investigates the validity of this assumption. 
This work shows that perturbations generated by $L_p$ norms do not entirely align with human perceptions, 
i.e., some changes with a small $L_p$ norm can be apparent to humans. 
In addition, adversarial examples with the minimum $L_p$ perturbation may be less effective and transferable than 
higher perturbation~\cite{Biggio:Roli:PR:2018,Rosenberg:Shabtai:Elovici:Rokach:CSUR:2021}. 
Hence, a number of approaches explore metrics for imperceptibility 
in computer vision and NLP domains~\cite{Fezza:Bakhti:Hamidouche:Deforges:QoMEX:2019,Mezher:Deng:Karam:EUVIP:2022, Zhang:Sheng:Alhazmi:Li:ACMTIST:2020}. 
Yet another issue with $L_p$ norms is that they cannot be used reliably in domains other than images. 
For example, in the case of software/malware, simply generating adversarial examples with $L_p$ norms 
may result in feature representations that are not possible in 
the problem space~\cite{Rosenberg:Shabtai:Elovici:Rokach:CSUR:2021,Pierazzi:Pendlebury:Cortellazz:Cavallaro:2020}. 

While all these works focus on the properties of adversarial examples, 
they are orthogonal to the topic of our survey, as we rather focus on how properties of the training data 
affect the success of adversarial examples.
}

%Gilmer et al.~\cite{Gilmer:Adams:Goodfellow:Anderson:Dahl:ArXiv:2018} argue that, while constraining the perturbations by sufficiently small $L_p$ norms can generate indistinguishable samples for most inputs, the actual imperceptibility of the changes depends on the input sample. 
%Several individual studies~\cite{Sharif:Bauer:Reiter:CVPRW:2018,Fezza:Bakhti:Hamidouche:Deforges:QoMEX:2019, Mezher:Deng:Karam:EUVIP:2022} find faults with using $L_p$ norms to generate adversarial examples. They show that the changes measured by $L_p$ norm, does not entirely align with human perceptions, i.e., some changes with a small $L_p$ norm appear apparent to humans. 
%In some domains adversarial examples do not need to be imperceptible but rather semantically preserving. 
%For example, in the case of Android malware~\cite{Rosenberg:Shabtai:Elovici:Rokach:CSUR:2021}, adversarial examples are small perturbations which fool a model while preserving the semantics of the sample, 
%i.e., a malware stays malicious even after the perturbation. 
%This highlights another problem with $L_p$ norm based adversarial examples as Dong et al.~\cite{Dong:Liu:Shang:NeurIPS:2022} show that the semantics of a sample change during adversarial training. 
%Hence, there is a need for metrics to measure the size of perturbations that is imperceptible or semantically preserving.
%Fezza et al.~\cite{Fezza:Bakhti:Hamidouche:Deforges:QoMEX:2019} and Mezher et al.~\cite{Mezher:Deng:Karam:EUVIP:2022} propose to use objective metrics for image quality to approximate the imperceptibility in the computer vision domain.
%Zhang et al.~\cite{Zhang:Sheng:Alhazmi:Li:ACMTIST:2020}, focusing on providing such a metric for Natural Language Processing.
%Vadillo et al.~\cite{Vadillo:Santana:CS:2022} also highlight conducted subject studies to evaluate the noticeability of audio adversarial examples.

%Even in computer vision, adversarial examples are not always imperceptible. For example, Machado et al.~\cite{Machado:Silva:Goldschmidt:CSUR:2021} find that visible perturbations such as adversarial patch~\cite{Brown:Mane:Roy:Abadi:Gilmer:ArXiv:2017}, and graffiti on stop signs~\cite{Eykholt:Evtimov:Fernandes:Li:Rahmati:Xiao:Prakash:Kohno:Song:CVPR:2018} are also considered adversarial examples in research.

%The aforementioned research examines the work on defining and creating adversarial examples, demonstrating the insufficiency of using conventional $L_p$ norms to evaluate the imperceptibility and semantics between clean and adversarial examples. 

\vspace{-0.1in}
\revadd{
\subsection{Non-Evasion Attacks}
\label{sec:relatedwork-poisoning}
Similar to evasion attacks, data poisoning and backdoor attacks aim to compromise model accuracy. 
However, they achieve it by tampering the training data to create deceptive model decision boundaries. 
%Data poisoning attacks involve modifying the training data to create deceptive decision boundaries, either to manipulate the prediction outcomes of a specific input or the entire model.
%Meanwhile, Backdoor attacks are a form of poisoning attacks where the attacker inject tempered training data with triggers 
% and then activates the attack by showing the trigger pattern at inference time.
In addition, backdoor attacks also require perturbing the test instance to result in a misclassification. 
This is achieved by introducing manipulated training data with triggers that can be activated during the testing phase.

Goldblum et al.~\cite{Goldblum:Tsipras:Xie:Chen:Schwarzchild:song:Madry:Li:Goldstein:TPAMI:2022} and Cinà et al.~\cite{Cina:Grosse:Demontis:Sebastiano:Zellinger:Moser:Oprea:Biggio:Pelillo:Roli:CSUR:2023} 
review recent literature on attack methodologies and countermeasures for both poisoning and backdoor attacks.
Both of these surveys found that existing research made overly-optimistic assumptions when designing / validating attack techniques, e.g., assuming the knowledge of a large portion of training data. 
They advocate for researchers to test proposed methods in more realistic situations to better assess the potential threats. 
Furthermore, they encourage exploration of the relationship between poisoning attacks and evasion attacks. 
This could lead to the creation of attacks that produce less noticeable poisoning examples, 
or defensive strategies that can safeguard models against both backdoor and evasion attacks.
%Their survey catalogs and systematizes the threats in the dataset creation process, and discuss the open problems that benefits the understanding of dataset security. 

In addition to undermining model accuracy, 
adversarial attacks also aim at breaching the privacy and confidentiality of training data. 
In particular, membership inference attacks~\cite{Shokri:Stronati:Song:Shmatikov:SP:2017} attempt to determine whether a specific data point was part of the training set used to train the model.
Hu et al.~\cite{Hu:Salcic:Sun:Dobbie:Yu:Zhang:CSUR:2022} present a comprehensive survey of existing research efforts on membership inference attacks. 
They find that, similar to evasion attacks, the membership inference attack success rate decreases as 
%the training data better represents the whole data distribution, i.e., 
the number of training samples increases.
%and model stealing attacks~\cite{Oliynyk:Mayer:Rauber:CSUR:2023} are designed to breach the privacy of training data and machine learning models. 
However, all these attacks are orthogonal to our survey, as we focus on adversarial evasion attacks.

%Li et al. ~\cite{Li:Jiang:Li:Xia:TNNLS:2022} 
%provide the first survey that focuses on backdoor attacks and identified common scenarios in which backdoor attack happen in real life. 
%Furthermore, they proposed a systematic taxonomy for backdoor attacks and defenses for researchers and practitioners to identify the characteristics and limitations of each method. 

%Wang et al.~\cite{Wang:Ma:Wang:Hu:Qin:Ren:CSUR:2022} and Tian et al.~\cite{Tian:Cui:Liang:Yu:CSUR:2022} argue federated learning~\cite{McMahan:Moore:Ramage:Hampson:Arcas:AISTATS:2017} 
%creates new venue for poisoning attack, and survey recent literature on poisoning attacks for both standard and federated learning scenarios. 
%They present a unified framework to categorize both data poisoning and model poisoning attacks, and compared the defense techniques proposed for each of the learning framework, analyzed their advantages and disadvantages.
}

\vspace{-0.1in}
\subsection{Effects of Training Data on Standard Generalization}
\label{sec:relatedwork-standard}
A number of surveys investigate the influence of data properties on standard
rather than robust generalization.
One of the earliest is probably the work of Raudys and Jain~\cite{Raudys:Jain:TPAMI:1991},
who review studies related to the influence of sample size on binary classifiers, showing that
a limited sample size usually leads to sub-optimal generalization.
%With the development of deep learning and the ever-increasing need for larger training datasets,
%a variety of data augmentation techniques have been proposed.
Bansal et al.~\cite{Bansal:Sharma:Kathuria:CSUR:2021} and
Bayer et al.~\cite{Bayer:Kaufhold:Reuter:CSUR:2022} also survey papers addressing the data scarcity problem,
focusing in particular on the recent advancements in data augmentation techniques in the fields of computer vision, security, and text classification.
Their results show that augmentation techniques %exist for various application domain and
can help improve a model's generalization by reducing the problem of model overfitting.
%They evaluate the effectiveness of such techniques in improving the accuracy of machine learning models.

%Limited sample size is also one of the culprit behind poor robust generalization~\cite{Schmidt:Santurkar:Tsipras:Talwar:Madry:NeurIPS:2018}, we collected a number of researches characterize the sample complexity for robust generalization or propose data augmentation techniques to fill in the sample complexity gap.

Label noise is another aspect of data that influences both standard and robust generalization.
Most works on this topic find that the presence of noisy labels increases the need for a greater number of training samples and may result in unnecessarily complex decision boundaries~\cite{Frenay:Verleysen:TNNLS:2014,Song:Kim:Park:Shin:Lee:TNNLS:2022}.
For example, Fr\'{e}nay and Verleysen~\cite{Frenay:Verleysen:TNNLS:2014} show
that overfitting to label noise greatly degrades a model's standard generalization;
the same effect has been observed in the case of robust generalization~\cite{Sanyal:Dokania:Kanade:Torr:ICLR:2021}.
Song et al.~\cite{Song:Kim:Park:Shin:Lee:TNNLS:2022} survey the impact of label noise in deep learning, arguing
that the presence of noisy labels is a more serious concern for deep models as they contain a larger number of parameters which makes them prone to overfitting to the noise in training data.
%They also point out the connection between adversarial poisoning attacks and noisy labels as
%the countermeasures for both share the goal of learning noise-resilient representations.
They mention that adversarial defense techniques, e.g., adversarial training, are effective against label noise~\cite{Zhu:Zhang:Han:Liu:Niu:Yang:Kankanhalli:Sugiyama:ArXiv:2021, Fatras:Damodaran:Lobry:Flamary:Tuia:Courty:TPAMI:2022}
but do not discuss how label noise influences a deep learning model's robustness under attacks.

Lorena et al.~\cite{Lorena:Garcia:Lehmann:Souto:Ho:CSUR:2020} identify a collection of 26 quantitative metrics that measure data complexity with respect to
(1) ambiguity of classes, i.e., whether the classes can be clearly distinguished with the given features,
(2) sparsity and dimensionality of data, 
%i.e., whether enough information are provided to learn confident decision boundaries, and
(3) complexity of boundary separating the classes, i.e., whether more intricate functions are required to describe the decision boundaries.
The authors also discuss how these metrics help estimate the difficulty of performing classification on a given dataset.
Similar to our survey, the authors show that high dimensionality and small separation between classes hinder standard generalization.
However, the relationship of some of the metrics reviewed by these authors, e.g.,
%faction of borderline points (i.e., a measure for the complexity of the required decision boundary) and
%the fraction of hyperspheres covering data (i.e.,
the number of non-intersecting spheres needed to enclose all data points of a class,
to robust generalization is not studied, according to our survey.

%Moreover, the effect of XXX on standard generalization needs future investigation as well (that is if we found something they do not have).

%Knowing the characteristics of a dataset according to these perspectives can assist researchers and practitioners to select optimal learning algorithms~\cite{Ho:Basu:TPAMI:2002}.

He and Garcia~\cite{He:Garcia:TKDE:2009} focus on the imbalance learning problem. %~--
%the disproportion in the number of samples belonging to each class in a given dataset.
The authors found that most standard algorithms %are designed with the assumption of a balanced class distribution.
%These algorithms
fail to reliably represent the characteristics of the imbalanced data and result in unfavorable performance across classes.
Furthermore, L\'{o}pez et al.~\cite{Lopez:Fernandez:Garcia:Palade:Herrera:InfSci:2013} discuss six intrinsic data characteristics that potentially complicate learning from imbalanced data:
low density, sample overlap between classes, noisy data, borderline instances,
dataset shift between training and testing distributions, and
small disjuncts, i.e., disperse small clusters of samples from a single class.
Their analysis concludes that while all these ``unfavorable'' data characteristics further complicate the data imbalance
issues, data overlap between classes is probably one of the most harmful.
To follow up on this point, Santos et al.~\cite{Santos:Henriques:Pedro:Japkowicz:Fernandez:Soares:Wilk:Santos:AIR:2022}
focus on the joint effect of data imbalance and class overlap on model generalization.
The negative impact of data imbalance, low separation, and noisy data on robust generalization was also discussed in our survey.
Yet, the compounding effect of these factors, as well as the effect of other properties,
on robust generalization needs future investigation.

Recently, Yang et al.~\cite{Yang:Jiang:Song:Guo:IJCV:2022} summarized relevant studies focusing on
long-tailed distributions in the field of Computer Vision.
% and categorize the main methods for alleviating the issues caused by long-tailed distribution.
%They present quantitative metrics for measuring data imbalance and .
This survey also includes work on the influence of long-tail distributions on a model's adversarial robustness~\cite{Wu:Liu:Huang:Wang:Lin:CVPR:2021}, which is covered in our survey.
%which is included in our survey,
The authors advocate for more research on adapting long-tailed-based approaches for standard generalization to improve robust generalization.

Finally, Moreno-Torres et al.~\cite{MorenoTorres:Raeder:Rodrigues:Chawla:Herrera:PR:2012} present a unifying framework to categorize existing definitions of dataset shift~-- the case where the joint distribution of inputs and outputs differs between training and testing data.
While ML models are normally trained under the premise that testing data has a similar distribution to the training data,
in reality, the observed data distribution may be different from the historical data that the model is trained on.
Such difference can substantially compromise the quality of model predictions.
The authors analyze the possible causes for dataset shift, e.g., malicious software that evolves over time, and
review the techniques dealing with dataset shift.
They characterize adversarial attacks as one form of dataset shift, where adversaries adaptively
change test instances to create a distribution that differs from training data.
%All works discussed in our survey assumed similar distribution on training and testing data, treating adversarial attacks as the only dataset shift in the problem setup.
%However, in real applications, the underlying data distribution itself can be non-stationary, and the characterize the influence of the dataset shift between training and testing data on the adversarial robustness is yet to be investigated.

\revadd{Overall, despite the similarities with our work, literature discussed in this section focuses on standard generalization while our survey discusses 
the effect of data on robust generalization.}

%More works use the connection between adversarial attacks and distributional shift to analyze the effect of adversaries on generalization performance~\cite{Tu:Zhang:Tao:NeurIPS:2019}.
%However, we do not discuss them in detail, as they focus more on models instead of data.
%\jr{How is that relevant to data properties section?} \gx{This can be removed, as it an individual work we filtered}

\vspace{-0.1in}
\subsection{Summary}
\revadd{
Our survey is the first to explicitly focus on properties of training data in the context of model robustness under evasion attacks.
Numerous other surveys on evasion attacks discuss attack and defense mechanisms, non-data-related reasons for adversarial vulnerability, and the different threat models. 
We identified only five surveys that considered data-related reasons for evasion attacks. 
However, as these surveys are older and do not focus on data in particular, our work provides a more extensive
and comprehensive view on this topic. 
By including more than 50 papers not covered in prior work, we were able to 
identify additional relevant properties, practical suggestions, and future research directions in this area. 

Additional work studies non-data-related reasons for evasion attacks, as well as non-evasion attacks, 
such as poisoning and backdoor. 
Yet another body of literature examines how data properties affect standard generalization. These works show that 
some of the properties discussed in our survey, such as 
the number of samples, dimensionality, and label quality, also affect clean accuracy. 
There are also additional data properties that are covered exclusively by these or by our work. 
Studying the interplay between data properties for clean and robust accuracy is an interesting research direction, 
which could be facilitated by our work. 
However, all these current works are orthogonal and complementary to ours.
}

%\ad{
%The related work of our survey can be categorized into four key topics: 
%The first topic examines data for other adversarial attacks, this include the research that investigates the link between the data characteristics and model's resilience against poisoning attacks as well as the studies that explore data poisoning and backdoor attacks and their countermeasures. \jr{same issues as before: this is meta-summary, we need a concrete summary.}
%These studies complement our survey as they highlight the threats directly aimed at data, thus emphasizing the importance of secure data collection. 
%The second topic focuses on the relationship between various properties of training data and model's standard generalization ability. 
%This body of work suggests that data traits such as number of samples, dimensionality, label quality also influence model's ability to generalize in standard classification. \jr{this looks more concrete!}
%
%The third strand of research concerns adversarial evasion attacks. 
%The work in this area encompasses the research frontier in evasion attacks and the countermeasures. 
%Due to the large volume of work in this area, there are numerous surveys that gives more detail on the advancement. 
%\jr{meta-summary again}
%In addition to attacks and defenses, one relevant line of work investigates the alignment of the conventional similarity metrics used for adversarial examples and human perception, showing the need for supplementary metrics. \jr{why important?}
%These studies \jr{which "these studies"?} collectively present an extensive overview of other types of work conducted on adversarial robustness.
%The last category of work proposes alternative explanations for model vulnerability to adversarial examples.
%These studies presented hypothesis showing the characteristics of machine learning models, e.g., nonlinearity, invariance to rotational shift etc, induces susceptibility to attacks, as well as limited computational resources and non-robust feature representations. \jr{all text based on previous related work looks somewhat concrete; the new additions should be at least at the same level, or better.}
%These studies supplement our work, offering a broader perspective of potential factors affecting model's robust generalization ability. }
%


\section{Contract Design as a Special Case of Scoring Rule Design}\label{sec:related to contract}
In this section, we compare the contract design framework to the scoring rule design framework and reduces the standard contract designing problem to a special case of the scoring rule designing problem.
In a contract designing problem, we refer to $\Omega$ as the outcome space. 
Consider the following contract designing problem:
\begin{mdframed}[style=box]
\textbf{Contract designing problem in the principal-agent framework}

\vspace{5pt}
\noindent
At the $t$-th round, the principal and the agent play as the following:
\begin{itemize}[noitemsep, topsep=3pt]
% [leftmargin=*,topsep=0pt]
% \setlength{\itemsep}{0pt}
    \item[1.] The principal announces a contract $C_t:\Omega\rightarrow \RR_+$ to the agent.
    \item[2.] Based on $C_t$, the agent chooses an action $b_{k_t}\in\cB$ indexed by $k_t$ and bears a cost $c_{k_t}\ge 0$. The action $b_{k_t}$ can be observed by the principal, but the cost $c_{k_t}$ is private to the agent.
    \item[3.] The stochastic environment then selects an outcome $\omega_t\in\Omega$ according to $p(\omega_t\given b_{k_t})$. The outcome $\omega_t$ is revealed as observation, but the generating process $p(\omega_t\given b_{k_t})$ is private to the agent.
    % \item[3.] 
    % % The agent computes her own belief $\sigma_t\in\Delta(\Omega)$ on the hidden state according to \eqref{eq:sigma}, 
    % The agent gives her report $\hat\omega_t\in\Omega$ on her observation.
    \item[4.] The principal makes a decision $a_t\in\cA$ based on $\omega_t$.
    \item[5.] In the end, the principal obtains her utility $u(a_t, \omega_t)$ and pays the agent by $C_t( \omega_t)$.
\end{itemize}
\end{mdframed}
The difference between this contract designing problem and the scoring rule designing problem is that $\omega_t$ is revealing, and the agent's action influences the principal's utility only through her action choice without giving any report. 
We remark that we can replace $u(a_t, \omega_t)$ by $u(\omega_t)=u(a^*(\omega_t), \omega_t)$ if the principal knows about the utility function and always takes the best action. In this contract design problem, the agent has an action policy $\pi:\cC\rightarrow [K]$, where $\cC$ is the contract space.
The principal targets at designing the optimal contract that maximizes her profit, i.e., utility minus payment, subject to the agent's best response given by maximizing the agent's profit, i.e., payment minus cost. The Stackelberg game for this contract designing problem can be formulated as
\begin{equation}
\begin{aligned}\label{eq:stackelberg-contract}
&\max_{C\in\cC}\quad \EE_{\omega\sim p(\cdot\given b_{\pi^*(C)})} \sbr{u(\omega) - C(\omega)},\\
&\mathrm{s.t.}\quad \pi^*(C)\in\argmax_{k\in[K]} \EE_{\omega\sim p(\cdot\given b_k)} C\rbr{\omega} - c_{k},
\end{aligned}
\end{equation}


% For any contract design problem with action space $[K]$ and outcome space $\Omega$, a contract is a mapping $C:\Omega\to \RR$ from the outcome space to a payment. After the principal posts a menu of contracts $\{C_i\}_i$, the agent selects a contract and best responds with an action. The principal receives a reward $u:\Omega\to \RR$ which is a function of the outcome. The principal designs optimal contract to maximize revenue, the expected reward minus the payment of the contract.

In the sequel, we aim to show in the scoring rule designing problem:
(i) If the hidden state is perfectly revealing, i.e., $o_t=\omega_t$ as the agent's observation after taking her action, there exists a class of scoring rules such that the above contract designing problem is equivalent to the scoring rule designing problem. 
(ii) Using proper scoring rules, the principal's optimal profit under the scoring rule framework is no less than the optimal profit under the contract framework.


To show (i), consider the scoring rule class 
$$
\sS^C=\cbr{S\in\sS\given S(\hat\sigma, \omega)=\ind(\hat\sigma=e_\omega)\cdot C(\omega), \forall C\in\cC}, 
$$
where $e_{\omega'}(\omega)=\ind(\omega=\omega')\in\Delta(\Omega)$.
Even though $S\in\sS^C$ might not be a proper scoring rule, the agent will always be truth-telling, i.e., $\hat\sigma=e_\omega$, since only by telling the truth can she gains nonzero payment.
Therefore, this hidden state $\omega_t$ is also revealed to the principal through the agent's report.
The Stackelberg game in \eqref{eq:stackelberg-2} in the scoring rule problem can therefore be written as,
\begin{equation}
\begin{aligned}\label{eq:stackelberg-scoring}
&\max_{S\in\sS^C}\quad \EE_{\omega\sim p(\omega\given b_{k^*(S)})} \sbr{u(\omega) - S(e_\omega, \omega)},\\
&\mathrm{s.t.}\quad k^*(S)\in\argmax_{k\in[K]} \EE_{\omega\sim p(\cdot\given b_{k})} S\rbr{e_\omega, \omega} - c_{k},
\end{aligned}
\end{equation}
By noting that $S(e_\omega, \omega)=C(\omega)$, we have that \eqref{eq:stackelberg-scoring} and \eqref{eq:stackelberg-contract} are actually the same problem.
We thus conclude that the contract designing problem is perfectly reduced to this scoring rule designing problem.

We also remark that even if the hidden state is perfectly revealing, the principal need not be aware in advance. By sticking to a proper scoring rule, the agent always tells the truth. Moreover, using the revelation principle stated in \Cref{lem:revelation}, for any $S\in\sS^C$, there always exists a proper scoring rule $\tilde S\in\cS$ that generates the same expected payment $\EE_{\omega\sim p(\cdot\given b_k)} \tilde S(e_\omega, \omega)=\EE_{\omega\sim p(\cdot\given b_k)} C(\omega)$ for the agent, even though $\tilde S(e_\omega, \omega)$ might not be equal to $C(\omega)$ pointwise. Therefore, statement (ii) is also justified and we conclude that the (proper) scoring rule framework has more power than the contract framework by asking one more question about the agent's belief.



\section{More Details on the Revelation Principle}\label{sec:proper scoring rule}
% In this section, we first formulate the problem of optimally acquiring information under the principal-agent framework, where the principal aims to find a scoring rule to pay the agent for giving reports and this scoring rule should maximize the principal's profit under the agent's best response. Then, we show that restricting to the class of proper scoring rule is without any loss of generality for the principal's purpose. Given that the information structure is unknown to the principal, we propose an online learning framework for the principal to optimally acquire information. 

% \subsection{Acquiring Information under the Principal-agent Framework}
% To formulate the problem of optimally acquiring information under the principal-agent framework, we consider a stochastic environment with a principal and an almighty agent. 
% At the $t$-th round, there is a hidden state $\omega_t\in\Omega$ that will affect the principal's utility, but unknown to both the agent and the principal until the end of this round. To elicit refined information from the agent,
% the principal moves first and offers a scoring rule to the agent, under which the agent receives a payment according to the quality of her report on the posterior of the hidden state. 
% The agent may choose from $K$ actions with some cost, obtain an observation related to the hidden state, and report to the principal her belief on the hidden state.
% After receiving the agent's report, the principal then makes a decision $a_t\in\cA$ and pays the agent according to the scoring rule. The information acquisition proceeds as the following:
% \begin{mdframed}[style=box]
% \textbf{Information acquisition via scoring rule}\\
% At the $t$-th round, the principal, and the agent play as the following:
% \begin{itemize}[leftmargin=*,topsep=0pt]
%     \item[1.] The principal announces a scoring rule $S_t:\Delta(\Omega)\times\Omega\rightarrow\RR_+$ to the agent.
%     \item[2.] The agent chooses an action from $\cB$ indexed by $k_t$ and bears a cost $c_{k_t}\ge 0$.
%     \item[3.] The stochastic environment selects the hidden state $\omega_t\in\Omega$ and emits an observation $o_t\in\cO$ to the agent.
%     \item[3.] The agent chooses a distribution $\hat\sigma_t\in\Delta(\Omega)$ and reports it to the principal as her estimation on the posterior of the hidden state.
%     \item[4.] Based on the report $\hat\sigma_t$ and the agent's action choice, the principal makes a decision $a_t\in\cA$.
%     \item[5.] When the hidden state $\omega_t$ is revealed, the principal obtains her utility $u(a_t, \omega_t)$, and pays the agent by $S_t(\hat\sigma_t, \omega_t)$.
% \end{itemize}
% \end{mdframed}
% \addtocounter{footnote}{-2} %3=n
% \stepcounter{footnote}
% \footnotetext{}

% Here, the scoring rule $\cS_t$ is a payment rule based on the agent's report $\hat\sigma$ and the real state $\omega_t$ with bounded norm $\nbr{S}_\infty\le B_S$. We let $B_S$ large enough such that all the agent's actions are inducible. The reward function $u(a, \omega)$ also has bounded norm $\nbr{u}_\infty\le B_u$. 
% We consider the agent's action set $\cB$ and the observation set $\cO$ to be finite.
% Specifically, $\cB=\{b_0, b_1,\dots,b_K\}$ has $K+1$ actions, with action $b_0$ being the null action and leading to a zero cost. We remark that $b_0$ captures the situation when the agent has no incentive to participate and just receives a null observation. The remaining $K$ actions are referred to as the effective actions in the sequel.
% Notably, we allow the hidden state to be influenced by the agent's action choice.
% Recall the previously discussed producer-consultant example, where the producer is the principal, the consultant is the agent, and the value of the customer group is the hidden state.
% In this example, the consultant may conduct some market investigations and collect relevant information for refining the customer group's value. Such actions may actually cause inference to the marketing and affect the customer's value. Therefore, the hidden state $\omega_t$ is endogenous in our setting.

% \paragraph{Information structure.} 
% In the remaining part of this subsection, we ignore the subscript $t$ for a while.
% In this information acquisition process, we consider the agent to be almighty who has full knowledge of the \emph{information structure}, i.e., each action's cost $c_k$ and the generating process $p(\omega, o\given b_k)$ for the hidden state and the observation under action $b_k$.
% Therefore, after obtaining the observation $o$, the agent is able to refine her belief on the hidden state as $\sigma(\omega)=p(\omega\given b_{k}, o)$. If $k=0$, the belief simply corresponds to the distribution $\sigma(\omega)=p(\omega)$.
% Note that $\sigma$ can also be viewed as a random variable mapping from the sample space to the probability space over the hidden state.
% In addition, $\sigma$ has support $\Sigma$ with cardinality $M\defeq \abr{\Sigma}\le K\times\abr{\cO}+1$.
% Let $q_k(\sigma)\in\Delta(\Sigma)$ be the distribution of $\sigma$ under the agent's action $k\in[K]^+$ where $[K]^+=\{0, 1,\dots,K\}$.
% Since $\sigma$ already captures all the information about the hidden state from the observation, we also refer to the costs $\{c_k\}_{k\in[K]^+}$ and the distributions of the belief $\{q_k\}_{k\in[K]^+}$ as the information structure.
% The information structure is private to the agent.
% We consider the situation where the principal only has knowledge of her utility function $u$ 
% \footnote{Since the utility and the hidden state are both known to the principal at the end of each round, the principal is able to obtain good estimation of the utility function. Therefore, we consider $u$ to be the principal's private information in Table \ref{table:info}.} and is able to observe the agent's action $b_{k}$
% \footnote{In cases where the principal cannot observe the agent's action, there are still ways to distinguish different actions. For instance, when $q_k$ has different support for different $k$ and the agent is truth-telling about her belief under proper scoring rules (see \S\ref{sec:proper scoring rule}), the principal is able to learn the support for a particular agent's action by repeating the same scoring rule multiple times. The next time the agent chooses the same action, the principal will be aware.}. The summary of different information types in this information acquisition process is available in Table \ref{table:info}.
% \begin{table}[h!]
% \centering
% \begin{tabular}{|c|c|c|c|}
%     \hline
%     Public Info & Agent's Private Info & Principal's Private Info & Delayed Info\\
%     \hline
%     {$S_t, k_t, \hat\sigma_t$} & $\{c_k\}_{k\in[K]^+}, \{q_k\}_{k\in[K]^+}, \sigma$ & $u$ & $w_t$\\
%     \hline
% \end{tabular}
% \caption{Table for the information types.}
% \label{table:info}
% \end{table}

% \paragraph{Information acquisition as a Stackelberg game.} 
% In this information acquisition process, the agent always aims to find her own action selection policy $k=\mu(S)$ and a reporting scheme $\hat\sigma=\nu(S, \sigma, k)$ that maximizes her own expected profit $g^{\mu,\nu}(S)$.
% Here, the reporting scheme does not depend on $o$ since $\sigma$ captures all the information about the hidden state.
% On the other hand, the principal aims to find the scoring rule $S$ and the best decision policy $a=\iota(S, \hat\sigma, k)$ that maximizes her own expected profit $h^\iota(S)$ under the agent's best response $(\mu^*,\nu^*)\in\argmax_{\mu,\nu} g^{\mu,\nu}(S)$. 
% The problem can therefore be formulated as a Stackelberg game,
% % \begin{mini!}|l|[3]
% %     {S\in\cS, \atop \pi:\cS\times\Delta(\Omega)\times[K]^+\rightarrow \cA}{h^\pi(S) 
% %     \defeq \EE_{\omega\sim\sigma \sigma\sim q_{\mu^*(S)}} \sbr{u(\pi(S, \nu^*(S, \sigma, k), k), \omega) - S(\nu^*(S, \sigma, k), \omega)}}
% % \end{mini!}
% \begin{equation}
% \begin{aligned}\label{eq:stackelberg-1}
% &\max_{\iota, S:\nbr{S}_\infty\le B_S}\quad h^\iota(S) \defeq \EE_{\omega\sim\sigma, \sigma\sim q_{\mu^*(S)}} \sbr{u\rbr{\iota\big(S, \nu^*(S, \sigma, \mu^*(S)), \mu^*(S)\big), \omega} - S(\nu^*(S, \sigma, \mu^*(S)), \omega)},\\
% &\qquad\mathrm{s.t.}\quad(\mu^*, \nu^*)\in\argmax_{\mu, \nu}  g^{\mu,\nu}(S)\defeq\EE_{\omega\sim\sigma, \sigma\sim q_{\mu(S)}} S\rbr{\nu(S, \sigma, \mu(S)), \omega} - c_{\mu(S)}.
% \end{aligned}
% \end{equation}
% Here, by taking the $(\mu^*, \nu^*)$ that maximizes the principal's profit within the agent's best response, we are actually assuming that the agent is tie-breaking in favor of the principal.

% \subsection{Eliciting Information via Proper Scoring Rules}
% Acquiring information with a proper scoring rule is a generalization of incentivizing information acquisition with contracts. In contract design, the agent selects a contract before taking an action; while in scoring rule design, the agent reports his signal after taking an action, which is equivalent to selecting a contract after taking the action. Hence, acquiring information with a proper scoring rule has more power than contract design. The following example shows that a contract design problem can be reduced to a scoring rule design problem in our paper.



%The scoring rule can be viewed as a special case of contracts in information acquisition \citep{10.1145/3490486.3538261}. 
In this section, we provide a formal argument on the revelation principle in our model. That is, it is without loss of generality to only design the proper scoring rules under which the agent is encouraged to be truth-telling. 
% In the original model, the agent is not guaranteed to report her actual belief under a general scoring rule. However, there is a class of proper scoring rules under which the agent is encouraged to be truth-telling \citep{gneiting2007strictly, dawid2007geometry}. 
\begin{definition}[Proper scoring rule]
    A scoring rule $S:\Delta(\Omega)\times\Omega\rightarrow \RR_+$ is proper if, for any belief $\sigma\in\Delta(\Omega)$ and any reported posterior $\hat\sigma\in\Delta(\Omega)$,  we have $\EE_{\omega\sim \sigma}S(\hat\sigma, \omega)\le \EE_{\omega\sim \sigma}S(\sigma, \omega)$. In addition, if the inequality holds strictly for any $\hat\sigma\neq \sigma$, the scoring rule $S$ is strictly proper. 
\end{definition}
Let $S$ be a proper scoring rule and fix the agent's policy $\mu(\cdot)$ for action selection. For a reporting scheme $\nu$ and any true belief $\sigma$, it follows from definition \ref{def:PSR} that
\begin{align*}
    g^{\mu,\nu}(S)= \EE_{\omega\sim\sigma} S(\nu(S, \sigma, \mu(S)), \omega) - c_{\mu(S)}\le \EE_{\omega\sim\sigma} S(\sigma, \omega) - c_{\mu(S)}.
\end{align*}
Therefore, the agent's expected payment is maximized by always being truth-telling about her belief under the class of proper scoring rule.
In the following, we let $S(\hat\sigma, \sigma)=\EE_{\omega\sim\sigma}S(\hat\sigma, \omega)$. 
To give an example of proper scoring rules, let us consider the binary hidden state space $\Omega=\{0, 1\}$ where the class of proper scoring rules admits the Schervish representation \citep{gneiting2007strictly}, i.e., $S(p, 1)=G(p) + (1-p) G'(p)$ and $S(p, 0)=G(p)-p G'(p)$ where $p\in[0, 1]$ and $G:[0, 1]\rightarrow\RR_+$ is a convex function.
Intuitively, the expected payment of a proper scoring rule $S$ given belief $\sigma$ and report $p$ is $S(p, \sigma)=G(p)+(\sigma-p)G'(p)$, which corresponds to the supporting line of $G$ at $p$.
In this example, the convexity of $G$ guarantees that $S(p, \sigma)=G(p)+(\sigma-p)G'(p)\le G(\sigma)=S(\sigma, \sigma)$.
Moreover, the next observation in Lemma \ref{lem:revelation} suggests that restricting to the class of proper scoring rules does not incur any loss of generality for the principal's purpose.
\begin{lemma}[Restatement of the revelation principle]
    There exists a proper scoring rule $S^*$ that is an optimal solution to \eqref{eq:stackelberg-1-concise} if the agent is truth-telling under any proper scoring rule.
    \begin{proof}
        We first prove that for any scoring rule $S$ such that $\nbr{S}_\infty\le B_S$, there always exists a \emph{proper} scoring rule $S'(\hat\sigma, \omega)=S(\nu^*(S, \hat\sigma, k), \omega)$ such that they make the same payment to the agent for any $\sigma\in\Delta(\Omega)$ and any agent's action choice. To prove that $S'$ is a proper scoring rule with norm bounded by $B_S$, we have $B_S\ge S(\nu^*(S, \hat\sigma, k), \omega)=S'(\hat\sigma, \omega)\ge 0$ and $$S'(\hat\sigma, \sigma)= S(\nu^*(S, \hat\sigma, k), \sigma)\le S(\nu^*(S, \sigma, k), \sigma)=S'(\sigma, \sigma).$$
        The fact that $S'$ makes the same payment can be verified by plugging in $\hat\sigma=\sigma$ in the definition of $S'$ since the agent is truth-telling under proper scoring rules, and taking expectation with respect to $\omega\sim\sigma$, i.e., $S'(\sigma, \sigma)=S(\nu^*(S, \sigma, k), \sigma)$, which proves the first part.
        
        Secondly, we prove that encouraging the agent to report the real belief $\sigma$ makes the principal's revenue nondecreasing. Note that
        \begin{align*}
            \max_{\iota} \EE_{\omega\sim\sigma, \sigma\sim q_k}\sbr{u(\iota(S, \nu^*(S, \sigma, k), k), \omega)} \le  \max_{\iota} \EE_{\omega\sim\sigma, \sigma\sim q_k}\sbr{u(\iota(S, \sigma, k), \omega)}=\EE_{\omega\sim\sigma, \sigma\sim q_k}\sbr{u(a^*(\sigma), \omega)},
        \end{align*}
        where $a^*(\sigma)=\argmax_{a\in\cA}\EE_{\omega\sim\sigma} u(a, \omega)$.
        Here, the inequality holds by noting that $\nu^*(S, \sigma, k)$ is a function of $\sigma$, and the equality holds by noting that $\omega\indep (S,k)\given \sigma$.
        To conclude, by choosing $S'$ instead of $S$, the payment is exactly the same while the principal's revenue is nondecreasing. Thus, the principal's profit is nondecreasing by choosing $S'$ and there must exist a proper scoring rule that is also an optimal scoring rule. 
    \end{proof}
\end{lemma}

% The power of a proper scoring rule is thus obvious -- not only paying the agent the exact amount \emph{any} scoring rule is capable of, but also encouraging truth-telling from the agent. 
Following Lemma \ref{lem:revelation}, the principal's optimal scoring rule lies within the class of proper scoring rules $\cS$ with bounded norm $\nbr{S}_\infty\le B_S$.
One concern about the use of proper scoring rules is that being truth-telling might not be the unique maximizer to the agent's utility. However, we note that the class of proper scoring rules is a convex hull with strictly proper scoring rules as the interior. Thus, adding an infinitesimal portion of a \emph{strictly proper} scoring rule to any \emph{proper} scoring rule always yields a \emph{strictly proper} scoring rule. In this sense, we can safely make the assumption that the agent always reports the true posterior under a proper scoring rule.
The following table summarizes all the different information types in our model.


% \newpage
\section{More Details on the Binary Search Algorithm}
In this section, we give a summary of the binary seach algorithm as follows, 
\begin{figure}
\makebox[\linewidth]{%
\begin{minipage}{\linewidth}
\begin{algorithm}[H]
\begin{algorithmic}[1]
\STATE {\bfseries Input:} $S_0, S_1, k^*(S_1), t, k_t, \{I_q^t(k)\}_{k\in[K]}$;
\STATE {\bfseries Output:} $t$;
\IF{$k_t = k^*(S_1)$}
    \STATE Break the binary search algorithm;
\ENDIF
\STATE Initiate $\lambda_{\min}\leftarrow 0, \lambda_{\max}\leftarrow 1, t_0\leftarrow t$;
\WHILE{$\lambda_{\max}-\lambda_{\min} \ge I_q^{t_0}(k_{t})\land I_q^{t_0}(k^*(S_1))$}
\STATE Start a new round $t\leftarrow t+1$;
\STATE Pick $\lambda \leftarrow \rbr{\lambda_{\min}+\lambda_{\max}}/2$ as the middle point; 
\STATE The principal announces scoring  rule $S_{t} = (1-\lambda) S_0 + \lambda S_1$ and obtain the agent's response $k_{t}$;
\IF{$k_{t}=k^*(S_1)$}
    \STATE Update $\lambda_{\max} \leftarrow \lambda$;
\ELSE
    \STATE Update $\lambda_{\min}\leftarrow \lambda$;
\ENDIF
\ENDWHILE
\end{algorithmic}
\caption{Binary Search {BS}($S_0, S_1, k^*(S_1)$, $t$) }\label{alg:BS}
\end{algorithm}
\end{minipage}}
\end{figure}
The binary searching algorithm at step $t$ works on the segment connecting two scoring rules $S_0$ and $S_1$, where the agent's best response under $S_1$ is $k^*(S_1)$. The goal of this binary search is to find the first switching point of the agent's best response from $k_1$ to another action on this segment. 
The binary searching algorithm keeps updating on $\lambda_{\max}$ and $\lambda_{\min}$ as the candidate interval that contains the first switching point. Note that any $\lambda\in[0, 1]$ corresponds to a scoring rule on the segment connecting $S_0, S_1$. Each time, the algorithm deploy a scoring rule corresponding to $\rbr{\lambda_{\min}+\lambda_{\max}}/2$ and the candidate interval is thus cut by half.
In addition, we consider binary searching to a finite depth $m$ such that the searching error respects the minimal uncertainty of $\hat q_{k'}^t, \hat q_{k_t^*}^t$, i.e., 
$2^{-m}\le I_{q}^{t}(k') \land I_{q}^{t}(k_t^*)$ in Algorithm \ref{alg:BS}. Thus, the binary search achieves sufficient accuracy with logarithmic searching time.

% To simplify our discussion, we introduce the concept of equivalent proper scoring rules.
% \begin{definition}[equivalence of proper scoring rules]
% we say that $S, S'\in\cS$ are equivalent proper scoring rules if $\EE_{\omega\sim\sigma}S(\sigma, \omega)=\EE_{\omega\sim\sigma}S'(\sigma, \omega)$ for any $\sigma\in\Delta(\Omega)$.
% \end{definition}
% As it turns out that even within the class of proper scoring rules, there might be more than one scoring rule resulting in the same payment for any $\sigma\in\Sigma$. 

 
% If $G$ is not smooth at $p_0$, then we have a class of proper scoring rules corresponding to multiple sub-gradient of $G$ but having the same payment $G(p)$. With a little abuse of notation, we use $S(\sigma)=\EE_{\omega\sim\sigma}S(\sigma, \omega)$ to represent the proper scoring rule class that is equivalent to $S(\sigma, \omega)$ and $\cS=\cbr{S(\sigma)}$ to represent all the equivalent proper scoring rules.

% \subsection{Online Learning for the Optimal Scoring Rule}
% When $S\in\cS$, the agent's best report scheme is $v^*(S, \sigma)=\sigma$ and the principal's best decision policy $\iota^*$ can be simplified to $a^*(\sigma)$ since $\omega\indep (S,k)\given \sigma$. 
% Using the notations $u(\sigma)=\EE_{\omega\sim\sigma}u(a^*(\sigma), \omega)$, $S(\sigma)=\EE_{\omega\sim\sigma}S(\sigma, \omega)$,  and $k^*(S)=\mu^*(S)$, we rewrite \eqref{eq:stackelberg-1} under the class of proper scoring rules $\cS$ as
% \begin{equation}
% \begin{aligned}\label{eq:stackelberg-2}
% &\max_{S\in\cS}\quad h(S) \defeq \EE_{\sigma\sim q_{k^*(S)}} \sbr{u(\sigma) - S(\sigma)},\\
% &\mathrm{s.t.}\quad k^*(S)\in\argmax_{k\in[K]^+} g(k,S)\defeq \EE_{\sigma\sim q_{k}} S\rbr{\sigma} - c_{k}.
% \end{aligned}
% \end{equation}
% The difficulty in solving \eqref{eq:stackelberg-2} is that the information structure, i.e., $q_k$ and $c_k$, is unknown to the principal. 
% In this work, we study the problem of online learning the optimal scoring rule for information acquisition. 
% Given $H_{t-1}=\cbr{(S_\tau, k_\tau, \sigma_\tau, \omega_\tau)}_{\tau\in[t-1]}\in\cH_{t-1}$ as the history observed by the principal before round $t$, the principal is able to deploy a policy for the next round's scoring rule $\pi_{t}:\cH_{t-1}\rightarrow \cS$.
% Hence, the data generating process is described as the following,
% \begin{align}\label{eq:data generating}
%     p^\pi(S_t, k_t, \sigma_t, \omega_t\given H_{t-1}) = \ind\rbr{S_t=\pi_t(H_{t-1}), k_t=k^*(S_t)} q_{k_t}(\sigma_t) \sigma_t(\omega_t), 
% \end{align}
% and the regret for this online policy $\pi=\cbr{\pi_t}_{t\in[T]}$ is defined as
% \begin{align*}
%     \Reg^\pi(T)\defeq T\cdot \max_{S\in\cS} h(S) - \EE_\pi\sbr{\sum_{t=1}^T u(a^*(\sigma_t), \omega_t) - S_t(\sigma_t, \omega_t)}, 
% \end{align*}
% where the expectation is taken with respect to the data generating process. We aim to develop an online policy $\pi_t$ that learns the optimal scoring rule with small regret.



% we remark that requiring the agent to report her action is without loss of generality when the agent is telling the truth. In cases where the agent is not required to report her action, the principal can always repeat the same scoring rules multiple rounds and distinguish the agent's different  actions via comparing the reports. Specifically, if $q_k$ and $q_{k'}$ are different in the sense $\nbr{q_k - q_{k'}}_1\ge \eta$ for any $k\neq k'$, it is sufficient to tell two different actions with high probability after $\cO(M/\eta^2)$ repetitions.

% Now we specify the agent's best response. 
% Based on our previous discussion,
% the best response of the agent under proper scoring rule $S\in\cS$ can be denoted by tuple $\BR{S}=\rbr{k^*(S), \hat\sigma^*}$ where $\hat\sigma^*=\sigma$ under proper scoring rules.
% To study the  agent's best action $k^*(S)$, 
% note that we consider the action set $\cB$ and the observation set $\cO$ to be finite. As a consequence, the set of possible posteriors $\sigma$ is also finite, and we denote it as $\Sigma\subset\Delta(\Omega)$ with $\abr{\Sigma}=M \le K \cdot \abr{O} + 1$.
% Consider $q_k\in\Delta(\Sigma)$ as the distribution of posterior $\sigma$ if the agent chooses action $b_k$.
% The agent's distribution belief when choosing $b_k$ can then be written as,
% \begin{align*}
%     p(\omega\given b_k) = \sum_{\sigma\in\Sigma}\sigma(\omega) q_k(\sigma).
% \end{align*}
% We let $g(k, S)$ denote the agent's expected profit by choosing action $b_k$ under scoring rule $S$.  Following the fact $\hat\sigma^*=\sigma$, we can characterize the agent's best response as
% \begin{align}
%     k^*(S)= \argmax_{k\in[K]\cup \cbr{0}}  g(k, S), \where g(k, S) \defeq \EE_{\sigma\sim q_k} S(\sigma) - c_k, \label{def:k^*}
% \end{align}
% where we recall $S(\sigma)=\EE_{\omega\sim\sigma}S(\sigma, \omega)$.
% When we talk about \say{best response}, we typically refer to the agent's best action $k^*(S)$ in the sequel.

% Upon receiving the agent's report, the principal will adopt the reported belief and chooses the best action $a^*(\sigma) = \argmax_{a\in\cA} \EE_{\omega\sim\sigma}u(a, \omega)$.
% In the sequel, we simply use $u(\sigma)\defeq\max_{a\in\cA} \EE_{\omega\sim\sigma}u(a, \omega)$ for the maximal utility achievable for the principal based on the report $\sigma$.
% and $S(\sigma)\defeq\EE_{\omega\sim\sigma}S(\sigma, \omega)$ for the expected payment received by the agent.
% After the principal carries out her action, the hidden state $\omega$ is revealed. The principal receives utility $u(a^*(\sigma), \omega)$ and pays the agent a total amount of $S(\sigma, \omega)$.
% Let $h(S)$ denote the principal's expected profit.
% Under the best response of the agent, the principal aims to find the best proper scoring rule $S^*$ that maximizes her expected profit, 
% \begin{align}
%     S^*\in \argmax_{S\in\cS} h(S), \where h(S)=\EE_{\sigma\sim q_{k^*(S)}} \sbr{u(\sigma) - S(\sigma)}.\label{def:S^*}
% \end{align}

\section{Proof of the main theorem}

Let $p$ be an odd prime and let $V$ be an $n$-dimensional vector space over $\mathbb{F}_p$ with basis $v_1,v_2,\dots,v_n$. The groups $G$ in (a),(b) are precisely those in (\ref{Gpi}), with $n=3$, associated to the linear maps
\begin{enumerate}[label = (\alph*)]
\item $\pi : V\rightarrow \Lambda^2V;\,$ $v_2^\pi = v_3^\pi  =1 $ and $v_1^\pi = (v_1\wedge v_2)$,
\item $\pi : V\rightarrow \Lambda^2V;\,$ $v_2^\pi = v_3^\pi =1 $ and $v_1^\pi = (v_2\wedge v_3)$.
\end{enumerate}
Similarly, the groups $G$ in (c),(d),(e) are precisely those in  (\ref{Gpi}), with $n=4$, associated to the linear maps
\begin{enumerate}[label = (\alph*)]\setcounter{enumi}{+2}
\item $\pi : V\rightarrow \Lambda^2V;\,$ $v_2^\pi = v_3^\pi = v_4^\pi =1 $ and $v_1^\pi = (v_1\wedge v_2)$,
\item $\pi : V\rightarrow \Lambda^2V;\,$ $v_2^\pi = v_3^\pi = v_4^\pi =1 $ and $v_1^\pi = (v_3\wedge v_4)$,
\item $\pi : V\rightarrow \Lambda^2V;\,$ $v_2^\pi = v_3^\pi = v_4^\pi =1 $ and $v_1^\pi = (v_1\wedge v_2)(v_3\wedge v_4)$.
\end{enumerate}
By Propositions \ref{rank one prop'} and \ref{rank one prop}, for $n=3,4$ and up to a change of basis, these are the only linear maps $\pi$ of rank one.

In this section, let us take $n=3,4$ and the symbol $\pi$ denotes one of the five linear maps above. As explained in Section \ref{group section}, we may identify
\[ G/G' = V\mbox{ and }G'=\Lambda^2V.\]
Moreover, we have a natural isomorphism
\[ \Aut^c(G) \simeq \Aut^c(\pi).\]
With these identifications, we may rephrase (\ref{Delta2}) as
\begin{equation}\label{Delta3}
\Delta(u^\alpha,v^\alpha) = \Delta(u,v)^{\hat{\alpha}}
\end{equation}
for all $u,v\in V$ and $\alpha\in\Aut^c(\pi)$. The $S$ and $S'$ in Section \ref{bilinear form sec} become
\begin{align*}
S &=  \{\mbox{symmetric bilinear $\Delta :V\times V\rightarrow \Lambda^2V$ satisfying (\ref{Delta3})}\}\\
S' &= \{\mbox{anti-symmetric bilinear  $\Delta :V\times V\rightarrow \Lambda^2V$ satisfying (\ref{Delta3})}\}
\end{align*}
in the current setting. The group $\Aut^c(\pi)$ was computed in Section \ref{group section}. Let $P$ and $Q$ denote the subgroups defined there. Then, we have
\[\Aut^c(\pi) = P\rtimes Q.\]
We shall also make the following assumption.

\begin{assume}Assume that $p\geq 5$ in the cases (a),(c),(e).
\end{assume}

We first show that the groups $G$ in question satisfy Assumption \ref{assumption} so that the discussion thereafter applies.  
\begin{lemma}\label{gamma lemma}
Let $\gamma : V\rightarrow\Aut^c(\pi)$ be an $\Aut^c(\pi)$-equivariant homomorphism and let $1\leq i,j\leq n$. Suppose that
\begin{enumerate}[label = $(\arabic*)$]
\item $\gamma(v_i)=1$,
\item $v_i^\alpha = v_iv_j$ for some $\alpha\in \Aut^c(\pi)$.
\end{enumerate}
Then $\gamma(v_j)=1$ also holds.
\end{lemma}

\begin{proof}Indeed, we have
\[ 1 = \gamma(v_i)^\alpha = \gamma(v_i^\alpha) = \gamma(v_i)\gamma(v_j) = \gamma(v_j)\]
by the hypotheses.
\end{proof}

\begin{prop}\label{gamma prop}There is no non-trivial $\Aut^c(\pi)$-equivariant homomorphism from $V$ to $\Aut^c(\pi)$.
\end{prop}

\begin{proof}Let $\gamma : V\rightarrow\Aut^c(\pi)$ be an $\Aut^c(\pi)$-equivariant homomorphism and observe that $\gamma(V)$ must be a normal $p$-subgroup of $\Aut^c(\pi)$. But
 \[ Q \simeq \begin{cases}
\mathbb{F}_p^\times\times \mathbb{F}_p^\times &\mbox{in case (a)}\\
\GL_2(\mathbb{F}_p)&\mbox{in cases (b) and (e)}\\
\mathbb{F}_p^\times \times \GL_2(\mathbb{F}_p)&\mbox{in cases (c) and (d)}
\end{cases}\]
has no non-trivial normal $p$-subgroup. Since $\Aut^c(\pi) = P\rtimes Q$, we see that $\gamma(V)$ must lie inside $P$. We now deal with each case separately.
\begin{enumerate}[label=(\alph*), wide=0pt]
\item It is clear from Proposition \ref{auto1'} that
\[ v_1^{\alpha_{12}} = v_1v_2\]
for some $\alpha_{12} \in P$, and so it is enough to show that $\gamma(v_1)=\gamma(v_3)=1$ by Lemma \ref{gamma lemma}, it. Let us put
\[ \gamma(v_1) = \begin{bmatrix}
1 & b_1 & 0 \\
0 & 1 & 0 \\
0 & c_1 & 1
\end{bmatrix}\mbox{ and }\gamma(v_3)= \begin{bmatrix}
1 & b_3 & 0 \\
0 & 1 & 0 \\
0 & c_3 & 1
\end{bmatrix} \]
From $\gamma(v_1^\alpha) = \gamma(v_1)^\alpha$ for $\alpha\in Q$ of the shape
\[ \alpha = \begin{bmatrix}s & 0 & 0 \\
0 & 1 & 0\\
0 & 0 & s\end{bmatrix} \mbox{ with } s\in \mathbb{F}_p^\times,\]
we get that $\gamma(v_1^\alpha) = \gamma(v_1)^s$ and
\[  \begin{bmatrix}
1 & sb_1 & 0 \\
0 & 1 & 0 \\
0 & sc_1 & 1
\end{bmatrix}= \begin{bmatrix}
1 & s^{-1}b_1 & 0 \\
0 & 1 & 0 \\
0 & s^{-1}c_1 & 1
\end{bmatrix}.\]
Since $p\geq 5$, there exists $s\in \mathbb{F}_p^\times$ with $s^2\neq 1$, and so $b_1=c_1=0$. We may obtain $b_3 = c_3 =0$ by the exact same calculation.
\item It is clear from Proposition \ref{auto2'} that
\[ v_1^{\alpha_{12}} = v_1v_2\mbox{ and } v_1^{\alpha_{13}} = v_1v_3\]
for some $\alpha_{12},\alpha_{13} \in P$, so it suffices to show that $\gamma(v_1)=1$ by Lemma \ref{gamma lemma}. Let us put
\[ \gamma(v_1) = \begin{bmatrix}1 & b_1 & c_1\\0& 1 & 0 \\ 0 & 0 & 1\end{bmatrix}.\]
From $\gamma(v_1^\alpha) = \gamma(v_1)^\alpha$ for $\alpha\in Q$ of the shape
\[\begin{bmatrix}
1 & 0 & 0\\
0 & s & 0\\
0 & 0 &s^{-1}
\end{bmatrix}\mbox{ with }s\in\mathbb{F}_p^\times,\]
we get that $\gamma(v_1^\alpha) = \gamma(v_1)$ and
\[  \begin{bmatrix}1 & b_1 & c_1\\0& 1 & 0 \\ 0 & 0 & 1\end{bmatrix}
=  \begin{bmatrix}1 & sb_1 & s^{-1}c_1\\0& 1 & 0 \\ 0 & 0 & 1\end{bmatrix}.\]
This yields $b_1=c_1=0$.
\item It is clear from Proposition \ref{auto1} that
\[ v_1^{\alpha_{12}} = v_1v_2\mbox{ and }v_3^{\alpha_{34}} = v_3v_4\]
for some $\alpha_{12}\in P, \alpha_{34}\in Q$, so it suffices to show that $\gamma(v_1)=\gamma(v_3)=1$ by Lemma \ref{gamma lemma}. Let us put
\[ \gamma(v_1) = \begin{bmatrix}
1 & b_1 & 0 & 0\\
0 & 1 & 0 & 0\\
0 & c_1 & 1 & 0\\
0 & d_1 & 0 & 1
\end{bmatrix}\mbox{ and }
 \gamma(v_3) = \begin{bmatrix}
1 & b_3 & 0 & 0\\
0 & 1 & 0 & 0\\
0 & c_3 & 1 & 0\\
0 & d_3 & 0 & 1
\end{bmatrix}.\]
From $\gamma(v_1^\alpha) = \gamma(v_1)^\alpha$ for $\alpha\in  Q$ of the shape
\[ \alpha =\begin{bmatrix}
s & 0 & 0 &0\\
0 & 1 & 0 & 0\\
0 & 0 & s & 0\\
0 & 0 & 0 & s
\end{bmatrix} \mbox{ with } s\in \mathbb{F}_p^\times,\]
we get that $\gamma(v_1^\alpha) = \gamma(v_1)^s$ and
\[ \begin{bmatrix}
1 & sb_1 & 0 & 0\\
0 & 1 & 0 & 0\\
0 & sc_1 &1 & 0\\
0 & sd_1 & 0 & 1
\end{bmatrix} = \begin{bmatrix}
1 & s^{-1}b_1 & 0 & 0\\
0 & 1 & 0 & 0\\
0 & s^{-1}c_1 &1 & 0\\
0 & s^{-1}d_1 & 0 & 1
\end{bmatrix} .\]
Since $p\geq 5$, there exists $s\in \mathbb{F}_p^\times$ with $s^2\neq 1$, and so $b_1=c_1=d_1=0$. We may obtain $b_3=c_3=d_3=0$ by the exact same calculation.
\item It is clear from Proposition \ref{auto2} that
\[ v_1^{\alpha_{12}} = v_1v_2,\,\ v_1^{\alpha_{13}} = v_1v_3,\,\ v_1^{\alpha_{14}} = v_1v_4.\]
for some $\alpha_{12},\alpha_{13},\alpha_{14}\in P$, and so it suffices to show that $\gamma(v_1)=1$ by Lemma \ref{gamma lemma}. Let us put
\[ \gamma(v_1) = \begin{bmatrix}
1 & b_1 & c_1 & e_1\\
0 & 1 & d_1 & f_1\\
0 & 0 & 1 & 0\\
0 & 0 & 0 & 1
\end{bmatrix}.\]
From $\gamma(v_1^\alpha) = \gamma(v_1)^\alpha$ for $\alpha\in \Aut^c(\pi)$ of the shape
\[ \alpha =\begin{bmatrix}
1 & 0 & 0 & 0\\
0 & 1 & g & 0\\
0 & 0 & s & 0\\
0 & 0 & 0 & s^{-1}
\end{bmatrix} \mbox{ with  } s\in \mathbb{F}_p^\times\mbox{ and }g\in \mathbb{F}_p,\]
we get that $\gamma(v_1^\alpha) = \gamma(v_1)$ and
\[ \begin{bmatrix}
1 & b_1 & c_1 & e_1\\
0 & 1 & d_1 & f_1\\
0 & 0 &1 & 0\\
0 & 0 & 0 & 1
\end{bmatrix} = \begin{bmatrix}
1 & b_1 & gb_1 + sc_1 & s^{-1}e_1\\
0 & 1 & sd_1 & s^{-1}f_1\\
0 & 0 & 1 & 0\\
0 & 0 & 0 &1 
\end{bmatrix}.\]
This yields $b_1 = c_1 = d_1=e_1=f_1=0$. 
\item It is clear from Proposition \ref{auto3} that
\[v_1^{\alpha_{12}} = v_1v_2,\,\ v_1^{\alpha_{13}} = v_1v_3,\,\ v_1^{\alpha_{14}} = v_1v_4\]
for some $\alpha_{12},\alpha_{13},\alpha_{14}\in P$, and so it suffices to show that $\gamma(v_1)=1$ by Lemma \ref{gamma lemma}. Let us put
\[ \gamma(v_1) = \begin{bmatrix}
1 & b_1 & -d_1 & c_1\\
0 & 1 & 0 & 0\\
0 & c_1 & 1 & 0\\
0 & d_1 & 0 & 1
\end{bmatrix}.\]
From $\gamma(v_1^\alpha) = \gamma(v_1)^\alpha$ for $\alpha\in Q$ of the shape
\[ \alpha =\begin{bmatrix}
s& 0 & 0 & 0\\
0 & 1 & 0 & 0\\
0 & 0 & s & 0\\
0 & 0 & 0 &1
\end{bmatrix} \mbox{ with } s\in \mathbb{F}_p^\times,\]
we get that $\gamma(v_1^\alpha) = \gamma(v_1)^{s}$ and
\[\begin{bmatrix}
1 & sb_1 & -sd_1 & sc_1\\
0 & 1 & 0 & 0\\
0 & sc_1 & 1 & 0\\
0 & sd_1 & 0 & 1
\end{bmatrix}= \begin{bmatrix}
1 & s^{-1}b_1 & -d_1 & s^{-1}c_1\\
0 & 1 & 0 & 0\\
0 & s^{-1}c_1 & 1 & 0\\
0 & d_1 & 0 & 1\end{bmatrix}.\]
This implies that $d_1=0$. Since $p\geq 5$, there exists $s\in \mathbb{F}_p^\times$ with $s^2\neq 1$, and we see that $b_1=c_1=0$ as well.
\end{enumerate}
In all cases, we have shown that $\gamma$ is trivial.
 \end{proof}
 
 Therefore, we may apply Theorem \ref{pre thm} to obtain
 \begin{equation}\label{T(G)} T(G) \simeq S \rtimes \res(\mathcal{S}').\end{equation}
It remains to determine the structure of $S$ and $\res(\mathcal{S}')$.

 \subsection{A module-theoretic approach} 
 
Observe that by the universal property of $S^2V$, the symmetric square of $V$, there is a natural correspondence between
\begin{itemize}
\item symmetric bilinear forms $V\times V\rightarrow\Lambda^2V$,
\item linear maps $S^2V\rightarrow \Lambda^2V$.
\end{itemize}
Similarly, there is a natural correspondence between
\begin{itemize}
\item anti-symmetric bilinear forms $V\times V\rightarrow\Lambda^2V$,
\item linear maps $\Lambda^2V\rightarrow \Lambda^2V$.
\end{itemize}
Since we are writing addition in $V$ multiplicatively, let us denote multiplication in $S^2V$ by $*$ to avoid confusion. Then, both $S^2V$ and $\Lambda^2V$ are naturally $\Aut^c(\pi)$-modules via the action
\[ (u* v)^{\alpha} = u^\alpha * v^\alpha\mbox{ and }(u\wedge v)^\alpha = u^\alpha \wedge v^\alpha\]
for all $u,v\in V$ and $\alpha\in \Aut^c(\pi)$. Taking (\ref{Delta3}) into account, it follows that elements of $S$ and $S'$, respectively, correspond to $\Aut^c(\pi)$-module homomorphisms $S^2V\rightarrow \Lambda^2V$ and $\Lambda^2V\rightarrow\Lambda^2V$.

Let us first restrict the action to $Q$. An $\Aut^c(\pi)$-module homomorphism is in particular a $Q$-module homomorphism. The latter is easier to understand because matrices in $Q$ are all block diagonal, and so we easily see that both $S^2V$ and $\Lambda^2V$, as $Q$-modules, are decomposable as a direct sum of irreducible submodules. In the tables below, we list a basis for each irreducible component, and we indicate the action of an arbitrary $\alpha\in Q$ in matrix form with respect to the given basis. Here
\[ \alpha = \begin{bmatrix} s & 0 & 0 \\ 0 & 1 & 0 \\ 0 & 0 &t\end{bmatrix},\begin{bmatrix}
|A| &  \begin{matrix} 0 & 0 \end{matrix}\\
 \begin{matrix} 0 \\ 0 \end{matrix} & A
\end{bmatrix}\]
in cases (a),(b), respectively, while 
\[ \alpha = \begin{bmatrix}
s & 0 & \begin{matrix} 0 & 0 \end{matrix}\\
0 & 1 & \begin{matrix} 0 & 0 \end{matrix}\\
\begin{matrix} 0 \\ 0 \end{matrix} & \begin{matrix} 0 \\ 0 \end{matrix} & A
\end{bmatrix},\begin{bmatrix}
|A| & 0 & \begin{matrix} 0 & 0 \end{matrix}\\
0 & s & \begin{matrix} 0 & 0 \end{matrix}\\
\begin{matrix} 0 \\ 0 \end{matrix} & \begin{matrix} 0 \\ 0 \end{matrix} & A
\end{bmatrix},\begin{bmatrix}
|A| & 0 & \begin{matrix} 0 & 0 \end{matrix}\\
0 & 1 & \begin{matrix} 0 & 0 \end{matrix}\\
\begin{matrix} 0 \\ 0 \end{matrix} & \begin{matrix} 0 \\ 0 \end{matrix} & A
\end{bmatrix}\]
in cases (c),(d),(e), respectively. The variables $s,t$ here range over $\mathbb{F}_p^\times$, and $A$ ranges over $\GL_2(\mathbb{F}_p)$.

 \begingroup
\setlength{\tabcolsep}{10pt} % Default value: 6pt
\renewcommand{\arraystretch}{1.15}
%\captionof{table}{}
 \begin{center}
  \begin{longtable}{ |c|c|}
  \multicolumn{2}{c}{Case (a)}\\
 \hline
 \hline
\multicolumn{2}{|c|}{Components of $S^2V$} \\
\hline
 Basis & Action of $\alpha\in Q$\\ \hline
 $v_1*v_1 $ & $s^2$ \\ 
 $v_1*v_2$ & $s$ \\ 
 $v_1*v_3$ & $st$ \\ 
 $v_2*v_2$ & $1$\\
 $v_2*v_3$ & $t$ \\
 $v_3*v_3$ & $t^2$\\
\hline\hline
\multicolumn{2}{|c|}{Components of $\Lambda^2V$}\\
\hline
 Basis & Action of $\alpha\in Q$ \\ \hline
 $v_1\wedge v_2 $ & $s$\\ 
 $v_1\wedge v_3$ & $st$  \\ 
 $v_2\wedge v_3$ & $t$\\ 
\hline
\end{longtable} 
 \begin{longtable}{ |c|c|}
  \multicolumn{2}{c}{Case (b)}\\
 \hline
 \hline
\multicolumn{2}{|c|}{Components of $S^2V$}\\
\hline
 Basis & Action of $\alpha\in Q$  \\ \hline
 $v_1*v_1 $ & $|A|^2$ \\ 
 $v_1*v_2,v_1*v_3$ & $|A|A$  \\ 
 $v_2*v_2, v_2*v_3,v_3*v_3$ & omitted  \\ 
\hline
\hline
\multicolumn{2}{|c|}{Components of $\Lambda^2V$}\\
\hline
 Basis & Action of $\alpha\in Q$ \\ \hline
 $v_1\wedge v_2 ,v_1\wedge v_3$ & $|A|A$  \\ 
 $v_2\wedge v_3$ & $|A|$  \\ 
\hline
\end{longtable}
 \begin{longtable}{ |c|c| }
  \multicolumn{2}{c}{Case (c)}\\
 \hline
 \hline
\multicolumn{2}{|c|}{Components of $S^2V$}\\
\hline
 Basis & Action of $\alpha\in Q$ \\ \hline
 $v_1*v_1 $ & $s^2$ \\ 
 $v_1*v_2$ & $s$ \\ 
 $v_1*v_3,v_1*v_4$ & $sA$  \\ 
 $v_2*v_2$ & $1$ \\
 $v_2*v_3,v_2*v_4$ & $A$ \\
 $v_3*v_3,v_3*v_4,v_4*v_4$ & omitted \\
\hline
\hline
\multicolumn{2}{|c|}{Components of $\Lambda^2V$} \\
\hline
 Basis & Action of $\alpha\in Q$  \\ \hline
 $v_1\wedge v_2 $ & $s$  \\ 
 $v_1\wedge v_3,v_1\wedge v_4$ & $sA$  \\ 
 $v_2\wedge v_3,v_2 \wedge v_4$ & $A$  \\ 
 $v_3\wedge v_4$ & $|A|$ \\
\hline
\end{longtable}
%\captionof{table}{The case when $v_1^\pi = v_1\wedge v_2$}\label{a sym}
 \begin{longtable}{ |c|c| }
 \multicolumn{2}{c}{Case (d)}\\
 \hline
 \hline
\multicolumn{2}{|c|}{Components of $S^2V$}\\
\hline 
Basis & Action of $\alpha\in Q$ \\ \hline
 $v_1*v_1 $ & $|A|^2$ \\ 
 $v_1*v_2$ & $s|A|$  \\ 
 $v_1*v_3,v_1*v_4$ & $|A|A$  \\ 
 $v_2*v_2$ & $s^2$\\
 $v_2*v_3,v_2*v_4$ & $sA$  \\
 $v_3*v_3,v_3*v_4,v_4*v_4$ & omitted \\
\hline
\hline
\multicolumn{2}{|c|}{Components of $\Lambda^2V$}\\
\hline 
Basis & Action of $\alpha\in Q$  \\ \hline
 $v_1\wedge v_2 $ & $s|A|$ \\ 
 $v_1\wedge v_3,v_1\wedge v_4$ & $|A|A$  \\ 
 $v_2\wedge v_3,v_2 \wedge v_4$ & $sA$  \\ 
 $v_3\wedge v_4$ & $|A|$ \\
\hline
\end{longtable}
%\captionof{table}{The case when $v_1^\pi = v_3\wedge v_4$}\label{a sym}
 \begin{longtable}{ |c|c| }
 \multicolumn{2}{c}{Case (e)}\\
 \hline
 \hline
\multicolumn{2}{|c|}{Components of $S^2V$}\\
\hline 
Basis & Action of $\alpha\in Q$ \\ \hline
 $v_1*v_1 $ & $|A|^2$ \\ 
 $v_1*v_2$ & $|A|$  \\ 
 $v_1*v_3,v_1*v_4$ & $|A|A$  \\ 
 $v_2*v_2$ & $1$ \\
 $v_2*v_3,v_2*v_4$ & $A$  \\
 $v_3*v_3,v_3*v_4,v_4*v_4$ & omitted \\
\hline
\hline
\multicolumn{2}{|c|}{Components of $\Lambda^2V$}\\
\hline 
Basis & Action of $\alpha\in Q$  \\ \hline
 $v_1\wedge v_2 $ & $|A|$ \\ 
 $v_1\wedge v_3,v_1\wedge v_4$ & $|A|A$  \\ 
 $v_2\wedge v_3,v_2 \wedge v_4$ & $A$  \\ 
 $v_3\wedge v_4$ & $|A|$  \\
\hline
\end{longtable}
%\captionof{table}{The case when $v_1^\pi = v_3\wedge v_4$}\label{a sym}
\end{center} 
\endgroup

\vspace{-0.55cm}
 
Under a $Q$-module homomorphism, an irreducible component of the domain either lies in the kernel or gets mapped to an isomorphic irreducible component of the codomain. From the stated action of $Q$, we can easily compare the isomorphism classes of the irreducible components of $S^2V$ and $\Lambda^2V$. Note that the omitted action does not matter because $\Lambda^2V$ does not have any $3$-dimensional irreducible component. The next two propositions are then immediate. 
 
 \begin{prop}\label{prelim prop sym}For any $\Delta\in S$, the following holds.
 \begin{enumerate}[label= $(\arabic*)$]
 \item In case (a), we have
\begin{align*}\Delta(v_1,v_1)&=1,\\
\Delta(v_2,v_2) &=1,\\
 \Delta(v_3,v_3)&=1.
\end{align*}
\item In case (b), we have
\begin{align*}
\Delta(v_1,v_1)& = 1,\\
\Delta(v_2,v_2) &= \Delta(v_2,v_3) =\Delta(v_3,v_3)=1.
\end{align*}
\item In cases (c),(d), and (e), we have
\begin{align*}
\Delta(v_1,v_1) &=1,\\
 \Delta(v_2,v_2) &=1,\\
 \Delta(v_3,v_3) &= \Delta(v_3,v_4)=\Delta(v_4,v_4) =1.
 \end{align*}
 \end{enumerate}
 \end{prop}
 
 \begin{prop}\label{prelim prop anti}
For any $\Delta\in S'$, the following holds.
 \begin{enumerate}[label= $(\arabic*)$]
\item In case (a), we have
\begin{align*}
\Delta(v_1,v_2) & \in \langle v_1\wedge v_2\rangle,\\
\Delta(v_1,v_3) & \in \langle v_1\wedge v_3\rangle,\\
\Delta(v_2,v_3) & \in \langle v_2\wedge v_3\rangle.
\end{align*}
\item In case (b), we have
\begin{align*}
\Delta(v_1,v_2),\Delta(v_1,v_3)& \in \langle v_1\wedge v_2,v_1\wedge v_3\rangle,\\
\Delta(v_2,v_3) & \in \langle v_2\wedge v_3\rangle.
\end{align*}
\item In cases (c) and (d), we have
\begin{align*}
 \Delta(v_1,v_2) & \in \langle v_1\wedge v_2\rangle,\\\
 \Delta(v_1,v_3),\Delta(v_1,v_4) & \in \langle v_1\wedge v_3, v_1\wedge v_4 \rangle,\\
 \Delta(v_2,v_3),\Delta(v_2,v_4) & \in \langle v_2\wedge v_3, v_2\wedge v_4 \rangle, \\
 \Delta(v_3,v_4) & \in \langle v_3\wedge v_4\rangle.
\end{align*}
\item In case (e), we have
\begin{align*}
 \Delta(v_1,v_2),\Delta(v_3,v_4) & \in \langle v_1\wedge v_2, v_3\wedge v_4\rangle,\\
 \Delta(v_1,v_3),\Delta(v_1,v_4) & \in \langle v_1\wedge v_3, v_1\wedge v_4 \rangle,\\
 \Delta(v_2,v_3),\Delta(v_2,v_4) & \in \langle v_2\wedge v_3, v_2\wedge v_4 \rangle
\end{align*}
\end{enumerate}  
 \end{prop}
  
We may refine parts of Proposition \ref{prelim prop anti} as follows.

 \begin{prop}\label{scalar prop} For any $\Delta\in S'$, the following holds.
 \begin{enumerate}[label= $(\arabic*)$]
 \item In case (b), there exists $\lambda\in\mathbb{F}_p$ such that
 \[ \begin{cases}
 \Delta(v_1,v_2) = (v_1\wedge v_2)^\lambda,\\
 \Delta(v_1,v_3) = (v_1\wedge v_3)^\lambda.
 \end{cases}\]
 \item In cases (c),(d), and (e), there exist $\lambda_1,\lambda_2\in \mathbb{F}_p$ such that
\[\begin{cases}
\Delta(v_1,v_3) = (v_1\wedge v_3)^{\lambda_1} \\
\Delta(v_1,v_4) = (v_1\wedge v_4)^{\lambda_1}
\end{cases}\,\
\begin{cases}
\Delta(v_2,v_3) = (v_2\wedge v_3)^{\lambda_2},\\
\Delta(v_2,v_4) = (v_2\wedge v_4)^{\lambda_2}.
\end{cases}\]
 \end{enumerate}
  \end{prop}
 
 \begin{proof} Consider case (b). We know from Proposition \ref{prelim prop anti} that $\Delta$ has to induce a $Q$-module endomorphism 
 \[ \delta : \langle v_1\wedge v_2,v_1\wedge v_3\rangle \rightarrow \langle v_1\wedge v_2,v_1\wedge v_3 \rangle.\]
If $\delta$ is trivial, then simply take $\lambda=0$. If $\delta$ is non-trivial, then it has to be invertible because $\langle v_1\wedge v_2,v_1\wedge v_3\rangle$ is irreducible. Say $\delta$ is given by the matrix $M\in \GL_2(\mathbb{F}_p)$. But $M$ must commute with the action of $Q$ and observe that $Q$ restricts to an $\SL_2(\mathbb{F}_p)$-action on $\langle v_1\wedge v_2,v_1\wedge v_3\rangle$.
Since the only matrices that centralize $\SL_2(\mathbb{F}_p)$ are the scalar multiples of the identity, it follows that $M = \left[\begin{smallmatrix} \lambda& 0\\ 0 & \lambda\end{smallmatrix}\right]$
for some $\lambda\in\mathbb{F}_p^\times$. The proves (1), and the same argument may be applied to prove (2).\end{proof}

\subsection{Computation of $S$ and $S'$} We shall now compute $S$ and $S'$ by taking the action of $P$ into account.

First, notice that a symmetric bilinear form $\Delta : V\times V\rightarrow \Lambda^2V$ is uniquely determined by
  \[ \Delta(v_i,v_j)\mbox{ for }1\leq i \leq j \leq n.\]
The next observation shall also be useful.

\begin{lemma}\label{sym lemma}Let $\Delta \in S$ and let $1\leq i, j \leq n$. If
\begin{enumerate}[label = $(\arabic*)$]
\item $\Delta(v_i,v_i) = \Delta(v_j,v_j)=1$,
\item  $v_i^{\alpha }= v_iv_j$ for some $\alpha\in \Aut^c(\pi)$,
\end{enumerate}
then $\Delta(v_i,v_j) =\Delta(v_j,v_i)= 1$ also holds.
\end{lemma}

\begin{proof}By the hypothesis and the condition (\ref{Delta3}), we have
\begin{align*}
1 & = \Delta(v_i,v_i)^{\hat{\alpha}}\\
& = \Delta(v_i^{\alpha} ,v_i^{\alpha})\\
& = \Delta(v_iv_j,v_iv_j)\\
& = \Delta(v_i,v_i)\Delta(v_i,v_j)\Delta(v_j,v_i)\Delta(v_j,v_j)\\
&=\Delta(v_i,v_j)\Delta(v_j,v_i)\\
& = \Delta(v_i,v_j)^2,
\end{align*}
where the last equality holds because $\Delta$ is symmetric. Since $p$ is odd, we may take the square root and so $\Delta(v_i,v_j)=\Delta(v_j,v_i)=1$.
\end{proof}

\begin{prop}\label{S=1} We have $S=1$ in all cases (a),(b),(c),(d), and (e).
\end{prop}

\begin{proof}Let $\Delta\in S$ be arbitrary. We consider each case separately.
\begin{enumerate}[label=(\alph*),wide=0pt]
\item It is clear from Proposition \ref{auto1'} that
\[ v_1^{\alpha_{12}} = v_1v_2\mbox{ and } v_3^{\alpha_{23}} = v_2v_3\]
for some $\alpha_{12},\alpha_{23}\in P$. We then have
\[ \Delta(v_i,v_j) = 1\mbox{ for all }1\leq i \leq j\leq 3\mbox{ with }(i,j)\neq (1,3) \]
by Proposition \ref{prelim prop sym} and Lemma \ref{sym lemma}. Comparing the irreducible components of $S^2V$ and $\Lambda^2V$ as $Q$-modules, we also see that
\[ \Delta(v_1,v_3) = (v_1\wedge v_3)^\lambda\]
for some $\lambda\in\mathbb{F}_p$. But consider the action of $\alpha\in P$ given by
\[ \alpha = \begin{bmatrix} 1 & 1 & 0 \\ 0 & 1 & 0 \\ 0 & 1 & 1\end{bmatrix}.\]
By the condition (\ref{Delta3}), we have
\begin{align*}
\Delta(v_1,v_3)^{\hat{\alpha}}  & = \Delta(v_1^\alpha,v_3^\alpha)\\
& = \Delta(v_1v_2,v_2v_3)\\
& = \Delta(v_1,v_2)\Delta(v_1,v_3)\Delta(v_2,v_2)\Delta(v_2,v_3)\\
& = \Delta(v_1,v_3).
\end{align*}
But the left hand side is equal to
\[(v_1v_2\wedge v_2v_3)^\lambda =  (v_1\wedge v_2)^\lambda  (v_2\wedge v_3)^\lambda\Delta(v_1,v_3).\]
It follows that $\lambda=0$ and so $\Delta(v_1,v_3)=1$ also holds.
\item It is clear from Proposition \ref{auto2'} that
\[ v_1^{\alpha_{12}} = v_1v_2\mbox{ and } v_1^{\alpha_{13}} = v_1v_3\]
for some $\alpha_{12},\alpha_{13}\in P$. We then have
\[ \Delta(v_i,v_j) = 1\mbox{ for all }1\leq i \leq j\leq 3 \]
by Proposition \ref{prelim prop sym} and Lemma \ref{sym lemma}. 
\item It is clear from Proposition \ref{auto1} that
\[ v_1^{\alpha_{12}} = v_1v_2,\, 
v_3^{\alpha_{23}} = v_2v_3,\, v_4^{\alpha_{24}} = v_2v_4\]
for some $\alpha_{12},\alpha_{23},\alpha_{24}\in P$. We then have
 \[ \Delta(v_i,v_j)=1\mbox{ for all }1\leq i \leq j \leq 4 \mbox{ with }(i,j)\not\in\{(1,3),(1,4)\}\]
by Proposition \ref{prelim prop sym} and Lemma \ref{sym lemma}. Comparing the irreducible components of $S^2V$ and $\Lambda^2V$ as $Q$-modules, we also see that
\[ \Delta(v_1,v_3),\Delta(v_1,v_4)\in \langle v_1\wedge v_3,v_1\wedge v_4\rangle\]
has to hold. Let us write
\[ \Delta(v_1, v_3) = (v_1\wedge v_3)^{\lambda}(v_1\wedge v_4)^{\kappa},\]
and consider the action of $\alpha_1\in P$ defined by
\[\alpha_1 =  \begin{bmatrix}1 & 1 & 0 & 0 \\
0 & 1 & 0 & 0\\
 0& 1 & 1 & 0\\
 0 & 0 & 0 & 1\end{bmatrix}.
 \]
Since $\Delta$ satisfies the condition (\ref{Delta3}), we get that
\begin{align*}
\Delta(v_1,v_3)^{\hat{\alpha}_1}
& = \Delta(v_1^{\alpha_1},v_3^{\alpha_1})\\
& = \Delta(v_1v_2,v_2v_3)\\
& =\Delta(v_1,v_2)\Delta(v_1,v_3)\Delta(v_2,v_2)\Delta(v_2,v_3)\\
& = \Delta(v_1,v_3).\end{align*}
But explicitly, the left hand side is given by
\[(v_1v_2\wedge v_2v_3)^\lambda (v_1v_2\wedge v_4)^{\kappa}  = (v_1\wedge v_2)^{\lambda} (v_2\wedge v_3)^\lambda(v_2\wedge v_4)^\kappa\Delta(v_1,v_3).\]
This shows that $\lambda = \kappa = 0$ and hence $\Delta(v_1,v_3) =1$. Since there exists $\alpha_2\in Q$ for which $v_1^{\alpha_2} = v_1$ and $v_3^{\alpha_2} = v_4$, we have
\[ 1 = \Delta(v_1,v_3)^{\hat{\alpha}_2} = \Delta(v_1^{\alpha_2},v_3^{\alpha_2} ) = \Delta(v_1,v_4).\]
We have thus shown that $\Delta(v_1,v_3) = \Delta(v_1,v_4)=1$ also holds.
\item It is clear from Proposition \ref{auto2} that 
\[ v_1^{\alpha_{12}} = v_1v_2,\,
v_1^{\alpha_{13}} = v_1v_3,\,
v_1^{\alpha_{14}} = v_1v_4,\,
v_2^{\alpha_{23}} = v_2v_3,\,
v_2^{\alpha_{24}} = v_2v_4\]
for some $\alpha_{12},\alpha_{13},\alpha_{14},\alpha_{23},\alpha_{24}\in P$. We then have
\[ \Delta(v_i,v_j) = 1\mbox{ for all }1\leq i \leq j\leq 4 \]
by Proposition \ref{prelim prop sym} and Lemma \ref{sym lemma}.  
\item It is clear from Proposition \ref{auto3} that
\[ v_1^{\alpha_{12}} = v_1v_2,\,
v_1^{\alpha_{13}} = v_1v_3,\,
v_1^{\alpha_{14}} = v_1v_4,\,
v_3^{\alpha_{23}} = v_2v_3,\,
v_4^{\alpha_{24}}= v_2v_4\]
for some $\alpha_{12},\alpha_{13},\alpha_{14},\alpha_{23},\alpha_{24}\in P$. We then have
\[ \Delta(v_i,v_j) = 1\mbox{ for all }1\leq i \leq j\leq 4 \]
by Proposition \ref{prelim prop sym} and Lemma \ref{sym lemma}.  
\end{enumerate}
In all cases, we have shown that $\Delta=1$, and so indeed $S=1$.
  \end{proof}
 
Next, note that an anti-symmetric bilinear form $\Delta :V \times V \rightarrow \Lambda^2V$ is uniquely determined by
\[ \Delta(v_i,v_j) \mbox{ for }1\leq i < j\leq n.\]
We also make the following observation.

\begin{lemma}\label{anti lemma}
Let $\Delta\in S'$ and let $1\leq i,j,k\leq n$ with $i\neq j,k$. If
\begin{enumerate}[label = $(\arabic*)$]
\item $\Delta(v_i,v_j) = (v_i\wedge v_j)^{\lambda_1}$ or equivalently $\Delta(v_j,v_i) = (v_j\wedge v_i)^{\lambda_1}$,
\item $\Delta(v_i,v_k) = (v_i\wedge v_k)^{\lambda_2}$ or equivalently $\Delta(v_k,v_i) = (v_k\wedge v_i)^{\lambda_2}$,
\item $v_i^\alpha = v_i,\, v_j^\alpha =v_jv_k$ for some $\alpha\in \Aut^c(\pi)$,\end{enumerate}
then $\lambda_1 = \lambda_2$ has to hold.
\end{lemma}

%Note that the equivalence holds because $\Delta$ is anti-symmetric.

\begin{proof}By the condition (\ref{Delta3}), we have
\[
\Delta(v_i,v_j)^{\hat{\alpha}} = \Delta(v_i^\alpha,v_j^\alpha)
= \Delta(v_i,v_jv_k)
=\Delta(v_i,v_j)\Delta(v_i,v_k). \]
Using the hypothesis, we rewrite this as
\[ (v_i\wedge v_j)^{\lambda_1}(v_i\wedge v_k)^{\lambda_1} = (v_i\wedge v_j)^{\lambda_1}( v_i\wedge v_k)^{\lambda_2},\] 
which implies that $\lambda_1 =\lambda_2$, as claimed.
\end{proof}

For each $\lambda\in\mathbb{F}_p$, as noted in Remark \ref{remark}, clearly
\[ \Delta_{[\lambda]} : V \times V\rightarrow \Lambda^2V;\,\ \Delta_{[\lambda]}(u,v) = (u\wedge v)^\lambda\]
is an anti-symmetric bilinear form satisfying (\ref{Delta3}), namely $\Delta_{[\lambda]}\in S'$.

\begin{prop}\label{S' prop}We have
\[ S' =\begin{cases}
  \{ \Delta_{[\lambda]} : \lambda\in \mathbb{F}_p\}&\mbox{in cases (a),(b),(c), and (d)},\\
  \{ \Delta_{[\lambda]}\Delta_{[\kappa]}^* : \lambda,\kappa\in \mathbb{F}_p\}&\mbox{in case (e)},
  \end{cases} \] 
where $\Delta_{[\kappa]}^* : V\times V\rightarrow \Lambda^2V$ denotes the anti-symmetric form defined by
\begin{align*}
\Delta_{[\kappa]}^*(v_1,v_2) & = (v_3\wedge v_4)^\kappa,&\Delta_{[\kappa]}^*(v_2,v_3) & = (v_2\wedge v_3)^{-\kappa},\\
\Delta_{[\kappa]}^*(v_1,v_3) & = (v_1\wedge v_3)^{-\kappa},&\Delta_{[\kappa]}^*(v_2,v_4)& = (v_2\wedge v_4)^{-\kappa},\\
\Delta_{[\kappa]}^*(v_1,v_4)& = (v_1\wedge v_4)^{-\kappa},&\Delta_{[\kappa]}^*(v_3,v_4) & = (v_1\wedge v_2)^{\kappa}.\end{align*}
\end{prop}

\begin{proof}Let $\Delta\in S'$ be arbitrary. We consider each case separately.
\begin{enumerate}[label=(\alph*),wide=0pt]
\item[(a),(b)] By Propositions \ref{prelim prop anti} and \ref{scalar prop}, we know that
\begin{align*} \Delta(v_1,v_2) &= (v_1\wedge v_2)^{\lambda_1}\\ 
\Delta(v_1,v_3) &= (v_1\wedge v_3)^{\lambda_2}\\
\Delta(v_2,v_3) &= (v_2\wedge v_3)^{\lambda_3}
\end{align*}
for some $\lambda_1,\lambda_2,\lambda_3 \in \mathbb{F}_p$. In case (a), by Proposition \ref{auto1'}, we have
\[ \begin{cases}
v_1^{\alpha_{12}} = v_1\\
v_3^{\alpha_{12}} = v_2v_3
\end{cases}\,\ \begin{cases}
v_3^{\alpha_{23}} = v_3\\
v_1^{\alpha_{23}} = v_1v_2
\end{cases}\]
for some $\alpha_{12},\alpha_{23}\in P$. In case (b), we already know from Proposition  \ref{scalar prop} that $\lambda_1=\lambda_2$, and by Proposition \ref{auto2'}, we have
\[ \begin{cases}
v_3^{\alpha_{23}} = v_3\\
v_1^{\alpha_{23}} = v_1v_2
\end{cases}\]
for some $\alpha_{23}\in P$. In both cases, we get that
\[\lambda :=\lambda_1 = \lambda_2 = \lambda_3\]
by Lemma \ref{anti lemma}. This shows that $\Delta = \Delta_{[\lambda]}$, as claimed.
\item[(c),(d)] By Propositions \ref{prelim prop anti} and \ref{scalar prop}, we know that
\begin{align*} \Delta(v_1,v_2) &= (v_1\wedge v_2)^{\lambda_1}&\Delta(v_2,v_3) & = (v_2\wedge v_3)^{\lambda_3}\\ 
\Delta(v_1,v_3) &= (v_1\wedge v_3)^{\lambda_2} & \Delta(v_2,v_4) &= (v_2\wedge v_4)^{\lambda_3}\\
\Delta(v_1,v_4) &= (v_1\wedge v_4)^{\lambda_2}&\Delta(v_3,v_4) &= (v_3\wedge v_4)^{\lambda_4}
\end{align*}
for some $\lambda_1,\lambda_2,\lambda_3,\lambda_4 \in \mathbb{F}_p$. In case (c), by Proposition \ref{auto1}, we have
\[ \begin{cases}
v_1^{\alpha_{12}} = v_1\\
v_3^{\alpha_{12}} = v_2v_3
\end{cases}\,\
\begin{cases}
v_3^{\alpha_{23}} = v_3\\
v_1^{\alpha_{23}} = v_1v_2
\end{cases}
\,\
\begin{cases}
v_4^{\alpha_{34}} = v_4\\
v_3^{\alpha_{34}} = v_2v_3
\end{cases}\]
for some $\alpha_{12},\alpha_{23},\alpha_{34}\in P$. In case (d), by Proposition \ref{auto2}, we have
\[ \begin{cases}
v_1^{\alpha_{12}} = v_1\\
v_2^{\alpha_{12}} = v_2v_3
\end{cases}\,\
\begin{cases}
v_3^{\alpha_{23}} = v_3\\
v_1^{\alpha_{23}} = v_1v_2
\end{cases}
\,\
\begin{cases}
v_4^{\alpha_{34}} = v_4\\
v_2^{\alpha_{34}} = v_2v_3
\end{cases}\]
for some $\alpha_{12},\alpha_{23},\alpha_{34}\in P$. In both cases, we get that 
\[\lambda :=\lambda_1 = \lambda_2 = \lambda_3= \lambda_4\]
by Lemma \ref{anti lemma}. This shows that $\Delta = \Delta_{[\lambda]}$, as claimed.
\item[(e)] By Propositions \ref{prelim prop anti} and \ref{scalar prop}, we know that 
\begin{align*} \Delta(v_1,v_2) &= (v_1\wedge v_2)^{\lambda_1}(v_3\wedge v_4)^{\kappa_1}&\Delta(v_2,v_3) & = (v_2\wedge v_3)^{\lambda_3}\\ 
\Delta(v_1,v_3) &= (v_1\wedge v_3)^{\lambda_2} & \Delta(v_2,v_4) &= (v_2\wedge v_4)^{\lambda_3}\\
\Delta(v_1,v_4) &= (v_1\wedge v_4)^{\lambda_2}&\Delta(v_3,v_4) &= (v_1\wedge v_2)^{\kappa_4}(v_3\wedge v_4)^{\lambda_4}
\end{align*}
 for some $\lambda_1,\lambda_2,\lambda_3,\lambda_4,\kappa_1,\kappa_4\in \mathbb{F}_p$. Consider $\alpha\in P$ given by
\[ \alpha = \begin{bmatrix} 1 & 0 & 0 & 1\\
0 & 1 & 0 & 0\\
 0 & 1 & 1 & 0 \\ 
 0 & 0 & 0 & 1\end{bmatrix},\]
and we compute that
 \begin{align*}
\Delta(v_1,v_2)^{\hat{\alpha}}& = (v_1v_4\wedge v_2)^{\lambda_1}(v_2v_3\wedge v_4)^{\kappa_1} \\
&= \Delta(v_1,v_2)(v_4\wedge v_2)^{\lambda_1-\kappa_1},\\
\Delta(v_1^\alpha,v_2^\alpha) & = \Delta(v_1v_4,v_2) \\
&=\Delta(v_1,v_2)(v_4\wedge v_2)^{\lambda_3},\\
\Delta(v_1,v_3)^{\hat{\alpha}} & = (v_1v_4\wedge v_2v_3)^{\lambda_2} \\
&= \Delta(v_1,v_3)(v_1\wedge v_2)^{\lambda_2}(v_4\wedge v_2)^{\lambda_2}(v_4\wedge v_3)^{\lambda_2},\\
\Delta(v_1^\alpha,v_3^\alpha) & = \Delta(v_1v_4,v_2v_3) \\
&= \Delta(v_1,v_3)(v_1\wedge v_2)^{\lambda_1-\kappa_4}(v_4\wedge v_2)^{\lambda_3}(v_4\wedge v_3)^{\lambda_4-\kappa_1},\\
\Delta(v_3,v_4)^{\hat{\alpha}} & = (v_1v_4\wedge v_2)^{\kappa_4}(v_2v_3\wedge v_4)^{\lambda_4}\\
& = \Delta(v_3,v_4)(v_2\wedge v_4)^{\lambda_4-\kappa_4},\\
\Delta(v_3^\alpha,v_4^\alpha) & = \Delta(v_2v_3,v_4)\\
&= \Delta(v_3,v_4)(v_2\wedge v_4)^{\lambda_3}.
\end{align*}
Since the condition (\ref{Delta3}) has to hold, we deduce that
\[ \lambda_3= \lambda_1 - \kappa_1,\,\
\lambda_2 = \lambda_1-\kappa_4 =\lambda_3=\lambda_4-\kappa_1,\,\ \lambda_3 = \lambda_4-\kappa_4.\]
Solving this system of equations, we get that
\[ \lambda := \lambda_1 = \lambda_4 ,\,\ \kappa:=\kappa_1=\kappa_4,\,\ \lambda_2 =\lambda_3 = \lambda -\kappa.\]
This shows that $\Delta = \Delta_{[\lambda]}\Delta_{[\kappa]}^*$. Conversely, for any $\lambda,\kappa\in\mathbb{F}_p$, we know that $ \Delta_{[\lambda]}\in S'$ already and it is straightforward to check that $\Delta_{[\kappa]}^*$ also satisfies (\ref{Delta3}), so then $\Delta_{[\lambda]}\Delta_{[\kappa]}^*\in S'$.
 \end{enumerate}
 This completes the proof.
\end{proof}
   
\subsection{The structure of $T(G)$} We shall now prove Theorem \ref{thm1}. We already know from (\ref{T(G)}) and Proposition \ref{S=1} that
\[ T(G) \simeq \res(\mathcal{S}').\]
In cases (a),(b),(c), and (d), the theorem follows because we have
\[ \res(\mathcal{S}') \simeq \mathbb{F}_p^\times\]
by Remark \ref{remark} and Proposition \ref{S' prop}. In case (e), by Proposition \ref{S' prop}, the elements of $S'$  are precisely the bilinear forms
\[ \Delta_{[\sigma]}: V\times V\rightarrow\Lambda^2V ;\,\  \Delta_{[\sigma]}(u,v) = (u\wedge v)^\sigma.\]
Here $\sigma$ is any endomorphism on $\Lambda^2V$ of the form
\begin{equation}\label{tau}
 \begin{bmatrix}
\lambda & &  & &&\kappa\\
 & \lambda-\kappa & & &&\\
 & & \lambda-\kappa & & &\\
 & & & \lambda-\kappa & &\\
 & & & &\lambda-\kappa &\\
\kappa & & & &&\lambda
\end{bmatrix} \mbox{ with }\lambda,\kappa\in \mathbb{F}_p,\end{equation}
written with respect to the basis
\[ v_1\wedge v_2, v_1\wedge v_3, v_1\wedge v_4,v_2\wedge v_3,v_2\wedge v_4,v_3\wedge v_4\]
of $\Lambda^2V$. By \cite[Example 3.4]{LMH}, we know that $N_{\Delta_{[\sigma]}}\simeq G$ occurs only for $1+2\sigma\in \GL(\Lambda^2V)$. Let us make a change of variables $\tau = 1+2\sigma$, and consider $\tau_{\lambda,\kappa}\in \GL(\Lambda^2V)$ of the form (\ref{tau}) but with the restriction $\kappa\neq\pm\lambda$. Observe that then
\[
\eta_{\lambda,\kappa}   = \begin{bmatrix}
\lambda+\kappa &&&\\
&(\lambda+\kappa)^{-1} &&\\
&&1&\\
&&&1
\end{bmatrix},\]
written with respect to the basis $v_1,v_2,v_3,v_4$ of $V$, in which case
\[\hat{\eta}_{\lambda,\kappa} = \begin{bmatrix}
1 &&&&&\\
 &\lambda+\kappa&&&&\\
 &&\lambda+\kappa&&&\\
 &&&(\lambda+\kappa)^{-1}&&\\
 &&&&(\lambda+\kappa)^{-1}&\\
 &&&&&1
\end{bmatrix},\]
yields a solution to $\pi\hat{\eta}_{\lambda,\kappa}\tau_{\lambda,\kappa} = \eta_{\lambda,\kappa}\pi$. From (\ref{S'}), we deduce that
\[ \res(\mathcal{S}') \simeq \{(\eta_{\lambda,\kappa},\hat{\eta}_{\lambda,\kappa}\tau_{\lambda,\kappa}) : \lambda,\kappa\in\mathbb{F}_p\mbox{ with }\kappa\neq\pm\lambda\}.  \]
It is straightforward to verify that
\[ \eta_{\lambda_{1},\kappa_1} \eta_{\lambda_{2},\kappa_{2}} = \eta_{\lambda,\kappa},\,\
 \hat{\eta}_{\lambda_1,\kappa_1}\tau_{\lambda_1,\kappa_1}\hat{\eta}_{\lambda_2,\kappa_2}\tau_{\lambda_2,\kappa_2} = \hat{\eta}_{\lambda,\kappa}\tau_{\lambda,\kappa}\]
 for any $\lambda_1,\lambda_2,\kappa_1,\kappa_2\in \mathbb{F}_p$ with $\kappa_1\neq\pm\lambda_1$ and $\kappa_2\neq\pm\lambda_2$, where
 \[\begin{bmatrix} \lambda & \kappa\\ \kappa &\lambda \end{bmatrix}= \begin{bmatrix} \lambda_1 &\kappa_1\\\kappa_1&\lambda_1\end{bmatrix}
 \begin{bmatrix}\lambda_2& \kappa_2\\ \kappa_2 & \lambda_2\end{bmatrix}
 =\begin{bmatrix} \lambda_1\lambda_2 + \kappa_1\kappa_2 & \lambda_1\kappa_2 +\lambda_2\kappa_1\\\lambda_1\kappa_2 +\lambda_2\kappa_1&\lambda_1\lambda_2 + \kappa_1\kappa_2 \end{bmatrix}.\]
It follows that $\res(\mathcal{S}')$ is isomorphic to the subgroup
\[ \left\{ \begin{bmatrix}\lambda & \kappa \\ \kappa & \lambda\end{bmatrix}  : \lambda,\kappa\in\mathbb{F}_p\mbox{ with }\kappa\neq\pm\lambda\right\}\]
of $\GL_2(\mathbb{F}_p)$, or conjugating by $\left[\begin{smallmatrix}1 & -1\\ 1 & 1 \end{smallmatrix}\right]$, the subgroup
\[ \left\{ \begin{bmatrix}\lambda + \kappa& 0 \\  0 & \lambda- \kappa\end{bmatrix}  : \lambda,\kappa\in\mathbb{F}_p\mbox{ with }\kappa\neq\pm\lambda\right\}\]
of $\GL_2(\mathbb{F}_p)$. This decomposes as
\[ \left\{ \begin{bmatrix} \lambda & 0 \\ 0 & 1 \end{bmatrix}: \lambda\in \mathbb{F}_p^\times \right\}\times  \left\{ \begin{bmatrix} 1 & 0 \\ 0 & \kappa \end{bmatrix}: \kappa\in \mathbb{F}_p^\times \right\}\]
and so is isomorphic to $\mathbb{F}_p^\times \times \mathbb{F}_p^\times$, as claimed in (e).
%We then have
%\begin{align*}
% \res(\mathcal{S'}) = & \{ (\eta,\hat{\eta}\tau)\Gamma(G) : \eta\in \GL(V),\, \tau\in \GL(\Lambda^2V)\\
% &\hspace{2.5cm}\mbox{of the shape (\ref{tau}) with $\kappa \neq \lambda,\pm2\lambda$,}\\
% &\hspace{2.5cm}\mbox{and the equation }\pi\hat{\eta}\tau = \eta\pi\mbox{ holds}\}
% \end{align*}
%by (\ref{res(S')}). Let us solve $\pi\hat{\eta}\tau = \eta\pi$ for such $\eta\in \GL(V)$ and $\tau\in \GL(\Lambda^2V)$ in a manner very similar to the proof of Proposition \ref{auto3}. Write 
%\begin{align*}
%v_1^\eta & = v_1^{n_{11}} v_2^{n_{12}} v_3^{n_{13}} v_4^{n_{14}}, \\
%v_2^\eta & = v_1^{n_{21}} v_2^{n_{22}} v_3^{n_{23}} v_4^{n_{24}},\\
%v_3^\eta & = v_1^{n_{31}} v_2^{n_{32}} v_3^{n_{33}} v_4^{n_{34}},\\
%v_4^\eta & = v_1^{n_{41}} v_2^{n_{42}} v_3^{n_{43}} v_4^{n_{44}}.
%\end{align*}
%Since $v_2^\pi =v_3^\pi = v_4^\pi = 1$, necessarily $n_{21} = n_{31} = n_{41} = 0$ and $n_{11}\neq0$. We may then simplify $v_1^{\pi\hat{\eta}\tau} = v_1^{\eta\pi}$ as
%\begin{align*}
%&((v_1^{n_{11}} v_2^{n_{12}} v_3^{n_{13}} v_4^{n_{14}} \wedge v_2^{n_{22}} v_3^{n_{23}} v_4^{n_{24}})( v_2^{n_{32}} v_3^{n_{33}} v_4^{n_{34}}\wedge v_2^{n_{42}} v_3^{n_{43}} v_4^{n_{44}}))^{\tau} \\
%&\hspace{7.25cm}= (v_1\wedge v_2)^{n_{11}} (v_3\wedge v_4)^{n_{11}}.\end{align*}
%Since $\langle v_1\wedge v_3\rangle$ and $\langle v_1\wedge v_4\rangle$
%are eigenspaces of $\tau$, which is taken to be invertible here, by comparing exponents, we see that $n_{23} = n_{24}=0$. and $n_{22}\neq0$. The above equation then becomes
%\begin{align*}
%&((v_1^{n_{11}} v_2^{n_{12}} v_3^{n_{13}} v_4^{n_{14}} \wedge v_2^{n_{22}})( v_2^{n_{32}} v_3^{n_{33}} v_4^{n_{34}}\wedge v_2^{n_{42}} v_3^{n_{43}} v_4^{n_{44}}))^{\tau} \\
%&\hspace{7.25cm}= (v_1\wedge v_2)^{n_{11}} (v_3\wedge v_4)^{n_{11}}.\end{align*}
%Since $\langle v_2\wedge v_3\rangle$ and $\langle v_2\wedge v_4\rangle$
%are eigenspaces of $\tau$, by comparing exponents, we similarly deduce that
%\[ -n_{13}n_{22} + \begin{vmatrix} n_{32} & n_{33} \\ n_{42} & n_{43} \end{vmatrix}
%= -n_{14}n_{22} + \begin{vmatrix} n_{32} & n_{34} \\ n_{42} & n_{44} \end{vmatrix} = 0.\]
%Finally, by comparing the $v_1\wedge v_2$ and $v_3\wedge v_4$ terms, we obtain
%\[ \begin{bmatrix}\lambda & \kappa \\ \kappa &\lambda\end{bmatrix}
%\begin{bmatrix}n_{11}n_{22} \\[4pt] \lvert\begin{smallmatrix} n_{33}&n_{34}\\n_{43}&n_{44}\end{smallmatrix}\rvert\end{bmatrix} 
%=\begin{bmatrix} n_{11}\\n_{11}\end{bmatrix}.\]
%The matrix on the left is taken to be invertible, so equivalently
%\[ \begin{bmatrix}n_{22}\\[4pt] n_{11}^{-1}\lvert\begin{smallmatrix} n_{33}&n_{34}\\n_{43}&n_{44}\end{smallmatrix}\rvert\end{bmatrix}=
%\begin{bmatrix} \lambda & \kappa \\ \kappa & \lambda\end{bmatrix}^{-1}\begin{bmatrix}1\\1\end{bmatrix} = \begin{bmatrix}(\lambda+\kappa)^{-1} 
%\\ (\lambda + \kappa)^{-1} 
%\end{bmatrix}.\]
%Put $s_{\lambda,\kappa} =\lambda +\kappa$. Then $\pi\hat{\eta}\tau = \eta\pi$ holds if and only if
%\begin{align*}
%\eta &= \begin{bmatrix}
%s_{\lambda,\kappa} \lvert\begin{smallmatrix}n_{33} & n_{34}\\n_{43} & n_{44}\end{smallmatrix}\rvert & n_{12} & s_{\lambda,\kappa} \lvert\begin{smallmatrix}n_{32} & n_{33} \\ n_{42} & n_{43}\end{smallmatrix}\rvert & s_{\lambda,\kappa} \lvert\begin{smallmatrix} n_{32} & n_{34} \\ n_{42} & n _{44}\end{smallmatrix}\rvert\\
%0 & s_{\lambda,\kappa}^{-1} & 0 & 0\\
%0 & n_{32} & n_{33} & n_{34}\\
%0 & n_{42} & n_{43} & n_{44}
%\end{bmatrix}\\
%& = \begin{bmatrix} s_{\lambda,\kappa} & 0 & 0& 0 \\
%0 & s_{\lambda,\kappa}^{-1}& 0 & 0 \\
% 0 & 0 & 1 & 0\\
% 0 & 0 & 0 & 1\end{bmatrix} \begin{bmatrix}
%\lvert\begin{smallmatrix}n_{33} & n_{34}\\n_{43} & n_{44}\end{smallmatrix}\rvert & s_{\lambda,\kappa}^{-1}n_{12} &\lvert\begin{smallmatrix}n_{32} & n_{33} \\ n_{42} & n_{43}\end{smallmatrix}\rvert & \lvert\begin{smallmatrix} n_{32} & n_{34} \\ n_{42} & n _{44}\end{smallmatrix}\rvert \\
%0 & 1 & 0 & 0\\
%0 & n_{32} & n_{33} & n_{34}\\
%0 & n_{42} & n_{43} & n_{44}
%\end{bmatrix},
%\end{align*}
%where this last matrix lies in $\Aut^c(\pi)$ by Proposition \ref{auto3}. The class of $(\eta,\hat{\eta}\tau)$ modulo $\Gamma(G)$ is not affected when $\eta$ is multiplied by an element of $\Aut^c(\pi)$. Thus, we may take
%\[ \eta = \begin{bmatrix} s_{\lambda,\kappa} &  & &  \\
% & s_{\lambda,\kappa}^{-1}&  &  \\
%  &  & 1 & \\
%  &  &  & 1\end{bmatrix},\,\ \hat{\eta} =
%\begin{bmatrix}
%  1 & & & & & \\
%  & s_{\lambda,\kappa}& &  &\\
%  &  & s_{\lambda,\kappa} & & &\\
%  & & & s_{\lambda,\kappa}^{-1} & &\\
%   &  & & &s_{\lambda,\kappa}^{-1} & \\
%  & &  & & & 1
%  \end{bmatrix}.\]
%% in which case we have
%% \begin{align*}
%%  \hat{\eta}\tau  =
%%\begin{bmatrix}
%%  \lambda & 0 & 0 & 0 & 0 &\kappa\\
%%  0 & (\lambda-\kappa)s_{\lambda,\kappa}& 0 & 0 &0 & 0\\
%%  0 & 0 & (\lambda-\kappa)s_{\lambda,\kappa} & 0 & 0 &0\\
%%  0 & 0 & 0 & (\lambda-\kappa)s_{\lambda,\kappa}^{-1}   & 0 & 0\\
%%   0 & 0 & 0 & 0 &(\lambda-\kappa)s_{\lambda,\kappa}^{-1} & 0 \\
%%\kappa & 0 & 0 & 0 & 0 & \lambda
%%\end{bmatrix}.
%%   \end{align*}
%To simplify notation, let us put
%\begin{align*}
%M_{\lambda,\kappa} & =  \left[ \begin{smallmatrix} \lambda + \kappa&&&\\ & (\lambda+\kappa)^{-1} &&\\ &&1&&\\&&&1\end{smallmatrix}\right],\\
% N_{\lambda,\kappa} & =  \left[\begin{smallmatrix}
% \lambda &&&&&\kappa\\
% & (\lambda-\kappa)(\lambda+\kappa) &&&&\\
% &&(\lambda-\kappa)(\lambda+\kappa) &&&\\\
% &&&(\lambda-\kappa)(\lambda+\kappa)^{-1}&&\\
% &&&&(\lambda-\kappa)(\lambda+\kappa)^{-1}&\\
%\kappa &&&&&\lambda
%\end{smallmatrix} \right].
%\end{align*}
%We then deduce that
%\[ \res(\mathcal{S}') \simeq \{ (M_{\lambda,\kappa} ,N_{\lambda,\kappa}) : \lambda,\kappa\in \mathbb{F}_p\mbox{ with }\kappa\neq \lambda,\pm2\lambda\},\]
%and it is not hard to show that this is isomorphic to 


\end{document}
