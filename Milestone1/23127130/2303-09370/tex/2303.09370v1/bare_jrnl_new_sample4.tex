\documentclass[lettersize,journal]{IEEEtran}
\usepackage{amsmath,amsfonts}
\usepackage{algorithmic}
\usepackage{algorithm}
\usepackage{array}
% \usepackage[caption=false,font=normalsize,labelfont=sf,textfont=sf]{subfig}
\usepackage{textcomp}
\usepackage{stfloats}
\usepackage{url}
\usepackage{verbatim}
\usepackage{graphicx}
\usepackage{cite}
\usepackage{multirow}
\usepackage{color,xcolor}
\usepackage{array}
\usepackage{booktabs}
\usepackage{epsfig}
\usepackage{amssymb}
\usepackage{graphicx}
\usepackage{subcaption}
% \usepackage{subfigure}
\hyphenation{op-tical net-works semi-conduc-tor IEEE-Xplore}

% Hide or show our comments
\newif\ifshowcomments
\showcommentstrue
%\showcommentsfatlse

% Coloring for our comments (use showcommentstrue or showcommentsfalse above to show/hide)
\ifshowcomments
\newcommand{\TMM}[1]{{\color{red}{[TMM: #1]}}}
\newcommand{\ACMMM}[1]{{\color[rgb]{0.2,0.7,0.2}{[ACMMM: #1]}}}
\else
\newcommand{\TMM}[1]{}
\newcommand{\ACMMM}[1]{}
\fi
% updated with editorial comments 8/9/2021

\begin{document}

\title{Learning Physical-Spatio-Temporal Features for Video Shadow Removal}

% \author{IEEE Publication Technology,~\IEEEmembership{Staff,~IEEE,}
        % <-this % stops a space
% \author{Z. Chen, L. Wang, and Y. Xiao are with the School of Computer Science and Technology, College of Intelligence and Computing, Tianjin University, Tianjin 300350, China. Email: zh_chen@tju.edu.cn, fnxyf@tju.edu.cn, lwan@tju.edu.cn.}
\author{Zhihao Chen, Liang Wan$^\dag$ \textit{Member, IEEE}\thanks{Z. Chen, L. Wang$^\dag$, and Y. Xiao are with the College of Intelligence and Computing, Tianjin University, Tianjin 300350, China. Email: zh\_chen@tju.edu.cn, lwan@tju.edu.cn, fnxyf@tju.edu.cn.}, Yefan Xiao, Lei Zhu \textit{Member, IEEE}\thanks{L. Zhu is with The Hong Kong University of Science and Technology (Guangzhou), Nansha, Guangzhou, 511400, Guangdong, China and The Hong Kong University of Science and Technology, Hong Kong SAR, China  (Email: leizhu@ust.hk) }, Huazhu Fu \textit{Senior Member, IEEE}\thanks{H. Fu is with the Institute of High Performance Computing (IHPC), Agency for Science, Technology and Research (A*STAR), Singapore 138632. (E-mail: hzfu@ieee.org)}\\
% }
% \thanks{This paper was produced by the IEEE Publication Technology Group. They are in Piscataway, NJ.}% <-this % stops a space
% \thanks{Manuscript received April 19, 2021; revised August 16, 2021.}
}

% The paper headers
% \markboth{IEEE Transactions on Circuits and Systems for Video Technology, IN SUBMISSION}%
\markboth{IN SUBMISSION}%
{Shell \MakeLowercase{\textit{et al.}}: A Sample Article Using IEEEtran.cls for IEEE Journals}

% \IEEEpubid{0000--0000/00\$00.00~\copyright~2021 IEEE}
% Remember, if you use this you must call \IEEEpubidadjcol in the second
% column for its text to clear the IEEEpubid mark.

\maketitle

\begin{abstract}
Shadow removal in a single image has received increasing attention in recent years. However, removing shadows over dynamic scenes remains largely under-explored.  In this paper, we propose the first data-driven video shadow removal model, termed PSTNet, by exploiting three essential characteristics of video shadows, i.e., physical property, spatio relation, and temporal coherence. Specifically, a dedicated physical branch was established to conduct local illumination estimation, which is more applicable for scenes with complex lighting and textures, and then enhance the physical features via a mask-guided attention strategy. Then, we develop a progressive aggregation module to enhance the spatio and temporal characteristics of features maps, and effectively integrate the three kinds of features. Furthermore, to tackle the lack of datasets of paired shadow videos, we synthesize a dataset (SVSRD-85) with aid of the popular game GTAV by controlling the switch of the shadow renderer. Experiments against 9 state-of-the-art models, including image shadow removers and image/video restoration methods, show that our method improves the best SOTA in terms of RMSE error for the shadow area by $14.7\%$. In addition, we develop a lightweight model adaptation strategy to make our synthetic-driven model effective in real world scenes. The visual comparison on the public SBU-TimeLapse dataset verifies the generalization ability of our model in real scenes.
% Shadow removal in a single image has recently received great research interest. However, the shadow removal over dynamic scenes is still under-exploring.
%%
%Different with the existing methods mainly based on the spatial and temporal characteristics featuring by CNN feature and optical flow, we introduce a novel physical characteristic to present the relation between the complex lighting and object shadow. 
%%
% In this paper, we develop the first data-driven video shadow removal model, termed PSTNet, by fusing three important representations of video shadows (i.e., physical, saptio and temporal). Specifically, PSTNet construct three independent branches to learn physical, saptio and temporal features and then fuse them to refine the the final video shadow removal results in a progressive manner. 
% %
% The physical branch uses a linear illumination transformation to model the shadow effects, which consider the illumination similarity of the shadow regions. In addition, to avoid the problem that global uniform lighting estimation is not applicable in scenes dealing with complex lighting and textures, we propose the Adaptive Exposure Estimation Module (AEEM) to construct the linear illumination transformation for different shadow regions. Next, the spatio and temporal branches are introduced to maintain the spatial resolution and temporal coherence of the removal results, respectively.
% Our method exploits three kinds of characteristics related to shadows in dynamic scenes, i.e. the physical characteristic, spatial characteristic, and temporal characteristic. 
%%
% We resort to the physical characteristic branch to mine the differences between shadow and non-shadow regions in the illumination model.
%mine the relation between complex lighting/texture conditions and image shadows. 
%
% In addition, to maintain the spatial resolution and temporal coherence of the prediction results, we introduce the spatio and temporal characteristic branches, respectively. 
%and develop an adaptive exposure estimation module and a mask-guided supervised attention module to extract reliable physical features. 
% And then, a dedicated feature fusion module aggregates multi-characteristic features to refine the shadow removal results in a progressive manner. 
% %%
% Furthermore, facing the lack of well-established datasets of paired shadow videos, we synthesize a dataset (SVSRD-85) in the popular game GTAV by controlling the switch of the shadow renderer.
% %%%
% Experiments in SVSRD-85 against 9 state-of-the-art models, including single image shadow removers and image/video restoration methods, 
% %%
% show that our method improves the state-of-the-art in terms of RMSE error for the shadow area by $14.7\%$.
% %%
% In addition, we propose a model transformation strategy to make our synthetic-driven model effective in real world scenes. The visual comparison in SBU-TimeLapse dataset can evaluate the generalization ability of our model in real scenes.

%Last, we also evaluate PSTNet on the SBU-Timelapse dataset that provides unpaired real shadow scenes, to prove the generalizability of our method.
\end{abstract}

\begin{IEEEkeywords}
Video shadow removal, physical-spatio-temporal features, synthetic scenes.
\end{IEEEkeywords}

\definecolor{mygreen}{RGB}{76, 146, 58}
\definecolor{mypurple}{RGB}{148, 103, 189}
\definecolor{myorange}{RGB}{255, 127, 14}
\definecolor{myred}{RGB}{214, 39, 40}
\definecolor{myblue}{RGB}{31, 119, 180}


\section{Introduction}

Sequence-to-sequence (Seq2Seq) \cite{sutskever2014sequence} models based on Recurrent Neural Networks \cite{mikolov2010recurrent} and its variant, such as Gated Recurrent Unit (GRU) \cite{cho2014learning} and Long Short-Term Memory (LSTM) cell \cite{hochreiter1997long}, has demonstrated high rate of success in areas such as question answering \cite{sukhbaatar2015end,dong2016language}, dialogue systems \cite{serban2016building}, machine translation \cite{sutskever2014sequence,luong2015multi}, and text generation \cite{wen2015stochastic, shang2015neural}. This paper focuses on a vanilla encoder-decoder based Seq2Seq model mostly following its original design in 2014 \cite{sutskever2014sequence}, as shown in Figure \ref{structure}, and the major difference is that we use GRU instead of LSTM. It is constructed by two recurrent blocks (encoder and decoder), an embedding layer to deal with inputs, and a fully connected layer to generate distributions of all tokens. In order to learn the output sequence distribution (target) given the input sequence (source), common training data are designed as input and output pairs: [$x_1,x_2,x_3$] $\rightarrow$ [$y_1,y_2,y_3,y_4$]. 


The inference process is described as follows: First the encoder encodes [$x_1,x_2,x_3$] to a dense vector $h_0$. Then at the first step of decoder, $h_0$ is transformed into $h_1$ after passing through the decoder unit once. $h_1$ then serves as input to the fully connected layer to produce the conditional probability of each single output token. Normally the token with highest probability becomes the first output token $\hat{y}_1$. $h_1$ shall be recursively passed down to the \emph{identical} decoder unit to generate conditional probability 
at subsequent time steps to produce tokens ($\hat{y}_2, \hat{y}_3, ...$) one by one.

%In addition that $h_1$ will be passed into the FC layer and converted into conditional probability of output tokens ($y_1$), it will recursively pass down to the \emph{identical and fixed} decoder units again to output other words ($y_2, y_3, ...$) to form a sequence. The model outputs the token, $o_t$ with maximum probability,  
%(3) Firstly, it will be converted to the conditional probability of the output token by the FC layer, $o_1 = FC (h_1)$, and the model will output the token with the highest probability: $\hat{y}_{1} = \mathop{\arg\max}\limits_{y \in \text{vocab}}o_1 =\mathop{\arg\max}\limits_{y \in \text{vocab}}Pr(y|\hat{y}_{t-1},h_{t-1})$.
%$\hat{y}_{t} = \mathop{\arg\max}\limits_{y \in \text{vocab}}Pr(y|\hat{y}_{t-1},h_{t-1})$. 
%(4) Secondly, $h_1$ (there is a slightly different between with LSTM or GRU base) will recursively pass down to the \emph{identical and fixed} decoder units again and again to output conditional probability for later time step, ($y_2, y_3, ...$), to form a sequence.

\begin{figure*}%[h]
  \centering
  \includegraphics[width=0.9\textwidth]{figures/structure.png}
  \caption{The Seq2Seq model combines classic encoder-decoder structure with one embedding layer and one fully connected layer without attention nor bi-directional structure.}
  \label{structure}
\end{figure*}

Previous research \cite{shen2019controlling} has shown that it is possible to train a Seq2Seq model to output a token at a specific position with high accuracy. In fact, with two control signals, \emph{token} and \emph{position}, attached to the input sequence directly, they showed that a transformer model can learn the meaning of the control signals and output the right token at the right position with accuracy close to 100\%. 
Furthermore, it is also possible to train a vanilla Seq2Seq model (i.e. no attention or bidirectional structure is needed) like the one shown in Figure \ref{structure} to achieve the same goal. More precisely, training with sufficient number of examples containing token and position signals at the end of the input, the model can eventually learn that it has to output the token at the target position (e.g. third) as shown below:


\begin{center}
\textit{
$x_1$, $x_2$, $x_3$, {\color{myred}token}, {\color{mygreen}3} $\rightarrow$ $y_1$, $y_2$, {\color{myred}token}, $y_4$
}
\end{center}
%\begin{center}
%"\textit{
%Ride in the back {\color{myred}doors} %{\color{mygreen}3} $\rightarrow$ lock the %{\color{myred}doors}
%}",
%\end{center}
%the model can eventually understand that it shall output the token (e.g. $y_3$) 
%at the target position (e.g. third) with close to 100\% accuracy. 
%the numerical position signal can be any number as long as this position has been seen in the training set.
 Note that a Seq2Seq auto-encoder\cite{ma2018autoencoder,xu2017variational} can be regarded as a realization of such token-positioning function since the model has to precisely outputs each token at the right position. 
 %For example, a well-trained vanilla Seq2Seq model can produce an outputting sequence whose third token is {\color{myred}doors} given the input "\emph{Ride in the back  {\color{myred}doors} {\color{mygreen}3}.

%$\rightarrow$ lock the {\color{myred}doors}}" uses \emph{doors} as the target token and \emph{3} as the target position.

%We will further confirm in the subsequent paragraphs that when the token is set to \emph{EOS} (a token to label end-of-sequence) or a certain rhyme, this model can also be used to control the length or rhyme of the output sentence.


We believe this finding is very interesting and potentially influential since the model in Figure \ref{structure} has no single component specifically designed for token positioning. It is exactly the same model used to perform machine translation as its original goal. % and the only modification made lies solely on the generation of training data (i.e. adding two control signals). 
Comparing with the state-of-the-art models that usually consist of much more complicated structure (e.g. transformer, GAN, etc.) or mechanism (e.g. attention, variational inference, etc.) to achieve the control of generation, we are quite surprised to learn that the simplest Seq2Seq model can already achieve fine-grained control of outputs given carefully designed training examples, which motivates us to focus on explaining the mechanism behind such capability. 

To begin with, we want to explain that generating a word at a target position is mathematically challenging for a vanilla Seq2Seq model as shown in Figure \ref{structure}. We start from Equation \ref{long-eq} that describes mathematically how the model produces output sequences during inference. %The key factor is that all learned decoder units \emph{are identical} (and so is the fully connected function). 
%mathematically we can write the generated sequence as ($\hat{y}_{1}$, $\hat{y}_2$, ..., $\hat{y}_T$). 
The equation tells us that the target output $y_t$ is a fairly complicated function that contains $argmax$ among probabilities of all tokens in the vocabulary generate by the fully-connected layer FC, which depends on the GRU mapping of $h_{t-1}$ and  $y_{t-1}$, and further depends recursively on $h_{t-2}$ and  $y_{t-2}$, and so on. Eventually we can regard every output token as coming from a very complicated recursive function of the encoder output $h_0$. In order to output one specific token at position $t$, $h_0$ needs to make sure that after entering the same GRU unit recursively for $t-1$ times, the fully connected layer can assign the largest probability to the target token among tens of thousands of possible candidates. Also note that since the decoder shares the same FC and GRU function in every step, all the dynamic information needs to be stored in the hidden state $h_t$ in order to output different tokens at different positions.  %Thus it is not intuitive why a memory-less model like the one shown in Figure \ref{structure} can deliver the correct output precisely at a specific position. %In order to output a specific token precisely at a later position, We find it surprising that the target token can still be produced at the right position after so many recursive calls and non-linear selections. 

\begin{comment}
Another two extended examples:
%\begin{center}
"\textit{
Ride in the back {\color{myred}EOS} {\color{mygreen}3} $\rightarrow$ lock the doors
}", and "\textit{
Ride in the back {\color{myred}or} {\color{mygreen}3} $\rightarrow$ lock the {\color{myred}doors}
}"
%\end{center}
, the model can learn to output \emph{EOS} to control the length of the outputs, or learn to output the word with rhyme '\emph{or}' at the third time step respectively. Therefore, with the format, 
%\begin{center}
"\textit{
    $x_1$, $x_2$, $x_3$, ${\color{myred}Token_{A}}$, ${\color{mygreen}Pos._{3}}$,  $\rightarrow$ $y_1$, $y_2$, ${\color{myred}Token_{A}}$, $y_4$
    }"
%\end{center}
, the model can learn to output a sequence with conditions. 
\end{comment}

%This is very different from previous works that most of them design a more complicated model structure to deal with the controlled conditions, i.g. concatenate the control signals in the embedding layer or hidden state (Need reference here). In addition, it is also different from the more intuitive approach which uses different structures to process different types of control signals, for example, design different models to control the length or rhyme. 
%, first of all, to produce the right token at a specific position, this token shall processes the highest probability (i.e. output from FC) among tens of thousands of possible tokens in the dictionary. Next, the generation of the first token $y_1$ comes from a complicated non-linear GRU function that involves the encoder outputs $h_0$, while the generation of $y_2$ recursively requires the embedding of requires generating the probabilities of all tokens in the dictionary and is a complicated non-linear mapping from $h_k-1$ that requires not only the process of GRU but also finding the tokens of the probabilities of all token 

%It is not only because model should learn that the two control signals $token$ and $position$ have their own function and shall be treated differently comparing to contain different information from other tokens in the input sequence, the more critical factor is about the recursive mechanism of the recurrent neural network itself: (1) As shown in Figure \ref{structure}, the decoder units (GRU in the figure) and fully connected layer \emph{are all identical}. 
Therefore, each outputs ($\hat{y}_{1}$, $\hat{y}_2$, ..., $\hat{y}_T$) can be written mathematically as follows: %However, each $h_t$ depends on the previous $h_{t−1}$ as well as the previous output $y_{t−1}$, which recursively depends on the previous states $h_{t−2}$, $y_{t−2}$, and so on. Therefore, 
% Mathematically we can write the recursive formula of the decoder as following,  

\begin{equation}
    \begin{aligned}
    \hat{y}_1 &= \mathop{\arg\max}\limits_{y \in \text{vocab}} FC(h_1)
            = \mathop{\arg\max}\limits_{y \in \text{vocab}} FC \big(GRU(h_{0}, Emb(\hat{y}_{0}))\big) \\
    \hat{y}_2 &= \mathop{\arg\max}\limits_{y \in \text{vocab}} FC(h_2)
            = \mathop{\arg\max}\limits_{y \in \text{vocab}} FC \big(GRU({\color{myorange}h_{1}}, Emb({\color{myblue}\hat{y}_{1}}))\big) \\
            & = \mathop{\arg\max}\limits_{y \in \text{vocab}} FC \Big(GRU\big({\color{myorange}GRU(h_{0},Emb(\hat{y}_{0}))}\big),
            \quad  Emb({\color{myblue}\mathop{\arg\max}\limits_{y \in \text{vocab}} FC \big(GRU(h_{0}, Emb(\hat{y}_{0})))}\big)\Big) \\
    \hat{y}_t &=...        
    \end{aligned}
    \label{long-eq}
\end{equation}

%That is, all outputs are generated by $h_0$ with different times GRU transformation and one time FC transformation, but it needs to learn to output a extremely huge possible combinations of words, that says, $20000  ^ {21}$ (20000 is the vocabulary size and 21 is the max length in our data). (2) A Seq2Seq model is unlike a memory-based model that the later one can save the information in the memory until it is needed. A RNN-based Seq2Seq model, especially without other additional bridge or transformer, can only save the information in its hidden state, i.e. only in a dense vector.

In order to accurately position a token, this paper uncovers how a Seq2Seq model can perform the following functions: (1) The model not only needs to \emph{store} the token information after it appears in the encoder, but also confines its impact till the target outputt time. %This is not an easy task because in each time step the hidden state enters an identical GRU function.
(2) In order to output at the right position, the Seq2Seq model requires a mechanism to \emph{count down}. (3) The \emph{storing} and \emph{counting} mechanisms need to interact in a certain way to ensure the fully connected layer assigns the largest probability to the target token. 
%For instance, this \emph{counter} is not likely to stored as a numerical value in a neuron since such neuron can hardly interact with \emph{storing} neurons to achieve (3).
%Note that different from a memory-based model such as xxx, an RNN-based Seq2Seq model doesn't have memory to store extra information, thus all required information to be passed to the next stage needs to be stored in the hidden state $h$. 
%we still need to answer how neurons interact to activate the output of the correct token. Hence, this paper is aim to provide an in-depth analysis on those questions to explain how a Seq2Seq model can learn to position a word accurately.
%The first step of our work is to train a Seq2Seq model on the data pairs mentioned above. Next, with the help of the auxiliary classifier and integrated gradient, the recurrent neurons can be categorized to four important neuron sets. These neuron sets perform four different basic functions of GRU cell, including \textit{storing}, \textit{counting}, \textit{triggering} and \textit{outputting}. Noting that every target token (20000 tokens in our output dictionary) has their own important neurons set. Afterwards, we replace the values of the important neuron sets to verify how these small portion of neurons influence the entire model. At this stage, we can answer the first two research questions by observing the values of storing and counting neuron set respectively, and also answer the third questions by analyzing how these neuron sets interact. 


The contributions of this paper can be summarized as: (1) We find that neurons in a Seq2Seq model can perform four basic functions, \emph{storing, counting, triggering,} and \emph{outputting}. (2) We discover the mechanism behind counting down and token positioning based on the interaction among different types of neurons. (3) We propose a series of strategies to identify neurons of specific purpose in a recursive neural network, which can potentially be exploited for other types of neuron-based analysis. %
%(3) We also report several other side-findings, such as how h, z and r gate play different roles to transmit the information, to provide a better understanding about a GRU-based Seq2Seq model.
%The paper is constructed as below: we first describe our experimental setup and research problems. In section 3, we propose a general strategy to identify and verify the important neuron set. Next, we explain how we use the strategy to find out the important neuron set, and try to explain these three research questions. In section 5, we summarize our findings and leave discussion.


% We leave LSTM-based cells as a future work since GRU is simpler than LSTM with one gate, but the difference of accuracy between these two is less than one percent. 


\begin{comment}
---
As shown in Figure \ref{structure}, at every step of the decoder, the same GRU outputs a hidden state ${h_t}$, which will be fed into a fully-connected layer (FC) to generates the conditional probabilities for each token. In order to output a specific token, the probability of this token has to be the largest one. However, each $h_t$ depends on the previous $h_{t−1}$ as well as the previous output $y_{t−1}$, which recursively depends on the previous states $h_{t−2}$, $y_{t−2}$, and so on. Mathematically we can write the recursive formula of the decoder as following,  
%\begin{small}
%\begin{equation}
\[
    \begin{aligned}
    \hat{y}_t &= \mathop{\arg\max}\limits_{y \in \text{vocab}} FC(h_t)
            = \mathop{\arg\max}\limits_{y \in \text{vocab}} FC \big(GRU({\color{myorange}h_{t-1}}, Emb({\color{myblue}\hat{y}_{t-1}}))\big)\\
            & = \mathop{\arg\max}\limits_{y \in \text{vocab}} FC \Big(GRU\big({\color{myorange}GRU(h_{t-2},Emb(\hat{y}_{t-2}))}\big),
            \quad  Emb({\color{myblue}\mathop{\arg\max}\limits_{y \in \text{vocab}} FC \big(GRU(h_{t-2}, Emb(\hat{y}_{t-2})))}\big)\Big) \\
            & = \text{a complicated function of } h_0...
    \end{aligned}
    %\label{tab:long-eq}
\]
%\end{small}
%\end{equation}
Precisely speaking, outputs ($y_{1}$, $y_2$, ..., $y_T$) are all generated by $h_0$ with different times GRU transformation and one time FC transformation. The model needs to learn to output a extremely huge possible combinations of words, that says, $20000  ^ {21}$ (20000 is the vocabulary size and 21 is the max length in our data), but it can only use a fixed GRU and FC weight. Therefore, this paper aims to provide an in-depth analysis on how a Seq2Seq model can learn to position a token accurately, as we called the token-positioning ability, given the recursive mechanism and such simple structure. It can be divided into three questions: (1) How does the model store the $token$ information? (2) How does the model use $position$ signal to know when to output? (3) How does the model combine $token$ and $position$ information to output? 
\end{comment}


\section{Related Works}
\label{sec:relatedworks}
%\noindent
\subsection{Image Shadow Removal}
Early research on image shadow removal employ prior information, e.g., gradient~\cite{gryka2015learning}, illumination~\cite{shor2008shadow,xiao2013fast,zhang2015shadow}, and region~\cite{guo2012paired,vicente2017leave}, for removing shadows. 
%%
However, conventional prior
based methods can just handle high-quality images with certain constraints of real scenes (e.g. ample lighting conditions and simple textures).
%%
In recent years, deep learning based methods boost the removal performance~\cite{qu2017deshadownet,ding2019argan,hu2019direction,cun2020towards}, since CNN networks have built-in inductive biases that make them well-suited to a wide variety of computer vision applications~\cite{liu2022convnet}. 
%%
In more detail, DeshadowNet~\cite{qu2017deshadownet} proposes a multi-context architecture, where the  shadow matte is predicted by embedding information from three different perspectives, i.e. global localization, appearance modeling and semantic modeling.
%%
ST-CGAN~\cite{wang2018stacked} leverages two stacked conditional GANs to jointly train shadow detection and removal tasks.
%%
DSC\cite{hu2019direction} designs direction-aware context to improve shadow detection and removal.
%%
DHAN~\cite{cun2020towards} analyses two types of shadow ghosts and proposes a dual hierarchical aggregation network and shadow augmentation method to boost shadow removal.
%%
%ARGAN~\cite{ding2019argan} treats shadow removal as a progressive optimization process and designs an attentive recurrent generative adversarial network to recover shadow free image step-by-step.
Zhu et al.~\cite{zhu2022bijective} argued that shadow removal and generation are interrelated. They proposed BMNet, which couples the learning procedures of shadow removal and shadow generation in a unified parameter-shared framework.

In the literature, some methods consider physical models of shadow formation and embed such information in the end-to-end deep learning manner. 
%%
SID~\cite{le2021physics} uses a linear illumination transformation to model the shadow effects in the image, which expresses the shadow image as a combination of the shadow-free image, the shadow parameters, and a matte layer. It then uses two deep networks to predict the shadow parameters and the shadow matte respectively.
%%
Fu et al.~\cite{fu2021auto} extended SID via employing multi-exposure image fusion strategy and proposes a boundary-aware RefineNet to eliminate the remaining ghost shadow further. 

In addition, to alleviate the requirements of capturing paired data, some GAN based methods~\cite{hu2019mask,liu2021LG,liu2021G2R} perform image shadow removal on unpaired shadow and shadow-free images.
%%
MaskShadowGAN~\cite{hu2019mask} learns to produce a shadow mask from the input shadow image and then takes the mask to guide the shadow generation via re-formulated cycle-consistency constraints. LG-ShadowNet~\cite{liu2021LG} improves MaskShadowGAN by introducing a brightness guidance strategy. G2R-ShadowNet~\cite{liu2021G2R} generates unpaired data given a set of shadow images and their corresponding shadow masks. Gao et al.~\cite{gao2022towards} proposed a shadow simulation method to simulate shadow on the grayscale, which can be applied to arbitrary shadow-free images and masks to generate corresponding shadow images.
%
However, due to the lack of accurate pixel-level shadow-free supervision, those unpaired-based methods still suffer from artifacts and image blur, and hence there exists a certain gap in performance compared to the fully supervised approaches on benchmarks.

%\noindent
\subsection{Video Shadow Removal}
Video shadow removal aims to remove shadows from each frame of a video. Existing methods almost all rely on hand-crafted features and there is no deep-learning based method for video shadow removal. 
%%
Existing methods usually assume that the background is relatively stationary, allowing the moving objects and shadows to be separated. 
%%
Nadimi \& Bhanu~\cite{nadimi2004physical} separated moving cast shadows from the moving objects in an outdoor environment. This approach is based on a spatio-temporal albedo test and dichromatic reflection model which accounts for both the sun and the sky illuminations. 
%%
Jung et al.~\cite{jung2009efficient} presented a statistical method for background subtraction and shadow removal for grayscale video sequences. 
%The background image is modeled using statistical descriptors, and a noise estimate is obtained. Foreground pixels are extracted, and a statistical approach combined with geometrical constraints are adopted to detect and remove shadows. 
%%
Wang et al.~\cite{wang2009real} presented an approach of moving vehicle detection and cast shadow removal for video based traffic monitoring. Based on conditional random field, spatial and temporal dependencies in traffic scenes are formulated under a probabilistic discriminative framework.
%, where contextual constraints during the detection process can be adaptively adjusted in terms of data-dependent neighborhood interaction. 

The above methods are suitable for surveillance videos with static backgrounds, but not applicable to remove shadows in scenes with rich motion conditions. In this paper, we develop the first deep-learning based method for video shadow removal. Furthermore, we  bypass the strict requirements of collecting paired videos from real scenes, and built one virtual dataset of paired shadow and shadow-free videos in synthetic scenes.

%The success of deep learning methods in image shadow removal task is inseparable from the successive proposals of several large-scale datasets~\cite{qu2017deshadownet,wang2018stacked,hu2019mask}. However, as for video shadow removal, no available dataset with paired shadow and shadow-free videos can be used to train a data-driven video shadow removal model. Paired videos in real-world scenes are difficult to collect, as it is impossible to reproduce the dynamics exactly as the first shot at the second shot. 
%In this paper, we bypass the collect paired videos from real scenes. Instead, we synthesize video pairs in synthetic scene via twice rendering with controlling the switch of shadow. In this way, we made it possible to perform video shadow removal by robust data-driven models.

% \noindent
% \textbf{Computer vision using synthetic scene}


\section{SVSRD-85: Synthetic Video Shadow Removal Dataset}
\label{sec:dataset}
%采用人工合成场景的动机,以及具体的生成方式。
%%%%%%%%%%%%%%%%%%%%%%%%%%%%%%%%%%%%%%%%%%%%%%%%%%%%%%%%%%%%%%%%%%%%% Figure: Dataset
\begin{figure*}
\centering
\includegraphics[width=\textwidth]{Figures/Dataset.pdf}
\caption{Example frames of our  synthetic dataset (SVSRD-85), and video shadow removal results by our proposed PSTNet. }
\label{fig:dataset}
\vspace{-2.5mm}
\end{figure*}
%%%%%%%%%%%%%%%%%%%%%%%%%%%%%%%%%%%%%%%%%%%%%%%%%%%%%%%%%%%%%%%%%%%%%
\noindent
\textbf{Motivation.}
%To boost the video shadow removal task, we first consider the acquisition of real data.
%%
In real world, though we can easily get a shadowed video, getting the synchronous shadow-free video is quite difficult since it's hard to restore the same shooting states between two shots. 
%%
Le~\textit{et al.} collected SBU-Timelapse~\cite{le2021physics}, a video dataset of 50 videos with time-lapse photography, in which each video contains a static scene without visible moving objects.
%%
They used ``max-min'' technique to obtain a single pseudo shadow-free frame for each video.
%and then evaluated some single image shadow removal methods in this dataset. 
However, this dataset has two deficiencies. First, the obtained pseudo shadow-free image still contains much static shadows (as shown in Figure~\ref{fig:dataset}). 
%that appear on all frames. 
Second, it does not allow for any movement of targets and perspectives, which greatly limits the motion richness.
%of the collected scenes.
%%%%%%%%%%%%%%%%%%%%%%%%%%%%%%%%%%%%%%%%%%%%%%%%%%%%%%%%%%%%%%%%%%%
\renewcommand\arraystretch{1.1}
\begin{table}[t]
\begin{center}
  \caption{SVSRD-85 vs. SBU-TimeLapse~\cite{le2021physics}.}
  \vskip -5pt
  \label{table:SVSRD_vs_SBUTimeLapse}
  \resizebox{0.46\textwidth}{!}{%
    \begin{tabular}{c|c|c|c|c}
        \toprule[1pt]
        Dataset & Type & Annotation & Scenario & \#Videos\\
        \specialrule{0em}{1pt}{1pt}
        \hline
        \specialrule{0em}{1pt}{1pt}
        SVSRD-85 & Synthetic & Paired & Dynamic & 85  \\
        SBU-TimeLapse~\cite{le2021physics} & Real & Unpaired & Static & 50  \\
        \bottomrule[1pt]
    \end{tabular}
    }
    \vspace{-5mm}
  \end{center}
\end{table}

\noindent
\textbf{Synthesize dataset from game scenes.}
In order to get the reliable video pairs to train, we consider rendering shadow and shadow-free video pairs in synthetic scenes.
%%
Thanks to the strong physical engine and the editing flexibility of the popular game: Grand Theft Auto V (GTAV), many works~\cite{richter2016playing,sidorov2019conditional} explored to drive various computer vision tasks based on this game. GTAV provides a range of APIs that can monitor the creation, modification, and deletion of resources used to specify the scene and synthesize images/videos. 
%%
What's more, it supports to control the switch of the shadow renderer, which can allow us to obtain the high-quality video shadow pairs.

In this paper, we collect a synthetic video shadow removal dataset (SVSRD-85) via the above manner. 
%%
Our dataset includes 85 videos with total 4250 frames, and each video contains a sequence of shadow images and corresponding shadow-free ground-truth. 
%%
We compute the pesudo shadow masks by operating Otsu’s algorithm to the difference between shadow and shadow-free images, similar to MaskShadow-GAN~\cite{hu2019mask}. %%
To provide guidelines for future works, we randomly split the dataset into training and testing sets with a ratio of 7:3, then obtain 59 training videos and 26 testing videos. 
%%
It is worth noting that since there are no restrictions on shooting conditions, SVSRD-85 includes rich scenes with more complex illustration and surface textures, compared with SBU-Timelapse. Table~\ref{table:SVSRD_vs_SBUTimeLapse} shows the main differences between SVSRD-85 and SBU-Timelapse. Some examples of video in SVSRD-85 can be found in Figure~\ref{fig:dataset}. In this paper, we also use some videos in SBU-Timelapse to verify the generalisation ability of our model in real scenes.
%%%%%%%%%%%%%%%%%%%%%%%%%%%%%%%%%%%%%%%%%%%%%%%%%%%%%%%%%%%%%%%%%%%%% Figure:framework
\begin{figure*}[!t]
\centering
\includegraphics[width=1\textwidth]{Figures/framework.pdf}
\caption{The schematic illustration of our proposed PSTNet. The current frame $\mathbf{I}_{t}$ is fed into three parallel branches to extract features of three different characteristic: physical characteristic (\textcolor[RGB]{113,151,152}{top branch}), spatio characteristic (\textcolor[RGB]{159,179,150}{middle branch}), and temporal characteristic (\textcolor[RGB]{237,145,147}{bottom branch}). In bottom branch, the adjacent frame $\mathbf{I}_{t+1}$ will also be used to compute the optical flow $\mathbf{O}_{t, t+1}$. Next, a tailored feature fusion module will aggregate those multi-characteristic features in a progressive manner, thereby performing the final shadow removal prediction $\mathbf{R}_{t}^{final}$. }
\label{fig:framework}
\vspace{-3mm}
\end{figure*}
%%%%%%%%%%%%%%%%%%%%%%%%%%%%%%%%%%%%%%%%%%%%%%%%%%%%%%%%%%%%%%%%%%%%%
\section{Proposed Method} \label{sec:method}
To the best of our knowledge, so far there is no deep-learning based method for video shadow removal. In this paper, we propose a baseline network for this task, termed \textbf{PSTNet}, which learns the union features of physical, spatio, and temporal characteristics to perform video shadow removal. 

\subsection{Overview of Our Network}
\label{sec:overview}
Figure~\ref{fig:framework} presents the schematic illustration of our PSTNet. The network takes the frame $\mathbf{I}_{t}$ and its next frame $\mathbf{I}_{t+1}$ as inputs, then outputs the shadow removed result of the $t$-th frame in an end-to-end manner. 
%%
The intuition behind our network is to leverage complementary information of physical, spatio, and temporal characteristics of moving shadows. 
%%
We explicitly extract the three kinds of features with three independent branches, followed by a multi-characteristics fusion module to achieve the hybrid features in a progressive manner. For convenience, we first introduce the workflows of the three branches.

\noindent
\textbf{Physical characteristic branch.} Shadows are regular physical phenomena produced by occluded lighting. The removal of shadows is inseparable from the analysis of physical lighting conditions. 
%%
Following SID~\cite{le2021physics}, shadow removal can be considered as the physical re-exposure problem. However, SID uses a uniform light model for different areas, ignoring the effects of complex lighting and background textures.
%%
To remedy this problem, we extract the physical characteristics by estimating the adaptive exposure parameters in each shadow region, and using an encoder-decoder sub-network to further learn the broad contextual information due to large receptive fields. 

%%
Specifically, the shadow image $\mathbf{I}_{t}$$\in$$\mathcal{R}^{3 \times W\times H}$ is first fed into AEEM (Adaptive Exposure Estimation Module) to estimate the re-exposure parameters, yielding the over-exposure image $\mathbf{L}_{t}$$\in$$\mathcal{R}^{3 \times W\times H}$. 
%%
Then we feed the concatenation of $\mathbf{I}_{t}$ and $\mathbf{L}_{t}$ into a encoder-decoder sub-network to learn the physical features in the hierarchical manner. 
%%
To make the physical feature more focused on the shadow regions, we append an extra  decoder served for shadow detection. Here, we utilize a widely-used UNet~\cite{UNet2015} structure as the encoder-decoder backbone. 
%%
By this way, we can obtain $\mathbf{F}_{t}^{rem}$ and $\mathbf{F}_{t}^{msk}$ from shadow removal decoder and shadow mask decoder, respectively. 
%%
To further enhance the physical features, we design a Mask-guided Supervised Attention Module (MSAM). 
%MASM provides ground-truth supervisory signals useful for shadow removal and shadow mask segmentation. In addition, 
It enhances the physical features $\mathbf{F}_{t}^{rem}$ to $\mathbf{F}_{t}^{ph}$ with the guidance of the ground-truth supervision through attention mechanism. 
%Last, physical feature $\mathbf{F}_{t}^{ph}$ and the hierarchical features in removal decoder will be passed to the fusion module for further processing.
% includes the Adaptive Exposure Estimation Module(AEEM), the couple-task encoder-decoder structure, and the Mask-guided Supervised Attention Module(MSAM). 

\noindent
\textbf{Spatio characteristic branch.} Shadow removal is a position-sensitive task, since it needs to establish the pixel-to-pixel correspondence from the input to the output.
%%
To preserve the desired fine texture in the final results, in the spatio branch, we employ a sub-network that operates on the original input frame resolution (without any downsampling operation).
%, thereby preserving the desired fine texture in the final shadow removal results. 
%%
To be specific, the sub-network extracts spatio features with a $3$$\times$$3$ convolution followed by a self channel attention block which has the same structure as Figure~\ref{fig:TAB} (b). Then we can get spatio features $\mathbf{F}_{t}^{sp}$. 
%as the output of the spatio branch.

\noindent
\textbf{Temporal characteristic branch} is committed to extract the temporal knowledge for the current frame. 
%%
In video restoration tasks~\cite{wang2019edvr, chan2021basicvsr}, the use of adjacent frames has been shown to facilitate the prediction of the current frame due to the continuity of the video. 
%%
The temporal information brought by adjacent frames are apt to distinguishes different objects in the current frame.
%%
Here, we model the temporal information by computing the optical flow map.
%%
Specifically, the current frame $\mathbf{I}_{t}$ and the next frame $\mathbf{I}_{t+1}$ are fed into a well-trained optical flow estimation network (here, we choose FlowNet2~\cite{ilg2017flownet}) to generate the optical flow $\mathbf{O}_{t, t+1}$$\in$$\mathcal{R}^{3 \times W\times H}$. 
%%
Next, $\mathbf{O}_{t, t+1}$ is further encoded by a $3$$\times$$3$ convolution followed by a self channel attention block which has the same structure as Figure~\ref{fig:TAB} (b), which yields the temporal features $\mathbf{F}_{t}^{te}$.

%%%%%%%%%%%%%%%%%%%%%%%%%%%%%%%%%%%%%%%%%%%%%%%%%%%%%%%%%%%%%%%%%%%%% Figure: AEEM
\begin{figure}[t]
\centering
\includegraphics[width=0.45\textwidth]{Figures/AEEM.pdf}
\caption{The schematic illustration of our Adaptive Exposure Estimation Module (AEEM). }
\label{fig:AEEM}
\vspace{-2.5mm}
\end{figure}
%%%%%%%%%%%%%%%%%%%%%%%%%%%%%%%%%%%%%%%%%%%%%%%%%%%%%%%%%%%%%%%%%%%%%

\subsection{Adaptive Exposure Estimation Module}
\label{sec:aeem}
Considering a shadow frame $\mathbf{I}_{t}$, some recent shadow removal methods~\cite{le2021physics,fu2021auto}, which use physical shadow models, mainly learn to over-exposure $\mathbf{I}_{t}$ to a lightened version, and then fuse them to acquire the  shadow-free image by estimating a shadow matte. 
%%
The common ground of these works is that for simplicity, only one group of linear parameters (denoted as $[\mathrm{w}, \mathrm{b}]$) is predicted for the entire frame $\mathbf{I}_{t}$. With estimated $[\mathrm{w}, \mathrm{b}]$, the over-exposure image $\mathbf{L}_{t}$ can be formulized as follow:
\begin{equation}\label{Equ:I2L}  
    \mathbf{L}_t = \mathrm{w}_t \times \mathbf{I}_t + \mathrm{b}_t.
\end{equation}
%, which greatly reduces the solution space of the problem. 
%%
However, due to the complex changes in lighting and texture, it may be difficult to fit different regions with only one set of exposure parameters. 
%%
In our work, we introduce the Adaptive Exposure Estimation Module (AEEM) to estimate respective $[\mathrm{w},\mathrm{b}]$ for different regions rather than the whole image. 
%The schematic illustration of AEEM is shown in Figure~\ref{fig:AEEM}. 
%%
%Its contributions are two-fold. 
%First, the input image is split into several patches and the estimations of $[\mathrm{w},\mathrm{b}]$ are generated for these patches rather than the whole image.
This can alleviate conflicts in case that  $[\mathrm{w},\mathrm{b}]$ for different regions are inconsistent. 
%%
What's more, we utilize the transformer encoder to build the long-range relationships between image patches, which smooths the estimations of each $[\mathrm{w},\mathrm{b}]$. After AEEM, we can obtain the over-exposure image $\mathbf{L}_{t}$.

As illustrated in Figure~\ref{fig:AEEM}, AEEM takes the shadow image $\mathbf{I}_t$ as input and first splits it into $N$ patches. In our experiments, we set $N$=$4$. 
%%
%After arranging these patches in regular order, the patch embedding and position embedding operations are used to generate the transformer input. 
With patch embedding and position embedding, we employ a transformer encoder to produce the $[\mathrm{w},\mathrm{b}]$ sequence, denoted as $\{[\mathrm{w}_t^n, \mathrm{b}_t^n]\}_{n=1}^N$. 
%%
For convenience, we inherit the same transformer encoder structure as the widely known ViT~\cite{dosovitskiy2020image}, which uses 6 transformer blocks with cascaded multi-head attention module and multi-layer perceptron.
%%
After obtaining $\{[\mathrm{w}_t^n, \mathrm{b}_t^n]\}_{n=1}^N$, we re-exposure the shadow image $\mathbf{I}_t$ to a lightened version $\mathbf{L}_t$ by:
\begin{equation}\label{Equ:It2Lt}  
    \mathbf{L}_t^n = \mathrm{w}_t^n \times \mathbf{I}_t^n + \mathrm{b}_t^n,
\end{equation}
where $\mathbf{I}_t^n$ is $n$-th patch of $\mathbf{I}_t$; $\mathbf{L}_t^n$ is the corresponding lightened version of $\mathbf{}_t^n$. 
%%
Then, we obtain the estimated $\mathbf{L}_t$ by assembling $\{\mathbf{L}_t^n\}^N_{n=1}$ in the original spatial order.
Compared with estimating uniform [w,b] for the whole image as SID\cite{le2021physics}, our proposed AEEM can be proved to handle more complex scenarios (see \ref{sec:ablation} for details).

\subsection{Mask-guided Supervised Attention Module}
\label{sec:msam}

Mask-guided Supervised Attention Module (MSAM) is employed to enhance the physical features $\mathbf{F}_{t}^{rem}$ to $\mathbf{F}_{t}^{ph}$ with the guidance of the ground-truth supervision through attention mechanism. 
%%
The motivations behind MSAM include third main aspects. 
%%
First, ground-truth supervisory signals is useful for progressive shadow removal. 
%%
Second, a well-segmented shadow mask can guide the network to suppress the less informative features (mainly existing in non-shadow regions) and enhance the useful features (mainly existing in shadow regions).
%%
Third, multi-task learning can leverage a stronger encoder via training with multiple types of supervised labels.

As illustrated in Figure~\ref{fig:MSAM}, on the one hand, MSAM takes the features $\mathbf{F}_t^{rem}$$\in$$\mathcal{R}^{C \times W\times H}$ from shadow removal decoder to generate the residual image $\mathbf{X}_t$$\in$$\mathcal{R}^{3 \times W\times H}$ with a simple $1$$\times$$1$ convolution, where $W\times H$ denotes the spatial dimension and $C$ is the number of channels. 
%%
The residual image $\mathbf{X}_t$ is added to the input image $\mathbf{I}_t$ to obtain the coarse shadow-free estimation $\mathbf{R}_t^{middle}$. 
%%
\begin{equation}\label{Equ:R_middle}  
    \mathbf{R}_t^{middle} = \mathbf{I}_t + \mathtt{Conv}(\mathbf{F}_t^{rem}),
\end{equation}
where $\mathtt{Conv}$ denotes the simple $1$$\times$$1$ convolution. On the other hand, MSAM takes the feature $\mathbf{F}_t^{msk}$$\in$$\mathcal{R}^{C \times W\times H}$ from shadow detection decoder to generate the shadow mask prediction $\mathbf{M}_t$$\in$$\mathcal{R}^{1 \times W\times H}$. 
%%
For these predicted image $\mathbf{R}_t^{middle}$ and $\mathbf{M}_t$, we exert explicit supervision with the ground-truth image. 
%%
Next, per-pixel attention maps $\mathbf{F}_t^{att}$$\in$$\mathcal{R}^{C \times W\times H}$ are generated from the hybrid feature, concatenated by $\mathbf{R}_t^{middle}$ and $\mathbf{M}_t$, using a simple $1$$\times$$1$ convolution followed by the sigmoid activation.
%%
Then, $\mathbf{F}_t^{att}$ are then employed to guide the transformed $\mathbf{F}_t^{rem}$ (obtained after $1$$\times$$1$ convolution), resulting in the attention-guided residual features which are added to the identity mapping path. Consequently, the attention-augmented $\mathbf{F}_t^{ph}$ can be formulized as follows:
\begin{equation}\label{Equ:F_ph}  
\begin{aligned}
    \mathbf{F}_t^{ph} &=& \mathbf{F}_t^{rem} + \mathtt{Conv}(\mathbf{F}_t^{rem}) \odot \mathbf{F}_t^{att} \ , \\
    where~\mathbf{F}_t^{att} &=& \sigma(\mathtt{Conv}(\mathtt{Cat}(\mathbf{R}_t^{middle}, \mathbf{M}_t))) \ ,
\end{aligned}
\end{equation}
%%
where $\sigma$ denotes sigmoid activation; $\mathtt{Cat}$ denotes concatenation operation. Then, the attention-augmented feature $\mathbf{F}_t^{ph}$ will be passed to the final multi-characteristics fusion.
%%%%%%%%%%%%%%%%%%%%%%%%%%%%%%%%%%%%%%%%%%%%%%%%%%%%%%%%%%%%%%%%%%%%% Figure: MSAM
\begin{figure}[]
\centering
\includegraphics[width=0.48\textwidth]{Figures/MSAM.pdf}
\caption{The schematic illustration of  Mask-guided Supervised Attention Module (MSAM).}
\label{fig:MSAM}
\vspace{-2.5mm}
\end{figure}
%%%%%%%%%%%%%%%%%%%%%%%%%%%%%%%%%%%%%%%%%%%%%%%%%%%%%%%%%%%%%%%%%%%%%

\subsection{Multi-Characteristics fusion}
\label{sec:mcf}
In this section, we present a dedicated feature fusion module to aggregate physical/spatio/temporal characteristic features. 
%%
Specifically, as shown in Figure~\ref{fig:framework}, first, the physical features $\mathbf{F}_t^{ph}$ and spatio features $\mathbf{F}_t^{sp}$ are concatenated. 
%%
Then we introduce three Temporal-aware Attention Blocks (TAB) to further fuse the temporal features $\mathbf{F}_t^{te}$.
%%
It is worth noting that due to the different ways in which temporal information and spatial information are encoded, we consider using temporal features for filtering the important channels of hybrid features instead of direct concatenation.
%%
After each TAB, the generated features will also be added with the upsampled features $\{\mathbf{D}_i\}_{i=1}^{3}$ from the shadow removal decoder in the physical branch to obtain richer scale features. The output of TABs can be formulized as:
\begin{equation}
\label{Equ:TAB}
\left\{
\begin{aligned}
\mathbf{B}^{out}_{1} &=& &\mathrm{TAB}_{1}(\mathtt{Cat}(\mathbf{F}_t^{ph}, \mathbf{F}_t^{sp}), \mathbf{F}_t^{te}), \\
\mathbf{B}^{out}_{i} &=& &\mathrm{TAB}_{i}(\mathbf{B}^{out}_{i-1} + \mathbf{D}_{i-1}, \mathbf{F}_t^{te}), i = 2, 3,
\end{aligned}
\right.
\end{equation}
where $\mathrm{TAB}_{i}$ denotes each TAB block; $\mathbf{B}^{out}_{i}$ denotes the output of each TAB block.
%%
%Finally, the last hybrid features will pass a single convolution layer then add to the original input $\mathbf{I}_{t}$ to get the final shadow removal output $\mathbf{R}_{t}^{final}$. 
Finally, the last hybrid features will add to the original input $\mathbf{I}_{t}$ to get the final shadow removal output $\mathbf{R}_{t}^{final}$ as follows:
\begin{equation}\label{Equ:TAB2}  
    \mathbf{R}_t^{final} = \mathbf{I}_t + \mathtt{Conv}(\mathbf{B}^{out}_{3} + \mathbf{D}_{3}),
\end{equation}
%%
Note that the fusion module operates on the original input image resolution.
%%
$\mathbf{F}_t^{ph}$ and $\mathbf{F}_t^{te}$ should be upsampled to $W \times H$ before being fed into the fusion module. 
%%
Meanwhile, there is no downsampling operation in fusion module, thereby preserving the desired fine texture in the final output image.

\noindent
\textbf{Temporal-aware Attention Block.} TAB mainly consists of two types of channel attention blocks. Previous works~\cite{zhang2018image,zamir2021multi} demonstrate that the stacking of channel attention modules can help to achieve significant performance gains in most image restoration tasks.
%%
In our work, we follow this progressive processing manner as \cite{zamir2021multi} and consider the extra temporal features $\mathbf{F}_t^{te}$ to make the model be aware of temporal information.
%In our work, we follow this progressive feature augmented with stack channel attentions and consider the extra temporal features $\mathbf{F}_t^{te}$ to make the feature be aware of temporal information.

As illustrated in Figure~\ref{fig:TAB} (a), TAB includes $k$ self channel attention blocks (SCAB) and one cross channel attention block (CCAB), while TAB adopts residual learning to make convergence easier. In our experiments, we set $k$=$8$. SCAB takes progressive hybrid features as input, and CCAB takes progressive hybrid features and temporal features $\mathbf{F}_t^{te}$ as inputs. 
%%
Figure~\ref{fig:TAB} (b) show the details of SCAB.
%%
It first encodes shadow removal features with two $3$$\times$$3$ convolution followed by the PReLU activation. The global information is extracted via a global average pooling operation (GAP), then fed into two $1$$\times$$1$ convolutions followed by a sigmoid activation to achieve the channel attention maps. Finally, the augmented features re-calibrated by these channel attention maps will be passed to the next block.
%%
Figure~\ref{fig:TAB} (c) show the details of CCAB.
%%
It first concatenates the $\mathbf{F}_t^{te}$ to the main branch hybrid features. Next, a simple $1$$\times$$1$ convolution is used to lower the dimension of these hybrid features, then the rest process is same as SCAB. 
%%
In this way, $\mathbf{O}_{t, t+1}$ is encoded into the shadow removal features and controls the expression of temporal information.
%Figure~\ref{fig:TAB} (b) show the details of SCAB. SCAB first encodes shadow removal features with two $3$$\times$$3$ convolution followed by the PReLU activation. The global average pooling operation (GAP) is used to extract the global information, then fed them into two $1$$\times$$1$ convolution followed by a sigmoid activation to achieve the channel attention maps. Finally, augmented features re-calibrated by these channel attention maps will be passed to the next block. Figure~\ref{fig:TAB} (c) show the details of CCAB. CCAB first concatenate the $\mathbf{F}_t^{te}$ to the main branch hybrid features. Next, a simple $1$$\times$$1$ convolution is used to lower the dimension of these hybrid features, then the rest of channel attention process is same as SCAB. In this way, $\mathbf{O}_{t, t+1}$ is encoded into the shadow removal features and control the expression of temporal information.
%%%%%%%%%%%%%%%%%%%%%%%%%%%%%%%%%%%%%%%%%%%%%%%%%%%%%%%%%%%%%%%%%%%%% Figure: TAB
\begin{figure}[]
\centering
\includegraphics[width=0.5\textwidth]{Figures/TAB.pdf}
\caption{(a) is the schematic illustration of our Temporal-aware Attention Block (TAB); (b) is one of the self channel attention block (SCAB) in (a); (c) is the cross channel attention block (CCAB) in (a). }
\label{fig:TAB}
\vspace{-2.5mm}
\end{figure}
%%%%%%%%%%%%%%%%%%%%%%%%%%%%%%%%%%%%%%%%%%%%%%%%%%%%%%%%%%%%%%%%%%%%%
% As the first work for video shadow removal, we propose a baseline multi-stage network which will be introduced in detail in Sec.~\ref{sec:overview}. This network mainly consists of three sub-modules, including AEEM, MSAM, and MAB. We will go into the motivations and details of each of the three modules in Sec.~\ref{sec:aeem}-Sec.~\ref{sec:mab}
% \subsection{Overview of Our Network}
% \label{sec:overview}
% Figure~\ref{fig:framework} shows the schematic illustration of our Multi-Stage Adaptive Exposure Network, termed MSAE-Net. The intuition behind our network is to leverage complementary of physics, semantic, and motion knowledge. To do so, MSAE-Net is consists of three corresponding stages. 

% The first stage try to consider physical properties of shadow producing. Inspired by xxx, shadow removal can be considered as the physics re-exposure problem. A single linear function can fit the re-exposure process: $\mathbf{L}_{t} = w \times \mathbf{I}_{t} + b$, where $\mathbf{I}_t$$\in$$\mathcal{R}^{3 \times W\times H}$ denote t-th frame in a video, $\mathbf{L}_{t}$$\in$$\mathcal{R}^{3 \times W\times H}$ denote the corresponding re-exposure image, and scalars [w, b] are parameters need to be estimated. Consequently, the complicated shadow removal task is converted to the simple regression problem of two scalars. In this paper, we design AEEM to realize the estimation of [w, b]. Different from xxx, transformer-based AEEM consider the inconsistencies in exposure in different regions and estimate the adaptive [w, b] in each region. After stage one, we obtain the re-exposure image $\mathbf{L}_{t}$.

% The second stage is based on encoder-decoder subnetworks that learn the broad contextual information due to large receptive fields. Here, we focuses on extracting the high-level semantic information to achieve the coarse shadow removal and shadow mask segmentation. As shown in Figure~\ref{fig:framework}, $\mathbf{I}_{t}$ and $\mathbf{L}_{t}$ are concatenated to fed into an encoder, then produce the shadow removal feature $\mathbf{F}_{t}^{rem}$ and shadow mask feature $\mathbf{F}_{t}^{msk}$ through the two branches decoder. We utilize a widely-used UNet~\cite{UNet2015} structure as the encoder-decoder backbone. To obtain the coarse shadow-free estimation $\mathbf{R}_{t}^{middle}$ and shadow mask $\mathbf{M}_{t}$, we further design a Mask-guided Supervised Attention Module (MSAM). MASM provides ground-truth supervisory signals useful for shadow removal and shadow mask segmentation. In addition, MSAM enhances the shadow removal feature $\mathbf{F}_{t}^{rem}$ to $\mathbf{F}_{t}^{mrem}$ with the supervised predictions $\mathbf{R}_{t}^{middle}$ and $\mathbf{M}_{t}$ through attention mechanism.

% In last two stages, only single image $\mathbf{I}_{t}$ is considered. However, in some similar video restoration tasks~\cite{}, the use of adjacent frames has been shown to facilitate the prediction of the current frame due to the continuity of the video. The motion information brought by adjacent frames are apt to distinguishes different objects in the current frame. Here, we obtain the motion information by computing the optical flow map. Specifically, the current frame $\mathbf{I}_{t}$ and next frame $\mathbf{I}_{t+1}$ are fed into a well-trained optical flow estimation network (here, we choose FlowNet2~\cite{}) to generate the optical flow $\mathbf{O}_{t, t+1}$$\in$$\mathcal{R}^{3 \times W\times H}$. The third stage concentrates on leveraging these motion information to refine the result of shadow removal. Firstly, $\mathrm{Conv}(\mathbf{I}_{t})$ and $\mathbf{F}_{t}^{mrem}$ are concatenated to $\mathbf{F}_{t}^{1}$ as the input of third stage, where $\mathrm{Conv}$ denotes a single convolutional layer. Then we introduce three Motion-aware Attention Blocks (MAB)[$\mathrm{MAB}_{1}, \mathrm{MAB}_{2}, \mathrm{MAB}_{3}$] to augment features with aggregating the extra motion information. MAB consists of some cascade self and cross channel attention blocks like figure~\ref{fig:MAB}(a). $\mathrm{MAB}_{k}$ take $\mathbf{F}_{t}^{k}$ and $\mathbf{O}_{t, t+1}$ as input then output the augmented features, where $(k$$\in$$\{1,2,3\})$. After each MAB, the output feature will be also added the upsampled features from second stage upper decoder to obtain richer scale features. Finally, the last feature will path a single convolutional layer then add to the original input $\mathbf{I}_{t}$ to get the final shadow removal output $\mathbf{R}_{t}^{final}$. It is worth noting that, the third stage employs a subnetwork that operates on the original input image resolution (without any downsampling operation), thereby preserving the desired fine texture in the final output image.



% \subsection{Adaptive Exposure Estimation Module}
% \label{sec:aeem}
% %%%%%%%%%%%%%%%%%%%%%%%%%%%%%%%%%%%%%%%%%%%%%%%%%%%%%%%%%%%%%%%%%%%%% Figure: AEEM
% \begin{figure}[!t]
% \centering
% \includegraphics[scale=.5]{Figures/AEEM.pdf}
% \vskip -5pt
% \caption{The schematic illustration of our Adaptive Exposure Estimation Module(AEEM); See Section~\ref{sec:aeem} for details.}
% \label{fig:AEEM}
% \vspace{-2.5mm}
% \end{figure}
% %%%%%%%%%%%%%%%%%%%%%%%%%%%%%%%%%%%%%%%%%%%%%%%%%%%%%%%%%%%%%%%%%%%%%
% Recent shadow removal methods~\cite{}, based on physical shadow models, mainly learn to re-exposure the shadow image to a lit version and then fuse them together to acquire the desired shadow-free image via a shadow matte. The common of these works is that only one group of linear parameters (denote as $[\mathrm{w}, \mathrm{b}]$) is predicted for each image, which greatly reduces the solution space of the problem. However, in some scenes, due to the changes in lighting and materials, the parameters $[\mathrm{w},\mathrm{b}]$ of different regions vary greatly. Instead, we introduce an Adaptive Exposure Estimation Module(AEEM), which leverage distinguishing $[\mathrm{w},\mathrm{b}]$ between different regions. The schematic diagram of AEEM is shown in Figure~\ref{fig:AEEM}, and its contributions are two-fold. First, the input image is split into several patches and the estimations of $[\mathrm{w},\mathrm{b}]$ are generated for these patches rather than whole image. It alleviates conflicts in which $[\mathrm{w},\mathrm{b}]$ for different regions are inconsistent. Second, we utilize the transformer encoder building the long-range relationships between patches, which smooths the estimations of each $[\mathrm{w},\mathrm{b}]$.

% As illustrated in Figure~\ref{fig:AEEM}, AEEM takes the shadow image $\mathbf{I}_t$ and first split it into $N$ patches, where t denotes the current t-th frame. In our experiments, we set $N$=$4$. After arranging these patches in regular order, the patch embedding and position embedding operations are used to generate the transformer input. Then a transformer encoder is employed to product the $[\mathrm{w},\mathrm{b}]$ sequence, denoted as $\{[\mathrm{w}_t^n, \mathrm{b}_t^n]\}_{n=1}^N$. For convenience, we inherit the same transformer encoder structure as widely known ViT~\cite{}, which use several transformer blocks with cascade multi-head attention module and multi-layer perceptron.

% After obtaining $\{[\mathrm{w}_t^n, \mathrm{b}_t^n]\}_{n=1}^N$, we re-exposure the shadow image $\mathbf{I}_t$ to a lit version $\mathbf{L}_t$ by:
% \begin{equation}\label{Equ:It2Lt}  
%     \mathbf{L}_t^n = \mathrm{w}_t^n \times \mathbf{L}_t^n + \mathrm{b}_t^n,
% \end{equation}
% where $\mathbf{I}_t^n$ is n-th patch of $\mathbf{I}_t$; $\mathbf{L}_t^n$ is the corresponding lit version of $\mathbf{}_t^n$. Then, we achieve $\mathbf{L}_t$ by assembling $\{\mathbf{L}_t^n\}^N_{n=1}$ in the original order.

% \subsection{Mask-guided Supervised Attention Module}
% \label{sec:msam}
% %%%%%%%%%%%%%%%%%%%%%%%%%%%%%%%%%%%%%%%%%%%%%%%%%%%%%%%%%%%%%%%%%%%%% Figure: MSAM
% \begin{figure}[!t]
% \centering
% \includegraphics[scale=.38]{Figures/MSAM.pdf}
% \vskip -5pt
% \caption{The schematic illustration of our Mask-guided Supervised Attention Module(MSAM); See Section~\ref{sec:msam} for details.}
% \label{fig:MSAM}
% \vspace{-2.5mm}
% \end{figure}
% In this section, we will introduce our Mask-guided Supervised Attention Module(MSAM) in details. The motivations behind MSAM include third main aspects. First, multi-task learning can leverage a stronger encoder via training with multiple types of supervised labels. Second, ground-truth supervisory signals is useful for progressive shadow removal. Third, a well-segmented shadow mask can guide the network to suppress the less informative features (mainly existing in non-shadow regions) and enhance the useful features (mainly existing in shadow regions).

% As illustrated in Figure~\ref{fig:MSAM}, on the one hand, MSAM take the incoming feature $\mathbf{F}_t^{rem}$$\in$$\mathcal{R}^{C \times W\times H}$ from upper UNet decoder output in second stage and then generate the residual image $\mathbf{X}_t$$\in$$\mathcal{R}^{3 \times W\times H}$ with a simple $1$$\times$$1$ convolution, where $W\times H$ denotes the spatial dimension and $C$ is the number of channels. The residual image $\mathbf{X}_t$ is added to the input image $\mathbf{I}_t$ to obtain the coarse shadow-free estimation $\mathbf{R}_t^{middle}$. On the other hand, MSAM take the incoming feature $\mathbf{F}_t^{msk}$$\in$$\mathcal{R}^{C \times W\times H}$ from lower UNet decoder output in second stage and then generate the shadow mask prediction$\mathbf{M}_t$$\in$$\mathcal{R}^{1 \times W\times H}$. To these predicted image $\mathbf{R}_t^{middle}$ and $\mathbf{M}_t$, we provide explicit supervision with the ground-truth image. Next, per-pixel attention maps $\mathbf{F}_t^{att}$$\in$$\mathcal{R}^{C \times W\times H}$ are generated from the hybrid feature, concatenated by $\mathbf{R}_t^{middle}$ and $\mathbf{M}_t$, using a simple $1$$\times$$1$ convolution followed by the sigmoid activation. These maps are then employed to guide the transformed $\mathbf{F}_t^{rem}$ (obtained after $1$$\times$$1$ convolution), resulting in attention-guided features which are added to the identity mapping path. Finally, the attention-augmented feature $\mathbf{F}_t^{mrem}$ is passed to the third stage for further processing.
%%%%%%%%%%%%%%%%%%%%%%%%%%%%%%%%%%%%%%%%%%%%%%%%%%%%%%%%%%%%%%%%%%%%%

% \subsection{Motion-aware Attention Block}
% \label{sec:mab}
% %%%%%%%%%%%%%%%%%%%%%%%%%%%%%%%%%%%%%%%%%%%%%%%%%%%%%%%%%%%%%%%%%%%%% Figure: MAB
% \begin{figure}[!t]
% \centering
% \includegraphics[scale=.55]{Figures/MAB.pdf}
% \vskip -5pt
% \caption{The schematic illustration of our Motion-aware Attention Block(MAB); See Section~\ref{sec:mab} for details.}
% \label{fig:MAB}
% \vspace{-2.5mm}
% \end{figure}
% %%%%%%%%%%%%%%%%%%%%%%%%%%%%%%%%%%%%%%%%%%%%%%%%%%%%%%%%%%%%%%%%%%%%%
% In this section, we will introduce our Motion-aware Attention Block(MAB). MAB mainly consists of two types of channel attention blocks. Some related works~\cite{} demonstrate that the stacking of channel attention modules can facilitates achieving significant performance gain in most image restoration tasks. In our work, we follow this progressive feature augmented with stack channel attentions and consider the extra optical flow to make the feature be aware of motion information.

% As illustrated in Figure~\ref{fig:MAB} (a), MAB includes $k$ self channel attention blocks (SCAB) and one cross channel attention block (CCAB). In our experiments, we set $k$=$8$. SCAB take progressive shadow removal features as input. CCAB take progressive shadow removal features and optical flow $\mathbf{O}_{t, t+1}$ as input. MAB adpots residual learning to make convergence easier. Figure~\ref{fig:MAB} (b) show the details of SCAB. SCAB first encode shadow removal features with two $3$$\times$$3$ convolution followed by the PReLU activation. Next, the global average pooling operation (GAP) is used to extract the global information, then fed them into two $1$$\times$$1$ convolution followed by a sigmoid activation to achieve the channel attention maps. Finally, augmented features re-calibrated by these channel attention maps will be passed to the next block. Figure~\ref{fig:MAB} (c) show the details of CCAB. The dispose of shadow removal features in CCAB is similar as SCAB. To the extra input $\mathbf{O}_{t, t+1}$, CCAB first encode $\mathbf{O}_{t, t+1}$ with three $3$$\times$$3$ convolutions followed by the PReLU activation, then concatenate these features and encoded shadow removal features. Next, a simple $1$$\times$$1$ convolution is used to lower the dimension of these hybrid features, then the rest of channel attention process is same as SCAB. In this way, $\mathbf{O}_{t, t+1}$ is encoded into the shadow removal features and control the expression of motion information.

\subsection{Loss Function}
\label{sec:loss}
There are three types of loss in our network: (1) regression loss ($\mathcal{L}_{reg}$) for over-exposure parameters $\{[\mathrm{w}_t^n, \mathrm{b}_t^n]\}_{n=1}^N$ in AEEM; (2) segmentation loss ($\mathcal{L}_{seg}$) for shadow detection mask $\mathbf{F}_{t}^{msk}$ in MSAM; (3) restoration loss ($\mathcal{L}_{res}$) for shadow removal predictions $\mathbf{R}_{t}^{middle}$ in MSAM and also $\mathbf{R}_{t}^{final}$ in the fusion module. 
%%
The total loss $\mathcal{L}_{total}$ can be written as follows:
\begin{equation}\label{Equ:TotalLoss}  
    \mathcal{L}_{total} = \alpha \mathcal{L}_{reg} + \beta \mathcal{L}_{seg} + \gamma \mathcal{L}_{res},
\end{equation}
where $[\alpha, \beta, \gamma]$ is the trade-off weight. In our experiments, we set them all to 1.

Regression loss $\mathcal{L}_{reg}$ in Equ~\ref{Equ:TotalLoss} can be further formulated as:
\begin{equation}\label{Equ:RegLoss}  
    \mathcal{L}_{reg} = \sum_{n=1}^{N}{\big{(}\mathrm{\Phi}_{char}(\mathbf{w}_{t}^{n}, \mathbf{\hat{w}}_{t}^{n}) + \mathrm{\Phi}_{char}(\mathbf{b}_{t}^{n}, \mathbf{\hat{b}}_{t}^{n})\big{)}},
\end{equation}
where $\mathrm{\Phi}_{char}$ is Charbonnier loss~\cite{charbonnier1994two} (similar as L1 loss and smoother than L1 loss around zero point); 
%%
$N$ represents the patch number for AEEM (in our experiments, we set $N$=$4$); $\mathbf{w}_{t}^{n}$ and $\mathbf{b}_{t}^{n}$ are $n$-th patch estimation of $[\mathrm{w}, \mathrm{b}]$ with AEEM for the $t$-th frame; $\mathbf{\hat{w}}_{t}^{n}$ and $\mathbf{\hat{b}}_{t}^{n}$ are the corresponding ground-truth. 
%%
The generation details of $\mathbf{\hat{w}}_{t}^{n}$ and $\mathbf{\hat{b}}_{t}^{n}$ are described in Section~\ref{sec:details}.

Segmentation loss $\mathcal{L}_{seg}$ in Equ~\ref{Equ:TotalLoss} can be formulated  as:
\begin{equation}\label{Equ:SegLoss}  
    \mathcal{L}_{seg} = \mathrm{\Phi}_{bce}(\mathbf{M}_{t}, \mathbf{\hat{M}}_t),
\end{equation}
where $\mathrm{\Phi}_{bce}$ is binary cross-entropy loss~\cite{rubinstein2004cross}; $\mathbf{M}_{t}$ represents shadow mask prediction for the $t$-th frame; $\mathbf{\hat{M}}_{t}$ is the corresponding ground-truth. 
% The generation details of $\mathbf{\hat{M}}_{t}$ are described in Section~\ref{sec:details}.

Restoration loss $\mathcal{L}_{res}$ in Equ~\ref{Equ:TotalLoss} can be further written as:
\begin{equation}\label{Equ:ResLoss}  
    \mathcal{L}_{res} = \mathrm{\Phi}_{char}(\mathbf{R}_{t}^{middle}, \mathbf{\hat{R}}_t) + \mathrm{\Phi}_{char}(\mathbf{R}_{t}^{final}, \mathbf{\hat{R}}_t),
\end{equation}
where $\mathrm{\Phi}_{char}$ is Charbonnier loss; $\mathbf{R}_{t}^{middle}$ and $\mathbf{R}_{t}^{final}$ denote coarse and refined shadow removal results for the $t$-th frame, respectively; $\mathbf{\hat{R}}_{t}$ is the corresponding ground-truth which denotes the shadow free image.

% \subsection{Real world video shadow removal by synthetic-to-real(S2R) strategy}
\subsection{Model adaptation in real world scenes}
\label{sec:S2R}
In the previous section, we introduce the PSTNet for video shadow removal. Due to the lack of available video shadow pairs in real world scene, in this paper, we build a synthetic dataset SVSRD-85 (details in Section~\ref{sec:dataset}) and train models with the synthetic video shadow pairs. However, there is large domain gap between synthetic scenes and real world scenes. When we trivially apply the models that trained in synthetic scenes to the real world scenes, the models tend to fail. In this section, we propose a lightweight synthetic-to-real strategy, termed S2R, to adapt the synthetic-driven models to the real world scenes without retraining.

Figure~\ref{fig:FDA} (lower) illustrate the process of S2R. First, each frame from a real world video is adapted to the synthetic domain by Fourier Domain Adaptation (FDA)~\cite{yang2020fda}. Second, a pretrained synthetic-driven video shadow removal model (e.g. PSTNet) is used to remove shadow in synthetic domain. Finally, the deshadow image in synthetic domain is adapted to real domain by FDA again, further produce the final shadow removal result. The total process can be formulized as:
\begin{equation}\label{Equ:S2R}
    \mathbf{R}_{rea} = \mathrm{FDA}(\mathrm{PSTNet}(\mathrm{FDA}(\mathbf{I}_{rea}, \mathbf{I}_{syn})), \mathbf{I}_{rea}),
\end{equation}
where $\mathbf{I}_{rea}$ and $\mathbf{I}_{syn}$ denote the input real image and synthetic image, respectively; $\mathbf{R}_{rea}$ denotes the shadow removed result of $\mathbf{I}_{rea}$; $\mathrm{FDA}(\mathtt{source}, \mathtt{target})$ denotes the FDA operation with inputs of source and target images. S2R requires only Fourier-based style transformation for the input video. With S2R, real world video can be produced trivially by synthetic-driven video shadow removal model without retraining.

\noindent
\textbf{Fourier Domain Adaptation (FDA).}
FDA~\cite{yang2020fda} is first introduced for unsupervised domain adaptation, whereby the discrepancy between the source and target distributions is reduced by swapping the low-frequency spectrum of one with the other. FDA does not require any training to perform the domain alignment, just a simple Fourier Transform and its inverse. Here, we employ FDA to reduce the domain gap between test data (from real world) and pretrained data (from synthetic scenes). Figure~\ref{fig:FDA} (upper) illustrate the process of FDA. FDA aims to adapt the source image to the target image style. First, the RGB-based source image and target image are transformed to amplitude and phase components via Fourier transform~\cite{frigo1998fftw}, respectively. Second, the low frequency part (controlled by a hyper parameter $\delta$) of the amplitude of the source image is replaced by the counterpart of the amplitude of the target image. Finally, inverse Fourier Transform~\cite{frigo1998fftw} reassemble the source phase and the new amplitude to obtain the `source image in target style'. The details of FDA can be found in \cite{yang2020fda}.

\noindent
\textbf{Choice of reference synthetic image}
Considering each frame in real world video, in S2R framework, we need to select a reference synthetic image for FDA transformation. In order to make the image transition natural, we tend to choose reference image with small gap in color space. In this paper, we use color histogram~\cite{novak1992anatomy} to filter the reference image. Specifically, we extract the first frame of each video in SVSRD-85 and compute the color histograms of them, termed $\{h_i\}_{i=1}^{s}$, where $i$ denotes the video index, $h_i$ denotes i-th color histogram and $s$ denotes number of videos in SVSRD-85. To a real world video to be processed, we also compute the color histogram of the first frame, termed $\tilde{h}$, and then computer the similarity scores between $\tilde{h}$ and $\{h_i\}_{i=1}^{s}$ with the following formulation: $score_i = |\tilde{h}-h_i|$. Finally, We choose the candidate image corresponding to the minimum value of similarity score as the final reference synthetic image as follows:
\begin{equation}\label{Equ:score}
    \hat{i} = \mathop{\mathrm{argmin}}\limits_{i\in\{1,...s\}}|\tilde{h}-h_i|,
\end{equation}
where $\hat{i}$ is the final selected reference synthetic image.


%%%%%%%%%%%%%%%%%%%%%%%%%%%%%%%%%%%%%%%%%%%%%%%%%%%%%%%%%%%%%%%%%%%%% Figure: FDA
\begin{figure}[t]
\centering
\includegraphics[width=0.45\textwidth]{Figures/FDA.pdf}
\caption{Illustration of our proposed S2R strategy. In above figure, `PSTNet' denotes our proposed PSTNet model that pretrained in synthetic dataset (SVSRD-85). `FDA' denotes the Fourier Domain Adaptation operation~\cite{yang2020fda}.}
\label{fig:FDA}
\vspace{-2.5mm}
\end{figure}
%%%%%%%%%%%%%%%%%%%%%%%%%%%%%%%%%%%%%%%%%%%%%%%%%%%%%%%%%%%%%%%%%%%%%
\section{Experiments}
\label{sec:experiments}
\subsection{Fundamental Settings}
\label{sec:details}
\noindent
\textbf{Evaluation measures.}
Following previous works~\cite{wang2018stacked,guo2012paired,qu2017deshadownet,le2021physics,fu2021auto}, we utilize the root mean square error (RMSE) in LAB color space between the predicted shadow removal result and the ground-truth image to evaluate different shadow removal methods. 
%We directly compare our PSTNet against several state-of-the-art methods on the SVSRD-85 test dataset in quantitative and qualitative ways.

\noindent
\textbf{Comparative Methods.}
%Since there is no existing method for video shadow removal, 
We make comparison against nine state-of-the-art methods for relevant tasks, including Guo \textit{et al.}~\cite{guo2012paired}, Gong \textit{et al.}~\cite{gong2014interactive}, DSC~\cite{hu2019direction}, SID~\cite{le2021physics}, DHAN~\cite{cun2020towards}, and Expo~\cite{fu2021auto} for single image shadow removal; MPRNet~\cite{zamir2021multi} for single image restoration; EDVR~\cite{wang2019edvr} for video restoration; and BasicVSR~\cite{chan2021basicvsr} for video super-resolution. 
We utilize their public codes, and re-train these methods on the SVSRD-85 training set to produce their best results for a fair comparison.

\noindent
\textbf{Implementation Details}
Our PSTNet is end-to-end trainable and requires no pre-training. 
The networks are trained on $256$$\times$$256$ patches on two NVIDIA GTX 2080Ti by using an Adam optimizer with a batch size of 6, 200 epochs, and an initial learning rate of $2$$\times$$10^{-4}$. 
The learning rate is then steadily decreased to $1$$\times$$10^{-6}$ using a cosine annealing strategy~\cite{loshchilov2016sgdr}. 
For data augmentation, horizontal and vertical flips are randomly applied. 
Note that our SVSRD-85 dataset provides the ground-truth of shadow-free image $\mathbf{\hat{R}_t}$ and shadow mask $\mathbf{\hat{M}_t}$ for each video frame.
Moreover, like SID~\cite{le2021physics}, we generate the ground-truth $\{[\mathrm{\hat{w}}_t^n, \mathrm{\hat{b}}_t^n]\}_{n=1}^N$ for AEEM, using a least squares method regression~\cite{chatterjee1986influential} via $\mathbf{\hat{I}_t}$, $\mathbf{\hat{R}_t}$ and $\mathbf{\hat{M}_t}$.

%%%%%%%%%%%%%%%%%%%%%%%%%%%%%%%%%%%%%%%%%%%%%%%%%%%%%%%%%%%%%%%%%%%%% Figure: visual SVSRD-85
\begin{figure*}[]
\centering
\includegraphics[width=1\textwidth]{Figures/SOTA.pdf}
\vskip -5pt
\caption{Qualitative comparison between our methods and other shadow removal methods on our SVSRD-85 dataset.}
\label{fig:visual_SVSRD-85}
\vspace{-2.5mm}
\end{figure*}
%%%%%%%%%%%%%%%%%%%%%%%%%%%%%%%%%%%%%%%%%%%%%%%%%%%%%%%%%%%%%%%%%%%%%
\subsection{Comparison with the State-of-the-arts}
%%%%%%%%%%%%%%%%%%%%%%%%%%%%%%%%%%%%%%%%%%%%%%%%%%%%%%%%%%%%%%%%%%%
% \renewcommand\arraystretch{1.1}
\begin{table}[t]
\begin{center}
  \caption{Comparing our network (PSTNet) against the state-of-the-art methods on our proposed SVSRD-85 dataset.}
  \vskip -5pt
  \label{table:state-of-the-art}
  \resizebox{0.46\textwidth}{!}{%
    \begin{tabular}{c|c|c|c|c}
        \toprule[1pt]
        Method~~ $\backslash$ ~~RMSE & Year & \textbf{Shadow} & \textbf{Non-Shadow} & \textbf{All} \\
        \specialrule{0em}{1pt}{1pt}
        \hline
        \specialrule{0em}{1pt}{1pt}
        Input Image & - & 42.08 & 6.26 & 11.82 \\
        \specialrule{0em}{1pt}{1pt}
        \hline
        \specialrule{0em}{1pt}{1pt}
        Guo \textit{et al.}~\cite{guo2012paired}& 2013 & 25.54 & 13.32 & 15.18 \\
        Gong \textit{et al.}~\cite{gong2014interactive} & 2014 & 18.90 & 6.91 & 8.67 \\
        DSC~\cite{hu2019direction} & 2019 & 18.28 & 9.91 & 11.19 \\
        SID~\cite{le2021physics} & 2019 & 15.69 & 8.69 & 9.49 \\
        DHAN~\cite{cun2020towards} & 2020 & 14.97 & 9.87 & 10.49 \\
        Expo~\cite{fu2021auto} & 2021 & 16.34 & 8.66 & 9.68 \\
        MPRNet~\cite{zamir2021multi} & 2021 & 16.57 & 7.73 & 8.99 \\
        \specialrule{0em}{1pt}{1pt}
        \hline
        \specialrule{0em}{1pt}{1pt}
        EDVR~\cite{wang2019edvr} & 2019 & 16.50 & 9.47 & 10.30 \\
        BasicVSR~\cite{chan2021basicvsr} &2021 & 17.78 & 9.91 & 10.93 \\
        \specialrule{0em}{1pt}{1pt}
        \hline
        \specialrule{0em}{1pt}{1pt}
        \textbf{PSTNet(ours)} & - & \textbf{12.77} & \textbf{6.18} & \textbf{6.93}\\
        \specialrule{0em}{1pt}{1pt}
        \bottomrule[1pt]
    \end{tabular}
    }
    \vspace{-5mm}
  \end{center}
\end{table}

Table~\ref{table:state-of-the-art} lists RMSE scores of our PSTNet and compared methods at shadow pixels, non-shadow pixels, and all pixels of the whole video frames from our SVSRD-85 dataset.
%%and corresponding shadow-free videos 
And the first row shows the RMSE values of the input shadow videos without any shadow removal operation.
%%
% Can be deleted
% From these RMSE scores, we can find that the shadow removal results of our PSTNet have achieved smaller RMSE scores than state-of-the-art image shadow removal methods that of the input shadow video frame.  many get larger RMSE scores for non-shadow regions
%%
Among all the methods, our PSTNet obtains the smallest RMSE scores at shadow regions, non-shadow regions, and the whole image, which indicate that our network has better video shadow removal performance than the compared methods. 
%in both shadow and non-shadow regions, leading to the lowest RMSE in the whole image. 
%%
Specifically, compared against two encoder-decoder based image shadow removal methods (i.e., DSC~\cite{hu2019direction} and DHAN~\cite{cun2020towards}), our PSTNet outperforms DSC by 30.1\%/37.6\% RMSE in shadow/non-shadow region and outperforms DHAN by 14.7\%/37.3\% in shadow/non-shadow region. 
%%
Compared with physical-based methods SID~\cite{le2021physics} and Expo~\cite{fu2021auto}, PSTNet outperforms SID by 18.6\%/28.8\% in shadow/non-shadow region and outperforms Expo by 21.8\%/28.6\% in shadow/non-shadow region. 
PSTNet also outperforms the image restoration method MPRNet~\cite{zamir2021multi} by 22.9\%/20.0\%. 
In addition, compared with the video restoration method EDVR~\cite{wang2019edvr} and video super-resolution method BasicVSR~\cite{chan2021basicvsr}, PSTNet also outperforms EDVR by 22.6\%/34.7\% in shadow/non-shadow region and outperforms BasicVSR by 28.1\%/37.6\% in shadow/non-shadow region.




Figure~\ref{fig:visual_SVSRD-85} visually compares the shadow removal results produced by our network and other methods on the SVSRD-85 dataset. 
For the shadow on the deck of the train (first case), traditional methods (Guo \textit{et al.}~\cite{guo2012paired} and Gong \textit{et al.}~\cite{gong2014interactive}) can not distinguish this region as shadow, thereby generating the predictions that are similar as input. 
Most data-driven methods can remove the shadow of the central part to some extent, while they all tend to generate the ghost shadow over the original shadow boundary.
%%
In comparison, our PSTNet can yield reasonable exposure estimation and smooth removal result on the shadow boundary. 
%%
For the second case, the ground over the manhole cover is white, which makes the illustration of the shadow cast on it higher than the ground regions. 
Most image-based shadow removal methods cannot remove the persons' shadow due to being cheated by the high values of shadows. 
%%
However, the video-based methods like EDVR~\cite{wang2019edvr} and our PSTNet take advantage of the hints of temporal information between frames, which help to distinguish shadows from lit regions. Furthermore, the AEEM in PSTNet can help generate adaptive exposure estimation for different regions, thereby preventing from generating the discontiguous prediction between adjacent regions with different textures, like EDVR.

%%%%%%%%%%%%%%%%%%%%%%%%%%%%%%%%%%%%%%%%%%%%%%%%%%%%%%%%%%%%%%%%%%%%% Figure: visual TimeLapse (old)
% \begin{figure*}[]
% \centering
% \includegraphics[width=1\textwidth]{Figures/visual_compare_TimeLapse.pdf}
% \vskip -13pt
% \caption{Qualitative comparison between our method and other shadow removal methods on the SBU-TimeLapse dataset. Here, we compare the image-based methods (\textit{i.e.,} DHAN~\cite{cun2020towards} and SID~\cite{le2021physics}) and the video-based methods (\textit{i.e.,} BasicVSR~\cite{chan2021basicvsr} and our PSTNet). The penultimate column is the pesudo ground-truth images which remove the moving-shadows and retain the static-shadows. The last colume is the corresponding moving-shadow masks.}
% \label{fig:visual_TimeLapse}
% \vspace{-2.5mm}
% \end{figure*}
%%%%%%%%%%%%%%%%%%%%%%%%%%%%%%%%%%%%%%%%%%%%%%%%%%%%%%%%%%%%%%%%%%%%%
%%%%%%%%%%%%%%%%%%%%%%%%%%%%%%%%%%
\begin{figure*}[t]
	\centering
	\vspace*{0.5mm}
    \begin{subfigure}{0.16\textwidth} %0.162\textwidth
		\includegraphics[width=\textwidth]{./Figures/TL_visual/0-input-garden_003.png}
	\end{subfigure}
	\begin{subfigure}{0.16\textwidth}
		\includegraphics[width=\textwidth]{./Figures/TL_visual/1-SID-garden_003.png}
	\end{subfigure}
	\begin{subfigure}{0.16\textwidth}
		\includegraphics[width=\textwidth]{./Figures/TL_visual/2-DHAN-garden_003.png}
	\end{subfigure}
	\begin{subfigure}{0.16\textwidth}
		\includegraphics[width=\textwidth]{./Figures/TL_visual/3-EDVR-garden_003.png}
	\end{subfigure}
	\begin{subfigure}{0.16\textwidth}
		\includegraphics[width=\textwidth]{./Figures/TL_visual/4-PSTNet-garden_003.png}
	\end{subfigure}
	\begin{subfigure}{0.16\textwidth}
		\includegraphics[width=\textwidth]{./Figures/TL_visual/5-PSTNet-FDA-garden_003.png}
	\end{subfigure}
	\ \\
	\vspace*{0.5mm}
    \begin{subfigure}{0.16\textwidth} %0.162\textwidth
		\includegraphics[width=\textwidth]{./Figures/TL_visual/0-input-pottedplant2_041.png}
	\end{subfigure}
	\begin{subfigure}{0.16\textwidth}
		\includegraphics[width=\textwidth]{./Figures/TL_visual/1-SID-pottedplant2_041.png}
	\end{subfigure}
	\begin{subfigure}{0.16\textwidth}
		\includegraphics[width=\textwidth]{./Figures/TL_visual/2-DHAN-pottedplant2_041.png}
	\end{subfigure}
	\begin{subfigure}{0.16\textwidth}
		\includegraphics[width=\textwidth]{./Figures/TL_visual/3-EDVR-pottedplant2_041.png}
	\end{subfigure}
	\begin{subfigure}{0.16\textwidth}
		\includegraphics[width=\textwidth]{./Figures/TL_visual/4-PSTNet-pottedplant2_041.png}
	\end{subfigure}
	\begin{subfigure}{0.16\textwidth}
		\includegraphics[width=\textwidth]{./Figures/TL_visual/5-PSTNet-FDA-pottedplant2_041.png}
	\end{subfigure}
	\ \\
	\vspace*{0.5mm}
    \begin{subfigure}{0.16\textwidth} %0.162\textwidth
		\includegraphics[width=\textwidth]{./Figures/TL_visual/0-input-tree1_092.png}
	\end{subfigure}
	\begin{subfigure}{0.16\textwidth}
		\includegraphics[width=\textwidth]{./Figures/TL_visual/1-SID-tree1_092.png}
	\end{subfigure}
	\begin{subfigure}{0.16\textwidth}
		\includegraphics[width=\textwidth]{./Figures/TL_visual/2-DHAN-tree1_092.png}
	\end{subfigure}
	\begin{subfigure}{0.16\textwidth}
		\includegraphics[width=\textwidth]{./Figures/TL_visual/3-EDVR-tree1_092.png}
	\end{subfigure}
	\begin{subfigure}{0.16\textwidth}
		\includegraphics[width=\textwidth]{./Figures/TL_visual/4-PSTNet-tree1_092.png}
	\end{subfigure}
	\begin{subfigure}{0.16\textwidth}
		\includegraphics[width=\textwidth]{./Figures/TL_visual/5-PSTNet-FDA-tree1_092.png}
	\end{subfigure}
	\ \\

	\vspace*{0.5mm}
	\begin{subfigure}{0.16\textwidth}
		\includegraphics[width=\textwidth]{./Figures/TL_visual/0-input-tower_103.png}
		\vspace{-5.5mm} \caption*{{\footnotesize Input}}
        \vspace{-2mm} \caption*{\hspace*{-0.7mm}{\footnotesize images}}
	\end{subfigure}
	\begin{subfigure}{0.16\textwidth}
		\includegraphics[width=\textwidth]{./Figures/TL_visual/1-SID-tower_103.png}
		\vspace{-5.5mm} \caption*{{\footnotesize SID}}
        \vspace{-2mm} \caption*{\hspace*{-0.7mm}{\footnotesize ~\cite{le2021physics}}}
	\end{subfigure}
	\begin{subfigure}{0.16\textwidth}
		\includegraphics[width=\textwidth]{./Figures/TL_visual/2-DHAN-tower_103.png}
		\vspace{-5.5mm} \caption*{{\footnotesize DHAN}}
        \vspace{-2mm} \caption*{\hspace*{-0.7mm}{\footnotesize ~\cite{cun2020towards}}}
	\end{subfigure}
	\begin{subfigure}{0.16\textwidth}
		\includegraphics[width=\textwidth]{./Figures/TL_visual/3-EDVR-tower_103.png}
		\vspace{-5.5mm} \caption*{{\footnotesize EDVR}}
        \vspace{-2mm} \caption*{\hspace*{-0.7mm}{\footnotesize ~\cite{wang2019edvr}}}
	\end{subfigure}
	\begin{subfigure}{0.16\textwidth}
		\includegraphics[width=\textwidth]{./Figures/TL_visual/4-PSTNet-tower_103.png}
		\vspace{-5.5mm} \caption*{{\footnotesize PSTNet}}
        \vspace{-2mm} \caption*{\hspace*{-0.7mm}{\footnotesize (ours)}}
	\end{subfigure}
	\begin{subfigure}{0.16\textwidth}
		\includegraphics[width=\textwidth]{./Figures/TL_visual/5-PSTNet-FDA-tower_103.png}
		\vspace{-5.5mm} \caption*{{\footnotesize PSTNet + S2R}}
        \vspace{-2mm} \caption*{\hspace*{-0.7mm}{\footnotesize (ours)}}
	\end{subfigure}
    \ \\
    \vspace{-1.5mm}
	\caption{Visual comparison on the real world scenes (SBU-TimeLapse dataset). 1st column is the input frame from a specific video; 2nd-5th columns are video shadow removal results produced by our PSTNet and other comparison methods. These methods are trained on the synthetic SVSRD-85 dataset. The last column, PSTNet + FDA, is video shadow removal results produced by our PSTNet and then boosted by the `S2R' transfer strategy (mentioned in section~\ref{sec:S2R}).}
	\label{fig:comparison_TL}
    \vspace{-4.5mm}
\end{figure*}


%%%%%%%%%%%%%%%%%%%%%%%%%%%%%%%%%%%%%%%%%%%%%%%%%%%%%%%%%%%%%%%%%%%
% \renewcommand\arraystretch{1.1}
\begin{table}[]
\begin{center}
  \caption{Quantitative results of ablation study experiments on our SVSRD-85 dataset.}
  \vskip -5pt
  \label{table:ablation}
  \resizebox{0.46\textwidth}{!}{%
    \begin{tabular}{c|c|c|c}
        \toprule[1pt]
        Method~~ $\backslash$ ~~RMSE & \textbf{Shadow} & \textbf{Non-Shadow} & \textbf{All} \\
        \specialrule{0em}{1pt}{1pt}
        \hline
        \specialrule{0em}{1pt}{1pt}
        Input Image & 42.08 & 6.26 & 11.82 \\
        \specialrule{0em}{1pt}{1pt}
        \hline
        \specialrule{0em}{1pt}{1pt}
        Physical & 16.42 & 6.91 & 8.25 \\
        %\multicolumn{2}{c|}{Spatio+Temporal} & 14.08 & 6.71 & 7.61 \\
        Physical+Temporal & 13.92 & 6.44 & 7.36 \\
        Physical+Spatio & 13.01 & 6.38 & 7.15 \\
        \specialrule{0em}{1pt}{1pt}
        \hline
        \specialrule{0em}{1pt}{1pt}
        \textbf{Our method} & \textbf{12.77} & \textbf{6.18} & \textbf{6.93} \\
        \specialrule{0em}{1pt}{1pt}
        \hline
        \specialrule{0em}{1pt}{1pt}
        w/o AEEM & 15.69 & 6.62 & 7.95 \\
        w/o AEEM-A & 13.64 & 6.46 & 7.35 \\
         w/o MSAM & 14.01 & 6.63 & 7.67 \\
         w/o MSAM-M & 13.67 & 6.90 & 7.70 \\
         w/o FM & 14.79 & 6.88 & 7.85 \\
         %& \textbf{w/ all(ours)} & \textbf{12.77} & \textbf{6.18} & \textbf{6.93} \\
        
        \specialrule{0em}{1pt}{1pt}
        \bottomrule[1pt]
    \end{tabular}
    }
    \vspace{-5mm}
  \end{center}
\end{table}

%\multirow{4}*{\shortstack{Spatio+\\Physical+\\Temporal}} & 

\subsection{Ablation Study}
\label{sec:ablation}
We perform ablation study experiments to verify the effectiveness of three shadow characteristics and some modules of our PSTNet. 

\vspace{2mm}
\noindent
\textbf{Effectiveness of three branches.}
Here, the first baseline ``Physical'' denotes that we only exploits the physical branch of our method to remove shadows of video frames. 
%The second ``Physical+Spatio'' is reconstructed by using physical branch, spatio branch, and feature fusion module to perform the shadow removal.
%%
The second ``Physical+Temporal'' utilizes the  physical branch, temporal branch, and feature fusion module to perform the shadow removal, 
while the third ``Physical+Spatio'' combines the physical branch, the spatio branch, and the feature fusion module to removal shadows. 


Table~\ref{table:ablation} summarizes the RMSE scores of our method and three reconstructed  baseline networks on the SVSRD-85 dataset. 
%%
From the quantitative results, we can find that
``Physical+Temporal'' has smaller RMSE scores that ``Physical'' at shadow pixels, non-shadow pixels, and all pixels of the whole video frames, which demonstrates that the temporal branch helps our method to remove shadows from video frames.
%%
Moreover, the smaller RMSE scores of ``Physical+Spatio'' over ``Physical'' indicates that considering the spatio branch in our method incurs a better video shadow removal performance. 
%%
More importantly, combining the three branches in our method has the best RMSE performance. 

%have the following observations: (i) Multi-characteristics baselines outperform the single characteristic. That shows these three properties are all relevant to the video shadow removal task. (ii) Physical characteristic is most important. 
%(iii) The tailored feature fusion module can greatly improve the results, compared with the single fusion.

%The fifth to the eighth baseline `Physical+Spatio+Temporal' denote considering all three characteristics to remove shadow. 
\vspace{2mm}
\noindent
\textbf{Effectiveness of AEEM.} \  
%%
We construct a baseline (``w/o AEEM'') by removing the adaptive exposure estimation network from our PSTNet, which means that the input shadow video frame is directly fed into the subsequent encoder for feature extraction. 
%%
Apparently, our method consistently has smaller RMSE scores at shadow regions, non-shadow regions, and the whole video frame than ``w/o AEEM'', which shows that the exposure estimation via AEEM helps our network to better removal shadow pixels from video frames. In addition, we also construct a baseline (``w/o AEEM-A'') by replacing the adaptive exposure estimation with a fixed exposure estimation, just like SID~\cite{le2021physics}. With this setting, the shadow performance reduces from 12.77 to 13.64. This shows that dynamic adjustment of parameter estimation is necessary in some complex cases.

\vspace{2mm}
\noindent
\textbf{Effectiveness of MSAM.} \ 
%%
We further construct a baseline (``w/o MSAM'') by removing mask-guided supervised attention module from PSTNet. Hence, ``w/o MSAM'' directly pass the decoder output of the physical branch to the subsequent fusion module without any mask supervision enhancement operation.
%%
According to Table~\ref{table:ablation}, we can find that removing MSAM from our PSTNet degrade its video shadow removal performance due to the superior RMSE results of our PSTNet over ``w/o MSAM''. In addition, when we just remove the shadow mask prediction branch in MSAM (``w/o MSAM-M''), shadow performance reduces from 12.77 to 13.67 and the non-shadow performance reduces from. 6.18 to 6.90. This indicates that joint learning of shadow detection and removal tasks helps to improve the performance of shadow removal.

\vspace{2mm}
\noindent
\textbf{Effectiveness of multi-characteristics fusion module.} \ 
%%
Lastly, a baseline (``w/o FM'') is reconstructed by replacing the multi-characteristics fusion module of PSTNet with a concatenation operation to assemble features form three branches. 
%The last `w/all' denotes our integrated PSTNet.
Apparently, our method outperforms ``w/o F.M.'' in terms of RMSE scores at shadow, non-shadow, and all pixels of the whole video frame.
It means that combining features from three branches via our multi-characteristics fusion module enables our method to reach a better video shadow removal result.

\subsection{Video shadow removal in real world scenes.}
In section~\ref{sec:S2R}, we introduce a lightweight model adaptation strategy `S2R' to adapt the well trained synthetic-driven video shadow removal model to real world scenes. We perform the generalisation experiments on the real world videos from SBU-Timelapse~\cite{le2021physics} to evaluate the effectiveness of S2R. In our experiments, we set the hyper parameters $\delta$ (mentioned in section~\ref{sec:S2R}) to 0.01. Since accurate shadow removal ground-truths in SBU-Timelapse are not available, we only perform visual comparisons on this dataset.

Figure~\ref{fig:comparison_TL} visually compares the video shadow removal results produced by our PSTNet (5-th column) and other methods (2-4 columns). The last column (6-th column) denotes the boosted PSTNet with S2R strategy. All models are trained with SVSRD-85 and the reference images in S2R are also from SVSRD-85. It can be found that S2R allows the model to adapt better to the real world scenes. For examples, in 3rd row, due to the overall bright color, the shadows of the trees are not recognized in vanilla PSTNet. However, in S2R boosted version, most of the shadows of the trees are reduced to the color of the background. In 4th row, vanilla PSTNet and other methods can only remove part of the shadows of tower, while S2R boosted version can remove almost all shadows.

% \ACMMM{
% \subsection{Generalisation Ability}
% % %%%%%%%%%%%%%%%%%%%%%%%%%%%%%%%%%%%%%%%%%%%%%%%%%%%%%%%%%%%%%%%%%%%
% \renewcommand\arraystretch{1.1}
\begin{table}[]
\begin{center}
  \caption{Evaluate the generalizability on SBU-TimeLapse dataset.}
  \vskip -5pt
  \label{table:timelapse}
  \resizebox{0.46\textwidth}{!}{%
    \begin{tabular}{c|c|c|c}
        \toprule[1pt]
        Method~~ $\backslash$ ~~RMSE & Type & Training Dataset & \textbf{In Moving Shadow Mask} \\
        \specialrule{0em}{1pt}{1pt}
        \hline
        \specialrule{0em}{1pt}{1pt}
        Input Image & - & - & 33.80  \\
        \specialrule{0em}{1pt}{1pt}
        \hline
        \specialrule{0em}{1pt}{1pt}
        DHAN~\cite{cun2020towards} & Image & SVSRD-85 & 31.16  \\
        SID~\cite{le2021physics} & Image & SVSRD-85 & 28.01  \\
        % \specialrule{0em}{1pt}{1pt}
        % \hline
        % \specialrule{0em}{1pt}{1pt}
        BasicVSR~\cite{chan2021basicvsr} & Video & SVSRD-85 & 25.17  \\
        \textbf{PSTNet(ours)} & Video & SVSRD-85& \textbf{18.93}\\
        \specialrule{0em}{1pt}{1pt}
        \bottomrule[1pt]
    \end{tabular}
    }
    \vspace{-5mm}
  \end{center}
\end{table}

% Following SID~\cite{le2021physics}, we also conduct an experiment to evaluate the generalisation capability of the proposed PSTNet.  
% %SBU-TimeLapse dataset by comparing it against DHAN~\cite{cun2020towards}, SID~\cite{le2021physics}, and BasicVSR~\cite{chan2021basicvsr}.
% %%
% As mentioned in section~\ref{sec:dataset}, SBU-TimeLapse includes 50 videos taken by time-lapse photography. Each video contains a static scene without visible moving objects, and thus a ``max-min'' technique can be used to obtain the pseudo shadow-free frame and moving shadow mask. 
% %%
% Like SID~\cite{le2021physics}, we train our network and four state-of-the-art methods on SVSRD-85 dataset and test the trained models on the SBU-TimeLapse dataset for fair comparisons.
% %%
% The four compared methods include two image-based methods (DHAN~\cite{cun2020towards} and SID~\cite{le2021physics}), and one video-based method (BasicVSR\cite{chan2021basicvsr}).
% %and our PSTNet) are trained from SVSRD-85 dataset. %%
% %%
% %Then, We test the trained models in SBU-TimeLapse dataset and report their RMSE results in Table~\ref{table:timelapse}.
% %Apparently, our network has the smaller RMSE scores than all compared methods. 
% %It indicates that our method can better remove shadows of the SBU-TimeLapse dataset.

% Table~\ref{table:timelapse} reports the RMSE results on the moving shadow mask of our method and four state-of-the-art methods.
% %%
% Apparently, BasicVSR has the smallest RMSE result among the three compared methods, and the RMSE score is 25.17.
% %%
% Clearly, our method further outperforms BasicVSR in terms of the RMSE score, and its RMSE score is 18.93.
% It indicates that our network has a better generalization capability on the SBU-TimeLapse dataset than state-of-the-art methods.
% %The quantitative results are reported in Table~\ref{table:timelapse}. 
% %The data from SBU-TimeLapse and ISTD dataset are all come from real scenes, which should be more similar compared with the synthetic data in our SVSRD-85. 
% %However, our method still outperforms SID~\cite{le2021physics} by 5.8\% on the RMSE metric in provided moving shadow mask. 
% %It shows the strong generalisation ability of our PSTNet. 
% Moreover, Figure~\ref{fig:comparison_TL} visually compares the shadow removal results on the images from the SBU-TimeLapse dataset, and it shows our method can effectively remove the shadow pixels while other compared shadow removal methods tend to maintain many shadows in their results.

% %We also show visualisation comparison results, see Figure~\ref{fig:visual_SVSRD-85} for more details.
% }
\section{Limitations}
% 泛化性差
% 整体光线暗的时候结果不好
%%%%%%%%%%%%%%%%%%%%%%%%%%%%%%%%%%%%%%%%%%%%%%%%%%%%%%%%%%%%%%%%%%%%% Figure: failure case
\begin{figure}[]
\centering
\includegraphics[width=0.5\textwidth]{Figures/failure_case.pdf}
\vskip -3pt
\caption{Some failure cases of our method.}
\label{fig:FC}
\vspace{-2.5mm}
\end{figure}
%%%%%%%%%%%%%%%%%%%%%%%%%%%%%%%%%%%%%%%%%%%%%%%%%%%%%%%%%%%%%%%%%%%%%
Figure~\ref{fig:FC} shows some failure cases of our method in synthetic (1st row) and real world (2nd row) scenes, respectively. From the 1st row, we can find that our method PSTNet has failed in the scenes with low illumination contrast. The same happens with DHAN~\cite{cun2020towards}, which is also based on deep learning. However, the traditional method Gong et al.~\cite{gong2014interactive} is instead very good at removing the shadows from people. From the 2nd row, we can find that there is a huge gap between synthetic and real world scenes. Despite the use of some domain adaptation approach (e.g. S2R), the shadow removal results still have large deviations in real world scenes.

\section{Conclusion}
%Normally for machines to achieve a certain task such as  control of writing, one needs to design some specific model (e.g. manipulate network structure or learning objectives) to ensure the important information can be learned and used. 
To perform fine-grained control on natural text generation, usually one would anticipate a more complicated structure such as attention-based or memory-based \cite{shen2013general,weston2014memory,sukhbaatar2015end} networks. What originally surprised us is that the vanilla Seq2Seq model can still achieve token positioning simply by learning from sufficient amount of training examples with control signals. The results from this paper show that a recurrent network with encoder-decoder structure could be much more powerful than one originally expected. Through the process of training, the neurons gradually developed their own capabilities to perform different functions such as storing, triggering, counting, etc. In the next stage, we will focus on uncovering the mystery behind the training process to learn how such dynamic neuron behavior can be trained with given samples.  % It automatically learns several special skills such as counting down and triggering using implicit information provided in the controls. This paper describes a very unique work to analyze the recurrent neural network down to the neuron level, which is by no means a trivial task as one needs to carefully trace how different neurons change through time and how one affects the other. In the next stage, we will focus on uncovering the mystery behind the training process to learn how such dynamic neuron behavior can be trained through time. 
%We have discovered other types of the control signals that is learnable in a Seq2Seq model, such as the one determines the position of a specific token and another controlling the character-level length of each outputting words.  In the next stage, we will focus on uncovering the mystery behind the training process to learn how such dynamic neuron behavior can be trained through time.








%\subsection{Visualization of samples}
%After training, we compare different control signals with the same previous sequence to see how output sequences vary. The output for 5 different control signals are shownin table \ref{Examples} and the hidden state of encoder and decoder are shown in Figure \ref{3d-embedding}.



%\subsection{Generalization}
%In this section, we generalize our assign word method to various case. For example, put the assign control signal in the front of input sentence, assign multiple words, and rhyme, POS tasks. Each task and its accuracy is describe in appendix. 



%Rhyme & 99.62 & \makecell{Length / rhyme controls\\
%		control the length and the rhyme \\of the output sentence}& 
%		\makecell{SOS it s a wonderful face EOS \\ IY1 NOE 7 NOR \\
%		 $\rightarrow$ SOS and it means something \\ special to me EOS \\
%	     SOS i m your song EOS NG NOE 10 NOR \\
%		 $\rightarrow$ SOS play me time and time again and \\ make me strong EOS} \\
%		\midrule

\bibliographystyle{IEEEtran}
\bibliography{sample-base}

\begin{IEEEbiography}[{\includegraphics[width=1in,height=1.25in,clip,keepaspectratio]{Figures/photos/czh.jpg}}]{Zhihao Chen}
received the B.S. degree in software engineering from the Tianjin University, Tianjin, China, in 2017. He is currently pursuing the Ph.D. degree in College of Intelligence and Computing from the Tianjin University, China. His research interests include computer vision and deep learning, specifically for shadow detection shadow removal, and medical image segmentation.
\end{IEEEbiography}

\begin{IEEEbiography}[{\includegraphics[width=1in,height=1.25in,clip,keepaspectratio]{Figures/photos/wl.jpg}}]{Liang Wan}
% is a full Professor in the College of Intelligence Computing, and deputy director of Medical College, Tianjin University, P. R. China. She obtained a Ph.D. degree in computer science and engineering from The Chinese University of Hong Kong in 2007,and worked as a PostDoc Research Associate/Fellow at City University of Hong Kong from 2007 to 2011. Her current research interests focus on image processing and computer vision, including image segmentation, low-level image restoration, and medical image analysis.
is a full Professor in the College of Intelligence Computing, and deputy director of Medical College, Tianjin University, P. R. China. She obtained a Ph.D. degree in computer science and engineering from The Chinese University of Hong Kong in 2007, and worked as a PostDoc Research Associate/Fellow at City University of Hong Kong from 2007 to 2011. Her current research interests focus on image processing and computer vision, including image segmentation, low-level image restoration, and medical image analysis.
\end{IEEEbiography}

\begin{IEEEbiography}[{\includegraphics[width=1in,height=1.25in,clip,keepaspectratio]{Figures/photos/xyf.png}}]{Yefan Xiao}
received the bachelor degree in software engineering from the Tianjin University, Tianjin, China, in 2020. He is currently pursuing the master degree in Tianjin University. His research interests include computer vision and deep learning, specifically for image polyp segmentation and video semantic segmentation.
\end{IEEEbiography}

\begin{IEEEbiography}[{\includegraphics[width=1in,height=1.25in,clip,keepaspectratio]{Figures/photos/zl.png}}]{Lei Zhu}
received the Ph.D. degree from the Department
of Computer Science and Engineering, The
Chinese University of Hong Kong. He is currently
working as an Assistant Professor with the ROAS
Thrust, HKUST (GZ), and also an affiliated Assistant
Professor in ECE with HKUST. Before that,
he was a Postdoctoral Researcher at DAMTP, University
of Cambridge. His research interests include
computer graphics, computer vision, medical image
processing, and deep learning.
\end{IEEEbiography}

\begin{IEEEbiography}[{\includegraphics[width=1in,height=1.25in,clip,keepaspectratio]{Figures/photos/fhz.png}}]{Huazhu Fu}
(SM'18) is a senior scientist at Institute of High Performance Computing (IHPC), A*STAR, Singapore. He received his Ph.D. from Tianjin University in 2013. Previously, he was a Research Fellow (2013-2015) at NTU, Singapore, a Research Scientist (2015-2018) at I2R, A*STAR, Singapore, and a Senior Scientist (2018-2021) at Inception Institute of Artificial Intelligence, UAE. His research interests include computer vision, AI in healthcare, and trustworthy AI. He received the Best Paper Award from ICME 2021. He has served as the AE of IEEE TMI, IEEE TNNLS, and IEEE JBHI, AC/Senior-PC for MICCAI, IJCAI, and AAAI. He is also a Member of the IEEE BISP TC.
\end{IEEEbiography}

\newpage
\vfill

\end{document}


