\section{Limitations}
% 泛化性差
% 整体光线暗的时候结果不好
%%%%%%%%%%%%%%%%%%%%%%%%%%%%%%%%%%%%%%%%%%%%%%%%%%%%%%%%%%%%%%%%%%%%% Figure: failure case
\begin{figure}[]
\centering
\includegraphics[width=0.5\textwidth]{Figures/failure_case.pdf}
\vskip -3pt
\caption{Some failure cases of our method.}
\label{fig:FC}
\vspace{-2.5mm}
\end{figure}
%%%%%%%%%%%%%%%%%%%%%%%%%%%%%%%%%%%%%%%%%%%%%%%%%%%%%%%%%%%%%%%%%%%%%
Figure~\ref{fig:FC} shows some failure cases of our method in synthetic (1st row) and real world (2nd row) scenes, respectively. From the 1st row, we can find that our method PSTNet has failed in the scenes with low illumination contrast. The same happens with DHAN~\cite{cun2020towards}, which is also based on deep learning. However, the traditional method Gong et al.~\cite{gong2014interactive} is instead very good at removing the shadows from people. From the 2nd row, we can find that there is a huge gap between synthetic and real world scenes. Despite the use of some domain adaptation approach (e.g. S2R), the shadow removal results still have large deviations in real world scenes.
