\section{Discussion}


Longitudinal studies are needed to understand the evolution of the state of D\&I and to discern if diversity initiatives have an impact. As part of our longitudinal study sponsored by the Apache diversity project, here we provide a pulse check of the state of D\&I across two surveys (across a three-year time period). Our results show improvements in the state of D\&I, which indicates the Apache projects are on their way to being more inclusive. In the following, we discuss where the results indicate improvements and where they indicate that progress is still needed.

\subsection{The state of D\&I is Improving}

Our findings indicate that the hurdles arising because of biases and stereotypes are diminishing, particularly for underrepresented groups such as respondents of gender-minority and respondents with limited English proficiency. We posit that this is due to increased awareness of the importance of D\&I among the public as well as the Apache's efforts to make the project more inclusive, such as its mentoring program and its efforts to create a welcoming workplace \cite{EDIgroup}. 

\textbf{Gender inclusive:} A key impact of the diversity gap in OSS and role stereotype is disparity in becoming a contributor and having contributions accepted~\cite{guizani2022debug}. Our results show that from 2020 to 2022 there is a marked improvement (30.52\%) among gender-minority respondents in making code-related contributions as well as improvements in their perception about their ability to contribute. 6.97\% fewer respondents in this class reported facing challenges frequently. Fewer participants also reported facing role stereotyping. 

One form of role stereotyping is that women are considered to be warm and nurturing and seen as parental figures. They end up taking a larger share of community organizing and mentoring ~\cite{trinkenreich2020hidden}. However, our survey indicates that more men are taking up mentoring activities and seen as parental figures.  

%When people face the term ``D\&I'', gender is almost often the first factor that springs to mind  In our results, gender was no longer a significant impact attribute when it came to contributors' perceptions of D\&I and their levels of engagement in different activities. We see that a more substantial proportion of gender-minority contributors agree on equality for their contributions. For instance, respondents of the non-majority gender were less engaged in mentoring and caregiving but more in code-related activities. 
%since men contributors are helping with burden-bearing. 

\textbf{Geo-location, English Proficiency:} Results indicate that respondents who are not from western countries and are not proficient in English are increasingly engaging in the majority of the activities. Further, the challenges and barriers that arise because of these demographic aspects are on the decline. This increase in their activities aligns with their perception of being able to contribute to OSS. As with gender-minority, fewer respondents in these categories reported facing challenges in 2022 as compared to 2020. More respondents in these categories also perceived that the PMC is diverse and they are represented in decision making.


\subsection{Progress is still needed}

\textbf{Developing a gender inclusive community:} There still exists a significant gap between the number of respondents who identify as men versus gender-minority respondents. Additionally, many biases and gender-barriers are entrenched in OSS, as men have (and still) dominate OSS community ~\cite{trinkenreich2022women}. Improving the state of D\&I is therefore a long-term goal. While there are improvements in OSS projects being gender inclusive, there is still more to be done. OSS community should continue in their outreach and awareness program to attract more contributors who are from gender minority.
In particular, OSS could be intentional in promoting and mentoring women to leadership roles, which will also help attract and retain newcomers from underrepresented groups~\cite{bosu2019diversity}.

\textbf{Welcoming newcomers to OSS:} How to appropriately and effectively onboard newcomers is always an ongoing challenge in OSS. OSS communities are trying hard to utilize mentoring programs for training future newcomers, such as Google Summer of Code \cite{googlecode,silva2020theory} and the Linux mentoring program \cite{linux}. Indeed, regardless of their demographic backgrounds, we found that respondents are mentoring other contributions more frequently. However, fewer respondents concurred that locating a desirable mentor is easy. 
Potential causes for this problem could be time zone conflicts, mentor-mentee interest mismatches, or negative sentiments about receiving negative feedback \cite{balali2018newcomers}.   

We believe there are two ways to mitigate this problem. First, promote and acknowledge mentoring activities. For example, Feng et al. \cite{feng2022case} presented a mechanism to identify implicit mentoring, where topical mentoring is done through code reviews. In their study,  over 90\% of their survey respondents found such mentoring to be more effective and beneficial for both mentors and mentees as compared to formal mentoring programs. OSS projects should, therefore, acknowledge and highlight such implicit mentors. Second, in addition to formal mentoring programs, OSS community should investigate organizing informal mentoring events where mentors and mentees can organically meet and create mentorship relations. 

%To mitigate this problem, as suggested by Feng et al. \cite{feng2022case}, the OSS community should promote implicit mentoring as an alternative, cheap, and more effective solution for training newcomers and experienced contributors. Indeed, over 90\% of respondents agreed this type of mentoring is more effective and beneficial to both mentors and mentees.

\textbf{Intersectionality and Multidimensionality:}
Our analysis revealed that the state of diversity in OSS is improving across multiple attributes. However, to further enhance the state of D\&I in OSS, it is essential to consider intersectionality, where contributors can be identified by multiple attributes, such as a woman contributor who is also a newcomer and lacks English confidence. Future research should investigate how these factors interact and impact contributors' daily OSS activities. For example, In what ways do gender minority senior contributors impact mentoring activities? How do gender minority contributors who are not confident in English experience challenges, and which D\&I interaction patterns are more likely to face challenges? Additionally, do perceptions associated with both seniority and English proficiency exacerbate their feeling of being patronized?

As diversity is a multidimensional concept, a unidimensional perspective is insufficient to comprehend the state of D\&I in OSS community in its entirety. To investigate the interplay between various D\&I attributes, a multidimensional perspective is necessary. Thus, the community can more effectively adapt and develop strategies to promote diversity and inclusion. Moreover, our survey design  and research approach can serve as a useful starting point for future research to explore how to tailor strategies to promote inclusivity in OSS for different groups of contributors with diverse intersectional identities.