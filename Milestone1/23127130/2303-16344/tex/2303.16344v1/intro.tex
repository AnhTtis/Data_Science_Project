%!TEX root = main.tex

\section{Introduction}
\label{sec:intro}


%Open Source Software (OSS) projects are self-organized, volunteer-driven communities where contributors from all over the world collaborate to create and maintain large, complex software \cite{de2019recommending, ducheneaut2005socialization, courant2006software, zhao2004user}. 
Improving the state of Diversity and Inclusion (D\&I) has become an increasingly important mandate for Open Source Software (OSS) communities \cite{bosu2019diversity, guizani2022perceptions, trinkenreich2022women}. A lack of diversity can result in projects missing out on a broad range of backgrounds, qualifications, and perspectives. Increasing diversity is also a ``social good" mandate, as a lack of diversity means that individuals from underrepresented groups miss out on the learning and potential job opportunities afforded by OSS~\cite{marlow2013impression, singer2013mutual}.


% diversity is a multifaceted concept
Diversity is a multifaceted concept. Projects can be diverse in terms of demographics such as age, gender, seniority, or ethnicity, as well as participant backgrounds (role, expertise, personality, or cognitive styles) \cite{guizani2022perceptions, ortu2017diverse, prana2021including}. While prior studies have looked at D\&I in OSS, these studies have typically focused on specific aspects of diversity \cite{marlow2013impression, singer2013mutual, prana2021including, trinkenreich2022women, guizani2022perceptions} and provide a single snapshot in time. 


% no longitudinal study
OSS communities have started several initiatives to improve the state of D\&I. For example, the Linux \cite{linux} and Apache Software  Foundations (ASF) \cite{EDIgroup} have launched D\&I initiatives that implement mentorship, scholarship, training, and certification programs and promote D\&I best practices such as inclusive naming. 

Thus far, to the best of our knowledge, no research empirically investigates the evolution of the state of D\&I in OSS. Without a pulse check on the state of D\&I, OSS community lacks an understanding of what has improved and what requires more attention. In our work, we investigate the following: 

\noindent\textbf{RQ: How much do contributors from minorities...}
\begin{enumerate}[wide=0pt, start=1,label={\bfseries Sub-RQ\arabic*:}]
\item ...engage in OSS activities?
\item ...face challenges?
\item ...perceive the state of D\&I? 
\end{enumerate}

To answer the research questions, we performed a longitudinal case study within the ASF projects. Thus far, we have conducted two cross-sectional surveys, one in 2020 and the other in 2022. The surveys use the same Likert-scale questions to allow us to evaluate the difference in contributors' perspectives and engagement over time, serving as a pulse check. 

Our work complements prior research by adopting multidimensional attributes to investigate changes in contributors' perceptions of diversity through four perspectives (gender, proficiency in English, seniority, and country of residence), as these four perspectives are significant and have been demonstrated to impact D\&I in OSS.  We believe deploying a pulse check on this multidimensional approach provides a nuanced and comprehensive understanding of the complex factors that lead to D\&I in OSS communities.


Our findings provide an in-depth analysis of the present state of D\&I in OSS communities. Specifically, we offer insights into the following: 1) a pulse check of the state of D\&I engagement across five distinct OSS daily activities; 2) the current state of challenges faced by contributors regarding their D\&I backgrounds; 3) the perception of the contemporary state of D\&I in four areas: role stereotypes, the ability to contribute, being represented, and following a code of conduct.




This pulse check can help OSS communities make informed decisions about proceeding with their D\&I initiatives. Understanding the state of D\&I in OSS can inform practitioners when developing and adjusting programs and initiatives to promote contributor diversity, ultimately benefiting OSS community and software industry by increasing diversity and equity. Additionally, studying the evolution of D\&I in OSS can help researchers in OSS understand inclusive community development. It can assist in uncovering diversity and inclusion factors and inform the design of future studies and experiments, leading to more effective and efficient research in OSS. These insights can also improve future research by informing the design of the study methodology.


