\section{Results And Recommendations}

We performed a pulse check on (1) engagement in activities (Q5), (2) challenges in OSS (Q7), and (3) perception of the state of D\&I (Q10) based on our demographic attributes of interest (gender, seniority, English proficiency, and region). In the following, we present the comparisons of the two editions of the survey and provide recommendations for practice. While there is still a significant gap in the representation of minority groups across all demographic attributes (\Cref{tab:demo}), in this paper, we focus on the participation trends within each group to investigate the evolving state of D\&I in OSS. 

\subsection{Engagement in OSS activities}

\begin{table*}[tbp]
\caption{Comparison of proportional activity engagement between 2020 and 2022 disaggregated by diversity lens.}
\centering
\resizebox{7in}{!}{
\begin{tabular}{lrl|rl|rl|rl|rl|rl|rl|rl}
 & \multicolumn{2}{c|}{\textbf{Man}} & \multicolumn{2}{c|}{\textbf{Minority gender}} & \multicolumn{2}{c|}{\textbf{Newcomer}} & \multicolumn{2}{c|}{\textbf{Not newcomer}} & \multicolumn{2}{c|}{\textbf{\begin{tabular}[c]{@{}c@{}}Comfortable using \\ English\end{tabular}}} & \multicolumn{2}{c|}{\textbf{\begin{tabular}[c]{@{}c@{}}Not comfortable using \\ English\end{tabular}}} & \multicolumn{2}{c|}{\textbf{\begin{tabular}[c]{@{}c@{}}From western \\ countries\end{tabular}}} & \multicolumn{2}{c}{\textbf{\begin{tabular}[c]{@{}c@{}}Not from western \\ countries\end{tabular}}} \\
\rowcolor[HTML]{EFEFEF} 
\textbf{\begin{tabular}[c]{@{}l@{}}Contributing/ reviewing \\ code\end{tabular}} & \greenup & {\color[HTML]{228B22} 1.62\%} & \greenup & {\color[HTML]{228B22} 30.52\%} & \greenup & {\color[HTML]{228B22} 19.87\%} & \greenup & {\color[HTML]{228B22} 3.20\%} & \greenup & {\color[HTML]{228B22} 3.97\%} & \greenup & {\color[HTML]{228B22} 3.41\%} & \greenup & {\color[HTML]{228B22} 2.91\%} & \greenup & {\color[HTML]{228B22} 7.37\%} \\
\textbf{\begin{tabular}[c]{@{}l@{}}Creating/maintaining \\ documentation\end{tabular}} & \greenup & {\color[HTML]{228B22} 3.34\%} & \greenup & {\color[HTML]{228B22} 5.95\%} & \reddown & {\color[HTML]{C70039} -4.64\%} & \greenup & {\color[HTML]{228B22} 4.56\%} & \greenup & {\color[HTML]{228B22} 2.24\%} & \greenup & {\color[HTML]{228B22} 11.13\%} & \greenup & {\color[HTML]{228B22} 0.74\%} & \greenup & {\color[HTML]{228B22} 18.44\%} \\
\rowcolor[HTML]{EFEFEF} 
\textbf{\begin{tabular}[c]{@{}l@{}}Participating in decision \\ making about the project \\ development\end{tabular}} & \greenup & {\color[HTML]{228B22} 4.13\%} & \reddown & {\color[HTML]{C70039} -2.41\%} & \greenup & {\color[HTML]{228B22} 7.68\%} & \greenup & {\color[HTML]{228B22} 1.02\%} & \greenup & {\color[HTML]{228B22} 3.68\%} & \greenup & {\color[HTML]{228B22} 1.67\%} & \greenup & {\color[HTML]{228B22} 0.35\%} & \greenup & {\color[HTML]{228B22} 19.85\%} \\
\textbf{\begin{tabular}[c]{@{}l@{}}Serving as a community \\ organizer\end{tabular}} & \greenup & {\color[HTML]{228B22} 2.46\%} & \reddown & {\color[HTML]{C70039} -11.07\%} & \reddown & {\color[HTML]{C70039} -9.34\%} & \greenup & {\color[HTML]{228B22} 0.60\%} & \reddown & {\color[HTML]{C70039} -0.43\%} & \reddown & {\color[HTML]{C70039} -2.23\%} & \reddown & {\color[HTML]{C70039} -3.24\%} & \greenup & {\color[HTML]{228B22} 17.98\%} \\
\rowcolor[HTML]{EFEFEF} 
\textbf{\begin{tabular}[c]{@{}l@{}}Mentoring other \\ contributors\end{tabular}} & \greenup & {\color[HTML]{228B22} 7.54\%} & \greenup & {\color[HTML]{228B22} 4.16\%} & \greenup & {\color[HTML]{228B22} 13.04\%} & \greenup & {\color[HTML]{228B22} 5.75\%} & \greenup & {\color[HTML]{228B22} 5.81\%} & \greenup & {\color[HTML]{228B22} 25.56\%} & \greenup & {\color[HTML]{228B22} 4.02\%} & \greenup & {\color[HTML]{228B22} 23.90\%}
\end{tabular}}
\label{tab:engagement}
\end{table*}



\Cref{tab:engagement} shows the proportion differences between the answers to the two surveys, disaggregated by the diversity attributes. We considered engagement in OSS activities in which respondents responded ``often" (more than once a week or more than once a month). The {\greenup} and {\reddown} indicate, respectively, a percentage increase and decrease between responses in 2020 and 2022. 

We can see an overall positive trend in \Cref{tab:engagement}, with more respondents engaging in the listed activities. The one exception is serving as community organizers, which sees a reduction. The maximum decrease is for those from gender minorities (11.07\%). This means that 11.07\% fewer respondents, who considered themselves as gender minority, participated as community organizers in 2022 compared to 2020. A reduction in this activity also occurs for newcomers and western respondents. This can be an impact of the COVID-19 pandemic with public event cancellations and travel limitations \cite{ford2021tale}, which might have affected the answers to the 2022 survey. However, a higher proportion of respondents from non-western countries served as community organizers (17.98\% increase).

Next, we discuss the engagement of contributors per demographic attributes. 

\textbf{Gender:} A bias against women arises from role incongruity---widespread cultural associations link men, but not women, with the intellectual aptitude required to work on computer science and OSS \cite{singh2019open, leslie2015expectations}. Such bias can result in fewer women making code contributions. It is heartening to see that when considering activities related to contributing or reviewing code, there is a 30.52\% increase in gender minority participation in 2022 compared to 2020 (there is also a slight increase--1.62\%--among men contributors who participate in contributing/reviewing code). 

There is also an increase among all genders in creating or maintaining documentation. This can be an instance of more contributors, perhaps newcomers, who are participating in OSS in a non-code capacity. The importance of non-code contributions, especially by women contributors, has been increasingly getting visibility and recognition \cite{trinkenreich2020hidden}.

Another bias that can impede women contributors is that women are stereotyped as nurturing, caring, and protective. As a result, women contributors are expected to play the primary caregiver role in their communities~\cite{ kaplan1994woman}, which can take away time from making code-related contributions~\cite{balali2018newcomers}. Mentoring is a role often connected to this stereotype. We see an increase in women mentors (4.16\%), but there is a bigger increase among men mentors (7.54\%). This possibly reflects an increased appreciation of the importance of mentoring activities among contributors from Apache projects, a trend that can help attract and retain newcomers. 

Despite the progress on these two fronts, more must be done. We saw a decline in gender minority respondents' engagement with project development decision-making (2.41\%). This reflects the need to promote women to leadership positions, which is considered an effective solution to foster diversity \cite{bosu2019diversity}. We recommend that the PMC and the board members actively engage those in gender minorities in their decision-making at the project and foundation levels. This strategy was also cited by Trinkenreich et al. \cite{trinkenreich2022women}.

\begin{mdframed}[roundcorner=10pt,nobreak=true]
\textbf{Observation 1:} More respondents from gender minorities are engaging in both code and non-code contributions, but there is an increasing gap in participation in decision-making activities.
\end{mdframed}

\textbf{Seniority:} There is an increase in engagement across all listed activities among non-newcomer respondents (seniority$>$1 year), ranging from 1\% to 6\% increase. The proportion of newcomer respondents who engage in code-related activities increased by 19.87\%, which may allude to a healthy trend of Apache projects being able to attract newcomers and ensure they succeed in their contributions. Additionally, newcomer respondents are becoming increasingly involved in decision-making (7.68\%) and mentoring others (13.04\%). OSS contributors frequently contribute to many projects and shift between them ~\cite{jergensen2011onion}. In such instances, they may acquire the necessary skills and experience from other sources and continue to share their knowledge while contributing to OSS, regardless of their seniority within the project. Such mentoring activities have been defined as implicit mentoring, where contributors guide each other in everyday OSS activities such as code review \cite{feng2022case}. This implies that OSS communities are not only recruiting newcomers but also newcomers feel engaged and become community members by implicitly mentoring other contributors.




One point of concern could be the decrease in the percentage of newcomers participating in creating or maintaining documentation (4.64\%), along with a 4.56\% increase among senior contributors. On the one hand, these results show that senior contributors are engaging more in non-code contributions, which is good for the sustainability of the community. On the other hand, it can become a cause of concern if the documentation ends up being written from the point of view of senior contributors, making it harder for newcomers. The literature shows that a lack of documentation and roadmap to participation and ambiguous and outdated documentation impedes newcomers \cite{steinmacher2014hard, steinmacher2015social}. We recommend that OSS projects review their documentation to ensure newcomers' documentation needs are met. Involving newcomers in these activities can help both to engage the newcomers and to ensure the documentation is accessible for those outside the project.

%Lack of documentation/road-map, particularly for contributors from different regions, can be related to the ``poor engineering environment'' that discourages contributions\cite{prana2020including}. Spread, ambiguous, and outdated documentation are impediments for newcomers \cite{steinmacher2014hard}. However, as the proportion of newcomer respondents who frequently engage in creating and maintaining documentation decreases by 4.64\%, the obstacles caused by documentation may continue to intensify, as junior and senior employees' understanding of what newcomers want is likely lacking without newcomers' participation.  

\begin{mdframed}[roundcorner=10pt,nobreak=true]
\textbf{Observation 2:} Newcomers engage more actively in code contribution, code review, mentoring other contributors, and decision-making processes. However, they tend to be less involved in community organizations and documentation-related activities.
\end{mdframed}

\textbf{English proficiency:} English is the most common language in  Apache projects. Therefore, we use it as a diversity attribute, as contributors who are not proficient in English can face additional barriers to their efforts \cite{guizani2021long, balali2018newcomers}. 

Contributors who considered themselves not proficient in English showed higher engagement across all activities (except, as already discussed, in the community organization). The largest increase (25.56\%) was in mentoring others. This can be a side effect of contributors feeling more comfortable seeking help from those with similar English proficiency levels. However, as this is an anonymous survey, we cannot associate the demographics of mentors with that of mentees, which can be a future investigation. We found that contributors who are proficient in English did not reveal significant variations in activity engagement between the two surveys (less than 6 percent).


Respondents not comfortable using English were also more involved in creating/maintaining documentation (11.13\%) and decision-making (1.67\%). This implies that the OSS community is becoming more inclusive of contributors whose native language is not English. 


\begin{mdframed}[roundcorner=10pt,nobreak=true]
\textbf{Observation 3:} Our results show that English proficiency seems to be becoming less of a barrier to participation in project activities, as non-native English-speaking respondents are increasingly involved in creating and maintaining project documentation and mentoring other contributors.
\end{mdframed}

\textbf{Geo-location:} A majority of Apache projects that successfully incubate are from North America and Europe, as per the Apache D\&I committee. Apache projects are actively looking at strategies to promote the incubation success of projects from Asian countries, especially China. In addition to the incubation process challenges that contributors face, such as ``compliance and project ascension to top level'' \cite{guizani2021long}, language barriers, cultural differences, and communication styles can impede contributors from non-western-centric countries \cite{chung2013time, steinmacher2016overcoming, guizani2021long}. 


We see an increase in the proportion of non-Western respondents engaged in the OSS activities, including being community organizers (an increase of 17.98\%, the largest increase in this activity across all demographic attributes). Most other activities also had double-digit improvements; a 23.9\% increase in being mentors, a 19.85\% increase in participating in decision-making, and an 18.44\% improvement in creating/maintaining documents. These numbers show that Apache projects are becoming more inclusive of contributors from non-western countries. The smallest improvement was in contributing to code/reviewing (7.37\%). These results show greater improvements in non-code related activities, which can provide a good pathway toward more code-related activities~\cite{trinkenreich2020hidden}. 

\begin{mdframed}[roundcorner=10pt,nobreak=true]
\textbf{Observation 4}: There is increased participation from respondents from non-western countries across all activities.
\end{mdframed}

\begin{figure}[t]
\centering
\setlength{\tabcolsep}{0.5pt}
\renewcommand{\arraystretch}{0.3}
\begin{tabular}{cccc}
\centeredtab{\rotatebox[origin=c]{90}{\scriptsize{Mip-NeRF 360}}} &
\centeredtab{\includegraphics[width=0.32\columnwidth]{fig/teaser/color_002.jpg}} &
\centeredtab{\includegraphics[width=0.32\columnwidth]{fig/teaser/color_053.jpg}} &
\centeredtab{\includegraphics[width=0.32\columnwidth]{fig/teaser/color_119.jpg}} \\
\centeredtab{\rotatebox[origin=c]{90}{\scriptsize{BARF}}} &
\centeredtab{\includegraphics[width=0.32\columnwidth]{fig/teaser/rgb_2.jpg}} &
\centeredtab{\includegraphics[width=0.32\columnwidth]{fig/teaser/rgb_53.jpg}} &
\centeredtab{\includegraphics[width=0.32\columnwidth]{fig/teaser/rgb_119.jpg}} \\
\end{tabular}%
\caption{
\textbf{Challenges of casual videos.} 
The scene's length prevent Mip-NeRF 360~\cite{barron2022mipnerf360}'s global scene parametrization from maintaining high resolution throughout the video and BARF~\cite{lin2021barf} fails to estimate poses accurately, leading to poor synthesis results.
}
\label{fig:challenges}
\end{figure}

\subsection{Challenges to contributing}


Past works have shown that respondents from non-majority groups face challenges in contributing to OSS \cite{steinmacher2016overcoming, terrell2017gender, singh2019open, prana2021including}. This is true for newcomers and existing contributors. Guizani et al. \cite{guizani2021long} categorized the challenges that contributors face into three groups: (1) technical challenges, which are related to technical hurdles with the project code, its infrastructure, or lack of documentation; (2) social challenges, which are related to the communication styles and (unwelcoming) project culture; and (3) process-related challenges, which are related to navigating the contribution process, getting started, and licenses. Here we build on this categorization and analyze if there was any improvement (or worsening situation) to the frequency of challenges contributors face, disaggregated by the demographic aspects. 

Overall, there was a reduction in the percentage of respondents who mentioned frequently facing challenges. \Cref{tab:challenge} shows that fewer respondents across all different demographic groups reported frequently facing challenges. Among the majority groups, the percentage reduction ranged between about 3\% to 5\%; 2.99\% fewer seniors and 4.60\% respondents from western countries reported facing challenges frequently.

The reduction in proportions was higher for the minority groups. The decline was highest among respondents from newcomers (8.93\%) followed by gender minority groups (6.97\%). This signals that the programs and initiatives to improve gender diversity and attract newcomers to Apache projects have a positive impact. The proportion of respondents from non-western countries and lower English proficiency reduced too, but in the 4\% range. We recommend OSS projects review its documentation and communication processes to further reduce challenges faced by contributors who are non-native English speakers. 

\begin{mdframed}[roundcorner=10pt,nobreak=true] 
\textbf{Observation 5}: The challenges faced by respondents have reduced across all demographic groups; the reduction is higher for those in minority groups.
\end{mdframed}

\begin{figure}[!tbp]
\centering
\includegraphics[width=3.5 in]{pic/challenges.png}
\caption{The three types of challenges that contributors face disaggregated by demographics in the 2022 survey.}
\label{fig:challenges}
\end{figure}


Next, we delve deeper into the types of challenges that contributors from minority groups are still facing. The goal of this pulse check on the types of challenges is to help the communities identify challenge ``hot spots", so they can prioritize their inclusion strategies.

\Cref{fig:challenges} shows three types of challenges that contributors from minority demographic groups faced, according to the 2022 survey. While the 2020 survey asked for challenges in an open-ended question, the 2022 survey used Likert scale questions on specific types of challenges based on Guizani et al.'s conceptual model of the challenges faced by contributors \cite{guizani2021long} (see Q6 in \cref{tab:survey}).


\textit{Technical challenges} were the most prevalent across all categories. The highest incidence is for newcomers (57.14\% respondents indicated that they ``often'' confront technical challenges). While newcomers facing technical challenges is to be expected, since they have to learn the project code and its infrastructure, a 57\% proportion is concerning. This was followed by gender-minority groups (45.95\%), which can be a result of OSS projects and documents not supporting the cognitive styles favored by women \cite{mendez2018open}. This was followed by respondents not proficient in English (31.82\%) and those from not western countries (26.53\%). These results indicate that perhaps the documentation can be more language-inclusive (e.g., not using idioms, jargon) to make it easier to comprehend. We recommend that OSS projects perform an inclusivity evaluation of its project and documentation, perhaps using the GenderMag method as done by Guizani et al. \cite{guizani2022debug}.

\textit{Social challenges} were more prominently reported by gender minorities (32.43\%) and newcomers (20\%). Social challenges to participation arising from non-inclusive communication, toxic culture, and stereotyping \cite{trinkenreich2022women, balali2018newcomers} are often barriers to participation by those in underrepresented groups. Similarly, when it comes to \textit{process issues}, we found that they were more frequent for gender minority  (31.94\%) newcomers (28.26\%) respondents. Potential explanations may be related to a lack of project orientation, a lack of up-to-date documentation, and reception problems resulting from bias \cite{steinmacher2014hard, trinkenreich2022women}.

Our findings indicate that while fewer respondents reported facing challenges frequently, OSS projects still need to strive to remove gender biases and stereotypes to create an inclusive and welcoming environment.

\subsection{Perceptions about the state of D\&I}

Feeling represented and valued by the community is important to being productive and satisfied~\cite{lim2008job,trinkenreich2023sense}. Past work found that certain factors can impede the feeling of being represented, such as the lack of interpersonal relationships in the community~\cite{tinto1987leaving} and the perception that one's voice is lost in an environment where the loudest voice prevails~\cite{nafus2012patches}. Therefore, here we analyze contributors' perceptions of their ability to contribute to Apache projects. 

As before, we compare the survey results regarding the perception of D\&I between 2020 and 2022 surveys as shown in \Cref{tab:statement}. 

\begin{table*}[tbp]
\caption{Comparison of respondents' agreement on the perceptions of D\&I between the years 2020 and 2022.}
\centering
\resizebox{\textwidth}{!}{
\begin{tabular}{clll|llll|ll|ll|ll|ll|ll}
\textbf{Perception of} & \multicolumn{1}{c}{\textbf{Questions}} & \multicolumn{2}{c|}{\textbf{Man}} & \multicolumn{2}{c|}{\textbf{Minority gender}} & \multicolumn{2}{c|}{\textbf{Newcomer}} & \multicolumn{2}{c|}{\textbf{Not newcomer}} & \multicolumn{2}{c|}{\textbf{\begin{tabular}[c]{@{}c@{}}Comfortable using\\ English\end{tabular}}} & \multicolumn{2}{c|}{\textbf{\begin{tabular}[c]{@{}c@{}}Not comfortable using \\ English\end{tabular}}} & \multicolumn{2}{c|}{\textbf{\begin{tabular}[c]{@{}c@{}}From western\\ countries\end{tabular}}} & \multicolumn{2}{c}{\textbf{\begin{tabular}[c]{@{}c@{}}Not from western \\ countries\end{tabular}}} \\ \hline
 & \cellcolor[HTML]{EFEFEF}\begin{tabular}[c]{@{}l@{}}Other members of the project \\ see me as a parental figure\end{tabular} & \cellcolor[HTML]{EFEFEF}\greenup & \cellcolor[HTML]{EFEFEF}{\color[HTML]{228B22} 10.15\%} & \cellcolor[HTML]{EFEFEF}\greenup & \multicolumn{1}{l|}{\cellcolor[HTML]{EFEFEF}{\color[HTML]{228B22} 14.13\%}} & \cellcolor[HTML]{EFEFEF}\greenup & \cellcolor[HTML]{EFEFEF}{\color[HTML]{228B22} 10.92\%} & \cellcolor[HTML]{EFEFEF}\greenup & \cellcolor[HTML]{EFEFEF}{\color[HTML]{228B22} 9.08\%} & \cellcolor[HTML]{EFEFEF}\greenup & \cellcolor[HTML]{EFEFEF}{\color[HTML]{228B22} 10.44\%} & \cellcolor[HTML]{EFEFEF}\greenup & \cellcolor[HTML]{EFEFEF}{\color[HTML]{228B22} 8.70\%} & \cellcolor[HTML]{EFEFEF}\greenup & \cellcolor[HTML]{EFEFEF}{\color[HTML]{228B22} 10.57\%} & \cellcolor[HTML]{EFEFEF}\greenup & \cellcolor[HTML]{EFEFEF}{\color[HTML]{228B22} 10.91\%} \\
 & \begin{tabular}[c]{@{}l@{}}I am expected to take care \\ of other  members of the\\ project more so than is usual\end{tabular} & \greenup & {\color[HTML]{228B22} 4.27\%} & \reddown & \multicolumn{1}{l|}{{\color[HTML]{C70039} -15.63\%}} & \greenup & {\color[HTML]{228B22} 8.16\%} & \greenup & {\color[HTML]{228B22} 1.23\%} & \greenup & {\color[HTML]{228B22} 2.34\%} & \greenup & {\color[HTML]{228B22} 8.41\%} & \greenup & {\color[HTML]{228B22} 2.37\%} & \reddown & {\color[HTML]{C70039} -1.06\%} \\
\multirow{-3}{*}{\textbf{\begin{tabular}[c]{@{}c@{}}Role \\ stereotyping\end{tabular}}} & \cellcolor[HTML]{EFEFEF}\begin{tabular}[c]{@{}l@{}}I feel some members of the\\ community are  patronizing \\ to me\end{tabular} & \cellcolor[HTML]{EFEFEF}\reddown & \cellcolor[HTML]{EFEFEF}{\color[HTML]{C70039} -2.33\%} & \cellcolor[HTML]{EFEFEF}\reddown & \multicolumn{1}{l|}{\cellcolor[HTML]{EFEFEF}{\color[HTML]{C70039} -12.28\%}} & \cellcolor[HTML]{EFEFEF}\reddown & \cellcolor[HTML]{EFEFEF}{\color[HTML]{C70039} -4.42\%} & \cellcolor[HTML]{EFEFEF}\reddown & \cellcolor[HTML]{EFEFEF}{\color[HTML]{C70039} -2.82\%} & \cellcolor[HTML]{EFEFEF}\reddown & \cellcolor[HTML]{EFEFEF}{\color[HTML]{C70039} -1.85\%} & \cellcolor[HTML]{EFEFEF}\reddown & \cellcolor[HTML]{EFEFEF}{\color[HTML]{C70039} -13.55\%} & \cellcolor[HTML]{EFEFEF}\reddown & \cellcolor[HTML]{EFEFEF}{\color[HTML]{C70039} -2.58\%} & \cellcolor[HTML]{EFEFEF}\reddown & \cellcolor[HTML]{EFEFEF}{\color[HTML]{C70039} -5.15\%} \\ \hline
 & \begin{tabular}[c]{@{}l@{}}I have an equal chance to\\ get contributions accepted\end{tabular} & \greenup & {\color[HTML]{228B22} 3.76\%} & \greenup & \multicolumn{1}{l|}{{\color[HTML]{228B22} 8.97\%}} & \greenup & {\color[HTML]{228B22} 9.95\%} & \greenup & {\color[HTML]{228B22} 1.45\%} & \greenup & {\color[HTML]{228B22} 0.20\%} & \greenup & {\color[HTML]{228B22} 19.12\%} & \greenup & {\color[HTML]{228B22} 0.93\%} & \greenup & {\color[HTML]{228B22} 4.78\%} \\

 
 & \cellcolor[HTML]{EFEFEF}\begin{tabular}[c]{@{}l@{}}Nothing keeps me from\\ contributing to the project\end{tabular} & \cellcolor[HTML]{EFEFEF}\reddown & \cellcolor[HTML]{EFEFEF}{\color[HTML]{C70039} -0.06\%} & \cellcolor[HTML]{EFEFEF}\greenup & \multicolumn{1}{l|}{\cellcolor[HTML]{EFEFEF}{\color[HTML]{228B22} 39.43\%}} & \cellcolor[HTML]{EFEFEF}\greenup & \cellcolor[HTML]{EFEFEF}{\color[HTML]{228B22} 10.45\%} & \cellcolor[HTML]{EFEFEF}\greenup & \cellcolor[HTML]{EFEFEF}{\color[HTML]{228B22} 0.70\%} & \cellcolor[HTML]{EFEFEF}\greenup & \cellcolor[HTML]{EFEFEF}{\color[HTML]{228B22} 0.57\%} & \cellcolor[HTML]{EFEFEF}\greenup & \cellcolor[HTML]{EFEFEF}{\color[HTML]{228B22} 14.11\%} & \cellcolor[HTML]{EFEFEF}\greenup & \cellcolor[HTML]{EFEFEF}{\color[HTML]{228B22} 0.77\%} & \cellcolor[HTML]{EFEFEF}\greenup & \cellcolor[HTML]{EFEFEF}{\color[HTML]{228B22} 2.41\%} \\

 
 & \begin{tabular}[c]{@{}l@{}}I have a solid network of\\ open source peers\end{tabular} & \reddown & {\color[HTML]{C70039} -2.09\%} & \greenup & \multicolumn{1}{l|}{{\color[HTML]{228B22} 18.08\%}} & \greenup & {\color[HTML]{228B22} 1.97\%} & \reddown & {\color[HTML]{C70039} -2.78\%} & \reddown & {\color[HTML]{C70039} -2.50\%} & \greenup & {\color[HTML]{228B22} 11.90\%} & \reddown & {\color[HTML]{C70039} -2.83\%} & \greenup & {\color[HTML]{228B22} 13.27\%} \\

 
 & \cellcolor[HTML]{EFEFEF}\begin{tabular}[c]{@{}l@{}}It was easy to find a mentor\\ with whom I  felt comfortable\end{tabular} & \cellcolor[HTML]{EFEFEF}\reddown & \cellcolor[HTML]{EFEFEF}{\color[HTML]{C70039} -5.86\%} & \cellcolor[HTML]{EFEFEF}\greenup & \multicolumn{1}{l|}{\cellcolor[HTML]{EFEFEF}{\color[HTML]{228B22} 11.76\%}} & \cellcolor[HTML]{EFEFEF}\reddown & \cellcolor[HTML]{EFEFEF}{\color[HTML]{C70039} -7.18\%} & \cellcolor[HTML]{EFEFEF}\reddown & \cellcolor[HTML]{EFEFEF}{\color[HTML]{C70039} -4.29\%} & \cellcolor[HTML]{EFEFEF}\reddown & \cellcolor[HTML]{EFEFEF}{\color[HTML]{C70039} -4.38\%} & \cellcolor[HTML]{EFEFEF}\reddown & \cellcolor[HTML]{EFEFEF}{\color[HTML]{C70039} -7.83\%} & \cellcolor[HTML]{EFEFEF}\reddown & \cellcolor[HTML]{EFEFEF}{\color[HTML]{C70039} -5.63\%} & \cellcolor[HTML]{EFEFEF}\reddown & \cellcolor[HTML]{EFEFEF}{\color[HTML]{C70039} -1.88\%} \\
\multirow{-5}{*}{\textbf{\begin{tabular}[c]{@{}c@{}}Ability to \\ contribute\end{tabular}}} & \begin{tabular}[c]{@{}l@{}}I have a hard time following\\ discussions because of technical \\ jargon\end{tabular} &  & {*} &  & \multicolumn{1}{l|}{*} & \greenup & {\color[HTML]{228B22} 3.04\%} & \reddown & {\color[HTML]{C70039} -1.65\%} & \reddown & {\color[HTML]{C70039} -1.17\%} & \reddown & {\color[HTML]{C70039} -0.26\%} & \reddown & {\color[HTML]{C70039} -1.53\%} & \greenup & {\color[HTML]{228B22} 0.08\%} \\ \hline
 & \cellcolor[HTML]{EFEFEF}\begin{tabular}[c]{@{}l@{}}The PMC represents a diverse \\ set of people\end{tabular} & \cellcolor[HTML]{EFEFEF}\greenup & \cellcolor[HTML]{EFEFEF}{\color[HTML]{228B22} 1.56\%} & \cellcolor[HTML]{EFEFEF}\reddown & \multicolumn{1}{l|}{\cellcolor[HTML]{EFEFEF}{\color[HTML]{C70039} -22.22\%}} & \cellcolor[HTML]{EFEFEF}\reddown & \cellcolor[HTML]{EFEFEF}{\color[HTML]{C70039} -7.69\%} & \cellcolor[HTML]{EFEFEF}\greenup & \cellcolor[HTML]{EFEFEF}{\color[HTML]{228B22} 1.15\%} & \cellcolor[HTML]{EFEFEF}\greenup & \cellcolor[HTML]{EFEFEF}{\color[HTML]{228B22} 0.21\%} & \cellcolor[HTML]{EFEFEF}\greenup & \cellcolor[HTML]{EFEFEF}{\color[HTML]{228B22} 13.54\%} & \cellcolor[HTML]{EFEFEF}\reddown & \cellcolor[HTML]{EFEFEF}{\color[HTML]{C70039} -2.58\%} & \cellcolor[HTML]{EFEFEF}\greenup & \cellcolor[HTML]{EFEFEF}{\color[HTML]{228B22} 14.44\%} \\
\multirow{-2}{*}{\textbf{\begin{tabular}[c]{@{}c@{}}Being \\ represented\end{tabular}}} & \begin{tabular}[c]{@{}l@{}}I feel represented in the decision \\ making  group\end{tabular} & \greenup & {\color[HTML]{228B22} 4.83\%} & \reddown & \multicolumn{1}{l|}{{\color[HTML]{C70039} -16.40\%}} & \greenup & {\color[HTML]{228B22} 7.28\%} & \greenup & {\color[HTML]{228B22} 2.36\%} & \greenup & {\color[HTML]{228B22} 1.22\%} & \greenup & {\color[HTML]{228B22} 8.53\%} & \greenup & {\color[HTML]{228B22} 1.40\%} & \greenup & {\color[HTML]{228B22} 13.75\%} \\ \hline
 & \cellcolor[HTML]{EFEFEF}\begin{tabular}[c]{@{}l@{}}I was made aware of the code\\ of conduct and how to report \\ violations\end{tabular} & \cellcolor[HTML]{EFEFEF}\greenup & \cellcolor[HTML]{EFEFEF}{\color[HTML]{228B22} 7.83\%} & \cellcolor[HTML]{EFEFEF}\greenup & \multicolumn{1}{l|}{\cellcolor[HTML]{EFEFEF}{\color[HTML]{228B22} 7.89\%}} & \cellcolor[HTML]{EFEFEF}\reddown & \cellcolor[HTML]{EFEFEF}{\color[HTML]{C70039} -13.15\%} & \cellcolor[HTML]{EFEFEF}\greenup & \cellcolor[HTML]{EFEFEF}{\color[HTML]{228B22} 9.47\%} & \cellcolor[HTML]{EFEFEF}\greenup & \cellcolor[HTML]{EFEFEF}{\color[HTML]{228B22} 7.98\%} & \cellcolor[HTML]{EFEFEF}\greenup & \cellcolor[HTML]{EFEFEF}{\color[HTML]{228B22} 8.88\%} & \cellcolor[HTML]{EFEFEF}\greenup & \cellcolor[HTML]{EFEFEF}{\color[HTML]{228B22} 5.47\%} & \cellcolor[HTML]{EFEFEF}\greenup & \cellcolor[HTML]{EFEFEF}{\color[HTML]{228B22} 14.90\%} \\
\multirow{-2}{*}{\textbf{\begin{tabular}[c]{@{}c@{}}The code\\  of conduct\end{tabular}}} & \begin{tabular}[c]{@{}l@{}}I felt safer and more empowered \\ to fully  participate in this project \\ because it followed the code of \\ conduct\end{tabular} & \greenup & {\color[HTML]{228B22} 6.69\%} & \greenup & \multicolumn{1}{l|}{{\color[HTML]{228B22} 11.90\%}} & \greenup & {\color[HTML]{228B22} 3.40\%} & \greenup & {\color[HTML]{228B22} 8.14\%} & \greenup & {\color[HTML]{228B22} 8.83\%} & \reddown & {\color[HTML]{C70039} -1.01\%} & \greenup & {\color[HTML]{228B22} 5.44\%} & \greenup & {\color[HTML]{228B22} 6.11\%} \\ \hline
\multicolumn{1}{l}{} & \multicolumn{17}{l}{\cellcolor[HTML]{EFEFEF}* We posit gender should not impact following technical discussion, thus, we dont present it in table. For reference, the breakdown is: man: -1.54\%;  minority-gender: -2.19\%.}
\end{tabular}}
\label{tab:statement}
\end{table*}

\textbf{Role stereotyping:} The proportion of respondents who perceived that they were seen as ``parental figures" and were ``expected to take care of others \textit{more than usual}'' increased for all demographics, except for those from non-gender minority for the ``taking care" question. An explanation of this trend could be that with the reduction in community organization or events along with an increase in the proportion of respondents contributing to code and non-code artifacts, there is an increased need for mentoring. Every demographic (majority and minority) reported a higher proportion of mentoring others. This might result in respondents feeling that they are perceived as parental figures and have to take care of others. Interestingly, gender minority respondents had a reduction (15.63\%) in feeling that they were expected to take care of others more than usual. It might be that because other demographics have stepped up that non-minority contributors feel a reduction, or it might be that these contributors have a different expectation of what is considered ``usual". We hope it is the former.

There is a positive trend in the community being more welcoming. The proportion of respondents who felt members were ``patronizing" reduced across all demographics, for both majority and minority. The reduction was in double digits for those in minority-gender (12.28\%) and not English proficient (13.55\%) groups. We hope this trend reflects OSS is becoming more inclusive and respectful of its members, including those traditionally underrepresented.

\begin{mdframed}[roundcorner=10pt,nobreak=true]
\textbf{Observation 6}: Respondents across all demographics feel they are sought out as parental figures, while fewer proportions feel patronized.
\end{mdframed}

\textbf{Ability to contribute:} Biases against gender minorities, language proficiency, and cultures can be seen as hurdles to contributing to OSS. Results indicate that contribution barriers have been reduced for all the analyzed demographic factors, with a higher proportion of contributors feeling they have an ``equal chance" of getting contributions accepted, especially minority-gender (8.97\%) and not English-proficient (19.12\%) contributors. We also observed a higher proportion of newcomers who feel positive (9.95\%) about having an ``equal chance" of getting contributions accepted. Similar improvements in sentiments are seen for the ``nothing keeps me from contributing" question (39.43\% for gender minorities and 14.11\% for non-English proficients). This positive trend can help Apache projects attract and retain contributors.

When considering ``network of peers" and ``finding a mentor they are comfortable with", we see mixed results. A higher proportion of contributors from gender minorities feel they have a good peer network and mentors they are comfortable with. All other minority demographics reported an improvement in having a solid network of peers but a reduction in the ability to find mentors they are comfortable with.  

These results reflect a positive trend of contributors having a network of peers, on who they can depend. This could also be a reason for us seeing a higher proportion of contributors performing mentoring activities (\Cref{tab:engagement}) and those who feel they are seen as parental figures. However, it appears that finding a desired mentor remains challenging. Another reason for this dichotomy (more contributors are mentoring, but fewer respondents are finding mentors they feel comfortable with) could be a result of the mentorship programs. Successful mentor-mentee relationships often occur organically based on topics of interest or alignment of personality and career goals. Formal mentorship programs can often feel forced or misaligned based on differences between mentor-mentee working styles and goals~\cite{balali2018newcomers}. We recommend that OSS community keep developing mentorship programs and investigate strategies to facilitate and encourage informal mentoring.

When considering following technical discussions, there is some improvement for newcomers (3.04\%), which indicates either newcomers are getting more familiar with technical jargon or that the discussions are using less jargon and being more inclusive. The differences for the rest of the demographics are marginal. 

\begin{mdframed}[roundcorner=10pt,nobreak=true]
\textbf{Observation 7}: Respondents agreeing on the equal chance to contribute has increased across all demographic factors.  However, when it comes to finding an ideal mentor, there is less agreement among all groups except for those who identify as a minority gender. 
\end{mdframed}




\textbf{Perception of being represented:}
There has been a significant impetus in increasing awareness of gender diversity, both in research and in practice. In fact, 12 out of 355 OSS websites have a ``women-only'' section \cite{Lee.Carver:2019}. Our results show that there is still work to be done at the leadership level. A higher proportion of those in gender minorities (22.22\%) and newcomers (7.69\%) disagreed about PMC representing a diverse set of people. Fewer proportion of those in gender minorities felt that they were represented in the decision-making group. These results imply that OSS needs to actively mentor and promote those in gender minorities to leadership positions to improve their representation.

It is heartening to see that those not in western countries and not proficient in English feel more representation---both in the PMC makeup and in decision-making groups.

\begin{mdframed}[roundcorner=10pt,nobreak=true]
\textbf{Observation 8}: Gender minority and newcomer respondents were less likely to concur that they felt represented by PMC and decision-making.
\end{mdframed}

\textbf{Code of conduct:}
A code of conduct that outlines the expected behaviors of its members can assist in creating a more pleasant and supportive social environment \cite{CoC}. Creating a code of conduct is one of the most widely used D\&I efforts in OSS \cite{guizani2022perceptions,trinkenreich2022women}. Apache projects rely primarily on English-language content, and respondents who are not proficient in English and are not from western countries may encounter language barriers. Due to differences in culture and geography, rules and procedures such as the code of conduct may lose attention and create additional communication hurdles \cite{li2021code}.

Our results indicate that the awareness of the code of conduct has increased by 8.88\% for respondents who are not comfortable using English and 14.90\% for respondents who are not from western countries. Similarly, respondents from non-Western countries and non-native English speakers feel more empowered to participate because their projects follow the code of conduct. 

Nonetheless, we found that the proportion of newcomers and those in gender minorities who were aware of the code of conduct decreased (13.15\% for newcomers). These trends might mean that more contributors are unaware of how to react to unwelcoming interactions, and this might be why we saw an increase in perceptions of ``members are patronizing to me'' for these subgroups. This lack of awareness can be a result of the reduced number of events where respondents from these groups could interact informally about an unwelcoming environment and talk about action plans.

On the flip side, more respondents from these groups (arguably, when they were aware of the code of conduct) felt empowered to participate fully in the project when it followed its code of conduct. This could be because increasingly code-of-conduct documents are being written not simply as an ``aspirational" document, but with actionable steps that need to be taken when someone violates the established practices and rules.

Our recommendation is twofold. First, the code-of-conduct document should be prominently listed in projects, perhaps even as a separate tab. Additionally, onboarding documents, training, and mentorship programs should emphasize the code of conduct. Second, projects should follow their code of conduct, ensuring that those who do not follow it are appropriately penalized.

\begin{mdframed}[roundcorner=10pt,nobreak=true]
\textbf{Observation 9}: When projects follow their code-of-conduct respondents feel empowered to participate.
\end{mdframed}

