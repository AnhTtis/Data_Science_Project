\section{method}

Longitudinal studies focus on discovering trends or changes in the characteristics of the studied population at both the aggregate and individual levels \cite{menard2002longitudinal}. We are conducting a prospective longitudinal study to observe the evolution of D\&I in Apache projects. Thus far, we have conducted two surveys over the course of two years.

We used Apache as our case study as it is the world's largest OSS foundation with over 430k people and more than 350 projects and initiatives~\cite{ASFwebsite}. Apache projects are investing in improving the state of D\&I and collaborated with us in designing the instruments. The Apache D\&I committee helped validate and pilot our survey questions and gave our survey legitimacy for the community. The Apache D\&I committee comprises 18 experienced contributors, including committers, the Project Management Committee (PMC), and board members. 

\subsection{Survey design}

\begin{table*}[tbp]
\centering
\caption{Likert-Scale sub questions}
\resizebox{7in}{!}{
\begin{tabular}{lll}
\rowcolor[HTML]{EFEFEF} 
\textbf{\begin{tabular}[c]{@{}l@{}}Q5. How often do you engage in the \\ following activities in the ASF?\end{tabular}} & \textbf{\begin{tabular}[c]{@{}l@{}}Q6. Do you face challenges when participating  \\ in the ASF (e.g., language differences,technical \\ expertise, cultural differences, etc)?\end{tabular}} & \cellcolor[HTML]{EFEFEF}\textbf{\begin{tabular}[c]{@{}l@{}}Q7. Thinking about your current \\ project(s), please rate the following statements\end{tabular}} \\
Contributing/reviewing code & \textbf{\begin{tabular}[c]{@{}l@{}}*If not ``No challenges'', follow-up question \\ (only in 2022 survey ): How often do you face \\ the following challenges when participating \\ in the ASF?:\end{tabular}} & \begin{tabular}[c]{@{}l@{}}Other members of the project see me as a \\ parental figure\end{tabular} \\
\rowcolor[HTML]{EFEFEF} 
\begin{tabular}[c]{@{}l@{}}Creating or maintaining \\ documentation\end{tabular} & \begin{tabular}[c]{@{}l@{}}Process related   challenges with getting \\ started on the project\end{tabular} & \cellcolor[HTML]{EFEFEF}I am expected to take care of others than is usual \\
\begin{tabular}[c]{@{}l@{}}Participating in decision making \\ about the project and  development\end{tabular} & \begin{tabular}[c]{@{}l@{}}Process related challenges with navigating \\ the contribution process\end{tabular} & \begin{tabular}[c]{@{}l@{}}I have a hard time following discussions because of \\ technical jargon.\end{tabular} \\
\rowcolor[HTML]{EFEFEF} 
Serving as a community organizer & \begin{tabular}[c]{@{}l@{}}Process related challenges with reception\\ issues in the project\end{tabular} & \cellcolor[HTML]{EFEFEF}\begin{tabular}[c]{@{}l@{}}I feel some members of the community are patronizing \\ to me.\end{tabular} \\
Mentoring other contributors & Process related challenges with licenses & I have an equal chance to get contributions accepted \\
\rowcolor[HTML]{EFEFEF} 
 & \begin{tabular}[c]{@{}l@{}}Social Challenges with communication\\ styles\end{tabular} & \cellcolor[HTML]{EFEFEF}Nothing keeps me from contributing to the project \\
 & \begin{tabular}[c]{@{}l@{}}Social Challenges with feeling imposter \\ syndrome/ fear of making mistakes\end{tabular} & I have a solid network of open-source peers \\
\rowcolor[HTML]{EFEFEF} 
 & \begin{tabular}[c]{@{}l@{}}Social Challenges with facing a lack of \\ recognition\end{tabular} & \cellcolor[HTML]{EFEFEF}\begin{tabular}[c]{@{}l@{}}It was easy to find a mentor with whom I felt \\ comfortable\end{tabular} \\
 & \begin{tabular}[c]{@{}l@{}}Social Challenges with toxic/ unwelcoming\\ environment\end{tabular} & The PMC represents a diverse set of people \\
\rowcolor[HTML]{EFEFEF} 
 & \begin{tabular}[c]{@{}l@{}}Social Challenges with located in a different\\ country/from a different nationality\end{tabular} & \cellcolor[HTML]{EFEFEF}I feel represented in the decision making group \\
 & Social Challenges with cultural   differences & \begin{tabular}[c]{@{}l@{}}I felt safer and more empowered to fully participate \\ in this project because it followed the code of conduct\end{tabular} \\
\rowcolor[HTML]{EFEFEF} 
 & Technical related challenges with  documentation & \cellcolor[HTML]{EFEFEF} \\
 & Technical related challenges with technical Hurdles & 
\end{tabular}}
\label{tab:survey}
\end{table*}



% Q7, How often do you consult the 

% following ASF policies, processes, 

% or guidelines regarding



% Q8. When you need to locate 

% information about ASF processes, 

% policies, or guidelines, what best 

% describes your experience


% Q9. How often do you engage in the 

% following activities in the ASF?


%  Q10. Thinking about your current 

% project(s), please rate the 

% following statements

%We worked closely with ASF's D\&I committee to prioritize and improve the instrument. 
Our survey comprises seven questions (see supplementary for the survey questions~\cite{supply}), a mix of multiple-choice and Likert scale questions (Table \ref{tab:survey}). We reused existing questions when possible. 
The four demographic questions (gender identification, seniority in Apache projects, country of residence, and English confidence) were adapted from the 2016 ``Apache Committer Diversity Survey'' \cite{chaossmetrics} and the ``Open Demographics Survey'' \cite{opendemographicsdoc}. The remaining questions are as follows: Q5 aims to understand the activities that contributors frequently engage in to understand the diversity of contributions, Q6 aims to understand the challenges that contributors face when contributing to Apache projects, and Q7 aims to understand contributors' experiences about aspects of D\&I.
In 2022, if respondents did not select ``not a challenge'' for Q6, we added a follow-up question to understand the frequency with which contributors face process, technical, and social challenges when contributing to Apache projects. The follow-up question was adapted from the high-level categories of the conceptual model of challenges that contributors face in a large OSS organization found in prior literature \cite{guizani2021long}. Our goal was to conduct a more in-depth analysis of the difficulties that arise when individuals contribute to OSS projects.
A cross-sectional analysis of the 2020 survey results are reported in \cite{guizani2022perceptions}.


\subsection{Data collection}

%survey deployment and advertisig D2
We used GPLv2-licensed Lime Survey \cite{Lime} to deploy our survey. With the help of Apache project managers, we sent out invitations to all ``apache.org'' email addresses and shared the survey on Apache developer mailing lists. Respondents were provided with a consent page that explained the purpose of the survey, the data collection and usage process, and provided a point of contact. Each of the two surveys was open for 45 days. We followed Apache's privacy policies and did not collect identifying information or IP addresses.
%respondents' email addresses were retained separately from the responses. 

%Respondant rate
Based on an estimated total community size of 7500 contributors, we obtained 624 responses in the 2020 survey, corresponding to an 8.32\% response rate, and 432 responses in 2022, corresponding to a 5.76\% response rate (see Table \ref{tab:demo}). These response rates are consistent with other studies in software engineering and OSS \cite{feng2022case, steinmacher2021being}. The reported demographic or survey respondents are consistent across the two surveys ($<5$\% differences for all demographic attributes). 


\begin{table}[tbp]
\caption{Demographics of the survey respondents (2020 \& 2022)}
\centering
\resizebox{\columnwidth}{!}{
\begin{tabular}{lllll}
\rowcolor[HTML]{EFEFEF} 
\multicolumn{1}{l|}{\cellcolor[HTML]{EFEFEF}\textbf{Demographics}} & \multicolumn{2}{c|}{\cellcolor[HTML]{EFEFEF}\textbf{2020}} & \multicolumn{2}{c}{\cellcolor[HTML]{EFEFEF}\textbf{2022}} \\ \hline
\multicolumn{5}{c}{\textbf{Gender}} \\ \hline
\rowcolor[HTML]{EFEFEF} 
\multicolumn{1}{l|}{\cellcolor[HTML]{EFEFEF}Men} & 545 & \multicolumn{1}{l|}{\cellcolor[HTML]{EFEFEF}87.34\%} & 374 & 86.57\% \\
\multicolumn{1}{l|}{Women+non-binary+self-describe} & 45 & \multicolumn{1}{l|}{7.21\%} & 29 & 6.71\% \\
\rowcolor[HTML]{EFEFEF} 
\multicolumn{5}{c}{\cellcolor[HTML]{EFEFEF}\textbf{Senority}} \\ \hline
\multicolumn{1}{l|}{less than 1 year (Newcomer)} & 66 & \multicolumn{1}{l|}{10.58\%} & 28 & 6.48\% \\
\rowcolor[HTML]{EFEFEF} 
\multicolumn{1}{l|}{\cellcolor[HTML]{EFEFEF}greater than 1 year} & 553 & \multicolumn{1}{l|}{\cellcolor[HTML]{EFEFEF}88.62\%} & 404 & 93.52\% \\
\multicolumn{5}{c}{\textbf{English Proficiency}} \\ \hline
\multicolumn{1}{l|}{Comfortable using English} & 551 & \multicolumn{1}{l|}{88.30\%} & 364 & 84.26\% \\
\rowcolor[HTML]{EFEFEF} 
\multicolumn{1}{l|}{\cellcolor[HTML]{EFEFEF}Not comfortable using English} & 56 & \multicolumn{1}{l|}{\cellcolor[HTML]{EFEFEF}8.97\%} & 31 & 7.18\% \\
\multicolumn{5}{c}{\textbf{Region}} \\ \hline
\multicolumn{1}{l|}{\begin{tabular}[c]{@{}l@{}}Western countries \\ (N. America + Europe)\end{tabular}} & 508 & \multicolumn{1}{l|}{81.41\%} & 340 & 78.70\% \\
\rowcolor[HTML]{EFEFEF} 
\multicolumn{1}{l|}{\cellcolor[HTML]{EFEFEF}Not western countries} & 93 & \multicolumn{1}{l|}{\cellcolor[HTML]{EFEFEF}14.90\%} & 68 & 15.74\%
\end{tabular}}
\label{tab:demo}
\end{table}

\subsection{Data analysis}
\textit{Demographics grouping:}
The first step in our analysis was to group respondents into the majority and minority groups for each demographic attribute. \Cref{tab:demo} presents the demographic distributions per attribute.

When considering \textit{gender}, the survey asked respondents to select their gender identity (options: man, woman, gender variant/non-conforming/non-binary, prefer to self-describe, and prefer not to say). Respondents who selected ``man" were in the majority; we grouped as ``gender-minority'' all other respondents except those who selected ``prefer not to say". We decided not to include the latter in our grouping, as we are interested in understanding the perspective of underrepresented genders and ``prefer not to say" does not give us the gender. Table \ref{tab:demo} presents the gender distribution of survey respondents in 2020 and 2022. 

The next attribute of interest is \textit{newcomers} to understand their perspectives. To remain innovative it is important for projects to attract new contributors who bring fresh ideas and bring a diversity of thought \cite{steinmacher2013newcomers}. We classified newcomers as those who have ``less than 1-year" experience in Apache projects (see \Cref{tab:demo}).

The default communication language among Apache projects is English, so we analyze the perspective of respondents from \textit{English proficiency} attributes. Non-native English speakers may have difficulty following discussions, especially those involving  idioms and technical jargon \cite{prana2020including}. In the survey, we asked respondents about their English skills from four perspectives, including questions regarding confidence in (1) speaking (face-to-face), participating in (2) technical or (3) non-technical discussion, and (4) conducting code reviews. Each question included Likert-scale options of very confident, confident, average, uncomfortable, and not confident. We averaged the responses across the four sub-questions. Respondents who scored ``average'' or above for all four questions were classified as ``Comfortable using English''; otherwise, they were classified as ``Not comfortable using English'' (see \Cref{tab:demo}).

Another aspect of diversity is regional diversity, which can serve as a proxy for different cultures and communication styles. Respondents were located across 65 countries (2020: 53 countries; 2022: 46 countries), with the majority coming from North America and Europe; the rest were mainly distributed across Asian countries, including India, China, and Japan. Given this distribution, we categorized regions into ``western" and ``not-western" countries. Western country culture is typically more individualistic, whereas Eastern culture is hierarchical and depends on consensus building when making decisions \cite{schwartz2006theory, bae2000organizational}. 


\textit{Question response grouping}:
We analyzed the proportion of responses for each of the listed questions (see \Cref{tab:survey}) per demographic attributes. 

To analyze the Likert-scale question Q5, we grouped the responses ``more than once a week'' and ``more than once a month" into \textit{often} and ``less than once a month" and ``never" into \textit{rarely to never}. We compare, between 2020 and 2022, the proportion of each demographic that \textit{often} engages in each of the listed activities (e.g., the proportion of gender minorities answering \textit{often} in the 2020 survey vs. the proportion of gender minorities answering \textit{often} in 2022). 

For Q6 (challenges), we grouped the responses ``many challenges" and ``several challenges'' into \textit{numerous challenges} and ``a few challenges" and ``no challenges'' into \textit{nearly no challenges}. Similarly, for Q7, we grouped the responses ``completely agree" and ``agree" into \textit{agree} and compared the proportion of each demographic that ``agreed" with each of the listed statements. 
