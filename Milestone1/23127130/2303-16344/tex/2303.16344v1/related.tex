\section{Related Work}
\label{sec:relatedwork}

%gender
In recent years, diversity in OSS projects has earned widespread attention, mirroring the growing awareness of the importance of diversity in broader workplace settings. Nevertheless, most software engineering research on diversity focuses on gender diversity, mostly on women contributors \cite{bosu2019diversity, ortu2017diverse, trinkenreich2022women, trinkenreich2021please}. By conducting a systematic literature survey, Trinkenreich et al. \cite{trinkenreich2022women} summarized ten strategies to mitigate challenges women contributors  encounter in OSS, such as ``Promote women-specific groups and events'' and ''Encourage and be welcoming to women''. A recent study has created and investigated a systematic debugging process to empower project leaders to reduce gender-biased contribution barriers within their project's workflow \cite{guizani2022debug}. Research has shown that teams with a more diverse gender composition perform better \cite{ortu2017diverse}. Perez \cite{perez2019invisible} discusses data biases that affect women.

%senority
Seniority has also been investigated in a few other studies. Senior contributors often lack the social motivation of their younger contributors \cite{morrison2016veteran}, and younger contributors are confronted with a wide range of technical and social issues \cite{davidson2014older}. When it comes to mitigating community smells, many contributors believe that experience is more significant than gender diversity \cite{catolino2019gender}. Steinmacher et al.~\cite{steinmacher2018let} present guidelines for OSS communities and newcomers to help mitigate barriers, and Guizani et al. \cite{guizani2022Dashboard} designed a maintainer dashboard to help attract and retain newcomers. Mentoring is also a strategy often used to support newcomer onboarding \cite{balali2018newcomers}.

%geo-location
Researchers have found that geo-location is an additional factor that impacts the contributors' activities; having a pull request submitted by someone in the same country as the person doing the integrating results in a higher acceptance rate \cite{rastogi2016geographical}. By building a conceptual model of the challenges faced by contributors at large OSS organizations, researchers have reported that some of the social challenges that contributors face are related to geographical isolation and the lack of representation of non-western countries and suggested diversifying event locations \cite{guizani2021long}. Recent research highlighted that gender diversity is low across all parts of the world, and while there is some variation across regions, the difference is insignificant \cite{prana2021including}. A study suggested using several communication methods to address geographic challenges such as cultural differences, time zone issues, and English language proficiency. Encouraging local language and culture groups can help contributors feel at home, especially newcomers \cite{balali2018newcomers}. 

Researchers in OSS communities have put in a lot of effort to make communities more welcoming and accepting of contributors of diverse backgrounds. This paper aims to provide a pulse check of the evolution of contributors' perspectives and engagement over time. We hope this study will help to guide new and existing D\&I initiatives. 
