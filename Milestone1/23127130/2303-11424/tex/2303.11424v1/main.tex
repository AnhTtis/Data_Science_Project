% CVPR 2023 Paper Template
% based on the CVPR template provided by Ming-Ming Cheng (https://github.com/MCG-NKU/CVPR_Template)
% modified and extended by Stefan Roth (stefan.roth@NOSPAMtu-darmstadt.de)

\documentclass[10pt,twocolumn,letterpaper]{article}

%%%%%%%%% PAPER TYPE  - PLEASE UPDATE FOR FINAL VERSION
%\usepackage[review]{cvpr}      % To produce the REVIEW version
%\usepackage{cvpr}              % To produce the CAMERA-READY version
\usepackage[pagenumbers]{cvpr} % To force page numbers, e.g. for an arXiv version

% Include other packages here, before hyperref.
\usepackage{graphicx}
\usepackage{amsmath}
\usepackage{amssymb}
\usepackage{booktabs}
\usepackage{url}            % simple URL typesetting
\usepackage{amsfonts}       % blackboard math symbols
\usepackage{nicefrac}       % compact symbols for 1/2, etc.
\usepackage{microtype}      % microtypography
\usepackage{xcolor}         % colors
\usepackage{paralist}
\usepackage{caption}
\usepackage{subcaption}
%\usepackage{subfigure}
\usepackage{etoolbox}
\usepackage{float}
\usepackage{tabularx}
\usepackage[accsupp]{axessibility}  % Improves PDF readability for those with disabilities.
% It is strongly recommended to use hyperref, especially for the review version.
% hyperref with option pagebackref eases the reviewers' job.
% Please disable hyperref *only* if you encounter grave issues, e.g. with the
% file validation for the camera-ready version.
%
% If you comment hyperref and then uncomment it, you should delete
% ReviewTempalte.aux before re-running LaTeX.
% (Or just hit 'q' on the first LaTeX run, let it finish, and you
%  should be clear).
\usepackage[pagebackref,breaklinks,colorlinks]{hyperref}


% Support for easy cross-referencing
\usepackage[capitalize]{cleveref}
\crefname{section}{Sec.}{Secs.}
\Crefname{section}{Section}{Sections}
\Crefname{table}{Table}{Tables}
\crefname{table}{Tab.}{Tabs.}


%%%%%%%%% PAPER ID  - PLEASE UPDATE
\def\cvprPaperID{5938} % *** Enter the CVPR Paper ID here
\def\confName{CVPR}
\def\confYear{2023}


\begin{document}

%%%%%%%%% TITLE - PLEASE UPDATE
\title{Polynomial Implicit Neural Representations For Large Diverse Datasets}

\author{Rajhans Singh  \qquad Ankita Shukla \qquad Pavan Turaga\\
 \qquad Geometric Media Lab, Arizona State University\\
{\tt\small \{rsingh70, ashukl20, pavan.turaga\}@asu.edu}
}
% For a paper whose authors are all at the same institution,
% omit the following lines up until the closing ``}''.
% Additional authors and addresses can be added with ``\and'',
% just like the second author.
% To save space, use either the email address or home page, not both
%\and
% Second Author\\
% Institution2\\
% First line of institution2 address\\
% {\tt\small secondauthor@i2.org}

% \and Geometric Media Lab\\
% Arizona State University\\
% {\tt\small rsingh70@asu.edu}
% }

% \twocolumn[{%
% \renewcommand\twocolumn[1][]{#1}%
% \maketitle
%   \centering
%   %\vspace{-0.3in}

%   \begin{subfigure}{0.99\textwidth}\centering
%      \centering
%         % \vspace{-0.15in}
%          \includegraphics[ width=0.99\textwidth]{cvpr2023-author_kit-v1_1-1/latex/figures/poly-teaser.pdf}
%         %\vspace{-0.1in}
% \end{subfigure}

% \captionof{figure}{Samples generated by our Poly-INR model on the ImageNet dataset at various resolutions. Our model generates images with high fidelity without using convolution, upsample, or self-attention layers, i.e., no interaction between the pixels.}
% \label{fig:teaser}

%   \vspace{3mm}
% }]
\twocolumn[{%
\renewcommand\twocolumn[1][]{#1}%
\maketitle
\begin{center}
    \centering
    \captionsetup{type=figure}
    \includegraphics[width=.99\textwidth]{cvpr2023-author_kit-v1_1-1/latex/figures/poly-teaser.pdf}
    \captionof{figure}{Samples generated by our Poly-INR model on the ImageNet dataset at various resolutions. Our model generates images with high fidelity without using convolution, upsample, or self-attention layers, i.e., no interaction between the pixels.}
    \label{fig:teaser}
\end{center}%
}]

\maketitle

%%%%%%%%% ABSTRACT
% Recently, there has been a significant interest in representing an image as a continuous function of its coordinate location and approximating that function with a neural network. Such representation of an image is called
% Implicit Neural Representations (INR) have gained significant popularity for image representation as a continuous function of the pixel location. 
\begin{abstract}
\vspace{-3mm}
Implicit neural representations (INR) have gained significant popularity for signal and image representation for many end-tasks, such as superresolution, 3D modeling, and more. Most INR architectures rely on sinusoidal positional encoding, which accounts for high-frequency information in data. However, the finite encoding size restricts the model's representational power. Higher representational power is needed to go from representing a single given image to representing large and diverse datasets. Our approach addresses this gap by representing an image with a polynomial function and eliminates the need for positional encodings. Therefore, to achieve a progressively higher degree of polynomial representation, we use element-wise multiplications between features and affine-transformed coordinate locations after every ReLU layer. The proposed method is evaluated qualitatively and quantitatively on large datasets like ImageNet. The proposed Poly-INR model performs comparably to state-of-the-art generative models without any convolution, normalization, or self-attention layers, and with far fewer trainable parameters. With much fewer training parameters and higher representative power, our approach paves the way for broader adoption of INR models for generative modeling tasks in complex domains. The code is available at \url{https://github.com/Rajhans0/Poly_INR}
\end{abstract}
\vspace{-4mm}
%%%%%%%%% BODY TEXT
%!TEX root = main.tex
\section{Introduction}

% GAN is awesome and sota, diffusion time taking
% conv based GAN and problem
% recent boom in coordinate based GAN, good things for inr and not been scaled to large diverse data
%Our method and inspiration
%contributions 1) INR in forms of polynomial 2) architecture only made up of linear and relu, 3) scale it to imagenet datasets 4) best of style gan and inr

Deep learning-based generative models are a very active area of research with numerous advancements in recent years \cite{kingma2013auto, goodfellow2020generative, dhariwal2021diffusion}. 
% In particular, generative adversarial networks (GANs) \cite{goodfellow2020generative} have shown remarkable performance for various generative tasks such as image-to-image translation \cite{liu2017unsupervised, isola2017image, zhu2017toward, park2019semantic}, image manipulation \cite{zhu2020domain, wang2022high, harkonen2020ganspace}, disentanglement \cite{karras2019style, tran2017disentangled, kazemi2019style}, inverse problems \cite{shah2018solving, yang2021gan}, domain adaptation \cite{huang2018auggan, tzeng2017adversarial}, and video generation \cite{skorokhodov2022stylegan, li2018video}. 
Most widely, generative models are based on convolutional architectures. However, recent developments such as implicit neural representations (INR)  \cite{mildenhall2021nerf, sitzmann2020implicit} represent an image as a continuous function of its coordinate locations, where each pixel is synthesized independently. Such a function is approximated by using a deep neural network. INR provides flexibility for easy image transformations and high-resolution up-sampling through the use of a coordinate grid. Thus, INRs have become very effective for $3$D scene reconstruction and rendering from very few training images \cite{mildenhall2021nerf, mescheder2019occupancy, barron2022mip, martin2021nerf,yu2021pixelnerf}. However, they are usually trained to represent a single given scene, signal, or image. Recently, INRs have been implemented as a generative model to generate entire image datasets \cite{anokhin2021image, skorokhodov2021adversarial}. They perform comparably to CNN-based generative models on perfectly curated datasets like human faces \cite{karras2019style}; however, they have yet to be scaled to large, diverse datasets like ImageNet \cite{deng2009imagenet}. 

% The weights in this CNN-based model are modulated by some low-dimension latent code sampled from a known distribution \cite{karras2019style}. The model can generate plausible images over the distribution of real datasets by modulating the CNN weights. However, CNN kernels are not inherently equivariant for transformation like rotation; hence, even simple image transformations in GAN space are not trivial. 
% Convolution neural network (CNN) is generally used in the generative model.


%inr, inrgan , blur mlp, siren
INR generally consists of a positional encoding module and a multi-layer perceptron model (MLP). The positional encoding in INR is based on sinusoidal functions, often referred to as Fourier features. Several methods \cite{mildenhall2021nerf, sitzmann2020implicit, tancik2020fourier} have shown that using MLP without sinusoidal positional encoding generates blurry outputs, i.e., only preserves low-frequency information. Although, one can remove the positional encoding by replacing the ReLU activation with a periodic or non-periodic activation function in the MLP \cite{sitzmann2020implicit, ramasinghe2022beyond, chng2022gaussian}. However, in INR-based GAN \cite{anokhin2021image}, using a periodic activation function in MLP leads to subpar performance compared to positional encoding with ReLU-based MLP.  

Sitzmann et al. \cite{sitzmann2020implicit} demonstrate that ReLU-based MLP fails to capture the information contained in higher derivatives. This failure to incorporate higher derivative information is due to ReLU's piece-wise linear nature, and second or higher derivatives of ReLU are typically zero. This can be further interpreted in terms of the Taylor series expansion of a given function. The higher derivative information of a function is included in the coefficients of a higher-order polynomial derived from the Taylor series. Hence, the inability to generate high-frequency information is due to the ineffectiveness of the ReLU-based MLP model in approximating higher-order polynomials. 

%non linear combination of periodic, not very natural
Sinusoidal positional encoding with MLP has been widely used, but the capacity of such INR can be limiting for two reasons. First, the size of the embedding space is limited; hence only a finite and fixed combination of periodic functions can be used, limiting its application to smaller datasets. Second, such an INR design needs to be mathematically coherent. These INR models can be interpreted as a non-linear combination of periodic functions where periodic functions define the initial part of the network, and the later part is often a ReLU-based non-linear function. Contrary to this, classical transforms (Fourier, sine, or cosine) represent an image by a linear summation of periodic functions. However, using just a linear combination of the positional embedding in a neural network is also limiting, making it difficult to represent large and diverse datasets. Therefore, instead of using periodic functions, this work models an image as a polynomial function of its coordinate location. 
% Therefore,  we design our MLP in a way that can represents higher-order polynomials.

The main advantage of polynomial representation is the easy parameterization of polynomial coefficients with MLP to represent large datasets like ImageNet. However, conventionally MLP can only approximate lower-order polynomials. One can use a polynomial positional embedding of the form $x^py^q$ in the first layer to enable the MLP to approximate higher order. However, such a design is limiting, as a fixed embedding size incorporates only fixed polynomial degrees. In addition, we do not know the importance of each polynomial degree beforehand for a given image. 

Hence, we do not use any positional encoding, but we progressively increase the degree of the polynomial with the depth of MLP. We achieve this by element-wise multiplication between the feature and affine transformed coordinate location, obtained after every ReLU layer. The affine parameters are parameterized by the latent code sampled from a known distribution. This way, our network learns the required polynomial order and represents complex datasets with considerably fewer trainable parameters. In particular,  the key highlights are summarized as follows:
\begin{compactitem}
\item We propose a Poly-INR model based on polynomial functions and design a MLP model to approximate higher-order polynomials.
\item Poly-INR as a generative model performs comparably to the state-of-the-art CNN-based GAN model (StyleGAN-XL \cite{sauer2022stylegan}) on the ImageNet dataset with $~3-4\times$ fewer trainable parameters (depending on output resolution).
\item Poly-INR outperforms the previously proposed INR models on the FFHQ dataset \cite{karras2019style}, using a significantly smaller model.
\item We present various qualitative results demonstrating the benefit of our model for interpolation, inversion, style-mixing, high-resolution sampling, and extrapolation.
\end{compactitem}








\section{Related work}
% There is extensive recent work on speeding up analytical queries due to the need for consistent execution times in the face of the explosive growth in the volume of available data.
% In this section, we divide existing work into two categories: maintaining data freshness in MVs (\Cref{sec:server_side}) and optimizations for minimizing ad-hoc query latency (\Cref{sec:client_side}).

% \subsection{Maintaining Data Freshness in MVs}
% \label{sec:server_side}
% There exists a variety of data warehousing applications aimed at supporting low-latency analytical queries on fresh data.
% In particular, these applications require efficiency in the propagation of newly ingested data into downstream MVs.
 
\mypara{Efficient MV Refresh}
Incremental view maintenance (IVM) aims to update MVs to reflect newly ingested data, taking advantage of already computed results to perform the update in a manner more efficient than computing from scratch (full refresh)
~\cite{ahmad2012dbtoaster,mcsherry2013differential,armbrust2013generalized,zeng2016iolap, palpanas2002incremental, griffin1995incremental, agiwal2021napa, braun2015analytics}. 
There is an abundance of work in IVM, including incremental updates on duplicate values~\cite{griffin1995incremental}, non-distributive aggregate functions~\cite{palpanas2002incremental}, higher-order views~\cite{ahmad2012dbtoaster}, and sliding windows~\cite{braun2015analytics}. 
More recent works also investigate the scalability aspect of IVM, proposing scale-independent updates~\cite{armbrust2013generalized} and sampled views~\cite{zeng2016iolap}. Since \system is applicable to arbitrary SQL statements, \system is orthogonal to and is fully compatible with existing IVM techniques.

\mypara{MV Refresh Scheduling}
There exist works on scheduling the refresh of a MV set focusing on resolving cyclic dependencies~\cite{folkert2005optimizing}, minimizing weighted average staleness~\cite{golab2009scheduling}, and prioritizing MVs with the highest speedups on predicted future queries~\cite{ahmed2020automated}.
\system's scheduling to speed up the end-to-end refresh of the MV set is not addressed in existing works.

\mypara{DAG Workflow Scheduling}
The execution of workloads consisting of individual jobs with acyclic dependencies is a well-studied topic~\cite{apacheoozie,sparkdag,marchal2018parallel,bathie2020revisiting,baruah2022ilp}; many of these techniques can be applied to MV refresh runs studied in this paper.
Existing workflow scheduling systems such as Apache Oozie~\cite{apacheoozie}, Apache Airflow~\cite{airflow}, and Spark DAG scheduler~\cite{sparkdag} automate the execution of user-defined workflows following a topological order.
There are recent works aimed at finding more optimal execution orders in terms of peak memory usage~\cite{marchal2018parallel, bathie2020revisiting} and execution time on parallel platforms~\cite{baruah2022ilp}.
While \system is designed for use with MV refresh runs/workloads, our technique on joint scheduling and optimization can be reasonably applied to general workloads as a possible future direction.

% \paragraph{Incremental MV indexing}
% Update-optimized indices such as the log-structured merge-trees (LSM)~\cite{o1996log} are used for indexing MVs due to frequent updates induced by data ingestion~\cite{gupta2016mesa,agiwal2021napa}.
% \system is orthogonal to indexing: \system is capable of efficiently performing MV refresh runs regardless of whether the individual MVs are indexed or not.

% \subsection{Ad-hoc Query Latency Reduction}
% \label{sec:client_side}

% The minimization of ad-hoc analytical query response times is a well-studied topic due to latency being negatively correlated with the productivity of a data analyst during a data analysis session~\cite{liu2014effects}.
% Sessions are commonly conducted within visualization systems that contain a variety of optimization techniques to ensure that query response times fall within a certain latency tolerance.

% \mypara{Data prefetching}
% Data is often loaded into memory on a by-need basis in visualization systems to minimize interference with user-issued query computations~\cite{mani2017effective,xin2021enhancing,galakatos2017revisiting, yan2020auto, battle2016dynamic, crotty2016case, jalaparti2018netco}. 
% Query-time data retrieval can be significantly expedited by anticipating the data usage of the user in future queries and pre-loading the data into memory during the downtime between user queries (`think time'). SMART~\cite{mani2017effective} prefetches data for modified versions of current user-issued queries with different filters and dimensions. A-WARE~\cite{crotty2016case} maintains a sample store constantly refined through ingesting data based on speculations of future plots.
% ForeCache~\cite{battle2016dynamic} uses an SVM to predict the user's current analysis phase and accordingly prefetches data tiles partitioned based on different numerical values. NetCo predicts future queries via log analysis, and solves an ILP formulation to prefetch data to maximize the number of SLO-meeting queries~\cite{jalaparti2018netco}.
% In the case of MV refresh workloads, `think time' is nonexistent as individual MVs are refreshed back-to-back, rendering data prefetching techniques non-applicable.

\mypara{Intermediate Data Caching}
Some existing data visualization systems cache user-defined variables to support the typical incremental construction of data visualizations~\cite{zgraggen2016progressive, eichmann2020idebench} during data analysis sessions~\cite{jupyter, rstudio, colab}. 
Recent work proposes a management scheme for these cached variables under a memory constraint that greedily keeps variables with the highest estimated time savings based on predicted future user behavior via neural networks~\cite{xin2021enhancing}.
While useful for data visualization, a greedy approach to memory management fails to achieve satisfactory results compared to \system.

\mypara{Intermediate Result Reuse}

There exist works on storing intermediate results from computations to speedup future computations~\cite{yang2018intermediate, dursun2017revisiting, nagel2013recycling, michiardi2019memory, galakatos2017revisiting}.
Studied topics include the identification of reuse opportunities by finding overlaps in computation graphs of successive jobs~\cite{yang2018intermediate, michiardi2019memory},
selective storage under a space constraint with heuristics such as reuse probability~\cite{dursun2017revisiting}, expected savings~\cite{yang2018intermediate}, and recompute-storage cost difference~\cite{nagel2013recycling},
and rewriting incoming jobs to take advantage of stored intermediates~\cite{galakatos2017revisiting}.
These works share similarity with \system in their selection of items to store under a memory constraint, however, \system's problem setting requires it to uniquely consider the joint (re)ordering of job executions along with the selection of items.

% work that considers both job execution (re)order as well as intermediate result caching with a bounded amount of memory. but notably lack the joint aspect of \system and cannot be used to achieve immediate speedup on an incoming MV refresh run if no intermediates are stored beforehand. 

\mypara{Incremental Query Processing} Incremental processing (IQP) is useful for cases where not all data required for a query is immediately available. Similar to online aggregation~\cite{hellerstein1997online}, initial results of a query are computed on a subset of required data and progressively refined as the rest of the required data arrives in a predictable pattern~\cite{tang2019intermittent,wangtempura}. Tang et al. propose a dynamic programming formulation to pick intermediate states to store in memory given a limited memory budget~\cite{tang2019intermittent}. Tempura rewrites the query plan for more efficient execution based on predicted data arrival patterns~\cite{wangtempura}. While similarities exist between the problem setting of IQP and \system, such as management of bounded memory, \system notably includes additional joint optimization for the order of MV updates.

% \paragraph{Sampling}
% Sampling has seen wide use in visualization systems for reducing the computation time of ad-hoc queries by computing an approximate result over a subset of data as exact results are not always required by the user~\cite{crotty2016case, mani2017effective, zgraggen2014panoramicdata, kraska2021northstar, galakatos2017revisiting, kandula2016quickr}. 
% Commonly studied topics in sampling for ad-hoc queries include complex query sampling~\cite{kandula2016quickr}, rare event aggregation~\cite{kraska2021northstar, galakatos2017revisiting}, and maintaining consistency between related sampled visualizations~\cite{zgraggen2014panoramicdata}.
% Sampling server-side at the MV level compromises the assumptions of downstream applications and is thus not considered in \system.

% \paragraph{Progressive visualization}
% The latency tolerance for time-consuming queries can be circumvented by presenting a partially-computed visualization to the user within the tolerance, which is then incrementally refined until it is fully accurate~\cite{rahman2017ve, zgraggen2016progressive, crotty2015vizdom, kraska2021northstar, kamat2017infiniviz}.
% Example plots which benefit from progressive visualization include bar charts~\cite{kamat2017infiniviz} and heatmaps~\cite{rahman2017ve}.
% Similar to sampling, study on this topic is orthogonal to \system as pushing out partially-updated MVs compromises downstream assumptions.
\section{Method}
\label{sec:method}

% \ml{``Inconsistent'' to ``large variation''}

% In this section, we propose our methods based on the observations in Section \ref{sec:motivation}.
In this section, we propose two techniques to further enhance the strong baseline to capture the variation of activation distributions better.
We first introduce spatial re-scaling to adapt the network to pixel-to-pixel variation.
We then propose channel-wise shifting and re-scaling to better capture the channel-to-channel variation.
Meanwhile, as both of the two methods are image-dependent, the image-to-image variation can be captured naturally.
By combining the two methods with our strong baseline, we build our enhanced BNN for SR, named EBSR.

% Because the activation distributions among pixels, channels and images have large variations \red{**are highly inconsistent} in SR networks, we introduce spatial re-scaling to adapt to pixel-wise variations and channel shift and re-scaling to adapt to channel-wise variations. And both of them are image-dependent to adapt to image-wise variations, which means during inference our network re-scales and shifts the distributions of activations flexibly for different input images. Based on these methods, we build an enhanced binary neural network for image super-resolution (EBSR).

% According to [3], the difference of activation magnitudes indicates different scaling factors are needed for each pixel.

\subsection{Spatial Re-scaling}
% It is better to use different scaling factors for different pixels to reduce the quantization error and retain more detailed information for image super-resolution. 

% \ml{In the main method, we do not need to introduce the previous works but can focus on introducing our own method. Channel rescaling in Real-to-binary Net is not relevant in this context.}

% Re-scaling the output of binary convolutions was proposed at the birth of BNN in XNOR-Net \cite{rastegari2016xnor} to reduce quantization error and improve accuracy for image classification tasks.
% It is computed as below:
% \begin{equation}
% \mathcal{A} * \mathcal{W} \approx(\operatorname{sign}(\mathcal{A}) \circledast \operatorname{sign}(\mathcal{W})) \odot \mathcal{K} \alpha
% \label{eq:xnor-net rescale}
% \end{equation}
% where $\circledast$ denotes the binary convolution and $\odot$ denotes the element-wise multiplication.
% $\mathcal{A}$, $\mathcal{W}$, $\alpha$, and $\mathcal{K}$ denote the activation, weight, weight scaling factor, and activation scaling factor, respectively.
%  Later in XNOR-Net++ \cite{bulat2019xnor}, Bulat et al. fuse the activation and weight scaling factors into a single one that is learned end-to-end based on gradients and this improves the classification accuracy on ImageNet dataset.

% % It is computed as Eq.~\ref{eq:xnor-net rescale}, where $\circledast$ denotes 
% %  the binary convolution and $\odot$ denotes the element-wise multiplication. The binary convolution of $\mathcal{A}$ and $\mathcal{W}$ is rescaled by the weight scaling factor $\alpha$ and the activation scaling factor $\mathcal{K}$, both of which are calculated analytically.


% \zc{Similarly, you should explain the meaning of A, W and the operators $\circledast$ in the formula}
% Then in Real-to-binary Net \cite{martinez2020training}, Martinez et al. used a data-driven channel re-scaling module that takes the pre-convolution activations as input to predict the activation scaling factor. Unlike that in XNOR-Net++ \cite{bulat2019xnor}, these scaling factors are not fixed during inference but rather inferred from data. By doing this, they further improved the classification accuracy on ImageNet over XNOR-Net++. 
As is shown in Figure \ref{fig:pixel}, activation distributions have large pixel-to-pixel variation in SR networks
and the difference of activation magnitudes indicates different scaling factors are preferred for different pixels.
Inspired by \cite{martinez2020training}, we propose spatial re-scaling to better adapt the network to the spatial variation
of activation distributions in SR networks.
% fit the various pixel-wise distributions in SR networks.
We take the real-valued activations $A$ before convolution as input and predict pixel-wise scaling factors $S(A)$, which re-scale the binary convolution output. Spatial re-scaling process can be formulated as follows:
\begin{equation}
A * W \approx(\operatorname{sign}(A) \circledast \operatorname{sign}(W)) \odot \alpha \odot S(A)
\label{eq:spatial rescale}
\end{equation}
where $\circledast$ denotes 
the binary convolution and $\odot$ denotes the element-wise multiplication. $A$, $W$, $\alpha$, and $S\left(A\right)$ denote real-valued activations, weights, the scaling factor of weights, and the spatial-wise scaling factor of activations respectively. $S\left(A\right) \in \mathbb{R}^{1\times H\times W}$ can be calculated with a convolution and a sigmoid function.
% as $\sigma\left( CONV\left(A\right)\right)$. 
As shown in Figure \ref{fig:method}(a), real-valued activations first go through a convolution layer,
which has an input channel of $C$ and an output channel of 1, 
and then pass through a sigmoid function to produce the scaling factors $S(A)$ along the spatial dimension.
During inference, the scaling factor will change dynamically according to different input feature maps.
By re-scaling binary convolution output using $S(A)$, we can reduce the quantization error and the original pixel-wise information in FP activation
will be preserved much better.
Spatial re-scaling leads to a large PSNR improvement of 0.24 dB (from 30.30 dB to 31.54 dB) on Set5 and 0.22 dB (from 25.09 dB to 25.31 dB)
on Urban100 compared with our strong baseline. 

\subsection{Channel-wise Shifting and Re-scaling}

\begin{table}[!tb]
\centering
\caption{Comparison between whether to fuse channel-wise shifting and re-scaling or not based on our baseline with spatial re-scaling. }
\label{tab:fusing}

\scalebox{0.65}{
\begin{tabular}{c|cc|cc|cc}
\hline
\multirow{2}{*}{Method}     & \multirow{2}{*}{OPs} & \multirow{2}{*}{Params} & \multicolumn{2}{c|}{Set5} & \multicolumn{2}{c}{Urban100} \\ \cline{4-7} 
                            &                      &                         & PSNR        & SSIM        & PSNR          & SSIM         \\ \hline
Baseline + spatial re-scale & 2.16G                & 0.05M                   & 31.54       & 0.883       & 25.31         & 0.759        \\
+ channel-wise shift and re-scale             & 2.34G                & 0.09M                   & 31.61       & 0.885       & 25.35         & 0.761        \\
+ w/ fusing                   & 2.27G                & 0.08M                   & \textbf{31.64}       & \textbf{0.885}       & \textbf{25.36}         & \textbf{0.761}        \\ \hline
\end{tabular}
}
\end{table}

In SR networks, activation distributions exhibit larger channel-to-channel variation (Figure \ref{fig:chl}).
Both the mean and magnitude of the activation distributions vary significantly across channels.
% Thus we use channel-wise shifting and re-scaling to adapt to various channel-wise distributions. 
\cite{martinez2020training} has proposed the data-driven channel re-scaling, 
but our method differs from them in further introducing data-driven thresholds to handle the channel-wise variation of both mean and magnitude.
Since the blocks to generate the scaling factors and thresholds are very similar, we further propose to fuse them into one module.
% and fusing channel-wise shifting and re-scaling into one module.
We evaluate the effect of fusing the two blocks in Table \ref{tab:fusing}.
With channel-wise shifting and re-scaling fused, our models have fewer operations and parameters overhead and slightly higher performance.

For the specific process, we take the real-valued activations as input and predict different thresholds and scaling factors for each channel. They are also image dependent, e.g., $\beta_{i}$ in Eq.\ref{eq:act_binarize} is no longer fixed during inference but generated according to different input feature maps. Channel-wise shifting and re-scaling can be formulated as follows:
\begin{equation}
A * W \approx(\operatorname{sign}(A-C_s(A)) \circledast \operatorname{sign}(W)) \odot \alpha \odot C_r(A)
\label{eq:channel-wise_shift_and_rescale}
\end{equation}
where $\circledast$ denotes 
the binary convolution and $\odot$ denotes the element-wise multiplication. $C_s(A), C_r(A) \in \mathbb{R}^{C\times1\times1}$ denote the channel-wise threshold and scaling factor, respectively. 
We show the block diagram in Figure \ref{fig:method}(b).
The real-valued input feature map is first squeezed to a ${C\times1\times1}$ vector by a global average pooling (GAP) layer.
The subsequent fully connected layers and ReLU learn the channel-wise information and output a ${2C\times1\times1}$ vector.
Then the ${2C\times1\times1}$ vector is split into two ${C\times1\times1}$ vectors.
We use the first $C$ channels as the channel-wise bias and pass the last $C$ channels through a sigmoid layer 
as the channel-wise scaling factor, which are used to shift the real-valued activations and re-scale the binary convolution output, respectively. 


% \ml{We can mention previously, channel-wise re-scale has been proposed. We propose to fuse them. Add the comparison between fuse v.s. no fuse.}

\begin{figure}[!tbp]%
  \centering
    \includegraphics[width=0.4\textwidth]{fig/methods.png}
  
% \subfloat[channel-wise shifting\&re-scale]{
%     \label{subfig:channel-wise shifting and re-scale}
%     \includegraphics[width=0.2\textwidth]{fig/chl shift and rescale.png}
%   }

  \caption{Block diagram for spatial re-scaling, and channel-wise shifting and re-scaling.} 
  % Input A is the real-valued activation tensor and C, H, and W denote its dimension. GAP stands for global average pooling. The reduction ratio r is set to 16 for a better trade-off between the performance and the number of operations and parameters.}
  \label{fig:method}
\end{figure}


\subsection{Network Structure}

Combining the spatial re-scaling and the channel-wise shifting and re-scaling methods, we construct the enhanced convolution layer (E-Conv).
Then we build our EBSR model based on E-Conv.
In Figure \ref{fig:E-conv}, we compare the binary convolution layer used in the baseline network and our proposed E-Conv.
We use spatial and channel-wise scaling factors to re-scale the binary convolution output,
and use channel-wise shifting to learn appropriate thresholds for each channel before binarization.
The scaling factors and threshold used in E-Conv are learnable and depend on the real-valued input activations.
In this way, our proposed EBSR can adapt to pixel-to-pixel, channel-to-channel, and image-to-image variations
to reduce the large binarization error and preserve more details.
% In this way, our proposed E-Conv reduces the large quantization error caused by binarization and keeps the original information of input feature maps to a large extent.


\begin{figure}[!tb]%
  \centering

    \includegraphics[width=0.5\textwidth]{fig/E-conv.png}

  \caption{Comparison of (a) the binary convolution layer with a skip connection used in our baseline network and (b) the proposed E-Conv.}
  \label{fig:E-conv}
\end{figure}


Figure \ref{fig:network} shows the basic block based on the E-Conv and our EBSR composed of the basic blocks. Following existing works, the convolution layers in the head and tail modules are not binarized. We choose the lightweight EDSR which has 16 basic blocks and 64 channels, and EDSR which has 32 basic blocks and 256 channels as our backbones, which correspond to EBSR-light and EBSR, respectively.

\begin{figure}[!tb]%
  \centering
  {
    \includegraphics[width=0.35\textwidth]{fig/network.png}
  }
  
  \caption{The structure of our proposed EBSR.  Convolution layers in purple are real-valued vanilla 3x3 convolutions.}
  \label{fig:network}
\end{figure}
We present in section~\ref{ssec:faces} an application of PnP-HVAE on face images, using a pretrained state-of-the-art hierarchical VAE. 
Next, we study the application of our framework to natural images. To that end, we introduce  in section~\ref{ssec:patchVDVAE}  a patch hierachical VAE architecture, that is able to model natural images of different resolutions. In section~\ref{ssec:app_nat}, we provide deblurring, super-resolution and inpainting experiments to demonstrate the relevance of the proposed method.

Additional results are presented in Appendix~\ref{app:add}. All experiments can be reproduced using the code available at \url{https://github.com/jprost76/PnP-HVAE}.



\subsection{Face Image restoration (FFHQ)}\label{ssec:faces}
We first demonstrate the effectiveness of PnP-HVAE on highly structured data, by performing face image restoration.
Latent variable generative models can accurately model structured images such as face images \cite{karras2019style,vahdat2020nvae,child2021very,kingma2018glow}, and then be used to produce high quality restoration of such data. 
In our experiments, we use the VDVAE model of~\cite{child2021very}, pre-trained on the FFHQ dataset~\cite{karras2019style}, as our hierarchical VAE prior.
VDVAE has $L=66$ latent variable groups in its hierarchy and generates images at resolution $256\times256$.

We compare PnP-HVAE with the intermediate layer optimization algorithm (ILO)~\cite{daras2021intermediate} that is based on a different class of generative models than HVAE. ILO is a GAN inversion method which optimizes the image latent code along with the intermediate layer representation of a StyleGAN to generate an image consistent with a degraded observation.
We use the official implementation of ILO, along with a StyleGAN2 model~\cite{karras2020analyzing, stylegan2pytorch}, that was trained for 550k iterations on images of resolution $256\times256$ from FFHQ.  
As VDVAE and StyleGAN models are not trained on the same train-test split of FFHQ, we chose to evaluate the methods on a subset of 100 images from the CelebA dataset~\cite{liu2018large}. 
For super-resolution, the degradation model corresponds to the application of a gaussian low-pass filter followed by a $\times 4$ sub-sampling, and the addition of a gaussian white noise with $\sigma=3$.
For the deblurring, we considered motion blur and  gaussian kernels, both with a noise level $\sigma=8$. %

We provide quantitative comparisons in table~\ref{table:comp_ILO}, along with a visual comparison of the results in figure~\ref{fig:face_restoration}.
PnP-HVAE has the best  PSNR and SSIM results for all the considered restoration tasks, while ILO provides better results  for the perceptual distance.
By jointly optimizing the image and its latent variable, PnP-HVAE provides  results that are both realistic and consistent with the degraded observation.
On the other hand,  ILO  only optimizes on an extended latent space. This method generates  sharp and realistic images with better LPIPS scores,   
but the results lack  of consistency with respect to the observation, which explains the overall lower PSNR performance. 






\subsection{PatchVDVAE: a HVAE for natural images}\label{ssec:patchVDVAE}
Available generative models in the literature operate on images of  fixed resolutions and
are either restrained to datasets of limited diversity, or even to registered face images~\cite{kingma2018glow,child2021very, vahdat2020nvae, karras2019style}, or requiring additional class information~\cite{brock2018large, dhariwal2021diffusion, song2020score, luhman2022optimizing}.
Fitting an unconditional model on natural images appears to be a more difficult task, as their resolution can change, and their content is highly diverse.
The complexity of the problem can be reduced by learning a prior model on patches of reduced dimension. 
For image restoration problems, the patch model can be reused on images of higher dimensions~\cite{zoran2011learning,prost2021learning,altekruger2022patchnr}. When the model is a full CNN, the prior on the set of the  patches can  be computed efficiently by applying the network on the full image~\cite{prost2021learning}.

We thus introduce  patchVDVAE, a fully convolutional hierarchical VAE.
Contrary to existing HVAE models whose resolution is constrained by the constant tensor at the input of the top-down block, patchVDVAE can generate images of different resolutions by controlling the dimension of the input latent. 
This amounts to defining a prior on patches whose dimension corresponds to the receptive field of the VAE. A similar model is used for image denoising in~\cite{prakash2021interpretable}.

 
For PatchVDVAE architecture, we use the same bottom-up and top-down blocks as VDVAE~\cite{child2021very}, and replace the constant trainable input in the first top-down block by a latent variable, to make the model fully convolutional (details on the  architecture are given in Appendix~\ref{app:details}). 
The training dataset is composed of $128\times 128$ patches extracted from a combination of DIV2K~\cite{agustsson2017ntire} and Flickr2K~\cite{Lim_2017_CVPR_workshops} datasets.
We perform data augmentation by extracting  patches at $3$ resolutions: HR-images and $\times 2$ and $\times 4$ downscaled images. 
The model is trained for $7.10^5$ iterations with a batch size of $64$. Following the recommendation of~\cite{hazami2022efficient}, we use Adamax optimizer with an exponential moving average and gradient smoothing of the variance.
We set the decoder model to be a gaussian with diagonal covariance, as in~\cite{luhman2022optimizing}.
PatchVDVAE is fully convolutional and can generate images of dimension that are multiples of $64$ as illustrated by
figure~\ref{fig:vdvae}.

\newlength{\patchwidth}
\setlength{\patchwidth}{0.135\columnwidth}
\begin{figure}[!ht]
    \centering
    \begin{subfigure}[t]{.34\columnwidth}\hspace{0.1cm}
        \setlength{\tabcolsep}{0.02pt}
\renewcommand{\arraystretch}{0}
        \begin{tabular}{*{2}{p{1.03\patchwidth}}}
            \includegraphics[width=\patchwidth]{figures_arxiv/patchVDVAE/samples/generated/64x64/setup-5-image-0018.png} &
            \includegraphics[width=\patchwidth]{figures_arxiv/patchVDVAE/samples/generated/64x64/setup-5-image-0016.png} \\
            \includegraphics[width=\patchwidth]{figures_arxiv/patchVDVAE/samples/generated/64x64/setup-5-image-0008.png} &
            \includegraphics[width=\patchwidth]{figures_arxiv/patchVDVAE/samples/generated/64x64/setup-5-image-0019.png}   
        \end{tabular}
    \end{subfigure}\hspace{-0.15cm}
    \begin{subfigure}[t]{.64\columnwidth}
\begin{tabular}{cc}\vspace{-0.1cm}
\includegraphics[width=2\patchwidth]{figures_arxiv/patchVDVAE/samples/generated/256x256/setup-2-image-0009.png}&
        \includegraphics[width=2\patchwidth]{figures_arxiv/patchVDVAE/samples/generated/256x256/setup-2-image-0002.png}\end{tabular}

    \end{subfigure}
    \caption{\label{fig:vdvae} Left: $64\times64$ patches samples from our patchVDVAE model trained on patches from natural images.
    Right: PatchVDVAE is fully convolutional and it can generate images of higher resolution (here: $128\times128$).\vspace{-0.2cm}}
\end{figure}

\subsection{Natural images restoration}\label{ssec:app_nat}
We  evaluate PnP-HVAE on natural image restoration.
For each task, we report the average value of the PSNR, the SSIM, and the LPIPS metrics on $20$ images from the test set of the BSD dataset~\cite{MartinFTM01}.\\


\noindent
{\bf Image deblurring.}
In the experiments, we consider $2$ gaussian kernels and $2$ motion blur kernels from~\cite{levin2009understanding}, with $3$ different noise levels 
$\sigma \in \{2.55, 7.65, 12.75\}$.
As a baseline we consider  EPLL~\cite{zoran2011learning}, which learns a prior on image patches with a gaussian mixture model.
We also compare PnP-HVAE  with PnP-MMO and GS-PnP, $2$ competing convergent Plug-and-Play methods based on CNN denoisers.
PnP-MMO~\cite{pesquet2021learning} restricts the denoiser to be contraction in order to guarantee the convergence of the PnP forward-backard algorithm. GS-PnP~\cite{hurault2022gradient} considers a gradient step denoiser and reaches state-of-the-art performances of non converging methods~\cite{zhang2021plug}.
We set the temperature $\tau$  in our method as $0.95$, $0.8$ and $0.6$ for noise levels $2.55$, $7.65$ and $12.75$ respectively, and we let it run for a maximum of $50$ iterations. 
For the three compared methods we use the official implementations and pre-trained models provided by the respective authors. 
Details on the choice of hyperparameters for the concurrent methods are provided in the Appendix~\ref{app:details}
Figure~\ref{fig:deblurring_bsd} illustrates that our method provides correct deblurring results. 

According to table~\ref{tab:deb}, the performance of PnP-HVAE is between those of EPLL and GS-PnP and it outperforms PnP-MMO for large noise levels.\\

\begin{table}
\begin{center}\footnotesize
    \begin{tabular}{>{\centering}m{.3cm}*{5}{c}}
    $\sigma$ &Method & PSNR$\uparrow$ & SSIM$\uparrow$ & LPIPS$\downarrow$  \\ 
    \hline
    \multirow{4}{*}{\vcell{$2.55$}}
    & PnP-HVAE & $27.75$ & $0.79$ & $0.31$\\
    & GS-PNP \cite{hurault2022gradient} & $\mathbf{29.59}$ & $\mathbf{0.84}$ & $\mathbf{0.22}$\\
    & EPLL \cite{zoran2011learning} & $26.49$ & $0.71$ & $0.36$\\ 
    & PnP-MMO \cite{pesquet2021learning} & $\underbar{29.50}$ & $\underbar{0.83}$ & $\underbar{0.20}$ \\ \hline
    \multirow{4}{*}{\vcell{$7.65$}}
    & PnP-HVAE & $\underbar{26.36}$ & $\underbar{0.72}$ & $\underbar{0.40}$\\
    & GS-PNP \cite{hurault2022gradient} & $\mathbf{27.33}$ & $\mathbf{0.77}$ & $\mathbf{0.31}$\\
    & EPLL \cite{zoran2011learning} & $24.04$ & $0.66$ & $0.45$ \\ 
    & PnP-MMO \cite{pesquet2021learning} & $25.34$ & $0.69$ & $0.34$\\
    \hline
    \multirow{4}{*}{\vcell{$12.75$}}
    & PnP-HVAE & $\underbar{25.12}$ & $\mathbf{0.73}$ & $\underbar{0.47}$\\
    & GS-PNP \cite{hurault2022gradient} & $\mathbf{26.32}$ & $\mathbf{0.73}$ & $\mathbf{0.37}$\\
    & EPLL \cite{zoran2011learning} & $23.28$ & $0.61$ & $0.51$ \\ 
    & PnP-MMO \cite{pesquet2021learning} & $22.42$ & $0.53$& $0.54$ \\
    \hline
    &\vspace*{-.3cm}\\
            \multicolumn{2}{c}{Blur and motion kernels}& \multicolumn{3}{c}{
        \includegraphics*[scale=1]{figures_arxiv/kernels/4.png}\;\includegraphics*[scale=1]{figures_arxiv/kernels/7.png}\;\includegraphics*[scale=1]{figures_arxiv/kernels/9.png}\;\includegraphics*[scale=1]{figures_arxiv/kernels/11.png}} 
    \end{tabular}
        \caption{\label{tab:deb}Comparison  of PnP-HVAE  and other restoration methods on deblurring. Results are averaged on $4$ kernels.\vspace{-0.2cm}}% on image deblurring.}
    \end{center}
\end{table}

\begin{figure}
    
    \begin{subfigure}[h]{\linewidth}
        \centering
        \includegraphics*[width=\columnwidth]{figures_arxiv/deb_s255_k7.pdf}\vspace{-0.1cm}
        \caption{Gaussian blur, $\sigma=2.55$}
    \end{subfigure}
    \begin{subfigure}[h]{\linewidth}
        \centering
        \includegraphics*[width=\columnwidth]{figures_arxiv/deb_s765_k11.pdf}\vspace{-0.1cm}
        \caption{Motion blur, $\sigma=7.65$}
    \end{subfigure}\vspace*{-0.1cm}
    \caption{\label{fig:deblurring_bsd} Natural image deblurring\vspace{-0.1cm}}
\end{figure}

\noindent {\bf Effect of the temperature.}
PnP-HVAE gives control on the temperature of the prior over the latent space.
In figure~\ref{fig:temp_effect}, we illustrate that reducing the temperature increases the strength of the regularization prior. In this example the tuning $\tau=0.7$ produces the best performance.\\
\begin{figure}[!ht]
   
    \includegraphics[width=\columnwidth]{figures_arxiv/demo_temp.pdf}\vspace{-0.15cm}
    \caption{ \label{fig:temp_effect} Effect of the temperature in PnP-VAE on a deblurring problem, with $\sigma=7.65$.\vspace{-0.15cm}}
\end{figure}


\noindent
{\bf Image inpainting.}
Next we consider the task of noisy image inpainting. 
We compose a test-set of 10 images from the validation set of BSD~\cite{MartinFTM01} and we create masks
  by occluding diverse objects of small size in the images. 
A gaussian white noise with $\sigma=3$ is added to the images.
As a comparaison, we still consider GS-PnP and EPLL.
For PnP-HVAE, the temperature is set to $\tau=0.6$, and the algorithm is run for a maximum of $200$ iterations, unless the residual $||\x_{k+1}-\x_k||$ is on a plateau.
We provide on Table~\ref{tab:inpainting_bsd} the distortion metrics with the ground truth, as well as a visual
\begin{table}



\begin{center}
    \begin{tabular}{cccc}
        & PSNR$\uparrow$ & SSIM$\uparrow$ &LPIPS$\downarrow$ \\\hline
        PnP-HVAE  & $\mathbf{29.54}$ & $\mathbf{0.93}$ & $\mathbf{0.06}$\\
        GS-PNP & $28.52$ & $\mathbf{0.93}$ & $0.09$\\
        EPLL & $\underline{29.16}$ & $\mathbf{0.93}$ & $\mathbf{0.06}$\\
    \end{tabular}
    \caption{\label{tab:inpainting_bsd}Quantitative evaluation for inpainting on BSD.}
    \end{center}
\end{table}
comparison on figure~\ref{fig:inpainting_bsd}. 
With its hierarchical structure,  PnP-HVAE outperforms the compared methods. \vspace{0.05cm}



\begin{figure}[!h]
    \includegraphics[width=\columnwidth]{figures_arxiv/demo_inp_bsd2.pdf}\vspace{-0.1cm}
    \caption{\label{fig:inpainting_bsd}Natural image inpainting\vspace{-0.3cm}}
\end{figure}












\section{Conclusion}
In this work, we propose polynomial function based implicit neural representations for large image datasets while only using Linear and ReLU layers. Our Poly-INR model captures high-frequency information and performs comparably to the state-of-the-art CNN-based generative models without using convolution, normalization, upsampling, or self-attention layers. The Poly-INR model outperforms previously proposed positional embedding-based INR GAN models. We demonstrate the effectiveness of the proposed model for various tasks like interpolation, style-mixing, extrapolation, high-resolution sampling, and image inversion. Additionally, it would be an exciting avenue for future work to extend our Poly-INR method for $3$D-aware image synthesis on large datasets like ImageNet.   

\section*{Acknowledgements}
This material is based upon work supported by the Defense Advanced Research Projects Agency (DARPA) under Agreement No. HR00112290073. Approved for public release; distribution is unlimited.

%%%%%%%%% REFERENCES
{\small
\bibliographystyle{ieee_fullname}
\bibliography{main}
}
\documentclass[./main.tex]{subfiles}
\begin{document}

\title{Supplemental Material\\From Clean Room to Machine Room: Commissioning of the First-Generation BrainScaleS Wafer-Scale Neuromorphic System}

\DeclareRobustCommand{\enumauthorrefmark}[1]{\smash{\textsuperscript{\footnotesize #1}}}

\newcommand{\contributedSymbol}{\IEEEauthorrefmark{1}}
\newcommand{\uheiSymbol}{\enumauthorrefmark{1}}
\newcommand{\ugoeSymbol}{\enumauthorrefmark{2}}


\author{
	\IEEEauthorblockN{%
		Hartmut Schmidt\contributedSymbol,
		José Montes\contributedSymbol,
		Andreas Grübl,
		Maurice Güttler,
		Dan Husmann,
		Joscha Ilmberger,\\
		Jakob Kaiser,
		Christian Mauch,
		Eric Müller,
		Lars Sterzenbach,
		Johannes Schemmel,
		Sebastian Schmitt\\
	}

	\thanks{
		\IEEEauthorblockA{%
		\contributedSymbol%
		Contributed equally\\
		}
	}
}

\maketitle
Next, we present the Supplementary Materials for the paper ``Re-ReND: Real-time Rendering of NeRFs across Devices''.
Specifically, in addition to the results reported in the paper, we report results of \methodname w.r.t. Image Quality~(Section~\ref{sec:im_qual}) and (Section~\ref{sec:quali}), Rendering Speed~(Section~\ref{sec:fps}), Mesh Size~(Section~\ref{sec:mesh_size} and Section~\ref{sec:meshi}), Disk Space~(Section~\ref{sec:disk_space}), validation of view-dependent effects (Section~\ref{sec:val}),  sensitivity to geometry variations (Section~\ref{sec:geo}) and Photo-metric quality w.r.t. embedding dimensionality $D$ (Section~\ref{sec:dim}).
Furthermore, we encourage the reviewers to watch the \textbf{associated video}, \texttt{Re-ReND.mp4}, demonstrating \methodname's capabilities of real-time rendering across devices.
% In particular, please refer to .
This video demonstrates how \methodname can render, in real time, a scene composed of tens (\Figure{composit}) or even thousands (\Figure{many_objects}) of objects. % , respectively. %  , or even with thousands of . %  in an AR headset.
\Figure{composit} illustrates such a scene, composed of moving chairs, hotdogs, the drumset, and a microphone.


% Finally, we also provide the PyTorch~\cite{NEURIPS2019_9015} and GLSL implementations of our method inside the folders called \texttt{Re-ReND\_Pytorch\_code} and \texttt{Re-ReND\_GLSL\_code}.

% \thispagestyle{empty}
% \appendix

%%%%%%%%% BODY TEXT - ENTER YOUR RESPONSE BELOW
% \section{The PyTorch code and GLSL code}

%  \begin{itemize}
%     \item Clean and README.md
%     \item Should I upload only pur method or MipNeRF and NeRF++?
%     \item Should I upload the generated data and the meshes in a google drive? What happens with anonymity?
% \end{itemize}

% \section{A video showing how we were measuring the FPS}
% \section{A video showing real scenes in comparison with MobileNeRF and SNeRG}
% \section{Qualitative Results}

%  \begin{itemize}
%     \item all objects visualizations 
% \end{itemize}

%-------------------------------------------------------------------------


\begin{figure}
    \centering
    \includegraphics[width=\linewidth]{pics/quantitative.pdf}
    \caption{Box plots of quantitative benchmarks MIG, FactorVAE, Disentanglement, and reconstruction error on dSprites and Shapes3D.}\label{fig:quantitative}
\end{figure}


\bibliographystyle{style/IEEEtran}
\bibliography{bib/vision}

\end{document}

\end{document}
