\section{Introduction} \label{sec:intro}

Preference-based many-to-many matching problems model applications like assigning students to courses \cite{CEFMMP2014} and workers to firms~\cite{roth1984stability} where vertices on both sides of the bipartition have preferences over their neighbors and can accept multiple partners. In the student-course allocation problem, students typically have a \emph{minimum} requirement on the number of courses they need to complete in a semester, and a course may be offered only if there is a \emph{minimum} number of registrants. Similarly, in the worker-firm assignment, certain workers \emph{must} be assigned jobs, and firms may need a \emph{minimum} number of workers for their operations. These \textit{minimum} requirements can be specified as \textit{lower quotas} by the corresponding agents to denote the minimum number of agents they need to get assigned to in any matching. Therefore, it is natural to have lower quotas on both sides of the bipartition. Lower quotas~\cite{BFIM10,HIM16,NN17} have been considered in the literature in the context of matching residents to hospitals (\textit{e.g.} National Resident Matching Program in the U.S.A.) where rural hospitals often face the problem of being understaffed~\cite{roth84,Roth86}. \textit{Ties} in preferences is yet another important practical consideration in matching problems. For instance, hospitals with a large number of applicants often find it difficult to generate strict preference lists~\cite{swat/IrvingMS00}. Similarly, in the case of course allocation, it is natural for teachers to have all the students with equal scores in a single tie in their preference lists. 


In this work, we explore the generalized matching model where \emph{ties} as well as \emph{lower quotas} on both sides of the bipartition are explored in the \textit{many-to-many setting}. 
Our setting generalizes several models, such as the one-to-one setting of the stable marriage problem~\cite{GS62,irving1994stable}, the many-to-one setting of the Hospital/Residents problem with ties~\cite{swat/IrvingMS00}, and the many-to-many assignment problem~\cite{brandl2019two} -- all of these works are without lower quotas. Recently, lower quotas have been considered in the presence of strict preference lists~\cite{BFIM10,HIM16,NN17,NNRS21,Kavitha2021,NNRS2024popular} and in the presence of ties in preferences~\cite{GokoMMY22,MakinoMY22}. However, none of these works consider ties as well as lower quotas on \textit{both sides} of the bipartition. In the conference version of our work ~\cite{nasre2023critical}, we consider this model for the one-to-one setting. 
In parallel and independent of our work, Cs{\'a}ji~\cite{csaji2024AAMAS} considered
a generalized problem involving critical edges and free edges for the same model in the one-to-one setting.
Therefore, it leaves the model of ties as well as lower quotas in the \textit{many-to-many} setting unexplored. Recently, in an unpublished work~\cite{csaji2024S}, Cs{\'a}ji generalized his work for the one-to-one setting to matroid constraints for the vertices. Ours is a simple  proposal based algorithm and can be viewed as an extension of the well-known Gale and Shapley algorithm~\cite{GS62} -- thus making it practical to implement.


 Formally, the input to our problem is a bipartite graph $G=(\AA \cup \BB,E)$, where $\AA$ and $\BB$ are two sets of vertices and $E$ denotes the set of all the acceptable vertex-pairs. Every vertex $u\in \AA\cup\BB$ ranks a subset of vertices in its neighborhood in $G$, and this ordering is called the {\em preference list} of $u$, denoted by $\prefu$. We say that a vertex strictly prefers a neighbor with a smaller rank over another neighbor with a larger rank. If a vertex is allowed to be indifferent between some of its neighbors and is allowed to assign the same rank to such neighbors, it is referred to as a {\em tie}.  If ties are not allowed, the preference lists are said to be {\em strict}. For a vertex $u$, let $v_1$ and $v_2$ be its neighbors in $G$. We use $v_1\succ_u v_2$ to denote that $u$ strictly prefers $v_1$ over $v_2$ and $v_1\succeq_u v_2$ to denote that $u$ either strictly prefers $v_1$ over $v_2$ or is indifferent between them. In addition, every vertex $u\in\AA\cup\BB$ has an associated lower quota $q^-(u)\in \mathbb{Z}^+\cup\{0\}$ and  an upper quota $q^+(u)\in \mathbb{Z}^+$ such that $q^-(u)\le q^+(u)$. The upper quota $q^+(u)$ denotes the maximum number of neighbors that $u$ can accommodate in any assignment. The lower quota $q^-(u)$ of a vertex $u$ denotes the minimum number of neighbors that must be assigned to $u$ in any assignment. If for a vertex $u$, $q^-(u)>0$ then we call $u$ an \emph{lq-vertex} otherwise $u$ is a non-\emph{lq} vertex. 

A \emph{matching} $M$ in $G$ is a subset of the edge set $E$ such that $|M(u)|\le q^+(u)$ for each vertex $u\in\AA\cup\BB$, where $M(u)$ denotes the set of neighbors assigned to $u$ in $M$. Note that this definition of matching deviates from the classical one used in graph theory. We still use the term \emph{matching} as done in the literature~\cite{brandl2019two,HIM16}. In the matching $M$, a vertex $u\in\AA\cup\BB$ is called \emph{fully-subscribed} if $|M(u)| = q^+(u)$, \emph{under-subscribed} if $|M(u)| < q^+(u)$, \emph{deficient} if $|M(u)| < q^-(u)$, and \emph{surplus} if $|M(u)| > q^-(u)$.  A matching $M$ is {\em feasible} if no vertex in $\AA \cup \BB$ is deficient in $M$.  Feasible matchings are desirable since they always satisfy the lower quotas of every vertex. Unfortunately, the existence of a feasible matching is not guaranteed~\cite{NNRS2024popular}. When a feasible matching does not exist, we seek to compute a \textit{critical matching} that fulfils lower quotas to the maximum possible extent. For a matching $M$, we define the deficiency of a vertex $u\in\AA\cup\BB$  as $\max\{0,q^-(u)-|M(u)|\}$. The deficiency, $\Df{M}$, of a matching $M$ is equal to the sum of the deficiencies of all the vertices $u\in\AA\cup\BB$ in $G$.  

\begin{definition}[Critical Matching]\label{def:critical}
A matching $M$ in $G$ is critical if there is no matching $N$ in $G$ such that $\Df{N}<\Df{M}$. 
\end{definition}

 In this work, we are interested in computing a critical matching that is {\em optimal} with respect to the preferences of the vertices. Two of the most extensively investigated optimality notions for the bipartite matching problem with preferences on both sides are \textit{stability} and \textit{popularity}. 


\begin{definition}[Stable Matchings]\label{def:bPT}
Given a matching $M$, a pair $(a, b) \in E \setminus M$ is called a blocking pair with respect to the matching $M$ if $(i)$ either $|M(a)| < q^+(a)$ or $b\succ_a b'$ for some $b' \in M(a)$ and $(ii)$ either $|M(b)| < q^+(b)$ or $a\succ_b a'$ for some $a'\in M(b)$. The matching $M$ is stable if there is no blocking pair w.r.t. $M$.
\end{definition}

We remark that for a pair $(a, b)$ to block a matching $M$, both $a$ and $b$ {\em strictly} prefer each other over some of their current partners in $M$. Note that this stability notion is referred to as the \textit{weak stability}~\cite{irving1994stable} as two other notions of stability, \textit{strong} and \textit{super}, are also formulated in the literature. In this paper, we do not consider strong and super stability as they are not guaranteed to exist, and we use the term \textit{stability} to indicate weak stability. 

\vspace{0.1in}

\noindent {\bf Stable matching in the presence of ties and \emph{lq}-vertices:} It is known that every instance of the stable marriage problem with ties admits a stable matching~\cite{irving1994stable}, and it can be efficiently computed. When preferences are strict, all stable matchings have the same size. However, this need not be the case in the presence of ties in preferences. That is, stable matchings need not have the same size in the presence of ties in preferences. Moreover, the problem of computing a maximum or minimum size stable matching is NP-hard~\cite{manlove2002hard}. When we have \emph{lq}-vertices as a part of the input, a stable matching, which is also critical, may not exist even without ties~\cite{NNRS2024popular}. Moreover, in the presence of ties, deciding whether the given instance admits a stable and critical matching is NP-hard (see Claim~\ref{cl:nphardCS} in Section~\ref{subsec:NPhard}).


As mentioned earlier, \textit{popularity} is another well-investigated notion of optimality for the matching problem with preferences on both sides. Informally, a matching $M$ is \emph{popular} in a set of matchings if no majority of vertices wish to deviate from $M$ to any other matching in that set. 

\vspace{0.1in}

\noindent \noindent {\bf Popular matching in the presence of \emph{lq}-vertices:}  Since in the presence of \textit{lq}-vertices, stability and criticality are not simultaneously guaranteed, this alternate notion of optimality is extensively investigated in the literature for the settings involving strict preferences~\cite{NN17,NNRS21,Kavitha2021,NNRS2024popular}. The goal is to compute {\em popular} among the set of critical matchings. It is known~\cite{Kavitha2021,NNRS21} that an instance with strict preference lists \emph{always} admits a matching that is popular amongst critical matchings, and such a matching can be computed efficiently. Hence, it is natural to consider popularity in the presence of \emph{lq}-vertices and ties. However, when ties are present in the preferences only on one side of the bipartition (even with no \emph{lq}-vertices), popular matchings are not guaranteed to exist~\cite{biro2010popular}, and deciding whether a popular matching exists is an NP-hard problem. In light of this, we explore the notion of  {\em relaxed stability}. 

\vspace{0.1in}

\noindent {\bf Relaxed stability in the presence of ties and \emph{lq}-vertices: } The notion of relaxed stability was introduced and  studied by Krishnaa~\etal~\cite{krishnaa2023envy} for the Hospital/Residents problem  with lower quotas (\HRLQ).  In their setting, preferences are assumed to be strict. The  \HRLQ\ setting is a many-to-one preference-based matching problem where each resident has an upper quota equal to one, and each hospital $h$ is associated with an upper quota $q^+(h)$ and a lower quota $q^-(h) \leq q^+(h)$. To minimize the deficiency of a matching, certain residents may be {\em forced} to be matched to some \textit{lq}-hospitals. The notion of relaxed stability ignores the blocking pairs involving only such residents. That is, in a matching $M$, if a resident matched to $h$ participates in a blocking pair, and the hospital $h$ is {\em surplus} (\textit{i.e.} $|M(h)| > q^-(h)$), then $M$ is not relaxed stable.



\begin{definition}[Relaxed Stability in \HRLQ~\cite{krishnaa2023envy}]\label{def:rsmHRLQ} A matching $M$ is relaxed stable if, no unmatched resident blocks $M$ and for a blocking resident $r$, the hospital $h = M(r)$ satisfies $|M(h)| \le q^-(h)$.
\end{definition}


A matching $M$ is called critical relaxed stable matching (\CRSM) if it is \emph{critical} as well as \emph{relaxed stable}. In the \HRLQ\ setting~\cite{krishnaa2023envy}, preferences are strict, upper quotas and lower quotas are allowed only for hospitals. In contrast, we allow ties in preferences and \emph{lq}-vertices to appear on {\em both sides} of the bipartition in the many-to-many setting. 

\vspace{0.1in}

\noindent\textbf{Our Contribution:}
Our first contribution is to generalize the above notion of relaxed stability to the many-to-many setting in the presence of ties in the preference lists and lower quotas on both sides of the bipartition. We prove that for an \HRLQ\ instance, our notion of relaxed stability (see Definition~\ref{def:rsmMM}) is equivalent to the one by Krishnaa~\etal~\cite{krishnaa2023envy} (Definition~\ref{def:rsmHRLQ}). Our next contribution is to show that a \CRSM\ always exists in our setting. We remark that when $q^-(u)=0$ and $q^+(u)=1$ for all $u\in \AA\cup\BB$, an instance of our setting is the same as the stable marriage setting with ties but without \emph{lq}-vertices, and hence the set of \CRSM\ is the same as the set of stable matchings. This implies that computing a maximum size critical \RSM\ is NP-hard~\cite{manlove2002hard} and hard to approximate~\cite{halldorsson2007improved}. For the problem of computing a stable matching of the maximum size in the presence of ties, the current best approximation factor \cite{Kiraly13,mcdermid20093,paluch2014faster} is $\frac{3}{2}$. The main result (Theorem~\ref{theo:main}) provides the same approximation size guarantee for a maximum size \CRSM\ problem in our generalized setting. More specifically, we give an efficient algorithm for computing a $\frac{3}{2}$-approximation of the maximum size \CRSM\ for the many-to-many setting where ties and lower quotas appear on both sides of the bipartition.


\begin{theorem}\label{theo:main}
 Let $G=(\AA\cup\BB,E)$ be a bipartite graph where each vertex $v\in\AA\cup\BB$ has an associated lower quota $q^-(v)$, an upper quota $q^+(v)\ge q^-(v)$ and a preference ordering possibly containing ties over its neighbors. Then $G$ always admits a matching $M$ such that $M$ is \CRSM\ and can be computed efficiently. Moreover, $|M|\ge \frac{2}{3}|M_{max}|$, where $M_{max}$ is a maximum size \CRSM\ in $G$.  
\end{theorem}



%\noindent\textbf{Challenges in many-to-many setting:} In general, it is nontrivial to adapt one-to-one or many-to-one matching algorithms to the many-to-many setting. This is pointed out in the literature (see \cite{kamiyama2019many,chen2010strongly}). Specifically, in our case, as pointed out in~\cite{NNRS2024popular}, there are challenges in extending the approaches considered earlier for computing popular feasible matching in the one-to-one and many-to-one settings to the many-to-many setting. The algorithms in \cite{NNRS21,Kavitha2021} both propose a reduction where the original instance with two-sided lower quotas is converted into an instance {\em without} lower quotas. The standard Gale-Shapley algorithm~\cite{GS62} is used to obtain a  stable matching in the reduced instance, and the edges in the stable matching are mapped to the edges in the original instance. The reduced instance constructed in \cite{NNRS21,Kavitha2021}  has  multiple {\em level} copies of the original vertices. The notion of levels for vertices will become clear when we give the description of our algorithm in Section~\ref{sec:Algo}.
%Extending a similar approach to the many-to-many setting can lead to a vertex $a$ being matched to $b$ multiple times in the reduced instance, and such matching cannot be mapped back to a feasible/critical matching in the original instance. Such a difficulty does not arise when the quota of at least one side of the bipartition is one which is indeed the case in the earlier works ~\cite{NN17,NNRS21,Kavitha2021} (see \ref{sec:challenges} for an example). In this work, we present a proposal-based algorithm inspired by a similar idea mentioned by Kavitha in~\cite{Kavitha2021}.



\noindent{\bf Related work:}
For strict preferences and lower-quotas, various optimality notions, that are relaxations of stability like envy-freeness~\cite{Yokoi20,krishnaa2023envy}, popularity~\cite{NN17,Kavitha2021,NNRS21,NNRS2024popular}, and relaxed stability~\cite{krishnaa2023envy} have been studied. Envy-freeness is a relaxation of stability and is defined by the absence of envy-pairs. However, critical envy-free matching is not guaranteed to exist~\cite{krishnaa2023envy}. Relaxed stability and popularity do not define the same set of matchings, even in the one-to-one strict-list setting where \emph{lq}-vertices are restricted to only one side of the bipartition. That is, neither one implies the other. This fact has already been pointed out by Krishnaa \etal~\cite{krishnaa2023envy}. Hamada~\etal~\cite{HIM16} consider the problem of computing a matching with the minimum number of blocking pairs or blocking residents among all \textit{feasible} matchings for the \HRLQ\ problem.

For the stable marriage problem with ties (without \emph{lq}-vertices), there is a long line of investigation \cite{kiraly2011better,mcdermid20093,Kiraly13,paluch2014faster,iwama201425,huang2015improved} in order to improve the approximation ratio under various restricted settings. The best-known approximation algorithm for the case of one-sided ties is by Lam and Plaxton~\cite{lam20191}, whereas the best-known algorithm for the case of two-sided ties ($\frac{3}{2}$-approximation) is by ~\cite{Kiraly13,mcdermid20093,paluch2014faster}. Dudycz~\etal~in~\cite{dudycz2022tight} showed that assuming the Small Set Expansion Hypothesis, there cannot be a $(\frac{3}{2}-\epsilon)$-approximation algorithm for the maximum size stable matching problem in the presence of ties.

 The model investigated in this paper which involves both ties and \emph{lq}-vertices on \textit{both} sides in the many-to-many setting, has not received much attention.  We mention works that consider ties and \emph{lq}-vertices in the restricted \HR\ setting.  Goko \etal~\cite{GokoMMY22} and Makino \etal~\cite{MakinoMY22}  focus on the many-to-one setting where \emph{lq}-vertices are allowed on only one side of the bipartition. They examine the problem of computing a weakly stable matching that maximizes the total \emph{satisfaction ratio} for lower quotas, where the satisfaction ratio of each hospital $h$ is given by $\min \{1,\frac{|M(h)|}{q^-(h)}\}$. Their goal is to compute a matching, among all stable matchings, that achieves the highest total satisfaction ratio. They refer to this problem as \textit{{\sf HRT} to Maximally Satisfy Lower Quotas} ({\HRTMSLQ}).  It is easy to observe that the \HRTMSLQ\ problem has a non-empty solution set, which is in contrast with the stable critical matching problem. We emphasize that the \HRTMSLQ\ problem prioritizes preference optimality (in this case, stability) over \textit{lq}-vertices. Therefore, the matching output by the algorithms considered in~\cite{GokoMMY22} and~\cite{MakinoMY22} is guaranteed to be stable, but it may not be critical. In our work, we prioritize lower quotas to be satisfied as much as possible, and the matching output by our algorithm is guaranteed to be critical. 
