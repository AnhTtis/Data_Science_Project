\begin{filecontents}{letter.bib}
@article{Kiraly13,
  author    = {Zolt{\'{a}}n Kir{\'{a}}ly},
  title     = {Linear {T}ime {L}ocal {A}pproximation {A}lgorithm for {M}aximum {S}table {M}arriage},
  journal   = {Algorithms},
  volume    = {6},
  number    = {3},
  pages     = {471--484},
  year      = {2013},
  url       = {https://doi.org/10.3390/a6030471}
}
\end{filecontents}


\documentclass{article}
\usepackage{hyperref}
\usepackage{cleveref}
\title{Index of Supplementary Materials}
\author{Meghana Nasre, Prajakta Nimbhorkar, Keshav Ranjan}
\date{}

\begin{document}

\maketitle

\section*{Title of Manuscript}
Critical Relaxed Stable Matchings with Ties in the Many-to-Many Setting

\section*{Description of Supplementary Materials}
The supplementary document presents an adaptation of Kir{\'a}ly's algorithm~\cite{Kiraly13} for computing a $\frac{3}{2}$-approximation of a maximum-size stable matching in the presence of two-sided ties, extended from the \textit{many-to-one setting} to the \textit{many-to-many setting}. 

\section*{File Listing}
\begin{itemize}
    \item \textbf{AppendixA.pdf} – This document contains the adaptation of Kir{\'a}ly's algorithm from the many-to-one setting to the many-to-many setting, along with its formal analysis.
\end{itemize}

\section*{Reason for Inclusion}
To the best of our knowledge, Kir{\'a}ly's algorithm has not been formally documented for the many-to-many setting. To ensure that our manuscript remains self-contained, we provide a generalized version of the algorithm along with its theoretical justification. 

\bibliographystyle{plain}
\bibliography{letter}

\end{document}
