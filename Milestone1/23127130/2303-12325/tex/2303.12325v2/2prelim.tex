\section{Preliminaries} \label{sec:prelim}

Our algorithm is a proposal-based algorithm that combines the ideas in 
(i) Kir{\'a}ly's algorithm~\cite{Kiraly13} for computing a $\frac{3}{2}$-approximation of a maximum size stable matching in the presence of two-sided ties \textit{without} lower quotas and (ii) multi-level algorithm for computing popular critical matching~\cite{NNRS2024popular} in the presence of strict preferences (without ties) and lower quotas on both sides of the bipartition. It is useful to extend the many-to-one variant of  Kir{\'a}ly's algorithm~\cite{Kiraly13} to the many-to-many setting. %We give this extension for the sake of completeness.

\subsection{Kir{\'a}ly's algorithm for the many-to-many setting}\label{sec:GenKiraly}
In this section, we present the extension of  Kir{\'a}ly's algorithm~\cite{Kiraly13} for many-to-one setting to the many-to-many setting for computing a $\frac{3}{2}$-approximation of a maximum size stable matching. The algorithm is a proposal-based algorithm where vertices in $\AA$ propose and vertices in $\BB$ accept or reject the proposals. As long as a vertex $a$ is under-subscribed and has not proposed to all vertices in its preference list $\prefa$, the vertex $a$ proposes to its \emph{favorite} neighbor (we define it formally in  Definition~\ref{def:favNbr}). If $a\in \AA$ remains under-subscribed after proposing all the vertices in $\prefa$, the vertex $a$ achieves the `$*$' status, denoted by $a^*$. Subsequently, $a^*$ starts proposing vertices from the beginning of $\prefa$.  As considered in~\cite{Kiraly13}, the $*$ status of a vertex $a$ can be interpreted as improving the rank of $a$ in the preference list of any neighbor $b$ by some $0<\epsilon<1$. Thus, for a vertex $a$, its neighbor, say $b$, prefers $a^*$ over any vertex $a'$ with which $a$ is in a tie in $b$'s preference list provided that $a'$ does not have the $*$ status. However, any vertex $a'$ such that $a'\succ_b a$ in the preference list of $b$ continues to remain better preferred even over $a^*$. We adapt the definition of \emph{uncertain proposal} and \textit{favorite neighbor} from~\cite{Kiraly13} for the one-to-one setting to the many-to-many setting.  To the best of our knowledge, Kir{\'a}ly's algorithm for the many-to-many settings has not been documented. Hence, for the sake of completeness, we give the generalized algorithm for computing a $\frac{3}{2}$-approximation of the maximum size stable matching in the many-to-many setting and its analysis in~\ref{append:kiralyMM}. 


\begin{definition}[Uncertain Proposal] \label{def:uncertainProp}
   Let $M$ be an intermediate matching computed by the algorithm and $b\in\BB$ be the $k^{th}$ ranked neighbor in $\prefa$ for some $a\in\AA$. During the course of the algorithm, a proposal from vertex $a$ to $b$ is labeled uncertain if it is the first proposal from $a$ to $b$, and there exists another $k^{th}$ ranked unproposed neighbor of $a$ which is under-subscribed with respect to $M$.
\end{definition}


Each time an  $a$ or $a^*$ proposes to its \emph{favorite} neighbor $b$,  the vertex $b$ accepts/rejects as follows: 
\begin{enumerate}[label=\arabic*]
    \item\label{itm:AccRej1} If $b$ is under-subscribed then $b$ accepts the proposal. 
    \item\label{itm:AccRej2} If $b$ is fully subscribed, and for some $a'\in M(b)$, $(a',b)$ is labeled as an uncertain proposal, then $b$ rejects $a'$ and accepts the proposal, irrespective of the ranks of $a$ and $a'$ in $\prefb$. In this case, $b$ is {\em marked} by $a'$. The reason for $a'$ \textit{marking} the vertex  $b$ is explained below.
    
    \item\label{itm:AccRej3} If $b$ is fully subscribed and no proposal to $b$ is labeled uncertain, then $b$ compares the proposing vertex with a least preferred partner in the set $M(b)$. Note that some vertices in $M(b)$  may be with the $*$ status, whereas others may not be with the $*$ status. 
    A least preferred partner of $b$ is a vertex at the highest rank in $M(b)$, which does not have a $*$ status if one exists. Otherwise, it is a vertex with $*$ status at the highest rank in $M(b)$. Therefore, let $a'$ be one of the least preferred partners in $M(b)$. If $a'$ is lower preferred than the proposing vertex $a$ or $a^*$, then $b$ rejects $a'$, otherwise $b$ rejects the proposing vertex $a$ or $a^*$.
    \end{enumerate}
    
    In  (\ref{itm:AccRej2}), $b$ rejects the uncertain proposal from $a'$ and accepts the proposing vertex {\em irrespective} of the preference of $b$ between $a$ and $a'$. Later, when $a'$ gets a chance to propose, and if it remains under-subscribed even after it has proposed to all the neighbors at the rank of $b$,  then  $a'$ must propose to the \textit{marked} vertex $b$ before proposing to the next higher-ranked neighbors. This ensures that the stability is respected.
    In contrast, in (\ref{itm:AccRej3}) above, when the proposal $(a', b)$ is not uncertain, and $a'$ is lower preferred than $a$, then $a'$ does not mark $b$.   
    
Now, we define the favorite neighbor of a vertex $a$, which is an adaptation of the definition in~\cite{Kiraly13}.
\begin{definition}[Favorite Neighbor of $a$] \label{def:favNbr}
  Assume that $k$ is the best rank at which some neighbor $b$ of $a$ exists in $\prefa$ such that $b$ is unproposed or marked by $a$. Then $b$ is the favorite neighbor of $a$ if:
    \begin{enumerate}[label=(\roman*)]
       \item\label{itm:FavNbrKiral1} there exists at least one under-subscribed neighbor of $a$  at $k^{th}$ rank in $\prefa$ which is unproposed by $a$, and $b$ has the lowest index among all such neighbors, or
        \item\label{itm:FavNbrKiral2} conditions in \ref{itm:FavNbrKiral1} do not hold, and there exists at least one fully subscribed neighbor of $a$ at $k^{th}$ rank which is unproposed by $a$ and $b$ is the lowest index among all such unproposed neighbors of $a$
        %all the $k^{th}$-ranked neighbors of $a$ are fully subscribed, and $b$ is the lowest index among all such neighbors which are unproposed by $a$, 
        or
        \item\label{itm:FavNbrKiral3} conditions in \ref{itm:FavNbrKiral1} and \ref{itm:FavNbrKiral2} do not hold, and $b$ has the lowest index among all the vertices, which are marked by $a$.
    \end{enumerate}
 \end{definition}

In the above definition (Definition~\ref{def:favNbr}), the vertices $a$ and $a^*$ (vertex $a$ with and without $*$ status) are considered as two different vertices. That is, it is possible that the vertex $a$ is matched to $b$, and the vertex $b$ is the favorite neighbor for $a^*$. This happens because when $a^*$ proposes, it starts proposing from the beginning of $\prefa$. We illustrate this using an example given below. Suppose the given instance contains only one edge $(a,b)$ and the capacities of $a$ and $b$ are $2$ and $1$, respectively. When $a$ proposes to $b$, it gets accepted by $b$. Since $a$ is under-subscribed and has proposed to all of its neighbors, $a$ achieves $*$ status. Subsequently, $a^*$ proposes from the beginning of $\prefa$ where $b$ is its favorite neighbor. The vertex $b$ accepts this proposal. It is equivalent to saying that $a$ is replaced with $a^*$, and our algorithm terminates with $a^*$ being matched to $b$.
 

\subsection{Overview of the multi-level popular critical matching algorithm}
We now briefly describe the algorithm by Nasre \etal~\cite{NNRS2024popular} for computing a maximum-size popular critical matching in the many-to-many setting with strict preferences and lower quotas on both sides of the bipartition. Let $\S$ and $\T$ denote the sum of lower quotas for all vertices in $\AA$ and $\BB$, respectively, i.e., $\S=\sum_{a\in\AA}q^-(a)$ and $\T = \sum_{b\in\BB}q^-(b)$. The algorithm in~\cite{NNRS2024popular} follows a multi-level approach, utilizing $\S+\T+2$ levels, indexed as $0,1,\ldots, \T, \T+1, \ldots, \S+\T+1$. All vertices in $\AA$ start at level $0$. During execution, a vertex $a \in \AA$ may increase its level multiple times, potentially reaching the highest level, $\S+\T+1$. A vertex $a$ at level $\ell$ is denoted as $a^\ell$. A vertex $b\in\BB$ prefers $a_i^\ell$ over $a_j^{\ell'}$ if either:
(i) $\ell > \ell'$, or
(ii) $\ell = \ell'$ and $a_i \succ_b a_j$.

The algorithm first achieves a $\BB$-critical matching, where the lower quotas of vertices in $\BB$ are maximally satisfied. It does so by allowing each under-subscribed vertex $a\in\AA$ to propose only to \emph{lq}-vertices on the $\BB$ side at levels $0,\ldots,\T-1$, subject to additional constraints on the capacities of vertices in $\BB$ (see Table~\ref{tab:quota}). At level $\T$, each vertex $a\in\AA$ is allowed to propose to \emph{all} its neighbors. If a vertex $a\in\AA$ remains under-subscribed even after exhausting all vertices in its preference list at level $\T$, it raises its level to $\T+1$ and continues proposing to its neighbors until it becomes fully subscribed or exhausts its preference list at level $\T+1$. If an \emph{lq}-vertex $a$ remains \textit{deficient}, it raises its level above $\T+1$ and continues proposing to all its neighbors until it either becomes non-deficient (that is, matched to at least $q^-(a)$ many neighbors) or exhausts all vertices in its preference list at the highest level, $\S+\T+1$. 
It is shown by Nasre \etal~\cite{NNRS2024popular} that the resulting matching is a maximum-size popular matching among all critical matchings.

\begin{table}
\centering
\renewcommand{\arraystretch}{1.3} % Increases row height
\begin{tabular}{| p{2.8 cm} |p{1cm} | p{1.7 cm} |p{1cm} p{4cm}|}
\hline 
      \textbf{level of $a$}  & \textbf{$c(a)$} & \textbf{preference list of $a$ }&  & \textbf{$\qquad \ \ c(b)$}  \\
      \hline
      \hline 
      $0, \ldots, \T  - 1$ &  & \preflqa & $q^-(b)$ & \\ [2pt]
      \cline{1-1} \cline{3-5} 
      $\T$, $\T + 1$ & $q^+(a)$ & \prefa & $q^-(b)$ & if $\exists\ a_i\in M(b)$ such that $a_i$ is at level $< \T$ \\ [2pt]
      \cline{1-2}
      $\T+2, \ldots, \S+\T+1$ & $q^-(a)$ &  & $q^+(b)$ & otherwise \\[2pt]
\hline     
\hline
\end{tabular}
\vspace{0.1in}
\caption{Let $a \in \AA$ be the vertex proposing to $b \in \BB$. Entries in the table give the capacity and preference list of vertex $a$ and the capacity of the vertex $b$ used by the algorithm in~\cite{NNRS2024popular}. We denote the capacity of a vertex $v$ by $c(v)$, and $\prefa$ restricted to \emph{lq}-vertices by $\preflqa$.}
\label{tab:quota}
\end{table}


\subsection{Stable critical matching problem is NP-complete }\label{subsec:NPhard}
In this section, we consider the stable critical matching problem. Given an instance $G=(\AA\cup\BB,E)$ of the many-to-many setting with two-sided ties and lower quotas on both sides of the bipartition, the stable critical matching problem asks whether $G$ admits a matching that is stable as well as critical. We show that the stable critical matching problem is NP-complete even for the many-to-one setting. As mentioned earlier, Goko \etal~\cite{GokoMMY22} considered a similar problem called \HRTMSLQ. In the \HRTMSLQ\ problem, we are given an instance $G=(\RR\cup\HH,E)$ where (i) $q^-(r)=0$ and $q^+(r)=1$ for all $r\in\RR$, (ii) $q^+(h)>0$ and $0\le q^-(h)\le q^+(h) \le |\RR|$ for all $h\in\HH$, and (iii) the preference lists for all $v\in \RR\cup\HH$ are complete and may contain ties. Recall that the satisfaction ratio for a hospital $h\in\HH$ w.r.t. a given matching $M$ is defined as $S_M(h)=\min\{1,\frac{|M(h)|}{q^-(h)}\}$. Here, the assumption is that if $q^-(h)=0$, then $S_M(h)=1$ because the lower quota is automatically satisfied in this case. The \HRTMSLQ\ problem asks to maximize the total satisfaction ratio over all stable matchings in the given instance. That is, if $\mathcal{M}_s$ denotes the set of all stable matchings in the given instance, then \HRTMSLQ\ problem asks us to find a matching $M\in\mathcal{M}_s$ such that $\sum_{h\in\HH}S_M(h)$ is maximized.


 
Theorem 13 in~\cite{GokoMMY22} shows that \HRTMSLQ\ problem is NP-hard for one-to-one setting. % even if there is master preference list of hospitals and ties appear only in the preference lists of residents or only in the preference lists of hospitals. 
Computing a critical matching is polynomial-time solvable~\cite{Kavitha2021,NNRS2024popular}. So we can verify whether the given solution is critical or not. Also, since the stability of a matching is verifiable in linear time, it is easy to see that our problem is in NP. In the following claim, we prove that when $q^+(h)=1$ for all $h\in\HH$, then the optimal solution of \HRTMSLQ\ problem for the given instance is the same as the stable critical matching in that instance. This implies that the stable critical matching problem is NP-complete even in a very restricted case of our setting.

\begin{cl}\label{cl:nphardCS}
Let $G$ be an instance of \HRTMSLQ\ problem with $q^+(h)=1$ for all $h\in \HH$. Then, $M$ is a stable critical matching in $G$ if and only if $M$ is a stable matching with the maximum satisfaction ratio in $G$.
\end{cl}

\begin{proof}
    Let $M$ be a critical stable matching in $G$. Clearly, $M$ is a stable matching. Thus, we need to show that $M$ has the maximum satisfaction ratio in $G$. Recall that the preference lists of all agents are complete as $G$ is an instance of \HRTMSLQ\ problem. If $|\RR|\ge |\HH|$, then all hospitals are matched in $M$, and $M$ trivially has the maximum satisfaction ratio in $G$. So, let us assume that $|\RR|<|\HH|$. Clearly, some hospitals must be unmatched in $M$. Assume that the satisfaction ratio of the matching $M$ is $k$. We claim that no stable matching $M'\neq M$ has a satisfaction ratio greater than $k$. This is because if the satisfaction ratio of $M'$ is greater than $k$, then $M'$ must match more hospitals with $q^-(h)=1$ than the matching $M$.  This contradicts that $M$ is a critical matching. Thus, $M$ has the maximum satisfaction ratio in $G$.

    Now, to prove the other direction, let us assume that $M$ is a stable matching with the maximum satisfaction ratio in $G$. If there exists any stable matching $M'\neq M$ such that $M'$ matches more hospitals with $q^-(h)=1$ than the matching $M$. Then, $\sum_{h\in\HH}S_M'(h) > \sum_{h\in\HH}S_M(h)$. This contradicts that $M$ is a stable matching with the maximum satisfaction ratio. Therefore, the number of hospitals with $q^-(h)=1$ matched in $M$ is the maximum for any stable matching. Thus, $M$ is a critical stable matching.
\end{proof}