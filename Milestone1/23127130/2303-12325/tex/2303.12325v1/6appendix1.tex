\newpage
\section{Relaxed stability versus popularity}\label{append:RSvsPop}
Relaxed stability and popularity do not define the same set of matchings even when preferences are strict and  critical vertices are restricted to only one side of the bipartition. That is, neither one implies the other. We give simple examples (from~\cite{krishnaa2023envy}).% shown in Figure~\ref{fig:rsmNOTpop} to justify the above claim.

Consider the example shown in Figure~\ref{subfig:popNOTrsm}. Notice that the popular matching $M_1= \{(a_1,b_2),(a_2,b_1)\}$ is not relaxed stable because the blocking pair $(a_1,b_1)$ is not justified (there are no critical vertices). Observe that in the absence of critical vertices, relaxed stability is the same as stability.

Now, consider the example shown in Figure~\ref{subfig:rsmNOTpop}. Notice that the matching  $M_2=\{(a_1,b_2)\}$ is relaxed stable as the only blocking pair $(a_1,b_1)$ is justified by $a_1$. But $M_3=\{(a_1,b_1)\}$ is more popular than $M_2$. Thus a relaxed stable matching $M_2$ is not popular. 


\begin{figure}[!h]
\centering
\begin{subfigure}{.5\textwidth}
\centering
\begin{tikzpicture}[thick,  ncnode/.style={draw=black,,circle,fill=black,scale=0.5},  cnode/.style={draw=red,circle,fill=red,scale=0.5},
  every fit/.style={ellipse,draw,inner sep=-2pt,text width=0.5cm},  -,shorten >= 3pt,shorten <= 3pt, scale=0.7]

% the vertices of A U B
  \node[ncnode] (a1) at (-1,3) {};
  \node[ncnode] (a2) at (-1,1.5) {};
  \node[ncnode] (b1) at (2.5,3) {};
  \node[ncnode] (b2) at (2.5,1.5) {};
    \node at (-1.8,3) {$a_1$};
    \node at (-1.8,1.5) {$a_2$};
    \node at (3.2,3) {$b_1$};
    \node at (3.2,1.5) {$b_2$};


% the set A
\node [fit=(a1) (a2),label=above:$\AA$] {};
% the set B
\node [fit=(b1) (b2),label=above:$\BB$] {};

% the edges
\draw(a1) -- (b1) node[pos=0.15,draw=white, fill=white,inner sep=0.1pt] {\textcolor{black}{\tiny 1}} node[pos=0.8,draw=white, fill=white,inner sep=0.1pt] {\textcolor{black}{\tiny 1}};
\draw(a1) -- (b2) node[pos=0.2,draw=white,fill=white,inner sep=0.1pt]{\textcolor{black}{\tiny 2}} node[pos=0.85,draw=white, fill=white,inner sep=0.1pt] {\textcolor{black}{\tiny 1}};
\draw(a2) -- (b1) node[pos=0.2,draw=white,fill=white,inner sep=0.1pt]{\textcolor{black}{\tiny 1}} node[pos=0.85,draw=white, fill=white,inner sep=0.1pt] {\textcolor{black}{\tiny 2}};
\end{tikzpicture}

\caption{}
\label{subfig:popNOTrsm}
\end{subfigure}%
\begin{subfigure}{.5\textwidth}
\centering
\begin{tikzpicture}[thick,  ncnode/.style={draw=black,,circle,fill=black,scale=0.5},  cnode/.style={draw=red,circle,fill=red,scale=0.5},
  every fit/.style={ellipse,draw,inner sep=-2pt,text width=0.5cm},  -,shorten >= 3pt,shorten <= 3pt, scale=0.7]

% the vertices of A U B
  \node[ncnode] (a1) at (-1,2.25) {};
  \node[cnode] (b1) at (2.5,3) {};
  \node[cnode] (b2) at (2.5,1.5) {};
    \node at (-1.8,2.25) {$a_1$};
    \node at (3.2,3) {$b_1$};
    \node at (3.2,1.5) {$b_2$};


% the set A
% \node [fit=(a1),label=above:$\AA$] {};
\node at (-1,3) {$\AA$};
% the set B
\node [fit=(b1) (b2),label=above:$\BB$] {};

% the edges
\draw(a1) -- (b1) node[pos=0.15,draw=white, fill=white,inner sep=0.1pt] {\textcolor{black}{\tiny 1}} node[pos=0.8,draw=white, fill=white,inner sep=0.1pt] {\textcolor{black}{\tiny 1}};
\draw(a1) -- (b2) node[pos=0.2,draw=white,fill=white,inner sep=0.1pt]{\textcolor{black}{\tiny 2}} node[pos=0.85,draw=white, fill=white,inner sep=0.1pt] {\textcolor{black}{\tiny 1}};
\end{tikzpicture}

\caption{}
\label{subfig:rsmNOTpop}
\end{subfigure}%
\caption{\label{fig:rsmNOTpop}%
The red vertices are critical and black vertices are non-critical. The numbers on the edges denote the ranking of the other endpoint in the preference list of that vertex.}
\end{figure}
