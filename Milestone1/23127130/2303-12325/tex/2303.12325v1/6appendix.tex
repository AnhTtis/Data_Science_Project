\newpage
\section{Relaxed stability versus popularity}\label{append:RSvsPop}
Relaxed stability and popularity do not define the same set of matchings even when preferences are strict and  critical vertices are restricted to only one side of the bipartition. That is, neither one implies the other. We give simple examples (from~\cite{krishnaa2023envy}).% shown in Figure~\ref{fig:rsmNOTpop} to justify the above claim.

Consider the example shown in Figure~\ref{subfig:popNOTrsm}. Notice that the popular matching $M_1= \{(a_1,b_2),(a_2,b_1)\}$ is not relaxed stable because the blocking pair $(a_1,b_1)$ is not justified (there are no critical vertices). Observe that in the absence of critical vertices, relaxed stability is the same as stability.

Now, consider the example shown in Figure~\ref{subfig:rsmNOTpop}. Notice that the matching  $M_2=\{(a_1,b_2)\}$ is relaxed stable as the only blocking pair $(a_1,b_1)$ is justified by $a_1$. But $M_3=\{(a_1,b_1)\}$ is more popular than $M_2$. Thus a relaxed stable matching $M_2$ is not popular. 


\begin{figure}[!h]
\centering
\begin{subfigure}{.5\textwidth}
\centering
\begin{tikzpicture}[thick,  ncnode/.style={draw=black,,circle,fill=black,scale=0.5},  cnode/.style={draw=red,circle,fill=red,scale=0.5},
  every fit/.style={ellipse,draw,inner sep=-2pt,text width=0.5cm},  -,shorten >= 3pt,shorten <= 3pt, scale=0.7]

% the vertices of A U B
  \node[ncnode] (a1) at (-1,3) {};
  \node[ncnode] (a2) at (-1,1.5) {};
  \node[ncnode] (b1) at (2.5,3) {};
  \node[ncnode] (b2) at (2.5,1.5) {};
    \node at (-1.8,3) {$a_1$};
    \node at (-1.8,1.5) {$a_2$};
    \node at (3.2,3) {$b_1$};
    \node at (3.2,1.5) {$b_2$};


% the set A
\node [fit=(a1) (a2),label=above:$\AA$] {};
% the set B
\node [fit=(b1) (b2),label=above:$\BB$] {};

% the edges
\draw(a1) -- (b1) node[pos=0.15,draw=white, fill=white,inner sep=0.1pt] {\textcolor{black}{\tiny 1}} node[pos=0.8,draw=white, fill=white,inner sep=0.1pt] {\textcolor{black}{\tiny 1}};
\draw(a1) -- (b2) node[pos=0.2,draw=white,fill=white,inner sep=0.1pt]{\textcolor{black}{\tiny 2}} node[pos=0.85,draw=white, fill=white,inner sep=0.1pt] {\textcolor{black}{\tiny 1}};
\draw(a2) -- (b1) node[pos=0.2,draw=white,fill=white,inner sep=0.1pt]{\textcolor{black}{\tiny 1}} node[pos=0.85,draw=white, fill=white,inner sep=0.1pt] {\textcolor{black}{\tiny 2}};
\end{tikzpicture}

\caption{}
\label{subfig:popNOTrsm}
\end{subfigure}%
\begin{subfigure}{.5\textwidth}
\centering
\begin{tikzpicture}[thick,  ncnode/.style={draw=black,,circle,fill=black,scale=0.5},  cnode/.style={draw=red,circle,fill=red,scale=0.5},
  every fit/.style={ellipse,draw,inner sep=-2pt,text width=0.5cm},  -,shorten >= 3pt,shorten <= 3pt, scale=0.7]

% the vertices of A U B
  \node[ncnode] (a1) at (-1,2.25) {};
  \node[cnode] (b1) at (2.5,3) {};
  \node[cnode] (b2) at (2.5,1.5) {};
    \node at (-1.8,2.25) {$a_1$};
    \node at (3.2,3) {$b_1$};
    \node at (3.2,1.5) {$b_2$};


% the set A
% \node [fit=(a1),label=above:$\AA$] {};
\node at (-1,3) {$\AA$};
% the set B
\node [fit=(b1) (b2),label=above:$\BB$] {};

% the edges
\draw(a1) -- (b1) node[pos=0.15,draw=white, fill=white,inner sep=0.1pt] {\textcolor{black}{\tiny 1}} node[pos=0.8,draw=white, fill=white,inner sep=0.1pt] {\textcolor{black}{\tiny 1}};
\draw(a1) -- (b2) node[pos=0.2,draw=white,fill=white,inner sep=0.1pt]{\textcolor{black}{\tiny 2}} node[pos=0.85,draw=white, fill=white,inner sep=0.1pt] {\textcolor{black}{\tiny 1}};
\end{tikzpicture}

\caption{}
\label{subfig:rsmNOTpop}
\end{subfigure}%
\caption{\label{fig:rsmNOTpop}%
The red vertices are critical and black vertices are non-critical. The numbers on the edges denote the ranking of the other endpoint in the preference list of that vertex.}
\end{figure}


\section{Omitted proofs from Section~\ref{sec:correct}} \label{append:proof}

\begin{appendix-claim}{\ref{lem:noedge11}}
Let  $(a,b)\in E\setminus M$ and $b$ be matched in $M$ to $\Tilde{a}$ at level $y$, that is, $M(b)=\Tilde{a}^y$. If the level $x$ of $a$ is at least 2 then $y\ge x-1$.
\end{appendix-claim}

\begin{proof}
Suppose for contradiction that there exists $\Tilde{a}\in \AA$ such that $(\Tilde{a}^y,b)\in M$ for $y<x-1$. The fact that $(a,b)\in E$ and $a$ achieves the level $x$ implies that $a$ remains unmatched after $a^{x-1}$ exhausted its preference list $\prefa$, $\prefsa$ or $\prefslqa$ as appropriate.  Note that if $b$ receives a proposal from a vertex $\Tilde{a}\in\AA$ at levels below $x-1$ then $b$ is also available to receive proposals from vertices in $\AA$ at levels $\ge x-1$. This is because when a vertex in $\AA$ transitions to a higher level, it proposes to possibly a superset of vertices that it proposes to in the lower level (recall that $\prefa$ and $\prefsa$ is a superset of $\prefslqa$). Furthermore, if a vertex in $\BB$ receives a proposal from some $a'\in \AA$ at a level $z$ then it is available to receive a proposal from all its neighbours proposing at level $z$.

Since $b$ is matched to a vertex at level $y<x-1$, it must be the case that $b$ has received a proposal from $a^{x-1}$ and it accepted this proposal by rejecting $\Tilde{a}^y$ because $y<x-1$. Recall that a vertex $b\in\BB$ always prefers $a$ over $\Tilde{a}$ if $a$ is at a higher level than that of $\Tilde{a}$. Thus, $(a,b)\in M$ and we get a contradiction to the fact that $(a,b)\notin M$.
\qed\end{proof}


\noindent\textbf{Justification for Property~\ref{property:GM}}

Note that only critical vertices in $\AA$ attain levels above $\T$. This implies that the partition set $\AA_{\T+1},\ldots,\AA_{\S+\T}$ contain only critical $a\in\AA$. Thus, we have Property~\ref{property:GM}(\ref{obs:partitionPorp2}).
Since each $a\in\AA$ at a level at most $\T-1$ does not propose to any non-critical vertex, the matched partner of each $a^x$ for $x\le \T-1$ is a critical vertex. Also, all the unmatched non-critical vertices are only in $\BB_{\T}$. This implies that the partition set $\BB_0,\ldots,\BB_{\T-1}$ contain only critical $b\in\BB$.  Thus, we have Property~\ref{property:GM}(\ref{obs:partitionPorp1}). 
 If a vertex $a$ remains unmatched in $M$ then by the design of our algorithm it must have exhausted its preference list at level $\S+\T$ (if it is a critical vertex) or at level $\T$, more specifically $\T^*$, (if it is a non-critical vertex)  and got rejected by each of its neighbours. Recall that each $b$ prefers a higher level $a$ over any lower level $a'$ irrespective of the ranks of $a$ and $a'$ in $\prefb$. Thus, Property~\ref{property:GM}(\ref{obs:partitionPorp4}) and Property~\ref{property:GM}(\ref{obs:partitionPorp3}) hold in $G_M$. Observe that if any vertex $b\in\BB$ receives a proposal then it cannot remain unmatched in $M$. This implies, if a critical $b$ is unmatched then none of its neighbours has proposed it at level 0, which further implies that they have not exhausted $\prefslqa$ at level 0. By construction, $b$ is in $\BB_0$. Hence, we have Property~\ref{property:GM}(\ref{obs:partitionPorp5}). Similarly, if $b$ is non-critical and unmatched then none of its neighbours can go to level $\T+1$ or above. By construction, $b$ is in $\BB_{\T}$. Hence, we have Property~\ref{property:GM}(\ref{obs:partitionPorp6}).\qed




Next, we prove the following claim which states that no matching matches more critical vertices from a particular side $\AA$ or $\BB$ than a critical matching. In other words, the number of critical vertices matched from $\AA$-side or $\BB$-side is optimum in any critical matching. This also implies that the number of critical vertices matched from $\AA$-side or $\BB$-side is invariant across all critical matchings. That is, if a critical matching $M_1$ matches $p$ number of critical vertices from $\AA$ and $q$ number of critical vertices from $\BB$ then another critical matching, say $M_2$, also matches $p$ many critical vertices from $\AA$ and $q$ many critical vertices from $\BB$. This claim is similar to the one in~\cite{nasre2022popular}.

\begin{cl}\label{lem:noLessDef}
Let $N$ be any critical matching and $M$ be any matching in $G$. Then the number of critical vertices matched in $N$ from $\AA$ is at least the number of critical vertices matched in $M$ from $\AA$. Similarly, the number of critical vertices matched in $N$ from $\BB$ is at least the number of critical vertices matched in $M$ from $\BB$.
\end{cl}
\begin{proof}
Here we will prove the first statement, that is, we show that the number of critical vertices matched in a critical matching $N$ from $\AA$ is at least the number of critical vertices matched in any matching $M$ from $\AA$. The proof for the second statement is symmetric. 


Consider the symmetric difference $N\oplus M$.  Suppose for contradiction that $N$ matches strictly less number of critical vertices from $\AA$ than that of $M$.  This implies there must exist a maximal alternating path $\rho=\langle u,u'\ldots,v\rangle$ starting at an unmatched critical vertex $u\in\AA$ in $N$ such that using $\rho$ we obtain a matching $N'=N\oplus \rho$ which matches strictly more number of critical vertices from $\AA$ than $N$ on $\rho$.  Note that $\rho$ is a maximal $M$-$N$ alternating path starting with an $M$ edge $(u,u')$ such that critical $u$ is unmatched in $N$ but matched in $M$. We consider the two cases below depending on the parity of the length of $\rho$.

If $\rho$ is of odd length then it is an augmenting path with respect to $N$. That is, critical $u$ becomes matched in $N'$ from unmatched in $N$  whereas the matched/unmatched status of all other vertices on $\rho$, except $v$, remains the same as in $N$. Note that $\rho$ is an augmenting path for $N$ and hence $v$ gets matched in $N'$ from unmatched in $N$.  Thus, $N'$ matches more number of critical vertices from $\AA$ than that of $N$. Also, note that all the vertices from $\BB$, except $v$, remain matched/unmatched as they were in $N$. That is, the number of matched critical vertices from $\BB$ either increases by 1 (when $v$ is critical) or remains the same. Thus $N'$ matches more number of critical vertices overall than that of a critical matching $N$ -- a contradiction. 

If $\rho$ is of even length then due to the maximality of $\rho$, the other endpoint $v\in\AA$ is unmatched in $M$ and hence, $\rho$ ends with an $N$-edge. Note that $v\in\AA$ is unmatched in $M$ and $u$ is critical but matched in $M$. If $v$ is a critical vertex then the number of critical vertices matched in $N'$ and $N$ remain the same. This contradicts the selection of our path $\rho$. Recall $\rho$ is an alternating path such that $N'=N\oplus \rho$ matches strictly more number of critical vertices from $\AA$. Thus, $v$ is not a critical vertex. But then the number of critical vertices matched in $N'$ is strictly less than that of $N$. This contradicts the selection of our path $\rho$.

 Thus, we conclude that such a path $\rho$ does not exist and hence the number of critical vertices matched in $N$ from $\AA$ is at least the number of critical vertices matched in $M$ from $\AA$.\qed
 \end{proof}


\begin{appendix-lemma}{\ref{lem:critical11}}
The output matching $M$ is critical for $G$.
\end{appendix-lemma}
\begin{proof}

We prove the criticality of $M$ by using the level structure of the graph $G_{M}$. The idea is to show that there is no alternating path $\rho$ in $G_M$ w.r.t. $M$ such that $M\oplus \rho$ results in more critical vertices matched than in $M$. We prove the criticality in two parts. First, we prove ($\AA$-part) where we show that $M$ matches the maximum possible critical vertices from the set $\AA\cap\CC$ and then we prove ($\BB$-part) where we show that $M$ matches the maximum possible critical vertices from the set $\BB\cap\CC$. Thus, by using Claim~\ref{lem:noLessDef} above, we conclude that $M$ is critical. Let $N$ be any critical matching in $G$.


\vspace{0.1in}


\noindent \textbf{Proof of ($\AA$-part):} Suppose for contradiction that $M$ does not match the maximum possible critical vertices from the set $\AA\cap\CC$. This implies that there exists an alternating path $\rho$ in $M\oplus N$ such that $N$ matches more critical vertices from $\AA$ on $\rho$ than in $M$. Let $\rho=\langle u_0, v_1, u_1, v_2, u_2, \ldots, v_k, u_k,\ldots \rangle$ where $(v_i,u_i)\in M$ and the other edges of $\rho$ are in the matching $N$. Furthermore, assume that the first vertex $u_0$ represents a vertex $a\in\AA$ such that critical $a$ is matched in $N$ but unmatchedy in $M$. Since critical $a$ is unmatched in $M$, by Property~\ref{property:GM}(\ref{obs:partitionPorp4}), $a\in\AA_{\S+\T}$. Thus, $\rho$ starts at level $\S+\T$ in $G_M$, that is, $u_0\in\AA_{\S+\T}$. Since $u_0=a$ is critical and unmatched in $M$, by Property~\ref{property:GM}(\ref{obs:partitionPorp4}), $v_1\in\AA_{\S+\T}$ and $u_1=M(v_1)$ is in $\AA_{\S+\T}$. The other end of $\rho$ can be in $\AA$ or in $\BB$. We consider both these cases below.

\noindent\textbf{The path $\rho$ ends at a vertex in $\AA$:} Suppose that the path ends at a vertex in $\AA_x$ for $x>\T$. By Property~\ref{property:GM}(\ref{obs:partitionPorp2}), all the vertices in $\AA_x$ for $x>\T$ are critical.  Thus, if $\rho$ ends at a vertex $u_i$ such that $u_i\in \AA_{x}$ for $x>\T$ then $N\oplus \rho$ matches the same number of critical vertices from $\AA$. This contradicts the choice of our path $\rho$ (recall that we selected $\rho$ such that $N$ matches more critical vertices from $\AA$ on $\rho$ than in $M$). This implies that the other endpoint of $\rho$ must be in $\AA_x$ for $x\le\T$. Lemma~\ref{pr:noedge11} implies that if $u_i\in\AA_x$ and $u_{i+1}\in\AA_y$ then $|y-x|\le 1$ for all indices $i$ on $\rho$. Hence, $\rho$ must contain at least one vertex from each $\AA_x$ for $\T+1\le x\le \S+\T$. We observe that $\rho$ contains at least two vertices $u_0$ and $u_1$ from $\AA_{\S+\T}$, and at least one vertex from each $\AA_x$ for $\T+1\le x\le \S+\T$.  Thus, $\rho$ contains at least $\S+1$ many critical vertices from $\bigcup_{x=\T+1}^{\S+\T}\AA_x$. By Property~\ref{property:GM}(\ref{obs:partitionPorp2}), the total number of vertices accommodated in these levels  is at most $\S$. Thus, we get a contradiction.

\noindent\textbf{The path $\rho$ ends at some vertex in $\BB$:} Note that $\rho$ has even length and hence the last vertex, say $v_{k+1}$, on $\rho$ remains unmatched in $M$. By construction, an unmatched vertex $b\in\BB$ are in $\BB_{\T}\cup\BB_{0}$. Thus, by Property~\ref{property:GM}(\ref{obs:partitionPorp5}) and Property~\ref{property:GM}(\ref{obs:partitionPorp6}), $u_k\in \AA_{x}$ for $x\le \T$. Since $\rho$ contains some vertex in $\AA_x$ for $x\le \T$, by using the same argument as in the previous case, we show that $\rho$ contains at least $\S+1$ many critical vertices from $\bigcup_{x=\T+1}^{\S+\T}\BB_x$ to get a contradiction. Hence, we conclude that such a path $\rho$ cannot exist. 

\vspace{0.1in}

\noindent \textbf{Proof of ($\BB$-part):} Suppose for contradiction that $M$ does not match the maximum possible critical vertices from the set $\BB\cap\CC$. This implies that there exists an alternating path $\rho$ in $M\oplus N$ such that $N$ matches more critical vertices from $\BB$ on $\rho$ than in $M$. Let $\rho=\langle v_0, u_1, v_1, u_2, v_2, \ldots, u_k, v_k,\ldots \rangle$ where $(u_i,v_i)\in M$ and the other edges of $\rho$ are in the matching $N$. Furthermore, assume that the first vertex $v_0$ represents a vertex $b\in\BB$ such that critical $b$ is matched in $N$ but unmatched in $M$. Since $b$ is critical and unmatched in $M$, by Property~\ref{property:GM}(\ref{obs:partitionPorp5}), $b\in\BB_0$. Thus, $\rho$ starts at level 0 in $G_M$, that is, $v_0\in\BB_0$. Since $v_0=b$ is critical and unmatched in $M$, by Property~\ref{property:GM}(\ref{obs:partitionPorp5}), $u_1\in\AA_0$ and $v_1=M(u_1)$ is in $\BB_0$. The other end of $\rho$ can be in $\BB$ or in $\AA$. We consider both these cases below.


\noindent\textbf{The path $\rho$ ends at a vertex in $\BB$:} Suppose that the path ends at a vertex in $\BB_x$ for $x<\T$. By Property~\ref{property:GM}(\ref{obs:partitionPorp1}), all the vertices in $\BB_x$ for $x<\T$ are critical.  Thus, if $\rho$ ends at a vertex $v_i$ such that $v_i\in \BB_{x}$ for $x<\T$ then $N\oplus \rho$ matches the same number of critical vertices from $\BB$. This contradicts the choice of our path $\rho$ (recall that we selected $\rho$ such that $N$ matches more critical vertices from $\BB$ on $\rho$ than in $M$). This implies that the other endpoint of $\rho$ must be in $\BB_x$ for $x\ge\T$. 
Lemma~\ref{pr:noedge11} implies that if $v_i\in\BB_x$ and $v_{i+1}\in\BB_y$ then $y-x\le 1$ for all indices $i$ on $\rho$. Hence, $\rho$ must contain at least one vertex from each $\BB_x$ for $1\le x\le \T-1$. We observe that $\rho$ contains at least two vertices $v_0$ and $v_1$ from $\BB_0$, and at least one vertex from each $\BB_x$ for $1\le x\le \T-1$.  Thus, $\rho$ contains at least $\T+1$ many critical vertices from $\bigcup_{x=0}^{\T-1}\BB_x$. By Property~\ref{property:GM}(\ref{obs:partitionPorp1}), the total number of vertices accommodated in these levels  is at most $\T$. Thus, we get a contradiction.

\noindent\textbf{The path $\rho$ ends at some vertex in $\AA$:} Note that $\rho$ has even length and hence the last vertex, say $u_{k+1}$, on $\rho$ remains unmatched in $M$. By construction, an unmatched vertex $a\in\AA$ are in $\AA_{\T}\cup\AA_{\S+\T}$. Thus, by Property~\ref{property:GM}(\ref{obs:partitionPorp3}) and Property~\ref{property:GM}(\ref{obs:partitionPorp4}), $v_k\in \BB_{x}$ for $x\ge \T$. Since $\rho$ contains some vertex in $\BB_x$ for $x\ge \T$, by using the same argument as in the previous case, we show that $\rho$ contains at least $\T+1$ many critical vertices from $\bigcup_{x=0}^{\T-1}\BB_x$ to get a contradiction. Hence, we conclude that such a path $\rho$ cannot exist. 

\qed
\end{proof}