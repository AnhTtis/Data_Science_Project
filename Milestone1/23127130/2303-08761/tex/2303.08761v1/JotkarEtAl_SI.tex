% ****** Start of file apssamp.tex ******
%
%   This file is part of the APS files in the REVTeX 4.1 distribution.
%   Version 4.1r of REVTeX, August 2010
%
%   Copyright (c) 2009, 2010 The American Physical Society.
%
%   See the REVTeX 4 README file for restrictions and more information.
%
% TeX'ing this file requires that you have AMS-LaTeX 2.0 installed
% as well as the rest of the prerequisites for REVTeX 4.1
%
% See the REVTeX 4 README file
% It also requires running BibTeX. The commands are as follows:
%
%  1)  latex apssamp.tex
%  2)  bibtex apssamp
%  3)  latex apssamp.tex
%  4)  latex apssamp.tex
%
\documentclass[onecolumn, aps, amsmath,notitlepage,amssymb,showkeys,a4paper]{revtex4-1}%{revtex4-1}
%\documentclass[onecolumn, amsmath,amssymb,preprint]{revtex4}%{revtex4-1}

\usepackage{amssymb}
\usepackage{amsbsy}
\usepackage{amsmath}
\usepackage{bbm}
\usepackage{xcolor}
\usepackage{hyperref}

%\usepackage{tikz}
%\usetikzlibrary{automata,topaths}

%\usepackage{CJK}

%\usepackage{psfrag}
%\usepackage{pstricks}
\usepackage{graphicx}% Include figure files
\usepackage{dcolumn}% Align table columns on decimal point
\usepackage{bm}% bold math
%\usepackage{subfigmat}
%\usepackage{gensymb}

\begin{document}

%\preprint{APS/123-QED}

\title{Supplemental Material: Diffusiophoresis and medium structure control macroscopic particle transport in porous media}


\author{Mamta Jotkar}
\email{mamta.jotkar@upm.es, mamta.jotkar@idaea.csic.es }
\affiliation{Universidad Polit\'ecnica de Madrid, Spain, \\ Institute of Environmental Assessment and Water Research, Spanish National Research Council, Barcelona, Spain} 
\author{Pietro de Anna}
\email{pietro.deana@unil.ch}
\affiliation{Institute of Earth Sciences, University of Lausanne, Switzerland} 
\author{Marco Dentz}
\email{marco.dentz@idaea.csic.es }
\affiliation{Institute of Environmental Assessment and Water Research, Spanish National Research Council, Barcelona, Spain} 
\author{Luis Cueto-Felgueroso}
\email{luis.cueto@upm.es}
\affiliation{Universidad Polit\'ecnica de Madrid, Spain}%, ETSI Caminos, Canales y Puertos, Spain}
\date{\today}% It is always \today, today,

\begin{abstract}

This supplemental material gives details on the setup of the detailed
numerical simulations of flow and solute and particle transport, and
the derivation of the one-dimensional analytical model for the
quantification of the fraction of particles trapped in the DEPs. 

\end{abstract}


\keywords{Diffusiophoresis, anomalous dispersion, porous media}
\date{\today}

\maketitle

% %%%%%%%%%%%%%%%%%%%%%%%%%%%%%%%%%%%%%%%%5
% \begin{figure}[h]
% \begin{center}
% \includegraphics[width=.6\textwidth]{./Figure/setup}
% \caption{Hyper-uniform porous structure characterised by dead-end
%   pores (DEPs) and transmitting pores (TPs). Computational domain with white spaces
%   indicating solid grains (bottom). \label{fig1}}
% \end{center}
% \end{figure}
% %%%%%%%%%%%%%%%%%%%%%%%%%%%%%%%%%%%%%%%%5


\section{Numerical simulations}
The computational domain is shown in figure 1 in the main text. The mean flow is
from left to right. This medium is statistically homogeneous such that
the distribution of the pore-size is narrow with a strong peak close
to mean pore-size $\lambda=30 \mu \text{m}$. It exhibits complex pore
network interspersed among disordered solid grains with a porosity
$\phi= 0.39$. The domain is initially saturated with solute at a lower concentration ($s_i=$0.1mM) and particles. A sharp
front of solute at a higher concentration ($s_H=$10mM) is then injected such that the
ratio $\chi=s_i/s_H=0.01$. The governing
equations (Eq.(2) in the article) are subjected to no slip and no
penetration flow at the solid surfaces, uniform flow with an average
fluid speed $U$ at the left inlet and constant pressure at the right
outlet. For the solute concentration, we use constant flux at the
inlet and no flux boundary condition everywhere else whereas, for the
particle concentration, we impose a conservative form of no flux
boundary condition, where the sum of diffusive and advective fluxes is
zero. As an initial condition, we assume initially no flow $\textbf{u}
= 0, p = 0$ and impose solute concentration $s_i = 0.1$mM and particle
concentration $c_i=0.1$mM. The numerical model has been validated for
a simpler micro-channel geometry \cite{JotkarCueto-Felgueroso_2021} as
well as for the hyper-uniform porous medium considered here by
approximating the case without DP with
Ref. \cite{BordoloiEtAl_2022}. We use COMSOL
Multiphysics\textregistered $\;$based on finite element for performing
the pore-scale simulations \cite{comsol}. For the physical parameters
we use the following values (closest to realistic ones): $\mu=10^{-3}
Pa.s$, $\rho=10^3 kg/m^3$, $D_s = 7 \times 10^{3}\mu \text{m}^2/s$,
$D_p = 7  \mu \text{m}^2/s$ and $U=175 \mu \text{m}/s$. This yields
a characteristic advection time of $\tau_v = 0.17$ s, the salt
diffusion time $\tau_{D_s} = 0.12$ s, the particle diffusion time
$\tau_{D_p} = 128$ s, and the   P\'eclet numbers $Pe_s = U \lambda /
D_s = 0.75$ based on the solute and $Pe_p = U \lambda / D_p = 750$
based on the particles. Particle concentrations at the outlet for different mobilities
$\Gamma_p$ are shown in figure \ref{fig2}. 

%%%%%%%%%%%%%%%%%%%%%%%%%%%%%%%%%%%%%%%%5
\begin{figure}[h]
\begin{center}
\includegraphics[width=0.7\textwidth]{2_SI_full_btc}
\caption{Numerical simulations. Arrival time distribution at the outlet for different diffusiophoretic mobilities
  $\Gamma_p/\lambda U$ ranging from -1.6, -1, -0.5, -0.01 (shades of
  blue, trapping) to 0 (black, no DP) to 0.02, 0.4, 0.8 (shades of red, extraction). Solid curves correspond to
  numerical simulations while dotted curves correspond to the
  one-dimensional travel-time model obtained using
  different fraction $\alpha$ of particles initially available within
  the DEPs. \label{fig2}}
\end{center}
\end{figure}
%%%%%%%%%%%%%%%%%%%%%%%%%%%%%%%%%%%%%%%%5

%%%%%%%%%%%%%%%%%%%%%%%%%%%%%%%%%%%%%%%%5
\section{Analytical model }
%%%%%%%%%%%%%%%%%%%%%%%%%%%%%%%%%%%%%%%%5

%%%%%%%%%%%%%%%%%%%%%%%%%%%%%%%%%%%%%%%%5
\begin{figure}[h]
\begin{center}
\includegraphics[width=0.7\textwidth]{1d_model_2}
\caption{Illustration of the one-dimensional analytical model. A single DEP of length $\ell_p$ is  connected to a (vertical) TP with a
  width of $\lambda$. The solute concentration within the DEP is
  illustrated by the red line, the blue line denotes the profile of $u_{dp}$. At short times, 
the diffusiophoretic drift is strongly localized at the interface between TP and DEP.   \label{1d_model}}
\end{center}
\end{figure}
%%%%%%%%%%%%%%%%%%%%%%%%%%%%%%%%%%%%%%%%5

We construct a one-dimensional model to quantify the dependence of the initial fraction of
particles $\alpha$ within the DEPs on the DP mobility $\Gamma_p$ (see
figure \ref{1d_model}). To this end, we assume that particle transport in
the DEP at short times is dominated by the diffusiophoretic drift such that
%
\begin{align}
\label{eq:adv}
\frac{\partial c}{\partial t} + \frac{\partial}{\partial x} u_{dp} c = 0. 
\end{align}
%
The total mass of particles in the DEP is given by
%
\begin{align}
\label{mint}
m_{dp} = w \int\limits_0^{\ell_p} dx c,
\end{align}
%
where $w$ is the pore-width. Since the only flux of particles
  toward or from the DEP is across  the DEP-TP junction, the temporal
  variability of the mass of particles in the DEP is equal to the mass flux at $x =
  0$ and controlled by DP. Spatial integration of
  Eq.~\eqref{eq:adv} according to Eq.~\eqref{mint} gives
%
\begin{equation}
\label{eq:mdp}
\frac{\partial m_{dp}}{\partial t} = w u_{dp}(x=0,t) c(x = 0,t),
\end{equation} 
%
where we used that there is no flux across the boundary at $x = \ell_p$. Thus, the added or extracted mass is given by 
%
\begin{align}
\label{mdptime}
m_{dp} = m_i + w \int\limits_0^\infty dt u_{dp}(x = 0,t) c(x = 0,t). 
\end{align}
%
In the following, we first determine the diffusiophoretic drift, then we deal with the cases of extraction ($\Gamma_p > 0$) and 
addition of particles  ($\Gamma_p < 0$) separately. 

% Furthermore, we assume that the particle concentration at $x = 0$ is
% constant and equal to $c_i$, that is, we do not account for depletion of
% particles from the transmitting pore, which is an issue for strong
% diffusiophoretic drift into the dead-end pore.

%%%%%%%%%%%%%%%%%%%%%%%%%%%%%%%%%%
\subsection{Diffusiophoretic drift}
%%%%%%%%%%%%%%%%%%%%%%%%%%%%%%%%%%
We focus here on estimating the drift $u_{dp}$. The P\'eclet number for salt is so low that we can
assume that diffusion dominates in the DEP. Thus, to obtain the salt
concentration $s$, we solve the diffusion equation
%
\begin{align}
\frac{\partial s}{\partial t} - D_s \frac{\partial^2 s}{\partial x^2} = 0. 
\end{align}
%
We consider the boundary conditions $s = s_H$ at $x = 0$ and $\partial s /
\partial x = 0$ at $x = L$. The initial condition is $s(x, t = 0) = s_i$. In
Laplace space we obtain the exact solution 
%
\begin{align}
s^\ast(x,\sigma) = \frac{s_i}{\sigma} + \frac{(s_H - s_i)}{\sigma}
\frac{\cosh[(1 - x/\ell_p) \sqrt{\sigma \tau_{D_s}}]}{\cosh(\sqrt{\sigma \tau_{D_s}})},
\end{align}
%
where $\tau_{D_s} = \ell_p^2/D_s$ and $\sigma$ is the Laplace variable. The
Laplace transform is defined in~\cite{AS1972}.

The Laplace transform of the diffusiophoretic velocity at $x = 0$ is then given by
%
\begin{align}
u_{dp}^\ast(x = 0,\sigma) = -\Gamma_p (1 - \chi) \frac{\tanh(\sqrt{\sigma
    \tau_{D_s}})}{\sqrt{\sigma D_s}},  
\end{align}
%
where we defined $\chi = s_i / s_H$. The integral of the drift from $t = 0$ to $\infty$ is given by 
%
\begin{align}
\label{intudpexact}
\int\limits_0^\infty dt u_{dp}(x = 0,t) =  u_{dp}^\ast(x = 0,\sigma = 0) = - \frac{\Gamma_p (1 - \chi) \ell_p}{D_s}. 
\end{align}
%
 This expression is used directly to estimate the total mass of trapped
  particles for the extraction case, as argued below. For for the trapping
  case, however, the full time dependence of $u_{dp}(,0,t)$ is required as can
  be seen from Eq.~\eqref{mdptime}.
%
Thus, we approximate the salt concentration
profile in the DEP by the
solution for a semi-infinite domain,
%
\begin{equation}
s^a(x,t)=s_i + (s_H-s_i)\text{erfc}(x/\sqrt{4D_s t}) \label{s_ana},
\end{equation}
%
where the superscript $a$ denotes approximation. 
Using this expression, the diffusiophoretic drift is given by
%
\begin{equation}
\label{udp_approx}
u^a_{dp}(x,t) =  - \Gamma_p (s_H-s_i) \frac{\exp{(-x^2/4D_s t)}}{s(x,t)\sqrt{\pi D_s t}}.
\end{equation}
%
 The drift at $x = 0$ then is given by 
%
\begin{equation}
u^a_{dp}(x = 0,t) =  - \frac{\Gamma_p (1 - \chi) }{\sqrt{\pi D_s t}},
\end{equation}
%
where we used that $s(x=0,t) = s_H$. For times larger than $\tau_{D_s} =
\ell_p^2/D_s$, the salt gradient decays exponentially fast with time. Thus, the
time integral over the drift can be written as
%
\begin{align}
\int\limits_0^\infty dt u_{dp}(x = 0,t) = \int\limits_0^{\tau_{D_s} a} dt
u^a_{dp}(x = 0,t) = - \sqrt{\frac{4}{\pi a}} \frac{\Gamma_p (1 - \chi) \ell_p}{\sqrt{D_s}}. 
\end{align}
%
In order to match the exact expression~\eqref{intudpexact}, we set $a = \pi/4$
and use the following the approximation 
%
\begin{align}
\label{udpa}
u^a_{dp}(x = 0,t) = - \frac{\Gamma_p (1 - \chi) }{\sqrt{4 \pi D_s t}}
H(\tau_{D_s} \pi - t),
\end{align}
%
where $H(t)$ denotes the Heaviside step function.  

%%%%%%%%%%%%%%%%%%%%%%%%%%%%%%%%%%%
\subsection{Extraction of particles}
%%%%%%%%%%%%%%%%%%%%%%%%%%%%%%%%%%%
In the case $\Gamma_p > 0$, particles are extracted from the DEP. The
particle concentration at $x = 0$, that is, at the interface 
with the TP is set equal to $c(x = 0,t) = c_i$, the resident particle concentration. 
Thus, we obtain by integration of Eq.~\eqref{eq:mdp} for the added particle mass
%
\begin{align}
\label{eq:mdp2}
m_{dp} = m_i + c_i w \int\limits_0^\infty dt u_{dp}(x = 0,t) =
m_i + c_i w  u_{dp}^\ast(x = 0,\sigma = 0) = m_i - \frac{m_i \Gamma_p (1 - \chi)}{D_s}, 
\end{align}
%
where $m_i = c_i w \ell_p$ is the initial particle mass and
$\chi=s_i/s_H$. Note that we used expression~\eqref{intudpexact} to
arrive at this result. If $\alpha_0$ is the fraction of particle mass inside the DEP
without DP, then the fraction $\alpha$ of particles after DP is 
%
\begin{align}
\alpha = \alpha_0 \frac{m_{dp}}{m_i}. 
\end{align}
%
Using Eq.~\eqref{eq:mdp2} and setting $\ell_p = \lambda$, we obtain
%
\begin{equation}
\alpha = \alpha_0 \left[1 - \Gamma_p^\ast Pe_s (1 - \chi) \right],
\label{alpha_mp}
\end{equation}
%
where $\Gamma_p^*=\Gamma_p/\lambda U$ is the dimensionless form of the
diffusiophoretic mobility and $Pe_s = \lambda U/D_s$ is the salt P\'eclet
number. 
%%%%%%%%%%%%%%%%%%%%%%%%%%%%%%%%%%%
\subsection{Trapping of particles}
%%%%%%%%%%%%%%%%%%%%%%%%%%%%%%%%%%%
In the case $\Gamma_p < 0$, particles are trapped from the TP into the
DEP. In order to determine $c(x = 0,t)$, we consider the balance of
fluxes across the interface. For $x < 0$, that is
within the TP, the particle flux transverse to the flow direction
is due to diffusion. For $x > 0$, that is, in the DEP the particle
flux is dominated by the diffusiophoretic drift. Thus, we can write
%
\begin{align}
\label{flux_balance}
- D_p \frac{c(x=0,t) - c_i}{\ell_0} = u_{dp}(x = 0,t) c(x = 0,t),
\end{align}
%
where $\ell_0$ is the concentration gradient scale. We assume that the
particle concentration in the flow past the interface
between TP and DEP is constant and equal to the initial concentration
$c_i$. We estimate $\ell_0$ as the length scale at which the diffusive flux
transverse to the flow direction toward the interface is of the same
order as the advective flux past the interface, that is,
%
\begin{align}
%\overline u c_i \sim D_p \frac{c_i}{\ell_0}. 
U c_i \sim D_p \frac{c_i}{\ell_0}. 
\end{align}
%
From this relation, we obtain the estimate
%
\begin{align}
%\ell_0 \sim \frac{D_p}{\overline u} = \frac{\ell_p}{Pe_p}. 
\ell_0 \sim \frac{D_p}U = \frac{\ell_p}{Pe_p}. 
\end{align}
%
That is, particles within the layer of thickness $\ell_0$ are available for
trapping in the DEP. \\
%
From~\eqref{flux_balance}, we obtain for $c_0(t) \equiv c(x = 0,t)$
%
\begin{align}
\label{c0}
c_0(t) = \frac{c_i}{1 + \frac{u_{0}(t) \ell_0}{D_p}},
\end{align}
% 
where we set $u_0(t) = u_{dp}(x=0,t)$. Inserting~\eqref{c0} into~\eqref{mdptime} gives 
%
\begin{align}
\label{mdptrap}
m_{dp} = m_i + w \int\limits_0^\infty dt u_{0}(t) \frac{c_i}{1 + \frac{u_{0}(t) \ell_0}{D_p}}. 
\end{align}
%
Note that here use the approximation~\eqref{udpa} for $u_0(t)$ to
derive an analytical expression for $m_{dp}$. 
%
%%%%%%%%%%%%%%%%%%%%%%%%%%%%%%%%%%%%%%%%%%%%%
\begin{figure}
\includegraphics[width=0.7\textwidth]{alpha.pdf}
\caption{Data for the dependence of $\alpha$ on $\Gamma_p^\ast$ (symbols) and
  the analytical model (solid line) for  $Pe_s = 0.75$, $Pe_p = 750$ and $\ell_0^\ast = 0.65/Pe_p $. }
\end{figure}
%%%%%%%%%%%%%%%%%%%%%%%%%%%%%%%%%%%%%%%%%%%%%
%


Inserting expression~\eqref{udpa}
into the right side of Eq.~\eqref{mdptrap} gives 
%
\begin{align}
m_{dp} = m_i + w  \int\limits_0^{\tau_{D_s} \pi/4} dt \frac{\hat \Gamma_p
  (1-\chi)}{\sqrt{\pi D_s t}} \frac{c_i}{1 + \frac{\hat \Gamma_p (1-\chi)
    \ell_0}{\sqrt{\pi D_s t}D_p}}. 
\end{align}
% 
where we set $\hat \Gamma_p = - \Gamma_p$. We can further write
%
\begin{align}
m_{dp} = m_i + w c_i  \int\limits_0^{\tau_{D_s} \pi/4} dt \frac{\hat \Gamma_p
  (1-\chi)}{\sqrt{\pi D_s t} + \frac{\hat \Gamma_p (1-\chi) \ell_0}{D_p}}. 
\end{align}
% 
Integration of the latter gives 
%
\begin{align}
m_{dp} = m_i + w \ell_p c_i \left\{\frac{\hat\Gamma_p (1-\chi)}{D_s} +
\frac{2 \hat \Gamma_p^2 (1 - \chi)^2 \ell_0}{\pi D_s D_p \ell_p} 
\ln\left[\frac{2 \hat \Gamma_p (1 - \chi)}{2 \hat \Gamma_p (1 - \chi) + \pi D_p \ell_p / \ell_0} \right] \right\}
\end{align}
%
Thus, we obtain for $\alpha$
%
\begin{align}
\alpha = \alpha _0 \left\{1 + \frac{\hat\Gamma_p (1-\chi)}{D_s} +
\frac{2 \hat \Gamma_p^2 (1 - \chi)^2 \ell_0}{\pi D_s D_p \ell_p} 
\ln\left[\frac{2 \hat \Gamma_p (1 - \chi)}{2 \hat \Gamma_p (1 - \chi) + \pi D_p \ell_p / \ell_0} \right]\right\}
\end{align}
%
We set $\ell_p = \lambda$ and define the dimensionless diffusiophoretic mobility and the dimensionless
diffusion layer scale as
%
\begin{align}
\label{eq:gamma}
\Gamma^\ast_p = - \frac{\hat \Gamma_p}{\lambda U}, && % \beta  = \frac{D_p}{D_s}
% \frac{\lambda}{\ell_0} = \frac{Pe_s}{Pe_{p}} \frac{1}{\ell_0^\ast} &&
\ell_0^\ast = \frac{\ell_0}{\lambda} \sim 1/Pe_p.  
\end{align}
%
Thus, we can write expression~\eqref{eq:gamma} in dimensionless form as
%
\begin{align}
\alpha = \alpha _0 \left\{1 - \Gamma_p^\ast (1-\chi)Pe_s + \frac{{2 \Gamma_p^\ast}^2
  (1 - \chi)^2 Pe_s Pe_{p} \ell_0^\ast}{\pi} 
\ln\left[\frac{2 \Gamma_p^\ast (1 - \chi) Pe_{p} \ell_0^\ast}{2 \Gamma_p^\ast (1 - \chi) Pe_{p} \ell_0^\ast- \pi
 } \right] \right\} .
\end{align}
%


\bibliographystyle{unsrt}
\bibliography{ref}% Produces the bibliography via BibTeX.


%======================================================

\end{document}
%
% ****** End of file apssamp.tex ******
