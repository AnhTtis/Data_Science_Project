%
\documentclass[twocolumn, aps, amsmath,amssymb,showkeys,a4paper]{revtex4-1}%{revtex4-1}
%\documentclass[onecolumn, amsmath,amssymb,preprint]{revtex4}%{revtex4-1}

\usepackage{amssymb}
\usepackage{amsbsy}
\usepackage{amsmath}
\usepackage{bbm}
\usepackage{xcolor}
%\usepackage{hyperref}

%\usepackage{tikz}
%\usetikzlibrary{automata,topaths}

%\usepackage{CJK}

%\usepackage{psfrag}
%\usepackage{pstricks}
\usepackage{graphicx}% Include figure files
\usepackage{dcolumn}% Align table columns on decimal point
\usepackage{bm}% bold math
\usepackage{subfigmat}
%\usepackage{gensymb}

\begin{document}

%\preprint{APS/123-QED}

\title{Diffusiophoresis and medium structure control macroscopic particle transport in porous media}

\author{Mamta Jotkar}
\email{mamta.jotkar@upm.es, mamta.jotkar@idaea.csic.es }
\affiliation{Universidad Polit\'ecnica de Madrid, Spain, \\ Institute
  of Environmental Assessment and Water Research, Spanish National
  Research Council, Barcelona, Spain} 
\author{Pietro de Anna}
\email{pietro.deana@unil.ch}
\affiliation{Institute of Earth Sciences, University of Lausanne, Switzerland} 
\author{Marco Dentz}
\email{marco.dentz@idaea.csic.es }
\affiliation{Institute of Environmental Assessment and Water Research, Spanish National Research Council, Barcelona, Spain} 
\author{Luis Cueto-Felgueroso}
\email{luis.cueto@upm.es}
\affiliation{Universidad Polit\'ecnica de Madrid, Spain}%, ETSI Caminos, Canales y Puertos, Spain}
\date{\today}% It is always \today, today,
             %  but any date may be explicitly specified

\begin{abstract}
% In this letter, we show that pore-scale diffusiophoresis due to solute
% concentration gradients manifests in the macroscopic dispersion 
% of  particles in a porous medium. To this end, we consider a
% hyper-uniform porous structure characterised by dead-end and
% transmitting pores. {\color{black} Diffusiophoresis induces 
%   motion along the local solute concentration gradients. The
%   diffusiophoretic drift decays rapidly as the solute 
% gradients attenuate, which occurs at times smaller than one pore
% volume. We show that, despite its transient character, 
% % And yet we show that, by altering the partition of colloids between transmitting and dead-end pores,
% this microscopic phenomenon controls the large-scale transport of
% particles within the porous medium by altering the partitioning 
% of colloids between transmitting and dead-end pores.} Depending on the
% sign of the diffusiophoretic mobility $\Gamma_p$, it is possible to
% either enhance particle mobilisation from or to promote trapping into
% dead-end pores. The impact of diffusiophoresis on large-scale
% dispersion is quantified in terms of the fraction of trapped
% particles, which depends non-linearly on the solute P\'eclet number
% and diffusiophoretic mobility. Our results suggest that
% diffusiophoresis provides a means for controlled manipulation and
% transport of particleal particles through porous media.

% In this letter, we show that pore-scale diffusiophoresis along local
%   salt gradients manifests in the macroscopic dispersion of colloids
%   in a porous medium. Despite its transient character, this microscopic phenomenon
%   controls the large-scale transport of particles by altering the
%   partitioning of colloids between transmitting and
%   dead-end pores. Depending on the diffusiophoretic mobility, colloids
%   can be mobilized from or trapped in dead-end pores, which provides 
%   means for controlled manipulation of particles in porous media. 
	
In this letter, we show that pore-scale diffusiophoresis of colloidal particles along local
  salt gradients manifests in the macroscopic dispersion of particles
  in a porous medium. Despite is transient character, this microscopic phenomenon
  controls large-scale particle transport by altering their
  partitioning between transmitting and dead-end pores. It determines
  the distribution of residence and arrival times in the
  medium. Depending on the diffusiophoretic mobility, particles
  can be mobilized from or trapped in dead-end pores, which provides a
  means for the controlled manipulation of particles in porous media. 

\end{abstract}
\keywords{Diffusiophoresis, anomalous dispersion, porous media}
\maketitle

Diffusiophoresis (DP) \cite{Derjaguin_1947} is the motion of
microscopic particles driven by local gradients of solute
concentration that has been demonstrated both
theoretically~\cite{PrieveAnderson_1984,Anderson_1989_ARFM,VelegolEtAl_2016,AultEtAl_2018}
and experimentally~\cite{AbecassisEtAl_2008,PalacciEtAl_2010_PRL,KarEtAl_2015,ShinEtAl_2016_PNAS,BattatEtAl_2019,SinghEtAl_2020_PRL}
to be a powerful particle manipulation tool. The physical mechanisms
that drive this complex physicochemical phenomenon can be broken down
to two components: chemiphoresis that occurs due the osmotic pressure
gradient along the surface of a charged particle (at the scale of the
particle) and electrophoresis arising due to the difference in the
diffusivities between the cation and the anion in the electrolyte
solution~\cite{VelegolEtAl_2016}. The physics behind this phenomenon
is well-established~\cite{PrieveAnderson_1984,Anderson_1989_ARFM}. Despite
the studies in relatively simpler microfluidic setups that demonstrate
particle focusing \cite{ShiEtAl_2016_PRL}, particle separation
\cite{ShinEtAl_2017_NatCommun,ShinEtAl_2018,RasmussenEtAl_2020,JotkarCueto-Felgueroso_2021},
particle banding \cite{StaffeldQuinn_1989}, particle trapping
\cite{SinghEtAl_2020_PRL}, etc., with the help of DP,
its effect on macroscopic transport within intricate and spatially
variable porous structures remains unexplored.

Flow and transport of dissolved solutes and suspended particles through
 porous and confined media are ubiquitous in natural and
engineered systems \cite{Bear}. Most geological and biological
porous media share the common feature of being spatially variable or
heterogeneous. The broad range of variability in pore size is known to
induce anomalous transport. {\color{black}Moreover,} the diversity in
shape of the constituent grains induces a rich flow organization that plays an
important role in groundwater contamination and
remediation~\cite{KahlerEtAl_2019_gw}, enhanced hydrocarbon
recovery~\cite{EOR}, transport through river sediments~\cite{Lei2022}
and water filtration systems~\cite{ShinEtAl_2017_NatCommun,miele2019}. {\color{black}The
  morphology of a porous medium is often characterized
  by the presence of cavities or dead-end pores (DEP), which represent
  the part of the system that cannot host net fluid transfer,
  resulting in stagnant flow~\cite{Lever1985, Nishiyama2017,
    Erktan2020}. These DEPs are connected via a network of
  percolating channels or transmitting pores (TP). Such DEP-TP}
structures characterize biological tissues~\cite{Hapfelmeier2010},
soil~\cite{Erktan2020} and filters~\cite{Phillip2011} and lead to
anomalous transport of passive tracers~\cite{BordoloiEtAl_2022}. {\color{black}To
date, the combined role of DP and the
complexity of the porous medium on particle transport remains elusive.} Here, we
use detailed numerical pore-scale simulations and analytical modeling to elucidate how
DP couples with the medium structure to alter macroscopic particles transport.

We study a {\color{black}fluid-}saturated porous system where a
particle suspension gets displaced by a continuously injected salt
solution. The velocity experienced by each transported 
particle results from advection in the flow field $\mathbf u$, which is
controlled by the medium structure and {\color{black} imposed flow
  rate}~\cite{Dentz2011,deAnnaPRF2017,Dentz2018}, and due to the
diffusiophoretic drift $\textbf{u}_{dp}$. In the thin Debye
layer limit, for dilute solutions with valence symmetric solutes
(e.g., LiCl, NaCl), the diffusiophoretic drift is proportional to the
{\color{black}gradient of the logarithm} of the solute concentration
%
\begin{equation}
	\label{udp}
	\textbf{u}_{dp} = \Gamma_p \nabla \ln s
\end{equation}
%
and the diffusiophoretic mobility $\Gamma_p$ is approximately a
constant. $\Gamma_p$ is determined by the size and surface charge of
the particle \cite{PrieveAnderson_1984,Anderson_1989_ARFM} 
%
\begin{align}
\nonumber
	\Gamma_p = \frac{\varepsilon k_B T}{\nu Z e } \left\{
          D\zeta - \frac{2k_B T}{Z e} \ln \left[1 - \tanh^2
            \left(\frac{Z e \zeta}{4k_B T}\right) \right]\right\},
\end{align}
%
where $\varepsilon$ is the dielectric permittivity of the medium,
$\zeta$ is the particle zeta potential, $\nu$ is
the kinematic viscosity of the medium, $k_B$ is the Boltzmann
constant, $T$ is the absolute temperature, $Z$ is the valence of the
constituent ions of the solute, $e$ is the proton charge and
$D=(D_+ - D_-)/(D_+ + D_-)$ measures the difference in
diffusivity $D_+$ of the cation and $D_-$ of the anion. The
logarithmic dependence of the particle velocity $\textbf{u}_{dp}$ on
the solute concentration gradients $\nabla s$ allows for rapid and
efficient particle motion, even in low concentration areas. 


We consider a porous system {\color{black}characterized by DEPs of
  different size connected to a network of TPs,}
similar to the one used in ref.~\cite{BordoloiEtAl_2022}. The computational
domain is shown in figure \ref{fig1}. {\color{black} The mean flow is
  driven from left to right with the flow rate $U$.} The medium is statistically homogeneous with a mean
pore-size $\lambda=30 \mu \text{m}$ and porosity $\phi= 0.39$. The dual
feature of this medium is characterized by TPs, and DEPs leading to stagnant flow. While solutes typically diffuse and dissipate
gradients within pores over time scales shorter than the time $\tau_L = L/U$
needed to elute a pore-volume, the junctures of TPs and DEPs serve as excellent
candidates to retain gradients of solute concentration, which in turn trigger
DP.

 \begin{figure}[t]
 \begin{center}
 \includegraphics[width=0.45\textwidth]{setup}
 \vspace{-0.4cm}
 \caption{Hyper-uniform porous structure characterized by DEPs and TPs. Computational domain with white spaces indicating solid grains (bottom). \label{fig1}}
 \end{center}
 \end{figure}
 
In the low Reynolds number limit, the
{\color{black}fluid-solute-particle dynamics} are governed by the Stokes
equation for fluid flow, and the advection-diffusion equations for the
solute and particle transport \cite{AultEtAl_2018}
%
\begin{subequations}
	\label{govern_eq}
	\vspace{-0.cm}
	\begin{equation}
		\label{nse}
		0 = \frac{1}{\rho} \left( - \nabla p  +  \mu{\nabla}^2 \textbf{u} \right), 
	\end{equation} \vspace{-0.8cm}
	\begin{equation}\label{cont}
		\nabla  \cdot \textbf{u} =0, 
	\end{equation}  \vspace{-0.9cm}
	\begin{equation}\label{scon}
		 \frac{\partial s}{\partial t} +  \nabla  \cdot (\textbf{u} s )=  D_s{\nabla}^2 s , 
	\end{equation} \vspace{-0.7cm}
	\begin{equation}\label{ccon}
		 \frac{\partial c}{\partial t} +  \nabla  \cdot (\textbf{u} +\textbf{u}_{dp})c =  D_p{\nabla}^2 c ,
	\end{equation} 
\end{subequations}
%
where $\textbf{u}$ is the two-dimensional velocity field, $p$ is the
pressure, $s$ is the solute concentration and $c$ is the 
particle concentration, $\mu$ is the dynamic viscosity of the fluid
and $\rho$ its density. $D_s$ and $D_p$ are the diffusion coefficient
of solute and particles, respectively. Typically, solutes diffuse much
faster than particles such that $D_s \gg D_p$. Solute and particle
transport can be characterized by the respective P\'eclet numbers,
$Pe_s = U \lambda / D_s $ and $Pe_p = U \lambda / D_p$, which compare
the characteristic diffusion time scales $\tau_{D_s} = \lambda^2 /
D_s$ and $\tau_{D_p} = \lambda^2 /D_p$ to the advection time $\tau_v =
\lambda / U$ over the {\color{black} mean pore length}. The
diffusiophoretic particle velocity $\textbf{u}_{dp}$ is given by
Eq.\eqref{udp}. The medium is initially saturated with particles and
solute with initial concentrations $c_i$ and $s_i$, respectively. At time $t> 0$, 
 a sharp front of solute at concentration
 $s_H \gg s_i$ is injected such that the ratio $\chi = s_i/s_H \ll
 1$. This solute front induces a dynamic and heterogeneous solute concentration gradient
that drives DP. The governing
equations~\eqref{govern_eq} are solved numerically for the pore
geometry detailed above. The numerical setup is described
in~\cite{SM}. 

%%%%%%%%%%%%%%%%%%%%%%%%%%%%%%%%%%%%%%%%%%%%%%%%%%%%%%
 \begin{figure*}[t]
 \begin{center}
 $t\sim35\tau_v$ \hspace{2.6cm} $t\sim70\tau_v$ \hspace{2.6cm} $t\sim210\tau_v$ \hspace{2.6cm} $t\sim665\tau_v$
 \subfigure[Particle trapping $\Gamma_p<0$]{\label{fig2a}\includegraphics[trim=0.9cm 1.9cm 3.5cm 2.9cm, clip=true, width=1.7in]{2a_1.png}
 \includegraphics[trim=0.9cm 1.9cm 3.5cm 2.9cm, clip=true, width=1.7in]{2a_2.png}
 \includegraphics[trim=0.9cm 1.9cm 3.5cm 2.9cm, clip=true, width=1.7in]{2a_3.png}
 \includegraphics[trim=0.9cm 1.9cm 3.5cm 2.9cm, clip=true, width=1.7in]{2a_4.png}
 } \\ 
 %
 \subfigure[No DP $\Gamma_p=0$]{\label{fig2b}
 \includegraphics[trim=0.9cm 1.9cm 3.5cm 2.9cm, clip=true, width=1.7in]{2b_1.png}
 \includegraphics[trim=0.9cm 1.9cm 3.5cm 2.9cm, clip=true, width=1.7in]{2b_2.png}
 \includegraphics[trim=0.9cm 1.9cm 3.5cm 2.9cm, clip=true, width=1.7in]{2b_3.png}
 \includegraphics[trim=0.9cm 1.9cm 3.5cm 2.9cm, clip=true, width=1.7in]{2b_4.png}
 }\\
 %
 \subfigure[Particle extraction $\Gamma_p>0$]{\label{fig2c}
 \includegraphics[trim=0.9cm 1.9cm 3.5cm 2.9cm, clip=true, width=1.7in]{2c_1.png}
 \includegraphics[trim=0.9cm 1.9cm 3.5cm 2.9cm, clip=true, width=1.7in]{2c_2.png}
 \includegraphics[trim=0.9cm 1.9cm 3.5cm 2.9cm, clip=true, width=1.7in]{2c_3.png}
 \includegraphics[trim=0.9cm 1.9cm 3.5cm 2.9cm, clip=true, width=1.7in]{2c_4.png}
 }
 \vspace{-0.3cm}
 \caption{Temporal evolution of the dimensionless particle
   distributions $c/c_i$ for (a) trapping, (b) no-DP, and (c)
   extraction cases. The concentration scale is logarithmic between $10^{-5}$
   (light) and $10$ (dark). White spaces indicate solid
   grains. \label{fig2}}
 \end{center}
 \end{figure*}
%%%%%%%%%%%%%%%%%%%%%%%%%%%%%%%%%%%%%%%%%%%%%%%%%%%%%%

Figure \ref{fig2} shows the temporal evolution of the particle
concentration field without DP ($\Gamma_p=0$), and for {\color{black}diffusiophoretic trapping ($\Gamma_p<0$), and extraction ($\Gamma_p>0$)}. In the absence of DP, majority of the particles get dispersed through the
TPs leaving behind a small fraction of particles that accumulate
within the DEPs, where the flow is stagnant, 
organized in convection rolls~\cite[][]{BordoloiEtAl_2022}. The only mechanism
through which these localized particles can escape into the TPs
is diffusion, the times scale of which is typically orders of
magnitude {\color{black} larger} than $\tau_L$. 

The injection of a sharp front of solute at a higher concentration
results in local gradients of solute concentration that 
drive DP within the DEPs. For $\Gamma_p < 0$, that is, when particles
migrate from high to low solute concentrations, DP leads
to the trapping of particles inside the DEPs, as shown in figure \ref{fig2a}.
% {\color{black} Note that this scenario is analogous to reversing the
% direction of the solute gradient when the displacing front carries
% solute at lower concentration ($\chi=s_i/s_H\gg 1$). (MAMTA:
% Something like this sentence is needed to give context to the 1st
% line in the very last paragraph about reversibility)}

When $\Gamma_p>0$, DP leads to 
particle mobilization out of the DEPs because the particles
move towards the higher solute concentration zones in the TPs. This is
seen in the particle distributions shown in Figure \ref{fig2c}, where the
particles within the DEPs rapidly escape the DEPs and leave
the domain from the right outlet. Figure~\ref{fig3} shows the time evolution of
the solute concentration at a point within a DEP {\color{black}close to the bottom of the DEP}. The increase or
decrease of concentration due to DP occurs on the time scale $\tau_{D_s}${\color{black}, which here is smaller than $\tau_{D_p}$ by a factor of $10^3$.} Thus, the action of DP on mass
transfer between TPs and DEPs is limited to
relatively short initial times.

To understand how DP controls the macroscopic fate of the suspended particles, 
we first estimate the fraction $\alpha$ of particles that are trapped in or
mobilized from the DEPs in the initial phase. 
To this end, we consider a single DEP connected to a TP~\cite{SM}. For this geometry, we can derive the solute concentration profile in the DEP, and
thus obtain explicit expressions for the diffusiophoretic drift $u_{dp}$. Combining the latter with conservation of particle flux at the interface between TP and DEP, we obtain the following expression for $\alpha$ as
a function of $\Gamma_p^\ast = \Gamma_p / U \lambda$ 
%
\begin{widetext}
\begin{align}
\alpha = \alpha _0 \left[1 - \Gamma_p^\ast (1-\chi)Pe_s\right] +
  H(-\Gamma_p^\ast) \left\{ \frac{{2 \Gamma_p^\ast}^2
  (1 - \chi)^2 Pe_s Pe_{p} \ell_0^\ast}{\pi} 
\ln\left[\frac{2\Gamma_p^\ast (1 - \chi) Pe_{p} \ell_0^\ast}{2\Gamma_p^\ast (1 - \chi) Pe_{p} \ell_0^\ast-  \pi
 } \right] \right\},
	\label{alpha_mp}
\end{align}
\end{widetext}
%
where $H(\cdot)$ is the Heaviside step function, and $\alpha_0$ the initial
fraction of particles in the DEPs {\color{black}in the absence of DP
  ($\Gamma_p=0$)}. The dimensionless length $\ell_0^\ast$ denotes 
a characteristic diffusion scale at the interface between TP and DEP.
It is inversely proportional to the particle P\'eclet
number, $\ell_0^\ast \sim 1/Pe_p$. We set in the following $\ell_0^\ast =
\beta/Pe_p$ with $\beta$ a number of the order of one. Expression~\eqref{alpha_mp} predicts that the
particles are depleted from the DEPs for $\Gamma_p^\ast \geq 1/(1 - \chi)
Pe_s$. Furthermore, for strongly negative diffusiophoretic mobility, the rate at which particles are transferred to
the interface is eventually smaller than the diffusiophoretic
drift. Thus, {\color{black} by taking the limit of $\Gamma_p^\ast \to -
  \infty$ in Eq. (\ref{alpha_mp}),} we find that $\alpha$ asymptotes toward
%
\begin{align}
\alpha_{\infty} = \alpha_0 \left(1 + \frac{\pi Pe_s}{\beta}\right).
\end{align}
%
The trapped particle fraction increases linearly with $Pe_s$ because
the solute gradients and thus the diffusiophoretic drift increase
with increasing $Pe_s$. 

%%%%%%%%%%%%%%%%%%%%%%%%%%%%%%%%%%%%%%%%%%%%%%%%%%%%%%
\begin{figure}[t]
 \begin{center}
 \includegraphics[width=0.45\textwidth]{trapping_release}
 \caption{Temporal evolution of dimensionless particle concentration
   ($c/c_i$) at an arbitrary point (indicated by grey) within a DEP
   shown in the inset for {\color{black}$\Gamma_p^*=-1$ (blue, trapping), $\Gamma_p^*=0$
   (black, no DP), and $\Gamma_p^*=0.4$} (red, extraction). The particle
   concentrations shown in the inset are displayed in a logarithmic scale and range 
   from $10^{-6}$ (light) to $10$ (dark).
   \label{fig3}}
 \end{center}
 \end{figure}
%%%%%%%%%%%%%%%%%%%%%%%%%%%%%%%%%%%%%%%%%%%%%%%%%%%%%%


%%%%%%%%%%%%%%%%%%%%%%%%%%%%%%%%%%%%%%%%%%%%%%%%%%%%%%
 \begin{figure}[t]
 \begin{center}
 \includegraphics[width=0.45\textwidth]{btcs} 
 \includegraphics[width=0.45\textwidth]{alpha_main} 
 \caption{(Top panel) Arrival time distribution $F(t)$ at the outlet
   for (blue circles)  {\color{black}$\Gamma_p^*= -1$, (black triangles) $0$ and
   (orange circles) $0.4$}. The solid lines denote the travel-time model in
   Eq.~\eqref{btc_ctrw}. (Bottom panel) Particle fraction $\alpha$ in
   the DEP from the (symbols) numerical data, and (solid line) the
   analytical expression~\eqref{alpha_mp} for $\ell_0^\ast = 0.65/Pe_p$. \label{fig4}}
 \end{center}
 \end{figure}
%%%%%%%%%%%%%%%%%%%%%%%%%%%%%%%%%%%%%%%%%%%%%%%%%%%%%%

Figure \ref{fig4} shows the distribution of arrival times of 
particles at the outlet. Similar to Ref.~\cite{BordoloiEtAl_2022}, we observe two transport
regimes. At times of the order of $\tau_L$, particles at the outlet
are produced by advection and dispersion from the TPs.
{\color{black}For times $t \gg \tau_L$, the arrival time distribution
  deviates from the exponential decay predicted under the classical
  dispersion framework and displays a power law tail. This is caused
  by the particles that are initially trapped within the DEPs.} For $\Gamma_p < 0$, {\color{black} an
  increasing fraction of} particles is trapped {\color{black}in the DEP}, which manifests in a stronger tail than without DP. 
 For $\Gamma_p > 0$, particles are  extracted from the DEP at initial
 times, and thus the tail is weaker than for $\Gamma_p \leq 0$.

The arrival time distribution can be modeled as
the superposition of the residence time distributions in the TPs and
DEPs~\cite{BordoloiEtAl_2022}
%
\begin{equation}
F(t) = (1-\alpha) % \frac{1}{L} \int_0^L \text{d}x f_0(t,x)
F_0(t)+ \alpha \int_0^\infty \text{d}\tau \frac{g(t/\tau)}{\tau} f_D(\tau),
\label{btc_ctrw}
\end{equation} 
%
respectively. Note that $\alpha$ is the fraction of trapped  particles
after the short initial phase, which can be approximated by
expression~\eqref{alpha_mp}.
We assume that transport in the TP can be
characterized by the mean flow velocity and a hydrodynamic dispersion
coefficient. Thus, $F_0(t)$ is given by the superposition of inverse
Gaussian distributions as, 
%
\begin{align}
F_0(t) = \frac{1}{L} \int\limits_0^L d x \frac{x \exp[-(x - \overline u t)^2/(4 D_h t)]}{\sqrt{4 \pi D_h t^3}}. 
\end{align}
%
%MARCO: What value was used for $D_h$? MAMTA: $D_h=0.1mm^2/s$
As we see in Figure~\ref{fig4}, this approximation is able to capture the early
arrival times, but underestimates the arrival time distribution at intermediate
times. This can be traced back to the velocity variability between
pores~\cite{Dentz2018}. The residence time distribution within the DEPs
is expressed in terms of {\color{black}a Gamma-}distribution $g(t')$ of dimensionless
residence times $\tau' = \tau/\tau_{D_p}$ in a single DEP and the
distribution $f_D$ of characteristic diffusion times $\tau_{D_p}$, {\color{black}given below}~\cite[][]{BordoloiEtAl_2022}
%
\begin{equation}\label{pdf_dep}
	g(t') = \frac{t'^{-2/3}\exp (-t')}{\Gamma (1/3)},
\end{equation}
%
while $f_D$ is obtained from the distribution $f_\Lambda$ of aspect ratios
$\Lambda$ through the map $\Lambda \to \tau_{D_p}  = (\lambda\Lambda)^2
/D_p$.

{\color{black}Figure \ref{fig4} shows the numerically estimated
  particle fractions versus the dimensionless diffusiophoretic
  mobility, and the analytical expression~\eqref{alpha_mp} for $\beta
  = 0.65$.}  
The analytical model describes the full dependence of $\alpha$ on $\Gamma_p$ for
both extraction from and trapping in DEPs, and thus seems to capture the controls of
DP on the macro-scale dispersion of  particles.

In conclusion, the microscopic interactions between DP and flow and transport
through porous media impact the macroscopic fate of 
particles. Depending on the diffusiophoretic mobility
$\Gamma_p$, DP may promote trapping {\color{black}within} or it may
lead to particle mobilization from the DEPs. This can be exploited
for preferential particle deposition or removal. DP is a short-term phenomenon
that persists as long as the solute gradients are not dissipated by diffusion,
which is characterized by the time scale $\tau_{D_s} = \lambda^2/D_s$ {\color{black} and is typically smaller than the characteristic
advection time across the medium i.e. $\tau_{D_s} \ll \tau_L$.} {\color{black} However, despite its short
  timespan, DP has a significant impact on macroscopic
  transport, by reorganizing the partitioning of particles between the
  DEPs and TPs, which is quantified by the fraction $\alpha$.}

Our results suggest that DP provides an {\color{black}efficient} way of
controlling particle transport through porous media in a reversible
manner by changing the direction of the solute gradient. Moreover, the existence of
DEPs is quite common in natural geological and biological porous
media, and serve as excellent candidates for retaining microscopic gradients of
solute concentration for relatively large times. This opens new
avenues for developing technological solutions to
various problems of socio-economic relevance such as groundwater
remediation, enhanced oil recovery, water-filtration systems, targeted drug
delivery and microfluidics for biomedical applications.

\begin{acknowledgments}
	This work has received funding from European Union's Horizon 2020
        research and innovation program under the Marie Sk\l{}odowska-Curie
        G.A. No. 895569. MD and LCF gratefully acknowledge funding from the
        Spanish Ministry of Science and Innovation through the project HydroPore
        (PID2019-106887GB-C31-C33). P.d.A. acknowledges the support of FET-Open
        project NARCISO (ID: 828890) and of Swiss National Science Foundation
        (grant ID 200021 172587).
       
\end{acknowledgments}


%\bibliographystyle{unsrt}
%\bibliography{ref}
\begin{thebibliography}{100}

\bibitem{Derjaguin_1947}
B.~V. Derjaguin, G.~P. Sidorenkov, E.~A. Zubashchenkov, and E.~V. Kiseleva.
\newblock Kinetic phenomena in boundary films of liquids.
\newblock {\em Colloid J. USSR}, 9:335--347, 1947.

\bibitem{PrieveAnderson_1984}
D.~C. Prieve, J.~L. Anderson, J.~P. Ebel, and M.~E. Lowell.
\newblock Motion of a particle generated by chemical gradients. part 2.
  {E}lectrolytes.
\newblock {\em J. Fluid Mech.}, 148:247--269, 1984.

\bibitem{Anderson_1989_ARFM}
J.~L. Anderson.
\newblock Colloid transport by interfacial forces.
\newblock {\em Annu. Rev. Fluid Mech.}, 21(1):61--99, 1989.

\bibitem{VelegolEtAl_2016}
D.~Velegol, A.~Garg, R.~Guha, A.~Kar, and M.~Kumar.
\newblock Origins of concentration gradients for diffusiophoresis.
\newblock {\em Soft Matter}, 12:4686--4703, 2016.

\bibitem{AultEtAl_2018}
J.~T. Ault, S.~Shin, and H.~A. Stone.
\newblock Diffusiophoresis in narrow channel flows.
\newblock {\em J. Fluid Mech.}, 854:420--448, 2018.

\bibitem{AbecassisEtAl_2008}
B.~Ab{\'e}cassis, C.~Cottin-Bizonne, C.~Ybert, A.~Ajdari, and L.~Bocquet.
\newblock Boosting migration of large particles by solute contrasts.
\newblock {\em Nat. Mater.}, 7(10):785--789, 2008.

\bibitem{PalacciEtAl_2010_PRL}
J.~Palacci, B.~Ab\'ecassis, C.~Cottin-Bizonne, C.~Ybert, and L.~Bocquet.
\newblock Colloidal motility and pattern formation under rectified
  diffusiophoresis.
\newblock {\em Phys. Rev. Lett.}, 104:138302, 2010.

\bibitem{KarEtAl_2015}
A.~Kar, T.-Y. Chiang, I.~Ortiz~Rivera, A.~Sen, and D.~Velegol.
\newblock Enhanced transport into and out of dead-end pores.
\newblock {\em ACS Nano}, 9(1):746--753, 2015.
\newblock PMID: 25559608.

\bibitem{ShinEtAl_2016_PNAS}
S.~Shin, E.~Um, B.~Sabass, J.~T. Ault, M.~Rahimi, P.~B. Warren, and H.~A.
  Stone.
\newblock Size-dependent control of colloid transport via solute gradients in
  dead-end channels.
\newblock {\em Proc. Natl. Acad. Sci.}, 113(2):257--261, 2016.

\bibitem{BattatEtAl_2019}
S.~Battat, J.~T. Ault, S.~Shin, S.~Khodaparast, and H.~A. Stone.
\newblock Particle entrainment in dead-end pores by diffusiophoresis.
\newblock {\em Soft Matter}, 15:3879--3885, 2019.

\bibitem{SinghEtAl_2020_PRL}
N.~Singh, Goran~T. Vladisavljevi\ifmmode~\acute{c}\else \'{c}\fi{},
  Fran\ifmmode \mbox{\c{c}}\else~\c{c}\fi{}ois Nadal, C\'ecile Cottin-Bizonne,
  Christophe Pirat, and Guido Bolognesi.
\newblock Reversible trapping of colloids in microgrooved channels via
  diffusiophoresis under steady-state solute gradients.
\newblock {\em Phys. Rev. Lett.}, 125:248002, 2020.

\bibitem{ShiEtAl_2016_PRL}
Nan Shi, Rodrigo Nery-Azevedo, Amr~I. Abdel-Fattah, and Todd~M. Squires.
\newblock Diffusiophoretic focusing of suspended colloids.
\newblock {\em Phys. Rev. Lett.}, 117:258001, 2016.

\bibitem{ShinEtAl_2017_NatCommun}
Shin S., Shardt O., Warren~P. B., and Stone~H. A.
\newblock Membraneless water filtration using $co_2$.
\newblock {\em Nat. Commun.}, 8:15181, 2017.

\bibitem{ShinEtAl_2018}
S.~Shin, P.~B. Warren, and H.~A. Stone.
\newblock Cleaning by surfactant gradients: Particulate removal from porous
  materials and the significance of rinsing in laundry detergency.
\newblock {\em Phys. Rev. Appl.}, 9:034012, 2018.

\bibitem{RasmussenEtAl_2020}
M.~K. Rasmussen, J.~N. Pedersen, and R.~Marie.
\newblock Size and surface charge characterization of nanoparticles with a salt
  gradient.
\newblock {\em Nat. Commun.}, 11(1):2337, 2020.

\bibitem{JotkarCueto-Felgueroso_2021}
Jotkar M. and Cueto-Felgueroso L.
\newblock Particle separation through diverging nanochannels via
  diffusiophoresis and diffusioosmosis.
\newblock {\em Phys. Rev. Applied}, 16:064067, 2021.

\bibitem{StaffeldQuinn_1989}
P.~O. Staffeld and J.~A. Quinn.
\newblock Diffusion-induced banding of colloid particles via diffusiophoresis:
  1. {E}lectrolytes.
\newblock {\em J. Colloid Interface Sci.}, 130(1):69--87, 1989.

\bibitem{Bear}
J.~Bear.
\newblock {\em Dynamics of fluids in porous media}.
\newblock Courier Corporation, 1988.

\bibitem{KahlerEtAl_2019_gw}
David~M. Kahler and Zbigniew~J. Kabala.
\newblock Acceleration of groundwater remediation by rapidly pulsed pumping:
  Laboratory column tests.
\newblock {\em Journal of Environmental Engineering}, 145(1):06018009, 2019.

\bibitem{EOR}
S.~Thomas.
\newblock {\em Oil and Gas Science and Technology - Rev. IFP}, 63 (1):9--19,
  2008.

\bibitem{Lei2022}
Liang Lei, Taehyung Park, Karl Jarvis, Lingli Pan, Imgenur Tepecik, Yumeng
  Zhao, Zhuan Ge, Jeong-Hoon Choi, Xuerui Gai, Sergio~Andres Galindo-Torres,
  Ray Boswell, Sheng Dai, and Yongkoo Seol.
\newblock Pore-scale observations of natural hydrate-bearing sediments via
  pressure core sub-coring and micro-ct scanning.
\newblock {\em Scentifici Reports}, 12(3471).

\bibitem{miele2019}
F.~Miele, P.~de~Anna, and M.~Dentz.
\newblock Stochastic model for filtration by porous materials.
\newblock {\em Phys. Rev. Fluids}, 4:094101, 2019.

\bibitem{Lever1985}
Lever D.A., Bradbury M.H., and Hemingway S.J.
\newblock The effect of dead-end porosity on rock-matrix diffusion.
\newblock {\em Journal of Hydrology}, 80:45--76, 1985.

\bibitem{Nishiyama2017}
Naoki Nishiyama and Tadashi Yokoyama.
\newblock Permeability of porous media: Role of the critical pore size.
\newblock {\em Journal of Geophysical Research: Solid Earth},
  122(9):6955--6971, 2017.

\bibitem{Erktan2020}
Amandine Erktan, Dani Or, and Stefan Scheu.
\newblock The physical structure of soil: Determinant and consequence of
  trophic interactions.
\newblock {\em Soil Biology and Biochemistry}, 148:107876, 2020.

\bibitem{Hapfelmeier2010}
Siegfried Hapfelmeier, Melissa A.~E. Lawson, Emma Slack, Jorum~K. Kirundi,
  Maaike Stoel, Mathias Heikenwalder, Julia Cahenzli, Yuliya Velykoredko,
  Maria~L. Balmer, Kathrin Endt, Markus~B. Geuking, Roy Curtiss, Kathy~D.
  McCoy, and Andrew~J. Macpherson.
\newblock Reversible microbial colonization of germ-free mice reveals the
  dynamics of iga immune responses.
\newblock {\em Science}, 328(5986):1705--1709, 2010.

\bibitem{Phillip2011}
William~A. Phillip, Rachel~Mika Dorin, J{\"o}rg Werner, Eric M.~V. Hoek, Ulrich
  Wiesner, and Menachem Elimelech.
\newblock Tuning structure and properties of graded triblock terpolymer-based
  mesoporous and hybrid films.
\newblock {\em Nano Letters}, 11(7):2892--2900, 2011.
\newblock PMID: 21648394.

\bibitem{BordoloiEtAl_2022}
A.~D. Bordoloi, D.~Scheidweiler, M.~Dentz, Bouabdellaoui M., Abbarchi M., and
  de~Anna~P.
\newblock Structure induced laminar vortices control anomalous dispersion in
  porous media.
\newblock {\em Nat. Commun.}, 13:3820, 2022.

\bibitem{Dentz2011}
Dentz Marco, Le~Borgne Tanguy, Englert Andreas, and Bijeljic Branko.
\newblock Mixing, spreading and reaction in heterogeneous media: A brief
  review.
\newblock {\em Journal of Contaminant Hydrology}, 120-121:1--17, 2011.

\bibitem{deAnnaPRF2017}
P.~de~Anna, B.~Quaife, G.~Biros, and R.~Juanes.
\newblock Prediction of velocity distribution from pore structure in simple
  porous media.
\newblock {\em Phys. Rev. Fluids}, 2:124103, 2017.

\bibitem{Dentz2018}
M.~Dentz, M~Icardi, and J.~J. Hidalgo.
\newblock Mechanisms of dispersion in a porous medium.
\newblock {\em J. of Fluid Mech.}, 841:851--882, 2018.

\bibitem{SM}
{See Supplmental Material at [{\em URL will be inserted by publisher}] for the
  setup of the detailed numerical simulations and the details on the analytical
  model.}

\end{thebibliography}



\end{document}