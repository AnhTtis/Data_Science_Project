\section{Beyond the Standard Model phenomenology with SMEFT}
\label{sect:practical}
In this section we showcase the practical application of effective field theory, and particularly of the SMEFT, in the search for new physics. Since the Standard Model predictions so far agree so well with experiment, we think that a detailed treatment is important. The bumpy history of some of the
precision results shows the importance of a systematic treatment. We will discuss the different EFTs involved, and how they are linked. We comment on existing work and consider in detail two explicit examples to illustrate all the steps needed in order to reliably constrain possible BSM scenarios.

\subsection{The SMEFT analysis workflow}
We start by discussing the general EFT workflow for new-physics searches, whereas detailed information on
the various steps involved will be discussed in the subsequent sections.
For a typical BSM analysis, we consider a tower of effective theories as illustrated in Fig.~\ref{fig:multiscale-eft}. Each of these EFTs provides an accurate description of nature at a given energy scale, only containing the relevant degrees of freedom at that energy and incorporating the effect of heavier states by higher-dimensional effective operators. To connect theories at different energy scales, matching computations are performed, which allow to integrate out the heavy particles from one theory to obtain the corresponding low-energy EFT. We already encountered a matching computation for the Fermi constant in Eq.~\eqref{eq:4fermi-matching}; further details on the procedure will be given in Sec.~\ref{sec:matching}.
Then, within each EFT, the corresponding renormalization group~(RG) equations can be used to evolve all couplings from the high energy scale, where the matching was performed, to the low energy scale of the heaviest particles remaining in the EFT (see Sec.~\ref{sec:rge}). These can then be integrated out in a next matching computation to obtain yet another EFT valid at even lower energies. This procedure has to be repeated until the desired energy scale --usually the energy range of experimental observables-- is reached. By using such a multi-step procedure of alternate matching and running, we can ensure a proper description of physics at all involved scales, as the RG~evolution allows to resum large logarithms that would appear if just a single matching would be performed at the lowest scale, or if we would even use only the full theory.
As one can easily imagine such a multi-scale computation can become highly involved  which requires automation of all the steps involved. There are already several computer tools available that automate some of these steps [for an overview see \cite{Proceedings:2019rnh,Aebischer:2023irs}]. However, complete automation is still a goal to be achieved in the future.

To be concrete, we can take a BSM theory containing some heavy particles with masses of order~$\mathcal{O}(\Lambda_\mathrm{BSM})$ that are not accessible at the energies of current experiments. 
We can then match such theory to the SMEFT where the effect of these heavy states is encoded in effective operators. 
In general, not all of the SMEFT operators will be generated by the matching, but only a certain subset. 
For example, a list of all operators that are generated at tree level in all possible BSM scenarios has been worked out in the UV/IR dictionary provided in \cite{deBlas:2017xtg}.
Consequently, the SMEFT RG~equations~\cite{Jenkins:2013zja,Jenkins:2013wua,Alonso:2013hga} are used to evolve the couplings from the matching scale~$\Lambda_\mathrm{BSM}$ down to the electroweak scale~$\sim m_W$. Through this RG~mixing further operators can be generated, that were absent in the matching. Then, at the electroweak scale spontaneous symmetry breaking takes place as discussed in Sec.~\ref{sec:EWSB-in-SMEFT}. After expanding the Higgs field around its vacuum expectation value, we end up with the SMEFT Lagrangian in the broken phase, invariant under the $\mathrm{SU}(3)_c \times \mathrm{U}(1)_e$ gauge symmetry, and containing only the physical Higgs~$h$ rather than the full Higgs doublet~$H$, which is only present in the unbroken phase above the electroweak scale. 
One can now integrate out the heaviest SM particles, i.e. the top quark~$t$, and the Higgs~$h$, $Z$,~and $W$~boson, to arrive at the~LEFT \cite{Jenkins:2017jig,Dekens:2019ept}.
Usually EWSB and the matching is performed at the single scale~$\sim m_W$. 
Since the masses of all particles that are integrated out here are very similar, no large logarithms arise even though we choose only a single matching scale.
Afterwards, the known LEFT~RG equations~\cite{Jenkins:2017dyc} can be used to evolve the theory down to the bottom-quark mass scale~$m_b$. Then, if necessary, the $b$~quark can be integrated out, and so on, until one reaches the energy scale of the experimental observables of interest. For example, for $B$~physics experiments it is enough to stop the procedure here, but for processes at even lower energies one might require also integrating out, e.g., the charm quark. See \cite{Buchalla:1995vs} for an excellent review of the EFTs at the $b$-scale.

\begin{figure}[t]
    \centering
    \includegraphics[width= 0.95\linewidth]{figures/EFT-multiscale-figure_fig10.pdf}
    \caption{%
    Tower of effective field theories ranging from far UV scales to the low energies where experiments are performed. Depending on the observable, different EFTs might be appropriate. The various EFTs can be connected through matching and RG~evolution. Moreover, spontaneous symmetry breaking may occur at intermediate steps.}
    \label{fig:multiscale-eft}
\end{figure}

Ultimately one could end up at the QCD confinement scale~$\Lambda_\mathrm{QCD}$, where one matches onto chiral perturbation theory~(ChPT). However, due to confinement, in this case the IR~degrees of freedom of~ChPT, i.e. the pions, are not the same as in the EFT above~$\Lambda_\mathrm{QCD}$, where we have the quarks. Because of the growth of the strong coupling constant, perturbativity is lost and the matching has to be performed non-perturbatively. The discussion of this process is, however, beyond the scope of this review and we refer to \cite{Bernard:2007zu} for further details.

The BSM theory we started with could well be just an EFT itself, originating from integrating out even heavier particles in some more fundamental theory valid at even higher scales~$\Lambda_\mathrm{BSM}^\prime$. 
In this EFT picture we can simply consider any theory as an effective description at a given energy containing only the relevant degrees of freedom at these energies and incorporating our agnosticism about the laws of nature valid at higher energies in terms of higher-dimensional effective operators. 
Relating the different EFTs through matching and running as discussed above, we can express the couplings in some UV theory through the couplings/Wilson coefficients of a low-energy EFT valid at the scale of experiments; therefore, allowing us to constrain these UV parameters from low-energy data.

Describing nature by a chain of EFTs is, by construction, always an approximation. However, it can be systematically improved by including higher orders in the EFT power counting, i.e., higher-dimensional operators, thereby allowing, in principle, to describe physical laws up to arbitrary precision. In practice, usually only the leading contributions given by the dimension-six operators and their interference with the~SM are relevant. 
Nevertheless, higher orders, such as dimension-six squared or $d\geq 8$~operators, can be relevant in the case where they introduce new interactions that are not generated by the leading order \cite{Hays:2018zze,Corbett:2021eux} or when they exhibit some energy enhancement as in the case of high-$p_T$ Drell-Yan tails that we will discuss in Sec.~\ref{sec:Drell-Yan}. 
For a discussion of higher-order operators, and the EFT validity see also Sec.~\ref{sec:d8-operators} and~\ref{subsec:SMEFT_Constraints}.

Any SMEFT computation always involves a double expansion in the EFT power-counting order and the loop order, and the truncation of both has to be chosen individually according to the desired precision and the process of interest.\footnote{Contrary to that, in the HEFT the two expansions are linked together and cannot be truncated individually.} Going to the one-loop level in the matching computation might be especially required in cases where the operators of interest are not generated at the tree level, i.e., for all loop generated operators (see Sec.~\ref{sec:tree-loop-generated-operators}). In principle,  consistency in the expansion parameters should be obeyed. For instance, a one-loop matching computation would require two-loop running to obtain scheme independent results. However, in practice nearly always only the one-loop RG equations are considered as the full two-loop SMEFT RG equations are not yet known and only partial results are available. Also, usually one-loop running suffices for the required precision and one-loop matching computations are only required for operators that cannot be generated at the tree level.


Depending on the question we ask, the EFT analysis starts at the top or bottom of the tower of EFTs shown in Fig.~\ref{fig:multiscale-eft}, which we already denoted as top-down or bottom-up studies.
In the top-down approach one starts with a given BSM theory and matches it to the SMEFT and consecutively to the LEFT and so on, to determine the implications of this given theory at low energies. The ultimate goal is usually to constrain specific parameters of the BSM theory from suitable low-energy measurements. The strong advantage of using EFTs in this approach is the simplification of the initial problem and the resummation of large logarithms.

In the bottom-up approach, on the other hand, the idea is trying to be agnostic about the UV completion of the SM. In this case, one can try to use many low-energy data sets to constrain a large number of SMEFT Wilson coefficients. In principle, the goal of this approach is to perform a global fit to determine all SMEFT parameters. In practice, due to the large number of free parameters, such a fit is (currently) unfeasible. These types of analyses only consider a certain subset of parameters
based either on general dynamical hypotheses and/or symmetry assumptions, such as the flavor symmetries discussed in Sec.~\ref{sec:GlobalSymmetries}.
Eventually, constraints provided in this manner should still be used to constrain different BSM scenarios that just have to be matched to the SMEFT, rather than having to perform the full analysis for every model individually.

Sometimes there is also no clear distinction between the two approaches. 
A~key point worth stressing is that performing an EFT analysis is especially useful if there is a signal for new physics. Without such a signal, one can only put constraints on the huge parameter space of the SMEFT, gaining limited knowledge about the underlying structure of new physics. This is particularly problematic as not all the SMEFT parameters are equally relevant for experimental observations in different BSM theories. 
In absence of a new-physics signal, the constraints are more efficiently expressed as direct bounds on possible deviations from the SM for a given set of observables. 
In this respect, an interesting approach is that of {\em pseudo-observables}~\cite{Bardin:1999gt,Passarino:2010qk}
i.e.~the definition of a suitable set of  on-shell
amplitudes unambiguously connected to measurable quantities,  
able to characterize in general terms deviations from the SM of short-distance 
origin. This approach, originally introduced to describe electroweak precision tests at the $Z$~pole \cite{Bardin:1999gt}, and later extended also to Higgs physics \cite{David:2015waa,Ghezzi:2015vva, Gonzalez-Alonso:2014eva, Greljo:2015sla,Passarino:2010qk}, can be viewed as an intermediate, consistent, and economical step between measurements and their possible EFT interpretation. 
Of course, also in absence of deviations from the SM the EFT interpretation of data provides a useful guiding principle in model building, but the strong power of the EFT approach in predicting new effects and testing the validity of a given BSM hypothesis cannot be exploited.

In the following we will discuss all the steps involved in a new-physics SMEFT analysis. We also analyze in detail one example of a top-down and bottom-up analysis each, for illustration.

% - - - - - - -
\subsection{Matching BSM models to the SMEFT}
\label{sec:matching}
The matching of a BSM theory to the SMEFT is quite involved and there are several variants. To expose the important points, we will consider the matching procedure in the context of EFTs in general, as similar computations are required in all the matching steps illustrated in Fig.~\ref{fig:multiscale-eft}.
By matching a given UV theory to its corresponding low-energy EFT, we fix the value of all the Wilson coefficients of the EFT such that both theories reproduce the same physics in the low-energy limit. Therefore, the Wilson coefficients have to be determined as functions of the UV parameters, and constraints on the EFT coefficients can be directly translated into bounds on the BSM parameters. 
There are two different approaches that allow us to ensure that two theories describe the same physics at low energies, known as \textit{off-shell} and \textit{on-shell} matching. 

The most restrictive requirement we can enforce is that the effective action~$\Gamma$ of both theories, taken as a function of the light fields~$\phi$ only, agrees 
\begin{align}
\Gamma_\mathrm{UV}[\phi]=\Gamma_\mathrm{EFT}[\phi] \,.
\label{eq:matching-condition-general}
\end{align}
This ensures that all off-shell amplitudes with light external particles agree in both theories at low energies. Therefore, this method is commonly known as \textit{off-shell} matching. The matching condition~\eqref{eq:matching-condition-general} then determines the value of all Wilson coefficient in terms of the UV parameters. 
One way to determine the effective action is by calculating all relevant Feynman diagrams with external light fields. Contrary to the usual computation of the effective action, where we need to consider all one-particle-irreducible~(1PI) diagrams, it is required in this case to compute all one-light-particle-irreducible~(1LPI) Feynman diagrams, i.e., the diagrams that cannot be split in two by cutting any light internal line. 
This is because we consider the effective action as a function of light fields only. 
More on this ``diagrammatic matching procedure''  is provided in Sec.~\ref{sec:diagrammatic-matching}. 
An alternative approach to calculating the effective action is through its path integral representation, which we discuss afterwards in Sec.~\ref{sec:functional-matching}.



Requiring off-shell amplitudes to agree is actually more restrictive than necessary. It suffices to ensure that all physical observables computed in either theory agree, which is equivalent to equating the $S$-matrices of UV and EFT for all scattering processes with only light particles in the external states:
\begin{align}
    \langle\phi| S_\mathrm{UV} |\phi\rangle = \langle\phi| S_\mathrm{EFT} |\phi\rangle \,.
\end{align}
This amounts to equating on-shell amplitudes and is therefore known as \textit{on-shell} matching. We can again compute the $S$-matrix diagrammatically, but contrary to the off-shell matching computation we now need to consider all contributing diagrams and not only the 1LPI ones. This can significantly increase the number of diagrams that need to be included in the computation. Also the matching of the reducible diagrams is computationally more challenging, which is why in practice mostly off-shell matching is used.

The main advantage of on-shell matching is that it suffices to consider EFT operators from a minimal basis for the matching computation.
This is not the case for off-shell matching, since for off-shell amplitudes further kinematical structures are allowed, and therefore additional operators need to be included. When discussing the construction of EFT bases in Sec.~\ref{sec:redundant-operators}, we used field redefinitions to reduce the operator list to a minimal basis. Recall that to do so, we argued that the LSZ-formula guarantees that physical observables remain unchanged under field redefinitions. For off-shell matching, however, we do not compute physical observables. Therefore, we are not allowed to use the LSZ-formula and have to use a larger set of operators, that is a basis up to field redefinitions. Such an operator set is commonly referred to as a Green's basis.
We will discuss the diagrammatic off-shell matching procedure with a concrete example and briefly mention the differences to the on-shell computation.
Afterwards, we provide a short introduction to functional matching.

Before we continue with the different matching prescriptions, a comment on the calculation of the loop integrals is in order.
Because loop integrals in the matching can depend on light~$m$ and heavy~$M$ scales/masses,
certain regions of the internal momentum~$k$ of the loop are enhanced. This is encapsulated in the method of \textit{expansion by regions}~\cite{Beneke:1997zp,Jantzen:2011nz} which can be applied in all of the previously mentioned matching techniques.
We can expand each integrand in the region where~$k$ is hard $(k \sim M \gg m)$, and where it is soft $(k\sim m \ll M)$. 
Then performing both integrals separately over the full $D$-dimensional space and summing their results yields the same outcome as computing the full integral and expanding afterwards in powers of~$m/M$. 
Applying the method of regions offers the advantage of separating the UV from the IR physics. 
By construction, the full theory and its corresponding EFT describe the same low-energy dynamics, i.e., IR physics. 
This means that the soft region of the full theory integrals must be equal to the soft region of the corresponding EFT diagrams. 
Thus, we only have to compute the hard region of all integrals to determine the matching conditions, since these incorporate all of the short distance dynamics that must be captured by the Wilson coefficients. 
Next, we notice that the hard region of EFT integrals only yields scaleless integrals, which vanish in dimensional regularization, since the EFT only depends analytically on~$M$. Therefore, we only need to consider tree-level EFT diagrams with insertions of one-loop coefficients and full theory diagrams with at least one heavy propagator in the loop to obtain the full matching conditions. See Appendix~\ref{app:method-of-regions} and \cite{Manohar:2018aog} for more details.


\subsubsection{Diagrammatic matching} \label{sec:diagrammatic-matching}
In order to demonstrate the diagrammatic matching procedure, we work with a concrete example. Consider the extension of the SM by a heavy colored scalar~$S_1$ transforming as~$(\boldsymbol{\bar{3}},\boldsymbol{1})_{1/3}$ under the SM gauge group. The $S_1$~leptoquark couples to both quarks and leptons (thus its name), and its BSM Lagrangian is
\begin{align}
\mathcal{L}_{S_1} \! = \mathcal{L}_\mathrm{SM} &+ (D_\mu S_1)^\dagger (D^\mu S_1) - M_{S}^2 S_1^\dagger S_1
\label{eq:S1}\\
&-\left[ \lambda_{pr}^L (\overline{q}^c_p \varepsilon \ell_r) S_1 + \lambda_{pr}^R (\overline{u}^c_p e_r) S_1 + \mathrm{h.c.} \right] \,, \nonumber
\end{align}
where we neglect any direct coupling of the $S_1$ to the Higgs doublet~$H$. 
For the off-shell matching we need to consider a Green's basis for the SMEFT, i.e., a basis of operators up to field redefinitions. Such basis is given in \cite{Gherardi:2020det}, where also the full matching computation for the given model is presented. Here, we will only reproduce partial results of this derivation to illustrate the procedure.

First, we realize that for tree-level matching only interaction terms of the Lagrangian~\eqref{eq:S1} with at most one heavy field can contribute. 
Operators with more heavy fields can only contribute at loop level. 
It is then obvious that only four-fermion operators are generated in the EFT at tree level. 
The resulting EFT Lagrangian is
\begin{align}
    \mathcal{L}_\mathrm{EFT} &= \mathcal{L}_\mathrm{SM} + \sum_X \frac{[C_X^{(R)}]_{prst}}{\Lambda^2} [R_X]_{prst} \,,
\end{align}
where the sum runs over the ``redundant'' operators
\begin{subequations}
\begin{align}
    [R_{q^c l}]_{prst} &= (\overline{q}^c_{i p} \ell_{j r}) (\overline{\ell}{}_{s}^j {q^c_{t}}^i) \,,
    \\
    [R_{q^c l}^\prime]_{prst} &= (\overline{q}^c_{i p} \ell_{j r}) (\overline{\ell}{}_{s}^i {q^c_{t}}^j) \,,
    \\
    [R_{e^c u}]_{prst} &= (\overline{e}^c_p u_r) (\overline{u}_s e_t^c) \,,
    \\
    [R_{u^c e l q^c}]_{prst} &= (\overline{u}^c_p e_r) \varepsilon_{ij} (\overline{\ell}{}_{s}^i {q_{t}^c}^j)
\end{align}%
\label{eq:operators_S1}%
\end{subequations}%
defined following the conventions in~\cite{Fuentes-Martin:2022vvu}.
The Feynman diagrams in the UV and the EFT relevant to the matching are shown on the left- and right-hand side of Fig.~\ref{fig:S1-tree-matching}, respectively. Inserting the appropriate interaction terms from the UV and EFT Lagrangian we can compute the corresponding amplitudes. Afterwards, we expand the UV amplitudes as power series in~$1/M_{S}$. Here we work to mass-dimension six, thus we need to truncate the results at~$\mathcal{O}(M_{S}^{-2})$. We can then equate the amplitudes of both EFT and UV to find the matching conditions
\begin{subequations}%
\begin{align}
    [C_{q^c l}^{(R)}]_{prst} &= -[C_{q^c l}^{\prime\,(R)}]_{prst} = \lambda^{L}_{p r} \lambda^{L\ast}_{t s} \,,
    \\
    [C_{e^c u}^{(R)}]_{prst} &= \lambda^{R}_{r p} \lambda^{R\ast}_{s t} \,,
    \\
    [C_{u^c e l q^c}^{(R)}]_{prst} &= -\lambda^{R}_{p r} \lambda^{L\ast}_{t s} \,,
\end{align}%
\end{subequations}%
where we also identify the new-physics scale~$\Lambda$ with the mass of the $S_1$~state: $\Lambda = M_S$.

\begin{figure}[t]
    \centering
    \includegraphics[width=0.80\linewidth]{figures/S1-matching-tree_fig11.pdf}
    \caption{Tree-level Feynman diagrams for the $S_1$~model (left) and the SMEFT (right), that are relevant to the matching of the four-fermion operators. The UV diagram has to be expanded in powers of $1/M_{S}$ before equating it with the EFT diagram.}
    \label{fig:S1-tree-matching}
\end{figure}

For convenience, we want to rewrite our result in the Warsaw basis. The operators in Eq.~\eqref{eq:operators_S1} are related to the operators $\smash{Q_{lq}^{(1,3)}}$, $\smash{Q_{eu}}$, and $\smash{Q_{lequ}^{(1,3)}}$ of the Warsaw basis through the Fierz transformations~\eqref{eq:Fierz-ids}. The matching conditions in the Warsaw basis read
\begin{subequations}
\begin{align}
    [C_{lq}^{(1)}]_{prst} &= \frac{1}{2} [C_{q^c l}^{(R)}]_{trps} + \frac{1}{4} [C_{q^c l}^{\prime\,(R)}]_{trps} \,,
    \\
    [C_{lq}^{(3)}]_{prst} &= \frac{1}{4} [C_{q^c l}^{\prime\,(R)}]_{trps} \,,
    \\
    [C_{eu}]_{prst} &= \frac{1}{2} [C_{e^c u}^{(R)}]_{rtsp} \,,
    \\
    [C_{lequ}^{(1)}]_{prst} &= -4 [C_{lequ}^{(3)}]_{prst} = -\frac{1}{2} [C_{u^c e l q^c}^{(R)}]_{trps} \,,
\end{align}%
\label{eq:S1-Fierz}%
\end{subequations}%
where evanescent operators can be ignored since we work at tree level. In the present case there are no integration by parts relations or field redefinitions required to reduce the matching result to the Warsaw basis.


Next, we want to perform the one-loop matching. Since the entire computations is rather lengthy and the full results are shown in~\cite{Gherardi:2020det}, we focus here only on the contributions to the leptonic dipole operators 
\begin{align}
    [Q_{eB}]_{pr} &= (\overline{\ell}_p \sigma^{\mu\nu} e_r) H B_{\mu\nu} \,,
    \label{eq:QeB}
    \\
    [Q_{eW}]_{pr} &= (\overline{\ell}_p \sigma^{\mu\nu} e_r) \tau^I H W_{\mu\nu}^I \,,
    \label{eq:QeW}
\end{align}
which can only be generated at loop level. 
We choose to use on-shell matching, which allows us to single out the dipole matching contributions, for which the relevant diagrams are shown in Fig.~\ref{fig:S1-loop}. 
The first four rows display the diagrams of the UV theory, whereas the diagram in the last row is the only EFT diagram. Recall that, since we employ the method of regions, we only have to consider EFT tree diagrams with one-loop coefficients, but no loop diagrams. 
After expanding in the hard loop-momentum region and performing the Dirac algebra, including the application of the spinor equations of motions and the Gordon identity, the diagrams in the third and fourth row provide a local contribution to the leptonic dipole operators.
If we had chosen off-shell matching instead, we had to consider further topologies that do not directly match onto the dipole, but do contribute to it only after applying field redefinitions to reduce the Green's basis to the Warsaw basis. 
However, it is not straight forward to identify which topologies to consider, which is why we choose to work out this explicit contribution on-shell.\footnote{In the case of off-shell matching we could also neglect the diagrams in the third and fourth row of Fig.~\ref{fig:S1-loop}, since we only have to consider 1LPI diagrams. Their contribution would be shifted to the additional operators in the Green's basis that reduce to the dipole by applying field redefinitions, such that the final results of both methods agree.}


\begin{figure}[t]
    \centering
    \includegraphics[height=1.9cm]{figures/S1-matching-loop_fig12a.pdf}
    \includegraphics[height=1.9cm]{figures/S1-matching-loop_fig12b.pdf}
    \\[0.1cm]
    \includegraphics[height=1.9cm]{figures/S1-matching-loop_fig12c.pdf}
    \\[0.1cm]
    \includegraphics[height=1.9cm]{figures/S1-matching-loop_fig12d.pdf}
    \includegraphics[height=1.9cm]{figures/S1-matching-loop_fig12e.pdf}
    \\[0.1cm]
    \includegraphics[height=1.9cm]{figures/S1-matching-loop_fig12f.pdf}
    \includegraphics[height=1.9cm]{figures/S1-matching-loop_fig12g.pdf}
    \\[0.2cm]
    \includegraphics[height=2cm]{figures/S1-matching-loop-EFT_fig12h.pdf}
    \caption{One-loop diagrams relevant to the on-shell matching of the leptonic dipole operators for the $S_1$~model. The first four rows show the diagrams of the UV theory, whereas the last row contains the only EFT diagram when using the method of regions.}
    \label{fig:S1-loop}
\end{figure}

Computing and equating the amplitudes corresponding to the diagrams shown in Fig.~\ref{fig:S1-loop}, where for the UV amplitudes we only keep the terms with the Lorentz structure matching that of the dipole, since the remaining terms will match onto other operators of the Warsaw basis that we are not interested in, we find at~$\mathcal{O}(M_{S}^{-2})$
\begin{align}
    \label{eq:S1-CeB}
    [C_{eB}]_{pr}
    &= \frac{1}{16\pi^2} \frac{g_1}{8} \Bigg\{ -[Y_e]_{pt} \lambda_{st}^{R\ast} \lambda^R_{sr}  \\
    &+ \lambda^{L\ast}_{sp} [Y_u^\ast]_{st} \lambda^R_{tr} \left[\frac{19}{2}+5\log\!\left(\!\frac{\mu_m^2}{M_{S}^2}\!\right)\!\right] \!\!\Bigg\} + [\Delta_{eB}]_{pr}  ,
    \nonumber\\[0.1cm]
    \label{eq:S1-CeW}
    [C_{eW}]_{pr}
    &= \frac{1}{16\pi^2} \frac{g_2}{8} \Bigg\{ \lambda_{sp}^{L\ast} \lambda^L_{st} [Y_e]_{tr} \\
    &- 3 \lambda^{L\ast}_{sp} [Y_u^\ast]_{st} \lambda^R_{tr} \left[\frac{3}{2}+\log\!\left(\!\frac{\mu_m^2}{M_{S}^2}\!\right)\!\right] \!\!\Bigg\} + [\Delta_{eW}]_{pr} , \nonumber
\end{align}
where $\mu_m$ is the matching scale, which we can conveniently choose as $\mu_m = M_{S}$ to eliminate all logarithms in the matching conditions.\footnote{Other choices for~$\mu_m$ are possible, but it should not be chosen too far away from the mass threshold to avoid large logarithms and a worsening of the perturbative expansion.}

However, since we work at the one-loop level now, we can no longer use the Fierz identities applied in the tree-level matching in Eq.~\eqref{eq:S1-Fierz}. 
As stated before, these are intrinsically four-dimensional identities that must not be applied in combination with a computation in dimensional regularization, since they lead to evanescent operators. 
Instead, we apply the corrected \textit{one-loop Fierz transformations} as discussed in Sec.~\ref{sec:evanescent} and \cite{Fuentes-Martin:2022vvu}, which effectively project the evanescent operators onto the physical four-dimensional Warsaw basis, allowing us to ignore evanescent contributions afterwards. 
We focus on the dipole operator, whose additional contributions arising due to the evanescent operators generated by applying the $D=4$~Fierz identities in Eq.~\eqref{eq:S1-Fierz} are labeled by $\Delta_{eB/eW}$ in Eqs.~\eqref{eq:S1-CeB}--\eqref{eq:S1-CeW}. The corresponding shift of the action was derived in Sec.~\ref{sec:evanescent} and is given in Eq.~\eqref{eq:evanescent-dipole-shift} with which we find
\begin{align}
    [\Delta_{eB}]_{pr} &= \!- \frac{1}{16\pi^2} \frac{5}{8} g_1 [Y_u^\ast]_{ts} (1-\xi_\mathrm{rp}) [C_{u^c e l q^c}^{(R)}]_{srpt},
    \\
    [\Delta_{eW}]_{pr} &= \frac{1}{16\pi^2} \frac{3}{8} g_2 [Y_u^\ast]_{ts} (1-\xi_\mathrm{rp}) [C_{u^c e l q^c}^{(R)}]_{srpt},
\end{align}
where $\xi_\mathrm{rp}$ is the parameter denoting the reading point ambiguity when using NDR to evaluate the loop integrals (see Sec.~\ref{sec:evanescent} and Appendix~\ref{app:gamma5} for more details). 
For convenience, we decide to read all EFT loop integrals ending with the EFT operator as suggested by \textcite{Fuentes-Martin:2022vvu}.
As mentioned before, we then have to follow this prescription for all subsequent computations within the EFT to obtain consistent results
independent of the choice for the reading point.
The given prescription yields $\xi_\mathrm{rp}=1$ (which also agrees with the results obtained in the 't~Hooft--Veltman scheme) and thus the evanescent contributions to the dipole happens to vanish.
Nevertheless, there are additional non-vanishing and unambiguous evanescent contributions to further operators, that we do not consider here. 
For more details on the reading point ambiguity see Appendix~\ref{app:gamma5} and \cite{Fuentes-Martin:2022vvu}.


\subsubsection{Functional matching}\label{sec:functional-matching}
We now recap the functional formalism  
worked out in \cite{Dittmaier:1995ee,Henning:2014wua,delAguila:2016zcb,Henning:2016lyp,Fuentes-Martin:2016uol,Zhang:2016pja,Cohen:2020fcu},
and employ it for a specific matching computation.
We make use of the background-field method \cite{Abbott:1980hw,Abbott:1983zw,Denner:1994xt,Denner:1996wn} by separating all fields~$\eta \to \hat\eta + \eta$ in a background field configuration~$\hat\eta$, which satisfies the classical equations of motion, and a pure quantum component~$\eta$. 
In Feynman diagrams $\hat\eta$ then corresponds to tree-level lines, whereas $\eta$ corresponds to lines in loops.
Expanding the action to one-loop accuracy we find
\begin{align}
	S[\hat\eta + \eta] &=S[\hat\eta] + \frac{\sigma_{\eta_j}}{2} \bar{\eta}_i \left. \frac{\delta^2 S}{\delta\bar\eta_i \delta\eta_j} \right|_{\eta=\hat\eta} \eta_j + \mathcal{O}(\eta^3) \,,
\end{align}
where $\sigma_{\eta_j}= 1$ if $\eta_j$ is bosonic and $\sigma_{\eta_j}=-1$ if it is Grassmann due to anti-commuting $\eta_j$ to the right-hand side of the above equation.
The linear term vanishes due to the equations of motion, and higher-order terms only contribute at two-loop order and beyond. We identify the term quadratic in the quantum fields as the \textit{fluctuation operator}
\begin{align}
    \Omega_{ij}[\hat\eta] &= \left. \sigma_{\eta_j} \frac{\delta^2 S}{\delta\bar\eta_i \, \delta\eta_j} \right|_{\eta=\hat\eta} \,.
    \label{eq:fluctuation_operator}
\end{align}
The effective action of the theory is then given by
\begin{align}
    \exp \! \big( i\Gamma[\hat\eta] \big) \! &= \!\! \int\!\!\mathcal{D}\eta \, \exp\!\left( \! i S[\hat\eta] + \frac{i}{2} \bar{\eta}_i \, \Omega_{ij}[\hat\eta] \, \eta_j + \mathcal{O}(\eta^3) \!\right) \,.
\end{align}
Thus, we find the tree-level effective action $\Gamma^{(0)}[\hat\eta]=S^{(0)}[\hat\eta]$ and the one-loop effective action
\begin{align}
\begin{split}
    \Gamma^{(1)}[\hat\eta] &= S^{(1)}[\hat{\eta}] - i\log \left(\mathrm{SDet}\,\Omega^{(0)}[\hat\eta]\right)^{-1/2} 
    \\
    &= S^{(1)}[\hat{\eta}] + \frac{i}{2} \mathrm{STr} \log \Omega^{(0)}[\hat\eta] \,,
\end{split}
\end{align}
where $S^{(1)}[\hat{\eta}]$ contains all local one-loop contributions; that is, in renormalizable theories $S^{(1)}[\hat{\eta}]$ contains only the counterterms required to renormalize the theory.
In EFTs the one-loop induced Wilson coefficients are included in~$S^{(1)}[\hat{\eta}]$, too.
We furthermore introduce the superdeterminant~($\mathrm{SDet}$) and the supertrace~($\mathrm{STr}$), which are generalizations of the determinant and trace to operators with mixed spin. The supertrace is a trace over all internal degrees of freedom and therefore involves an integration over all loop momenta
\begin{align}
    \mathrm{STr} \, \log \Omega[\hat\eta] &= \pm \int \frac{\mathrm{d}^D k}{(2\pi)^D} \langle k | \mathrm{tr} \, \log \Omega[\hat\eta] | k \rangle \,,
\end{align}
where $\mathrm{tr}$ denotes the regular trace over all internal degrees of freedom apart from momentum, and the sign depends on the spin of the considered field, with $+~(-)$ for bosonic (fermionic) states.
These operator traces can be evaluated using the so-called \textit{covariant derivative expansion} \cite{Gaillard:1985uh,Chan:1986jq,Cheyette:1987qz}. However, a~discussion of the supertrace evaluation is beyond the scope of this review and we refer to~\cite{Henning:2014wua,Cohen:2020fcu,Fuentes-Martin:2020udw} for further details.

Since the path integral formulation allows to compute the effective action, we can also use it to calculate the off-shell matching condition in Eq.~\eqref{eq:matching-condition-general}. At tree level we find
\begin{align}
    S^{(0)}_\mathrm{EFT}[\hat\eta_L] &= S^{(0)}_\mathrm{UV}[\hat\eta_L,\hat\eta_H] \,,
\end{align}
where we separated the fields into light~$\hat\eta_L$ and heavy~$\hat\eta_H$. The heavy background fields are understood as the solution to their equation of motion as a power series in~$1/M$, where $M$~is their mass, so that they can be entirely expressed in terms of the light fields $\hat\eta_H=\hat\eta_H[\hat\eta_L]$. 

Taking the Lagrangian~\eqref{eq:S1} of the $S_1$~leptoquark example, we find the equation of motion for~$S_1$
\begin{align}
    D^2 S_1 + M_{S}^2 S_1 - \lambda^{L\ast}_{pr} (\overline{\ell}_r \varepsilon q_p^c) + \lambda^{R\ast}_{pr} (\overline{e}_r u_p^c) &= 0 \,.
\end{align}
Its power series solution is given by
\begin{align}
    S_1 &= \frac{1}{M_{S}^2} \left[ \lambda^{L\ast}_{pr} (\overline{\ell}_r \varepsilon q_p^c) - \lambda^{R\ast}_{pr} (\overline{e}_r u_p^c) \right] + \mathcal{O}(M_{S}^{-4}) \,.
\end{align}
Substituting this solution back into the Lagrangian~\eqref{eq:S1} yields the same matching condition as in the diagrammatic computation shown in Eq.~\eqref{eq:operators_S1}. 

For the one-loop matching it is convenient to split the fluctuation operator into a kinetic and an interaction term
\begin{align}
    \Omega_{ij} &\equiv \delta_{ij} \Delta_i^{-1} - X_{ij} ,
    \ \text{with} \ 
    \Delta_i^{-1} = \left\{ \begin{matrix} -(D^2 + M_i^2) \\ i\slashed{D} - M_i \\ g^{\mu\nu} (D^2 + M_i^2) \end{matrix} \right. ,
\end{align}
for scalars, fermions, and vector bosons, respectively.
For simplicity we use the Feynman gauge for the quantum fluctuations of the gauge fields. 
This does not imply any particular choice for the gauge of the background fields, which remain in the general $R_\xi$~gauge \cite{Henning:2014wua}. For more details on gauge fixing the SMEFT in the background field method see \cite{Helset:2018fgq}. 
The interaction terms~$X_{ij}$ are implicitly defined by the above equation. This allows to write the one-loop effective action of the UV theory as
\begin{align}
    \Gamma_\mathrm{UV}^{(1)} &= \frac{i}{2} \mathrm{STr} \log \Delta^{-1} + \frac{i}{2} \mathrm{STr} \log (1-\Delta X) \,.
\end{align}
We can again apply the method of regions splitting $\Gamma_\mathrm{UV}^{(1)}$ into a hard and soft part, which are computed by expanding the loop integrands in the soft or hard momentum region, respectively. By construction we have $\Gamma_\mathrm{EFT}^{(1)}|_\mathrm{soft} = \Gamma_\mathrm{UV}^{(1)}|_\mathrm{soft}$, which ensures that both theories describe the same long distance dynamics. Therefore, we find the one-loop EFT Lagrangian to be given by $\int \mathrm{d}^ D x\,\mathcal{L}_\mathrm{EFT}^{(1)} = \Gamma_\mathrm{UV}^{(1)}|_\mathrm{hard}$, and thus
\begin{align}
    \int \!\!\mathrm{d}^D x\,\mathcal{L}_\mathrm{EFT}^{(1)} 
    &= \!\frac{i}{2} \mathrm{STr} \log \Delta^{-1} \bigg|_\mathrm{hard} \!\!\!\!\! + \!\frac{i}{2} \sum_{n=0}^\infty \frac{1}{n} \mathrm{STr} (\Delta X)^n \bigg|_\mathrm{hard}
    \label{eq:matching-master-formula}
\end{align}
where we expanded the logarithm in the latter term. This is the master formula for functional one-loop matching, expressing the EFT Lagrangian in terms of \textit{log-type} and \textit{power-type} supertraces. These can be evaluated using the covariant derivative expansion as discussed in \cite{Cohen:2020fcu,Fuentes-Martin:2020udw}. 
The main advantage of the functional formalism is that Eq.~\eqref{eq:matching-master-formula} directly yields all generated EFT operators and, unlike the diagrammatic approach, no a~priori knowledge of an operator basis is required. However, the Lagrangian obtained by Eq.~\eqref{eq:matching-master-formula} is in a non-minimal form, and redundant operator need to be removed to recover the EFT in a minimal basis.

A~computation for the $S_1$ example discussed before using functional methods is rather tedious and thus not discussed here, but further details can be found in \cite{Fuentes-Martin:2020udw,Dedes:2021abc}.\footnote{For the application of the functional matching formalism to other simple BSM theories see, e.g., \cite{Zhang:2021jdf,Li:2022ipc,Liao:2022cwh,Dittmaier:2021fls,Du:2022vso}.} 
However, it is a purely algebraic problem that can be solved by a computer. 
The Mathematica package \cmd{Matchete} \cite{Fuentes-Martin:2022jrf} is the first tool that fully automatizes the functional one-loop matching.\footnote{Earlier codes such as \cmd{STrEAM} \cite{Cohen:2020qvb} and \cmd{SuperTracer} \cite{Fuentes-Martin:2020udw} allow only to compute the supertraces, but do not perform the full matching computation. In particular they do not perform operator reductions on the resulting EFT Lagrangian. See also \cmd{MatchingTools} \cite{Criado:2017khh} for a pure tree-level matching implementation.}
Previously, the diagrammatic one-loop matching technique was already automated in the \cmd{MatchMakerEFT} \cite{Carmona:2021xtq} tool.
This greatly simplifies phenomenological BSM analyses and gives the possibility of validating matching results with different methods.
Another tool for one-loop matching is \cmd{CoDEx} \cite{DasBakshi:2018vni} using the universal one-loop effective action~(UOLEA) \cite{Drozd:2015rsp,Ellis:2016enq,Ellis:2017jns,Ellis:2020ivx,Kramer:2019fwz}, which is also based on the path integral approach explained above. A~more detailed discussion of the UOLEA technique is, however, beyond the scope of this review.


\subsection{Renormalization group evolution}
\label{sec:rge}
The Wilson coefficients of the SMEFT Lagrangian obtained from the matching are related to the UV parameters at the matching scale~$\mu_m$, usually taken at the mass threshold $\mu_m \sim M$. 
Next, we have to evolve the coefficients down to the electroweak scale~$(\sim \! m_W)$
using the SMEFT RG~equations. 
These have been computed at one loop for the dimension-six operators of the Warsaw basis shown in Tab.~\ref{tab:Warsaw-basis} in \cite{Jenkins:2013zja,Jenkins:2013wua,Alonso:2013hga}.
The RG equations of the baryon- and lepton-number violating operators listed in Tab.~\ref{Tab:Warsaw-basis_BV} have been derived in \cite{Alonso:2014zka} also including operators with right-handed neutrinos. The RG~equations for the dimension-five and -seven operators have been derived in \cite{Babu:1993qv,Davidson:2018zuo,Liao:2016hru,Liao:2019tep}, whereas for dimension eight only partial results are yet available \cite{Chala:2021pll,DasBakshi:2022mwk}. 
Results for specific sectors of 
the two-loop anomalous dimension matrix 
have been derived in \cite{Aebischer:2022anv,Bern:2020ikv}.
For some recent phenomenological analyses of the SMEFT RG~mixing effects see, e.g., \cite{Chala:2021juk,Kumar:2021yod,Isidori:2021gqe,Aoude:2022aro}. 
A~careful analysis of the flavor structure of the 2499-by-2499 anomalous-dimension matrix of the SMEFT is presented in \cite{Machado:2022ozb}.

An important feature of the RG~evolution is the mixing of different operator classes. 
In particular, an operator that is not generated by the matching can obtain a non-vanishing coefficient through the running. 
This leads to non-trivial relations among different operator types, that need to be carefully considered in a phenomenological analysis.


As an example, we consider the RG~evolution of the leptonic dipole operators in Eq.~\eqref{eq:QeB} and~\eqref{eq:QeW}, and the Yukawa interactions, that are described by
\begin{align}
    \mu\frac{\mathrm{d}}{\mathrm{d}\mu} [C_{X}]_{pr} &= \frac{1}{16\pi^2} [\beta_X]_{pr}
\end{align}
with the beta-functions given by
\begin{subequations}
\begin{align}
    [\beta_{eB}]_{pr} 
    &=  3 |y_t|^2 [C_{eB}]_{pr} - 10 g_1 y_t^\ast [C_{lequ}^{(3)}]_{pr33} \,, 
    \label{eq:CeB-RGE}
    \\
    [\beta_{eW}]_{pr} 
    &= 3 |y_t|^2 [C_{eW}]_{pr} + 6 g_2 y_t^\ast [C_{lequ}^{(3)}]_{pr33} \,,
    \label{eq:CeW-RGE}
    \\
    [\beta_{Y_e}]_{pr}
    &= 3 \lambda \frac{v^2}{\Lambda^2} \left( [C_{eH}]_{pr} - y_t^\ast [C_{lequ}^{(1)}]_{pr33} \right) \approx 0 \,,
    \label{eq:Y-RGE}
    \\
    [\beta_{eH}]_{pr}
    &= 9 |y_t|^2 [C_{eH}]_{pr} + 12 y_t^\ast |y_t|^2 [C_{lequ}^{(1)}]_{pr33} \,,
    \label{eq:CeH-RGE}
\end{align}%
\label{eq:SMEFT-beta-functions}%
\end{subequations}%
where for simplicity we only keep numerically relevant terms, i.e., top Yukawa~$(y_t)$ enhanced terms that are not multiplied by~$\lambda$.
Thus, we can write the Wilson coefficients at a low scale~$\mu_l$, in terms of the coefficients at the matching scale~$\mu_m$ with one-loop accuracy as
\begin{align}
    [C_{X}]_{pr} (\mu_l) &= [C_{X}]_{pr} (\mu_m) + \frac{1}{16\pi^2} \log\left(\frac{\mu_l}{\mu_m}\right) [\beta_{X}]_{pr} \,.
    \label{eq:SMEFT-one-loop-running}
\end{align}

The RG~evolution of the Warsaw basis operators is also automated in computer programs such as \cmd{DSixTools} \cite{Celis:2017hod,Fuentes-Martin:2020zaz} and \cmd{Wilson} \cite{Aebischer:2018bkb}, making a phenomenological analysis using the full 2499-by-2499 anomalous-dimension matrix of the $d=6$~SMEFT feasible. 


\subsection{Low-energy constraints in the LEFT}
\label{sec:g-2_SMEFT}

Having discussed the matching of the BSM model defined in Eq.~\eqref{eq:S1}
onto the dipole operators $Q_{eB}$ and $Q_{eW}$, we now relate these to the photon dipole operator
\begin{align}
    [\mathcal{Q}_{e\gamma}]_{pr} = \frac{v}{\sqrt{2}} \overline{e}_p^L \sigma^{\mu\nu} e_r^R F_{\mu\nu} \,.
    \label{eq:photon-dipole}
\end{align} 
This allows us to illustrate how the low-energy constraints on this effective operators can be used 
for constraining the high-energy couplings of the $S_1$~field.

To this end, we write the SMEFT Lagrangian in the broken phase\footnote{Notice that for convenience we use here a different definition for the Yukawa and mass matrices compared to Eq.~\eqref{eq:LEFT-mass-Yukawa} . Moreover, for the dipole operators we directly apply the SMEFT instead of the LEFT power counting.}
\begin{align}
    \Delta\mathcal{L}^\mathrm{broken} 
    = &-[\mathcal{Y}_e]_{pr} \frac{v}{\sqrt 2} (\bar e_{p}^L e_{r}^R )
- [\mathcal{Y}_{he}]_{pr} \frac{h}{\sqrt{2}} (\bar e_{p}^L e_{r}^R )  
\nonumber\\
&+ \frac{[\mathcal{C}_{e\gamma}]_{pr}}{\Lambda^2} \frac{v}{\sqrt 2} (\bar e_{p}^L \sigma^{\mu\nu} e_{r}^R) F_{\mu\nu}
\\
&+ \frac{[\mathcal{C}_{eZ}]_{pr}}{\Lambda^2} \frac{v}{\sqrt 2} (\bar e_{p}^L \sigma^{\mu\nu} e_{r}^R) Z_{\mu\nu} + \ldots
\nonumber
\end{align}
Here, we also included the mass term, the Yukawa, and the $Z$-boson dipole, where the latter two are phenomenologically not relevant for the present analysis.

Assuming that new physics is not affecting the electroweak symmetry breaking pattern, 
i.e.~assuming the relations between quantities in the broken and unbroken phase are the same as in the SM (e.g. $\overline{g}_1=g_1$, $\overline{s}_\theta=s_\theta$, $v_T=v$~\ldots),
we can use the results presented in Sec.~\ref{sect:LEFT}
to relate the coefficients of the broken phase Lagrangian to the ones of the unbroken phase by
\begin{align}
\begin{pmatrix}
	 [\mathcal{C}_{e\gamma}]_{pr} \\[0.1cm]
     [\mathcal{C}_{eZ}]_{pr}
\end{pmatrix}
&=
\begin{pmatrix}
	c_\theta & -s_\theta \\[0.1cm]
	-s_\theta & -c_\theta
\end{pmatrix}
\begin{pmatrix}
	[C_{eB}]_{pr} \\[0.1cm]
	[C_{eW}]_{pr}
\end{pmatrix} \,, 
\label{eq:dipole-rotation}
%
\\[0.2cm]
%
\begin{pmatrix}
	[\mathcal{Y}_e]_{pr} \\[0.1cm]
	[\mathcal{Y}_{he}]_{pr}
\end{pmatrix} 
&=
\begin{pmatrix}
	1 & -\frac{1}{2} \\[0.1cm]
	1 & -\frac{3}{2}
\end{pmatrix}
\begin{pmatrix}
	[Y_e]_{pr} \\[0.1cm]
	\frac{v^2}{\Lambda^2}[C_{eH}]_{pr}
\end{pmatrix} \, ,
\label{eq:Yukawa-rotation}
\end{align}
where
\begin{align}
    c_\theta &= \frac{g_2}{\sqrt{g_1^2 + g_2^2}} = \frac{e}{g_1} \,, \quad  s_\theta = \frac{g_1}{\sqrt{g_1^2 + g_2^2}} = \frac{e}{g_2} \,.
\end{align}

We can now combine our results for the relations to the broken phase, shown in Eqs.~\eqref{eq:dipole-rotation} and~\eqref{eq:Yukawa-rotation}, with the RG~evolution equations above the electroweak scale in Eqs.~\eqref{eq:SMEFT-beta-functions} and~\eqref{eq:SMEFT-one-loop-running} to express the electromagnetic dipole and the mass Yukawa at the electroweak scale~$\mu_w$ in terms of the SMEFT Wilson coefficients at the new-physics/matching scale~$\mu_m\!\sim\!\Lambda$:
\begin{align}
\begin{split}
    [\mathcal{C}_{e\gamma}]_{pr} (\mu_w) 
    &= 
    \brackets{1 - 3 \hat{L} y_t^2 } [\mathcal{C}_{e\gamma}]_{pr} (\mu_m) 
    \\
    &\qquad + 16 \hat{L} y_t e \, [C_{lequ}^{(3)}]_{pr33} (\mu_m) \,,
\end{split}
\label{eq:physical_dipole_low}
\\[0.1cm]
% \begin{split}
    [\mathcal{Y}_e]_{pr} (\mu_w) 
    &= 
    \squarebrackets{Y_{e}}_{pr}(\mu_m) - \frac{v^2}{2 \Lambda^2} [C_{eH}]_{pr}(\mu_m) 
    \label{eq:physical_Yukawa_low}\\
    &\qquad + 6 \frac{v^2}{\Lambda^2} \hat{L} \left[ y_t^3 [C_{lequ}^{(1)}]_{pr33} +  \frac34 y_t^2 [C_{eH}]_{pr} \right]_{\mu_m} \!\!\!\! ,
    \nonumber
% \end{split}
\end{align}
where we assume the Yukawa couplings to be real, and we define $\hat{L}\equiv (1/16\pi^2) \log(\mu_m/\mu_w)$.
We find that the semileptonic triplet operator~$\smash{Q_{lequ}^{\scriptscriptstyle (3)}}$ can generate the electromagnetic dipole~$\mathcal{Q}_{e\gamma}$ at the low scale, whereas the semileptonic singlet operator~$\smash{Q_{lequ}^{\scriptscriptstyle (1)}}$ as well as~$Q_{eH}$ run into the mass terms~$\mathcal{Y}_e$.


We can now investigate the RG evolution below the electroweak scale, which is given by \cite{Jenkins:2017dyc}
\begin{align}
    \mu \frac{\dd}{\dd\mu} [\mathcal{C}_{e\gamma}]_{pr} &= \frac{1}{16\pi^2} \frac{170}{9} e^2 [\mathcal{C}_{e\gamma}]_{pr} \,,
    \label{eq:Cegamma-RGE}
    \\
    \mu \frac{\dd}{\dd\mu} [\mathcal{Y}_{e}]_{pr} &= -\frac{1}{16\pi^2} 6e^2 [\mathcal{Y}_{e}]_{pr} \,,
    \label{eq:Ye-RGE}
\end{align}
where we consider all other operators to be turned off,\footnote{For the SMEFT, the only numerically relevant contributions in the running are due to the $y_t$~enhanced terms. In the LEFT, however, the top quark is integrated out and top loops cannot contribute, thus no such RG effects are present below the electroweak scale.}
and thus we only have the self renormalization of the dipole and the mass term, which leave the flavor structure unchanged. 
Notice also that the LEFT dipole operator in Eq.~\eqref{eq:photon-dipole} is a dimension-five operator, thus, in principle, we had to consider double insertions of this operator for the RG~evolution. 
However, from the matching conditions~\eqref{eq:S1-CeB} and~\eqref{eq:S1-CeW} we know that such contribution is of order~$\mathcal{O}(M_{S}^{-4})$ in the SMEFT power counting and can thus be neglected.
Equation~\eqref{eq:Cegamma-RGE} then allows to evolve the photon dipole to the low-energy scales of experimental measurements, which for muons is~$\mu_l\sim m_\mu$. Notice that in the present case it is not required to integrate out any other particles, such as the $b$~quark, since these do not affect the RG~evolution in good approximation due to their small Yukawa couplings.

Experimental measurements (usually) constrain couplings in the mass basis, whereas our Wilson coefficients are given in the generic flavor basis of the UV theory. Thus, rotating the fermion fields to the mass basis is the last missing piece of our analysis. To do this, we need to diagonalize the mass matrix $[\mathcal{Y}_e]_{pr}$ which is determined in terms of the SMEFT operators in Eq.~\eqref{eq:Yukawa-rotation}. Assume the mass term is diagonalized $\smash{(\mathrm{diag.} = U_L \, \mathcal{Y}_e \, U_{R}^\dagger)}$ when rotating the lepton fields~by
\begin{align}
    e_L^\prime = U_L \, e_L \,, \qquad e_R^\prime = U_R \, e_R \,,
\end{align}
where $U_{L,R}$ are unitary matrices and $e_{L,R}^\prime$ denote the mass-basis fields. Then the mass-basis dipole~$\mathcal{C}_{e\gamma}^\prime$ is given~by
\begin{align}
    \mathcal{C}_{e\gamma}^\prime &= U_L \, \mathcal{C}_{e\gamma} \, U_R^\dagger \,.
    \label{eq:Cegamma-mass-basis-rotation}
\end{align}

The most sensitive probe of this operator is the lepton flavor violating transition $\mu \to e \gamma$; however, also the anomalous magnetic moment of the muon $(g-2)_\mu$ is interesting, especially given the tension of the recent FNAL measurement \cite{Muong-2:2021ojo} with the SM prediction by \textcite{Aoyama:2020ynm}, summarized in Eq.~\eqref{eq:gm2exp}. 
For mere illustrative purposes, we take the latter result as reference input of our analysis, despite the recent doubts on its validity mentioned in Sec.~\ref{sect:gm2intro}.
Taking into account also the upper bound on the branching ratio $\mathcal{B}(\mu^+ \to e^+ \gamma)$ determined by the MEG experiment \cite{MEG:2016leq},
we can then write
\begin{align}
\begin{split}
    \mathcal{B}(\mu^+ \to e^+ \gamma) &= \frac{m_\mu^3 v^2}{8\pi \Gamma_\mu} \frac{\big|[\mathcal{C}_{e\gamma}^\prime]_{12}\big|^2 + \big|[\mathcal{C}_{e\gamma}^\prime]_{21}\big|^2}{\Lambda^4}
    \\
    &< 4.2 \times 10^{-13} \quad \text{(90\% CL)} \,,
\end{split}
    \\[0.2cm]
    \Delta a_\mu  
    &\equiv a_\mu^\mathrm{Exp} - a_\mu^\mathrm{SM} 
    = -\frac{4m_\mu}{e} \frac{v}{\sqrt{2}} \frac{\mathrm{Re}[\mathcal{C}_{e\gamma}^\prime]_{22}}{\Lambda^2}
    \nonumber\\
    &= (251 \pm 59) \times 10^{-11} \,,
\end{align}
which leads to
\begin{align}
    \left|\frac{[\mathcal{C}_{e\gamma}^\prime]_{12(21)}}{\Lambda^2}\right| &\lesssim 2.1 \times 10^{-10} \,\mathrm{TeV}^{-2} \,,
    \label{eq:Cegamma12-constraint}
    \\[0.1cm]
    \frac{\mathrm{Re} [\mathcal{C}_{e\gamma}^\prime]_{22}}{\Lambda^2} &\simeq -1.0 \times 10^{-5} \,\mathrm{TeV}^{-2} \,.
    \label{eq:Cegamma22-constraint}
\end{align}

We can now combine all our results: the low-energy constraints in Eqs.~\eqref{eq:Cegamma12-constraint}--\eqref{eq:Cegamma22-constraint}, the rotation to the mass basis~\eqref{eq:Cegamma-mass-basis-rotation}, the LEFT RG equations~\eqref{eq:Cegamma-RGE}--\eqref{eq:Ye-RGE}, the EWSB relations~\eqref{eq:dipole-rotation}--\eqref{eq:Yukawa-rotation}, the SMEFT running~\eqref{eq:SMEFT-beta-functions}, and the matching conditions~\eqref{eq:S1-CeB}--\eqref{eq:S1-CeW}, where the last three results have already been combined in Eq.~\eqref{eq:physical_dipole_low} and~\eqref{eq:physical_Yukawa_low}.\footnote{Notice that we have chosen $\xi_\mathrm{rp}=1$ for convenience, which fixes the NDR reading point that has to be used in all consecutive EFT calculations.}



For simplicity we also consider $[C_{eH}]_{pr}=0$, which holds at tree level in the considered $S_1$~model. 
We also assume $Y_e$ to be diagonal such that the mass matrix is already diagonal and we can set~$U_{L,R}=\mathds{1}$. 
Notice that this is a strong assumption on a marginal operator appearing in the~UV, and in general we have to consider rotation matrices~$U_{L,R} \neq \mathds{1}$. 
The resulting constraints on the $S_1$~couplings, assuming these are real quantities, are shown in Fig.~\ref{fig:S1-constraints}, where we set the leptoquark mass to~$M_{S}=2\,\text{TeV}$. In the upper plot, the constraints derived from the $\Delta a_\mu$~measurement are shown, whereas the lower plot shows the constraints from the $\mu \to e \gamma$ decay, where we set $\lambda^R_{31}=0$ for simplicity. Also couplings to quarks other than the top are neglected as these are not $y_t$~enhanced. 

As can be seen, the scales of the two figures are very different, signaling that underlying models 
able to explain the $(g-2)_\mu$ anomaly, while being consistent with $\mu \to e \gamma$, require a 
peculiar flavor-alignment mechanism.
A~more detailed phenomenological analysis of the given model and a discussion of the implied flavor structure can be found in \cite{Isidori:2021gqe}, see also \cite{Aebischer:2021uvt}.

\begin{figure}[t]
    \centering
    \includegraphics[width=0.8\linewidth]{figures/S1-constraint-22_fig13a.pdf}
    \\[0.2cm]
    \includegraphics[width=0.8\linewidth]{figures/S1-constraint-12_fig13b.pdf}
    \caption{Constraints on the $S_1$~leptoquark couplings derived from the measurements of the $\mu \to e \gamma$ transition (upper plot), and from the $(g-2)_\mu$ measurement (lower plot). The leptoquark mass is chosen as $M_{S}=2\,\text{TeV}$, and only top-Yukawa enhanced contributions are considered in the numerical analysis. See text for more details.}
    \label{fig:S1-constraints}
\end{figure}

There are also tools automating large parts of such analysis. For example the \cmd{flavio}~\cite{Straub:2018kue} package has a large set of low-energy measurements implemented that can be used to constrain Wilson coefficients. Also the SMEFT to LEFT matching as well as the RG~evolution in both ETFs is available in the code [trough the \cmd{Wilson} package \cite{Aebischer:2018bkb}; see also \texttt{DsixTools} \cite{Celis:2017hod}]. A~global likelihood based on the data available in \cmd{flavio} can be constructed with the \cmd{smelli} package \cite{Aebischer:2018iyb}, which can simplify analyses. 


\subsection{\texorpdfstring{SMEFT at high-$p_T$}{SMEFT at high-pT} and global fits}
While the SMEFT (in combination with the LEFT) is very practical to relate low-energy measurements to UV parameters, it can also be used to analyze measurements from higher energies in a model independent way. This makes it a powerful tool for combined analyses of multiple data sets from various types of processes at different energy scales. This is in particular advantageous in light of the plethora of measurements of different processes performed at LHC and~LEP. We can use the SMEFT for phenomenological analyses of all these observables in 
Higgs \cite{Ellis:2014dva,Corbett:2015ksa,Corbett:2012ja}, 
Di-boson \cite{Grojean:2018dqj,Gomez-Ambrosio:2018pnl,Butter:2016cvz,Biekoetter:2018ypq}, 
and top physics \cite{Hartland:2019bjb,Aoude:2022deh,Aoude:2022aro,Brivio:2019ius}, 
as well as for electroweak precision studies \cite{Falkowski:2019hvp,Han:2004az,Breso-Pla:2021qoe,Efrati:2015eaa,Falkowski:2014tna,Almeida:2021asy}, 
and Drell-Yan tails \cite{Allwicher:2022gkm,Greljo:2022jac}. 
Global fits considering multiple of the above data sets have been performed, e.g., in \cite{Ellis:2018gqa,Ellis:2020unq,Ethier:2021bye,daSilvaAlmeida:2018iqo}, see also \cite{Dawson:2020oco}.
Such combined analyses of different types of data are necessary since in any reasonable new-physics model multiple SMEFT operators are generated when integrating out the heavy particles \cite{Jiang:2016czg}. 
These operators can contribute to different processes that can be probed at various energies. 
Also RG~mixing can generate further operators contributing to even more processes. 
Therefore, to carefully evaluate the plausibility of a given BSM theory, it is not enough to look at only a single measurement, but we have to perform a global SMEFT fit.


One of the main challenges for these fits are the large number of free parameters in the SMEFT. 
Thus, one has to apply some simplifying assumptions to reduce the degrees of freedom in a fit. 
For example, one can decide only to look at a specific set of operators that is particularly relevant for a given set of observables (e.g.~those involving only top and bottom quarks, and electroweak gauge bosons).
Moreover, one can apply some flavor symmetry assumptions as discussed in Sec.~\ref{sect:Flavor}. 
As shown in Tab.~\ref{tab:U3new}, the latter allow to significantly lower the number of parameters that we have to fit, while still allowing to describe the SM flavor structure in good approximation.



\begin{figure*}[t]
\begin{center}
\includegraphics[width=1.0\linewidth]{figures/Globalfit_fig14.pdf}
\caption{Bounds on SMEFT effective coefficients as obtained by~\textcite{Ellis:2020unq}.
The top panel indicates the bounds on the 
coefficients assuming 
a reference effective scale of 1~TeV.
The corresponding bounds on the effective scales, for different reference 
hypotheses for the Wilson coefficients,
are shown in the bottom panel. 
The light yellow points are obtained in the 
$U(3)^5$ symmetric limit. The remaining points are obtained 
employing the  $U(2)^3\times  U(2)_u \times U(2)_q$ flavor symmetry,
which allow us to treat separately top-physics observables.} 
\label{fig:Globalfits}
\end{center}
\end{figure*}


On the one hand, if experimental data show deviations from the SM predictions, global fits are essential to determine the best-fit values of all relevant Wilson coefficients in order to be simultaneously compatible with multiple possibly correlated measurements.
On the other hand, if no clear signal for new physics is present in the data, global fits only allow us to put upper bounds on the coefficients. In general, the constraints obtained  
depend on the assumptions entering the fit. Since a truly global fit with all 2499~parameters of the $d=6$~SMEFT is unfeasible, a selection of certain operators, e.g., by choosing a specific flavor symmetry, has to be made. Therefore, one should keep in mind that 
the results of the  simplified fit cannot necessarily be applied to generic BSM scenarios.

As an illustration of the present 
state of the art of global fits, in Fig.~\ref{fig:Globalfits} we report the results of one of the most updated and extensive global analysis of SMEFT coefficients~\cite{Ellis:2020unq}.
The results are obtained considering all the relevant 
operators constrained by 
electroweak precision observables, 
di-boson processes, and top-physics measurements from the LHC.
The flavor symmetries 
$\mathrm{U}(3)^5$ or $\mathrm{U}(2)^3\times \mathrm{U}(2)^2$
are employed  (see  Sec.~\ref{sect:Flavor}).
The results for each Wilson coefficient are obtained marginalizing over the remaining ones.
Despite not fully generic, the number
of independent coefficients varied at the same time is quite impressive. 
One the most important message emerging from this analysis is that, under
motivated flavor-symmetry assumptions, present data are compatible with an effective cutoff scale for the SMEFT in the few-TeV domain.


\subsubsection{Drell-Yan tails}
\label{sec:Drell-Yan}
%
In this section we analyze in detail the specific case of the Drell-Yan process $p p \to \ell^+ \ell^-$, which represents a good example of a high-energy transition  constraining SMEFT  Wilson coefficients. 
In the SM this process is mediated by the photon and the $Z$-boson, whereas in the SMEFT the dominant contributions are given by four-fermion operators~$(\psi^4)$, dipole operators~$(\psi^2 X H)$, and operators modifying the $Z$-boson couplings~$(\psi^2 H^2 D)$. 
The relevant tree-level Feynman diagrams are shown in Fig.~\ref{fig:Feynman-diagrams-Drell-Yan}. 
The plot on the left-hand side shows the SM contribution and the center plot the contribution by $\psi^4$~contact interactions. 
The diagram for the $(\psi^2 X H)$ and $(\psi^2 H^2 D)$ operators is similar to the SM diagram with the SM interaction vertices replaced by the respective SMEFT interactions. 
The dominant contribution depends on the energy range we are investigating. 
The operators modifying the $Z$-boson couplings can be best probed at the $Z$-pole, i.e., for invariant masses of the dilepton system of around~$m_{\ell\ell} \sim m_Z$. 
At higher energies the four-fermion contact interaction yield the dominant contribution since their amplitude is energy enhanced compared to the~SM. 
This is what allows us to probe effects due to the exchange of resonances 
with a mass even above the center-of-mass energy of the collider, as pointed out by \textcite{Greljo:2017vvb}.


\begin{figure}[t]
    \centering
    \includegraphics[width=0.95\linewidth]{figures/U1-Drell-Yan_fig15.pdf}
    \caption{Tree-level Feynman diagrams contributing to Drell-Yan in the SM~(left), the SMEFT~(center), and the $U_1$~leptoquark model~(right).}
    \label{fig:Feynman-diagrams-Drell-Yan}
\end{figure}

In the following, we will focus on the high-$p_T$ constraints on $\psi^4$~operators involving mainly third-generation fermions, which have received considerable interest in the recent literature \cite{Allwicher:2022gkm,Allwicher:2022mcg,Greljo:2022jac,Angelescu:2020uug,Fuentes-Martin:2020lea,Faroughy:2016osc,Dawson:2018dxp,Greljo:2018tzh,Endo:2021lhi,Marzocca:2020ueu,Jaffredo:2021ymt,Boughezal:2023nhe,Boughezal:2021tih,Boughezal:2022nof,Alioli:2020kez}. 
We will also consider measurements of low-energy meson decays that are mediated by the same effective operators. 
Therefore, we can utilize the SMEFT framework to combine these complementary high-$p_T$ and low-energy constraints to asses the validity of a given BSM scenario.
The first analysis of this type, focused of light-generation four-fermion operators, has been presented in \cite{Cirigliano:2012ab}.


In this example, we consider the $U_1$~vector leptoquark contributing to Drell-Yan (see the diagram on the right-hand side of Fig.~\ref{fig:Feynman-diagrams-Drell-Yan}) and to charged-current semileptonic $B$-meson decays with the underlying $b \to c \tau \nu$ transition. We will follow the discussion laid out in \cite{Aebischer:2022oqe}.
The example is particularly interesting due to deviations currently observed in these low-energy decays, known as the $B$-anomalies, mentioned already in Sec.~\ref{sec:LFUV-intro}. We are especially interested in the lepton-flavor-universality ratios~$R_{D^{(\ast)}}$ defined in Eq.~\eqref{eq:RD-ratio} currently showing a $3.1\,\sigma$ discrepancy with the SM expectation \cite{HFLAV:2022pwe}.\footnote{Notice that while the fate of this anomaly, as for any anomaly, is unclear, the discussion presented here still remains an illustrative example of a SMEFT analysis.}

Consider the $U_1$~Lagrangian
\begin{align}
    \mathcal{L}_{U_1}
    &= \mathcal{L}_\mathrm{SM} -\frac{1}{2} U_{\mu\nu}^\dagger U^{\mu\nu} + M_{U}^2 U_\mu^\dagger U^\mu + \left( U_\mu J^\mu + \mathrm{h.c.} \right) \,,
    \\
    J^\mu &= \frac{g_U}{\sqrt{2}} \left[ \beta^L_{pr} \left( \overline{q}_p \gamma^\mu \ell_r \middle)  + \beta^R_{pr}  \middle(\overline{d}{}_p \gamma^\mu e_r \right) \right] \,.
\end{align}
We now integrate out the $U_1$ at tree-level using its equation of motion $\smash{U_\mu = -J_\mu^\dagger \big/ M_{U}^2 + \mathcal{O}(M_{U}^{-4})}$, we find 
\begin{align}
    \mathcal{L}_\mathrm{EFT} &= \mathcal{L}_\mathrm{SM} - \frac{1}{M_{U}^2} J_\mu^\dagger J^\mu \,.
\end{align}
Then, using the Fierz identities in Eqs.~\eqref{eq:Fierz-ids} and~\eqref{eq:SUN-Fierz}
we find the EFT Lagrangian in the Warsaw basis 
\begin{align}
\begin{split}
    \mathcal{L}_\mathrm{W}
    &= \mathcal{L}_\mathrm{SM} -\frac{g_U^2}{2 M_{U}^2} \bigg\{ \frac{1}{2} \beta^L_{pr} \beta^{L\ast}_{st} \left( [Q_{lq}^{(1)}]_{trps} + [Q_{lq}^{(3)}]_{trps}  \right)
    \\
    & + \beta^R_{pr} \beta^{R\ast}_{st} [Q_{ed}]_{trps} - \left( 2 \beta^R_{pr} \beta^{L\ast}_{st} [Q_{ledq}]_{trps} + \mathrm{h.c.} \right) \! \bigg\} \,.
\end{split}
\label{eq:U1-Warsaw}
\end{align}
Notice that since we restrict our analysis to the tree level, we do not have to consider evanescent contributions here.

This Lagrangian provides the appropriate description for interactions at energies above the electroweak scale but below~$M_{U}$. Thus, we can use it to describe the tails of Drell-Yan distributions where we consider events with~$200\,\text{GeV} \lesssim m_{\ell\ell} \lesssim M_{U}$. For a discussion of the EFT validity in the case where the EFT cutoff scale~$M_{U}$ is not sufficiently high, see the end of this section and \cite{Allwicher:2022gkm}.

The event yield~$\mathcal{N}$ in a given bin of the measured $m_{\ell\ell}$~distribution can then be schematically written as
\begin{align}
    \mathcal{N} &= \mathcal{L}_\mathrm{int} \, (\mathcal{A}\times\epsilon) \int_{m_{\ell\ell,\,\mathrm{min}}^2}^{m_{\ell\ell,\,\mathrm{max}}^2} \mathrm{d}s \, \frac{\mathrm{d}\sigma}{\mathrm{d}s} \,,
\end{align}
where $\mathcal{L}_\mathrm{int}$ is the integrated luminosity and $(\mathcal{A}\times\epsilon)$ parametrizes the acceptance and efficiency of the detector and has to be extracted using Monte Carlo simulations. The cross section~$\sigma$ is computed as a function of the Wilson coefficients or new-physics couplings, thus allowing to constrain these. For more details see \cite{Allwicher:2022gkm}. The event yields can also be automatically extracted using codes like \cmd{HighPT} \cite{Allwicher:2022mcg} or \cmd{flavio} \cite{Greljo:2022jac}.

The operators in Eq.~\eqref{eq:U1-Warsaw} contribute also to low-energy processes, of course. In particular, $\smash{[Q_{lq}^{(3)}]_{3323}}$ and $\smash{[Q_{ledq}]_{3332}}$ can contribute to the $b \to c \tau \nu$ transitions that we are interested in.
The relevant low-energy Lagrangian can be written as
\begin{align}
\begin{split}
    \mathcal{L}_{b \to c}
    = -\frac{4 \, G_F}{\sqrt{2}} V_{23} \Big[ &\left( 1 + \mathcal{C}_{LL}^c \right) \left( \overline{c}_L \gamma^\mu b_L \middle) \middle( \overline{\tau}_L \gamma_\mu \nu_L \right)  
    \\
    &- 2\,\mathcal{C}_{LR}^c \left( \overline{c}_L b_R \middle) \middle( \overline{\tau}_R \nu_L \right) \Big]
\end{split}
\end{align}
where $G_F$ is Fermi's constant and $V_{23}=V_{cb}$ is a CKM matrix element.
The coefficients are related to the Warsaw basis Wilson coefficients by
\begin{align}
    \mathcal{C}_{LL}^c
    &= -\frac{1}{\sqrt{2} G_F} \frac{1}{M_{U}^2} \sum_{k=1}^3 \frac{[C_{lq}^{(3)}]_{33k3} V_{2k}}{V_{23}} \,,
    \\
    \mathcal{C}_{LR}^c
    &= \frac{1}{4\sqrt{2} G_F} \frac{1}{M_{U}^2} \sum_{k=1}^3 \frac{[C_{ledq}^\ast]_{333k} V_{2k}}{V_{23}} \,,
\end{align}
where we assume that the flavor basis of the new physics is given by the down-quark and charged-lepton mass basis so that we can write
\begin{align}
    q_p = \! \begin{pmatrix}
        V_{rp}^\ast u_r^L \\[0.05cm] d_p^L
    \end{pmatrix}\!,
    \, 
    u_p = u_p^R ,
    \ 
    d_p = d_p^R ,
    \ 
    \ell_p = \! \begin{pmatrix}
        \nu_p^L \\[0.05cm] e_p^L
    \end{pmatrix}\!,
    \ 
    e_p = e_p^R . 
\end{align}

Following \cite{Cornella:2021sby}, we can express the LFU ratios~$R_{D^{(\ast)}}$ in terms of these parameters as 
\begin{align}
    \frac{R_D}{R_D^\mathrm{SM}} &= \left| 1 \! + \mathcal{C}_{LL}^c \right|^2 \! - 3.0 \mathrm{Re} \big[\! \left( 1 \! + \mathcal{C}_{LL}^c \right) \mathcal{C}_{LR}^{c\ast} \big] + 4.12 \left| \mathcal{C}_{LR}^c \right|^2  \!,
    \\
    \frac{R_{D^\ast}}{R_{D^\ast}^\mathrm{SM}} &= \left| 1 \! + \mathcal{C}_{LL}^c \right|^2 \! - 0.24 \mathrm{Re} \big[\! \left( 1 \! + \mathcal{C}_{LL}^c \right) \mathcal{C}_{LR}^{c\ast} \big] + 0.16 \left| \mathcal{C}_{LR}^c \right|^2  \!\!.
\end{align}
As numerical input we use the world average for the experimental measurements and SM predictions for these observables as provided by the HFLAV collaboration in \cite{HFLAV:winter2023,HFLAV:2022pwe}, respectively:
\begin{align}
    R_{D} &= 0.356 \pm 0.029 \,,
    &
    R_{D}^\mathrm{SM} &= 0.298(4) \,,
    \\
    R_{D^\ast} &= 0.284 \pm 0.013 \,,
    &
    R_{D^\ast}^\mathrm{SM} &= 0.254(5) \,.
\end{align}

The LEFT beta-functions of the coefficients are given by \cite{Jenkins:2017dyc}
\begin{align}
    \beta_{\mathcal{C}_{LL}^c} &= -4 e^2 \mathcal{C}_{LL}^c \,, 
    & 
    \beta_{\mathcal{C}_{LR}^c} &= \left( \frac{4}{3} e^2 - 8 g_3^2 \right) \mathcal{C}_{LR}^c \,.
\end{align}
We use the LEFT RG~equations\footnote{The dominant contribution is due to the strong coupling constant~$\alpha_s=g_3^2/4\pi$, which runs as $\alpha_s(\mu)=\frac{4\pi}{\beta_0 \ln(\mu^2/\Lambda_\mathrm{QCD}^2)}$ at one loop, with the one-loop QCD beta-function~$\beta_0$.} to directly run the low-energy coefficients from the scale~$\mu \sim m_b$ up to $\mu = 1\,\text{TeV}$, which is the appropriate scale for measurements of the high-$p_T$ Drell-Yan tails at LHC. There we directly match to the SMEFT and neglect the SMEFT running in good approximation since it only yields a small logarithmic contribution.

To perform the combined fit of the high-$p_T$ Drell-Yan data and the low-energy measurements of~$R_{D^{(\ast)}}$, we assume that all couplings except for $\beta_{33}^{L/R}$ and $\beta_{23}^{L}$ vanish, i.e., the~$U_1$ couples dominantly to the third generation. Furthermore, we choose to set $\beta_{33}^L=-\beta_{33}^R=1$ and $\beta_{23}^L=2V_{ts}$,
adopting the hypothesis of a minimal breaking of the flavor symmetry  \cite{Aebischer:2022oqe}.
The combined constraints on the $U_1$~model in the coupling versus mass plane are shown in Fig.~\ref{fig:U1-fit}. We used the \cmd{HighPT} package \cite{Allwicher:2022mcg} to derive the constraints from the Drell-Yan search for new physics in $pp \to \tau\tau$ scattering by the ATLAS collaboration \cite{ATLAS:2020zms}. The $95\%$~CL region preferred by our low-energy constraint discussed above is shown in light orange, whereas the region excluded at $95\%$ by LHC is shown in dark gray. In combination, only a fraction of parameter space is left viable, thus showing the complementarity of the low- and high-energy constraints.\footnote{Interestingly enough, CMS data currently indicates a $3\sigma$ excess of events in $pp \to \tau\bar\tau$, well compatible 
with a possible $U_1$ contribution in this parameter region~\cite{CMS:2022zks}. }
For more details on this analysis see \cite{Aebischer:2022oqe,Cornella:2021sby}.

\begin{figure}[t]
    \centering
    \includegraphics[width=0.85\linewidth]{figures/U1-fit_fig16.pdf}
    \caption{Constraints on the $U_1$~model in the coupling~$g_U$ versus mass~$M_{U}$ plane. Shown in light orange is the region preferred by the low-energy fit of the $R_{D^{(\ast)}}$~anomalies, and in dark gray we show the parameter space excluded by the ATLAS search \cite{ATLAS:2020zms} for new physics in $pp \to \tau\tau$ scatterings. Both constraints are given at $95\%$~CL.}
    \label{fig:U1-fit}
\end{figure}

 
In the case of very low masses of the leptoquark $(M_U \sim 1\,\text{TeV})$ one might question the validity of the EFT approach to Drell-Yan measurements, since the kinematical distributions contain events with corresponding center-of-mass energies~$\sqrt{s}$ of the same order. 
Therefore the EFT expansion in~$s/M_U^2$ can converge poorly or even break down. To improve the convergence one can include higher-dimensional operators.  
We can either fit them as additional free parameters, marginalize over them,\footnote{Notice that when marginalizing over $d=8$ operators no correlation among the $d=6$ and $d=8$ operators is assumed, which is not true in concrete BSM scenarios. In particular the interference of $d=6$ and $d=8$ operators with the SM amplitude are allowed to have opposite sign, leading to cancellations.} or we can match them to the parameters of a given UV model, such as the $U_1$~leptoquark, depending on the scenario we are considering.
If we are too close to the mass threshold of the heavy BSM states there might be no way to analyze the high-energy data apart from using a concrete UV model. However, in this case the model independence of the EFT approach might be less important as the signal for a concrete new-physics model should be stronger.
A~short discussion of the EFT validity in Drell-Yan tails can be found in \cite{Allwicher:2022gkm}. For more details see also Sec.~\ref{sect:dim8} and \cite{Brivio:2022pyi}. 

