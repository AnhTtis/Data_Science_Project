\section{Dimensional regularization in the SMEFT}
\label{app:dim-reg}
As in any QFT, divergences can occur in the computation of one-loop diagrams in EFTs, which need to be regulated. 
Afterwards, the theory can be renormalized, and physical predictions can be derived.
By far the most commonly used regularization scheme in SMEFT computations is dimensional regularization, which we also use throughout this review, where we work in $D=4-2\epsilon$ spacetime dimensions. 
The most common renormalization scheme for SMEFT computations is the modified minimal subtraction~$\smash{(\overline{\mathrm{MS}})}$ scheme, which we also use in most of this work. 
The only exception is when we deal with evanescent operators, where we choose to work in an evanescent free version of~$\smash{\overline{\mathrm{MS}}}$ (see Sec.~\ref{sec:evanescent}). 
To be precise, we work in a modified version of $\smash{\overline{\mathrm{MS}}}$ that contains additional finite counterterms that compensate the effects of evanescent operators to physical observables so that these can be neglected in all computations.

In the following, we discuss two topics related to the use of dimensional regularization. First, we discuss the method of regions that is often used in EFT computations, e.g., when computing one-loop matching conditions. Afterwards, we comment on the issue of chiral fermions in dimensional regularization, i.e., the generalization of the $\gamma^5$~matrix to $D$~dimensions.

\subsection{The method of regions}
\label{app:method-of-regions}
\begin{figure*}[t]
    \centering
    \includegraphics[width=0.9\linewidth]{figures/method-of-regions_fig17.pdf}
    \caption{Schematic illustration of the method of regions applied for EFT matching. The separation of the full integration region into a hard and soft region (green frames) is shown for the UV~theory and the corresponding~EFT. For each region the UV~(IR) poles are highlighted in red (yellow) at the top (bottom) of each box. The UV~divergences require counterterms and allow to extract the RGE and the contributions of evanescent operators. The artificial divergences introduced by the method of region are shown in gray and cancel between the soft and the hard region as indicated  by the dashed lines connecting them. Divergences that are equal are connected by a solid line, whereas divergences that have the same magnitude but opposite sign are linked by dashed lines. The soft region of both theories are equal by construction, and the hard region of the EFT only contains scaleless integrals and thus vanishes in dimensional regularization.}
    \label{fig:method-of-regions}
\end{figure*}

The method of \textit{expansion by regions} \cite{Beneke:1997zp,Jantzen:2011nz} simplifies the calculation of multi-scale loop integrals in the presence of a power counting. Loop integrals can depend on several different scales (e.g. masses or external momenta), each defining an integration region. For each scale we can expand the loop integrand in the quantities that are small in the respective region and then perform the resulting integral over the entire $D$-dimensional space. The method of regions states that doing this for all regions and summing the results yields the same answer as performing the full original integral and expanding afterwards.

As an example, consider the loop integral
\begin{align}
    \mathcal{I} &= \int \frac{\dd^D k}{(2\pi)^D} \, \frac{1}{k^2-M^2} \, \frac{1}{k^2-m^2}
    \\
    &= \!\frac{i}{16\pi^2} \!\!\left[ \frac{1}{\epsilon} \!+\! \log\!\left(\!\frac{\mu^2}{M^2}\!\right)\! +\! 1 \!+ \!\frac{m^2}{M^2} \log\!\left(\!\frac{m^2}{M^2}\!\right) \!\right] \!+ \mathcal{O}(M^{-4})
    \nonumber
\end{align}
which entails two regions called soft~$(k \sim m)$ and hard~$(k \sim M)$.
Expanding the propagators in the soft $(k^2 \sim m^2 \ll M^2)$ and hard $(k^2 \sim M^2 \gg m^2)$ region before the integration
\begin{align}
    \frac{1}{k^2-M^2} &= -\frac{1}{M^2} \left[ 1 + \frac{k^2}{M^2} + \mathcal{O}\left(\frac{k^4}{M^4}\right) \right] \,,
    \\
    \frac{1}{k^2-m^2} &= \frac{1}{k^2} \left[ 1 + \frac{m^2}{k^2} + \mathcal{O}\left(\frac{m^4}{k^4}\right) \right] 
\end{align}
we find the corresponding integrals in each region
\begin{align}
    \mathcal{I}\big|_\mathrm{soft} &= -\frac{1}{M^2} \int \frac{\dd^D k}{(2\pi)^D} \, \left[ \frac{1}{k^2-m^2} + \ldots \right]
    \\
    &= - \frac{i}{16\pi^2} \frac{m^2}{M^2} \left[ \frac{1}{\epsilon} + \log\left(\frac{\mu^2}{m^2}\right) + 1 \right] + \mathcal{O}(M^{-4}) \,,
    \nonumber\\[0.1cm]
    \mathcal{I}\big|_\mathrm{hard} &= \int \frac{\dd^D k}{(2\pi)^D} \, \frac{1}{k^2} \, \frac{1}{k^2-M^2} \left[ 1 + \frac{m^2}{k^2} + \ldots \right]
    \\
    &= \! \frac{i}{16\pi^2} \!\! \left[ \frac{1}{\epsilon} + \log\!\left(\!\frac{\mu^2}{M^2}\!\right)\! + 1 \right]\!\! \left(\! 1 +\! \frac{m^2}{M^2} \!\right) \!+ \mathcal{O}(M^{-4}) \,.
    \nonumber
\end{align}
Thus, we find working at the order~$\mathcal{O}(M^{-2})$
\begin{align}
    \mathcal{I} &= \mathcal{I}\big|_\mathrm{hard} + \mathcal{I}\big|_\mathrm{soft} + \mathcal{O}(M^{-4}) \,,
\end{align}
as dictated by the method of regions.

As discussed in Sec.~\ref{sec:matching}, the method of regions provides a very powerful tool for EFT matching computations. These are, of course, multi-scale problems and applying this method allows for a separation of the hard UV dynamics from the soft IR behavior. In these computations we have to determine Green's functions in the UV~theory and the corresponding~EFT. In both theories we can split these into a hard and soft region. Since we require both theories to describe the same IR physics the soft regions --which contain exactly the low-energy dynamics-- of both theories must agree. Thus, the one-loop matching conditions for the EFT Wilson coefficients are solely determined by the hard regions encoding the UV~dynamics. However, since the EFT does by definition not contain any UV scales, its hard scale loop integrals must be scaleless and thus vanish exactly in dimensional regularization. This holds for all integrals apart from
\begin{align}
    \int \frac{\dd^D k}{(2\pi)^D} \, \frac{1}{k^4} &= \frac{i}{16\pi^2} \left( \frac{1}{\epsilon_{\scriptscriptstyle\mathrm{UV}}} - \frac{1}{\epsilon_{\scriptscriptstyle\mathrm{IR}}} \right) = 0 \,,
    \label{eq:scaleless-integral}
\end{align}
which vanishes only since we identify $\epsilon_{\scriptscriptstyle\mathrm{UV}} = \epsilon_{\scriptscriptstyle\mathrm{IR}}$ by analytic continuation in dimensional regularization.
Therefore, it is enough to only consider the hard region of the Green's functions of the UV theory which contains all information required to determine the EFT Wilson coefficients. The entire procedure of applying the method of regions is schematically shown in Fig.~\ref{fig:method-of-regions}.

This illustration also highlights the connection of the different UV and IR~divergences encountered in the computation. The UV~poles (red) and IR~poles (yellow) of a theory must match the corresponding poles in hard and soft region, respectively. However, applying the method of regions introduces additional artificial divergences (gray) in both regions. But since the sum of both regions must yield back the full solution, these must cancel between soft and hard region. We recall that also the UV and IR~poles of the hard EFT region must cancel due to Eq.~\eqref{eq:scaleless-integral}. In Fig.~\ref{fig:method-of-regions} cancelling divergences are connected by dashed lines, whereas equal poles are linked by solid lines.
When performing a computation we use the renormalized versions of these theories, i.e., we introduce counterterms cancelling the UV~poles. As mentioned before, for a matching computation we only need to compute the hard region of the UV~theory. The $\epsilon_{\scriptscriptstyle\mathrm{UV}}$~poles of this region are cancelled by the appropriate counterterms, and from Fig.~\ref{fig:method-of-regions} we see that the artificial IR~poles provide exactly the right counterterms to cancel the UV~poles of the resulting EFT. Thus the EFT is automatically renormalized. 

Eventually, notice that the method of regions is also useful to extract only the UV~divergences of a theory, since these are entirely encoded in its hard region. Therefore, it simplifies the extraction of the RG equations of a theory, and also the computation of the physical effect of evanescent operators (see Sec.~\ref{sec:evanescent}), since both are entirely determined by~$\epsilon_{\scriptscriptstyle\mathrm{UV}}$.

\subsection{Treatment of \texorpdfstring{$\gamma_5$}{gamma5} in \texorpdfstring{$D$}{D}~dimensions}
\label{app:gamma5}
When working in $D$~dimensions the Dirac algebra is infinite dimensional for non-integer~$D$, as mentioned already in Sec.~\ref{sec:evanescent}. While the usual Dirac matrices are defined by interpolation of the $D$~dimensional Dirac basis~$\gamma^\mu$ for $\mu \in \{0,\ldots,D=2n\}$ with an integer~$n\geq2$, the $\gamma_5$~matrix is not easily generalizable to $D \neq 4$~dimensions.
This is due to the intrinsically four-dimensional relation 
$\gamma_5 = -\frac{i}{4!} \varepsilon_{\mu\nu\rho\sigma} \gamma^\mu \gamma^\nu \gamma^\rho \gamma^\sigma$
linking it to the Levi-Civita tensor that can only be defined for~$D=4$.
Thus, any regularization and renormalization scheme must provide a prescription for treating $\gamma_5$ in dimensional regularization.

Throughout this review, we employed the (semi-) na\"ive dimensional regularization~(NDR) scheme, which assumes that the four-dimensional anti-commutation relations \cite{Korner:1991sx,Kreimer:1989ke,Nicolai:1980km}
\begin{align}
    \{\gamma^\mu,\gamma^\nu\} &= 2 g^{\mu\nu} \,,
    &
    \{\gamma^\mu,\gamma_5\} &= 0 \,,
    &
    \gamma_5^2 &= \mathds{1}
\end{align}
hold also away from~$D=4$. This is inconsistent with the cyclicity of the trace and $\mathrm{tr} \left( \gamma^\mu \gamma^\nu \gamma^\rho \gamma^\sigma \gamma_5 \right) \neq 0$. To reproduce the correct four-dimensional limit we formally substitute
\begin{align}
    \mathrm{tr} \left( \gamma^\mu \gamma^\nu \gamma^\rho \gamma^\sigma \gamma_5 \right) &= -4i\varepsilon^{\mu\nu\rho\sigma} \,,
    \label{eq:NDR-trace}
\end{align}
with $\varepsilon^{0123}=+1$. This breaks the cyclicity of traces with six or more $\gamma^\mu$-matrices and an odd number of~$\gamma_5$, thus introducing a reading point ambiguity. That means these traces depend on which $\gamma$-matrix is put first/last in the trace. 
For example, when computing the Feynman diagrams in Fig.~\ref{fig:evanescent-dipole} with insertions of the operator $Q_{lequ}^{(3)}$ we find, depending on where we start reading the closed fermion loop, the two Dirac traces
\begin{align}
    \mathrm{tr}_1 &\equiv \mathrm{tr}\left( \gamma^\alpha \gamma^\rho \gamma^\sigma \gamma_\alpha \gamma^\mu \gamma^\nu \gamma_5 \right) = 4i (4-D) \varepsilon^{\mu\nu\rho\sigma} \,,
    \\
    \mathrm{tr}_2 &\equiv \mathrm{tr}\left( \gamma^\rho \gamma^\sigma \gamma_\alpha \gamma^\mu \gamma^\nu \gamma_5 \gamma^\alpha \right) = -4i (4-D) \varepsilon^{\mu\nu\rho\sigma} \,,
\end{align}
which can be shown using Eq.~\eqref{eq:NDR-trace} and $\gamma^\alpha \gamma^\mu \gamma^\nu \gamma_\alpha = 4 g^{\mu\nu} \mathds{1} - (4-D) \gamma^\mu \gamma^\nu$.
We thus find
\begin{align}
     \mathrm{tr}_1 - \mathrm{tr}_2 = \mathcal{O}(\epsilon) \neq 0
\end{align}
in contradiction to the cyclicity of the trace. 
In EFT analyses using the NDR scheme, we must therefore carefully apply a consistent reading point prescription throughout all computations to obtain consistent results \cite{Fuentes-Martin:2020udw,Fuentes-Martin:2022vvu,Carmona:2021xtq}.

If one wants to avoid the ambiguities related to the reading point of Dirac traces one can resort to the 't~Hooft--Veltman~(HV) scheme \cite{tHooft:1972tcz,Breitenlohner:1977hr}, which is the only $\gamma_5$--scheme that is proven to be self-consistent to all orders.
In this scheme we define
\begin{align}
    \{\gamma^\mu,\gamma_5\} &= 0 & &\text{for } \mu \in \{0,1,2,4\} \,,
    \\
     [\gamma^\mu,\gamma_5] &= 0 & &\text{otherwise} \,.
\end{align}
While being the only known self-consistent scheme, HV comes with the subtlety that it breaks chiral symmetry and thus the Ward identities, which need to be restored by finite renormalizations. Also the HV scheme is computationally more expensive than NDR due to the splitting of the Dirac algebra in a four and a $D-4$~dimensional part. We therefore stick to the NDR scheme throughout this review, which is sufficient for the topics discussed here.

For a more detailed discussion of regularization schemes in $D$~dimensions and the problems of extending $\gamma_5$ to $D$~dimensions see \cite{Gnendiger:2017pys,Jegerlehner:2000dz} and references therein.
