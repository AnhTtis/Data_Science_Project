\section{Non-linear realization of electroweak symmetry breaking}
\label{sec:HEFT}

Within the SM,  the spontaneous breaking of the electroweak symmetry occurs through the non-vanishing 
vacuum expectation value of the $\mathrm{SU}(2)_{L}$--doublet scalar $H$.
Expanding around the minimum of $H$
as in  Eq.~(\ref{eq:Hdec}), one identifies the 
massive field with $h$ and the three 
Goldstone bosons with $\varphi_{1,2,3}$.
From measurements of various electroweak observables
and high-energy processes, the existence of the three Goldstone bosons and the massive scalar~$h$ is well established. However, it is not yet evident that they are necessarily embedded into the four components of a single 
$\mathrm{SU}(2)_L$--doublet~$H$, as in Eq.~(\ref{eq:Hdec}).
We refer to this embedding as 
the linear realization of the 
electroweak symmetry breaking mechanism. 
As we shall discuss shortly,
in principle, other embeddings are still viable. 

The inclusion of the scalar state~$h$ in the theory, and the relations among the different couplings provided by embedding $h$ and the three~$\varphi_{1,2,3}$ into the doublet~$H$, is essential to ensure the unitarity of scattering amplitudes for longitudinally polarized electroweak gauge bosons at high energies.  
However, within an EFT approach the loss of unitarity is  not a problem as long as it happens above the cutoff scale of the theory  [see e.g.~\cite{Brivio:2017vri}]. As a consequence, in a general EFT 
approach to physics beyond the SM we are allowed to relax the strict
constraints following from the linear embedding of 
$h$ and the three~$\varphi_{1,2,3}$ into~$H$ and consider a more general structure. 

To this end, we employ the Callan-Coleman-Wess-Zumino~(CCWZ) formalism \cite{Coleman:1969sm,Callan:1969sn} and proceed similarly to the construction of chiral perturbation theory. In other words, we construct an EFT with a non-linear realization of the electroweak symmetry breaking mechanism. 
To make contact with the SM Lagrangian, it is convenient to decompose the matrix field~$\Sigma$ introduced in Eq.~\eqref{eq:Higgs-matrix-field} as
\begin{align}
    \Sigma(x) &= \frac{v+\hat{h}(x)}{\sqrt{2}} U(x) \quad\text{with}\quad U(x) = \exp \left( i\frac{\boldsymbol{\tau}\cdot\boldsymbol{\pi}(x)}{v} \right) ,
    \label{eq:HEFT-non-linear-field-redefinition}
\end{align}
where we introduced the unitary dimensionless field~$U(x)$, and $\boldsymbol{\tau}$ denotes the three-vector of Pauli matrices. The new fields~$\smash{\hat{h}}$ and $\boldsymbol{\pi}=\smash{\left(\pi_1,\pi_2,\pi_3\right)^\intercal}$, where the latter are the counterpart of the pions in two-flavor QCD, are related to the original fields $h$ and $\boldsymbol{\varphi}$ by a non-linear field redefinition.  We can still interpret~$\smash{\hat{h}}$ as the physical Higgs boson and $\boldsymbol{\pi}$ as the vector containing the three Goldstone bosons. The fields transform under the custodial symmetry group~\eqref{eq:custSU2} as
\begin{align}
    \hat{h}(x) &\rightarrow \hat{h}(x) \,,
    &
    &\text{and}
    &
    U(x) &\rightarrow V_L \, U(x) \, V_R^\dagger \,.
\end{align}
Following the discussion in \cite{Longhitano:1980tm,Longhitano:1980iz,Appelquist:1980vg,Appelquist:1980ae,Feruglio:1992wf,Grinstein:2007iv,Stoffer:EFT} and substituting Eq.~\eqref{eq:HEFT-non-linear-field-redefinition} into \eqref{eq:L-linear-sigma-model}, we can express the scalar part of the SM Lagrangian, including also the interactions with the weak gauge bosons,  as
\begin{align}
\begin{split}
    \mathcal{L}_{p^2}^\mathrm{scalar} &= \frac{1}{2} \left(\partial_\mu\hat{h}\right)\left(\partial^\mu\hat{h}\right) - \frac{1}{2} m_h^2 \hat{h}^2 
    \\
    &+ \frac{v^2}{4} \mathcal{F}\left(\frac{\hat{h}}{v}\right) \mathrm{tr}\left[ (D_\mu U)^\dagger (D^\mu U) \right] - V\left( \frac{\hat{h}}{v} \right)
\end{split}
\label{eq:non-linear-SM-Lagrangian}
\end{align}
with the mass~$m_h^2=2m^2=\lambda v^2$ of the physical Higgs boson~$\hat{h}$,
% \footnote{Notice the distinction between the Lagrangian mass parameter~$m^2$ for Higgs doublet~$H$ and the physical Higgs boson~($\hat{h}$) mass~$m_h^2$. \felix{Delete this footnote?}} 
and where we have defined
\begin{align}
    \mathcal{F}\left(\frac{\hat{h}}{v}\right) &= \left( 1 + \frac{\hat{h}}{v} \right)^{\!\!2} \,,
    \label{eq:F-heft}
    \\
    V\left( \frac{\hat{h}}{v} \right) &= v^4 \left[ \frac{m_h^2}{2v^2} \left( \frac{\hat{h}}{v} \right)^{\!\!3} + \frac{m_h^2}{8v^2} \left( \frac{\hat{h}}{v} \right)^{\!\!4} \right].
    \label{eq:V-heft}
\end{align}
We now have a non-linear formulation of the custodial symmetry breaking in close analogy to chiral perturbation theory, the only difference being the presence of the additional singlet state~$\hat{h}$. To write down Eq.~\eqref{eq:non-linear-SM-Lagrangian} we promoted the global custodial symmetry to a local one by introducing two ${2 \times 2}$~matrix spurion fields~$\hat{W}_\mu$ and~$\hat{B}_\mu$ as the gauge bosons of the chiral $\mathrm{SU}(2)_{L}$ and $\mathrm{SU}(2)_{R}$ groups, respectively. These fields must transform in the adjoint representation of the chiral groups\footnote{The transformation rules read $\smash{\hat{W}_\mu \to V_L \hat{W}_\mu V_L^\dagger + i V_L (\partial_\mu V_L)^\dagger}$, and $\smash{\hat{B}_\mu \to V_R \hat{B}_\mu V_R^\dagger + i V_R (\partial_\mu V_R)^\dagger}$, respectively.} to make~\eqref{eq:non-linear-SM-Lagrangian} formally invariant. The covariant derivative then reads
\begin{align}
    D_\mu U &= \partial_\mu U - i \hat{W}_\mu U + i U \hat{B}_\mu
\end{align}
and we obtain the SM case by fixing the spurions to %their values
\begin{align}
    \hat{W}_\mu &\to g_2 \frac{\tau^I}{2} W_\mu^I \,,
    &
    \hat{B}_\mu &\to g_1 \frac{\tau^3}{2} B_\mu \,,
\end{align}
where $W_\mu$ and $B_\mu$ are the weak gauge bosons of the~SM.
This breaks the chiral symmetry down to its gauged SM subgroup
\begin{align}
    \mathrm{SU}(2)_L \otimes \mathrm{SU}(2)_R \rightarrow \mathrm{SU}(2)_L \otimes \mathrm{U}(1)_Y \,.
\end{align}

% % % % % % % % % % % % % %
\subsection{The Higgs effective field theory}
\label{sect:HEFT1}
After adding the fermion and gauge sectors to the chiral Lagrangian given in Eqs.~\eqref{eq:non-linear-SM-Lagrangian}--\eqref{eq:V-heft} it is equivalent to the usual SM~Lagrangian written in terms of the Higgs doublet~$H$ shown in Eq.~\eqref{eq:SM_Lagrangian}. The two Lagrangians are related by the non-linear field redefinition~\eqref{eq:HEFT-non-linear-field-redefinition} which leaves the physical observables invariant.

However, going beyond the SM, i.e., considering the EFT extension of~\eqref{eq:non-linear-SM-Lagrangian}, the equivalence between the linear and non-linear realization can be broken as we will discuss shortly. Amending the Lagrangian~\eqref{eq:non-linear-SM-Lagrangian} to an EFT, we can express the functions~$\mathcal{F}$ and~$V$ as generic power series in their argument~$\hat{h}/v$
\begin{align}
    \mathcal{F}\left(\frac{\hat{h}}{v}\right) &= 1 + \sum_{n=1}^\infty a_n \left(\frac{\hat{h}}{v}\right)^{\!\!n} \,,
    \\
    V\left(\frac{\hat{h}}{v}\right) &= v^4 \sum_{n=3}^\infty b_n \left(\frac{\hat{h}}{v}\right)^{\!\!n} \,,
\end{align}
where we reproduce the SM by choosing $a_1=2$, $a_2=1$, $b_3=\lambda/2$, and $b_4=\lambda/8$ with all other coefficients vanishing. Given $\hat{h}$ is a singlet, in this generic EFT approach 
the coefficients~$a_n,b_n$ are free parameters, not fixed by the symmetries of the theory, and have to be determined experimentally. Since their determination requires the measurement of multi Higgs processes, the current experimental constraints on these parameters are rather imprecise. 
%\felix{Maybe be a bit more precise and give some numerical constraints and/or projections for HL-LHC.}
Of course, allowing for generic coefficients $a_n$ and~$b_n$ is incompatible with the field redefinition~\eqref{eq:HEFT-non-linear-field-redefinition} that we used to relate the chiral SM Lagrangian~\eqref{eq:non-linear-SM-Lagrangian} to the linear one. In general, it is not possible to find a field redefinition that brings the generic non-linear EFT case
back to the linear one, while the reverse processes is always possible.

The above statement  implies that  the EFT constructed from the chiral Lagrangian~\eqref{eq:non-linear-SM-Lagrangian} is more general than the SMEFT which is built under the assumption of a linear realization of electroweak symmetry 
breaking mechanism. 
This effective theory is known as the Higgs effective field theory, or HEFT \cite{Feruglio:1992wf,Alonso:2012px,Buchalla:2013rka,Pich:2016lew}. The HEFT contains the SMEFT as the special case 
where a non-linear field redefinition can be found to map the scalar components ($\smash{\hat{h}}$~and~$\pi_{1,2,3}$) into a single $\mathrm{SU}(2)_{L}$~doublet~($H$). For more details on distinguishing SMEFT and HEFT see the discussion in Sec.~\ref{sec:geometric-formulation} and \cite{Falkowski:2019tft}.
When adding the Yukawa interactions to the EFT, these can of course also be multiplied by an arbitrary power of~$\hat{h}/v$, thus leading to an expansion of these similar to Eqs.~(\ref{eq:F-heft})--(\ref{eq:V-heft}). The NLO~Lagrangian can be constructed in close analogy to chiral perturbation theory, with the only difference being the presence of the additional singlet~$\hat{h}$ that allows every operator to be multiplied by a generic function of~$\hat{h}/v$.

The HEFT is therefore a combination of the fermionic and gauge sectors of the SMEFT, with their power counting in canonical mass dimension, and the scalar sector of chiral perturbation theory with a chiral power counting. The HEFT is based on the same gauge symmetry as the SMEFT and contains the same degrees of freedom apart from the Higgs doublet~$H$ which is replaced by the scalar singlet~$\hat{h}$ and the Goldstone boson matrix~$U$. This decorrelates the interactions of (multiple)~$\hat{h}$ and the Goldstone bosons, therefore allowing to cover more general BSM scenarios. 
%\felix{We should probably mention for which scenarios HEFT is necessary: composite Higgs, NP affecting EWSB/scalar sector, ...}
A~complication in HEFT is that the matrix field~$U$ as well as $\hat{h}/v$ are adimensional: $[U]=[\hat{h}/v]=1$. Therefore, the EFT series cannot be truncated only according to the canonical mass dimension as in the SMEFT case, but we also need to consider a chiral power counting, i.e., an expansion in the number of derivatives. This is possible since $U^\dagger U=\mathds{1}$, and thus the field~$U$ must always be derivatively coupled, as generically expected for Goldstone bosons.  
Due to the mixture of the different countings, there is no unique way to define a consistent power counting for the HEFT, but a commonly used counting \cite{Gavela:2016bzc} is NDA (c.f.~Sec.~\ref{sec:powercounting}). 
%\felix{Maybe we should be a bit more precise for the power counting here.}
This counting has been employed for the construction of the NLO HEFT basis in \cite{Brivio:2016fzo}, see also \cite{Sun:2022ssa}. A discussion of the counting of the HEFT operators using the Hilbert series technique can be found in \cite{Sun:2022aag,Graf:2022rco}.

%For more details on the HEFT see \cite{Brivio:2017vri} and references therein.
%\felix{Maybe we should somewhere add a bit more discussion of the differences of SMEFT and HEFT.}
% \felix{Discuss power counting mixture of SMEFT, chiral power counting (like $\chi$PT) mixed with counting in canonical dimension, NDA.}
% \felix{How to built basis \cite{Brivio:2016fzo}.}
% \felix{Phenomenological differences to SMEFT. See also \cite{Brivio:2017vri,Falkowski:2019tft} and references therein.}
% \noindent\felixx{\rule{\linewidth}{2.0pt}}


% - - - - - - - - - - - - - - - - - - - - - - - - - - - - - - - - - - - - - - - - - - - - - - - - - - - - - - - - - - - - - - - - - -
\subsection{Geometric interpretation for the scalar sector}
\label{sec:geometric-formulation}

% \felix{This subsection is based on Refs.~\cite{Alonso:2015fsp,Alonso:2016btr,Alonso:2016oah,Helset:2018fgq,Corbett:2019cwl,Helset:2020yio}.}
% \\
% When working with the SMEFT we often implicitly assume that we work in the unbroken phase of the theory, meaning that we consider energy scales above the electroweak symmetry breaking scale~$\vT$ such that the Higgs doublet~$H$ does not acquire a \vev. However, we can also consider the spontaneous breaking of the electroweak symmetry in the SMEFT. This is necessary when considering energies below~$\vT$ foe which the Higgs acquires a \vev. This corresponds to the broken phase of the SMEFT. Going to this broken phase of the theory allows us to formulate the EFT in a geometric way, which we will discuss below.\footnote{The geometric formulation of SMEFT is apparent in the broken phase of the theory, however it can equally well be used to describe the unbroken phase.\felix{Check this footnote.}}

An interesting approach to better appreciate the
difference between SMEFT and HEFT is the 
geometric interpretation of the scalar 
sectors of these EFTs, which we review in this section.
This technique,  initially developed 
in the context of non-linear sigma models,
has been extensively applied to analyze the 
scalar sectors of the SMEFT and HEFT \cite{Alonso:2015fsp,Alonso:2016btr,Alonso:2016oah,Helset:2018fgq,Corbett:2019cwl,Helset:2020yio}. 

The starting point of the geometric formulation is the observation that, after spontaneous symmetry breaking, a tower of higher-dimensional operators collapses into a single composite operator form \cite{Helset:2020yio}.
%\footnote{The geometric formulation of SMEFT is apparent in the broken phase of the theory, however it can equally well be used to describe the unbroken phase.\felix{Check this footnote.}} 
Consider for instance the SMEFT, with the Higgs vev defined by $v_T \equiv \sqrt{2 \langle H^\dagger H \rangle}$.\footnote{Note that, in general, we have $v \neq v_T$ due to the presence of higher-dimensional operators in the scalar potential of the SMEFT as we will discuss in Sec.~\ref{sect:LEFT}.} An example for the breakdown of a tower of higher-dimensional operators can be observed in the effective Yukawa interactions.
The operators of interest are of the form $\brackets{H^\dagger H}^n \brackets{\overline{\psi}_L H \psi_R}$ for some $n \in \mathbb{N}$. 
When the Higgs field acquires a vev, $\langle H^\dagger H\rangle \to v_T^2/2$, these higher-dimensional operators collapse into a number multiplying the (effective) SM~Yukawa operator, as illustrated in Fig.~\ref{fig:geoSMEFT}. A~similar breakdown of higher-dimensional operators is perceptible for other interactions as well. Thus, the interactions of all the particles, including the physical Higgs~$h$ itself, can be thought of as taking place in a Higgs-medium \cite{Helset:2020yio}. This medium can be described by a scalar field manifold~$\mathcal{M}$ with coordinates defined by the scalar fields. The $S$-matrix of the theory is invariant under scalar field redefinitions which in this case are equivalent to coordinate transformations on~$\mathcal{M}$. These coordinate redefinitions also leave invariant the geometry of the scalar manifold~$\mathcal{M}$. Therefore, the $S$-matrix, and thus all physical observables, only depend on the geometric properties\footnote{We are considering here the geometry of the scalar field space~$\mathcal{M}$, contrary to general relativity which considers the geometry of spacetime. However, many of the concepts are similar.} of~$\mathcal{M}$, but not on the choice of coordinates \cite{Alonso:2016oah}.

The geometric formulation leads to a factorization of the EFT power counting expansions. In the SMEFT there are two distinct expansions that are often not properly distinguished. The first expansion~$(i)$ is in the ratio of the electroweak scale~$v_T$ to the new physics scale~$\Lambda$, whereas the second expansion~$(ii)$ is in the ratio of the kinematical scale~$p$ for the process of interest relative to the new physics scale
\begin{align*}
	(i): & \quad \frac{v_T}{\Lambda} \, , &
	(ii): & \quad \frac{p^2}{\Lambda^2} \, ,
\end{align*}
where $p^2$ is some kinematic Lorentz invariant. In the geometric formulation, expansion $(i)$~is largely factorized out, as it can be linked to the curvature of the scalar manifold, whereas $(ii)$~is determined by the derivative expansion \cite{Alonso:2016oah}. This factorization of the power counting allows to define the SM Lagrangian parameters to all orders in the SMEFT power counting as shown in \cite{Helset:2020yio}. 

\begin{figure}
\centering
\includegraphics[width=0.98\linewidth]{figures/smeft-yukawa-diagrams.pdf}
\caption{Feynman diagrams contributing to the (effective) Yukawa interactions in the SMEFT after taking the \vev $v_T$ symbolized by the blue crossed dot {\color{\myBlue}$\otimes$} in the diagrams by replacing $\langle H^\dagger H \rangle = v^2_T / 2$.
\label{fig:geoSMEFT}
}
\end{figure}

The geometric interpretation of the SM is particularly
simple. As we have already seen in Sec.~\ref{sect:custodial}, the scalar sector of the SM 
is invariant under a global $\mathrm{O}(4)$~symmetry,
and the minimum of the scalar potential~$V(\phi)$ 
defines a three-sphere~$S^3$ with 
radius~$v= \sqrt{\langle\boldsymbol{\phi}\cdot\boldsymbol{\phi}\rangle}$.
Conventionally, we align the vev of $\boldsymbol{\phi}$ to its fourth component, i.e., $\langle \boldsymbol{\phi} \rangle = \brackets{0,0,0,v}^\intercal$. This triggers the breaking of the custodial symmetry group~$\mathcal{G}=\mathrm{O}(4)$ down to the subgroup~$\mathcal{H}=\mathrm{O}(3)$, acting only on the first three components of~$\boldsymbol{\phi}$.
Expressing~$\phi$ in terms of the radial component~$h$ and the 
three Goldstone bosons as in~\eqref{eq:geoSMEFT_Cartesian_coordinates},
the scalar Lagrangian~\eqref{eq:geoSMEFT_SM_Lagrangian} assumes the form
\begin{align}
\begin{split}
	\L_\varphi 
	\!=\! \frac{1}{2} \!\brackets{D_\mu \boldsymbol{\varphi}}\! \cdot\! \brackets{D_\mu \boldsymbol{\varphi}} 
	+\! \frac{1}{2}\! \brackets{\partial_\mu h}^2
    -\! \frac{\lambda}{8}\! \brackets{h^2 + 2hv + \boldsymbol{\varphi}\!\cdot\!\boldsymbol{\varphi}}^2 \!.
\end{split}	
\label{eq:geoSMEFT_SM_Lagrangian_2}
\end{align}
The real scalar fields~$\varphi^a$ with $a \in \{1,2,3\}$ transform in the vector representation of~$\mathcal{H}$, whereas the physical Higgs~$h$ transforms as a singlet under~$\mathcal{H}$. Together these four real scalar fields constitute coordinates in the scalar field space of the~SM.


\subsubsection{Geometric formulation of the SMEFT}
\label{sec:geometric_formulation_SMEFT}
The generic kinetic term for a scalar field~$\Phi^i$ in a general scalar field space is
\begin{align}
	\L_\mathrm{kin} 
	= \frac{1}{2} g_{ij}(\Phi) \brackets{D_\mu \Phi}^i \brackets{D_\mu \Phi}^j \, ,
	\label{eq:geoSMEFT_SM_kinetic}
\end{align}
where $g_{ij}(\Phi)$ is the metric of the scalar field space. Comparing Eq.~\eqref{eq:geoSMEFT_SM_Lagrangian} with~\eqref{eq:geoSMEFT_SM_kinetic}, by choosing $\Phi^i=\phi^i$ and promoting partial to covariant derivatives, we find ${g_{ij}(\boldsymbol{\phi})=\delta_{ij}}$. Therefore, the scalar field manifold of the Standard Model is $\mathcal{M}_\mathrm{SM}=\mathbb{R}^4$, i.e., the scalar field space is flat four-dimensional Euclidean space and the fields~$\phi^i$ (with $i\in\{1,2,3,4\}$) or equivalently $\varphi^a$ and~$h$ (with $a\in\{1,2,3\}$) define a Cartesian coordinates system on~$\mathcal{M}_\mathrm{SM}$, as shown at the top of Fig.~\ref{fig:geoSMEFT_SM_manifold}. The black dot in the center represents $\boldsymbol{\phi}=\boldsymbol{0}$ or equivalently~$H=0$. The blue circle symbolizes the Goldstone boson vacuum manifold~$S^3$ given by the coset space~$\mathcal{G}/\mathcal{H}$. The dashed blue arrow points to the physical vacuum denoted by the green dot, where the Cartesian coordinate system is centered. The direction~$h$ is orthogonal to~$S^3$ and $\varphi^a$ are the remaining three orthogonal directions to~$h$.

\begin{figure}
\centering
\includegraphics[width=0.75\linewidth]{figures/flat-field-manifold.pdf}
\caption{Illustration of the SM flat scalar field manifold $\mathcal{M}_\mathrm{SM}=\mathbb{R}^4$ adapted from figure~1 in \cite{Alonso:2016oah}. The vacuum manifold~$S^3=\mathcal{G}/\mathcal{H}$ is represented by the blue circle with radius~$\langle\phi\rangle=v$, the green dot represents the physical vacuum, and the green axes symbolize the scalar field coordinate system.
In the top figure Cartesian coordinates $\{\varphi^1,\, \varphi^2,\, \varphi^3,\, h\}$ centered at the physical vacuum are chosen, whereas in the figure on the bottom polar coordinates $\{\pi^1,\, \pi^2,\, \pi^3,\, \hat{h}\}$ are used, where $\hat{h}$~is the radial coordinate and the three~$\pi^a$ form a vector~$\boldsymbol{n}(\pi^a) \in S^3$.
\label{fig:geoSMEFT_SM_manifold}}
\end{figure}

We have seen above that the SM corresponds to the simple case of a flat four-dimensional scalar manifold. In the following we generalize our previous considerations by extending the SM by higher-dimensional operators and analyzing the geometric properties of the SMEFT. 
The scalar kinetic term in the SMEFT consists of all terms containing only Higgs doublets and exactly two derivatives acting on them. At dimension six in the Warsaw basis \cite{Grzadkowski:2010es} it reads
\begin{align}
\begin{split}
\L_\mathrm{SMEFT}^{H,\mathrm{kin}} 
&= \brackets{D_\mu H}^\dagger \brackets{D^\mu H} \\
&+ \frac{C_{HD}}{\Lambda^2} \brackets{H^\dagger D_\mu H}^\ast \brackets{H^\dagger D^\mu H} \\
&+ \frac{C_{H\Box}}{\Lambda^2} \brackets{H^\dagger H} \Box \brackets{H^\dagger H}
+ \ord{\Lambda^{-4}} \, .
\label{eq:geoSMEFT_L_kin1}
\end{split}
\end{align}
Using only $\SU{2}_L \otimes \mathrm{U}(1)_Y$ gauge invariance and the $\mathrm{SU}(2)$~Fierz identity in Eq.~\eqref{eq:SUN-Fierz} it is straight forward to show that at higher powers only two different and independent operator structures can appear at each mass dimension\footnote{See also \cite{Helset:2020yio} and references therein. Notice, however, that we use a different operator definition, and thus a different basis. The two bases differ only by a Fierz redefinition and thus reproduce the same metric.}
\begin{align}
    Q_{H,\mathrm{kin}}^{(8+2n)} &= \left( H^\dagger H \right)^{n+2} \brackets{D_\mu H}^\dagger \brackets{D^\mu H} \,,
    \label{eq:K-kin-1}
    \\
    Q_{H\!D}^{(8+2n)} &= \left( H^\dagger H \right)^{n+1} \brackets{H^\dagger D_\mu H}^\ast \brackets{H^\dagger D^\mu H} \,.
    \label{eq:K-kin-2}
\end{align}

Using Eq.~\eqref{eq:geoSMEFT_H_to_phi} we can express~$\L_\mathrm{SMEFT}^{H,\mathrm{kin}}$ (and these operators) in terms of the real scalar coordinates~$\phi^i$. The general expression for the kinetic term of the SMEFT in terms of the coordinates~$\phi^i$ is given by\footnote{Similar expressions have been given in \cite{Alonso:2016oah} for the SMEFT in the custodial limit. Notice, however, that the operator~$Q_{H\!D}$ breaks custodial symmetry. Therefore, the formulae presented here are more general.}
\begin{align}
\begin{split}
\L_\mathrm{SMEFT}^\mathrm{kin} 
= &\frac{1}{2} \left[A\brackets{\!\frac{\boldsymbol{\phi}\cdot\boldsymbol{\phi}}{\Lambda^2}\!} \brackets{D_\mu \boldsymbol{\phi}}\cdot\brackets{D^\mu \boldsymbol{\phi}} \right.
\\
&\left. +B\!\brackets{\!\frac{\boldsymbol{\phi}\cdot\boldsymbol{\phi}}{\Lambda^2}\!}\! \frac{(D_\mu \phi)^i \, \mathfrak{f}_{ij}(\phi) \, (D_\mu \phi)^j }{\Lambda^2} \right] ,
\end{split}
\label{eq:geoSMEFT_L_kin2}
\end{align}
where we have defined
\begin{align}
    \mathfrak{f}^{ij}(\phi) &= \begin{pmatrix}
	a & 0 & b & c \\
	0 & a & c & -b \\
	b & c & d & 0 \\
	c & -b & 0 & d
\end{pmatrix}
\,,
&
\begin{pmatrix}
    a\\b\\c\\d
\end{pmatrix}
&=
\begin{pmatrix}
    (\phi^1)^2+(\phi^2)^2
    \\
    \phi^1\phi^3 - \phi^2\phi^4
    \\
    \phi^1\phi^4 + \phi^2\phi^3
    \\
    (\phi^3)^2+(\phi^4)^2
\end{pmatrix}
.
\end{align}
This expression is, of course, only valid for a specific choice for the operator basis. Here, $A$~and~$B$ are defined through a power series expansion in their argument $z \equiv \brackets{\boldsymbol{\phi}\cdot\boldsymbol{\phi}}/\Lambda^2$, which simplifies to the usual EFT expansion in~$v_T^2 / \Lambda^2$ after spontaneous symmetry breaking. 
Since the above equation has to reduce to the SM case in the limit $\Lambda \to \infty$ we must have $A(0)=1$ and~$B(0)=0$. Comparing again to the general scalar kinetic term on a curved manifold in Eq.~\eqref{eq:geoSMEFT_SM_kinetic}, we find the SMEFT scalar field space metric
\begin{align}
g_{ij} (\phi) 
&= A\!\brackets{\!\frac{\boldsymbol{\phi}\cdot\boldsymbol{\phi}}{\Lambda^2}\!} \delta_{ij}
+ B\!\brackets{\!\frac{\boldsymbol{\phi}\cdot\boldsymbol{\phi}}{\Lambda^2}\!} \frac{\mathfrak{f}_{ij}(\phi)}{\Lambda^2} \, .
\label{eq:geoSMEFT_metric}
\end{align}
\newline
This metric describes, in general, a curved manifold~$\mathcal{M}_\mathrm{SMEFT}$ and only for~$B=0$ the manifold is flat. In the limit~$\Lambda \to \infty$ we find that the SMEFT metric reduces to the SM metric in Cartesian coordinates
\begin{align}
g_{ij}^\mathrm{SMEFT} (\phi) \quad \xrightarrow{~\Lambda\to\infty~} \quad \delta_{ij} = g_{ij}^\mathrm{SM} \, .
\end{align}

Therefore, the curvature of~$\mathcal{M}_\mathrm{SMEFT}$ is determined entirely by the EFT expansion parameter~$\smash{{v_T^2}/{\Lambda^2}}$. Furthermore, we can deduce all kinds of geometric quantities such as Christoffel symbols, Riemann curvature tensors, etc. 
%\felix{Should we actually give some example here?} from the metric in Eq.~\eqref{eq:geoSMEFT_metric}.  GI: No !

Taking Eq.~\eqref{eq:geoSMEFT_L_kin1} we find the scalar metric of the SMEFT up to dimension six \cite{Helset:2018fgq}
\begin{align}
g_{ij}(\phi) &= \delta_{ij} + \frac{C_{HD}}{2\Lambda^2} \mathfrak{f}_{ij}(\phi) -2 \frac{C_{H\Box}}{\Lambda^2} \phi_i \phi_j \,.
\label{eq:SMEFT-metric-d6}
\end{align}
Notice that the last term in the above expression does not match the general form of the metric in Eq.~\eqref{eq:geoSMEFT_metric}. This is because the operator~$Q_{H\Box}$ of the Warsaw basis does not agree with the definitions in Eqs.~\eqref{eq:K-kin-1}--\eqref{eq:K-kin-2}. It could, of course, be rewritten in that form by using integration by parts identities.
Apart from $\smash{Q_{H,\mathrm{kin}}^{(6)}}$, this would introduce further operators that have to be removed using field redefinitions.\footnote{The replacement reads $Q_{H\Box}=2Q_{H,\mathrm{kin}}^{(6)}+\ldots$, where the ellipses denote terms that do not contribute to the metric after applying the appropriate redefinition of the Higgs field~$H$.} The latter, however, change the scalar field space metric. Thus, the last term in Eq.~\eqref{eq:SMEFT-metric-d6} could be removed in favor of a term proportional to~$\delta_{ij}$, but for consistency we decided to stick with the Warsaw basis at dimension six.
In this example we see that the explicit form of the metric is basis dependent. However, a geometric formulation of the SMEFT exists in every basis \cite{Helset:2020yio}.

We can now use Eqs.~\eqref{eq:K-kin-1}--\eqref{eq:K-kin-2} and our previous results to define the scalar field metric to all orders in the EFT power counting
\begin{align}
\begin{split}
g_{ij} 
&= \squarebrackets{1 + \sum_{n=0}^{\infty} \brackets{\frac{\boldsymbol{\phi}\cdot\boldsymbol{\phi}}{2}}^{\!n+2} \frac{C_{H,\mathrm{kin}}^{(8+2n)}}{\Lambda^{4+2n}}} \delta_{ij}  
\\
&\ +\frac{1}{2} \squarebrackets{C_{H\!D}^{(6)} + \sum_{n=0}^\infty \! \brackets{\frac{\boldsymbol{\phi}\cdot\boldsymbol{\phi}}{2}}^{\!n+1} \frac{C_{H\!D}^{(8+2n)}}{\Lambda^{4+2n}}} \! \mathfrak{f}_{ij}(\phi) 
\\
&\ -2 \frac{C_{H\Box}}{\Lambda^2} \phi_i \phi_j\, .
\end{split}
\end{align}
% 
% \begin{widetext}
% \begin{align}
% g_{ij} 
% &= \squarebrackets{1 + \sum_{n=0}^{\infty} \brackets{\frac{\boldsymbol{\phi}\cdot\boldsymbol{\phi}}{2}}^{\!n+2} \frac{C_{H,\mathrm{kin}}^{(8+2n)}}{\Lambda^{4+2n}}} \delta_{ij}  
% + \frac{1}{2} \squarebrackets{C_{H\!D}^{(6)} + \sum_{n=0}^\infty \brackets{\frac{\boldsymbol{\phi}\cdot\boldsymbol{\phi}}{2}}^{\!n+1} \frac{C_{H\!D}^{(8+2n)}}{\Lambda^{4+2n}}} \mathfrak{f}_{ij}(\phi) - 2 \frac{C_{H\Box}}{\Lambda^2} \phi_i \phi_j\, .
% \end{align}
% \end{widetext}
% 
Of course, this entails a choice of basis, nevertheless it is remarkable that we are able to define this geometric quantity to all orders in the EFT power counting.

The ideas discussed so far in this section apply to the Higgs two-point function leading to the scalar field space metric. Following \cite{Helset:2020yio}, we can generalize the concepts to higher $n$-point functions and other types of field connections by factorizing the operators in the SMEFT Lagrangian
\begin{align}
\L_\mathrm{SMEFT} &= \sum_{n} f_n\brackets{\mu,\alpha,\ldots} \, G_n\brackets{I,A,\ldots} \, .
\end{align}
The factors~$f_n$ are composite operator forms containing all non-scalar fields and all dependence on spacetime indices, i.e. Lorentz~$(\mu,\ldots)$, and spinor~$(\alpha,\ldots)$ indices. The~$f_n$ can only depend on the scalar field coordinates through derivatives acting on the scalars, e.g.~$\brackets{D_\mu H}$.
The factors~$G_n$, on the other hand, depend on the non-spacetime group indices~$(I,A,\ldots)$ and contain only scalar field coordinates and symmetry generators acting on them, i.e., expressions built only out of~$H^{(\dagger)}$ and~$\tau^I$. It is evident that after electroweak symmetry breaking the~$G_n$ collapse to a number and an appropriate power of Higgs~$h$ emissions, largely factoring out the expansion in~$v_T^2 / \Lambda^2$ from the remaining composite operator form~$f_n$, whereas the latter~$(f_n)$ contain the derivative expansion in~$p^2 / \Lambda^2$ and only retain a minimal dependence on the scalar coordinates and~$v_T$ mixing the two expansions \cite{Helset:2020yio}.

This allows us to define the scalar field metric
\begin{align}
g_{ij} \brackets{\phi} &= \left. \frac{g^{\mu\nu}}{D} \frac{\delta^2 \L_\mathrm{SMEFT}}{\delta\!\brackets{D^\mu \phi}^i \, \delta\!\brackets{D^\nu \phi}^j} \right|_{f_n\to 0} \, .
\end{align}
% where $\L\brackets{\alpha,\beta,\ldots}$ denotes all parts of the Lagrangian that depend on spacetime indices. 
Similarly, we can now define all sorts of field-space connection, e.g., the Yukawa-type connection \cite{Helset:2020yio} we already encountered
\begin{align}
[\mathsf{Y}_{\psi}]_{pr} \brackets{\phi_i} &= \left. \frac{\delta \L_\mathrm{SMEFT}}{\delta ( \overline{\psi}{}^{L,i}_{p} \psi^R_{r} ) } \right|_{f_n\to 0} \, .
\end{align}
For this case we find $\smash{f_n} = \smash{\overline{\psi}{}^{L,i}_{p} \psi^R_{r}}$ containing the fermion bilinear and the factor $\smash{G_n}\sim \smash{\sum_k (H^\dagger H)^k H_i}$. 
Comparing to Fig.~\ref{fig:geoSMEFT} we see that $f_n$~corresponds to the fermion current, whereas the~$G_n$ corresponds to the emission of~$h$ and the vev~(marked by {\color{\myBlue}$\otimes$}). The operators contributing to~$G_n$ at all orders in this case are \cite{Helset:2020yio}
\begin{align}
	[Q_{\psi H}^{(6+2n)}]_{pr} &= \left(H^\dagger H\right)^{n+1} \left(\overline{\psi}{}^{L,i}_{p} \psi^R_{r} H_i\right)
\end{align}
leading to the all-order Yukawa connection
\begin{align}
\begin{split}
	[\mathsf{Y}_\psi]_{pr} \brackets{\phi} = &-H\brackets{\phi} [Y_\psi]_{pr} 
    \\
    &+ H\brackets{\phi} \sum_{n=0}^{\infty} \frac{[C_{\psi H}^{(6+2n)}]_{pr}}{\Lambda^{2+2n}} \brackets{\frac{\boldsymbol{\phi}\cdot\boldsymbol{\phi}}{2}}^{n+1}.
\end{split}
\end{align}
For more details and the definition of other field-space connections, as well as for more all order results, see \cite{Helset:2020yio}.
The key advantage of this formulation is the reduction of the number of relevant structures, especially when going beyond dimension six, and obtaining all-order results (or better results independent of the operator power counting) 
for a series of relevant quantities, such as the physical fermion masses.


\subsubsection{Geometric formulation of the HEFT}
So far we have used Cartesian coordinates to describe the scalar field manifold~$\mathcal{M}$. We can equally well choose polar coordinates on~$\mathcal{M}$, since any measurable quantity does not depend on the choice of the coordinate system.
Following \cite{Alonso:2016oah}, we can use polar coordinates to write the real scalar fields as
\begin{align}
\boldsymbol{\phi} = \brackets{v+\hat{h}} \boldsymbol{n}(\pi) \quad \text{where} \quad \boldsymbol{n}(\pi)\in S^3 \sim \mathcal{G}\big/\mathcal{H}
\label{eq:geoSMEFT_polar_coordinates_parametriztion}
\end{align}
with the radial coordinate~$\hat{h}$ and the three angular coordinates~${\pi^a}/{v}$ associated to the Goldstone bosons of the broken generators. The three angular coordinates~$\pi^a$ form a four-dimensional unit vector~$\boldsymbol{n}(\pi^a) \in S^3$. The polar coordinate system is shown on the bottom of Fig.~\ref{fig:geoSMEFT_SM_manifold}. In the polar coordinates parametrization~\eqref{eq:geoSMEFT_polar_coordinates_parametriztion} there is no obvious relation among the physical Higgs field~$\hat{h}$ and the Goldstone bosons in~$\boldsymbol{n}$, contrary to the case of Cartesian coordinates in Eq.~\eqref{eq:geoSMEFT_Cartesian_coordinates} where such a relation is implicit as $h$~and~$\varphi^a$ transform together in the vector representation of~$\mathcal{G}=\mathrm{O}(4)$ as $\boldsymbol{\phi}\to O\boldsymbol{\phi}$ with~$O\in\mathcal{G}$.

The transformation properties of the coordinates under the chiral symmetry group~$\mathcal{G}$ are
\begin{align}
\hat{h} \xrightarrow{~\mathcal{G}~} \hat{h} \, , && \boldsymbol{n} \xrightarrow{~\mathcal{G}~} O\,\boldsymbol{n}\quad \text{with } O \in \mathcal{G} \, .
\label{eq:geoSMEFT_polar_transformation_laws}
\end{align}
The field~$\hat{h}$ is a singlet, whereas~$\boldsymbol{n}$ transforms linearly under~$\mathcal{G}$. However, due to the constraint $\boldsymbol{n}\cdot\boldsymbol{n}=1$, this four-component vector has only the three independent components~$\pi^a$ with~$a\in\{1,2,3\}$. Therefore, the~$\pi^a$ do not transform linearly under~$\mathcal{G}$, which is why this choice of coordinates is called the non-linear representation.\footnote{Contrary to Eq.~\eqref{eq:HEFT-non-linear-field-redefinition}, we use the vector notation with the field~$\boldsymbol{n}$ here rather than the matrix notation with the field~$U$. Nevertheless, the two formulation are completely equivalent.} 
Possible parametrizations are the square root parametrization $\boldsymbol{n} (\pi) = \smash{\left(\pi^1, \pi^2, \pi^3, \sqrt{v^2 - \boldsymbol{\pi}\cdot\boldsymbol{\pi}} \right)^\intercal \!\!\big/ v}$, and the exponential representation
\begin{align}
\begin{split}
    \boldsymbol{n} (\pi) 
    \!=\! \exp\!\! \left(\!\! \frac{1}{v}\!\! \left[\begin{matrix}
        0 & \!0 & \!0 & \!\pi^1
        \\
        0 & \!0 & \!0 & \!\pi^2
        \\
        0 & \!0 & \!0 & \!\pi^3
        \\
        -\pi^1 & \!-\pi^2 & \!-\pi^3 & \!0
    \end{matrix} \right]\!\right)\!\!\! 
    \begin{pmatrix}
        0 \\ 0 \\ 0 \\ 1
    \end{pmatrix} 
    \!\!=\!\!
    \begin{pmatrix}
        \!\sin\! \left(\!\frac{|\boldsymbol{\pi}|}{v}\!\right) \!\frac{\pi^1}{|\boldsymbol{\pi}|}\!\!\!
        \\
        \!\sin\! \left(\!\frac{|\boldsymbol{\pi}|}{v}\!\right) \!\frac{\pi^2}{|\boldsymbol{\pi}|}\!\!\!
        \\
        \!\sin\! \left(\!\frac{|\boldsymbol{\pi}|}{v}\!\right) \!\frac{\pi^3}{|\boldsymbol{\pi}|}\!\!\!
        \\
        \!\cos\! \left(\!\frac{|\boldsymbol{\pi}|}{v}\!\right)
    \end{pmatrix}\!,
\end{split}
\end{align}
where $|\boldsymbol{\pi}|=\sqrt{\boldsymbol{\pi}\cdot\boldsymbol{\pi}}$.
The latter corresponds to the standard coordinates of CCWZ \cite{Alonso:2016oah}. However, we will not pick any explicit parametrization here.

Using Eq.~\eqref{eq:geoSMEFT_polar_coordinates_parametriztion} to express the scalar part of the SM Lagrangian~\eqref{eq:geoSMEFT_SM_Lagrangian} in polar coordinates yields
\begin{align}
\begin{split}
\L 
= \frac{1}{2} & \brackets{v+\hat{h}}^2 \!\brackets{D_\mu \boldsymbol{n}} \! \cdot \! \brackets{D^\mu \boldsymbol{n}}
+ \frac{1}{2} \brackets{\partial_\mu \hat{h}} \! \brackets{\partial^\mu \hat{h}}
\\
&- \frac{\lambda}{8} \brackets{\hat{h}^2 + 2v\hat{h}}^2 \, ,
\end{split}
\label{eq:geoSMEFT_L_SM_polar}
\end{align}
where the Goldstone bosons of~$\boldsymbol{n}$ are only derivatively coupled and the potential is independent of the angular coordinates~$\pi^a$, contrary to the case of Cartesian coordinates in Eq.~\eqref{eq:geoSMEFT_SM_Lagrangian_2}. Instead of changing the coordinate system on the scalar field manifold~$\mathcal{M}$, we could equally well do a field redefinition. Using Eqs.~\eqref{eq:geoSMEFT_Cartesian_coordinates} and~\eqref{eq:geoSMEFT_polar_coordinates_parametriztion} we find $(v+h)^2+\boldsymbol{\varphi}\cdot\boldsymbol{\varphi} = (v+\hat{h})^2$ which yields
\begin{align}
	\hat{h} &= h + \frac{\boldsymbol{\varphi}\cdot\boldsymbol{\varphi}}{2v} - \frac{h}{2} \frac{\boldsymbol{\varphi}\cdot\boldsymbol{\varphi}}{v^2} + \mathcal{O}(v^{-3}) \,,
	\label{eq:geoSMEFT_higgs_field_relation}
\end{align}
where $h$ and $\hat{h}$ are the Higgs fields in Cartesian and polar coordinates, respectively.

As we have discussed before, in the Cartesian coordinate system, the Higgs field in Eq.~\eqref{eq:geoSMEFT_H_to_phi} or the corresponding real scalar fields~$\boldsymbol{\phi}$ in Eq.~\eqref{eq:geoSMEFT_Cartesian_coordinates} transform linearly under~$\mathcal{G}$ or the electroweak symmetry group. On the contrary, in the polar coordinate system the scalar fields~$\boldsymbol{\pi}$ do not transform linearly. However, physical observables must be independent of the choice of coordinates and, therefore, the SM~Lagrangians in Eq.~\eqref{eq:geoSMEFT_L_SM_polar} and Eq.~\eqref{eq:geoSMEFT_SM_Lagrangian_2} are equivalent as they only differ by a coordinate redefinition.
The question whether the Higgs transforms linearly or non-linearly under the electroweak symmetry group is thus depending on the choice of coordinate system and is therefore unphysical \cite{Alonso:2016oah}. The appropriate question is whether it is always possible to pick a coordinate system in which the Higgs field transforms linearly. As we have seen above this is true for the SM, but as we will discuss below this is not possible in general for all EFT extensions of the~SM. 

In fact, it is only possible if and only if the scalar field manifold~$\mathcal{M}$ has a $\mathcal{G}$~invariant fixed point \cite{Alonso:2016oah}. In a neighborhood of this fixed point it is then possible to pick a coordinate system in which the Higgs field transforms linearly under~$\mathrm{O}(4)$. For the SM, this fixed point is the origin~$\boldsymbol{\phi}=\boldsymbol{0}$ (black dot) as can be seen in both parts of Fig.~\ref{fig:geoSMEFT_SM_manifold}.

We have seen in the Sec.~\ref{sec:geometric_formulation_SMEFT} that the SMEFT is the extension of the SM with higher-dimensional operators using Cartesian coordinates on~$\mathcal{M}$. The Higgs field~$H$ transforms linearly under~$\mathcal{G}$ or the electroweak gauge group and the SMEFT has a $\mathcal{G}=\mathrm{O}(4)$~fixed point at the origin~$\boldsymbol{\phi}=\boldsymbol{0}$. 
On the contrary, the HEFT is the EFT extension of the SM using polar coordinates on the scalar field manifold~$\mathcal{M}_\mathrm{HEFT}$. The corresponding scalar part of the HEFT Lagrangian is a generalization of Eq.~\eqref{eq:geoSMEFT_L_SM_polar} and given by
\begin{align}
\begin{split}
\L_\mathrm{HEFT} = &\frac{1}{2} v^2 F(\hat{h})^2 \brackets{D_\mu \boldsymbol{n}}\cdot\brackets{D^\mu \boldsymbol{n}}
\\
&+ \frac{1}{2} \brackets{\partial_\mu \hat{h}}\brackets{\partial^\mu \hat{h}} - V(\hat{h}) \, ,
\end{split}
\label{eq:geoSMEFT_L_HEFT}
\end{align}
where $V(\hat{h})$ is the scalar potential that only depends on the radial coordinate~$\hat{h}$, and $F(\hat{h})$ is a generic dimensionless function that is defined by a power expansion in~$\hat{h}/v$ with~$F(0)=1$ such that the radius of the vacuum manifold is fixed by~$v$ \cite{Alonso:2016oah}. The $\mathcal{G}$~transformation rules for the fields~$\hat{h}$ and~$\boldsymbol{n}$ are the same as in Eq.~\eqref{eq:geoSMEFT_polar_transformation_laws}. We can write
\begin{align}
F(\hat{h}) &= 1 + c_1 \brackets{\frac{\hat{h}}{v}} + c_2 \brackets{\frac{\hat{h}}{v}}^2 + \cdots
%\ord{\brackets{\frac{\mathsf{h}}{v}}^3} 
\, 
\end{align}
and, as already stated in Sec,~\ref{sect:HEFT1}, we recover 
the SM case $F_\mathrm{SM} (\hat{h}) = (1+\hat{h}/v)$ for $c_1=1$ and~$c_{n\geq 2}=0$. 

Defining the HEFT scalar field space metric as in Eq.~\eqref{eq:geoSMEFT_SM_kinetic} by choosing $\Phi=\smash{\big(\pi^1,\pi^2,\pi^3,\hat{h}\big)^\intercal}$, we find \cite{Alonso:2015fsp}
% Comparing Eq.~\eqref{eq:geoSMEFT_L_HEFT} with Eq.~\eqref{eq:geoSMEFT_SM_kinetic} (by choosing $\Phi=(\pi^1,\pi^2,\pi^3,\hat{h})^\intercal$) we find the general scalar bilinear HEFT metric~\cite{Alonso:2015fsp}
\begin{align}
g_{ij}^\mathrm{HEFT} (\phi) &= \begin{pmatrix}
	F(\hat{h}) \, g_{ab}(\pi) & 0 \\
	0 & 1
\end{pmatrix}_{\!\!ij} \,,
\end{align}
where $g_{ab}$~is the $\mathcal{H}=\mathrm{O}(3)$~invariant metric on the coset space~$S^3=\mathcal{G}/\mathcal{H}$ for the angular coordinates~$\boldsymbol{\pi}$. 

The scalar field manifold~$\mathcal{M}_\mathrm{HEFT}$ for HEFT is shown in Fig.~\ref{fig:geoSMEFT_HEFT_manifold}. The manifold consists of~$\hat{h}$ (green arrow) and a sequence of three-spheres~$(S^3)$ of radius~$vF(\hat{h})$ fibered over any value of~$\hat{h}$ (the blue circle symbolizes the sphere for one particular value of~$\hat{h}$). From Eq.~\eqref{eq:geoSMEFT_polar_transformation_laws} we know that $\mathcal{G}$~acts on any point~$\boldsymbol{n}\in S^3$ by rotations on the surface of~$S^3$, i.e., rotations along the blue circle. Therefore, it is only possible to have an $\mathcal{G}=\mathrm{O}(4)$~invariant fixed point if the radius of the vacuum sphere is vanishing, meaning if there exists some~$\hat{h}_\ast$ for which we have~$F(\hat{h}_ \ast)=0$. In the SM this is the case for~$\hat{h}_\ast^\mathrm{SM}=-v$. However such a value~$\hat{h}_\ast$ does not exist in general as can be seen by the example~$F(\hat{h})=e^{\hat{h}/v} \smash{\cosh\big(1+\hat{h}/v\big)}$ which is non-vanishing for all~$\hat{h}$ \cite{Alonso:2016oah}. In this case the green dashed range of~$\hat{h}$ in Fig.~\ref{fig:geoSMEFT_HEFT_manifold} does not exist.

\begin{figure}
\centering
\includegraphics[]{figures/HEFT-field-manifold.pdf}
\caption{The scalar field manifold~$\mathcal{M}_\mathrm{HEFT}$ of the HEFT is fibered with a three-sphere~$S^3$ of radius~$vF(\hat{h})$ for every~$\hat{h}$. An $\mathrm{O}(4)$~fixed point~$\phi_0$ does only exist if there is a value~$\smash{\hat{h}_\ast}$ such that the radius of the three-sphere vanishes~$\smash{F(\hat{h}_\ast)=0}$. If no fixed point~$\phi_0$ exists, the green dashed region does not exist and the manifold might either be smoothly connected without a fixed point or extend to infinity. The SMEFT corresponds to the theories with an $\mathrm{O}(4)$~fixed point at~$\phi_0=0$. For this type of theory it is possible to change from polar coordinates to Cartesian coordinates and vice versa in a neighborhood of the fixed point. Figure adapted from \cite{Alonso:2016oah}.
\label{fig:geoSMEFT_HEFT_manifold}
}
\end{figure}

In summary, we have found that the most general EFT extension of the SM is the HEFT using polar coordinates on the scalar field manifold. In this framework the Goldstone bosons transform non-linearly under the electroweak symmetry group or the larger custodial symmetry group~$\mathcal{G}=\mathrm{O}(4)$. The subcategory of EFTs that have a $\mathcal{G}$~fixed point at the origin belong to the SMEFT class. For these theories it is possible to pick coordinates around this fixed point in which the Higgs field transforms linearly. Eventually, the SM is a subcategory of SMEFT with a flat scalar field manifold~$\mathcal{M}_\mathrm{SM}=\mathbb{R}^4$. We can therefore schematically write
\begin{align}
	\mathrm{SM} \subseteq \mathrm{SMEFT} \subseteq \mathrm{HEFT} \, .
  \label{eq:SMEFTHEFT}
\end{align}

Given the relation~\eqref{eq:SMEFTHEFT}, 
the last few years have seen an intense activity in 
identifying  concrete examples of UV models that cannot be well described by the 
SMEFT. As pointed out in~\cite{Cohen:2020xca}, this can happen under two conditions:
i)~when (non-SM) particles which acquire mass via electroweak symmetry breaking are integrated out, introducing non-analytic dependence from the Higgs 
field in the corresponding EFT~\cite{Falkowski:2019tft};
ii)~when additional sources of electroweak symmetry breaking are present besides a single scalar doublet. 
These two conditions signal that the $\mathrm{O}(4)$~fixed point 
in the scalar manifold is not the most convergent choice 
as origin for a Taylor expansion. However, such point can still exists also in this class of models and the SMEFT description can still be quite effective.
An instructive comparison of SMEFT and HEFT for a concrete UV model can be found in \cite{Buchalla:2016bse}.

On general grounds, a breakdown of the SMEFT description happens only if the non-SM states integrated out, connected to the mechanism of electroweak symmetry breaking
(either as sources of the breaking or because they acquire mass via this breaking), are sufficiently close to the electroweak scale~\cite{Banta:2021dek}. It is fair to say that
no indication of such states is present in current high-energy data.


At low energies, 
a pragmatic way to distinguish the two EFTs is by looking at transition amplitudes with identical electroweak and flavor structure, 
that differ only for the number of (massive) Higgs fields \cite{Isidori:2013cga,Isidori:2013cla,Brivio:2013pma}.
In the SMEFT, the linear realization implies a well-defined relation 
among all these processes at a given order in the EFT expansion. This relation can be broken at higher orders; however, the effect is expected to be small according to the power counting. On the contrary, in the HEFT the 
$F(\hat{h})$ function, and its analog for other electroweak structures, lead to a potential complete decoupling among processes with different number 
of Higgs fields. A particularly interesting study case is provided 
by non-universal corrections to the $Z\to f\bar f$ couplings \cite{Isidori:2013cga}.
Measurements at the $Z$~pole imply very small deviations from the SM, implying strong bounds on several operators in class~7 of Tab.~\ref{tab:Warsaw-basis} which control these effects. 
Within the SMEFT, this implies in turn tiny deviations from the SM
in the related processes $h\to Z f\bar f$ and $f \bar f \to Z h$.
A large deviation from the SM in the latter processes could occur naturally in the HEFT, while it would imply a breakdown of the SMEFT power counting.



