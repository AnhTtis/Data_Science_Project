\section{Low-Energy effective field theory}
\label{sect:LEFT}
% This section is about the matching of the SMEFT to the LEFT
% \begin{itemize}
% 	\item Some overview~\cite{Buchalla:1995vs}
% 	\item Classification of operators and matching to SMEFT~\cite{Jenkins:2017jig} = paper 1
% 	\item RG evolution of the LEFT operators~\cite{Jenkins:2017dyc} = paper 2
% 	\item Matching of SMEFT to LEFT at one-loop~\cite{Dekens:2019ept} = paper 3
% 	\item A dimension-eight basis~\cite{Murphy:2020cly}
% \end{itemize}

\subsection{Introduction and overview}

The success of the SM rests to a large degree on tests in low-energy processes such as decays of Kaons, $D$-mesons, and even more on $B$~physics, because these processes can be calculated with rather high precision.\footnote{See \cite{Buchalla:1995vs} for an early review.} The tool for this is the so called low-energy effective field theory~(LEFT),\footnote{Sometimes this theory is also called the weak effective theory~(WET).} which is derived from the SM by integrating out the Higgs boson~($h$), the weak gauge bosons~($\mathcal{Z},\mathcal{W}$), as well as the top quark~($t_L, t_R$). Of course, this is a generalization of the original Fermi theory with the four-fermion interaction 
\begin{equation}
-\frac{4 G_F}{\sqrt{2}} \left( \overline{\psi}\gamma_\mu\psi)(\overline{\psi}\gamma^\mu\psi \right) \,,
\label{eq:4fermi}
\end{equation}
where the Fermi constant~$G_F$ is related to the vacuum expectation value~$v$ by $G_F = 1/(\sqrt{2} v^2)$. The method works so well because the relevant energies~$E$ are much smaller than $v$~or~$m_W$ (and of course than~$\Lambda$) and due to the asymptotic freedom of the strong interactions. Since the pioneering work in the mid 1970s, this theory has been developed to an astonishing degree of precision by including all kinds of strong and electromagnetic corrections. See, e.g., \cite{Buras:2020xsm} for a recent review. 

The LEFT is thus an $\mathrm{SU}(3)_c \times \mathrm{U}(1)_e$ invariant effective theory valid below the electroweak symmetry breaking scale containing five quark flavors $(u,d,s,c,b)$, three charged leptons $(e,\mu,\tau)$, three left-handed neutrinos~$\smash{(\nu_e,\nu_\mu,\nu_\tau)}$, the gluons, and the photon. The LEFT Lagrangian is the sum of the Lagrangians of QCD and QED of these particles and the mass terms of the fermions
\begin{align}
\mathcal{L}_\mathrm{SM}^\mathrm{broken} 
= &-\frac{1}{4}F_{\mu\nu}F^{\mu\nu} -\frac{1}{4}G_{\mu\nu}^A G^{A\,\mu\nu} -\theta_3\frac{g_3^2}{32\pi^2}G_{\mu\nu}^A \tilde{G}^{A\,\mu\nu}  
\nonumber\\
&+ \sum_{\psi=u,d,e,\nu_L} \sum_{X=L,R} \left( \overline{\psi}{}^X_{p} \, i\slashed{D} \,\psi^X_{p} \right)
\label{QFT}\\
&-  \left[ \sum_{\psi=u,d,e} [\mathcal{M}_\psi]_{pr} \, \left( \overline{\psi}{}^L_{p} \psi^R_{r} \right) + \mathrm{h.c.} \right]
\nonumber
\end{align}
and a series of higher-dimensional operators~$(\mathcal{Q})$, to be made precise later
\begin{align}
\mathcal{L}_\mathrm{LEFT} &= \mathcal{L}_\mathrm{SM}^\mathrm{broken} + \mathcal{L}_\mathrm{EFT} \,,
\\
\mathcal{L}_\mathrm{EFT} &=  \sum_{n=-1}^\infty \sum_i \frac{\mathcal{C}_i^{(n)}(\mu)}{v^n} \mathcal{Q}_i^{(n)} (\mu)
\label{leff}
\end{align}
arising from the interactions with the heavy particles that were integrated out; the best known being the four-fermion operator in Eq.~\eqref{eq:4fermi}. Here the flavor indices~$p,r$ run over the values~$1,2,3$ for~$\psi=d,e,\nu$ and over~$1,2$ for~$\psi=u$.
These operators are organized by their dimension, starting with terms of dimension three, and increasing powers of~$1/v$ (often expressed by the Fermi constant~$G_F$). Sometimes also $m_W={\mathcal{O}(1) \times v}$ is used as expansion parameter instead. In a SMEFT theory, there is an additional expansion in powers of $1/\Lambda = 1/v \times (v/\Lambda)$, and of course, more operators than in the~SM. The LEFT Wilson coefficients~$\mathcal{C}(\mu)$ multiplying the operators depend on the renormalization scale~$\mu$. As a rule, the renormalization scale should be chosen near
to the physically relevant energy, in order to avoid additional large corrections in matrix elements of the operators. On the other hand, the Wilson coefficients~$\mathcal{C}(v)$ from the matching to the underlying model, be it the SM or the SMEFT, are given at the weak scale,~$\mu \approx m_W$. 
The connection between the two scales is realized by the renormalization group and the running of the~$\mathcal{C}(\mu)$ described by the renormalization group equation
\begin{equation}
\dot{\mathcal{C}} = 16 \pi^2 \mu \frac{\dd}{\dd\mu} \mathcal{C} = \beta_\mathcal{C} \,,
\label{eq:RGE}
\end{equation}
where $\beta_\mathcal{C}$~is the beta-function of the coefficient~$\mathcal{C}$.
This implies that the Wilson coefficients can pick up large logarithmic correction of the form~$\log(m_b/m_W)$. Furthermore,  the running of the Wilson coefficients of lower-dimensional operators can be proportional the coefficients of higher-dimensional operators, due to the presence of light scales/masses in the theory. 

As mentioned, a lot of work has been done in developing the LEFT from the~SM. If the underlying theory is the SMEFT rather than the~SM, we need to match the Wilson coefficients of the LEFT to the coefficients in SMEFT. We have to do this matching in the broken phase of the SMEFT, in the same way we do for the~SM.
This implies that additional terms suppressed by appropriate powers of~$1/\Lambda$ must be added in the matching equations for the LEFT coefficients at the scale~$m_W$.


\subsection{Electroweak symmetry breaking in the SMEFT}
\label{sec:EWSB-in-SMEFT}
% following ~\cite{Jenkins:2017jig}, called paper1.
The Lagrangian for the SMEFT in the \textit{unbroken phase}, i.e., above the electroweak symmetry breaking~(EWSB) scale~$\sim v$, has been discussed in Sec.~\ref{smeft}. 
Here we will now consider EWSB in the SMEFT determining the Lagrangian in the \textit{broken phase}. We especially emphasize how EWSB is altered compared to the SM due to the presence of the additional higher-dimensional operators, which modify the definition of several SM parameters at tree level. For this, we will follow the discussions presented in \cite{Alonso:2013hga,Jenkins:2017jig}.

\subsubsection{The Higgs sector}
% This is achieved by a rescaling of the Higgs field and a shift of the vacuum expectation value of the Higgs field by a factor
% \begin{equation}
% 1 + a/ {\lambda^2} + b/ {\lambda^4} + .....
% \label{scaling}
% \end{equation}
% where $a \sim 1 {\lambda^2}$,   $b \sim 1/ {\lambda^4}$  etc. 
% \felix{Not sure is the above is very illustrative.}
In the scalar sector both the Higgs kinetic term as well as the scalar potential are modified in the SMEFT and read
\begin{align}
    \mathcal{L}_{H} 
    &= (D_\mu H)^\dagger (D^\mu H) + m^2 H^\dagger H - \frac{\lambda}{2} (H^\dagger H)^2 
    \nonumber\\
    &+ \frac{C_{H\Box}}{\Lambda^2} (H^\dagger H) \Box (H^\dagger H) + \! \frac{C_{H\! D}}{\Lambda^2} (H^\dagger D_\mu H)^\ast (H^\dagger D_\mu H)
    \nonumber\\
    &+ \frac{C_H}{\Lambda^2} (H^\dagger H)^3 + \mathcal{O}(\Lambda^{-4}) \,.
    \label{eq:SMEFT-higgs-Lagrangian}
\end{align}
In unitary gauge we can write the Higgs doublet as
\begin{align}
H &= \frac{1}{\sqrt{2}} \begin{pmatrix}
    0
    \\
    [1+c_{H,\mathrm{kin}}] h + v_T
\end{pmatrix}\,, \quad \text{where}
\label{eq:Higgs-SMEFT-parametrization}
\\[0.2cm]
c_{H,\mathrm{kin}} &\equiv \left(\! C_{H\Box}-\frac{1}{4}C_{H\!D} \!\right) \frac{v^2}{\Lambda^2} \,, \quad 
v_T \equiv \left(\! 1+\frac{3\,C_{H}}{4\lambda} \frac{v^2}{\Lambda^2} \right)v \,.
\label{shift}
\end{align}
Here, $c_{H,\mathrm{kin}}$ guarantees a canonical normalization of the kinetic term of the physical real Higgs~$h$, and $v_T$~is the vacuum expectation value of the complex Higgs doublet~$H$ in the SMEFT, whereas $v=\sqrt{2 m^2/\lambda}$ is the vev of~$H$ in the SM.\footnote{Notice that we have $v_T^2=v^2+\mathcal{O}(v^2/\Lambda^2)$, and thus, when working at dimension six, we can always replace~$v_T$ by~$v$ when it multiplies a $d=6$~operator.} Substituting the above equations back into~\eqref{eq:SMEFT-higgs-Lagrangian} we find all self-interactions of the physical Higgs~$h$. For example, its mass term~$m_h$, defined by $\smash{\mathcal{L}_h \supset \frac{1}{2} m_h^2 h^2}$, reads
\begin{align}
    m_h^2 &= \lambda v_T^2 \left( 1-\frac{3 C_H}{\lambda} \frac{v^2}{\Lambda^2} + 2 c_{H,\mathrm{kin}} \right) \,.
\end{align}

Equally, all masses and couplings of the particles are modified. This concerns the fermions and their Yukawa couplings as well as the masses and couplings of the gauge bosons, which are modified by similar shifts. 
%The precise expressions can be found in an appendix and in~\cite{Jenkins:2017jig}. For each observable that is tested by experiments, these shifts must be included.
 % Formeln aus paper 1 diskutieren

\subsubsection{The Yukawa sector}
The fermion masses and Yukawa couplings in the broken phase 
\begin{align}
    \mathcal{L}_\mathrm{Yukawa}^\mathrm{broken}
    &= - [\mathcal{M}_\psi]_{pr} (\overline{\psi}{}^L_p \psi^R_r) - [\mathcal{Y}_{\psi h}]_{pr} (\overline{\psi}{}^L_p \psi^R_r) h + \mathrm{h.c.}
    \label{eq:LEFT-mass-Yukawa}
\end{align}
are determined by the parameters~$Y_\psi$ and~$C_{\psi H}$ through
\begin{align}
    [\mathcal{M}_\psi]_{pr} &= \frac{v_T}{\sqrt{2}} \left( [Y_\psi]_{pr} - \frac{1}{2} \frac{v^2}{\Lambda^2} [C_{\psi H}]_{pr} \right) \,,
    \\[0.2cm]
    [\mathcal{Y}_{\psi h}]_{pr} &= \frac{1}{\sqrt{2}} \left( (1+c_{H,\mathrm{kin}}) [Y_\psi]_{pr} - \frac{3}{2} \frac{v^2}{\Lambda^2} [C_{\psi H}]_{pr} \right)
\end{align}
for $\psi \in \{u,d,e\}$. Notice that contrary to the SM, the Yukawa matrices~$\mathcal{Y}_{\psi h}$ are no longer proportional to the mass matrices~$\mathcal{M}_\psi$. Therefore, both cannot be diagonalized simultaneously in general,\footnote{Both matrices also have different RG~equations.} and hence, when working in the mass basis, the Higgs boson~$h$ will have flavor violating couplings starting at~$\mathcal{O}(\Lambda^{-2})$ \cite{Alonso:2013hga}.

Similarly, the $d=5$~Weinberg operator~\eqref{eq:Weinberg-Operator} of the SMEFT yields a neutrino Majorana-mass matrix in the LEFT
\begin{align}
    \mathcal{L} &\supset -\frac{1}{2} [\mathcal{M}_\nu]_{pr} \left( \overline{\nu}{{}^L_{p}}^c \nu^L_{r} \right) + \mathrm{h.c.} \,,
\end{align}
where $\smash{[\mathcal{M}_\nu]_{pr}}=\smash{-[C_\mathrm{Weinberg}]_{pr} \, v_T^2 \big/ \Lambda_{\slashed{L}}}$. Here, $\Lambda_{\slashed{L}}$ denotes the new physics scale of the Weinberg operator, where lepton number is violated by $\Delta L = 2$. As already mentioned, this scale is not necessarily related to the new physics scale~$\Lambda$ of other operators and known to be very high $\Lambda_{\slashed{L}} \gtrsim 10^{13}\,\text{GeV}$, for an $\mathcal{O}(1)$ coefficient~$C_\mathrm{Weinberg}$, in order to explain the tiny neutrino masses of the order of~$\sim 1\,\text{eV}$. Notice that $\smash{[\mathcal{M}_\nu]_{pr}}$ is symmetric in the flavor indices~$p$ and~$r$, and that this is the only dimension-three operator present in the LEFT.

In general, the mass matrices~$\mathcal{M}_\psi$, for $\psi \in \{\nu,e,u,d\}$, are non-diagonal. To go to the mass basis we need to diagonalize them by unitary rotations~$U_{\psi_{L\!/\!R}}$ of the fermion fields $\psi_{L\!/\!R} \to U_{\psi_{L\!/\!R}} \psi_{L\!/\!R}$, such that 
\begin{align}
    U_{\psi_L}^\dagger \, \mathcal{M}_\psi \, U_{\psi_R} &\equiv \mathrm{diag}(m_{\psi_1},m_{\psi_2},m_{\psi_3}) \,.
\end{align}
Of course, in general we have $\smash{U_{d_L} \neq U_{u_L}}$ and $\smash{U_{e_L} \neq U_{\nu_L}}$ leading to the CKM and PMNS~matrix respectively
\begin{align}
    V_\mathrm{CKM} &= U_{u_L}^\dagger \, U_{d_L} \,,
    &
    V_\mathrm{PMNS} &=   U_{e_L}^\dagger \, U_{\nu_L}\, ,
\end{align}
which contribute to charged current interactions.
In the SM it is conventional to align the mass- and weak-eigenstate bases either in the up-sector~$(U_{u_L}=\mathds{1})$ or down-sector~$(U_{d_L}=\mathds{1})$. In the SM these two choices, and any other arbitrary alignment choice, are equivalent since the CKM~matrix is the only source of flavor violation in the SM, and it is determined experimentally. On the contrary, in the SMEFT there are potentially other sources of flavor violation due to the higher-dimensional operators. Therefore, the alignment of mass- and weak-eigenstates is crucial as different choices lead to different physics results (for a given set of Wilson coefficients). For example, for a four-fermion operator we find 
\begin{gather}
    [C]_{prst} \left( \overline{\psi}_{1,p} \Gamma \psi_{2,r} \right) \! \left( \overline{\psi}_{3,s} \Gamma \psi_{4,t} \right) 
    \nonumber\\
    \downarrow
    \\
    [C]_{prst} [U_1^\dagger]_{p^\prime \! p} [U_2]_{r r^\prime} [U_3^\dagger]_{s^\prime \! s} [U_4]_{t t^\prime} \! \left( \overline{\psi}_{1,p^\prime} \Gamma \psi_{2,r^\prime} \!\right) \!\! \left( \overline{\psi}_{3,s^\prime} \Gamma \psi_{4,t^\prime} \!\right)\! ,
    \nonumber
\end{gather}
where $\Gamma$~denotes some Dirac structure, possibly in combination with generators. We see that different alignment choices, i.e., different choices for the $U_n$~matrices, lead to different results for the operators in the mass basis.


\subsubsection{The gauge sector}
The kinetic terms for the gauge bosons in the broken phase receive additional contributions from the operators $Q_{HG}$, $Q_{HW}$, $Q_{HB}$, and~$Q_{HW\!B}$. To properly normalize the kinetic terms, we redefine the gauge fields and couplings \cite{Alonso:2013hga}
\begin{align}
    G_\mu^A &= \mathcal{G}_\mu^A \left(\! 1+ \frac{v_T^2}{\Lambda^2} C_{HG} \!\right) , 
    &
    \overline{g}_3 &= g_3 \left(\! 1+ \frac{v_T^2}{\Lambda^2} C_{HG} \!\right) ,
    \\
    W_\mu^I &= \mathcal{W}_\mu^I \left(\! 1+ \frac{v_T^2}{\Lambda^2} C_{HW} \!\right) ,
    &
    \overline{g}_2 &= g_2 \left(\! 1+ \frac{v_T^2}{\Lambda^2} C_{HW} \!\right) ,
    \\
    B_\mu &= \mathcal{B}_\mu \left(\! 1+ \frac{v_T^2}{\Lambda^2} C_{H\!B} \!\right) ,
    &
    \overline{g}_1 &= g_1 \left(\! 1+ \frac{v_T^2}{\Lambda^2} C_{H\!B} \!\right)
\end{align}
so that their products are left invariant (e.g. $\smash{g_2 W_\mu^I} = \smash{\overline{g}_2 \mathcal{W}_\mu^I}$).
This leads to canonically normalized kinetic terms for the gluons~$\smash{\mathcal{G}_\mu^A}$, but not for the weak gauge bosons $\smash{\mathcal{W}_\mu^I}$ and~$\smash{\mathcal{B}_\mu}$ due to the kinetic mixing induced by~$\smash{Q_{HW\!B}}$, which mixes the $\smash{\mathcal{W}_\mu^3}$~state with the $\smash{\mathcal{B}_\mu}$~state. For the kinetic and mass terms we find
\begin{align}
    \mathcal{L}_\mathrm{gauge}^\mathrm{broken} = 
    &-\frac{1}{2} \mathcal{W}_{\mu\nu}^{+} \mathcal{W}^{\mu\nu}_{-} -\frac{1}{4} \mathcal{W}_{\mu\nu}^{3} \mathcal{W}^{\mu\nu}_{3} +\frac{1}{4} \overline{g}_2^2 v_T^2 \mathcal{W}_\mu^{+} \mathcal{W}^{\mu}_{-}
    \nonumber\\
    &-\frac{1}{4} \mathcal{B}_{\mu\nu} \mathcal{B}^{\mu\nu} -\frac{1}{2} \frac{v_T^2}{\Lambda^2} C_{HW\!B} \mathcal{W}_{\mu\nu}^3 \mathcal{B}^{\mu\nu}
    \\
    &+\frac{1}{8}v_T^2 \left( 1 + \frac{1}{2} \frac{v_T^2}{\Lambda^2} C_{H\!D} \right) \left( \overline{g}_2 \mathcal{W}_\mu^3 - \overline{g}_1 \mathcal{B}_\mu \right)^2 \,,
    \nonumber
\end{align}
where $\smash{\mathcal{W}_\mu^\pm}=\smash{(\mathcal{W}_\mu^1 \mp i\mathcal{W}_\mu^2)/\sqrt{2}}$, and similarly for the field-strength tensors.
We can apply two rotations \cite{Grinstein:1991cd}
\begin{align}
    \begin{pmatrix}
        \mathcal{W}_\mu^3 
        \\ 
        \mathcal{B}_\mu
    \end{pmatrix}
    &=
    \begin{pmatrix}
        1 & \epsilon 
        \\
        \epsilon & 1
    \end{pmatrix}
    \begin{pmatrix}
        \overline{c}_\theta & \overline{s}_\theta
        \\
        -\overline{s}_\theta & \overline{c}_\theta
    \end{pmatrix}
    \begin{pmatrix}
        \mathcal{Z}_\mu 
        \\ 
        \mathcal{A}_\mu
    \end{pmatrix} \,,
\end{align}
where $\epsilon=- v_T^2 C_{HW\!B} \big/ (2\Lambda^2)$, to diagonalize the kinetic terms and go to the mass-eigenstate basis containing the photon~$\mathcal{A}_\mu$ and the $\mathcal{Z}_\mu$~boson.
The rotation angles are
\begin{align}
    \overline{s}_\theta &\equiv \sin\overline{\theta} = \frac{\overline{g}_1}{\sqrt{\overline{g}_1^2+\overline{g}_2^2}} \left[ 1 + \epsilon \, \frac{\overline{g}_2}{\overline{g}_1} \frac{\overline{g}_1^2-\overline{g}_2^2}{\overline{g}_1^2+\overline{g}_2^2} \right] \,,
    \\
    \overline{c}_\theta &\equiv \cos\overline{\theta} = \frac{\overline{g}_2}{\sqrt{\overline{g}_1^2+\overline{g}_2^2}} \left[ 1 - \epsilon \, \frac{\overline{g}_1}{\overline{g}_2} \frac{\overline{g}_1^2-\overline{g}_2^2}{\overline{g}_1^2+\overline{g}_2^2} \right] \,.
\end{align}
Of course, the photon~$\mathcal{A}_\mu$ remains massless due to $\mathrm{U}(1)_e$~gauge invariance, whereas the $\mathcal{W}^{\pm}_\mu$ and $\mathcal{Z}_\mu$~boson acquire the masses
\begin{align}
    m_W^2 &= \frac{\overline{g}_2^2 v_T^2}{4} \,,
    \\
    m_Z^2 &= \frac{v_T^2}{4}  \left[ \left( \overline{g}_1^2 + \overline{g}_2^2 \right)  \left( 1 + \frac{1}{2} \frac{v_T^2}{\Lambda^2} C_{H\!D} \right) - 4 \overline{g}_1 \overline{g}_{2} \epsilon \right].
%
    % \frac{v_T^2}{4} \! \left( \overline{g}_Y^2 \!+ \overline{g}_L^2 \right) \! \left(\! 1 + \frac{1}{2} \frac{v_T^2}{\Lambda^2} C_{H\!D} \!\right) \!+ \frac{1}{2} \frac{v_T^4}{\Lambda^2} \overline{g}_Y \overline{g}_L C_{HW\!B}.
\end{align}
We can then define the gauge couplings 
\begin{align}
    \overline{e} &= \overline{g}_2 \left( \overline{s}_\theta + \epsilon \, \overline{c}_\theta \right) \,,
    &
    \overline{g}_Z &= \frac{\overline{e}}{\overline{s}_\theta \overline{c}_\theta} \left( 1-\frac{\epsilon}{\overline{s}_\theta \overline{c}_\theta} \right)
    %\frac{\overline{e}}{\overline{s}_\theta \overline{c}_\theta} \left( 1 + \frac{\overline{g}_Y^2 + \overline{g}_L^2}{2 \overline{g}_Y \overline{g}_L} \frac{v_T^2}{\Lambda^2} C_{HW\!B} \right) 
\end{align}
and the covariant derivative of the broken phase
\begin{align}
\begin{split}
    D_\mu = \partial_\mu &- i\, \overline{g}_2 \left( \mathcal{W}_\mu^+ t^+ + \mathcal{W}_\mu^- t^- \right) 
    \\
    &- i\, \overline{g}_Z \left( T_3 - \overline{s}_\theta^2 Q \right) \mathcal{Z}_\mu - i\, \overline{e}\, Q \mathcal{A}_\mu \,,
\end{split}
\end{align}
where the electric charge is $Q=T_3+Y$ with the hypercharge~$Y$ and the third component of weak isospin~$T_3$. Furthermore, we defined $\smash{t^\pm}=\smash{(t^1 \pm i t^2)/\sqrt{2}}$, where $t^I$ are the $\mathrm{SU}(2)_L$~generators.
Also, the couplings of the $\mathcal{W}$ and $\mathcal{Z}$~boson to fermions are modified by operators of the class~$\psi^2 H^2 D$. 
For example, in the broken phase the operator $\smash{[Q_{Hl}^{(3)}]_{pr}}$ yields an interaction term of the form 
\begin{align}
    [Q_{Hl}^{(3)}]_{pr} 
    &\rightarrow 
    v_T^2 \bigg\{\frac{\overline{g}_Z}{2}\big[\!\left(\overline{\nu}{}^L_p \gamma^\mu \nu^L_r\right) - \left(\overline{e}{}^L_p \gamma^\mu e^L_r\right) \big] \mathcal{Z}_\mu
    \label{eq:QHl3}\\
    &+ \frac{\overline{g}_2}{\sqrt{2}} \big[ \! \left( \overline{\nu}{}^L_p \gamma^\mu e^L_r \right) \mathcal{W}_\mu^+ + \left( \overline{e}{}^L_p \gamma^\mu \nu^L_r \right) \mathcal{W}_\mu^- \big] \!\bigg\} + \mathcal{O}(h)
    \nonumber
\end{align}
additionally to the SM interactions.
For more details see, e.g., \cite{Jenkins:2017jig}.\footnote{Notice also that in the SMEFT the $\mathcal{W}$~boson can couple to right-handed fermions through the $Q_{Hud}$~operator.}

% \felix{Move this one section below}
% The modification of the tree-level relations of SM parameters in the SMEFT requires extra care in extracting these from experimental measurements. For example, the Fermi constant~$G_F$ is determined from muon decays $\mu^- \to e^- + \nu_\mu + \overline{\nu}_e$. Following Eq.~\eqref{eq:4fermi} we can define the local four-Fermion interaction $(\mathcal{G}_F/\sqrt{2}) (\overline{\nu}_L^\mu \gamma^\mu \mu_L)(\overline{e}_L \gamma_\mu \nu_L^e)$, where we find
% \begin{align}
%     -\frac{4\mathcal{G}_F}{\sqrt{2}}
%     = - \frac{2}{v_T^2} &+ \frac{1}{\Lambda^2} \Big( [C_{ll}]_{1221} + [C_{ll}]_{2112} \Big) 
%     \\
%     &- \frac{2}{\Lambda^2}  \Big( [C_{Hl}^{(3)}]_{11} + [C_{Hl}^{(3)}]_{22} \Big) \,.
% \end{align}
% Thus, what is actually extracted from experimental data is~$\mathcal{G}_F$ rather than the SM value~$\smash{G_F=(\sqrt{2}\,v^2)^{-1}}$.
% For more detail on the extraction of the SM parameters and a discussion of appropriate input schemes for SMEFT computations see e.g. \cite{Brivio:2021yjb,Brivio:2017bnu}.


% % % % % % % % % % % % % % % % % % % % % % % % % % % % % % % % % % % % % % % % % % % % % % % % % % % % % % 
\subsection{Integrating out the weak-scale particles in SMEFT}
After having derived the SMEFT Lagrangian in the broken phase, we can construct the LEFT Lagrangian by removing the heaviest particles from the theory, i.e., the Higgs~$h$, $W/Z$-bosons, and the top quark~$t$. This procedure will be discussed in general in Sec.~\ref{sec:matching}, and here we only anticipate the determination of the Fermi interaction~\eqref{eq:4fermi} as an example.

For illustration, consider the four-fermion interaction
\begin{align}
    -\frac{4 \mathcal{G}_F}{\sqrt{2}} \left( \overline{\nu}{}^L_{\mu} \gamma^\mu \mu^L \middle)\middle(\overline{e}{}^L \gamma_\mu \nu^L_{e} \right)
\end{align}
in the LEFT mediating the muon decay $\mu^- \to e^- + \nu_\mu + \overline{\nu}_e$, whose measurement allows to determine the value of the coupling constant~$\mathcal{G}_F$.
In the SMEFT this decay is mediated through the exchange of a $\mathcal{W}_\mu^-$~boson, either with its SM couplings or with the modified coupling due to~$\smash{Q_{Hl}^{(3)}}$ as shown in Eq.~\eqref{eq:QHl3}, or through the four-fermion SMEFT operator~$Q_{ll}$. The corresponding tree-level Feynman diagrams are shown in Fig.~\ref{fig:GFermi-matching}.

% \onecolumngrid
% \begin{figure}[h] 
%     \centering
%     \includegraphics[width=0.98\textwidth]{figures/GFermi-matching.pdf}
%     \caption{Tree-level diagrams contributing to the SMEFT to LEFT matching for the Fermi constant. The $W$~boson propagators in the SMEFT diagrams are understood to be expanded in~$p^2/m_w^2$ yielding local contributions.}
%     \label{fig:GFermi-matching}
% \end{figure}    
% \twocolumngrid

\begin{figure}[t] 
    \centering
    \includegraphics[width=0.98\linewidth]{figures/GFermi-matching-short.pdf}
    \caption{Tree-level diagrams contributing to the SMEFT to LEFT matching for the Fermi constant. The $\mathcal{W}$~boson propagators in the SMEFT diagrams are understood to be expanded in~$p^2/m_W^2$ yielding local contributions.}
    \label{fig:GFermi-matching}
\end{figure}    

We can now compute the tree-level amplitudes in the LEFT and SMEFT, expanding the $\mathcal{W}$-propagators as $1/(p^2-m_W^2)=-1/m_W^2+\mathcal{O}(p^2 \big/ m_W^2)$, where $p$~is the momentum carried by the $\mathcal{W}$-boson, which in the range of validity of the LEFT is small $p^2 \ll m_W^2$. Equating our results we find \cite{Alonso:2013hga}
\begin{align}
\begin{split}
    -\frac{4\mathcal{G}_F}{\sqrt{2}}
    =- \frac{2}{v_T^2} &- \frac{2}{\Lambda^2}  \Big( [C_{Hl}^{(3)}]_{11} + [C_{Hl}^{(3)}]_{22} \Big) 
    \\
    &+ \frac{2}{\Lambda^2} \Re \big( [C_{ll}]_{1221} \big) \,,
\end{split}
\label{eq:4fermi-matching}
\end{align}
where we used $[C_{ll}]_{2112}=[C_{ll}^\ast]_{1221}$.
The above equation is called a matching condition and determines the LEFT coefficient~$\mathcal{G}_F$ in terms of the SMEFT parameters.
We see that the LEFT and SMEFT power expansions get mixed up in this case. In general LEFT operators of dimension~$d$ are suppressed by 
\begin{equation}
\frac{1}{\Lambda^a}\frac{1}{v^b}\,, \quad a+b=d-4\,, \quad a \geq 0 \,, 
\label{dimension}
\end{equation}
where~$a$ is always positive, since the SMEFT contains~$\Lambda$ only as multiplicative prefactor with negative powers, and~$b$ can be negative due to Higgs vev insertions. 

From Eq.~\eqref{eq:4fermi-matching} we see that what is actually extracted from experimental data on the muon decay is~$\mathcal{G}_F$ rather than the SM value~$\smash{G_F=(\sqrt{2}\,v^2)^{-1}}$. The modification of the tree-level relations of SM parameters like the above in the SMEFT requires extra care in extracting these from experiment. For more detail on the determination of the SM parameters and a discussion of appropriate input schemes for SMEFT computations see, e.g., \cite{Brivio:2021yjb,Brivio:2017bnu}.

The construction of a complete basis of the LEFT up to dimension six, and the derivation of all tree-level relations of the LEFT Wilson coefficients to the SMEFT Wilson coefficients is presented in \cite{Jenkins:2017jig}.
Subsequently, also LEFT operator bases at dimension seven \cite{Liao:2020zyx}, eight \cite{Murphy:2020cly}, and nine \cite{Li:2020tsi} have been derived in the literature.
At the one-loop level, the matching requires the calculation of a large number of loop diagrams. This monumental program was taken up by~\textcite{Dekens:2019ept}. 
% The coefficients are understood to be renormalized in the $\overline{\mathrm{MS}}$~scheme, and implicitly dependent on the matching scale~$\mu \sim m_W$.
Obviously these are lengthy expressions and are most conveniently and usable given directly in digital form as in Appendix~G and the supplementary material of \cite{Dekens:2019ept}.\footnote{The SMEFT to LEFT one-loop matching is also implemented in codes such as \cmd{Wilson} \cite{Aebischer:2018bkb} and \cmd{DsixTools} \cite{Celis:2017hod,Fuentes-Martin:2020zaz}.}
Therein, the SMEFT coefficients appearing in the matching conditions are understood to be renormalized in the $\overline{\mathrm{MS}}$~scheme, and implicitly dependent on the matching scale~$\mu \sim m_W$.
When using these one-loop matching results it is important to strictly follow all the conventions used in their derivation also in all subsequent computation with the resulting LEFT Lagrangian to avoid any inconsistencies.
Eventually, we point out that the one-loop anomalous dimensions of all $d \leq 6$ LEFT coefficients have been derived in \cite{Jenkins:2017dyc} which complements the toolbox for LEFT computations. 


% \noindent {\color{RedOrange}\rule{\linewidth}{2.pt}}
% \felix{Compactify below and/or remove to avoid overlap with the following section.}

% \felix{Compactify and quickly discuss tree-level matching properties. Just quote Peter on one-loop.}
% The next step in the transition to the LEFT consists in integrating out the weak scale masses,~$M$, that is the Higgs field, the $W, Z$~bosons and the $t$~quark.\footnote{We also note that  since $M \sim g v$, $M/v$ is of the order of a gauge coupling} That is, we consider graphs, where these heavy particles appear only at internal lines, at tree level or in loops. This procedure gives operators of only light particles, that is LEFT operators. Tree graphs give propagators 
% with powers of  $1/M \sim 1/v$, loops might add positive powers of~$M$. \felix{Mention somewhere the power-counting subtleties that arise due to this.} This yields LEFT operators of dimension~$D$ with 
% \begin{equation}
% \frac{1}{\Lambda^a}\frac{1}{v^b}\,, \quad a+b=D-4\,, \quad a \geq 0 \,, 
% \label{dimension}
% \end{equation}
% where $a$ is always positive and $b$ can be negative. 
% The explicit construction of the possible LEFT operators out of those of the SMEFT is described in~\cite{Jenkins:2017jig}, in particular in section 4. 
% The procedure is totally analogous to the construction of the four-Fermi interaction out of two $W$ couplings to Fermion pairs connected by a $W$-propagator. Of course, there are operators that do not require a heavy internal line, such as a four-Fermion interaction in SMEFT. The resulting
% operators for tree level graphs are listed in appendix C of~\cite{Jenkins:2017jig} up to dimension 
% %give the left operators
% At the one-loop level, the matching requires the calculation of a large number of loop diagrams. This monumental program was taken up in~\cite{Dekens:2019ept},
% where the relevant effective couplings in the LEFT were calculated. Apart from a direct (tree level) term already present in the low energy Lagrangian in Eq.~\eqref{QFT},  the following four can contribute: Tree level terms of the SM and the SMEFT and one-loop terms with couplings from the SM and/or the SMEFT.
% %here possiblyx some graphs.
% Graphs with only SM vertices are of order $1/{v^2}$, those with  at least one SMEFT vertex include a factor $1/{\Lambda^2}$. Mixed graphs, with both kinds of may give rise to positive powers of $M$ in some cases. As an example of a result, we refer to eq (6.1) of~\cite{Dekens:2019ept}, where
% the dipole gluonic coupling of a light up-type quark is displayed. The parameters are understood to be renormalized in the $\overline{\mathrm{MS}}$~scheme, and implicitly
% dependent on the matching scale~$\mu_W$.  Obviously these are lengthy expressions and are most conveniently and usable  given directly
% in digital form; see appendix G of \cite{Dekens:2019ept} and the supplemental material there. \felix{They are also implemented in \cmd{Wilson} and \cmd{DsixTools}, which might be even more convenient.}

% As stressed in \cite{Dekens:2019ept}, the results depend on the basis chosen for the SMEFT. In~\cite{Dekens:2019ept}, the renormalized $\overline{\mathrm{MS}}$ quark mass matrix are diagonalized by $\mu_W$ dependent matrices~$U$ and~$V$ which are kept explicitly rather than being reabsorbed. This allows
% for a systematic listing of all contributions, but they need to be carefully followed. 

% \subsection{Renormalization group evolution of the LEFT coefficients}
% \felix{Drop this an refer to next section.}

% As mentioned, the relevant scale for the low energy processes is $m_b \sim$ few GeV. Therefore the parameters in the LEFT discussed above must
% be taken at this scale.  The transition from the matching scale $m_W$ to the physical scale $m_b$ is provided by the solutions to the renormalization group equation~\eqref{eq:RGE},~\cite{Jenkins:2017dyc}. Without much details, we give the results in appendix. \felix{You want to give the entire RGE in the appendix?}
%What do we need here?


% \felix{The next two example sections have been moved to section~6.}
% \subsubsection{The anomalous magnetic moment of the muon in the SMEFT}
% \label{sec:g-2_SMEFT}


% \subsubsection{Semi-leptonic processes in the SMEFT}
% \label{sec:semmileptonic-processes}
% E.g. look at \cite{Cornella:2021sby}.

% This section is about the matching of the \SMEFT to the LEFT
% \begin{itemize}
% 	\item Some overview~\cite{Buchalla:1995vs}
% 	\item Classification of operators and matching to \SMEFT~\cite{Jenkins:2017jig}
% 	\item RG evolution of the LEFT operators~\cite{Jenkins:2017dyc}
% 	\item Matching of \SMEFT to LEFT at one-loop~\cite{Dekens:2019ept}
% 	\item A dimension-eight basis~\cite{Murphy:2020cly}
% \end{itemize}

% \subsection{Introduction}

% The typical energy scales $E$ for weak processes such as decays of Kaons, D-mesons and B mesons are much smaller than the mass of the $W$ bosons. We therefore expect an expansion in powers of $E/m_w$ to give an adequate description of these processes. This is
% of course the basis of the original 4-Fermi interaction, e.g.
% \begin{equation}
% \end{equation}
% \subsection{The broken phase of \SMEFT}

% \subsection{Matching to LEFT}

% \subsubsection{The anomalous magnetic moment of the muon in the \SMEFT}
% \label{sec:g-2_SMEFT}

% \subsubsection{Semi-leptonic processes in the \SMEFT}
% \label{sec:semileptonic-processes}
% E.g. look at \cite{Cornella:2021sby}.

%@article{Buras:2022qip,
%    author = "Buras, Andrzej J.",
%    title = "{Standard Model Predictions for Rare K and B Decays without New Physics Infection}",
%    eprint = "2209.03968",
%    archivePrefix = "arXiv",
%    primaryClass = "hep-ph",
%    month = "9",
%    year = "2022"
%}
% a few papers to look at:
%https://arxiv.org/abs/1709.04486
%https://arxiv.org/abs/1711.05270
%https://arxiv.org/abs/1908.05295
%und evtl. noch https://arxiv.org/abs/2012.13291



%@book{Buras:2020xsm,
%    author = "Buras, Andrzej",
%    title = "{Gauge Theory of Weak Decays}",
%    doi = "10.1017/9781139524100",
%    isbn = "978-1-139-52410-0, 978-1-107-03403-7",
%    publisher = "Cambridge University Press",
%    month = "6",
%    year = "2020"