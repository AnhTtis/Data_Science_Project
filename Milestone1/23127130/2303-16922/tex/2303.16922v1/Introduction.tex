\section{Introduction}
\begin{quotation}
\noindent {\it Maybe the main point of our analysis is that it demonstrates explicitly  how remarkable the standard electroweak theory is. 
\cite{Buchmuller:1985jz}}
\end{quotation}
The Standard Model~(SM) of particle physics, formulated some 50~years ago, and judiciously completed over the years, forms the basis of our understanding of the fundamental interactions. More precisely, the SM is the quantum field theory~(QFT) that describes how the basic matter constituents (quarks and leptons) interact at the microscopic level via weak, strong, and electromagnetic forces. While all data from earth-based laboratory experiments agrees with the SM predictions (possibly with a few exceptions that we will comment on later), there is some indirect evidence, derived from cosmological observations, that the model is not complete: this model does not explain the baryon asymmetry of the universe, dark matter, and dark energy. 
These are all phenomena that could naturally find their explanation in the domain of particle physics or, more generally, within~QFT.
There are also theoretical concerns about the SM itself, such as the strong sensitivity of the Higgs mass term to high-energy modes in the renormalization procedure (the so-called ``hierarchy problem''), the absence of an explanation for the hierarchical structure of the fermion spectrum, and the lack of a bridge to quantum gravity. Last but not least, non-vanishing neutrino masses cannot be accounted for by the ``classical version'' of the Standard Model, containing only left-handed neutrinos and only renormalizable interactions.
%\footnote{Only dimension four operators \gino{GI: {\em I don't think this footnote is necessary. This is the intro, we clarify the point later.}}}

In order to address these problems, a large number of new ``fundamental'' theories beyond the Standard Model~(BSM) were formulated over the last 40--50~years. In fact, the 1980s and, to a lesser extent, the 1990s saw a downright explosion of model building. While some of them addressed specific questions, others offered veritable extensions of the basics of the SM, such as supersymmetric models, or models with composite Higgs sectors, and/or composite quarks and leptons: concepts that might become important again in the future, possibly within a new context, such as string theory. These models have new particles and interactions, generally at energies (well) above the Fermi scale. They were designed to explain some of the facts that the SM cannot, such as the %origin of Baryon violation,
quantization of the electric charge, the hierarchical generational structure of quarks and leptons, the possible unification of interaction strengths, etc. Unfortunately, many of these models have been shown to be inconsistent with data 
%or with mathematical demands
or are not testable with present and near-future experimental facilities.

In order to look for these new physics scenarios, most likely manifested in small discrepancies between the SM predictions and the observations, both theoretical and experimental progress is necessary. 
Over many years, and with increasing intensity and success in the new century, theoretical work on the Standard Model has improved enormously. Apart from devising new calculation techniques, this progress has been made possible by developing and applying the concepts of effective field theory~(EFT) in several relevant areas. 
Roughly speaking, a quantum EFT is a quantum field theory which is not considered to be ``fundamental'', being valid only in a limited range of energies or distances, or even in specific kinematic configurations. The wide separation of the Fermi scale (or the $W$-boson mass,~$m_W$) and of the masses of the $B$-meson or the charmed particles has allowed to successfully use EFT and renormalization group techniques to calculate the expected (inclusive) decay rates of these mesons with astonishing accuracy. The formulation of new quantum EFTs like HQET (heavy quark effective theory) and SCET (soft collinear effective theory) have lead to accurate predictions also for exclusive decays. Equally, high-energy calculations, such as jet dynamics at the LHC, have benefited from EFT techniques. Also the oldest effective field theory of the SM, namely ChPT (chiral perturbation theory), has been extensively used to obtain precision results for low-energy meson dynamics. 
Embracing the philosophy of EFT for the Standard Model has led to theoretical predictions which have reached levels of precision not imaginable twenty years ago. A~similarly impressive progress has been achieved by experiments, both at low and high energies. 
%As mentioned, this has excluded many BSM ideas or moved their characteristic energy scales high up  making them considerably less attractive. 
We expect this quest for ever higher precision, both on the theoretical and the experimental side, to continue, in the hope to find deviations from the SM that, as argued above, are well motivated. 
%, especially since it is still widely thought that the SM is not the whole story, and there must be physics beyond the SM.\footnote{as mentioned, particle physics experiments do not show any deviations from the SM; an apparent breakdown of lepton universality in $B$-decays has disappeared \cite{LHCb:2022qnv}.}

In this perspective, it is very natural to consider the original formulation of the SM as the effective low-energy ``remnant'' of a more fundamental theory, whose new heavy degrees of freedom are removed 
% ('integrated out') 
in favor of generating additional effective contact interactions between the known SM fields. As argued by \textcite{Wilson:1983xri}, the true physics of the ``full'' theory below the cutoff scale can be recovered by including all possible interactions allowed by the particles and symmetries of the theory. 
The effective Lagrangian thus obtained consists of a string of local interaction terms (operators), each characterized by an appropriate coefficient (effective coupling/Wilson coefficient), organized in a series of increasing dimensionality, corresponding to the expected decreasing relevance. 
As usual for EFTs, this construction is not renormalizable in the usual strict sense, because it involves an infinite number of coupling constants. It is however renormalizable order by order in an energy/momentum expansion reflected in the operator expansion. Actually the independence of the renormalization scale of physical amplitudes can be exploited by the renormalization group flow of the operator coefficients, allowing to identify and resum the largest quantum corrections. 

Given the success of effective theories so far, this approach seems a good way to access the next layer of physics, as proposed by \textcite{Buchmuller:1985jz} even before the last building blocks of the SM where experimentally identified. 
In this review, we will trace its development and highlight some of the most recent results.
Our main scope is to illustrate, in practice, how considering the SM as an EFT can help in identifying properties of new physics and single out future research directions. The EFT approach provides indeed not only a systematic way for analyzing experimental results, but also a precious tool to correlate different observables obtaining a deeper insights on where to look to discover the next layer. 

The review is organized as follows: in the rest of this section we introduce the SM, briefly recalling also the motivations why we want to go beyond it, we review general aspects of EFT, and finally introduce the so-called Standard Model effective field theory~(SMEFT). A detailed analysis of the SMEFT, with special focus on the structure of operators of dimension six, is presented in 
Sec.~\ref{sect:SMEFT}. The role of global symmetries in the SMEFT, with a particular emphasis on exact and approximate flavor symmetries, is discussed in Sec.~\ref{sec:GlobalSymmetries}. Section~\ref{sec:HEFT} is devoted to a discussion of the differences between the SMEFT and the more general case of a non-linearly realized electroweak symmetry. In Sec.~\ref{sect:LEFT} we briefly review the low-energy effective field theory as the low-energy limit of the SMEFT. Finally, in Sec.~\ref{sect:practical} we present two concrete examples of the SMEFT at work, i.e., of applications of the SMEFT to analyze concrete phenomenological problems. 
In Appendix~\ref{app:dim-reg} we discuss some technical details of dimensional regularization showing up in SMEFT computations.

% - - - - - - - - - - - - - - - - - - - - - - - - - - - - - - - - - - - - - - - - - - - - - - - - - - - -
% - - - - - - - - - - - - - - - - - - - - - - - - - - - - - - - - - - - - - - - - - - - - - - - - - - - -

\subsection{The Standard Model of particle physics}

Within the Standard Model the three fundamental 
forces are described via the principle of gauge 
invariance, requiring the theory to be 
invariant under the local symmetry group
\begin{align}
\mathcal{G}_\mathrm{SM} &= \SU{3}_c \otimes \SU{2}_L \otimes \mathrm{U}(1)_Y \, .
\end{align}
The quantum fields can be divided in three categories: $i)$~the gauge fields associated to the local gauge symmetry groups~$(G_\mu,W_\mu,B_\mu)$; $ii)$~the matter (fermion) fields~($\ell,e,q,u,d$); $iii)$~the Higgs boson doublet~$H$ responsible for the breaking of the electroweak subgroup of~$\mathcal{G}_\mathrm{SM}$ down to the QED group~$\mathrm{U}(1)_e$
\begin{align}
\SU{2}_L \otimes \mathrm{U}(1)_Y \longrightarrow \mathrm{U}(1)_e \, .
\end{align}
The field content of the SM is shown in Tab.~\ref{tab:SM_field-content} together with the transformation properties of each field under the different gauge groups and the hypercharge assignments.\footnote{In principle, one could extend the fermion content including right-handed neutrinos. However, these fields would be completely neutral under~$\mathcal{G}_\mathrm{SM}$. We prefer to define the SM as the theory of the chiral fermions with non-trivial transformation properties under~$\mathcal{G}_\mathrm{SM}$, that acquire mass via the Higgs mechanism. As such, right-handed neutrinos are not SM fields.}
The basic fermion family ($\ell,e,q,u,d$) is replicated three times. 


\begin{table}%[H]
\centering 
\begin{tabular}{ | c | c c c c c | c | c c c | }
\hline
& $\ell$ & $e$ & $q$& $u$ & $d$ & $H$ & $G$ & $W$ & $B$ \\ \hline
%
$\SU{3}_c$ representation & $\mathbf{1}$ & $\mathbf{1}$ & $\mathbf{3}$ & $\mathbf{3}$ & $\mathbf{3}$ & $\mathbf{1}$ & $\mathbf{8}$ & $\mathbf{1}$ & $\mathbf{1}$ \\ \hline
%
$\SU{2}_L$ representation & $\mathbf{2}$ & $\mathbf{1}$ & $\mathbf{2}$ & $\mathbf{1}$ & $\mathbf{1}$ & $\mathbf{2}$ & $\mathbf{1}$ & $\mathbf{3}$ & $\mathbf{1}$ \\ \hline
%
$\mathrm{U}(1)_Y$ charge & $-\frac{1}{2}$ & $-1$ & $\frac{1}{6} $ & $\frac{2}{3}$ & $-\frac{1}{3}$ & $\frac{1}{2}$ & $0$ & $0$ & $0$ \\ \hline
\end{tabular}
\caption{Standard Model field content with the transformation properties of the fields under $\SU{3}_c \otimes \SU{2}_L$, and the hypercharge assignments. The fields are divided into fermions~$(\ell,e,q,u,d)$, the Higgs doublet~$(H)$, and gauge fields~$(G,W,B)$.
\label{tab:SM_field-content}
}
\end{table}

The SM Lagrangian 
%of the Standard Model of particle physics in the unbroken phase, i.e. before the electroweak symmetry breaking, is given by 
is the most general renormalizable expression that can be constructed out of the fields in Tab.~\ref{tab:SM_field-content} that is invariant under~$\mathcal{G}_\mathrm{SM}$:
\begin{align}
& \L_\mathrm{SM} = -\frac{1}{4} G_{\mu\nu}^A G^{A\mu\nu} -\frac{1}{4} W_{\mu\nu}^I W^{I\mu\nu} -\frac{1}{4} B_{\mu\nu} B^{\mu\nu} 
\nonumber\\
%
&-  \frac{\theta_3g_3^2 }{32\pi^2}  G_{\mu\nu}^A \widetilde{G}^{A\mu\nu} - \frac{\theta_2g_2^2 }{32\pi^2}  W_{\mu\nu}^I \widetilde{W}^{I\mu\nu} - \frac{\theta_1g_1^2 }{32\pi^2} B_{\mu\nu} \widetilde{B}^{\mu\nu}  
\nonumber\\
%
&+ i \brackets{\overline{\ell}_p \slashed{D} \ell_p + \overline{e}_p \slashed{D} e_p + \overline{q}_p \slashed{D} q_p + \overline{u}_p \slashed{D} u_p + \overline{d}_p \slashed{D} d_p} 
\label{eq:SM_Lagrangian}\\
%
&+ \brackets{D_\mu H}^\dagger \brackets{D^\mu H} + m^2 H^\dagger H - \frac{\lambda}{2} \brackets{H^\dagger H}^2 
\nonumber\\
%
&- \brackets{[Y_e]_{pr}\, \overline{\ell}_p e_r H + [Y_u]_{pr}\, \overline{q}_p u_r \widetilde{H} + [Y_d]_{pr}\, \overline{q}_p d_r H + \mathrm{h.c.}} .
\nonumber
\end{align}
% \felixx{where $g_{3,2,1}$ are the gauge couplings associated to the group factors of~$\mathcal{G}_\mathrm{SM}$, and $\theta_{3,2,1}$ are the charge-parity~(CP) violating phases \felix{Maybe mention that we can ignore them for this review?}. The gauge indices of the adjoint representations of $\mathrm{SU}(3)_c$ and~$\mathrm{SU}(2)_L$ are labeled $A$ and~$I$, respectively, the fundamental indices are suppressed here, the Lorentz indices are~$\mu,\nu$, and flavor indices are dubbed $p$ and~$r$. The dual field-strength tensors are defined, e.g., by $\widetilde{G}^{A,\mu\nu}=\smash{\epsilon^{\mu\nu\rho\sigma}G^A_{\rho\sigma}\big/2}$, with the Levi-Civita tensor~$\epsilon^{0123}=+1$, and Higgs conjugate is defined by $\widetilde{H}^i=\epsilon^{ij} H^\dagger_j$, where $i,j$ are fundamental $\mathrm{SU}(2)_L$~indices and the anti-symmetric tensor defined by $\epsilon^{01}=+1$.}

\subsubsection{The gauge sector}
The first three lines of Eq.~\eqref{eq:SM_Lagrangian} contain all gauge interaction in the SM. The gauge couplings associated to the gauge groups $\SU{3}_c$, $\SU{2}_L$, and~$\mathrm{U}(1)_Y$ are $g_3$, $g_2$, and~$g_1$. The indices $A=1,...,8$ and $I=1,2,3$ denote adjoint $\SU{3}_c$ or $\SU{2}_L$ gauge indices, respectively.
In the first line of Eq.~\eqref{eq:SM_Lagrangian} the field-strength tensors are defined by 
\begin{subequations}
\begin{align}
G_{\mu\nu}^A &= \partial_\mu G_\nu^A - \partial_\nu G_\mu^A + g_3 f^{ABC} G_\mu^B G_\nu^C \, , 
\\
W_{\mu\nu}^I &= \partial_\mu W_\nu^I - \partial_\nu W_\mu^I + g_2 \varepsilon^{IJK} W_\mu^J W_\nu^K \, , 
\label{eq:SM_fieldstrength_W}
\\
B_{\mu\nu} &= \partial_\mu B_\nu - \partial_\nu B_\mu \, ,
\end{align}
\end{subequations}
where $\smash{f^{ABC}}$ and $\smash{\varepsilon^{IJK}}$ are the totally anti-symmetric structure constants of $\smash{\SU{3}_c}$ and~$\smash{\SU{2}_L}$. They contain the kinetic terms for the gauge fields as well as all interactions among the gauge fields themselves. 

In the second line, the dual field-strength tensors are defined by $\smash{\widetilde{F}^{\mu\nu}=\frac{1}{2}\varepsilon^{\mu\nu\rho\sigma}F_{\rho\sigma}}$ for $\smash{F=G^A,W^I,B}$ with the totally anti-symmetric Levi-Civita tensor defined by~$\varepsilon^{0123}=-\varepsilon_{0123}=+1$. The Lagrangian terms containing dual field-strength tensors are proportional to total derivatives, meaning we can rewrite them as $\smash{G_{\mu\nu}^A \tilde{G}^{A,\mu\nu} = 2 \varepsilon^{\mu\nu\alpha\beta}\partial_\mu \big( G_\nu^A \partial_\alpha G_\beta^A + \frac{1}{3} g_3 f^{ABC} G_\nu^A G_\alpha^B G_\beta^C \big)}$. Therefore, they can only contribute to topological effects. For simplicity, 
we drop them from here on.

The third line comprises the kinetic terms of the fermion fields, as well as their gauge interactions. The latter are encoded in the gauge covariant derivative
%\footnote{We use the slash notation, i.e. $\slashed{D}=\gamma^\mu D_\mu$ where $\gamma_\mu$ is a Dirac $\gamma$-matrix.} 
\begin{align}
D_\mu &= \partial_\mu - i g_3 T^A G_\mu^A - i g_2 t^I W_\mu^I - i g_1 \mathsf{y} B_\mu \, ,
\end{align}
where $T^A = \lambda^A / 2$ and $t^I = \tau^I / 2$  are the generators of the fundamental representation of $\SU{3}_c$ and $\SU{2}_L$, respectively, with the Gell-Mann matrices~$\lambda^A$ and the Pauli matrices~$\tau^I$. The hypercharge generator is denoted~$\mathsf{y}$. 
%The indices $p$ and~$r$ on the fermion fields are flavor or generation indices, which run from 1~to~3. 

\subsubsection{The Higgs sector}
The last two lines of Eq.~\eqref{eq:SM_Lagrangian} include the Higgs and Yukawa sector of the~SM 
written in a symmetric notation before electroweak symmetry  breaking. The complex Higgs doublet is denoted by~$H$ and we define $\widetilde{H}=i \tau_2 H^\ast$. Minimizing the scalar potential
\begin{equation}
V(H) = -m^2 H^\dagger H + \frac{\lambda}{2}\brackets{H^\dagger H}^2
\label{eq:Hpotential}
\end{equation}
 yields a non-vanishing \vev~(vev) for the Higgs field,
$v^2 = 2 \langle 0 | H^\dagger H| 0  \rangle$, whose tree-level 
expression reads $v^2= 2 m^2/\lambda$. Considering the breaking of the electroweak symmetry, it is convenient to re-write
the Higgs doublet as
\begin{align}
H &= \frac{1}{\sqrt{2}} 
	\begin{pmatrix}
		\varphi^2 + i \varphi^1 \\
		v + h - i \varphi^3
	\end{pmatrix}
 \label{eq:Hdec}
\end{align}
where $h$ is the massive physical Higgs boson and~$\varphi^a$ denote the three Goldstone bosons, that, in the unitary gauge, are ``eaten'' by the massive gauge bosons. The tree-level mass of the physical Higgs is~$m^2_h = 2 m^2$. 



The Yukawa couplings~$[Y_i]_{pr}$ for $i=e,u,d$ are complex ${3 \times 3}$~matrices in flavor space contracted to the fermion fields via the global flavor indices $p$ and~$r$,
% \footnote{Note that all other types of indices can only be contracted in a unique way and are therefore suppressed in the Lagrangian for better readability. \felix{I think this foot note can be removed.}} 
which run from 1~to~3.
After electroweak symmetry breaking the Yukawa interactions in Eq.~\eqref{eq:SM_Lagrangian} yield the fermion mass terms as well as the Yukawa interactions with the physical Higgs boson~$h$. The Yukawa matrices~$Y_{u,d}$ are the only source of flavor violation in the SM,
%\footnote{Neutrino oscillations introduce flavor violation also in the lepton sector, however, since the origin of the non-zero neutrino masses responsible for the oscillation is not well understood, these effects are usually not included in what we call Standard Model.} 
as the gauge interactions are all flavor diagonal. They are also the only source of CP~violation in the SM, apart from the topological terms associated to the dual field-strength tensors.


\subsubsection{The success of the Standard Model}
With the discovery of the Higgs boson by the ATLAS \cite{ATLAS:2012yve} and CMS \cite{CMS:2012qbp} experiments at the Large Hadron Collider~(LHC) in 2012, the last missing piece of the Standard Model was observed. The measurement of the Higgs mass also made it possible to complete the determination of all the free parameters of the SM Lagrangian, but for the topological terms.
The overall agreement of the theoretical predictions of the SM with the plethora of available experimental data is remarkable. Especially in the electroweak sector the achieved precision is very high \cite{Haller:2018nnx,deBlas:2021wap}, as highlighted by the results in Fig.~\ref{fig:pulls2}. It is worth stressing that the results shown in this figure are only a small subset of the many tests successfully passed by the SM in the last few years, including also flavor-violating transitions of both quarks and leptons \cite{Isidori:2014rba,Bona:2022xnf}, and high-energy processes \cite{Boyd:2020qox}. In particular, no clear deviation from the SM predictions has been observed in the high-energy distributions analyzed so far by the ATLAS and CMS experiments, which collected an integrated luminosity of about $140\,\text{pb}^{-1}$ each in proton-proton collisions at an energy of $\sqrt{s}=13\,\text{TeV}$ at the~LHC.



\begin{figure}[t]
\begin{center}
\includegraphics[width=0.75\linewidth]{figures/PullPlot.pdf}
\caption{{\em Pulls} of the electroweak observables as obtained by a global SM fit, namely differences between SM predictions and direct measurements, normalized to the experimental uncertainties. From \cite{Haller:2022eyb}, see also \cite{Haller:2018nnx}.}
\label{fig:pulls2}
\end{center}
\end{figure}


% - - - - - - - - - - - - - - - - - - - - - - - - - - - - - - - - - - - - - - - - - - - - - - - - - - - -
% - - - - - - - - - - - - - - - - - - - - - - - - - - - - - - - - - - - - - - - - - - - - - - - - - - - -

\subsection{Motivations and hints for new physics}

Despite the outstanding agreement of the SM with experimental data, there are well known deficiencies that hint at a more fundamental theory. The most important is arguably the lack to incorporate gravity, the fourth known fundamental force of nature, into a coherent QFT framework  valid at arbitrary energy scales. 
As anticipated, the SM does not provide an explanation for cosmological observations such as the baryon asymmetry, dark matter, and dark energy. These phenomena do not necessarily need to find an explanation in the domain of particle physics. However, no convincing alternative explanations have been provided yet and, if interpreted in a QFT framework, they unavoidably point to the existence of new degrees of freedom beyond the SM ones. 

The clear experimental evidence of non-vanishing neutrino masses is also an unambiguous indication that the SM Lagrangian in (\ref{eq:SM_Lagrangian}) is not complete. As we shall discuss in Sec.~\ref{subsec:SMEFT_Operator-basis}, a natural solution to this problem is obtained when interpreting (\ref{eq:SM_Lagrangian}) as the  first  part --more precisely, the leading part containing operators of dimension up to four-- of a more general EFT Lagrangian.
A serious consistency problem of the SM is also the instability of the Higgs quadratic term in (\ref{eq:SM_Lagrangian}) with respect to quantum corrections, the so-called electroweak hierarchy problem~\cite{Barbieri:2017uzd}.
While none of the problems mentioned above points to a well-defined energy scale for the breakdown of the SM, a solution of the 
electroweak hierarchy problem would necessarily require new physics close to the Fermi scale. The fact that no clear evidence of new physics has been found yet at the LHC has let to consider explanations of this problem beyond the EFT framework~\cite{Giudice:2017pzm}.
However, many solutions within the EFT domain are still possible, motivating a deeper study of the SM as the low-energy limit of a more complete theory with new degrees of freedom not far from the Fermi scale and thus potentially detectable in near-future experiments.

Beside these general considerations, there are a few specific hints of deviations from the SM predictions observed in precision measurements. None of these hints is statistically compelling yet. However, they provide a clear illustration of the type of deviations we can expect in the near future, and of the type of effects we can describe within the EFT approach to new physics. This is why we discuss two such hints in more detail below: we will use these results in Sec.~\ref{sect:practical} to illustrate, in practice, the power of the EFT  approach.

\subsubsection{Muon anomalous magnetic moment}
\label{sect:gm2intro}
A long-standing discrepancy between SM predictions and observations concerns the anomalous magnetic moment of the muon.
The magnetic moment of the muon,~$\boldsymbol{\mu}_\mu$, is defined as
\begin{align}
\boldsymbol{\mu}_\mu &= g_\mu \brackets{\frac{e}{2m_\mu}} \boldsymbol{s} \, ,
\end{align}
where $\boldsymbol{s}$ denotes the muon spin and $g_\mu$ is 
the so-called $g$-factor. The prediction from the Dirac equations is $g_\mu = 2$; however, in QFT this value is modified by quantum effects sensitive to heavy degrees of freedom. The interesting quantum effects are parametrized by the anomalous magnetic moment, $a_\mu = \frac{1}{2}\brackets{g_\mu - 2}$. According to the detailed analysis by~\textcite{Aoyama:2020ynm}, the current SM prediction is ${a_\mu^\mathrm{SM}=116 591 810(43) \times 10^{-11}}$. The E989 experiment at FNAL~\cite{Muong-2:2021ojo} recently measured a deviation from this value, that combined  with the previous BNL E821 experiment~\cite{Muong-2:2006rrc}, yields a $4.2\,\sigma$ discrepancy:
\begin{equation}
%%a_\mu^\mathrm{Exp} &= 116 592 061(41) \times 10^{-11} \, ,
\Delta a_\mu = a_\mu^\mathrm{Exp} - a_\mu^\mathrm{SM} = \brackets{251 \pm 59} \times 10^{-11} \, .
\label{eq:gm2exp}
\end{equation}
The chance of a statistical fluctuation of this size is below $0.003\,\%$ making this an interesting hint of possible BSM dynamics. We will discuss the possible interpretation of this effect in terms of the SM effective field theory 
in Sec.~\ref{sec:g-2_SMEFT}. However, we warn the reader that there is an intense debate on the reliability of the error in the SM prediction entering 
(\ref{eq:gm2exp}). The main uncertainty is due to hadronic contributions to the photon vacuum-polarization amplitude. The latter is computed either via $\sigma(e^+e^- \to {\rm hadron})$ data and dispersion relations, or via lattice~QCD.  
Recent results from lattice QCD \cite{Borsanyi:2020mff} [see also \cite{FermilabLattice:2022izv,Ce:2022kxy,Alexandrou:2022amy}]
hint at a possibly smaller deviation from the SM than what was obtained in \cite{Aoyama:2020ynm} using dispersive techniques, see also \cite{Colangelo:2022vok}.
More recently, a new measurements of $\smash{\sigma(e^+e^- \to {\rm hadron})}$, presented in \cite{CMD-3:2023alj},
 also shows some discrepancies with previous experimental inputs used in the dispersive approach.
%by \textcite{Aoyama:2020ynm}.
 

\subsubsection{Lepton universality violation}
\label{sec:LFUV-intro}
Deviations from the SM predictions have recently been reported in tests of lepton flavor universality in semileptonic $B$-meson decays. 
These tests are performed via
universality ratios, such as 
\begin{align}
\label{eq:RD-ratio}
R_{D^{(\ast)}} = \frac{\mathcal{B}\!\brackets{B \to D^{(\ast)} \tau \nu_\tau}}{\mathcal{B}\!\brackets{B \to D^{(\ast)} \ell \nu_\ell}} \,,
\end{align}
where $\ell \in \{\mu,e\}$,
probing the quark-level amplitude  ${b \to c \ell \nu}$,
and similar ratios in neutral-current processes of the type 
${b \to s \ell \ell}$. These ratios can be predicted with high accuracy within the SM due the cancellation of hadronic uncertainties. The latest results on 
$R_{D^{(\ast)}}$ indicate a $3.1\,\sigma$ deviation from the SM predictions \cite{HFLAV:2022pwe}.
We will discuss the possible interpretation of this effect in terms of the SM effective field theory 
in Sec.~\ref{sec:Drell-Yan}. 
Till recently, an even more significant deviations was reported by the LHCb experiment in  universality ratios 
in  ${b \to s \ell \ell}$ decays; however, this effect has not been confirmed by the latest analysis \cite{LHCb:2022qnv}. 


% - - - - - - - - - - - - - - - - - - - - - - - - - - - - - - - - - - - - - - - - - - - - - - - - - - - -
% - - - - - - - - - - - - - - - - - - - - - - - - - - - - - - - - - - - - - - - - - - - - - - - - - - - -


\subsection{Effective field theories}

In physics we are interested in very different length or energy scales. Starting from the scale of the whole universe for cosmological studies all the way down to the scales of elementary particle physics at the LHC, the relevant energy scales indeed vary by many orders of magnitude. Each energy region usually requires its own physical theory to describe its phenomena. Remarkably, 
we often do not need to know in detail the laws at all energies if we want to describe processes at a given scale: it often suffices to set scales that are small or large compared to the process of interest to zero or to infinity, respectively, to get correct results. This is the basic principle of effective theory. We state it as principle; however, in a wide class of quantum field theories, and specifically when considering effective theories with an ultraviolet cutoff, this principle follows from the decoupling theorem 
~\cite{Appelquist:1974tg}.\footnote{Possible exceptions are discussed in~\cite{Donoghue:2009mn}.} 


Computations in an effective theory are usually simpler than in the full theory and reproduce the complete results with a  degree of accuracy that can be systematically improved. A~common example is Newtonian mechanics, which is the effective theory of special relativity in the limit of small energies and small velocities. Relativistic (or post-Newtonian) corrections are included by an expansion in the small parameter $v^2/c^2$ to the desired accuracy. 
An excellent description about the essence of effective quantum field theories
% is given in~\cite{Georgi:1994qn,Weinberg:2016kyd,Manohar:2018aog,Skiba:2010xn},
is the review by \textcite{Georgi:1994qn}, the article by \textcite{Weinberg:2016kyd}, or \cite{Manohar:2018aog,Skiba:2010xn},
on which our discussion below is based. Recent and further information can be found in the {\em All Things EFT} lecture series \cite{AllThings}.

Quantum EFTs as we use them today grew out of the
attempts to simplify and systematize the calculations of low-energy pion observables, originally based on current algebra techniques. \textcite{Weinberg:1978kz}\footnote{For early work see \cite{Weinberg:1966fm,Dashen:1969ez}. See also~\cite{Weinberg:1980wa, Weinberg:2009bg}.} argued that adhering just to the relevant symmetry properties embodied in current algebra, it is possible to construct an effective Lagrangian of the pion fields able to reproduce all known results and greatly simplifying the treatment. This, together with the work of \textcite{Wilson:1969zs}, that clarified the concept of integrating out heavy states in QFT obtaining universal results, and the related decoupling theorem by~\textcite{Appelquist:1974tg},
put EFTs on a solid basis. Starting from this basis, the systematic construction of the
effective theory of low-energy QCD, namely ChPT, was developed by \textcite{Gasser:1984gg, Gasser:1983yg}. This theory, whose leading expansion parameter is $E/\Lambda_\mathrm{QCD}$, where $E$ is the energy of the process, has been applied with great success to describe with high precision a multitude of low-energy systems. For a recent review see~\cite{Ananthanarayan:2023gzw}.  

Another well-known example of an effective theory is Fermi's theory of weak interactions~\cite{Fermi:1934hr}, 
which is part of the EFT of the Standard Model, and actually the first quantum EFT considered in particle physics (although its recognition as quantum theory, valid also beyond lowest order in the loop expansion, arrived much later).  
While certain amplitudes of the Fermi theory diverge at high energies, thereby violating unitarity, this does not spoil the low-energy limit of the SM, and particularly the 
infrared~(IR) behavior of QCD and~QED, which are still correctly reproduced.

To elucidate in simple terms the basic concepts of quantum EFT,
let's consider a theory containing two types of fields $\phi_L$ and $\phi_H$. Let's further assume $m \ll M$, where $m$ denote the mass of the excitations of $\phi_L$, and $M$ the one associated to $\phi_H$. 
The generating functional of the sources~$J_L$ associated to the light fields, and the corresponding 
EFT Lagrangian, can be obtained by performing the path integral over the heavy fields
%\footnote{We use dimensional regularization in $D=4-2\epsilon$ dimensions.}
\begin{align}
Z[J_L] &= \! \int \!\!  \mathcal{D} \phi_L \exp\! \left[ \int \!\! \dd^4 x  \left(\L_\mathrm{EFT}\!\brackets{\phi_L} + \phi_L J_L\right) \right] 
\label{eq:EFT_integrate-out}
\\
&=\! \int \!\!  \mathcal{D} \phi_H   \mathcal{D} \phi_L \exp\!\left[ \int \!\! \dd^4 x \left( \L\!\brackets{\phi_L , \phi_H} + \phi_L J_L\right) \right]. 
\nonumber
\end{align}
This formal manipulation, usually referred to as \textit{integrating out} the heavy degrees of freedom, essentially amounts to averaging over all $\phi_H$ configurations. The $\L_\mathrm{EFT}\brackets{\phi_L}$ thus
obtained contains non-local operators built only out of the light fields. Using an operator product expansion we can then express $\L_\mathrm{EFT}$ as a generally infinite sum of higher-dimensional operators
\begin{align}
\L_\mathrm{EFT} &= \L_{d \leq 4} + \sum_{d=5}^\infty \frac{1}{M^{d-4}} \sum_{i=1}^{n_d} C_i^{(d)} Q_i^{(d)} \, ,
\label{eq:EFT_Lagrangian}
\end{align}
where $d$ is the (mass) dimension of the operator~$Q_i^{(d)}$, 
and $n_d$ is the number of independent operators at a given dimension~$d$, 
which is always finite. The effective couplings~$\smash{C_i^{(d)}}$, 
associated to each operator, are dubbed \textit{Wilson coefficients}.
This procedure of integrating out the heavy fields changes the ultraviolet~(UV) structure of the theory, but it ensures that the EFT is constructed in such a way as to reproduce the same low-energy behavior as the original theory.
 
Higher-dimensional operators are suppressed by inverse powers of the mass scale of the heavy fields~$M$. Computing physical observables using $\L_\mathrm{EFT}$ thus leads to an expansion in powers of $E/M$, where $E$ is the typical energy scale of the process of interest. For the case $E \ll M$, which is exactly the validity region of the EFT, it is then sufficient to truncate the sum over~$d$ in Eq.~\eqref{eq:EFT_Lagrangian} at some finite order depending on the required accuracy of the result.

Since the operators~$\smash{Q_i^{(d)}}$ in Eq.~\eqref{eq:EFT_Lagrangian} are of mass dimension~$d>4$, these terms are non-renormalizable in the traditional sense, that is all infinities cannot be absorbed in a finite number of coefficients. For example, a divergent Feynman graph with two insertions of a $d=5$ operator is of order~$\smash{\ord{M^{-2}}}$ and therefore requires a counterterm of mass dimension~$d=6$. A diagram with two insertions of this counterterm would then require a $d=8$ counterterm and so on. Thus an infinite set of operators would be required to render the theory finite. However, the EFT comes with an associated expansion in powers of~$E/M$: if all terms with more than $k$~powers of this parameter are neglected, only a finite set of parameter remains and the theory can be renormalized in the usual sense. This means that all infinities up to terms of order~$\smash{\brackets{E/M}^k}$ can be canceled by a finite set of couplings, and that the corresponding renormalization group~(RG) equations can be derived.

The procedure of integrating out heavy particles as shown in Eq.~\eqref{eq:EFT_integrate-out} can be performed repeatedly. Suppose we have a theory with particles at several well separated mass scales~$\Lambda_1 \gg \Lambda_2 \gg \Lambda_3 \gg \ldots$. We can first integrate out the heavy particles at the scale~$\Lambda_1$ then compute the RG equations of the resulting EFT to run the theory down from~$\Lambda_1$ to~$\Lambda_2$. Next, we can integrate out the particles at the mass scale~$\Lambda_2$ obtaining a second EFT only containing practices with masses~$\lesssim \Lambda_3$. Again we can compute the RG equations of the new EFT to run down to the scale~$\Lambda_3$ and so on, until we reach the desired mass scale. The advantage of this multi-step procedure is the systematic resummation of large logarithms, that would appear in the matching steps if we would only do a single matching at the desired scale and integrate out all heavy particles at once.

The scenario as described above can be viewed as \textit{top-down} approach to EFTs: we start with a known theory at the high scale and integrate out the heavy particles. This is the adequate procedure when we strive to make precise predictions from a known theory with known UV behavior.
However, EFTs can also be a useful tool if the full theory at the high scale is unknown, but only some of its features. This was for instance the case for the strong interactions before the discovery of the $\mathrm{SU}(3)_c$~gauge theory which was helped by the work on current algebra and chiral perturbation theory. This scenario is often referred to as \textit{bottom-up approach}. This is also the case for  the present situation, where the Standard Model is known and one would like to understand the underlying theory. This is the approach of the SM effective theory. In this case the operators~$\smash{Q_i^{(d)}}$ in Eq.~\eqref{eq:EFT_Lagrangian} do not emerge in the matching procedure, but have to be constructed using symmetry arguments. Suppose we want to find the EFT operators for the~SM. In this case we have $\L_{d \leq 4}=\L_\mathrm{SM}$ and for the EFT operators~$\smash{Q_i^{(d)}}$ at a given mass dimension~$d$ we simply have to construct all structures that are invariant under the local and global symmetries of the theory of interest, i.e., the~SM. In this bottom-up setup we usually replace the explicit mass~$M$ in Eq.~\eqref{eq:EFT_Lagrangian} by a generic UV scale~$\Lambda$, that can be identified with a heavy BSM mass scale once the EFT is matched to some UV theory. In the remainder of this review we will focus on these EFT extensions of the SM, with particular emphasis on the so-called~SMEFT.


	

% - - - - - - - - - - - - - - - - - - - - - - - - - - - - - - - - - - - - - - - - - - - - - - - - - - - -
% - - - - - - - - - - - - - - - - - - - - - - - - - - - - - - - - - - - - - - - - - - - - - - - - - - - -



\subsection{The Standard Model as an effective theory}
\label{smeft}

As mentioned above, the Standard Model can be interpreted as the leading-order dimension-four piece of a larger effective theory. This EFT must have the same gauge  symmetries as the~SM. There are actually two candidate EFTs that are distinguished only by their assumptions on the realization of the electroweak symmetry group. The Standard Model effective field theory~(SMEFT) assumes that the electroweak symmetry is realized linearly, whereas the Higgs effective field theory~(HEFT) assumes a non-linear realization.\footnote{The HEFT is sometimes also called the electroweak chiral Lagrangian~(EWChL).} In the SM both versions are equivalent as they are related by a field redefinition. However, they lead to different EFT descriptions  as in the EFT framework it is not always possible to find a field redefinition to go from a non-linear to a linear realization of the electroweak symmetry. We review this issue in more detail in Sec.~\ref{sec:HEFT}. The HEFT is thus a more general theory containing the~SMEFT as special case. 

In this review we will focus mainly on the SMEFT, on the one hand because of its ``simplicity'', on the other hand because  present data on SM precision tests and Higgs couplings seem to favor a linearly realized electroweak symmetry, i.e., a fundamental (or quasi-fundamental) Higgs field transforming as as doublet of~$\mathrm{SU}(2)_L$. For an extensive discussion about differences between  HEFT and SMEFT we refer to~\cite{Brivio:2017vri}.

Applying the general concepts of EFT discussed in the previous section, we can decompose the SMEFT Lagrangian as
\begin{samepage}
\begin{align}
\L_\mathrm{SMEFT}(\psi,H,A) &= \L_\mathrm{SM}(\psi,H,A) 
\label{eq:SMEFT_Lagrangian_general}
\\
&\quad+ \sum_{d=5}^{\infty} \sum_{i=1}^{n_d} \frac{C_i^{(d)}}{\Lambda^{(d-4)}} Q_i^{(d)}(\psi,H,A)\, .
\nonumber
\end{align}
\end{samepage}
Here, $\psi$, $H$, and~$A$ collectively denoted the  SM fields listed in Tab.~\ref{tab:SM_field-content}. 
% Comparing to Eq.~\eqref{eq:EFT_Lagrangian} we replaced the mass of the heavy particles~$M$ by some generic UV scale~$\Lambda$ far above the electroweak scale $(\Lambda \gg v)$. 
The key assumption of this construction is indeed the hypothesis that physics beyond the SM is characterized by one or more heavy scales. 
As in most literature, we adopt the convention where 
the Wilson coefficients~$C_i^{(d)}$ are dimensionless quantities, this is why we
pull out explicitly the factor $\Lambda^{(4-d)}$ in the effective couplings. 
In principle, the sum on $d$ runs over all possible values; however, the majority of our discussion will be focused on operators up to dimension~six, and therefore we often drop the superscript~$(d)$ denoting the operator dimension.

After fixing the mass dimension up to which we expand the EFT, which is equivalent to determining the desired accuracy of our result, $\L_\mathrm{SMEFT}$ is capable of describing the low-energy signatures of generic UV completions of the~SM. 
One of the less trivial aspect of this approach is the construction 
of a suitable basis of operators 
at a given dimension. Not surprisingly, a long time passed from the initial attempt 
to define a consistent basis for the SMEFT at dimension~six by \textcite{Buchmuller:1985jz}, till the identification of a complete and non-redundant basis by \textcite{Grzadkowski:2010es}.
We will review in detail how this is done in general, 
and specifically for the SMEFT up to dimension~six, in Sec.~\ref{subsec:SMEFT_Operator-basis}.

In many realistic UV completions, 
the physics above the electroweak scale is characterized by several mass scales. 
What matters to determine the convergence of the EFT expansion is the lowest of such scales, that we can identify with~$\Lambda$. 
However, the presence of additional energy scales can play a role 
in determining the size of the~$\smash{C_i^{(d)}}$, given the conventional choice of assuming a 
unique normalization scale $\Lambda$ in (\ref{eq:SMEFT_Lagrangian_general}). We will come back to this point in more detail at the end of Sec.~\ref{sect:SMEFT} and in Sec.~\ref{sec:GlobalSymmetries}.

The two key assumptions of this construction in describing generic extensions of the SM is that no unknown light particles exist and the electroweak symmetry is linearly realized. 
Under these hypotheses, any experimental result on the search for new physics can be given in the framework of the SMEFT, i.e.~in terms of bounds on the Wilson coefficients, if the energies probed in the experiment are well below the scale of new physics. At the same time, different models of new physics can be matched onto the SMEFT Lagrangian by integrating out the heavy particles in each theory. More interestingly, if a deviation from the SM emerges, the SMEFT can be used to test its consistency pointing out correlated observables and discriminating among large varieties of UV completions. Illustrating all this with concrete examples is the subject of Sec.~\ref{sect:practical}.

The absence of light new particles is definitely a strong hypothesis. Several examples of light new states, such as axion-like particles or the dilaton, are well motivated and can originate by physics at energies far beyond the weak scale. However, such new states are necessarily very weakly coupled to the SM fields (otherwise they would have already been discovered). This implies we can neglect their effect in a large class of observables, for which the description in terms of the SMEFT remains a very efficient tool. Of course, to describe in full generality these frameworks requires to add the corresponding light fields in the EFT. This can be done, case by case, according the nature of the new degrees of freedom, but is beyond the scope of this review. 