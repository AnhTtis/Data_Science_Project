\section{Global symmetries}
\label{sec:GlobalSymmetries}

\subsection{The role of accidental symmetries}
\label{sect:SymmA}
A key concept in any EFT is that of {\em accidental symmetries},
i.e., symmetries that arise in the lowest-dimensional operators as
indirect consequences of the field content and the symmetries explicitly imposed on the theory.
Within the SMEFT, two well-known examples are baryon number~($B$) and lepton number~($L$).
These are exact accidental global symmetries of the $d=4$ part of the Lagrangian, or the SM:
they do not need to be imposed in the SM because gauge invariance forbids to write any 
$d=4$~operator violating $B$~or~$L$. 

If the accidental symmetries are not respected by the underlying UV~completion,
we expect them to be violated by the higher-dimensional operators. 
The strong bounds on $B$-violating terms from proton stability, and the tiny coefficient of the $L$-violating 
Weinberg operator in Eq.~\eqref{eq:Weinberg-Operator} from neutrino masses, indicate that such symmetries 
remain almost unbroken in the SMEFT. This observation can be interpreted in a natural way assuming that the fundamental 
interactions responsible for $B$ and $L$ violation appear at very high energy scales, therefore, assuming 
a very high cutoff scale for these operators. 
This is not in contradiction with the possibility of having a lower cutoff scale 
for the $d=6$~SMEFT operators preserving $B$~and~$L$, since the symmetry-preserving sector cannot 
induce violations of the global symmetries. In other words, accidental global symmetries allow us to 
define a stable partition of the tower of effective operators into different sectors characterized 
by different cutoff scales, reflecting a possible multi-scale structure of the underlying theory.
The key point is that this partition is stable with respect to quantum corrections. 

Besides $B$~and~$L$, the SM Lagrangian (or better the SMEFT at~$d=4$) has two additional exact accidental
global symmetries related to the individual lepton flavor, that we can conventionally choose as 
$L_{e-\mu}$ and~$L_{\mu-\tau}$ (combined with~$L$, these correspond to the conservation of each 
individual lepton flavor). However, a much larger number of {\em approximate accidental symmetries} 
appears in the limit where we neglect the tiny Yukawa couplings of the light families and the small
off-diagonal entries of the Cabibbo-Kobayashi-Maskawa matrix. These approximate flavor symmetries 
are responsible for the smallness of flavor-changing neutral-current (FCNC) processes, such as $B$--$\overline{B}$
and $K$--$\overline{K}$ mixing, which are severely constrained by data. 
Despite the precision and the energy scales involved are very different, the situation is 
similar to that of $B$~and~$L$: the experimental bounds on FCNC processes 
imply high cutoff scales for the $d=6$~operators violating the approximate SM flavor symmetries. 
In practice, there is no real difference between exact and approximate accidental symmetries: in both cases we can 
conceive a multi-scale structure in order to preserve a low effective scale in the symmetry-preserving sector of the theory.  
More precisely, the scale of the symmetry-preserving sector of the SMEFT can be as low as few TeV, if at that  
scale not only $B$~and~$L$, but also the tightly constrained accidental flavor symmetries, remain valid, 
or are broken only by small symmetry-breaking terms. 

The technical implementation of the concept of small symmetry-breaking terms, in presence of approximate 
(or exact) symmetries in the low-energy sector of the EFT, is obtained via the {\em spurion} technique, 
discussed in Sec.~\ref{sect:Flavor}. Generalizing the case of exact accidental symmetries, this technique 
can be viewed as a consistent partitioning of the tower of effective operators, reflecting a possible 
underlying multi-scale structure. This classification is particularly important in the SMEFT, given the 
large number of flavor-violating operators at~$d=6$, and the very different bounds on the 
symmetry-preserving and symmetry-breaking terms. If we do not conceive an underlying multi-scale 
structure, we are unavoidably led to the conclusion that the cutoff scale of the SMEFT is extremely high,
preventing the observation of any deviation from the SM except in rare $B$- or $L$-violating processes.


\subsection{Baryon and lepton number}
As already discussed in Sec.~\ref{subsec:SMEFT_Operator-basis},  
the unique $d=5$~operator of the SMEFT is the $L$-violating term in Eq.~\eqref{eq:Weinberg-Operator}.
This operator provides a clear illustration of the general concept of accidental symmetries discussed above:
it describes well all phenomena related to neutrino masses,
hence it provides an (indirect) evidence\footnote{Alternative description of neutrino masses
not involving the Weinberg operator and preserving~$L$ are possible, but require the enlargement of the field 
content or the inclusion of operators of even higher dimension, see e.g. \cite{Gonzalez-Garcia:2007dlo}.} that~$L$ is violated beyond~$d=4$.
On the other hand, its coupling inferred from neutrino masses points to a very high effective scale:
$10^{14}\,{\rm TeV} < \Lambda < 10^{15}\,{\rm TeV}$ for $\mathcal{O}(1)$~coefficients
(following from $0.03\,{\rm eV} < \sum m_\nu  < 0.3\,{\rm eV}$).

Possible baryon number violating terms appear first at~$d=6$. The complete list of the $B$~(and~$L$) violating $d=6$~operators is shown in Tab.~\ref{Tab:Warsaw-basis_BV}. These operators satisfy the SM gauge symmetries because of the $\SU{3}_c$~property $\boldsymbol{3} \otimes \boldsymbol{3} \otimes \boldsymbol{3} \sim \boldsymbol{1}$. This is also the reason why there are no baryon and lepton number violating operators with three leptons and one quark at dimension six, and why $B-L$ is conserved at this order. The strong bound from proton decay implies severe bounds on some of these operators: $\Lambda > 10^{16}\,\mathrm{GeV}$, for $\mathcal{O}(1)$~coefficients, for terms involving only first-generation fermions.  
The constraints are significantly weaker for operators involving heavy fermions which cannot contribute to proton decay at the tree 
level \cite{Nikolidakis:2007fc}.

\begin{table}[t]
\centering
\renewcommand{\arraystretch}{1.5}
\begin{tabular}{| lc |}
	\hline
	\multicolumn{2}{| c |}{Baryon number violating $\psi^4$ operators}
	\\ \hline
	$Q_{duql}$ & $\varepsilon^{abc}\varepsilon^{ij} \squarebrackets{{(d_{a p})}^\intercal C u_{b r}} \squarebrackets{{(q_{c i s})}^\intercal C \ell_{j t}}$ 
	\\
	$Q_{qque}$ & $\varepsilon^{abc}\varepsilon^{ij} \squarebrackets{{(q_{a i p})}^\intercal C q_{b j r}} \squarebrackets{{(u_{c s})}^\intercal C e_t}$
	\\
	$Q_{qqql}$ & $\varepsilon^{abc}\varepsilon^{il}\varepsilon^{jk} \squarebrackets{{(q_{a i p})}^\intercal C q_{b j r}} \squarebrackets{{(q_{c k s})}^\intercal C \ell_{l t}}$
	\\
	$Q_{duue}$ & $\varepsilon^{abc} \squarebrackets{{(d_{a p})}^\intercal C u_{b r}} \squarebrackets{{(u_{c s})}^\intercal C e_t}$ 
	\\[0.1cm]
	\hline
\end{tabular}
\caption{Baryon number violating dimension-six operators in the Warsaw basis \cite{Grzadkowski:2010es}, with the operator labels adopted from \cite{Alonso:2014zka}. The color indices are labeled~$\{a,b,c\}$, the indices of $\SU{2}_L$ are~$\{i,j,k,l\}$, the flavor indices read~$\{p,r,s,t\}$, the charge conjugation matrix is~$C=i\gamma^2\gamma^0$, and $\varepsilon$ denotes the totally antisymmetric rank two or three tensor respectively.
\label{Tab:Warsaw-basis_BV}}
\end{table}



\subsection{Flavor symmetries}
\label{sect:Flavor}

After imposing exact $B$~and~$L$ conservation, the number of independent electroweak structures at~$d=6$ amounts 
to the $59$~terms listed in Tab.~\ref{tab:Warsaw-basis}.
The huge proliferation in the number of independent coefficients in the SMEFT at~$d=6$ occurs when all the possible flavor structures for these 
terms are taken into account: in absence of any flavor symmetry, they amount to 1350 CP-even and 1149 CP-odd independent coefficients for the dimension-six operators \cite{Alonso:2013hga}.  

Among these couplings, those contributing at tree level to flavor-violating observables, in particular 
meson--antimeson mixing and lepton-flavor violating processes are strongly constrained: 
these set bounds of $\mathcal{O}(10^{5}\,\text{TeV})$ on~$\Lambda$ for $\mathcal{O}(1)$~coefficients \cite{Isidori:2010kg}. 
If this high scale were the overall cutoff scale of the SMEFT, it would imply that all the other $d=6$~operators play an irrelevant 
role in current experiments, making the whole construction not very interesting from the phenomenological point of view.  
On the other hand, from the known structure of the SM Yukawa couplings, we know that flavor is highly non-generic, 
at least in the $d=4$~sector of the SMEFT. As anticipated, it is conceivable to assume this being the result of an underlying 
multi-scale structure, leading to approximate flavor symmetries in the whole SMEFT also beyond~$d=4$.
This assumption allows us to reduce, in a consistent way, the number of relevant parameters, making the whole 
construction more consistent and  more interesting from the phenomenological point of view, with competing constraints from  
flavor-conserving and flavor-violating processes on a given effective operator.

The price to pay to achieve this goal is the choice of the flavor symmetry and symmetry-breaking sector, which necessarily introduces some model dependence, given there is no exact flavor symmetry to start with (contrary to the case of $B$~and~$L$). 
If we are interested in symmetries and symmetry-breaking patterns able to successfully reproduce the SM Yukawa couplings and, at the same time, suppress non-standard contributions to flavor-violating observables, the choice is limited.  
Here, we analyze in some detail two cases which are particularly motivated from this point of view: the flavor symmetries $\mathrm{U}(3)^5$ and~$\mathrm{U}(2)^5$, with possible minor variations. In both cases the starting point is the flavor symmetry allowed by the SM gauge group. 

The $\mathrm{U}(3)^5$~symmetry is the maximal flavor symmetry allowed by the SM gauge group, while $\mathrm{U}(2)^5$ is the corresponding  subgroup acting only on the first two (light) generations. The $\mathrm{U}(3)^5$~symmetry allows us to implement the minimal flavor violation~(MFV) hypothesis \cite{Chivukula:1987py,DAmbrosio:2002vsn}, which is the most restrictive consistent hypothesis we can utilize in the SMEFT  to suppress non-standard contributions to flavor-violating observables \cite{DAmbrosio:2002vsn}. The $\mathrm{U}(2)^5$ symmetry with minimal breaking \cite{Barbieri:2011ci,Barbieri:2012uh,Blankenburg:2012nx} is quite interesting since it retains most of the MFV virtues, but it allows us to have a much richer structure as far as third-generation dynamics is concerned. 


%============================================================================
\subsubsection{\texorpdfstring{$\mathrm{U}(3)^5$}{U(3)5} and minimal flavor violation}
\label{sec:U3}
%============================================================================
The largest group of flavor-symmetry transformations compatible with 
the gauge symmetries of the SM Lagrangian is \cite{Gerard:1982mm,Chivukula:1987py}
\bea
	\cG_f &=&  \mathrm{U}(3)_\ell \otimes \mathrm{U}(3)_q \otimes \mathrm{U}(3)_e \otimes \mathrm{U}(3)_u \otimes \mathrm{U}(3)_d \no\\
&\equiv& \mathrm{U}(3)^5 = \mathrm{SU}(3)^5 \otimes \mathrm{U}(1)^5 \,.
\eea
Within the SM, the Yukawa couplings ($Y_{e,u,d}$) are the only 
source of breaking of~$\cG_f$. They break this global symmetry as follows 
\bea
\cG_f = \left\{
\ba{l}
\mathrm{SU}(3)^5 \\[3pt]   \mathrm{U}(1)^5 
\ea \right. \!\!
\stackrel{ Y_{e,u,d}\neq0 }{\longrightarrow} 
\ba{l}
 \mathrm{U}(1)_{e - \mu} \otimes \mathrm{U}(1)_{\tau - \mu}  \\
 \mathrm{U}(1)_B \otimes \mathrm{U}(1)_L \otimes \mathrm{U}(1)_Y  \\
\ea \,, \quad
\eea
where we separated explicitly the flavor--universal and flavor--non-universal subgroups.
The three unbroken flavor--universal $\mathrm{U}(1)$ groups are baryon number, lepton number, and 
hypercharge.

Most of the $d=6$~SMEFT operators can be viewed as independent $\cG_f$--breaking terms,
hence they can be classified according to their transformation properties under~$\cG_f$.
To start with, let us consider the limit of unbroken~$\cG_f$: retaining only the $\cG_f$~invariant operators at~$d=6$ 
is not a fully consistent hypothesis, since $\cG_f$
is broken in the $d=4$~sector. However, it is a useful starting point for the classification of the operators,
and it is a coherent hypothesis to be implemented in the SMEFT in the limit where we neglect $\cG_f$--breaking terms also in the SM sector,
i.e., in the limit where we neglect the SM Yukawa couplings.

The number of independent $d=6$~terms respecting~$\cG_f$ is reported in Tab.~\ref{tab:U3new}
under the ``Exact''~$\mathrm{U}(3)^5$ column: the left (right) value in each entry indicates 
the number of CP-even (CP-odd) coefficients. For comparison, the counting of independent
coefficients if no symmetry is imposed, or if a single generation of fermions is considered, is also shown. 
As can be seen, the number of independent coefficients respecting the $\cG_f$~symmetry is smaller 
than in the single-generation case: this is because $\cG_f$ forbids bilinear couplings of fermions with 
different gauge quantum numbers, such as those appearing in the Yukawa couplings. 

% % % % % % % % % % % % % % % % % % % % % % % % % % % % % % % % % % % % % % % % % % % % 
\begin{table*}
\centering
\begin{center}
	\renewcommand{\arraystretch}{1.2} % set row height
	\begin{tabular}{c   l   ||   rr   |   rr   ||    rr   |   rr  |    rr  ||   rr  |   rr  |   rr   }
	& &   \multicolumn{4}{c||}{	No symmetry }  &  \multicolumn{6}{c||}{ $U(3)^5$} & \multicolumn{6}{c}{ $U(2)^5$}   \\
   Class & Operators & \multicolumn{2}{c|}{  3 Gen.}  & \multicolumn{2}{c||}{  1 Gen.}  &  \multicolumn{2}{c|}{  Exact } &  \multicolumn{2}{c|}{ $\cO(Y_{e,d,u}^1)$ }   &  \multicolumn{2}{c||}{ $\cO(Y_{e}^1, Y_d^1 Y^2_u)$} & \multicolumn{2}{c|}{Exact} & \multicolumn{2}{c|}{$\mathcal{O}(V^1)$} & \multicolumn{2}{c}{$\mathcal{O}(V^2,\Delta^1)$}   \\  \hline
		1--4 	& $X^3$, 	$H^6$, $H^4 D^2$, $X^2 H^2$		& 9 		& 6 		& 9	& 6 	& 9	& 6  	& 9	& 6   &  9  &  6  	& 9 	& 6	& 9	& 6	 & 9		& 6	\\	\hline
		5 					& $\psi^2 H^3$ 			& 27 		& 27 		& 3	& 3	& -- 	& -- 	& 3	& 3	&  4	& 4	& 3 	& 3	& 6	& 6	 & 9		& 9	\\	
		6 					& $\psi^2 X H$ 			& 72		& 72		& 8	& 8	& --	& --	& 8	& 8	&  11	& 11	& 8	& 8	& 16	& 16	 & 24		& 24	\\
		7 					& $\psi^2 H^2 D$		& 51		& 30		& 8	& 1	& 7	& --	& 7	& --   & 11	& 1  	& 15	& 1	& 19	& 5	 & 23		& 5		\\ 	\hline
							& $(\bar{L}L)(\bar{L}L)$	& 171	& 126	& 5	& --	& 8	& --    & 8	& --	& 14 &-- 	& 23	& --	& 40	& 17	 & 67		& 24	  	\\
							& $(\bar{R}R)(\bar{R}R)$	& 255	& 195	& 7	& --	& 9	& --	& 9	& --	& 14 &-- 	& 29	& --	& 29	& --	 & 29		& -- 		\\
		8					& $(\bar{L}L)(\bar{R}R)$	& 360	& 288	& 8	& --	& 8	& --	& 8	& --	& 18 &-- 	& 32	& --	& 48	& 16  & 69		& 21			\\
							& $(\bar{L}R)(\bar{R}L)$ 	& 81		& 81		& 1	& 1	& --	& --	& --	& --	& --	& --	& 1	& 1	& 3	& 3	 & 6		& 6	\\	
							& $(\bar{L}R)(\bar{L}R)$ 	& 324	& 324	& 4	& 4	& --	& --	& --	& --	& 4	& 4	& 4	& 4	& 12	& 12	 & 28		& 28		\\ 	\hline
		\multicolumn{2}{c||}{\bf total:}					&1350	&1149	& 53	& 23 & 41	& 6	& 52 & 17	& 85 & 26 & 124& 23	&182	& 81	 & 264	& 123
			\end{tabular}
	\caption{Number of independent $d=6$~SMEFT operators without any symmetry for three and one generation(s), and when imposing a
	$\mathrm{U}(3)^5$ or $\mathrm{U}(2)^5$ flavor symmetry with different powers of symmetry-breaking terms \cite{Faroughy:2020ina}.
	In each column the left (right) number corresponds to the number of CP-even (CP-odd) coefficients.  
	$\cO(X^n)$ stands for including terms up to $\cO(X^n)$. 
	\label{tab:U3new}}
	\end{center}
\end{table*}
% % % % % % % % % % % % % % % % % % % % % % % % % % % % % % % % % % % % % % % % % % % % 

The MFV~hypothesis is the assumption that the SM Yukawa couplings are the only sources of $\mathrm{U}(3)^5$~breaking \cite{Chivukula:1987py,DAmbrosio:2002vsn}. 
The exact $\mathrm{U}(3)^5$~limit can be viewed as employing the MFV~hypothesis and working to zeroth order in the symmetry-breaking terms.
To go beyond the leading order, we promote the SM Yukawa couplings to become $\mathrm{U}(3)^5$~spurions, i.e., non-dynamical fields with 
well-defined transformation properties under~$\mathrm{U}(3)^5$. The latter are deduced by the 
structure of the SM Lagrangian \cite{DAmbrosio:2002vsn}:
\begin{align}
    Y_u&=\left( \boldsymbol{1},\boldsymbol{3},\boldsymbol{1}, \bar{\boldsymbol{3}},\boldsymbol{1} \right)\,,
    &
    Y_d&=\left( \boldsymbol{1},\boldsymbol{3},\boldsymbol{1},\boldsymbol{1},\bar{\boldsymbol{3}} \right)\,, 
    \nonumber\\
	Y_e&=\left( \boldsymbol{3},\boldsymbol{1},\bar{\boldsymbol{3}},\boldsymbol{1},\boldsymbol{1} \right)\,.
\end{align}
With these transformation properties, the $d=4$~sector of the theory is formally invariant under~$\mathrm{U}(3)^5$.
The MFV~hypothesis consist in constructing the higher-dimensional operators using SM fields and 
spurions, such that the EFT remains formally invariant under~$\mathrm{U}(3)^5$ to all orders, 
and the breaking occurs only via the appropriate insertions of the spurions~$Y_{u,d,e}$.

In principle, the spurions can appear with arbitrary powers both in the renormalizable ($d=4$)~part of the Lagrangian 
and in the dimension-six effective operators. 
However, via a suitable redefinition of both fermion fields and spurions, we can always put the $d=4$~Lagrangian 
to its standard expression, identifying the spurions with the SM Yukawa couplings. 
This implies we can always choose a flavor basis where the spurions are 
completely determined in terms of fermion masses and the Cabibbo-Kobayashi-Maskawa~(CKM) matrix,~$V_{\rm CKM}$. A~representative 
example is the down-quark mass-eigenstate basis, where
\bea
&& Y_e =  \textrm{diag}(y_e,y_\mu,y_\tau)\,, \qquad
Y_d  = \textrm{diag}(y_d,y_s,y_b)\,,  \no \\
&& Y_u =  V_{\rm CKM}^\dagger \times \textrm{diag}(y_u,y_c,y_t)\,.
\label{eq:d-basis}
\eea
The key point is that there are no free (observable) parameters in the structure of the MFV spurions. 
A~related important point is the fact that, knowing the structure of the spurions, 
we know that they are all small but for the top Yukawa~$y_t$. We can thus limit the spurion expansion to a few terms. 

The overall number of independent terms allowed by the MFV hypothesis with at most one ``small'' Yukawa coupling, namely 
$Y_d$ or~$Y_e$, and up to two powers of~$Y_u$ is shown in the last column of Tab.~\ref{tab:U3new} \cite{Faroughy:2020ina}.
As can be seen, this number is almost two orders of magnitudes smaller than what is obtained 
in absence of any symmetry (for three generations) and quite close to the single generation case.
With the corresponding set of operators we can describe the SM spectrum and possible
deviations from the SM in a series of rare flavor-violating processes \cite{DAmbrosio:2002vsn}.
A~representative set of these operators is shown in Tab.~\ref{tab:MFV}. 


\begin{table}[t]
\begin{center}
\resizebox{\linewidth}{!}{%
\begin{tabular}{c | c | c}
EW type & possible MFV form & ~Bound on $\Lambda$ \\
\hline
$Q_{dB}$ 			& $ [\bar q_r (Y_u Y_u^\dagger Y_d)_{rp}   \sigma^{\mu\nu} d_p] H (g_1 B_{\mu\nu})  \phantom{\frac{P^X}{2}}$ 	& $6.1$~TeV  \\[2pt]   
$Q_{dG}$ 		& $ [\bar q_r (Y_u Y_u^\dagger Y_d)_{rp}  \sigma^{\mu\nu} T^A   d_p)  H (g_3 G_{\mu\nu}^A)$ 					& $3.4$~TeV  \\[2pt] 
$Q_{Hq}^{(1)}$   	& $(H^\dagger i \overleftrightarrow{D}_\mu H)  [{\bar q}_r (Y_u Y_u^\dagger)_{rp}  \gamma_\mu q_p]$ 			& $2.3$~TeV  \\[2pt]   
$Q_{qq}^{(1)}$ 		& $[\bar{q}_r  (Y_u Y_u^\dagger)_{rp} \gamma_{\mu} q_p][\bar{q}_r  (Y_u Y_u^\dagger)_{rp} \gamma_{\mu} q_p]$ 	& $5.9$~TeV  \\[2pt]     
$Q_{q e}$   		& $[{\bar q}_r (Y_u Y_u^\dagger)_{rp} \gamma_\mu q_p] [ {\bar e}_s \gamma_\mu e_s]$  					& $2.7$~TeV  \\[2pt]   
$Q_{\ell q}^{(1)}$	& $[{\bar q}_r (Y_u Y_u^\dagger)_{rp}  \gamma_\mu q_p]  [\bar{\ell}_s \gamma_\mu \ell_s]$ 					& $1.7$~TeV  \\[2pt]   
\end{tabular}
}
\end{center}
\caption{\label{tab:MFV} Representative set of SMEFT operators with their flavor structure determined according to the MFV~hypothesis. 
Each electroweak structure (first column) can admit different MFV implementations: in the second column we indicate the one more constrained 
by flavor-violating processes in the quark sector. The corresponding bounds on the effective scale set by $B$- and $K$-meson physics measurements is reported in the 
third column (95\%\,C.L.~bound, assuming an effective coupling $\sim\pm 1/\Lambda^2$, considering each operator separately).}
\end{table}


The number of insertions of the (large) $Y_u$~spurions has been limited to two since,  
in the reference basis~\eqref{eq:d-basis}, one gets 
\be
\left[  Y_u (Y_u)^\dagger \right]^n_{r\not = p} ~\approx~
y_t^{2n} V^*_{tr} V_{tp} \propto  [Y_u (Y_u)^\dagger]_{r\not = p} \,.
\label{eq:basicspurion}
\ee
This result implies that within the MFV hypothesis rare FCNC processes that, within the SM, are dominated by short-distance contributions induced
by the large top-quark mass (such as $B^0$--$\overline{B}{}^0$ and $K^0$--$\overline{K}{}^0$ mixing, $b\to s \gamma$, $b\to s \ell^+\ell^-$, \ldots),
receive exactly the same CKM suppression as in the SM:
\begin{align}
\mathcal{A}(d^i \to d^j)_{\rm MFV} &=  (V^*_{ti} V_{tj}) \mathcal{A}^{(\Delta F=1)}_{\rm SM}
\!\!\left[ 1 + a_1 \frac{ 16 \pi^2 M^2_W }{ \Lambda^2 } \right], \!\!\!\! \no \\  
\mathcal{A}(M_{ij}\!-\!{\overline{M}}_{ij})_{\rm MFV}  &= \! (V^*_{ti} V_{tj})^2
\mathcal{A}^{(\Delta F=2)}_{\rm SM} \!\!\left[ 1 + a_2 \frac{ 16 \pi^2 M^2_W }{ \Lambda^2 } \!\right]\!,
 %\no \\  
 % &\quad \times \left[ 1 + a_2 \frac{ 16 \pi^2 M^2_W }{ \Lambda^2 } \right]~,
\label{eq:FC}
\end{align}
where $\mathcal{A}^{(i)}_{\rm SM}$ denote the SM loop amplitudes and $a_i$ are $\mathcal{O}(1)$~parameters.
The~$a_i$ depend on different combinations of SMEFT coefficients but are flavor independent. 



%%%%%%%%%%%%%%%%%%%%%%%%%%%%%%%%%%%%%%%%%%%%%%%%%%%%%%%%%%%%%%%%%%%%%%%%%%%%%%%%


\subsubsection{The \texorpdfstring{$\mathrm{U}(2)^5$}{U(2)5} symmetry}



The $\mathrm{U}(2)^5$~flavor symmetry is the subgroup of the $\mathrm{U}(3)^5$ global symmetry   
that, by construction, distinguishes the first two generations of fermions 
from the third one \cite{Barbieri:2011ci,Barbieri:2012uh,Blankenburg:2012nx}.
For each set of SM fermions with the same gauge quantum numbers, the first two generations form a doublet of a given $\mathrm{U}(2)$ subgroup, whereas the third one transforms as a singlet. 
Denoting the five independent flavor doublets as $L,Q,E,U,D$, the flavor symmetry decomposes as 
\be
	\mathrm{U}(2)^5 = \mathrm{U}(2)_L \otimes \mathrm{U}(2)_Q \otimes \mathrm{U}(2)_E \otimes \mathrm{U}(2)_U \otimes \mathrm{U}(2)_D \,.
 \label{eqU25}
\ee
In the limit of unbroken~$\mathrm{U}(2)^5$, only third-generation 
fermions can have non-vanishing Yukawa couplings,
which is an excellent first-order approximation for the SM Lagrangian. 
This is why, contrary to the MFV case, the $\mathrm{U}(2)^5$~symmetry allows us to build an EFT where all the symmetry-breaking terms are small.

A~$\mathrm{U}(2)^3$~symmetry in the quark sector can be viewed as the result of a generalized MFV framework,
taking into account arbitrary insertions of the third-generation Yukawa couplings
without suppression (the so-called non-linear representation of MFV \cite{Feldmann:2008ja} or general MFV \cite{Kagan:2009bn}
hypothesis). However, this interpretation is less motivated in the lepton sector and it also 
implies a rather strict structure for the symmetry-breaking terms. On the other hand, the symmetry group in~\eqref{eqU25} with 
the symmetry-breaking terms discussed below, can be viewed as an effective way to describe in general terms 
the large class of SM extensions where the third-generation of fermions plays a special role. 


\paragraph{Yukawa couplings and spurion structures.}
A~set of symmetry-breaking terms sufficient to reproduce 
the complete structure of the SM Yukawa couplings is \cite{Barbieri:2011ci}
\begin{align}
V_\ell &\sim\left(\boldsymbol{2},\boldsymbol{1},\boldsymbol{1},\boldsymbol{1},\boldsymbol{1}\right)\,,
&
V_q &\sim\left(\boldsymbol{1},\boldsymbol{2},\boldsymbol{1},\boldsymbol{1},\boldsymbol{1}\right)\,,  \no 
\\ 
\Delta_e &\sim\left(\boldsymbol{2},\boldsymbol{1},\bar{\boldsymbol{2}},\boldsymbol{1},\boldsymbol{1}\right)\,,
&
\Delta_{u(d)} &\sim\left(\boldsymbol{1},\boldsymbol{2},\boldsymbol{1},\bar{\boldsymbol{2}}(\boldsymbol{1}),\boldsymbol{1}(\bar{\boldsymbol{2}})\right)\,.
\label{eq:U2spur}
\end{align}
By construction, $V_{q,\ell}$ are complex two-vectors and $\Delta_{e,u,d}$ are complex $2\times 2$~matrices. In terms of these spurions, we can express the Yukawa couplings as
\bea
&&	Y_e = y_\tau\left(\begin{matrix}
		\Delta_e	 & x_\tau V_\ell \\
		0			 & 1
	\end{matrix}\right)\,, \quad
	 Y_u = y_t\left(\begin{matrix}
		\Delta_u	 & x_t V_q \\
		0			 & 1
	\end{matrix}\right)\,,  \no\\
&&	 Y_d = y_b\left(\begin{matrix}
		\Delta_d	 & x_b V_q \\
		0			 & 1
	\end{matrix}\right),
	\label{eq:YU2_5}
\eea
where $y_{\tau,t,b}$ and $x_{\tau,t,b}$ are free complex parameters expected to be of order~$\mathcal{O}(1)$. 

The spurion set in Eq.~\eqref{eq:YU2_5} is minimal in terms of 
independent $\mathrm{U}(2)^5$~structures (at least as far as the quark sector is concerned), 
and leads to spurions which are small and hierarchical in size. 
Contrary to the MFV~framework, in this case 
we cannot determine completely the spurions in terms of SM parameters. 
However, we can constrain their size requiring no tuning in the  $\cO(1)$~parameters. 
In particular, from the $2 \leftrightarrow 3$~mixing in the CKM matrix we deduce 
$|V_q| = \cO(|V_{cb}|)$, while light-quark and lepton masses imply $|\Delta_{u,d,e}|_{ij} \ll |V_{q}|$.

There are no unambiguous constraints about the size of~$V_\ell$. Actually, the SM lepton Yukawa coupling 
can be reproduced even setting~$V_\ell=0$. On the other hand, assuming 
a common structure for the three Yukawa couplings, as suggested by the similar hierarchies 
observed in the eigenvalues, it is natural to assume~$|V_\ell|\sim |V_q|$.
The assumption that $V_{q,\ell}$ are the leading 
$\mathrm{U}(2)^5$--breaking spurions ensures a suppression of 
flavor-violating terms in the quark sector, via higher-dimensional operators, as effective as the one implied by the MFV hypothesis.

It is convenient to define as reference (or interaction) basis the flavor basis 
in $\mathrm{U}(2)^5$~space where $ V_{q,\ell}=  |V_{q,\ell}| \times \vec n$,  with $\vec n ={(0,1)}^\intercal$, and $\Delta_{u,d,e}^\dagger \Delta_{u,d,e}$ 
are diagonal.
After the $\mathrm{U}(2)^5$~symmetry is broken, the residual flavor symmetry implies that the Yukawa matrices 
in the interaction basis can be written 
in the following form \cite{Fuentes-Martin:2019mun} 
\begin{subequations}%
\begin{align}
Y_u &= |y_t|
\begin{pmatrix}
U_q^\dagger O_u^\intercal\, \hat\Delta_u & |V_q|\,|x_t|\,e^{i\phi_q}\,\vec n\\
0 & 1
\end{pmatrix}
\,, \\
Y_d &=|y_b|
\begin{pmatrix}
\;\;\;U_q^\dagger \hat\Delta_d& |V_q|\,|x_b|\,e^{i\phi_q}\,\vec n\\
\;\;\;0 & 1
\end{pmatrix}
\,, \\
Y_e &=|y_\tau|
\begin{pmatrix}
\;\;O_e^\intercal\,\hat\Delta_e\;\;& |V_\ell|\,|x_\tau|\,\vec n\\
\;\;0 & 1
\end{pmatrix}
\,, 
\end{align}\label{eq:U2param}%
\end{subequations}%
where $\hat\Delta_{u,d,e}$ are $2\times 2$~diagonal positive matrices, $O_{u,e}$ are $2\times2$ orthogonal matrices, 
and $U_q$ is a complex unitary matrix. The unitary matrices that diagonalize the above Yukawa matrices 
 can be found in \cite{Fuentes-Martin:2019mun}.
After expressing the free parameters in terms of fermion masses and CKM elements, 
the residual terms which cannot be determined in terms of SM parameters are:
\begin{itemize}
\item{} {\em quark sector}:  $2\leftrightarrow 3$  mixing angle in the down sector, $s_b \approx |x_b|\,|V_q|$, and CP-violating phase $\phi_q$;
\item{} {\em lepton sector}: $2\leftrightarrow 3$ mixing angle $s_\tau \approx |x_\tau|\,|V_\ell|$ and $1\leftrightarrow 2$ mixing angle $s_e$ (which appears in $O_e$).
\end{itemize}

As pointed out in \cite{Greljo:2022cah}, the parametrization in~\eqref{eq:U2param} is redundant and all the 
(non-SM) parameters listed above can be eliminated via a suitable change of basis consistent within the $\mathrm{U}(2)^5$~framework. 
For instance, in the quark sector both~$s_b$ and~$\phi_q$ can be eliminated by a  
transformation mixing the $\mathrm{U}(2)_Q$~singlet field with the $\mathrm{U}(2)_Q$~doublet appropriately contracted with spurions. 
While this is certainly correct, this change of basis implies a shift in the tower of higher-dimensional operators. In a pure bottom-up approach, this shift has no practical consequences, hence the redundancy can safely be removed. On the other hand, keeping the redundant formulation in~\eqref{eq:U2param}
is particularly useful when matching to a specific UV theory: it highlights the fact that the third generation, with a special role in the UV, is not unambiguously determined by the SM Yukawa couplings.  

\paragraph{Higher-dimensional operators.}
The exact $\mathrm{U}(2)^5$~symmetry is the natural (and unavoidable) starting point to 
describe all processes where we can neglect light-fermion masses. 
This is why the SMEFT with unbroken~$\mathrm{U}(2)^5$ is employed to
describe  top-quark physics and related processes at colliders \cite{Aguilar-Saavedra:2018ksv}. 
The number of relevant operators is listed in Tab.~\ref{tab:U3new}.

In the same table also the terms obtained with one $V$~spurion, or two of them and one $\Delta$~spurion are shown. 
The higher-dimensional operators built in terms of a single $V_{q,\ell}$~spurion %($V_q$ or $V_\ell$) 
contribute to flavor-violating transitions which involve only left-handed fields and connect 
only the $2\leftrightarrow 3$~sectors in the interaction basis.
Considering terms with two $V_q$~spurions is the analog of considering two $Y_u$~insertions in~MFV.
Compared to the latter case, the $\mathrm{U}(2)^5$~hypothesis leads to more freedom (differentiating, for instance, 
effective operators contributing to flavor-violating process in $B$- and $K$-meson physics)
but also more terms. This latter statement can be understood by looking at the 
number of independent invariant $\bar{q}_r \gamma^{\mu} q_p$ bilinears 
in the two cases:\footnote{Here the flavor indices $\{r,p\}$ run from 1 to 3, whereas $\{i,j\}$ only between 1 and 2.}
\be
\left.
\begin{array}{l}
  	\bar{q}_r \gamma^{\mu} q_r  \\  
  	\bar{q}_r  (Y_u Y_u^\dagger)_{rp} \gamma^{\mu} q_p 
\end{array}
\right|_{\mathrm{U}(3)^5}
 	\to
\left.
\begin{array}{l}
	\bar{q}_3 \gamma^{\mu} q_3 \\  
	\bar{q}_i \gamma^{\mu} q_i  \\
	{\bar q}_i (V_q)_{i}  \gamma^\mu q_3 + \textrm{h.c.}\\  
	{\bar q}_i (V_q)_i \gamma^\mu (V_q^\dagger)_{j} q_j  
\end{array}	 
\right|_{\mathrm{U}(2)^5}
\nonumber
\ee

We stress that the hypothesis of a $\mathrm{U}(2)^5$~flavor symmetry broken by the minimal set of 
spurions in Eq.~\eqref{eq:YU2_5} naturally implies lepton flavor violation in charged leptons. This is controlled by 
the size of~$V_\ell$ and~$s_e$, which are left unconstrained by the SM Yukawa couplings.
This is one of the most evident differences between the genuine $\mathrm{U}(2)^5$~approach and the 
non-linear MFV hypothesis \cite{Feldmann:2008ja,Kagan:2009bn}. 


\subsubsection{Other options and running}
The $\mathrm{U}(2)^5$~case discussed above is the prototype of a series of 
symmetry groups providing a suppression similar to MFV in the quark sector, but allowing 
more general breaking terms. The common ground is the presence of the (chiral) non-Abelian 
group~$\mathrm{U}(2)^3$ acting in the quark sector. The variations come from obtaining this 
group as a subgroup of possible larger symmetries, such as $\mathrm{U}(2)^2\times \mathrm{U}(3)_d$
or $\mathrm{U}(2)^3\times \mathrm{U}(1)_d$ \cite{Greljo:2022cah,Faroughy:2020ina}.
Given the smaller set of phenomenological constraints,
a larger set of variations have been proposed in the lepton sector \cite{Greljo:2022cah}.

A somehow different approach is that of using only $\mathrm{U}(1)$~groups, as originally proposed 
by \textcite{Froggatt:1978nt}. Recent analyses of this type can be found in \cite{Smolkovic:2019jow,Bordone:2019uzc}.

To conclude the discussion about flavor symmetries, it is worth mentioning that the approximate symmetries present in the SM are responsible for
a series of powerful (approximate) selection rules in the renormalization group evolution of the SMEFT~\cite{Feldmann:2015nia,Machado:2022ozb}. 
These are nothing but the manifestations of the statement made in 
Sec.~\ref{sect:SymmA}
that the 
partitioning of the EFT due to global symmetries is
stable with respect to quantum corrections. 
These selection rules become manifest when working in a basis of flavor invariants, where the apparently large  anomalous dimension matrix of dimension-six current-current operators is reduced to a block-diagonal structure with several blocks of small dimension~\cite{Machado:2022ozb}. 


\subsection{Custodial symmetry}
\label{sect:custodial}
The large number of fermions in the SM implies that most of the exact or approximate  global symmetries of the theory are related to the fermion sector, as discussed so far. However, there is one important symmetry that involves mainly (but not only) the scalar sector.

Custodial symmetry is an exact symmetry of the pure Higgs sector of the~SM,
\be
\cL_H = \partial_\mu H^\dagger \partial^\mu H - V(H)\,,
\ee
with the scalar potential defined as in (\ref{eq:Hpotential}). 
The simplest way  to realize the global symmetry of~$\cL_H$ 
is to write the complex Higgs doublet in terms 
of four independent real scalar components~$\phi^i$ as in~\eqref{eq:Hdec},
% with $\phi^4=v + h$, namely 
\begin{align}
	H = \frac{1}{\sqrt{2}} 
	\begin{pmatrix}
		\phi^2 + i \phi^1 \\
		\phi^4 - i \phi^3
	\end{pmatrix}\,.
\label{eq:geoSMEFT_H_to_phi}
\end{align}
We find
\begin{align}
\begin{split}
    \L_{\phi}
	&= \frac{1}{2} \brackets{\partial_\mu \boldsymbol{\phi}} \cdot \brackets{\partial^\mu \boldsymbol{\phi}} 
	+ \frac{m^2}{2} \boldsymbol{\phi} \cdot \boldsymbol{\phi} - \frac{\lambda}{8} \left( \boldsymbol{\phi} \cdot \boldsymbol{\phi} \right)^2 \, ,
\end{split}	
\label{eq:geoSMEFT_SM_Lagrangian}
\end{align}
where we have defined
\begin{align}
	\boldsymbol{\phi}=\begin{pmatrix}
		\phi^1 \\ \phi^2 \\ \phi^3 \\ \phi^4
	\end{pmatrix} = \begin{pmatrix}
		\varphi^1 \\ \varphi^2 \\ \varphi^3 \\ v + h
	\end{pmatrix} 
\label{eq:geoSMEFT_Cartesian_coordinates}
\end{align}
with the Higgs vev~$v$, the physical Higgs bosons~$h$, and $\varphi^a$ being the Goldstone bosons of electroweak symmetry breaking.
It is easy to verify that $\cL_H$~or~$\cL_\phi$ depend only 
on $\boldsymbol{\phi}\cdot\boldsymbol{\phi} = 2H^\dagger H$ and are thus invariant under a global 
$\mathrm{O}(4)$~symmetry, with
% Defining $\smash{\boldsymbol{\phi} = \brackets{\varphi^1 \, \varphi^2 \, \varphi^3 \, \varphi^4}^\intercal}$, 
the symmetry transformation $\boldsymbol{\phi} \to O \, \boldsymbol{\phi}$ for $O \in \mathrm{O}(4)$.
The minimum of the Higgs potential is the three-sphere~$S^3$ with radius~$v$, defined by $\langle\boldsymbol{\phi}\cdot\boldsymbol{\phi} \rangle = v^2$. Hence, the $\mathrm{O}(4)$~global symmetry
of~$\cL_H$
is spontaneously broken by the Higgs~vev to its subgroup~$\mathrm{O}(3)$. The corresponding Goldstone bosons $\boldsymbol{\varphi}=\smash{\left(\varphi^1,\varphi^2,\varphi^3\right)^\intercal}$ transform under this group as $\boldsymbol{\varphi} \to \tilde{O}\,\boldsymbol{\varphi}$, where $\tilde{O} \in \mathrm{O}(3)$.

This $\mathrm{O}(3)$~global symmetry of the Higgs sector after electroweak symmetry breaking 
is responsible, among other things, for the tree-level relation~$\rho=1$, where 
\be 
\rho  \equiv \frac{m^2_W}{m_Z^2} \frac{g_1^2 + g_2^2}{g_2^2}\,.
\ee
This relation is tested to the permil level finding good 
agreement with the SM prediction, after taking into account 
the small deviations generated beyond the tree level. 
On the other hand, adding to~$\cL_H$ 
generic dimension-six operators compatible only with the gauge symmetry of the SM,
one would expect $\rho-1 = \mathcal{O}(v^2/\Lambda^2)$. 

Custodial symmetry is explicitly broken in the SM,
both by the electroweak gauge symmetry, which acts differently on the different $\phi^i$~components,
and by the Yukawa interactions. These breaking terms are responsible for the deviation from~$\rho=1$ generated beyond the tree level.  In particular, the leading contribution induced by the top Yukawa coupling reads
\be 
(\rho - 1)^{y_t}_{\rm SM} = \frac{ 3 y_t^2}{32 \pi^2}
 \approx 1\%\,.
 \ee
Given the strong constraint on the SMEFT imposed by the $\rho$~parameter, 
it is interesting to conceive the case of new physics models 
where the breaking of custodial symmetry is small as in the SM,
originating only from the gauge and the Yukawa sector. 
In other words, in close analogy to the flavor symmetries discussed above,
it is interesting to treat custodial symmetry as an approximate global symmetry 
of the SMEFT broken by a well-defined set of spurion terms. 

In order to describe the explicit breaking of custodial symmetry occurring in 
the~SM, it is more convenient to express the symmetry
in a different way, taking into account the (local) equivalence
of the $\mathrm{SO}(4)$~group with the product of two $\mathrm{SU}(2)$~groups:
\begin{align}
   \mathrm{O}(4) \simeq \SU{2}_L \otimes \SU{2}_R \, .
   \label{eq:custSU2}
\end{align}
To see how the 
$\mathrm{SU}(2)$~groups in~\eqref{eq:custSU2} act on the Higgs field, we can combine~$H$ and its conjugate, $\widetilde H = \varepsilon H^\ast$, where~$\varepsilon=i\tau^2$ is the totally anti-symmetric $\mathrm{SU}(2)$~tensor,
to form the $2\times 2$~matrix field~$\Sigma$, 
transforming as a $(\boldsymbol{2}_L,  \bar{\boldsymbol{2}}_R)$ 
under~\eqref{eq:custSU2}, namely 
\begin{align}
    \Sigma \equiv \left(\widetilde{H},H\right) \rightarrow V_L \, \Sigma \, V_R^\dagger
    \label{eq:Higgs-matrix-field}
\end{align}
with $V_{L(R)} \in \mathrm{SU}(2)_{L(R)}$.
We find $\smash{\mathrm{tr}\left[\Sigma^\dagger \Sigma\right]}=\smash{2\,H^\dagger \! H}=\smash{\boldsymbol{\phi}\cdot\boldsymbol{\phi}}$ allowing us to write the Higgs Lagrangian~$\mathcal{L}_H$ as 
\begin{align}
    \mathcal{L}_\Sigma
    &= 
    \frac{1}{2} \mathrm{tr} \left[ \left(\partial_\mu\Sigma\right)^\dagger \! \left(\partial^\mu\Sigma\right) \right] 
    +\frac{m^2}{2}\mathrm{tr}\left[\Sigma^\dagger \Sigma\right]
    -\frac{\lambda}{8} \left(\mathrm{tr}\left[\Sigma^\dagger \Sigma\right]\right)^2 \!\! .
    % V(\Sigma)=-\frac{m^2}{2}\mathrm{tr}\left(\Sigma^\dagger \Sigma\right)+\frac{\lambda}{8} \mathrm{tr}\left(\Sigma^\dagger \Sigma\right)^2 \,.
    \label{eq:L-linear-sigma-model}
\end{align}
With this notation it is easy to verify that $\cL_H$~or~$\cL_\Sigma$ is invariant 
under $\SU{2}_L \otimes \SU{2}_R$~global transformations, and that the spontaneous symmetry breaking due to the Higgs vev corresponds to $\mathrm{O}(4) \simeq \mathrm{SU}(2)_L \otimes \mathrm{SU}(2)_R \to \mathrm{SU}(2)_{L+R} \simeq \mathrm{O}(3)$.
While the $\mathrm{SU}(2)_L$~group is fully gauged in the SM, only a part of the $\SU{2}_R$~group is gauged, 
leading to an explicit breaking of custodial symmetry. More precisely, gauging hypercharge in the Higgs sector 
is equivalent to gauging only the $\mathrm{U}(1)$~subgroup of~$\SU{2}_R$ corresponding to the diagonal generator~$T^3_R$.

Up to this level, i.e., when considering only the gauge sector, the identification of the explicit breaking of 
custodial symmetry from a general EFT point of view is unambiguous.
An ambiguity arises when considering also the fermion sector, given the action of~$T^3_R$ is not sufficient to 
describe fermion hypercharges. On general grounds, we can extend the symmetry to \cite{Elias-Miro:2013mua}
\be
\cG_{\rm cust} = \mathrm{SU}(2)_L \otimes \SU{2}_R \otimes \mathrm{U}(1)_X\,,
\ee 
such that hypercharge reads
\be 
Y = T^3_R +X\,.
\ee 
However, different embeddings of the SM fermions in~$\cG_{\rm cust}$ are possible. The simplest one corresponds to the choice
$X=(B-L)/2$. In such case, all right-handed fermions belong to doublets of~$\SU{2}_R$, with an incomplete 
doublet in the lepton sector due to the absence of right-handed neutrinos, while all left-handed fermions are assumed to be singlets of~$\SU{2}_R$. But other options are also possible \cite{Elias-Miro:2013mua}.

Once a representation of the SM fermions under~$\cG_{\rm cust}$ is chosen, we have all the ingredients to define 
a consistent EFT based on the hypothesis of minimal breaking of custodial symmetry, 
able to reproduce all SM properties.  
First, all representations of~$\cG_{\rm cust}$ including some SM fermions are promoted to be complete representations 
by introducing appropriate spurions (unphysical) fields which are set to zero in physical processes \cite{Elias-Miro:2013mua}.
Second, $\SU{2}_R$~breaking terms in the Yukawa couplings, such as the one responsible for the 
top-bottom splitting when $t_R$~and~$b_R$ are embedded in the same $\SU{2}_R$~multiplet,
are also promoted to be spurion fields \cite{Isidori:2009ww}. Finally, also spurion gauge bosons are introduced, 
so that the whole group~$\cG_{\rm cust}$ is formally gauged, and the SM is recovered as the limit obtained 
setting the spurion fields to zero \cite{Gonzalez-Alonso:2014eva}.

The consequences of these hypotheses for various subsets of
SMEFT operators (or physical amplitudes evaluated at~$d=6$ 
in the SMEFT) have been discussed in \cite{Contino:2013kra,Elias-Miro:2013mua,Gonzalez-Alonso:2014eva}. 






