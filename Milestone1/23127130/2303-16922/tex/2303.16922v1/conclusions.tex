\section{Conclusion}
The Standard Model has set a natural and successful framework for the qualitative and quantitative understanding of the elementary particles and their interactions. It has been possible to calculate its predictions with enormous precision, allowing comparison with a similar progress on the experimental side. 
On the other hand, as already stated in the introduction, there is a number of observational and theoretical issues with the SM, such as neutrino masses, baryon asymmetry, a natural bridge to gravity, and the instability of the Higgs quadratic term. 
This is why it is widely believed, and we share this point of view, that the SM is the remnant of a more complete theory with new degrees of freedom showing up at some higher energy scale. By this statement we do not imply there cannot be also other light states beyond the SM ones, but rather that the SM fields are embedded into a more complete QFT in the UV, addressing many of the currently open issues.

The outstanding agreement between experiment and theory, that in various cases reach the sub-percent level, suggests that the scale where new heavy particles will appear, and the SM will manifestly become an incomplete description of nature, is well above the electroweak scale. This fact does not prevent the observation of effects related to the new degrees of freedom in current and near-future experiments. However, these effects will be indirect manifestation of new physics, and their interpretation in terms of hypothetical new dynamics require a suitable effective theory approach.

In this article we have reviewed the EFT approach to physics beyond the SM, focusing in particular on the linear realization of the mechanism of electroweak symmetry breaking, 
i.e., the SMEFT. Given all measurements of the 125\,GeV scalar particle discovered at the LHC are consistent with the properties expected for the SM Higgs boson, the SMEFT emerges as most natural EFT approach to physics beyond the~SM. In Sec.~\ref{sect:SMEFT} we extensively reviewed the construction of the basis of effective operators, the power counting, and various other technical aspects of this EFT. In Sec.~\ref{sec:HEFT} we also illustrated the more general approach represented by the~HEFT, or the possibility of a non-linear realization of the mechanism of electroweak symmetry breaking. An option that, despite being not favored by current data, cannot be excluded at present. 

An important role in effective field theories is played by exact and approximate symmetries
emerging in the low-energy limit of the theory, the so-called accidental symmetries.
We extensively reviewed this aspect in Sec.~\ref{sec:GlobalSymmetries}, focusing in particular on flavor symmetries, which represent the vast majority of possible global symmetries in the SMEFT. As we argued, in the absence of flavor symmetries the SMEFT approach is not particularly useful: severe bounds from flavor-violating observables would imply a very high scale of new physics, rendering the whole construction not particularly appealing. 
On the other hand, with the help of motivated hypotheses about symmetry and symmetry-breaking, 
resulting from general dynamical hypothesis in the~UV, it is possible to consistently reduce the bounds on the new-physics scales and provide an {\em a~posteriori} justification for the observed mass hierarchies. 
In this motivated limit, we can both 
reduce the number of free parameters of the SMEFT, and
combine information from flavor-changing and flavor-violating processes. 


In Sec.~\ref{sect:LEFT} and, especially, in Sec.~\ref{sect:practical} we have reviewed in detail the techniques used to put the SMEFT at work in analyzing data and possibly extracting information about physics beyond the~SM. These involve a large array of theoretical concepts and methods developed over the last decades, which we bring together here. From the use of low-energy effective theories valid below the electroweak scale, to the running of the SMEFT, and finally the 
matching to explicit beyond-the-SM theories. We reviewed various technical aspects of this workflow, both in a bottom-up perspective as well as in top-down approaches. 

The SMEFT is already a mature subject and many studies exist in the literature, 
including excellent reviews such as the one by \textcite{Brivio:2017vri}.
However, most of the existing studies are focused mainly on the use of this tool in setting bounds on possible new physics scenarios. In this review, we emphasized the advantage of using the SMEFT 
in case of a “positive” signal of new physics.  While new-physics bounds can efficiently be set, in many cases, directly at the level of the observables, the full power of the EFT approach manifests itself in presence of a new-physics signal. In this case the SMEFT, being a consistent QFT, allows us to connect a signal in one observable to those in other processes and possibly recognise the underlying origin of the new dynamics.
We illustrated this chain via two specific examples in Sec.~\ref{sect:practical}, inspired by `anomalies' (i.e.~deviations from the SM predictions) present in current data: the $(g-2)_\mu$ anomaly and the deviation from lepton-flavor universality in $b\to c\ell\nu$ decays. While none of these effects is statistically compelling, we have analyzed them in detail since they provide a very clear and rather general illustration of the power of the EFT approach. 

The applicability of the SMEFT rests on the validity of the effective theory approach. 
This itself relies on the hypothesis of having identified all degrees of freedom and symmetries
relevant at low energies. In this respect, the wide class of SM extensions with light new degrees of freedom, such as axions or axion-like particles, is not covered by the SMEFT. Their inclusion is conceptually simple once the properties of the new fields are specified [see e.g. \cite{Bauer:2020jbp,Agrawal:2021dbo,Galda:2021hbr}], but is clearly more model dependent and something that goes beyond the scope of this review. More generally, EFT approaches are based on the concept of scale separation, a key paradigm which guided the progress in particle physics for several decades by now. The absence of TeV-scale new physics, as expected by na\"ive EFT considerations, has recently stimulated to  consider alternatives to this paradigm \cite{Giudice:2017pzm}. 
While this is certainly an interesting possibility, we believe that our knowledge of TeV-scale physics is still far from being complete. The possibility of new physics just around the corner of the current energy and precision frontiers remains an extremely motivated option, and the SMEFT represents the most suitable tool to analyze it. 


