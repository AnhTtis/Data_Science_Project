\documentclass{article}



\usepackage{arxiv}

\usepackage[utf8]{inputenc} % allow utf-8 input
\usepackage[T1]{fontenc}    % use 8-bit T1 fonts
\usepackage{hyperref}       % hyperlinks
\usepackage{url}            % simple URL typesetting
\usepackage{booktabs}       % professional-quality tables
\usepackage{amsfonts}       % blackboard math symbols
\usepackage{nicefrac}       % compact symbols for 1/2, etc.
\usepackage{microtype}      % microtypography
\usepackage{lipsum}		% Can be removed after putting your text content
\usepackage{graphicx}
\usepackage{natbib}
\usepackage{doi}

\usepackage{blindtext}
\usepackage{color}
\usepackage{graphicx}
\usepackage{amsthm}
\usepackage{mathtools}
\usepackage{amsfonts}
\usepackage{amssymb,amsmath}
\usepackage{mathtools}
\usepackage{subcaption}
\usepackage{caption}
\usepackage{amsmath}
\usepackage{tcolorbox}
\usepackage{titlesec}
\usepackage{stackengine}
\usepackage{caption}
\usepackage{algorithm}
\usepackage{algpseudocode}
\usepackage{bigints}
\newtheorem{theorem}{Theorem}
\newtheorem{corollary}{Corollary}[theorem]
\newtheorem{prop}{Proposition}
\newtheorem{definition}{Definition}
\newtheorem{problem}{Problem}
\newtheorem{remark}{Remark}
\newtheorem{example}{Example} 
\newtheorem{lemma}[theorem]{Lemma}
\newtheorem{SampleEnv}{Sample Environment}[section]
\usepackage{tikz}
\usepackage{calrsfs}
\DeclareMathAlphabet{\pazocal}{OMS}{zplm}{m}{n}
\DeclarePairedDelimiter\ceil{\lceil}{\rceil}
\DeclarePairedDelimiter\floor{\lfloor}{\rfloor}
\usepackage{thmtools, thm-restate}
\newtheorem{conjecture}[theorem]{Conjecture}

\title{Externalities in Chore Division}

%\date{September 9, 1985}	% Here you can change the date presented in the paper title
%\date{} 					% Or removing it

\author{ \href{https://orcid.org/0000-0001-5030-9645}{\includegraphics[scale=0.06]{orcid.pdf}\hspace{1mm}Mohammad Azharuddin Sanpui} \\
	Department of Mathematics\\
	Indian Institute of Technology Kharagpur\\
	 India\\
	\texttt{azharuddinsanpui123@gmail.com} \\
	%% examples of more authors
	%% \AND
	%% Coauthor \\
	%% Affiliation \\
	%% Address \\
	%% \texttt{email} \\
	%% \And
	%% Coauthor \\
	%% Affiliation \\
	%% Address \\
	%% \texttt{email} \\
	%% \And
	%% Coauthor \\
	%% Affiliation \\
	%% Address \\
	%% \texttt{email} \\
}

% Uncomment to remove the date
\date{}

% Uncomment to override  the `A preprint' in the header
%\renewcommand{\headeright}{}
%\renewcommand{\undertitle}{}
%\renewcommand{\shorttitle}{}

%%% Add PDF metadata to help others organize their library
%%% Once the PDF is generated, you can check the metadata with
%%% $ pdfinfo template.pdf
\hypersetup{
pdftitle={A template for the arxiv style},
pdfsubject={q-bio.NC, q-bio.QM},
pdfauthor={David S.~Hippocampus, Elias D.~Striatum},
pdfkeywords={First keyword, Second keyword, More},
}

\begin{document}
\maketitle

\begin{abstract}
The chore division problem simulates the fair division of a heterogeneous undesirable resource among several agents. In the fair division problem, each agent only gains value from its own piece. Agents may, however, also be concerned with the pieces given to other agents; these externalities naturally appear in fair division situations. Branzei et ai. (Branzei et al., IJCAI 2013) generalize the classical ideas of proportionality and envy-freeness while extending the classical model to account for externalities. Our results clarify the relationship between these expanded fairness conceptions, evaluate whether fair allocations exist, and examine whether fairness in the face of externalities is computationally feasible. In addition, we identify the most efficient allocations among all the allocations that are proportional and swap envy-free. We present tractable methods to achieve this under different assumptions about agents' preferences..
\end{abstract}

% keywords can be removed
\keywords{Cake Cutting \and Game Theory \and Economics}
\section{Introduction}
The problem of allocating a heterogeneous, divisible resource among a set of $n$ agents with varying preferences is essentially described as the problem of cutting a cake in intuitive concepts. The cake cutting problem is a fundamental topic in the theory of fair division  \cite{brams1996fair,Cake,moulin2004fair,brandt2016handbook} and it has received a significant amount of attention in the domains of mathematics, economics, political science, and computer science \cite{TheEfficiencyoffairdivision,cakecuttingreallyisnotapieceofcake,feeiciencyoffairdivisionwithconnectedpiece,ChildrenCryingatbirthdayparties,cakecuttingnotjustchildplay,TheQuerycomplexityofcakecutting}. Dividing a cake fairly among agents is a challenging task.
\textit{Envy-freeness} and \textit{proportionality} are the most important criteria of a fair allocation in the cake-cutting literature. In an envy-free allocation, every agent is pleased with the pieces they are allocated as opposed to any other agent's allocation. In a proportional allocation, each agent receives at least $\frac{1}{n}$ of the value he estimates to the cake. When all of the cake has been divided, envy-freeness entails proportionality. It is generally known that envy-free allocations always exists \cite{ANENVY-FREECAKEDIVISIONPROTOCOL} and even if we specify that each agent must receive a connected piece \cite{rental,Stromquist1980HowTC}. In addition to the existence, the algorithmic design aspect of the process has also been thought about for a long time \cite{ADiscreteandBoundedEnvy-FreeCakeCuttingProtocolforAnyNumberofAgents,Aboundedandenvy-freecakecuttingalgorithm,Adiscreteandboundedenvyfreecakecuttingprotocolforfouragents,Howtocutacakefairly,ANoteOnCakeCutting,Stromquist2008EnvyFreeCD}. For any number of agents, we are able to calculate a proportional allocation \cite{Howtocutacakefairly,ANoteOnCakeCutting} as well as an envy-free allocation \cite{ADiscreteandBoundedEnvy-FreeCakeCuttingProtocolforAnyNumberofAgents}. 

The two concepts of fairness are obviously fundamentally distinct upon closer examination. The concept of envy-freeness itself assumes that agents compare their own allocations to those of others, while proportionality just needs each agent to assess the quality of their own allocation in comparison to their best possible allocation.
This first assumption, which comes from psychological research, argues that agents are influenced both by their own piece of cake and the pieces of other agents. This type of influences is classified as "externalities". These externalities may be either positive or negative.

There are several reasons for including externalities into consideration in an allocation problem. Externalities are usually prices or extra advantages that aren't passed on to customers through pricing and may be paid for by someone outside of the transaction. A scenario like this exists often, primarily while allocating resources. When agents share resources, an agent may be able to use resources given to a friend or family member because the agent has the right to use the resource. There are several merit commodities that provide positive demand externalities. In the medical field, vaccinations have positive externalities since they reduce the danger of infection for others around them. In various settings for resource allocation, negative externalities may also occur. For instance, when allocating resources to opposing groups, allocating a crucial resource to another group could make it more difficult for one group to flourish. Another example of a negative externality is the fact that when scheduling advertising time slots, Coca-Cola hurts if Pepsi gets the better slots.

 
Velez \cite{Velez2016FairnessAE} first addresses the study of externalities in fair division of indivisible goods and extends the notion of envy-freeness: \textit{swap envy-freeness} in which an agent cannot enhance by swapping its allocation with that of another agent. Branzei et al. \cite{Brnzei2013ExternalitiesIC} initiate the idea of externalities in fair division of heterogeneous divisible resources (or cakes) and extend the fairness notions: proportionality and swap stability. Swap stability requires that no agent can benefit by swapping the allocations between any pair of agents. Proportionality requires that each agent get the value of at least $1 \slash n$ in the comparison with her best possible allocation.   


In contrast, the dual problem of cake-cutting, also known as chore division, seeks to allocate an undesirable resource to a set of $n$ agents, with each agent wishing to receive as little of the resource as possible. Chore division might model the allocation of chores within a household, liabilities in a bankrupt company, etc. Similar to the cake-cutting problem, dividing a chore fairly is also a challenging problem. The most important criteria of a fair allocation, similar to cake cutting, are \textit{envy-freeness} and \textit{proportionality}. For most questions in cake cutting, there exist parallel questions in chore division; for instance, an $n$-person envy-free cake cutting algorithm, in which each agent is satisfied that no other agent has received a bigger piece in their estimation, was found by Brams and Taylor \cite{ANENVY-FREECAKEDIVISIONPROTOCOL}, while the equivalent chore-cutting result was found by Peterson and Su \cite{Npersonchoredivision}. We know how to compute a proportional allocation ( similar to the cake cutting) and an envy-free allocation for any number of agents \cite{envy-freechoredivisionforanyagents} . Though several algorithms for cake cutting also apply to chore division, the theoretical properties of the two problems differ in many cases, and much less work has been done for chore division than for cake cutting \cite{AlgorithmicSolutionsforEnvy-FreeCakeCutting,AlgorithmsforCdivisionofchores,complexitychoredivisio,DividingConnectedChoresfairly,peterson1998exact,Chaudhury2020DividingBI}.\newline
The problem of chore division with externalities, commonly referred to as "negative externalities," is the inverse of the problem of cake-cutting with externalities.
Our aim throughout the entire paper is to study chore division with externalities using the extended model of cake-cutting with externalities presented by Branzei et al. \cite{Brnzei2013ExternalitiesIC}. 
\subsection{Related Works}
Externality theory is researched extensively in economics \cite{Ayres1969ProductionCA,Katz1985NetworkEC}, but recent time it have also been getting more attention from researchers of computer science. Several research include the study of externalities in voting \cite{Alon2012SequentialVW}, auctions \cite{Krysta2010CombinatorialAW,Haghpanah2011OptimalAW}, coalitional games \cite{Michalak2009OnRC}, and matchings \cite{Brnzei2013MatchingsWE}. Velez addresses externalities in fair division of indivisible items and money (e.g., tasks and salary) \cite{Velez2016FairnessAE}. He independently proposes swap envy-freeness, where no agent may improve their allocation by swapping it with another. His remarkable discoveries can be applied to cake cutting, however the results are limited. Branzei et al. first present a generalized cake-cutting model with externalities \cite{Brnzei2013ExternalitiesIC}. They introduce the idea of swap stable, which is a more robust version of swap envy-freeness and asserts that no agent can profit from a swap involving any two agents. Several AI articles explore the fair division of indivisible items with externalities \cite{ghodsi2018fair,mishra2022fair,seddighin2019externalities,aziz2021fairness}.
\subsection{Our Results}
Throughout the entire paper, we focus on the analysis of the chore division problem with externalities. This is the mirror image of the cake-cutting problem with externalities that Branzei et al. \cite{Brnzei2013ExternalitiesIC} proposed. Our focus is on three extended fairness notions: \textit{proportionality}, \textit{swap envy-freeness}, and \textit{swap stability}. Proportionality among $n$ agents requires that each agent needs to obtain at most 1/n of the value she gains based on the worst possible allocation from her point of view. Swap envy-freeness requires that no agent can make their allocation better by swapping it with another agent’s. Swap stability requires that no agent can benefit from swapping the allocations between any two agents. In Section $3$, we demonstrate the relationship between fairness properties. In Section $4$, we display the lower and upper bounds of the required number of cuts to get various fair allocations. In Section $5$, we affirm that the generalized Robertson-Webb communication model cannot provide a finite proportional protocol even for two agents. In Section $6$, we try to obtain the most efficient proportional and swap envy-free allocations. We describe tractable methods for just doing so when the agents' preferences are piecewise constant valuations.

\section{Preliminaries}
In intuitive terms, the challenge of allocating a heterogeneous, divisible resource among n agents with each agent influencing other agents' allocations is known as the "problem of cake-cutting with externalities." Branzei et al. \cite{Brnzei2013ExternalitiesIC} first proposed the general model for cake cutting with externalities. On the other hand, the dual problem of cake-cutting with externalities, also known as chore division with externalities, tries to divide an unwanted resource among n agents in a way that takes into account how each agent's allocation affects the allocations of the other agents. 

 We consider a setting where a heterogeneous undesirable resource ( or chore) is represented by the interval $[0,1]$ is to be allocated among a set $N=[n]=\{1,2,.....,n\}$ of agents under externalities. Each agent $i$ is endowed with $n$ integrable, non-negative value density functions $v_{i,j}:[0,1]\rightarrow \mathbb{R}_{\geq 0}$ which captures how the agent $i$ influences on different parts of the chore for the agent $j$. By $v_{i,j}(x)$ we denote the value that agent $i$ obtains when $x$ is allocated to agent $j$. A piece of chore $X$ is a countable union of disjoint subintervals of $[0,1]$. The value that agent $i$ achieves from a piece $X$ that is assigned to agent $j$ is $V_{i,j}(X)=\int_{X}v_{i,j}(x)dx$. Defined in this manner, agent valuations are additive, i.e. $V_{i,j}(X\cup Y ) = V_{i,j}(X) + V_{i,j}(Y)$ if $X$ and $Y$ are disjoint, and non-atomic, i.e., $V_{i,j}([x,x]) = 0$. Because of the latter property, we can treat open and closed intervals as
equivalent. In the classical model of cake cutting, $V_{i,j}(X)=0$ for all pieces X and $i \neq j$. 

An allocation $A = (A_1,A_2,.....,A_n)$ is an assignment of a piece of cake $A_i$ to each agent $i$, such that the pieces are disjoint and $\bigcup_{i\in N} A_i\subseteq [0,1]$.  Moreover, each piece $A_i$ is a possibly infinite set of disjoint subintervals of $[0,1]$. The value of agent $i$ under allocation $A$ is $V_i(A)=\sum_{j=1}^n V_{i,j}(A_i)$.\newline
We make the assumption that the sum of each agent's multiple valuations should be normalized to 1 i.e, $\sum_{j=1}^n V_{i,j}([0,1])=1$ for all $i$. In the classical model of chore-division, $V_{i,j}([0,1])=0$ for all $i\neq j$ and then the assumption becomes $V_{i,i}([0,1])=1$ for all $i$, this represents the chore division problem. Moreover, this may happen that for each agent $i$, $V_{i}(\Tilde{A}_i)=1$ where $\Tilde{A}_i$ is the worst possible allocation for agent $i$.\newline
\textbf{Example 1.} Suppose there are $n$ agents. Consider the value density functions: $v_{i,i}(x)=1, \forall x\in [\frac{i-1}{n},\frac{i}{n}]$ and $v_{i,j}(x)=1, \forall x\in [\frac{j-1}{n},\frac{j}{n}]$ and for all $i\neq j$. Then $A=(A_1,A_2,......,A_n)$ where $A_i=[\frac{i-1}{n},\frac{i}{n}]$ for all $i$, is the allocation such that $V_i(A)=1$. Thus $A$ is the worst allocation for every agent $i$.



\subsection{Fairness notions}
\textit{Envy-freeness} and \textit{proportionality} are the most important fairness criteria in the theory of fair division. Branzei et al. \cite{Brnzei2013ExternalitiesIC} extened the definition of Proportionality in the following way. 
\\
\textbf{Definition 1 (Proportionality).} An allocation $A=(A_1,A_2,....,A_n)$ is said to be \textit{proportional} if for every agent $i\in N$, $V_i(A)\leq \frac{1}{n}$ i.e., $\sum_{j=1}^n V_{i,j}(A_i)\leq \frac{1}{n}$.
\vspace{.2cm}

{Clearly state that each agent needs to obtain at most $1/n$ of the value she gains based on the worst possible allocation from her point of view. This definition addresses and guarantees the classical definition: when there are no externalities, each agent receives at most her average share of the entire chore.}

Velez \cite{Velez2016FairnessAE} defines the notion of swap envy-freeness in the fair division of indivisible goods and money with externalities, in which each agent has no interest to exchange her allocation with that of another agent. 
\\
\textbf{Definition 2 (Swap Envy-freeness).} An allocation $A=(A_1,A_2,......,A_n)$ is \textit{swap envy-free} if for any two players $i,j\in N $, $V_{i,i}(A_i)+V_{i,j}(A_j)\leq V_{i,i}(A_j)+V_{i,j}(A_i)$.


{In other words, no agent can make their allocation better by swapping it with another agent's. This definition directly addresses and indicates the classical definition of envy-freeness when there is no externalities. }

Branzei et al. introduce swap stability \cite{Brnzei2013ExternalitiesIC}, which is a stronger version of swap envy-freeness in which no agent can benefit by swapping the allocations between any two agents.
\\
\textbf{Definition 3 (Swap Stability).} An allocation $A=(A_1,A_2,......,A_n)$ is \textit{swap stable} if for any three players $i,j,k\in N $, $V_{i,j}(A_j)+V_{i,k}(A_k)\leq V_{i,j}(A_k)+V_{i,k}(A_j)$.
\vspace{0.2cm}

{Note that the definition of swap stability implies swap envy-freeness.}

\section{Relationship Between Fairness Properties}
The division of chores is an example of a fair division problem in which the resource being divided is undesired and each player desires the least amount possible. It is the inverse of the cake-cutting problem, in which the shared resource is desired and each player wants to get the maximum amount. In the general model of cake-cutting with externalities proposed by Branzei et al. \cite{Brnzei2013ExternalitiesIC}, swap stability necessitates swap envy-freeness and guarantees proportionality when the whole chore is allotted. In contrast, using this model for chore division with externalities, we show the relationship among proportionality, swap envy-freeness and swap stability.\newline
\textbf{Proposition 1.} Swap envy-freeness guarantees proportionality when $n=2$.\newline
\textit{Proof.} Assume that $A=(A_1,A_2)$ be any swap envy-free allocation. By the definition of swap envy-freeness we have
\begin{equation}
    V_{i,i}(A_i)+V_{i,3-i}(A_{3-i})\leq  V_{i,i}(A_{3-i})+V_{i,3-i}(A_i)
\end{equation}
The inequality (1) can be rewritten as
\begin{equation*}
\begin{split}
    2 \left(V_{i,i}(A_i)+V_{i,3-i}(A_{3-i})\right)&\leq  V_{i,i}(A_{3-i})+V_{i,3-i}(A_i)\\& +V_{i,i}(A_i)+V_{i,3-i}(A_{3-i})
\end{split}
\end{equation*}
equivalently,
\begin{equation*}
    2 \left(V_{i,i}(A_i)+V_{i,3-i}(A_{3-i})\right)\leq V_{i,i}([0,1])+V_{i,3-i}([0,1])
\end{equation*}
equivalently,
\begin{equation*}
    V_i(A)=V_{i,i}(A_i)+V_{i,3-i}(A_{3-i})\leq \frac{1}{2}
\end{equation*}

Thus, swap envy-freeness always guarantees the existence of proportionality when $n=2$.\qed
\newline
But the converse may not be true: proportionality does not ensure envy-freeness. Indeed, the following example provides a proportional but not swap envy-free allocation.
\newline
\textbf{Example 2.} \textit{Consider the value density functions: $v_{1,1}(x)=\frac{3}{4},\forall x\in [0,\frac{1}{2}]$ and $v_{1,1}(x)=\frac{1}{4},\forall x\in [\frac{1}{2},1]$; $v_{1,2}(x)=\frac{1}{2},\forall x\in [0,1]$; $v_{2,1}(x)=\frac{1}{2},\forall x\in [0,1]$; $v_{2,2}(x)=\frac{1}{4},\forall x\in [0,\frac{1}{2}]$ and $v_{2,2}(x)=\frac{3}{4}, \forall x\in [\frac{1}{2},1]$. The allocation $A=(A_1,A_2)$, where $A_1=[0,\frac{1}{2}]$ and $A_2=[\frac{1}{2},1]$ is proportional, but not swap envy-free, since both agents can minimize their total valuation by swapping their pieces among them.}
\newline
Moreover, swap envy-freeness does not imply proportionality when $n>2$, as the following example shows.\\
\textbf{Example 3.} \textit{Suppose there are $3$ agents. Consider the value density functions: $v_{1,1}(x)=1,\forall x\in [0,\frac{1}{3}]$; $v_{1,2}(x)=1, \forall x \in [\frac{1}{3},\frac{2}{3}]$; $v_{1,3}(x)=1, \forall x \in [\frac{2}{3},1]$; $v_{2,1}(x)=\frac{1}{2},\forall x\in [0,1]$; $v_{2,2}(x)=0,\forall x\in [0,1]$; $v_{2,3}(x)=\frac{1}{2},\forall x\in [0,1]$; $v_{3,1}(x)=v_{3,2}(x)=v_{3,3}(x)=1,\forall x\in [0,1]$. The allocation $A=(A_1,A_2,A_3)$ where $A_1=[0,\frac{1}{3}]$, $A_2=[\frac{1}{3},\frac{2}{3}]$ and $A_3=[\frac{2}{3},1]$ is swap envy-free but not proportional. Because agent $1$ receives utility $V_1(A)=V_{1,1}(A_1)+V_{1,2}(A_2)+V_{1,3}(A_3)=\frac{1}{3}+\frac{1}{3}+\frac{1}{3}=1$.}
\vspace{0.2cm}

{Branzei et al. \cite{Brnzei2013ExternalitiesIC} have shown that swap stability implies proportionality when the entire cake is allocated. Now we demonstrate that the same result occurs even when the whole chore is not allotted.}
\newline
\textbf{Theorem 1.} \textit{Swap stability implies proportionality even when the allocation is not complete.}\\
\textit{Proof.} Suppose that $A=(A_1,A_2,.....,A_n)$ be any swap stable allocation. Due to the definition of swap stability, we have that for any three agents $i,j,k\in N$,
\begin{equation}
    V_{i,j}(A_j)+V_{i,k}(A_k)\leq V_{i,j}(A_k)+V_{i,k}(A_j)
\end{equation}
Summing inequality $(2)$ over all $k\in N$, we have:
\begin{equation*}
   \sum\limits_{k=1}^n V_{i,j}(A_j)+\sum\limits_{k=1}^n V_{i,k}(A_k)\leq \sum\limits_{k=1}^n V_{i,j}(A_k)+\sum\limits_{k=1}^n V_{i,k}(A_j)
\end{equation*}
Because of $V_i(A)=\sum\limits_{k=1}^n V_{i,k}(A_k)$ and $\bigcup_{k=1}^n A_k \subseteq [0,1]$, we get from the above inequality:
\begin{equation}
     n V_{i,j}(A_j)+V_i(A)\leq V_{i,j}([0,1])+\sum\limits_{k=1}^n V_{i,k}(A_j)
\end{equation}
Again summing inequality $(3)$ over all $j\in N$, we obtain:
\begin{equation*}
    n \sum\limits_{j=1}^n V_{i,j}(A_j)+\sum\limits_{j=1}^n V_i(A) \leq \sum\limits_{j=1}^n V_{i,j}([0,1])+\sum\limits_{j=1}^n \sum\limits_{k=1}^n V_{i,k}(A_j)
\end{equation*}
 equivalently,
\begin{equation*}
    n V_{i}(A)+n V_i(A) \leq \sum\limits_{j=1}^n V_{i,j}([0,1])+\sum\limits_{k=1}^n \sum\limits_{j=1}^n V_{i,k}(A_j)
\end{equation*}
 equivalently,
\begin{equation*}
    2n V_{i}(A)\leq \sum\limits_{j=1}^n V_{i,j}([0,1])+\sum\limits_{k=1}^n V_{i,k}([0,1])
\end{equation*}

Since $\sum\limits_{j\in N} V_{i,j}([0,1])=1$, we have
\begin{equation*}
    2n V_{i}(A)\leq \sum\limits_{j=1}^n V_{i,j}([0,1])+\sum\limits_{k=1}^n V_{i,k}([0,1])=1+1=2
\end{equation*}
Thus $V_i(A)\leq \frac{1}{n}$ for any $i\in N$. So $A=(A_1,A_2,....,A_n)$ is proportional.\qed
\vspace{0.3cm}

{Since swap stability follows swap envy-freeness by definition. In contrast, in the next example we show that proportionality and swap envy-freeness are not sufficient to guarantee swap stability.}
\newline
\textbf{Example 4.} \textit{ Suppose there $3$ agents. Consider the value density functions: $v_{1,1}(x)=\frac{1}{3},\forall x\in [0,1]$; $v_{1,2}(x)=\frac{1}{3},\forall x\in [0,\frac{1}{3}]$ and $v_{1,2}(x)=\frac{2}{3},\forall x\in [\frac{1}{3},\frac{2}{3}]$; $v_{1,3}(x)=\frac{1}{3},\forall x\in [0,\frac{1}{3}]$ and $v_{1,3}(x)=\frac{2}{3}, \forall x\in [\frac{2}{3},1]$; $v_{2,1}(x)=v_{2,2}(x)=v_{2,3}(x)=\frac{1}{3}, \forall x\in [0,1]$; $v_{3,1}(x)=v_{3,2}(x)=\frac{1}{2},\forall x\in [0,1]$; $V_{3,3}(x)=0,\forall x\in [0,1]$. Consider the allocation $A=(A_1,A_2,A_3)$ where $A_1=[0,\frac{1}{3}]$, $A_2=[\frac{1}{3},\frac{2}{3}]$ and $[\frac{2}{3},1]$. Each agent receives a value exactly $\frac{1}{3}$ and no agent can minimize their utility by swapping his allocation with that of another agent. So the allocation $A=(A_1,A_2,A_3)$ is proportional as well as swap envy-free but not swap stable, since agent $1$ would like to swap the pieces between agents $2$ and $3$, which minimize his utility to $\frac{1}{9}$ (compared to $\frac{1}{3}$ under $A$).}


\section{Existence of Fair Allocations}
In the classical model, when there are two agents, an envy-free (and therefore proportional) allocation is possible if the cake is cut into two pieces that one agent values equally and the other agent chooses its favorite piece. In the presence of externalities, the equivalent result holds. Theorem $2$ leads to the conclusion. Branzei et al. \cite{Brnzei2013ExternalitiesIC} show that there exists a proportional and swap envy-free allocation that requires a single cut when $n=2$. Now we present that the same result occurs in a chore division with externalities. 
\newline
\textbf{Theorem 2.} \textit{Let n = 2. Then there is an allocation that is proportional, swap envy-free, and therefore only needs one cut.}\newline
\textit{Proof.} The scenario of the proof is to find a point $\Tilde{y}\in [0,1]$ such that $[0,\Tilde{y}]$ and  $[\Tilde{y},1]$ is a proportional and swap envy-free allocation.\newline
Now define a function $\mathcal{F}:[0,1]\rightarrow{\mathbb{R}}$ such that for all $x\in [0,1]$:
\begin{equation*}
    \mathcal{F}(x)=V_{2,1}([0,x])+V_{2,2}([x,1])-V_{2,1}([x,1])-V_{2,2}([0,x]).
\end{equation*}
It is clear that $\mathcal{F}$ is a continuous function and the values of $\mathcal{F}$ at the points $0$ and $1$ are
$\mathcal{F}(0)=V_{2,2}([0,1])-V_{2,1}([0,1])$ and 
$\mathcal{F}(1)=V_{2,1}([0,1])-V_{2,2}([0,1])$ respectively.
Thus we get $\mathcal{F}(0)+\mathcal{F}(1)=0$. So by the intermediate value theorem, there exists a point $\Tilde{y}\in [0,1]$ such that $\mathcal{F}(\Tilde{y})=0$.

Now we show that the allocation in which agent $1$ chooses his most preferred piece among $\{[0,\Tilde{y}],[\Tilde{y},1]\}$ and allocating the remaining piece to agent $2$, is proportional and swap envy-free.
\newline
Suppose that agent $1$ selects a piece from $\{[0,\Tilde{y}].[\Tilde{y},1]\}$ that is, $[\Tilde{y},1]$. Then the corresponding allocation is $A=(A_1,A_2)$ where $A_1=[\Tilde{y},1]$ and $A_2=[0,\Tilde{y}]$.

Since agent $1$ chooses his most preferred piece $[\Tilde{y},1]$, we must have:
\begin{equation*}
    V_{1,1}([\Tilde{y},1])+V_{1,2}([0,\Tilde{y}])\leq V_{1,1}([0,\Tilde{y}])+V_{1,2}([\Tilde{y},1])
\end{equation*}
Thus agent $1$ is swap envy-free. Now, via contradiction, we show that agent $1$ has a worth below $\frac{1}{2}$. Suppose that $V_1(A)>\frac{1}{2}$. Then we have 
\begin{align*} % no numbers with starred version
 \frac{1}{2} &< V_1{A} = V_{1,1}([\Tilde{y},1])+V_{1,2}([0,\Tilde{y}]) \\
       &\leq V_{1,1}([0,\Tilde{y}])+V_{1,2}([\Tilde{y},1]).
\end{align*}
From the above inequality, we get 
\begin{align*} % no numbers with starred version
 1 &<V_{1,1}([\Tilde{y},1])+V_{1,2}([0,\Tilde{y}])+V_{1,1}([0,\Tilde{y}])+V_{1,2}([\Tilde{y},1]) \\
       &= V_{1,1}([0,1)+V_{1,2}([0,1])=1
\end{align*}
This is a contradiction, thus $V_1(A)\leq \frac{1}{2}$.\newline
We have seen that the allocation $A$ satisfies fairness notions proportionality and swap envy-freeness for agent $1$.\newline
Now we show that the allocation $A$ also satisfies fairness for agent $2$.\newline By the choice of $\Tilde{y}$, $V_{2,1}([\Tilde{y},1])+V_{2,2}([0,\Tilde{y}])=V_{2,1}([0,\Tilde{y}])+V_{2,2}([\Tilde{y},1])$, and so agent $2$ is not swap-envious. Besides,
\begin{align*} % no numbers with starred version
 2V_2(A) &=2(V_{2,1}([\Tilde{y},1])+V_{2,2}([0,\Tilde{y}]))\\
       &= V_{2,1}([\Tilde{y},1])+V_{2,2}([0,\Tilde{y}])+V_{2,1}([0,\Tilde{y}])+V_{2,2}([\Tilde{y},1])\\
       &=V_{2,1}([0,1])+V_{2,2}([0,1])=1
\end{align*}
So we get $V_2(A)=\frac{1}{2}$. Thus the allocation $A$ is proportional, swap envy-free, and therefore only needs single cut.\qed
\newline
The classical model ensures that envy-free (and thus proportional) allocations achieved with only n-1 cuts \cite{rental}. It stands to reason that at least that many cuts are required because each agent must get a piece.
In contrast, we show there are some instances where zero cuts are needed to achieve swap stability. To see this: consider an instance where $v_{i,1}(x)=0, \forall x\in [0,1], \forall i \in N$ and the others density functions are $v_{i,j}(x)=\frac{1}{n-1},\forall x\in [0,1]$ and allocate the entire cake to agent $1$.

Branzei et al. \cite{Brnzei2013ExternalitiesIC} show that a proportional and swap envy-free allocation can require strictly more than $n-1$ cuts. Now we demonstrate how a chore division with externalities achieves the same outcome. This lower bound also applies to swap stability, which entails both proportionality and  swap envy-freeness.
\newline
\textbf{Theorem 3.} \textit{There may be strictly more than $n-1$ cuts necessary for a proportional and envy-free swap allocation.}\newline
\textit{Proof.} Here we consider an instance and then show that a proportional allocation where $n-1$ cuts are used is not swap envy-free. Consider the value density functions:
$v_{i,i}(x)=1-\frac{\epsilon}{n},\forall x\in [0,1]$ and for all $i\neq j\in N$,
\begin{equation*}
   v_{i,j}=
   \begin{cases}
    \frac{\epsilon}{n-1} & \text{if } x \in [\frac{i-1}{n},\frac{i}{n}]\\
    0 & \text{otherwise}
    \end{cases}
\end{equation*}
where $\epsilon$ is small positive number and $i\neq 1$.
\begin{equation*}
    v_{1,1}(x)=
    \begin{cases}
        1 & \text{if } x\in [0,\frac{1}{n}]\\
        0 & \text{otherwise}
    \end{cases}
\end{equation*}
and 
\begin{equation*}
    v_{1,j}=
    \begin{cases}
       0 & \text{if } x\in [\frac{j-1}{n},\frac{j}{n}] \\
       \frac{1}{n-1}& \text{otherwise}
    \end{cases}
\end{equation*}
where $j=2,3,....,n$.\newline
Since any proportional allocation of the chore division needs at least $n-1$ cuts, we allocate the piece $[\frac{i-1}{n},\frac{i}{n}]$ to the agent $i$. Thus the resulting allocation $A=(A_1,A_2,.....,A_n)$ is proportional where $A_i=[\frac{i-1}{n},\frac{i}{n}]$ for all $i\in N$. Nevertheless, the allocation $A$ is not swap envy-free, since due to swap envy-freeness condition for agent $1$, we get
\begin{equation*}
  V_{1,1}(A_1)+ V_{1,j}(A_j)\leq V_{1,1}(A_j)
  + V_{1,j}(A_1)
\end{equation*}
equivalently,
\begin{equation*}
    \frac{1}{n}+ 0 \leq 0 +\frac{1}{n(n-1)}
\end{equation*}

This is a contradiction when $n\geq 3$, thus the allocation $A$ with $n-1$ cuts is not swap envy-free but proportional. This concludes that for a proportional and swap envy-free allocation at least $n$ cuts are required.\qed 
\newline
In addition to demonstrating the existence of swap stability under weak assumptions, Branzei et al. \cite{Brnzei2013ExternalitiesIC} also provide an upper limit on the number of cuts required. This result is also valid for the existence of swap-stable allocation for chore division with externalities, where the main idea used by Branzei et al. \cite{Brnzei2013ExternalitiesIC} is the following lemma by Alon \cite{alon1987splitting}. 
\newline
\textbf{Lemma 1 (Alon 1987 \cite{alon1987splitting}).}  Let $\mu_1, \mu_2,.....,\mu_t$ be $t$ continuous
probability measures on the unit interval. Then it is possible to cut the interval in $(k-1). t$ places and partition the $(k-1). t+1$ resulting intervals into $k$ families $\mathcal{F}_1,\mathcal{F}_2,....,\mathcal{F}_k$ such that $\mu_i(\bigcup \mathcal{F}_j)=1 \slash k$, for all 
$1\leq i \leq t, 1 \leq j \leq  k$. The number $(k-1).t$ is best possible.
\newline
\textbf{Theorem 4 (Branzei et al. 2013 \cite{Brnzei2013ExternalitiesIC}).} \textit{ Assume that the value density functions are continuous. Then swap stable allocations are assured and need at most $(n-1)n^2$ cuts.}


In the following passage, we show that it is always possible to make fair allocations when value density functions are piecewise continuous.
\subsection*{Piecewise Constant Valuations}
Let the chore be given as a set of intervals $\mathcal{J}=(I_1,I_2,....,I_m)$ such that for all $i,j\in N$, the influence of agent $j$ on agent $i$ in the interval $k$ is given by a value density function constant on $I_k:$ $v_{i,j}(x)=c_{i,j,k}$, $\forall x \in I_k$ where $\bigcup I_J=[0,1]$.
\newline
\textbf{Definition 4} (Uniform Allocation). \textit{An  allocation is regarded as uniform if each agent receives a contiguous piece of each interval $I_j$ of length $|I_j|/n$ in the context of a chore division problem with piecewise constant valuations over a set of intervals $\mathcal{J}=\{I_1,I_2,....,I_m\}$.}
\newline
\textbf{Theorem 5.} \textit{Consider a chore division instance with externalities, where the value density functions are piecewise constant. Then the uniform allocation is swap stable and $(n-1)(m-1)$ cuts are needed.}
\newline
\textit{Proof.} The uniform allocation is swap stable, since all agents receive identical pieces. That is, $V_{i,j}(A_p)=V_{i,j}(A_q)$ for all $i,j,p,q \in [n]$ where $A=(A_1,A_2,....,A_n)$ is the uniform allocation. Thus the required number of cuts is $(n-1)(m-1)$. 
\subsection*{Piecewise Linear Valuations}
Let the chore be given as a set of intervals $\mathcal{J}=(I_1,I_2,....,I_m)$ such that for all $i,j\in N$, the influence of agent $j$ on agent $i$ in the interval $k$ is given by a value density function linear on $I_k:$ $v_{i,j}(x)=a_{i,j,k}+b_{i,j,k}. x$, $\forall x \in I_k$ where $\bigcup I_J=[0,1]$.
\newline
\textbf{Definition 5} (Sandwich Allocation). \textit{The sandwich allocation of a set of interval $\mathcal{J}=\{I_1,I_2,.....,I_m\}$ is the allocation such that each agent $i$ receives a piece of cake $X_{ij}$ from every interval $I_j$ and $X_{ij}$ is defined as:
\begin{equation*}
    X_{ij}=[a_j+(i-1)\alpha_j,a_j+i\alpha_j]\cup [b_j-i\alpha_j,b_j-(i-1)\alpha_j]
\end{equation*}
where $I_j=[a_j,b_j]$, $\alpha_j=\frac{b_j-a_j}{2n}$.}
\newline
\textbf{Theorem 6.} \textit{Consider a chore division instance with externalities, where the value density functions are piecewise constant. Then the sandwich allocation is swap stable and $2(n-1)(m-1)$ cuts are required.} 
\newline
We require the following well-known property of piecewise linear valuations \cite{brams2012maxsum,kurokawa2013cut,chen2013truth}.
\newline
\textbf{Lemma 2.} \textit{Assume that the interval $[a,b]$ is partitioned into $2n$ equal pieces and that an agent has a linear value density on this interval. Let the piece constructed by joining the $i$-th piece from the left (going right) with the $i$-th piece from the right (moving left) be denoted by $X_i$ for $i\in [n]$. That is, $X_1$ is the leftmost and rightmost piece, $X_2$ is the second from the left
combined with the second from the right, etc. Then the agent
is indifferent between the $X_i$.}
\newline
\\
\textit{Proof of the theorem 6.} The sandwich allocation is swap stable, as all agents have identical pieces due largely to Lemma $2$. That is, $V_{i,j}(A_p)=V_{i,j}(A_q)$ for all $i,j,p,q \in [n]$ where $A=(A_1,A_2,....,A_n)$ is the sandwich allocation. Moreover, $2(n-1)(m-1)$ cuts are needed.
\section{Complexity Considerations}
The common communication model for cake cutting was introduced by Robertson and Webb \cite{Cake}; it limits the connection between the algorithm and agents to two types of queries:
\begin{enumerate}
    \item $\textbf{Evaluate}_{i}(x,y)$: Output $V_{i}([x,y])$
    \item $\textbf{Cut}_{i}(x,\alpha)$: Output $y$ such that $V_{i}([x,y])=\alpha$.
\end{enumerate}
Branzei et al. \cite{Brnzei2013ExternalitiesIC} extend the Robertson-Webb query model \cite{Cake} to include the following types of queries involving externalities:
\begin{enumerate}
    \item $\textbf{Evaluate}_{i,j}(x,y)$:\newline
           Agent $i$ outputs $\alpha$ such that $V_{i,j}([x,y])=\alpha$.

    \item $\textbf{Cut}_{i,j}(x,\alpha)$:\newline
           Agent $i$ outputs $y$ such that $V_{i,j}([x,y])=\alpha$.
\end{enumerate}
Branzei et al. \cite{Brnzei2013ExternalitiesIC} demonstrate that, even with two agents, the generalized Robertson-Webb communication model cannot provide a proportional protocol. We affirm that this assumption remains true even when chores have externalities. Due to Proposition $1$, we conclude that even with two agents, the generalized Robertson-Webb communication model cannot provide a swap envy-free protocol.
\newline
\textbf{Theorem 7 (Branzei et al. 2013\cite{Brnzei2013ExternalitiesIC}).} \textit{ There exists no finite protocol that can compute a proportional allocation of the entire cake even for two
agents in the extended Robertson-Webb model.}
\newline
\textbf{Theorem 8.} \textit{ In the extended Robertson-Webb model, there is no finite protocol that can determine a proportional allocation of the entire chore, even for two agents.}\newline
\textit{Sketch of the proof.} Suppose that $A=(A_1,A_2)$ be any complete proportional allocation of a chore. by the condition of proportionality, we have 
\begin{equation}
    V_{i,i}(A_i)+V_{i,3-i}(A_{3-i})\leq \frac{1}{2}
\end{equation}
Moreover, we have
\begin{equation*}
    V_{i,i}(A_i)+V_{i,i}(A_{3-i})+V_{i,3-i}(A_i)+V_{i,3-i}(A_i)=1
\end{equation*}
The inequality $(4)$ can be rewritten as:
\begin{equation} \label{eq1}
\begin{split}
V_{i,i}(A_i)+V_{i,3-i}(A_{3-i})&\leq\frac{1}{2}(V_{i,i}(A_i)+V_{i,i}(A_{3-i})
 \\
 & +V_{i,3-i}(A_i)+V_{i,3-i}(A_i)
\end{split}
\end{equation}
Equivalently,
\begin{equation}
 V_{i,i}(A_i)+V_{i,3-1}(A_{3-i})\leq V_{i,i}(A_{3-i})+ V_{i,3-i}(A_i)   
\end{equation}
We conclude that any complete proportional allocation with arbitrary valuations satisfies the inequality $(6)$ .\newline
Our goal is to show that there is no finite protocol
that can compute a complete proportional allocation in the extended Robertson-Webb model.\newline
For our convenience, we consider a chore division instance where the two agents have symmetric valuations. That is, $v_{1,1}(x)=v_{2,2}(x)$ and $v_{1,2}(x)=v_{2,1}(x)$, $\forall x\in [0,1]$. Moreover, we assume that $V_{1,1}([0,1])=\frac{2}{3}$ and $V_{1,2}([0,1])=\frac{1}{3}$. Due to symmetry of the valuations, the inequality $(6)$ can be rewritten as:
\begin{equation}
    V_{3-i,3-i}(A_i)+V_{3-i,i}(A_{3-i})\leq V_{3-i,3-i}(A_{3-i})+ V_{3-i,i}(A_i)
\end{equation}
From the inequalities $(6)$ (when $i=1$) and $(7)$ (when $i=2$) we obtain
\begin{equation}
  V_{1,2}(A_1)+V_{1,1}(A_2)= V_{1,2}(A_2)+ V_{1,1}(A_1)   
\end{equation}
Since $V_{1,1}(A_1)+V_{1,1}(A_2)=\frac{2}{3}$ and $V_{1,2}(A_1)+V_{1,2}(A_2)=\frac{1}{3}$, the equation $(8)$ can be rewritten as:
\begin{align*}
    V_{1,1}(A_1)-V_{1,2}(A_1)&=\left(\frac{2}{3}-V_{1,1}(A_1)\right)-\left( \frac{1}{3}-V_{1,2}(A_2)\right)
\end{align*}
Equivalently,
\begin{equation}
    V_{1,1}(A_1)-V_{1,2}(A_1)=\frac{1}{6}
\end{equation}
Due to symmetry valuations, we must obtain
\begin{equation}
  V_{2,2}(A_2)-V_{2,1}(A_2)=\frac{1}{6}  
\end{equation}
Thus to achieve proportional allocation, the allocated pieces satisfy the equations $(9)$ and $(10)$.
So we conclude that there is no finite protocol that computes the required allocation following the proof of Theorem 5 by Branzei et al. \cite{Brnzei2013ExternalitiesIC}.\qed
\section{Optimal Allocation}
Our goal in this section is to interpret the most efficient allocations among all the allocations that are proportional and swap envy-free. We describe tractable algorithms to do this under different assumptions about agents' preferences. Cohler et al. \cite{Cohler2011OptimalEC} provide a linear program for such allocations in the standard cake cutting model. Branzai \cite{Siminathesis} gives a linear program for the extended cake-cutting model with externalities to compute such allocations. Several AI articles explore the efficient allocations of cake \cite{Bei2012OptimalPC,Aumann2012ComputingSC,Arunachaleswaran2019FairAE}.
\newline
\textbf{Proposition 2.} \textit{Consider a chore division instance with externalities, where the value density functions are piecewise constants. Then optimal allocation requires at most $m-1$ cuts and can be computed in $mn^2$ queries, where $m$ is the number of intervals in the representation.}
\newline
\textit{Proof.} The optimal allocation allocates the interval $I_k$ to the agent $j$, when $\sum_{i=1}^n V_{i,j}(I_k)$ gives the minimum value where $k\in [m]$ and $j\in N$. Thus the required number of eval queries is $mn^2$ and the allocation needs at most $m-1$ cuts.\qed
\newline
In her Ph.D. dissertation, Branzei \cite{Siminathesis} gives an algorithm that returns an optimal swap envy-free and proportional allocation in time polynomial in $m$ and $n$, following Cohlar et al.\cite{Cohler2011OptimalEC}, where m is the number of intervals in the representation. Cohler et al.\cite{Cohler2011OptimalEC} apply the conclusion that an algorithm for linear programming executes in polynomial time if all outputs are $O(k)$-bit rational \cite{Karmarkar1984ANP}. A $k$-bit rational is a rational number of the form $a\slash b$, where each $a$ and $b$ is a $k$-bit integer.
\newline
\textbf{Theorem 9.} \textit{Assume that agents have piecewise linear valuations over a set of intervals $\mathcal{J}=\{I_1,I_2,....,I_m\}$. Then Algorithm $1$ produces an optimal swap envy-free and proportional allocation in polynomial time.} 
\newline
\textbf{Lemma 3.} \textit{Given $\epsilon >0$ and general value density functions are $v_{i,j}$. Assume that $v_{i,j}^\prime$ are piecewise constant functions such that for all  $i,j\in N$,
\begin{equation}
    v_{i,j}(x)-\frac{\epsilon}{n}\leq v_{i,j}^\prime (x) \leq v_{i,j}(x)
\end{equation}
Let $A=(A_1,A_2,.....,A_n)$ and $A^\prime=(A_1^\prime,A_2^\prime,......,A_n^\prime)$ be optimal proportional allocations with respect to the valuations $V_{i,j}$ (induced $v_{i,j}$) and $V_{i,j}^\prime$ (induced by $v_{i,j}^\prime$), respectively. Then $A^\prime$ is proportional with respect to $V_{i,j}$ and $e(A^\prime)\leq e(A)+\epsilon$.}
\newline
\textit{Proof.} To show $A=(A_1,A_2,.....,A_n)$ is proportional with respect to $V_{i,j}^\prime$, we have
\begin{equation}
   V_{i,j}^\prime(A_j) \leq V_{i,j}(A_j) 
\end{equation}
Summing the inequality $(15)$ over $j\in N$, we get
\begin{equation}
    V_i^\prime (A)= \sum_{j=1}^n V_{i,j}^\prime(A_j) \leq V_i(A)=\sum_{j=1}^n V_{i,j}(A_j)
\end{equation}
Due to the fact that $A=(A_1,A_2,.....,A_n)$ is proportional with respect to $V_{i,j}$, we can deduce from inequality $(16)$ that
\begin{equation*}
   V_i^\prime (A)= \sum_{j=1}^n V_{i,j}^\prime(A_j) \leq V_i(A)=\sum_{j=1}^n V_{i,j}(A_j) \leq \frac{1}{n} 
\end{equation*}
Thus we have $V_i^\prime(A)\leq \frac{1}{n}$ for all $i\in N$. So $A=(A_1,A_2,......,A_n)$ is proportional with respect to $V_{i,j}^\prime$.
\newline
Next, we assert that 
\begin{equation*}
 \sum_{i}^n V_i^\prime(A^\prime)\leq \sum_{i}^n V_i^\prime(A).  
\end{equation*}
Since $A^\prime=(A_1^\prime,A_2^\prime,.....,A_n^\prime)$ is an optimal proportional allocation with respect to $V_{i,j}^\prime$ and $A=(A_1,A_2,.......,A_n)$ is a proportional allocation with respect to $V_{i,j}^\prime$, so we must obtain
\begin{equation*}
   \sum_{i=1}^n V_{i}^\prime(A)\leq \sum_{i=1}^n V_{i}(A)
\end{equation*}
Furthermore, it preserves 
\begin{equation*}
\begin{split}
 \sum_{i=1}^n V_i(A^\prime)=\sum_{i=1}^n \sum_{j=1}^n  V_{i,j}(A_j^\prime)&=\sum_{i=1}^n \sum_{j=1}^n \int_{A_j^\prime} v_{i,j}(x) dx\\
  &\leq \sum_{i=1}^n \sum_{j=1}^n \int_{A_j^\prime}(v_{i,j}(x)+\frac{\epsilon}{n})dx\\&=\sum_{i=1}^n \sum_{j=1}^n \int_{A_j^\prime}v_{i,j}(x)dx + \epsilon \\ &=\sum_{i=1}^n V_i(A)+\epsilon 
\end{split}
\end{equation*}
equivalently,    $e(A^\prime)\leq e(A)+\epsilon$.\qed
\newline
\textbf{Lemma 4.} \textit{Given $\epsilon >0$ and general value density functions are $v_{i,j}$. Assume that $v_{i,j}^\prime$ are piecewise constant functions such that for all $i,j\in N$,
\begin{equation}
    v_{i,j}(x)-\frac{\epsilon}{4}\leq v_{i,j}^\prime (x) \leq v_{i,j}(x)
\end{equation}
Let $A=(A_1,A_2,.....,A_n)$ be an optimal proportional and swap envy-free allocation with respect to the valuations $V_{i,j}$ (induced $v_{i,j}$), and let $A^\prime=(A_1^\prime,A_2^\prime,......,A_n^\prime)$ be an optimal proportional and $\frac{\epsilon}{2}$-swap envy-free allocation with respect to $V_{i,j}^\prime$ (induced by $v_{i,j}^\prime$), respectively. Then $A^\prime$ is proportional and $\epsilon$-swap envy-free with respect to $V_{i,j}$ and
$e(A^\prime)\leq e(A)+\frac{n \epsilon}{4}$.}
\newline
\textit{Proof.} To demonstrate that $A^\prime$ is $\epsilon$-swap envy-free with respect to $V_{i,j}$ follow the lead of
\begin{equation*}
\begin{split}
     V_{i,i}(A_i^\prime)+V_{i,j}(A_j^\prime)&\leq (V_{i,i}^\prime(A_i^\prime)+\frac{\epsilon}{4})+(V_{i,j}^\prime(A_j^\prime)+\frac{\epsilon}{4})\\ &= V_{i,i}^\prime(A_i^\prime)+V_{i,j}^\prime(A_j^\prime )+\frac{\epsilon}{2}\\&\leq (V_{i,i}^\prime(A_j^\prime)+V_{i,j}^\prime(A_i^\prime )+\frac{\epsilon}{2})+\frac{\epsilon}{2}\\&=V_{i,i}^\prime(A_j^\prime)+V_{i,j}^\prime(A_i^\prime )+\epsilon\\&=V_{i,i}(A_j^\prime)+V_{i,j}(A_i^\prime )+\epsilon
\end{split}
\end{equation*}
Again, $A^\prime$ is proportional allocation with respect to $V_{i,j}$ followed from Lemma $4$. Thus $A^\prime$ is a proportional and $\epsilon$-swap envy-free allocation with respect to $V_{i,j}$.\newline
Next, we assume that
\begin{equation}
    \sum_{i=1}^n V_i^\prime(A^\prime)\leq \sum_{i=1}^n V_i^\prime(A).
\end{equation}
Because of $A^\prime$ is an optimal proportional and $\frac{\epsilon}{2}$-swap envy-free allocation with respect $V_{i,j}^\prime$, and $A$ is a proportional allocation with respect to $V_{i,j}^\prime$, it is enough to show that $A$ is an $\frac{\epsilon}{2}$-swap envy-free allocation with respect to $V_{i,j}^\prime$.
\begin{equation*}
    \begin{split}
      V_{i,i}^\prime(A_{i})+V_{i,j}^\prime(A_{j})&\leq V_{i,i}(A_{i})+V_{i,j}(A_{j})\\&=V_{i,i}(A_{j})+V_{i,j}(A_{i})\\&\leq (V_{i,i}^\prime(A_{j})+\frac{\epsilon}{4})+(V_{i,j}^\prime(A_{i})+\frac{\epsilon}{4})\\&=V_{i,i}^\prime(A_{j})+V_{i,j}^\prime(A_{i})+\frac{\epsilon}{2}
    \end{split}
\end{equation*}
Furthermore, it yields
\begin{equation*}
\begin{split}
 \sum_{i=1}^n V_i(A^\prime)=\sum_{i=1}^n \sum_{j=1}^n  V_{i,j}(A_j^\prime)&=\sum_{i=1}^n \sum_{j=1}^n \int_{A_j^\prime} v_{i,j}(x) dx\\
  &\leq \sum_{i=1}^n \sum_{j=1}^n \int_{A_j^\prime}(v_{i,j}(x)+\frac{\epsilon}{4})dx\\&=\sum_{i=1}^n \sum_{j=1}^n \int_{A_j^\prime}v_{i,j}(x)dx + \frac{n \epsilon}{4} \\ &=\sum_{i=1}^n V_i(A)+\frac{n \epsilon}{4}
\end{split}
\end{equation*}
equivalently, $e(A^\prime)\leq e(A)+\frac{n \epsilon}{4}$\qed
\newline
\textbf{Theorem 10.} \textit{ Suppose that there are $n$ agents whose value density functions $v_{i,j}$ are $K$-Lipschitz with $M\leq v_{i,j}(x)$ for some $M\in \mathbb{N}$ , all $i,j\in N$. For any  $\epsilon>0$, there is an algorithm that runs in time polynomial in $n,\log M, K, \frac{1}{\epsilon}$
and computes a proportional allocation whose efficiency is within $\epsilon$ of the optimal proportional allocation.}
\newline
\textit{Proof.} Our approach will depend on an Algorithm via Lemma reduction. Finding a collection of piecewise constant value density functions $v_{i,j}^\prime$ that closely resembles the set of real value density functions $v_{i,j}$ is our first goal.
To find $v_{i,j}^\prime$ as required by Lemma 3, we know that $v_{i,j}$ are $k$-Lipschitz, i.e, for all $i,j\in N$ and for all $x, y \in [0,1]$,
\begin{equation*}
    |v_{i,j}(x)-v_{i,j}(y)|\leq K.|x-y|.
\end{equation*}
Now, split $[0,1]$ in $\lceil \frac{2nK}{\epsilon} \rceil$ subintervals of size at most $\frac{\epsilon}{2nK}$. Call this new set of subintervals $\mathcal{J}$. For all 
$I_k \in \mathcal{J}$,\newline define
\begin{center}

     $v_{i,j}^* (I_k)=\min\limits_{x\in I_k} v_{i,j}(x)$ \text{and}  $S=\{
 \frac{r}{2^a}: r\in [0,M2^a]\}$
\end{center}
Where $M$ is a lower bound on $v_{i,j}(x)$ for all $i,j\in N$ and $x\in [0,1]$, and $a=\lceil {2+ \log(\frac{1}{\epsilon})} \rceil $. For each $I_k\in \mathcal{J}$ and $x\in I_k$, let $v_{i,j}^\prime(x)=p^*(I_k)$, where
\begin{equation*}
    p^*(I_k)=\max \{s\in S:s\leq v_{i,j}^*(I_k)\}
\end{equation*}
The $K$-Lipschitz condition ensures that the density function $v_{i,j}$ varies at most $\frac{\epsilon}{2n}$ on each subinterval $I_k$, where $I_k\in \mathcal{J}$, i.e,
\begin{equation*}
    |v_{i,j}(x)-v_{i,j}(y)|\leq \frac{\epsilon}{2n}
\end{equation*}
for any $x,y \in I_k, I_k\in \mathcal{J}$.\newline
Since $a=\lceil {2+ \log(\frac{1}{\epsilon})} \rceil $, $p^*(I_k)-v_{i,j}^*(I_k)\leq \frac{\epsilon}{2n}$ for each $I_k\in \mathcal{J}$, so $v_{i,j}^\prime$ satisfies the condition (11). By Lemma 3, then, an optimal $\frac{\epsilon}{2}$-EF allocation with respect to $v_{i,j}^\prime$ will be  $\epsilon$-EF with respect to the true value density functions $v_{i,j}$ and have more social welfare. 

Since the $v_{i,j}^\prime$ are piecewise constant, their encodings\footnotemark{} \footnotetext{The notations $\Theta_{i,j}$ and $\Phi_{i;j}$ are used, followed by Cohler et al. \cite{Cohler2011OptimalEC}.}are given by $\Theta_{i,j}=\mathcal{J}$ and $\Phi_{i;j}=\{\beta_{k}\}$, where $\beta_{k}=p^*(I_k)$, for $k=1,2,....,m$ and $i,j\in N$.\newline
We can easily find such an allocation by submitting $\Theta_{i,j},\Phi_{i,j}$ to a modified version of Algorithm 1, where the condition (19) is omitted.

Finally, we specify the running time of the algorithm. First we have that $|\Theta_{i,j}|=|\Phi_{i,j}|=\lceil {\frac{4K}{\epsilon}}\rceil$. Now the values of $p^*(I_k)$ are $(\log_2 M+a)$-bit rationals \cite{Papadimitriou1979EfficientSF}. We can easily choose the boundaries of the intervals in $\Theta_{i,j}$ to be $O(\log_2 M+a)$-bit rationals, so that all inputs to Algorithm 1 are $\log_2 M, K$, and $\frac{1}{\epsilon}$. The theorem follows.\qed
\newline
\textbf{Theorem 11.} \textit{ Suppose that there are $n$ agents whose value density functions $v_{i,j}$ are $K$-Lipschitz with $M\leq v_{i,j}(x)$ for some $M\in \mathbb{N}$ , all $i,j\in N$. For any  $\epsilon>0$, there is an algorithm that runs in time polynomial in $n,\log M, K, \frac{1}{\epsilon}$
and computes a proportional and $\epsilon$-swap envy-free allocation whose efficiency is within $\frac{n \epsilon}{4}$ of the optimal proportional and swap envy-free allocation.}
\newline
\textit{Sketch of the proof.} Split $[0,1]$ in $\lceil\frac{8K}{\epsilon}\rceil$ subintervals of size at most $\frac{8K}{\epsilon}$. So, following from the first portion of the proof of Theorem 7, we get a collection of piecewise constant value density functions $v_{i,j}^\prime$ that satisfy condition $(14)$.
We can easily find such an allocation by submitting $\mathcal{J}$, $\Phi_{i,j}$ to a modified version of Algorithm 1 via Lemma 4, where the condition (19) is replaced by
\begin{equation*} 
\sum_{k=1}^m x_{i,k}V_{i,i}(I_k)+x_{j,k} V_{i,j}(I_k) \leq\sum_{k=1}^m x_{j,k}V_{i,i}(I_k)+x_{i,k} V_{i,j}(I_k)+\frac{\epsilon}{2},\forall i,j \in N
\end{equation*} 
This follows the theorem.\qed
\begin{algorithm}
 \caption{}\label{alg2}   
\begin{algorithmic}[1]
\State Let $\mathcal{J}=\{I_1,I_2,....,I_m\}$ be the set of intervals formed the chore $[0,1]$.
\State  Solve the following linear program:

\begin{align}
\min \quad
& \sum_{i,j =1}^{n} \sum_{k=1}^m x_{j,k} V_{i,j}(I_k)
\end{align}
s.t.
\begin{align}
\sum_{i=1}^n x_{i,k}&
= 1 && \forall k\in [m]\\
\sum_{k=1}^m \sum_{j=1}^n x_{j,k}V_{i,j}(I_k)&\leq \frac{1}{n} && \forall i\in N \\
\sum_{k=1}^m x_{i,k}V_{i,i}(I_k)+x_{j,k} V_{i,j}(I_k) & \leq\sum_{k=1}^m x_{j,k}V_{i,i}(I_k)+x_{i,k} V_{i,j}(I_k) && \forall i,j \in N \\
x_{i,k} &\geq 0&& \forall i\in N, \forall k\in [m]
\end{align}



\State Return an allocation which for all $i\in N$ and $I_k\in \mathcal{J}$ allocates an $x_{i,k}$ fraction of subinterval $I_k$ to agent $i$.
\end{algorithmic}
\end{algorithm}

\section{Discussion}
We focus the entire paper on the study of chore division with externalities using the extended model of the cake-cutting problem with externalities presented by Branzei et al. \cite{Brnzei2013ExternalitiesIC}. We also make the additional assumption that the sum of each agent's multiple valuations should be normalized to 1. We show that a proportional and swap envy-free allocation can require at least $n$ cuts. We propose that proportionality and swap envy-freeness can often be calculated with at most $n-1$ cuts when needed individually. The existence of a query model and computationally efficient protocols for the computing of swap envy-free and swap stable allocations for any number of agents is one of the significant open problems.  

  



\section{Acknowledgements}

The author would like to thank Bodhayan Roy for his suggestions which have improved the presentation of this paper significantly. 

 
\bibliographystyle{unsrtnat}
\bibliography{references}  

\end{document}
