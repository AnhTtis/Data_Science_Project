Robots extensively rely on sensor observations to detect static and dynamic obstacles. However, sensors inherently face limitations, primarily due to occlusion or range constraints. For instance, consider a common scenario illustrated in Figure \ref{fig:front_figure} where a green vehicle can only observe a 
nearby red vehicle, which occludes the orange vehicles in the horizontal lanes. If the green vehicle maintains its speed, it risks colliding with the occluded vehicles. To navigate safely in such conditions, the green vehicle can consider two alternatives: 1) no occluded vehicles exist, making it safe to drive forward, and 2) occluded vehicles exist, which is especially likely if the green vehicle observes the red vehicle decelerating. 
A responsible human driver would choose to proceed cautiously, balancing the risk of collision with an occluded vehicle against the (perhaps more likely) scenario in which it is safe to proceed. A reasonable driver would also recognize that the horizontal lanes will not remain occluded for very long, at which point any occluded vehicles will be visible and the optimal behavior will be unambiguous.

This scenario highlights how interactions among agents influence their observed behavior and, conversely, how observed behavior can reveal underlying interaction structures. For example, the deceleration of the red vehicle suggests the existence of occluded orange vehicles. Therefore, robots can anticipate and respond to potential occluded agents promptly by observing the behavior of visible agents.

\begin{figure}[!t]
\centering
\includegraphics[width=0.47\textwidth]{figures/front_figure.png}
\caption{Occlusion-aware contingency game planner in an intersection scenario. The green vehicle can only see the red vehicle in the adjacent lane and is uncertain about the existence of occluded vehicles in the horizontal lanes.
The green vehicles makes two assumptions: either 1) occluded vehicles exist. It can then use our proposed approach to estimate their trajectories by observing the red vehicle’s deceleration and plans the dark green trajectory; or 2) there are no occluded vehicles. It then only interacts with the red vehicle and plans the light green trajectory. Our contingency planning approach blends these two alternative strategies, while accounting for the fact that any occluded agents will be visible in the near future as the green vehicle approaches the intersection.}
\label{fig:front_figure}
\vspace{-4ex}
\end{figure}

Existing game-theoretic methods
 %\cite{fisac2019hierarchical,wang2021game,fridovich2020efficient,peters2021inferring,peters2023ijrr,li2023cost,peters2023learning, liu2023learning}
effectively solve multi-agent interaction and planning problems % either
by computing agents' trajectories as Nash
equilibria of noncooperative, dynamic games \cite{fisac2019hierarchical,fridovich2020efficient,laine2021multi,chiu2021encoding,wang2021game,le2022algames,laine2023computation,zhu2024sequential}.
%or by 
Recent efforts are also capable of identifying the game model that best describes the observed agent behaviors \cite{le2021lucidgames,peters2021inferring,peters2023ijrr,li2023cost, liu2023learning,mehr2023maximum, khan2024leadership}. These methods perform well when all agents are perfectly visible. However, the presence of occluded agents, a common occurrence in real-world scenarios, has been largely overlooked.

To address the challenge of occluded agents, we make the following contributions which are illustrated in Figure~\ref{fig:front_figure}: 1) an occlusion-aware game-theoretic estimator that simultaneously identifies unknown parameters in each agent’s cost function and estimates the trajectories of both occluded and visible agents in a Nash game, based solely on noisy position observations of visible agents; and 2) a receding horizon occlusion-aware contingency game planner, structured to account for the possible existence or nonexistence of occluded agents. We evaluate our method across various traffic scenarios.
Monte Carlo experiments indicate that our estimation approach provides accurate trajectory estimates for both visible and occluded agents. Moreover, our planning pipeline computes safer navigation decisions compared to the existing baseline method.
