\subsection{Social Occlusion Inference}\label{soi_intro}
We begin by introducing social occlusion inference methods. Many works \cite{hara2018recognizing,hara2020predicting, afolabi2018people,itkina2022multi,mun2023occlusion} have explored inference methods based on observations of visible agents' behavior in social settings. These works leverage the influence of the surrounding environment on an individual's behavior. Representative works often employ computer vision techniques \cite{hara2018recognizing,hara2020predicting} and occupancy grid map generation techniques \cite{afolabi2018people,itkina2022multi,mun2023occlusion} to infer occupancy of occluded areas. While these methods primarily focus on determining whether the occluded area is occupied or if an agent is emerging from it, additional techniques must be applied to provide precise information about agents' behavior or to compute safe control policies for robots\cite{mun2023occlusion}.

\subsection{Planning-based Agent Behavior Modeling}\label{planning_intro}
To enable precise behavior modeling, many works utilize planning-based methods. These approaches assume that agents make rational decisions about their trajectories, which follow specific rules. Inferring these rules is the key to behavior modeling tasks. Inverse optimal control (IOC) \cite{kalman1963linear,maroger2020walking,albrecht2011imitating, maroger2022inverse} and inverse reinforcement learning (IRL)\cite{ng2000algorithms,ziebart2009planning,kitani2012activity,monfort2015intent,previtali2016predicting} are two prominent frameworks in this area. These approaches focus on recovering the rules that underlie agent-environment interactions and use these rules to predict behavior. Recent multi-agent IRL techniques \cite{henry2010learning,pfeiffer2016predicting,kim2016socially,kretzschmar2016socially,morales2018towards,sun2019behavior} explore rules governing agents' social behaviors and apply to robot navigation \cite{kim2016socially,kretzschmar2016socially} and autonomous driving \cite{morales2018towards,sun2019behavior} tasks.

\subsection{Dynamic Games and Inverse Dynamic Games}\label{game_intro}
As a subcategory of planning-based methods, game-theoretic methods\cite{fisac2019hierarchical,fridovich2020efficient,laine2021multi,chiu2021encoding,wang2021game,le2022algames,laine2023computation,zhu2024sequential} extend single-agent optimal control techniques to model multi-agent interactions as Nash equilibria of noncooperative, dynamic games\cite{bacsar1998dynamic}. Recent efforts \cite{le2021lucidgames,peters2021inferring,peters2023ijrr,li2023cost,liu2023learning,mehr2023maximum} aim to model agent behaviors more precisely by introducing inverse dynamic games. Instead of directly computing Nash trajectories, inverse games aim to learn the parameters of a game model that best explains the observed agents' behavior.

In general, planning-based and game-theoretic methods assume that agents' behaviors directly result from %their decision-making under certain rules, which motivates our work. 
their attempts to optimize specific preferences which are encoded as objective functions.
However, these approaches share a common limitation: they require observations from all agents and do not account for occluded agents, leading to imprecise inference of visible agents' objectives and suboptimal navigation outcomes in the presence of occlusions. 
To address this challenge, we propose an occlusion-aware game-theoretic technique for estimating both observed and occluded agents' state and intentions, from only observations of visible agents. 

\subsection{Contingency Planning and Contingency Games}
Contingency planning has been studied extensively as a technique for managing uncertainty in navigation tasks. Contingency planning methods \cite{hardy2013contingency, chen2022interactive, nair2022stochastic} design a control strategy with multiple branches representing both the different possible scenarios and the times at which an agent would be sure of which scenario it is in.
% in response to all possibilities of the scenario. 
Recent works \cite{rhinehart2021contingencies, tolstaya2021identifying} also propose an end-to-end contingency planning framework that enables both intention prediction and motion planning. In multi-agent scenarios, contingency games can additionally account for agents' noncooperative behavior \cite{peters2024contingency}.
% has also been incorporated with dynamic games into a contingency game \cite{peters2024contingency}. 

In dealing with potentially occluded agents during navigation, our method extends the idea in \cite{peters2024contingency} by proposing an occlusion-aware contingency game planner. We introduce a receding-horizon estimation and planning pipeline that enhances accuracy in modeling agent behavior and promotes safety in scenarios with potentially occluded agents.