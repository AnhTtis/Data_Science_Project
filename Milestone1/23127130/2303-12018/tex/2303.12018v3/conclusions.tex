\section{Conclusions and outlook}
\label{sec:conclu}
Recently, upon the inclusion of the
full Run~2 data set and substantially
refined analysis techniques,
the CMS collaboration has
confirmed an excess of \GW{about} $3\,\sigma$
local significance at about $95\gev$
in the low-mass Higgs boson searches
in the di-photon final state.
\GW{An excess at this mass value with similar significance}
had previously been reported
based on the $8\tev$ Run~1 and the first-year
Run~2 data set.
We have 
\GW{investigated the interpretation of this excess as a 
di-photon resonance arising from the production of} 
a Higgs boson in the
\GW{Two-Higgs doublet model that is extended by a}
complex singlet 
(S2HDM).
We have shown that a good description of the
excess is possible in the Yukawa
type~II and~IV, while being in agreement
with all other collider searches for additional
Higgs bosons, the measurements of the 
\GW{properties of the}
SM-like
Higgs boson at $125\gev$,
\GW{and further experimental and theoretical constraints}. 
At the same time,
the model can account for all or a large
fraction of the observed 
DM relic abundance in agreement with the measurements
of the Planck satellite.

Previously, a signal strength
for the di-photon excess observed by CMS 
of $\mu_{\gamma\gamma}^{\rm exp} = 0.6 \pm 0.2$
had been obtained 
utilizing the 
\GW{data from the
first year of Run~2 and of Run~1}.
This \GW{relatively high central value of the signal strength 
gave rise to a} 
preference to
a type~II Yukawa structure,
in which larger signal rates
of the state at $95\gev$ can be achieved
compared to the type~IV.
After the inclusion of the remaining
Run~2 data
and 
\GW{performing various improvements of the}
experimental analysis, 
the new CMS 
\GW{result 
shows an excess with a local significance that is essentially unchanged 
compared to the previous result but which yields an
interpretation in terms of a smaller central value of the signal 
strength with reduced uncertainties,
$\mu_{\gamma\gamma}^{\rm exp} =
\muCMS^{+\dmuCMSpl}_{-\dmuCMSmi}$}.
We have shown that
as a result of the smaller \GW{central} value of
$\mu_{\gamma\gamma}^{\rm exp}$ 
\GW{both Yukawa types provide an equally well description} 
of the di-photon excess in the S2HDM.


The di-photon excess observed at CMS
is especially intriguing in view of
additional excesses that appeared at
approximately the same mass.
An excess of events above the SM
expectation with \GW{about} $2\,\sigma$ local
significance was observed at LEP in searches
\GW{for}
Higgsstrahlung production
of a scalar state that then decays to a pair
of bottom quarks. Moreover, CMS observed
an excess with \GW{about} $3\,\sigma$ local significance
consistent with a mass of about
$95\gev$ in searches 
\GW{for}
the production
of a Higgs boson via gluon fusion and subsequent
decay into \GW{tau} pairs.

We have demonstrated that the S2HDM type~II can
simultaneously describe
the CMS di-photon excess and the $b \bar b$
excess observed at LEP, whereas no significant
signal for the CMS di-tau excess is possible 
\GW{in this model}.
In the S2HDM type~IV, on the other hand, in addition
also a sizable signal 
\GW{strength in the di-tau channel can occur}.
However,
even in type~IV
the 
\GW{maximally reachable}
signal rates are
smaller than the
signal strengths
that \GW{would be}
required to describe the
di-tau excess at the level of~$1\,\sigma$.

Our analysis in the S2HDM serves as
an example study from which more 
\GW{general}
conclusions valid for a wider class of
extensions of the SM can be drawn.
Notably, supersymmetric extensions
were previously shown to be able to accommodate a
di-photon signal at about $95\gev$ that 
\GW{turns out to}
be in good agreement with the updated
experimental value of $\mu_{\gamma\gamma}^{\rm exp}$.

In the near future,
\GW{the possible presence}
of a Higgs boson at $95\gev$
\GW{can be directly tested by the eagerly awaited results from the}
corresponding \mbox{ATLAS} searches in the di-photon
and the di-tau final states covering the
mass region below $125\gev$ and utilizing
the full Run~2 data.
\GW{Further into the future,}
the scenarios 
\GW{discussed} here will
be tested in a twofold way at future Runs
of the (HL)-LHC, where the direct searches
for $h_{95}$ and the coupling measurements
of $h_{125}$ will benefit 
\GW{in particular from a significant}
increase of statistics.
Nevertheless, we have shown that the experimental
precision of the coupling measurements of
the Higgs boson at $125\gev$ might not be
sufficient to exclude the S2HDM interpretation
of the excesses at $95\gev$, or conversely 
confirm a deviation from the SM predictions.


Going beyond the (HL-)LHC projections,
we have discussed the experimental prospects
at a future $e^+ e^-$ collider, considering
as an example the ILC operating
at $250\gev$ with an integrated luminosity 
of $2~\mathrm{ab}^{-1}$. At the ILC250,
the couplings of $h_{125}$ could be determined
in an effectively model independent way at
sub-percent level precision.
Assuming that no deviations from the SM
predictions would be observed, the measurements
of the couplings of $h_{125}$ would
significantly disfavour
the S2HDM interpretation of the excess
at $95\gev$.
Conversely, 
\GW{a clear deviation from the SM predictions will be established}
if the coupling measurements of $h_{125}$
will be according to the predictions
of any S2HDM parameter
point describing the excess.

Although the possible state
at $95\gev$ has suppressed couplings compared
to $h_{125}$, the ILC could produce
$h_{95}$ in large numbers 
\GW{if it has a sufficiently large coupling to $Z$ bosons}.
We have shown that
the clean environment of an $e^+e^-$ collider
would allow for a determination of
the couplings of $h_{95}$
at percent-level precision.
As such, we demonstrated that the ILC,
in contrast to the HL-LHC,
could distinguish between a type~II and
a type~IV description of the excesses.


