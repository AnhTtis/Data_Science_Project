\begin{flushright}
\footnotesize
  KA-TP-03-2023, ~~ 
  DESY-23-033, ~~ 
  IFT--UAM/CSIC-23-028 ~~
\end{flushright}

\begin{center}
{\large
\textbf{
The CMS di-photon excess at 95 GeV 
in view of the LHC Run~2 results
}
}
\vspace{0.4cm}

Thomas Biekötter$^1$\footnote{thomas.biekoetter@desy.de},
Sven Heinemeyer$^2$\footnote{Sven.Heinemeyer@cern.ch} and
Georg Weiglein$^{3,4}$\footnote{georg.weiglein@desy.de}\\[0.2em]

{\small

  $^1${\textit{
   Institute for Theoretical Physics,
   Karlsruhe Institute of Technology,\\
   Wolfgang-Gaede-Str.~1, 76131 Karlsruhe, Germany
 }}
 
 $^2${\textit{
   Instituto de Física Teórica UAM-CSIC, Cantoblanco, 28049,
   Madrid, Spain
 }}
 
 $^3${\textit{
   Deutsches Elektronen-Synchrotron DESY,
     Notkestr.~85, 22607 Hamburg, Germany
  }}
  
 $^4${\textit{
   II. Institut für Theoretische Physik, Universität Hamburg,\\
    Luruper Chaussee 149, 22761 Hamburg, Germany\\[0.4em]
  }}
  
}


\begin{abstract}
The CMS collaboration has recently reported the
results of a low-mass Higgs-boson search in
the di-photon final state based on the full
Run 2 data set with refined analysis 
techniques. The new results 
show an excess
of events at a mass of about $95\gev$ with a local significance of
$\sigCMS\,\sigma$, confirming a previously reported excess at about the 
same mass and similar significance
based
on the first-year Run 2 plus Run 1 data. 
The observed excess is compatible
with the limits obtained in the corresponding
ATLAS searches.
In this work, we discuss the di-photon excess
and show that it can be interpreted as the
lightest Higgs boson in the Two-Higgs doublet
model that is extended by a complex singlet (S2HDM) of
Yukawa types~II and~IV. 
We show that the second-lightest Higgs
boson is in good
agreement with the current
LHC Higgs-boson measurements of the state at 125 GeV,
and that the full scalar sector is compatible with
all theoretical and experimental constraints.
Furthermore, we discuss the di-photon excess
in conjunction with an excess in the
$b \bar b$ final state observed at LEP and
an excess observed by CMS in the di-tau
final state, which were found at comparable
masses with local significances of about
$2\sigma$ and $3\sigma$, respectively.
We find that the $b \bar b$ excess can
be well described together with the
di-photon excess in both types of the S2HDM.
However, the di-tau excess can only be
accommodated at the level of $1\sigma$ in type IV.
We also comment on the compatibility with 
supersymmetric scenarios and other extended Higgs 
sectors, and
we discuss how the potential signal
can be further analyzed at the LHC and at
future $e^+e^-$ colliders.
\end{abstract}


\end{center}