\section{Numerical discussion}
\label{sec:num}

In order to address the question whether
a description of the CMS di-photon excess 
can be realized in the S2HDM,
possibly in combination with
the excesses in the $b \bar b$ and the
di-tau final states, we performed a
parameter scan in the Yukawa types~II
and~IV of the S2HDM.
\GW{We investigated the theoretical predictions in comparison to 
the experimental results for the observed excesses near $95\gev$, 
ensuring at the same time that the properties of the}  
Higgs boson at $125\gev$ 
\GW{are}
in good agreement with the most
up-to-date LHC signal rate measurements.
To this end, we implemented a
genetic algorithm (using the python
package \texttt{DEAP}~\cite{DEAP_JMLR2012})
that minimizes
a loss function constructed 
\GW{from}
$\chi^2_{125}$ (obtained using
\texttt{HiggsSignals}) and the three
contributions $\chi^2_{\gamma\gamma}$,
$\chi^2_{bb}$,
and $\chi^2_{\tau\tau}$
quantifying the
\GW{compatibility with}
the excesses at $95\gev$,
where we define the latter as
\begin{equation}
\chi^2_{\gamma\gamma,\tau\tau,bb} =
\frac{
(\mu_{\gamma\gamma,\tau\tau,bb} -
\mu_{\gamma\gamma,\tau\tau,bb}^{\rm exp})^2 }{
(\Delta \mu_{\gamma\gamma,\tau\tau,bb}^{\rm exp})^2} \ .
\label{eq:chisq95indi}
\end{equation}
Here the experimental central
values and the uncertainties were
stated in \refse{sec:intro}, and
$\mu_{\gamma\gamma,\tau\tau,bb}$
are the theoretically predicted values.
Since $\mu_{\gamma\gamma}^{\rm exp}$ has
asymmetric uncertainties, we define
$\chi^2_{\gamma\gamma}$ in such a way that
the lower uncertainty is used if
$\mu_{\gamma\gamma} < \mu_{\gamma\gamma}^{\rm exp}$,
and the upper uncertainty
is used if
$\mu_{\gamma\gamma} > \mu_{\gamma\gamma}^{\rm exp}$.
To obtain the predictions for
$\mu_{\gamma\gamma}$ and
$\mu_{\tau\tau}$,
we used \texttt{HiggsTools}
to derive the gluon-fusion cross section
of the state at $95\gev$
via a re-scaling of the SM predictions
as a function of $c_{h_{95} t \bar t}$ and
$c_{h_{95} b \bar b}$. To compute
$\mu_{b b}$, we \GW{approximated}
the cross section
ratio as $\sigma / \sigma_{\rm SM}
= c_{h_{95} VV}^2$.
The branching ratios
of $h_{95}$ were obtained with the help
of \texttt{N2HDECAY} (see
also the discussion in \refse{sec:constraints}).

The set of parameter points obtained by
the minimization of
the loss function
was then confronted with
the constraints discussed in \refse{sec:constraints}.
\GW{Parameter points 
that did not pass the applied constraints were rejected.}
For the generation of the S2HDM parameter
points and the application of the
constraints, we used the program
\texttt{s2hdmTools}~\cite{Biekotter:2021ovi,
Biekotter:2022bxp}, which features interfaces
to \texttt{HiggsBounds}, \texttt{HiggsSignals},
\texttt{micrOMEGAs} and \texttt{N2HDECAY}.

We chose the values of the free parameters in
our scan as follows. The mass of $h_{95}$
was varied in the region in which the
di-photon excess is most pronounced,
i.e.~$94\gev \leq m_{h_{95}} \leq 97\gev$.
The mass of the second-lightest Higgs boson
\GW{was} set to $m_{h_{125}} = 125.09\gev$, and
the third heavier Higgs boson, denoted $H$
in the following, was scanned freely up to an
upper limit of \GW{$m_H = 1\tev$}.
The same upper limit was chosen for the masses
of the DM state $\chi$, the 
\GW{CP-odd}
Higgs boson $A$, and the charged Higgs bosons
$H^\pm$, where for the latter additionally
the lower limit $m_{H^\pm} > 600\gev$ was applied
\GW{arising from}
the flavour constraints.
Moreover, we varied $\tan\beta$ in the
range $1.5 \leq \tan\beta\leq 10$, and for the
singlet vev we chose $40\gev \leq v_S \leq 2\tev$.
Finally, the scan range of the parameter
$m_{12}^2$ was determined by the condition
$400\gev \leq M \leq 1\tev$, where
$M^2 = m_{12}^2 / (\sin\beta \cos\beta)$.



\subsection{Description of the
di-photon excess}

\begin{figure}[t]
\centering
\includegraphics[width=0.9\columnwidth]{mh1_xsyy.pdf}~
\vspace*{-0.4cm}
\caption{\small
S2HDM parameter points
\GW{passing the applied constraints}
in the \plane{m_{h_{95}}}{\mu_{\gamma\gamma}}
for the type~II (blue) and the type~IV (orange).
The expected and observed
cross section limits 
\GW{obtained}
by CMS are indicated
\GW{by}
the black dashed and solid lines, respectively,
and the $1\sigma$ and $2\sigma$ uncertainty intervals
are indicated 
\GW{by}
the green and yellow
bands, respectively. The value of
$\mu_{\gamma\gamma}^{\rm exp}$ and its uncertainty
is shown with the magenta error bar at the mass
\GW{value}
at which the excess is most pronounced.}
\label{fig:gamgam}
\end{figure}

In \reffi{fig:gamgam} we show the
predictions for $\mu_{\gamma\gamma}$ for the
S2HDM parameter points that are in agreement
with the applied constraints.
The type~II parameter points are shown
in blue, and the parameter points of type~IV
are shown in orange. The expected and observed
cross section limits 
\GW{obtained}
by CMS are indicated
\GW{by}
the black dashed and solid lines, respectively,
and the $1\sigma$ and $2\sigma$ uncertainty intervals
are indicated 
\GW{by}
the green and yellow
bands, respectively~\cite{CMSnew}.
The value of
$\mu_{\gamma\gamma}^{\rm exp}$ and its uncertainty
is shown with the magenta error bar at the mass
\GW{value}
at which the excess is most pronounced.
One can see that both types of the S2HDM
considered here can accommodate the observed
excess. As expected from the discussion
in \refse{sec:quanti},
type~II 
\GW{can give rise to larger predicted}
values
of $\mu_{\gamma\gamma}$ due to the additional
suppression of the $h_{95} \to \tau^+ \tau^-$
decay mode. 
\GW{The points featuring the largest values of $\mu_{\gamma\gamma}$ in
type~II are seen to exceed the observed limit of the new CMS analysis 
(which is not applied as a constraint via \texttt{HiggsBounds} in this 
plot). On the other hand, both type~II and type~IV give rise to 
predictions for $\mu_{\gamma\gamma}$ that are very well compatible with 
the new} 
experimental value of $\mu_{\gamma\gamma}^{\rm exp}$
\GW{obtained by CMS}
after the inclusion of the second- and third-year
Run~2 data.\footnote{\GW{As discussed above,} in
type~I and type~III no significant enhancement
of the di-photon branching ratio of
$h_{95}$ is possible, and one finds
$\mu_{\gamma\gamma} \approx \mu_{bb}
\lesssim c_{h_{95} VV}^2$.
Thus, $\mu_{\gamma\gamma}$-values 
\GW{close to}
$\mu_{\gamma\gamma}^{\rm exp}$ require
values of $c_{h_{125}VV}^2 \approx
1 - c_{h_{95}VV}^2$ that are in
significant tension
with the coupling measurements of $h_{125}$.}



\subsection{Combined description of the
excesses}

\begin{figure*}[t]
\centering
\includegraphics[width=0.82\columnwidth]{muyy_mubb_chisq125_tp2.pdf}~~~
\includegraphics[width=0.82\columnwidth]{muyy_mubb_chisq125_tp4.pdf}\\[0.4em]
\includegraphics[width=0.82\columnwidth]{muyy_mull_chisq125_tp2.pdf}~~~
\includegraphics[width=0.82\columnwidth]{muyy_mull_chisq125_tp4.pdf}
\vspace*{-0.4cm}
\caption{\small
S2HDM parameter points 
\GW{passing the applied constraints}
in the
\plane{\mu_{\gamma\gamma}}{\mu_{bb}}
(top row) and the
\plane{\mu_{\gamma\gamma}}{\mu_{\tau\tau}}
(bottom row) for type~II (left)
and type~IV (right). The colors of the
points indicate the value of $\Delta \chi^2_{125}$.
The black dashed lines 
indicate the
regions \GW{in} which the two excesses considered
in each plot are accommodated at a level of
$1\sigma$ or better, i.e.~$\chi^2_{\gamma\gamma}
+ \chi^2_{bb} \leq 2.3$ (top row) and
$\chi^2_{\gamma\gamma}
+ \chi^2_{\tau\tau} \leq 2.3$ (bottom row).}
\label{fig:yyllbb}
\end{figure*}

We demonstrated in the previous section that
both \GW{the Yukawa types II and IV}
can describe the
excess in the di-photon channel observed
by CMS. Now we turn to the question whether
additionally also the $b \bar b$ excess
observed at LEP and the $\tau^+ \tau^-$
excess at CMS can be accommodated.

Starting with the $b \bar b$ excess,
we show in the top row of \reffi{fig:yyllbb}
the parameter points 
\GW{passing the applied constraints}
in the 
\plane{\mu_{\gamma\gamma}}{\mu_{b b}}.
The parameter points of type~II and type~IV
are shown in left and the right plot, respectively.
The colors of the points indicate the
value of $\Delta \chi^2_{125}$ showing
the compatibility with the LHC rate
measurements of $h_{125}$.
The black dashed lines 
indicate the
region in which the excesses are described
at a level of $1\sigma$ or better,
i.e.~$\chi^2_{\gamma\gamma} +
\chi^2_{bb} \leq 2.3$ 
(see \refeq{eq:chisq95indi}).
The shape of these lines is asymmetrical
due to the asymmetrical
uncertainties of $\mu_{\gamma\gamma}^{\rm exp}$
used in the definition of $\chi^2_{\gamma\gamma}$
in \refeq{eq:chisq95indi}.

One can see that we find points
inside the $1\sigma$ preferred region 
in the upper left and 
right plots.
Thus, both type~II and type~IV
are able to describe the di-photon excess
and the $b \bar b$ excess
simultaneously.
At the same time the properties
of the second-lightest scalar
$h_{125}$ are such that the
LHC rate measurements
can be accommodated 
\GW{at the same $\chi^2$ level}
as
in the SM,
i.e.~$\Delta \chi^2_{125} \approx 0$,
or even marginally
better, i.e.~$\Delta \chi^2_{125} < 0$.
At the current level of experimental
precision, the description of both
excesses is therefore possible in combination
with the presence of
a Higgs boson at $125\gev$ that
would so far be indistinguishable from
a SM Higgs boson.

Turning to the di-tau excess,
we show in the bottom row of
\reffi{fig:yyllbb} the 
parameter points 
\GW{passing the applied constraints}
in the \plane{\mu_{\gamma\gamma}}{\mu_{\tau\tau}}.
As before, the colors of the points
indicate the values of $\Delta \chi^2_{125}$,
and the black dashed lines 
indicate the region in which the di-photon excess and
the di-tau excess are described at a level
of $1\sigma$ or better, i.e.~$\chi^2_{\gamma
\gamma} + \chi^2_{\tau\tau} \leq 2.3$.

In the lower left plot, showing
the parameter points of the scan in
type~II, one can see that there are no
points within or close to the black line. 
This finding is in agreement with the
\GW{discussion} in \refse{sec:quanti}.
It is also \SH{qualitatively} unchanged as compared to
the results of \citere{Biekotter:2022jyr},
where $\mu_{\gamma\gamma}^{\rm exp} = 0.6 \pm 0.2$
was used: the new and somewhat lower experimental 
\GW{central}
value of $\mu_{\gamma\gamma}^{\rm exp}$ has no impact on the
(non-)compatibility of the $\gamma\gamma$ and the
$\tau^+\tau^-$ excesses in Yukawa type~II.

The lower right plot shows
the 
parameter points 
\GW{passing the applied constraints}
from the
scan in type~IV. One can observe that
the values of $\mu_{\tau\tau}$ overall increase
with increasing value of $\mu_{\gamma\gamma}$.
The parameter points that predict
the largest values for the signal rates
reach the lower edge of the black line 
that indicates the preferred region
regarding the two excesses.
However, even these points lie substantially
below the central value of $\mu_{\tau\tau}^{\rm exp}$.
A simultaneous description
of both excesses at $95\gev$ observed by CMS
is therefore possible only at the
level of $1\,\sigma$ at best.
Although larger values of $\mu_{\tau\tau}$
are theoretically possible in
type~IV~\cite{Biekotter:2022jyr},
the application of cross-section limits
from Higgs-boson searches exclude such
parameter points.
These constraints arise in particular 
from recent searches performed by CMS 
for the production of a Higgs boson
in association with a top-quark pair or
in association with a $Z$~boson, with subsequent
decay into tau pairs~\cite{CMS-PAS-EXO-21-018}.

Constraints on the interpretation
of the di-tau excess 
\GW{as an additional Higgs boson}
were also derived from
cross-section measurements of the Higgs boson
at $125\gev$.
In particular, \citere{Iguro:2022dok}
investigated the sensitivity of the ATLAS
measurement assuming the production
of $h_{125}$ in association
with a top-quark pair and subsequent decay
into di-tau
pairs~\cite{ATLAS:2022yrq}.\footnote{In
\citere{Coloretti:2023wng} the invariant-mass
spectra of the $h_{125} \to WW^*$ decay
channel measured by ATLAS~\cite{ATLAS:2022ooq}
and CMS~\cite{CMS:2022uhn}
were considered. However, the decay of
$h_{95} \to WW^*$ is highly off-shell,
suppressing the corresponding branching
ratio by orders of magnitude compared to
the one of $h_{125}$.
As a result, there is no sensitivity
in this decay channel to the presence
of $h_{95}$ according to our model
interpretation of the excesses.}
The ATLAS analysis considered
an invariant di-tau mass in the range between
$50\gev$ and $200\gev$ and is based
on the full Run~2 data \GW{set}. 
We emphasize, however, that the
constraints extracted
in \citere{Iguro:2022dok}
are affected by the lack of publicly available
information on
the correlations between the different mass bins.

In summary, the S2HDM type~II can
simultaneously describe
the CMS di-photon excess and the $b \bar b$
excess observed at LEP, whereas no significant
\GW{contribution to the
signal strength of} the CMS di-tau excess is 
\GW{generated.}
In type~IV, in addition also a 
\GW{contribution to the
di-tau signal strength can occur}, although
the largest possible signal rates of about
$\GW{\mu_{\tau\tau}} = 0.5$ are somewhat
below the 
\GW{experimentally preferred}
range of $\mu_{\tau\tau}^{\rm exp}
= 1.2 \pm 0.5$.

\medskip
\SH{Our results in the S2HDM}
\GW{can be generalised to other 
extended Higgs sectors containing at least a
second Higgs doublet and
at least one scalar singlet.
Our analysis indicates that
the conclusions in various models
that have previously been considered
as an explanation for the di-photon excess are expected to be affected by the
modified value
of $\mu_{\gamma\gamma}^{\rm exp}$. 
This applies in particular to}
\SH{supersymmetric extensions
of the SM, which were shown to be able to accommodate
a signal at about $95\gev$ with
a signal strength that in most cases
was predicted to be
at the lower end of
the previous
$\mu_{\gamma\gamma}^{\rm exp}$-range~\cite{Biekotter:2017xmf,
Domingo:2018uim,
Hollik:2018yek,
Biekotter:2019gtq,
Choi:2019yrv,
Cao:2019ofo,
Biekotter:2021qbc}.
Requiring also agreement with the LEP excess
resulted in $\mu_{\gamma\gamma} \approx
0.3$~\cite{Domingo:2018uim,Biekotter:2019gtq,Biekotter:2021qbc},
which}
\GW{turns out to 
be in very good
agreement with the updated result from CMS.}


\subsection{Prospects at future colliders}

We finally discuss how future collider
experiments will shed light on the
possible presence
of a Higgs boson below $125\gev$ as considered here.
In the S2HDM the mixing between the singlet-like
state at $95\gev$ and the SM-like state
at $125\gev$ determines the strengths of the
couplings of $h_{95}$ to fermions and
gauge bosons. Thus,
in addition to directly searching for
$h_{95}$, a
complementary -- although more model-dependent -- strategy consists in
the search for modifications of the cross sections
of $h_{125}$ compared to the ones of
a SM Higgs boson. We start with discussing this
approach in the following.

Currently, the experimental precision
of the observed couplings of $h_{125}$
\GW{is} at the level of ten to twenty
percent~\cite{CMS:2022dwd,ATLAS:2022vkf}.
During the high-luminosity phase of the LHC \GW{(HL-LHC)},
the experimental precision of these couplings
\GW{is expected to be reduced}
to the level of
a few percent~\cite{Cepeda:2019klc}.\footnote{Here
it is assumed that no undetected decay mode
of $h_{125}$ 
\GW{into} BSM particles is
present.}
A future $e^+ e^-$ collider with sufficient
energy to produce $h_{125}$
could further improve
the experimental precision to the
sub-percent level.
As an example, we will consider
in the following the expected precision
of the International Linear Collider
(ILC) operating at a center-of-mass
energy of $250\gev$ and collecting
$2~\mathrm{ab}^{-1}$ of
integrated luminosity~\cite{Bambade:2019fyw}.
We note that here and in the following 
the specific example of the projections for the 
ILC is meant to showcase
the potential impact of the coupling
measurements at a future $e^+ e^-$ collider.
In fact, very similar results would be obtained
considering the other proposals for a
``Higgs factory'' operating at about~250~GeV, such
as CLIC, CEPC or the FCC-ee~\cite{deBlas:2019rxi}.

\begin{figure}[t]
\centering
\includegraphics[width=0.96\columnwidth]{h125cpls.pdf}
\vspace*{-0.4cm}
\caption{\small
S2HDM parameter points 
\GW{passing the applied constraints}
that
predict a di-photon signal strength
in the 
\GW{preferred range of $0.21 \leq \mu_{\gamma\gamma}
\leq 0.52$
in view of the excess observed by}
CMS~\cite{CMSnew} in the
\plane{|c_{h_{125} \tau^+ \tau^-}|}{|c_{h_{125} VV}|}.
The type~II and the type~IV parameter
points are shown in blue and orange,
respectively.
The green dotted and the magenta
dashed ellipses indicate the 
projected experimental precision of the
coupling measurements at the
HL-LHC~\cite{Cepeda:2019klc} and the
ILC250~\cite{Bambade:2019fyw}, respectively,
with their centers located at the SM values.}
\label{fig:h125cpls}
\end{figure}

In \reffi{fig:h125cpls} we show
the 
parameter points 
\GW{passing the applied constraints}
of the scan in type~II (blue)
and in type~IV (orange) that
provide a good description
of the di-photon excess,
i.e.~$0.21 \leq \mu_{\gamma\gamma} \leq 0.52$,
in the \plane{|c_{h_{125} \tau^+ \tau^-}|}{|c_{h_{125} VV}|}.
Here $c_{h_{125} \tau^+ \tau^-}$
and $c_{h_{125} VV}$
are the effective coefficients of the
coupling of $h_{125}$ to tau-leptons
and the gauge bosons $V=Z,W$, respectively.
These coefficients are normalized such that they
are equal to one in the SM.
Centered at the SM prediction, we also
indicate with the green dotted ellipse the
expected precision on the coupling coefficients
after the 
\GW{HL-LHC}
will have collected $3000~\mathrm{fb}^{-1}$
of integrated luminosity.
Finally, the magenta dashed ellipse indicates
the expected experimental precision
after a combination of the HL-LHC data
and the ILC data collected 
at $\sqrt{s} = 250 \gev$ (ILC250) with an 
integrated luminosity of $2~\mathrm{ab}^{-1}$.
We note that these experimental
projections have been obtained assuming
that the cross section measurements are
according to the predictions of the SM.

One can see that the points of both
types all lie outside of the green ellipse.
For the points with the largest deviations
from the SM, the anticipated HL-LHC precision
would be sufficient to distinguish between 
\GW{SM-like properties of $h_{125}$ and the 
predictions of the S2HDM for parameter regions that are in accordance 
with the observed di-photon excess}.
However, for the \GW{S2HDM points that are} closest to the 
SM value, no distinction at the $2\,\sigma$ 
level could be established. Consequently, the
HL-LHC will not be able to entirely probe
the S2HDM interpretation of the di-photon
excess at $95\gev$ \GW{based on the coupling measurements of $h_{125}$}.
Moreover, 
\GW{for many of the displayed blue and orange points the 
expected HL-LHC precision, indicated by the}
size of the green ellipse,
will
not be sufficient to distinguish between
a type~II and a type~IV interpretation.

Now we compare the model predictions with
the expected precision at the ILC250,
indicated by the magenta ellipse.
One can see that under the assumption that
no modifications of the properties of
$h_{125}$ will be observed even at the
ILC, all parameter points would be excluded
with 
\GW{high} experimental significance.
\GW{On the other hand,}
for each point in the S2HDM 
describing the di-photon excess, 
\GW{a clear deviation of the properties of $h_{125}$ from the SM 
predictions could be established via the}
coupling measurements.
The ILC also has a significantly larger
potential to distinguish between a type~II
and a type~IV scenario, although even the
ILC precision might not be sufficient to
distinguish between the types for the
parameter points with the largest values
of $c_{h_{125} \tau^+ \tau^-}$
and $c_{h_{125} VV}$.
\GW{Information about} the direct production of
$h_{95}$ and its coupling measurements
\GW{will of course be instrumental to further}
probe the S2HDM scenarios. 

In our S2HDM interpretation of the di-photon
excess, $h_{95}$ is required to have a
non-vanishing coupling to
top quarks, and thus also to gauge bosons,
in order to be the origin of this excess.
Moreover, a sizable coupling of $h_{95}$
\GW{to the $Z$ boson}
is required if this state is also supposed to
be the origin of the $b \bar b$ excess
observed at LEP.
In this case, a future lepton collider running at
$250\gev$ has the capability to produce
$h_{95}$ in large numbers~\cite{Drechsel:2018mgd,
Wang:2020lkq}.
From the resulting
cross-section measurements, the couplings
of $h_{95}$ 
\GW{could}
be determined with a
precision that is expected to greatly improve on
the precision achievable at
the~LHC.\footnote{Experimental
projections for Higgs coupling measurements at the HL-LHC
are only publicly available for the discovered Higgs boson 
at 125~GeV. In contrast to the cleaner experimental environment at 
an $e^+ e^-$ collider, at the LHC it is not feasible
to obtain projections for the accuracy of coupling measurements 
for additional Higgs bosons without detailed simulations
taking into account systematical uncertainties. Since such a dedicated
simulation would be beyond the scope of the present paper, we do not attempt 
to provide precise quantitative estimates for the achievable accuracy 
on the couplings of $h_{95}$ at the HL-LHC.
However, a rough estimate of the precision
for the signal rates in the di-photon and
di-tau channel assuming $3~\mathrm{ab}^{-1}$
can be achieved by a simple rescaling
with the square root of the luminosity,
yielding a precision of about 
10\% for the di-photon and the di-tau channel.}
Thus, if a new state at $95\gev$ exists,
a future $e^+ e^-$ collider such as the
ILC 
\GW{is expected to}
be of vital importance for the
determination of the underlying model
that is realized in nature.

\begin{figure}[t]
\centering
\includegraphics[width=0.96\columnwidth]{cplsprec.pdf}
\vspace*{-0.4cm}
\caption{\small
S2HDM parameter points passing the
applied constraints that
predict a di-photon signal strength
in the preferred range $0.21 \leq \mu_{\gamma\gamma}
\leq 0.52$
in view of the excess observed by
CMS~\cite{CMSnew} in the
\plane{|c_{h_{95} \tau^+ \tau^-}|}{|c_{h_{95} VV}|}.
The type~II and the type~IV parameter
points are shown in blue and orange,
respectively. The shaded ellipses around
the dots indicate the projected experimental
precision with which the couplings of
$h_{95}$ could be measured at the ILC250 
\GW{with}
$2~\mathrm{ab}^{-1}$ of integrated
luminosity, which we evaluated according to 
\citere{Heinemeyer:2021msz}.}
\label{fig:h95cpls}
\end{figure}

In order to showcase the potential
of the ILC for
discriminating different models that
give rise to the state at $h_{95}$,
we show in \reffi{fig:h95cpls}
the parameter points
of our scans in the
\plane{|c_{h_{95} \tau^+ \tau^-}|}{|c_{h_{95} VV}|}.
Here, $c_{h_{95} \tau^+ \tau^-}$
and $c_{h_{95} VV}$ are the effective
coefficients for the couplings of $h_{95}$
to tau-leptons and gauge bosons, respectively.
These coefficients are normalized
such that they are equal to one for a
hypothetical SM Higgs boson at the mass
of $h_{95}$. As in \reffi{fig:h125cpls},
the parameter points of
type~II and type~IV are shown in
blue and orange, respectively,
and we only depict the parameter points
that provide a good description of
the di-photon excess observed by CMS.
In addition to the theoretical prediction
of the coupling coefficients, indicated
with the dots, we also indicated the
experimental precision with which the
respective couplings could be measured
at the ILC 
\GW{by means of} the shaded ellipses
around each dot.
We estimated the experimental precision
of the coupling measurements 
\GW{for the ILC250 with $2~\mathrm{ab}^{-1}$
of integrated luminosity}
according
to the approach discussed in
\citere{Heinemeyer:2021msz}.

One can observe in \reffi{fig:h95cpls}
that the blue points
and the orange points are clearly
separated from each other.
For a fixed value of the gauge-boson
coupling, the parameter
points of type~IV predict larger couplings
to tau-leptons compared to the parameter
points of type~II. This is in line with
the discussion in \refse{sec:quanti}: In type~II
one has
$c_{h_{95} \tau^+ \tau^-} = c_{h_{95} b \bar b}$,
such that the enhancement of the di-photon
branching ratio via the condition
$|c_{h_{95} t \bar t} / c_{h_{95} b \bar b}| > 1$
is achieved in the regime
in which $c_{h_{95} \tau^+ \tau^-}$
is suppressed. On the other hand,
in type~IV one has
$c_{h_{95} \tau^+ \tau^-} = c_{h_{95} t \bar t}$,
such that the coupling to tau-leptons
is less suppressed in the regime in which the
di-photon branching ratio is enhanced.

As a consequence of the separation of
the points of the two types, combined with the 
high anticipated precision of the $h_{95}$ coupling 
measurements at the ILC250, there are
no blue or orange ellipses that overlap.
Thus, the coupling
measurements of $h_{95}$ at the ILC
would be sufficient to distinguish
between a type~II or a type~IV interpretation.
In combination with the experimental
observation regarding $h_{125}$ (see
discussion above), a lepton collider like
the ILC would be able to 
\GW{scrutinize} the underlying physics model that is realized
in nature.

