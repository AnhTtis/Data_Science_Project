\section{Introduction}
\label{sec:intro}

In
the year 2012 the ATLAS and CMS collaborations
discovered a new
particle~\cite{Aad:2012tfa,Chatrchyan:2012xdj}.
Within the current experimental and theoretical uncertainties the properties 
of the observed particle are
consistent with the predictions for 
the Higgs boson of the
Standard Model (SM) with a mass
of approximately
$125\gev$~\cite{CMS:2022dwd,ATLAS:2022vkf}, 
but they are also compatible with many scenarios 
of physics beyond the SM (BSM).  
While the minimal scalar sector of the SM features
only one physical scalar particle,
BSM physics often gives rise to
extended Higgs sectors in which additional
scalar particles are present.
Accordingly, one of the primary objectives of the
LHC is the search for additional Higgs bosons,
which is of crucial importance for exploring the
underlying physics of electroweak symmetry 
breaking.\blfootnote{$^*$thomas.biekoetter@kit.edu,
$^\dagger$sven.heinemeyer@cern.ch,
$^\ddagger$georg.weiglein@desy.de
}

Searches for Higgs bosons below
$125\gev$ have been performed at
LEP~\cite{Abbiendi:2002qp,Barate:2003sz,
Schael:2006cr},
the Tevatron~\cite{Group:2012zca} and the
LHC~\cite{CMS:2015ocq,CMS:2018cyk,Sirunyan:2018aui,
CMS:2018rmh,ATLAS:2018xad,CMS:2022goy,ATLAS:2022abz,
CMSnew}.
Among them, searches for di-photon resonances are
particularly intriguing, as this decay mode 
constituted one of the
two discovery channels of the
Higgs boson at $125\gev$.
CMS has performed searches for scalar di-photon
resonances at~$8\tev$ and $13\tev$.
Results based on the $8\tev$ data and the
first year of Run~2 data at $13\tev$,
corresponding to an integrated luminosity of
$19.7\,\mathrm{fb}^{-1}$ and $35.9\,\mathrm{fb}^{-1}$,
respectively,
showed a local excess of $2.8\,\sigma$
at $95.3 \gev$~\cite{CMS:2015ocq,
Sirunyan:2018aui}.
This excess, which is present in both the $8\tev$
and the $13\tev$ data set, 
received considerable attention 
already soon after it was made public,
see e.g.~\citeres{Cao:2016uwt,
Fox:2017uwr,
Richard:2017kot,
Haisch:2017gql,
Biekotter:2017xmf,
Liu:2018xsw,
Domingo:2018uim,
Biekotter:2019kde,
Cline:2019okt,
Cao:2019ofo,
Aguilar-Saavedra:2020wrj}.

Recently, CMS published the result based on
their full Run~2 data set
and with substantially refined analysis
techniques. This new analysis
confirmed the excess of di-photon
events at about $95\gev$~\cite{CMSnew}.
By combining the data from the first,
second, and third years of Run~2,
which were collected at
$13\tev$ and correspond to integrated
luminosities of $36.3\,\mathrm{fb}^{-1}$,
$41.5\,\mathrm{fb}^{-1}$ and
$54.4\,\mathrm{fb}^{-1}$, respectively,
CMS finds an excess with a local
significance of $\sigCMS\,\sigma$ at a
mass of $95.4\gev$.
This ``di-photon excess'' can be
described by a scalar
resonance with a signal strength
of~\cite{CMSnewtalk}
\begin{equation}
\mu_{\gamma\gamma}^{\rm exp} =\frac{\sigma^{\rm exp} \left( gg \to \phi \to \gamma\gamma \right)}
         {\sigma^{\rm SM}\left( gg \to H \to \gamma\gamma \right)}
     = \muCMS^{+\dmuCMSpl}_{-\dmuCMSmi} \ .
\label{muCMS}
\end{equation}
Here $\sigma^{\rm SM}$ denotes the cross section
for a hypothetical SM
Higgs boson at the same mass.
In comparison to the previously reported results that 
were based just on the Run~1 and the first-year Run~2 
data~\cite{CMS:2018cyk},
the inclusion of the data collected
in the second and third year of Run~2 and
the refined analysis techniques 
yield a local significance of the excess that is almost 
unchanged, 
while the central value of the signal strength
$\mu_{\gamma\gamma}^{\rm exp}$
in \refeq{muCMS}
is substantially smaller than the
value $\mu_{\gamma\gamma}^{\rm exp} = 0.6 \pm 0.2$
extracted 
from the previous 
results~\cite{CMS:2018cyk}.



Regarding the interpretation of the new result from CMS 
it is important to note
that the updated analysis not only considered more
data, but in comparison to \citere{CMS:2018cyk}
it also improves the background suppression
of misidentified $Z \to e^+ e^-$
Drell-Yan events, and it includes
further event classes requiring the presence
of additional jets. Since 
a possible signal at about $95\gev$
giving rise to a relatively small number of events 
would occur 
on top of a fluctuating background,
one cannot necessarily rely on the the naive expectation
that the significance of 
an excess caused by a statistical fluctuation should be 
reduced by the inclusion of more data while it should be 
increased in case of an actual signal.
In fact, even in the latter case 
the excess of events observed in the different
data sets and evaluated at a fixed mass value
would still be expected to fluctuate.
From our point of view the fact that the inclusion of 
the additional data sets and the improvements in the 
analysis have led to an excess of events at approximately 
the same mass as previously reported with a statistical 
significance that has not been reduced strengthens the 
motivation for exploring a possible BSM origin of the 
observed results.

ATLAS reported results of searches
in the di-photon final state below $125\gev$
using $80~\mathrm{fb}^{-1}$ of Run~2 data
in 2018~\cite{ATLAS:2018xad}.
The ATLAS search 
found only a mild excess of about $1\,\sigma$
local significance
at masses around $95\gev$.
However, the cross section limits
obtained in the ATLAS analysis
are substantially weaker than the corresponding
CMS limits, even in the mass range where
CMS reported the excess~\cite{Heinemeyer:2018wzl},
and the excess observed in CMS is therefore compatible 
with the ATLAS limits.

If the origin of the di-photon excesses
at $95\gev$ is a new particle, the question
arises whether it is also detectable in
other collider channels, and whether additional
indications for this new particle 
might have already occurred
in other existing searches.
Notably, LEP reported a local $2.3\,\sigma$ excess
in the~$e^+e^-\to Z(H\to b\bar{b})$
searches\,\cite{Barate:2003sz}, which would
be consistent with a
scalar particle with a mass of about 
$95\gev$.\footnote{Due to
the $b \bar b$ final state the 
mass resolution is significantly worse
compared to the resolution of
searches in the di-photon final state.}
This ``$b \bar b$ excess'' corresponds to
a signal strength of
$\mu_{bb}^{\rm exp} =
0.117 \pm 0.057$~\cite{Cao:2016uwt,Azatov:2012bz}.
Moreover, CMS observed another excess
compatible with a mass of $95\gev$ in
the Higgs-boson searches utilizing
di-tau final states~\cite{CMS:2022goy}.
This excess was
most pronounced at a mass of $100\gev$
with a local significance of $3.1\,\sigma$,
but it is also well compatible with a mass
of $95\gev$, where the local significance
amounts to $2.6\,\sigma$. For this
``di-tau excess'', the best-fit
signal strength for a mass hypothesis
of $95\gev$ was determined to be
$\mu^{\rm exp}_{\tau\tau} =
1.2 \pm 0.5$.
It is noteworthy that, to date,
ATLAS has not published a search in
the di-tau final state that covers the
mass range around 95~GeV.

Given that the excesses observed by CMS
and LEP occurred at a similar mass,
the intriguing question arises whether
the excesses in the three different channels might arise from the 
production of a single new particle.
This triggered activities in the literature regarding possible 
model interpretations that could account 
for the various excesses
while also satisfying all other
measurements related to the Higgs sector.
Models in which the previously observed
two excesses in the di-photon and the $b \bar b$
final states can be described simultaneously
(with the CMS excess based only on the Run~1 and
first year Run~2 data) were reviewed
in~\citeres{Heinemeyer:2018jcd,Heinemeyer:2018wzl}.
In \citere{Biekotter:2019kde}
those two excesses were studied in the \GW{Two-Higgs doublet model (2HDM)} with
an additional real singlet (N2HDM), with several follow-up
analyses~\cite{Biekotter:2021ovi,
Biekotter:2021qbc,Heinemeyer:2021msz},
while in \citeres{Biekotter:2022jyr,Biekotter:2022abc}
also the more recently observed excess
in the di-tau searches was taken into account.


\medskip
Since the new result obtained by CMS
confirmed the previously observed
di-photon excess
at about $95\gev$ 
but resulted in a significant change in
the required signal rate
$\mu_{\gamma\gamma}^{\rm exp}$, 
it is of interest to assess the implications of the new result on possible
model interpretations. In the present paper we focus in particular on 
the extension of the 2HDM by a complex singlet
(S2HDM) as a template for a model where 
a mostly gauge-singlet scalar particle
obtains its couplings to fermions
and gauge bosons via the mixing with
the SM-like Higgs boson at $125\gev$.
We will demonstrate that 
this kind of scenario is suitable for describing
the di-photon excess. 
In this context we will in particular investigate the impact of the 
reduced central value of the signal strength of
$\mu_{\gamma\gamma}^{\rm exp} = \muCMS$~\cite{CMSnew}
compared to the result of $\mu_{\gamma\gamma}^{\rm exp} = 0.6$
that was obtained based on the previous analysis~\cite{CMS:2018cyk}.
Moreover, we will discuss the possibility
of simultaneously
describing the $b\bar b$ excess
and the di-tau excess.
We will further discuss
possible ways in which the presented scenario
could be confirmed or excluded
experimentally in
the near future.

The paper is structured as follows. In \refse{sec:modeldef} 
we introduce the S2HDM and 
define our notation.
In \refse{sec:quanti}
we qualitatively discuss how
sizable signal rates
in the three channels in which the excesses
have been observed can arise.
The relevant theoretical and experimental
constraints on the model parameters
are discussed in \refse{sec:constraints}.
We present our numerical results and
discuss their implications in \refse{sec:num}, including
an analysis of future experimental prospects.
The conclusions and an outlook are given
in \refse{sec:conclu}.
