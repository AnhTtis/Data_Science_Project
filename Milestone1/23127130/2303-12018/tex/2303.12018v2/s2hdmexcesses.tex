\section{A 95~GeV Higgs boson in the S2HDM}

In this section we briefly summarize the
scalar sector of
S2HDM and how the excesses at $95\gev$
can be accommodated in this model.
We also discuss the relevant experimental
and theoretical constraints that we
apply in our numerical analysis.


\subsection{Model definitions}
\label{sec:modeldef}
In the SM the Higgs sector
contains a single SU(2) doublet $\Phi_1$.
The S2HDM extends the SM by a second
Higgs doublet field $\Phi_2$
and an additional
complex gauge-singlet field
$\Phi_S$~\cite{Jiang:2019soj,
Biekotter:2021ovi}.
The richer structure of the scalar sector is
motivated for instance by the possibility
of a first-order electroweak phase
transition~\cite{Biekotter:2021ysx},
and the related phenomenology, including
electroweak baryogenesis, or the
presence of a stochastic primordial
gravitational-wave
background.
From a more theoretical perspective,
the presence of a second
Higgs doublet field arises in several
extensions of the SM that address the
hierarchy problem in the context of
supersymmetry~\cite{Fayet:1976et}
or compositeness~\cite{Mrazek:2011iu},
and in many models addressing the
strong CP problem of QCD~\cite{Kim:1986ax}.
Due to the presence of the complex scalar
singlet field, the S2HDM can accommodate
a dark-matter candidate in the form of
pseudo-Nambu-Goldstone (pNG) dark
matter~\cite{Biekotter:2022bxp}.
As will be discussed below,
among the various
proposed WIMP dark-matter candidates,
pNG dark matter is 
in particular motivated in view of the
existing limits from
dark-matter
direct-detection
experiments~\cite{PandaX-4T:2021bab,
XENON:2018voc,LZ:2022ufs}.

The vacuum state of the S2HDM is characterized
by non-zero vacuum expectation values (vev)
$v_1$ and $v_2$
for the neutral CP-even components of the
Higgs doublets fields $\Phi_1$
and $\Phi_2$, respectively.
The presence of these vevs
leads to the spontaneous breaking of the
electroweak symmetry. As in the usual 2HDM,
one defines the parameter $\tan\beta = v_2 / v_1$,
where $v_1^2 + v_2^2 = v^2 \approx (246 \gev)^2$
corresponds to the SM vev squared.
In addition, the real component of the
singlet field has the non-zero vev $v_S$,
which breaks a global U(1) symmetry under which
only $\Phi_S$ is charged.
If this symmetry 
was exact initially,
the imaginary component of $\Phi_S$
would act as a massless Goldstone boson.
Therefore, one introduces a soft breaking
via a bilinear term
$-m_\chi^2 (\Phi_S^2 + \mathrm{h.c.})$,
which gives rise to a mass $m_\chi$
for the imaginary component of $\Phi_S$,
which then
plays the role of the pNG
dark-matter state.

Neglecting possible sources of CP violation,
as we do throughout this paper,
the physical scalar spectrum of the S2HDM consists
of three CP-even Higgs bosons $h_{1,2,3}$
with masses $m_{h_{1,2,3}}$ that
are mixed states composed of the neutral real
components of $\Phi_{1,2}$ and the real
component of $\Phi_S$. The imaginary
component of $\Phi_S$ does not mix 
with other states and results in
a stable scalar dark-matter particle which is labeled
$\chi$ in the following. Moreover, as in the
CP-conserving 2HDM, the scalar spectrum contains
a pair of charged Higgs bosons $H^\pm$ and
a CP-odd Higgs boson $A$ with masses
$m_{H^\pm}$ and $m_A$, respectively.

For the presence of two Higgs doublets,
the most general
gauge invariant Yukawa sector gives rise
to flavour-changing neutral currents (FCNC) at
the tree-level. These are, however, strongly
constrained experimentally.
In order to avoid FCNC at the tree-level,
we impose an additional 
$Z_2$ symmetry under which
one of the doublet fields changes the sign,
which is only softly-broken via a term of
the form $-m_{12}^2(\Phi_1^\dagger \Phi_2
+ \mathrm{h.c.})$.
This symmetry can be extended to the fermion
sector such that either $\Phi_1$ or $\Phi_2$
(but not both) couples
to either the charged leptons $\ell$, the up-type
quarks $u$ or the down-type quarks $d$.
There are four different possibilities to
assign conserved charges for the three kinds
of fermions, giving rise to the four Yukawa
types~I, II, III (lepton-specific) and
IV (flipped) that are known from the
$Z_2$-symmetric 2HDM
(see e.g.~\citere{Branco:2011iw}).

For the Yukawa types~II and~IV, $\Phi_1$
is coupled to down-type quarks
and $\Phi_2$ is coupled to up-type
quarks. In this case
an independent modification of the couplings
of the Higgs bosons $h_i$ to bottom quarks and
top quarks is possible.
These two types are therefore of particular interest 
regarding the prediction of a sufficiently large 
di-photon signal rate~\cite{Biekotter:2019kde}.



\subsection{Interpretation of the excesses}
\label{sec:quanti}
In the following discussion,
the lightest of the three CP-even Higgs
bosons of the S2HDM $h_1$
serves as the possible particle state
at $95\gev$, also denoted $h_{95}$ from here on.
We furthermore assume that
the second lightest Higgs boson, $h_2 = h_{125}$,
corresponds to the state discovered
at about $125 \gev$.
The key aspect of the signal interpretation
presented here is that $h_{95}$ obtains
its couplings to the
fermions and gauge
bosons as a result of the mixing with the
CP-even components of the two doublets.
In order to comply
with the constraints from the
Higgs-boson searches at LEP 
\GW{in the mass region of about $95\gev$}
and the LHC cross section measurements 
\GW{for the detected state at $125\gev$, 
the state} $h_{95}$ must have couplings
to gauge-bosons that are reduced by roughly
one order of magnitude as compared to the
couplings of a SM Higgs boson \GW{of the same mass}. 
As a consequence,
in the S2HDM interpretation $h_{95}$ is dominantly
singlet-like.

Despite the \GW{predominant singlet-like character} of $h_{95}$,
sizable decay rates into di-photon pairs can
be achieved via a suppression of the otherwise
dominating decay into $b$-quark 
pairs (see also \citere{Barbieri:2013nka}).
At the same time, 
\GW{no such suppression should occur for
the coupling} to top quarks,
whose loop contribution gives rise to the
decay into photons 
\GW{(and also governs the production process via gluon fusion)}.
As a result, large signal rates $\mu_{\gamma\gamma}$
can \GW{occur} in the S2HDM if
$|c_{h_{95} t \bar t} / c_{h_{95} b \bar b}| > 1$,
where the coupling coefficients $c_{h_{95} t \bar t}$
and $c_{h_{95} b \bar b}$ are the couplings
of $h_{95}$ to the respective quark normalized
to the couplings of a hypothetical
SM Higgs boson of the same mass.
It becomes apparent that
the Yukawa types~I and~III, 
\GW{for which}
$c_{h_{95} t \bar t} = c_{h_{95} b \bar b}$ 
\GW{applies,
do not feature the conditions for a sufficiently large
di-photon branching ratio in accordance with}
the CMS excess.
On the other hand, in type~II and type~IV
\GW{the two} coupling coefficients can be modified
independently. This can potentially
enhance the di-photon
branching ratio by up to
an order of magnitude~\cite{Biekotter:2019kde,
Biekotter:2022jyr}, such that
sizable values of $\mu_{\gamma\gamma}$
can be accommodated even for a relatively small
mixing with the detected Higgs boson
at $125\gev$ (and thus suppressed
cross sections).\footnote{An
additional, although not as significant,
enhancement of $\mu_{\gamma\gamma}$ can
\GW{occur} if $c_{h_{95} t \bar t}$ and
$c_{h_{95} b \bar b}$ carry a relative minus
sign. This relative sign gives \GW{rise to} constructive interference
effects in the loop-induced couplings of
$h_{95}$ to gluons and photons, hence enhancing
both the production and the decay rate in
the $gg \to h_{95} \to \gamma\gamma$ channel.}

Since larger values of $\mu_{\gamma\gamma}$
can be achieved in type~II and~IV
compared to type~I and type~III
as discussed above,
we will focus on the
type~II and the type~IV
in the following.
Between these two types, an important difference
arises from the fact that 
$c_{h_{95} \tau^+ \tau^-} =
c_{h_{95} b \bar b}$ 
\GW{holds}
in type~II, whereas
\GW{the corresponding relation in type~IV is}
$c_{h_{95} \tau^+ \tau^-} =
c_{h_{95} t \bar t}$.
Accordingly, 
\GW{in the parameter regions of
type~II where the di-photon signal rate is
enhanced as a consequence of the suppression of its coupling to
$b$-quark pairs}
the coupling of $h_{95}$
to tau-leptons is \GW{simultaneously}
suppressed.
Hence, type~II is not expected to yield
sizable signal rates in the $\tau^+ \tau^-$
decay channel if the di-photon excess
is accommodated.
On the other hand, given that
$c_{h_{95} t \bar t}$ 
\GW{should be}
unsuppressed for a description of
the di-photon excess,
\GW{type~IV can give rise to a simultaneous description of 
the CMS di-tau excess}~\cite{Biekotter:2022jyr}.


\subsection{Constraints}
\label{sec:constraints}

The parameter space that is relevant for
a possible description of the excesses at
$95\gev$ is subject to various theoretical
and experimental constraints. We will briefly
discuss the relevant constraints in the following.

Theoretical constraints that we \GW{apply in our analysis}
ensure that the perturbative treatment of
the scalar sector of the S2HDM is valid.
To this end, we demand
that the eigenvalues of the scalar
$2\times 2$ scattering
matrix in the high-energy limit
are smaller than $8\pi$, giving
rise to the so-called tree-level perturbative
unitarity constraints~\cite{Biekotter:2021ovi}.
In addition, using the approach
described in \citere{Biekotter:2021ovi}
we 
\GW{apply a condition on the stability of} 
the electroweak vacuum (\GW{see}
\refse{sec:modeldef}) 
by requiring that
the tree-level scalar potential
is bounded from below, and that the
electroweak vacuum corresponds to the
global minimum of the
potential.

Moreover, the parameters of the
S2HDM are constrained by various
experimental 
\GW{results.}
With regards to the collider phenomenology,
we check whether the parameter points
are in agreement with the cross section
limits from collider searches for
BSM Higgs bosons by making use of the
public code
\texttt{HiggsBounds v.6}~\cite{Bechtle:2008jh,
Bechtle:2011sb,
Bechtle:2013wla,
Bechtle:2020pkv,
Bahl:2022igd} (as part of the new code 
\texttt{HiggsTools}~\cite{Bahl:2022igd}).
A parameter point is rejected if
\GW{the} signal rate of one of the Higgs bosons
in the most sensitive search channel
(based on the expected limits) is larger
than the experimentally observed
limit at the 95\% confidence level.

In order to ensure that 
\GW{the properties of $h_{125}$ are}
in agreement
with the measured signal rates from the LHC,
we make use of the public code
\texttt{HiggsSignals v.3}~\cite{Bechtle:2013xfa,
Bechtle:2014ewa,
Bechtle:2020uwn,
Bahl:2022igd} (as part of the new code 
\texttt{HiggsTools}~\cite{Bahl:2022igd}).
This code performs a $\chi^2$ fit to
a large dataset of LHC cross section measurements
in the different channels in which the
SM-like Higgs boson was observed.
As a requirement for accepting or rejecting
a parameter point, we use the condition
$\chi^2_{125} \leq \chi^2_{{\rm SM},125} + 6.18$,
where $\chi^2_{125}$ is the fit value of
the S2HDM parameter point under consideration,
and $\chi^2_{{\rm SM},125} = 146.15$ is the fit result
assuming a Higgs boson at $125\gev$ that behaves
according to the predictions of the SM.
In two-dimensional parameter 
\GW{planes}
the above condition ensures that the 
\GW{selected}
S2HDM parameter points are not disfavoured
by more than $2\,\sigma$ in comparison to the SM
\GW{regarding the properties of $h_{125}$}.

Both \texttt{HiggsBounds} and \texttt{HiggsSignals}
require as input the cross sections and the
branching ratios of the scalar state 
\GW{for the considered parameter point}.
The cross sections were derived internally
in \texttt{HiggsBounds} from the
effective couplings coefficients.
For the computation of the branching ratios,
we applied the
library \texttt{N2HDECAY}~\cite{Muhlleitner:2016mzt,
Engeln:2018mbg},
which we
modified to account for decays
of the Higgs bosons into pairs of
the DM state $\chi$~\cite{Biekotter:2021ovi}.

Indirect experimental constraints on the
Higgs sector can be obtained from flavour-physics
observables and \GW{from} electroweak precision
observables.
Lacking precise theoretical predictions
for the different flavour observables in
the S2HDM,
we apply conservative
lower limits of $\tan\beta > 1.5$ and $m_{H^\pm} > 600\gev$
in our S2HDM parameter scans in type~II and type~IV
to \GW{ensure}
agreement with
the flavour-physics constraints~\cite{Haller:2018nnx}.
With regards to the electroweak precision observables,
we apply constraints in terms of the oblique parameters
$S$, $T$ and $U$ which we computed according to
\citere{Grimus:2008nb} at the one-loop level.
We required that the predicted values of the
oblique parameters are in agreement with the
fit result of \citere{Haller:2018nnx}
within a confidence level
of $2\,\sigma$.\footnote{The fit result of the oblique
parameters was obtained before the recent
CDF measurement of $M_W$~\cite{CDF:2022hxs},
which showed a significant
upward deviation with respect to the SM prediciton.
We demonstrated in \citere{Biekotter:2022abc}
that a larger value for the $W$-boson
mass, even as large as the
central value of the
CDF measurement, can
be accommodated in a 2HDM that is 
extended by a singlet if there
are sizable mass splittings between the heavy
BSM Higgs bosons $h_3$, $A$ and $H^\pm$,
while in addition
the excesses at $95\gev$ can be accommodated
in the same way as presented here.}

As a consequence of the presence of
the stable scalar state
$\chi$, further constraints on the S2HDM
\GW{parameter space} arise from the measurements of the
dark-matter relic abundance of the universe.
Assuming the freezout mechanism for the
production of $\chi$ in the early universe,
we applied the Planck measurement of today's
relic abundance
of $h^2 \Omega = 0.119$~\cite{Planck:2018vyg}
as an upper limit, thus avoiding overproduction
of dark matter. The theoretical predictions for
the relic abundance of $\chi$ were obtained by
making use of the public code
\texttt{micrOMEGAs}~\cite{Belanger:2018ccd}.

Given its nature 
\GW{as}
a pNG
boson of the softly-broken global U(1) symmetry,
the cross sections for the scattering of
$\chi$ on nuclei are highly suppressed in the
limit of small momentum transfer as relevant
for dark-matter direct detection
experiments~\cite{Barger:2008jx}.
As a result,
it has been shown that even including
loop corrections the current
direct detection
constraints are 
\GW{of minor importance}
in the
S2HDM~\cite{Biekotter:2022bxp}.
We nevertheless applied the
currently strongest
spin-independent
cross section limits for the scattering of
$\chi$ on nucleons obtained by the LZ
collaboration~\cite{LZ:2022ufs},
where we \GW{used} the one-loop predictions
of the scattering cross sections as computed
in \citere{Biekotter:2022bxp}.\footnote{Dark-matter
indirect detection experiments can so far only
probe a very limited mass window of $m_\chi$
once the experimental upper limit on the
relic abundance is applied~\cite{Biekotter:2021ovi}.
Thus, we do not
consider additional constraints from
indirect-detection experiments.}

