% SIAM Shared Information Template
% This is information that is shared between the main document and any
% supplement. If no supplement is required, then this information can
% be included directly in the main document.


% Packages and macros go here
\usepackage{lipsum}
\usepackage{amsfonts}
\usepackage{graphicx}
\usepackage[caption=false,farskip=0pt]{subfig}
\usepackage{epstopdf}
\usepackage{algorithmic}
\ifpdf
  \DeclareGraphicsExtensions{.eps,.pdf,.png,.jpg}
\else
  \DeclareGraphicsExtensions{.eps}
\fi

% Add a serial/Oxford comma by default.
\newcommand{\creflastconjunction}{, and~}

% Used for creating new theorem and remark environments
\newsiamremark{remark}{Remark}
\newsiamremark{hypothesis}{Hypothesis}
\crefname{hypothesis}{Hypothesis}{Hypotheses}
\newsiamthm{claim}{Claim}
\def\dfrac{\displaystyle\frac}

% Sets running headers as well as PDF title and authors
\headers{Barycentric Interpolation Based on Equilibrium Potential}{Kelong Zhao, and Shuhuang Xiang}

% Title. If the supplement option is on, then "Supplementary Material"
% is automatically inserted before the title.
\title{Barycentrical Interpolation Based on Equilibrium Logarithmic Potential\thanks{Submitted to the editors DATE.
\funding{This work was funded by National Science Foundation of China (No. 12271528).}}}

% Authors: full names plus addresses.
\author{Kelong Zhao\thanks{School of Mathematics and Statistics, Central South University, Changsha 410083, Hunan, People's Republic of China 
  (\email{clonezhao.1994@gmail.com}).}
\and Shuhuang Xiang\thanks{School of Mathematics and Statistics, Central South University, Changsha 410083, Hunan, People's Republic of China 
  (\email{xiangsh@csu.edu.cn}) (corresponding author).}}

\usepackage{amsopn}
\DeclareMathOperator{\diag}{diag}


%%% Local Variables: 
%%% mode:latex
%%% TeX-master: "ex_article"
%%% End: 
