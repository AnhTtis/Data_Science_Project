\vspace{-0.1in}
\section{Decentralized Energy Selection Game}\label{sec:game}

In this section, we study the uncoordinated decisions of $N$ self-interested consumers in an energy sharing community, formulated as a non-cooperative decentralized Energy Selection Game (ESG).

\vspace{-0.2in}
\subsection{Energy Sharing Community Set Up}
\vspace{-0.05in}
We consider an energy sharing community, consisting of $N$ consumers $i \in \mathcal{N}= \{1,...,N\}$ with heterogeneous preferences. Each day is split into two time intervals, namely, daytime and nighttime. And for both time intervals of the day, the consumers in this community have access to multiple energy sources in order to serve their daily flexible loads.

%%%
Firstly, we consider that the community owns RESs that can produce energy during the daytime only and that can be consumed locally. More precisely, it is assumed that during the daytime interval, the consumers in the community have access to a limited RESs production $\mathcal{ER}>0$ (in energy units) at a low cost $c_{RES}$ (per unit of energy). If the total aggregate demand for RESs exceeds the available RESs production during the day, the community manager allocates the RESs production among the consumers based on the PA policy, which satisfies the well-accepted notion of \textit{proportional fairness}. Under PA, a consumer receives a share of the RESs production proportional to their demand for it. Secondly, the consumers in the community have access to energy from the grid, priced using time-of-use tariffs.
These prices are specified by an energy provider in order to (i) maximize the local consumption of RESs by the community, and (ii) incentivize the consumers to shift their flexible loads from peak hours during the daytime interval to the nighttime interval. We can express this tariff w.r.t. to the RESs cost, such that the daytime interval price is $c_{grid,d}= \gamma c_{RES}$, and the nighttime interval price is $c_{grid,n} = \beta c_{RES}$, with $\gamma>\beta>1$.

%%% assumptions
\vspace{-0.15in}
\subsection{Non-cooperative Game Formulation} \label{sec:game_def}
\vspace{-0.05in}
Based on the aforementioned set up, at the beginning of each day, each self-interested consumer $i$ independently decides during which time interval, i.e., daytime or nighttime, she will schedule her daily flexible load based on her type $\vartheta_i \in \Theta = \{1,...,M\}$, consisting of her daily flexible load $U_{\vartheta_i}$ and risk-aversion degree $0\leq \mu_{\vartheta_i} \leq 1$. %Each consumer type $\ell \in \Theta = \{1,...,M\}$ represents the daily flexible load $U_{\ell}$ and risk-aversion degree $\mu_{\ell}\in [0,1]$ of the consumers $i \in \ell$. Note that the preferences of each consumer may vary from day to day.
Once scheduled during a time interval, consumer loads cannot be interrupted or shifted to the other time interval. If a consumer schedules her flexible load during the day, she competes with other consumers to use the limited low-cost RESs and incurs a financial risk. Indeed, if too many consumers compete for RESs and their aggregate daytime demand exceeds the available RES capacity, the part of their load that cannot be covered by RESs must be bought from the grid during the same time interval at the high daytime tariff $c_{grid,d}$.
%%%%Note that these parameters are known at the beginning of each day, but may vary from day to day.

Therefore, we consider that a consumer $i$ may be willing to "risk" only part of her daily flexible load, defined as $E_{\vartheta_i} = \mu_{\vartheta_i}U_{\vartheta_i}$, when competing for the RESs. The parameter $\mu_{\vartheta_i}$ stands for the risk aversion degree of consumer $i$ of type $\vartheta_i$. Therefore, if a consumer decides to compete for RESs, she schedules a share of her daily flexible (deferrable) load equal to $E_{\vartheta_i}$ during the day, and the remainder of her daily flexible load $(1-\mu_{\vartheta_i})U_{\vartheta_i}$ is transferred to the next day that the game is played again. Note that whether or not the remaining flexible load, $(1-\mu_{\vartheta_i})U_{\vartheta_i}$, will be scheduled the next day depends on the outcome of the game the corresponding day. If this consumer decides not to compete for the RESs, her total daily flexible load $U_{\vartheta_i}$ is scheduled during the night. With these explanations, $\mu_{\vartheta_i}=1$ represents a risk-seeking consumer, and $0 \leq \mu_{\vartheta_i}<1$ a risk-conservative consumer.
%%
Note that these individual preferences may vary from day to day.
%%
This behavior represents consumers with a broad range of flexible loads, including shiftable appliances, such as washing machines that do not need to run every day, as well as EVs, water heaters, and batteries that do not need to be fully charged at the end of a given day. For instance EV owners would compute their minimum (inflexible) daily energy load, representing the energy needed to cover their transportation needs for the day, as well as their flexible energy load, representing the additional energy needed to fully charge their EV. Then, if they decide to compete for RESs, they may be willing to engage only part of their daily flexible load to mitigate the risk of paying for the high-priced peak-load production. At the beginning of the following day that they play this game, they would update their daily inflexible and flexible loads and their risk attitude based on their new state-of-charge and transportation needs.

In the following, we provide the mathematical definition of this uncoordinated \emph{Energy Selection Game (ESG)} for one single day.

\vspace{-0.05in}
\begin{definition}\label{def:energy_source_game}
An \emph{Energy Selection Game (ESG)} is a tuple
\\$\Gamma=(\mathcal{N}, \mathcal{ER},  \{A_{i}\}_{i\in\mathcal{N}}, \{\vartheta_{i}\}_{i\in \mathcal{N}}, \mathbf{r}, \{v_{A_{i},\vartheta_{i}}\}_{i\in \mathcal{N}})$, where:\\
$\bullet$ $\mathcal{N}=\{1,...,N\}$, is the set of energy consumers. 
\\
$\bullet$  $\mathcal{ER}>0$ is the limited RES capacity in energy units.
\\
$\bullet$ $A_i$ is the action of player $i$ taking values in the set of potential pure strategies $\mathcal{A}=\{RES,grid\}$. $\mathcal{A}$ consists of the choices to schedule the consumer's daily flexible load during the day to compete for RESs ($RES$) or schedule it during the night ($grid$).\\
$\bullet$ $\vartheta_{i} \in \Theta=\{1,...,M\}$ ($M \leq N$) is the type of consumer $i \in \mathcal{N}$, consisting of the daily flexible load $U_{\vartheta_i}$ and \textit{risk-aversion degree} $\mu_{\vartheta_i}\leq 1$. The risk-aversion degree represents the aversion of consumers to risk their entire daily flexible load during the day if playing $RES$. \\
$\bullet$ $\mathbf{r}=[r_1,...,r_{M}]^T$ is a distribution of the consumers types, with $0\leq r_{\ell} \leq 1$ being the probability that a consumer is of type $\ell \in \Theta$.
\\
$\bullet$ $\upsilon_{A_i,\vartheta_i}(.): \mathbb{R}^M \rightarrow \mathbb{R}$ is the cost function of player $i$ of type $\vartheta_i$ and with action $A_i$. 
\end{definition}
\vspace{-0.1in}
In this paper, we study the ESG equilibria under \textit{mixed strategies}. A \textit{mixed strategy} can be represented by a probability distribution $\mathbf{p}_{\vartheta_i}=[p_{RES,\vartheta_i}, p_{grid,\vartheta_i}]^T$, with $p_{RES,\vartheta_i}$ determining the probability that a consumer $i$ competes for RESs, and $p_{grid,\vartheta_i}$ the probability that she does not compete for RESs. This mixed strategy can intuitively be interpreted as the consumer "splitting" her daily flexible load between the day and the night, such that she schedules on average a load equal to $p_{RES,\vartheta_i}\cdot E_{\vartheta_i}$ during the day, and a load equal to $p_{grid,\vartheta_i}\cdot U_{\vartheta_i}$ during the night. We denote with $\mathbf{p}=[\mathbf{p_1}^\top;...;\mathbf{p_{M}}^\top]$ the vector of mixed strategies for all consumer types, i.e., a consumer $i$ of type $\vartheta_i$ with a mixed strategy $\mathbf{p}_{\bm{\vartheta_i}}$ plays the game by randomly selecting an action $a \in \mathcal{A} = \{RES,grid\}$, with probability $p_{a,\vartheta_i}$\footnote{Note that a \textit{pure strategy} is a special case of a mixed strategy where one action has a probability equal to 1 (and the remaining have 0).}.
%Then, each consumer $i$ tries to maximize its expected profit by scheduling a load equal to $E_{\vartheta_i}$ during the day with probability $p_{RES,\vartheta_i}$, and a load equal to $U_{\vartheta_i}$ during the night with probability $p_{grid,\vartheta_i}$. 
%


When choosing an optimal mixed strategy, each consumer $i$ has information on (i) the price parameters ($c_{RES}, \beta, \gamma$), (ii) the available RES production ($\mathcal{ER}$), (iii) their own type $\vartheta_i \in \Theta$ (i.e., $\mu_{\vartheta_i}$ and $U_{\vartheta_i}$), and (iv) the probability distribution of the other consumers' types, and therefore the maximum expected aggregate demand for RESs during the day (i.e., if all consumers compete for RESs with $p_{RES,\ell}=1, \forall \ell \in \Theta$) that is equal to $D^{Total} = N  \sum_{\ell \in \Theta} r_{\ell  }~\mu_{\ell} ~U_{\ell}$.


For notational simplicity, in the remainder of the paper, we introduce $\varepsilon_{\vartheta_i}=\frac{1}{\mu_{\vartheta_i}}$, such that the daily flexible load of consumer $i$ of type $\vartheta_i$ is $U_{\vartheta_i}=\varepsilon_{\vartheta_i} \cdot E_{\vartheta_i}$. Thus, $\varepsilon_{\vartheta_i}=1$ represents a risk-seeking consumer $i$, and $\varepsilon_{\vartheta_i}>1$ a risk-conservative consumer. Finally, we assume without loss of generality that $E_1\leq E_2 \leq ...\leq E_{M}$.


\vspace{-0.15in}

\subsection{Individual Cost Functions}\label{sec:costs} \label{sec:policies}
\vspace{-0.05in}
Based on the above game definition, the expected aggregate demand for RESs during the day can be expressed as

\vspace{-0.05in}
\begin{small}
\begin{equation}
    D (\mathbf{p}) = N\sum_{\ell\in \Theta} r_{\ell }~p_{RES,\ell}~E_{\ell}. \label{eq:demand}
\end{equation}
\end{small}
\vspace{-0.1in}

\noindent Under the chosen PA policy, if this expected aggregate demand exceeds the available RES production $\mathcal{ER}$, a consumer $i$ of type $ \vartheta_i \in \Theta$ who plays the pure strategy $RES$ (i.e., $p_{RES,\vartheta_i}=1$) receives a share $rse_{\vartheta_i}^{PA}(\mathbf{p})$ of the RESs production proportional to their daytime energy demand $E_{\vartheta_i }$:

\vspace{-0.1in}
\begin{small}
\begin{eqnarray}
rse_{\vartheta_i}^{PA}(\mathbf{p}) &=& \frac{E_{\vartheta_i}}{\max(\mathcal{ER} , D (\mathbf{p}))}\mathcal{ER}.
\label{eq:prop_alloc_energy}
\end{eqnarray}
\end{small}
\vspace{-0.1in}

\noindent Note that \eqref{eq:prop_alloc_energy} depends on the strategies of all consumers $\mathbf{p}$.
%where $n_0,n_1$ denote the number of competitors %of low-to-moderate and high energy profile \textit{but} player $i$, respectively, as also defined in Section 
%\ref{sec:prop_alloc}, and $n_0+n_1+1$ is the total number of competitors.
%where $\sum_{\ell \in \Theta } %n_{\ell , i}+1$ is the total number of consumers competing for RES,where $ n_{\ell , i}$ is explained above Eq. \eqref{eq:demand_RES}. 
%Taking the minimum ensures that, no consumer $ i \in \mathcal{N}$ receives a share of the available RES capacity $\mathcal{ER}$ that is greater than its energy demand profile $E_{\vartheta_i}$.%, with $n_0,n_1$ as in Section \ref{sec:prop_alloc}.
As a result, the cost function $\upsilon_{RES,\vartheta_i}(.)$ for this consumer\footnote{Note that the remaining share of the daily flexible load $(\varepsilon_{\vartheta_i}-1)E_{\vartheta_i}$ is transferred to the next day and does not incur any cost.} can be expressed as

\vspace{-0.1in}
\begin{small}
\begin{align}
& \upsilon_{RES,\vartheta_i}(\mathbf{p}) =
rse_{\vartheta_i}^{PA}(\mathbf{p}) \cdot c_{RES} + (E_{\vartheta_i}-rse_{\vartheta_i}^{PA}(\mathbf{p}) ) \cdot c_{grid,d}.
\label{eq:RES_cost}
\end{align}
\end{small}
\vspace{-0.1in}

\noindent The cost $\upsilon_{grid,\vartheta_i}(.)$ for a consumer $i$ of type $\vartheta_i$ that plays the pure strategy $grid$ (i.e., $p_{grid,\vartheta_i}=1$) is 

\vspace{-0.1in}
\begin{small}
\begin{align}
\upsilon_{grid,\vartheta_i}=\varepsilon_{\vartheta_i} \cdot E_{\vartheta_i} \cdot c_{grid,n},
\label{eq:nonRES_cost2}
\end{align}
\end{small}
\vspace{-0.1in}

\noindent and solely depends on that consumer's strategy.