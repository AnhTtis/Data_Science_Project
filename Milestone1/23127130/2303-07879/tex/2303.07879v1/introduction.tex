\vspace{-0.18in}
\section{Introduction, Background \& Contributions}\label{sec:intro}
\vspace{-0.05in}
%In the pursuit of a connected community that meets the needs of governments and their employees, citizens and businesses, the endowment of the physical capital that incorporates mobile communications, multimedia services, data storage and ubiquitous computing should be integrated with the intellectual and social capital, namely all dimensions of human, collective, and artificial intelligence. In this respect, new insights into creating a data-driven approach to urban design and planning can be provided so that more reliable knowledge can be built and rapid reactions can be triggered. To this end, 
%Smart grids are envisioned as an advanced power network architecture that enhances the reliability, efficiency and safety of energy distribution as well as its conservation, through integrated metering, networking and control. 


The large scale penetration of distributed, stochastic and non-dispatchable Renewable Energy Sources (RESs) has
triggered the need for energy management solutions in distribution systems 
%in order to improve power systems efficiency, reliability and resilience
\cite{muruganantham2017challenges,alam2019energy}.
%%%
In the meantime, growing environmental and societal awareness, coupled with advances in metering and control technologies have allowed for a more active involvement of end-users in managing their energy consumption \cite{SCHWEIGER2020110359}.
%%%
In this context, energy communities, which locally coordinate distributed energy resources production and consumption, have become a viable and efficient solution to facilitate the integration of renewable energy sources (RESs) into distribution grids and reduce procurement costs for consumers \cite{abada2020viability}.


%%%%%%%%%% REFERENCES!!!!!!!!!!!!!!!!!!!!!!!!!!!!!!!!!!!!!!!!!!!!!!!!!

%%%
%In mart-grids, designing appropriate decentralized demand response programs (DRPs), which incentivize self-interested consumers to independently and in an uncoordinated manner adapt their consumption in response to price signals has become a central question \cite{deng15,akbari2020concept}.

Several works in the literature 
%demand response programs (DRPs) as an essential mechanism of smart-grids 
have shown the benefits of energy communities to reduce consumers' costs and increase energy justice by focusing on peer-to-peer (P2P) energy trading mechanisms \cite{sousa2019peer}. However, the development of these energy communities with consumer-owned RESs may be limited due to high investment costs \cite{rodrigues2020battery}. In contrast, recent regulatory changes have provided an unprecedented opportunity for the development of P2P energy sharing mechanisms, in which the community-owned RESs are allocated efficiently and fairly among the consumers \cite{roberts2020power}. Various works in the literature showed the potential economic benefits of these energy sharing communities, both for individual consumers and the community as a whole \cite{lowitzsch2019investing,minuto2022energy,jia16}.

%%
However, these energy sharing communities require the design of fair and efficient mechanisms to allocate the limited community resources among consumers who have equal claim to these resources but different levels of demand.
As highlighted in \cite{young1995equity}, due to the subjective nature of \textit{fairness}, various well-established notions of fairness, such as proportional and egalitarian, have been introduced in the literature and no allocation policy is universally accepted as "the most fair".
Additionally, different allocation policies satisfying one notion of fairness or another, may result in different levels of efficiency and stability. The work in \cite{bertsimas2011price} showed that allocation policies satisfying the notion of proportional fairness, such as the well-known \textit{proportional allocation (PA)} policy, may provide substantially higher efficiency and a lower "cost of fairness" than other axiomatically justified notions of fairness (e.g., egalitarian) by being more considerate to ``strong players'', i.e., consumers with high demand. Furthermore, the PA provides a trade-off between efficiency and fairness, since proportional fairness has been shown to be both Pareto optimal and a Nash bargaining solution \cite{boche2009nash}. 
%%
However, the PA may lack stability, as ``weak players'', i.e., consumers with low demand, may continuously change strategies to improve their allocation \cite{kulmala2021comparing}.
%%
On the other hand, the well-established \textit{equal sharing (ES)} allocation policy, which satisfies an egalitarian notion of fairness, is known to provide greater stability than the PA since it allows small players to be fully satisfied and prevents strong players from obtaining more resources than other players. Yet, ES may result in highly inefficient and wasteful utilization of energy resources. In the context of demand response programs (DRPs) and the allocation of renewable resources, this is a major limitation to the application of ES.

%such as reducing costs of energy production/transmission/distribution and increasing the integration of the environmental friendly renewable energy without increasing the cost of reserves
Additionally, in order to be beneficial for the overall system, the decisions of the consumers in energy sharing communities can be influenced by a third-party, e.g., an energy provider, in charge of interfacing with the market and system operator \cite{moret2018energy}. In particular, DRPs can harness the flexibility of the community and provide services to the grid (e.g., load shifting and peak-load reduction). A thorough literature review of the benefits and challenges of various DRPs can be found in \cite{pinson2014benefits,hussain2018review,deng15,akbari2020concept}. Most of these approaches focus on indirect DRPs, which, in contrast with direct load control, incentivize self-interested consumers to independently and in an uncoordinated manner adapt their consumption in response to price signals.
While indirect DRPs may not lead to a globally-optimal energy dispatch, they present benefits, namely (i) scalability and (ii) privacy awareness, which render them desirable compared to traditional direct load control approaches \cite{maharjan16,jacquot18}.
%%%%%%%%

%Decentralized DRPs have been studied at length in the literature. 
%In \cite{bitar17}, the authors introduce a DRP where consumers can reduce their energy procurement costs by requesting longer deadline periods, which provides flexibility to utility companies to satisfy demand. Contrary to our work, a centralized entity directly controls the deferrable loads so that the consumers' deadlines are respected and social welfare is maximized.
%Furthermore, there is no differentiation in pricing when RES or conventional generation are used to satisfy the demand, but, only the deadline determines the relative price paid by the consumer.

%%%
As consumers participating in energy communities are self-interested and their decisions are independent and uncoordinated, game theory is the most suitable tool to analyze the impact of these DRPs on consumers in energy communities \cite{mei2017engineering,abapour2020game}. In the literature, a focus has been placed on designing efficient price signals for individual consumers with deferrable loads using game theoretic tools. The dynamic pricing scheme introduced in \cite{Caron10} incentivizes consumers to adapt their load profile so that they are conveniently supplied by the providers. 
%This work highlights the importance of information (particularly of load profiles) sharing in order to reduce cost overheads. 
Similarly, the decentralized DRP developed in \cite{ibars2010distributed} uses non-cooperative game theory to design appropriate dynamic prices to control the grid load at peak hours. In \cite{Joe_Wong12}, prices are derived by formulating the electricity provider's cost minimization problem, which considers consumers' device-specific scheduling flexibility and the provider's cost of purchasing electricity. In addition, in \cite{Lulu2022}, a dynamic price DRP is proposed and analyzed based on Stackelberg games and with the goal to reduce the demand peaks. These game-theoretical frameworks developed in the literature rely on models of the consumers' decisions in response to price signals.


%The aforementioned works rely on the exchange of information between a centralized 3rd party entity and the consumers to design efficient pricing schemes. 
%Further works have focused on developing distributed control algorithms to achieve a Nash equilibrium (NE) in a decentralized manner in indirect DRPs while preserving privacy and independence of the decision of the consumers.
%In \cite{stephens15}, a multi-period DRP in presence of storage is performed via Model Predictive Control (MPC) over a Nash non-cooperative game. Under perfect information, the unique NE strategies are shown to be Pareto optimal. The use of forecasts breaks the condition of perfect information but the NE are re-computed using MPC with updated forecasts so as to be closer to Pareto optimality. Similarly, the work in \cite{jacquot18} thoroughly analyzes  multi-period decentralized DRPs with hourly prices via a game theoretic viewpoint and proposes MPC algorithms to reach a NE while respecting users' privacy.


While the aforementioned works show the existence of NE for various decentralized DRPs, they fail to derive closed-form solutions of these NE and to formally analyze the loss of efficiency resulting from the self-interested behavior of consumers compared to direct load control approaches. The authors in \cite{ma2011decentralized} have shown that the NE in a decentralized DRP for an infinite population of consumers and electric vehicles with identical technical characteristics and preferences is efficient. However, these assumptions are impractical and quite restrictive.
%The authors in \cite{jia16} compare the centralized or decentralized ownership and control of RESs and storage using Stackelberg game models. Yet, no bounds on the loss of efficiency is provided. 
%Further works have focused on quantifying the loss of efficiency in decentralized DRPs, using the so-called Price of Anarchy (PoA) metric. %resulting from the self-interested behavior of consumers compared to a centralized dispatch.
Similarly, the authors in \cite{chakraborty2017distributed} study a decentralized DRP in which the central coordinator aims at setting the optimal price signals for deferrable loads solely based on renewable energy production forecasts. A game-theoretic analysis of this mechanism shows the existence of NE both for price-taker and price-anticipating consumers, and derives a bound on the Price of Anarchy (PoA). However, the aforementioned papers focus on the design of decentralized DRPs in which consumers with homogeneous preferences compete for a single and unlimited resource across multiple time steps, and do not account for competition across multiple energy resources.


In view of this state-of-the-art, a gap remains to better understand the impact of decentralized DRPs on the self-interested behavior of consumers who compete for access to multiple energy sources in an energy sharing community.
%%%%%%%%%%%%%%%%%%%%% RELATED 
This paper differentiates itself from the existing literature by addressing these limitations.
The considered setting is similar to the multi-energy energy communities in \cite{maharjanuser,mitridati2021design}, and extends the preliminary work in \cite{stai2022}.
%%%%
More precisely, we consider self-interested consumers in an energy community, participating in (i) a decentralized P2P energy sharing mechanism, in which they are in competition with each other for access to the low-priced but limited production of the community-owned RESs, and (ii) a decentralized DRP, in which an energy provider defines time-of-use tariffs to incentivize them to shift their consumption from the grid across multiple time intervals. The consumers are considered to have heterogeneous preferences with respect to their energy demand and attitude towards risk. In the absence of a centralized dispatch, consumers independently decide during which time interval to schedule their flexible loads based on the prices and availability of different energy sources. This general resource selection problem with a ternary cost structure can model a wealth of resource selection cases beyond smart grids, such as in the case of parking resources \cite{kokolaki2013}.

Without adequately-designed price incentives and allocation policies for the different resources, the intuitive tendency of self-interested consumers to opt for low-priced but scarce RESs would lead to high demand for RESs, congestion, and lowered social-welfare. Therefore, an adequate decentralized DRP and P2P energy sharing mechanism should aim at incentivizing consumers to efficiently utilize the available resources. Additionally, to facilitate social-acceptance and consumers engagement, ensuring the \textit{fairness} of the P2P energy sharing mechanism in the community is also crucial in practice.

 
%Consumers who schedule their flexible loads during the day compete among each others for the low-priced but limited RESs production. Indeed, if the aggregate demand for RESs exceeds its available capacity, the available RES capacity is allocated fairly among consumers based on a given \textit{allocation policy}, and their excess daytime demand is purchased from the grid at a high daytime price. Alternatively, consumers may choose to schedule their flexible loads during the night and pay a risk-free medium night-time price. In essence, each consumer faces the dilemma of competing or not for a limited inexpensive resource based on limited availability, and time-dependent prices: if they compete and are successful they incur a low resource cost; if they compete and fail, they incur a high cost; if they decide not to compete, they incur a medium cost. 


%We consider that nonRES has much higher capacity (regarded as unlimited) than the maximum possible demand. Also, we assume that the consumers have flexible loads that can be time-shifted over the 24-hour time period of a day. If a load is engaged at some time, it cannot be interrupted (i.e., it should be totally served) and it will be served by these two available energy resources, i.e. RESs and nonRES, depending on the available capacity of each of them with priority given to RES. There is no storage, thus, the optimal usage of RES requires its immediate and local consumption. There exist two prices for nonRES: a high-price, which is effective over the day-zone, and a medium-price that is effective over the night-zone. % per unit of time (\ie a properly defined short time period). 
%Therefore, even if the RESs capacity can be predicted with high accuracy, each consumer will lack knowledge on the loads engaged by the other consumers, i.e., it is possible that the RESs capacity is not adequate to serve all the consumers. Consequently, if the aggregate load engaged by all consumers exceeds the available RES, the excess loads will be served by the power grid at the day-zone price.%according to the price in effect. If the time unit of engagement falls in the day-zone then the incurred cost for serving the excess loads would be higher than the one had the excess loads been shifted to the night-zone. 
%Ideally, if the consumers knew the available low-priced RES capacity, they would engage their loads when RES was available otherwise they would engage them during the medium-priced night zone. 

%In this paper we thoroughly investigate the distributed and uncoordinated demand-side decision making on engaging loads or not during the day-zone so that the utilization of RES is maximized and that of the nonRES is minimized. This is not a trivial decision for the consumer; if a consumer decides to compete for RESs, then he/she may end up paying the high-priced nonRES day-rate, in the case that the aggregate load exceeds the available RESs capacity, as explained before. %Essentially, the main dilemma faced by consumers is whether to compete for RES or not and pay for the medium-priced nonRES (night-rate). 
%Therefore, the consumer decisions are interdependent while consumers are strategic and want to minimize their energy consumption costs. In this context, game theory is the most suitable tool for modeling and analysis. Consumers are envisioned as rational and strategic selfish agents that try to minimize their energy consumption cost by deciding whether to compete or not for RESs. We assume that there exist a finite number of energy demand profiles; each consumer has its demand profile, but they do not share this knowledge with the rest of the consumers. Specifically, consumers are presented with probabilistic information about the energy demand levels and perfect information (e.g., accurate forecats) about the RES capacity. In addition, a-priori known pricing of all available energy options is assumed. Finally, different levels of consumers' risk attitude are considered by adjusting the demand that consumers gamble in the game for RESs according to their associated risk attitude level. 

%We assume that our decision mechanism is applied once during the daytime and each consumer decides its demand to be engaged during the day-zone and the one to be engaged during the night-zone. However, this is not a restrictive assumption, as it can be applied multiple times during the day-zone, e.g., in a model predictive control fashion, after updating the demand levels and the forecasts. Two important features of our distributed, uncoordinated DRP are (i) its scalability and (ii) its privacy awareness since it does not require private-information sharing among consumers \cite{maharjan16}. According to \cite{jacquot18}, it is a common belief that distributed mechanisms are essential in presence of an increasing number of user appliances and due to the user privacy concerns.

% In the absence of coordination, potential consumers independently and selfishly decide whether or not to engage their loads when renewable energy is produced, aiming at taking advantage of its low-price. In their decisions they factor in the inherent competition for the renewable energy and the risk of ending up being supported by the high-priced main power distribution network instead, if the total engaged load exceeds the limited renewable energy capacity. Alternatively, consumers could decide to refrain from competing for the low-priced energy (if chances are that demand would exceed its available capacity) and plan to engage their loads over known periods when (fossil-fuel-based) energy is medium-priced (\eg night rate) and unlimited.


%Consumers are envisioned as rational and strategic selfish agents that try to minimize the cost for the acquired energy by deciding whether to compete or not for renewable energy to serve their demand during particular time periods. In their decisions, they factor in the inherent competition for the low-priced energy facilities and the risk of incurring the associated congestion cost effects if the demand load exceeds the renewable energy capacity. In fact, we consider automatic agent implementations rather than human decision-makers yet, the actual human incurs the cost associated with machines' suggestions with the assumption that he fully complies with them. The different possible demand levels are abstracted in two broader classes of consumers also accounting for different levels of consumers' risk attitude that amount to adjusting the demand that consumers gamble in the game according to the level of risk undertaken by each choice. Through smart metering, decision-makers are presented with probabilistic information about the levels of demand and perfect information about the renewable energy capacity. In addition, a priori known pricing is implied, as opposed to real-time pricing, allowing consumers to further decrease the uncertainty in their decisions. 



%Based on the aforementioned state-of-the-art, a remaining gap in the literature is to design and study the efficiency of decentralized DRPs with competition over multiple energy sources, and heterogeneous consumer preferences. Therefore, 


Given the described research gaps, the contributions of this paper are the following:\\
$\bullet$ Firstly, we present a novel game-theoretic formulation of a decentralized DRP and P2P energy sharing mechanism in which consumers in an energy community compete for multiple energy resources over different time intervals. We model a wide range of consumers with heterogeneous preferences, namely energy demand and risk attitudes. The defined game provides a novel application for the PA policy with multiple energy sources.
We theoretically analyze it and derive closed-form expressions for its stable operational points (NE).
To the best of our knowledge, this paper is the first to propose, thoroughly analyze and evaluate a game-theoretic framework for the distributed, uncoordinated competition of consumers across different energy sources and time intervals.\\
$\bullet$ Secondly, we formulate the centralized coordinated energy source allocation mechanism that minimizes the social-cost for all consumers within the community; the results serve as a benchmark for assessing the efficiency of the decentralized game-theoretic mechanism. We quantify the efficiency of the distributed, uncoordinated energy selection using the PoA metric with respect to the benchmark centralized solution. \\
$\bullet$ Third, we provide a distributed, uncoordinated, iterative algorithm for choosing consumers' decisions so that they coincide with those prescribed by a NE. By choosing actions using our proposed algorithm, consumers do not need to reveal privacy-sensitive information such as their individual loads and constraints.\\
$\bullet$ Finally, we provide thorough numerical studies and comparisons with emphasis on the PoA metric. The efficiency of the proposed decentralized sharing mechanism is compared to that of the ES allocation policy. Additional fairness properties introduced via the distributed algorithm's design are studied.


The rest of the paper is organized as follows. %In Section \ref{sec:related}, we position our paper within the related literature. 
In Section \ref{sec:game}, we introduce the game-theoretic modeling approach. In Section \ref{sec:uncoordinated_extra_demand}, we study the NE mixed strategies under different parameter values and the corresponding expected aggregate demand and expected aggregate cost. In Section \ref{sec:coordinated}, we investigate the solution via a centralized mechanism. Section \ref{sec:algorithms} provides a distributed, uncoordinated algorithm with which the players can choose NE mixed strategies. In Section \ref{sec:eval}, we perform numerical evaluations and comparisons. Finally, Section \ref{sec:conclusions} concludes the paper. 
%In Section \ref{sec:game} we introduce a generic game-theoretic framework for the energy resource selection adopting abstract references to normalized cost and energy values. Two different policies for renewable energy allocation are devised and presented in Section \ref{sec:policies} enabling varying notions of fairness and affecting both the stability and efficiency of the energy allocation. In Section \ref{sec:uncoordinated_extra_demand}, % \ref{sec:uncoordinated_no_extra_demand} and \ref{sec:uncoordinated_extra_demand}, 
%we derive the stable operational states in which all competing influences are balanced assuming consumers of various risk-attitudes. The efficiency of the uncoordinated energy demand process is assessed in Section \ref{sec:coordinated} by deriving the related Price of Anarchy values, which compare the induced equilibrium social cost against the optimal one, \ie the social cost under an ideal centralized mechanism possessing perfect information about energy availability and the associated pricing, enabling coordination of the distributed decisions, reserving and allocating energy resources. Numerical results are provided in Section \ref{sec:results}. Indeed, the associated Price of Anarchy results reveal deviations from optimal that grow with increasing price differentials between energy sources and decreasing level of risk for risk-seeking consumers, whereas under risk-conservative attitude, they suggest conditions that generate a counteracting effect between fairness and efficiency in energy allocation. Ultimately, we explore decentralized coordination mechanisms that cope with the emerging inefficiencies through strategic dimensioning of energy source capacity and manipulation of pricing. We close in Section \ref{sec:conclusions} presenting collectively major conclusions and outlining possible directions for future research.}



%The integration of advances in power systems, Information and Communication Technologies and networking services have transformed power networks into large-scale, automated cyber-physical systems paving the way for the so-called \textit{smart grids}. The inherited heterogeneous and widely distributed nature of smart grids both in the energy supply and demand side motivates various technical challenges with respect to the efficiency of their design and the reliability of their implementation. In this paper, we consider multi-energy environments consisting of smart-grids that foster low-priced but limited-capacity renewal energy facilities and operate in conjunction with the higher and demand-response priced, yet free of capacity constraints, main power distribution network. Consumers with asymmetric load levels are viewed as strategic and rational decisions-makers that seek to minimize the cost of the acquired energy by deciding whether to opt for renewal energy and gain the price differentials between the two energy sources, factoring in their decisions the inherent competition for the low-priced energy facilities and the risk of incurring the associated congestion cost effects. In this respect, we investigate the efficiency of the uncoordinated energy source selection through a game-theoretic demand-side management framework. Four different game variants capture consumers' interaction when (a) they decide under two different levels of risk-aversion and (b) the smart-grid operator(s) employ(s) two different renewal energy allocation policies. We analyze the equilibrium outcomes and compared them against coordination systems that optimally regulate the load by either a centralized reservation mechanism or a decentralized mechanism providing incentives to shift consumption to less-loaded energy facilities. We compute the related Price of Anarchy metric and provide insights to the dynamics of consumers' strategies and theoretical support for the practical management of energy sources with capacity limitations.

