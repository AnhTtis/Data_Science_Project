
\vspace{-0.1in}
\section{Numerical Evaluations}
\label{sec:eval}
\vspace{-0.05in}
\subsection{Case Study Setup}
\vspace{-0.03in}
We consider a smart grid with $N=1000$ consumers, divided into $5$ distinct consumer types, with a maximum total demand for RESs $D^{Total} = 4250$ kWh. Table \ref{tab:residential} summarizes the consumer types parameters.
\begin{table}[t]
    \centering
    \begin{small}
    \begin{tabular}{|c||c|c|c|c|c|}
        \hline
        Type $\ell$ & 0 & 1 & 2 & 3 & 4 \\
        \hline 
        $E_\ell$ (kWh)& 2 & 3 & 5 & 10 & 15 \\
        \hline
        $r_\ell$ & 0.20 & 0.40 & 0.30 & 0.07 & 0.03 \\
        \hline
    \end{tabular}
    \caption{Game parameters for residential smart-grid.}
    \label{tab:residential}
    \end{small}
    \vspace{-0.2in}
\end{table}
The consumer type distribution and the daytime energy demand levels are selected to be consistent with European households \cite{enerdata}. Most households are moderately energy efficient (types $1$ and $2$), combined with many highly efficient households (type $0$) and few inefficient ones (types $3$ and $4$). Consumers of type $0$ are assumed to be risk-seeking ($\varepsilon_0=1$) and the risk-aversion degrees of all other types are determined by  \eqref{eq:relation_E_0_E_1_pa_ne_extra_demand}, but are close to $1$. We set the RES price as $c_{RES}=1$ \euro/kWh.


The proposed DRP with the PA policy is compared to a DRP with the ES policy for reference. Under ES, a so-called \textit{fair share} of RESs capacity is computed as

\vspace{-0.1in}
\begin{small}
\begin{align} \label{eq:fairshare}
 sh(\mathbf{p^{NE}}) 
& =\frac{\mathcal{ER}}{ N \sum_{ \ell \in \Theta} r_{\ell} ~ p_{RES,\ell}^{NE}}.
\end{align}
\end{small}
\vspace{-0.1in}

\noindent 
%where $  N \sum_{ \ell=0}^{M-1}  r_{\ell}  p_{RES,\ell}^{NE}$ represents the number of consumers competing for $RES$ during the day.
Under ES, consumers of type $\ell \in \Theta$ that play $RES$ and have a daytime demand $E_{\ell} \leq sh(\mathbf{p^{NE}})$ are allocated their full daytime demand $E_{\ell} $, as well as an extra energy equal to $sh(\mathbf{p^{NE}})-E_{\ell}$ that will remain unused. %The remaining RESs capacity is allocated equally between the consumers of type $\ell \in \Theta$ that play $RES$ and have a demand $E_{\ell}$ larger than the fair share $sh(\mathbf{p^{NE}})$. 
On the contrary, the consumers of type $\ell \in \Theta$ that play $RES$ and have a daytime demand $E_{\ell}>sh(\mathbf{p^{NE}})$ will be allocated the fair share and their remaining daytime energy demand $E_{\ell}-sh(\mathbf{p^{NE}})$ will be served by the highly priced peak-load generation. Therefore, the share of RESs received by a consumer $i$ of type $ \vartheta_i \in \Theta$ that plays $RES$ is $rse^{ES}_{\vartheta_i}(\mathbf{p^{NE}}) = \min\left( E_{\vartheta_i}, sh(\mathbf{p^{NE}}\right))$. Note that this allocation policy may result in large inefficiencies due to unused RES capacity, even when the total aggregate demand for RESs $D (\mathbf{p^{NE}})$ is higher than $\mathcal{ER}$. Therefore, this allocation policy is solely used as a base-case comparison to the PA allocation. The definitions and/or analysis of the ESSG and the centralized EAM under the ES policy are provided in Appendix C.

Finally, the convergence properties of the proposed decentralized algorithm under PA are studied for three capping systems, namely, (i) equal cap: $cap$ stays constant and equal to $0.1$; (ii) random cap: $cap$ is sampled from the uniform distribution $cap \sim U (0,1)$ (evaluated over multiple trials with varying values of $cap$); and (iii) no cap: equivalent to $cap=1$.

\vspace{-0.15in}
\subsection{Numerical Results}

\subsubsection{Social Cost and PoA under Varying Parameters}
\vspace{-0.07in}

The first set of numerical evaluations studies the proposed DRP under various values of the price parameters $\beta = \{2 , 2.5 \}$ and $\gamma = 3$, and varying available RES capacities $\mathcal{ER}$, ranging from $5\%$ to $125\%$ of $D^{Total}$.


\begin{figure}[ht!]
\vspace{-0.1in}
     \centering
     \subfigure[Social Cost (in eurocents). \label{fig:eval_1a}]{
         \includegraphics[width=0.23\textwidth]{plots/social_cost_PA_residential.eps}}
  \subfigure[PoA. \label{fig:eval_1b}]{
         \includegraphics[width=0.23\textwidth]{plots/PoA_PA_residential.eps}}
        \caption{Social cost and PoA under PA rule for residential grid with $\beta=2$.}
        \label{fig:eval_1}
        \vspace{-0.2in}
\end{figure}

As illustrated in Fig. \ref{fig:eval_1a}, the optimal social cost (given by Eq. \eqref{eq:social_cost_pa_extra_demand_2}) given by the centralized mechanism, denoted by OPT, decreases linearly with $\mathcal{ER}$. Indeed, since all risk-aversion degrees are equal or close to $1$, the cost function can be approximated as {\small $ \mathcal{ER} (1-\gamma) c_{RES} +N \sum_{\ell \in \Sigma_2} \left[ r_{\ell} E_{\ell} \left(\gamma -  \beta \right) p^{OPT}_{RES,\ell} \right] c_{RES}+ D^{Total} \beta c_{RES} \approx \mathcal{ER}(1-\beta) c_{RES}+ D^{Total} \beta c_{RES}$}, which is constant with respect to the competing probabilities and linearly decreasing with $\mathcal{ER}$. Note that since $\gamma=3$, $\beta=2$ and all risk aversion degrees are close to $1$, all consumers belong in the set $\Sigma_2$. Furthermore, we have observed that the minimization by the centralized mechanism results in "big players" competing for RESs at the expense of smaller ones. It is indeed observed that consumers with lower daytime energy demand compete for RES with non-zero probability only if there is remaining RES capacity when all consumers with higher daytime energy demand compete for RES with probability 1. This is aligned with the theoretical solution of the centralized mechanism in Section \ref{sec:centralsol}, since given that consumers are ranked with increasing daytime demand, according to Remark 2, the larger the player is the lower her risk aversion degree should be. Thus, larger players are prioritized in getting the highest probabilities values for competing for RES also according to the theoretical analysis.   %with probability $1$. %with type $4$ start competing for RES and only when their competing probability reaches $1$ (as $\mathcal{ER}$ increases) consumers with type $3$ start competing and so on for consumers of types 2 and 1.
%the night-time cost is mostly affected by the energy demand of each type. This explains why for values of $\mathcal{ER} < \mathcal{ER}_{max}$, %the centralized approach favours the types with the highest demand, namely, as $\mathcal{ER}$ increases, consumers with type $4$ start competing for RES and only when their competing probability reaches $1$, only then consumers with type $3$ start competing. Following the trend, type $2$ consumers compete for RES only when type $3$ consumers compete with probability equal to $1$, and so on. Therefore, 
%the PA rule indeed favours "big players" at the expense of smaller ones at the optimal allocation. %As a final note, naturally, for $\mathcal{ER} \geq \mathcal{ER}_{max}$, RES capacity adequately covers all the consumers, thus consumers of all types compete for RES.

On the other hand, as seen in Fig. \ref{fig:eval_1a}, for the decentralized DRP under the PA rule, the social cost at NE (denoted as EQ) decreases linearly with $\mathcal{ER}$ only for $\mathcal{ER}\in [0.5 D^{Total}, D^{Total}]$. The social cost is almost constant with the initial increase in RES capacity due to the fact that consumers tend to over-compete for RES, even for lower values of RES capacity, as it is observed in the obtained values of the competing probabilities. However, for $\mathcal{ER}\in [0.5 D^{Total}, D^{Total}]$, the social cost starts decreasing when $\mathcal{ER}$ increases, because there exists less excess demand for RES and thus the amount of required highly priced daytime non-RES energy is reduced.
%%%%%%%%%% ?????????????????????????????????????????????????
%Note that, the mixed strategy NE solution should satisfy \eqref{eq:probrelation1}, except if this does not give acceptable probability values. Since the maximum value that the right hand side of \eqref{eq:probrelation1} can take is $A^{Total}=D^{Total}- \sum_{ {\vartheta_l}\in \Theta} r_{\vartheta_l} E_{\vartheta_l} $ (when all competing probablilities are equal to $1$), this requirement is only possible if the left-hand side is less or equal than $A^{Total}$. With the applied parameter values, the left-hand side of \eqref{eq:probrelation1} is equal to $A^{Total}$ for 
%$\mathcal{ER} \approx 50\% D^{Total}$. Therefore, for $\mathcal{ER} \geq 50\% D^{Total}$ \eqref{eq:probrelation1} cannot be strictly satisfied and the closest we can get to its satisfaction is by setting the competing probabilities of all consumers equal to $1$. Thus, all consumers compete for RES with probability equal to $1$ and there is extra demand, which costs the high daytime prices. 
%%%%%%%%%% ?????????????????????????????????????????????????

As illustrated in Fig. \ref{fig:eval_1b}, the PoA values are rather small for all values of $\mathcal{ER}$. The PoA peaks for $\mathcal{ER} \approx 0.5 \cdot D^{Total}$, which is the point at which the social cost for the uncoordinated mechanism begins decreasing. This graph can provide valuable insights into how much RES capacity should be installed to increase the efficiency of the outcomes of the decentralized DRP. We can identify two zones of high efficiency, namely for low and high RES capacity. In the first zone, this is due to the small gains in cost offered by low RES capacity. In the second, the NE solution has almost converged to the optimal solution and thus social costs are optimal. 

In addition, the value of $\mathcal{ER}$ at which the PoA reaches its peak (most inefficient outcome) depends on the system model parameters and most importantly on the price parameters $\beta$ and $\gamma$. %since the left-hand side of \eqref{eq:probrelation1} depends on the quantity $\frac{(\gamma-1)}{(\gamma-\varepsilon_{\vartheta_i}\beta)}$. 
In paticular, from Fig. \ref{fig:eval_2a} we observe that both the centralized and the uncoordinated social
costs are higher for a value of the parameter $\beta=2.5$, compared to $\beta=2$ (Fig. \ref{fig:eval_1b}), because it results in higher night-time costs.
\begin{figure}[t] 
\vspace{-0.1in}
     \centering
     \subfigure[Social Cost (in eurocents). \label{fig:eval_2a}]{
         \includegraphics[width=0.23\textwidth]{plots/social_cost_PA_residential_beta.eps}}
  \subfigure[PoA. \label{fig:eval_2b}]{
         \includegraphics[width=0.23\textwidth]{plots/PoA_PA_residential_beta.eps}}
        \caption{Social cost and PoA under PA rule for residential grid with $\beta=2.5$.}\vspace{-0.2in}
        \label{fig:eval_2}
       % \vspace{-0.1in}
\end{figure}
Moreover, for $\beta=2.5$, the uncoordinated social cost curve starts decreasing at lower values of available RES capacity, namely
at $\mathcal{ER} = 20 \% \cdot D^{Total}$. However, as seen in Fig. \ref{fig:eval_2b}, the PoA attains significantly lower values for higher $\beta$ and peaks at
around $1.16$. Therefore, when the night time cost increases, the uncoordinated DRP behaves closer to the optimal solution. More results on how the price changes (via the parameters $\gamma$ and $\beta$) affect the NE can be found in \cite{stai2022}.


 






Finally, in Fig. \ref{fig:eval_poarisk}, the PoA is compared for different values of risk aversion of the energy community with $\beta=2$.  In particular, the inverse risk aversion degree of consumers of type $0$ are set to values between $\varepsilon_0=1$ and $\varepsilon_0=2$ (as indicated in the legend) and the risk-aversion degrees of all other consumer types are determined by  \eqref{eq:relation_E_0_E_1_pa_ne_extra_demand}. It turns out that all risk-aversion degrees are either equal or very close to $\varepsilon_0$, and, thus, the consumers in the energy community have all approximately the same risk aversion. It can be observed that as consumers become less risk seeking (i.e., $\varepsilon_{\ell}$ increases and thus $\mu_{\ell}$ decreases), the PoA values decrease for all $\mathcal{ER} /D^{Total}$ ratios exceeding $50\%$ in this plot. Thus, our proposed distributed scheme reveals that the achieved social cost of a less risk seeking community moves closer to the optimal for all possible NE and in particular, for $\varepsilon_{\ell}\geq 1.5$ (or for $\mu_{\ell}\leq 0.67$) the PoA values are optimal (i.e., equal to $1$) for all values of $\mathcal{ER} /D^{Total}$. As a conclusion, under conditions such as those associated with Fig. \ref{fig:eval_poarisk}, less risk seeking behavior by the community can yield NE inducing a social cost arbitrarily close to the optimal (PoA be reduced to as low as 1). Notice from Fig. \ref{fig:eval_poarisk} that a deviation of the social cost of about $33\%$ from the optimal social cost ($PoA=1.33$ for $\varepsilon_{\ell}=1$ and $\mathcal{ER} /D^{Total}$ ratio of $50\%$) can be entirely eliminated by adopting a less risk seeking behavior ( $\varepsilon_{\ell}\geq 1.5$).
 
 
 




\subsubsection{Comparison to ES Policy}
\label{sec:comptoES}
%%%%%%%%%%%%%%%%
As observed in Fig. \ref{fig:eval_3a}, both the centralized mechanism and the uncoordinated DRP yield higher social costs under the ES than under the PA for all values of RES capacity.
\begin{figure}[t] 
%\vspace{-0.1in}
     \centering
     \subfigure[Social Cost (in eurocents). \label{fig:eval_3a}]{
         \includegraphics[width=0.23\textwidth]{plots/social_cost_ES_residential.eps}}
  \subfigure[PoA. \label{fig:eval_3b}]{
         \includegraphics[width=0.23\textwidth]{plots/PoA_ES_residential.eps}}
        \caption{Social cost and PoA under ES for residential smart grid with $\beta=2$.}
        \label{fig:eval_3}
        \vspace{-0.2in}
\end{figure}

\begin{figure}[t] 
%\vspace{-0.1in}
     \centering
         \includegraphics[width=0.32\textwidth]{plots/PoAvsRisk.eps}
        \caption{PoA vs (inverse) risk aversion degree.}
        \label{fig:eval_poarisk}
        \vspace{-0.2in}
\end{figure}
This is due to i) the unused RES capacity by consumers' types whose demand for RES is lower than the fair share; and ii) the resulting increased daytime non-RES energy needed to cover the unsatisfied demand of consumers' types whose demand for RES is higher than the fair share.
In addition, with ES, even for $\mathcal{ER}=125 \% \cdot \mathcal{ER}_{max}$ and even for the centralized mechanism the competing probabilities may not be all equal to $1$. The mechanism may reduce the competing probabilities of smaller players in order to increase the RES utilization. The curves for both the centralized and uncoordinated cases decrease with increasing $\mathcal{ER}$, but not linearly contrary to the PA policy, due to the non-linearity of the cost functions with respect to $\mathcal{ER}$ under the ES policy. Moreover, we observe that the uncoordinated social cost under ES follows a similar trend as under PA, namely, it is constant for small values of $\mathcal{ER}$ and then starts to decrease. This shows that, similarly to the PA rule, consumers tend to over-compete for RES under the ES policy, especially for lower values of the RES capacity.

%%%%%%%%%%%%
Additionally, as seen in Fig. \ref{fig:eval_3b}, the ES policy achieves lower PoA than the PA policy for most values of the RES capacity. However, the decentralized DRP under ES achieves $100\%$ efficiency only when the RES capacity reaches $\mathcal{ER}=125 \% \cdot D^{Total}$, whereas, for the PA policy, the PoA is equal to $1$ for lower values of RES capacity $\mathcal{ER} \geq 110\% \cdot D^{Total}$. Hence, using the ES policy may be more expensive in case that $100\%$ efficiency of the DRP is required for which it requires more RES capacity than PA. Furthermore, due to the non-linearity of the social cost function with $\mathcal{ER}$, the PoA curve does not decrease monotonously after the initial peak.


%This is because the social cost for ES is not linear with $\mathcal{ER}$ (see for example \eqref{eq:social_cost_es_sc}). Moreover, we observe that the uncoordinated curve follows the same trend as the PA rule, namely, it is flat for small values of $\mathcal{ER}$ and then it decreases. This is because similarly to the PA rule, consumers tend to overcompete for RES also under the ES rule. %This is seen in the competing probability values obtained in our evaluations and can be explained analytically in a similar way as with the PA rule (using \eqref{eq:probrelationes1}). 
%Also, we should note that in the optimal allocation the competing probabilities may not be all equal to $1$ even for $\mathcal{ER}=125\%\mathcal{ER}_{max}$. We can attribute this observation to the fact that smaller players may need to reduce their competing probability in order to increase the RES share and subsequently the RES utilization, even if that means they forgo the chance to pay less. In addition, the social cost for ES is higher than for PA, regardless of the $\mathcal{ER}$ value. This can be easily explained by observing that, under the ES rule, either part of the RES capacity is unused because some consumers have lower demand than their share, or consumers are forced to pay the day-zone nonRES because their share is less than their demand. In both cases, the social cost increases. 


\subsubsection{Evaluation of Distributed Algorithm}

Here, we evaluate the performance and convergence of the distributed algorithm \ref{algorithm}. For easier visualization, we have implemented the algorithm in a smart grid with $N=500$ consumers divided into two consumer types, using the following parameter values: $E_0=100$ kWh, $E_1=200$ kWh, $D^{Total}=65000$ kWh, $r_0=0.7$, $r_1=0.3$, $\varepsilon_0=1$, $\varepsilon_1=1.004$, $c_{RES}=100$ \euro/kWh, $\beta=2$, $\gamma=4$, and $\mathcal{ER}=25\% \cdot D^{Total}=16250$ kWh.

Tables \ref{tab:algo_sc_PA} summarizes the evaluation results on the social cost, the demand for RESs and the PoA for the optimal centralized solution as well as for the solution of the distributed algorithm for the three capping systems.

\begin{table}[b] \vspace{-0.15in}
\begin{small}
    \resizebox{\linewidth}{!}{
        \begin{centering}
    \begin{tabular}{|c||c|c|c|c|}
        \hline
        & \textbf{Social Cost} ($10^6$) & \textbf{Demand} & \textbf{PoA} & \textbf{Number of steps} \\
        \hline 
        Centralized & 11.37 & 16250 & 1 & - \\
        %\hline
       % Worst case & 13.01 & 24324 & 1.14\\
        \hline
        Equal cap & 13.00 & 24321 & 1.14 & 18  \\
        \hline
        Random cap & 12.99 & 24286 & 1.14 & 17-27 \\
        \hline
        No cap & 13.01 & 24324 & 1.14 &  14 \\
        \hline
    \end{tabular}
    \end{centering}
    }
    \caption{Social cost, demand, PoA, and number of iterations until convergence under the PA rule for centralized and distributed algorithmic solutions.}
    \label{tab:algo_sc_PA}
    \vspace{-0.15in}
    \end{small}
\end{table}

All three capping methods lead to similar social cost and PoA values. Thus, the choice of capping method mostly influences the competing probabilities to introduce an additional fairness level for sharing the RES capacity among the consumer types, without affecting the social cost. To clarify, the fairness level introduced by the capping system is with respect to the mixed strategies level due to the fact that the order that consumers play has an influence; whereas the fairness of the allocation policy is with respect to the assignment of the available RES to those that finally compete for RES. Furthermore, Table \ref{tab:algo_sc_PA} highlights that if we do not apply a capping scheme the algorithm converges the fastest \footnote{The tolerance is set to $tol=10^{-4}$.} at the expense of fairness. This is because we do not restrict the rate at which the solution reaches a NE. Introducing a constant capping system slightly deteriorates convergence, but it stays within the same order of magnitude. Lastly, the random capping system provides no control over the convergence speed, and we observe a large variance in the required number of steps (outer loops) to convergence. Note that lower $cap$ values increase the required number of steps for convergence. Most importantly, for all three capping systems, we observe that the number of steps until convergence is much lower than the number of players ($N=500$), which showcases the efficiency of the algorithm.


Figure \ref{fig:eval_4a} illustrates the solution paths given by the distributed algorithm for all three capping systems. It can be observed that all solution paths converge to a theoretically proven NE, represented by the blue line. For the constant cap ($cap=0.1$), the solution path oscillates around the $45^{\circ}$ line. Therefore, the achieved NE solution consists of similar competing probability values for both consumer types. Lower constant $cap$ values increase fairness among consumer types, and greatly dampen any bias towards any type. If the random cap method is implemented, the solution path is naturally random. Lastly, with the no cap system, the consumer type that plays first gains a considerable advantage.


\begin{figure}[t] 
%\vspace{-0.1in}
     \centering
     \includegraphics[width=0.27\textwidth]{plots/algo_sol_small_grid_PA.eps}
     \caption{Decentralized algorithmic solutions for different capping systems.}\label{fig:eval_4a}
    \vspace{-0.2in}
\end{figure}

