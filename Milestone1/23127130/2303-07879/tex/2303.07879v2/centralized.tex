\section{Centralized Energy Sharing Mechanism}\label{sec:coordinated}

In this section we study an ideal centralized scheduling problem, in which an energy community manager with perfect knowledge of the available energy sources and types of the consumers in the community, centrally schedules their daily flexible loads.

\subsection{Problem Formulation}

%Therefore, the aggregate amount of daytime loads is equal to $D^{Total}$. 

Based on the available information, the community manager aims at finding the optimal load schedule of each consumer type, which minimize the social cost of the community under the chosen PA and payment policy. The community's social cost $C^{PA}(\bm{p})$ can be expressed as a function of the \textit{expected} aggregate daytime energy demand ($D^d(\bm{p})$) and nighttime energy demand ($D^n(\bm{p})$) of the community (as defined in Section \ref{sec:game_def}), such that:

\vspace{-0.15in} 
\begin{small}
	\begin{align}
	C^{PA}(\bm{p}) &=  \min\{\mathcal{RE}, D^d(\bm{p})\} \cdot c^{RES} \nonumber \\
    &+ \max\{0,D^d(\bm{p})-\mathcal{RE}\} \cdot c^{grid,d} + D^n(\bm{p}) \cdot c^{grid,n},
\label{eq:social_cost_pa_extra_demand}
	\end{align}
\end{small}

\noindent where the probabilities $p^d_{\vartheta}$ and $p^n_{\vartheta}$ (as defined in Section \ref{sec:game_def}) can be interpreted as the proportion of consumers of type $\vartheta$ that the community manager schedules during daytime and nighttime, respectively.
Although this objective cost is non-convex, we observe that during daytime, for any expected aggregate load schedule, the community manager minimizes the cost from grid imports. Therefore, by introducing the optimization variable $D^{grid}$ representing the expected aggregate grid imports during daytime, we can write the community manager's optimal load scheduling problem under the PA policy as a linear optimization problem, as follows:

 \vspace{-0.1in} 
 \begin{small}
 \begin{subequations} \label{eq:social_cost_x_opt}
\begin{alignat}{2}
& \min_{\mathbf{p},D^{grid}} \ && c^{grid,d}  D^{grid} + c^{RES}  \left(N\sum_{\vartheta \in \Theta} r_{\vartheta} p_{\vartheta}^d E_{\vartheta} - D^{grid}\right) \nonumber \\
& \quad && + c^{grid,n}  N \sum_{\vartheta \in \Theta} r_{\vartheta} {p}_{\vartheta}^{n}U_{\vartheta} \label{eq:opt_1} \\
 & \text{s.t. } &&  p^{d}_{\vartheta} + p^{n}_{\vartheta} = 1 , \ \forall \vartheta \in \Theta, \label{eq:opt_2.1} \\
  & \quad && 0 \leq p^{d}_{\vartheta},  p^{n}_{\vartheta},  \ \forall \vartheta \in \Theta, \label{eq:opt_2.2} \\
 & \quad && D^{grid} \geq \mathcal{ER} - N \sum_{\vartheta \in \Theta} r_{\vartheta} {p}_{\vartheta}^{d} E_{\vartheta}, \label{eq:opt_3.1} \\
 & \quad && D^{grid} \geq 0. \label{eq:opt_3.2}
 \end{alignat}
 \end{subequations}
\end{small} \vspace{-0.1in}  

\noindent This optimization problem minimizes the social cost of the community \eqref{eq:opt_1}, subject to constraints on the daytime and nighttime probabilities \eqref{eq:opt_2.1}-\eqref{eq:opt_2.2}  as well as to lower bounds on the expected aggregate grid imports during daytime \eqref{eq:opt_3.1}-\eqref{eq:opt_3.2}.

\subsection{Solution Analysis} \label{sec:centralsol}

In the following, we provide insights and analytical formulations of the optimal solutions $\mathbf{p^{*}}$ of this centralized mechanism in different cases. The proofs are available in the {Appendix \ref{appendix:dual}} of \cite{arxiv_version}.
 
\subsubsection*{\textbf{Case $1$: $\bm{\mathcal{RE}}$ exceeds $\bm{D^{Total}}$}}

In this trivial case, the optimal solutions to the C-ESM is to schedule all consumers' daily flexible loads during daytime, such that $p^{d,*}_{\vartheta}=1$, $\forall \vartheta \in \Theta$, and the expected grid imports $D^{grid,*} =0$.

\subsubsection*{\textbf{Case $2$: $\bm{\mathcal{RE}}$ is lower than $\bm{D^{Total}}$}}

In this case, it is optimal for the centralized ESM to schedule loads during the day so that the total RES capacity is fully utilized. To perform the analysis, we use the two complementary subsets of consumer types, $\Sigma_1$ and $\Sigma_2$, as those are defined in Section \ref{sec:gameanalysis}. 

For all consumers whose type $\vartheta \in \Sigma_1$, it is optimal for the community to schedule them during daytime, such that $p^{d,*}_{\vartheta}=1 $. For the optimal load schedule of the remaining consumers whose type $\vartheta \in \Sigma_2$, we observe that the consumer types are scheduled during daytime in order of increasing risk aversion (i.e., decreasing $\varepsilon_\vartheta$), until the local RESs production is fully utilized. Therefore, the optimal competing probabilities for the consumers whose types are in $\Sigma_2=\{\tilde{\vartheta}^1, \tilde{\vartheta}^2, \dots, \tilde{\vartheta}^K \}$, can be expressed as:

 \vspace{-0.1in} 
 \footnotesize
\begin{align}
 & p^{d,*}_{\tilde{\vartheta}^k} = \max \Bigg\{ \min \Bigg\{ 1, \dfrac{\left( \mathcal{RE} - D^{Total}_{\Sigma_1} - N \sum_{i=1}^{k-1}r_{{\tilde{\vartheta}^i}} E_{{\tilde{\vartheta}^i}} p^{d,*}_{\tilde{\vartheta}^i}  \right)}{N r_{{\tilde{\vartheta}^k}} E_{{\tilde{\vartheta}^k} }}\Bigg\} , 0 \Bigg\} , \nonumber \\
&  \forall k \in \{1,...,K\},
\end{align}
\normalsize 
\vspace{-0.1in}  

\noindent where the consumer types in $\Sigma_2$ are ordered such that $\varepsilon_{\tilde{\vartheta}^1} \geq \varepsilon_{\tilde{\vartheta}^2} \geq ... \geq \varepsilon_{\tilde{\vartheta}^K}$.

%If $N\sum_{{\vartheta} \in \Sigma_1}r_{{\vartheta}} E_{{\vartheta}} \geq \mathcal{RE}$, i.e., the daytime consumption of the consumers whose type $\vartheta \in \Sigma_1$ fully utilizes the RES capacity, then the solution to this optimization problem is trivial, and for all consumers whose type $\vartheta \in \Sigma_2$, $p^{}_{RES,\vartheta}=0$.


\section{Efficiency Loss of D-ESM vs. C-ESM}\label{sec:efficiency}
The (in)efficiency of equilibrium strategies in the D-ESM compared to the optimal C-ESM solution is quantified by the Price of Anarchy (PoA) metric \cite{Koutsoupias09}, representing the ratio of the worst case social cost among all mixed strategy NE, denoted as $C^{PA,NE}_{WC}$, over the optimal minimum social cost of the C-ESM, such that:

%\vspace{-5pt}
 \vspace{-0.1in} 
 \small
\begin{align}
 \hspace{-5pt} \textit{PoA}  = \frac{C^{PA,NE}_{WC}}{C^{PA}(\mathbf{p^{^*}})}.
\label{eq:poa_pa}
\end{align}
\normalsize 
\vspace{-0.1in}  

First observe that $C^{PA}(\mathbf{p^{^*}})$ is uniquely determined for each particular case (Section \ref{sec:coordinated}). Now, in order to obtain $C^{PA,NE}_{WC}$ when there exist multiple possible NE, we can maximize the social cost $C^{PA}(\mathbf{p^{NE}})$ (Eq. \eqref{eq:social_cost_pa_extra_demand}) with respect to $\mathbf{p^{NE}}$. 


%Note that $C^{PA,NE}(\mathbf{p^{NE}})$ takes its optimal value (i.e, minimum value) when the night-time cost is minimized. This solution coincides with the optimal centralized solution and thus in this case PoA takes the optimal (unity) value. 


 %The last observation is the fact that the demand $D^{PA,NE}$ is constant with respect to $\mathbf{p^{PA,NE}}$. Also, in the special case of the risk-seeking consumers, from Eq. \eqref{eq:social_cost_pa_sc}, the social cost is constant with respect to the probabilities $\mathbf{p^{PA,NE}}$ for each case of energy profile values. Thus, $C^{PA,NE}_w$ is given by Eq. \eqref{eq:social_cost_pa_sc}.

%$2$. If having risk-conservative consumers, there exist multiple possible equilibria in the sub-cases 2(a) and 2(c) as well as in case 4. Of course, the multiple combinations of probabilities can lead to NE with possibly different social cost values. Based on the Remark \ref{rem:risk_degrees_relation}, the existence of NE is possible if consumers with lower energy demands have lower risk aversion degrees. Thus, the night cost is minimized if the optimal probabilities for RES take lower values for consumers with lower energy demands. 
