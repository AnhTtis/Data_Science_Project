\newpage

\section{Proofs for Case $2$ of the D-ESM}
\label{sec:proofsESSG}
\vspace{-0.05in}
For the consumers in $\Sigma_1$, we need to show that  $\upsilon^d_{\vartheta}(\mathbf{p})<\upsilon^n_{ \vartheta}(\mathbf{p})$, $\forall \mathbf{p}$ and $\forall \vartheta \in \Sigma_1$. Assume a consumer type $\vartheta\in \Sigma_{1}$ and that her allocated RES energy is $E'$. Then, we have that $\upsilon^d_{ \vartheta}(\mathbf{p})= E'  \cdot c^{RES}+(E_{\vartheta}-E')\cdot \gamma \cdot  c^{RES}$ and 
$\upsilon^n_{ \vartheta}(\mathbf{p})= \varepsilon_\vartheta \cdot E_{\vartheta} \cdot \beta \cdot c^{RES}$. The inequality $\upsilon^d_{\vartheta}(\mathbf{p})<\upsilon^n_{ \vartheta}(\mathbf{p})$ is then equivalent to $ E'  (1-\gamma) \cdot c^{RES} <  E_\vartheta \cdot (\varepsilon_\vartheta \cdot \beta -\gamma) \cdot c^{RES}$, which is true by assumption, since $(1-\gamma)<0$ and $(\varepsilon_\vartheta \cdot \beta -\gamma)>0$.

                                     
  Next, for the consumers in $\Sigma_{2,1}$, we need to show that $\upsilon^d_{ \vartheta}(\mathbf{p})>\upsilon^n_{ \vartheta}(\mathbf{p})$, $\forall \mathbf{p}$ and $\forall \vartheta\in \Sigma_{2,1}$. Assume a consumer type $\vartheta \in \Sigma_{2,1}$ and that her allocated RES energy is $E'$. Then, the inequality $\upsilon^d_{\vartheta}(\mathbf{p})>\upsilon^n_{ \vartheta}(\mathbf{p})$ is equivalent to the inequality $E_\vartheta >E' \frac{(\gamma-1)}{(\gamma-\varepsilon_\vartheta\beta)}$, which is true by assumption, since $E'<\mathcal{RE}$.

Now, we prove the condition of existence of a mixed strategies NE for the consumers in $\Sigma_{2,2}$. Recall that in the ESG under the PA policy, a mixed strategy NE, $\mathbf{p^{NE}}$, among consumers in $\Sigma_{2,2}$ exists under the condition
\vspace{-0.05in}

\small
\begin{equation}\label{eq:condition_PA_NE_2}
res_{\vartheta}^{PA}(\mathbf{p}^{NE}) =res_{\vartheta}^{NE}(\mathbf{p}^{NE}), \forall \vartheta \in \Sigma_{2,2}. \end{equation}
\normalsize
%Next we give the conditions such that either \eqref{eq:condition_PA_NE} holds and mixed NE exist or there exist dominant strategies. For this study, we distinguish cases with respect to the RES capacity, the risk aversion degree values and the daytime energy demand levels. 

To derive condition \eqref{eq:relation_E_0_E_1_pa_ne_extra_demand} we re-write \eqref{eq:condition_PA_NE} first with assuming that a consumer $i$ of type $\vartheta_i \in \Sigma_{2,2}$ plays the pure strategy $A_i=d$ (in \eqref{eq:probrelation1}) and second with assuming that a consumer $j$ with type $\vartheta_j \in \Sigma_{2,2} \setminus \{\vartheta_i\}$ plays the pure strategy $A_j=d$ (in \eqref{eq:probrelation2}):

\vspace{-0.1in}
\begin{small}
\begin{align}
&  \mathcal{RE}\frac{(\gamma-1)}{(\gamma-\varepsilon_{\vartheta_i}\beta)}-E_{\vartheta_i}= D^{Total}_{\Sigma_1}+\sum_{ {\vartheta'}\in \Sigma_{2,2}} r_{\vartheta'}~ (N-1)~E_{\vartheta'}~p^{d,NE}_{\vartheta'},
    \label{eq:probrelation1}\\
  &  \mathcal{RE}\frac{(\gamma-1)}{(\gamma-\varepsilon_{\vartheta_j}\beta)}-E_{\vartheta_j}=D^{Total}_{\Sigma_1}+  \sum_{ {\vartheta'}\in \Sigma_{2,2}} r_{\vartheta'} ~(N-1)~E_{\vartheta'}~ p^{d,NE}_{\vartheta'}.
    \label{eq:probrelation2}  
\end{align}
\end{small}

%Eq. \eqref{eq:condition_PA_NE} can be re-written in a similar way for any other type $\vartheta_j \in \Theta$. 
Note that to derive (\ref{eq:probrelation1}) we consider that if a consumer $i$ in $\Sigma_{2,2}$ of type $\vartheta_i$ plays  the pure strategy $A_i=d$, then, the aggregate expected daytime energy of the consumers in $\Sigma_{2,2}$, $D_{\Sigma_{2,2}}(\mathbf{p^{NE}})$ can be expressed as $E_{\vartheta_i}+ \sum_{ {\vartheta'}\in \Sigma_{2,2}} r_{\vartheta'}~ (N-1)~E_{\vartheta'}~p^{d,NE}_{\vartheta'}$ for a large number of consumers and similarly also for (\ref{eq:probrelation2}). Then, since the right-hand sides of \eqref{eq:probrelation1}-\eqref{eq:probrelation2} are equal, the left-hand sides will be also equal and \eqref{eq:relation_E_0_E_1_pa_ne_extra_demand} derives. %\eqref{eq:probrelation1}-\eqref{eq:probrelation2} 

To derive the probability bounds, we re-write \eqref{eq:condition_PA_NE} assuming that all consumers of the same type play the same mixed strategy, i.e., 

\vspace{-0.1in}
\begin{small}
\begin{align}
&  \mathcal{RE}\frac{(\gamma-1)}{(\gamma-\varepsilon_{\vartheta_i}\beta)}= D^{Total}_{\Sigma_1}+N\sum_{ {\vartheta'}\in \Sigma_{2,2}} r_{\vartheta'}~ E_{\vartheta'}~p^{d,NE}_{\vartheta'}.
    \label{eq:probrelation3}  
\end{align}
\end{small}

The minimum bound on the probability for competing for RESs, $p_{\vartheta}^{\min}$, derives by setting in (\ref{eq:probrelation3}) $p^{d,NE}_{\tilde{\vartheta}}=1$, $\forall \tilde{\vartheta}\in \Sigma_{2,2}$ with $\tilde{\vartheta}\neq \vartheta=\vartheta_i$. Similarly, the maximum bound on the probability for competing for RESs, $p_{\vartheta}^{\max}$, derives by setting in (\ref{eq:probrelation3}) $p^{d,NE}_{\tilde{\vartheta}}=0$, $\forall \tilde{\vartheta}\in \Sigma_{2,2}$ with $\tilde{\vartheta}\neq \vartheta=\vartheta_i$. 

Finally, the expression for the aggregate expected daytime energy demand given in \eqref{eq:demand1_2c} is constructed as follows. First we can write that 
\begin{align}
 D^{d,NE} = D^{Total}_{\Sigma_1} +N\sum_{ {\vartheta'}\in \Sigma_{2,2}} r_{\vartheta'}~E_{\vartheta'}~p^{d,NE}_{\vartheta'}. \label{eq:totdemand}
 \end{align}

Second, by multiplying \eqref{eq:probrelation1} with $\frac{N}{N-1}$, we obtain:


\begin{small}
\begin{align}
&  N\sum_{ {\vartheta'}\in \Sigma_{2,2}} r_{\vartheta'}~E_{\vartheta'}~p^{d,NE}_{\vartheta'}=\frac{N}{N-1}\left[\frac{\mathcal{RE}(\gamma-1)}{(\gamma-\varepsilon_{\vartheta_i}\beta)}-E_{\vartheta_i}-D^{Total}_{\Sigma_1}\right].
    \label{eq:probrelation3}
\end{align}
\end{small}

Third, by replacing \eqref{eq:probrelation3} in \eqref{eq:totdemand} we obtain \eqref{eq:demand1_2c}, where the $\min\{.\}, ~\max\{.\}$ operators account for the case that the initially obtained probability values by  \eqref{eq:probrelation1} do not lie in the range $[0,1]$ and should be set to the values $1$ or $0$, correspondingly. 

\section{Proofs for Case $2$ of the C-ESM}
\label{appendix:dual}

In this case, it is optimal for the C-ESM to schedule loads during the day so that the total RES capacity is fully utilized, i.e., the expected aggregate daytime energy demand is greater than or equal to the RES capacity:

 \vspace{-0.1in} 
 \small
\begin{align}
N \sum_{{\vartheta} \in \Theta}r_{{\vartheta}} ~E_{{\vartheta}}~p^{d}_{{\vartheta}} \geq \mathcal{RE}.
\label{eq:optimal}
\end{align}
\normalsize 
\vspace{-0.1in}


\noindent Therefore, the social cost reduces to:

 \vspace{-0.1in} 
 \begin{small}
\begin{align}
C(\mathbf{p}) &=  \mathcal{RE} \cdot c^{RES}   + \left[N \sum_{{\vartheta} \in \Theta} r_{\vartheta} ~p_{{\vartheta}}^{d}~  E_{\vartheta} - \mathcal{RE}\right]  \gamma \cdot c^{RES} \nonumber \\
& + N \left[ \sum_{{\vartheta} \in \Theta} r_{\vartheta } \left(1-p^{d}_{{\vartheta}}\right)\varepsilon_{{\vartheta} }~E_{{\vartheta}}\right] \beta \cdot c^{RES},
 \label{eq:social_cost_pa_extra_demand_2}
 \end{align}
\end{small} \vspace{-0.1in}

\noindent and the C-ESM optimization problem \eqref{eq:social_cost_x_opt} is equivalent to minimizing $ N \sum_{\vartheta \in \Theta} \left[ r_{\vartheta} E_{\vartheta} \left(\gamma - \varepsilon_{\vartheta} \beta \right) p^{d}_{\vartheta} \right] c^{RES}$, subject to constraints \eqref{eq:opt_2.1}-\eqref{eq:opt_3.2} and \eqref{eq:optimal}. Below, we derive closed-form expressions of the solutions of this linear optimization problem.

We define two complementary subsets of consumer types, depending on their risk aversion degrees: $\Sigma_1 = \Bigl\{\vartheta \in \Theta : \varepsilon_{\vartheta} \geq \gamma/\beta \Bigr\} \subset \Theta$, and $\Sigma_2 = \Bigl\{ \vartheta \in \Theta :1\leq \varepsilon_{\vartheta} < \gamma/\beta \Bigr\} \subset \Theta$.

For all consumers whose type $\vartheta \in \Sigma_1$, it is optimal for the C-ESM to schedule them during daytime, such that $p^{d,*}_{\vartheta}=1 $. Therefore, the optimal schedule for the remaining consumers whose type $\vartheta \in \Sigma_2$ can be found by solving the following linear optimization problem: 


%\vspace{-0.1in} 
 \begin{small}
 \begin{subequations} \label{eq:social_cost_x_opt_2}
\begin{alignat}{2}
& \min_{\mathbf{p}} \ &&  N \sum_{\vartheta \in \Sigma_2} \left[ r_{\vartheta} ~E_{\vartheta} \left(\gamma - \varepsilon_{\vartheta} \beta \right) p^{d}_{\vartheta} \right] c^{RES} \label{eq:opt_S2_1} \\
 & \text{s.t. } && \eqref{eq:opt_2.1}-\eqref{eq:opt_3.2} \label{eq:opt_S2_2}\\
 & \quad && N \sum_{{\vartheta} \in \Sigma_2}r_{{\vartheta}} ~E_{{\vartheta}}~p^{d}_{{\vartheta}} \geq \left( \mathcal{RE} - N \sum_{{\vartheta} \in \Sigma_1}r_{{\vartheta}} E_{{\vartheta}}\right). \label{eq:opt_S2_3} 
 \end{alignat}
 \end{subequations}
\end{small} %\vspace{-0.1in}

\noindent And the dual function of this optimization problem is 

\begin{footnotesize}
 \begin{align} \label{eq:social_cost_x_opt_2_dual}
 \max_{\lambda \geq 0}\min_{\mathbf{p}} \quad & N \sum_{\vartheta \in \Sigma_2} \left[ r_{\vartheta} E_{\vartheta} \left(\gamma - \varepsilon_{\vartheta} \beta \right) p^{d}_{\vartheta} \right] c^{RES} \nonumber \\&-\lambda\left( N \sum_{{\vartheta} \in \Sigma_2}r_{{\vartheta}} E_{{\vartheta}}p^{d}_{{\vartheta}} - \left( \mathcal{RE} - N \sum_{{\vartheta} \in \Sigma_1}r_{{\vartheta}} E_{{\vartheta}}\right)\right),
 \end{align}
\end{footnotesize} \vspace{-0.1in}

\hspace{-0.2in} subject to \eqref{eq:opt_S2_2}, where $\lambda$ represents the dual variable associated with \eqref{eq:opt_S2_3} and let $\lambda^*$ represent its optimal value.

It results that:\\
$\bullet$ for all $\vartheta \in \Sigma_{2}$ where $1 \leq \varepsilon_\vartheta < \dfrac{\gamma c^{RES} - \lambda^*}{\beta c^{RES}}$, $p^{d,*}_{\vartheta}=0$,\\
$\bullet$ for all $\vartheta \in \Sigma_2$ where $ \varepsilon_\vartheta = \dfrac{\gamma c^{RES} - \lambda^*}{\beta c^{RES}}$, $0<p^{d,*}_{\vartheta}<1$,\\
$\bullet$  for all $\vartheta \in \Sigma_2$ where $ \dfrac{\gamma c^{RES} - \lambda^*}{\beta c^{RES}} < \varepsilon_\vartheta <\dfrac{\gamma}{\beta}$, $p^{d,*}_{\vartheta}=1$.

This means that the consumer types are fully dispatched during the day in the order of increasing risk aversion degree (or decreasing $\varepsilon_\vartheta$), until constraint \eqref{eq:opt_S2_3} is satisfied. 


\section{Analysis For the ES Allocation Policy}
\subsection{Decentralized Energy Sharing Mechanism Under ES}
The analysis and proofs of this section follow similar lines as the analysis and proofs for the PA policy. Most proofs are however omitted for brevity.

In the ESG with the ES policy, a mixed-strategy NE exists under the condition:


 %\vspace{-0.1in} 
 \small
\begin{equation}\label{eq:condition_ES_NE}
rse^{ES}_{\vartheta_i}(\mathbf{p^{NE}}) =res_{\vartheta}^{NE}(\mathbf{p}^{NE}), ~\forall \vartheta \in \Theta. \end{equation}
\normalsize 
\vspace{-0.1in}  

\noindent 
%Therefore, for all cases, any existing mixed-strategy NE competing probabilities, $\mathbf{p}^{NE}$, are obtained by resolving condition \eqref{eq:condition_ES_NE}.
Let us distinguish the following cases:

\subsubsection*{\textbf{Case $1$: $\bm{\mathcal{RE}}$ exceeds $\bm{D^{Total}}$}}

As consumers have knowledge of $\mathcal{RE}$ and $D^{Total}$, it is straightforward to show that the dominant-strategy for all consumers is to schedule their daily flexible loads during daytime. As a result, the competing probabilities that lead to equilibrium states are equal to $p_{\vartheta}^{d,NE} = 1$ for all consumer types $\vartheta \in \Theta$.


\subsubsection*{\textbf{Case $2$: $\bm{\mathcal{RE}}$ is lower than $\bm{D^{Total}}$}} 

In this case, the strategies of the consumers depend on their respective risk aversion degrees and the TOU tariffs. We define two complementary subsets of consumer types, depending on their risk aversion degrees: $\Sigma_1 = \Bigl\{\vartheta \in \Theta : \varepsilon_{\vartheta} \geq \gamma/\beta \Bigr\} \subset \Theta$, and $\Sigma_2 = \Bigl\{ \vartheta \in \Theta :1\leq \varepsilon_{\vartheta} < \gamma/\beta \Bigr\} \subset \Theta$.


Firstly, the dominant strategy for all consumers $i$ whose type $\vartheta_i$ is in the set $\Sigma_1$ is to schedule their daily flexible loads during daytime, i.e., to play the pure strategy $A_i = d$ with probability $p_{\vartheta_i}^{d,NE} = 1$. Their expected aggregate daytime energy demand is then $D^{Total}_{\Sigma_1}=N\sum_{\theta \in \Sigma_1}r_{\theta} E_{\theta}$.

Secondly, the strategies of the consumers $i$ whose type $\vartheta_i$ is in the set $\Sigma_2$ depends on their daily flexible loads and risk-aversion degrees. Therefore, we define two distinct subsets of consumer types in $\Sigma_2$: $\Sigma_{2,1} = \left\{ \vartheta \in \Sigma_2 : E_{\vartheta} > \mathcal{RE}\frac{(\gamma-1)}{(\gamma-\varepsilon_{\vartheta}\beta)} \right\}$ and $\Sigma_{2,2} = \left\{ \vartheta \in \Sigma_2 : E_{\vartheta} \leq \mathcal{RE}\frac{(\gamma-1)}{(\gamma-\varepsilon_{\vartheta}\beta)}\right\}$.

%$\Sigma_{2,1} = \left\{ \vartheta \in \Sigma_2 : E_{\vartheta} > \left(\mathcal{RE}-D^{Total}_{\Sigma_1}\right)\frac{(\gamma-1)}{(\gamma-\varepsilon_{\vartheta}\beta)} \right\}$ and $\Sigma_{2,2} = \left\{ \vartheta \in \Sigma_2 : E_{\vartheta} \leq \left(\mathcal{RE}-D^{Total}_{\Sigma_1}\right)\frac{(\gamma-1)}{(\gamma-\varepsilon_{\vartheta}\beta)}\right\}$

For consumers $i$ whose type $\vartheta_i$ is in the set $\Sigma_{2,1}$, the dominant strategy is to schedule their daily flexible loads during nighttime, i.e., to play the pure strategy $A_i=n$ with probability $p^{n,NE}_{\vartheta_i}=1$, and $A_i=d$ with probability $p^{d,NE}_{\vartheta_i}=0$.

For consumers whose types are in the set $\Sigma_{2,2}$, a mixed-strategy NE with the ES policy exists if and only if the following condition holds:

 \vspace{-0.1in} 
 \small
\begin{equation}\label{eq:relation_E_0_E_1_es_ne_extra_demand}
(\gamma-\varepsilon_{\vartheta}\beta)\cdot E_{\vartheta} = (\gamma-\varepsilon_{\tilde{\vartheta} }\beta)\cdot E_{\tilde{\vartheta}} , \ \forall \vartheta , \tilde{\vartheta} \in \Sigma_{2,2}.
\end{equation}
\normalsize 


To derive condition \eqref{eq:relation_E_0_E_1_es_ne_extra_demand} we re-write \eqref{eq:condition_ES_NE} first with assuming that a consumer $i$ of type $\vartheta_i \in \Sigma_{2,2}$ plays the strategy $A_i=d$ with probability $p^{d,NE}_{\vartheta_i}=1$ (in \eqref{eq:probrelation1es}) and second with assuming that a consumer $j$ with type $\vartheta_j \in \Sigma_{2,2} \setminus \{\vartheta_i\}$ plays the strategy $A_j=d$ with probability $p^{d,NE}_{\vartheta_j}=1$ (in \eqref{eq:probrelation2es}).

\vspace{-0.1in}
\begin{small}
\begin{align} \label{eq:probrelation1es}
  D^{Total}_{\Sigma_1}+1+\sum_{ {\vartheta'}\in \Sigma_{2,2}} r_{\vartheta'}~ (N-1)~p^{d,NE}_{\vartheta'}=\frac{\mathcal{RE}(\gamma-1)}{E_{\vartheta_i}(\gamma-\varepsilon_{\vartheta_i}\beta)},
\end{align}
\end{small}
\vspace{-0.1in}

\begin{small}
\begin{align} \label{eq:probrelation2es}
  D^{Total}_{\Sigma_1}+1+\sum_{ {\vartheta'}\in \Sigma_{2,2}} r_{\vartheta'}~ (N-1)~p^{d,NE}_{\vartheta'}=\frac{\mathcal{RE}(\gamma-1)}{E_{\vartheta_j}(\gamma-\varepsilon_{\vartheta_j}\beta)}.
\end{align}
\end{small}
Then, since the right-hand sides of \eqref{eq:probrelation1es}-\eqref{eq:probrelation2es} are equal, the left-hand sides will be also equal and \eqref{eq:relation_E_0_E_1_es_ne_extra_demand} derives.


Additionally, for the consumers of type $\vartheta \in \Sigma_{2,2}$, the competing probabilities that lead to NE states lie in the range $p^{min}_{\vartheta} \leq p^{d,NE}_{{\vartheta}} \leq p^{max}_{\vartheta}$, where:

 \vspace{-0.1in} 
 \footnotesize
\begin{align}
&  p^{min}_{\vartheta}=\nonumber\\&\max \left\{0,\frac{\frac{\mathcal{RE}(\gamma-1)}{E_{\vartheta}(\gamma-\varepsilon_{\vartheta}\beta)}-
  \sum\limits_{\tilde{\vartheta} \in \Sigma_{2,2} \cup \Sigma_1 \setminus \{\vartheta\}}N r_{\tilde{\vartheta}} }{N r_{\vartheta}} \right\}, \label{eq:plminbound_appendix}\\
& p^{max}_{\vartheta} = \min \left\{1,\frac{\frac{\mathcal{RE}(\gamma-1)}{E_{\vartheta}(\gamma-\varepsilon_{\vartheta}\beta)}-\sum\limits_{\tilde{\vartheta} \in  \Sigma_1 }N r_{\tilde{\vartheta}} }{N r_{\vartheta}}\right\}.
    \label{eq:plmaxbound_appendix}
\end{align}
\normalsize  
\vspace{-0.1in}

To derive the probability bounds, we re-write \eqref{eq:condition_ES_NE} assuming that all consumers of the same type play the same mixed strategy, i.e., 

\vspace{-0.1in}
\begin{small}
\begin{align} \label{eq:probrelation3es}
  D^{Total}_{\Sigma_1}+\sum_{ {\vartheta'}\in \Sigma_{2,2}} N~r_{\vartheta'}~p^{d,NE}_{\vartheta'}=\frac{\mathcal{RE}(\gamma-1)}{E_{\vartheta_i}(\gamma-\varepsilon_{\vartheta_i}\beta)}.
\end{align}
\end{small}

The minimum bound on the probability for playing RES, $p_{\vartheta}^{\min}$, derives by setting in (\ref{eq:probrelation3es}) $p^{d,NE}_{\tilde{\vartheta}}=1$, $\forall \tilde{\vartheta}\in \Sigma_{2,2}$ with $\tilde{\vartheta}\neq \vartheta=\vartheta_i$. Similarly, the maximum bound on the probability for playing RES, $p_{\vartheta}^{\max}$, derives by setting in (\ref{eq:probrelation3es}) $p^{d,NE}_{\tilde{\vartheta}}=0$, $\forall \tilde{\vartheta}\in \Sigma_{2,2}$ with $\tilde{\vartheta}\neq \vartheta=\vartheta_i$. 

The Remarks 3 and 4, which are stated for the PA allocation policy in Section \ref{sec:gameanalysis}, also hold in case of the ES allocation policy. 

The social cost under the ES policy can be expressed as 

\footnotesize
\begin{align}
&C^{ES}(\mathbf{p^{NE}}) =  N \sum_{\vartheta \in \Theta} r_{\vartheta}~ \min\{sh(\mathbf{p^{NE}}), E_{\vartheta}\} ~p_{\vartheta}^{d,NE}~ c^{RES} \nonumber\\ + &\left[D(\mathbf{p^{NE}}) -N \sum_{\vartheta \in \Theta} r_{\vartheta} ~\min\{sh(\mathbf{p^{NE}}), E_{\vartheta}\} ~p_{ \vartheta}^{d,NE}\right]
 ~c^{grid,d}\nonumber\\ + &
 N \left[ \sum_{\vartheta \in \Theta} r_{\vartheta }~ p^{n,NE}_{\vartheta}~ \varepsilon_{\vartheta }~E_{\vartheta}\right] c^{grid,n}.
 \label{eq:social_cost_es_sc}
 \end{align}
\normalsize

\subsection{Centralized Energy Sharing Mechanism Under ES Policy}
Similar to C-ESM under the PA policy (Section \ref{sec:coordinated}), the C-ESM under the ES policy is modeled as an optimization problem, defined as:

 \vspace{-0.1in} 
 \begin{small}
 \begin{subequations} \label{eq:social_cost_x_opt_es}
\begin{alignat}{2}
& \min_{\mathbf{p}} \ && C^{ES}(\mathbf{p}) \label{eq:opt_1_es} \\
 & \text{s.t. } &&p^{d}_{\vartheta},~ p^{n}_{\vartheta}\geq 0  ,  \ \forall \vartheta \in \Theta \label{eq:opt_2_es} \\
 & \quad && p^{d}_{\vartheta} + p^{n}_{\vartheta} = 1,  \ \forall \vartheta \in \Theta. \label{eq:opt_4_es}
 \end{alignat}
 \end{subequations}
\end{small} 

The problem \eqref{eq:social_cost_x_opt_es} is non-convex due to its objective function and the form of the equal share $sh(\mathbf{p^{NE}})$ (Eq. \eqref{eq:fairshare}). In our simulations in Section \ref{sec:comptoES}, we solve it with genetic algorithms using the Global Optimization Toolbox of MATLAB. 
