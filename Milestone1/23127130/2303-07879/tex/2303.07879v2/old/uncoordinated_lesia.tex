\section{Theoretical Game Analysis} \label{sec:uncoordinated_extra_demand}

Here, we study the conditions on the parameter values for the existence of dominant strategies and mixed-strategy NE under PA, and provide close form formulations of these states, i.e. ranges on the values of the vector of mixed strategies at NE $\mathbf{p^{NE}}$. The proofs of the theoretical results presented below are available in the online appendix \cite{}.

First, we recall that, for a mixed-strategy NE to exist, the expected costs of each consumer for all pure strategies in the support of the mixed-strategy NE must be equal. Using the expressions of the costs in \eqref{eq:RES_cost} and \eqref{eq:nonRES_cost2}, we obtain that the amount of RESs allocated to the type $\ell \in \Theta$ at a NE should satisfy:

%First, we recall that any mixed strategy NE $\mathbf{p^{NE}}$ must fulfill

%\small
%\begin{equation}\label{eq:cost_equality_mixed}
%\upsilon_{RES, \ell}(\mathbf{p^{NE}})= \upsilon_{nonRES, \ell}(\mathbf{p^{NE}}), ~\forall \ell \in \Theta.
%\end{equation}
%\normalsize

%\noindent Namely, the expected costs of each pure strategy in the support of the mixed-strategy equilibrium are equal. By substituting the expressions of \eqref{eq:RES_cost} and \eqref{eq:nonRES_cost2} in \eqref{eq:cost_equality_mixed}, we obtain that the amount of RESs allocated to $\ell \in \Theta$ at a NE should satisfy:

 \vspace{-0.1in} 
\small
\begin{equation}\label{eq:conditionEQ_extra_demand}
rse^{PA, NE}_{\ell}(\mathbf{p^{NE}}) = \frac{\gamma-\epsilon_{\ell}\beta}{\gamma-1}E_{\ell},~ \forall \ell \in \Theta.
\end{equation}
\normalsize 
\vspace{-0.1in}  

\noindent Thus, in the ESSG with the PA policy, a mixed-strategy NE exists under the condition:


 \vspace{-0.1in} 
 \small
\begin{equation}\label{eq:condition_PA_NE}
rse_{\ell}^{PA}(\mathbf{p}^{NE}) =rse_{\ell}^{PA,NE}(\mathbf{p}^{NE}), \forall \ell \in \Theta. \end{equation}
\normalsize 
\vspace{-0.1in}  

\noindent 
Therefore, for all cases, any existing mixed-strategy NE competing probabilities, $\mathbf{p}^{NE}$, are obtained so as to satisfy the condition \eqref{eq:condition_PA_NE}. In the following, we distinguish cases with respect to the RES capacity, as well as the preferences, i.e. the risk aversion degree values and the day-time energy demand, of all consumers. 

\subsection{\textbf{Case $1$: The RES capacity, $\mathcal{ER}$, exceeds the maximum total demand for RES $D^{Total}$ ($\mathcal{ER} \geq D^{Total}$).}}

As the consumers have knowledge of $\mathcal{ER}$ and $D^{Total}$, it is straightforward to show that the dominant-strategy for all consumers is to select the strategy $RES$. As a result, the competing probabilities that lead to equilibrium states are equal to $1$ for all consumer types, and the expected aggregate demand for RES at NE is equal to $D^{Total}$.

In all the remaining cases, we assume that $\mathcal{ER}< D^{Total}$. Three distinct cases are defined with respect to the risk aversion degrees of the consumers and energy prices.

\subsection{\textbf{Case $2$: The risk aversion degrees for all consumer types satisfy $1\leq \epsilon_{\ell}<\gamma/\beta$, $\forall \ell \in \Theta$.}}

We distinguish the following sub-cases with respect to the day-time energy demand levels. 

\subsubsection{\textbf{Sub-case $2(a)$: The day-time energy demand levels satisfy  $E_{\ell } \leq \mathcal{ER}\frac{(\gamma-1)}{(\gamma-\epsilon_{\ell }\beta)}$,  for all $\ell \in \Theta$}}

In this case, a mixed-strategy NE with the PA policy exists if and only if for every pair of consumers, $i,j$ with day-time energy demand levels $E_{\vartheta_i}$, $E_{\vartheta_j}$, respectively, it holds that

 \vspace{-0.1in} 
 \small
\begin{eqnarray}\label{eq:relation_E_0_E_1_pa_ne_extra_demand}
\mathcal{ER}\frac{(\gamma-1)}{(\gamma-\epsilon_{\vartheta_i }\beta)}-E_{\vartheta_i} &=& \mathcal{ER}\frac{(\gamma-1)}{(\gamma-\epsilon_{\vartheta_j }\beta)}-E_{\vartheta_j}.
 \end{eqnarray}
\normalsize 
\vspace{-0.1in}  


\begin{proof}
To derive the condition \eqref{eq:relation_E_0_E_1_pa_ne_extra_demand} we first consider a consumer $i$ with type $\vartheta_i \in \Theta$ that plays $RES$. Then, we can substitute the left-hand side of \eqref{eq:condition_PA_NE} with \eqref{eq:prop_alloc_energy} where the demand for RESs is expressed as $D(\mathbf{p^{NE}}) = E_{\vartheta_i}+ \sum_{ {\ell}\in \Theta} r_{\ell}~ (N-1)~E_{\ell}~p^{NE}_{RES,\ell}$. And, by also substituting the right-hand side of \eqref{eq:condition_PA_NE} with \eqref{eq:conditionEQ_extra_demand} for consumer $i$, we obtain:

\begin{small}
\begin{align}
&  \mathcal{ER}\frac{(\gamma-1)}{(\gamma-\epsilon_{\vartheta_i}\beta)}-E_{\vartheta_i}= \sum_{ {\ell}\in \Theta} r_{\ell}~ (N-1)~E_{\ell}~p^{NE}_{RES,\ell},
    \label{eq:probrelation1}
\end{align}
\end{small}

\noindent By analogy, we can re-write \eqref{eq:probrelation1} for a consumer $j$ with type $\vartheta_j \in \Theta \setminus \{\vartheta_i\}$ that plays $RES$ as:

\begin{small}
\begin{align}
  &  \mathcal{ER}\frac{(\gamma-1)}{(\gamma-\epsilon_{\vartheta_j}\beta)}-E_{\vartheta_j}=  \sum_{ {\ell}\in \Theta} r_{\ell} ~(N-1)~E_{\ell}~ p^{NE}_{RES,\ell}.
    \label{eq:probrelation2}  
\end{align}
\end{small}

\noindent Since the right-hand sides of \eqref{eq:probrelation1}-\eqref{eq:probrelation2} are equal, the left-hand sides will be also equal and \eqref{eq:relation_E_0_E_1_pa_ne_extra_demand} derives.
\end{proof}


Under condition \eqref{eq:relation_E_0_E_1_pa_ne_extra_demand}, the competing probabilities $p^{NE}_{RES,{\vartheta_i}}$ for each consumer $i$ of type $\vartheta_i \in \Theta$ that lead to equilibrium states lie in the range:


 \vspace{-0.1in} 
 \footnotesize
\begin{align}
&  \Biggl[ \max \Biggl\{0,\frac{\mathcal{ER}\frac{(\gamma-1)}{(\gamma-\epsilon_{\vartheta_i}\beta)}-E_{\vartheta_i}-
  \sum_{ {\ell}\in \Theta \setminus \{{\vartheta_i}\}} r_{\ell} (N-1)E_{\ell}}{r_{\vartheta_i}(N-1)E_{\vartheta_i}}\Biggl\}, \nonumber \\
 &      \min\Biggl\{1,\frac{\mathcal{ER}\frac{(\gamma-1)}{(\gamma-\epsilon_{\vartheta_i}\beta)}-E_{\vartheta_i}}{r_{\vartheta_i}(N-1)E_{\vartheta_i}}\Biggl\}
    \Biggr].
    \label{eq:prop_alloc_pa_bounds_0_extra_demand}
\end{align}
\normalsize 
\vspace{-0.1in}  

Furthermore, at NE, the expected aggregate demand for RES for any consumer type $\ell \in \Theta$ can be expressed as:

 \vspace{-0.1in} 
 \footnotesize
\begin{align}
& D(\mathbf{p}^{NE})= \min\Bigl\{D^{Total}, \max\Bigl\{\left[\mathcal{ER}\frac{(\gamma-1)}{(\gamma-\epsilon_{\ell}\beta)}-E_{\ell}\right]\frac{N}{(N-1)},0\Bigr\}\Bigr\}.
    \label{eq:demand1}
\end{align}
\normalsize 
\vspace{-0.1in}  

\begin{remark}\label{rem:risk_seeking}If all consumers are risk-seeking (i.e., $\epsilon_{\vartheta_i}= 1, \forall i \in \mathcal{N}$), a NE can exist only if  $E_{0}=E_{1}=...=E_{M-1}$.%for every pair of $\vartheta_i, \vartheta_j \in \Theta$. %On the contrary, if there are risk-conservative consumers, they may need to have asymmetric energy profiles for a NE to exist.
\end{remark}

\begin{remark} \label{rem:risk_degrees_relation}
Note that condition \eqref{eq:relation_E_0_E_1_pa_ne_extra_demand} can hold, and therefore a NE can exist, only if $\epsilon_0\leq \epsilon_1 \leq..\leq \epsilon_{M-1}$. Since by assumption, $E_0\leq E_1 \leq ...\leq E_{M-1}$, this means that consumers with lower day-time energy demand levels should be less risk-averse than those with higher ones.
\end{remark}

 
\subsubsection{\textbf{Sub-case $2(b)$: The day-time energy demand levels satisfy $E_{\ell} > \mathcal{ER}\frac{(\gamma-1)}{(\gamma-\epsilon_{\ell}\beta)}$  for all $\ell \in \Theta$}} 
In this case, the dominant strategy for all consumers is to play $nonRES$.

As result, the competing probabilities that lead to equilibrium states are equal to $0$ for all consumer types $\vartheta_i\in \Theta$, and the expected aggregate demand for RES at NE is equal to $0$.

\subsubsection{\textbf{Sub-case $2(c)$: There exist two distinct subsets of consumer types, $\Sigma_1 , \Sigma_2 \subset \Theta$, such that $\{E_{\ell} > \mathcal{ER}\frac{(\gamma-1)}{(\gamma-\epsilon_{\ell}\beta)}, ~\forall \ell \in \Sigma_1\}$ and $\{E_{\ell} \leq \mathcal{ER}\frac{(\gamma-1)}{(\gamma-\epsilon_{\ell}\beta)}, ~ \forall \ell \in \Sigma_2\}$}}

For consumers whose types are in the set $\Sigma_1$, the dominant strategy is to play $nonRES$. 
For consumers whose types are in the set $\Sigma_2$, the mixed strategy NE is determined under the condition of \eqref{eq:relation_E_0_E_1_pa_ne_extra_demand}.

Additionally, for consumers whose types are in $\Sigma_1$, the competing probabilities at NE are equal to $0$, whereas, for consumers $i$ of type $\vartheta_i \in \Sigma_2$, the competing probabilities that lead to a NE states lie in the range:

 \vspace{-0.1in} 
 \footnotesize
\begin{align}
&  \Biggl[ \max \biggl\{0,\frac{\mathcal{ER}\frac{(\gamma-1)}{(\gamma-\epsilon_{\vartheta_i}\beta)}-E_{\vartheta_i}-
  \sum_{\ell \in \Sigma_2 \setminus \{\vartheta_i\}} r_{\ell} (N-1)E_{\ell}}{r_{\vartheta_i}(N-1)E_{\vartheta_i}}\Biggr\}, \nonumber \\
    & \min \Biggl\{1,\frac{\mathcal{ER}\frac{(\gamma-1)}{(\gamma-\epsilon_{\vartheta_i}\beta)}-E_{\vartheta_i}}{r_{\vartheta_i}(N-1)E_{\vartheta_i}}\Biggr\}
    \Biggr].
    \label{eq:prop_alloc_pa_bounds_0_extra_demandnew}
\end{align}
\normalsize  
\vspace{-0.1in}  

Finally, the aggregate demand for RESs in $\Sigma_1$ is equal to $0$, and the total aggregate demand for RESs can be expressed as

 \vspace{-0.1in} 
 \footnotesize
\begin{align}
& D(\mathbf{p}^{NE})= \min\Bigl\{ D^{Total}_{\Sigma_2} , \max\Bigl\{\left[\mathcal{ER}\frac{(\gamma-1)}{(\gamma-\epsilon_{\ell}\beta)}-E_{\ell}\right]\frac{N}{(N-1)},0\Bigr\}\Bigr\},
    \label{eq:demand1_2c}
\end{align}
\normalsize 
\vspace{-0.1in}  

\noindent where $D^{Total}_{\Sigma_2} = N  \sum_{\tilde{\ell } \in \Sigma_2} r_{\tilde{\ell }  }E_{\tilde{\ell } }$ is the maximum demand for RESs of the consumers whose types are in $\Sigma_2$.

\begin{remark}
We note that Remark \ref{rem:risk_degrees_relation} now holds for all consumer types in $\Sigma_2$.
\end{remark}

\subsection{\textbf{Case $3$: The risk aversion degrees satisfy $\epsilon_{\ell} \geq \gamma/\beta$, $\forall \ell \in \Theta$.}}

In this case, the dominant strategy for all consumers is to play $RES$.

Therefore, the competing probabilities that lead to equilibrium states are equal to $1$ for all consumer types, and the aggregate demand for RESs at NE is equal to $D^{Total}$.

\subsection{\textbf{Case $4$: There exist two distinct subsets of consumer types, $\Sigma_1 , \Sigma_2 \subset \Theta$, such that $\Bigl\{\epsilon_{\ell} \geq \gamma/\beta, ~ \forall \ell \in \Sigma_1 \Bigr\}$, and $\Bigl\{\epsilon_{\ell} < \gamma/\beta,~\forall \ell \in \Sigma_2 \Bigr\}$.}} 


By analogy with Case $3$, for the consumers whose type is in the set $\Sigma_1$, the dominant strategy is to compete for RESs.
By analogy with Case $2$, for the consumers whose type is in the set $\Sigma_2$, three sub-cases $4(a)-(c)$ are defined with respect to their day-time energy demand levels. All the conditions derived in Case $2$ can straightforwardly be extended to the consumer types in $\Sigma_2$, considering that all consumers in $\Sigma_1$ play $RES$. To do so, we consider that the available RES capacity for consumers in $\Sigma_2$ is equal to $\mathcal{ER}$ minus the aggregate demand for RESs of consumer types in $\Sigma_1$. For the sake of concision we do not detail these conditions.
