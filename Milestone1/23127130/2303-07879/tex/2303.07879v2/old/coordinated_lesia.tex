x\section{Centralized Energy Source Allocation Mechanism}\label{sec:coordinated}

In this section, we study the case in which a central energy community coordinator allocates the different energy resources among the $N$ consumers in the energy community so as to minimize the social cost of the overall community. This centralized energy source allocation mechanism (ESAM) is used as a ideal benchmark against which to evaluate the efficiency of the proposed decentralized ESSG.

\subsection{Optimization Problem Formulation}

In practice, in the ESAM, all consumers $i$ with type $\vartheta_i$ communicate their daily flexible load $U_{\vartheta_i}$ and risk-aversion degree $\mu_{\vartheta_i}$ (such that their day-time energy demand $E_{\vartheta_i}= \mu_{\vartheta_i}U_{\vartheta_i}$) to a central energy community coordinator. 
%Therefore, the aggregate amount of day-time loads is equal to $D^{Total}$. 
This coordinator optimally schedules the loads of the consumers during each time period (i.e. day or night) by selecting the optimal vector of mixed-strategies $\mathbf{p}^{OPT}=[\mathbf{p_0}^{OPT^\top};...;\mathbf{p_{M-1}}^{OPT^\top}]$ that minimizes the social cost $C(\mathbf{p}^{OPT})$ of the overall community. For each consumer $i$ of type $\vartheta_i$, the vector $\mathbf{p}^{OPT}_{\vartheta_i} = [\mathbf{p}^{OPT}_{RES,\vartheta_i},\mathbf{p}^{OPT}_{nonRES,\vartheta_i}]$ is defined, similarly to $\mathbf{p}^{NE}_{\vartheta_i}$, such that the share of their daily flexible load scheduled during the day equals $p^{OPT}_{RES,\vartheta_i}E_{\vartheta_i}$, and the share of their daily flexible load scheduled during the night equals $p^{OPT}_{nonRES,\vartheta_i}U_{\vartheta_i}$. 

As a result, this centralized ESAM is modelled as an optimization problem, defined as:

 \vspace{-0.1in} 
 \begin{small}
 \begin{subequations} \label{eq:social_cost_x_opt}
\begin{alignat}{2}
& \min_{\mathbf{p}^{OPT}} \ && C(\mathbf{p}^{OPT}) = \min\Bigl\{ \mathcal{ER}, N \sum_{{\ell} \in \Theta} r_{\ell} p_{RES,{\ell}}^{OPT}  E_{\ell}\Bigr\}  c_{RES} \nonumber \\ 
& \quad && + \max \Bigl\{ 0,N \sum_{{\ell} \in \Theta} r_{\ell} p_{RES,{\ell}}^{OPT}  E_{\ell}- \mathcal{ER}\Bigr\}  c_{nonRES}\nonumber \\
& \quad && + N  \sum_{{\ell} \in \Theta} r_{{\ell} }p^{OPT}_{nonRES,{\ell}}\varepsilon_{{\ell} }E_{{\ell}} c_{nonRES,N} \label{eq:opt_1} \\
 & \text{s.t. } && 0 \leq p^{OPT}_{RES,\ell} \leq 1 ,  \ \forall \ell \in \Theta \label{eq:opt_2} \\
  & \quad && 0 \leq p^{OPT}_{nonRES,\ell} \leq 1 ,  \ \forall \ell \in \Theta \label{eq:opt_3} \\
 & \quad && p^{OPT}_{RES,\ell} + p^{OPT}_{nonRES,\ell} = 1 , \ \forall \ell \in \Theta, \label{eq:opt_4}
 \end{alignat}
 \end{subequations}
\end{small} \vspace{-0.1in}  

\noindent which minimizes the objective function \eqref{eq:opt_1}, representing the social cost of the community, subject to constraints \eqref{eq:opt_2}-\eqref{eq:opt_4} defining the mixed-strategy probabilities. 
%\noindent where, the decision variables of the centralized mechanism are the optimal probability distributions ($p^{OPT}_{RES,\vartheta_i}$, $p^{OPT}_{nonRES,\vartheta_i}$), which represent the optimal probability that consumer $i$ with type $\vartheta_i$ engages its load during day, or during the night, respectively, for all consumers $i \in \mathcal{N}$. \footnote{The vector $\mathbf{p^{OPT}}$ is defined similarly to $\mathbf{p^{NE}}$.}

%where $N  \sum_{{\ell}\in \Theta} r_{\ell} p_{RES,{\ell}}^{OPT}  E_{\ell} $ represents the total RES production allocated in the day-zone, and $N  \sum_{{\ell} \in \Theta} r_{{\ell} }p^{OPT}_{nonRES,{\ell}}\varepsilon_{{\ell} }E_{{\ell}}$ represents the total nonRES production allocated in the night-zone. %The first summand refers to the consumers who are served by low-cost RES capacity during the day. The second summand is for the consumers who are served $nonRES$ duing the night-zone. 
%In other words, the part of the optimal social cost that corresponds to the day-zone peak-load production will be zero. %The social costs under the PA and ES policies are defined similarly to Eqs. \eqref{eq:social_cost_pa_extra_demand} and \eqref{eq:social_cost_es}, respectively. 


%The optimal solutions of this centralized mechanism will provide a benchmark against which to evaluate the side-effects that stem from the distributed, uncoordinated energy source selection, in terms of social cost. 
Although this optimization problem is nonconvex, due to its objective function, it can be linearized or formulated as an equivalent optimization problem depending on the values of RES capacity and the preferences of the consumers, i.e. their daily flexible loads and risk-aversion degrees.
In the following, we provide insights and analytical formulations of the optimal solutions $\mathbf{p^{OPT}}$ and objective cost $C(\mathbf{p^{OPT}})$ of this centralized ESAM in each one of the four cases described in Section \ref{sec:uncoordinated_extra_demand}. 

%and the PoA under the PA mechanism. %For the ES mechanism, the centralized decisions for minimizing the social cost are harder to derive analytically and we will present numerical evaluations in Section \ref{sec:eval} using software optimization tools, namely, Mathematica and MATLAB.

\subsection{Optimal Centralized Solutions}

%%%%%%%%% define these costs under each allocation policy!!!????
%We derive the properties of the solution provided by the centralized mechanism, $\mathbf{p^{OPT}}$, in four cases, identical to the ones defined in Section \ref{sec:uncoordinated_extra_demand}.
 
\subsubsection{\textbf{Case $1$: The RES capacity exceeds the maximum day-time energy demand}}

In this trivial case, the social cost reduces to:

 \vspace{-0.1in}
 \begin{small}
\begin{align}
C(\mathbf{p}^{OPT}) & = N \sum_{{\ell} \in \Theta} r_{\ell} p_{RES,{\ell}}^{OPT}  E_{\ell}   c_{RES}  \nonumber \\
& + N  \sum_{{\ell} \in \Theta} r_{{\ell} }p^{OPT}_{nonRES,{\ell}}\varepsilon_{{\ell} }E_{{\ell}} c_{nonRES,N},
 \label{eq:social_cost_x_opt_case1}
 \end{align}
\end{small} \vspace{-0.1in}  

\noindent and the optimal solutions to the centralized ESAM is to schedule all consumers during the day,i.e. $p^{OPT}_{RES,\vartheta_i}=1$, and $p^{OPT}_{nonRES,\vartheta_i}=0$, $\forall \vartheta_i \in \Theta$.

In all other cases considered, the RES capacity is lower than the maximum day-time energy demand.

\subsubsection{\textbf{Case $2$: The risk-aversion degrees of all consumer types are lower than $\gamma/\beta$}}

In this case, it is optimal for the centralized ESAM to schedule loads during the day so that the total RES capacity is utilized, and to schedule the remaining load during the night. Specifically, the social cost reduces to:

 \vspace{-0.1in} 
 \begin{small}
\begin{align}
C(\mathbf{p}^{OPT}) &=  \mathcal{ER} \cdot c_{RES}   \nonumber \\
&+N \left[ \sum_{{\ell} \in \Theta} r_{\ell }p^{OPT}_{nonRES,{\ell}}\varepsilon_{{\ell} }E_{{\ell}}\right] c_{nonRES,N}
 \label{eq:social_cost_pa_extra_demand_2}
 \end{align}
\end{small} \vspace{-0.1in}  

\noindent Therefore, the centralized ESAM optimization problem \eqref{eq:social_cost_x_opt} is equivalent to minimizing the night-time cost $N \left[ \sum_{{\ell} \in \Theta} r_{\ell }p^{OPT}_{nonRES,{\ell}}\varepsilon_{{\ell} }E_{{\ell}}\right] c_{nonRES,N}$, subject to constraints \eqref{eq:opt_2}-\eqref{eq:opt_4} and the condition that the total energy demand for RES equals the RES capacity:

 \vspace{-0.1in} 
 \small
\begin{align}
N \sum_{{\ell} \in \Theta}r_{{\ell}} E_{{\ell}}p^{OPT}_{RES,{\ell}}=\mathcal{ER}.
\label{eq:optimal}
\end{align}
\normalsize 
\vspace{-0.1in}  

\noindent As a result, the optimal competing probabilities $p^{OPT}_{RES,\vartheta_i}$ for all $ \vartheta_i \in \Theta$ lie in the range:

 \vspace{-0.1in} 
 \footnotesize
\begin{align}
 \hspace{-5pt} 
 \Biggl[max\left\{0,\frac{\left[\mathcal{ER} -\sum_{{\ell} \in \Theta \setminus \{\vartheta_i\}}r_{{\ell}}N E_{{\ell}}\right]}{r_{\vartheta_i}N E_{\vartheta_i}}\right\},
   %\notag\\
    % min\left(1,\mathcal{ER} \frac{1}{rNE_0}\right)
    min\left\{1,\frac{\mathcal{ER}}{r_{\vartheta_i}N E_{\vartheta_i}}\right\}
    \Biggr].
    \label{eq:prop_alloc_opt_bounds_0}
\end{align}
\normalsize 
\vspace{-0.1in}  

\subsubsection{\textbf{Case $3$: The risk-aversion degrees of all consumer types are greater than $ \gamma/\beta$}}

In this case, the optimal solutions to the centralized ESAM is to schedule all consumers during the day, i.e. $p^{OPT}_{RES,\vartheta_i}=1$ and $p^{OPT}_{onRES,\vartheta_i}=0$ $\forall \vartheta_i \in \Theta$. This result is derived in a similar fashion with the result of Case $3$ in Section \ref{sec:uncoordinated_extra_demand}.
 
\subsubsection{\textbf{Case $4$: The risk aversion degrees of certain consumer types are lower than $ \gamma/\beta$ (set $\Sigma_1$), while others are greater than $ \gamma/\beta$ (set $\Sigma_2$)}}

Then, it is optimal for the centralized ESAM to schedule all consumers whose types are in the set $\Sigma_1$ during the day, i.e., $p^{OPT}_{RES,\vartheta_i}=1$ $\forall \vartheta_i \in \Sigma_1$. Furthermore, the social cost of the remaining consumers whose types are in $\Sigma_2$ can be expressed as: 

 \vspace{-0.1in} 
 \begin{small}
\begin{align}
C_{\Sigma_2}(\mathbf{p}^{OPT}) &=  (\mathcal{ER} - \sum_{\ell \in \Sigma_1} r_\ell N E_\ell )\cdot c_{RES}   \nonumber \\
&+N \left[ \sum_{{\ell} \in \Theta} r_{\ell }p^{OPT}_{nonRES,{\ell}}\varepsilon_{{\ell} }E_{{\ell}}\right] c_{nonRES,N}
 \label{eq:social_cost_pa_extra_demand_2}
 \end{align}
\end{small} \vspace{-0.1in}  

\noindent where $ \mathcal{ER} - \sum_{\ell \in \Sigma_1} r_\ell N E_\ell $ represents the remaining available RES capacity, i.e., the available RES capacity minus the aggregate demand of consumers whose types are in $\Sigma_1$ and are scheduled during the day. 



Therefore, the probability that consumers whose type is in the set $\Sigma_2$ play $RES$ is optimized to minimize their night-time cost, while ensuring that they totally utilize the remaining RES capacity, such that:


 \vspace{-0.1in} 
 \small
\begin{align}
N \sum_{{\ell} \in \Theta}r_{{\ell}} E_{{\ell}}p^{OPT}_{RES,{\ell}}=\mathcal{ER} - \sum_{\ell \in \Sigma_1} r_\ell N E_\ell.
\label{eq:optimal}
\end{align}
\normalsize 
\vspace{-0.1in}  

This centralized problem can be solved using a software optimization tool. And, we show that the optimal competing probabilities $p^{OPT}_{RES,\vartheta_i}$ for all $ \vartheta_i \in \Sigma_2$ lie in the range:

 \vspace{-0.1in} 
 \footnotesize
\begin{align}
&\Biggl[\max\left\{0,\frac{\mathcal{ER}-\sum_{\ell \in \Theta \setminus \{\vartheta_i\}}r_{\ell}N E_{\ell}}{r_{\vartheta_i}N E_{\vartheta_i}}\right\}, \nonumber \\ & \min\left\{1,\frac{\mathcal{ER}-\sum_{ \ell \in \Sigma_1}r_{\ell}N E_{\ell}}{r_{\vartheta_i}N E_{\vartheta_i}}\right\}  \Biggr].
\label{eq:prop_alloc_opt_p_cond}
\end{align}
\normalsize 
\vspace{-0.1in}  

\noindent By analogy with Case $2$, this range is derived similarly to the one defined in \eqref{eq:prop_alloc_opt_bounds_0}.

\subsection{PoA}

The (in)efficiency of equilibrium strategies in the decentralized, uncoordinated mechanism is quantified by the Price of Anarchy (PoA) \cite{Koutsoupias09}. The PoA is expressed as the ratio of the worst case social cost among all mixed strategy NE, denoted as $C^{PA,NE}_w$, over the optimal minimum social cost of the centralized mechanism, denoted as $C(\mathbf{p^{OPT^*}})$, such that:

%\vspace{-5pt}
 \vspace{-0.1in} 
 \small
\begin{align}
 \hspace{-5pt} \textit{PoA}^{PA}  = \frac{C^{PA,NE}_w}{C(\mathbf{p^{OPT^*}})}.
\label{eq:poa_pa}
\end{align}
\normalsize 
\vspace{-0.1in}  

First observe that $C(\mathbf{p^{OPT^*}})$ is uniquely determined for each particular case. Now, in order to obtain $C^{PA,NE}_w$ when there exist multiple possible NE (e.g., in Cases $2$ and $4$), we can solve a simple optimization problem to maximize the social cost $C^{PA,NE}(\mathbf{p^{NE}})$ with respect to $\mathbf{p^{NE}}$ and subject to the corresponding probability constraints for NE defined in Section \ref{sec:uncoordinated_extra_demand}, using a software optimization tool. 

Note that $C^{PA,NE}(\mathbf{p^{NE}})$ takes its optimal value (i.e, minimum value) when the night-time cost is minimized. This solution coincides with the optimal centralized solution and thus in this case PoA takes the optimal (unity) value. 


 %The last observation is the fact that the demand $D^{PA,NE}$ is constant with respect to $\mathbf{p^{PA,NE}}$. Also, in the special case of the risk-seeking consumers, from Eq. \eqref{eq:social_cost_pa_sc}, the social cost is constant with respect to the probabilities $\mathbf{p^{PA,NE}}$ for each case of energy profile values. Thus, $C^{PA,NE}_w$ is given by Eq. \eqref{eq:social_cost_pa_sc}.

%$2$. If having risk-conservative consumers, there exist multiple possible equilibria in the sub-cases 2(a) and 2(c) as well as in case 4. Of course, the multiple combinations of probabilities can lead to NE with possibly different social cost values. Based on the Remark \ref{rem:risk_degrees_relation}, the existence of NE is possible if consumers with lower energy demands have lower risk aversion degrees. Thus, the night cost is minimized if the optimal probabilities for RES take lower values for consumers with lower energy demands. 
