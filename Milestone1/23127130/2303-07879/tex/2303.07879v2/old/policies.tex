\subsection{Renewable Energy Allocation Policies}\label{sec:policies}

The allocation of scarce resources to human-driven agents with distinct demands, is challenging. In this paper, we examine two types of allocations with different fairness properties, namely, the proportional allocation and the equal sharing.%The engagement of humans (behind the machines) transfuse social features and require careful consideration of the fairness properties of the resource allocation policy, as it impacts on users' satisfaction, participation, compliance pervasion, and ultimately, on system's efficiency \cite{Busqu12}.

%In multiplexing settings, the division of scarce resources motivates a number of challenges with respect to the properties that should characterize the allocation mechanism. Most commonly, the allocation mechanisms pursue an utilitarian view of the allocation (\ie aggregation of agents' utility on the allocated resources). However, practical circumstances that engage humans (and machines) transfuse social features to the allocation process and raise some concerns about the level of fairness that should vouch for. The fairness properties of an allocation mechanism have a direct impact on agents' satisfaction, participation, compliance pervasion, and ultimately, on system's efficiency \cite{Busqu12}.

\paragraph{Proportional Allocation}\label{sec:prop_alloc}

The proportional allocation allocates resources in proportion to the consumer demand. Under this policy, the amount of RES received by a consumer of type $\vartheta_i$ is given by,
% all users a level of service proportional to their actual needs. In turn, applying this allocation rule in the distribution of RES capacity, consumers that compete for RES capacity are endowed with an amount of energy that is given by 

\vspace{-5pt}
\footnotesize
\begin{eqnarray}
rse^{PA}_{\vartheta_i}(\mathbf{n}) &=& \frac{E_{\vartheta_i}}{\max(\mathcal{ER} ,(\sum_{\ell \in \Theta } n_{\ell  } E_{\ell  }+ E_{\vartheta_i })}\mathcal{ER},
\label{eq:prop_alloc_energy}
\end{eqnarray}
\normalsize

%where $n_0=n_{RES,0}$, $n_1=n_{RES,1}$ and 
where $\sum_{\ell \in \Theta } n_{\ell  } E_{\ell  }+ E_{\vartheta_i}$ is the total energy demand.

\paragraph{Equal Sharing}\label{sec:eq_sharing}

Under equal sharing, all competing consumers will obtain the same amount of resources (as long as they have a need for). Under this policy, the amount of RES received by a consumer of type $ \vartheta_i$ is given by,
%benefits from accessing a particular service. In the energy front, all consumers that compete for RES capacity are endowed with an amount of energy that is given by

\vspace{-5pt}
\footnotesize
\begin{eqnarray}
\hspace{-30pt} && rse^{ES}_{\vartheta_i}(\mathbf{n}) = \min\left( E_{\vartheta_i}, \frac{\mathcal{ER}}{\sum_{\ell \in \Theta } n_{\ell }+1}\right),
\label{eq:eq_sharing_energy}
\end{eqnarray}
\normalsize
%where $n_0,n_1$ denote the number of competitors %of low-to-moderate and high energy profile \textit{but} player $i$, respectively, as also defined in Section 
%\ref{sec:prop_alloc}, and $n_0+n_1+1$ is the total number of competitors.
where $\sum_{\ell \in \Theta } n_{\ell }+1$ is the total number of competitors.%, with $n_0,n_1$ as in Section \ref{sec:prop_alloc}.

Generally, policies for sharing resources that accommodate a level of fairness that prioritizes classes of users with specific characteristics may provide high efficiency (by prioritizing ``strong players'') but lack stability (``weak players'' may continuously change behavior/strategy to improve the acquired service). For instance, the proportional allocation may cause unstable operational states since it entails the risk of starvation of users with low energy demand, and eventually result in fewer satisfied players. On the other hand, equal sharing of resources favors stability since it prevents strong energy players (consumers with high demand) from obtaining more resources that any other player. Yet, equal sharing may result in inefficient and wasteful utilization of energy resources, since the ``fair share'' of energy resources may exceed the energy requirements of particular consumers of low energy demand.

In the following sections, we analytically and numerically study the distributed, uncoordinated energy selection game. We study the stability and efficiency of energy resource allocation by considering the two different notions of fairness above. Specifically, for each allocation policy we analytically derive its equilibrium states (stability) and we numerically compare the costs incurred by the consumers' strategies against the costs of an optimal centralized coordinated operation (efficiency).

%
%enabling varying notions of fairness
%derive the stable operational states under the two allocation policies in terms of the associated equilibrium strategies, assuming fully rational players, free of knowledge and computational constraints.% and emotional biases.
%


% and analyze the resulting dynamics under different allocation policies that conceptualize various levels of fairness in energy resource sharing. In all cases, we derive the stable operational conditions (\ie equilibria) and the associated costs incurred by the players, and assess their efficiency by comparing them with those under optimal coordinated energy source allocation. 


\subsection{Demand}\label{sec:demand}
The expected aggregate demand for RES at NE, denoted as $D^{x,NE}$, is given as: %$N\sum_{\vartheta_j \in \Theta} r_{\vartheta_j }~p^{PA,NE}_{RES,\vartheta_j}~E_{\vartheta_j}$. % with respect to the energy profile 
\vspace{-5pt}
\footnotesize
\begin{eqnarray}
\hspace{-30pt} && D^{x,NE} (\mathbf{p^{x,NE}})= N\sum_{\ell\in \Theta} r_{\ell }~p^{x,NE}_{RES,\ell}~E_{\ell},
\label{eq:expected_demand}
\end{eqnarray}\normalsize where $x$ is replaced by $PA$ for the proportional allocation and by $ES$ for the equal sharing, and $p^{x,NE}_{RES,\vartheta_j}$ stands for the Nash equilibrium probability of profile $\vartheta_j \in \Theta$ competing for RES for each $x$ value. 

In Section, \ref{sec:uncoordinated_extra_demand}, we will give more explicit expressions for the expected aggregate demand dependent on the case.

\subsection{Social Cost}\label{sec:socialcost}
The social cost is computed differently for the proportional allocation and the equal sharing. In proportional allocation and at NE, 

\vspace{-5pt}
\footnotesize
\begin{align}
C^{PA,NE}(\mathbf{p}^{PA,NE}) &=  \min(\mathcal{ER}, D^{PA,NE}(\mathbf{p^{PA,NE}})) c_{RES} \nonumber \\&+  \max(0,D^{PA,NE}(\mathbf{p^{PA,NE}}))-\mathcal{ER})c_{nonRES,D} \nonumber \\
&+N \left[ \sum_{\ell \in \Theta} r_{\ell }p^{PA,NE}_{nonRES,\ell}\varepsilon_{\ell }E_{\ell}\right] c_{nonRES,N}.
 \label{eq:social_cost_pa_extra_demand}
 \end{align}
\normalsize

For the equal sharing, we need to account for the extra renewable energy allocated to users, e.g., when all users (both profiles) compete for RES and possibly the lower energy profile users are allocated more RES than they need for or when the amount of RES is higher than the overall users' demand. The fair share of RES at NE, $sh(\mathbf{p^{ES,NE}})$, can be approximated by 

\vspace{-5pt}
\footnotesize
\begin{eqnarray}
sh(\mathbf{p^{ES,NE}}) \hspace{-5pt} &=& \hspace{-5pt} \frac{\mathcal{ER}}{ N \sum_{ \ell=0}^{M-1} r_{\ell}  E_{\ell} p_{RES,\ell}^{ES,NE}}.
 \label{eq:sharing_es}
 \end{eqnarray}
\normalsize

Therefore,
\vspace{-5pt}
\footnotesize
\begin{align}
&C^{ES,NE}(\mathbf{p^{ES,NE}}) \nonumber\\ &=  d(\mathbf{p^{ES,NE}}) c_{RES} +(D^{ES,NE} -d(\mathbf{p^{ES,NE}}))c_{nonRES,D}\nonumber \\
&+N \left[ \sum_{\ell\in \Theta} r_{\ell}p^{ES,NE}_{nonRES,\ell}\varepsilon_{\ell }E_{\ell}\right] c_{nonRES,N},
 \label{eq:social_cost_es}
 \end{align}
\normalsize

with \footnotesize$d(\mathbf{p^{ES,NE}})= N \sum_{ \ell=0}^{M-1} r_{\ell} p_{RES,\ell}^{ES,NE}  \min(sh(\mathbf{p^{ES,NE}}), E_{\ell})$.\normalsize
