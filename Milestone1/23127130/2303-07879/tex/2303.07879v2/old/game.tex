\section{Modeling the energy source selection}\label{sec:game}
%\textbf{we need to specify a time-unit of taking the decisions: if we take them only once then (i) ER cannot be known, (ii) why not centralized approach. This is studied for one time interval. Can I decide at a next time interval to move back some loads sent to night rate at a previous interval to ER - intra-day re- scheduling per interval? Nice to describe how this is done. }


We assume a micro-grid with $N$ (eligible) consumers. Let $\mathcal{ER}$ stand for the available RES capacity in energy units (\eg kWh) for the considered time-period in the day-zone. Also, we define the following per unit of energy cost values, namely, (i) $c_{RES}$ for RES, (ii) $c_{nonRES,N} = \beta \cdot c_{RES}$, for night-zone nonRES with $\beta>1$, and (iii) $c_{nonRES,D}= \gamma \cdot  c_{RES}$ for day-zone nonRES with $ \gamma>\beta$. Notice that an \emph{excess penalty cost}, equal to $(\gamma-\beta) \cdot c_{RES}$ per unit of energy load, will be paid by consumers that was engaged in the day-zone but not served due to the limited RES capacity. % and captures the impact of congestion (in case of the uncoordinated competition) for RES. 
All aforementioned costs, as well as the amount of the available RES, $\mathcal{ER}$, are assumed as known to the consumers. %We also assume that the consumers do not increase their demand when opting for the night-zone. 
%The challenge addressed in this paper is finding a distributed and uncoordinated scheduling of the consumer loads in day and night-zones so that the consumers' average cost is minimized. Based on the prices, if the consumers' average cost is minimized, the amount of RES actually used (and not curtailed) should be maximized. %More specifically, we solve the problem of the energy source selection, between RES and night-zone nonRES, by the consumers themselves. %We first present a short discussion on a benchmark centralized solution and afterwards we will discuss the simpler to implement distributed one. 
%
%In time zones when the capacity of RES suffices to serve the entire energy demand, consumers can readily opt for using it and gain the price differential between the two resources. When, however, the demand exceeds the limited RES, an inherent competition emerges that should be factored by consumers in their decision to opt for requesting this resource or not at the particular time period. The underlying assumption is that the decision to opt for the finite RES bears the risk of experiencing the congestion penalties induced by the lack of coordination and pay for the more expensive, top-priced nonRES. Essentially, the main dilemma faced by consumers is whether to compete or not for requesting RES or settle for less favoured time zones and pay for medium-priced energy (assumed being covered by a nonRES facility).
%
%In order to analytically investigate the suffered congestion penalties due to the uncoordinated access to the limited resource, it is important to identify and model those key features and critical parameters that tune the intensity of competition. In this respect, we may consider $N$ consumers that decide between RES and nonRES. The first facility is capacity-constrained and able to serve a total energy demand of $\mathcal{ER}$ energy units (\eg kilowatt-hours, kWh) while the second one responds to every possible level of energy demand, yet heavily burdens the environment producing large amounts of pollutants. In the pricing arena, RES is offered at $c_{RES}$ cost units, whereas for nonRES different rates are used for daily and night energy consumption: moderate-priced night energy rate of $c_{nonRES,N} = \beta \cdot c_{RES}, \beta>1$, units and top-priced daily energy rate of $c_{nonRES,D}= \gamma \cdot  c_{RES}, \gamma>\beta$, units. The excess penalty cost $\delta\cdot c_{RES}$, with $\delta=\gamma-\beta>0$, captures the impact of congestion for RES.

%\textbf{Optimal coordinated RES allocation:} 
The allocation of the available RES, $\mathcal{ER}$, to the consumers could be carried out optimally through a fully coordinated centralized approach. %Typically, smart grid-enabled information flow from and to the consumers could be exploited to ensure that (a) the RES is fully used and fairly distributed among the consumers and (b) no consumer would pay the excess congestion penalty. 
This can be implemented by having consumers issue energy demand requests to a central server that processes the requests; if the aggregate amount of loads does not exceed $\mathcal{ER}$, they are all selected during the day-zone, otherwise, certain loads are selected so that their aggregate amount is less or equal to $\mathcal{ER}$ and the remaining non-selected loads are shifted to the night-zone. Besides the increased communication and processing requirements, a coordinated RES allocation is a complex process that needs to take into account fairness criteria, the complex interplay between energy consuming devices, %the heterogeneity of the micro-grid environments often consisting of fairly diverse components such as electric cars, wind turbines, solar farms and all kinds of different appliances with various capabilities and requirements, weather conditions, 
user preferences/comfort, \etc, and it is not scalable. In addition, a major concern of a centralized solution is the privacy of consumers, since they should reveal their demands/constraints to a central server. Also, there is no guarantee that consumers will report their demands truthfully.
%Note that when consumers increase their demand when avoiding the competition, the optimal allocation becomes more complex since it should compare the aggregate cost for consuming nonRES at day-zone against the aggregate cost for consuming (more) nonRES at night-zone. These decisions are communicated to the consumers through the smart-grid facility. 

%
%
%Under the optimal coordinated energy source allocation scheme, %the full information processing and decision-making tasks lie with a central entity.
%micro-grid operators %, operators of energy efficient neighbourhoods and operators of building blocks that host smart metering 
%use the smart grid-related information so as to coordinate the actions of all different stakeholders involved and secure that (a) the local RES - and storage systems - are optimally used and fairly distributed among consumers and (b) no one would pay the excess congestion penalty. In essence, individual customers/consumers that range from owners/operators of large building and facilities to house customers, are expected to issue energy demand requests to a central server, which monitors RES, possesses precise information about its availability, and assigns it so that the aggregate consumers' cost is minimized. Ideally, the efficient exploitation of the smart meter attributes requires the use of a highly intelligent mechanism which takes into account the complex interplay between energy consuming devices, weather conditions, building dynamics, user preferences and comfort, grid status, availability of RES or storage systems \etc.
%
%Overall, in an micro-grid environment where the aggregate demand exceeds the RES capacity of $\mathcal{ER}$ energy units, %with RES capacity of $\mathcal{ER}$ energy units, whereby such an coordination system serves the energy requests of $N$ consumers that jointly shape an energy demand exceeding the RES capacity, 
%exactly $\mathcal{ER}$ units of demand would be directed to RES and no one would pay the excess congestion penalty cost. Advanced decentralized realizations of optimal coordination mechanisms may optimally shape the $\mathcal{ER}$ capacity, pricing and the energy demand of all different consumers within the operator's entity so as the locally available RES is exploited to the maximum while the needs from purchasing from the top-priced conventional grid are minimized.

%\textbf{Uncoordinated energy source selection:} As already mentioned, the design and implementation of a coordinated energy source selection is complex due to the required information exchange, fairness criteria to be considered, heterogeneity of the micro-grid environments (often consisting of fairly diveRES components such as electric cars, wind turbines, solar farms and all kinds of different appliances with various capabilities and requirements), \etc. 
Consequently, in this paper we investigate a simpler to implement distributed, uncoordinated energy source selection approach. Each consumer acts selfishly aiming at minimizing the cost of the acquired energy. Based on the prices' definitions, if the consumers' average cost is minimized, the amount of RES actually used (and not curtailed) should be maximized. As aforementioned in the introduction, the intuitive tendency to opt for RES combined with its scarcity foster the tragedy of commons effects and highlight the game-theoretic dynamics behind the energy source selection task. Thus, the decision-making on energy source selection can be formulated in game-theoretic terms, whereby $N$ rational players/consumers compete against each other for a capacity of $\mathcal{ER}$ low-cost energy units. We will compare the optimal (theoretical) performance (expressed in terms of the minimum average aggregate consumer cost) achieved via the centralized coordinated energy allocation, against the one under the distributed, uncoordinated approach proposed in this paper. 



%Advanced decentralized realizations of the aforementioned optimal centralized coordination may also be considered, but this is outside the focus of this paper. We are only interested in comparing the optimal (theoretical) performance (expressed in terms of the minimum average aggregate consumer cost) delivered through centralized or decentralized coordinated energy allocation, against that under the (simpler to implement) uncoordinated approach considered in this paper. 

%
%Typically, the research in the optimization of energy grid system operation, amounts to defining and solving a system-level optimization problem based on a centralized objective function. However, under heterogeneous micro-grid networks, often consisting of different components such as electric cars, diesel generators, wind turbines, solar farms and all different appliances with various capabilities and objectives, specific objective function for each individual component should be taken into account. Indeed, considering the demand side, in the absence of central coordination, each consumer acts selfishly aiming at minimizing the cost of the acquired energy. The intuitive tendency to opt for RES combined with its scarcity, foster tragedy of commons effects and highlight the game-theoretic dynamics behind the energy source selection task.
%
 %coming up with optimal decisions for operating their associated energy consuming devices. 
%In doing so, they are assumed here to hold precise information about the RES capacity and the pricing on all energy sources involved (RES, day and nigh time grid energy). They also may or may not hold precise information about the actual demand, \ie competition. These assumptions are fairly realistic and feasible since the decision-making task can be assigned to intelligent meters that leverage the available information (through the communication network layer within the smart grid infrastructure) to perform complex computations and best respond to others' actions. Furthermore, these meters may account for user preferences and comfort in the reasoning process. To this end, we assume \textit{two} generic energy consumption profiles corresponding to low-to-moderate and high consumption level energy users.

\vspace{-2pt}
\subsection{Game setting}\label{sec:game_definition}

Overall, the distributed, uncoordinated decision-making on energy source selection can be expressed via the following game: 
\begin{definition}\label{def:energy_source_game}
An \emph{Energy Source Selection Game} is a tuple
\\$\Gamma=(\mathcal{N}, \mathcal{ER}, \{\vartheta_{i}\}_{i\in \mathcal{N}}, \{A_{i}\}_{i\in\mathcal{N}},  \{w_{A_{i},\vartheta_{i}}\}_{i\in \mathcal{N}})$, where:
\begin{itemize}
\item $\mathcal{N}=\{1,...,N\}$, $N>1$, the set of energy consumers/users acting as players,
\item $\mathcal{ER}>0$, the RES capacity in energy units,
\item $\vartheta_{i} \in \Theta=\{0,1,...,M-1\}$ is the type of user $i$ ($i\in \mathcal{N}$) corresponding to an energy consumption profile (load) with demand of $E_{\vartheta_i}$ energy units. It should hold that $M \leq N$ and it is assumed that $E_0<E_1<...<E_{M-1}$, %where $0$ ($1$) stands for low-to-moderate (high) energy level users with demand of $E_0$ ($E_1$, $E_1 \geq E_0$) energy units,
\item $A_{i} $ is the pure strategy of player $i$ taking values in the set of potential pure strategies $\mathcal{A}=\{RES,nonRES\}$ (same for all consumers $i\in \mathcal{N}$), consisting of the choices ``low-priced renewable-source energy'' (RES) and ``medium-priced night-zone non-renewable-source energy'' (nonRES),
%\item $s_{i}: \Theta \rightarrow A$, the strategy function for each consumer $i\in \mathcal{N}$,
\item $w_{RES,\vartheta_i}$, $w_{nonRES,\vartheta_i}: \mathbb{R}^2 \rightarrow \mathbb{R}$ are cost functions indexed by the pure strategy RES or nonRES, and the user type $i\in \mathcal{N}$. 
\end{itemize}
\end{definition}
In addition, 
\begin{itemize}
\item a consumer has an energy demand level $E_{\ell}$ with probability $r_{\ell}$, where $\sum_{\ell\in \Theta} r_{\ell}=1$,
\item $\mathbf{p}_{\boldsymbol \vartheta_i}=(p_{RES,\vartheta_i}, p_{nonRES,\vartheta_i})^T$ is the mixed strategy of user $i$ with profile $\theta_i$ and let $\mathbf{p}=(\mathbf{p_0}^T;\mathbf{p_1}^T;...;\mathbf{p_{M-1}}^T)$,
\item $\upsilon_{RES, \theta_i}(p)$,  $\upsilon_{nonRES, \theta_i}(p)$ are expected cost values for each pure strategy RES, nonRES, respectively, of user $i$ with type $\theta_i$ given the mixed strategies of all other users; the expected value is over both the strategy choice by other consumers and their energy profile.   
%\item $c_{i}^{N_0}(A_i,\vartheta_i)$ are cost functions under particular type, $\vartheta_i$, and strategy, $A_i$, that are basically determined based on $w_{RES,\vartheta_i}(\cdot)$ and $w_{nonRES,\vartheta_i}(\cdot)$.
%$c_{i}^{(\cdot)}(s(\vartheta),\vartheta)= c_{i}^{(\cdot)}(s_{i}(\vartheta_{i}),s_{-i}(\vartheta_{-i}),\vartheta_{i},\vartheta_{-i})$, are the expected cost functions for each consumer $i\in \mathcal{N}$, for every type profile $\vartheta\in \times_{k=1}^{N}\Theta_{k}$ and strategy profile $s(\vartheta)\in \times_{k=1}^{N}S_{k}$, ($c_{i}^{N_0}(s(\vartheta),\vartheta)$ are cost functions under particular type, $\vartheta$, and strategy, $s(\vartheta)$, profile with $N_0=N-\sum_k \vartheta_k$), that are functions of $w_{RES,\vartheta_i}(\cdot)$ and $w_{nonRES,\vartheta_i}(\cdot)$,
\end{itemize}

The number of users with energy profile $E_\ell$, $\ell \in \Theta$, taking action $RES$, excluding player $i$, is denoted as $n_{\ell}$. %and it is equal to $\sum_{j \in \mathcal{N}\setminus {i}} \mathbf{1}_{(A_j=RES)}\cdot \mathbf{1}_{(\theta_j=\ell)}$. %and similarly the corresponding number of competitors of high energy profile is $n_{1}$ ($=\sum_{j \in \mathcal{N}\setminus {i}} A_j \mathbf{1}_{(\theta_j=1)}$). 
The cost function $w_{RES,\vartheta_i}(.)$, is non-decreasing with the {number} of players competing for RES ($n_{\ell}$,  $\ell \in \Theta$) and has the same value for all players of the {same} energy profile that play the same strategy. Let us denote as $\mathbf{n}=(n_0, n_1,..., n_{M-1})^T$. %consumers who decide to compete for RES undergo the risk of incurring congestion penalties. 
In addition, $w_{RES,\vartheta_i}(.)$ is determined %based on the actions taken by the entire population, on the one side, and 
based on the allocation rule that the micro-grid operator adopts to distribute RES among those who compete for it. %, on the other side. %We denote an action profile by the vector $a=(a_{i},a_{-i})\in\times_{k=1}^{N}A_{k}$, where $a_{-i}$ denotes the actions of all other consumers but player $i$ in the profile $a$. The full set of $2^N$ different action profiles maps into $(N_0+1)(N-N_0+1)$ different action \emph{meta-profiles}, where $N_0$ is the number of low-to-moderate energy consumption users. Each meta-profile $a(k_0,k_1)$, with $k_0$, $k_1$ denoting the number of competitors of low-to-moderate and high energy profile, respectively, encompasses all different action profiles that correspond to the same number of consumers of the two energy profiles competing for RES. The expected costs for these $k_0$, $k_1$ consumers and for the $N-(k_0+k_1)$ ones choosing directly the medium-priced nonRES alternative are functions of $a(k_0,k_1)$ rather than the exact action profile.
Thus, the cost for a consumer $i\in \mathcal{N}$ of consumption profile $\vartheta_i$ (and associated energy demand $E_{\vartheta_i}$) that plays the action $RES$ is:
\begin{small}
\begin{align}
w_{RES,\vartheta_i}(n) =
rse_{\vartheta_i}(n) \cdot c_{RES} + (E_{\vartheta_i}-rse_{\vartheta_i}(n) ) \cdot c_{nonRES,D},
\label{eq:RES_cost}
\end{align}
\end{small}
where $rse_{\vartheta_i}(n) $ is the amount of RES allocated to consumer profile $\vartheta_i$. Note that all consumers $j \neq i, ~j \in \mathcal{N}$ with $\vartheta_j=\vartheta_i$ and strategy RES are granted with the same amount of RES. On the other hand, the cost for those avoiding competition, $w_{nonRES,\vartheta_i}$, is oblivious to others' actions, that is, 
%\vspace{-7pt}
%\small
\begin{equation}
w_{nonRES,\vartheta_i}=E_{\vartheta_i} \cdot c_{nonRES,N}.
\label{eq:nonRES_cost}
\end{equation}
\normalsize
%In general, the cost $c_i^{N_0}(a_i,a_{-i})$ for consumer $i$ under the action profile $a=(a_i,a_{-i})$ is
%
%%\vspace{-10pt}
%\small
%\begin{eqnarray}
%\hspace{-1pt}c_{i}^{N_0}(a_{i},a_{-i})=\left\{
%\begin{array}{l l}
%\hspace{-5pt}w_{RES,\vartheta_i}(N_{RES,0}(a), n_{1}(a)), \text{if~ $a_{i}=RES$} \nonumber\\
%\hspace{-5pt}w_{nonRES,\vartheta_i}, \text{if~ $a_{i}=nonRES$} \nonumber\\
%\end{array} \right.
%\label{equ:consumer_cost_profile}
%\end{eqnarray}
%\normalsize
%
%where $N_{RES,0}(a), n_{1}(a)$ are the numbers of competitors of low-to-moderate and high energy profile \textit{but} player $i$, respectively, under action profile $a$.
Besides the two pure strategies, RES, nonRES, the consumers may also randomize over them\footnote{Note that a pure strategy is a special case of a mixed strategy.}. We mainly draw our attention on symmetric mixed strategies for each energy consumption profile, since these can be more helpful in dictating practical strategies in real systems. In particular, as defined above a player's \emph{mixed strategy} corresponds to a probability distribution $\mathbf{p}_{\boldsymbol \vartheta_i}$. %=(p_{RES,\vartheta_i},p_{nonRES,\vartheta_i})$, where $p_{RES,\vartheta_i}$ and $p_{nonRES,\vartheta_i}$ are the probabilities of selecting the pure strategies RES and nonRES, respectively, with $p_{RES,\vartheta_i}+p_{nonRES,\vartheta_i}=1$ ($\theta_i \in \Theta$). %Also, let $p=(p_0^T;p_1^T)$ be the matrix with lines the symmetric mixed strategies for each energy profile, namely, $p_0=(p_{RES,0}; p_{nonRES,0})$ and $p_1=(p_{RES,1}; p_{nonRES,1})$. %associated with the two energy profiles. 
Assume that consumers play the mixed-action strategy $\mathbf{p}$. In this case, the expected costs for player $i$, with energy profile $\theta_i$, when playing RES (nonRES) is %is a weighted sum of the cost functions $w_{RES,\vartheta_i}(\cdot)$ and $w_{nonRES,\vartheta_i}(\cdot)$ and is given by
%\vspace{-15pt}
%\small
%\begin{eqnarray}\label{eq:costs_symRES}
%    c_{i}(RES,p, p_{0}, p_{1}) = \sum_{N_0=0}^{N-1}\sum_{N_{RES,0}=0}^{N_0}\sum_{n_{1}=0}^{N-1-N_0}    w_{RES,\vartheta_i}(\cdot)  && \nonumber\\
%    \cdot B(n_{1};N-1-N_0,p_{RES,1}) \cdot  B(N_{RES,0};N_0,p_{RES,0}) && \nonumber\\
%      \cdot B(N_0;N-1,p) &&
%    \vspace{-10pt}
%\end{eqnarray}
%\normalsize
%\begin{align}
%   \upsilon_{ \theta_i}^{N_0}(p_0,p_1) = &  p_{RES,\theta_i} ~\upsilon_{RES, \theta_i}^{N_0}(p_0,p_1)  \nonumber \\&  + p_{nonRES,\theta_i} \upsilon_{nonRES, \theta_i}^{N_0}(p_0,p_1),\label{eq:costs_tot}
%\end{align}
%\noindent where the expected costs of the pure strategies RESs, nonRES of consumer $i$ are expressed as
\begin{align}& \upsilon_{RES, \theta_i}(\mathbf{p})=  \sum_{N_{0}=0}^{N}~ \sum_{N_{1}=0}^{N-N_0} ... \sum_{N_{M-1}=0}^{N-\sum_{k=0}^{M-2}{N_k}}\nonumber\\  & \Bigl( \sum_{n_{0}=0}^{N_0-\mathbf{1}_{(\theta_i=0)}}~\sum_{n_{1}=0}^{N_1-\mathbf{1}_{(\theta_i=1)}}...  \sum_{n_{M-1}=0}^{N_{M-1}-\mathbf{1}_{(\theta_i=M-1)}} ~w_{RES,\vartheta_i}(n) \cdot \nonumber\\
 &B(n_{0};N_0-\mathbf{1}_{(\theta_i=0)},p_{RES,0}) \cdot \nonumber \\& B(n_{1};N_1-\mathbf{1}_{(\theta_i=1)},p_{RES,1}) \cdot... \nonumber \\& B(n_{M-1};N_{M-1}-\mathbf{1}_{(\theta_i=M-1)},p_{RES,M-1})\Bigr) \cdot... \nonumber \\&B(N_0;N,r_0)B(N_1;N,r_1)...B(N_{M-1};N,r_{M-1}),  \label{eq:costs_symRES}
\end{align}
\begin{align}
\label{eq:costs_symnonRES}
   \upsilon_{nonRES, \theta_i}(\mathbf{p}) = w_{nonRES,\vartheta_i},
\end{align}
\noindent where $N_{\vartheta_i}$ is the number of consumers of type $\theta_i\in \Theta$, $B(n_{0};N_0-\mathbf{1}_{(\theta_i=0)},p_{RES,0}) $ is the value of the Binomial probability distribution with parameters $N_0-\mathbf{1}_{(\theta_i=0)}$ and $p_{RES,0}$, for $n_{0}$ consumers of energy profile $E_0$ competing for RES, and $B(N_0;N,r_0)$ is the value of the Binomial distribution for $N_0$ consumers of energy profile $E_0$ with parameters $N$ and $r_0$. %and $B(N_0;N-1,p)$ is the Binomial probability distribution with parameters $N-1$ and $p$, for $N_0$ consumers of low-to-moderate energy consumption users.

The Energy Source Selection Game has at least one (mixed) Nash Equilibrium, as it is a game with finite set of players and finite strategy sets (Nash's original result). To derive the equilibria, we recall that any mixed strategy equilibrium $\mathbf{p^{NE}}=(\mathbf{p_0^{NE}}, \mathbf{p_1^{NE}}, ..., \mathbf{p_{M-1}^{NE}})^T$ must fulfill

\vspace{-5pt}
\small
\begin{equation}\label{eq:cost_equality_mixed}
\upsilon_{RES, \theta_i}(\mathbf{p^{NE}})= \upsilon_{nonRES, \theta_i}(\mathbf{p^{NE}}), ~\forall \theta_i \in \Theta.
\end{equation}
\normalsize

Namely, the expected costs of each pure strategy that belongs to the support of the equilibrium mixed strategy are equal. %By Eq. (\ref{eq:costs_symnonRES}), the expected cost of action nonRES, $ \upsilon_{nonRES, \theta_i}(p_0^{NE}, p_1^{NE})$, is insensitive to $N_0$ (or $N-N_0$), that is, the cardinality of the two sets of consumers. 
%In addition, in order to accommodate different risk attitudes, it becomes $c_{i}^{(\cdot)}(nonRES,p) = (\epsilon_{\vartheta_i}  E_{\vartheta_i})  c_{nonRES,N}$. 
%Likewise, by Definition \ref{def:energy_source_game}, the expected cost of action RES is given by $c_{i}^{(\cdot)}(RES,p) =\sum_{N_0=0}^{N}c_{i}^{N_0}(RES,p) B(N_0;N,r)$, where the cost $c_{i}^{N_0}(RES,p)$ follows equation (\ref{eq:costs_symRES}). 
The cost for the RES pure strategy, $\upsilon_{RES, \theta_i}(p^{NE})$, given by Eq. (\ref{eq:costs_symRES}), under large populations of consumers, can be approximated by

\vspace{-5pt}
\small
\begin{align}
&\upsilon_{RES, \theta_i}(\mathbf{p^{NE}})=w_{RES,\vartheta_i}(\mathbf{n}),\nonumber \\&
\mathbf{n}=(r_0 p_{RES,0}^{NE}, r_1 p_{RES,1}^{NE}, ...,  r_{M-1} p_{RES,M-1}^{NE})^T\cdot (N-1).\label{eq:gen_costs_symRES}
\end{align} \normalsize
%In the rest of the paper, we will apply the definition of $n$ of Eq. \eqref{eq:gen_costs_symRES}.


%As a result, the player $i$, with energy profile $\theta_i$, and mixed strategy $p_{\vartheta_i}=(p_{RES,\vartheta_i};p_{nonRES,\vartheta_i})$ will have an expected cost equal to
%\begin{align}
%   \upsilon_{ \theta_i}^{N_0}(p_0,p_1) = &  p_{RES,\theta_i} ~\upsilon_{RES, \theta_i}^{N_0}(p_0,p_1)  \nonumber \\&  + p_{nonRES,\theta_i} \upsilon_{nonRES, \theta_i}^{N_0}(p_0,p_1).\label{eq:costs_tot}
%\end{align}
\subsection{Risk Attitude of the Consumers}
%Various expressions of consumers' risk attitude are considered yielding different variants of the game prescribed under Definition \ref{def:energy_source_game}. 
In the game of Definition \ref{def:energy_source_game}, we take into account various risk attitudes of the consumers.  %Specifically, the risk attitude of a consumer is expressed by adjusting the reported energy demand to its level of risk. 
A \textit{risk-seeking} consumer when playing the RES strategy will gamble its total energy demand in the game; that is, it will attempt to have its entire possible energy demand served by the RES and minimize its cost. On the contrary, a \textit{risk-conservative} consumer with RES strategy gambles part of its possible total energy demand. 

The above described risk attitude is captured by adjusting the energy demand of consumer $i\in \mathcal{N}$ when served at the night zone as $\epsilon_{\vartheta_i} E_{\vartheta_i}$, where the parameter $\epsilon_{\vartheta_i}$ is the ``{risk-aversion degree}'' and $\epsilon_{\vartheta_i}\geq 1$. For example, $\epsilon_{\vartheta_i}=1$ in case of a risk-seeking consumer $i$ and $\epsilon_{\vartheta_i}>1$ in case of a risk-conservative consumer $i$. The consumer's $i$ energy demand when competing for RES remains equal to $E_{\vartheta_i}$. Note that we assume that consumers with the same energy profile $\theta_i$ will have the same risk attitude. In order to capture these risk attitudes, Eq. \eqref{eq:nonRES_cost} should take the form
\begin{small}
\begin{equation}
w_{nonRES,\vartheta_i}=\epsilon_{\vartheta_i} \cdot E_{\vartheta_i} \cdot c_{nonRES,N}.
\label{eq:nonRES_cost2}
\end{equation}
\end{small}