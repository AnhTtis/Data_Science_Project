\vspace{-0.1in}
\section{Conclusions}\vspace{-0.05in}
\label{sec:conclusions}

In this paper, we analyze the uncoordinated decisions of self-interested risk-aware consumers participating in an energy sharing community and a decentralized ESM through a non-cooperative game-theoretic framework. We prove the existence of dominant solutions and/or NE for different energy tariff values and renewable energy allocation policies, as well as under various levels of consumers' risk-aversion and energy demand. %We obtain closed-form expressions of these NE. %and quantify their efficiency compared to the solutions of a centralized RESs allocation problem using the PoA metric. 
For low and medium values of RESs production, consumers are shown to over-compete for RESs compared to the optimal solution giving rise to higher cost values. However, the incorporation of consumers’ attitude toward risk in the model considered in this work has revealed that the PoA peaks can be reduced and even alleviated as the energy community becomes more risk conservative. Moreover, the PA policy outperforms ES in terms of social cost. Finally, choosing a fair NE among all possible ones is also studied using a distributed algorithm for choosing consumers' actions.

From a methodological point of view, a natural direction for further investigation is to account for more complex behavioral human-driven models of consumers' decision-making (\eg \cite{McKP95}). %Furthermore, our previous work in \cite{stai2022} showed the variability of the efficiency of decentralized DRPs based on the available RESs production in the community. Therefore, I
Also, we could incorporate and compare various tariff schemes, which account both for competition across multiple energy sources and time steps. Finally, the authors in \cite{rodriguez2021value} showed that information has a major impact on the efficiency loss in decentralized DRPs. Thus, future work will analyze consumer competition in an energy community under more realistic assumptions of imperfect information.




