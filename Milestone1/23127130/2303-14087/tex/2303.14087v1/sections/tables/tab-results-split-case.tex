\begin{table}
\resizebox{\linewidth}{!}{
\begin{tabular}{@{} l r rr rr @{}}
\toprule
 & No AO $\% \uparrow$ & \multicolumn{2}{c}{Single AO $\% \uparrow$} & \multicolumn{2}{c}{Multiple AO $\% \uparrow$} \\
\cmidrule(l{0pt}r{2pt}){2-2} \cmidrule(l{2pt}r{0pt}){3-4} \cmidrule(l{2pt}r{0pt}){5-6}
Model & \textbf{Accuracy} & \partdet & +\mtype{}\axis{}\orig & \partdet & +\mtype{}\axis{}\orig  \\
\midrule
\opdrcnnc~\cite{jiang2022opd}  & \best{58.6} & 43.6 & 11.8 & 37.6 & 8.6\\
\opdrcnno~\cite{jiang2022opd}  & 57.5 & 34.8 & 0.5 & 30.5 & 0.4\\
\opdrcnnop & 50.8 & 40.0 & 10.0 & 34.0 & 8.9 \\
\opdformerc & 27.3 & 60.1 & 21.9 & 36.1 & 14.6 \\
\opdformero  & 16.7 & 59.4 & 2.5 & 35.8 & 1.5\\
\opdformerp & 35.0 & \best{61.4} & \best{28.7} & \best{40.2} & \best{15.2} \\
\bottomrule
\end{tabular}
}
\caption{We compare the performance of the models for images with no/one/multiple articulated objects (AO) on the \ourdatamulti validation set.  For `No AO', we compute the percent of frames for which the method correctly predicted there was no openable parts.
}
\label{tab:results-OPDMulti-split}
\end{table}