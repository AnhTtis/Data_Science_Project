\begin{figure*}
\centering
\setkeys{Gin}{width=\linewidth}
\begin{tabularx}{\linewidth}{Y Y Y Y Y Y}
\toprule
\small{GT} &
\imgclip{0}{figure/qua_single/gt/174-8460__0.png} & 
\imgclip{0}{figure/qua_single/gt/174-8460__1.png} &
\imgclip{0}{figure/qua_single/gt/174-8460__2.png} &
\imgclip{0}{figure/qua_single/gt/241-5820__0.png} &
\imgclip{0}{figure/qua_single/gt/241-5820__1.png}\\

\small{\opdrcnnop} &
\imgclip{0}{figure/qua_single/opdrcnnp/174-8460__1.png} & 
\imgclip{0}{figure/qua_single/opdrcnnp/174-8460__0.png} &
Miss &
\imgclip{0}{figure/qua_single/opdrcnnp/241-5820__0.png} &
\imgclip{0}{figure/qua_single/opdrcnnp/241-5820__1.png}\\
\small{Axis (origin) error} & 7.537 & 2.098 & - & 6.078 (0.067) & 8.752 (0.149)\\

\small{\opdformerp} &
\imgclip{0}{figure/qua_single/opdformerp/174-8460__1.png} & 
\imgclip{0}{figure/qua_single/opdformerp/174-8460__0.png} &
\imgclip{0}{figure/qua_single/opdformerp/174-8460__2.png} &
\imgclip{0}{figure/qua_single/opdformerp/241-5820__0.png} &
\imgclip{0}{figure/qua_single/opdformerp/241-5820__1.png}\\
\small{Axis (origin) error} & 2.131 & 2.956 & 2.153 & 3.247 (0.019) & 1.975 (0.06)\\

\midrule

\small{GT} &
\imgclip{0}{figure/qua_multi/gt/106-5040__0.png} & 
\imgclip{0}{figure/qua_multi/gt/106-5040__1.png} &
\imgclip{0}{figure/qua_multi/gt/252-4800__0.png} &
\imgclip{0}{figure/qua_multi/gt/252-4800__1.png} &
\imgclip{0}{figure/qua_multi/gt/252-4800__2.png}\\

\small{\opdrcnnop} &
\imgclip{0}{figure/qua_multi/opdrcnnp/106-5040__0.png} & 
Miss &
\imgclip{0}{figure/qua_multi/opdrcnnp/252-4800__1.png} &
\imgclip{0}{figure/qua_multi/opdrcnnp/252-4800__2.png} &
\imgclip{0}{figure/qua_multi/opdrcnnp/252-4800__0.png}\\
\small{Axis (origin) error} & 4.458 (0.013) & - & 6.329 (0.057) & 5.757 (0.056) & 8.305 (0.029)\\

\small{\opdformerp} &
\imgclip{0}{figure/qua_multi/opdformerp/106-5040__0.png} & 
\imgclip{0}{figure/qua_multi/opdformerp/106-5040__1.png} &
\imgclip{0}{figure/qua_multi/opdformerp/252-4800__0.png} &
\imgclip{0}{figure/qua_multi/opdformerp/252-4800__2.png} &
\imgclip{0}{figure/qua_multi/opdformerp/252-4800__1.png}\\
\small{Axis (origin) error} & 2.021 (0.087) & 4.099(0.234) & 1.038 (0.058) & 1.619 (0.063) & 1.569 (0.096)\\

\bottomrule
\end{tabularx}
\caption{
Example predictions on the \ourdatamulti val split. 
The first row in each group is the ground truth (GT) with the motion axis in green.  
The following rows are predictions from \opdrcnnop and \opdformerp with axis error and origin error indicated if the motion type is rotation.
The GT axis is in \textcolor{blue}{blue} and the predicted axis is in \textcolor{green}{green} if it is within $5^\circ$ of the GT, \textcolor{orange}{orange} if between $5^\circ$ and $10^\circ$, and \textcolor{red}{red} if the angle difference is greater than $10^\circ$.
The axis origin is visualized with the same color scheme using error thresholds of 0.1 and 0.25.
Overall, \opdformerp provides significantly more accurate openable part predictions, in particular for the scenarios in the bottom that contain multiple objects or multiple parts.
}
\label{fig:vis-compare-real-multi}
\end{figure*}
