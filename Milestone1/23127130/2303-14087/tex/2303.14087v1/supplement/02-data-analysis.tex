\section{Additional dataset details}
\label{sec:supp:opdmulti}

We provide more statistics for the \ourdatamulti dataset (\Cref{sec:supp-parts-dist}), examples of different part ratio coverages in frames (\Cref{sec:supp-part-coverage}), details about correcting inaccurate masks (\Cref{sec:supp-mask-correction}), and distribution of part and motion types for the train/val/test splits (\Cref{sec:supp-part-dist-expr}).  We also provide example visualization clarifying the difference between the different coordinate frames we consider for our models (\Cref{sec:supp-coord-frames}) and information about how we extract object poses for training (\Cref{sec:supp-object-pose}).







\subsection{Distribution of openable parts}
\label{sec:supp-parts-dist}

The \ourdatamulti dataset we constructed is composed of approximately $64K$ RGBD frames extracted from the MultiScan dataset~\cite{mao2022multiscan}.
Since our task focuses on detecting openable parts in images from more realistic real-world scenes with variable number of parts, we report the distribution of the openable parts across the dataset frames (see \Cref{tab:data-opdmulti-partstats}).
In \ourdatamulti, around $60\%$ of the frames that have at least one openable object (ignoring the frames that has 0 part) contain 1 part.
This distribution is approximately the same across train/val/test sets.
Overall, we observe that there is a long-tail distribution for the number of openable parts observed in these RGBD frames captured by people from real scenes.

\begin{table}
\centering
\begin{tabular}{@{} l rrrrr @{}}
\toprule
& \multicolumn{5}{c}{\# Parts}\\
\cmidrule(lr{0pt}){2-6}
split & 0 & 1 & 2 & 3 & 4+
\\
\midrule
train & 31189 & 7705 & 3183 & 1261 & 664 \\
val & 6424 & 1853 & 1055 & 516 & 320 \\
test & 6818 & 1839 & 826 & 336 & 224 \\
\midrule
total & 44431 & 11397 & 5064 & 2113 & 1208 \\
\bottomrule
\end{tabular}
\caption{Distribution of openable parts over train/val/test splits, indicating the number of frames having different numbers of openable parts.}
\label{tab:data-opdmulti-partstats}
\end{table}


\subsection{Part mask image coverage ratio examples}
\label{sec:supp-part-coverage}


We show examples of parts with different part coverage ratios (e.g. fraction of the frame covered by the part) in \Cref{fig:pix-ratio}. 
We note that images with low part coverage ratio (below 5\%) do not provide sufficient information for openable part detection, thus we exclude these parts from our evaluation.


\begin{figure}
\centering
\setkeys{Gin}{width=\linewidth}
\begin{tabularx}{\linewidth}{ Y Y Y Y}

\toprule

$(0-5] \%$ &
$(5-10] \%$ &
$(10-15] \%$ &
$\ge15 \%$ \\

\midrule
\imgclip{0}{figure/pixel_vis/pix_1_5/0.png} &
\imgclip{0}{figure/pixel_vis/pix_6_10/0.png} &
\imgclip{0}{figure/pixel_vis/pix_11_15/0.png} &
\imgclip{0}{figure/pixel_vis/pix_16/0.png}\\

\imgclip{0}{figure/pixel_vis/pix_1_5/1.png} &
\imgclip{0}{figure/pixel_vis/pix_6_10/1.png} &
\imgclip{0}{figure/pixel_vis/pix_11_15/1.png} &
\imgclip{0}{figure/pixel_vis/pix_16/1.png}\\


\bottomrule
\end{tabularx}
\caption{
Examples of \ourdatamulti frames with different per-part pixel ratio (percent of pixels in that frame for a given part).  We note that when the pixel ratio for a part is extremely low (<5\%), it is challenging for humans to recognize the part.  Thus, we do not include such parts in our evaluation.  
}
\label{fig:pix-ratio}
\end{figure}


\begin{figure}[t]
\includegraphics[width=1\linewidth]{figure/mask_bad_vs_new/mask_bad_vs_new.pdf}
\caption{Examples of inaccurate openable part masks and corresponding updated masks. The left column shows the masks with errors, while the right column shows the updated maskes. The first row is an example of the incomplete mask error, and the second row shows the shifted mask error.
}
\label{fig:mask_correct}
\end{figure}
\subsection{Mask correction}
\label{sec:supp-mask-correction}

As noted in the main paper, it is possible for some frames to have inaccurate masks due to issues in the reconstruction.  Since the annotations are on the 3D reconstructions, if the reconstruction is incomplete or the camera pose for the frame is inaccurate, then the annotated mask when projected onto the image will be inaccurate.
\Cref{fig:mask_correct} shows examples of inaccurate projected mask annotations.  From the top row, we see an example of an incomplete mask due to a shiny surface that is not reconstructed well.
The bottom row shows an example of a shifted mask due to an inaccurate camera pose estimate for the frame.
We note that despite the inaccurate masks, the projected motion axis is good.  

To ensure that the data for our evaluation is accurate and high-quality, we manually inspect all frames in the validation and test splits and flag frames with bad mask annotations.
We then use the Toronto Annotation Suite \cite{torontoannotsuite} to re-annotate the bad masks manually.
Considering there may be multiple parts in one frame, we directly re-annotate on the images with bad part masks.
During the annotation procedure, we draw the polygon segmentation for the specific openable part.
In total, we re-annotate 2261 openable part mask segmentations across both the validation and test sets.

\begin{table}[t]
\resizebox{\linewidth}{!}{
\centering
{
\begin{tabular}{@{} l r rrr rr@{}}
\toprule
& & \multicolumn{3}{c}{Part Type} & \multicolumn{2}{c}{Motion Type}\\
\cmidrule(l{0pt}r{2pt}){3-5} \cmidrule(l{2pt}r{2pt}){6-7} 
split & \# part & \door & \drawer & \lid & \mtrot & \mttrans\\
\midrule
train & 18479 & 15904 & 2097 & 478 & 15760 & 2719 \\
val & 6077 & 5124 & 752 & 201 & 4695 & 1382\\
test & 4704 & 3592 & 981 & 131 & 3449 & 1255 \\
\midrule
total & 29260 & 24620 & 3830 & 810 & 23904 & 5356 \\
\bottomrule
\end{tabular}
}}
\caption{OPDMulti statistics for train/val/test splits for part type and motion type.  These statistics are computed after discarding small parts that occupy less than 5\% of a frame (see \Cref{sec:supp-part-coverage}).  Most of the parts we observe in this real world dataset is revolute doors.
}
\label{tab:stats-motion-part}
\end{table}


\subsection{Distribution of motion and openable part type}
\label{sec:supp-part-dist-expr}

We report the distribution of different motion types and openable part types in the \ourdatamulti dataset in \Cref{tab:stats-motion-part}.
This distribution is counted from the data we used for our experiments, which means it is after excluding small parts as described in \Cref{sec:supp-part-coverage}.
From \Cref{tab:stats-motion-part}, we see that \door is the most frequent part type with around $83\%$ ratio in all splits, and \mtrot is more common than \mttrans.
Since \ourdatamulti is constructed from real-world indoor scenes, the distribution shows the frequency of openable parts in a realistic setting.
We analyze the model performance for different part types in \Cref{sec:supp-analysis}.
