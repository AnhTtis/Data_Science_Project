\begin{figure}
\centering
\setkeys{Gin}{width=\linewidth}
\begin{tabularx}{\linewidth}{ Y Y Y Y}

\toprule

$(0-5] \%$ &
$(5-10] \%$ &
$(10-15] \%$ &
$\ge15 \%$ \\

\midrule
\imgclip{0}{figure/pixel_vis/pix_1_5/0.png} &
\imgclip{0}{figure/pixel_vis/pix_6_10/0.png} &
\imgclip{0}{figure/pixel_vis/pix_11_15/0.png} &
\imgclip{0}{figure/pixel_vis/pix_16/0.png}\\

\imgclip{0}{figure/pixel_vis/pix_1_5/1.png} &
\imgclip{0}{figure/pixel_vis/pix_6_10/1.png} &
\imgclip{0}{figure/pixel_vis/pix_11_15/1.png} &
\imgclip{0}{figure/pixel_vis/pix_16/1.png}\\


\bottomrule
\end{tabularx}
\caption{
Examples of \ourdatamulti frames with different per-part pixel ratio (percent of pixels in that frame for a given part).  We note that when the pixel ratio for a part is extremely low (<5\%), it is challenging for humans to recognize the part.  Thus, we do not include such parts in our evaluation.  
}
\label{fig:pix-ratio}
\end{figure}
