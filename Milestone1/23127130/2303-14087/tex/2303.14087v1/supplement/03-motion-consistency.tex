\subsection{Analysis of object part consistency}
\label{sec:supp-part-consistency}

As noted in the main paper, the motion parameters are highly correlated within an object, as well as across objects.
In a single object, parts with the same part category are likely to have similar motion types (e.g., all doors in one cabinet are either all rotating or sliding).  Similarly, the motion axes tend to be consistent across the parts (e.g. all drawers for a cabinet will tend to translate in the same direction, while a cabinet with both drawers and door will tend to have rotation axis for the doors that are perpendicular to the drawers' translation axes). Thus, the motion axes of different parts are likely parallel or perpendicular to each other.  Rotational motion origins follow a similar layout for rotating parts (e.g., the edge of rotating parts is constrained by the part position).

We investigate the consistency of the ground truth (GT) motions in the datasets, as well as how consistent the predictions from our model variants are.  We show that our \opdformerp provides the most consistent predictions.

\begin{table}[t]
\centering
\resizebox{\linewidth}{!}
{
\begin{tabular}{@{} ll rrr r  @{}}
\toprule 
      &       & \multicolumn{3}{c}{axis $\uparrow$} & type $\uparrow$ \\
\cmidrule(l{0pt}r{2pt}){3-5} \cmidrule(l{2pt}r{0pt}){6-6}
Dataset & Model & $1^\circ$ & $5^\circ$ & $10^\circ$ & \\
\midrule
\multirow{4}{*}{\ourdatacad} 
& \opdrcnnc~\cite{jiang2022opd} & 0.05 & 0.47 & 0.75 & 0.99 \\
& \opdrcnnop & 0.15 & 0.75 & 0.89 & 0.99 \\
& \opdformerc & 0.46 & 0.87 & \best{0.92} & 0.99  \\
& \opdformerp &  \best{0.72} & \best{0.89} & \best{0.92} & 0.99 \\
\midrule
\multirow{4}{*}{\ourdatareal} 
& \opdrcnnc & 0.02 & 0.20 & 0.43  & 0.95 \\
& \opdrcnnop & 0.03 & 0.38 & 0.68 & 0.93\\
& \opdformerc & 0.15 & 0.78 & \best{0.89} & 0.99 \\
& \opdformerp & \best{0.35} & \best{0.82} & \best{0.89} & 0.98 \\
\midrule
\multirow{4}{*}{\ourdatamulti} 
& \opdrcnnc & 0.02 & 0.29 & 0.57 & 0.97 \\
& \opdrcnnop & 0.02 & 0.28 & 0.56 & 0.96 \\
& \opdformerc & 0.18 & 0.79 & 0.88 & 0.98\\
& \opdformerp & \best{0.33} & \best{0.84} & \best{0.90} & 0.98 \\
\bottomrule
\end{tabular}
}
\caption{
Consistency of motion type and motion axis predictions on the val set of the three datasets.
For each part pair we check whether motion axes are parallel or perpendicular within three thresholds ($1^\circ$, $5^\circ$, and $10^\circ$).
For each part pair of the same category we check whether predicted motion types are the same.
We report averaged scores over all valid images.
For experiments in \ourtask we evaluate the consistency for each object in the image instead of the whole image.
}
\label{tab:results-consistency}
\end{table}


\mypara{GT Motion Consistency.}
We check how \emph{consistent} the motion types and motion axes are across the single object datasets, \ourdatacad and \ourdatareal, by measuring the percentage of part pairs in the same object that: 1) have the same motion type given the same part category; and 2) are parallel or perpendicular to each other.
For motion type, all part pairs in both datasets are consistent according to our observation.
For the motion axes, we measure the axis consistency for three different angle thresholds (1$^\circ$, 5$^\circ$, and 10$^\circ$).
\ourdatacad matches our hypothesis for all objects across all thresholds, while in \ourdatareal all pairs matched at $10^\circ$, 97\% matched at $5^\circ$, and 68\% pairs at $1^\circ$.
Upon inspection, many non-consistent cases are due to small inaccuracies in the ground truth annotations for the motion axis in \ourdatareal.
This inconsistency is likely due to noise in the annotation.
\Cref{fig:vis-bad-gt} shows an example where there is a slight inconsistency.
The motion axis should have the same direction between the two drawers, but the GT motion axis is slightly different with a 6.29-degree error.

\begin{figure}
\centering
\setkeys{Gin}{width=\linewidth}
\begin{tabularx}{\linewidth}{Y Y}
\imgclip{0}{figure/bad-gt/736-1980__0.png} &
\imgclip{0}{figure/bad-gt/736-1980__1.png} \\
\end{tabularx}
\caption{Example of inaccurate axis annotations in \ourdatareal from \citet{jiang2022opd}.
The parallel motion axes of the two drawers in the same cabinet have about 6.29 degree error. }
\label{fig:vis-bad-gt}
\end{figure}


\mypara{Prediction Consistency.}
To evaluate motion consistency for predicted joints, we also measure pair-wise consistency of part motion predictions.
\Cref{tab:results-consistency} shows the results for \ourdatacad, \ourdatareal and \ourdatamulti.
We see that \opdformerbaseline predictions are much more consistent than \opdrcnn.
This is likely due to the overall better pose prediction and the self-attention mechanism in \opdformerbaseline that allows it to better capture relationships with other parts.
In addition, almost all predictions on \ourdatacad are parallel or perpendicular within $5^\circ$ and $10^\circ$, where \opdformerp has the best consistency. For \ourdatareal and \ourdatamulti, the consistency is low at $1^\circ$ but fairly good at $5^\circ$ and $10^\circ$.


