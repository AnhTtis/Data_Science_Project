\usepackage{iccv}      
\usepackage{times}

% Include other packages here, before hyperref.
\usepackage{graphicx}
\usepackage{amsmath}
\usepackage{amssymb}
\usepackage{booktabs}

% It is strongly recommended to use hyperref, especially for the review version.
% hyperref with option pagebackref eases the reviewers' job.
% Please disable hyperref *only* if you encounter grave issues, e.g. with the
% file validation for the camera-ready version.
%
% If you comment hyperref and then uncomment it, you should delete
% ReviewTempalte.aux before re-running LaTeX.
% (Or just hit 'q' on the first LaTeX run, let it finish, and you
%  should be clear).
\usepackage[pagebackref=true,breaklinks=true,colorlinks,bookmarks=false]{hyperref}

% Support for easy cross-referencing
\usepackage[capitalize]{cleveref}
\Crefname{section}{Sec.}{Secs.}
\crefname{section}{Sec.}{Secs.}
\Crefname{table}{Tab.}{Tabs.}
\crefname{table}{Tab.}{Tabs.}
\Crefname{figure}{Fig.}{Figs.}
\crefname{figure}{Fig.}{Figs.}
\Crefname{appendix}{App.}{Apps.}
\crefname{appendix}{App.}{Apps.}

% Todonotes is useful during development; simply uncomment the next line
%    and comment out the line below the next line to turn off comments
%\usepackage[disable,textsize=tiny]{todonotes}
% \usepackage[textsize=tiny]{todonotes}

% \usepackage{times}
\usepackage[utf8]{inputenc} % allow utf-8 input
\usepackage[T1]{fontenc}    % use 8-bit T1 fonts
\usepackage{url}            % simple URL typesetting
\usepackage{amsfonts}       % blackboard math symbols
\usepackage{nicefrac}       % compact symbols for 1/2, etc.
% \usepackage[nopatch=eqnum]{microtype}      % microtypography
\usepackage{microtype}      % microtypography
\usepackage{xcolor}         % colors

\usepackage{epsfig}
\usepackage{tabularx}
\usepackage{bbm}
\usepackage{color}
\usepackage{wrapfig}
\usepackage{subcaption}
\usepackage{float}
\usepackage{makecell}
\usepackage[percent]{overpic}
\usepackage[numbers,compress]{natbib}
\usepackage{adjustbox}

\usepackage{multirow}
\usepackage{inconsolata}
\usepackage{comment}
\usepackage{xspace}


\newcommand{\stdhide}[1]{}
\newcommand{\std}[1]{\scriptsize{$\pm$#1}}
\newcommand{\stdnew}[1]{\scriptsize{$\pm$#1}}

\newcommand{\ourdatareal}{OPDReal\xspace}
\newcommand{\ourdatacad}{OPDSynth\xspace}
\newcommand{\ourdatamulti}{OPDMulti\xspace}

\newcommand{\ourtask}{OPDMulti\xspace}

\newcommand{\opdformerbaseline}{\textsc{OpdFormer}\xspace}
\newcommand{\opdformer}{\textsc{OpdFormer + WMA}\xspace}

\newcommand{\opdformerowma}{\textsc{OpdFormer-O-W}\xspace}
\newcommand{\opdformerc}{\textsc{OpdFormer-C}\xspace}
\newcommand{\opdformero}{\textsc{OpdFormer-O}\xspace}
\newcommand{\opdformerp}{\textsc{OpdFormer-P}\xspace}

\newcommand{\opdrcnn}{\textsc{OPDRCNN}\xspace}
\newcommand{\opdrcnnc}{\textsc{OPDRCNN-C}\xspace}
\newcommand{\opdrcnno}{\textsc{OPDRCNN-O}\xspace}
\newcommand{\opdrcnnop}{\textsc{OPDRCNN-P}\xspace}
\newcommand{\opdnet}{\textsc{OpdRcnn}\xspace}

\newcommand{\drawer}{\texttt{drawer}\xspace}
\newcommand{\door}{\texttt{door}\xspace}
\newcommand{\lid}{\texttt{lid}\xspace}
\newcommand{\mttrans}{\texttt{prismatic}\xspace}
\newcommand{\mtrot}{\texttt{revolute}\xspace}


% metric name macros
\newcommand{\edir}{$\epsilon_\text{d}$\xspace}
\newcommand{\eorig}{$\epsilon_\text{o}$\xspace}
\newcommand{\axis}{\textbf{A}\xspace}
\newcommand{\orig}{\textbf{O}\xspace}
\newcommand{\mtype}{\textbf{M}\xspace}
\newcommand{\partdet}{\textbf{PDet}\xspace}
\newcommand{\motiondet}{\textbf{MDet}\xspace}


% Highlighting stuff that's changed in the revision
\newcommand{\revised}[1]{{\leavevmode\color{teal}#1}}
% \newcommand{\revised}[1]{#1}

% The set of real numbers
\newcommand{\R}{\ensuremath{\mathbb{R}}}

% The indicator function
\newcommand{\indicator}{\ensuremath{\mathbbm{1}}}

%% Units of measurement
\newcommand{\unit}[1]{\ensuremath{\,\mathrm{#1}}}

\newcommand{\denselist}{\itemsep 0pt\parsep=0pt\partopsep 0pt}
\newcommand{\mypara}[1]{\noindent\textbf{#1}}

\newcommand*{\vcent}[1]{\vcenter{\hbox{#1}}}

\DeclareMathOperator{\sigmoid}{sigmoid}
\DeclareMathOperator{\abs}{abs}

\newcommand\bbox{\text{bbox}}
\newcommand\best[1]{\textbf{#1}}
\newcommand\bestalmost[1]{\textbf{\textit{#1}}}
\newcommand\markbest[1]{#1}

\renewcommand\tabularxcolumn[1]{m{#1}}% for vertical centering text in X column
\newcolumntype{Y}{>{\centering\arraybackslash}X}
\newcommand\imgclip[2]{\adjincludegraphics[Clip={#1\width} {#1\height} {#1\width} {#1\height}]{#2}}
