\section{Conclusion}\label{ch:conclusion}
%In this paper, we proposed the attack method to deviate a system state while reducing an output error, where an input of a sampled-data system is quantized. At that time, we considered a system which has unstable zeros.\par
%We proposed two types of method. The first method modifies the conventional zero dynamics attack by the dynamic quantizer. The second method directly calculates the quantized attack. These methods achieve the goal to deviate a system state. However, the effect on a system output is not same. Though the former method is less influential when the zero dynamics attack is statically quantized, the effect on an output depends on the number of unstable zeros in a system and its values. On the other hand, the latter method is independent of them. Moreover, its effect on an output is less than the former method.\par
%Future work is to propose the quantized attack method on a feedback condrol system. In this paper, we assumed a feedforward control system. So we were able to consider the effect of the quantized attack independently of control law. On the other hand, in a feedback control system, the quantization of the attack affects not only the system output but also the control input, so the problem formulation become more complicated. However, since a feedback control system is widely used in practical applications, this seems to be an important issue.

In this paper, we have studied the effects of quantization on sampled-data systems under zero dynamics attacks. We have proposed methods for generating quantized attacks exploiting the unstable zeros in the plant after discretization. Under these attacks, the system state diverges while the error in the output due to quantization is kept small. 
In particular, we have proposed two types of methods. The first one utilizes the conventional zero dynamics attacks and modifies such attack signals by dynamic quantization. In contrast, in the second method, slightly modified version of zero dynamics attacks is 
quantized. These methods have been compared with the simple
approach of directly quantizing the conventional zero
dynamics attacks, and our analysis shows the they perform
strictly better. 

Our study in this paper has been limited in two respects.
One is that the system is assumed to have unstable zeros.
The case without such zeros can be handled if there are
redundancy in the inputs compared to the outputs 
\cite{zsa}, and we will pursue such an approach.
The other is that the control system setting is in the 
feedforward form. 
Hence, for our future work, we must look at the feedback control case, where the chosen control laws may affect the performance of the quantized attack. We should note that these two system settings have been separately treated for the dynamic quantization in the literature \cite{dquant}. This may indicate the differences in the nature of the problems.
