\section{Numerical Example}\label{ch:example}
In this section, we illustrate the results of our paper through numerical simulations. \par
Consider the following stable continuous-time system with quantized input under attack \cite{enhancement}:
\begin{equation*}
    \begin{split}
        \dot{x}(t)&=\begin{bmatrix}
        0 & 0 & 1 & 0\\
        0 & 0 & 0 & 1 \\
        -2 & 1 & -1 & 1\\
        1 & -1 & 1 & -1
        \end{bmatrix}x(t)+\begin{bmatrix}
        0\\0\\0\\1
        \end{bmatrix}(v(t)+b(t)),\\
        y(t)&=\begin{bmatrix}
        1&0&0&0
        \end{bmatrix}x(t).
    \end{split}
\end{equation*}
This system has one stable zero whose value is $-1$. When discretized using sampling period $T=0.5$, the system has three zeros at $-3.21$, $-0.24$ and $0.61$ in the discrete-time domain. In particular, there is one unstable zero $\lambda=-3.21$. We set the initial state as $x_0=0$ and the input as $u[k]=\sin0.05\pi k+0.5\cos0.025\pi k$. Let the quantization width be  $d=1$. We compare three methods for quantized attacks from Section \ref{ch:nonmin}:~(i)~Static quantization, (ii)~dynamic quantization, and (iii)~$\epsilon$-stealthy approach.\par
(i) First, a static quantizer is used for zero dynamics attacks as described in Section \ref{se:sqzda}. In Fig.~\ref{fig:no:respnonminsta}, the time responses of the output of the plant are shown. The blue line is the inter-sample response under attack while the black circles indicate the values at sampling times. The red line is the output when no attack is present. Fig.~\ref{fig:no:stazda} displays the quantized signals of the control input and the sum of the input and the attack. We observe that the input and thus the plant states are diverging, but this is difficult to detect from the sampled output, which is fairly close to the expected behavior of the output under normal conditions.  From \eqref{eq:no:staticeps}, the theoretical bound on the error between the two output signals is $\sum_{k=0}^\infty\left|CA^kB\right|\frac{d}{2}=5.717$, which is clearly satisfied in the simulation results.

\begin{figure}[t]
\vspace*{2mm}
    \begin{minipage}[h]{1\linewidth}
        \centering
        \includegraphics[width=1\linewidth]{workspace/picture/subpic2/nonmin_sta_res.png}
        \subcaption{Outputs of the system in continuous time and discrete time  \label{fig:no:respnonminsta}}
    \end{minipage}\\
    \vspace*{2mm}
    \begin{minipage}[h]{1\linewidth}
        \centering
        \includegraphics[width=1\linewidth]{workspace/picture/subpic2/input_sta.png}
        \subcaption{Quantized control inputs before and after attacks  \label{fig:no:stazda}} 
    \end{minipage}
    \caption{Time responses of the system under the statically quantized zero dynamics attack. \label{fig:no:stzdaresult}}
\end{figure} 


(ii) Next, we look at attacks which are dynamically quantized from Section \ref{se:dqzda}. Fig.~\ref{fig:no:respnonmin1} presents the output of the plant similarly to the static quantization case in Fig.~\ref{fig:no:respnonminsta}. In Fig.~\ref{fig:no:diffy1}, the difference between the two outputs  with and without attacks is plotted, where the theoretical bound \eqref{eq:no:eps} from Theorem $1$ is $|CB|(1+|\lambda|)\frac{d}{2}=0.038$. In comparison, it is impressive that the difference between the sampled output under attack is much closer to the normal output than in the case of static quantization, which shows the effectiveness of the approach. Fig.~\ref{fig:no:dqzda} displays the control input with and without attack (as in Fig.~\ref{fig:no:stazda}). Note that the attack effects do not appear for a while in the output. This is because the zero dynamics attack signal is initially small.

\begin{figure}[t]
\vspace*{2mm}
    \begin{minipage}[h]{1\linewidth}
        \centering
        \includegraphics[width=1\linewidth]{workspace/picture/subpic2/nonmin_dq_res.png}
        \subcaption{System responses   \label{fig:no:respnonmin1}}
    \end{minipage}\\
\vspace*{2mm}
    \begin{minipage}[h]{1\linewidth}
        \centering
        \includegraphics[width=1\linewidth]{workspace/picture/subpic2/nonmin_dq_diffy.png}
        \subcaption{Difference in the outputs of the system    \label{fig:no:diffy1}}
    \end{minipage}\\
\vspace*{2mm}
    \begin{minipage}[h]{1\linewidth}
        \centering
        \includegraphics[width=1\linewidth]{workspace/picture/subpic2/input_dqzda.png}
        \subcaption{Control inputs and inputs contaminated by the attack.  \label{fig:no:dqzda}}
    \end{minipage}
    \caption{Time responses of the system under the quantized zero dynamics attack. \label{fig:no:dqzdaresult}}
\end{figure} 

(iii) Finally, we discuss the quantized attack based on $\epsilon$-stealthy approach from Section \ref{se:calc}.  The results are shown in Figs.~\ref{fig:no:respnonmin2}--\ref{fig:no:qzdag} as in the dynamic quantization case. Notice that the sampled output is closer to the attack-free one than in the previous two cases. In fact, the difference between attacked/original outputs remains below the bound $\epsilon=|CB|d+10^{-6}=0.018$ provided in Theorem \ref{theo:2}. Moreover, unlike the previous methods, in Fig.~\ref{fig:no:diffy2}, the attack effects appear at time $k=0$. This is because the initial value of the attack should be non-zero. 

\begin{figure}[t]
\vspace*{2mm}
    \begin{minipage}[h]{1\linewidth}
        \centering
        \includegraphics[width=1\linewidth]{workspace/picture/subpic2/nonmin_direct_res.png}
        \subcaption{System responses  \label{fig:no:respnonmin2}}
    \end{minipage}\\
\vspace*{2mm}
    \begin{minipage}[h]{1\linewidth}
        \centering
        \includegraphics[width=1\linewidth]{workspace/picture/subpic2/nonmin_direct_diffy.png}
        \subcaption{Difference in the outputs of the system  \label{fig:no:diffy2}}
    \end{minipage}\\
\vspace*{2mm}
    \begin{minipage}[h]{1\linewidth}
        \centering
        \includegraphics[width=0.96\linewidth]{workspace/picture/subpic2/input_qa.png}
        \subcaption{Control inputs and inputs contaminated by the attack.  \label{fig:no:qzdag}}
    \end{minipage}
    \caption{Time responses of the system under the attack via $\epsilon$-stealthy approach. \label{fig:no:qaresult}}
\end{figure} 

