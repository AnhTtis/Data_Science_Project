\section{Introduction}\label{ch:introduction}

Recent advances in communication technology are
enabling various cyber-physical systems (CPSs)
to be further connected by networks, leading to 
enhancements in efficiency and flexibility for
their operation and control. Application domains
where such changes have brought significant 
progress include large-scale plants, smart grids, 
and traffic networks. It is however inevitable that 
along with the increase in networks,
cyber-security issues have gained much more attention. 
Some of the major incidents include 
the attacks against nuclear research facilities in Iran
\cite{Farwell2011StuxnetAT}, 
power grids in Ukraine \cite{ukraine}, and 
sewage systems in Australia \cite{Slay2007LessonsLF}.

To maintain the defense in depth for security and safety
of such CPSs, it is of critical
importance to strengthen security methods from the
perspective of systems control
to complement conventional security methods based on 
information technologies \cite{FerTei:22,IshZhu:22}.
In particular, malicious cyber attacks against CPSs
may have physical consequences, which can 
potentially result in damages in control devices and
facilities. 

In the systems control area, recent research has studied
analyses on impacts of such attacks,
detection techniques, as well as resilient control methods.
Representative classes of attacks that may lead to manipulating
vulnerable physical plants include those of replay attacks, 
Denial-of-Service (DoS) attacks, and 
false-data injection (FDI) attacks;
see, for example, \cite{MoKimBra:12,contsec,TeiShaSan:15}
and the references therein.

The focus of this paper is on the class of FDI attacks
known as the zero dynamics attacks \cite{ParShiLee:19,TeiShaSan:15}. 
An attacker who is aware of the system dynamics may
modify the control signals in such a way that the
internal states of the system are manipulated by the
attacker and can diverge. 
They can be generated by taking advantage of 
system zeros and, in particular, the unstable ones. 
The difficulty in dealing with such attacks is that
the behavior of the system output remains the same 
as that without the attacks. Hence, it is hard to detect 
them, e.g.,
by conventional fault detection techniques.

In this paper, we formulate a networked control 
problem in the sampled-data setting, where 
the continuous-time plant is controlled by a digital 
compensator and the control 
input is transmitted over a network channel.
We place particular attention to the role played by 
quantization in the control signals. Note that zero dynamics
attacks normally require the attack signal to be continuous.
Hence, quantization will introduce certain errors
not only in the attacks but also in the system outputs,
which may make it easier to detect the attacks.
In this sense, the security level can be enhanced by 
using quantization and especially when it is coarse. 
Clearly, there will be a certain tradeoff between
the security level and the control performance since
coarse quantization can in general degrade the system
performance. 

It is important to note that sampled-data systems
can introduce vulnerability in the context of zero 
dynamics attacks \cite{ShiBacEun:22}. 
Even if the continuous-time plant
originally has no unstable zero, when discretized,
the socalled sampling zeros will appear and
some of them can be unstable depending on the 
sampling rate \cite{astrom}. Such unstable zeros
can be exploited by the attacker. 

Quantization in networked control has been studied 
from various perspectives in the past two decades;
see, e.g., \cite{IshFra_book:02,NaiFagZamEva:07}.
%A well-known line of studies is related to the 
%level of quantization necessary to stabilize unstable
%systems based on the minimum data rate 
%(e.g., \cite{}) and the coarsest quantizers (e.g., \cite{}).
For systems under DoS attacks, the effects of quantization
have been addressed in \cite{katoDoS,wakaikiDoS,fengDoS}. 
The general implication found there is that using finer 
quantization, which
requires higher data rate for the communication of
control signals, would improve the robustness of the 
system against DoS attacks and vice versa.

To the best of our knowledge, however, 
quantization and their influence on FDI attacks have
not been studied in the literature.
Here, we will provide an analysis on the error in 
the system output caused by quantizing zero dynamics 
attacks as well as their capability to destabilize the
system. Our problem setting is limited to systems under
feedforward control, but 
it will be shown that by taking account of quantization 
effects, attack signals can be generated
resulting in a lower error level in comparison to 
the simple approach of directly quantizing the 
conventional zero dynamics attacks. 
To this end, we propose two quantized attack methods, one
based on dynamic quantizers \cite{dquant,sythdq} and the other
using a modified version of zero dynamics attacks. 
These methods are constructive in that the attacker
can specify the size of tolerable errors in the system
output and generate attack signals accordingly. 


\if0
For networked control systems, 
the control method which considers a quantization 
of signals caused by actuator capabilities and 
limitations of network communication. 
However, the attack methods previously mentioned 
assume the continuous signals. 
Therefore, it is necessary to study on an attack 
method which takes signal quantization effect into account.

In this paper, we consider a sampled-data system 
whose input signals are quantized. If a system 
input is continuous, the zero dynamics attack 
reduces effect on a system output by adjusting 
its initial value. If not, quantization effect 
appears with respect to its width and a system 
state and output are affected. Consequently, 
the attack performance such as the difficulty 
of detection is seemed to be spoiled.

Therefore, we propose the construction methods 
of the attack and analyze its effect, which reduce 
quantization effect on a system output and deviate 
a system state while a system input is quantized. 
In particular, for the cases where a sampled-data 
system has unstable zeros, we propose the two types 
of methods as follows.
\begin{itemize}
    \item the method utilizes a continuous attack signal.
    \item the method directly calculates a quantized attack signal.
\end{itemize}
\fi

The reminder of this paper is organized as follows: 
Section~\ref{ch:problem} describes the networked control 
system, the input quantizer and the class of the attacks. 
In Section~\ref{ch:dq}, we briefly overview 
the approach of dynamic quantizers, which will be used for
attack signal generation in one of the proposed methods.
In Section~\ref{ch:nonmin}, we explain the two methods 
of quantized attacks for a non-minimum phase 
sampled-data systems and analyze their effects. 
In Section~\ref{ch:example}, we illustrate the effectiveness
of our results via a numerical example.  
In Section~\ref{ch:conclusion}, we provide 
concluding remarks. 

\textbf{Notation}: We denote by $\mathbb{R}$, $\mathbb{N}$ and $\mathbb{Z}$ the sets of real numbers, natural numbers and integers respectively. %Moreover, $\mathbb{R}^n$, $\mathbb{R}^{n\times m}$ are $n$ dimensional real vectors and $n$ by $m$ real matrices respectively.
For $d>0$, $d\mathbb{Z}$ is the set of numbers which can be expressed as $dz$ by using an integer $z\in\mathbb{Z}$. 
The space of bounded sequences is denoted by $l_\infty$. 
%Non-negative integers are denoted as $\mathbb{Z}_+$. 
%For a matrix $A:=[A_{ij}]$, $\abs(A):=[|A_{ij}|]$. The $\infty$-norm of the vector $v$ is written as $\left\| v\right\|_\infty=\max_i|v_i|$. Similarly, $\left\| A\right\|_\infty$ is the $\infty$-norm of the matrix $A$, which is induced by vector's $\infty$-norm. 