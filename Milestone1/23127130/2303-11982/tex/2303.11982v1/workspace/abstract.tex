For networked control systems, cyber-security issues have gained much attention in recent years. In this paper, we consider the so-called zero dynamics attacks, which is a form of false data injection attacks, with a special focus on the effects of quantization in the sampled-data control setting. When the attack signals must be quantized, some error will be necessarily introduced, potentially increasing the chance of detection through the output of the system. In this paper, we show however that the attacker may reduce such errors by avoiding to directly quantize the attack signal. We look at two approaches for generating quantized attacks which can keep the error in the output smaller than a specified level by using the knowledge of the system dynamics. The methods are based on a dynamic quantization technique 
and a modified version of zero dynamics attacks. Numerical examples are provided to verify the effectiveness of the proposed methods.