\section{Dynamic quantization}\label{ch:dq}
In this section, we briefly introduce the method called dynamic quantization from the works \cite{dquant,dqseri,sythdq}. 

We consider the two feedforward discrete-time systems shown in Figs.~\ref{fig:dq:osys} and \ref{fig:dq:qsys}, 
where the controllers apply the same input $u[k]$ 
to both systems. The plant is from \eqref{eq:pr:dsys} 
without the attack signal, i.e., $b[k]\equiv0$. 
The difference between the two systems is that the 
first system in Fig.~\ref{fig:dq:osys} is the original 
one with the ideal output denoted by $y_I$ whereas 
in the second system in Fig.~\ref{fig:dq:qsys}, 
the input is quantized, resulting in the output 
$y$ with some errors from $y_I$. 
The problem is to find a quantizer with dynamics 
to minimize the error $y-y_I$ based on the past 
inputs $u[k]$ and their quantized values $v[k]$.
In particular, the measure of output error is defined as
\begin{equation}
    E(Q) = \sup_{u,k}\left|y[k]-y_I[k]\right|.
\label{eq:dq:error}
\end{equation}\par


\begin{figure}[t]
    \begin{minipage}[t]{1\linewidth}
\vspace*{1mm}
        \centering        
\includegraphics[width=0.7\linewidth]{workspace/picture/dq_osys.png}
\vspace*{1mm}
        \caption{Controller and plant}
        \label{fig:dq:osys}
    \end{minipage}
    \vspace{0.04\linewidth}
    \\
    \begin{minipage}[t]{1\linewidth}
        \centering
     \includegraphics[width=1\linewidth]{workspace/picture/dq_qsys.png}
\vspace*{1mm}
        \caption{Controller and plant with an input quantizer}
        \label{fig:dq:qsys}
    \end{minipage}
\end{figure}

Before we proceed to dynamic quantizers, we can
look at the case of static quantization 
when the input $u[k]$ 
is directly quantized as $v[k]=q(u[k])$.
%i.e., with $G=O$ and $H=I$ in \eqref{eq:dq:dquant}, 
In this case, it is known that 
the output error \eqref{eq:dq:error} can be obtained as 
\begin{equation}
 E(Q)
  = \frac{1}{2}\sum\limits_{k=0}^\infty\left|CA^kB\right| d.
\label{eq:dq:staticeps}
\end{equation}
Since the plant is assumed to be stable, this value is
finite. This formula suggests that for systems with
slower convergence, quantization has larger impact
on the output, and moreover this error is linear with
respect to the quantization width $d$.




\subsection{Basic structure of dynamic quantizers}\label{se:dq}
Here, we assume that the plant is minimum phase, i.e., its zeros are stable.\par
The dynamic quantizer is given by a linear system whose output is quantized by the uniform quantizer in Section \ref{se:plant}:
\begin{equation}
    Q\left\{
    \begin{aligned}
    \xi[k+1]&=E\xi[k]+F_1v[k]+F_2u[k],\\
    v[k]&=q(G\xi[k]+Hu[k]),
    \end{aligned}
    \right.\label{eq:dq:dquant}
\end{equation}
where the initial state is given by $\xi[0]=0$. 
We would like to design the dynamic quantizer 
so as to minimize the output error in \eqref{eq:dq:error}.

The dynamic quantizer is known to be optimal if
\begin{equation}
  \begin{split}
     E&=A,~~F_1=-F_2=B,\\
     G&=-\frac{CA}{CB},~~H=I
    \end{split}
\label{eq:dq:param}
\end{equation}
and then the minimum output error is given by \cite{dquant}
\begin{equation}
    E(Q)=\frac{1}{2}\left|CB\right|d.
\label{eq:dq:erroropt}
\end{equation}
It is obvious that with dynamic quantization,
the output error becomes strictly smaller than
that for the static case in \eqref{eq:dq:staticeps}.
In particular, $E(Q)$ in \eqref{eq:dq:erroropt} is
independent of the system matrix $A$. So different from
the static case, the output error does not depend on
the convergence rate of the plant.

However, we must note that this design method can be 
applied only to minimum phase systems. 
In fact, when the system \eqref{eq:pr:dsys} has 
unstable zeros, the dynamic quantizer outlined 
above becomes unstable. The case for nonminimum 
phase systems is discussed next. 

\subsection{Dynamic quantizer based on serial decomposition}\label{se:dqseri}
We describe the dynamic quantization approach of 
\cite{dqseri} based on decomposition of the plant. 
Suppose that the system \eqref{eq:pr:dsys} has unstable zeros. We denote by $P$ the system from the quantized input $v[k]$ to the output $y[k]$ in \eqref{eq:pr:dsys}. This is decomposed serially to subsystems $P_s$ and $P_u$ as $P=P_s\cdot P_u$ such that the following hold:
\begin{enumerate}
    \item $P_s$ is stable, minimum phase and strictly proper, and
    \item $C_sB_s\neq0$,
\end{enumerate}
where the realizations of $P_s$ and $P_u$ are, respectively, given by $(A_s,B_s,C_s,0)$ and $(A_u,B_u,C_u,D_u)$. Note that the minimum phase property of $P_s$ implies that $A_s-\frac{B_sC_sA_s}{C_sB_s}$ is a stable matrix\cite{dquantstable}.\par
Dynamic quantization based on serial decomposition generates the quantized signal $v[k]$ as follows:
\begin{equation}
  Q_s\left\{
    \begin{aligned}
    \xi[k+1]
      &= A_s\xi[k]+B_sv[k]-B_su[k], \\
    v[k]
      &= q\left(-\frac{C_sA_s}{C_sB_s}\xi[k]+u[k]\right).
    \end{aligned}
  \right.
\label{eq:dq:dqseri}
\end{equation}
Notice that this is the optimal quantizer for the system $P_s$
for \eqref{eq:dq:param}.
The output error in \eqref{eq:dq:error} for this case 
can be expressed as
\begin{equation}
E(Q_s)
 = \frac{1}{2}\left\|P_u\right\|_{i_\infty}
    \left|C_sB_s\right|d,
\label{eq:dq:dqserierroropt}
\end{equation}
where $\left\|P_u\right\|_{i\infty}$ is the induced-$l_\infty$ norm of the subsystem $P_u$ given by
\begin{equation}
    \left\|P_u\right\|_{i\infty}:=\sup\limits_{r\in l_\infty,r\neq0}\frac{\left\|P_ur\right\|_\infty}{\left\|r\right\|_\infty}.\label{eq:dq:linfnorm}
\end{equation}
This is the contribution of $P_u$ to the output error 
in \eqref{eq:dq:dqserierroropt}.

Note that in this method, there is some freedom in the choices of the subsystems $P_s$ and $P_u$. One difficulty is that if $P$ has multiple unstable zeros, then the optimal way of decomposition is not known in general. One exception is when the plant $P$ has only one unstable zero. In that case, the optimal decomposition and output error can be obtained explicitly. Let $\lambda$ be the unstable zero $(|\lambda|>1)$ of $P$. Then, $P_u$ should be taken in transfer function form as $P_u(z)=(z-\lambda)/z$. The optimal output error can be obtained as
\begin{equation}
    E(Q_s)=\frac{1}{2}(1+\left|\lambda\right|)\left|C_sB_s\right|d.\label{eq:dq:dqseriopt}
\end{equation}


