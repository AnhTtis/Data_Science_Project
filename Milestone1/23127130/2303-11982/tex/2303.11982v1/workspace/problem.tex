\section{Problem formulation}\label{ch:problem}
In this section, we formulate the problem of quantized FDI attacks studied in this paper.
\subsection{Sampled-data system under quantized input}\label{se:plant}
Consider the feedforward sampled-data control system shown in Fig.~\ref{fig:pr:sys}, where the controller is connected with the plant via a network. The control input is generated by the controller and is quantized before it is sent to the plant over the network. A malicious attacker has access to the network and may modify the input signal though the attack signal must use the same quantization scheme as the controller. 

\begin{figure}[t]
 \vspace*{3mm}
 \centering
  \includegraphics[width=1\linewidth]{workspace/picture/system1.png}
 \vspace*{1mm}
  \caption{Networked sampled-data control system under attack}
  \label{fig:pr:sys}
\end{figure}

In Fig.~\ref{fig:pr:sys},
the plant is a single-input single-output linear time-invariant (LTI) system whose state-space equation is given by
\begin{equation}
    \begin{split}
        \dot{x}(t)&=A_cx(t)+B_c(v(t)+b(t)),\\
        y(t)&=Cx(t),
    \end{split}\label{eq:pr:csys}
\end{equation}
where $x(t)\in\mathbb{R}^n$ is the state, $y(t)\in\mathbb{R}$ is the output, $A_c\in\mathbb{R}^{n\times n}$, $B_c\in\mathbb{R}^{n\times 1}$ and $C\in\mathbb{R}^{1\times n}$ are the system matrices. Moreover, the inputs $v(t)$, $b(t)\in d\mathbb{Z}$ are respectively the quantized signals of the feedforward input $u(t)$ and the attack signal $a(t)$. The specific quantizer used will be introduced later. We assume that the system is stable and controllable, and 
the initial state is $x(0)=x_0$.\par
In this paper, the input $v(t)$ is injected through  the zero-order hold (ZOH) whose sampling period is $T>0$. Denote by $v[k]$ the input calculated by the controller at time $t=kT$. The continuous-time signal applied to the plant is $v(t):=v [k]$ for $t\in[kT,(k+1)T)$. On the other hand, the output $y(t)$ of the plant is sampled at the same rate as the ZOH with sampling period $T$. So the measured output is $y [k] :=y(kT)$. We adopt the notation to write the system state and output of \eqref{eq:pr:csys} driven by input $u(t)$
as $x_u(t)$ and $y_u(t)$ respectively.\par
As the quantizer, we use the uniform quantizer $q:\mathbb{R}\rightarrow d\mathbb{Z}$, whose width is denoted by $d>0$. We use the nearest neighbor quantization towards $-\infty$ \cite{dquant}; it maps $\mu\in\mathbb{R}$ to $q(\mu)$ which is the optimal solution to $\min_{w\in d\mathbb{Z}}(w-\mu)^2$. In Fig.~\ref{fig:pr:sys}, the quantized input is given by
\begin{equation}
   v[k]=q(u[k]).
\label{eqn:vk}
\end{equation}
\subsection{Quantized FDI attacks}
As mentioned above, the adversary is capable to modify the quantized control input $v[k]$ to $v[k]+b[k]$ by injecting the quantized attack signal $b[k]$. It is assumed that the adversary has the full information of the controller and the plant including their dynamics whereas the system operator has no information regarding the attacks. The adversary's objective is to disrupt the operation of the system without being detected by the system operator who may be monitoring the sampled system output $y[k]$. \par
More specifically, the adversary attempts to generate the attack signal $b[k]$ satisfying the following two conditions:\par
(i)~\textbf{$\epsilon$-stealthy condition}: The attack signal $\{b[k]\}_{k=0}^{\infty}$ is said to be $\epsilon$-stealthy for given $\epsilon >0$ if the following inequality is satisfied for every $k$:
\begin{equation}
    \left|y_{v+b}[k]-y_v[k]\right|\leq\epsilon,\label{eq:pr:estealth}
\end{equation}
where $y_v$ is the original output of the system under the input $v$ and $y_{v+b}$ is the output of the attacked system.\par
(ii)~\textbf{$\{H_k\}_{k=0}^{\infty}$-disruptive condition}: Given a sequence of nonnegative numbers $\{H_k\}_{k=0}^{\infty}$, the attack signal $\{b[k]\}_{k=0}^{\infty}$ is said to be  $\{H_k\}_{k=0}^{\infty}$-disruptive if there exists a time sequence $\{t_k\}_{k=0}^{\infty}$ such that the following inequality is satisfied for every $k$:
\begin{equation}
    \left\|x_{v+b}(t_k)-x_v(t_k)\right\|_2\geq H_k.\label{eq:pr:disrupt}
\end{equation}
\quad The problem addressed in the paper can be stated as follows.\par
\textit{Problem}: Let the positive scalar $\epsilon$ and the sequence $\{H_k\}_{k=0}^{\infty}$ of nonnegative numbers be given. Then find the quantized attack signal $\{b[k] \}_{k=0}^{\infty}$ which satisfies both $\epsilon$-stealthy and $\{H_k\}_{k=0}^{\infty}$-disruptive conditions.\par
Particularly, we examine the problem above for the case where the sampled-data system has unstable zeros. \par
If the attacker can inject continuous-valued signals, attack methods satisfying the two conditions above are proposed, which are known as the zero dynamics attacks (ZDA) \cite{TeiShaSan:15,ParShiLee:19}. This type of attacks takes advantage of system zeros. Notably, if the system has unstable zeros, states may diverge and it is difficult to detect such attacks from the output since it will remain almost the same with or without the attack.\par
However, if we place an input quantizer, the system state/output under attacks should be influenced. Consequently, the stealthy and disruptive properties may be lost to some extent. In fact, if the attack signals are statically quantized, then the output receives major influences as we will see later. Hence, quantization may be used as a means to detect attacks on the system. Note that however that quantizing the input would reduce the control performance at the same time. So there is a tradeoff between the attack detection and control performance. The objective of this paper is to demonstrate that different methods for quantization of attack signals can result in different properties and in particular there are ways to reduce the effects caused by quantizers.
\subsection{Zero dynamics attacks}\label{se:zda}
The sampled-data system \eqref{eq:pr:csys} can be expressed as the following discrete-time linear system under ZOH and sampling:
\begin{equation}
    \begin{split}
        x[k+1]&=Ax[k]+B(v[k]+b[k]),\\
        y[k]&=Cx[k],
    \end{split}\label{eq:pr:dsys}
\end{equation}
where $x[k]:=x(kT)$, $A:=e^{A_cT}$ and $B:=\int_0^Te^{A_c\tau}d\tau B_c$. We assume that this system is controllable and the vector product $CB$ is nonsingular, which means the relative degree of this discretized system is $1$; these properties hold for almost any $T>0$ \cite{astrom}. \par
Note that when a continuous-time system is discretized, zeros which do not exist in the original system may appear. These are called the sampling zeros. As mentioned above, regardless of the relative degree of the original system \eqref{eq:pr:csys}, the discretized system \eqref{eq:pr:dsys} has relative degree of $1$ for almost any sampling period $T$. It is known that when $T$ is small, some of them will be unstable \cite{astrom}. Hence, even if the original system has no unstable zero, there is a chance that discretization will introduce some. In such a case, if the input is not quantized, then the zero dynamics attacks proposed in \cite{TeiShaSan:15,ParShiLee:19} can be applied. We note that such attacks can have arbitrarily small effects on the output by taking the initial state accordingly.\par
We summarize how to construct ZDA signals as follows: In this part, we make the following assumption: The system \eqref{eq:pr:dsys} has nonminimum phase zeros. Here, we consider the case where the inputs are not quantized, i.e., $v[k]=u[k]$, $b[k]=a[k]$. According to \cite{TeiShaSan:15}, we can construct the attack signal $\{a[k]\}_{k=0}^{\infty}$ by
\begin{equation}
    \begin{split}
        z[k+1]&=\left(A-\frac{BCA}{CB}\right)z[k],\\
        a[k]&=-\frac{CA}{CB}z[k],
    \end{split}\label{eq:pr:zda}
\end{equation}
where $z[k]\in\mathbb{R}^n$, $z[0]\in\ker(C)$ and $z[0]\neq 0$. Note that the nonminimum phase property of \eqref{eq:pr:dsys} implies that the system matrix $A-\frac{BCA}{CB}$ is unstable. The core idea of zero dynamics attacks is that this attack signal $a[k]$ will make the system states diverge but remain in the undetectable region. Regarding this type of attacks, the following result is fundamental \cite{TeiShaSan:15}.
\begin{lem}\label{lem:zda}
Consider the system in \eqref{eq:pr:dsys}. For any $\epsilon>0$, there exist attack signals such that they are $\epsilon$-stealthy and $\left\|x_{u+a}[k]-x_u[k]\right\|_2$ diverges as $k\rightarrow\infty$.
\end{lem}