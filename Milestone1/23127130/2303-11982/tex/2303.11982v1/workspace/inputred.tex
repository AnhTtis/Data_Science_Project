\section{Quantized attack for an input redundant system}\label{ch:inputred}
In this section, we explain the quantized attack $\{b[k]\}_{k=0}^{\infty}$ where the system \eqref{eq:csys} satisfies following condition.
\begin{itemize}
    \item $N>1$
\end{itemize}
Concretely, we propose the two types of methods.
\begin{enumerate}
    \item the method modifies the continuous attack by using the dynamic quantizer to reduce influence on the output
    \item the method directly search the attack which satisfies the conditions at each time.
\end{enumerate}
The attack by first method reduces the output error compared with the case where the attack is statically quantized. Moreover, if $A-B(CB)^{-1}CA$ is a stable matrix, then we obtain the optimal the dynamic quantizer unlike the method in \ref{se:dqzda}. The attack by second method solves a lattice searching problem at each sampling time. That means this method requires high computational costs. However, this method probably achieve less output error compared with the first method. Therefore, there is the trade-off of the first and second method. Though the lower limit of the output error by the second method is not revealed, there exists the numerical example where the second method achieves better performance than the first method.\par
If $N>1$, the system \eqref{eq:csys} can be equivalently expressed as the lifted system given by
\begin{equation}
\begin{split}
    \tilde{x}[k+1]&=\tilde{A}\tilde{x}[k]+\tilde{B}_N(v[k]+b[k]),\\
    y[k]&=\tilde{C}\tilde{x}[k],
\end{split}\label{eq:re:syslift}
\end{equation}
where $\tilde{x}[k]:=x[Nk]=x(NkT)$. The lifted form of the control input $v[k]$ is 
\begin{equation}
    \tilde{v}[k]:=\begin{bmatrix}   
    v[Nk] \\
    \vdots \\
    v[N(k+1)-1]
    \end{bmatrix}.\label{eq:re:vlift}
\end{equation}
Analogously, other signals $u[k]$, $b[k]$, $a[k]$ can be transformed. Define $\tilde{A}:=A^N$, $\tilde{B}_l:=[A^{l-1}B\cdots B]$, $\tilde{C}:=C$ by using the system matrices $A$ and $B$ in \eqref{eq:pr:dsys}.\par
If the input is not quantized and either $\tilde{A}-\tilde{B}_N(\tilde{C}\tilde{B}_N)^+\tilde{C}\tilde{A}$ or $A-B(CB)^{-1}CA$ is unstable, then zero dynamics attack in \ref{se:zda} or the attack by \eqref{eq:no:zdag} in \ref{se:calc} is $\epsilon$-stealthy and $\{H_k\}_{k=0}^{\infty}$-disruptive with respect to the given scalar $\epsilon>0$ and the sequence $\{H_k\}_{k=0}^{\infty}$. Here, $(\tilde{C}\tilde{B}_N)^+$ is the pseudo inverse of $\tilde{C}\tilde{B}_N$. Note that zero dynamics attack is implemented as follows when $\tilde{A}-\tilde{B}_N(\tilde{C}\tilde{B}_N)^+\tilde{C}\tilde{A}$ is unstable. First, calculate the attack $\{\tilde{a}[k]\}_{k=0}^{\infty}$ with respect to the lifted system \eqref{eq:re:syslift}. At that time, $\tilde{a}[k]\in\mathbb{R}^{Nm}$ for every $k$. After that, the attack $\{a[k]\}_{k=0}^{\infty}$ is given by
\begin{equation*}
    \begin{bmatrix}
        a[Nk]\\
        \vdots \\
        a[Nk+N-1]
    \end{bmatrix}=\tilde{a}[k].
\end{equation*}
Above manipulation is the inverse of lifting.\par
However, if the above matrices are stable, zero dynamics attack and the attack by \eqref{eq:no:zdag} may not be $\{H_k\}_{k=0}^{\infty}$-disruptive. Therefore, the attack method which achieve $\epsilon$-stealthiness with respect to $\epsilon=0$ and $\{H_k\}_{k=0}^{\infty}$-disruption is proposed by taking advantage of the input redundancy, which is called zero stealthy attack (ZSA) \cite{zsa}.

\subsection{Zero stealthy attack}\label{se:zsa}
Consider the situation where the input is not quantized, which means $\tilde{v}[k]=\tilde{u}[k]$, $\tilde{b}[k]=\tilde{a}[k]$. At first, deal with the difference between the output under the attack and the attack-free output. Denote $\Delta\tilde{x}[k]:=\tilde{x}_{u+a}[k]-\tilde{x}_u[k]$, $\Delta \tilde{y}[k]:=\tilde{C}\Delta\tilde{x}[k]$, then
\begin{equation}
    \begin{split}
    \Delta\tilde{x}[k+1]&=\tilde{A}\Delta\tilde{x}[k]+\tilde{B}_N\tilde{a}[k],\\
    \Delta\tilde{y}[k]&=\tilde{C}\Delta\tilde{x}[k],
    \end{split}\notag%\label{eq:re:difflift}
\end{equation}
where $\Delta\tilde{x}[0]=0$. Define the attack $\{a[k]\}_{k=0}^{\infty}$ as following form.
\begin{equation}
     \tilde{a}[k]=\kappa_k\eta+\xi_k\label{eq:re:atk}
\end{equation}
Here, $\kappa_k\in\mathbb{R}$, $\eta$, $\xi_k\in\mathbb{R}^{mN}$.\par
Consider the difference between the outputs $\Delta\tilde{y}[k]$. If the parameters are defined as follows, then the attack is $\epsilon$-stealthy with respect to $\epsilon=0$.
\begin{equation}
    \begin{split}
    &\tilde{C}\tilde{B}_N \eta=0\\
    &\tilde{C}\tilde{B}_N\xi_k+\tilde{C}\tilde{A}\Delta\tilde{x}[k]=0
    \end{split}\notag
\end{equation}\par
Let us move on to the difference between the states $\Delta\tilde{x}[k]$. In the sequel, the sequence $\{h_k\}_{k=0}^{\infty}$ is given. Moreover, the time sequence $\{t_k\}_{k=0}^{\infty}$ is of the form; $t_k=TN(k+\theta_k)$. We consider the condition for the attack $\{b[k]\}_{k=0}^{\infty}$ such that $\left\|x_{u+a}(t_k)-x_u(t_k)\right\|>=h_k$. For every $t_k$, the difference between the states is denoted $x_{u+a}(t_k)-x_u(t_k)=\bar{A}\Delta\tilde{x}[k]+\Phi_k\tilde{a}[k]$ by using following matrices.
\begin{equation}
    \begin{split}   
    \bar{A}_k&:=e^{A_c\theta_kNT}\\
    \Phi_k&:=[\Phi_k(1),\cdots,\Phi_k(N)]\\
    \Phi_k(l)&:=\left\{
        \begin{aligned}
    &e^{A_cT(N\theta_k-l)} &l&=1,\cdots,\lfloor N\theta_k\rfloor \\
    &\int_0^{N\theta_k-\lfloor N\theta_k\rfloor} e^{A_c\tau}d\tau B_c & l&=\lfloor N\theta_k\rfloor+1 \\
    &0 & &\text{otherwise}
        \end{aligned}
        \right.
    \end{split}\notag
\end{equation}
Therefore, so as to $\left\|x_{u+a}(t_k)-x_u(t_k)\right\|\geq h_k$, the parameter $\kappa_k$ should be 
\begin{equation}
    \kappa_k\geq\frac{h_k+\left\|\bar{A}_k\Delta\tilde{x}[k]+\Phi_k\xi_k\right\|_2}{\left\|\Phi_k\eta\right\|_2}\notag
\end{equation}
for each $k$.\par
Sufficient condition to exist the parameters $\kappa_k$, $\eta$, $\xi_k$ is as follows.
\begin{enumerate}
    \renewcommand{\labelenumi}{(\alph{enumi})}
      \setlength{\topsep}{0pt}
      \setlength{\partopsep}{0pt}
      \setlength{\itemsep}{0pt}
      \setlength{\parsep}{0pt}
      \setlength{\parskip}{1pt plus 1pt minus 1pt}
    \item $\ker \tilde{C}\tilde{B}_N\neq\{0\}$
    \item For $\{t_k\}_{k=0}^{\infty}$, $\ker \tilde{C}\tilde{B}_N\nsubseteq\ker\Phi_k$
    \item $\im \tilde{C}\tilde{A} \subset \im \tilde{C}\tilde{B}_N$
\end{enumerate}
If $B$ has full column rank and (a) is satisfied, then we can construct the time sequence $\{t_k\}_{k=0}^{\infty}$ which satisfies (b). In particular, we obtain it which is of the form 
\begin{equation}
    \begin{split}
        t_k&=T(Nk+l^*)\\
        &=Tn_k
    \end{split}\label{eq:re:time}
\end{equation}
for each $k$ where $l^*\in\mathbb{N}$. In this section, we assume $B$ has full column rank, both (a) and (c) is satisfied and $l^*$ is given. At this time, if the sequence $\{h_k\}_{k=0}^{\infty}$ is given, then we can calculate the attack $\{a[k]\}_{k=0}^{\infty}$ at each sampling time of sensor. In the sequel, the sequence $\{h_k\}_{k=0}^{\infty}$ is called the attack parameter of the attack.\par
\textbf{Remark}: If the attack by \eqref{eq:re:atk} is statically quantized, it is $\epsilon$-stealth with respect to $\epsilon$ which satisfies \eqref{eq:re:staticeps}.
\begin{equation}
 \epsilon
 \geq \frac{\left\|\sum\limits_{k=0}^\infty
       \abs(\tilde{C}\tilde{A}^k\tilde{B}_N)\right\|_\infty d}{2}
\label{eq:re:staticeps}
\end{equation}
The right side of this inequality is same as \eqref{eq:dq:staticeps} where $A\rightarrow\tilde{A}$, $B\rightarrow\tilde{B}_N$, $C\rightarrow\tilde{C}$.

\subsection{Proposed method 3}\label{se:dqzsa}
In this subsection, we propose the method modifies zero stealthy attack by using the dynamic quantizer before implementation the attack. It is assumed $CB$ is non-singular. Since it is also assumed that $A-B(CB)^{-1}CA$ is stable in this section, the dynamic quantizer in \ref{se:dq} can be utilized.\par 
The method in this subsection generates the quantized attack $\{b[k]\}_{k=0}^{\infty}$ as follows from the attack $\{a[k]\}_{k=0}^{\infty}$ in previous subsection.
\begin{equation*}
    \begin{split}
        \xi[k+1]&=A\xi[k]+B(b[k]-a[k])\\
        b[k]&=q\left(-(CB)^{-1}CA\xi[k]+a[k]\right)
    \end{split}
\end{equation*}
The above equation is updated at each sampling time of the input, not the output. According to \eqref{eq:dq:erroropt}, following inequality is held for every $k$.
\begin{equation*}
    \left\|Cx_b[k]-Cx_a[k]\right\|_\infty\leq\frac{\left\|\sum\limits_{k=0}^\infty\abs(CB)\right\|_\infty d}{2}
\end{equation*}
This inequality also implies 
\begin{equation*}
\begin{split}
        \left\|y_b[k]-y_a[k]\right\|_\infty&=\left\|Cx_b[Nk]-Cx_a[Nk]\right\|_\infty\\&\leq\frac{\left\|\sum\limits_{k=0}^\infty\abs(CB)\right\|_\infty d}{2}.
\end{split}
\end{equation*}
The attack by the above method is characterized by the following theorem. The matrix $\bar{A}$ is given by
\begin{equation}
    \bar{A}:=\begin{bmatrix}
    A & -B(CB)^{-1}CA \\
    O & A-B(CB)^{-1}CA
    \end{bmatrix}.\notag
\end{equation}
\begin{theo}
The sequence $\{H_k\}_{k=0}^{\infty}$ and quantization width $d$ are given. Consider zero stealthy attack $\{a[k]\}_{k=0}^{\infty}$ in \ref{se:zsa} whose attack parameter is $h_k:=H_k+\frac{\sqrt{n}}{2}\alpha_k$ is modified by the dynamic quantizer in \ref{se:dq}, let this attack be $\{b[k]\}_{k=0}^{\infty}$. This attack is $\epsilon$-stealth for $\epsilon$ satisfies \eqref{eq:re:eps} and $\{H_k\}_{k=0}^{\infty}$-disruptive.
\begin{equation}
    \epsilon\geq\frac{\left\|\abs(CB)\right\|_\infty d}{2}\label{eq:re:eps}
\end{equation}
Here,
\begin{equation}
    \alpha_k:=\left\|\sum\limits_{l=0}^{n_k-1}\abs\biggm(\begin{bmatrix}
    I & O
    \end{bmatrix}\bar{A}^{l}\begin{bmatrix}
    B \\
    B
    \end{bmatrix}\biggm)\right\|_\infty.\label{eq:re:alpha}
\end{equation}
\end{theo}
\begin{proof}
According to the triangle inequality, the output error satisfies
\begin{equation}
\left\|y_{v+b}[k]-y_v[k]\right\|_\infty\leq\left\|y_{v+b}[k]-y_{v+a}[k]\right\|_\infty+\left\| y_{v+a}[k]-y_v[k]\right\|_\infty\notag
\end{equation}
for each $k$. The second term of the right side is always 0, because zero stealthy attack $\{a[k]\}_{k=0}^\infty$ holds $y_{v+a}[k]=y_v[k]$ for every $k$. From linearity, the first term of the right side coincides with $\left\| y_b[k]-y_a[k]\right\|_\infty$. Since $a[k]$ is dynamically quantized, 
\begin{equation}
\sup\limits_k\left\| y_b[k]-y_a[k]\right\|_\infty\leq\left\|\abs(CB)\right\|_\infty\frac{d}{2}.\notag
\end{equation}
So if $\epsilon\geq\left\|\abs(CB)\right\|_\infty \frac{d}{2}$, $\sup\limits_{v,k}\left\| y_{v+b}[k]-y_v[k]\right\|_\infty\leq\epsilon$ is held. It means the attack $\{b[k]\}_{k=0}^\infty$ is $\epsilon$-stealth with respect to $\epsilon$ which satisfies \eqref{eq:re:eps}.\par
Next, consider the state difference. According to the triangle inequality, the state difference satisfies
\begin{equation}
\left\| x_{v+b}[k]-x_v[k]\right\|_2\geq\left\| x_{v+a}[k]-x_v[k]\right\|_2-\left\| x_{v+b}[k]-x_{v+a}[k]\right\|_2\label{eq:re:statediff}
\end{equation}
for each $k$. From linearity, the second term of the right side is equal to $x_b[k]-x_a[k]$. Then the states of the plant and the dynamic quantizer obey following state equation.
\renewcommand{\arraystretch}{0.9}%%%
\begin{equation}
\begin{bmatrix}
x_{b}[k+1] \\
\xi[k+1]\\
x_{a}[k+1]
\end{bmatrix}\hskip-3pt=\hskip-3pt\left[\begin{array}{ccc}
A&O&O\\
O&A&O\\
O&O&A 
\end{array}\right]\hskip-5pt\begin{bmatrix}
x_{b}[k] \\
\xi[k]\\
x_{a}[k]
\end{bmatrix}+\begin{bmatrix}
Bb[k] \\
B(b[k]-a[k])\\
Ba[k]
\end{bmatrix}\notag
\end{equation}
If we define
\begin{equation}
w_q[k]:=q(-(CB)^{-1}CA\xi[k]+a[k])-\{-(CB)^{-1}CA\xi[k]+a[k]\},\notag
\end{equation}
then the quantized attack is expressed as $b[k]=a[k]+w_q[k]-(CB)^{-1}CA\xi[k]$. Note that $\left\|w_q[k]\right\|_\infty\leq d/2$.\par
By using this notation, the above equation is 
\begin{equation}
\begin{bmatrix}
x_{b}[k+1] \\
\xi[k+1]\\
x_{a}[k+1]
\end{bmatrix}\hskip-3pt=\hskip-3pt\left[\begin{array}{@{\hskip-1pt}c|c@{\hskip-1pt}}
\raisebox{-0.5em}{\mbox{\LARGE $\bar{A}$}} & \begin{matrix}
O \\ O
\end{matrix}
\\
\hline
 \begin{matrix}
O & O
\end{matrix} & A
\end{array}\right]\hskip-5pt\begin{bmatrix}
x_{b}[k] \\
\xi[k]\\
x_{a}[k]
\end{bmatrix}+\begin{bmatrix}
B(a[k]+w_q[k]) \\
Bw_q[k]\\
Ba[k]
\end{bmatrix}.\notag
\end{equation}
\renewcommand{\arraystretch}{1}%%%
Applying $x_{b}[k]-x_a[k]=[I~O~-I][x_{b}[k]~\xi[k]\ x_{a}[k]]^T$ and $\xi[0]=0$, the state difference is
\begin{equation}
x_{b}[k]-x_a[k]=\sum\limits_{l=0}^{k-1}\begin{bmatrix}
I & O
\end{bmatrix}\bar{A}^{l}\begin{bmatrix}
B \\
B
\end{bmatrix}w_q[k-l].\notag
\end{equation}
At the time $n_k$,
\begin{equation*}
    \left\|x_{b}[n_k]-x_a[n_k]\right\|_\infty\leq\left\|\sum\limits_{l=0}^{n_k-1}\abs\left(\begin{bmatrix}
    I & O
    \end{bmatrix}\bar{A}^l\begin{bmatrix}
    B \\
    B
    \end{bmatrix}\right)\right\|_\infty\frac{d}{2}\notag   
\end{equation*}
is held \cite{dquant}. Then, the square norm of the state difference satisfies
\begin{equation*}
    \left\| x_{b}[n_k]-x_a[n_k]\right\|_2\\\leq\left\|\sum\limits_{l=0}^{n_k-1}\abs\left(\begin{bmatrix}
    I & O
    \end{bmatrix}\bar{A}^l\begin{bmatrix}
    B \\
    B
    \end{bmatrix}\right)\right\|_\infty\frac{\sqrt{n}d}{2}.\notag  
\end{equation*}
Let the attack parameter of the attack be $\{h_k\}_{k=0}^\infty$, then the first term of \eqref{eq:re:statediff} is equal or greater than $h_k$. Therefore, by using $\alpha_k$ satisfies \eqref{eq:re:alpha}, 
\begin{equation}
\left\| x_{v+b}[n_k]-x_{v}[n_k]\right\|_2\geq h_k-\frac{\sqrt{n}d}{2}\alpha_k\notag
\end{equation}
is held. If $h_k=H_k+\frac{\sqrt{n}d}{2}\alpha_k$, the attack $\{b[k]\}_{k=0}^\infty$ is $\{H_k\}_{k=0}^\infty$-disruptive.
\end{proof}
\textbf{Remark}: By the method in \ref{se:dqzda}, the parameters of the dynamic quantizer and its optimal output error vary depending on the number of unstable zeros the system has and their values. The method in this subsection is independent of such influences. This means the attacker has advantages when the plant has input redundancy, which is an intuitive result. 
\subsection{Proposed method 4}\label{se:search}
In this subsection, we propose the method directly search the quantized attack $\{b[k]\}_{k=0}^{\infty}$ at each sampling time. This is a lattice searching problem under some constraints. If the state difference $\Delta\tilde{x}[k]:=\tilde{x}_{v+b}[k]-\tilde{x}_v[k]$ is given, the output error at next step $\Delta y[k+1]:=y_{v+b}[k+1]-y_v[k+1]$ is 
\begin{equation}
\Delta \tilde{y}[k+1]=\tilde{C}\tilde{A}\Delta\tilde{x}[k]+\tilde{C}\tilde{B}_N\begin{bmatrix}
b[kN] \\
\vdots \\
b[(k+1)N-1]
\end{bmatrix}.\notag
\end{equation}
For given $\epsilon$, if $\left\|\Delta y[k+1]\right\|_\infty\leq\epsilon$ for every $k\geq0$, then the attack is $\epsilon$-stealth. This condition is equivalent to following inequality. 
\begin{equation}
\left\|\frac{\tilde{C}\tilde{A}}{d}\Delta\tilde{x}[k]+\tilde{C}\tilde{B}_N\begin{bmatrix}
b[kN] \\
\vdots \\
b[(k+1)N-1]
\end{bmatrix}\right\|_\infty\leq\frac{\epsilon}{d}\label{eq:re:stealth}
\end{equation}
$\frac{b[k]}{d}$ is an integer vector. Hence, so as to satisfy \eqref{eq:re:stealth}, we should the integer vector $v\in\mathbb{Z}^{mN}$ such that
\begin{equation}
\left\|\frac{\tilde{C}\tilde{A}}{d}\Delta\tilde{x}[k]+\tilde{C}\tilde{B}_Nv\right\|_\infty\leq\frac{\epsilon}{d}.\notag
\end{equation}\par
Next, consider the condition for $\{H_k\}_{k=0}^{\infty}$-disruptive. At time $n_k$, the state difference is
\begin{equation}
x_{v+b}[n_k]-x_v[n_k]=A^{k^*}\Delta\tilde{x}[k]+\tilde{B}_{l^*}\begin{bmatrix}
b[kN] \\
\vdots \\
b[(kN+l^*-1]
\end{bmatrix}.\notag
\end{equation}
If we search an integer vector $v_1\in\mathbb{Z}^{mk^*}$ such that 
\begin{equation}
\left\|\frac{A^{k^*}}{d}\Delta\tilde{x}[k]+\tilde{B}_{k^*}v_1\right\|_2\geq\frac{H_k}{d}\notag
\end{equation}
and let $[b[kN]^T\cdots b[kN+k^*-1]^T]^T=v_1d$, then the attack is $\{H_k\}_{k=0}^\infty$-disruptive. Hence, the following theorem is held.
\begin{theo}
It is assumed that the sequence $\{H_k\}_{k=0}^\infty$ and the quantized width $d$ is given. Consider the integer vectors $v_1\in\mathbb{Z}^{ml^*}$ and $v_2\in\mathbb{Z}^{m(N-l^*)}$ which satisfy \eqref{eq:re:cri}. If $[a[kN]^T\cdots a[(k+1)N-1]^T]=d[v_1^T,v_2^T]$, then $b[k]=a[k]$ for every $k$ and the attack is $\epsilon$-stealthy with respect to the $\epsilon$ which satisfies \eqref{eq:re:epsopt} and $\{H_k\}_{k=0}^{\infty}$-disruptive.
\begin{equation}
    \begin{split}
        &\left\|\frac{\tilde{C}\tilde{A}}{d}\Delta\tilde{x}[k]+\tilde{C}\tilde{B}_Nv\right\|_\infty\leq\frac{\epsilon}{d}\\
        &\left\|\frac{A^{k^*}}{d}\Delta\tilde{x}[k]+\tilde{B}_{k^*}v_1\right\|_2\geq\frac{H_k}{d}
    \end{split}\label{eq:re:cri}
\end{equation}
\begin{equation}
   \epsilon\geq\sup\limits_k\epsilon_k\label{eq:re:epsopt}
\end{equation}
Here,
\begin{equation}
\begin{split}
    \epsilon_k:=\sup\limits_{x\in\mathbb{R}^n}\inf_{\substack{v_1\in V_k(x)\\ v_2\in\mathbb{Z}^{m(N-l^*)}}}\left\|\frac{\tilde{C}\tilde{A}}{d}x+\tilde{C}\tilde{B}_N\begin{bmatrix}
    v_1 \\
    v_2 
    \end{bmatrix}\right\|_\infty d\\
    V_k(x):=\biggm\{v\in\mathbb{Z}^{l^*m}\biggm|\left\|\frac{A^{l^*}}{d}x+\tilde{B}_{l^*}v\right\|_2>\frac{H_k}{d}\biggm\}.
\end{split}
\end{equation}
\end{theo}
\textbf{Remark}: The method adopting the dynamic quantizer in \ref{se:dqzsa} calculates the attack by updating a state space equation and solving an equation. On the other hand, the method in this subsection solves the lattice searching problem under the constraints \eqref{eq:re:cri}, which means this type of method costs higher than the method in \ref{se:dqzsa}. However, this method may achieve $\epsilon$-stealthy with respect to lower $\epsilon$. Note that the evaluation of the right side of \eqref{eq:re:epsopt} is the future work.
\subsection{Numerical example}
Consider the following continuous time system for Fig.~\ref{fig:pr:sys}.
\begin{equation}
\begin{split}
\dot{x}(t)&=\begin{bmatrix}
-5 & 1 \\
0 & -2
\end{bmatrix}x(t)+\begin{bmatrix}
0 \\ 1
\end{bmatrix}(v(t)+b(t))\\
y(t)&=\begin{bmatrix}
1 & 1
\end{bmatrix}x(t)
\end{split}\notag
\end{equation}
The sampling period of the control input is $T=0.1[s]$ and $N=2$. The initial state $x_0=0$. It is assumed $u[k]=\sin{0.1\pi k}+2\cos{0.05\pi k}$. The quantization width $d=1$ and $l^*=1$. Let the sequence $\{H_k\}_{k=0}^\infty$ be $H_k=0.01\times1.2^k$ for each $k$.\par
First, the results are depicted from the statically quantized zero stealthy attack in \ref{se:zsa}. Fig.~\ref{fig:re:responseredsta} shows the inter-sample responses under the attack and without the attack. It also shows the system output under the attack at each sampling time $2kT_s~(k=0,1,\ldots)$. In this case, the right side of \eqref{eq:re:staticeps} is $\left\|\sum_{k=0}^\infty\abs(\tilde{C}\tilde{A}^k\tilde{B}_N)\right\|_\infty\frac{d}{2}=0.300$. Fig.~\ref{fig:re:stazsa} shows the control input and sum of the input and the attack.\par
\begin{figure}[t]
 \centering
 \includegraphics[width=0.75\linewidth]{workspace/picture/responsesredsta.eps}
 \caption{System responses under static quantized zero stealthy attack.  \label{fig:re:responseredsta}}
\end{figure}
\begin{figure}[t]
 \centering
 \includegraphics[width=0.75\linewidth]{workspace/picture/ua_stazsa.eps}
 \caption{Control inputs and inputs contaminated by static quantized zero stealthy attack.  \label{fig:re:stazsa}}
\end{figure}\par
Next, the results are depicted from the dynamically quantized zero stealthy attack, which is explained in \ref{se:dqzsa}. Fig.~\ref{fig:re:responsered1} shows the inter-sample responses under the attack and without the attack. It also shows the system output under the attack at each sampling time $2kT_s~(k=0,1,\ldots)$. Fig.~\ref{fig:re:y_1} shows the difference between the output under the attack and attack-free output. Though the inter-sample response under the attack is largely deviated, the sampled outputs are closer to the attack-free ones. In fact, the difference between attacked/original outputs are not greater than $\epsilon=\left\|\abs(CB)\right\|_\infty\frac{d}{2}=0.0473$. Fig.~\ref{fig:re:xnorm} is $H_k$ and $\left\|x_{v+b}[n_k]-x_v[n_k]\right\|_2$ at each $k$. It means always $\left\|x_{v+b}[n_k]-x_v[n_k]\right\|_2\geq H_k$. Therefore, the attack is $\{H_k\}_{k=0}^{\infty}$-disruptive. Fig.~\ref{fig:re:dqzsa} shows the control input and sum of the input and the attack.\par
\begin{figure}[t]
 \centering
 \includegraphics[width=0.75\linewidth]{workspace/picture/responsered1.eps}
 \caption{System response under quantized zero-stealthy attack.  \label{fig:re:responsered1}}
\end{figure} 
\begin{figure}[t]
 \centering
 \includegraphics[width=0.75\linewidth]{workspace/picture/y_1.eps}
 \caption{Difference of outputs of system under quantized zero-stealthy attack.  \label{fig:re:y_1}}
\end{figure}
\begin{figure}[t]
 \centering
 \includegraphics[width=0.75\linewidth]{workspace/picture/xnorm.eps}
 \caption{Series $\{H_k\}_{k=0}^{\infty}$ and norm of states difference $\left\|x_{v+b}[n_k]-x_v[n_k]\right\|_2$ under quantized zero-stealthy attack. \label{fig:re:xnorm}}
\end{figure}
\begin{figure}[t]
 \centering
 \includegraphics[width=0.75\linewidth]{workspace/picture/ua_dqzsa.eps}
 \caption{Control inputs and inputs contaminated by quantized zero stealthy attack. \label{fig:re:dqzsa}}
\end{figure} \par
Next, the results are depicted from the attack method by searching the integer vector in \ref{se:search}. Similarly, Fig.~\ref{fig:re:responsered2} shows the inter-sample responses under the attack and without the attack. It also shows the system output under the attack at each sampling time $2kT_s~(k=0,1,\ldots)$. Fig.~\ref{fig:re:y_2} shows the difference between the output under the attack and attack-free output. Here, $\epsilon=\left\|\abs(CB)\right\|_\infty\frac{d}{8}$. For this $\epsilon$, though the attack method in \ref{se:dqzsa} may not hold $\epsilon$-stealthy property, this type of method is $\epsilon$-stealthy. Fig.~\ref{fig:re:xnorm2} shows this attack is $\{H_k\}_{k=0}^{\infty}$-disruptive. Fig.~\ref{fig:re:epsatk} shows the control input and sum of the input and the attack. Note that the input signal under the attack largely fluctuate compared with the previous two methods. In this case, $\epsilon$ is small so the number of lattices $[v_1^T,v_2]^T$ which satisfy \eqref{eq:re:cri} decrease. Then the attacker results in picking a lattice where $v_1$ is larger than it needs to be.
\begin{figure}[t]
 \centering
 \includegraphics[width=0.75\linewidth]{workspace/picture/responsered2.eps}
 \caption{System responses under integer attack.  \label{fig:re:responsered2}}
\end{figure} 
\begin{figure}[t]
 \centering
 \includegraphics[width=0.75\linewidth]{workspace/picture/y_2.eps}
 \caption{Difference of outputs of system under integer attack.  \label{fig:re:y_2}}
\end{figure}
\begin{figure}[t]
 \centering
 \includegraphics[width=0.75\linewidth]{workspace/picture/xnorm2.eps}
 \caption{Series $\{H_k\}_{k=0}^{\infty}$ and norm of states difference $\left\|x_{v+b}[n_k]-x_v[n_k]\right\|_2$ under integer attack. \label{fig:re:xnorm2}}
\end{figure}
\begin{figure}[t]
 \centering
 \includegraphics[width=0.75\linewidth]{workspace/picture/ua_epsatk.eps}
 \caption{Control inputs and inputs contaminated by integer attack.   \label{fig:re:epsatk}}
\end{figure} 