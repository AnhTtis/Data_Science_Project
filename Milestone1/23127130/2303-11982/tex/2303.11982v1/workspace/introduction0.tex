\section{Introduction}\label{ch:introduction}

Recently, advances in communication technology and 
computer led cyber-physical system technology to 
be more practical, which improves efficiency and 
flexibility of physical systems by using computers 
to supervise and control. This technology is applied 
to various objects such as large-scale plants, 
smart grid and traffic networks. 
On the other hand, its security problems also gets 
attention. Recent example of incidents include the 
cyber attack on Iran's nuclear plant by the 
Stuxnet and the Ukrainian power grid, the massive 
power blackouts in Brazil and the malfunction of 
sewage system in Australia \cite{Farwell2011StuxnetAT,Brazil,Slay2007LessonsLF,ukraine}. 
These examples reveals the importance of prevention and 
detection of the vulnerabilities of cyber-physical 
systems and the unexpected incidents. 
In particular, necessity of the system control 
security is pointed out in addition to conventional 
security based on information technologies\cite{teixeirabook,ishiibook},

So far, the vulnerabilities of cyber-physical systems 
are discussed from the system control point of view. 
For example, the false data injection attack, 
DoS attack and the replay attack are analyzed. 
Also, it is studied to detect them \cite{FDIs,Mo2010FalseDI,replay,contsec,katoDoS,wakaikiDoS,fengDoS}. 

The zero dynamics attack \cite{zda,revealing} 
is the attack method taking advantage of system zeros. 
In particular, if a system has unstable zeros, 
then this attack is difficult to detect from 
a system output and a system state diverges. 
In general, networked control systems are 
deployed as sampled-data systems, which is 
composed of a continuous time plant and a 
discrete time digital controller. 
In this case, the zeros which do not exist in 
a continuous plant may appear because of 
discretization, which are called discrete 
zeros \cite{astrom}. Because there may exist 
unstable discrete zeros, the zero dynamics attack 
is seemed to be important method for sampled-data 
systems. 

For networked control systems, 
the control method which considers a quantization 
of signals caused by actuator capabilities and 
limitations of network communication. 
However, the attack methods previously mentioned 
assume the continuous signals. 
Therefore, it is necessary to study on an attack 
method which takes signal quantization effect into account.

In this paper, we consider a sampled-data system 
whose input signals are quantized. If a system 
input is continuous, the zero dynamics attack 
reduces effect on a system output by adjusting 
its initial value. If not, quantization effect 
appears with respect to its width and a system 
state and output are affected. Consequently, 
the attack performance such as the difficulty 
of detection is seemed to be spoiled.

Therefore, we propose the construction methods 
of the attack and analyze its effect, which reduce 
quantization effect on a system output and deviate 
a system state while a system input is quantized. 
In particular, for the cases where a sampled-data 
system has unstable zeros, we propose the two types 
of methods as follows.
\begin{itemize}
    \item the method utilizes a continuous attack signal.
    \item the method directly calculates a quantized attack signal.
\end{itemize}

The reminder of this paper is organized as follows: 
Section~\ref{ch:problem} describes the system considered 
in this paper, the input quantizer and the conditions 
of the attack. In Section~\ref{ch:dq}, we briefly overview 
the approach of dynamic quantizers, which is used to 
adjust a continuous valued attack. 
In Section~\ref{ch:nonmin}, we explain the methods 
of quantized attacks for a non-minimum phase 
sampled-data systems and analyze their effects. 
In Section~\ref{ch:example}, we illustrate our
results via a numerical example.  
In Section~\ref{ch:conclusion}, we provide 
concludiong remarks. 

\textbf{Notation}: We denote by $\mathbb{R}$, $\mathbb{N}$ and $\mathbb{Z}$ the sets of real numbers, natural numbers and integers respectively. Moreover, $\mathbb{R}^n$, $\mathbb{R}^{n\times m}$ are $n$ dimensional real vectors and $n$ by $m$ real matrices respectively. For $d>0$, $d\mathbb{Z}$ is the set of numbers which can be expressed as $dz$ by using an integer $z\in\mathbb{Z}$. The set of finite sequences are denoted as $l_\infty$. 
%Non-negative integers are denoted as $\mathbb{Z}_+$. 
%For a matrix $A:=[A_{ij}]$, $\abs(A):=[|A_{ij}|]$. The $\infty$-norm of the vector $v$ is written as $\left\| v\right\|_\infty=\max_i|v_i|$. Similarly, $\left\| A\right\|_\infty$ is the $\infty$-norm of the matrix $A$, which is induced by vector's $\infty$-norm. 