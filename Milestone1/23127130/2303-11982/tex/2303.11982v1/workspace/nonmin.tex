\section{Quantized attacks for non-minimum phase systems}\label{ch:nonmin}
In this section, we discuss quantized attacks for the case when the sampled-data system \eqref{eq:pr:csys} has unstable zeros. Concretely, we consider three types of methods as follows:

(i)~The first method is simple, where zero dynamics attacks as described in Section~II-C are generated and quantized directly using the static quantizer.\par
(ii)~The second approach employs the dynamic quantizer outlined above to modify the zero dynamics attacks in real values. We can show that this method can achieve smaller output error than statically quantizing the zero dynamics attacks as in the first method. However, it is affected by the number of unstable zeros and their values. \par
(iii)~The third method quantizes attack signals which are slightly different from the conventional zero dynamics attacks and may induce small but bounded influence on the system output. This signal can be constructed without the influence of unstable zeros. Moreover, as we will see in the analysis as well as numerical examples, this method is capable to outperform the other two methods in terms of the output error.\par
In this section, we look at these three attack methods and analyze the resulting output errors. 
\subsection{Quantized attack method 1 via static quantization}\label{se:sqzda}
We start with the simplest quantization approach where the attack signal $a[k]$ generated by \eqref{eq:pr:zda} is directly 
quantized in a static manner. That is, the attack
signal $b[k]$ to be injected to the quantized control input $v[k]$ 
in Fig.~1 is given by
\begin{equation}
    b[k] = q(a[k]).
    \label{eq:no:bqa}
\end{equation} 
For this method, we obtain the following result.
%Regarding the quantized attack $\{b[k]\}_{k=0}^{\infty}$ 
%which is calculated by \eqref{eq:pr:zda} and the static 
%quantizer, we obtain the following result.
\begin{prop}
For the quantized networked system under attacks in Fig.~\ref{fig:pr:sys}, suppose that the quantization width $d$ is given and the scalar $\epsilon$ satisfies
\begin{equation}
    \epsilon>\frac{1}{2}\sum\limits_{k=0}^\infty\left|CA^kB\right|d.\label{eq:no:staticeps}
\end{equation}
Moreover, suppose that the attack signal generated by \eqref{eq:pr:zda} is statically quantized as in \eqref{eq:no:bqa}. Then, for any sequence $\{H_k\}_{k=0}^{\infty}$, there exists a quantized attack signal $\{b[k]\}_{k=0}^{\infty}$ that is $\epsilon$-stealthy and $\{H_k\}_{k=0}^{\infty}$-disruptive.
\end{prop}
\begin{sketchofproof}
First, we can upper bound the output error as
\begin{multline*}
    \left|y_{v+b}[k]-y_v[k]\right|\\\leq\left|y_{v+b}[k]-y_{v+a}[k]\right|+\left|y_{v+a}[k]-y_v[k]\right|.
\end{multline*}
The first term on the right-hand side is less than or equal to $\frac{1}{2}\sum_{k=0}^\infty\left|CA^kB\right|d$ from \eqref{eq:dq:staticeps}. By Lemma.~1 and \eqref{eq:no:staticeps}, we can choose the attack signal $\{a[k]\}_{k=0}^{\infty}$ which is $\left(\epsilon-\frac{1}{2}\sum_{k=0}^\infty\left|CA^kB\right|d\right)$-stealthy. Then, the right-hand side of the above inequality is less than or equal to $\epsilon$. Therefore the attack $\{b[k]\}_{k=0}^{\infty}$ is $\epsilon$-stealthy.\par
Next, we look at the difference in states. We have
\begin{align*}
  &\left\| x_{v+b}[k]-x_v[k]\right\|_2\\
   &~~\geq \left\| x_{v+a}[k]-x_v[k]\right\|_2
       - \left\| x_{v+b}[k]-x_{v+a}[k]\right\|_2.
\end{align*}
On the right-hand side, the first term diverges as $k\rightarrow\infty$. The second term on the right-hand side coincides with the state which is driven by the quantization error; since the quantization error is bounded and the system \eqref{eq:pr:dsys} is stable, this term is also bounded. It means that $\left\| x_{v+b}[k]-x_v[k]\right\|_2$ diverges as $k\rightarrow\infty$. Therefore, we conclude that the attack $\{b[k]\}_{k=0}^{\infty}$ is $\{H_k\}_{k=0}^{\infty}$-disruptive.
\end{sketchofproof}
\subsection{Quantized attack method 2 via dynamic quantization}\label{se:dqzda}
In this subsection, we propose a method which modifies the continuous zero dynamics attacks in Section \ref{ch:problem} by using dynamic quantization from the previous section. This method adopts dynamic quantizer based on serial decomposition in Section \ref{se:dqseri}. \par
Specifically, we apply \eqref{eq:dq:dqseri} to generate attack signals after obtaining the serial decompositions $P_s$, $P_u$ of the system \eqref{eq:pr:dsys} satisfying conditions outlined there. The effects to output signals caused by this attack is determined by the choices of $P_s$, $P_u$. As we discussed earlier, while this method can be utilized for systems with any number of unstable zeros, the optimal value for the output error is known only for the case of one unstable zero. The following theorem is the first main result of this paper stating that dynamic quantizer can be efficient in reducing the error due to quantized attacks. 
\begin{theo}\label{theo:dqzda}
For the quantized networked system under attacks in Fig.~\ref{fig:pr:sys}, suppose that the plant is decomposed serially as described in Section \ref{se:dqseri} and $\epsilon$ satisfies
\begin{equation}
    \epsilon>\frac{1}{2}\left\|P_u\right\|_{i\infty}\left|C_sB_s\right|d,\label{eq:no:eps}
\end{equation}
where $B_s$ and $C_s$ are the system matrices of $P_s$. Then, for any sequence $\{H_k\}_{k=0}^\infty$, there exists a quantized attack signal $\{b[k]\}_{k=0}^\infty$ generated by \eqref{eq:pr:zda} and dynamic quantization that is $\epsilon$-stealthy and $\{H_k\}_{k=0}^\infty$-disruptive.
\end{theo}\par
\begin{proof}
By Lemma \ref{lem:zda}, we can construct the attack signal $\{a[k]\}_{k=0}^{\infty}$ such that it is $\left(\epsilon-\frac{1}{2}\left\|P_u\right\|_{i\infty}\left|C_sB_s\right|d\right)$-stealthy and $\left\|x_{u+a}[k]-x_u[k]\right\|_2$ diverges. Consider the attack $\{b[k]\}_{k=0}^{\infty}$ which is made from $\{a[k]\}_{k=0}^{\infty}$ and the dynamic quantizer in Section \ref{se:dqseri}. We verify that this attack satisfies both $\epsilon$-stealthy and $\{H_k\}_{k=0}^{\infty}$-disruptive properties.\par
We can upper bound the output error as
\begin{align}
  & |y_{v+b}[k]-y_v[k]|\\
  &~~~\leq|y_{v+b}[k]-y_{v+a}[k]|
          +| y_{v+a}[k]-y_v[k]|\label{eq:no:ineqy}
\end{align}
for every $k$. Since the system is linear, the first term on the right-hand side coincides with $|y_b[k]-y_a[k]|$. By the bound on the output error \eqref{eq:dq:dqserierroropt} for the dynamic quantizer, this term can be bounded as
\begin{equation}
    \sup\limits_k|y_b[k]-y_a[k]|\leq\frac{1}{2}\left\|P_u\right\|_{i\infty}|C_sB_s|d.\notag
\end{equation}
By the construction of $\{a[k]\}_{k=0}^{\infty}$, the second term on the right-hand side of \eqref{eq:no:ineqy} satisfies the following inequality:
\begin{equation}
    \sup\limits_k|y_{v+a}[k]-y_v[k]|\leq\epsilon-\frac{1}{2}\left\|P_u\right\|_{i\infty}|C_sB_s|d.\notag
\end{equation}
Hence, from \eqref{eq:no:ineqy}, we have $\sup_k| y_{v+b}[k]-y_v[k]|\leq\epsilon$. Consequently, the attack $\{b[k]\}_{k=0}^\infty$ is $\epsilon$-stealthy.\par
Next, we look at the difference in states. We have
\begin{multline}
    \left\| x_{v+b}[k]-x_v[k]\right\|_2\\\geq\left\| x_{v+a}[k]-x_v[k]\right\|_2-\left\| x_{v+b}[k]-x_{v+a}[k]\right\|_2.\label{eq:no:statediff}
\end{multline}
Because of the choice of $\{a[k]\}_{k=0}^{\infty}$, the first term on the right-hand side diverges as $k\rightarrow\infty$. Since the system is linear, it follows that $x_{v+b}[k]-x_{v+a}[k]=x_b[k]-x_a[k]$. The state-space equation of the plant and the dynamic quantizer can be written as
\begin{equation}
\begin{bmatrix}
x_{b}[k+1] \\
\xi[k+1]\\
x_{a}[k+1]
\end{bmatrix}\hskip-3pt=\hskip-3pt\left[\begin{array}{@{\hskip-1pt}c@{\hskip1pt}c@{\hskip1pt}c@{\hskip-1pt}}
A&O&O\\
O&A_s&O\\
O&O&A 
\end{array}\right]\hskip-5pt\begin{bmatrix}
x_{b}[k] \\
\xi[k]\\
x_{a}[k]
\end{bmatrix}\hskip-2pt+\hskip-2pt\begin{bmatrix}
Bb[k] \\
B_s(b[k]-a[k])\\
Ba[k]
\end{bmatrix}.\label{eq:no:xbxixa}
\end{equation}
Let $w_q[k]$ be
\begin{equation*}
        w_q[k]:=q\left(-\frac{C_sA_s}{C_sB_s}\xi[k]+a[k]\right)-\left\{-\frac{C_sA_s}{C_sB_s}\xi[k]+a[k]\right\}.
\end{equation*}
Then, the quantized attack is expressed as $b[k]=a[k]+w_q[k]-\frac{C_sA_s}{C_sB_s}\xi[k]$. Note that $\left\|w_q[k]\right\|_\infty\leq d/2$. Also, let the system matrix $\bar{A}$ be
\begin{equation*}
    \bar{A}:=\begin{bmatrix}
    A & -\frac{BC_sA_s}{C_sB_s} \\
    O & A_s-\frac{B_sC_sA_s}{C_sB_s}
    \end{bmatrix}.
\end{equation*}
\quad Then, the state-space equation in \eqref{eq:no:xbxixa} can be rewritten into 
\begin{equation}
\begin{bmatrix}
x_{b}[k+1] \\
\xi[k+1]\\
x_{a}[k+1]
\end{bmatrix}\hskip-3pt=\hskip-3pt\left[\begin{array}{@{\hskip-1pt}c|c@{\hskip-1pt}}
\raisebox{-0.5em}{\mbox{\LARGE $\bar{A}$}} & \begin{matrix}
O \\ O
\end{matrix}
\\
\hline
 \begin{matrix}
O & O
\end{matrix} & A
\end{array}\right]\hskip-5pt\begin{bmatrix}
x_{b}[k] \\
\xi[k]\\
x_{a}[k]
\end{bmatrix}+\begin{bmatrix}
B(a[k]+w_q[k]) \\
B_sw_q[k]\\
Ba[k]
\end{bmatrix}.\notag
\end{equation}
Applying $x_{b}[k]-x_a[k]=[I~O~-I][x_{b}[k]~\xi[k]\ x_{a}[k]]^T$ and $\xi[0]=0$ to it, we can obtain the relation
\begin{equation}
x_{b}[k]-x_a[k]=\sum\limits_{l=0}^{k-1}\begin{bmatrix}
I & O
\end{bmatrix}\bar{A}^{l}\begin{bmatrix}
B \\
B_s
\end{bmatrix}w_q[k-l].\notag
\end{equation}
Since the system \eqref{eq:pr:dsys} is stable and the subsystem $P_s$ is minimum phase, the matrix $\bar{A}$ is also stable. Thus, the sequence $x_{b}[k]-x_a[k]$ is bounded, implying that the second term on the right-hand side of \eqref{eq:no:statediff} is bounded. Consequently, $\left\| x_{v+b}[k]-x_v[k]\right\|_2\rightarrow\infty$ as $k\rightarrow\infty$. Therefore, we conclude that the quantized attack signal $\{b[k]\}_{k=0}^\infty$ is $\{H_k\}_{k=0}^{\infty}$-disruptive.
\end{proof}
\subsection{Quantized attack method 3 via $\epsilon$-stealthy approach}\label{se:calc}

The third method employs a continuous attack signal 
which is a slightly modified version of the zero 
dynamics attack in \eqref{eq:pr:zda} and then quantizes 
it directly. 

To this end, for given $\epsilon>0$, let us consider 
the attack signal $\{a[k]\}_{k=0}^{\infty}$ generated by
\begin{equation}
    \begin{split}
        z[k+1]&=\left(A-\frac{BCA}{CB}\right)z[k]+\frac{B}{CB}\epsilon,\\
        a[k]&=-\frac{CA}{CB}z[k]+\frac{\epsilon}{CB},
    \end{split}\label{eq:no:zdag}
\end{equation}
where $z[0]=0$. We can easily show that this system 
is unstable and the state difference difference $x_{u+a}[k]-x_u[k]$ is equal to $z[k]$. 
It means that $y_{u+a}[k]-y_u[k]=Cz[k]$ and this value 
coincides with $\epsilon$ for $k\geq1$. 
Hence, this attack is $\epsilon$-stealthy in the sense of \eqref{eq:pr:estealth} (without quantization).

The system \eqref{eq:no:zdag} motivates us to generate attack
signals using a system which has constant input proportional 
to $\epsilon$ and then to quantize the signal. 
This can be achieved by introducing the following quantized system:
%a quantized version which generates the quantized attack signal $\{b[k]\}_{k=0}^{\infty}$ as
\begin{equation}
 \begin{split}
   z[k+1]&=Az[k]+Bb[k],\\
        b[k]&=q\left(-\frac{CA}{CB}z[k]+\frac{\sign(CB)\epsilon}{CB}-\frac{d}{2}+\Delta\right),
    \end{split}\label{eq:no:dzdag}
\end{equation}
where $\Delta$ is a sufficiently small positive number. \par
Now, we are ready to state our second main result of the paper. 
\begin{theo}\label{theo:2}
For the quantized networked system under attacks in Fig.~\ref{fig:pr:sys}, suppose that $\epsilon$ is taken as $\epsilon \geq |CB|d$. Then, for any sequence $\{H_k\}_{k=0}^{\infty}$, the quantized attack $\{b[k]\}_{k=0}^{\infty}$ generated by \eqref{eq:no:dzdag} is $\left(\epsilon + |CB|\Delta\right)$-stealthy and $\{H_k\}_{k=0}^{\infty}$-disruptive. 
%if the following conditions are satisfied:
%\begin{itemize}
%    \item Consider the Jordan decomposition of $\left(A-\frac{BCA}{CB}\right)=P\Lambda P^{-1}$. Moreover, let the diagonal elements of $\Lambda$ be $\sigma_1,\ldots,\sigma_n$ and $\sigma_l,\ldots,\sigma_n$ be unstable ones. Then at least one of the $l-th$ through the last element of $B':=P^{-1}B$ is not $0$.
%\end{itemize}
\end{theo}
\begin{proof}
It is straightforward to show that by \eqref{eq:no:zdag}, the differences in states and output can be written as $x_{v+b}[k]-x_v[k]=z[k]$ and $y_{v+b}[k]-y_v[k]=Cz[k]$ respectively. Let  
\[
  w[k]:=(-\frac{CA}{CB}z[k]+\frac{\sign(CB)\epsilon}{CB}+\Delta)-b[k]. 
\]
Then, \eqref{eq:no:dzdag} becomes
\begin{multline}
    z[k+1]=\left(A-\frac{BCA}{CB}\right)z[k]\\+B\left(\frac{\sign(CB)\epsilon}{CB}+\Delta-w[k]\right),
\end{multline}
where $0<w[k]\leq d$. Since $z[0]=0$, we have 
\begin{align}
&z[k]\notag\\
&~~= \sum_{l=0}^{k-1}\left(A-\frac{BCA}{CB}\right)^{k-l-1}
   B\left(\frac{\sign(CB)\epsilon}{CB}+\Delta-w[l]\right).
\label{eq:no:zk}
\end{align}
Note that it holds $C\left(A-\frac{BCA}{CB}\right)=0$ and $Cz[k]=\sign(CB)\epsilon+CB\Delta-CBw[k-1]$ for all $k\geq1$. Since $\epsilon\geq|CB|d$, it holds 
\[
  -\left(\epsilon+|CB|\Delta\right)
    \leq|Cz[k]|\leq\epsilon+|CB|\Delta. 
\]
It means that the attack $\{b[k]\}_{k=0}^{\infty}$ is $\left(\epsilon+|CB|\Delta\right)$-stealthy.\par
Next, to establish that $\{b[k]\}_{k=0}^{\infty}$ is $\{H_k\}_{k=0}^{\infty}$-disruptive, we must show that $z[k]$ diverges, which implies that $\left\|x_{v+b}[k]-x_v[k]\right\|_2$ also diverges. Let 
\[
  w'[k]:=\frac{\sign(CB)\epsilon}{CB}+\Delta-w[k].
\] 
Now, based on \eqref{eq:no:zk}, we can rewrite the system of $z[k]$ and
obtain
\begin{equation*}
   z[k+1] = \left(A-\frac{BCA}{CB}\right) z[k] + B w'[k].
\end{equation*}
Since $z[0]=0$, we must make sure that in this system, the input $w'[k]$ enters and excites the unstable mode. \par 
Recall that $\left(A,B\right)$ is controllable, and thus $\left(A-\frac{BCA}{CB}, B\right)$ is also controllable. Hence, it follows that the input $w'[k]$ enters every mode of the system. Finally, we can confirm that $w'[k]$ is not zero at all times. By assumption, $\epsilon\geq|CB|d$, and this implies that $w'[k]\geq\Delta$ at each $k$. Therefore, the state $z[k]$ diverges.
%Next, we look at $z[k]$. The matrix $A-\frac{BCA}{CB}$ can be denoted as $P\Lambda P^{-1}$ by Jordan decomposition. Moreover, we can let the diagonal elements of $\Lambda$ be $\sigma_1,\ldots,\sigma_n$ and $\sigma_l,\ldots,\sigma_n$ be unstable ones.  Since $\epsilon\geq|CB|d$, the new variable $w'[k]:=\frac{\sign(CB)}{CB}\epsilon+\Delta-w[k]$ satisfies $w'[k]\geq\Delta$. Then, \eqref{eq:no:zk} becomes
%\begin{equation}
%    z[k]=\sum_{l=0}^{k-1}P\Lambda^{k-l-1}B'w'[l],\label{eq:no:zk2}
%\end{equation}
%where $B':=P^{-1}B$. Since $\left(A, B\right)$ is controllable, $\left(A-\frac{BCA}{CB}, B\right)$ is also controllable. Therefore, the vector $B'$ has nonzero elements and the last one has index greater than or equal to $l$. Here, we denote this index by $l^*$. Then the state $P^{-1}z[k]$ can be expressed as
%\begin{equation}
%    P^{-1}z[k]=\begin{bmatrix}
%    *\\
%    \vdots\\
%    *\\
%    \sum_{l=0}^{k-1}\sigma_{l^*}^{k-l-1}B_{l^*}'w'[l]\\
%    0\\
%    \vdots\\
%    0
%    \end{bmatrix}.\label{eq:no:zk3}
%\end{equation}
%The square norm of $P^{-1}z[k]$ can be bounded as
%\begin{equation*}
%        \left\|P^{-1}z[k]\right\|_2\leq\left\|P^{-1}\right\|_2\left\|z[k]\right\|_2.
%\end{equation*}
%Then, that of $z[k]$ can be bounded as
%\begin{equation*}
%    \begin{split}
%        \left\|z[k]\right\|_2&\geq\frac{1}{\left\|P^{-1}\right\|_2}\left\|P^{-1}z[k]\right\|_2\\
%        &\geq\frac{1}{\left\|P^{-1}\right\|_2}\left|\sum_{l=0}^{k-1}\sigma_{l^*}^{k-l-1}B_{l^*}'w'[l]\right|\rightarrow\infty.
%    \end{split}
%\end{equation*}
%Hence, $\left\|x_{v+b}[k]-x_v[k]\right\|_2$ also diverges. Consequently, the attack $\{b[k]\}_{k=0}^{\infty}$ is  $\{H_k\}_{k=0}^{\infty}$-disruptive.
\end{proof}
\quad We remark that to achieve the disruptive property, the assumption that $\epsilon \geq|CB|d$ is critical for the quantized attack \eqref{eq:no:dzdag}. This is because the initial state is $z[0]=0$, if $\epsilon<|CB|d$, then $z[k]\equiv0$. This will further result in $b[k]\equiv0$, that is, no quantized attack.\par
Furthermore, compared with the attack approach based on dynamic quantization  of Section \ref{se:dqzda}, the attack \eqref{eq:no:dzdag} of this method has the following feature: If there is only one unstable zero, then the output error is strictly less than 
the right-hand side of \eqref{eq:no:eps}. This holds because with sufficiently small $\Delta$, we have
\begin{equation*}
  \frac{1}{2}\left|CB\right|
    \left(1+|\lambda|\right)d
   >\left|CB\right|\left(d+\Delta\right)
\end{equation*}
where $\lambda$ is the unstable zero with $|\lambda|>1$.
We will confirm this difference between the two methods
in numerical examples presented in the next section.
