%%%%%%%%%%%%%%%%%%%%%%%%%%%%%%%%%%%%%%%%%%%%%%%%%%%%%%%%%%%%%%%%%%%%%%%%%%%%%%%%
%2345678901234567890123456789012345678901234567890123456789012345678901234567890
%        1         2         3         4         5         6         7         8

\documentclass[letterpaper, journal]{IEEEtran}  % Comment this line out if you need a4paper

%\documentclass[a4paper, 10pt, conference]{ieeeconf}      % Use this line for a4 paper

%\IEEEoverridecommandlockouts                              % This command is only needed if 
                                                          % you want to use the \thanks command

%\overrideIEEEmargins                                      % Needed to meet printer requirements.


%In case you encounter the following error:
%Error 1010 The PDF file may be corrupt (unable to open PDF file) OR
%Error 1000 An error occurred while parsing a contents stream. Unable to analyze the PDF file.
%This is a known problem with pdfLaTeX conversion filter. The file cannot be opened with acrobat reader
%Please use one of the alternatives below to circumvent this error by uncommenting one or the other
%\pdfobjcompresslevel=0
%\pdfminorversion=4

% See the \addtolength command later in the file to balance the column lengths
% on the last page of the document

% The following packages can be found on http:\\www.ctan.org
\usepackage{graphics} % for pdf, bitmapped graphics files
%\usepackage{epsfig} % for postscript graphics files
%\usepackage{mathptmx} % assumes new font selection scheme installed
%\usepackage{times} % assumes new font selection scheme installed
\usepackage{amsmath} % assumes amsmath package installed
\usepackage{amssymb}  % assumes amsmath package installed
\usepackage[dvipdfmx]{graphicx}
\usepackage{booktabs}
\usepackage{cite}
\usepackage{enumerate}
\usepackage{cleveref}
\usepackage{url}
\usepackage{titlesec}
\usepackage[subrefformat=parens]{subcaption}
\newcommand{\sign}{\operatorname{sign}}
\newtheorem{theo}{Theorem}
\newtheorem{assump}{Assumption}
\newtheorem{lem}{Lemma}
\newtheorem{prop}{Proposition}
\newenvironment{proof}{\textit{Proof:}}{\hfill  $\blacksquare$}
\newenvironment{sketchofproof}{\textit{Sketch of proof:}}{\hfill  $\blacksquare$}

\begin{document}

\title{Quantized Zero Dynamics Attacks against\\ Sampled-data Control Systems}


\author{Kosuke Kimura and Hideaki Ishii,~\IEEEmembership{Fellow,~IEEE}
        % <-this % stops a space
\thanks{This work was supported in the part by JSPS under Grants-in-Aid for 
	Scientific Research Grant No.~18H01460 and 22H01508.}
\thanks{%
K.~Kimura is with the Department of 
System Control and H.~Ishii is with the
Department of Computer Science, both at
the Tokyo Institute of Technology, 
Yokohama, 226-8502, Japan. 
(e-mail: kimura@sc.dis.titech.ac.jp, ishii@c.titech.ac.jp).}}

\maketitle
%\thispagestyle{empty}
%\pagestyle{empty}


%%%%%%%%%%%%%%%%%%%%%%%%%%%%%%%%%%%%%%%%%%%%%%%%%%%%%%%%%%%%%%%%%%%%%%%%%%%%%%%%
\begin{abstract}
For networked control systems, cyber-security issues have gained much attention in recent years. In this paper, we consider the so-called zero dynamics attacks, which 
form an important class of false data injection attacks, with a special focus on the effects of quantization in a sampled-data control setting. When the attack signals must be quantized, some error will be necessarily introduced, potentially increasing the chance of detection through the output of the system. In this paper, we show however that the attacker may reduce such errors by avoiding to directly quantize the attack signal. We look at two approaches for generating quantized attacks which can keep the error in the output smaller than a specified level by using the knowledge of the system dynamics. The methods are based on a dynamic quantization technique 
and a modified version of zero dynamics attacks. Numerical examples are provided to verify the effectiveness of the proposed methods.
\end{abstract}
\begin{IEEEkeywords}
Cyber security, false-data injections, networked control systems, quantization, 
zero dynamics attacks
\end{IEEEkeywords}


%%%%%%%%%%%%%%%%%%%%%%%%%%%%%%%%%%%%%%%%%%%%%%%%%%%%%%%%%%%%%%%%%%%%%%%%%%%%%%%%
%\section{Introduction}
\label{sec:introduction}
% \begin{itemize}
%     % Diffusion of FL
%     \item {\st{Diffusion of FL}}
%     % Security threats to FL
%     \item {\st{Security threats to FL with particular focus on model poisoning}}
%     % Limitations of existing countermeasures
%     \item {\st{Current countermeasures (e.g., KRUM) and their limitations}}
%     % Proposed method and its advantages
%     \item {\st{Intuitive description of the proposed method and its difference (i.e., advantages) w.r.t. state of the art}}
%     % Main contributions
%     \item {\st{Summary of the main contributions of this work}}
%     % Paper's structure and organization
%     \item {\st{Paper's structure and organization}}
% \end{itemize}

% Diffusion of FL
Recently, {\em federated learning} (FL) has emerged as the leading paradigm for training distributed, large-scale, and privacy-preserving machine learning (ML) systems~\cite{mcmahan2017googleai,mcmahan2017aistats}. 
The core idea of FL is to allow multiple edge clients to collaboratively train a shared, global model without disclosing their local private training data.
%Specifically, an FL system consists of a central server and many edge clients; 
A typical FL round involves the following steps: {\em(i)} the server randomly picks some clients and sends them the current, global model; {\em(ii)} each selected client locally trains its model with its own private data; then, it sends the resulting local model to the server;\footnote{Whenever we refer to global/local model, we mean global/local model {\em parameters}.} {\em(iii)} the server updates the global model by computing an \emph{aggregation function}, usually the average (FedAvg), on the local models received from clients.
% \begin{enumerate}
%     \item[{\em(i)}] the server sends the current, global model to the clients and appoints some of them for training;
%     \item[{\em(ii)}] each selected client locally trains its copy of the global model with its own private data; then, it sends the resulting local model back to the server;\footnote{Whenever we refer to global/local model, we mean global/local model {\em parameters}.}
%     \item[{\em(iii)}] the server updates the global model by computing an \emph{aggregation function} on the local models received from clients (by default, the average, also referred to as FedAvg~\cite{mcmahan2017aistats}).
% \end{enumerate}
This process goes on until the global model converges. %(e.g., after a certain number of rounds or other similar stopping criteria).
%\\
% The advantages of FL over the traditional, centralized learning paradigm are undoubtedly clear in terms of flexibility/scalability (clients can join/disconnect from the FL network dynamically), network communications (only model weights\footnote{We will use \textit{parameters} and \textit{weights} interchangeably.} are exchanged between clients and server), and privacy (each client's private training data is kept local at the client's end and not uploaded to the server).
\\
% Security threats to FL
%However, the growing adoption of FL also raises security concerns~\cite{costa2022covert}, particularly about its confidentiality, integrity, and availability.
Although its advantages over standard ML, FL also raises security concerns~\cite{costa2022covert}. %, particularly about its confidentiality, integrity, and availability~\cite{costa2022covert}.
% OLD, LONG VERSION
% Indeed, some work deals with privacy leakage that may expose the local data of some clients~\cite{melis2019sp}. 
% A large body of work, instead, investigates attacks that usually aim to detriment the predictive accuracy of the learned global model. For instance, \emph{data poisoning} attacks achieve this goal by letting an adversary pollute the training set of some corrupt FL clients with maliciously crafted examples~\cite{jagielski2018sp}.
% Similarly, in \emph{model poisoning} the attacker attempts to tweak the global model weights~\cite{bhagoji2019pmlr} by directly perturbing the local model's weights of some infected FL clients before these are sent to the central server for aggregation, usually via so-called Byzantine attacks. 
% It turns out that Byzantine model poisoning attacks severely impact standard FedAvg; therefore, more robust aggregation functions must be designed to make FL systems secure.
Here, we focus on \emph{untargeted model poisoning} attacks~\cite{bhagoji2019pmlr}, where an adversary attempts to tweak the global model weights %\footnote{We will use the terms \textit{parameters} and \textit{weights} interchangeably.} 
by directly perturbing the local model's parameters of some infected clients before these are sent to the central server for aggregation.
In doing so, the adversary aims to jeopardize the global model \textit{indiscriminately} at inference time.
Such model poisoning attacks severely impact standard FedAvg; therefore, more robust aggregation functions must be designed to secure FL systems.
\\
% In this paper, we focus on designing a novel robust aggregation scheme at the server's end to contrast the effect of Byzantine model poisoning attacks.
%
% Current countermeasures and their limitations
%Several countermeasures have been proposed in the literature to combat model poisoning attacks on FL systems.
% Some methods use simple statistics more robust than plain average to smooth the impact of malicious updates (e.g., Trimmed Mean and FedMedian~\cite{yin2018icml}). 
% Other defenses implement outlier detection techniques to discard malicious updates from the aggregation performed at the server's end. Those are either based on heuristics (e.g., Krum/Multi-Krum~\cite{blanchard2017nips} and Bulyan~\cite{mhamdi2018pmlr}) or data-driven approaches (e.g., K-means clustering~\cite{shen2016acm} or DnC via spectral analysis~\cite{shejwalkar2021ndss}). 
% Finally, some strategies rely on a centralized ``source of trust'' to spot potential malicious updates (e.g., FLTrust~\cite{cao2020fltrust}).
% Several countermeasures have been proposed in the literature to combat model poisoning attacks on FL systems, i.e., to discard possible malicious local updates from the aggregation performed at the server's end. 
% These techniques range from simple statistics more robust than plain average (e.g., Trimmed Mean and FedMedian~\cite{yin2018icml}) to outlier detection heuristics (e.g., Krum/Multi-Krum~\cite{blanchard2017nips} and Bulyan~\cite{mhamdi2018pmlr}) or data-driven approaches (e.g., spectral analysis via K-means clustering~\cite{shen2016acm} or spectral analysis), or methods based on ``source of trust'' (e.g., FLTrust~\cite{cao2020fltrust}).
% OLD, LONG VERSION
%Several countermeasures have been proposed in the literature to combat Byzantine model poisoning attacks on FL systems.
% Descriptive statistics
% For example, Trimmed Mean and FedMedian aggregate local model updates using more robust statistics than standard average~\cite{yin2018icml}.
%
% % Heuristics for outlier detection
% Many existing Byzantine-resilient strategies implement some outlier detection heuristics to discard the model updates sent by potentially malicious clients from the input of the aggregation function.
% One of the most popular heuristics is Krum~\cite{blanchard2017nips}.
% This strategy tries to mitigate the impact of Byzantine attacks by selecting as a global model the local model with the smallest sum of Euclidean distances to {\em all} the other local models.
% Although powerful, Krum requires the server to know (or, at least, estimate) the number of malicious FL clients upfront, which is generally impossible in a realistic attack scenario. %
% Moreover, Krum may become ineffective for complex, high-dimensional model parameter spaces due to the curse of dimensionality.
% Bulyan~\cite{mhamdi2018pmlr} tries to overcome this issue by combining Krum with a variant of Trimmed Mean.
% % Data-driven outlier detection
% Other strategies use data-driven outlier detection techniques -- e.g., via K-means clustering~\cite{shen2016acm} -- to spot potential malicious local model updates. 
% %For instance, Shen et al. propose to cluster local model updates with K-means and thus identify outliers.
%
% % Other techniques
% As far as the server is concerned, any local model received can be from a potential malicious client. 
% FLTrust~\cite{cao2020fltrust} assumes the server acts as a client, i.e., trains a local model on an additional {\em trustworthy} dataset at the server's end and compares it against all the local models from other clients. 
% This way, the server can rely on some ``source of trust'' when discarding potentially malicious clients.
%\\
% Limitations of existing Byzantine-resilient strategies
Unfortunately, existing defense mechanisms either rely on simple heuristics (e.g., Trimmed Mean and FedMedian by~\cite{yin2018icml}) or need strong and unrealistic assumptions to work effectively (e.g., foreknowledge or estimation of the number of malicious clients in the FL system, as for Krum/Multi-Krum~\cite{blanchard2017nips} and Bulyan~\cite{mhamdi2018pmlr}, which, however, cannot exceed a fixed threshold).
Furthermore, outlier detection methods using K-means clustering~\cite{shen2016acm} or spectral analysis like DnC~\cite{shejwalkar2021ndss} do not directly consider the temporal evolution of local model updates received.
Finally, strategies like FLTrust~\cite{cao2020fltrust} require the server to collect its own dataset and act as a proper client, thereby altering the standard FL protocol.
\\
% OLD, LONG VERSION
% Overall, existing Byzantine-resilient strategies are either simple heuristics (e.g., FedMedian) or, if they are more complex, they rely on strong and unrealistic assumptions to work effectively (e.g., knowing the number of malicious clients in the FL system in advance, as for Krum and alike).
% Furthermore, data-driven outlier detection methods do not consider the temporary evolution of local model updates received (e.g., K-means clustering). 
% Finally, strategies like FLTrust requires the server to collect its own dataset and act as a proper client, thereby altering the standard FL protocol.
%
% Description of the proposed method
This work introduces a novel pre-aggregation \textit{filter} robust to untargeted model poisoning attacks. Notably, this filter $(i)$ operates without requiring prior knowledge or constraints on the number of malicious clients and $(ii)$ inherently integrates temporal dependencies. 
The FL server can employ this filter as a preprocessing step before applying \textit{any} aggregation function, be it standard like FedAvg or robust like Krum or Bulyan.
Specifically, we formulate the problem of identifying corrupted updates as a multidimensional (i.e., matrix-valued) time series anomaly detection task. 
The key idea is that legitimate local updates, resulting from well-calibrated iterative procedures like stochastic gradient descent (SGD) with an appropriate learning rate, show \textit{higher predictability} compared to malicious updates. This hypothesis stems from the fact that the sequence of gradients (thus, model parameters) observed during legitimate training exhibit regular patterns, as validated in Section~\ref{subsec:intuition}. %until convergence. 
%This regularity may be more pronounced for smooth convex loss functions, but it can still be captured within an appropriate time window, even for more complex and convoluted loss surfaces. 
%We provide evidence of this claim in Appendix~B, where we show that the average mutual information (i.e., ``predictability''), calculated over pairs of legitimate model updates sent at different FL rounds, is significantly higher than the corresponding computation for a malicious client.
\\
Inspired by the matrix autoregressive (MAR) framework for multidimensional time series forecasting~\cite{chen2021je}, we propose the FLANDERS ({\em \textbf{F}ederated \textbf{L}earning meets \textbf{AN}omaly \textbf{DE}tection for a \textbf{R}obust and \textbf{S}ecure}) filter.
The main advantages of FLANDERS over existing strategies like FLDetector~\cite{zhao2020multivariate} are its resilience to large-scale attacks, where $50\%$ or more FL participants are hostile, and the capability of working under realistic non-iid scenarios.
We attribute such a capability to two key factors: $(i)$ FLANDERS works without knowing a priori the ratio of corrupted clients, and $(ii)$ it embodies temporal dependencies between intra- and inter-client updates, quickly recognizing local model drifts caused by evil players. Below, we summarize our main contributions:

\begin{itemize}
\item[{\em(i)}]
We provide empirical evidence that the sequence of models sent by legitimate clients is more predictable than those of malicious participants performing untargeted model poisoning attacks.
\\
\item[{\em(ii)}] 
We introduce FLANDERS, the first pre-aggregation filter for FL robust to untargeted model poisoning based on multidimensional time series anomaly detection.
\\
\item[{\em(iii)}] 
We integrate FLANDERS into Flower,\footnote{\scriptsize{\url{https://flower.dev/}}} a popular FL simulation framework for reproducibility.
\\
\item[{\em(iv)}] 
We show that FLANDERS improves the robustness of the existing aggregation methods under multiple settings: different datasets, client's data distribution (non-iid), models, and attack scenarios.
\\
\item[{\em(v)}] 
We publicly release all the implementation code of FLANDERS along with our experiments.\footnote{\scriptsize{\url{https://anonymous.4open.science/r/flanders_exp-7EEB}}}
\end{itemize}

% Paper's structure and organization
The remainder of the paper is structured as follows. %some related work and the current state-of-the-art solutions to security issues that FL entails. 
Section~\ref{sec:background} covers background and preliminaries. 
In Section~\ref{sec:related}, we discuss related work.
Section~\ref{sec:problem} and Section~\ref{sec:method} describe the problem formulation and the method proposed. % to tackle it. 
Section~\ref{sec:experiments} gathers experimental results. %, and Section~\ref{sec:limitations} discusses some limitations of this work.
Finally, we conclude in Section~\ref{sec:conclusion}.
 %discusses the limitations of this work and draws future research directions.
%reports conclusions and draws perspectives for future research directions.

%%%%%%% OLD %%%%%%%
%to overcome the resilience of Byzantine failures in distributed Stochastic Gradient Descent computations. 
% The strength of Krum is its time complexity, which is linear in the gradient dimension. 
% However, the robustness of the approach is guaranteed for gradient-based learning applications only when the majority of the clients are not compromised. 
% Besides, the aggregation mechanism of Krum, as well as that of similar methods, is robust from a coarse-grained perspective and does not provide solutions to errors and perturbations that may occur at inference time.
%A related approach to~\cite{blanchard2017nips} is the work of Su et al.~\cite{su2016dc}. Here, the authors propose an iterated approximate agreement to tackle a multi-layer scenario attacked by Byzantine agents. 
%However, the method works efficiently on the sole discrete context and it is inapplicable to continuous state environments.
%\gabri{Maybe, we should just talk about the main limitations of existing countermeasures without digging into their details (or, we can just mention Krum as this is the most popular one). I will move the description of all these methods to the Related Work section.}
\section{Introduction}\label{ch:introduction}

Recent advances in communication technology are
enabling various cyber-physical systems (CPSs)
to be further connected by networks, leading to 
enhancements in efficiency and flexibility for
their operation and control. Application domains
where such changes have brought significant 
progresses include large-scale plants, smart grids, 
and traffic networks. It is however inevitable that 
along with the increase in network connectivity,
cyber-security issues have gained much more attention. 
Some of the major incidents include 
the attacks against nuclear research facilities in Iran
\cite{Farwell2011StuxnetAT}, 
power grids in Ukraine \cite{ukraine}, and 
sewage systems in Australia \cite{Slay2007LessonsLF}.

To maintain the defense in depth for security and safety
of such CPSs, it is of critical
importance to strengthen security methods from the
perspective of systems control
to complement conventional security methods based on 
information technologies \cite{FerTei:22,IshZhu:22}.
In particular, malicious cyber attacks against CPSs
may have physical consequences, which can 
potentially result in damages in control devices and
facilities. 

In the systems control area, recent research has studied
analyses on impacts of such attacks,
detection techniques, as well as resilient control methods.
Representative classes of attacks that may lead to manipulating
vulnerable physical plants include those of replay attacks, 
Denial-of-Service (DoS) attacks, and 
false-data injection (FDI) attacks;
see, for example, \cite{MoKimBra:12,contsec,TeiShaSan:15}
and the references therein.

The focus of this paper is on the class of FDI attacks
known as the zero dynamics attacks \cite{ParShiLee:19,TeiShaSan:15}. 
An attacker who is aware of the system dynamics may
modify the control signals in such a way that the
internal states of the system are manipulated by the
attacker and can diverge. 
They can be generated by taking advantage of 
system zeros and, in particular, the unstable ones. 
The difficulty in dealing with such attacks is that
the behavior of the system output remains the same 
as that without the attacks. Hence, it is hard to detect 
them, e.g.,
by conventional fault detection techniques.

In this paper, we formulate a networked control 
problem in the sampled-data setting, where 
the continuous-time plant is controlled by a digital 
compensator and the control 
input is transmitted over a network channel \cite{CheFra:95}.
We place particular attention to the role played by 
quantization in the control signals. 
Since zero dynamics
attacks normally require the attack signals to be continuous,
discretizing them through quantization will introduce certain errors
not only in the attacks but also in the system outputs,
This may make the attacks more visible from the system operators,
helping them to detect the malicious activities. 
In this sense, the security level can be enhanced by 
using quantization and especially when it is coarse. 
Clearly, there will be a certain tradeoff between
the security level and the control performance since
coarse quantization can in general degrade the system
performance. 

It is important to note that sampled-data control
can introduce vulnerability in the context of zero 
dynamics attacks \cite{ShiBacEun:22}. 
Even if the continuous-time plant
originally is minimum phase without any unstable zero, 
when discretized, 
the socalled sampling zeros will appear and
some of them can be unstable depending on the 
sampling rate \cite{astrom}. Such unstable zeros
can be exploited by the attacker; 
the plant dynamics in the continuous-time domain 
can be excited in such a way that the sampled output
shows no sign of irregular trajectories.

Quantization in networked control has been studied 
in the past two decades from the perspective of reducing
data transmissions for feedback control purposes;
see, e.g., \cite{EliMit:01,IshFra_book:02,Liberzon:03,NaiFagZamEva:07,OkaIsh:17,YouXie:10}.
%A well-known line of studies is related to the 
%level of quantization necessary to stabilize unstable
%systems based on the minimum data rate 
%(e.g., \cite{}) and the coarsest quantizers (e.g., \cite{}).
For systems under DoS attacks, the effects of quantization
have been addressed in \cite{katoDoS,wakaikiDoS,fengDoS}. 
The general implication found there is that using finer 
quantization, which
requires higher data rate for the communication of
control signals, would improve the robustness of the 
system against DoS attacks and vice versa.

To the best of our knowledge, however, 
quantization and their influence on FDI attacks have
not been studied in the literature.
Here, we will provide an analysis on the error in 
the system output caused by quantizing zero dynamics 
attacks as well as their capability to destabilize the
system. Our problem setting is limited to systems under
feedforward control, but 
it will be shown that by taking account of quantization 
effects, attack signals can be generated
resulting in a lower error level in comparison to 
the simple approach of directly quantizing the 
conventional zero dynamics attacks. 
To this end, we propose two quantized attack methods, one
based on dynamic quantizers \cite{dquant,sythdq} and the other
using a modified version of zero dynamics attacks. 
These methods are constructive in that the attacker
can specify the size of tolerable errors in the system
output and generate attack signals accordingly. 


\if0
For networked control systems, 
the control method which considers a quantization 
of signals caused by actuator capabilities and 
limitations of network communication. 
However, the attack methods previously mentioned 
assume the continuous signals. 
Therefore, it is necessary to study on an attack 
method which takes signal quantization effect into account.

In this paper, we consider a sampled-data system 
whose input signals are quantized. If a system 
input is continuous, the zero dynamics attack 
reduces effect on a system output by adjusting 
its initial value. If not, quantization effect 
appears with respect to its width and a system 
state and output are affected. Consequently, 
the attack performance such as the difficulty 
of detection is seemed to be spoiled.

Therefore, we propose the construction methods 
of the attack and analyze its effect, which reduce 
quantization effect on a system output and deviate 
a system state while a system input is quantized. 
In particular, for the cases where a sampled-data 
system has unstable zeros, we propose the two types 
of methods as follows.
\begin{itemize}
    \item the method utilizes a continuous attack signal.
    \item the method directly calculates a quantized attack signal.
\end{itemize}
\fi

This paper is organized as follows: 
Section~\ref{ch:problem} describes the networked control 
system, the input quantizer and the class of the attacks. 
In Section~\ref{ch:dq}, we briefly overview 
the approach of dynamic quantizers, which will be used for
attack signal generation in one of the proposed methods.
In Section~\ref{ch:nonmin}, we explain the two methods 
of quantized attacks for a non-minimum phase 
sampled-data systems and analyze their effects. 
In Section~\ref{ch:example}, we illustrate the effectiveness
of our results via a numerical example.  
In Section~\ref{ch:conclusion}, we provide 
concluding remarks. 
The material of this paper appeared as \cite{KimIsh:22}
in a preliminary form; the current version contains the proofs
of the main results and further discussions. 

\textbf{Notation}: We denote by $\mathbb{R}$, $\mathbb{N}$, 
and $\mathbb{Z}$ the sets of real numbers, natural numbers, and integers, respectively. %Moreover, $\mathbb{R}^n$, $\mathbb{R}^{n\times m}$ are $n$ dimensional real vectors and $n$ by $m$ real matrices respectively.
For $d>0$, $d\mathbb{Z}$ is the set of numbers which can be expressed as $dz$ by using an integer $z\in\mathbb{Z}$. 
The space of bounded sequences is denoted by $l_\infty$. 
%Non-negative integers are denoted as $\mathbb{Z}_+$. 
%For a matrix $A:=[A_{ij}]$, $\abs(A):=[|A_{ij}|]$. The $\infty$-norm of the vector $v$ is written as $\left\| v\right\|_\infty=\max_i|v_i|$. Similarly, $\left\| A\right\|_\infty$ is the $\infty$-norm of the matrix $A$, which is induced by vector's $\infty$-norm. 

%------------------------
%\section{The Semi-Oblivious Chase Procedure}\label{sec:semi}
%

The semi-oblivious chase (or simply chase) takes as input a database $D$ and a set $\dep$ of TGDs, and constructs an instance that contains $D$ and satisfies $\dep$.
%
A central notion in this context is that of trigger.
%are those of trigger, active trigger, and trigger application.

\begin{definition}%[\textbf{Trigger Application}]
	Given a set $\dep$ of TGDs and an instance $I$, a {\em trigger} for $\dep$  on $I$ is a pair $(\sigma,h)$, where $\sigma \in \dep$ and $h$ is a homomorphism from $\body{\sigma}$ to $I$.
	%
	The {\em result} of $(\sigma,h)$, denoted $\result{\sigma}{h}$, is the set $\mu(\head{\sigma})$, where $\mu : \var{\head{\sigma}} \ra \ins{C} \cup \ins{N}$ is defined as follows:
	%
	%$\mu(x) = h(x)$ if $x \in \fr{\sigma}$, and $\mu(x) = \bot_{\sigma,h_{|\fr{\sigma}}}^{x}$ otherwise,
	\[
	\mu(x)\
	=\ \left\{
	\begin{array}{ll}
	h(x) & \quad \text{if } x \in \fr{\sigma}\\
	&\\
	\bot_{\sigma,h_{|\fr{\sigma}}}^{x} & \quad \text{otherwise}
	\end{array} \right.
	\]
	where $\bot_{\sigma,h_{|\fr{\sigma}}}^{x} \in \ins{N}$.  Let $T(\dep,I)$ be the set of triggers for $\dep$ on $I$.	\hfill\markfull
\end{definition}




Observe that in the definition of $\result{\sigma}{h}$, each existentially quantified variable $x$ of $\head{\sigma}$ is mapped by $\mu$ to a null value of $\ins{N}$ whose name is uniquely determined by the trigger $(\sigma,h)$ and the variable $x$ itself. This means that, given a trigger $(\sigma,h)$, we can unambiguously construct the set of atoms $\result{\sigma}{h}$.
%
The central idea of the chase is, starting from a database $D$, to exhaustively apply triggers for the given set $\dep$ of TGDs on the instance constructed so far.
%
More precisely, given a database $D$ and a set $\dep$ of TGDs, let
\[
\mathsf{chase}^{0}(D,\dep)\ =\ D,
\]
and for each $i>0$, let
\[
\mathsf{chase}^{i}(D,\dep)\ =\ \mathsf{chase}^{i-1}(D,\dep)\ \cup\ \bigcup_{(\sigma,h) \in S} \result{\sigma}{h},
\]
where $S = T(\dep,\mathsf{chase}^{i-1}(D,\dep))$. 
%
We finally define {\em the result of the chase of $D$ w.r.t.~$\dep$} as the (possibly infinite) instance
\[
\chase{D}{\dep}\ =\ \bigcup_{i \geq 0} \mathsf{chase}^{i}(D,\dep).
\]


\ignore{
The semi-oblivious chase procedure (or simply chase) takes as input a database $D$ and a set $\dep$ of TGDs, and constructs an instance that contains $D$ and satisfies $\dep$.
%
Central notions in this context are those of trigger, active trigger, and trigger application.

\begin{definition}%[\textbf{Trigger Application}]
	Given a set $\dep$ of TGDs and an instance $I$, a {\em trigger} for $\dep$  on $I$ is a pair $(\sigma,h)$, where $\sigma \in \dep$ and $h$ is a homomorphism from $\body{\sigma}$ to $I$.
	%
	The {\em result} of $(\sigma,h)$, denoted $\result{\sigma}{h}$, is the set $\mu(\head{\sigma})$, where $\mu : \var{\head{\sigma}} \ra \ins{C} \cup \ins{N}$ is defined as follows:
	%
	%$\mu(x) = h(x)$ if $x \in \fr{\sigma}$, and $\mu(x) = \bot_{\sigma,h_{|\fr{\sigma}}}^{x}$ otherwise,
	\[
	\mu(x)\
	=\ \left\{
	\begin{array}{ll}
	h(x) & \quad \text{if } x \in \fr{\sigma}\\
	&\\
	\bot_{\sigma,h_{|\fr{\sigma}}}^{x} & \quad \text{otherwise}
	\end{array} \right.
	\]
	where $\bot_{\sigma,h_{|\fr{\sigma}}}^{x}$ is a null value from $\ins{N}$.
	%
	The trigger $(\sigma,h)$ is {\em active} if $\result{\sigma}{h} \not\subseteq I$.
	%
	The {\em application} of $(\sigma,h)$ to $I$ returns the instance $J = I \cup \result{\sigma}{h}$ and is denoted as $I \app{\sigma}{h} J$.
	\hfill\markfull
\end{definition}


Observe that in the definition of $\result{\sigma}{h}$ above, each existentially quantified variable $x$ of $\head{\sigma}$ is mapped by $\mu$ to a null value of $\ins{N}$ whose name is uniquely determined by the trigger $(\sigma,h)$ and the variable $x$ itself. This means that, given a trigger $(\sigma,h)$, we can unambiguously extract the set of atoms 
$\result{\sigma}{h}$.



%\medskip

%\noindent
%\textbf{Semi-Oblivious Chase.}
The central idea of the chase is, starting from a database $D$, to exhaustively apply active triggers for the given set $\dep$ of TGDs on the instance constructed so far. This is formalized via the notion of (semi-oblivious) chase derivation, which can be finite or infinite.


\begin{definition}
	Consider a database $D$ and a set $\dep$ of TGDs.
	%We consider the two cases where a derivation is finite or infinite:
	\begin{itemize}
		\item A finite sequence $(I_i)_{0 \leq i \leq n}$ of instances, with $D = I_0$ and $n \geq 0$, is a {\em chase derivation} of $D$ w.r.t.~$\dep$ if, for each $i \in \{0,\ldots,n-1\}$, there is an active trigger $(\sigma,h)$ for $\dep$ on $I_i$ with $I_i \app{\sigma}{h} I_{i+1}$, and there is no active trigger for $\dep$ on $I_n$. The {\em result} of such a chase derivation is the instance $I_n$.
		
		
		\item An infinite sequence $(I_i)_{i \geq 0}$ of instances, with $D = I_0$, is a {\em chase derivation} of $D$ w.r.t.~$\dep$ if, for each $i \geq 0$, there is an active trigger $(\sigma,h)$ for $\dep$ on $I_i$ such that $I_i \app{\sigma}{h} I_{i+1}$. Moreover, $(I_i)_{i \geq 0}$ is {\em fair} if, for each $i \geq 0$, and for every active trigger $(\sigma,h)$ for $\dep$ on $I_i$, there exists $j > i$ such that $(\sigma,h)$ is not an active trigger for $\dep$ on $I_j$. 
		%The latter is known as the {\em fairness condition}, and guarantees that all the active triggers will be deactivated. %
		The {\em result} of such a chase derivation is the instance $\bigcup_{i \geq 0} \, I_i$.
	\end{itemize}
	%
	%The {\em result} of a chase derivation is defined as the union of all the instances occurring in it. 
	A chase derivation is {\em valid} if it is finite or infinite and fair.  \hfill\markfull
\end{definition}


Let us stress that infinite but unfair chase derivations are not considered as valid ones since they do not serve the main purpose of the chase, that is, to build an instance that satisfies the given set of TGDs. Indeed, given the set $\dep$ consisting of the TGDs
\[
\sigma\ =\ R(x,y) \ra \exists z \, R(y,z) \qquad \sigma'\ =\ R(x,y) \ra P(x,y),
\]
the result of the unfair chase derivation of $D = \{R(a,b)\}$ w.r.t.~$\dep$ that involves only triggers of the form $(\sigma,\cdot)$, i.e., only the TGD $\sigma$ is used, does not satisfy $\sigma'$, and thus, it does not satisfy $\dep$.
%
Interestingly, for every database $D$ and set $\dep$ of TGDs, any two valid chase derivations of $D$ w.r.t.~$\dep$ have always the same result, which implies that all valid chase derivations are either finite or infinite~\cite{GrOn18}. Therefore, in the rest of the paper, we can safely refer to {\em the} result of the chase of $D$ w.r.t. $\dep$, which we will denote by $\chase{D}{\dep}$. 
}


%\subsection{Non-Uniform Chase Termination}\label{sec:problem}
%

\medskip

\noindent
\textbf{Chase Termination.}
The result of the chase may be infinite even for very simple settings: it is easy to see that for $D = \{R(a,b)\}$ and $\dep = \{R(x,y) \ra \exists z \, R(y,z)\}$, $\chase{D}{\dep}$ is infinite.
%; in particular, $\chase{D}{\dep} = \{R(a,b),R(b,\bot_1),R(\bot_1,\bot_2),R(\bot_2,\bot_3),\ldots\}$, where $\bot_1,\bot_2,\ldots$ are null values.
%
This leads to the following problem, parameterized by a class $\class{C}$ of TGDs such as $\class{SL}$ (the class of simple-linear TGDs) and $\class{L}$ (the class of linear TGDs):


\medskip

\begin{center}
	\fbox{
		\begin{tabular}{ll}
			%{\small PROBLEM} : & %$\mathsf{ChaseTermination}(\class{C})$
			%\\
			{\small INPUT} : & A database $D$ and a set $\dep$ of TGDs from $\class{C}$.
			\\
			{\small QUESTION} : &  Is the instance $\chase{D}{\dep}$ finite?
	\end{tabular}}
\end{center}

\medskip

\noindent This problem has been recently studied in~\cite{CaGP22} for the classes of simple-linear and linear TGDs. Interestingly, for both classes, the finiteness of the result of the chase has been syntactically characterized by exploiting the notion of non-uniform weak-acyclicity. 
%
We proceed to recall this acyclicity notion, and then present the characterizations established in~\cite{CaGP22}, which in turn lead to simple algorithms for checking the finiteness of the result of the chase.
%
Note that, for the sake of clarity, in the rest of the paper we assume TGDs with a non-empty frontier, i.e., we assume that there is at least one variable in a TGD $\sigma$ that occurs both in $\body{\sigma}$ and $\head{\sigma}$. This assumption can be made without loss of generality since, given a database $D$ and a set $\dep$ of TGDs, we can easily construct a set $\dep'$ of TGDs with a non-empty frontier by slightly modifying $\dep$ such that $\chase{D}{\dep}$ is finite iff $\chase{D}{\dep'}$ is finite.


\medskip

\noindent
\textbf{Non-Uniform Weak-Acyclicity.} Weak-acyclicity was introduced in~\cite{FKMP05} as the main formalism for data exchange purposes, which guarantees the finiteness of the result of the chase for {\em every} input database. Non-uniform weak-acyclicity is the database-dependent variant of weak-acyclicity introduced in~\cite{CaGP22}. We proceed to give the formal definitions.
%
We first need to recall the notion of the {\em dependency graph} of a set $\dep$ of TGDs, 
%which symbolically encodes how terms may propagate during the chase.
%The {\em dependency graph} of set $\dep$ of TGDs 
defined as a directed multigraph $\depg{\dep}=(N,E)$, where $N = \pos{\sch{\dep}}$ and $E$ contains {\em only} the following edges.
%
For each TGD $\sigma \in \dep$ with $\head{\sigma} = \{\alpha_1,\ldots,\alpha_k\}$, for each $x \in \frontier{\sigma}$, and for each position $\pi \in \posvar{\body{\sigma}}{x}$:
\begin{itemize}
	\item For each $i \in [k]$ and for each $\pi' \in \posvar{\alpha_i}{x}$, there exists a \emph{normal} edge $(\pi,\pi') \in E$.
	%
	\item For each existentially quantified variable $z$ in $\sigma$, $i \in [k]$, and $\pi' \in \posvar{\alpha_i}{z}$, there is a \emph{special} edge $(\pi,\pi') \in E$.
\end{itemize}
%
We further need to define when a predicate is reachable from another predicate. 
%
Given predicates $R,P \in \sch{\dep}$, {\em $P$ is reachable from $R$ (w.r.t.~$\dep$)} if $R = P$, or there exists a path in $\depg{\dep}$ from a position of the form $(R,i)$ to a position of the form $(P,j)$.
%
%we write $R \ra_\dep P$  if $R = P$, or there exists a TGD $\sigma \in \dep$ such that $R$ occurs in $\body{\sigma}$ and $P$ occurs in $\head{\sigma}$. We say that {\em $P$ is reachable from $R$ (w.r.t.~$\dep$)}, denoted $R \reach{\dep} P$, if (i) $R \ra_\dep P$, or (ii) there exists $T \in \sch{\dep}$ such that $R \reach{\dep} T$ and $T \ra_\dep P$.
%in $\depg{\dep}$, denoted $R \reach{\dep} P$, if there exists a path in $\depg{\dep}$ from a position $(R,i)$ to a position $(P,j)$, for some $i \in [\arity{R}]$ and $j \in [\arity{P}]$.
Given a database $D$, we say that a (not necessarily simple and possibly cyclic) path $C$ in $\depg{\dep}$ is \emph{$D$-supported} if there exists an atom $R(\bar t) \in D$ and a node of the form $(P,i)$ in $C$ such that $P$ is reachable from $R$.
%
We are now ready to recall (non-uniform) weak-acyclicity.



\begin{definition}\label{def:dwa}
	Consider a database $D$ and a set $\dep$ of TGDs. We say that $\dep$ is {\em weakly-acyclic w.r.t.~$D$}, or {\em $D$-weakly-acyclic}, if there is no $D$-supported cycle in $\depg{\dep}$ with a special edge. 
	%
	We say that $\dep$ is {\em weakly-acyclic} if there is no cycle in $\depg{\dep}$ with a special edge. \hfill\markfull
\end{definition}


\smallskip

\noindent
\textbf{Characterizing the Finiteness of the Chase.}
It is not very difficult to show that whenever a set $\dep$ of TGDs (not necessarily linear) is $D$-weakly-acyclic, then the instance $\chase{D}{\dep}$ is finite. In other words, the $D$-weak-acyclicity of $\dep$ is a sufficient condition for the finiteness of $\chase{D}{\dep}$. What is more interesting is that, assuming that $\dep$ is a set of simple-linear TGDs, the $D$-weak-acyclicity of $\dep$ is also a necessary condition for the finiteness of $\chase{D}{\dep}$. This leads to the following characterization established in~\cite{CaGP22}:

\begin{theorem}\label{the:characterization-simple-linear}
	Consider a database $D$ and a set $\dep \in \class{SL}$ of TGDs. It holds that $\chase{D}{\dep}$ is finite iff $\dep$ is $D$-weakly-acyclic.
\end{theorem}

For linear TGDs, it turned out that non-uniform weak-acyclicity is not powerful enough for characterizing the finiteness of the chase instance. Here is an example given in~\cite{CaGP22} that illustrates this fact:
%This is illustrated by the following example.


\begin{example}
	Consider the database $D = \{R(a,b)\}$ and the singleton set $\dep$ consisting of the (non-simple) linear TGD
	\[
	R(x,x)\ \ra\ \exists z \, R(z,x). 
	\]
	It is easy to see that there is no trigger for $\dep$ on $D$. This means that $\chase{D}{\dep} = D$ is finite, whereas $\dep$ is {\em not} $D$-weakly-acyclic. \hfill\markfull
\end{example}


To obtain a characterization analogous to Theorem~\ref{the:characterization-simple-linear}, the authors of~\cite{CaGP22} used the technique of {\em simplification} to convert linear TGDs into simple-linear TGDs, while preserving the finiteness of the chase instance. We proceed to recall this technique.
%
Let $\bar t = (t_1,\ldots,t_n)$ be a tuple of (not necessarily distinct) terms. We write $\unique{\bar t}$ for the tuple obtained from $\bar t$ by keeping only the first occurrence of each term in $\bar t$.
%
For example, if $\bar t = (x,y,x,z,y)$, then $\unique{\bar t} = (x,y,z)$.
%
For each $i \in [n]$, the \emph{identifier of $t_i$ in $\bar t$}, denoted $\id{\bar t}{t_i}$, is the integer that identifies the position of $\unique{\bar t}$ at which $t_i$ appears. 
%
We write $\id{}{\bar t}$ for the tuple $(\id{\bar t}{t_1},\ldots,\id{\bar t}{t_n})$.
%
For example, if $\bar t = (x,y,x,z,y)$, then $\id{}{\bar t} = (1,2,1,3,2)$.
%
For an atom $\alpha = R(\bar t)$, the {\em simplification of $\alpha$}, denoted $\simple{\alpha}$, is the atom $R_{\id{}{\bar t}}(\unique{\bar t})$, whereas the {\em shape of $\alpha$}, denoted $\shape{\alpha}$, is the predicate $R_{\id{}{\bar t}}$. We can naturally refer to the simplification and the shape of a set of atoms.
%
For a tuple of variables $\bar x = (x_1,\ldots,x_n)$, a \emph{specialization of $\bar x$} is a function $f$ from $\bar x$ to $\bar x$ such that $f(x_1) = x_1$, and $f(x_i) \in \{f(x_1),\ldots,f(x_{i-1}),x_i\}$, for each $i \in \{2,\ldots,n\}$.
We write $f(\bar x)$ for $(f(x_1),\ldots,f(x_n))$. We are now ready to recall how a set of linear TGDs is converted into a set of simple-linear TGDs.

\begin{definition}\label{def:simplification}
	Consider a linear TGD $\sigma$ of the form
	\[
	R(\bar x) \ra \exists \bar z\, \psi(\bar y,\bar z), 
	\]
	where $\bar y \subseteq \bar x$, and a specialization $f$ of $\bar x$. The {\em simplification of $\sigma$ induced by $f$} is the simple-linear TGD
	\[
	\simple{R(f(\bar x))} \rightarrow \exists \bar z\, \simple{\psi(f(\bar y),\bar z)}.
	\]
	We write $\simple{\sigma}$ for the set of all simplifications of $\sigma$ induced by some specialization of $\bar x$.
	%
	For a set $\dep \in \class{L}$ of TGDs, the {\em simplification of $\dep$} is defined as the set
	\[
	\simple{\dep}\ =\ \bigcup_{\sigma \in \dep} \simple{\sigma}
	\]
	consisting only of simple-linear TGDs. \hfill\markfull
\end{definition}

We can now recall the characterization for the finiteness of the chase instance for linear TGDs, established in~\cite{CaGP22}, which is similar to the one for simple-linear TGDs, with the key difference that first we need to simplify both the database and the set of linear TGDs:

\begin{theorem}\label{the:characterization-linear}
	Consider a database $D$ and a set $\dep \in \class{L}$ of TGDs. Then, $\chase{D}{\dep}$ is finite iff $\simple{\dep}$ is $\simple{D}$-weakly-acyclic.
\end{theorem}

It is clear that Theorems~\ref{the:characterization-simple-linear} and~\ref{the:characterization-linear} provide simple algorithms for checking whether the chase instance is finite. In particular, given a database $D$ and a set $\dep$ of simple-linear TGDs, we simply need to check whether $\dep$ is $D$-weakly-acyclic, in which case the algorithm returns \true; otherwise, it returns \false. The same holds when $\dep$ is a set of linear TGDs, with the difference that the algorithm first needs to simplify $D$ and $\dep$, and then perform the acyclicity check.
%
Our goal is to experimentally evaluate the above algorithms with the aim of understanding which input parameters affect their performance, clarifying whether they can be applied in a practical context, and revealing their performance limitations. Of course, a naive implementation of the above algorithms, especially for linear TGDs where the expensive simplification must be applied, will lead to poor performance, and thus, will not be very useful towards our goal. Hence, we need to somehow convert the above theoretical algorithms into practical algorithms that are amenable to efficient implementations. This is the subject of the next section.
\section{Problem formulation}\label{ch:problem}
In this section, we formulate the problem of quantized FDI attacks studied in this paper.
\subsection{Sampled-data system under quantized input}\label{se:plant}
Consider the feedforward sampled-data control system shown in Fig.~\ref{fig:pr:sys}, where the controller is connected with the plant via a network. The control input is generated by the controller and is quantized before it is sent to the plant over the network. A malicious attacker has access to the network and may modify the input signal though the attack signal must use the same quantization scheme as the controller. 

\begin{figure}[t]
 \vspace*{1mm}
 \centering
  \includegraphics[width=0.9\linewidth]{workspace/picture/system1.png}
% \vspace*{-1mm}
  \caption{Networked sampled-data control system under attack}
  \label{fig:pr:sys}
\end{figure}

In Fig.~\ref{fig:pr:sys},
the plant is a single-input single-output linear time-invariant (LTI) system whose state-space equation is given by
\begin{equation}
    \begin{split}
        \dot{x}(t)&=A_cx(t)+B_c(v(t)+b(t)),\\
        y(t)&=Cx(t),
    \end{split}\label{eq:pr:csys}
\end{equation}
where $x(t)\in\mathbb{R}^n$ is the state, $y(t)\in\mathbb{R}$ is the output, $A_c\in\mathbb{R}^{n\times n}$, $B_c\in\mathbb{R}^{n\times 1}$, and $C\in\mathbb{R}^{1\times n}$ are the system matrices. Moreover, the inputs $v(t)$ and $b(t)\in d\mathbb{Z}$ are, respectively, the quantized signals of the feedforward input $u(t)$ and the attack signal $a(t)$. The specific quantizer used will be introduced later. We assume that the system is stable and controllable, and 
the initial state is $x(0)=x_0$.

The input $v(t)$ is injected through  the zero-order hold (ZOH) whose sampling period is $T>0$. Denote by $v[k]$ the input calculated by the controller at time $t=kT$. The continuous-time signal applied to the plant is $v(t):=v [k]$ for $t\in[kT,(k+1)T)$. On the other hand, the output $y(t)$ of the plant is sampled at the same rate as the ZOH with sampling period $T$. So the measured output is $y [k] :=y(kT)$. We adopt the notation to write the system state and output of \eqref{eq:pr:csys} driven by input $u(t)$
as $x_u(t)$ and $y_u(t)$, respectively.

As the quantizer, we employ the uniform quantizer $q:\mathbb{R}\rightarrow d\mathbb{Z}$, 
whose width is denoted by $d>0$. 
We use the nearest neighbor quantization towards $-\infty$; 
it maps $\mu\in\mathbb{R}$ to $q(\mu)$ which is the optimal solution obtained as
\begin{equation}
  q(\mu) := \min_{w\in d\mathbb{Z}}\left|w-\mu\right|. 
\label{eqn:qmu}
\end{equation}
In Fig.~\ref{fig:pr:sys}, 
the most simple approach for obtaining the quantized input $v[k]$ is
to directly quantize the input $u[k]$ as 
\begin{equation}   
  v[k] = q(u[k]).
\label{eqn:vk}
\end{equation}
This approach will be referred to as static quantization.


\subsection{Quantized FDI attacks}

As mentioned above, the adversary is capable to modify the quantized control input $v[k]$ to $v[k]+b[k]$ by injecting the quantized attack signal $b[k]$. It is assumed that the adversary has the full information of the plant and the controller including their dynamics whereas the system operator has no information regarding the attacks. The adversary's objective is to disrupt the operation of the system without being detected by the system operator who may be monitoring the sampled system output $y[k]$. 

More specifically, the adversary attempts to generate the quantized 
attack signal $b[k]$ satisfying the following two conditions:\par
(i)~\textit{$\mathbf{\epsilon}$-stealthy condition}: 
The attack signal $\{b[k]\}_{k=0}^{\infty}$ is said to be 
$\epsilon$-stealthy for given $\epsilon >0$ if 
\begin{equation}
    \left|y_{v+b}[k]-y_v[k]\right|\leq\epsilon
\label{eq:pr:estealth}
\end{equation}
for all $k$, where $y_v[k]$ is the original output of the system under the input 
$v[k]$, and $y_{v+b}[k]$ is the output of the attacked system.\par
(ii)~\textit{$\{H_k\}_{k=0}^{\infty}$-disruptive condition}: Given a sequence of nonnegative numbers $\{H_k\}_{k=0}^{\infty}$, the attack signal $\{b[k]\}_{k=0}^{\infty}$ is said to be  $\{H_k\}_{k=0}^{\infty}$-disruptive if there exists a time 
sequence $\{t_k\}_{k=0}^{\infty}$ such that for any $k$, it holds
\begin{equation}
    \left\|x_{v+b}(t_k)-x_v(t_k)\right\|_2\geq H_k.
\label{eq:pr:disrupt}
\end{equation}

The problem of quantized zero dynamics attacks addressed in the paper can be stated as follows:

\textit{Problem}: Consider the quantized networked system in Fig.~\ref{fig:pr:sys}. 
Let the positive scalar $\epsilon$ and the sequence $\{H_k\}_{k=0}^{\infty}$ of nonnegative numbers be given. Then find the quantized attack signal $\{b[k] \}_{k=0}^{\infty}$ which satisfies both the $\epsilon$-stealthy and the $\{H_k\}_{k=0}^{\infty}$-disruptive conditions.

An important aspect of our problem setting is that we examine the case where 
the sampled-data system has unstable zeros in the discrete-time domain after discretization.
As we will see, this may hold even if the original continuous-time plant  
does not have unstable zeros \cite{ShiBacEun:22}.

If the attacker can inject real-valued signals (i.e., without quantization), attack methods satisfying the two conditions above have been proposed, which are known as the zero dynamics attacks  \cite{TeiShaSan:15,ParShiLee:19}. This type of attacks takes advantage of system zeros. Notably, if the system has unstable zeros, states may diverge and it is difficult to detect such attacks from the output since it will remain almost the same with or without the attack.\par
However, if we place an input quantizer, the system state/output under attacks should be influenced. Consequently, the stealthy and disruptive properties may be lost to some extent. In fact, if the attack signals are statically quantized, then the output may receive direct and large influences as we will see later. Hence, quantization may be used as a means to detect attacks on the system. Note that however that quantizing the input would reduce the control performance at the same time. So there is a tradeoff between the attack detection and control performance. The objective of this paper is to demonstrate that different methods for quantization of attack signals can result in different properties and in particular there are ways to reduce the effects caused by quantizers.

\subsection{Zero dynamics attacks}\label{se:zda}

The sampled-data system \eqref{eq:pr:csys} can be expressed as the following discrete-time linear system under ZOH and sampling:
\begin{equation}
    \begin{split}
        x[k+1]&=Ax[k]+B(v[k]+b[k]),\\
        y[k]&=Cx[k],
    \end{split}\label{eq:pr:dsys}
\end{equation}
where $x[k]:=x(kT)$ is the sampled state, $A:=\text{e}^{A_cT}$, and $B:=\int_0^T\text{e}^{A_c\tau}d\tau B_c$. We assume that this system is controllable and the vector product $CB$ is nonzero, which means the relative degree of this discretized system is $1$; these properties hold for almost any $T>0$ \cite{astrom}. \par
Note that when a continuous-time system is discretized, zeros which do not exist in the original system may appear. These are called the sampling zeros. As mentioned above, regardless of the relative degree of the original system \eqref{eq:pr:csys}, the discretized system \eqref{eq:pr:dsys} has relative degree of $1$ for almost any sampling period $T$. It is known that when $T$ is small, some of them will be unstable \cite{astrom}. Hence, even if the original system has no unstable zero, there is a chance that discretization will introduce some. In such a case, if the input is not quantized, then the zero dynamics attacks proposed in \cite{TeiShaSan:15,ParShiLee:19} can be applied. We note that such attacks can have arbitrarily small effects on the output by taking the initial state accordingly.

In view of the above, we impose the following assumption throughout the paper
unless otherwise noted:

\begin{assump}\label{assump:1}
The discretized plant \eqref{eq:pr:dsys} has at least one nonminimum phase zero 
(i.e., unstable zero).
\end{assump}

We summarize how to construct zero dynamics attack signals under this assumption as follows: 
%In this part, we make the following assumption: The system \eqref{eq:pr:dsys} has nonminimum phase zeros. 
Here, we consider the case where the control input and the attack signal are not quantized, i.e., $v[k]=u[k]$ and $b[k]=a[k]$. According to \cite{TeiShaSan:15}, we can construct the attack signal $\{a[k]\}_{k=0}^{\infty}$ by
\begin{equation}
    \begin{split}
        z[k+1]&=\left(A-\frac{BCA}{CB}\right)z[k],\\
        a[k]&=-\frac{CA}{CB}z[k],
    \end{split}\label{eq:pr:zda}
\end{equation}
where $z[k]\in\mathbb{R}^n$ is the state whose initial value satisfies 
$z[0]\neq 0$ and $z[0]\in\ker(C)$ with small 
$\left\|z[0]\right\|_2$.
Note that the nonminimum phase property of \eqref{eq:pr:dsys} implies that the system matrix $A-{BCA}/{(CB)}$ is unstable. The core idea of zero dynamics attacks is that this attack signal $a[k]$ will make the system state diverge but keep it in the undetectable region. Regarding this type of attacks, the following result is fundamental \cite{TeiShaSan:15}.
\begin{lem}\label{lem:zda}
Consider the system in \eqref{eq:pr:dsys}. For any $\epsilon>0$, there exist attack signals such that they are $\epsilon$-stealthy and $\left\|x_{u+a}[k]-x_u[k]\right\|_2$ diverges as $k\rightarrow\infty$.
\end{lem}

%------------------------
%\section{Dynamic quantization}\label{ch:dq}
In this section, we briefly introduce the method called dynamic quantization from the works \cite{dquant,dqseri,sythdq}. 

We consider the two feedforward discrete-time systems shown in Figs.~\ref{fig:dq:osys} and \ref{fig:dq:qsys}, 
where the controllers apply the same input $u[k]$ 
to both systems. The plant is from \eqref{eq:pr:dsys} 
without the attack signal, i.e., $b[k]\equiv0$. 
The difference between the two systems is that the 
first system in Fig.~\ref{fig:dq:osys} is the original 
one with the ideal output denoted by $y_I$ whereas 
in the second system in Fig.~\ref{fig:dq:qsys}, 
the input is quantized, resulting in the output 
$y$ with some errors from $y_I$. 
The problem is to find a quantizer with dynamics 
to minimize the error $y-y_I$ based on the past 
inputs $u[k]$ and their quantized values $v[k]$.
In particular, the measure of output error is defined as
\begin{equation}
    E(Q) = \sup_{u,k}\left|y[k]-y_I[k]\right|.
\label{eq:dq:error}
\end{equation}\par


\begin{figure}[t]
    \begin{minipage}[t]{1\linewidth}
\vspace*{1mm}
        \centering        
\includegraphics[width=0.7\linewidth]{workspace/picture/dq_osys.png}
\vspace*{1mm}
        \caption{Controller and plant}
        \label{fig:dq:osys}
    \end{minipage}
    \vspace{0.04\linewidth}
    \\
    \begin{minipage}[t]{1\linewidth}
        \centering
     \includegraphics[width=1\linewidth]{workspace/picture/dq_qsys.png}
\vspace*{1mm}
        \caption{Controller and plant with an input quantizer}
        \label{fig:dq:qsys}
    \end{minipage}
\end{figure}

Before we proceed to dynamic quantizers, we can
look at the case of static quantization 
when the input $u[k]$ 
is directly quantized as $v[k]=q(u[k])$.
%i.e., with $G=O$ and $H=I$ in \eqref{eq:dq:dquant}, 
In this case, it is known that 
the output error \eqref{eq:dq:error} can be obtained as 
\begin{equation}
 E(Q)
  = \frac{1}{2}\sum\limits_{k=0}^\infty\left|CA^kB\right| d.
\label{eq:dq:staticeps}
\end{equation}
Since the plant is assumed to be stable, this value is
finite. This formula suggests that for systems with
slower convergence, quantization has larger impact
on the output, and moreover this error is linear with
respect to the quantization width $d$.




\subsection{Basic structure of dynamic quantizers}\label{se:dq}
Here, we assume that the plant is minimum phase, i.e., its zeros are stable.\par
The dynamic quantizer is given by a linear system whose output is quantized by the uniform quantizer in Section \ref{se:plant}:
\begin{equation}
    Q\left\{
    \begin{aligned}
    \xi[k+1]&=E\xi[k]+F_1v[k]+F_2u[k],\\
    v[k]&=q(G\xi[k]+Hu[k]),
    \end{aligned}
    \right.\label{eq:dq:dquant}
\end{equation}
where the initial state is given by $\xi[0]=0$. 
We would like to design the dynamic quantizer 
so as to minimize the output error in \eqref{eq:dq:error}.

The dynamic quantizer is known to be optimal if
\begin{equation}
  \begin{split}
     E&=A,~~F_1=-F_2=B,\\
     G&=-\frac{CA}{CB},~~H=I
    \end{split}
\label{eq:dq:param}
\end{equation}
and then the minimum output error is given by \cite{dquant}
\begin{equation}
    E(Q)=\frac{1}{2}\left|CB\right|d.
\label{eq:dq:erroropt}
\end{equation}
It is obvious that with dynamic quantization,
the output error becomes strictly smaller than
that for the static case in \eqref{eq:dq:staticeps}.
In particular, $E(Q)$ in \eqref{eq:dq:erroropt} is
independent of the system matrix $A$. So different from
the static case, the output error does not depend on
the convergence rate of the plant.

However, we must note that this design method can be 
applied only to minimum phase systems. 
In fact, when the system \eqref{eq:pr:dsys} has 
unstable zeros, the dynamic quantizer outlined 
above becomes unstable. The case for nonminimum 
phase systems is discussed next. 

\subsection{Dynamic quantizer based on serial decomposition}\label{se:dqseri}
We describe the dynamic quantization approach of 
\cite{dqseri} based on decomposition of the plant. 
Suppose that the system \eqref{eq:pr:dsys} has unstable zeros. We denote by $P$ the system from the quantized input $v[k]$ to the output $y[k]$ in \eqref{eq:pr:dsys}. This is decomposed serially to subsystems $P_s$ and $P_u$ as $P=P_s\cdot P_u$ such that the following hold:
\begin{enumerate}
    \item $P_s$ is stable, minimum phase and strictly proper, and
    \item $C_sB_s\neq0$,
\end{enumerate}
where the realizations of $P_s$ and $P_u$ are, respectively, given by $(A_s,B_s,C_s,0)$ and $(A_u,B_u,C_u,D_u)$. Note that the minimum phase property of $P_s$ implies that $A_s-\frac{B_sC_sA_s}{C_sB_s}$ is a stable matrix\cite{dquantstable}.\par
Dynamic quantization based on serial decomposition generates the quantized signal $v[k]$ as follows:
\begin{equation}
  Q_s\left\{
    \begin{aligned}
    \xi[k+1]
      &= A_s\xi[k]+B_sv[k]-B_su[k], \\
    v[k]
      &= q\left(-\frac{C_sA_s}{C_sB_s}\xi[k]+u[k]\right).
    \end{aligned}
  \right.
\label{eq:dq:dqseri}
\end{equation}
Notice that this is the optimal quantizer for the system $P_s$
for \eqref{eq:dq:param}.
The output error in \eqref{eq:dq:error} for this case 
can be expressed as
\begin{equation}
E(Q_s)
 = \frac{1}{2}\left\|P_u\right\|_{i_\infty}
    \left|C_sB_s\right|d,
\label{eq:dq:dqserierroropt}
\end{equation}
where $\left\|P_u\right\|_{i\infty}$ is the induced-$l_\infty$ norm of the subsystem $P_u$ given by
\begin{equation}
    \left\|P_u\right\|_{i\infty}:=\sup\limits_{r\in l_\infty,r\neq0}\frac{\left\|P_ur\right\|_\infty}{\left\|r\right\|_\infty}.\label{eq:dq:linfnorm}
\end{equation}
This is the contribution of $P_u$ to the output error 
in \eqref{eq:dq:dqserierroropt}.

Note that in this method, there is some freedom in the choices of the subsystems $P_s$ and $P_u$. One difficulty is that if $P$ has multiple unstable zeros, then the optimal way of decomposition is not known in general. One exception is when the plant $P$ has only one unstable zero. In that case, the optimal decomposition and output error can be obtained explicitly. Let $\lambda$ be the unstable zero $(|\lambda|>1)$ of $P$. Then, $P_u$ should be taken in transfer function form as $P_u(z)=(z-\lambda)/z$. The optimal output error can be obtained as
\begin{equation}
    E(Q_s)=\frac{1}{2}(1+\left|\lambda\right|)\left|C_sB_s\right|d.\label{eq:dq:dqseriopt}
\end{equation}



\section{Dynamic quantization}\label{ch:dq}
In this section, we briefly introduce the method called dynamic quantization from the works \cite{dquant,dqseri,sythdq}.
This will later be applied for the quantization of attack signals.

We consider the two feedforward discrete-time systems shown in Figs.~\ref{fig:dq:osys} and \ref{fig:dq:qsys}, 
where the controllers apply the same input $u[k]$ 
to both systems. The plant is from \eqref{eq:pr:dsys} 
without the attack signal, i.e., $b[k]\equiv0$. 
The difference between the two systems is that the 
first system in Fig.~\ref{fig:dq:osys} is the original 
one with the ideal output denoted by $y_I[k]$ whereas 
in the second system in Fig.~\ref{fig:dq:qsys}, 
the input is quantized, resulting in the output 
$y[k]$ with some deviations from $y_I[k]$. 
The problem is to find a quantizer with dynamics 
to minimize the error $y[k]-y_I[k]$ based on the past 
inputs $u[k]$ and their quantized values $v[k]$.
In particular, the measure of output error is defined as
\begin{equation}
    E_q := \sup_{u,k}\left|y[k]-y_I[k]\right|.
\label{eq:dq:error}
\end{equation}\par


\begin{figure}[t]
    \begin{minipage}[t]{1\linewidth}
%\vspace*{1mm}
        \centering        
\includegraphics[width=0.7\linewidth]{workspace/picture/dq_osys.png}
\vspace*{1mm}
        \caption{Controller and plant}
        \label{fig:dq:osys}
    \end{minipage}
    \vspace{0.04\linewidth}\\
    \begin{minipage}[t]{1\linewidth}
        \centering
     \includegraphics[width=1\linewidth]{workspace/picture/dq_qsys.png}
\vspace*{-2mm}
        \caption{Controller and plant with an input quantizer}
        \label{fig:dq:qsys}
    \end{minipage}
\end{figure}

Before we proceed to dynamic quantizers, we can
look at the case of static quantization 
when the input $u[k]$ is directly quantized as $v[k]=q(u[k])$
as in \eqref{eqn:vk}.
%i.e., with $G=O$ and $H=I$ in \eqref{eq:dq:dquant}, 
In this case, it is known that 
the output error \eqref{eq:dq:error} can be obtained as 
\begin{equation}
 E_q
  = \frac{1}{2}\sum\limits_{k=0}^\infty\left|CA^kB\right| d.
\label{eq:dq:staticeps}
\end{equation}
Since the plant is assumed to be stable, this value is
finite. This formula suggests that for systems with
slower convergence, quantization has a larger impact
on the output, and moreover this error is linear with
respect to the quantization width $d$.




\subsection{Basic structure of dynamic quantizers}\label{se:dq}
In this subsection, we assume that the plant \eqref{eq:pr:dsys} is minimum phase (i.e., 
Assumption~\ref{assump:1} is not imposed).
The dynamic quantizer is given by a linear system whose output is quantized by 
the uniform quantizer \eqref{eqn:qmu}:
\begin{equation}
    \begin{aligned}
    \xi[k+1]&=E\xi[k]+F_1v[k]+F_2u[k],\\
    v[k]&=q(G\xi[k]+Hu[k]),
    \end{aligned}
    \label{eq:dq:dquant}
\end{equation}
where $\xi[k]\in\mathbb{R}^{n_q}$ is the state and its initial value is taken as $\xi[0]=0$;
the system matrices are of appropriate sizes. 
We would like to design the dynamic quantizer 
so as to minimize the output error in \eqref{eq:dq:error}.
Note that the static quantization in \eqref{eqn:vk} 
is a special case, corresponding to $G=O$ and $H=I$.

The dynamic quantizer is known to be optimal if
\begin{equation}
  \begin{split}
     E&=A,~~F_1=-F_2=B,~~
     G=-\frac{CA}{CB},~~H=I,
    \end{split}
\label{eq:dq:param}
\end{equation}
and then the minimum output error is given by \cite{dquant}
\begin{equation}
    E_q = \frac{1}{2}\left|CB\right|d.
\label{eq:dq:erroropt}
\end{equation}


We observe that with dynamic quantization,
the output error may become smaller than
that for the static case in \eqref{eq:dq:staticeps}.
It is noted that $E_q$ in \eqref{eq:dq:erroropt} is
independent of the system matrix $A$. So different from
the static case, the output error does not depend on
the convergence rate of the plant.

However, we must note that this design method can be 
applied only to minimum phase systems. 
In fact, when the system \eqref{eq:pr:dsys} has 
unstable zeros, the dynamic quantizer outlined 
above becomes unstable. The case for nonminimum 
phase systems is discussed next. 

\subsection{Dynamic quantizer based on serial decomposition}\label{se:dqseri}
We describe the dynamic quantization approach of 
\cite{dqseri} based on decomposition of the plant. 
Suppose that the system \eqref{eq:pr:dsys} has unstable zeros. We denote by $P$ the system from the quantized input $v[k]$ to the output $y[k]$ in \eqref{eq:pr:dsys}. This is decomposed serially to subsystems $P_s$ and $P_u$ as $P=P_s\cdot P_u$ such that the following hold:
\begin{enumerate}
    \item $P_s$ is stable, minimum phase and strictly proper, and
    \item $C_sB_s\neq0$,
\end{enumerate}
where the realizations of $P_s$ and $P_u$ are, respectively, given by $(A_s,B_s,C_s,0)$ and $(A_u,B_u,C_u,D_u)$. Note that the minimum phase property of $P_s$ implies that $A_s-{B_sC_sA_s}/{(C_sB_s)}$ is a stable matrix.

Dynamic quantization based on serial decomposition generates the quantized signal $v[k]$ as follows:
\begin{equation}
%  Q_s\left\{
    \begin{aligned}
    \xi[k+1]
      &= A_s\xi[k]+B_sv[k]-B_su[k], \\
    v[k]
      &= q\left(-\frac{C_sA_s}{C_sB_s}\xi[k]+u[k]\right),
    \end{aligned}
%  \right.
\label{eq:dq:dqseri}
\end{equation}
where $\xi[0]=0$.
Notice that this is the optimal quantizer for the system $P_s$
from \eqref{eq:dq:param} and \eqref{eq:dq:dquant}.
The output error in \eqref{eq:dq:error} for this case 
can be expressed as
\begin{equation}
E_q
 = \frac{1}{2}\left\|P_u\right\|_{i_\infty}
    \left|C_sB_s\right|d,
\label{eq:dq:dqserierroropt}
\end{equation}
where $\left\|P_u\right\|_{i\infty}$ is the induced-$l_\infty$ norm of the subsystem $P_u$ given by
%\begin{equation}
    $\left\|P_u\right\|_{i\infty}
      :=\sup_{r\in l_\infty,r\neq0}
%         \frac{\left\|P_ur\right\|_\infty}{\left\|r\right\|_\infty}.
          {\left\|P_ur\right\|_\infty}/{\left\|r\right\|_\infty}$.
%\label{eq:dq:linfnorm}
%\end{equation}
This is the contribution of $P_u$ to the output error 
in \eqref{eq:dq:dqserierroropt}.

Note that in this method, there is some freedom in the choices of the subsystems $P_s$ and $P_u$. One difficulty is that if $P$ has multiple unstable zeros, then the optimal way of decomposition is not known in general. One exception is when the plant $P$ has only one unstable zero. In that case, the optimal decomposition and output error can be obtained explicitly. Let $\lambda$ be the unstable zero $(|\lambda|>1)$ of $P$. Then, $P_u$ should be taken in the transfer function form as $P_u(z)=(z-\lambda)/z$. The optimal output error 
in \eqref{eq:dq:dqserierroropt} can be obtained as
\begin{align}
    E_q &= \frac{1}{2}(1+\left|\lambda\right|)\left|C_sB_s\right|d %\notag\\
        = \frac{1}{2}(1+\left|\lambda\right|)\left|C B\right|d.
\label{eq:dq:dqseriopt}
\end{align}
The second equality can be shown by the serial decomposition 
and by the choice of $P_u$, resulting in the direct feedthrough term 
$D_u=1$ in its state-space form.



%------------------------
%\section{Quantized attacks for non-minimum phase systems}\label{ch:nonmin}
In this section, we discuss quantized attacks for the case when the sampled-data system \eqref{eq:pr:csys} has unstable zeros. Concretely, we consider three types of methods as follows:

(i)~The first method is simple, where zero dynamics attacks as described in Section~II-C are generated and quantized directly using the static quantizer.\par
(ii)~The second approach employs the dynamic quantizer outlined above to modify the zero dynamics attacks in real values. We can show that this method can achieve smaller output error than statically quantizing the zero dynamics attacks as in the first method. However, it is affected by the number of unstable zeros and their values. \par
(iii)~The third method quantizes attack signals which are slightly different from the conventional zero dynamics attacks and may induce small but bounded influence on the system output. This signal can be constructed without the influence of unstable zeros. Moreover, as we will see in the analysis as well as numerical examples, this method is capable to outperform the other two methods in terms of the output error.\par
In this section, we look at these three attack methods and analyze the resulting output errors. 
\subsection{Quantized attack method 1 via static quantization}\label{se:sqzda}
We start with the simplest quantization approach where the attack signal $a[k]$ generated by \eqref{eq:pr:zda} is directly 
quantized in a static manner. That is, the attack
signal $b[k]$ to be injected to the quantized control input $v[k]$ 
in Fig.~1 is given by
\begin{equation}
    b[k] = q(a[k]).
    \label{eq:no:bqa}
\end{equation} 
For this method, we obtain the following result.
%Regarding the quantized attack $\{b[k]\}_{k=0}^{\infty}$ 
%which is calculated by \eqref{eq:pr:zda} and the static 
%quantizer, we obtain the following result.
\begin{prop}
For the quantized networked system under attacks in Fig.~\ref{fig:pr:sys}, suppose that the quantization width $d$ is given and the scalar $\epsilon$ satisfies
\begin{equation}
    \epsilon>\frac{1}{2}\sum\limits_{k=0}^\infty\left|CA^kB\right|d.\label{eq:no:staticeps}
\end{equation}
Moreover, suppose that the attack signal generated by \eqref{eq:pr:zda} is statically quantized as in \eqref{eq:no:bqa}. Then, for any sequence $\{H_k\}_{k=0}^{\infty}$, there exists a quantized attack signal $\{b[k]\}_{k=0}^{\infty}$ that is $\epsilon$-stealthy and $\{H_k\}_{k=0}^{\infty}$-disruptive.
\end{prop}
\begin{sketchofproof}
First, we can upper bound the output error as
\begin{multline*}
    \left|y_{v+b}[k]-y_v[k]\right|\\\leq\left|y_{v+b}[k]-y_{v+a}[k]\right|+\left|y_{v+a}[k]-y_v[k]\right|.
\end{multline*}
The first term on the right-hand side is less than or equal to $\frac{1}{2}\sum_{k=0}^\infty\left|CA^kB\right|d$ from \eqref{eq:dq:staticeps}. By Lemma.~1 and \eqref{eq:no:staticeps}, we can choose the attack signal $\{a[k]\}_{k=0}^{\infty}$ which is $\left(\epsilon-\frac{1}{2}\sum_{k=0}^\infty\left|CA^kB\right|d\right)$-stealthy. Then, the right-hand side of the above inequality is less than or equal to $\epsilon$. Therefore the attack $\{b[k]\}_{k=0}^{\infty}$ is $\epsilon$-stealthy.\par
Next, we look at the difference in states. We have
\begin{align*}
  &\left\| x_{v+b}[k]-x_v[k]\right\|_2\\
   &~~\geq \left\| x_{v+a}[k]-x_v[k]\right\|_2
       - \left\| x_{v+b}[k]-x_{v+a}[k]\right\|_2.
\end{align*}
On the right-hand side, the first term diverges as $k\rightarrow\infty$. The second term on the right-hand side coincides with the state which is driven by the quantization error; since the quantization error is bounded and the system \eqref{eq:pr:dsys} is stable, this term is also bounded. It means that $\left\| x_{v+b}[k]-x_v[k]\right\|_2$ diverges as $k\rightarrow\infty$. Therefore, we conclude that the attack $\{b[k]\}_{k=0}^{\infty}$ is $\{H_k\}_{k=0}^{\infty}$-disruptive.
\end{sketchofproof}
\subsection{Quantized attack method 2 via dynamic quantization}\label{se:dqzda}
In this subsection, we propose a method which modifies the continuous zero dynamics attacks in Section \ref{ch:problem} by using dynamic quantization from the previous section. This method adopts dynamic quantizer based on serial decomposition in Section \ref{se:dqseri}. \par
Specifically, we apply \eqref{eq:dq:dqseri} to generate attack signals after obtaining the serial decompositions $P_s$, $P_u$ of the system \eqref{eq:pr:dsys} satisfying conditions outlined there. The effects to output signals caused by this attack is determined by the choices of $P_s$, $P_u$. As we discussed earlier, while this method can be utilized for systems with any number of unstable zeros, the optimal value for the output error is known only for the case of one unstable zero. The following theorem is the first main result of this paper stating that dynamic quantizer can be efficient in reducing the error due to quantized attacks. 
\begin{theo}\label{theo:dqzda}
For the quantized networked system under attacks in Fig.~\ref{fig:pr:sys}, suppose that the plant is decomposed serially as described in Section \ref{se:dqseri} and $\epsilon$ satisfies
\begin{equation}
    \epsilon>\frac{1}{2}\left\|P_u\right\|_{i\infty}\left|C_sB_s\right|d,\label{eq:no:eps}
\end{equation}
where $B_s$ and $C_s$ are the system matrices of $P_s$. Then, for any sequence $\{H_k\}_{k=0}^\infty$, there exists a quantized attack signal $\{b[k]\}_{k=0}^\infty$ generated by \eqref{eq:pr:zda} and dynamic quantization that is $\epsilon$-stealthy and $\{H_k\}_{k=0}^\infty$-disruptive.
\end{theo}\par
\begin{proof}
By Lemma \ref{lem:zda}, we can construct the attack signal $\{a[k]\}_{k=0}^{\infty}$ such that it is $\left(\epsilon-\frac{1}{2}\left\|P_u\right\|_{i\infty}\left|C_sB_s\right|d\right)$-stealthy and $\left\|x_{u+a}[k]-x_u[k]\right\|_2$ diverges. Consider the attack $\{b[k]\}_{k=0}^{\infty}$ which is made from $\{a[k]\}_{k=0}^{\infty}$ and the dynamic quantizer in Section \ref{se:dqseri}. We verify that this attack satisfies both $\epsilon$-stealthy and $\{H_k\}_{k=0}^{\infty}$-disruptive properties.\par
We can upper bound the output error as
\begin{align}
  & |y_{v+b}[k]-y_v[k]|\\
  &~~~\leq|y_{v+b}[k]-y_{v+a}[k]|
          +| y_{v+a}[k]-y_v[k]|\label{eq:no:ineqy}
\end{align}
for every $k$. Since the system is linear, the first term on the right-hand side coincides with $|y_b[k]-y_a[k]|$. By the bound on the output error \eqref{eq:dq:dqserierroropt} for the dynamic quantizer, this term can be bounded as
\begin{equation}
    \sup\limits_k|y_b[k]-y_a[k]|\leq\frac{1}{2}\left\|P_u\right\|_{i\infty}|C_sB_s|d.\notag
\end{equation}
By the construction of $\{a[k]\}_{k=0}^{\infty}$, the second term on the right-hand side of \eqref{eq:no:ineqy} satisfies the following inequality:
\begin{equation}
    \sup\limits_k|y_{v+a}[k]-y_v[k]|\leq\epsilon-\frac{1}{2}\left\|P_u\right\|_{i\infty}|C_sB_s|d.\notag
\end{equation}
Hence, from \eqref{eq:no:ineqy}, we have $\sup_k| y_{v+b}[k]-y_v[k]|\leq\epsilon$. Consequently, the attack $\{b[k]\}_{k=0}^\infty$ is $\epsilon$-stealthy.\par
Next, we look at the difference in states. We have
\begin{multline}
    \left\| x_{v+b}[k]-x_v[k]\right\|_2\\\geq\left\| x_{v+a}[k]-x_v[k]\right\|_2-\left\| x_{v+b}[k]-x_{v+a}[k]\right\|_2.\label{eq:no:statediff}
\end{multline}
Because of the choice of $\{a[k]\}_{k=0}^{\infty}$, the first term on the right-hand side diverges as $k\rightarrow\infty$. Since the system is linear, it follows that $x_{v+b}[k]-x_{v+a}[k]=x_b[k]-x_a[k]$. The state-space equation of the plant and the dynamic quantizer can be written as
\begin{equation}
\begin{bmatrix}
x_{b}[k+1] \\
\xi[k+1]\\
x_{a}[k+1]
\end{bmatrix}\hskip-3pt=\hskip-3pt\left[\begin{array}{@{\hskip-1pt}c@{\hskip1pt}c@{\hskip1pt}c@{\hskip-1pt}}
A&O&O\\
O&A_s&O\\
O&O&A 
\end{array}\right]\hskip-5pt\begin{bmatrix}
x_{b}[k] \\
\xi[k]\\
x_{a}[k]
\end{bmatrix}\hskip-2pt+\hskip-2pt\begin{bmatrix}
Bb[k] \\
B_s(b[k]-a[k])\\
Ba[k]
\end{bmatrix}.\label{eq:no:xbxixa}
\end{equation}
Let $w_q[k]$ be
\begin{equation*}
        w_q[k]:=q\left(-\frac{C_sA_s}{C_sB_s}\xi[k]+a[k]\right)-\left\{-\frac{C_sA_s}{C_sB_s}\xi[k]+a[k]\right\}.
\end{equation*}
Then, the quantized attack is expressed as $b[k]=a[k]+w_q[k]-\frac{C_sA_s}{C_sB_s}\xi[k]$. Note that $\left\|w_q[k]\right\|_\infty\leq d/2$. Also, let the system matrix $\bar{A}$ be
\begin{equation*}
    \bar{A}:=\begin{bmatrix}
    A & -\frac{BC_sA_s}{C_sB_s} \\
    O & A_s-\frac{B_sC_sA_s}{C_sB_s}
    \end{bmatrix}.
\end{equation*}
\quad Then, the state-space equation in \eqref{eq:no:xbxixa} can be rewritten into 
\begin{equation}
\begin{bmatrix}
x_{b}[k+1] \\
\xi[k+1]\\
x_{a}[k+1]
\end{bmatrix}\hskip-3pt=\hskip-3pt\left[\begin{array}{@{\hskip-1pt}c|c@{\hskip-1pt}}
\raisebox{-0.5em}{\mbox{\LARGE $\bar{A}$}} & \begin{matrix}
O \\ O
\end{matrix}
\\
\hline
 \begin{matrix}
O & O
\end{matrix} & A
\end{array}\right]\hskip-5pt\begin{bmatrix}
x_{b}[k] \\
\xi[k]\\
x_{a}[k]
\end{bmatrix}+\begin{bmatrix}
B(a[k]+w_q[k]) \\
B_sw_q[k]\\
Ba[k]
\end{bmatrix}.\notag
\end{equation}
Applying $x_{b}[k]-x_a[k]=[I~O~-I][x_{b}[k]~\xi[k]\ x_{a}[k]]^T$ and $\xi[0]=0$ to it, we can obtain the relation
\begin{equation}
x_{b}[k]-x_a[k]=\sum\limits_{l=0}^{k-1}\begin{bmatrix}
I & O
\end{bmatrix}\bar{A}^{l}\begin{bmatrix}
B \\
B_s
\end{bmatrix}w_q[k-l].\notag
\end{equation}
Since the system \eqref{eq:pr:dsys} is stable and the subsystem $P_s$ is minimum phase, the matrix $\bar{A}$ is also stable. Thus, the sequence $x_{b}[k]-x_a[k]$ is bounded, implying that the second term on the right-hand side of \eqref{eq:no:statediff} is bounded. Consequently, $\left\| x_{v+b}[k]-x_v[k]\right\|_2\rightarrow\infty$ as $k\rightarrow\infty$. Therefore, we conclude that the quantized attack signal $\{b[k]\}_{k=0}^\infty$ is $\{H_k\}_{k=0}^{\infty}$-disruptive.
\end{proof}
\subsection{Quantized attack method 3 via $\epsilon$-stealthy approach}\label{se:calc}

The third method employs a continuous attack signal 
which is a slightly modified version of the zero 
dynamics attack in \eqref{eq:pr:zda} and then quantizes 
it directly. 

To this end, for given $\epsilon>0$, let us consider 
the attack signal $\{a[k]\}_{k=0}^{\infty}$ generated by
\begin{equation}
    \begin{split}
        z[k+1]&=\left(A-\frac{BCA}{CB}\right)z[k]+\frac{B}{CB}\epsilon,\\
        a[k]&=-\frac{CA}{CB}z[k]+\frac{\epsilon}{CB},
    \end{split}\label{eq:no:zdag}
\end{equation}
where $z[0]=0$. We can easily show that this system 
is unstable and the state difference difference $x_{u+a}[k]-x_u[k]$ is equal to $z[k]$. 
It means that $y_{u+a}[k]-y_u[k]=Cz[k]$ and this value 
coincides with $\epsilon$ for $k\geq1$. 
Hence, this attack is $\epsilon$-stealthy in the sense of \eqref{eq:pr:estealth} (without quantization).

The system \eqref{eq:no:zdag} motivates us to generate attack
signals using a system which has constant input proportional 
to $\epsilon$ and then to quantize the signal. 
This can be achieved by introducing the following quantized system:
%a quantized version which generates the quantized attack signal $\{b[k]\}_{k=0}^{\infty}$ as
\begin{equation}
 \begin{split}
   z[k+1]&=Az[k]+Bb[k],\\
        b[k]&=q\left(-\frac{CA}{CB}z[k]+\frac{\sign(CB)\epsilon}{CB}-\frac{d}{2}+\Delta\right),
    \end{split}\label{eq:no:dzdag}
\end{equation}
where $\Delta$ is a sufficiently small positive number. \par
Now, we are ready to state our second main result of the paper. 
\begin{theo}\label{theo:2}
For the quantized networked system under attacks in Fig.~\ref{fig:pr:sys}, suppose that $\epsilon$ is taken as $\epsilon \geq |CB|d$. Then, for any sequence $\{H_k\}_{k=0}^{\infty}$, the quantized attack $\{b[k]\}_{k=0}^{\infty}$ generated by \eqref{eq:no:dzdag} is $\left(\epsilon + |CB|\Delta\right)$-stealthy and $\{H_k\}_{k=0}^{\infty}$-disruptive. 
%if the following conditions are satisfied:
%\begin{itemize}
%    \item Consider the Jordan decomposition of $\left(A-\frac{BCA}{CB}\right)=P\Lambda P^{-1}$. Moreover, let the diagonal elements of $\Lambda$ be $\sigma_1,\ldots,\sigma_n$ and $\sigma_l,\ldots,\sigma_n$ be unstable ones. Then at least one of the $l-th$ through the last element of $B':=P^{-1}B$ is not $0$.
%\end{itemize}
\end{theo}
\begin{proof}
It is straightforward to show that by \eqref{eq:no:zdag}, the differences in states and output can be written as $x_{v+b}[k]-x_v[k]=z[k]$ and $y_{v+b}[k]-y_v[k]=Cz[k]$ respectively. Let  
\[
  w[k]:=(-\frac{CA}{CB}z[k]+\frac{\sign(CB)\epsilon}{CB}+\Delta)-b[k]. 
\]
Then, \eqref{eq:no:dzdag} becomes
\begin{multline}
    z[k+1]=\left(A-\frac{BCA}{CB}\right)z[k]\\+B\left(\frac{\sign(CB)\epsilon}{CB}+\Delta-w[k]\right),
\end{multline}
where $0<w[k]\leq d$. Since $z[0]=0$, we have 
\begin{align}
&z[k]\notag\\
&~~= \sum_{l=0}^{k-1}\left(A-\frac{BCA}{CB}\right)^{k-l-1}
   B\left(\frac{\sign(CB)\epsilon}{CB}+\Delta-w[l]\right).
\label{eq:no:zk}
\end{align}
Note that it holds $C\left(A-\frac{BCA}{CB}\right)=0$ and $Cz[k]=\sign(CB)\epsilon+CB\Delta-CBw[k-1]$ for all $k\geq1$. Since $\epsilon\geq|CB|d$, it holds 
\[
  -\left(\epsilon+|CB|\Delta\right)
    \leq|Cz[k]|\leq\epsilon+|CB|\Delta. 
\]
It means that the attack $\{b[k]\}_{k=0}^{\infty}$ is $\left(\epsilon+|CB|\Delta\right)$-stealthy.\par
Next, to establish that $\{b[k]\}_{k=0}^{\infty}$ is $\{H_k\}_{k=0}^{\infty}$-disruptive, we must show that $z[k]$ diverges, which implies that $\left\|x_{v+b}[k]-x_v[k]\right\|_2$ also diverges. Let 
\[
  w'[k]:=\frac{\sign(CB)\epsilon}{CB}+\Delta-w[k].
\] 
Now, based on \eqref{eq:no:zk}, we can rewrite the system of $z[k]$ and
obtain
\begin{equation*}
   z[k+1] = \left(A-\frac{BCA}{CB}\right) z[k] + B w'[k].
\end{equation*}
Since $z[0]=0$, we must make sure that in this system, the input $w'[k]$ enters and excites the unstable mode. \par 
Recall that $\left(A,B\right)$ is controllable, and thus $\left(A-\frac{BCA}{CB}, B\right)$ is also controllable. Hence, it follows that the input $w'[k]$ enters every mode of the system. Finally, we can confirm that $w'[k]$ is not zero at all times. By assumption, $\epsilon\geq|CB|d$, and this implies that $w'[k]\geq\Delta$ at each $k$. Therefore, the state $z[k]$ diverges.
%Next, we look at $z[k]$. The matrix $A-\frac{BCA}{CB}$ can be denoted as $P\Lambda P^{-1}$ by Jordan decomposition. Moreover, we can let the diagonal elements of $\Lambda$ be $\sigma_1,\ldots,\sigma_n$ and $\sigma_l,\ldots,\sigma_n$ be unstable ones.  Since $\epsilon\geq|CB|d$, the new variable $w'[k]:=\frac{\sign(CB)}{CB}\epsilon+\Delta-w[k]$ satisfies $w'[k]\geq\Delta$. Then, \eqref{eq:no:zk} becomes
%\begin{equation}
%    z[k]=\sum_{l=0}^{k-1}P\Lambda^{k-l-1}B'w'[l],\label{eq:no:zk2}
%\end{equation}
%where $B':=P^{-1}B$. Since $\left(A, B\right)$ is controllable, $\left(A-\frac{BCA}{CB}, B\right)$ is also controllable. Therefore, the vector $B'$ has nonzero elements and the last one has index greater than or equal to $l$. Here, we denote this index by $l^*$. Then the state $P^{-1}z[k]$ can be expressed as
%\begin{equation}
%    P^{-1}z[k]=\begin{bmatrix}
%    *\\
%    \vdots\\
%    *\\
%    \sum_{l=0}^{k-1}\sigma_{l^*}^{k-l-1}B_{l^*}'w'[l]\\
%    0\\
%    \vdots\\
%    0
%    \end{bmatrix}.\label{eq:no:zk3}
%\end{equation}
%The square norm of $P^{-1}z[k]$ can be bounded as
%\begin{equation*}
%        \left\|P^{-1}z[k]\right\|_2\leq\left\|P^{-1}\right\|_2\left\|z[k]\right\|_2.
%\end{equation*}
%Then, that of $z[k]$ can be bounded as
%\begin{equation*}
%    \begin{split}
%        \left\|z[k]\right\|_2&\geq\frac{1}{\left\|P^{-1}\right\|_2}\left\|P^{-1}z[k]\right\|_2\\
%        &\geq\frac{1}{\left\|P^{-1}\right\|_2}\left|\sum_{l=0}^{k-1}\sigma_{l^*}^{k-l-1}B_{l^*}'w'[l]\right|\rightarrow\infty.
%    \end{split}
%\end{equation*}
%Hence, $\left\|x_{v+b}[k]-x_v[k]\right\|_2$ also diverges. Consequently, the attack $\{b[k]\}_{k=0}^{\infty}$ is  $\{H_k\}_{k=0}^{\infty}$-disruptive.
\end{proof}
\quad We remark that to achieve the disruptive property, the assumption that $\epsilon \geq|CB|d$ is critical for the quantized attack \eqref{eq:no:dzdag}. This is because the initial state is $z[0]=0$, if $\epsilon<|CB|d$, then $z[k]\equiv0$. This will further result in $b[k]\equiv0$, that is, no quantized attack.\par
Furthermore, compared with the attack approach based on dynamic quantization  of Section \ref{se:dqzda}, the attack \eqref{eq:no:dzdag} of this method has the following feature: If there is only one unstable zero, then the output error is strictly less than 
the right-hand side of \eqref{eq:no:eps}. This holds because with sufficiently small $\Delta$, we have
\begin{equation*}
  \frac{1}{2}\left|CB\right|
    \left(1+|\lambda|\right)d
   >\left|CB\right|\left(d+\Delta\right)
\end{equation*}
where $\lambda$ is the unstable zero with $|\lambda|>1$.
We will confirm this difference between the two methods
in numerical examples presented in the next section.

\section{Quantized attacks for non-minimum phase systems}\label{ch:nonmin}
In this section, we discuss quantized attacks for the case when the sampled-data system \eqref{eq:pr:csys} has unstable zeros. Concretely, we consider three types of methods as follows:

(i)~The first method is simple, where zero dynamics attacks as described in Section~II-C are generated and quantized directly using the static quantizer.\par
(ii)~The second approach employs the dynamic quantizer outlined above to modify the zero dynamics attacks in real values. We can show that this method can achieve smaller output error than statically quantizing the zero dynamics attacks as in the first method. However, it is affected by the number of unstable zeros and their values. \par
(iii)~The third method quantizes attack signals which are slightly different from the conventional zero dynamics attacks and may induce small but bounded influence on the system output. This signal can be constructed without the influence of unstable zeros. Moreover, as we will see in the analysis as well as numerical examples, this method is capable to outperform the other two methods in terms of the output error.\par
In this section, we look at these three attack methods and analyze the resulting output errors. 
\subsection{Quantized attack method 1 via static quantization}\label{se:sqzda}
We start with the simplest quantization approach where the attack signal $a[k]$ generated by \eqref{eq:pr:zda} is directly 
quantized in a static manner. That is, the attack
signal $b[k]$ to be injected to the quantized control input $v[k]$ 
in Fig.~1 is given by
\begin{equation}
    b[k] = q(a[k]).
    \label{eq:no:bqa}
\end{equation} 
For this method, we obtain the following result.
%Regarding the quantized attack $\{b[k]\}_{k=0}^{\infty}$ 
%which is calculated by \eqref{eq:pr:zda} and the static 
%quantizer, we obtain the following result.
\begin{prop}
For the quantized networked system under attacks in Fig.~\ref{fig:pr:sys}, suppose that the quantization width $d$ is given and the scalar $\epsilon$ satisfies
\begin{equation}
    \epsilon
     > \frac{1}{2}\sum\limits_{k=0}^\infty\left|CA^kB\right|d.
\label{eq:no:staticeps}
\end{equation}
Moreover, suppose that the attack signal generated by \eqref{eq:pr:zda} is statically quantized as in \eqref{eq:no:bqa}. Then, for any sequence $\{H_k\}_{k=0}^{\infty}$, there exists a quantized attack signal $\{b[k]\}_{k=0}^{\infty}$ that is $\epsilon$-stealthy and $\{H_k\}_{k=0}^{\infty}$-disruptive.
\end{prop}

\begin{proof}
This can be shown easily, and we provide only the outline
of the proof. First, we upper bound the output error as
\begin{multline*}
   \left|y_{v+b}[k]-y_v[k]\right|\\
     \leq \left|y_{v+b}[k]-y_{v+a}[k]\right|
           + \left|y_{v+a}[k]-y_v[k]\right|.
\end{multline*}
The first term on the right-hand side is less than or equal to $({1}/{2})\sum_{k=0}^\infty\left|CA^kB\right|d$ from \eqref{eq:dq:staticeps}. By Lemma.~1 and \eqref{eq:no:staticeps}, we can choose the attack signal $\{a[k]\}_{k=0}^{\infty}$ which is $\left(\epsilon-({1}/{2})\sum_{k=0}^\infty\left|CA^kB\right|d\right)$-stealthy. Then, the right-hand side of the above inequality is less than or equal to $\epsilon$. Therefore the attack $\{b[k]\}_{k=0}^{\infty}$ is $\epsilon$-stealthy.\par
Next, we look at the difference in the states. We have
\begin{align*}
  &\left\| x_{v+b}[k]-x_v[k]\right\|_2\\
   &~~\geq \left\| x_{v+a}[k]-x_v[k]\right\|_2
       - \left\| x_{v+b}[k]-x_{v+a}[k]\right\|_2.
\end{align*}
On the right-hand side, the first term diverges as $k\rightarrow\infty$. The second term on the right-hand side coincides with the state which is driven by the quantization error; since the quantization error is bounded and the system \eqref{eq:pr:dsys} is stable, this term is also bounded. It means that $\left\| x_{v+b}[k]-x_v[k]\right\|_2$ diverges as $k\rightarrow\infty$. Therefore, we conclude that the attack $\{b[k]\}_{k=0}^{\infty}$ is $\{H_k\}_{k=0}^{\infty}$-disruptive.
\end{proof}

\subsection{Quantized attack method 2 via dynamic quantization}\label{se:dqzda}

In this subsection, we propose a method which modifies the continuous zero dynamics attacks in Section \ref{ch:problem} by using dynamic quantization from the previous section. This method adopts dynamic quantizer based on serial decomposition in Section \ref{se:dqseri}. \par

Specifically, we apply \eqref{eq:dq:dqseri} to generate attack signals after obtaining the serial decompositions $P_s$ and $P_u$ of the system \eqref{eq:pr:dsys} satisfying conditions outlined there. The effects to output signals caused by this attack is determined by the choices of 
$P_s$ and $P_u$. As we discussed earlier, while this method can be utilized for systems with any number of unstable zeros, the optimal value for the output error is known only for the case of one unstable zero. The following theorem is the first main result of this paper stating that dynamic quantizer can be efficient in reducing the error due to quantized attacks. 
\begin{theo}\label{theo:dqzda}
For the quantized networked system under attacks in Fig.~\ref{fig:pr:sys}, suppose that the plant is decomposed serially as described in Section \ref{se:dqseri} and $\epsilon$ satisfies
\begin{equation}
    \epsilon
     > \frac{1}{2}\left\|P_u\right\|_{i\infty}\left|C_sB_s\right|d,
\label{eq:no:eps}
\end{equation}
where $B_s$ and $C_s$ are the system matrices of $P_s$. Then, for any sequence $\{H_k\}_{k=0}^\infty$, there exists a quantized attack signal $\{b[k]\}_{k=0}^\infty$ generated by \eqref{eq:pr:zda} and dynamic quantization that is $\epsilon$-stealthy and $\{H_k\}_{k=0}^\infty$-disruptive.
\end{theo}\par

\begin{proof}
By Lemma \ref{lem:zda}, we can construct the attack signal $\{a[k]\}_{k=0}^{\infty}$ such that it is $\left(\epsilon-({1}/{2})\left\|P_u\right\|_{i\infty}\left|C_sB_s\right|d\right)$-stealthy and $\left\|x_{u+a}[k]-x_u[k]\right\|_2$ diverges. Consider the attack $\{b[k]\}_{k=0}^{\infty}$ which is made from $\{a[k]\}_{k=0}^{\infty}$ and the dynamic quantizer in Section \ref{se:dqseri}. We verify that this attack satisfies both $\epsilon$-stealthy and $\{H_k\}_{k=0}^{\infty}$-disruptive properties.\par
We can upper bound the output error as
\begin{align}
  & |y_{v+b}[k]-y_v[k]| \notag\\
  &~~~\leq|y_{v+b}[k]-y_{v+a}[k]|
          +| y_{v+a}[k]-y_v[k]|\label{eq:no:ineqy}
\end{align}
for every $k$. Since the system is linear, the first term on the right-hand side coincides with $|y_b[k]-y_a[k]|$. By the bound on the output error \eqref{eq:dq:dqserierroropt} for the dynamic quantizer, this term can be bounded as
\begin{equation}
    \sup\limits_k|y_b[k]-y_a[k]|
     \leq \frac{1}{2}\left\|P_u\right\|_{i\infty}|C_sB_s|d.\notag
\end{equation}
By the construction of $\{a[k]\}_{k=0}^{\infty}$, the second term on the right-hand side of \eqref{eq:no:ineqy} satisfies the following inequality:
\begin{equation}
    \sup\limits_k|y_{v+a}[k]-y_v[k]|
       \leq \epsilon - \frac{1}{2}\left\|P_u\right\|_{i\infty}|C_sB_s|d.\notag
\end{equation}
Hence, from \eqref{eq:no:ineqy}, we have $\sup_k| y_{v+b}[k]-y_v[k]|\leq\epsilon$. Consequently, the attack $\{b[k]\}_{k=0}^\infty$ is $\epsilon$-stealthy.\par
Next, we look at the difference in states. We have
\begin{multline}
    \left\| x_{v+b}[k]-x_v[k]\right\|_2\\\geq\left\| x_{v+a}[k]-x_v[k]\right\|_2-\left\| x_{v+b}[k]-x_{v+a}[k]\right\|_2.\label{eq:no:statediff}
\end{multline}
Because of the choice of $\{a[k]\}_{k=0}^{\infty}$, the first term on the right-hand side diverges as $k\rightarrow\infty$. Since the system is linear, it follows that $x_{v+b}[k]-x_{v+a}[k]=x_b[k]-x_a[k]$. The state-space equation of the plant and the dynamic quantizer can be written as
\begin{equation}
\begin{bmatrix}
x_{b}[k+1] \\
\xi[k+1]\\
x_{a}[k+1]
\end{bmatrix}\hskip-3pt=\hskip-3pt\left[\begin{array}{@{\hskip-1pt}c@{\hskip1pt}c@{\hskip1pt}c@{\hskip-1pt}}
A&O&O\\
O&A_s&O\\
O&O&A 
\end{array}\right]\hskip-5pt\begin{bmatrix}
x_{b}[k] \\
\xi[k]\\
x_{a}[k]
\end{bmatrix}\hskip-2pt+\hskip-2pt\begin{bmatrix}
Bb[k] \\
B_s(b[k]-a[k])\\
Ba[k]
\end{bmatrix}.\label{eq:no:xbxixa}
\end{equation}
Let $w_q[k]$ be
\begin{equation*}
 w_q[k]
  := q\left(-\frac{C_sA_s}{C_sB_s}\xi[k]+a[k]\right)
      - \left(-\frac{C_sA_s}{C_sB_s}\xi[k]+a[k]\right).
\end{equation*}
Then, the quantized attack is expressed as $b[k]=a[k]+w_q[k]-{C_sA_s}/({C_sB_s})\xi[k]$. 
Note that $\left\|w_q[k]\right\|_\infty\leq d/2$. Also, let %the system matrix $\bar{A}$ be
\begin{equation*}
    \bar{A}:=\begin{bmatrix}
    A & -\frac{BC_sA_s}{C_sB_s} \\
    O & A_s-\frac{B_sC_sA_s}{C_sB_s}
    \end{bmatrix}.
\end{equation*}
Since the system \eqref{eq:pr:dsys} is stable and the subsystem $P_s$ is minimum phase, 
the matrix $\bar{A}$ is also stable. 

The state-space equation in \eqref{eq:no:xbxixa} can be rewritten as
\begin{equation}
\begin{bmatrix}
x_{b}[k+1] \\
\xi[k+1]\\
x_{a}[k+1]
\end{bmatrix}\hskip-3pt=\hskip-3pt\left[\begin{array}{@{\hskip-1pt}c|c@{\hskip-1pt}}
\raisebox{-0.5em}{\mbox{\LARGE $\bar{A}$}} & \begin{matrix}
O \\ O
\end{matrix}
\\
\hline
 \begin{matrix}
O & O
\end{matrix} & A
\end{array}\right]\hskip-5pt\begin{bmatrix}
x_{b}[k] \\
\xi[k]\\
x_{a}[k]
\end{bmatrix}+\begin{bmatrix}
B(a[k]+w_q[k]) \\
B_sw_q[k]\\
Ba[k]
\end{bmatrix}.\notag
\end{equation}
From this system, we can obtain the following 
one of reduced dimension that takes
$x_{b}[k]-x_a[k]$ as the state:
%=[I~O~-I][x_{b}[k]~\xi[k]\ x_{a}[k]]^T$ and $\xi[0]=0$, we can obtain the relation
%\begin{equation}
%x_{b}[k]-x_a[k]=\sum\limits_{l=0}^{k-1}\begin{bmatrix}
%I & O
%\end{bmatrix}\bar{A}^{l}\begin{bmatrix}
%B \\
%B_s
%\end{bmatrix}w_q[k-l].\notag
%\end{equation}
\begin{equation}
\begin{bmatrix}
 x_{b}[k+1]-x_a[k+1]\\
 \xi[k+1]
\end{bmatrix}
 = \bar{A} 
    \begin{bmatrix}
       x_{b}[k]-x_a[k]\\
       \xi[k]
    \end{bmatrix}
  +  \begin{bmatrix}
        B\\
        B_s
     \end{bmatrix} w_q[k].
\end{equation}
As $\bar{A}$ is stable, 
the sequence $x_{b}[k]-x_a[k]$ is bounded, implying that the second 
term on the right-hand side of \eqref{eq:no:statediff} is bounded. 
Consequently, $\left\| x_{v+b}[k]-x_v[k]\right\|_2\rightarrow\infty$ as $k\rightarrow\infty$. Therefore, we conclude that the quantized attack signal $\{b[k]\}_{k=0}^\infty$ is $\{H_k\}_{k=0}^{\infty}$-disruptive.
\end{proof}

\subsection{Quantized attack method 3 via $\epsilon$-stealthy approach}\label{se:calc}

The third method employs a continuous attack signal 
which is a slightly modified version of the zero 
dynamics attack in \eqref{eq:pr:zda} and then quantizes 
it directly. 

To this end, for given $\epsilon>0$, let us consider 
the attack signal $\{a[k]\}_{k=0}^{\infty}$ generated by
\begin{equation}
    \begin{split}
        z[k+1]&=\left(A-\frac{BCA}{CB}\right)z[k]+\frac{B}{CB}\epsilon,\\
        a[k]&=-\frac{CA}{CB}z[k]+\frac{\epsilon}{CB},
    \end{split}\label{eq:no:zdag}
\end{equation}
where $z[0]=0$. We can easily show that this system 
is unstable and the state difference difference $x_{u+a}[k]-x_u[k]$ is equal to $z[k]$. 
It means that $y_{u+a}[k]-y_u[k]=Cz[k]$ and this value 
coincides with $\epsilon$ for $k\geq1$. 
Hence, this attack is $\epsilon$-stealthy in the sense of \eqref{eq:pr:estealth} (without quantization).

The system \eqref{eq:no:zdag} motivates us to generate attack
signals using a system which has constant input proportional 
to $\epsilon$ and then to quantize the signal. 
This can be achieved by introducing the following quantized system:
%a quantized version which generates the quantized attack signal $\{b[k]\}_{k=0}^{\infty}$ as
\begin{equation}
 \begin{split}
   z[k+1]&=Az[k]+Bb[k],\\
        b[k]&=q\left(-\frac{CA}{CB}z[k]+\frac{\sign(CB)\epsilon}{CB}-\frac{d}{2}+\Delta\right),
    \end{split}\label{eq:no:dzdag}
\end{equation}
where $z[0]=0$ and 
$\Delta$ is a sufficiently small positive number. \par
Now, we are ready to state our second main result of the paper. 
\begin{theo}\label{theo:2}
For the quantized networked system under attacks in Fig.~\ref{fig:pr:sys}, suppose that $\epsilon$ is taken as $\epsilon \geq |CB|d$. Then, for any sequence $\{H_k\}_{k=0}^{\infty}$, the quantized attack $\{b[k]\}_{k=0}^{\infty}$ generated by \eqref{eq:no:dzdag} is $\left(\epsilon + |CB|\Delta\right)$-stealthy and $\{H_k\}_{k=0}^{\infty}$-disruptive. 
%if the following conditions are satisfied:
%\begin{itemize}
%    \item Consider the Jordan decomposition of $\left(A-\frac{BCA}{CB}\right)=P\Lambda P^{-1}$. Moreover, let the diagonal elements of $\Lambda$ be $\sigma_1,\ldots,\sigma_n$ and $\sigma_l,\ldots,\sigma_n$ be unstable ones. Then at least one of the $l-th$ through the last element of $B':=P^{-1}B$ is not $0$.
%\end{itemize}
\end{theo}
\begin{proof}
It is straightforward to show that by \eqref{eq:no:zdag}, the differences 
in the states and outputs can be written as $x_{v+b}[k]-x_v[k]=z[k]$ 
and $y_{v+b}[k]-y_v[k]=Cz[k]$, 
respectively. Let  
\[
  w[k]
   := \left(-\frac{CA}{CB}z[k]+\frac{\sign(CB)\epsilon}{CB}+\Delta\right)
       - b[k]. 
\]
Then, \eqref{eq:no:dzdag} becomes
\begin{multline}
    z[k+1]=\left(A-\frac{BCA}{CB}\right)z[k]\\+B\left(\frac{\sign(CB)\epsilon}{CB}+\Delta-w[k]\right),
\end{multline}
where $0<w[k]\leq d$. Since $z[0]=0$, we have 
\begin{align}
&z[k]\notag\\
&~~= \sum_{l=0}^{k-1}\left(A-\frac{BCA}{CB}\right)^{k-l-1}
   B\left(\frac{\sign(CB)\epsilon}{CB}+\Delta-w[l]\right).
\label{eq:no:zk}
\end{align}
Note that it holds $C\left(A-{BCA}/({CB})\right)=0$, and thus, $Cz[k]=\sign(CB)\epsilon+CB\Delta-CBw[k-1]$ for all $k\geq1$. 
Since $\epsilon\geq|CB|d$, we have
\[
  -\left(\epsilon+|CB|\Delta\right)
    \leq|Cz[k]|\leq\epsilon+|CB|\Delta. 
\]
It means that the attack $\{b[k]\}_{k=0}^{\infty}$ is $\left(\epsilon+|CB|\Delta\right)$-stealthy.\par
Next, to establish that $\{b[k]\}_{k=0}^{\infty}$ is $\{H_k\}_{k=0}^{\infty}$-disruptive, we must show that $z[k]$ diverges, which implies that $\left\|x_{v+b}[k]-x_v[k]\right\|_2$ also diverges. Let 
\[
  w'[k]:=\frac{\sign(CB)\epsilon}{CB}+\Delta-w[k].
\] 
Now, based on \eqref{eq:no:zk}, we can rewrite the system of $z[k]$ and
obtain
\begin{equation*}
   z[k+1] = \left(A-\frac{BCA}{CB}\right) z[k] + B w'[k].
\end{equation*}
Since $z[0]=0$, we must make sure that in this system, the input $w'[k]$ enters and excites the unstable mode. \par 
Recall that $\left(A,B\right)$ is controllable, and thus $\left(A-{BCA}/{(CB)}, B\right)$ is also controllable. Hence, it follows that the input $w'[k]$ enters every mode of the system. Finally, we can confirm that $w'[k]$ is not zero at all times. By assumption, $\epsilon\geq|CB|d$, and this implies that $w'[k]\geq\Delta$ at each $k$. Therefore, the state $z[k]$ diverges.
\end{proof}

%Next, we look at $z[k]$. The matrix $A-\frac{BCA}{CB}$ can be denoted as $P\Lambda P^{-1}$ by Jordan decomposition. Moreover, we can let the diagonal elements of $\Lambda$ be $\sigma_1,\ldots,\sigma_n$ and $\sigma_l,\ldots,\sigma_n$ be unstable ones.  Since $\epsilon\geq|CB|d$, the new variable $w'[k]:=\frac{\sign(CB)}{CB}\epsilon+\Delta-w[k]$ satisfies $w'[k]\geq\Delta$. Then, \eqref{eq:no:zk} becomes
%\begin{equation}
%    z[k]=\sum_{l=0}^{k-1}P\Lambda^{k-l-1}B'w'[l],\label{eq:no:zk2}
%\end{equation}
%where $B':=P^{-1}B$. Since $\left(A, B\right)$ is controllable, $\left(A-\frac{BCA}{CB}, B\right)$ is also controllable. Therefore, the vector $B'$ has nonzero elements and the last one has index greater than or equal to $l$. Here, we denote this index by $l^*$. Then the state $P^{-1}z[k]$ can be expressed as
%\begin{equation}
%    P^{-1}z[k]=\begin{bmatrix}
%    *\\
%    \vdots\\
%    *\\
%    \sum_{l=0}^{k-1}\sigma_{l^*}^{k-l-1}B_{l^*}'w'[l]\\
%    0\\
%    \vdots\\
%    0
%    \end{bmatrix}.\label{eq:no:zk3}
%\end{equation}
%The square norm of $P^{-1}z[k]$ can be bounded as
%\begin{equation*}
%        \left\|P^{-1}z[k]\right\|_2\leq\left\|P^{-1}\right\|_2\left\|z[k]\right\|_2.
%\end{equation*}
%Then, that of $z[k]$ can be bounded as
%\begin{equation*}
%    \begin{split}
%        \left\|z[k]\right\|_2&\geq\frac{1}{\left\|P^{-1}\right\|_2}\left\|P^{-1}z[k]\right\|_2\\
%        &\geq\frac{1}{\left\|P^{-1}\right\|_2}\left|\sum_{l=0}^{k-1}\sigma_{l^*}^{k-l-1}B_{l^*}'w'[l]\right|\rightarrow\infty.
%    \end{split}
%\end{equation*}
%Hence, $\left\|x_{v+b}[k]-x_v[k]\right\|_2$ also diverges. Consequently, the attack $\{b[k]\}_{k=0}^{\infty}$ is  $\{H_k\}_{k=0}^{\infty}$-disruptive.
%\end{proof}

We remark that to achieve the disruptive property, the assumption that $\epsilon \geq|CB|d$ is critical for the quantized attack \eqref{eq:no:dzdag}. This is because the initial state is $z[0]=0$, if $\epsilon<|CB|d$, then $z[k]\equiv0$. This will further result in $b[k]\equiv0$, that is, no quantized attack.
Furthermore, compared with the attack approach based on dynamic quantization of
Section \ref{se:dqzda}, the attack \eqref{eq:no:dzdag} of this method has 
an advantage in the specific case when the plant has only one unstable zero:
Then, the output error of this approach 
can be made strictly smaller than that for dynamic quantization.
 whose $\epsilon$ 
is lower bounded as shown in \eqref{eq:no:eps}. 

This holds because with sufficiently small $\Delta$, we have
\begin{equation*}
  \frac{1}{2}\left|CB\right|
    \left(1+|\lambda|\right)d
   >\left|CB\right|\left(d+\Delta\right),
\end{equation*}
where $\lambda$ is the unstable zero with $|\lambda|>1$.
We will confirm this difference between the two methods
in numerical examples presented in the next section.


%------------------------
% \begin{figure}[t]
        \centering
            \begin{subfigure}{0.48\linewidth}
            \centering
            {\includegraphics[width=0.95\linewidth]{figures/qual_wo.png}}
            \caption{\method w/o TA}\label{fig:quala}
           \end{subfigure}
           \begin{subfigure}{0.48\linewidth}
            \centering
            {\includegraphics[width=0.95\linewidth]{figures/qual_w.png}}
            \caption{\method w/ TA}\label{fig:qualb}
           \end{subfigure}\vspace{-7pt}
        \caption{Visualization of attention maps (a) without temporal adaptation (TA) and (b) with temporal adaptation for the action 'Spinning [something] that quickly stops spinning' in SSv2~\cite{ssv2}.}\vspace{-5pt}
    \label{fig:qual}
    \end{figure}
\section{Numerical Example}\label{ch:example}
In this section, we illustrate the results of our paper through numerical simulations. \par
Consider the following stable continuous-time system with quantized input under attack \cite{enhancement}:
\begin{equation*}
    \begin{split}
        \dot{x}(t)&=\begin{bmatrix}
        0 & 0 & 1 & 0\\
        0 & 0 & 0 & 1 \\
        -2 & 1 & -1 & 1\\
        1 & -1 & 1 & -1
        \end{bmatrix}x(t)+\begin{bmatrix}
        0\\0\\0\\1
        \end{bmatrix}(v(t)+b(t)),\\
        y(t)&=\begin{bmatrix}
        1&0&0&0
        \end{bmatrix}x(t).
    \end{split}
\end{equation*}
This system has one stable zero whose value is $-1$. When discretized using sampling period $T=0.5$, the system has three zeros at $-3.21$, $-0.24$, and $0.61$ in the discrete-time domain. In particular, there is one unstable zero $\lambda=-3.21$. We set the initial state as $x_0=0$ and the input as $u[k]=\sin0.05\pi k+0.5\cos0.025\pi k$. Let the quantization width be  $d=1$. We compare the three methods for quantized attacks from Section \ref{ch:nonmin}:~(i)~Static quantization, (ii)~dynamic quantization, and (iii)~$\epsilon$-stealthy approach.

\begin{figure}[t]
%\vspace*{2mm}
    \begin{minipage}[h]{1\linewidth}
        \centering
        \includegraphics[width=1\linewidth]{workspace/picture/subpic2/nonmin_sta_res.png}
        \subcaption{Outputs of the system in continuous time and discrete time  \label{fig:no:respnonminsta}}
    \end{minipage}\\
    \begin{minipage}[h]{1\linewidth}
    \vspace*{4mm}
        \centering
        \includegraphics[width=1\linewidth]{workspace/picture/subpic2/input_sta.png}
        \subcaption{Quantized control inputs before and after attacks  \label{fig:no:stazda}} 
    \vspace*{1mm}
    \end{minipage}
    \caption{Time responses of the system under the statically quantized zero dynamics attack. \label{fig:no:stzdaresult}}
\end{figure} 

(i) First, a static quantizer is used for zero dynamics attacks as described 
in Section~\ref{se:sqzda}. 
In Fig.~\ref{fig:no:respnonminsta}, the time responses of the output of the plant 
are shown. The blue line is the continuous-time output (before sampling) under attack while 
the black circles indicate the output at sampling times $t=kT$ for $k\in\mathbb{Z}_+$. 
The red line is the continuous-time output when no attack is present. 
Fig.~\ref{fig:no:stazda} displays the quantized signals of the control input and the sum of the input and the attack. We observe that the input and thus the plant states are diverging, but this is difficult to detect from the sampled output, which is fairly close to the expected behavior of the output under normal conditions.  From \eqref{eq:no:staticeps}, the theoretical bound on the error between the two output signals is $\sum_{k=0}^\infty\left|CA^kB\right|{d}/{2}=5.717$, which is clearly satisfied in the simulation results.


(ii) Next, we look at attacks which are dynamically quantized from Section \ref{se:dqzda}. Fig.~\ref{fig:no:respnonmin1} presents the output of the plant similarly to the static quantization case in Fig.~\ref{fig:no:respnonminsta}. In Fig.~\ref{fig:no:diffy1}, the difference between the two outputs  with and without attacks is plotted, where the theoretical bound \eqref{eq:no:eps} from Theorem $1$ is $|CB|(1+|\lambda|){d}/{2}=0.038$. In comparison, it is impressive that the difference between the sampled output under attack is much closer to the normal output than in the case of static quantization, which shows the effectiveness of the approach. Fig.~\ref{fig:no:dqzda} displays the control input with and without attack (as in Fig.~\ref{fig:no:stazda}). Note that the attack effects do not appear for a while in the output. This is because the zero dynamics attack signal is initially small.


(iii) Finally, we discuss the quantized attack based on $\epsilon$-stealthy approach from Section \ref{se:calc}.  The results are shown in Figs.~\ref{fig:no:respnonmin2}--\ref{fig:no:qzdag} as in the dynamic quantization case. Notice that the sampled output is closer to the attack-free one than in the previous two cases. In fact, the difference between attacked/original outputs remains below the bound $\epsilon=|CB|d+10^{-6}=0.018$ provided in Theorem \ref{theo:2}. Moreover, unlike the previous methods, in Fig.~\ref{fig:no:diffy2}, the attack effects appear at time $k=0$. This is because the initial value of the attack should be non-zero. 





\begin{figure}[t]
%\vspace*{2mm}
    \begin{minipage}[h]{1\linewidth}
        \centering
        \includegraphics[width=1\linewidth]{workspace/picture/subpic2/nonmin_dq_res.png}
        \subcaption{System responses   \label{fig:no:respnonmin1}}
    \end{minipage}\\
\vspace*{1mm}
    \begin{minipage}[h]{1\linewidth}
       \vspace*{4mm}
        \centering
        \includegraphics[width=1\linewidth]{workspace/picture/subpic2/nonmin_dq_diffy.png}
        \subcaption{Difference in the outputs of the system    \label{fig:no:diffy1}}
    \end{minipage}\\
\vspace*{1mm}
    \begin{minipage}[h]{1\linewidth}
        \vspace*{4mm}
        \centering
        \includegraphics[width=1\linewidth]{workspace/picture/subpic2/input_dqzda.png}
        \subcaption{Control inputs and inputs contaminated by the attack.  \label{fig:no:dqzda}}
\vspace*{1mm}
    \end{minipage}
    \caption{Time responses of the system under the quantized zero dynamics attack based on
dynamic quantization
\label{fig:no:dqzdaresult}}
\end{figure} 



%------------------------
%\section{Conclusion}\label{sec:conclusion}
In this work, we focus on addressing the fundamental challenge of OOD detection tasks, which is how to fully understand the semantic discrepancy between the ID/OOD samples. We reveal that the key to success in the realistic SCOOD task is to allocate as many ID samples in the unlabeled set correctly as possible. To this end, we propose a novel uncertainty-aware optimal transport scheme that introduces class-specific energy scores as guidance for effective label assignment. Experimental results show that our method achieves better performance than previous state-of-the-art methods on SCOOD benchmarks.

\textbf{Limitations.} In addition to temperature scaling, other techniques such as feature clipping applied in ReAct~\cite{sun2021react} also enhance the performance of energy score, so how to obtain an OOD score that best fits the SCOOD task can be further explored. Moreover, a setting highly related to SCOOD has been proposed in \cite{katz2022training} and formulated as a constrained optimization problem. We will also theoretically analyze these practical OOD settings in our feature work.

% \section*{Acknowledgments}
\textbf{Acknowledgments.} 
This work is supported by National Key R\&D Program of China under Grant 2020AAA0105701, National Natural Science Foundation of China (NSFC) under Grants 61872327, Major Special Science and Technology Project of Anhui, National Natural Science Foundation of China (62033012) and Ant Group through Ant Research Intern Program.


\section{Conclusion}\label{ch:conclusion}
%In this paper, we proposed the attack method to deviate a system state while reducing an output error, where an input of a sampled-data system is quantized. At that time, we considered a system which has unstable zeros.\par
%We proposed two types of method. The first method modifies the conventional zero dynamics attack by the dynamic quantizer. The second method directly calculates the quantized attack. These methods achieve the goal to deviate a system state. However, the effect on a system output is not same. Though the former method is less influential when the zero dynamics attack is statically quantized, the effect on an output depends on the number of unstable zeros in a system and its values. On the other hand, the latter method is independent of them. Moreover, its effect on an output is less than the former method.\par
%Future work is to propose the quantized attack method on a feedback condrol system. In this paper, we assumed a feedforward control system. So we were able to consider the effect of the quantized attack independently of control law. On the other hand, in a feedback control system, the quantization of the attack affects not only the system output but also the control input, so the problem formulation become more complicated. However, since a feedback control system is widely used in practical applications, this seems to be an important issue.

In this paper, we have studied the effects of quantization on sampled-data 
systems under zero dynamics attacks. We have proposed methods for generating 
quantized attacks exploiting the unstable zeros in the plant after discretization. 
%Under these attacks, the system state diverges while the error in the output 
%due to quantization is kept small. 
In particular, we have proposed two types of methods. The first one utilizes 
the conventional zero dynamics attacks and modifies them
via dynamic quantization. In the second method, slightly 
modified version of zero dynamics attacks is 
quantized. These methods have been compared with the simple
approach of directly quantizing the conventional zero
dynamics attacks, and our analysis shows that they perform
strictly better. 
In future research, we will study the effects of quantized attacks 
against discretized plants which are minimum phase as well as 
systems having feedback structures.



\if0
Our study in this paper has been limited in two respects.
One is that the system is assumed to have unstable zeros.
The case without such zeros can be handled if there is
redundancy in the inputs compared to the outputs (e.g., in terms of 
dimensions) \cite{zsa}, and we will pursue such an approach.
The other is that the control system setting is in the 
feedforward form. 
Hence, for our future work, we must look at the feedback control case, where the chosen control laws may affect the performance of the quantized attack. We should note that these two system settings have been separately treated for the dynamic quantization in the literature \cite{dquant}. This may indicate the differences in the nature of the problems.
\fi

\begin{figure}[t]
%\vspace*{2mm}
    \begin{minipage}[h]{1\linewidth}
        \centering
        \includegraphics[width=1\linewidth]{workspace/picture/subpic2/nonmin_direct_res.png}
        \subcaption{System responses  \label{fig:no:respnonmin2}}
    \end{minipage}\\
\vspace*{1mm}
    \begin{minipage}[h]{1\linewidth}
        \vspace*{4mm}
        \centering
        \includegraphics[width=1\linewidth]{workspace/picture/subpic2/nonmin_direct_diffy.png}
        \subcaption{Difference in the outputs of the system  \label{fig:no:diffy2}}
    \end{minipage}\\
\vspace*{1mm}
    \begin{minipage}[h]{1\linewidth}
        \centering
        \vspace*{4mm}
        \includegraphics[width=1\linewidth]{workspace/picture/subpic2/input_qa.png}
        \subcaption{Control inputs and inputs contaminated by the attack.  \label{fig:no:qzdag}}
        \vspace*{1mm}
    \end{minipage}
    \caption{Time responses of the system under the attack via $\epsilon$-stealthy approach. \label{fig:no:qaresult}}
\end{figure} 





%\bibliographystyle{junsrt}

\bibliographystyle{plain}
\bibliography{main,BibDataBase_jun18a,BibDataBase_jun18}
\addtolength{\textheight}{-12cm}   % This command serves to balance the column lengths
                                  % on the last page of the document manually. It shortens
                                  % the textheight of the last page by a suitable amount.
                                  % This command does not take effect until the next page
                                  % so it should come on the page before the last. Make
                                  % sure that you do not shorten the textheight too much.

%%%%%%%%%%%%%%%%%%%%%%%%%%%%%%%%%%%%%%%%%%%%%%%%%%%%%%%%%%%%%%%%%%%%%%%%%%%%%%%%



%%%%%%%%%%%%%%%%%%%%%%%%%%%%%%%%%%%%%%%%%%%%%%%%%%%%%%%%%%%%%%%%%%%%%%%%%%%%%%%%



%%%%%%%%%%%%%%%%%%%%%%%%%%%%%%%%%%%%%%%%%%%%%%%%%%%%%%%%%%%%%%%%%%%%%%%%%%%%%%%%
%\section*{APPENDIX}
%\section*{ACKNOWLEDGMENT}



%%%%%%%%%%%%%%%%%%%%%%%%%%%%%%%%%%%%%%%%%%%%%%%%%%%%%%%%%%%%%%%%%%%%%%%%%%%%%%%%



%\begin{thebibliography}{99}
%\end{thebibliography}




\end{document}
