\documentclass[10pt,twocolumn,letterpaper]{article}

\usepackage{iccv}
\usepackage{times}
\usepackage{epsfig}
\usepackage{graphicx}
\usepackage{amsmath}
\usepackage{amssymb}

\usepackage{booktabs}
\usepackage{caption}
\usepackage{amsfonts}

\usepackage{color}
\usepackage{multirow}
\usepackage{multicol}
\usepackage{float}

\usepackage{subfigure}
\usepackage{lipsum}

\usepackage[breaklinks=true,bookmarks=false]{hyperref}
\usepackage{cleveref}
\pagenumbering{arabic}
\iccvfinalcopy

\def\httilde{\mbox{\tt\raisebox{-.5ex}{\symbol{126}}}}


\newcommand\ours{\texttt{3DFuse}\xspace}

\begin{document} 

\title{Let 2D Diffusion Model Know 3D-Consistency for Robust Text-to-3D Generation}

\author{Junyoung Seo\thanks{Equal contribution.} $\,^\text{1}$~~~~Wooseok Jang\footnotemark[1] $\,^\text{1}$~~~~Min-Seop Kwak\footnotemark[1] $\,^\text{1}$~~~~Jaehoon Ko$^\text{1}$~~~~Hyeonsu Kim$^\text{1}$ \\
Junho Kim$^\text{2}$~~~~Jin-Hwa Kim\thanks{Co-corresponding author.} $\,^\text{2}$~~~~Jiyoung Lee\footnotemark[2] $\,^\text{2}$~~~~Seungryong Kim\footnotemark[2] $\,^\text{1}$\\
\\
$^\text{1}$Korea University~~~~~~$^\text{2}$NAVER AI Lab
}


\twocolumn{
\maketitle
}



%%%%%%%%% ABSTRACT
\begin{abstract}
Text-to-3D generation has shown rapid progress in recent days with the advent of score distillation, a methodology of using pretrained text-to-2D diffusion models to optimize neural radiance field (NeRF) in the zero-shot setting. However, the lack of 3D awareness in the 2D diffusion models destabilizes score distillation-based methods from reconstructing a plausible 3D scene. To address this issue, we propose \ours, a novel framework that incorporates 3D awareness into pretrained 2D diffusion models, enhancing the robustness and 3D consistency of score distillation-based methods. We realize this by first constructing a coarse 3D structure of a given text prompt and then utilizing projected, view-specific depth map as a condition for the diffusion model. Additionally, we introduce a training strategy that enables the 2D diffusion model learns to handle the errors and sparsity within the coarse 3D structure for robust generation, as well as a method for ensuring semantic consistency throughout all viewpoints of the scene. Our framework surpasses the limitations of prior arts, and has significant implications for 3D consistent generation of 2D diffusion models. Project page is available at \textnormal{\url{https://ku-cvlab.github.io/3DFuse/}}.
\end{abstract}

% \begin{figure}[t]
%     % \begin{subfigure}{1\linewidth}
%     %   \centering
%     % %   \includegraphics[width=1\linewidth]{figs/fig_1_moti_textattn.pdf}  
%     % %   \includegraphics[width=1\linewidth]{figs/fig_1_moti_textattn_v2.pdf}  
%     %   \includegraphics[width=1\linewidth]{figs/fig_1_moti_textattn_v5.pdf}  
%     %   \vspace{-0.5cm}
%     %     \caption{Amount of attention added to each video clip from the source video and query text in the self-attention layers of Moment-DETR encoder.}
%     %     % \caption{Distribution of attention for source and query in Moment-DETR encoder}
%     %     % Visualization of video clip's self-attention score in Moment-DETR encoder.
%     %   \label{fig:fig1_text_attn_ex}
%     % \end{subfigure}%\hfill% or  or \hspace{0.3\textwidth}
%     \vspace{0.2cm}
%     % \begin{subfigure}{1\linewidth}
%       \centering
%     %   \includegraphics[width=1\linewidth]{figs/fig1_moti_negattn.pdf}  
%       \includegraphics[width=1\linewidth]{figs/fig1_moti_negattn_v3.pdf}  
%       \vspace{-0.4cm}
%     %   \caption{Correspondence of saliency scores on the relevance between video clips and the text query.}
%     % \caption{Predicted saliency scores against the video relevant positive query and video irrelevant negative query}
%       \label{fig:fig1_neg_attn_ex}
%     % \end{subfigure}%\hfill% or  or \hspace{0.3\textwidth}
%     \caption{
%     % 원준 원본
%     % (a) Comparison between attention scores of source and query for each video clip~(We sum the attention scores from video and text). 
%     % We observe that the attention scores are dominated by other clips in the source video. 
%     % Text queries do not account for much attention regardless of the relevance to the video clips.
%     % \textbf{(a)} Inspection of the query dependency in Moment-DETR encoder.
%     % % We visualize the attention score of video tokens in the transformer encoder and observe that text query accounts for only a low portion of attention.
%     % % This tendency occurs regardless of the relevance between the text query and video clips. 
%     % We visualize the attention score of video tokens in the transformer encoder and observe 1) text query only accounts for a low portion of attention, and 2) relevance between video-query pair does not affect the attention scores ratio of text.
%     \textbf{(b)} Comparison of highlight-ness when relevant and non-relevant queries are input.
%     As observed in , existing work only uses queries to play an insignificant role, thereby may not be capable of detecting false queries and considering the video-query relevance even when the problem in (a) is resolved. 
%     % \SE{} % 이 부분이 "not capable of" 란 용어가 세다는 피드백이 있는 듯 합니다. 이러한 능력이 없다는 것은 굉장히 강한 어조인거 같기는 하고, 이러한 경우들이 종종 있다거나 좀 약화시킬 필요가 있어보이긴 하네요.
%     On the other hand, our QD-DETR yields a query-dependent representation that the relevance between the source video and query text is updated in the saliency scores.
%     There is a large gap between positive and negative saliency scores, and scores are consistent since the clips are all highly correlated to others.
%     }
%     \label{fig:motivation_ex}
%     % \captionsetup{belowskip=13pt}
%     % \setlength{\belowcaptionskip}{-10pt}
% \end{figure}
\begin{figure}
    \centering
    \includegraphics[width=1\linewidth]{figs/fig1_moti_negattn_1111.pdf}
    % \includegraphics[width=1\linewidth]{figs/fig1_moti_negattn_1109.pdf}
    % \includegraphics[width=1\linewidth]{figs/fig1_moti_negattn_stat.pdf}
    \vspace{-0.6cm}
    \caption{
        % \SE{} % 수정 필요
        Comparison of highlight-ness~(saliency score) when relevant and non-relevant queries are given.
        We found that the existing work only uses queries to play an insignificant role, thereby may not be capable of detecting negative queries and video-query relevance; saliency scores for clips in ground-truth~(GT) moments are low and equivalent for positive and negative queries.
        % This also results in mispredicted moments when ground-truth~(GT) moment is dominated by clips unrelated to GT since their prediction is highly focused on the video.
        % \SE{} % 여기 한번 더 보면 좋을 듯 합니다. GT moment에 unrelated한 clip이 많으면? label이 틀렷을 경우를 말씀하시는건지?
        % As observed in saliency graph, existing work only uses queries to play an insignificant role, thereby may not be capable of detecting false queries and considering the video-query relevance.
        On the other hand, query-dependent representations of QD-DETR result in corresponding saliency scores to the video-query relevance and precisely localized moments.
        % On the other hand, our QD-DETR yields a query-dependent representation that the
        % saliency scores are in accordance with the relevance between the video and query.
        % text is in accordance with the saliency scores.
        % There is a large gap between positive and negative saliency scores, and scores are consistent since the clips are all highly correlated to others.
}
    \label{fig:motivation_ex}
\end{figure}


\section{Introduction}
% 원준 원본
% Along with the advance of digital devices and platforms, video is now one of the most desired data type for consumers. However, although the large information capacity of videos may be beneficial in many aspects, e.g., informative and entertaining, on the contrary perspective, videos are time-consuming, and hard to search for desirable moments. 
% This has led many creators to use extra manpower to crop and edit the video to generate highlight clips to gain the consumer’s attention.
Along with the advance of digital devices and platforms, video is now one of the most desired data types for consumers~\cite{apostolidis2021video,wu2017deep}.
% SE: Video aware deep learning application & survey papers?
Although the large information capacity of videos might be beneficial in many aspects, e.g., informative and entertaining, inspecting the videos is time-consuming, so that it is hard to capture the desired moments~\cite{anne2017localizing,apostolidis2021video}. 
% This has led many creators to use extra manpower to crop and edit the video to generate highlight clips to gain the consumer’s attention.


% On the other side, 
Indeed, the need to retrieve user-requested or highlight moments within videos is greatly raised.
Numerous research efforts were put into the search for the requested moments in the video~\cite{anne2017localizing, gao2017tall, liu2015multi, escorcia2019temporal} and summarizing the video highlights~\cite{zhang2016video, mahasseni2017unsupervised, badamdorj2022contrastive, wei2022learning}.
% Numerous research efforts were put into the search for the requested moments in the video~\cite{anne2017localizing, gao2017tall, liu2015multi, escorcia2019temporal}, summarizing the video to generate highlights was another popular topic~\cite{zhang2016video, mahasseni2017unsupervised, badamdorj2022contrastive, wei2022learning}.
Recently, Moment-DETR~\cite{momentdetr} further spotlighted the topic by proposing a QVHighlights dataset that enables the model to perform both tasks, retrieving the moments with their highlight-ness, simultaneously.

% 원준 원본
% To detect the desired moments, previous works employed transformer encoder-decoder architectural designs to fuse the text query into the video representations. Moment-DETR~\cite{mDETR} modified detection transformer to process capture the moment as a set, and UMT~\cite{umt} implemented transformer decoder as to output clip-wise saliency. 
% Yet to their outstanding breakthroughs in the literature of moment retrieval with the seminal architectures, their limitation is that the role of the given text query is insignificant in representing the query-conditioned video representation; the attention mechanism of moment DETR is not explicitly conditioned on the text query, and the text query is conditioned on multi-modal clips where the differences between the clips are smoothed after encoding process in UMT.



% \begin{figure}[t]
% \centering
%     \begin{subfigure}[l]{0.37\linewidth}
%       \centering
%       \vspace{0.20cm}
%     %   \includegraphics[width=1\linewidth]{figs/fig_1_moti_textattn.pdf}  
%     %   \includegraphics[width=1\linewidth]{figs/fig_1_moti_textattn_v2.pdf}  
%       \includegraphics[width=1\linewidth]{figs/fig1_moti_violin_a.pdf}  
%       \vspace{-0.60cm}
%     %   \caption{text attention}
%         \caption{Importance of queries in video representation}
%       \label{fig:fig1_text_attn}
%     \end{subfigure}%\hfill% or  or \hspace{0.3\textwidth}
%     \vspace{0.2cm}
%     \begin{subfigure}[r]{0.61\linewidth}
%       \centering
%     %   \includegraphics[width=1\linewidth]{figs/fig1_moti_negattn.pdf}  
%       \includegraphics[width=1\linewidth]{figs/fig1_moti_violin_b.pdf}  
%     %   \caption{neg attention}
%         % \caption{Relation between the highlight-ness and the relevance between videos and query texts.}
%         \caption{Highlight-ness~(saliency) histogram of positive and negative video-query pairs\SE{}}
%       \label{fig:fig1_neg_attn}
%     \end{subfigure}%\hfill% or  or \hspace{0.3\textwidth}
%     % \vspace{-0.2cm}
%     \caption{Overall statistics for attention scores in Fig.~\ref{fig:motivation_ex} in QVHighlights dataset. 
%     (a) For the attention scores that measure how much the text query is generally involved in video representation, we use violin plots to show the probability density. We plot the score for each layer in the encoder.
%     % (b) Using the histogram, we compare how the baseline and QD-DETR yield different salient scores given the positive and negative video-text pairs.
%     (b) Saliency histogram shows the distributional gap between positive and negative video-text query pairs of baseline~(Moment-DETR) and proposed QD-DETR.\SE{}
%     }
%     \label{fig:motivation}
%     % \captionsetup{belowskip=13pt}
%     % \setlength{\belowcaptionskip}{-10pt}
% \end{figure}

% \begin{figure}[t]
% \centering

%     \begin{subfigure}[r]{1\linewidth}
%       \centering
%       \hspace{-0.2cm}
%     %   \includegraphics[width=1\linewidth]{figs/fig1_moti_negattn.pdf}  
%       \includegraphics[width=1.1\linewidth]{figs/fig1_moti_violin_a_v2.pdf}  
%     %   \caption{neg attention}
%         % \caption{Relation between the highlight-ness and the relevance between videos and query texts.}
%         \vspace{-0.5cm}
%         % \caption{Saliency histogram of positive and negative video-query pairs}
%         \caption{We plot the histograms and its average value~(dotted line) to compare saliency scores when true and false text queries are given for each method. (left) Since the video representations do not include much textual information, both the true and false queries yield similar saliency scores. (Middle) Even when the video representation is enforced to be updated with the textual information, the issue is not much resolved. (Right) By extracting discriminative features in the text query, distributions are differentiated.
%         % \SE{} % R1@0.5 설명
%         Also, R1@0.5 indicates evaluation metric, Recall at 1 with IoU 0.5 threshold on QVhighlight \textit{val} set.
%         }
%       \label{fig:fig1_neg_attn}
%     \end{subfigure}%\hfill% or  or \hspace{0.3\textwidth}
%     \\
%     \begin{tabular}{cc}
%     \hspace{-0.2cm}
%         \begin{minipage}{.4\linewidth}
%             \begin{subfigure}[l]{1\linewidth}
%               \centering
%             %   \vspace{0.20cm}
%             %   \includegraphics[width=1\linewidth]{figs/fig_1_moti_textattn.pdf}  
%             %   \includegraphics[width=1\linewidth]{figs/fig_1_moti_textattn_v2.pdf}  
%               \includegraphics[width=1\linewidth]{figs/fig1_moti_violin_a.pdf}  
%               \vspace{-0.60cm}
%             %   \caption{text attention}
%                 \caption{Importance of queries in video representation}
%               \label{fig:fig1_text_attn}
%             \end{subfigure}%\hfill% or  or \hspace{0.3\textwidth}
%         \end{minipage}
        
%         \begin{minipage}{.6\linewidth}
%             \vspace{-0.2cm}
%             \caption{Overall statistics of Fig.~\ref{fig:motivation_ex} in QVHighlights dataset. 
%             (a) Saliency histogram shows the distributional gap between positive and negative video-text query pairs.
%             % (a) For the attention scores that measure how much the text query is generally involved in video representation, we use violin plots to show the probability density. We plot the score for each layer in the encoder.
%             % (b) Using the histogram, we compare how the baseline and QD-DETR yield different salient scores given the positive and negative video-text pairs.
%             % (b) Text ratio in self-attention layer to  of Moment-DETR
%             % (b) Ratio of text when representing video tokens in self-attention of Moment-DETR.
%             % (b) Magnitude of attention text query involved.
%             % (b) Attention score of video tokens
%             % (b) Magnitude of text query to refine the video tokens in self-attention layer of Moment-DETR.
%             (b) Probability density depicting the weight of the text query in attention score for video clips. Scores are from the self-attention layers in Moment-DETR encoder.
%             % (b) The text query ratio in attention score of video clips (Self-attention layer in Moment-DETR encoder). We use violin plots to show probability density.
%             % 텍스트 쿼리가, 비디오 피쳐에 얼만큼 attend 하는지
%             }
%         \end{minipage}
    
%     \end{tabular}
%     \vspace{-0.5cm}
%     \label{fig:moti}
%     % \captionsetup{belowskip=13pt}
%     % \setlength{\belowcaptionskip}{-10pt}
% \end{figure}


% \begin{figure}
%     \centering
%     % \includegraphics[width=1\linewidth]{figs/fig1_moti_negattn_1109.pdf}
%     \includegraphics[width=1\linewidth]{figs/fig1_moti_negattn_stat_v2.pdf}
%     \vspace{-0.8cm}
%     \caption{
%         Histogram of saliency when the positive and negative queries are given. We plot the histograms and its average value~(dotted line) to compare saliency scores when relevant~(positive) and irrelevant~(negative) text queries are given for each method. (Left) Since the video representations do not properly reflect textual information, both the positive and negative queries yield similar saliency scores. 
%         % (Middle) Even when the video representation is enforced to be updated with the textual information, the issue is not much resolved. 
%         (Right) By representing video clips in query-dependent manner, distributions are differentiated.
%     }
%     \vspace{-0.6cm}
%     \label{fig:motivation}
% \end{figure}


% One of the demanding task is moment retrieval task, which is detecting the desired moments from the given query, typically the text query.
When describing the moment, one of the most favored types of query is the natural language sentence~(text)\cite{anne2017localizing}. 
While early methods utilized convolution networks~\cite{zhang2020learning, gao2021fast, wang2020temporally}, recent approaches have shown that deploying the attention mechanism of transformer architecture is more effective to fuse the text query into the video representation.
% To handle these modalities, previous works simply employed the attention mechanism of transformer architecture to fuse the text query into the video representation.
For example, Moment-DETR~\cite{momentdetr} introduced the transformer architecture which processes both text and video tokens as input by modifying the detection transformer~(DETR), and UMT~\cite{umt} proposed transformer architectures to take multi-modal sources, e.g., video and audio. 
Also, they utilized the text queries in the transformer decoder.
Although they brought breakthroughs in the field of MR/HD with seminal architectures, they overlooked the role of the text query.
To validate our claim, we investigate the Moment-DETR~\cite{momentdetr} in terms of the impact of text query in MR/HD~(Fig.\ref{fig:motivation_ex}).
Given the video clips with a relevant positive query and an irrelevant negative query, we observe that the baseline often neglects the given text query when estimating the query-relevance scores, i.e., saliency scores, for each video clip.
% the output saliency score, i.e. query-relevance scores.
% Based on the observation, we traced the actual saliency prediction of the model against both the video-relevant query and the irrelevant dummy one where we find that the baseline often neglects the given text query when estimating the query-relevance scores of video clips.
% For example, in Fig.~\ref{fig:motivation_ex}, saliency scores are not affected even when the query is substituted with the dummy.
% % General statistics for Fig.~\ref{fig:motivation_ex} is shown in Fig.~\ref{fig:motivation}. 
% General statistics corresponding to Fig.~\ref{fig:motivation_ex} are also shown in Fig.~\ref{fig:motivation}.



% The limitation of the concrete baseline~\cite{momentdetr} is inspected in two different aspects; 1) Utilization of text-query in the encoding process and 2) the output saliency score, i.e. query-relevance scores.
% Firstly, we visualize the attention score when video clips are given as a query in self-attention. 
% We observe that the text queries have relatively small impacts compared to other video features, as shown in Fig.~\ref{fig:fig1_text_attn_ex}.
% That is, the text does not account for much in representing every video clip, although the goal of MR/HD is to detect query-relevant moments.
% Based on the observation, we traced the actual saliency prediction of the model against both the video-relevant query and the irrelevant dummy one where we find that the baseline often neglects the given text query when estimating the query-relevance scores of video clips.
% For example, in Fig.~\ref{fig:motivation_ex}, saliency scores are not affected even when the query is substituted with the dummy.
% % General statistics for Fig.~\ref{fig:motivation_ex} is shown in Fig.~\ref{fig:motivation}. 
% General statistics are also shown in Fig.~\ref{fig:motivation}.

% Consequently, in Fig.~\ref{fig:fig1_neg_attn_ex}~(b), we found that the baseline often neglects the given text query when estimating the query-relevance scores of video clips; 
% For example, 


% We validate the previous work sometimes neglects the given query when estimating the saliency of video clips.
% For example, there is an example that the saliency scores from positive and negative queries cannot be distinguishable, as shown in Fig.~\ref{fig:fig1_neg_attn_ex}.
% % 우리는 추가로 text attention을 추가도 해봤지만, 효과가 있긴 했으나, still 이슈가 있는 것을 확인하였다?
% % Still, we observe that assuring the high attendance of text queries does not resolve the overlap which motivates us to question the quality of the naive use of task-agnostic text representation~\cite{momentdetr, umt}.
% We found that introducing the text-attention for ensuring the high attendance of text queries relieve the overlap, but there still be a severe overlap.


% To validate their limitations, we inspect the impacts of text queries in the concrete baseline~\cite{momentdetr} with the two different aspects, 1) tendency of attention in self-attention layer and 2) saliency score, i.e. query-relevance scores. \SE{} % attention 이 갑자기 등장하는가?
% Firstly, we visualize the attention score when video clips are given as a query in self-attention. We observe the text queries have relatively low attention scores compared to the video features, as shown in Fig.~\ref{fig:fig1_text_attn_ex}.
% That is, the text does not account for much in representing every video clip, although the goal of MR/HD is to detect query-relevant moments.
% Based on this observation, we trace the actual saliency prediction of the model against both positive and negative text queries.
% We validate the previous work sometimes neglects the given query when estimating the saliency of video clips.
% For example, there is an example that the saliency scores from positive and negative queries cannot be distinguishable, as shown in Fig.~\ref{fig:fig1_neg_attn_ex}.
% % 우리는 추가로 text attention을 추가도 해봤지만, 효과가 있긴 했으나, still 이슈가 있는 것을 확인하였다?
% % Still, we observe that assuring the high attendance of text queries does not resolve the overlap which motivates us to question the quality of the naive use of task-agnostic text representation~\cite{momentdetr, umt}.
% We found that introducing the text-attention for ensuring the high attendance of text queries relieve the overlap, but there still be a severe overlap.



% Thus, we 
% query dependency를 높이기 위해 
% Cross-attention? text-attention? detailed explanation on text-attention should be needed?
% By handling these two issues, we find that more precise retrieval can be achieved.
% 
% 
%
% By projecting video-discriminative text features with high text attendance to source video, we f 
% We also find the need to improve the quality of query features since assuring high text attendance also results in...
% pairs are not finetuned to be discriminative that even the similarity within the pairs does not reflect the relevance between the query and the video clips.
% General statistics for Fig.~\ref{fig:motivation_ex} is shown in Fig.~\ref{fig:motivation}. 
% \SE{} % 이거 ??로 뜨는데, 위처럼 figure 그리면 label이 안되는걸까요
% \SE{}
% 형님 아래 사항 생각 좀 해보는게 좋을 거 같아요.
% fig 1. (a) 그림만 봤을 때 모든 clip에 대해 text attention이 일정이상 존재하긴 하니까, 뭔가 not assured to be conditioned가 와닿지 않는거 같아요.
% + 왜 text가 항상 attend 해야하나?
% not assured to be conditioned --> text shows relatively low affects compared to video 같이 실제 나타난 현상까지 같이 적으면 어떨까 싶어요.
% fig 1. (b) 덜 반영한다?

% \SU{}
% 일단 text가 attend 잘 되어야 한다는 것에 좀 궁금점이 생깁니다. 결국에는 text와 관련있는 frame들을 attend해서 higlight를 찾아야 하는게 아닐까요? 그리고, 현제 저희의 모델 구조상 text query가 Key와 Value로 거의 활용되고 있는데 그렇다면 결국에는 해당 모델은 text에 대한 attention이 전혀 없다고 봐도 무방하지 않을까요? 그런 면에서 text attention을 강조하는게 좀 걸리긴 합니다.

% Specifically, the text query is not assured to be explicitly conditioned on every clip of the video, and as the query texts are evenly treated, discriminative keywords may not be spotlighted.
% attention mechanism of Moment-DETR is not explicitly conditioned on the text query as shown in Fig~\ref{}(d), and in UMT, the text are only used for conditioning the queries while the video representation are refined itself by self-attention.

% \begin{figure}[t]
%     \begin{subfigure}{1\linewidth}
%       \centering
%     %   \includegraphics[width=1\linewidth]{figs/fig_1_moti_textattn.pdf}  
%     %   \includegraphics[width=1\linewidth]{figs/fig_1_moti_textattn_v2.pdf}  
%       \includegraphics[width=1\linewidth]{figs/fig_1_moti_textattn_v4.pdf}  
%       \vspace{-0.5cm}
%     %   \caption{text attention}
%         \caption{Distribution of attention scores in Moment-DETR encoder}
%       \label{fig:fig1_text_attn}
%     \end{subfigure}%\hfill% or  or \hspace{0.3\textwidth}
%     \vspace{0.2cm}
%     \begin{subfigure}{1\linewidth}
%       \centering
%     %   \includegraphics[width=1\linewidth]{figs/fig1_moti_negattn.pdf}  
%       \includegraphics[width=1\linewidth]{figs/fig1_moti_negattn_v2.pdf}  
%       \vspace{-0.5cm}
%     %   \caption{neg attention}
%         \caption{Saliency score against positive and negative text queries}
%       \label{fig:fig1_neg_attn}
%     \end{subfigure}%\hfill% or  or \hspace{0.3\textwidth}
%     \vspace{0.2cm}
%     \begin{subfigure}{1\linewidth}
%       \centering
%     %   \includegraphics[width=1\linewidth]{figs/fig1_moti_violin.pdf}  
%       \includegraphics[width=1\linewidth]{figs/fig1_moti_violin_v2.pdf}  
%       \vspace{-0.5cm}
%       \caption{violin}
%       \label{fig:fig1_violin}
%     \end{subfigure}%\hfill% or  or \hspace{0.3\textwidth}
%     \vspace{-0.2cm}
%     \caption{(a) 1. portion of text attention vs. video attention 2. relation with text query and content (e.g. fg, bg) of clip seems not to affect the attention score
%     (b) 1. high variability even though entire clips are highly correlated with the given text query 2. positive and negative query makes overlaps on saliency score distribution
%     (3) actual distribution on validation dataset.}
%     \label{fig:motivation}
%     % \captionsetup{belowskip=13pt}
%     % \setlength{\belowcaptionskip}{-10pt}
% \end{figure}

To this end, we propose Query-Dependent DETR~(QD-DETR) that produces query-dependent video representation.
% Our key focus is to ensure each clip in predicted moments is explicitly conditioned by the query, particularly on the video-descriptive portion of the text query.
% Our key focus is to ensure that query-relevant clips are predicted by enforcing each clip to be explicitly conditioned by the query.
%Our key focus is to ensure that the model prediction for each clip is highly relevant to the query.
Our key focus is to ensure that the model's prediction for each clip is highly dependent on the query.
% by enforcing each clip to be explicitly conditioned by the query. :)
% hmm...
% \SE {} % "query-relevant clips are predicted" 이 문장이 좀 애매한거 같습니다. relevant 클립을 놓지지 않고 찾는 것을 보장한다? 이런 느낌인지 아니면 높은 saliency 를 주는게 목적이다? model prediction이 query-relevance를 반영하는 것을 보장한다?
% Our key focus is to ensure that the model prediction reflects query-relevance of clips by enforcing each clip to be explicitly conditioned by the query.
First, to fully utilize the contextual information in the query, we revise the transformer encoder to be equipped with cross-attention layers at the very first layers.
% 상익's thought :  single video - query간의 관계만 고려 - 같은 word가 더 많이 쓰이는 것을 보고 
% 교수님's thought : neg pair 를 쓰면 쿼리를 보지 않고서는 video clip간만 고려하는 것이 사라짐. 왜냐면 0으로 내보내야 하기 때문. --> SE: relative difference 만 고려하다가, 
By inserting a video as the query and a text as the key and value of the cross-attention layers, our encoder enforces the engagement of the text query in extracting video representation.
% 원준 교수님 코멘트 반영해서 다시
Then, in order to not only inject a lot of textual information into the video feature but also make it fully exploited, we leverage the negative video-query pairs generated by mixing the original pairs.
Specifically, the model is learned to suppress the saliency scores of such  negative~(irrelevant) pairs.
Our expectation is the increased contribution of the text query in prediction since the videos will be sometimes required to yield high saliency scores and sometimes low ones depending on whether the text query is relevant or not.
% \SE{}
% learns to?
% By suppressing the saliency scores of the irrelevant video-query pairs, the model learns to spotlight only the video-specific discriminative words in the query.
% % \SE{} % ====================== 상익 수정 ========================
% However, this architectural design still lacks the capability of identifying the video-descriptive keywords in the query.
% % However, this architectural design still lacks in identifying proper query relevance.
% This is because the current training scheme only focuses on the interactions of video and clips within a single video while neglecting information shared throughout the entire video.
% % We argue the problem of the current training scheme that only focuses on distinguishing the clips in a single video while neglecting information shared throughout the entire video.
% Therefore, we leverage the negative video-query relationships to enhance the capability of identifying the contextual similarity of query and video clips.
% 
% 원준 원본 
% However, this architectural design heavily relies on the quality of the text query.
% Therefore, we leverage the negative video-query relationships to enable the model to emphasize key corresponding query features.
% By suppressing the saliency scores of the irrelevant video-query pairs, the model learns to spotlight only the video-specific discriminative words in the query.
% =========================================================
Lastly, to apply the dynamic criterion to mark highlights for each instance, we deploy a saliency token to represent the entire video and utilize it as an input-adaptive saliency criterion. 
With all components combined, our QD-DETR produces query-dependent video representation by integrating source and query modalities.
This further allows the use of positional queries~\cite{dabdetr} in the transformer decoder.
% Furthermore, we can exploit the advanced DETR decoder architectures using the positional information, e.g., DAB-DETR, since our encoded tokens consist of identical position representations from a single modality.
% \SE{} % ====================== 상익 수정 ========================
% Furthermore, we can exploit the advanced DETR decoder architectures using the positional information, e.g., DAB-DETR, since our video clip tokens consist of identical position representations from a single modality.
% 원준 원본
% It also enables the use of advanced DETR decoder architectures, e.g., DAB-DETR, for the first time, as these works exploit the position information within a single modality.
% =========================================================
Overall, our superior performances over the existing approaches validate the significance of the role of text query for MR/HD.
% Our extensive experiments on QVHighlights, TVSum, and Charades-STA datasets validate the significance of considering the role and the quality of text query.

% All components combined with dynamic anchor moments for the query of decoder, our FOQUE fosters the query-dependent video representation, thereby making the 
% All components combined, our modified transformer encoding process fosters the query-dependent video representation thereby achieving the state-of-the-art results on various benchmarks of moment-retrieval and highlight detection.
	
% -	Video Platform & Streamer & Consumer의 증가. 
% Video는 다른 데이터 타입보다 정보가 많아 유용하지만, 이는 다른 말로 해석하면 video를 보는 것은 time-consuming 하고, 원하는 것을 찾아보기에는 힘들 수 있음.
% 따라서, 많은 매체에서는 사람들의 더 많은 이목을 끌기 위해 highlight 비디오라는 것을 편집하여 공유도 함.
% 하지만, highlight video를 만들기 위해 사람의 노력이 필요한 현 시점에서, This spotlights the need to retrieve the user-requested / Highlight moments in the video.

% -	이전에도 이러한 문제를 해결하기 위해 (asdfasdf) for moment retrieval, (asdfasdf) for highlight detection 등이 제안 되었지만, 이들은 비디오의 특정 영역을 찾는다는 공통된 목적을 가지고 있으면서도, 데이터 셋의 한계로 인해 따로 연구되었음. 이를 문제 삼으며, 최근에는 두 task를 동시에 학습할 수 있는 dataset이 소개 되었는데, 컴퓨터비전에서 최근 각광을 받고 있는 Transformer 모델 도입과 함께 큰 발전을 거듭하고 있음.

% -	구체적으로, 이 두가지 task를 수행하기 위해서는 transformer를 두가지 방법으로 이용할 수 있는데, moment-DETR 처럼 moment 를 clip의 set 단위로 예측할 수 있고, UMT 처럼 clip-wise prediction을 할 수 있음. 하지만, 이들은 query를 condition이 아닌 video와 동등한 레벨로 취급하거나 [mDETR], 매 클립이 self-attention으로 mixing 된 후에 condition을 걸어주어 clip간의 차이를 확실하지 이용하지 못하였고, 또한, 확실하게 condition으로 주지 못하였고, video와 query 사이의 관계를 한정적으로만 이용하였다.

% -	따라서, we explore three different ways to fully exploit query information. First, we design one-way cross-attention layer to condition every clip with the query features. Then, we utilized the negative video-text pairs to better model the relationships between the video and the text embeddings. Lastly, we define the saliency token to be the video-query dependent saliency estimator.


















% ===================== neg pair 부분 ===========================
% Nevertheless, the current training scheme, only considering the given video-query pair, still disturbs the model from identifying proper query-relevance prediction.
% In detail, the model focus on learning the fine-grained discrepancy between video clips, while neglecting the information they share, which contains significant clues to understand the context of video.
% Therefore, we leverage the negative video-query relationships to enhance the capability of identifying the contextual similarity of query and video clips.
% Therefore, we leverage the negative video-query relationships by suppressing those pairs, so that enhance the capability of identifying the contextual similarity of query and video clips.
% We hypothsize the diversity in query-video pairs are insufficient to learn the general relationship between text query and video.
% Therefore, we leverage the negative video-query relationships by suppressing the saliency scores of the irrelevant video-query pairs.
% However, this architectural design still lacks in identifying proper query relevance.
% We argue that the current training scheme only focuses on learning the fine-grained discrepancy between clips in a single video, while neglecting the information they share, which contains significant clues to understand the context of the video.
% Therefore, we leverage the negative video-query relationships to enhance the capability of identifying the contextual similarity of query and video clips.
% However, this architectural design still lacks in identifying proper query relevance.
% We argue the problem of the current training scheme that only focuses on learning the fine-grained discrepancy between clips in a single video.
% That is, the current design neglects the information shared throughout the video, although it contains significant clues to understand the context of the video.


\begin{figure}[t]
\centering
\renewcommand{\thesubfigure}{}
\subfigure[(a) Naive score distillation]
{\includegraphics[width=0.49\textwidth]{fig/fig2_a}}
\label{fig:fig2_a}
\vspace{-5pt}\\
\subfigure[(b) 3D-aware score distillation (Ours)]
{\includegraphics[width=0.49\textwidth]{fig/fig2_b}}
\label{fig:fig2_b}
\vspace{-10pt}\\
    \caption{\textbf{Motivation.} (a) Previous methods~\cite{poole2022dreamfusion,song2020score} only use noisy rendered images and prompt itself for score distillation through diffusion model, resulting in poor 3D coherence. (b) Our \ours addresses this issue and shows robust performance in recovering 3D-consistent scene.}
    \label{fig:fig2}\vspace{-5pt}
\end{figure}



\section{Related Work}
\label{sec:related_work}
\subsection{Co-Speech Gesture Synthesis}
The early approaches for generating co-speech gestures often involve creating linguistic rules to translate speech input into a sequence of pre-collected gesture segments, which are typically referred to as rule-based methods \cite{cassell1994rulefullbody,cassell2001beat,kipp2004gesture,kopp2006bml}. \citet{wagner2014rulereview} provide a comprehensive review of these methods. Rule-based methods produce interpretable and controllable results, but creating gesture datasets and rules requires significant effort. To alleviate the manual effort of designing rules in rule-based methods, data-driven approaches have gradually become predominant in this field. \citet{nyatsanga2023data_driven_gesture_survey} offer a thorough survey of these methods. Early data-driven approaches aim to directly learn mapping rules from data through statistical models \cite{neff2008videogesture,levine2009prosodygesture,levine2010gesturecontroller} and combine them with predefined gesture units for gesture generation. Later, the powerful modeling capability of deep neural networks makes it possible to train complex end-to-end models using raw speech-gesture data directly. One option is deterministic models, such as MLP \cite{kucherenko2020gesticulator}, CNN \cite{habibie2021videogesture}, RNN \cite{yoon2019robot,yoon2020trimodalgesture,bhattacharya2021affectivegesture,liu2022hierarchicalgesture}, and Transformer \cite{bhattacharya2021text2gestures}. Another choice is generative models, including flow-based models \cite{alexanderson2020stylegesture,ye2022styleflowgesture}, VAEs \cite{li2021audio2gesture,ghorbani2022zeroeggs}, and VQ-VAE \cite{yi2022talkshow,yazdian2022gesture2vec,liu2022vqgesturevideo}. Due to the inherent many-to-many relationship between speech and gesture, end-to-end models can generate natural-looking gestures but face challenges in ensuring content matching between speech and generated gestures \cite{yoon2022genea}. To address this issue, some neural systems aim to explicitly model both rhythm and semantics from the perspective of model structure \cite{kucherenko2021speech2properties2gestures,ao2022rhythmicgesticulator,liu2022disco} or training supervision strategy \cite{liang2022seeg}. Furthermore, hybrid systems, such as the combination of deep features and motion graphs \cite{zhou2022gesturemaster}, have been proposed to harness the advantages of different approaches. Recently, diffusion models \cite{sohldickstein2015diffusion,song2020improvedscore,ho2020ddpm} have demonstrated impressive results in image synthesis \cite{ramesh2022dalle2} and human motion generation \cite{tevet2022humanmotiondiffusion, zhang2022motiondiffuse}. Inspired by these works, our system adapts the latent diffusion model \cite{rombach2022latentdiffusion} for the co-speech gesture generation task and achieves appealing results.

\subsection{Style Control for Human Motion}
A typical approach to style control for human motion involves specifying a motion clip as a reference and transferring the reference clip's style to the source motion. This task is also known as \emph{style transfer}. Early works in motion style transfer integrate traditional machine learning techniques with manually defined features to infer motion styles \cite{hsu2005motion_style_translation,ma2010motion_style_transfer,xia2015realtime_motion_style_transfer,yumer2016spectral_motion_style_transfer}. Recently, deep learning-based methods have significantly enhanced motion quality. \citet{holden2016deepmotion} first propose a learning framework enabling motion style control through optimization in the motion manifold space. \citet{du2019stylemotioncvae} improve transfer efficiency by training a conditional VAE. \citet{mason2018few-shot_motion_style_transfer} use few-shot learning to generate stylized locomotion. \citet{aberman2020adain} employ a temporally invariant adaptive instance normalization (AdaIN) layer for target style injection, eliminating the need for paired data during training. \citet{wen2021stylemotionflow} achieve unsupervised style transfer using a flow model. \citet{jang2022motionpuzzle} introduce a method capable of controlling styles for individual body parts.

Previous co-speech gesture synthesis systems with style control can be categorized based on whether or not they require style labels. For methods needing labeled data, early works can only learn an individual style for one generator \cite{levine2010gesturecontroller,neff2008videogesture,ginosar2019stylegesture}. \citet{ahuja2022lowresource} propose a strategy that efficiently adapts the source generator to another speaker style using low-resource data. Some works learn a speaker style embedding space with labeled speaker-motion data, enabling gesture style control by sampling from this space \cite{ahuja2020stylegesture,yoon2020trimodalgesture,bhattacharya2021affectivegesture}. \citet{alexanderson2020stylegesture} aimat controlling fine-grained styles, such as gesturing speed and spatial scope, using preprocessed control signal-motion data. Their later work \cite{alexanderson2022diffusiongesture} utilizes a diffusion model for audio-driven motion synthesis, achieving label-based style control by training the model on labeled data. For methods not requiring style labels, \citet{habibie2022motionmatching} propose a motion matching framework to achieve flexible style control. Other studies achieve arbitrary style control by imitating an example given as a video \cite{liu2022hierarchicalgesture} or a motion clip \cite{ghorbani2022zeroeggs,ye2022styleflowgesture,kuriyama2022tokenizedgestures}.  In this work, we utilize a CLIP-based encoder to extract a style embedding from an arbitrary text prompt and incorporate it into the generator via an AdaIN layer, guiding the synthesis of stylized gestures. Our system supports fine-grained multimodal style prompts as opposed to label-based style control. It employs a self-supervised learning scheme and eliminates the need for labeled data. Additionally, we use an autoregressive model rather than a parallel model, making it potentially suitable for real-time applications.


\section{Preliminaries}
\section{Preliminaries}\label{sec:preliminaries}

%We leverage the principle of the normalized cut algorithm~\cite{shi2000normalized_cut} (NCut) to identify potential instances for 3D pseudo masks, lifted to the high-dimensional 3D scenario by using a geometric primitive basis, as discussed in Section~\ref{sec:oversegmentation}.
\NEW{We employ the principle of the NCut approach for pseudo-generation similarly to \cite{wang2023cut}, but lift it to support high-resolution 3D segmentation by operating on segment-level geometric primitives from self-supervised 2D and 3D features.}

NCut maximizes similarities within partitions and dissimilarities across partitions by minimizing the cost of a graph cut. This is formalized as: 
%
\begin{equation}
    NCut(A,B) = \frac{cut(A,B)}{assoc(A,V)} + \frac{cut(A,B)}{assoc(B,V)},
\end{equation}
where $A$ and $B$ are disjoint bipartitions from a full graph $V$, $cut$ measures the degree of dissimilarity computed as the total weight of edges that have been removed, and  $assoc$ represents the total connection within the partition.
Normalizing the cut cost function with the size of the partitions can solve the problem of single outlier node removal, which is one of the biggest difficulties of other graph cut algorithms~\cite{244673}.

While the minimum solution of this problem is intractable for practical applications, it can be rewritten as a generalized eigenvalue problem with adjacency matrix $W$ and degree matrix $D$, where $D(i,i) = \Sigma_jW(i,j)$:
%
\begin{equation} \label{eq:general_eigenval}
    (D-W)v = \lambda D v,
\end{equation}
%
Finding the second smallest eigenvalue $\lambda$  and its corresponding eigenvector $v$ is a close approximation for the minimized cost. 
From $v$, we obtain foreground separation by taking all node activations where the eigenvector components were larger than their mean. 
This method has been shown to be effective on intensity images, but combining it with deep features has demonstrated even stronger potential in the image domain \cite{wang2022tokencut,lis2022attentropy,wang2023cut}. 
While this multiple foreground objects could be directly predicted with a single pass by taking the eigenvectors in order, it was shown in \cite{wang2022tokencut} that a greedy iterative approach produces better results.




\begin{figure*}[t]
\begin{center}
\includegraphics[width=1\textwidth,]{fig/fig3.pdf}
\end{center}
\vspace{-15pt}
\caption{\textbf{Overall architecture of \ours.} In the framework, semantic code is sampled to reduce the text prompt ambiguity by generating an image based on the text prompt and then optimizing the prompt's embedding to match the generated image. Our consistency injection module receives this semantic code to synthesize view-specific depth maps as a condition to the diffusion U-net. The module also consists of a sparse depth injector to implicitly incorporate 3D awareness by utilizing an external 3D prior, and LoRA~\cite{hu2021lora} layers to maintain semantic consistency. 
}
\label{fig:network_overall}
\end{figure*}


\section{Method}
\subsection{Motivation and overview}

The score distillation-based text-to-3D methods~\cite{poole2022dreamfusion,lin2022magic3d,wang2022score} assume that maximizing the likelihood of images rendered from arbitrary viewpoints of a NeRF can be translated as maximizing the likelihood of the overall NeRF. Although this is a reasonable assumption, 2D diffusion models lack 3D awareness, which leads to inconsistent and distorted geometry of generated NeRF. To overcome this challenge and ensure NeRF's 3D-consistency, we incorporate 3D awareness into the diffusion model.

Previous works~\cite{poole2022dreamfusion, wang2022score} attempt this by using the text prompts that roughly describe the camera viewpoint (\textit{e.g.}, ``\textit{side view}''). However, this ad-hoc approach is severely limited: the ambiguity caused by the same text prompt representing a wide range of different pose values leaves NeRF generation vulnerable to geometric inconsistencies. A diffusion model directly conditioned on camera pose value $\pi$ would be an ideal solution; however, this is not feasible due to the ambiguity in defining a canonical space for each 3D scene, and the difficulty of acquiring camera pose data.

To overcome these limitations, we present \ours, our novel framework for effectively incorporating 3D awareness into pretrained text-to-image diffusion models~\cite{ho2020denoising, rombach2022high}. Instead of having the diffusion model explicitly model camera pose $\pi$, our method constructs a coarse point cloud of an initially generated image $\hat{x}$ through an off-the-shelf model~\cite{nichol2022point, wu2023multiview} and gives its viewpoint-specific depth map as a condition for the diffusion model. As the sparse depth map contains rich 3D information describing the scene from a given viewpoint, this approach effectively enables the diffusion model to generate NeRF in a 3D aware manner. In addition, to ensure the semantic similarity of NeRF across viewpoints, we introduce semantic code sampling, which constrains the entire 3D scene to a single semantic identity. 

In the following, we describe our proposed methodology in detail. Our overall architecture is described in Fig.~\ref{fig:network_overall}.

\subsection{Semantic code sampling}
\label{method:semcode}
The task of text-to-3D generation comes with a problem of inherent ambiguity within the text prompt. For instance, the text prompt ``\textit{a cute cat}'' has color ambiguity, as it could refer to either a black or white cat. This ambiguity leads to the freedom to generate any image within this range, which harms the quality and coherence of the generated NeRF. Each score distillation step may guide NeRF toward widely different textures and semantics, which could result in a lack of coherence in the generated output. In Sec.~\ref{abl:semcode}, we provide a detailed demonstration of this phenomenon.

We introduce a simple yet effective technique to counter this text prompt ambiguity, which we call semantic code sampling. To specify the semantic identity of the scene we optimize and thus reduce ambiguity, we first generate a 2D image $\hat{x}$ from the text prompt $c$. Then, we optimize the text prompt embedding $e$ to better fit the generated image, similarly to the textual inversion~\cite{gal2022image}:
\begin{equation}
{e}^* = \underset{e}{\mathrm{argmin}} \ || {\epsilon}_{\theta}(\hat{x}_t,{e}) - {\epsilon} ||^2_2,
 \label{equation:embedding_optimize}
\end{equation}
where $\hat{x}_t$ is a noised image of the generated image $\hat{x}$ with the noise $\epsilon$ and the noise level $t$. 

We refer to the pair of the generated image $\hat{x}$ and the optimized embedding ${e}^*$ as semantic code ${s}$, \textit{i.e.}, $s := (\hat{x}, {e}^*)$, which are the inputs for our consistency injection module. 

\subsection{Incorporating a coarse 3D prior}
\label{method:3dprior}

Our approach aims to incorporate 3D awareness into pre-trained 2D diffusion models, and to achieve this we construct a coarse 3D representation of a given initial image and project it to a target viewpoint to make a sparse depth map. This sparse depth map is leveraged in our consistency injection module as a condition for 3D awareness.
         
Specifically, an off-the-shelf model ${D}(\cdot)$ receives an image as input and outputs a coarse 3D representation, which is a sparse 3D point cloud in our architecture.  We can choose ${D}(\cdot)$ from a wide variety of models: it could be a point cloud generative model such as Point-E~\cite{nichol2022point} or a single-image reconstruction model such as MCC~\cite{wu2023multiview}. Using them as 3D priors, we construct a sparse point cloud and project it to get a sparse depth map ${P}$ corresponding to the camera pose $\pi$:
\begin{align}
 {P} = \mathcal{P}({D}(\hat{x}),\pi),
 \label{equation:projection}
\end{align}
where $\mathcal{P}(\cdot)$ is a depth-projection function. Subsequently, adopting the architecture of ControlNet~\cite{zhang2023adding}, our sparse depth injector $E_\phi$ receives the sparse depth map $P$, and its output features are added to the intermediate features within diffusion U-net of $\epsilon_\theta \bigl(\hat{x}_t,{e}^*\bigr)$, which can be further formulated such that $\epsilon_\theta \bigl(\hat{x}_t,{e}^*;E_\phi(P) \bigr)$.

This approach brings significant advantages to score distillation-based NeRF generation. Unlike previous methods that directly optimize NeRF using only global text prompts, our \ours conditions the 3D optimization explicitly on the semantic code and its view-specific depth map. Not only does this enhance the 3D-consistency and fidelity of NeRF as intended, but it also encourages the 3D scene to be faithful to the semantic code, ensuring both geometric and semantic robustness of the generated 3D scene.  

\subsection{Training the sparse depth injector}
\label{method:sparsedepth}
The point cloud obtained by the off-the-shelf 3D model inevitably contains errors and artifacts. It naturally causes its depth map to also have artifacts, as shown in Fig.~\ref{fig:depthcomp}(a). Therefore, our module must be able to handle both sparse geometry and the errors of the projected depth map.

To this end, we employ two training strategies for our sparse depth injector ${E}_\phi(\cdot)$. First, we train our injector using sparse depth maps, acquired by projecting point clouds from a point cloud dataset~\cite{reizenstein2021common} to known viewpoints. By training our module with sparse depth map–image pairs, our model learns to interpolate and infer dense structural information from sparse depth. We impose strong augmentations on the point cloud data by randomly subsampling from it and adding randomly generated noisy points, which increases the robustness of our model against errors and noises present in the predicted sparse depth map. The texts for the corresponding images are obtained using the image caption model~\cite{dosovitskiy2020image,kumar2022imagecaptioning}.

Second, the injector ${E}_\phi(\cdot)$ is also trained on predicted dense depth maps of text-to-image pairs, acquired using MiDaS~\cite{ranftl2020towards}. This strengthens of model’s generalization capability, enabling it to infer structural information from categories that were not included in the 3D point cloud dataset for sparse depth training.


Combining the two, given the depth map $P$ along with the corresponding image $y$ and caption $c$, the training objective of the depth injector $E_\phi$ is as follows:
\begin{equation}
    \mathcal{L}_\mathrm{inject}(\phi)= \mathbb{E}_{y,c,P,t,\epsilon}\Bigl[ || \epsilon_\theta \bigl(y_t,c;E_\phi(P) \bigr)-\epsilon ||^2_2 \Bigr],
\end{equation}
which is similar to Eq.~\ref{equation:diffloss}, but only tunes the depth injector $E_\phi$ while the diffusion model remains frozen.
These training strategies enable our model to receive sparse and noisy depth maps directly as input and successfully infer dense and robust structural information from them, without needing any auxiliary depth completion network. Fig.~\ref{fig:depthcomp} illustrates the effectiveness of our approach: our approach successfully generates realistic results without being restricted to the domain of the point cloud dataset. Note that the owl and eagle used in the illustration of Fig.~\ref{fig:depthcomp} are not included in the category of the point cloud dataset~\cite{reizenstein2021common} we use.

\begin{figure}[t]
    \centering
    \small
    \setlength\tabcolsep{0.8pt}
    {
    % \renewcommand{\arraystretch}{0.5}
    % \resizebox{\columnwidth}{!}{%
    \begin{tabular}{cccc}
     \includegraphics[width=0.249\linewidth]{fig/depthcomp_a} &
      \includegraphics[width=0.249\linewidth]{fig/depthcomp_b} &
      \includegraphics[width=0.249\linewidth]{fig/depthcomp_c} &
      \includegraphics[width=0.249\linewidth]{fig/depthcomp_d} \\
      \includegraphics[width=0.249\linewidth]{fig/depthcomp_a2} &
      \includegraphics[width=0.249\linewidth]{fig/depthcomp_b2} &
      \includegraphics[width=0.249\linewidth]{fig/depthcomp_c2} &
      \includegraphics[width=0.249\linewidth]{fig/depthcomp_d2} \\
       (a) & (b) & (c) & (d)\\
    \end{tabular} }
    % }
    \vspace{-10pt}
    \caption{\textbf{Qualitative results conditioned on the sparse depth map.} Given sparse depth maps in (a), (b) are the results of depth-conditional Stable Diffusion, (c) are the results of ControlNet~\cite{zhang2023adding} trained on MiDaS~\cite{ranftl2020towards} depths only, and (d) are our \ours results. The given text prompts are ``\textit{a front view of an owl}'' and ``\textit{a majestic eagle}''.}
    \label{fig:depthcomp} 
    \vspace{-5pt}
\end{figure}

\begin{figure*}[t]
\begin{center}
\includegraphics[width=0.89\linewidth]{fig/Arxiv_qual_comparison.pdf}
\end{center}
\vspace{-10pt}
\caption{\textbf{Qualitative comparisons for text-to-3D generation.} We compare our approach with Stable-DreamFusion~\cite{poole2022dreamfusion,stable-dreamfusion} and SJC~\cite{wang2022score}. Notice that all the results are rendered with a fixed random seed for a fair comparison. The geometric consistency of our approach can also be found at the \textbf{video results} provided in the project page.}
\label{qual_textto3d}
% \vspace{-5pt}
\end{figure*}

\subsection{Pivotal tuning for semantic consistency}
\label{method:pivotal}
To accomplish our objective, the diffusion model should produce a score that generates identical objects as much as possible from different camera poses, based on a semantic code. Although optimized embedding ${e}^*$ preserves the semantics, we further enhance this by adopting LoRA~\cite{hu2021lora} technique motivated by \cite{lora_diff}. LoRA layers $\psi$ consist of linear layers, inserted into the residual path of the attention layers in the diffusion U-net. Specifically, at test time, given an image $\hat{x}$ generated from text prompt $c$, we fix the optimized embedding ${e}^*$ and tune the LoRA layers $\psi$~\cite{roich2022pivotal}:
\begin{equation}
    \mathcal{L}_\mathrm{LoRA}(\psi) =\mathbb{E}_{\epsilon,t}\Bigl[ || \epsilon_\theta(\hat{x}_t,e^*;\psi) - \epsilon ||^2_2 \Bigr].
\end{equation}
Note that we only train the LoRA layers instead of the entire diffusion model to avoid overfitting to a specific viewpoint.


\section{Experiments}

% \begin{table}[t]
%     \tablestyle{2pt}{1.05}
    
%     \centering
%     %\resizebox{1\columnwidth}{!}{
%     \begin{tabular}{@{}l|ccccccc}
%     	\toprule
%             \multicolumn{8}{c}{Untrimmed Spatial-Temporal Grounding}
%     	\toprule
%     	\multicolumn{1}{c}{} & \multicolumn{7}{c}{GroundingYouTube}  \\ 
%     	\cmidrule(lr){2-8} 
%     	\multirow{2}{*}{\textbf{Method}}    & \multirow{2}{*}{IoU+Point} &\multicolumn{6}{c}{mAP}  \\ 
%     	                                    &  & 0.1 & 0.2 & 0.3 & 0.4 & 0.5  & 0.1:0.5 \\ 
%     	\midrule
%     	MIL-NCE \citep{miech2020end} & 4.67 & 33.94 & 25.16 & 12.65 & 3.42 & 0.41  & 15.11 \\
%          CoMMA* \citep{tan2021look}   & 1.02 & 2.18  & 1.72 & 1.11 & 0.93 & 0.37 & 1.26\\
%              %Ours S3D                         & 7.78 & 39.43 & 31.47 & 19.38 & 9.14 & 3.79  & 20.64  \\
%              Ours S3D                      & 9.12 & 42.70  & 35.49 & 25.16 & 16.22 & 10.05  & 25.92 \\
%              \midrule
%             CLIP \citep{radford2021learning}  & 3.59 & 29.54  & 22.15 & 9.16 & 2.48 & 0.39 & 12.74 \\
%             CoMMA$\dagger$              & 1.68 & 3.51 & 2.32 & 1.88 & 0.99 & 0.40 & 1.82 \\
%     	   Ours                        & 10.09 & 42.81  & 36.05 & 25.84 & 17.10 & 11.35  & 26.63 \\
%             \midrule
%             GLIP \citep{li2022grounded}      &  1.24 & 2.83 & 2.10 & 1.52 & 0.96 & 0.37 & 1.56 \\
%     	\bottomrule
%     \end{tabular}
%     %\vspace{+0.3cm}
%     \caption{\textbf{Spatio-temporal localization on full videos}. Since our model learned global representations encoding temporal information and spatial correspondences across modalities, it achieves the best performance in spatio-temporal evaluation.
%     % \caption{\textbf{Spatial-temporal localization on full videos}. Our model learned both global representation which encodes temporal information. It also learned spatial correspondence across modalities, which ends up with the best performance in spatial temporal evaluation.
%     \label{tab:st_long}
%     %\vspace{-0.7cm}
%     }
%     %}
% \end{table}
\begin{table*}[t]
    \tablestyle{4pt}{1.05}
    \tiny
    \centering
    \resizebox{2\columnwidth}{!}{
    \begin{tabular}{@{}l|ccccccccccc}
    	\toprule
    	\multicolumn{5}{c}{} &\multicolumn{7}{c}{GroundingYoutube}  \\ 
    	\cmidrule(lr){6-12} 
    	\multirow{2}{*}{\textbf{Method}}  & \multirow{2}{*}{\textbf{Backbone}} & \multirow{2}{*}{\textbf{DataSet}} & \multirow{2}{*}{\textbf{Supervision}} & \multirow{2}{*}{\textbf{Modality}}  & \multirow{2}{*}{IoU+Point} &\multicolumn{6}{c}{mAP}  \\ 
    	  & & & & & & 0.1 & 0.2 & 0.3 & 0.4 & 0.5  & 0.1:0.5 \\ 
    	\midrule
    	
         CoMMA$\dagger$ \citep{tan2021look}  & S3D &HT250K & Self &VT& 1.02 & 2.18  & 1.72 & 1.11 & 0.93 & 0.37 & 1.26\\
         MIL-NCE \citep{miech2020end} & S3D* &HT100M & Self &VT& 4.67 & 33.94 & 25.16 & 12.65 & 3.42 & 0.41  & 15.11 \\
             %Ours S3D                         & 7.78 & 39.43 & 31.47 & 19.38 & 9.14 & 3.79  & 20.64  \\
             %\midrule
             Ours                   & S3D &HT100M & Self &VT  & \textbf{9.12} & \textbf{42.70}  & \textbf{35.49} & \textbf{25.16} & \textbf{16.22} & \textbf{10.05}  & \textbf{25.92} \\
             \midrule
            GLIP \citep{li2022grounded}   & Swin-L*  & Cap24M & Weak & IT &  1.24 & 2.83 & 2.10 & 1.52 & 0.96 & 0.37 & 1.56 \\
            CoMMA$\ddagger$   \citep{tan2021look} & CLIP &HT100M& Self & VT & 1.68 & 3.51 & 2.32 & 1.88 & 0.99 & 0.40 & 1.82 \\
            CLIP \citep{radford2021learning}& CLIP &HT100M & Self & IT& 3.59 & 29.54  & 22.15 & 9.16 & 2.48 & 0.39 & 12.74 \\
            RegionCLIP \citep{zhong2022regionclip}   & ResNet-101*  & CC3M & Weak & IT &  5.65 & 35.65 & 27.43 & 15.69 & 4.31 & 0.86 &  16.78 \\
            %\midrule
    	   Ours       & CLIP &HT100M & Self &VT  &10.09 & 42.81  & 36.05 & 25.84 & 17.10 & 11.35  & 26.63 \\
               Ours                    & CLIP* &HT100M & Self &VT  & \textbf{11.53} & \textbf{43.64}  & \textbf{36.94} & \textbf{26.78} & \textbf{19.45} & \textbf{14.61}  & \textbf{28.26} \\
               \midrule
               MIL-NCE(temp.)+RegionCLIP(spa.)   &  -  & - & - & VT  & 9.21  & 40.54  & 34.97  & 22.38  &  13.79 & 9.18  &  22.33  \\
    	\bottomrule
    \end{tabular}}
    %\vspace{+0.3cm}
    \caption{\textbf{Spatio-temporal grounding on GroundingYouTube full videos}.   
The proposed model learns global representations encoding global information and spatial correspondences across modalities, achieving a better performance in spatio-temporal evaluation compared to models trained on only spatial or temporal grounding. 
(V: video, I: image, T: text.) $^*$ indicates finetuned backbone.
    % \caption{\textbf{Spatial-temporal localization on full videos}. Our model learned both global representation which encodes temporal information. It also learned spatial correspondence across modalities, which ends up with the best performance in spatial temporal evaluation.
    \label{tab:st_long}
    \vspace{-0.3cm}
    }
    %}
\end{table*}

\section{Experiments}

\subsection{Datasets} \label{dataset}
\noindent \textbf{Training Data:} \textbf{HowTo100M dataset} contains 1.2M instructional videos along with their corresponding automatically generated speech (ASR).
The narrations may be inaccurate and do not always accurately depict the video scene.
%We randomly selected 200K  video clips from the \textit{Food and Entertaining} category for training. %, and thus, we mainly focus on instructional videos in the area of cooking and kitchen tasks. 
%
%\begin{table*}%[htpb]
    \centering
    \small
    \setlength{\tabcolsep}{4pt}
    \resizebox{\textwidth}{!}{
    \begin{tabu}{lr|ccccccccc|ccccccccc}
        \toprule
        \multirow{2}{*}{\bf Method} & \multirow{2}{*}{\bf \#Pairs} & \multicolumn{9}{c|}{\bf FT Retrieval \ \ R@1 / R@5 / R@10} & \multicolumn{9}{c}{\bf ZS Retrieval \ \ R@1 / R@5 / R@10} \\
        & & \multicolumn{3}{c}{MSRVTT} & \multicolumn{3}{c}{DiDeMo} & \multicolumn{3}{c|}{ActivityNet} & \multicolumn{3}{c}{MSRVTT} & \multicolumn{3}{c}{DiDeMo} & \multicolumn{3}{c}{ActivityNet}  \\
        \midrule
        ClipBERT~\cite{lei2021less}  & 5.4M & 22.0 & 46.8 & 59.9 & 20.4 & 48.0 & 60.8 & 21.3 & 49.0 & 63.5 & -& -& -& -& -& -& -& -& -\\
        VideoCLIP~\cite{xu2021videoclip}  & 136M & 30.9 & 55.4 & 66.8 & -& -& -& -& -& -& 10.4 & 22.2 & 30.0 & 16.6 & 46.9 & -& -& -& -\\
        Frozen~\cite{bain2021frozen}  & 5M & 31.0 & 59.5 & 70.5 & 34.6 & 65.0 & 74.7  & -& -& -& 18.7 & 39.5 & 51.6 & 20.2 & 46.4 & 58.5 & -& -& -\\
        ALPRO~\cite{li2022align}  & 5M & 33.9 & 60.7 & 73.2 & 35.9 & 67.5 & 78.8 & -& -& -& 24.1 & 44.7 & 55.4 & 23.8 & 47.3 & 57.9 & -& -& -\\
        VIOLET~\cite{fu2021violet}  & 138M & 34.5 & 63.0 & 73.4 & 32.6 & 62.8 & 74.7 & -& -& -& 25.9 & 49.5 & 59.7 & 23.5 & 49.8 & 59.8 & -& -& - \\
        All-in-one~\cite{wang2022all} & 138M & 37.9 & 68.1 & 77.1 & 32.7 & 61.4 & 73.5 & 22.4 & 53.7 & 67.7 & -& -& -& -& -& -& -& -& -\\
        LAVENDER~\cite{li2022lavender} & 30M & 40.7 & 66.9 & 77.6 & 53.4 & 78.6 & 85.3 &  - & -& -& -& -& -& -& -& -& -& -& -\\
        Singularity~\cite{lei2022revealing} & 17M & 42.7 & 69.5 & 78.1 & 53.1 & 79.9 & 88.1 & 48.9 & 77.0 & 86.3 & 34.0 & 56.7 & 66.7 & 37.1 & 61.7 & 69.9 & 30.6 & 55.6 & 66.9 \\
        OmniVL~\cite{wang2022omnivl} & 17M & 47.8 & 74.2 & 83.8 & 52.4 & 79.5 & 85.4 & -& -& -& 34.6 & 58.4 & 66.6 & 33.3 & 58.7 & 68.5 & -& -& -\\ 
        VINDLU~\cite{Cheng2022VindLUAR} & 25M & 46.5 & 71.5 & 80.4 & 61.2 & 85.8 & 91.0 & 55.0 & 81.4 & 89.7 & 32.0 & 54.6 & 62.0 & 36.9 & 61.7 & 70.5 & 30.9 & 57.0 & 68.2 \\
        \rowfont{\color{Gray}}
        CLIP4Clip~\cite{luo2022clip4clip} & 400M & 44.5 & 71.4 & 81.6 & 42.8 & 68.5 & 79.2 & 40.5 & 72.4 & 83.4 & 31.2 & 53.7 & 64.2 & -& -& -& -& -& -\\
        % \rowfont{\color{Gray}}
        % CLIP-Hhiker~\cite{bain2022clip} & 400M & 47.7 & 74.1 & 82.9 & -& -& -& 44.0 & 74.9 & 86.1 & -& -& -& -& -& -& -& -& -\\
        \rowfont{\color{Gray}}
        CLIP-ViP~\cite{xue2022clip} & 500M & 54.2 & 77.2 & 84.8 & 50.5 & 78.4 & 87.1 & 53.4 & 81.4 & 90.0 & -& -& -& -& -& -& -& -& -\\
        \rowfont{\color{Gray}}
        InternVideo~\cite{Wang2022InternVideoGV} & 646M & 55.2 & 79.6 & 87.5 & 57.9 & 82.4 & 88.9 & 62.2  & 85.9 & 93.2 & 40.7 & 65.3 & 74.1 & 31.5 & 57.6 & 68.2 & 30.7 & 57.4 & 70.2 \\
        \midrule
        \multirow{3}{*}{\Modelname-Base} & 5M & 46.3 & 72.7 & 82.0 & 54.8 & 83.0 & 89.0 & 52.1 & 80.5 & 89.6 & 29.6 & 52.8 & 61.9 & 33.4 & 58.3 & 67.0 & 28.3 & 53.0 & 64.2 \\
        & 17M & 50.6 & 75.4 & 83.5 & 60.8 & 85.1 & 91.0 & 56.1 & 82.5 & 91.2 & 35.5 & 59.3 & 68.6 & 41.9 & 66.7 & 75.0 & 33.8 & 59.1 & 70.4 \\
        & 25M & 51.0 & 76.5 & 84.2 & 61.6 & 86.8 & 91.5 & 58.3 & 83.9 & 91.5 & 35.2 & 57.8 & 66.0 & 41.2 & 65.4 & 74.9 & 35.5 & 60.6 & 71.8 \\
        \hline
        \multirow{3}{*}{\Modelname-Large} & 5M & 53.3 & 76.6 & 83.9 & 59.7 & 84.9 & 90.8 & 58.1 & 85.5 & 92.9 & 33.3 & 58.1 & 66.7 & 34.0 & 60.4 & 68.7 & 31.9 & 60.2 & 72.0 \\
        & 17M & \underline{56.5} & \underline{80.1} & \underline{87.4} & \underline{66.6} & \underline{89.9} & \textbf{93.7} & \underline{66.6} & \underline{88.6} & \underline{94.7} & \textbf{42.6} & \textbf{64.4} & \textbf{73.1} & \underline{46.4} & \underline{70.0} & \underline{78.8} & \textbf{42.8} & \textbf{69.6} & \textbf{79.8} \\
        & 25M & \textbf{58.8} & \textbf{81.0} & \textbf{87.1} & \textbf{70.4} & \textbf{90.1} & \underline{93.5} & \textbf{66.8} & \textbf{89.1} & \textbf{94.9} & \underline{40.7} & \underline{63.4} & \underline{71.8} & \textbf{48.6} & \textbf{72.9} & \textbf{80.0} & \underline{41.9} & \underline{68.9} & \underline{80.3} \\
        \bottomrule
    \end{tabu}
    }
    \vspace{-0.3cm}
    \caption{Comparison to the state-of-the-art text-to-video retrieval methods on MSRVTT, DiDeMo and AcitivityNet.
    \#Pairs denotes the number of pre-training pairs.
    ``FT'' and ``ZS'' refer to the fine-tuning and zero-shot results.
    }
    \label{tab:retrieval}
\end{table*}
%\input{tables/spatio_vhico}

%\vspace{-0.3cm}
\noindent\textbf{Downstream Datasets:} %\textbf{YouCook2}: For the text-to-video retrieval downstream task, we use the common YouCook2 dataset , which provides a human-generated caption for 3.5K video clips for cooking instruction. 
%blah blah ... \hkc{add some details here?}
\textbf{GroundingYoutube (GYT)} is used to evaluate the task of multi-action spatio-temporal grounding as described in Section \ref{sec:dataset:annotation}.
% , we annotated the dense spatio-temporal location information as described in Section \ref{sec:dataset:annotation}.
%for 512 verb-noun phrases. All occurrences of the specific phrase in the test video are hence annotated, allowing us to evaluate spatio-temporal grounding in full untrimmed videos.
\noindent\textbf{MiningYoutube (MYT)} \citep{kuehne2019mining} %: To evaluate the temporal grounding abilities, we leverage the MiningYoutube \citep{kuehne2019mining} dataset, as it 
provides temporal annotation and is limited to the domain of cooking instruction videos. %The dataset features 250 full instructional videos, which are annotated with 512 action classes and temporal boundary information. 
%We use it to evaluate the temporal grounding abilities.
%Here, temporal alignment, the task of finding the right temporal boundaries given the sequences of actions, is used during evaluation to relax the task of temporal detection. 
\noindent\textbf{YouCook-Interaction (YC-Inter)} \citep{tan2021look} is an extension of the YouCook2 dataset \citep{zhou2018towards} for cooking instruction providing bounding boxes for 6K selected frames. The bounding boxes usually comprise the hand and the tool mentioned in the respective sentence-wise annotation. %We evaluate the spatial grounding abilities of models on this dataset.
% \noindent\textbf{YouCook2-Interaction (YC-Inter)}  To evaluate the spatial grounding abilities of our system, we use the YouCook2-Interaction dataset \citep{tan2021look}, an extension of a subset of the YouCook2 dataset \citep{zhou2018towards} for cooking instruction, which provides bounding boxes for 6K selected frames. The bounding boxes usually comprise the hand and the tool mentioned in the respective sentence-wise annotation.    
To further benchmark on general video domains on the \textbf{V-HICO} dataset~\citep{li2021weakly} with 6.5k videos with human-object interaction bounding boxes annotations, 
% that have been semi-automatically curated from sentence captions, 
and \textbf{Daly} action dataset~\citep{weinzaepfel2016human}, featuring videos consisting of daily actions such as ``brushing teeth''.% and ``cleaning windows''.



\subsection{Baseline methods}

%The proposed system is compared to various multimodal methods based on self- and weak supervision: 
\textbf{Temporal}: MIL-NCE~\citep{miech2020end} utilizes S3D~\citep{xie2018rethinking} and word2vec~\citep{mikolov2013efficient}. CLIP~\citep{radford2021learning}, an image-text model with transformer. 
\textbf{Spatial}:
CoMMA~\citep{tan2021look}, SSL model ($\dagger$ for weights shared by the author\footnote{We thank the authors for providing code and weights.} $\ddagger$ trained with CLIP);  
GLIP~\citep{li2022grounded}, RegionCLIP~\citep{zhong2022regionclip}, SOTA weakly supervised grounding model. % trained with image-text pairs.
\textbf{Spatio-temporal}: We construct MIL-NCE+RegionCLIP following the inference pipeline in Figure \ref{fig:inference}. 
TubeDETR~\citep{yang2022tubedetr} and STCAT \citep{jin2022embracing} are supervised. 
More descriptions of the baselines are given in the Appendix \ref{sup:baseline}.
Details of the implementation and experimental settings can be found in the appendix \ref{backbone_and_training}. Inference setups for each baseline are described in Section \ref{inference_sup}.

%% \begin{table}[t]
%     \tablestyle{2pt}{1.05}
    
%     \centering
%     %\resizebox{1\columnwidth}{!}{
%     \begin{tabular}{@{}l|ccccccc}
%     	\toprule
%             \multicolumn{8}{c}{Untrimmed Spatial-Temporal Grounding}
%     	\toprule
%     	\multicolumn{1}{c}{} & \multicolumn{7}{c}{GroundingYouTube}  \\ 
%     	\cmidrule(lr){2-8} 
%     	\multirow{2}{*}{\textbf{Method}}    & \multirow{2}{*}{IoU+Point} &\multicolumn{6}{c}{mAP}  \\ 
%     	                                    &  & 0.1 & 0.2 & 0.3 & 0.4 & 0.5  & 0.1:0.5 \\ 
%     	\midrule
%     	MIL-NCE \citep{miech2020end} & 4.67 & 33.94 & 25.16 & 12.65 & 3.42 & 0.41  & 15.11 \\
%          CoMMA* \citep{tan2021look}   & 1.02 & 2.18  & 1.72 & 1.11 & 0.93 & 0.37 & 1.26\\
%              %Ours S3D                         & 7.78 & 39.43 & 31.47 & 19.38 & 9.14 & 3.79  & 20.64  \\
%              Ours S3D                      & 9.12 & 42.70  & 35.49 & 25.16 & 16.22 & 10.05  & 25.92 \\
%              \midrule
%             CLIP \citep{radford2021learning}  & 3.59 & 29.54  & 22.15 & 9.16 & 2.48 & 0.39 & 12.74 \\
%             CoMMA$\dagger$              & 1.68 & 3.51 & 2.32 & 1.88 & 0.99 & 0.40 & 1.82 \\
%     	   Ours                        & 10.09 & 42.81  & 36.05 & 25.84 & 17.10 & 11.35  & 26.63 \\
%             \midrule
%             GLIP \citep{li2022grounded}      &  1.24 & 2.83 & 2.10 & 1.52 & 0.96 & 0.37 & 1.56 \\
%     	\bottomrule
%     \end{tabular}
%     %\vspace{+0.3cm}
%     \caption{\textbf{Spatio-temporal localization on full videos}. Since our model learned global representations encoding temporal information and spatial correspondences across modalities, it achieves the best performance in spatio-temporal evaluation.
%     % \caption{\textbf{Spatial-temporal localization on full videos}. Our model learned both global representation which encodes temporal information. It also learned spatial correspondence across modalities, which ends up with the best performance in spatial temporal evaluation.
%     \label{tab:st_long}
%     %\vspace{-0.7cm}
%     }
%     %}
% \end{table}
\begin{table*}[t]
    \tablestyle{4pt}{1.05}
    \tiny
    \centering
    \resizebox{2\columnwidth}{!}{
    \begin{tabular}{@{}l|ccccccccccc}
    	\toprule
    	\multicolumn{5}{c}{} &\multicolumn{7}{c}{GroundingYoutube}  \\ 
    	\cmidrule(lr){6-12} 
    	\multirow{2}{*}{\textbf{Method}}  & \multirow{2}{*}{\textbf{Backbone}} & \multirow{2}{*}{\textbf{DataSet}} & \multirow{2}{*}{\textbf{Supervision}} & \multirow{2}{*}{\textbf{Modality}}  & \multirow{2}{*}{IoU+Point} &\multicolumn{6}{c}{mAP}  \\ 
    	  & & & & & & 0.1 & 0.2 & 0.3 & 0.4 & 0.5  & 0.1:0.5 \\ 
    	\midrule
    	
         CoMMA$\dagger$ \citep{tan2021look}  & S3D &HT250K & Self &VT& 1.02 & 2.18  & 1.72 & 1.11 & 0.93 & 0.37 & 1.26\\
         MIL-NCE \citep{miech2020end} & S3D* &HT100M & Self &VT& 4.67 & 33.94 & 25.16 & 12.65 & 3.42 & 0.41  & 15.11 \\
             %Ours S3D                         & 7.78 & 39.43 & 31.47 & 19.38 & 9.14 & 3.79  & 20.64  \\
             %\midrule
             Ours                   & S3D &HT100M & Self &VT  & \textbf{9.12} & \textbf{42.70}  & \textbf{35.49} & \textbf{25.16} & \textbf{16.22} & \textbf{10.05}  & \textbf{25.92} \\
             \midrule
            GLIP \citep{li2022grounded}   & Swin-L*  & Cap24M & Weak & IT &  1.24 & 2.83 & 2.10 & 1.52 & 0.96 & 0.37 & 1.56 \\
            CoMMA$\ddagger$   \citep{tan2021look} & CLIP &HT100M& Self & VT & 1.68 & 3.51 & 2.32 & 1.88 & 0.99 & 0.40 & 1.82 \\
            CLIP \citep{radford2021learning}& CLIP &HT100M & Self & IT& 3.59 & 29.54  & 22.15 & 9.16 & 2.48 & 0.39 & 12.74 \\
            RegionCLIP \citep{zhong2022regionclip}   & ResNet-101*  & CC3M & Weak & IT &  5.65 & 35.65 & 27.43 & 15.69 & 4.31 & 0.86 &  16.78 \\
            %\midrule
    	   Ours       & CLIP &HT100M & Self &VT  &10.09 & 42.81  & 36.05 & 25.84 & 17.10 & 11.35  & 26.63 \\
               Ours                    & CLIP* &HT100M & Self &VT  & \textbf{11.53} & \textbf{43.64}  & \textbf{36.94} & \textbf{26.78} & \textbf{19.45} & \textbf{14.61}  & \textbf{28.26} \\
               \midrule
               MIL-NCE(temp.)+RegionCLIP(spa.)   &  -  & - & - & VT  & 9.21  & 40.54  & 34.97  & 22.38  &  13.79 & 9.18  &  22.33  \\
    	\bottomrule
    \end{tabular}}
    %\vspace{+0.3cm}
    \caption{\textbf{Spatio-temporal grounding on GroundingYouTube full videos}.   
The proposed model learns global representations encoding global information and spatial correspondences across modalities, achieving a better performance in spatio-temporal evaluation compared to models trained on only spatial or temporal grounding. 
(V: video, I: image, T: text.) $^*$ indicates finetuned backbone.
    % \caption{\textbf{Spatial-temporal localization on full videos}. Our model learned both global representation which encodes temporal information. It also learned spatial correspondence across modalities, which ends up with the best performance in spatial temporal evaluation.
    \label{tab:st_long}
    \vspace{-0.3cm}
    }
    %}
\end{table*}

\begin{table*}[h]
    \tablestyle{7pt}{1.05}
    \tiny
    \centering
    \resizebox{2\columnwidth}{!}{
    \begin{tabular}{@{}l| cccc | c |c c| c c | c c }
    	\toprule
    	\multicolumn{4}{c}{} & \multicolumn{1}{c}{ } & \multicolumn{1}{c}{YC-Inter} & \multicolumn{2}{c}{GroundingYT}  & \multicolumn{2}{c}{V-HICO}   & \multicolumn{2}{c}{Daly}\\ 
    	\cmidrule(lr){6-6} \cmidrule(lr){7-8}  \cmidrule(lr){9-10} \cmidrule(lr){11-12}  
    	Method  & Backbone &Data&Super.&Mod.& Acc &  Acc & mAP &  Acc & mAP  &  Acc & mAP \\ 
    	\midrule
        MIL-NCE \citep{miech2020end} & S3D* &HT100M & Self &VT& 23.67  & 27.45  & 8.21 & 12.65 & 11.23 & 13.84 & 24.23 \\
    	CoMMA$\dagger$ \citep{tan2021look} & S3D &HT250K & Self &VT& 48.63   & 47.68 & 23.38 & 40.97 & 21.45 & 54.48 & 33.39 \\
        %\midrule
        Ours                       & S3D &HT100M & Self &VT & \textbf{53.98}   & \textbf{60.62} & \textbf{44.93} & \textbf{44.32} & \textbf{24.31} & \textbf{66.35} & \textbf{45.93} \\
         \midrule
         CLIP   \citep{radford2021learning}            & CLIP&HT100M & Self &IT &    14.10    & 12.50  & 3.49 &  29.23 & 12.51  & 18.02 & 27.28  \\
         CoMMA$\ddagger$  \citep{tan2021look}            & CLIP  &HT100M & Self &VT&   52.65     & 47.56 & 36.42 & 55.20 &  34.54& 61.06 & 44.37  \\
             RegionCLIP   \citep{zhong2022regionclip}            & RN50x4* & CC3M & Weak &IT &   51.56     &   52.84 &  23.42 & 57.92 & 37.82 & 67.12 & 48.62 \\
            GLIP   \citep{li2022grounded}            & Swin-L*&Cap24M & Weak &IT &   52.84      &   53.62 & 24.73 & \textbf{66.05} & 41.17 & - & - \\
            %\midrule
            Ours         & CLIP &HT100M & Self &VT& 57.10    &   55.49 & 43.12 & 60.71& 39.28 & 70.08 & 50.56 \\
            Ours                       & CLIP* &HT100M & Self &VT& \textbf{58.35}    &   \textbf{56.98} & \textbf{45.32} & 62.34& \textbf{41.56} & \textbf{71.35} & \textbf{52.78} \\
            %V-HICO   \citep{}            &  Faster R-CNN &  -      &  - & - & & 67.21 & - & - \\
            \midrule
            {\color{gray}TubeDETR \citep{yang2022tubedetr}}    &  {\color{gray}MDETR} & {\color{gray}Vid-STG} & {\color{gray} Full} & {\color{gray}VT} & {\color{gray}51.63}    &   {\color{gray}53.24} & {\color{gray} 41.76} & {\color{gray}63.23} & {\color{gray}40.87 } & {\color{gray}84.21} & {\color{gray} 62.98} \\
            {\color{gray}STCAT \citep{jin2022embracing}}    &  {\color{gray}ResNet-101} & {\color{gray}Vid-STG} & {\color{gray} Full} & {\color{gray}VT} & {\color{gray}54.47}    &   {\color{gray} 55.90} & {\color{gray}44.21 } & {\color{gray}65.34} & {\color{gray} 41.10 } & {\color{gray}85.42} & {\color{gray} 63.94} \\
    	\bottomrule
    \end{tabular}
    }
    \vspace{-0.2cm}
    \caption{\textbf{Video spatial grounding}. We evaluate the accuracy of the pointing game and the mean average precision. 
    We listed CNN-based methods on top and transformer-based methods in the middle. 
    Models learning global representations (MIL-NCE, CLIP) don't perform well on localization tasks, while our model outperforms other grounding methods. $^*$ indicates finetuned backbone.
    %Models learning global representations (MIL-NCE, CLIP) don't perform well on localization tasks, while our model outperforms other grounding methods. %We listed CNN-based methods on top and transfomer-based methods at the bottom. 
    %(Mod. indicates the modality used, where V: video, I: image, T: text. Super. indicates supervision.)
    %Our method generalized well on both video and image architectures. 
    % Daly GLIP is not workable since every class is action. OOV. V-HICO dataset the CLIP  generalized better to OOV, while word2vec getting worse performance. \bc{maybe we can add supervision: weakly, SSL} \bc{add pretraining data}
    \label{tab:spatial}
    \vspace{-0.5cm}
    }
   
    
\end{table*}

% \begin{table}[t]
%     % \tablestyle{2pt}{1.05}
    
%     \centering
%     %\resizebox{1\columnwidth}{!}{
%     \begin{tabular}{@{}l|cc|cc}
%     	\toprule
%     	\multicolumn{1}{c}{} & \multicolumn{2}{c}{YouCook-Interaction} & \multicolumn{2}{c}{MiningYoutube Grounding}  \\ 
%     	\cmidrule(lr){2-3} \cmidrule(lr){4-5} 
%     	Method  & Acc & IoU   & Acc & IoU \\ 
%     	\midrule
%     	CoMMA* \citep{tan2021look}   & 48.63 & -  & 47.68 & -  \\
%     	MIL-NCE \citep{miech2020end} & 23.67 & -  & 27.45 & -  \\
%     	Ours                        & 48.03 & -  & 47.35 & -  \\
%     	\bottomrule
%     \end{tabular}
%     \vspace{+0.3cm}
%     \caption{Evaluation on spatial-only evaluation using pointing game accuracy and attention heatmap IoU with GT bounding box. Models learning global representation doesn't perform well on localization tasks, while our model maintain comparable performance.
%     \label{tab:spatial}
%     %\vspace{-0.2cm}
%     }
%     %}
    
% \end{table}

\subsection{Downstream Tasks}


%We compare to the SOTA self-supervised method evaluated on spatial \citep{tan2021look} and temporal \citep{kuehne2019mining} grounding.

We considered the following downstream tasks to evaluate spatio-temporal grounding abilities of various models (detailed description is included in the appendix \ref{eval_metric}):

\noindent (i) \textbf{Spatio-temporal grounding in untrimmed video} is evaluated on the proposed Grounding Youtube dataset. The entire video and the respective pool of action instructions were provided. The model needs to localize each action step in time (start-time/end-time) and space (location in the video) as described in Figure \ref{fig:inference}. 
% We evaluate in two metrics: \textbf{IoU+Pointing game} combines spatial grounding~\citep{akbari2019multi} and temporal grounding~\citep{kuehne2019mining} metrics. %For each video frame, the prediction is correct when the model predicts the correct action for the frame. Also, given the predicted action as a query, the maximum point of the heatmap aims to lie within the desired bounding box. We then compute the Intersection over Union (IoU) over all the predictions with the GT to acquire the final score. 
% We also compute \textbf{video mAP} following previous evaluation~\citep{gu2018ava}, where we set IoU threshold between GT and predicted spatio-temporal tubes. A prediction is correct when it surpasses the IoU threshold. We compute the mAP over all classes. %We form a 3D prediction mask following Figure \ref{fig:inference} and compute IoU between our 3D heatmap and 3D tube.
We evaluate in two metrics: \textbf{IoU+Pointing game} combines the evaluation setting from the spatial grounding \citep{akbari2019multi} and temporal grounding \citep{kuehne2019mining} metrics. For each video frame, the prediction is correct when the model predicts the correct action for the frame. Also, given the predicted action as a query, the maximum point of the heatmap aims to lie within the desired bounding box. We then compute the Intersection over Union (IoU) over all the predictions with the GT to acquire the final score. 
We also compute \textbf{video mAP} following previous evaluation \citep{gu2018ava}, where we set IoU threshold between GT and predicted spatio-temporal tubes. A prediction is correct when it surpasses the IoU threshold. We then compute the mAP over all classes. We form a 3D prediction mask following Figure \ref{fig:inference} and compute IoU between our 3D heatmap and 3D tube.

\noindent (ii) \textbf{Spatial grounding} is given a text description to localize the region in the trimmed video. %We use GroundingYoutube, Youcook-Interaction, V-HICO, and Daly for evaluation. %Note that the evaluation is spatial only. It evaluates the results for each frame separately without considering the temporal information. 
It is evaluated using the \textbf{pointing game accuracy}. %Given the query text and video, we compute the attention heatmap on the video as described in Figure \ref{fig:inference}(b). 
If the predicted point lies in the ground truth bounding box, the result counts as a ``hit" and counts as ``miss" otherwise. The final accuracy is calculated as a ratio between hits to the total number of predictions $\frac{\text{\# hits}}{\text{\# hits} + \text{\# misses}}$. 
We also report the mean average precision \textbf{(mAP)} following the settings from V-HICO~\citep{li2021weakly}. %Given a human-object category as the text query, we aim to localize the spatial location in the video frame.
%The predicted location is correct if their Intersection over-Union (IoU) with ground truth bounding boxes is larger than 0.3. 
%Since we do not use any bounding box proposal tools or supervision, we create an attention heatmap as described in Figure \ref{fig:inference}(b) to create a mask for IoU computation. 
%We follow \citep{li2021weakly} and compute the mAP over all verb-object classes.


\noindent (iii) \textbf{Temporal grounding} \label{temporal_grounding}
provides videos with the respective actions and their ordering, including the background. The goal is to find the correct frame-wise segmentation of the video. We follow the inference procedure in \citep{kuehne2019mining} to compute the alignment given the similarity input matrix. The task is evaluated by intersection over detection (IoD), defined as $\frac{G \cap D}{D}$ the ratio between the intersection of ground-truth action $G$ and prediction $D$ to prediction $D$, and the Jaccard index, which is an (IoU) given as $\frac{G \cap D}{G \cup D}$.



\subsection{Comparison with state-of-the-art methods}\label{sota}
\noindent (i) \textbf{Spatio-temporal grounding in untrimmed video:}
We first compare the proposed method with other approaches designed for spatial or temporal grounding in Table \ref{tab:st_long}.
It shows that models without specific loss designs for spatial grounding (MIL-NCE~\citep{miech2020end}, CLIP~\citep{radford2021learning}) show good mAP scores but lower pointing game accuracy. Out of the two weakly supervised methods, GLIP~\citep{li2022grounded} and RegionCLIP~\citep{zhong2022regionclip}), trained with aligned image-text, RegionCLIP show significantly better performance in this setting, while both perform in a similar range in the spatial grounding scenario (see Table~\ref{tab:spatial}). We attribute this behavior to the fact that RegionCLIP distinguishes frames with relevant queries better from background than GLIP, leading to better temporal localization. 
We finally compare the strong baseline MIL-NCE+RegionCLIP, which combines two approaches specialized in temporal and spatial aspects, to our task. 
It shows that the proposed method improves over all other baselines underlining the need to incorporate global (temporal) and local (spatial) representations. 
%Experiments showed that combining a joint objective that learns spatial and temporal information jointly results in better performance than simply applying the best temporal and spatial model. 
% Also, such a combined objective also benefits more when the visual backbone is finetued as well. 
% We construct a split with single action shown in appendix \ref{single_action_stg}.
%Models designed for trimmed videos (CoMMA\citep{tan2021look}) or trained with aligned image-text (GLIP\citep{li2022grounded}, RegionCLIP\citep{zhong2022regionclip}) failed to capture the temporal dynamics, while models without specific loss designs for spatial grounding (MIL-NCE\citep{miech2020end}, CLIP\citep{radford2021learning}) were not able to ground the action in the correct region.
%Note that supervised spatio-temporal grounding approaches~\citep{yang2022tubedetr,jin2022embracing} are not directly applicable in this evaluation since such methods assume the given text query to be ground-truth. %The model must distinguish the correct text query from a pool of action lists. 
%We include an evaluation setting in the supplement where the GT-text queries were provided. \hkc{Do we? If not, we can probably comment the last 2 sentences}
%More experiment setting is in the supplement.

\begin{table}[h]
     \tablestyle{2pt}{1.05}
    
    \centering
    %\resizebox{1\columnwidth}{!}{
    \begin{tabular}{@{}l|ccccc}
    	\toprule
    	%\multicolumn{4}{c}{} &\multicolumn{2}{c}{MiningYoutube}  \\ 
    	%\cmidrule(lr){5-6} 
    	Method   & Backbone &Data & Super. & IoU & IoD \\ 
    	\midrule
    	Mining: MLP \cite{miech2020end} & TSM & MiningYT & Weak & 9.80 & 19.20    \\
             CoMMA* \cite{tan2021look} & S3D-word2vec & HT250K & Self & 2.05 & 5.63    \\
    	MIL-NCE \cite{miech2020end} & S3D-word2vec & HT100M & Self & 18.69 & 26.74    \\
    	Ours                       & S3D-word2vec & HT200K & Self  & 19.18 & 27.65   \\
    	%Ours                       & VAT& S3D-g  & 19.40 & 28.48   \\
            Ours                       & CLIP & HT200K & Self &  \textbf{19.88} & \textbf{28.50}   \\
             %\midrule
            % MCN \cite{chen2021multimodal}      &VAT& R152+RX101   & 23.10 & 32.04    \\
    	\bottomrule
    \end{tabular}
    \vspace{-0.3cm}
    \caption{\textbf{Temporal Grounding on MiningYoutube.} %Spatial-focused model CoMMA is not trained for temporal detection, which results in lower performance, while the proposed model combines global and local representation resulting in better temporal localization than one alone. %\bc{we should include setting without knowing the order}
    %\vspace{-0.5cm}
    \label{tab:temporal}
%    \vspace{-0.4cm}
    }
    %}
\end{table}

\noindent (ii)~\textbf{Spatial grounding: } 
 %We do not report mAP on Youcook interaction since the input is sentence descriptions instead of class.
Second, we compare the performance of the proposed framework to other methods on the task of spatial grounding, including models with weak supervision, as well as models trained in a fully supervised setting in Table \ref{tab:spatial}.
%As shown in Table \ref{tab:spatial}, models trained with global representations such as MIL-NCE and CLIP were not able to localize the text description compared to models learning local representations such as CoMMA, GLIP, RegionCLIP and our approach. 
In the instruction video domain (GYT and YC-Inter), the proposed approach achieves the best result among all weakly and self-supervised trained methods. In the general domain (V-HICO and Daly), the method also achieves competitive results, showing the generalizability of the model to other domains. 
%We attribute this to the transformer architecture in the text branch inheriting knowledge from the open domain during large-scale training, while in contrast the model's performance using word2vec dropped in these datasets. 
Note that in the Daly dataset, the classes are verbs, which are not detectable by the object-focused model GLIP. 
Compared to their weakly trained counterparts, fully-supervised model (TubeDETER~\citep{yang2022tubedetr}, STCAT~\citep{jin2022embracing}) achieve competitive performance in the general domain (V-HICO, Daly) and slightly lower performance in instruction domain (GYT, YC-Inter) due to the domain gap with respect to the training data.
\begin{figure}
       \centering
        \setlength{\tabcolsep}{1pt}
        {\scriptsize
        \begin{tabular}{c c c c c c c }
            { Original } &
            \multicolumn{2}{c}{  } &
            \multicolumn{4}{c}{$\longleftarrow$ Object level variations $\longrightarrow$} \\
            \includegraphics[width=0.185\linewidth]{images/ablation/chair.jpg} &
            \multicolumn{2}{c}{  } &
            \includegraphics[width=0.185\linewidth]{images/ablation/1_only_prompt_mixing/bench.jpg} &
            \includegraphics[width=0.185\linewidth]{images/ablation/1_only_prompt_mixing/stool.jpg} &
            \includegraphics[width=0.185\linewidth]{images/ablation/1_only_prompt_mixing/armchair.jpg} &
            \includegraphics[width=0.185\linewidth]{images/ablation/1_only_prompt_mixing/saddle.jpg} \\
            \multicolumn{3}{c}{  } &
            \multicolumn{4}{c}{ Only Prompt Mixing } \\
            \multicolumn{3}{c}{ } &
            \includegraphics[width=0.185\linewidth]{images/ablation/2_with_self_attn_injection/bench.jpg} &
            \includegraphics[width=0.185\linewidth]{images/ablation/2_with_self_attn_injection/stool.jpg} &
            \includegraphics[width=0.185\linewidth]{images/ablation/2_with_self_attn_injection/armchair.jpg} &
            \includegraphics[width=0.185\linewidth]{images/ablation/2_with_self_attn_injection/saddle.jpg} \\
            \multicolumn{3}{c}{  } &
            \multicolumn{4}{c}{ + Attention-Based Shape Localization } \\
            \multicolumn{3}{c}{ } &
            \includegraphics[width=0.185\linewidth]{images/ablation/3_background_blending/bench.jpg} &
            \includegraphics[width=0.185\linewidth]{images/ablation/3_background_blending/stool.jpg} &
            \includegraphics[width=0.185\linewidth]{images/ablation/3_background_blending/armchair.jpg} &
            \includegraphics[width=0.185\linewidth]{images/ablation/3_background_blending/saddle.jpg} \\
            \multicolumn{3}{c}{  } &
            \multicolumn{4}{c}{ + Controllable Background Preservation } \\
        \end{tabular}
        }
    \vspace{1mm}
    \captionof{figure}{
    Ablating our full object variations pipeline. Original image was crated using the prompt ``A \emph{chair} with a dog on it''. 
    }
    \vspace{-10pt}
    \label{fig:ablation}
\end{figure}

\section{Visualization On Demand} %Visualization Elements
\label{sec:visrisk}
Based on environment data and trajectory evaluation, we now present ways of communicating the situation and risks on a visual display to achieve an ADAS.
In this context, we employ a renderer that visualizes all the information in a joint Cartesian coordinate system (see section \ref{subsec:sim}). 
Once driving risks are detected, design elements are overlayed on the display with section \ref{subsec:active} and section \ref{subsec:warning}. 

\subsection{Simulator Environment}
\label{subsec:sim}
Nodes of the R-LDM have a range of potential attributes, such as the 3D position or geometrical shape of objects. 
% For instance, the road centerline is a polyline with bounderies to the left and right. Crosswalks have a defined width and buildings a polygonal outline description. 
In the renderer, we always visualize static and quasi-static data that lie in the field of view from the ego vehicle. 
For this, a local 3D model is generated by converting geographic points with (lat, lon, alt) into Cartesian coordinates of (x, y, z). 
% and project the positonal relations from a view perspective with a transformation matrix. 
Fig. \ref{fig:3Dsimulator} depicts an exemplary map section having several intersections in bird's-eye view.
% with several intersections, stop lines and crosswalks. 
On the top right, the first person view of a vehicle approaching a crosswalk is shown. 

The dynamic data is then added to this static view. A zoomed-in excerpt from the map is given at the bottom of Fig. \ref{fig:3Dsimulator} that includes a recorded GNSS trace (red).
We project the trace onto the connected lane center, which is pictured in green. 
% Because we project the ego position on the closest lane segment, on the bottom right the measured trace is changed in red and the aligned trace is marked in green.
Consequently, the virtual horizon and its possible paths are retrieved as described in section \ref{subsec:ldm}. 
We can lastly update and move the excerpt with the current position from the GNSS to obtain a live simulation.

\subsection{Proactive Support}
\label{subsec:active}
Communication of spatial as well as spatio-temporal relations is crucial for risk-averse driver support. 
% This has the reason that humans can estimate the time better than positions (especially for risks). 
% Velocity contains implicitly the time as well. 
Further sources of information are cause, likelihood and severity of a potential risks.  
% if a collision happens. 
The next step for RNS is the choice of suitable design elements. 
In this process, we suppose that we know where the ego vehicle is driving (i.e., the ego path) from its navigation route. 
Yet, for surrounding vehicles, all paths are considered.

\subsubsection{Hazard Route Element}
The so-called hazard route in Fig. \ref{fig:charts} is a concept that consists of a scale portraying distances to an upcoming risk element.
Furthermore, the geometrical area or length of risks is considered.
Risk is thus measured with respect to the ego path, ranging from the current position  $\Delta l \hspace{-0.03cm}=\hspace{-0.03cm} \unit[0]{m}$ to the end of the path $\Delta l_{h}$.
Here, the length $\Delta l_{h}$ can be chosen according to own preferences. 

At an upcoming intersection, risk is defined by the section of the path that lies within the junction.
Since risk corresponds to exposition time, we encode the path part from the intersection $I_z$ with a color, ranging from green for short intersections to red for long ones. 
%allgemein risiko entlang des pfades zu intersection zone
%share of junction segment to navigation route + 
%one case with large intersection far and one case with small intersection close
Fig. \ref{fig:charts}~a) gives two examples of the hazard route.
The left bar shows a large intersection (e.g. multi-lane four-way stop) in vicinity and the right bar has a small and consecutive medium junction. 
% In the case of collision risk, the intersection zone $I_z$ can be used.
% Depending on the value of $I_z$ (low, medium and large), the area is marked from green, to yellow until red for conveying the criticality. 
This emphasizes that we may include more than one intersection in our warnings.

\begin{figure}[t]
  \centering
  \includegraphics[width=0.95\linewidth]{./img/simulator.png}
  \caption{Rendered road network from two perspectives with the ego position being projected on the navigation route. \vspace{0.45cm}}
  \label{fig:3Dsimulator}
\end{figure}

\begin{figure}[t]
  \centering
  \resizebox{\linewidth}{!}{
  \import{img/}{velocity_scale_new.pdf_tex}}  
  \caption{Chart elements for proactive support. Hazard route (left) and velocity scale (right).} %\vspace{-0.3cm}}
  \label{fig:charts} 
\end{figure} 

\subsubsection{Velocity Scale Element}
The velocity scale, Fig. \ref{fig:charts}~b), is a second chart element which qualifies the difference between the current velocity of the vehicle $v_0$ and the target velocity $v_{\text{tar}}$ from the trajectory evaluation of section \ref{subsec:trajeval}. 
The scale shows possible velocity values, from standstill $v\hspace{-0.05cm}=\hspace{-0.05cm}\unit[0]{m/s}$ to a maximal velocity $v_{\text{max}}$. Depending on the difference $|v_0 \hspace{0.05cm} - \hspace{0.05cm} v_{\text{tar}}|$, the situation is rated as safe with $v_0 \hspace{-0.042cm} \approx \hspace{-0.042cm} v_{\text{tar}}$ (green, left), as dangerous with e.g. $v_0 \hspace{-0.05cm} < \hspace{-0.05cm} v_{\text{tar}}$ (yellow, middle) to critical with $v_0 \hspace{-0.07cm} \ll \hspace{-0.07cm} v_{\text{tar}}$ (red, right). The same cases hold true for the opposite circumstances, i.e., $v_0 \hspace{-0.032cm} > \hspace{-0.032cm} v_{\text{tar}}$. 
This velocity scale can be employed for curve or regulatory risks. 
Moreover, we may set an enforced speed limit as the target velocity $v_{\text{tar}}$ for proactive behavior, once there is no risk ahead. 
%\noindent -Warning vs behavior support \\
%-Ghost vehicle as in game \\

\subsection{Short-Term Warning Elements}
\label{subsec:warning}
In order to emphasize the criticality of the situation, we propose to add further intuitive warning elements as e.g. pop-up signs and lane colorings. 
The following elements augment the proactive elements.

\subsubsection{Pop-up Signs}
Explicit symbols indicate the risk cause accompanied with the event time for collisions ($s_E$), distances to the risk spot for turns (i.e., right curve with $d_r$ and left curve with $d_l$) or stopping distance for crosswalks ($d_c$). In Fig. \ref{fig:popups}~a), the pop-up signs are pictured. 
% Besides the velocity difference, the risk type is an indication for the severity of the situation.
%Examples for collision risk are car-to-car crash., curve risk can be  as a single-car accident and regulatory risks will be a car-to-object collision. 
We want to stress that this is just a selection and more risk causes can be added. 
The purpose is also to clarify the reason for the warning and give more human-understandable information.

\subsubsection{Colored Events}
Finally, we highlight lane parts or positions according to the corresponding risks.  
% the determined color rating from the hazard route and velocity scale and relate the risks to the simulator environment. 
In the instance of curve and regulatory risk, the lane is colored from the ego position up to the point of maximal risk. 
For collision risk, we mark the point of the closest encounter as a red cube.
An illustration for regulatory risk induced from a stop line is depicted in Fig. \ref{fig:popups}~b). Again, the color is defined by the deviation $|v_0-v_{\text{tar}}|$. It also shows the therein considered navigation route with length $\Delta l_h$ and another unlikely path. 

It should be noted that the visualization of warnings only occurs if the risks are actually present. 
%\textcolor{red}{improve language, repeat intersection zone and navigation route}
%eingrauen unlikely paths and navigation path and describe in text, maybe delete Iz -> put line from unlikely path to green arrow
Altogether, the RNS provides a variety of tools to analyze and circumvent critical situations in intersection scenarios, while not overloading the driver's awareness.

\begin{figure}[t]
  \centering
  \resizebox{\linewidth}{!}{
  \import{img/}{colored_lane_new.pdf_tex}}  
  \vspace{-0.53cm}
  \caption{Short-term warning elements. Selected pop-up warnings (left) and colored lane (right).}
  \label{fig:popups} 
\end{figure} 



\noindent (iii)~\textbf{Temporal grounding:}
We evaluate temporal grounding in Table \ref{tab:temporal}. Here, it shows that global representations also profit from local representation learning.%, achieving state-of-the-art results in temporally localizing actions in untrimmed videos. 
This hypothesis is further validated in the ablation studies in Table~\ref{tab:train_ablations}, where we ablate both losses for all three settings and show a consistent improvement in the joint loss formulation. 

%Our model achieved comparable results with 
%Called action step localization. Evaluated on Mining Youtube. 




%\input{tables/spatial_temporal_clip}






% \noindent (iv) \textbf{Spatio-temporal Clip :}
% \label{ST_clip}
% Following the current spatio-temporal datasets \citep{jiang2014thumos,gu2018ava} which aim to discriminate the action class from the background class in a short clip, we construct a clip level evaluation where the clip varies from 9 sec to 60 Section  Given an action step, we append the video segments before and after the steps with the same time length of the action step to form the final video clip. This results in 2,895 clips for the spatio-temporal clip grounding evaluation.
% For each clip, the  temporal action intervals occupy 33\% of corresponding videos, which demonstrates the difficulty of the setting. As shown in Table \ref{tab:st_clip}, we observe a similar trend as the full video evaluation where our model outperforms all the baselines. 




\subsection{Ablation study} 
%\vspace{-1mm}
We perform ablation studies with respect to all three settings, spatio-temporal grounding, as well as spatial and temporal grounding alone, reporting performance for spatio-temporal grounding on GroundingYT using mAP with IoU@0.4, on temporal grounding using MiningYT IoU, and on spatial grounding using YC-Inter. pointing game. Additional ablation are in appendix \ref{ablation_sup}. %For each setting, we use the same feature extractor for three modalities as described in Sec 4.1 for a fair comparison. 

% add summary here?
% as they are the less computational evaluation tasks.
%This subset of downstream tasks has been chosen for their simplicity of evaluation and because they cover a wide range of tasks.

\noindent\textbf{Frame selection strategy.} 
We perform an ablation on the possible frame selection strategies for our method (Figure \ref{fig:pipeline}(b) and Section \ref{sinkhorn_main}). In Table \ref{tab:frame_ablations}, \textit{None} uses all frames within the ASR boundary ($U=T$) as our video training data. 
\textit{Global} represents the [CLS] token in text and video. \textit{Local} uses the words and spatio-temporal tokens. In the setting Sinkhorn was not applied, the top $T$ frames with the highest similarity score were selected. When we set spatio-temporal tokens as the selection target, we sum over the scores with respect to each frame to acquire the frame similarity score.
%\textit{Global} utilizes the global sentence resp. frame [CLS] token as the query to rank the top $T$ similar frames as the selected frames for training. \textit{Local} uses the words resp spatial-temporal tokens instead of the CLS token as a query and selects the frames with the closest feature distance. 
It shows that selecting frames based on Sinkhorn selection leads to consistently better results as it enforces more variety of visual concepts but also captures frames with possible groundable objects. It further shows that word tokens are more suitable than the global text CLS token for frame selection. Finally, we see that depending on the task (spatial vs. temporal), a local resp. global representation is better, and a combination of both works best for spatio-temporal grounding. 
%, which improves overall performance.%, leading to better supervision.
We provide runtime analysis of such frame selection strategy in the appendix \ref{runtime}.
% \noindent\textbf{Number of frames for training.} We tested different video lengths $T$ used for training. As shown in Table \ref{subtab:ablations2}, selecting less frames for training significantly causes the performance to drop. We hypothesize that not only does the model fail to capture the temporal dynamics with less frames, but loses some frames with groundable objects in the sentence while training. We also found that when the number of frames increases, more irrelevant frames might be selected during training, which decreases the performance.
\begin{table}[!t]
  \centering\small
  \caption{%
    Ablation study on dual-form approximate rank loss.
  }
  \vspace{-3pt}
  % \renewcommand{\arraystretch}{0.8}
  \setlength{\tabcolsep}{2.4mm}{
    \begin{tabular}{l|cccccc}
    \toprule
    \multirow{2}{*}{Loss} & \multicolumn{2}{c}{\textbf{IoU = 0.1}} & \multicolumn{2}{c}{\textbf{IoU = 0.3}} & \multicolumn{2}{c}{\textbf{IoU = 0.5}} \\
    & R@1 & R@5 & R@1 & R@5 & R@1 & R@5  \\
    \midrule
    $\mathcal{L}_{bce}$  & 0.05  & 0.51 & 0.01 & 0.10 & 0.00 & 0.01 \\
    $\mathcal{L}_{nce}$  & 5.26  & 13.65 & 4.09 & 10.90 & 2.32 & 6.73 \\
    $\mathcal{L}_{ar}$   & 10.08  & 22.02 & 8.15 & 18.47 & 4.80 & 12.04 \\
    \midrule
    $\mathcal{L}_{dar}$  & \textbf{11.03}  & \textbf{22.99} & \textbf{8.83} & \textbf{19.48} & \textbf{5.23} & \textbf{13.18} \\
    \bottomrule
    \end{tabular}
  }
  \vspace{-8pt}
  \label{tab:ablation_loss}
\end{table}

%\vspace{-0.1cm}
\noindent\textbf{Global and local loss.} As mentioned in the spatio-temporal evaluation, both features contribute to the final grounding result. We test the model by ablating out each loss. 
Table \ref{tab:train_ablations} shows that each loss not only contributes to the spatio-temporal grounding on the GYT, but also that the whole is more than the sum of its parts (losses) since this task requires both spatial and temporal detection. The reduced impact of the global loss in the case of YC-Inter is that this is a pure spatial grounding dataset (no background frames) without temporal detection, and the local loss plays a more critical role. We observe the same patterns in the temporal grounding result for MYT, where spatial localization is not directly contributing to the final performance. We tried out the same ablation using in the S3D backbone in supplement.
%We provide runtime analysis of different losses in the appendix \ref{runtime}.
%By comparing the results for spatio-temporal grounding in untrimmed videos (Table 1) vs. spatial grounding in trimmed videos (Table 3),  we can further see the impact of the proposed joint representation.

%\bc{to appendix}




% adding the global loss improves the ground performance. This results also shows that spatial grounding benefits from global representation learning. In the spatio-temporal setting, the performance without a global or local loss outperforms other baselines.

% \noindent\textbf{Dataset for training.} As mentioned in Section \ref{dataset}, we trained models with data with food categories. In Table \ref{subtab:ablations4}, we also tested our model trained with a larger set of food and entertaining called HowTo370K used in \citep{han2022temporal}. The full set of HowTo100M contains a total of 1M long videos, which is five times the size of our dataset. We found training with our 200K videos reaches similar performance with much less training hours.

% \noindent\textbf{Affect of audio in training and testing.} Unlike text which describes a discrete concept as a target to ground, audio serves as a continuous representation that is highly relevant to the temporal information. For example, we can determine an action started when we hear a ``cracking'' sound. In Table \ref{subtab:ablations5}, we tested our model using the additional audio modality by expanding our architecture and loss from VT to VAT. We found when training and testing with audio, the spatio-temporal result increases while the spatial-only result remains the same. This validates our assumption that audio contributes more to temporal understanding. When we trained on audio and tested without audio, the performance increases over the VT model, showing that the audio serves as useful supervision for better video/text representations. More details are presented in the supplement. 

\subsection{Qualitative results}
\vspace{-1mm}
We visualize our spatio-temporal result in Figure \ref{fig:visualization}. For the GLIP model, we output the bounding box with the highest confidence score and visualize its center point. We found GLIP model focuses on the salient object while our model focuses more on human-object interaction.



 \section{Conclusion}
 In this paper, we have presented a tactile manipulation system that is able to rotate different objects without vision. We showed an end-to-end reinforcement learning framework to learn tactile dexterity over the proposed system. We carried out experiments both in simulation and real to demonstrate its effectiveness. Our work demonstrated that we are able to achieve tactile dexterity as humans in real for the first time. In the future, there are many promising future directions to investigate, such as exploring the use of a more dense contact sensor array and scaling up the system to solve more diverse tasks. We hope that our work can pave the way for more intelligent robot hands.

{\small
\bibliographystyle{ieee_fullname}
\bibliography{egbib}
}

\end{document}
