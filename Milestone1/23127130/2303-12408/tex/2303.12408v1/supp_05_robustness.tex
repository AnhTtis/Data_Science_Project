\section{Robustness Study}

\begin{table}
    \centering
    \resizebox{\linewidth}{!}{
    \begin{tabular}{@{}l|cccccccc@{}}
    \toprule
    $r_0$ (m) & 0.01 & 0.03 & $0.05^*$ & 0.1 & 0.15 & 0.2 & 0.3 & 0.5 \\
    PSNR & 32.95 & 33.70 & 34.07 & 34.50 & 34.39 & 33.93 & 33.34 & 31.80\\
    \midrule
    $R_{\text{max}}$ (m) & 25 & 50 & 100 & 200 & $300^{*}$ & 500 & 1000 & \\
    PSNR & 28.43 & 33.96 & 34.31 & 34.23 & 34.07 & 33.93 & 33.69 & \\
    \bottomrule
    \end{tabular}
    }
    \caption{Quantitative results in \textit{BistroBike} (max depth$\seq 220$) for different hyperparameters. We mark * for the default values.}
    \label{tab:hyperparameter}
\end{table}

\begin{table}
    \centering
    \resizebox{\linewidth}{!}{
    \begin{tabular}{@{}l|ccccccccc@{}}
    \toprule
    $r$ (m) & 0 & 0.1 & 0.2 & 0.5 & $1^*$ & 1.5 & 2 & 3 & 5 \\
    PSNR & 35.38 & 35.47 & 35.45 & 35.26 & 34.74 & 33.15 & 31.17 & 27.36 & 22.02\\
    \bottomrule
    \end{tabular}
    }
    % \caption{Out of distribution test. $r$ is the distance between the center of training camera trajectory and position of test view. $r=1^*$ is the distance of training positions.}
    \caption{Out of distribution test. $r$ is the distance between the center of training camera trajectory and position of test view.}
    % \caption{Out-of distribution test. $r\seq1$ is radius of captured trajectory.}
    % \caption{Out-of distribution test. 1m is captured trajectory's radius.}
    \label{tab:ood}
\end{table}
In this section, we analyze the effects of various components to the reconstruction quality.
In ~\cref{tab:hyperparameter}, we study the effects of hyperparameters.
EgoNeRF shows robust performance regardless of the choice of $r_0$ and $R_{\text{max}}$ unless $R_{\text{max}}$ is too small compared to the scene size.

\begin{figure}
    \centering
    \includegraphics[width=\linewidth]{supp_figures/rebuttal_depth.pdf}
    \caption{Out of distribution test. $r$ is the distance between the center of training camera trajectory and position of test view.}
    \label{fig:depth_study}
\end{figure}

Also, we compare the quality of reconstructed images at various depths in ~\cref{fig:depth_study}.
EgoNeRF outperforms regular spherical grid and Cartesian grid, especially in the near region.
It supports our claim that Cartesian grid or regular spherical grid has insufficient resolution at nearby regions and is extravagant for far objects.

We further provide the quality of rendering at various distances from the original trajectory ($r=1$) in ~\cref{tab:ood}.
We noticed only minimal quality degradation for $r<1$ and $1<r\leq 3$.
Only when the viewing position is extremely far ($r\geq5$), there exists a noticeable performance decrease due to the unseen regions and the unconstrained scene depth.