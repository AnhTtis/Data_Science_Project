
\section{Vector-Matrix Decomposition of Balanced Spherical Feature Grids}
In this section, we describe the vector-matrix factorization of our balanced spherical feature grids.
As mentioned in Sec. 3.1. of the main manuscript, we model the radiance fields as 3D/4D tensors, which map a 3D position vector to volume density $\sigma$ and appearance feature vector.
Inspired by \cite{chen2022tensorf}, we decompose the 3D/4D tensor into low-rank tensor components.

\paragraph{VM Decomposition}
VM decomposition or vector-matrix decomposition, proposed by~\cite{chen2022tensorf}, decomposes a 3D tensor $\mathbf{T}\in \mathbb{R}^{I\times J\times K}$ into multiple vectors and matrices:
\begin{equation}
    \mathbf{T} = \sum_{n=1}^{N_1}\mathbf{v}_n^1 \otimes \mathbf{M}_n^{2,3}+\sum_{n=1}^{N_2}\mathbf{v}_n^2\otimes\mathbf{M}_n^{3,1}+\sum_{n=1}^{N_3}\mathbf{v}_n^3\otimes\mathbf{M}_n^{1,2},
\label{eq:vm_decomposition}
\end{equation}
where $\otimes$ denotes outer product, $\mathbf{v}_n^1\in\mathbb{R}^I$, $\mathbf{v}_n^2\in\mathbb{R}^J$, $\mathbf{v}_n^3\in\mathbb{R}^K$, and $\mathbf{M}_n^{2,3}\in\mathbb{R}^{J\times K}$, $\mathbf{M}_n^{3,1}\in\mathbb{R}^{K\times I}$, $\mathbf{M}_n^{1,2}\in\mathbb{R}^{I\times J}$ are vector and matrix factors for three modes of $n$th component respectively.
In general, $N_1, N_2, N_3$ have different values, but we use the same number of components for each mode for simplicity. i.e. $N_1=N_2=N_3=N$.
Then, \cref{eq:vm_decomposition} can be expressed as
\begin{equation}
    \mathbf{T} = \sum_{n=1}^N\sum_{m\in\{1, 2, 3\}}\mathcal{A}_{n}^m,
\label{eq:vm_decomposition_simple}
\end{equation}
where $\mathcal{A}_n^1 = \mathbf{v}_n^1\otimes\mathbf{M}_n^{2,3}$, $\mathcal{A}_n^2 = \mathbf{v}_n^2\otimes\mathbf{M}_n^{3,1}$, $\mathcal{A}_n^3 = \mathbf{v}_n^3\otimes\mathbf{M}_n^{1,2}$.

\paragraph{VM Decomposition of Balanced Spherical Feature Grids}
Our density feature grid $\mathcal{G}_\sigma$ is a 3D tensor of $\mathbb{R}^{2N_r^y\times N_\theta^y \times N_\phi^y}$.
The overset grid $\mathcal{G}_\sigma$ is a union of two tensors $\mathcal{G}_\sigma^{\text{Yin}}$ and $\mathcal{G}_\sigma^{\text{Yang}}\in\mathbb{R}^{N_r^y\times N_\theta^y \times N_\phi^y}$.
Each 3D tensor is further decomposed into vector and matrix factors using \cref{eq:vm_decomposition_simple}:
\small
\begin{align}
    \mathcal{G}_\sigma^y &\seq \sum_{n=1}^{N_\sigma}\mathbf{v}_{\sigma,n}^{y,R}\sotimes\mathbf{M}_{\sigma,n}^{y,\Theta\Phi}\splus\mathbf{v}_{\sigma,n}^{y,\Theta}\sotimes\mathbf{M}_{\sigma,n}^{y,\Phi R}\splus\mathbf{v}_{\sigma,n}^{y,\Phi}\sotimes\mathbf{M}_{\sigma,n}^{y,R\Theta}\nonumber\\
    &\seq \sum_{n=1}^{N_\sigma} \sum_{m\in{R\Theta\Phi}} \mathcal{A}_{\sigma, n}^{y, m}\text{,}\quad y\in\{\text{Yin, Yang}\}\text{.}
\end{align}
\normalsize

In contrast, our appearance gird $\mathcal{G}_a\in\mathbb{R}^{2N_r^y\times N_\theta^y\times N_\phi^y\times C}$ is a 4D tensor which has additional $C$-dimensional neural appearance features.
Since the mode of appearance feature does not need high dimension as spatial modes ($R,\Theta,\Phi$), we assign only vector components $\mathbf{b}$ for this mode, instead of matrix components from~\cite{chen2022tensorf}.
Specifically, $\mathcal{G}_a$ also consists of two tensors $\mathcal{G}_a^{\text{Yin}}$ and $\mathcal{G}_a^{\text{Yang}}\in\mathbb{R}^{N_r^y\times N_\theta^y\times N_\phi^y \times C}$ and each are factorized as following:
\small
\begin{align}
    \mathcal{G}_a^y &\seq \sum_{n=1}^{N_a}\mathbf{v}_{a,n}^{y,R}\sotimes \mathbf{M}_{a,n}^{y,\Theta\Phi}\sotimes\mathbf{b}_{3n-2}^y \splus\mathbf{v}_{a,n}^{y,\Theta}\sotimes \mathbf{M}_{a,n}^{y,\Phi R}\sotimes\mathbf{b}_{3n-1}^y \nonumber\\
    &\qquad \splus\mathbf{v}_{a,n}^{y,\Phi}\sotimes \mathbf{M}_{a,n}^{y,R\Theta}\sotimes\mathbf{b}_{3n}^y \nonumber\\
    &\seq \sum_{n=1}^{N_a} \mathcal{A}_{a,n}^{y,R}\sotimes \mathbf{b}_{3n-2}^{y} \splus \mathcal{A}_{a,n}^{y,\Theta}\sotimes \mathbf{b}_{3n-1}^{y}\splus\mathcal{A}_{a,n}^{y,\Phi}\sotimes \mathbf{b}_{3n}^{y}.
    \label{eq:vm_decomposition_appearance}
\end{align}
\normalsize
$\mathbf{B}\in\mathbb{R}^{C\times6N_a}$ in Fig.4 of the main manuscript is a matrix obtained by stacking all $\mathbf{b}^y$s columnwise.
By using $\mathbf{B}$ matrix, we can calculate \cref{eq:vm_decomposition_appearance} with simple matrix multiplication.

\paragraph{Querying Values from Grids}
In the volume rendering pipeline, the volume density $\sigma$ and color $c$ are queried from our feature grids along the camera rays:
\begin{equation}
    \sigma(\mathbf{x})=\mathcal{T}(\mathcal{G}_\sigma,\mathbf{x}), \quad c(\mathbf{x}, \mathbf{d})=f_{\text{MLP}}(\mathcal{T}(\mathcal{G}_a, \mathbf{x}),\mathbf{d}),
\end{equation}
where $\mathbf{x}$, $\mathbf{d}$ are querying position and viewing direction respectively, and $\mathcal{T}$ is a trilinear interpolation operator, as denoted in Eq. (5) of the main manuscript.
Furthermore, we can reduce computational burden by replacing trilinear interpolation with linear/bilinear interpolation of vector/matrix factors.