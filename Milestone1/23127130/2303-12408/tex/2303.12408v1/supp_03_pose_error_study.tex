\newcommand{\ssim}{\,{\sim}\,}
\section{Analysis on Camera Parameter Error}
\begin{figure}
    \centering
    \includegraphics[width=\linewidth]{supp_figures/noise_study.pdf}
    \caption{Qualitative results under the Gaussian perturbation $\epsilon\ssim(0,0.005)$ in different models (a) EgoNeRF, (b) EgoNeRF with coarser grid ($100^3$), and (c) MLP-based method mip-NeRF 360~\cite{Barron_2022_CVPR}. (d) Quantitative results of injecting different levels of Gaussian noise into camera poses.}
    \label{fig:noise_study}
\end{figure}
In this section, we analyze the effects of errors in camera parameters.
While EgoNeRF is able to reconstruct precise 3D scenes and synthesize high-quality novel views under perfect camera parameters in synthetic \textit{OmniBlender} dataset, our approach shows a degraded performance when the camera poses have errors in real-world scenes like prior works.
To further study the effects of the camera pose error, we train EgoNeRF with different grid resolutions ($300^3$ which is identical to our original setup, and $100^3$) and MLP-based approach mip-NeRF 360~\cite{Barron_2022_CVPR} under various levels of camera pose errors.
We perturb the camera pose by adding Gaussian noises $\epsilon\ssim(0,\sigma^2)$ with different levels of variance in the \textit{BarberShop} scene in \textit{OmniBlender}.

As shown in \cref{fig:noise_study} (d), injecting a higher level of noise reduces the performance across all the methods consistently.
EgoNeRF with the default parameter (resolution of $300^3$) outperforms other baselines amidst a negligible amount of noise (variance of 0.001), which is coherent with the main results.
When the level of noise increases, however, the performance of our model with fine resolution degrades rapidly and reaches a similar level of EgoNeRF with coarse resolution.
The MLP-based approach~\cite{Barron_2022_CVPR} shows better performance in the presence of high level of noise (greater than 0.01).

In \cref{fig:noise_study} (a) to (c), we visualize the qualitative results in the noise level 0.005 of which PSNR values from different methods are comparable.
As shown in the red box of \cref{fig:noise_study} (a), we observe a noisy artifact nearby the fine structure of the close objects in EgoNeRF.
On the other hand, EgoNeRF with coarse resolution does not show such a phenomenon in the red box of \cref{fig:noise_study} (b).
In contrast, EgoNeRF with both fine and coarse resolution does not make the noisy artifact in the far-away regions (yellow box).
We hypothesize that if the camera pose noise is non-negligible with relative to the grid size, the wrong camera parameters cause the camera ray for fine objects to hit the wrong neighborhood grids, which leads to multiple erroneous reconstructions of fine structures.
Since our distance-adaptive balanced spherical feature grid has a small grid size near the center and the grid has a larger volume at far regions, the noisy artifact only appears at the close region.
As shown in the blue box in \cref{fig:noise_study}, the MLP-based method shows blurry artifacts amidst the noise in the camera pose in contrast to the noisy artifact in grid-based methods.
This may be because MLP output naturally interpolates the values observed in the training set.
\vspace{-0.5em}