% CVPR 2023 Paper Template
% based on the CVPR template provided by Ming-Ming Cheng (https://github.com/MCG-NKU/CVPR_Template)
% modified and extended by Stefan Roth (stefan.roth@NOSPAMtu-darmstadt.de)

\documentclass[10pt,twocolumn,letterpaper]{article}

%%%%%%%%% PAPER TYPE  - PLEASE UPDATE FOR FINAL VERSION
% \usepackage[review]{cvpr}      % To produce the REVIEW version
% \usepackage{cvpr}              % To produce the CAMERA-READY version
\usepackage[pagenumbers]{cvpr} % To force page numbers, e.g. for an arXiv version

% Include other packages here, before hyperref.
\usepackage{graphicx}
\usepackage{amsmath}
\usepackage{amssymb}
\usepackage{booktabs}
\usepackage{multirow}
\usepackage{array}
\usepackage[accsupp]{axessibility}
\usepackage{listings}

% It is strongly recommended to use hyperref, especially for the review version.
% hyperref with option pagebackref eases the reviewers' job.
% Please disable hyperref *only* if you encounter grave issues, e.g. with the
% file validation for the camera-ready version.
%
% If you comment hyperref and then uncomment it, you should delete
% ReviewTempalte.aux before re-running LaTeX.
% (Or just hit 'q' on the first LaTeX run, let it finish, and you
%  should be clear).
\usepackage[pagebackref,breaklinks,colorlinks]{hyperref}
\usepackage{subcaption}

% Support for easy cross-referencing
\usepackage[capitalize]{cleveref}
\crefname{section}{Sec.}{Secs.}
\Crefname{section}{Section}{Sections}
\Crefname{table}{Table}{Tables}
\crefname{table}{Tab.}{Tabs.}
\Crefname{figure}{Figure}{Figures}
\crefname{figure}{Fig.}{Figs.}


%%%%%%%%% PAPER ID  - PLEASE UPDATE
\def\cvprPaperID{361} % *** Enter the CVPR Paper ID here
\def\confName{CVPR}
\def\confYear{2023}

\usepackage{colortbl}
\usepackage[table,dvipsnames]{xcolor}
\usepackage{xcolor}
\def\todo#1{{\color{purple}{\small\bf\sf#1}}}
\def\diff#1{{\color{magenta}{\small\bf\sf#1}}}

\definecolor{tab_yellow}{rgb}{1, 1, 0.7}
\definecolor{tab_orange}{rgb}{1, 0.85, 0.7}
\definecolor{tab_red}{rgb}{1, 0.7, 0.7}

\begin{document}

%%%%%%%%% TITLE - PLEASE UPDATE
\title{Balanced Spherical Grid for Egocentric View Synthesis}

\author{Changwoon Choi$^1$, Sang Min Kim$^1$, Young Min Kim$^{1,2}$\\
{\small $^1$Dept. of Electrical and Computer Engineering, Seoul National University, Korea}\\
{\small $^2$Interdisciplinary Program in Artificial Intelligence and INMC, Seoul National University}\\
% Dept. of Electrical and Computer Engineering, Seoul National University, Korea\\
% {\tt\small changwoon.choi00@gmail.com, \{tkdals9082, youngmin.kim\}@snu.ac.kr}
}


\twocolumn[{
\renewcommand\twocolumn[1][]{#1}
\maketitle
\begin{center}
    \centering
        \captionsetup{type=figure}
        % \includegraphics[width=\linewidth]{figures/egonerf_teaser.pdf}
        \includegraphics[width=\linewidth]{figures/egonerf_teaser_lowres.pdf}
        \captionof{figure}{We propose a practical solution to reconstruct large-scale scenes from a short egocentric  video. (a) Our scalable capturing setup observes the holistic environment by casually swiping a selfie stick with an omnidirectional camera attached.
        %We take a casually captured omnidirectional video as our input. 
        (b) Then we optimize our balanced spherical feature grids which are tailored for the outward-looking setup. 
        (c) EgoNeRF can quickly train and render high-quality images at nearby positions.
        Project page: \url{https://changwoon.info/publications/EgoNeRF}
        %(c) We demonstrate the rendered images of our approach from a novel viewpoint.
        } 
        \label{fig:teaser}

\end{center}
}]


\maketitle

%%%%%%%%% ABSTRACT %%%%%%%%%
\begin{abstract}
\vspace{-0.5em}
We present EgoNeRF, a practical solution to reconstruct large-scale real-world environments for VR assets.
% Given a few seconds of casually captured 360 video, EgoNeRF can efficiently build neural radiance fields which enable high-quality rendering from novel viewpoints.
Given a few seconds of casually captured 360 video, EgoNeRF can efficiently build neural radiance fields.
Motivated by the recent acceleration of NeRF using feature grids, we adopt spherical coordinate instead of conventional Cartesian coordinate.
Cartesian feature grid is inefficient to represent large-scale unbounded scenes because it has a spatially uniform resolution, regardless of distance from viewers.
%Cartesian feature grid which has uniform resolution ignorant of distance from viewers' position is inefficient to represent large-scale or unbounded scenes.
The spherical parameterization better aligns with the rays of egocentric images, and yet enables factorization for performance enhancement.
However, the na\"ive spherical grid suffers from singularities at two poles, and also cannot represent unbounded scenes. % or points far from the origin.
To avoid singularities near poles, we combine two balanced grids, which results in a quasi-uniform angular grid.
% We also adapt to sample density by partitioning the radial grid exponentially to the radius and eventually placing an environment map at infinity. 
We also partition the radial grid exponentially and place an environment map at infinity to represent unbounded scenes. 
% The sparse density fields can stably converge by deploying resamping techniques.
% Together, we increase the number of valid samples to train NeRF volume for the omnidirectional set-up.
Furthermore, with our resampling technique for grid-based methods, we can increase the number of valid samples to train NeRF volume.
We extensively evaluate our method in our newly introduced synthetic and real-world egocentric 360 video datasets, and it consistently achieves state-of-the-art performance.
\vspace{-0.5em}
\end{abstract}

%%%%%%%%% BODY TEXT %%%%%%%%%
\section{Introduction}

The ability to reason about plans is critical for performing long-horizon tasks \citep{erol1996hierarchical, sohn2018hierarchical, sharma-etal-2022-skill}, compositional generalization \citep{corona-etal-2021-modular} and generalization to unseen tasks and environments \citep{shridhar2020alfred}.
Consider a simple long-horizon planning scenario where a robot is tasked with preparing a meal and serving it on the table. 
This presents a non-trivial planning problem since the agent needs to understand the sequence of operations required to perform the task and search for the relevant objects in the unfamiliar environment by interacting with various objects. %



Large language models have been recently shown to possess commonsense knowledge about the world such as object affordances and physical dynamics \citep{ouyang2022training,chowdhery2022palm}.
Early approaches considered text based environments and fine-tuned PLMs to predict actions given the history of past observations and actions \citep{jansen-2020-visually,micheli-fleuret-2021-language,yao-etal-2020-keep}.
Recent work has used this ability to reason about plans from text instructions in simulated household environments with simplifying assumptions such as text-only environment observations or feedback \citep{huang2022language,ahn2022can,li2022pre,logeswaran-etal-2022-shot}.


We focus on \emph{visually grounded planning} with PLMs --- the ability to adapt plans based on interaction and visual feedback from the environment.
While PLMs have strong planning commonsense priors, predictions from a PLM may not be directly realizable in the environment since the observation and action spaces are unknown.
This requires \emph{grounding} the PLM in the environment and adapting it to observe visual feedback, which is highly non-trivial.
Some prior works assume the availability of a pre-trained affordance function \citep{ahn2022can} or a success detector \citep{mirchandani2021ella}.
Notably, SayCan \citep{ahn2022can} completely decouples the PLM from observation information by selecting actions that have both high affordability (through a pre-trained affordance model) and high PLM likelihood.
Although this partially addresses the grounding problem, the use of visual feedback for action affordance alone is limited.
Often an agent must choose one of many affordable actions using information from observations.
For example, a driving agent should re-navigate and possibly turn around when encountering a ``road closed'' sign, but both turning around and driving forward are indistinguishable to SayCan because they are both affordable and the PLM is blind to observations.

Another workaround explored in prior work is translating the information in the visual observations to text using a pre-trained captioning system \citep{shridhar2021alfworld,huang2022language}.
However, it can be difficult to faithfully describe an image in words and information is lost in this inherently noisy process, which limits the information available to the planner.



Recent work shows that PLMs can be adapted for various natural language tasks by inserting tunable embeddings or soft prompts at the input of the PLM (also called prompt tuning or prefix tuning)~\citep{li-liang-2021-prefix,lester-etal-2021-power}.
This approach also extends to multi-modal understanding tasks such as image captioning \citep{mokady2021clipcap} and VQA \citep{tsimpoukelli2021multimodal} where images are encoded as soft prompts and finetuned for the target task.
Transformer based architectures have also been successfully applied to offline Reinforcement Learning in recent work \citep{chen2021decision,janner2021offline,li2022pre,reid2022can}.

Taking inspiration from these works, we propose the simple approach of embedding visual observations (`visual prompts') and \textit{directly inserting them as PLM input embeddings}.
The visual encoder and PLM are jointly trained for the target task, an approach we call \textbf{\oursfull}~(\ours).
By teaching the PLM to use observations for planning in an end to end manner, we remove the dependency on external data such as captions and affordability information that was used in prior work.
We show that this simple approach performs better than prior PLM-based planning approaches on two embodied planning benchmarks based on ALFWorld~\citep{shridhar2021alfworld} and Virtualhome~\cite{puig2018virtualhome}.



\section{Related work}
\noindent \textbf{Implict Neural Representation}.
Implicit neural representations (also known as coordinate-based representations) are a popular way to parameterize content of all kinds, such as audio, images, video, or 3D scenes~\cite{FFL, siren, srn, NeRF}.
Recent works \cite{NeRF, DeepSDF, occnet, srn} build neural implicit fields for geometric reconstruction and novel view synthesis achieving outstanding performance.
The implicit neural representation is continuous, resolution-independent, and expressive, and is capable of reconstructing geometric surface details and rendering photo-realistic images. 
%
While explicit representations like point clouds\cite{points1, NHR}, meshes\cite{NT}, and voxel grids\cite{deepvoxels, occnet, NeuralVolume, voxel1} are usually limited in resolution due to memory and topology restrictions.
%
One of the most popular implicit representations - Neural Radiance Field (NeRF) \cite{NeRF} -  proposes to combine the neural radiance field with differentiable volume for photo-realistic novel views rendering of static scenes. However, NeRF requires optimizing the 5D neural radiance field for each scene individually, which usually takes hours to converge. Recent works\cite{PixelNeRF, ibrnet, MVSNeRF} try to extend NeRF to generalization with sparse input views.
%
In this work, we extend the neural radiance field to a general human reconstruction scenario by introducing conditional geometric code and appearance code. 


\noindent \textbf{3D Model-based Human Reconstruction}
With the emergence of human parametric models like SMPL\cite{SMPL,SMPLX} and SCAPE\cite{SCAPE}, many model-based 3D human reconstruction works have attracted wide attention from academics. Benefiting from the statistical human prior, some works\cite{tex2shape, Multi-Garment, expose, VIBE} can reconstruct the rough geometry from a single image or video. 
However, limited by the low resolution and fixed topology of statistical models, these methods cannot represent arbitrary body geometry, such as clothing, hair, and other details well. 
To address this problem, some works\cite{PIFu, pifuhd} propose to use pixel-aligned features together with neural implicit fields to represent the 3D human body, but still have poor generalization for unseen poses. To alleviate such generalization issues, \cite{pamir, arch, doublefield} incorporate the human statistical model SMPL\cite{SMPL, SMPLX} into the implicit neural field as a geometric prior, which improves the performance on unseen poses. 
Although these methods have achieved stunning performance on human reconstruction, high-quality 3D scanned meshes are required as supervision, which is expensive to acquire in real scenarios. Therefore, prior works\cite{PIFu, pifuhd, pamir, arch} are usually trained on synthetic datasets and have poor generalizability to real scenarios due to domain gaps. To alleviate this limitation, 
some works\cite{neuralbody, Anim-NeRF, animnerf_zju, humannerf, arah, a-nerf}  combine neural radiance fields\cite{NeRF} with SMPL\cite{SMPL} to represent the human body, which can be rendered to 2D images by differentiable rendering. 
Currently, some works\cite{gpnerf, genebody, NHP, keypointNeRF, doublediffuse, doublefield} can quickly create neural human radiance fields from sparse multi-view images without optimization from scratch.
While these methods usually rely on accurate SMPL estimation which is not always applicable in practical applications. 
% We introduce a xxx

% identity-specific models, like NeuralBody\cite{neuralbody}
% generalizable models, 
% SMPL\cite{SMPL}, SMPLX\cite{SMPLX}, SCAPE\cite{SCAPE}, Tex2Shape\cite{tex2shape}, Multi-Garment Net\cite{Multi-Garment}, VIBE\cite{VIBE}, Expose\cite{expose}, NeuralBody\cite{neuralbody}, Anim-NeRF\cite{animnerf_zju, animnerf}, Neural Actor\cite{neuralactor}, SelfRecon\cite{selfrecon}, HumanNeRF\cite{humannerf}, PIFu\cite{PIFu}, PIFuHD\cite{pifuhd}, Pamir\cite{pamir}, Arch\cite{arch}, Double Field\cite{doublefield}, GNR\cite{genebody}, NHP\cite{NHP}, GPNeRF\cite{gpnerf}, KeypointNeRF\cite{keypointNeRF}, DoubleDiffuse\cite{doublediffuse}











\begin{figure*}[ht]
    \centering
    \vspace{-1em}
    \includegraphics[width=1.0\linewidth]{figures/method/pipeline.png}
    \vspace{-1.5em}
    \caption{\textbf{The architecture of our method}. Given $m$ calibrated multi-view images and registered SMPL, we build the generalizable model-based neural human radiance field. First, we utilize the image encoder to extract multi-view image features, which are used to provide geometric and appearance information, respectively. In order to adequately exploit the geometric prior, we propose the visibility-based attention mechanism to construct a structured geometric body embedding, which is further diffused to form a geometric feature volume. For any spatial point $\mathbf{x}$, we trilinearly interpolate the feature volume $\mathcal{G}$ to obtain the geometric code $\mathbf{g}(\mathbf{x})$. In addition, we also propose geometry-guided attention to obtain the appearance code $\mathbf{a}(\mathbf{x}, \mathbf{d})$ directly from the multi-view image features. We then feed the geometric code $\mathbf{g}(\mathbf{x})$ and appearance code $\mathbf{a}(\mathbf{x}, \mathbf{d})$ into the MLP network to build the neural feature field $(\mathbf{f}, \sigma) = F(\mathbf{g}(\mathbf{x}), \mathbf{a}(\mathbf{x}, \mathbf{d}))$. Finally, we employ volume rendering and neural rendering to generate the novel view image.
    % \Liqian{1) Add section ref. 2) add detailed caption. 3) Modulate the fig, each module corresponds to a sub-section. 4) keep fig text  consistent with method text}
    }
    \vspace{-1em}
    \label{fig:architecture}
\end{figure*}

\section{Feature Grid Representation for EgoNeRF}
\label{sec:feature_grid}

\begin{figure*}
    \centering
    \includegraphics[width=\linewidth]{figures/method_overview_v1.pdf}
    \caption{Overview of our method. (a) We represent radiance fields as features stored in balanced feature grids $\mathcal{G}_\sigma\text{, } \mathcal{G}_a$, (b) which are further decomposed into vector and matrix components. (c) The hierarchical sampling is conducted by obtaining a coarse density grid from the density feature grid on the fly during optimization. (d) The balanced feature grids are optimized with photometric loss.}
    \label{fig:method_overview}
\end{figure*}

EgoNeRF utilizes feature grids to accelerate the neural volume rendering of NeRF.
Feature grids in previous works employ a Cartesian coordinate system, which regularly partition the volume in $xyz$ axis~\cite{liu2020neural, yu2021plenoctrees, hedman2021baking}.
To better express the egocentric views captured from omnidirectional videos, we use a spherical coordinate system.
We modify the spherical coordinate in both angular and radial partitions to efficiently express outward views of the surrounding environment, as described in \cref{subsec:grid_description}.
For rendering and training, the values are interpolated from the feature grid, which can be further factorized to reduce the memory and accelerate the learning~\cite{chen2022tensorf} (\cref{subsec:grid_NeRF}).
With our balanced feature grid, individual cells produce a uniform hitting rate of rays.

\subsection{Balanced Spherical Grid}
\label{subsec:grid_description}

Our balanced spherical grid is composed of the angular partition and the radial partition.
% \vspace{-0.5em}
\paragraph{Angular Partitions}
The desirable angular partition should result in regular shapes and be easily parameterized.
When we regularly partition on the angle parameters, the na\"ive spherical coordinate system results in irregular grid partitions, which severely distort the two polar regions.
Existing regular partitions do not maintain orthogonal axis parameterization~\cite{greger1998irradiance}, which hinders further factorization.

As a simple resolution, we only use the quasi-uniform half of the ordinary spherical coordinate system and combine two of them~\cite{kageyama2004yin}.
The two grids are referred to as the Yin grid and Yang grid, respectively, which have identical shapes and sizes as shown in \cref{fig:teaser} (b) and \cref{fig:method_overview} (a).
Together they can cover the entire sphere with minimal overlap, similar to the two regions of a tennis ball.


The Yin grid is defined as:
\begin{equation}
    (\pi/4 \leq \theta \leq 3\pi/4) \cap (-3\pi/4 \leq \phi \leq 3\pi/4),
\end{equation}
where $\theta$ is colatitude and $\phi$ is longitude.
The axis of another component grid, namely the Yang grid, is located at the equator of the Yin grid:
\begin{equation}
    \begin{bmatrix}x^{\text{Yin}}\\ y^{\text{Yin}}\\ z^{\text{Yin}}\end{bmatrix} = M \begin{bmatrix}x^{\text{Yang}}\\ y^{\text{Yang}}\\ z^{\text{Yang}}\end{bmatrix},
    M =
    \begin{bmatrix}
        -1 & 0 & 0\\
        0 & 0 & 1\\
        0 & 1 & 0
    \end{bmatrix}.
\end{equation}
%
We discretize the angular grid of Yin and Yang grid by $N_\theta^y$ and $N_\phi^y$ partitions for $\theta^y, \phi^y$ axis respectively, where $y\in\{\text{Yin}, \text{Yang}\}$.
The partition is uniform in angles leading to the grid size of
% \vspace{-0.5em}
\begin{equation}
    \Delta \theta^y = {\pi\over 2} {1\over N_\theta^y}, \; \Delta \phi^y = {3\pi\over 2} {1\over N_\phi^y}.
% \vspace{-0.5em}
\end{equation}
% \vspace{-1em}


\paragraph{Radial Partitions}
By adopting the spherical coordinate system, the grid cells cover larger regions as $r$ increases.
This is desired in the egocentric setup, as the panoramic image capture more detailed close-by views of central objects while distant objects occupy a small area on the projected images.
We further make the grid along the $r$ axis increase exponentially for far regions such that the resulting cell exhibit similar lengths in the angular and radial direction.

Specifically, if we denote the radial scales of both the Yin and Yang grids as $r^y$,
\begin{equation}
    r_i^y = r_0 k^{i-1}, \; R_{\text{max}} = r_0 k^{N_r^y - 1},
\end{equation}
where $R_{\text{max}}$ is the radius of the scene bounding sphere and constant value $r_0$ is the radius of the first spherical shell.
We set the grid interval to $r_0$ for the grid interval less than $r_0$.

We can optionally use the environment map for outdoor or large indoor environments. 
Our spherical grid is still bounded by $R_\text{max}$, limiting the size of the environment.
The environment map denoted as $\mathcal{E} \in \mathbb{R}^{H\times W \times 3}$, is a simple equirectangular image and represents what is visible at an almost infinite distance.

\if 0
In this section, we would first introduce our coordinate system and discrete grid suited for the egocentric setup.
Then we describe our radiance field representation, namely geometric and appearance feature grids with vector-matrix decomposed formulation.

%\subsection{Coordinate System}
%\label{subsec:coordinate_system}
The coordinate system and discrete grid for egocentric scene representation should 1) assign higher spatial resolution near the center 2) have uniform grid size on the plane orthogonal to $r$ direction.
The latitude-longitude grid in the spherical polar coordinate satisfies the first condition since the volume of spherical frustum increases along $+r$ direction.
On the other hand, the grid convergence near the pole makes the basic spherical grid not satisfy the second condition.
Therefore we adopt the Yin-Yang grid~\cite{kageyama2004yin} which is an effectively quasi-uniform on the sphere.

The Yin-Yang grid has two component grids, namely Yin grid and Yang grid, which have identical shape and size as shown in \todo{Fig.X}.
Each component grid is a part of latitude-longitude grid and the Yin grid is defined as:
\begin{equation}
    (\pi/4 \leq \theta \leq 3\pi/4) \cap (-3\pi/4 \leq \phi \leq 3\pi/4)
\end{equation}
where $\theta$ is colatitude and $\phi$ is longitude.
The axis of another component grid, namely Yang grid, is located in the equator of the Yin grid and the relation between Yin coordinates and Yang coordinates in the Cartesian coordinates can be expressed with matrix form below:
\begin{equation}
    \begin{bmatrix}x^{\text{Yin}}\\ y^{\text{Yin}}\\ z^{\text{Yin}}\end{bmatrix} = M \begin{bmatrix}x^{\text{Yang}}\\ y^{\text{Yang}}\\ z^{\text{Yang}}\end{bmatrix},
    M =
    \begin{bmatrix}
        -1 & 0 & 0\\
        0 & 0 & 1\\
        0 & 1 & 0
    \end{bmatrix}.
\end{equation}

We discretize the Yin and Yang grid with resolution of $N_r^y$, $N_\theta^y$, $N_\phi^y$ for $r^y, \theta^y, \phi^y$ axis respectively, where $y\in\{\text{Yin}, \text{Yang}\}$.
For $\theta^y$ and $\phi^y$ axis, we divide the grid uniformly.
Namely, the grid size for both Yin and Yang grid becomes
\begin{equation}
    \Delta \theta^y = {\pi\over 2} {1\over N_\theta^y}, \; \Delta \phi^y = {3\pi\over 2} {1\over N_\phi^y}.
\end{equation}
For $r^y$ direction, we increases the grid size exponentially.
When $r_i^y$ is the radius of $i$th spherical shell,
\begin{equation}
    r_i^y = r_0 k^{i-1}, \; R_{\text{max}} = r_0 k^{N_r^y - 1},
\end{equation}
where $R_{\text{max}}$ is the radius of the scene bounding sphere and $r_0$ is the radius of the first spherical shell which is a constant value.
To ensure the monotonically increasing grid size along $+r^y$ direction, we set the grid interval to $r_0$ for the grid interval less than $r_0$.
\fi

\subsection{Feature Grid as Radiance Field}
\label{subsec:grid_NeRF}
Now we describe our radiance field representation with the balanced spherical feature grid.
Given a set of egocentric images with corresponding camera parameters, EgoNeRF aims to reconstruct 3D scene representation and synthesize novel view images.
Instead of regressing for the volume density $\sigma$ and color $c$ from MLP~\cite{mildenhall2021nerf}, we build explicit feature grids of the density $\mathcal{G}_\sigma$ and the appearance $\mathcal{G}_a$ which serve as the mapping function.
Both grids are composed of our balanced spherical grids of resolution $2N_r^y\times N_\theta^y \times N_\phi^y$, as defined in~\cref{subsec:grid_description}.
The density grid $\mathcal{G}_\sigma \in \mathbb{R}^{2N_r^y\times N_\theta^y \times N_\phi^y}$ has a single channel which stores the explicit volume density value, and the appearance grid $\mathcal{G}_a \in \mathbb{R}^{2N_r^y \times N_\theta^y \times N_\phi^y \times C}$ stores $C$-dimensional neural appearance features.
The volume density and color at position $\mathbf{x}$ and viewing direction $\mathbf{d}$ are obtained by
\begin{equation}
    \sigma(\mathbf{x}) = \mathcal{T}(\mathcal{G}_\sigma, \mathbf{x}), \, c(\mathbf{x}, \mathbf{d}) = f_{\text{MLP}}(\mathcal{T}(\mathcal{G}_a , \mathbf{x}), \mathbf{d}),
    \label{eq:querying}
\end{equation}
where $\mathcal{T}$ denotes a trilinear interpolation, and $f_{\text{MLP}}$ is a tiny MLP that decodes the neural feature to color.

Inspired by~\cite{chen2022tensorf}, we further decompose the feature tensor into vectors and matrices as shown in \cref{fig:method_overview} (b):
% to reduce horizontal space
\newcommand{\seq}{\,{=}\!} 
\newcommand{\sotimes}{\,{\otimes}\,} 
\newcommand{\ssplus}{\!{+}\,} 
\newcommand{\splus}{\,{+}\,}
\small
\begin{align}
    \mathcal{G}_\sigma^y &\seq \sum_{n=1}^{N_\sigma} \mathbf{v}_{\sigma, n}^{y, R} \sotimes \mathbf{M}_{\sigma, n}^{y, \Theta \Phi} \ssplus \mathbf{v}_{\sigma, n}^{y, \Theta} \sotimes \mathbf{M}_{\sigma, n}^{y, \Phi R} \ssplus \mathbf{v}_{\sigma, n}^{y, \Phi} \sotimes \mathbf{M}_{\sigma, n}^{y, R \Theta} \nonumber \\
    &\seq \sum_{n=1}^{N_\sigma} \sum_{m\in{R\Theta\Phi}} \mathcal{A}_{\sigma, n}^{y, m}\text{,}
\end{align}
\begin{equation}
    \mathcal{G}_a^y  \seq \sum_{n=1}^{N_a} \mathcal{A}_{a, n}^{y, R}\sotimes \mathbf{b}_{3n-2}^y \splus \mathcal{A}_{a, n}^{y, \Theta}\sotimes \mathbf{b}_{3n-1}^y \splus \mathcal{A}_{a, n}^{y, \Phi}\sotimes \mathbf{b}_{3n}^y\text{,}
\end{equation}
\begin{equation}
    \mathcal{G}_\sigma = \bigcup\limits_{y\in Y} \mathcal{G}_\sigma^y, \mathcal{G}_a = \bigcup\limits_{y\in Y} \mathcal{G}_a^y, Y = \{\text{Yin, Yang}\}\text{,}
\end{equation}
\normalsize
where $\otimes$ represents the outer product and  $\mathbf{v}, \mathbf{b}, \mathbf{M}$ represents vector and matrix factors.
% \todo{TODO: need more descriptions for $\textbf{v}, \textbf{M}, \textbf{b}$.}
This low-rank tensor factorization significantly reduces the space complexity from $\mathcal{O}(n^3)$ to $\mathcal{O}(n^2)$.
With the minimal overhead of storing two grids, we can maintain regular angular components and yet factorize the grid using spherical parameterization.
The full decomposed formulation is described in the supplementary material.
% We describe the detailed vector-matrix decomposed formulation and efficient interpolation strategy in supplementary material.

% \todo{trilinear interpolation and B matrix $\rightarrow$ supplementary}
% efficient trilinear interpolation --> bilinear interpolation + linear interpolation

\if 0
%\subsection{Feature grid as Radiance Field}
%\label{subsec:radiance_feature_grid}
Now, we describe our radiance field representation with explicit feature grid.
Original NeRF~\cite{mildenhall2021nerf} models radiance field as a mapping function with MLP which takes input as 5D coordinate and outputs volume density $\sigma$ and color $c$.
% Then they obtain pixel values by volume rendering technique with querying $\sigma$ and $c$ along camera rays.
We use density grid $\mathcal{G}_\sigma$ and appearance grid $\mathcal{G}_a$ as a mapping function where both are our balanced spherical grid defined in~\cref{subsec:coordinate_system}.
3D tensor $\mathcal{G}_\sigma \in \mathbb{R}^{2N_r\times N_\theta \times N_\phi}$ has a single-channel which stores the volume density value, 4D tensor $\mathcal{G}_a \in \mathbb{R}^{2N_r \times N_\theta \times N_\phi \times C}$ stores $C$-dimensional neural appearance features.
Specifically, the volume density and color from position $\mathbf{x}$ and viewing direction $d$ are obtained from:
\begin{equation}
    \sigma(\mathbf{x}) = \mathcal{T}(\mathcal{G}_\sigma(\mathbf{x})), \, c(\mathbf{x}, d) = f_{\text{MLP}}(\mathcal{T}(\mathcal{G}_a (\mathbf{x}, d)))\texxt{,}
    \label{eq:querying}
\end{equation}
where $\mathcal{T}$ is trilinear interpolation operator, $f_{\text{MLP}}$ is a tiny MLP that decodes neural feature to color.

Inspired from~\cite{chen2022tensorf}, we further decompose the feature tensor into vectors and matrices:
% to reduce horizontal space
\newcommand{\seq}{\,{=}\!} 
\newcommand{\sotimes}{\,{\otimes}\,} 
\newcommand{\ssplus}{\!{+}\,} 
\newcommand{\splus}{\,{+}\,}
\begin{align}
    \mathcal{G}_\sigma^y &\seq \sum_{n=1}^{N_\sigma} \mathbf{v}_{\sigma, n}^{y, R} \sotimes \mathbf{M}_{\sigma, n}^{y, \Theta \Phi} \ssplus \mathbf{v}_{\sigma, n}^{y, \Theta} \sotimes \mathbf{M}_{\sigma, n}^{y, \Phi R} \ssplus \mathbf{v}_{\sigma, n}^{y, \Phi} \sotimes \mathbf{M}_{\sigma, n}^{y, R \Theta} \nonumber \\
    &\seq \sum_{n=1}^{N_\sigma} \sum_{m\in{R\Theta\Phi}} \mathcal{A}_{\sigma, n}^{y, m}\text{,}
\end{align}
\begin{equation}
    \mathcal{G}_a^y  \seq \sum_{n=1}^{N_a} \mathcal{A}_{a, n}^{y, R}\sotimes \mathbf{b}_{3n-2} \splus \mathcal{A}_{a, n}^{y, \Theta}\sotimes \mathbf{b}_{3n-1} \splus \mathcal{A}_{a, n}^{y, \Phi}\sotimes \mathbf{b}_{3n}
\end{equation}
\begin{equation}
    \mathcal{G}_\sigma = \bigcup\limits_{y\in Y} \mathcal{G}_\sigma^y\text{,} \mathcal{G}_a = \bigcup\limits_{y\in Y} \mathcal{G}_a^y, Y = \{\text{Yin, Yang}\}\text{,}
\end{equation}
where $\otimes$ represents cross product and  $\mathbf{v}, \mathbf{b}, \mathbf{M}$ represents vector and matrix factors.
% \todo{TODO: need more descriptions for $\textbf{v}, \textbf{M}, \textbf{b}$.}
This low-rank tensor factorization strategy significantly reduces the space complexity from $\mathcal{O}(n^3)$ to $\mathcal{O}(n^2)$.
% efficient trilinear interpolation --> bilinear interpolation + linear interpolation
\fi
\section{Training EgoNeRF}
\label{sec:EgoNeRF}

We utilize the balanced spherical grids to represent the volume density $\sigma$ and color $c$, which are stored in $\mathcal{G}_\sigma$ and $\mathcal{G}_a$, respectively.
In this chapter, we describe the technical details of the optimization process of our proposed method.


\if 0
Given a set of egocentric captured images/video with corresponding camera parameters, EgoNeRF aims to reconstruct 3D scene representation and synthesize novel view images. %% moved to 3.2
Further elaborate, we obtain pixel values through volume rendering by querying volume density $\sigma$ and color $c$, which are obtained from $\mathcal{G}_\sigma$ and $\mathcal{G}_a$ along camera rays.
In this chapter, we describe technical details in the optimization process of our proposed method.
\fi

\subsection{Hierarchical Density Adaptation}
%\subsection{Importance Sampling for Grid-Based Models}
\label{subsec:importance_sampling}

\if 0
We first train the density volume $\mathcal{G}_\sigma$.
As the scenes typically contain sparse occupied regions, we adapt the coarse-to-fine strategy of the original NeRF~\cite{mildenhall2021nerf}.
While other recent variants using feature grid~\cite{muller2022instant,Hu_2022_CVPR,Sun_2022_CVPR} maintain a dedicated data structure for the coarse grid, we store the values of coarse sample in the same structure as we store the fine grid.
When we update the grid with fine samples, we effectively use filtered values of initial coarse estimate.
This approach efficiently saves memory footprint and shortens the optimization time.

The coarse-to-fine strategy first samples coarse $N_c$ points along the ray to train a density estimate $\sigma$ from which we can sample fine $N_f$ points with importance sampling.
To share the estimates from coarse sample, we distill the value into the neighborhood of the fine grid with the convolution kernel $K$:
\begin{equation}
    \sigma(\mathbf{x_{\text{coarse}}}) = \mathcal{T}(\mathcal{G}_\sigma^{c}(\mathbf{x}_{\text{coarse}})) = \mathcal{T}(K * \mathcal{G}_\sigma(\mathbf{x}_{\text{fine}})).
\end{equation}
We use average pooling kernel as $K$.
It is reasonable to define a coarse grid by convolving the dense grid because our density grid $\mathcal{G}_\sigma$ stores the volume density itself, which has physical meaning, not neural features.

From the volume density values of coarse sampled points, we calculate weights for importance sampling by
\begin{equation}
    w_i = \tau_i (1 - e^{-\sigma_i \delta_i}), i\in [1, N_c],
\end{equation}
where $\delta_i$ is the distance between coarse samples, $\tau_i =\  e^{-\sum_{j=1}^{i-1}{\sigma_j\delta_j}}$ is transmittance.
Then the fine $N_f$ locations are sampled from the probability distribution. %, which is obtained by normalizing the coarse weights, using inverse transform sampling following original NeRF.
The volume density values of fine samples are queried from $\mathcal{G}_\sigma$ as described in Eq.~\ref{eq:querying}.
Finally, the volume density $\sigma$ and color $c$ at $N_c + N_f$ samples are used to render pixel.
\fi

As the scenes typically contain sparse occupied regions, we adapt the hierarchical sampling strategy of the original NeRF~\cite{mildenhall2021nerf} for feature grids.
While other recent variants using feature grid~\cite{muller2022instant,Hu_2022_CVPR,Sun_2022_CVPR} maintain a dedicated data structure for the coarse grid, we exploit our dense geometry feature grid $\mathcal{G}_\sigma$ for the first coarse sampling stage without allocating additional memory for the coarse grid.
% Then the positions of fine ray samples are sampled from the distribution obtained from coarse ray samples.

The hierarchical sampling strategy first samples coarse $N_c$ points along the ray to obtain a density estimate $\sigma$ from which we can sample fine $N_f$ points with importance sampling.
However, evaluating $\sigma$ with dense $\mathcal{G}_\sigma$ at the coarsely sampled points might skip the important surface regions.
Therefore, we obtain $\sigma$ value from a coarser density feature grid which can be obtained on the fly by applying a non-learnable convolution kernel $K$:
\begin{equation}
    \sigma(\mathbf{x_{\text{coarse}}}) = \mathcal{T}(\mathcal{G}_\sigma^{c}, \mathbf{x}_{\text{coarse}}) = \mathcal{T}(K * \mathcal{G}_\sigma, \mathbf{x}_{\text{coarse}}).
\end{equation}
We use the average pooling kernel as $K$.
It is reasonable to define a coarse grid by convolving the dense grid because our density grid $\mathcal{G}_\sigma$ stores the volume density itself, which has a physical meaning, not neural features.

From the volume density values of coarsely sampled points, we calculate weights for importance sampling by
\begin{equation}
    w_i = \tau_i (1 - e^{-\sigma_i \delta_i}), i\in [1, N_c],
\end{equation}
where $\delta_i$ is the distance between coarse samples, $\tau_i =\  e^{-\sum_{j=1}^{i-1}{\sigma_j\delta_j}}$ is the accumulated transmittance.
Then the fine $N_f$ locations are sampled from the filtered probability distribution. %, which is obtained by normalizing the coarse weights, using inverse transform sampling following original NeRF.
% The volume density values of fine samples are queried from $\mathcal{G}_\sigma$ as described in ~\cref{eq:querying}.
Finally, the volume density $\sigma$ and color $c$ at $N_c + N_f$ samples are used to render pixels.



\if 0
Most of the real-world scenes are dominated by unoccupied regions and occluded regions are frequently occur.
To exploit the sparsity and avoid inefficient samples in free space and occluded region, NeRF~\cite{mildenhall2021nerf} takes hierarchical sampling strategy.
They first sample coarse $N_c$ points along ray and query volume density $\sigma$ from network.
Then they sample fine $N_f$ points with importance sampling technique from the distribution of weights obtained from coarse volume densities.
To evaluate volume density values of fine points, NeRF needs another network.
In the same context, instant-NGP~\cite{muller2022instant} maintains additional multiscale occupancy grids to skip ray marching steps, EfficientNeRF~\cite{Hu_2022_CVPR} allocates dense momentum $\sigma$ voxels for valid sampling, and DVGO~\cite{Sun_2022_CVPR} also uses extra coarse density voxel grid.
Maintaining additional coarse feature grids or neural networks not only requires additional memory, but also increases computational burdens which leads slower convergence.
However, we propose an efficient way to draw samples proportional to their contribution in the volume rendering process.


We exploit our dense geometry feature grid $\mathcal{G}_\sigma$ for the first coarse sampling stage without allocating additional memory for coarse grid and avoiding optimizing features.
However, evaluating $\sigma$ with dense $\mathcal{G}_\sigma$ at the coarse sampled points might skip the important surface regions.
Therefore, we obtain $\sigma$ value from a wide range of density feature grid by applying a non-learnable convolution kernel $K$:
\begin{equation}
    \sigma(\mathbf{x_{\text{coarse}}}) = \mathcal{T}(\mathcal{G}_\sigma^{c}(\mathbf{x}_{\text{coarse}})) = \mathcal{T}(K * \mathcal{G}_\sigma(\mathbf{x}_{\text{coarse}})).
\end{equation}
We use average pooling kernel as $K$.
Obtaining coarse grid by convolving a kernel to dense grid is a reasonable approach since our density feature grid $\mathcal{G}_\sigma$ stores the volume density itself, which has physical meaning, not a neural latent feature.
This approach efficiently saves memory footprint and shortens the optimization time.

From the volume density values of coarse sampled points, we calculate weights of each sample by
\begin{equation}
    % w_i = \tau_i (1 - \exp{(-\sigma_i \delta_i)}, i\in [1, N_c],
    w_i = \tau_i (1 - e^{-\sigma_i \delta_i}), i\in [1, N_c],
\end{equation}
where $\delta_i$ is distance between coarse sample, $\tau_i = \exp{(-\sum_{j=1}^{i-1}{\sigma_j\delta_j})}$ is transmittance.
Then the fine $N_f$ locations are sampled from the probability distribution, which is obtained by normalizing the coarse weights, using inverse transform sampling following original NeRF.
The volume density values of fine samples are queried from $\mathcal{G}_\sigma$ as described in Eq.~\ref{eq:querying}.
Finally, the volume density $\sigma$ and color $c$ at $N_c + N_f$ samples are used to render pixel.
\fi

% \subsection{Combining Appearance and Environment Map}
\subsection{Optimization}
%\subsection{Training EgoNeRF}
\label{subsec:training}

The images of EgoNeRF are synthesized by applying the volume rendering equation along the camera ray~\cite{mildenhall2021nerf} and the optional environment map.
Specifically, the points $\mathbf{x}_i = \mathbf{o} + t_i\mathbf{d}$ along the camera ray from camera position $\mathbf{o}$ and ray direction $\mathbf{d}$ are accumulated to find the pixel value by
\begin{equation}
    % C=\sum_{i=1}^{N_c + N_f}\tau_i (1-\exp{(-\sigma_i \delta_i)}c_i, \tau_i = \exp{(-\sum_{j=1}^{i-1}\sigma_j\delta_j)}
    \hat{C}=\sum_{i=1}^{N}\tau_i (1-e^{-\sigma(\mathbf{x}_i) \delta_i})c(\mathbf{x}_i,\mathbf{d}) + \tau_{N+1}c_{\text{env}}(\mathbf{d}).
    \label{eq:rendering}
\end{equation}
% $\tau_i = e^{-\sum_{j=1}^{i-1}\sigma(\mathbf{x}_j)\delta_j}$ is the transmittance, and $\delta_i = t_{i+1}-t_i$ represents  the interval between adjacent samples. $N=N_c + N_f$ is the number of samples as described in~\cref{subsec:importance_sampling}.
$N=N_c + N_f$ is the number of samples as described in~\cref{subsec:importance_sampling}.
$\sigma(\mathbf{x})$ and $c(\mathbf{x}, \mathbf{d})$ are obtained from our balanced feature grids in~\cref{eq:querying}.
Since the size of our feature grid is exponentially increasing along the $r$ direction, we distribute $N_c$ coarse samples exponentially rather than uniformly.
%
The second term in~\cref{eq:rendering} is fetched from the environment map
\begin{equation}
    c_{\text{env}}(\mathbf{d}) = \mathcal{E}(u, v; \mathbf{d}),
\end{equation}
where the sampling position $(u,v)$ is only dependent on the viewing direction $\mathbf{d}$.
% Environment map is jointly optimized with our rendering pipeline as described in Eq.~\ref{eq:rendering}
The effect of the environment map is further discussed in~\cref{subsec:ablation}.

Finally, we optimize the photometric loss between rendered images and training images
\begin{equation}
    \mathcal{L} = \frac{1}{|\mathcal{R}|}\sum_{\mathbf{r} \in \mathcal{R}} \left\lVert \hat{C}(\mathbf{r}) - C(\mathbf{r}) \right\rVert_2^2,
\end{equation}
where $\mathcal{R}$ is a randomly sampled ray batch, $\hat{C}(\mathbf{r}), C(\mathbf{r})$ are rendered and the ground-truth color of the pixel corresponding to ray $\mathbf{r}$.
With the simple photometric loss, our feature grids $\mathcal{G}_\sigma, \mathcal{G}_a$, decoding MLP $f_{\text{MLP}}$, and environment map $\mathcal{E}$ are jointly optimized.
For real-world datasets, in which camera poses are not perfect, we additionally optimize a TV loss~\cite{rudin1994total} at our feature grid to reduce noise.
Furthermore, since our balanced feature grid guarantees a nearly uniform ray-grid hitting rate, EgoNeRF does not need a coarse-to-fine reconstruction approach for robust optimization used in other feature grid-based methods~\cite{Sun_2022_CVPR, chen2022tensorf}.

\if 0
%\paragraph{Rendering Pipeline}
To synthesize images from arbitrary viewpoints, we volume render through camera rays following NeRF~\cite{mildenhall2021nerf}.
Specifically, from camera position $\mathbf{o}$ and ray direction $\mathbf{d}$, one can obtain pixel value by:
\begin{equation}
    % C=\sum_{i=1}^{N_c + N_f}\tau_i (1-\exp{(-\sigma_i \delta_i)}c_i, \tau_i = \exp{(-\sum_{j=1}^{i-1}\sigma_j\delta_j)}
    \hat{C}=\sum_{i=0}^{N - 1}\tau_i (1-e^{-\sigma(\mathbf{x}_i) \delta_i})c_i(\mathbf{x}_i,\mathbf{d}) + \tau_{N}c_{\text{env}}(\mathbf{d}),
    \label{eq:rendering}
\end{equation}
\if 0
\begin{equation}
    \tau_i = e^{-\sum_{j=1}^{i-1}\sigma_j(\mathbf{x}_j)\delta_j}, \delta_i = t_{i+1}-t_i, \mathbf{x}_i = \mathbf{o} + t_i\mathbf{d}
\end{equation}
\fi
where transmittance $\tau_i = e^{-\sum_{j=1}^{i-1}\sigma_j(\mathbf{x}_j)\delta_j}$, interval between adjacent samples $\delta_i = t_{i+1}-t_i$, and querying position $\mathbf{x}_i = \mathbf{o} + t_i\mathbf{d}$, and number of samples $N=N_c + N_f$ which are obtained from~\cref{subsec:importance_sampling}.
$\sigma(\mathbf{x})$ and $c(\mathbf{x}, \mathbf{d})$ is obtained from our balanced feature grid in Eq.~\ref{eq:querying}.
Since the size of our feature grid is exponentially increasing along $r$ direction, we distribute $N_c$ coarse samples exponentially rather than uniformly.

%\paragraph{Environment Map}
%Although our balanced spherical feature grid is able to cover large-scale scenes with efficient memory, we use an additional environment map to be capable of modeling infinite-distant objects.
%The environment map $\mathcal{E}$ is a simple equirectangular image, or equivalent to $\mathbb{R}^{H\times 2H \times 3}$ tensor.
One can fetch color from environment map with arbitrary viewing direction $\mathbf{d}$:
\begin{equation}
    c_{\text{env}} = \text{Sample}(\mathcal{E}(u, v; \mathbf{d})),
\end{equation}
where the sampling position $(u,v)$ is only dependent to the viewing direction $\mathbf{d}$.
% Environment map is jointly optimized with our rendering pipeline as described in Eq.~\ref{eq:rendering}
The effect of environment map is further discussed in~\cref{subsec:results}.

%\paragraph{Optimization}
Finally, we optimize the photometric loss between rendered images and training images
\begin{equation}
    \mathcal{L} = \frac{1}{|\mathcal{R}|}\sum_{\mathbf{r} \in \mathcal{R}} \left\lVert \hat{C}(\mathbf{r}) - C(\mathbf{r}) \right\rVert_2^2,
\end{equation}
where $\mathcal{R}$ is a randomly sampled ray batch, $\hat{C}(\mathbf{r}), C(\mathbf{r})$ are rendered and ground-truth color of pixel corresponding to ray $\mathbf{r}$.
With the simple photometric loss, our feature grids $\mathcal{G}_\sigma, \mathcal{G}_a$, decoding MLP $f_{\text{MLP}}$, and environment map $\mathcal{E}$ are jointly optimized.
For real-world dataset, in which camera poses are not perfect, we additionally optimize a TV loss~\cite{rudin1994total} at our feature grid to reduce noise.
Furthermore, since our balanced feature grid guarantees nearly uniform ray-grid hitting rate, EgoNeRF do not need coarse-to-fine reconstruction approach for robust optimization used in other feature grid base methods~\cite{Sun_2022_CVPR, chen2022tensorf}.
\fi
%\section{Experiments}
\label{sec:experiments}
\subsection{Experimental details}
\paragraph{Datasets} We use three benchmark datasets in CZSL problem, namely Clothing16K~\cite{zhang2022learning}, UT-Zappos50K~\cite{yu2014fine}, and C-GQA~\cite{naeem2021learning}. Clothing16K~\cite{zhang2022learning} contains different types of clothing (\eg, shirt, pants) with color attributes (\eg, white, black). UT-Zappos50K~\cite{yu2014fine} is a fine-grained dataset consisting of different kinds of shoes (\eg, sneakers, sandals) with texture attributes (\eg, leather, canvas). C-GQA~\cite{naeem2021learning} is a split built on top of Stanford GQA dataset~\cite{hudson2019gqa}, composed of extensive common attribute concepts (\eg, old, wet) and object concepts (\eg, dog, bus) in real life. We follow the common data splits of these three datasets 
(see~\cref{tab:data-splits}). 

\begin{table}[h]
    \centering
    \scalebox{0.54}{
    \begin{tabular}{cccccccccc}
        \toprule
         & \multicolumn{3}{c}{Composition} & \multicolumn{2}{c}{Train} & \multicolumn{2}{c}{Val} & \multicolumn{2}{c}{Test}  \\
         \cmidrule(lr){2-4} \cmidrule(lr){5-6} \cmidrule(lr){7-8} \cmidrule(lr){9-10}
         Datasets & $|\mathcal{A}|$ & $|\mathcal{O}|$ & $|\mathcal{A}|\times|\mathcal{O}|$ & $|\mathcal{C}_{s}|$ & $|\mathcal{X}|$ & $|\mathcal{C}_{s}|$ / $|\mathcal{C}_{u}|$ & $|\mathcal{X}|$ & $|\mathcal{C}_{s}|$ / $|\mathcal{C}_{u}|$ & $|\mathcal{X}|$
         \\ \midrule
        Clothing16K~\cite{zhang2022learning} & 9 & 8 & 72 & 18 & 7242 & 10 / 10 & 5515 & 9 / 8 & 3413\\
        UT-Zappos50K~\cite{yu2014fine} & 16 & 12 & 192 & 83 & 22998 & 15 / 15 & 3214 & 18 / 18 & 2914 \\
        C-GQA~\cite{naeem2021learning} & 413 & 674 & 278362 & 5592 & 26920 & 1252 / 1040 & 7280 & 888 / 923 & 5098 \\
        \bottomrule
    \end{tabular}}
    \caption{Summary of data split statistics.}
    \label{tab:data-splits}
    \vspace{-10pt}
\end{table}

\paragraph{Open-world setting} In addition to the standard closed-world setting, we also evaluate our model on the open-world setting~\cite{mancini2021open}, which is neglected by most previous works. The open-world setting considers all possible compositions, which requires a much larger testing space than the closed-world setting during inference. Taking UT-Zappos50K as an example (see~\cref{tab:data-splits}), the closed world only considers 36 compositions in the testing set while the open world considers total 192 compositions, in which $\sim$81\% are ignored under the standard closed-world setting.

\paragraph{Evaluation metrics} Since CZSL models have an inherent bias for seen compositions, we follow the generalized CZSL evaluation protocol~\cite{purushwalkam2019task}. To overcome the negative bias for seen compositions, we apply different calibration terms to unseen compositions and compute the corresponding top-1 accuracy of seen and unseen compositions, where a larger bias makes higher unseen accuracy and lower seen accuracy, and vice versa.  We treat seen accuracy as $x$-axis and unseen accuracy as $y$-axis to derive an unseen-seen accuracy curve. We can then compute the area under curve (AUC), the best harmonic mean, the best seen accuracy, and the best unseen accuracy from the curve. In our experiments, we report these four metrics for evaluation, among which AUC is the most representative and stable metric for measuring CZSL model performance. \hsz{Note that the attribute accuracy or the object accuracy alone does not reflect CZSL performance, because the individual accuracy on attribute or object does not necessarily decide the accuracy of their composition.}

\paragraph{Implementation details}
We use a frozen ViT-B-16~\cite{dosovitskiy2020vit} backbone pretrained with DINO~\cite{caron2021emerging} on ImageNet~\cite{deng2009imagenet} in a self-supervised manner as our visual feature extractor. The ViT-B-16 outputs contain 197 tokens (1 \texttt{[CLS]} and 196 patch tokens) of 768 dimensions. For three attention disentangler modules, we implement one-layer multi-head attention framework following~\cite{vaswani2017attention}, changing the single input to paired inputs for cross-attentions. The embedders $\pi_a$, $\pi_c$, $\pi_o$ are the two-layer MLPs following the previous works~\cite{mancini2021open, zhang2022learning}, projecting the 768-dimension visual features to 300-dimension word embedding space. The word embedding prototypes are initialized with word2vec~\cite{mikolov2013distributed} for all datasets and learnable during training. The composition function $\psi$ is one linear layer. We train our model using Adam optimizer~\cite{kingma2015adam} with a learning rate of $5\times 10^{-6}$ for UT-Zappos50K and Clothing16K, and $5\times 10^{-5}$ for C-GQA. All models are trained with 128 batch size for 300 epochs.

\begin{table*}[t]
    \centering
    \scalebox{0.75}{
    \begin{tabular}{l>{\columncolor{tabcolor}}cccccc>{\columncolor{tabcolor}}cccccc>{\columncolor{tabcolor}}cccccc}
        \toprule
         Closed-world & \multicolumn{6}{c}{Clothing16K} & \multicolumn{6}{c}{UT-Zappos50K} & \multicolumn{6}{c}{C-GQA} \\
         \cmidrule(lr){2-7} \cmidrule(lr){8-13} \cmidrule(lr){14-19}
         Models & AUC & HM & Seen & Unseen & Attr & Obj & AUC & HM & Seen & Unseen & Attr & Obj & AUC & HM & Seen & Unseen & Attr & Obj \\
         \midrule
         SymNet~\cite{li2020symmetry} & 78.8 & 79.3 & 98.0 & 85.1 & 75.6 & 84.1 & 32.6 & 45.6 & 60.6 & 68.6 & 48.2 & 77.0 & 3.1 & 13.5 & 30.9 & 13.3 & 11.4 & 34.6 \\
         CompCos~\cite{mancini2021open} & 90.3 & 87.2 & 98.5 & 96.8 & \textbf{90.2} & 91.8 & 31.8 & 48.1 & 58.8 & 63.8 & 45.5 & 72.4 & 2.9 & 12.8 & 30.7 & 12.2 & 10.4 & 33.9 \\
         GraphEmb~\cite{naeem2021learning} & 89.2 & 84.2 & 98.0 & 97.4 & 90.0 & 93.1 & 34.5 & 48.5 & 61.6 & \textbf{70.0} & \textbf{50.8} & \textbf{77.1} & 3.8 & 15.0 & 32.3 & 14.9 & 13.8 & 33.2 \\
         Co-CGE~\cite{mancini2022learning} & 88.3 & 87.9 & 98.5 & 94.7 & 87.4 & 91.4 & 30.8 & 44.6 & 60.9 & 62.6 & 46.0 & 73.5 & 3.6 & 14.7 & 31.6 & 14.3 & 12.6 & 34.6 \\
         SCEN~\cite{li2022siamese} & 78.8 & 78.5 & 98.0 & 89.6 & 81.2 & 85.4 & 30.9 & 46.7 & \textbf{65.7} & 62.9 & 44.0 & 74.4 & 3.5 & 14.6 & 31.7 & 13.4 & 10.7 & 31.4 \\ 
         IVR~\cite{zhang2022learning} & 90.6 & 86.6 & \textbf{99.0} & 97.0 & 89.3 & \textbf{93.6} & 34.3 & 49.2 & 61.5 & 68.1 & 48.4 & 74.6 & 2.2 & 10.9 & 27.3 & 10.0 & 10.3 & \textbf{37.5} \\
         OADis~\cite{Saini_2022_CVPR} & 88.4 & 86.1 & 97.7 & 94.2 & 84.9 & 93.1 & 32.6 & 46.9 & 60.7 & 68.8 & 49.3 & 76.9 & 3.8 & 14.7 & 33.4 & 14.3 & 8.9 & 36.3 \\
         \midrule
         \framework (ours) & \textbf{92.4} & \textbf{88.7} & 98.2 & \textbf{97.7} & \textbf{90.2} & \textbf{93.6} & \textbf{35.1} & \textbf{51.1} & 63.0 & 64.3 & 46.3 & 74.0 & \textbf{5.2} & \textbf{18.0} & \textbf{35.0} & \textbf{17.7} & \textbf{16.8} & 32.3\\ 
        \bottomrule
    \end{tabular}}
    \caption{Closed-world results on three datasets. We report the area under curve (AUC), the best harmonic mean (HM), the best seen accuracy (Seen), the best unseen accuracy (Unseen), the attribute accuracy (Attr), and the object accuracy (Obj) of the unseen-seen accuracy curve under the closed world-setting. AUC is the core CZSL metric. All models use the same DINO ViT-B-16 backbone.} 
    \label{tab:cw-results}
\end{table*}

\begin{table*}[t]
    \centering
    \scalebox{0.75}{
    \begin{tabular}{l>{\columncolor{tabcolor}}cccccc>{\columncolor{tabcolor}}cccccc>{\columncolor{tabcolor}}cccccc}
        \toprule
         Open-world & \multicolumn{6}{c}{Clothing16K} & \multicolumn{6}{c}{UT-Zappos50K} & \multicolumn{6}{c}{C-GQA} \\
         \cmidrule(lr){2-7} \cmidrule(lr){8-13} \cmidrule(lr){14-19}
         Models & AUC & HM & Seen & Unseen & Attr & Obj  & AUC & HM & Seen & Unseen & Attr & Obj & AUC & HM & Seen & Unseen & Attr & Obj  \\
         \midrule
         SymNet~\cite{li2020symmetry} & 57.4 & 68.3 & 98.2 & 60.7 & 57.6 & 81.2 & 25.0 & 40.6 & 60.4 & 51.0 & 38.2 & \textbf{75.0} & 0.77 & 4.9 & 30.1 & 3.2 & 18.4 & 37.5 \\
         CompCos~\cite{mancini2021open} & 64.1 & 70.8 & 98.2 & 69.8 & 71.7 & 83.7 & 20.7 & 36.0 & 58.1 & 46.0 & 36.4 & 71.1 & 0.72 & 4.3 & 32.8 & 2.8 & 15.1 & 37.8 \\
         GraphEmb~\cite{naeem2021learning} & 62.0 & 68.3 & 98.5 & 69.7 & 71.8 & 82.4 & 23.5 & 40.0 & 60.6 & 47.0 & 37.1 & 69.3 & 0.81 & 4.8 & 32.7 & 3.2 & 17.2 & 36.7 \\
         Co-CGE~\cite{mancini2022learning} & 59.3 & 69.2 & 98.7 & 63.8 & 68.5 & 76.2 & 22.0 & 40.3 & 57.7 & 43.4 & 33.9 & 67.2 & 0.48 & 3.3 & 31.1 & 2.1 & 15.5 & 35.7 \\
         SCEN~\cite{li2022siamese}& 53.7 & 61.5 & 96.7 & 62.3 & 63.6 & 79.1 & 22.5 & 38.0 & \textbf{64.8} & 47.5 & 34.9 & 73.3 & 0.34 & 2.5 & 29.5 & 1.5 & 14.8 & 32.3 \\ 
         IVR~\cite{zhang2022learning} & 63.6 & 72.0 & 98.7 & 69.0 & 70.3 & 84.8 & 25.3 & 42.3 & 60.7 & 50.0 & 38.4 & 71.4 & 0.94 & 5.7 & 30.6 & 4.0 & 16.9 & 36.5 \\
         OADis~\cite{Saini_2022_CVPR} & 53.4 & 63.2 & 98.0 & 58.6 & 57.3 & \textbf{85.4} & 25.3 & 41.6 & 58.7 & \textbf{53.9} & \textbf{40.3} & 74.7 & 0.71 & 4.2 & 33.0 & 2.6 & 14.6 & \textbf{39.7} \\
         \midrule
         \framework (ours) & \textbf{68.0} & \textbf{74.2} & \textbf{99.0} & \textbf{73.1} & \textbf{75.0} & 84.5 & \textbf{27.1} & \textbf{44.8} & 62.4 & 50.7 & 39.9 & 71.4 & \textbf{1.42} & \textbf{7.6} & \textbf{35.1} & \textbf{4.8} & \textbf{22.4} & 35.6 \\ 
        \bottomrule
    \end{tabular}}
    \caption{Open-world results on three datasets. Different from~\cref{tab:cw-results}, open-world setting considers all possible compositions in testing.} 
    \label{tab:ow-results}
\end{table*}

\subsection{Comparison}
\hsznew{To ensure a fair comparison and demonstrate that our improvement over baseline models is not merely by ViT, we adopt ViT backbone to state-of-the-art CZSL models and \emph{re-train} all models.} We compare our method with them: \hsznew{(1)~OADis~\cite{Saini_2022_CVPR} disentangles attribute and object features from spatial convolutional maps;} (2)~SymNet~\cite{li2020symmetry} introduces the symmetry principle of attribute-object transformation and group theory as training objectives; (3)~CompCos~\cite{mancini2021open} extends CZSL to an open-world setting considering all possible compositions during inference, proposing a feasibility score based on data statistics to remove unfeasible compositions; (4)~GraphEmb~\cite{naeem2021learning} and Co-CGE~\cite{mancini2022learning} propose to use graph convolutional networks (GCN) to represent attribute-object relationships and compositions; (5)~SCEN~\cite{li2022siamese} projects visual features to a Siamese contrastive space to capture concept prototypes, and introduces complex state transition module to produce virtual compositions; (6)~IVR~\cite{zhang2022learning} proposes to disentangle visual features into concept-invariant domains from a perspective of domain generalization, by masking specific channels of visual features. 

\paragraph{Closed-world evaluation} In~\cref{tab:cw-results}, we compare our \framework model with the state-of-the-art methods. \framework consistently outperforms others by a significant margin. \framework increases the core metric AUC by 1.8 on Clothing16K, 0.6 on UT-Zappos50K, and 1.4 on C-GQA ($\sim$37\% relatively). Similarly, \framework increases the best harmonic mean (HM) by 0.8\% on Clothing16K, 1.9\% on UT-Zappos50K, and 3.0\% on C-GQA. We notice that SymNet~\cite{li2020symmetry} and SCEN~\cite{li2022siamese} perform badly on Clothing16K. The reason might be that not learning concept prototypes harms the word embedding expressivity on small-scale concepts. We also notice that IVR~\cite{zhang2022learning} performs very well on curated datasets Clothing16K and UT-Zappos50K but badly on larger-scale real-world dataset C-GQA. We hypothesize ideal concept-invariant domains might be difficult to learn from natural images and large-scale concepts of C-GQA. In contrast, our \framework model achieves state-of-the-art performance on all datasets.

\paragraph{Open-world evaluation} In~\cref{tab:ow-results}, we consider the open-world setting to compare our \framework with other methods. Likewise, \framework also performs the best among all methods under open-world setting. \framework increases AUC by 3.9 on Clothing16K, 1.8 on UT-Zappos50K, and 0.48 on C-GQA ($\sim$51\% relatively). \framework also increases the best harmonic mean (HM) by 2.2\% on Clothing16K, 2.5\% on UT-Zappos50K, and 1.9\% on C-GQA ($\sim$33\% relatively). From the above results, \framework surpasses others by a larger margin on open-world AUC and HM than closed-world ones, indicating \framework maintains utmost efficiency when turning from the closed world to the open world. It is worth mentioning that \framework does not apply any special operations (\eg, feasibility masking~\cite{mancini2021open}) for the open world and deals with the two settings in exactly the same way. 
IVR~\cite{zhang2022learning} keeps its performance to a great extent but still lags behind our method significantly.

\begin{table*}[t]
\begin{minipage}[t]{0.44\linewidth}
    \centering
    \scalebox{0.72}{
    \begin{tabular}{lcccc>{\columncolor{tabcolor}}cccc}
        \toprule
         & CA & AA & OA & Reg & AUC & HM & Seen & Unseen\\
         \midrule
         (0) & \xmark & \xmark & \xmark & \xmark & 23.8 & 41.1 & 59.0 & 48.9 \\
         (1) & self & \xmark & \xmark & \xmark & 25.3 & 42.3 & 61.1 & 49.9 \\
         (2) & self & self & self & \xmark & 26.7 & 44.6 & 61.9 & 49.8 \\
         (3) & self & cross & cross & \xmark & 26.9 & 44.5 & \textbf{63.4} &48.7 \\
         (4) & self & cross & cross & \cmark &  \textbf{27.1} & \textbf{44.8} & 62.4 & \textbf{50.7}\\
         
        \bottomrule
    \end{tabular}}
    \caption{Ablate the components in \framework on open-world UT-Zappos50K. CA, AA, and OA denote composition, attribute, and object attention. Reg denotes the regularization term. We test self- or cross-attention for AA and OA.} 
    \label{tab:model-ab}
\end{minipage}
\hspace{2mm}
\begin{minipage}[t]{0.54\textwidth}
\centering
    \scalebox{0.68}{
    \begin{tabular}{ll>{\columncolor{tabcolor}}cccc>{\columncolor{tabcolor}}cccc}
        \toprule
        & & \multicolumn{4}{c}{C-GQA} & \multicolumn{4}{c}{Clothing16K} \\
        \cmidrule(lr){3-6} \cmidrule(lr){7-10}
        & Inference formulation  & AUC & HM & Seen & Unseen & AUC & HM & Seen & Unseen\\
        \midrule
        (0) & $p(c)$ & 4.6 & 16.8 & \textbf{35.1} & 16.0 & \textbf{92.4} & \textbf{88.8} & \textbf{98.2} & \textbf{97.7} \\
        (1) & $p(a) \cdot p(o)$ &  4.0 & 15.8 & 31.4 & 15.1 & 57.3 & 66.3 & 96.7 & 63.1\\
        (2) & $p(c) + p(a) \cdot p(o)$ & \textbf{5.2} & \textbf{18.0} & 35.0 & \textbf{17.7} & 90.4 & 85.9 & 98.2 & 97.0\\
        (3) & $p(c) + \beta \cdot p(a) \cdot p(o)$ & \textbf{5.2} & \textbf{18.0} & 35.0 & \textbf{17.7} & \textbf{92.4} & 88.7 & \textbf{98.2} & \textbf{97.7} \\
        \bottomrule
    \end{tabular}}
    \caption{Results on closed-world Clothing16K and C-GQA using different inference formulations. Rows (0)-(2) respectively represents the cases when $\beta=0.0$, $\beta=+\infty$, and $\beta=1.0$. Row (3) is our inference formulation, which applies an $\beta$ optimized on the validation set.} 
    \label{tab:eval-ab}
\end{minipage}
\end{table*}

\subsection{Ablation study}
\paragraph{Backbone: ResNet \textit{vs} ViT}
\hsznew{Our work leverages ViT as the default backbone to excavate more high-level sub-space information, while ResNet18 is the most common backbone in previous works. 
In~\Cref{tab:backbone}, we compare our \framework to OADis~\cite{Saini_2022_CVPR}  with both backbones. Our \framework performs similarly to OADis with ResNet18, but outperforms it significantly with ViT. Additionally, we present an ablation study on different components of our method with the ResNet18 backbone in the Appendix. These experiments indicate that our model benefits from ViT and all components of our method are effective regardless of the backbone.}

\begin{table}[h]
    \centering
    \scalebox{0.75}{
    \begin{tabular}{ll>{\columncolor{tabcolor}}cc>{\columncolor{tabcolor}}cc>{\columncolor{tabcolor}}cc}
        \toprule
          \multicolumn{2}{l}{Closed-world} & \multicolumn{2}{c}{Clothing16K} & \multicolumn{2}{c}{UT-Zappos50K} & \multicolumn{2}{c}{C-GQA} \\
         \cmidrule(lr){3-4} \cmidrule(lr){5-6} \cmidrule(lr){7-8}
         Backbone & Models & AUC & HM & AUC & HM & AUC & HM \\
         \midrule
         \multirow{2}{*}{ResNet18} & OADis~\cite{Saini_2022_CVPR} & 85.5 & 84.7 & \textbf{30.0} & 44.4 & \textbf{3.1} & 13.6 \\
          & \framework (ours) & \textbf{87.2} & \textbf{85.1}  & 29.5 & \textbf{47.0} & \textbf{3.1} & \textbf{13.7} \\
         \midrule
         \multirow{2}{*}{ViT-B-16} & OADis~\cite{Saini_2022_CVPR} & 88.4 & 86.1  & 32.6 & 46.9 & 3.8 & 14.7 \\
          & \framework (ours) & \textbf{92.4} & \textbf{88.7}  & \textbf{35.1} & \textbf{51.1} & \textbf{5.2} & \textbf{18.0} \\
        \bottomrule
    \end{tabular}}
    \caption{Compare \framework and OADis~\cite{Saini_2022_CVPR} with ResNet18 and ViT.} 
    \label{tab:backbone}
    \vspace{-5pt}
\end{table}

\paragraph{Different parts of \framework} We evaluate the effectiveness of attention disentanglers (composition, attribute, and object attention) and the regularization term in our model. We report the ablation study results on the open-world UT-Zappos50K in~\cref{tab:model-ab}. Rows~(0)-(2) show attention disentanglers can significantly improve the performance. Rows~(2)-(3) show that cross-attention learns disentangled concepts better than self-attention for AA and OA. Rows~(3)-(4) show the regularization term can further benefit the visual disentanglement, improving the unseen accuracy and overall AUC.

\paragraph{Inference formulation} We also investigate the effect of our inference formulation $p(c) + \beta \cdot p(a) \cdot p(o)$ in~\cref{tab:eval-ab}. We report the results with extreme values of $\beta$, \ie, $\beta=0.0$ and $\beta=1.0$. Note that $\beta=0.0$ means only using composition probability for prediction. In addition, we also test the performance only using the product of attribute and object probabilities $p(a) \cdot p(o)$. We can observe that the best fixed $\beta$ value is unfixed among datasets. For example, $\beta=1.0$ gives the highest AUC for C-GQA in row (2) while $\beta=0.0$ for Clothing16K in row (0). In contrast, our validated $\beta$ consistently gives the best inference results for both datasets. Another observation on C-GQA is that $p(a) \cdot p(o)$ alone is not a good prediction, but adding it to $p(c)$ can increase the unseen accuracy. This indicates that the disentangled attribute prediction $p(a)$ and object prediction $p(o)$ indeed enhance the unseen generalization for CZSL problem.

\paragraph{Effect of regularization term} We propose an EMD-adapted regularization term at the attention level to force attentions to disentangle the concept of interest. We also investigate the effect of applying the regularization term at the feature level. Specifically, 
we compare our EMD-based distance to the cosine and euclidean feature distances. The results on open-world UT-Zappos50K are shown in~\cref{tab:reg-ab}. Our EMD-based regularization outperforms other distance forms, because our attention-level EMD distance considers token-wise similarity capturing the specific concept-related attention responses.
\begin{table}[h]
   \centering
   \setlength{\tabcolsep}{20pt}
    \scalebox{0.7}{
    \begin{tabular}{l>{\columncolor{tabcolor}}cccc}
        \toprule
        Reg & AUC & HM & Seen & Unseen \\
        \midrule
        Cosine & 26.8 & 44.7 & \textbf{63.0} & 48.6 \\
        Euclidean & 26.2 & 44.3 & 62.6 & 47.5 \\
        Ours (EMD) & \textbf{27.1} & \textbf{44.8} & 62.4 & \textbf{50.7}\\
        \bottomrule
    \end{tabular}}
    \caption{Comparison of different regularization terms on open-world UT-Zappos50K.} 
    \label{tab:reg-ab} 
    \vspace{-5pt}
\end{table}

\subsection{Qualitative analysis}
Visual disentanglement in feature space is hard to visualize~\cite{Saini_2022_CVPR}. Inspired by previous work attempts~\cite{Saini_2022_CVPR,zhang2022learning,li2020symmetry,nagarajan2018attributes}, we conduct qualitative analysis of image and text retrieval to show how our \framework model correlates the visual image and the concept composition. In addition, to further validate \framework is efficient to disentangle visual concepts, we conduct unseen-to-seen image retrieval based on their visual concept features extracted by attribute and object attentions.

\begin{figure*}[t]
     \centering
     \begin{subfigure}[b]{0.32\textwidth}
         \centering
         \includegraphics[width=0.92\linewidth]{images/txt2img.pdf}
    \caption{Top-5 text-to-image retrieval.}
    \label{fig:wrd2img}
     \end{subfigure}
     \hfill
     \begin{subfigure}[b]{0.35\textwidth}
    \centering
    \includegraphics[width=\linewidth]{images/img2wrd.pdf}
    \caption{Top-5 image-to-text retrieval.}
    \label{fig:img2wrd}
     \end{subfigure}
     \hfill
     \begin{subfigure}[b]{0.32\textwidth}
    \centering
    \includegraphics[width=0.9\linewidth]{images/retrieve_visual.pdf}
    \caption{Top-5 visual concept retrieval.}
    \label{fig:concept-retrieve}
     \end{subfigure}
    \caption{\hsznew{Qualitative analysis. (a) In the last row of ``suede sandals", the wrong image (red box) is ``fake leather sandals". (b) Each image has the ground-truth label (black text) and 5 retrieval results (colored text), in which the green text is the correct prediction. (c) We retrieve images sharing the same visual concepts by their visual concept features for unseen images of ``yellow skirt" and ``pink pants".}}
    \label{fig:qualitative}
    \vspace{-4pt}
\end{figure*}

\paragraph{Image and text retrieval}
We first consider text-to-image retrieval. Given a text composition, \eg, ``leather heels", we embed it and retrieve the top-5 closest visual features based on the feature distance. We display four text compositions of the different objects sharing the same attributes and vice versa in~\cref{fig:wrd2img}. We can observe that the retrieved images are correct in most of the cases. One exception is when retrieving ``suede sandals", the third closest image is ``fake leather sandals". Although ``suede sandals" and ``fake leather sandals" are not the same composition, they are quite visually similar. We then consider image-to-text retrieval, shown in~\cref{fig:img2wrd}. Given an image, \eg, the image of a ``brown zebra", we extract its visual feature and retrieve the top-5 closest text composition embeddings. It is difficult to retrieve the ground-truth label in the top-1 closest text composition, but all top-5 results are all semantically related to the image. We take the image of ``blond person" (row 3, col 2) as an example. Although the text composition ``blond person" is not retrieved in the top-5 matches, the retrieved results ``white shirts", ``white outfit", ``white shorts", ``white pants", and ``young girl" are all reasonable and actually present in the image. Image and text retrieval experiments validate that our \framework efficiently projects visual features and word embeddings into a uniform space.
\vspace{-2pt}

\paragraph{Visual concept retrieval}
Because the attribute and the object are visually coupled in an image to a high degree of entanglement, it is challenging to visualize the disentanglement in feature space~\cite{Saini_2022_CVPR}. Saini \etal~\cite{Saini_2022_CVPR} retrieve single attribute or object text from test images. However, this process is the same as multi-label classification and insufficient to validate that disentangled visual concepts are learned from images. \hsz{Based on the disentanglement ability of ADE, we construct a visual concept retrieval experiment to investigate the distances between visual concept features, \ie, the embedded attribute feature $\pi_a(v_a)$ and the embedded object feature $\pi_o(v_o)$, extracted from different images. Prior to our work, no existing models can do so, because none of them extracts concept-exclusive features like ADE.} The results are shown in~\cref{fig:concept-retrieve}. We first extract attribute features and object features from all seen images. Given an unseen image, we retrieve the top-5 closest images by measuring the feature distance between the attribute feature of the given image and that of all seen images, and the same goes for the object feature. For the image of ``yellow skirt", all retrieval results for the visual concept ``yellow" are all  ``yellow \texttt{[OBJ]}", and all retrieval results for ``skirt" are ``\texttt{[ATTR]} skirt". For the ``pink pants" image, our model also perfectly retrieves the visual concepts, \ie, the attribute ``pink" and the object ``pants". Our experimental results demonstrate that our \framework model is effective to disentangle visual concepts from seen compositions and combine learned concept knowledge into unseen compositions.

\begin{table*}

\centering
\resizebox{0.98\linewidth}{!}{
\begin{tabular}{@{}l@{\:}l| ccccc | ccccc| ccccc}
\toprule
\multirow{3}{*}{Step} & \multirow{3}{*}{Method} & \multicolumn{10}{c|}{\textit{OmniBlender}} & \multicolumn{5}{c}{\textit{Ricoh360}}\\
% \cmidrule{3-12}
& & \multicolumn{5}{c|}{Indoor} & \multicolumn{5}{c|}{Outdoor} & \\
 & & PSNR & $\text{PSNR}^{\text{WS}}$ & LPIPS& SSIM & $\text{SSIM}^{\text{WS}}$ & PSNR &  $\text{PSNR}^{\text{WS}}$& LPIPS & SSIM & $\text{SSIM}^{\text{WS}}$ & PSNR &  $\text{PSNR}^{\text{WS}}$& LPIPS & SSIM & $\text{SSIM}^{\text{WS}}$\\
 
 \midrule
 
 \multirow{5}{*}{5k} & NeRF~\cite{mildenhall2021nerf} &\cellcolor{tab_orange}26.25 &\cellcolor{tab_orange}27.27 &\cellcolor{tab_orange}0.500 &\cellcolor{tab_orange}0.726 &\cellcolor{tab_orange}0.710 &\cellcolor{tab_yellow}22.36 &\cellcolor{tab_yellow}23.62 &\cellcolor{tab_yellow}0.524 &\cellcolor{tab_yellow}0.651 &\cellcolor{tab_yellow}0.611 & 22.09 & 23.82 & 0.576 & 0.649 & 0.623\\\
 & mip-NeRF 360~\cite{Barron_2022_CVPR} & 23.51 & 24.41 & 0.628 & 0.649 & 0.613 & 21.76 & 23.03 & 0.545 & 0.614 & 0.568 & 22.30 & 24.12 &\cellcolor{tab_yellow}0.555 & 0.632 & 0.604\\
 & TensoRF~\cite{chen2022tensorf} &\cellcolor{tab_yellow}25.91 &\cellcolor{tab_yellow}26.93 &\cellcolor{tab_yellow}0.553 &\cellcolor{tab_yellow}0.722 &\cellcolor{tab_yellow}0.708 &\cellcolor{tab_orange}23.21 &\cellcolor{tab_orange}24.74 &\cellcolor{tab_orange}0.500 &\cellcolor{tab_orange}0.672 &\cellcolor{tab_orange}0.645 & \cellcolor{tab_orange}23.20 &\cellcolor{tab_orange}25.16 &\cellcolor{tab_orange}0.542 &\cellcolor{tab_orange}0.676 &\cellcolor{tab_orange}0.658\\
 & DVGO~\cite{Sun_2022_CVPR} & 24.26 & 25.29 & 0.633 & 0.689 & 0.666 & 21.70 & 23.15 & 0.570 & 0.642 & 0.605 &\cellcolor{tab_yellow}22.45 &\cellcolor{tab_yellow}24.59 & 0.573 &\cellcolor{tab_yellow}0.664 &\cellcolor{tab_yellow}0.646 \\
 & EgoNeRF & \cellcolor{tab_red}28.87 & \cellcolor{tab_red}30.06 & \cellcolor{tab_red}0.310 & \cellcolor{tab_red}0.803 &\cellcolor{tab_red}0.803 &\cellcolor{tab_red}27.90 & \cellcolor{tab_red}29.31 & \cellcolor{tab_red}0.167 & \cellcolor{tab_red}0.844 & \cellcolor{tab_red}0.832 &\cellcolor{tab_red}24.52 &\cellcolor{tab_red}26.74 &\cellcolor{tab_red}0.331 &\cellcolor{tab_red}0.737 &\cellcolor{tab_red}0.729\\

\midrule
 
 \multirow{5}{*}{10k} & NeRF~\cite{mildenhall2021nerf} & \cellcolor{tab_orange}27.66 & \cellcolor{tab_orange}28.80 & \cellcolor{tab_yellow}0.425 & \cellcolor{tab_yellow}0.756 & \cellcolor{tab_yellow}0.749 & 23.63 & 24.90 & 0.458 & 0.686 & 0.650 & 22.78 & 24.49 & 0.538 & 0.663 & 0.638 \\
 & mip-NeRF 360~\cite{Barron_2022_CVPR} & \cellcolor{tab_yellow}27.41 & \cellcolor{tab_yellow}28.47 & \cellcolor{tab_orange}0.412 & \cellcolor{tab_orange}0.763 & \cellcolor{tab_orange}0.755 & \cellcolor{tab_orange}25.57 & \cellcolor{tab_orange}26.80 & \cellcolor{tab_orange}0.306 & \cellcolor{tab_orange}0.769 & \cellcolor{tab_orange}0.741 & \cellcolor{tab_orange}24.28 & \cellcolor{tab_orange}26.28 & \cellcolor{tab_orange}0.384 & \cellcolor{tab_orange}0.725 & \cellcolor{tab_orange}0.710  \\
 & TensoRF~\cite{chen2022tensorf} & 26.96 & 26.98 & 0.469 & 0.751 & 0.743 & \cellcolor{tab_yellow}24.09 & \cellcolor{tab_yellow}25.71 & \cellcolor{tab_yellow}0.436 & \cellcolor{tab_yellow}0.696 & \cellcolor{tab_yellow}0.676 & \cellcolor{tab_yellow}23.82 & \cellcolor{tab_yellow}25.75 & \cellcolor{tab_yellow}0.481 & \cellcolor{tab_yellow}0.694 & \cellcolor{tab_yellow}0.678  \\
 & DVGO~\cite{Sun_2022_CVPR} & 25.44 & 26.53 & 0.556 & 0.715 & 0.699 & 22.54 & 24.06 & 0.518 & 0.659 & 0.628 & 23.08 & 25.28 & 0.529 & 0.678 & 0.664\\
 & EgoNeRF & \cellcolor{tab_red}30.23 & \cellcolor{tab_red}31.47 & \cellcolor{tab_red}0.248 & \cellcolor{tab_red}0.840 & \cellcolor{tab_red}0.841 & \cellcolor{tab_red}28.81 & \cellcolor{tab_red}30.21 & \cellcolor{tab_red}0.136 & \cellcolor{tab_red}0.868 & \cellcolor{tab_red}0.859 & \cellcolor{tab_red}24.71 & \cellcolor{tab_red}26.98 & \cellcolor{tab_red}0.314 & \cellcolor{tab_red}0.746 & \cellcolor{tab_red}0.740 \\
 
 \midrule
 \multirow{5}{*}{100k} & NeRF~\cite{mildenhall2021nerf} & \cellcolor{tab_orange}31.67 & \cellcolor{tab_orange}33.08 & \cellcolor{tab_yellow}0.240 & \cellcolor{tab_yellow}0.852 & \cellcolor{tab_yellow}0.853 & \cellcolor{tab_yellow}27.12 & \cellcolor{tab_yellow}28.54 & \cellcolor{tab_yellow}0.269 & \cellcolor{tab_yellow}0.789 & \cellcolor{tab_yellow}0.772 & 24.91 & 26.65 & 0.384 & 0.721 & 0.702  \\
 & mip-NeRF 360~\cite{Barron_2022_CVPR} & \cellcolor{tab_yellow}31.12 & \cellcolor{tab_yellow}32.41 & \cellcolor{tab_orange}0.225 & \cellcolor{tab_orange}0.859 & \cellcolor{tab_orange}0.859 & \cellcolor{tab_orange}29.34 & \cellcolor{tab_orange}30.63 & \cellcolor{tab_orange}0.135 & \cellcolor{tab_orange}0.879 & \cellcolor{tab_orange}0.867 & \cellcolor{tab_red}25.57 & \cellcolor{tab_red}27.62 & \cellcolor{tab_red}0.268 & \cellcolor{tab_red}0.778 & \cellcolor{tab_red}0.770 \\
 & TensoRF~\cite{chen2022tensorf} & 29.25 & 30.57 & 0.376 & 0.791 & 0.793 & 25.68 & 27.47 & 0.344 & 0.734 & 0.726 & \cellcolor{tab_yellow}25.16 & 27.13 & \cellcolor{tab_yellow}0.376 & \cellcolor{tab_yellow}0.732 & 0.724 \\
 & DVGO~\cite{Sun_2022_CVPR} & 28.84 & 30.23 & 0.348 & 0.798 & 0.803 & 24.87 & 26.73 & 0.363 & 0.720 & 0.711 & 24.90 & \cellcolor{tab_yellow}27.28 & \cellcolor{tab_yellow}0.376 & \cellcolor{tab_yellow}0.732 & \cellcolor{tab_yellow}0.729 \\
 & EgoNeRF & \cellcolor{tab_red}33.11 & \cellcolor{tab_red}34.53 & \cellcolor{tab_red}0.142 & \cellcolor{tab_red}0.902 & \cellcolor{tab_red}0.904 & \cellcolor{tab_red}30.56 & \cellcolor{tab_red}32.04 & \cellcolor{tab_red}0.087 & \cellcolor{tab_red}0.904 & \cellcolor{tab_red}0.901 & \cellcolor{tab_orange}25.25 & \cellcolor{tab_orange}27.50 & \cellcolor{tab_orange}0.286 & \cellcolor{tab_orange}0.763 & \cellcolor{tab_orange}0.758 \\
\bottomrule
\end{tabular}
}
\caption{Quantitative results in outward-looking \textit{OmniBlender} and \textit{Ricoh360} dataset. Top results are colored as \colorbox{tab_red}{top1}, \colorbox{tab_orange}{top2}, and \colorbox{tab_yellow}{top3}.}
\label{tab:quantitative}
\end{table*}

\section{Experiments}
\label{sec:experiments}

We demonstrate that EgoNeRF can quickly capture and synthesize novel views of large-scale scenes. 
We describe full implementation details including hyperparameter setup in the supplementary material.

\if 0
\paragraph{Implementation Details}
We implement EgoNeRF with PyTorch~\cite{paszke2019pytorch} without any customized CUDA kernels for optimization.
For all scenes, we use $300^3$ voxels for both $\mathcal{G}_\sigma$ and $\mathcal{G}_a$ with $N_r^y:N_\theta^y:N_\phi^y=1:\frac{2\sqrt{3}}{3}:2\sqrt{3}$.
The dimension of appearance feature $C$ is 27 and we use two-layer MLP of 128 hidden units for decoding network $f_{\text{MLP}}$.
We use the same $r_0=0.005$ for all scenes and the size of convolution kernel $K$ for obtaining a coarse grid is 2.
We describe full implementation details in the supplementary material.
\fi

\paragraph{Datasets}
Since many of the existing datasets for NeRF are dedicated to a setup where a bounded object is captured from outside-in viewpoints, we propose new synthetic and real datasets of large-scale environments captured with omnidirectional videos.
\textit{OmniBlender} is a realistic synthetic dataset of 11 large-scale scenes with detailed textures and sophisticated geometries in both indoor and outdoor environments, 25 images for both train and test, respectively.
It consists of omnidirectional images along a relatively small circular camera path.
The spherical images are rendered using Blender's Cycles path tracing renderer~\cite{blender} with 2000$\times$1000 resolution.
\textit{Ricoh360} is a real-world 360$^\circ$ video dataset captured with a Ricoh Theta V camera with 1920$\times$960 resolution.
We record video on the circular path by rotating an omnidirectional camera fixed with a selfie stick as shown in \cref{fig:teaser} (a).
The dataset consists of 11 diverse indoor and outdoor scenes, 50 images for train and test, respectively.
With the benefit of the simple procedure, the whole data acquisition is finished in less than 5 seconds, which enables capturing the surrounding scene while it remains nearly static.
We obtain camera poses using SfM library OpenMVG~\cite{moulon2016openmvg}.
A detailed description of our dataset can be found in the supplementary material.


\paragraph{Baselines}
EgoNeRF is a variant of NeRF~\cite{mildenhall2021nerf}, which synthesizes novel views of the scene using the neural volume trained with multi-view images.
However, the original NeRF utilizes an MLP to represent the scene volume.
There also exists a recent variant called mip-NeRF 360~\cite{Barron_2022_CVPR}, which combines many techniques to increase the quality of the results, including the adaptation to unbounded scenes by warping space farther than a certain radius.
EgoNeRF employs feature grids and vector-matrix factorization in the balanced spherical grid. DVGO~\cite{Sun_2022_CVPR} exploits feature grids in a Cartesian coordinate with great acceleration, whereas TensoRF~\cite{chen2022tensorf} deploys factorization, also in Cartesian coordinate.
For all the methods, we train with the same ray batch size and the same number of feature grids (for DVGO and TensoRF) with one RTX-3090 GPU for a fair comparison.



\subsection{Quantitative Results}
The quantitative results are reported in mean PSNR, SSIM~\cite{wang2004image}, and LPIPS~\cite{zhang2018unreasonable} across test images in \textit{OmniBlender} and \textit{Ricoh360} dataset in \cref{tab:quantitative}.
Since equirectangular images in our datasets have distortion near poles, we additionally measure weighted-to-spherically uniform PSNR and SSIM~\cite{sun2017weighted} ($\text{PSNR}^{\text{WS}}$ and $\text{SSIM}^{\text{WS}}$), which place smaller weights near the poles when evaluating the metrics.

\Cref{tab:quantitative} shows that EgoNeRF outperforms all compared methods across all error metrics in the \textit{OmniBlender} dataset. 
With the efficient grid structure of EgoNeRF, the difference is more significant in earlier iterations.
Considering the time for each iteration, the efficiency gap is even more significant, which is also visualized in ~\cref{fig:time-PSNR}.
% Our approach shows high performance even at the early 5k steps, which takes 10 minutes of wall-clock time, in contrast to mip-NeRF 360 needs 22.6 hours to train 100k steps to outperform our results at 5k steps.
Our approach shows high performance even at the early 5k steps, which takes 10 minutes of wall-clock time. In contrast, mip-NeRF 360 needs approximately 8 hours to outperform our results at 5k steps.
%% discussion about artifacts with camera poses -- added below
In \textit{Ricoh360}, EgoNeRF surpasses other methods in 5k and 10k training steps, and shows comparable results in 100k steps.
However, our approach sometimes produces spotty artifacts in real-world datasets because the camera pose estimates can be erroneous.
On the other hand, MLP-based methods show blurry rendering, which sporadically achieves better scores in error metrics.
%However, our approach sometimes shows slight noisy artifacts which lower error metrics for some scenes in the real-world dataset when the camera poses are wrong, while compared methods show blurry artifacts for those scenes.
Such a phenomenon is prominent when the error in the camera pose makes the rays hit neighboring cells in the feature grid, which is further discussed in the supplementary material.
% Further discussion regarding the effects of grid resolution and camera pose errors on rendering results can be found in the supplementary material.

More importantly, the feature grid using the Cartesian coordinate system (TensoRF and DVGO) results in inferior performance, especially in outdoor scenes.
This supports our main claim that the Cartesian grid is inadequate to represent large-scale scenes captured from egocentric viewpoints.
In contrast, MLP-based methods (NeRF and mip-NeRF 360) achieve moderate results.

\if 0

\begin{table}
    \centering
    \resizebox{\linewidth}{!}{
    \begin{tabular}{lccc|ccc}
    \toprule
    \multicolumn{1}{l|}{Method} & PSNR & LPIPS & SSIM & PSNR & LPIPS & SSIM \\
    \midrule
    \multicolumn{1}{l|}{NeRF~\cite{mildenhall2021nerf}} & 31.01 & 0.081 & 0.947 & 24.85 & 0.426 & 0.659 \\
    \multicolumn{1}{l|}{mip-NeRF~\cite{Barron_2021_ICCV}} & 32.63 & 0.047 & 0.958 & \cellcolor{tab_yellow}25.12 & \cellcolor{tab_yellow}0.414 & \cellcolor{tab_yellow}0.672 \\
    \multicolumn{1}{l|}{mip-NeRF 360~\cite{Barron_2022_CVPR}} & \cellcolor{tab_red}33.25 & 0.039 & \cellcolor{tab_orange}0.962 & \cellcolor{tab_red}29.23 & \cellcolor{tab_red}0.207 & \cellcolor{tab_red}0.844 \\
    \multicolumn{1}{l|}{TensoRF~\cite{chen2022tensorf}} & 
    \cellcolor{tab_orange}33.14 & \cellcolor{tab_red}0.027 & 
    \cellcolor{tab_red}0.963 & 22.75 & 0.619 & 0.558 \\
    \multicolumn{1}{l|}{DVGO~\cite{Sun_2022_CVPR}} & \cellcolor{tab_yellow}32.80 & \cellcolor{tab_red}0.027 & \cellcolor{tab_yellow}0.961 & 20.67 & 0.490 & 0.575 \\
    \multicolumn{1}{l|}{EgoNeRF} & 31.51 & \cellcolor{tab_yellow}0.037 & 0.952 & \cellcolor{tab_orange}25.83 & \cellcolor{tab_orange}0.320 & \cellcolor{tab_orange}0.701 \\
    \bottomrule
     & \multicolumn{3}{c}{(a) Synthetic-NeRF} & \multicolumn{3}{c}{(b) mip-NeRF 360}
    \end{tabular}
    }
    \caption{Quantitative results in inward-facing datasets (a) Synthetic-NeRF~\cite{mildenhall2021nerf} and (b) mip-NeRF 360~\cite{Barron_2022_CVPR}.}
    % The results of all compared methods on Synthetic-NeRF dataset and the results of NeRF and Mip-NeRF 360 on Mip-NeRF 360 dataset are excerpted from their original paper.}
    \label{tab:inward_facing}
\end{table}
Although our approach is not designed to reconstruct small bounded objects from inward-facing images, we also report results from widely-used datasets for novel view synthesis in \cref{tab:inward_facing}.
EgoNeRF shows comparable results in the Synthetic-NeRF dataset~\cite{mildenhall2021nerf}, which contains 8 synthetic objects.
In mip-NeRF 360 dataset~\cite{Barron_2022_CVPR}, which contains inward-facing objects but has unbounded background scenes, EgoNeRF outperforms other baselines except mip-NeRF 360.
\fi


\begin{figure*}
    \centering
    \begin{tabular}{@{}c@{\,}c@{\,}c@{\,}|@{\,}c@{\,}c@{\,}}
    & \small \textit{BarberShop} & \small \textit{BistroBike} & \small \textit{Bricks} & \small \textit{Poster} \\
    
    \rotatebox{90}{\qquad\quad\:\;\: \small G.T.} & \includegraphics[width=0.23\linewidth]{figures/OmniBlender/barbershop/gt_box_4.jpg} & \includegraphics[width=0.23\linewidth]{figures/OmniBlender/bistro_bike/gt_box_4.jpg}& \includegraphics[width=0.23\linewidth]{figures/Ricoh/bricks/gt_box_4.jpg}&\includegraphics[width=0.23\linewidth]{figures/Ricoh/poster/gt_box_4.jpg}\\
    
    \rotatebox{90}{\qquad\;\, \small NeRF~\cite{mildenhall2021nerf}} & \includegraphics[width=0.23\linewidth]{figures/OmniBlender/barbershop/NeRF_box_4.jpg} & \includegraphics[width=0.23\linewidth]{figures/OmniBlender/bistro_bike/NeRF_box_4.jpg}& \includegraphics[width=0.23\linewidth]{figures/Ricoh/bricks/NeRF_box_4.jpg}&\includegraphics[width=0.23\linewidth]{figures/Ricoh/poster/NeRF_box_4.jpg}\\
    
    \rotatebox{90}{\quad\; \small mip-NeRF 360~\cite{Barron_2022_CVPR}} & \includegraphics[width=0.23\linewidth]{figures/OmniBlender/barbershop/MipNeRF360_box_4.jpg} & \includegraphics[width=0.23\linewidth]{figures/OmniBlender/bistro_bike/MipNeRF360_box_4.jpg}& \includegraphics[width=0.23\linewidth]{figures/Ricoh/bricks/MipNeRF360_box_4.jpg}&\includegraphics[width=0.23\linewidth]{figures/Ricoh/poster/MipNeRF360_box_4.jpg}\\
    
    \rotatebox{90}{\qquad\; \small TensoRF~\cite{chen2022tensorf}} & \includegraphics[width=0.23\linewidth]{figures/OmniBlender/barbershop/TensoRF_box_4.jpg} & \includegraphics[width=0.23\linewidth]{figures/OmniBlender/bistro_bike/TensoRF_box_4.jpg}& \includegraphics[width=0.23\linewidth]{figures/Ricoh/bricks/TensoRF_box_4.jpg}&\includegraphics[width=0.23\linewidth]{figures/Ricoh/poster/TensoRF_box_4.jpg}\\
    
    \rotatebox{90}{\qquad\:\: \small DVGO~\cite{Sun_2022_CVPR}} & \includegraphics[width=0.23\linewidth]{figures/OmniBlender/barbershop/DVGO_box_4.jpg} & \includegraphics[width=0.23\linewidth]{figures/OmniBlender/bistro_bike/DVGO_box_4.jpg}& \includegraphics[width=0.23\linewidth]{figures/Ricoh/bricks/DVGO_box_4.jpg}&\includegraphics[width=0.23\linewidth]{figures/Ricoh/poster/DVGO_box_4.jpg}\\
    
    \rotatebox{90}{\qquad\quad \small EgoNeRF} & \includegraphics[width=0.23\linewidth]{figures/OmniBlender/barbershop/EgoNeRF_box.jpg} & \includegraphics[width=0.23\linewidth]{figures/OmniBlender/bistro_bike/EgoNeRF_box.jpg}& \includegraphics[width=0.23\linewidth]{figures/Ricoh/bricks/EgoNeRF_box.jpg}&\includegraphics[width=0.23\linewidth]{figures/Ricoh/poster/EgoNeRF_box.jpg}\\
    
    & \multicolumn{2}{c}{\small (a) \textit{OmniBlender}} & \multicolumn{2}{c}{\small (b) \textit{Ricoh360}} 
    
    \end{tabular}
    \vspace{-0.1in}
        
    \caption{Comparative results of novel view synthesis on the outward-looking (a) synthetic \textit{OmniBlender} dataset and (b) real-world \textit{Ricoh360} dataset. Best viewed on screen.}
    \label{fig:qualitative_comparison}
\end{figure*}



\if 0
\subsection{Dataset \& Implementation Details}
\label{subsec:dataset_implementation_details}
% 
% version 2
\begin{table*}
\centering
\resizebox{0.7\linewidth}{!}{
\begin{tabular}{@{}l@{\:}l| ccccc | ccccc}
\toprule
\multirow{2}{*}{Step} & \multirow{2}{*}{Method} & \multicolumn{5}{c|}{Indoor} & \multicolumn{5}{c}{Outdoor} \\
 & & PSNR & $\text{PSNR}^{\text{WS}}$ & LPIPS& SSIM & $\text{SSIM}^{\text{WS}}$ & PSNR &  $\text{PSNR}^{\text{WS}}$& LPIPS & SSIM & $\text{SSIM}^{\text{WS}}$ \\
 
 \midrule
 
 \multirow{5}{*}{5k} & NeRF &\cellcolor{tab_orange}26.25 &\cellcolor{tab_orange}27.27 &\cellcolor{tab_orange}0.500 &\cellcolor{tab_orange}0.726 &\cellcolor{tab_orange}0.710 &\cellcolor{tab_yellow}22.36 &\cellcolor{tab_yellow}23.62 &\cellcolor{tab_yellow}0.524 &\cellcolor{tab_yellow}0.651 &\cellcolor{tab_yellow}0.611\\
 & mip-NeRF 360 & 23.51 & 24.41 & 0.628 & 0.649 & 0.613 & 21.76 & 23.03 & 0.545 & 0.614 & 0.568\\
 & TensoRF &\cellcolor{tab_yellow}25.91 &\cellcolor{tab_yellow}26.93 &\cellcolor{tab_yellow}0.553 &\cellcolor{tab_yellow}0.722 &\cellcolor{tab_yellow}0.708 &\cellcolor{tab_orange}23.21 &\cellcolor{tab_orange}24.74 &\cellcolor{tab_orange}0.500 &\cellcolor{tab_orange}0.672 &\cellcolor{tab_orange}0.645 \\
 & DVGO & 24.92 & 26.02 & 0.560 & 0.718 & 0.698 & 21.70 & 23.15 & 0.570 & 0.642 & 0.605\\
 & EgoNeRF & \cellcolor{tab_red}28.87 & 3\cellcolor{tab_red}0.06 & \cellcolor{tab_red}0.310 & \cellcolor{tab_red}0.803 &\cellcolor{tab_red} 0.803 &\cellcolor{tab_red} 27.90 & \cellcolor{tab_red}29.31 & \cellcolor{tab_red}0.167 & \cellcolor{tab_red}0.844 & \cellcolor{tab_red}0.832 \\

\midrule
 
 \multirow{5}{*}{10k} & NeRF & \cellcolor{tab_orange}27.66 & \cellcolor{tab_orange}28.80 & \cellcolor{tab_yellow}0.425 & \cellcolor{tab_yellow}0.756 & \cellcolor{tab_yellow}0.749 & 23.63 & 24.90 & 0.458 & 0.686 & 0.650\\
 & mip-NeRF 360 & \cellcolor{tab_yellow}27.41 & \cellcolor{tab_yellow}28.47 & \cellcolor{tab_orange}0.412 & \cellcolor{tab_orange}0.763 & \cellcolor{tab_orange}0.755 & \cellcolor{tab_orange}25.57 & \cellcolor{tab_orange}26.80 & \cellcolor{tab_orange}0.306 & \cellcolor{tab_orange}0.769 & \cellcolor{tab_orange}0.741\\
 & TensoRF & 27.03 & 28.19 & 0.470 & 0.748 & 0.743 & \cellcolor{tab_yellow}24.09 & \cellcolor{tab_yellow}25.71 & \cellcolor{tab_yellow}0.436 & \cellcolor{tab_yellow}0.696 & \cellcolor{tab_yellow}0.676 \\
 & DVGO & 25.85 & 26.98 & 0.487 & 0.739 & 0.725 & 22.54 & 24.06 & 0.518 & 0.659 & 0.628\\
 & EgoNeRF & \cellcolor{tab_red}30.23 & \cellcolor{tab_red}31.47 & \cellcolor{tab_red}0.248 & \cellcolor{tab_red}0.840 & \cellcolor{tab_red}0.841 & \cellcolor{tab_red}28.81 & \cellcolor{tab_red}30.21 & \cellcolor{tab_red}0.136 & \cellcolor{tab_red}0.868 & \cellcolor{tab_red}0.859\\
 
 % \multirow{5}{*}{50k} & NeRF & \cellcolor{tab_orange}30.59 & \cellcolor{tab_orange}31.92 & \cellcolor{tab_yellow}0.276 & \cellcolor{tab_yellow}0.831 & \cellcolor{tab_yellow}0.831 & \cellcolor{tab_yellow}26.21 & \cellcolor{tab_yellow}27.58 & \cellcolor{tab_yellow}0.315 & \cellcolor{tab_yellow}0.763 & \cellcolor{tab_yellow}0.741\\
 % & Mip-NeRF 360 & \cellcolor{tab_orange}30.59 & \cellcolor{tab_yellow}31.86 & \cellcolor{tab_orange}0.245 & \cellcolor{tab_orange}0.848 & \cellcolor{tab_orange}0.848 & \cellcolor{tab_orange}28.84 & \cellcolor{tab_orange}30.11 & \cellcolor{tab_orange}0.151 & \cellcolor{tab_orange}0.868 & \cellcolor{tab_orange}0.854\\
 % & TensoRF & 29.21 & 30.60 & 0.360 & 0.798 & 0.802 & 25.31 & 27.06 & 0.364 & 0.726 & 0.715\\
 % & DVGO & 28.34 & 29.46 & 0.314 & 0.829 & 0.829 & 24.70 & 26.52 & 0.378 & 0.715 & 0.705\\
 % & EgoNeRF (Ours) & \cellcolor{tab_red}32.55 & \cellcolor{tab_red}33.94 & \cellcolor{tab_red}0.160 & \cellcolor{tab_red}0.892 & \cellcolor{tab_red}0.894 & \cellcolor{tab_red}30.18 & \cellcolor{tab_red}31.65 & \cellcolor{tab_red}0.097 & \cellcolor{tab_red}0.896 & \cellcolor{tab_red}0.892 \\
 
 \midrule
 \multirow{10}{*}{100k} & NeRF & \cellcolor{tab_orange}31.67 & \cellcolor{tab_orange}33.08 & \cellcolor{tab_yellow}0.240 & \cellcolor{tab_yellow}0.852 & \cellcolor{tab_yellow}0.853 & \cellcolor{tab_yellow}27.12 & \cellcolor{tab_yellow}28.54 & \cellcolor{tab_yellow}0.269 & \cellcolor{tab_yellow}0.789 & \cellcolor{tab_yellow}0.772\\
 & mip-NeRF 360 & \cellcolor{tab_yellow}31.12 & \cellcolor{tab_yellow}32.41 & \cellcolor{tab_orange}0.225 & \cellcolor{tab_orange}0.859 & \cellcolor{tab_orange}0.859 & \cellcolor{tab_orange}29.34 & \cellcolor{tab_orange}30.63 & \cellcolor{tab_orange}0.135 & \cellcolor{tab_orange}0.875 & \cellcolor{tab_orange}0.867\\
 & TensoRF & 29.85 & 31.34 & 0.339 & 0.807 & 0.814 & 25.68 & 27.47 & 0.344 & 0.734 & 0.726\\
 & DVGO & 28.68 & 29.86 & 0.298 & 0.834 & 0.836 & 24.87 & 26.73 & 0.363 & 0.720 & 0.711\\
 & EgoNeRF & \cellcolor{tab_red}33.11 & \cellcolor{tab_red}34.53 & \cellcolor{tab_red}0.142 & \cellcolor{tab_red}0.902 & \cellcolor{tab_red}0.904 & \cellcolor{tab_red}30.56 & \cellcolor{tab_red}32.04 & \cellcolor{tab_red}0.087 & \cellcolor{tab_red}0.904 & \cellcolor{tab_red}0.901\\
 \cmidrule{2-12}
 & w/o exp $R$ grid & 31.32 & 32.75 & 0.188 & 0.871 & 0.873 & 26.66 & 28.66 & 0.187 & 0.792 & 0.802\\
 & w/o YinYang grid & 30.53 & 31.88 & 0.191 & 0.860 & 0.864 & 26.74 & 28.71 & 0.160 & 0.806 & 0.814\\
 & Spherical Grid & 30.78 & 32.23 & 0.209 & 0.858 & 0.862 & 26.25 & 28.25 & 0.213 & 0.773 & 0.780 \\
 & w/o Resampling & 32.40 & 33.90 & 0.167 & 0.886 & 0.891 & 30.12 & 31.79 & 0.105 & 0.891 & 0.891\\
 & w/o Envmap &  &  & - &  &  & 30.04 & 31.59 & 0.107 & 0.891 & 0.887\\
\bottomrule
\end{tabular}
}
\caption{Quantitative results in \textit{OmniBlender} dataset.}
\label{tab:quantitative_omniblender}
\end{table*}
% 
\begin{table}
\centering
\resizebox{0.9\linewidth}{!}{
% \begin{tabular}{@{}ll| ccccc @{}}
\begin{tabular}{@{}l@{\:}l|ccccc}
\toprule
Step & Method & PSNR & $\text{PSNR}^{\text{WS}}$ & LPIPS & SSIM & $\text{SSIM}^{\text{WS}}$\\

 \midrule
 
 \multirow{5}{*}{5k} & NeRF & 22.09 & 23.82 & 0.576 & 0.649 & 0.623\\
 & mip-NeRF 360 & 22.30 & 24.12 &\cellcolor{tab_yellow}0.555 & 0.632 & 0.604\\
 & TensoRF &\cellcolor{tab_orange}23.20 &\cellcolor{tab_orange}25.16 &\cellcolor{tab_orange}0.542 &\cellcolor{tab_orange}0.676 &\cellcolor{tab_orange}0.658 \\
 & DVGO &\cellcolor{tab_yellow}22.45 &\cellcolor{tab_yellow}24.59 & 0.573 &\cellcolor{tab_yellow}0.664 &\cellcolor{tab_yellow}0.646\\
 & EgoNeRF &\cellcolor{tab_red}24.52 &\cellcolor{tab_red}26.74 &\cellcolor{tab_red}0.331 &\cellcolor{tab_red}0.737 &\cellcolor{tab_red}0.729 \\
 
\midrule
 \multirow{5}{*}{10k} & NeRF & 22.78 & 24.49 & 0.538 & 0.663 & 0.638 \\
 & mip-NeRF 360 & \cellcolor{tab_orange}24.28 & \cellcolor{tab_orange}26.28 & \cellcolor{tab_orange}0.384 & \cellcolor{tab_orange}0.725 & \cellcolor{tab_orange}0.710 \\
 & TensoRF & \cellcolor{tab_yellow}23.31 & 25.21 & \cellcolor{tab_yellow}0.507 & \cellcolor{tab_yellow}0.684 & \cellcolor{tab_yellow}0.666 \\
 & DVGO & 23.08 & \cellcolor{tab_yellow}25.28 & 0.529 & 0.678 & 0.664\\
 & EgoNeRF & \cellcolor{tab_red}24.71 & \cellcolor{tab_red}26.97 & \cellcolor{tab_red}0.314 & \cellcolor{tab_red}0.746 & \cellcolor{tab_red}0.740 \\
 
\midrule

%  \multirow{5}{*}{50k} & NeRF & 24.38 & 26.09 & 0.428 & 0.704 & 0.683 \\
%  & Mip-NeRF 360 & \cellcolor{tab_red}25.40 & \cellcolor{tab_red}27.44 & \cellcolor{tab_red}0.281 & \cellcolor{tab_red}0.772 & \cellcolor{tab_red}0.763 \\
%  & TensoRF & 24.24 & 26.14 & 0.433 & 0.710 & 0.697 \\
%  & DVGO & \cellcolor{tab_yellow}24.66 & \cellcolor{tab_yellow}27.00 & \cellcolor{tab_yellow}0.395 & \cellcolor{tab_yellow}0.725 & \cellcolor{tab_yellow}0.720 \\
%  & EgoNeRF (Ours)& \cellcolor{tab_orange}25.13 & \cellcolor{tab_orange}27.39 & \cellcolor{tab_orange}0.292 & \cellcolor{tab_orange}0.760 & \cellcolor{tab_orange}0.755 \\
% \midrule
 
 \multirow{5}{*}{100k} & NeRF & \cellcolor{tab_yellow}24.91 & 26.65 & 0.384 & 0.721 & 0.702 \\
 & mip-NeRF 360 & \cellcolor{tab_red}25.57 & \cellcolor{tab_red}27.62 & \cellcolor{tab_red}0.268 & \cellcolor{tab_red}0.778 & \cellcolor{tab_red}0.770 \\
 & TensoRF & 24.66 & 26.57 & 0.406 & 0.721 & 0.709 \\
 & DVGO & 24.90 & \cellcolor{tab_yellow}27.28 & \cellcolor{tab_yellow}0.376 & \cellcolor{tab_yellow}0.732 & \cellcolor{tab_yellow}0.729 \\
 & EgoNeRF & \cellcolor{tab_orange}25.25 & \cellcolor{tab_orange}27.50 & \cellcolor{tab_orange}0.286 & \cellcolor{tab_orange}0.763 & \cellcolor{tab_orange}0.758 \\
\bottomrule
\end{tabular}
}
\caption{Quantitative results in real-world \textit{Ricoh360} dataset.}
\label{tab:quantitative_ricoh}
\end{table}

\begin{table*}

\centering
\resizebox{0.98\linewidth}{!}{
\begin{tabular}{@{}l@{\:}l| ccccc | ccccc| ccccc}
\toprule
\multirow{3}{*}{Step} & \multirow{3}{*}{Method} & \multicolumn{10}{c|}{\textit{OmniBlender}} & \multicolumn{5}{c}{\textit{Ricoh360}}\\
% \cmidrule{3-12}
& & \multicolumn{5}{c|}{Indoor} & \multicolumn{5}{c|}{Outdoor} & \\
 & & PSNR & $\text{PSNR}^{\text{WS}}$ & LPIPS& SSIM & $\text{SSIM}^{\text{WS}}$ & PSNR &  $\text{PSNR}^{\text{WS}}$& LPIPS & SSIM & $\text{SSIM}^{\text{WS}}$ & PSNR &  $\text{PSNR}^{\text{WS}}$& LPIPS & SSIM & $\text{SSIM}^{\text{WS}}$\\
 
 \midrule
 
 \multirow{5}{*}{5k} & NeRF~\cite{mildenhall2021nerf} &\cellcolor{tab_orange}26.25 &\cellcolor{tab_orange}27.27 &\cellcolor{tab_orange}0.500 &\cellcolor{tab_orange}0.726 &\cellcolor{tab_orange}0.710 &\cellcolor{tab_yellow}22.36 &\cellcolor{tab_yellow}23.62 &\cellcolor{tab_yellow}0.524 &\cellcolor{tab_yellow}0.651 &\cellcolor{tab_yellow}0.611 & 22.09 & 23.82 & 0.576 & 0.649 & 0.623\\\
 & mip-NeRF 360~\cite{Barron_2022_CVPR} & 23.51 & 24.41 & 0.628 & 0.649 & 0.613 & 21.76 & 23.03 & 0.545 & 0.614 & 0.568 & 22.30 & 24.12 &\cellcolor{tab_yellow}0.555 & 0.632 & 0.604\\
 & TensoRF~\cite{chen2022tensorf} &\cellcolor{tab_yellow}25.91 &\cellcolor{tab_yellow}26.93 &\cellcolor{tab_yellow}0.553 &\cellcolor{tab_yellow}0.722 &\cellcolor{tab_yellow}0.708 &\cellcolor{tab_orange}23.21 &\cellcolor{tab_orange}24.74 &\cellcolor{tab_orange}0.500 &\cellcolor{tab_orange}0.672 &\cellcolor{tab_orange}0.645 & \cellcolor{tab_orange}23.20 &\cellcolor{tab_orange}25.16 &\cellcolor{tab_orange}0.542 &\cellcolor{tab_orange}0.676 &\cellcolor{tab_orange}0.658\\
 & DVGO~\cite{Sun_2022_CVPR} & 24.26 & 25.29 & 0.633 & 0.689 & 0.666 & 21.70 & 23.15 & 0.570 & 0.642 & 0.605 &\cellcolor{tab_yellow}22.45 &\cellcolor{tab_yellow}24.59 & 0.573 &\cellcolor{tab_yellow}0.664 &\cellcolor{tab_yellow}0.646 \\
 & EgoNeRF & \cellcolor{tab_red}28.87 & \cellcolor{tab_red}30.06 & \cellcolor{tab_red}0.310 & \cellcolor{tab_red}0.803 &\cellcolor{tab_red}0.803 &\cellcolor{tab_red}27.90 & \cellcolor{tab_red}29.31 & \cellcolor{tab_red}0.167 & \cellcolor{tab_red}0.844 & \cellcolor{tab_red}0.832 &\cellcolor{tab_red}24.52 &\cellcolor{tab_red}26.74 &\cellcolor{tab_red}0.331 &\cellcolor{tab_red}0.737 &\cellcolor{tab_red}0.729\\

\midrule
 
 \multirow{5}{*}{10k} & NeRF~\cite{mildenhall2021nerf} & \cellcolor{tab_orange}27.66 & \cellcolor{tab_orange}28.80 & \cellcolor{tab_yellow}0.425 & \cellcolor{tab_yellow}0.756 & \cellcolor{tab_yellow}0.749 & 23.63 & 24.90 & 0.458 & 0.686 & 0.650 & 22.78 & 24.49 & 0.538 & 0.663 & 0.638 \\
 & mip-NeRF 360~\cite{Barron_2022_CVPR} & \cellcolor{tab_yellow}27.41 & \cellcolor{tab_yellow}28.47 & \cellcolor{tab_orange}0.412 & \cellcolor{tab_orange}0.763 & \cellcolor{tab_orange}0.755 & \cellcolor{tab_orange}25.57 & \cellcolor{tab_orange}26.80 & \cellcolor{tab_orange}0.306 & \cellcolor{tab_orange}0.769 & \cellcolor{tab_orange}0.741 & \cellcolor{tab_orange}24.28 & \cellcolor{tab_orange}26.28 & \cellcolor{tab_orange}0.384 & \cellcolor{tab_orange}0.725 & \cellcolor{tab_orange}0.710  \\
 & TensoRF~\cite{chen2022tensorf} & 26.96 & 26.98 & 0.469 & 0.751 & 0.743 & \cellcolor{tab_yellow}24.09 & \cellcolor{tab_yellow}25.71 & \cellcolor{tab_yellow}0.436 & \cellcolor{tab_yellow}0.696 & \cellcolor{tab_yellow}0.676 & \cellcolor{tab_yellow}23.82 & \cellcolor{tab_yellow}25.75 & \cellcolor{tab_yellow}0.481 & \cellcolor{tab_yellow}0.694 & \cellcolor{tab_yellow}0.678  \\
 & DVGO~\cite{Sun_2022_CVPR} & 25.44 & 26.53 & 0.556 & 0.715 & 0.699 & 22.54 & 24.06 & 0.518 & 0.659 & 0.628 & 23.08 & 25.28 & 0.529 & 0.678 & 0.664\\
 & EgoNeRF & \cellcolor{tab_red}30.23 & \cellcolor{tab_red}31.47 & \cellcolor{tab_red}0.248 & \cellcolor{tab_red}0.840 & \cellcolor{tab_red}0.841 & \cellcolor{tab_red}28.81 & \cellcolor{tab_red}30.21 & \cellcolor{tab_red}0.136 & \cellcolor{tab_red}0.868 & \cellcolor{tab_red}0.859 & \cellcolor{tab_red}24.71 & \cellcolor{tab_red}26.98 & \cellcolor{tab_red}0.314 & \cellcolor{tab_red}0.746 & \cellcolor{tab_red}0.740 \\
 
 \midrule
 \multirow{5}{*}{100k} & NeRF~\cite{mildenhall2021nerf} & \cellcolor{tab_orange}31.67 & \cellcolor{tab_orange}33.08 & \cellcolor{tab_yellow}0.240 & \cellcolor{tab_yellow}0.852 & \cellcolor{tab_yellow}0.853 & \cellcolor{tab_yellow}27.12 & \cellcolor{tab_yellow}28.54 & \cellcolor{tab_yellow}0.269 & \cellcolor{tab_yellow}0.789 & \cellcolor{tab_yellow}0.772 & 24.91 & 26.65 & 0.384 & 0.721 & 0.702  \\
 & mip-NeRF 360~\cite{Barron_2022_CVPR} & \cellcolor{tab_yellow}31.12 & \cellcolor{tab_yellow}32.41 & \cellcolor{tab_orange}0.225 & \cellcolor{tab_orange}0.859 & \cellcolor{tab_orange}0.859 & \cellcolor{tab_orange}29.34 & \cellcolor{tab_orange}30.63 & \cellcolor{tab_orange}0.135 & \cellcolor{tab_orange}0.879 & \cellcolor{tab_orange}0.867 & \cellcolor{tab_red}25.57 & \cellcolor{tab_red}27.62 & \cellcolor{tab_red}0.268 & \cellcolor{tab_red}0.778 & \cellcolor{tab_red}0.770 \\
 & TensoRF~\cite{chen2022tensorf} & 29.25 & 30.57 & 0.376 & 0.791 & 0.793 & 25.68 & 27.47 & 0.344 & 0.734 & 0.726 & \cellcolor{tab_yellow}25.16 & 27.13 & \cellcolor{tab_yellow}0.376 & \cellcolor{tab_yellow}0.732 & 0.724 \\
 & DVGO~\cite{Sun_2022_CVPR} & 28.84 & 30.23 & 0.348 & 0.798 & 0.803 & 24.87 & 26.73 & 0.363 & 0.720 & 0.711 & 24.90 & \cellcolor{tab_yellow}27.28 & \cellcolor{tab_yellow}0.376 & \cellcolor{tab_yellow}0.732 & \cellcolor{tab_yellow}0.729 \\
 & EgoNeRF & \cellcolor{tab_red}33.11 & \cellcolor{tab_red}34.53 & \cellcolor{tab_red}0.142 & \cellcolor{tab_red}0.902 & \cellcolor{tab_red}0.904 & \cellcolor{tab_red}30.56 & \cellcolor{tab_red}32.04 & \cellcolor{tab_red}0.087 & \cellcolor{tab_red}0.904 & \cellcolor{tab_red}0.901 & \cellcolor{tab_orange}25.25 & \cellcolor{tab_orange}27.50 & \cellcolor{tab_orange}0.286 & \cellcolor{tab_orange}0.763 & \cellcolor{tab_orange}0.758 \\
\bottomrule
\end{tabular}
}
\caption{Quantitative results in outward-looking \textit{OmniBlender} and \textit{Ricoh360} dataset. Top results are colored as \colorbox{tab_red}{top1}, \colorbox{tab_orange}{top2}, and \colorbox{tab_yellow}{top3}.}
\label{tab:quantitative}
\end{table*}
\paragraph{Outward-Looking 360$^\circ$ Video Data}
Most of the existing widely used datasets for novel view synthesis capture the interest object around the outside-in viewpoints.
In contrast, we aim to reconstruct the large-scale environmental scene.
To this end, we evaluate our model on novel datasets of outward-looking omnidirectional videos.
\textit{OmniBlender} is a realistic synthetic dataset consisting of omnidirectional images along a relatively small circular camera path compared to the entire scene.
The spherical images are rendered using Blender's Cycles path tracing renderer~\cite{blender} with 2000$\times$1000 resolution.
OmniBlender contains 11 large-scale scenes with detailed textures and sophisticated geometries in both indoor and outdoor environments.
\textit{Ricoh360} is a real-world 360$^\circ$ video dataset captured with conventional Ricoh Theta V camera with 1920$\times$960 resolution.
We record video on the circular path by rotating an omnidirectional camera fixed with a selfie stick.
With the benefit of the simple procedure, the whole data acquisition is finished in less than 5 seconds, which enables capturing the surrounding scene while remaining nearly static.
The dataset is consist of 11 diverse indoor and outdoor scenes.
We obtain camera pose using structure-from-motion library OpenMVG~\cite{moulon2016openmvg}.
A detailed description of our dataset can be found in the supplementary material.

\paragraph{Inward-facing Data}
Although EgoNeRF is not best designed to represent small objects from inward-facing images, we also report results from widely-used outside-in viewing datasets.
We evaluate our approach on Synthetic-NeRF~\cite{mildenhall2021nerf}, which contains 8 synthetic objects and real-world data from mip-NeRF 360~\cite{Barron_2022_CVPR}.

\paragraph{Implementation Details}
We implement EgoNeRF with PyTorch~\cite{paszke2019pytorch} without any customized CUDA kernels for optimization.
For all scenes, we use $300^3$ voxels for both $\mathcal{G}_\sigma$ and $\mathcal{G}_a$ with $N_r^y:N_\theta^y:N_\phi^y=1:\frac{2\sqrt{3}}{3}:2\sqrt{3}$.
The dimension of appearance feature $C$ is 27 and we use two-layer MLP of 128 hidden units for decoding network $f_{\text{MLP}}$.
We use the same $r_0=0.005$ for all scenes and the size of convolution kernel $K$ for obtaining a coarse grid is 2.
We describe full implementation details in the supplementary material.

\begin{figure*}
    \centering
    \begin{tabular}{@{}c@{\,}c@{\,}c@{\,}|@{\,}c@{\,}c@{\,}}
    & \small \textit{BarberShop} & \small \textit{BistroBike} & \small \textit{Bricks} & \small \textit{Poster} \\
    
    \rotatebox{90}{\qquad\quad\:\;\: \small G.T.} & \includegraphics[width=0.23\linewidth]{figures/OmniBlender/barbershop/gt_box_4.jpg} & \includegraphics[width=0.23\linewidth]{figures/OmniBlender/bistro_bike/gt_box_4.jpg}& \includegraphics[width=0.23\linewidth]{figures/Ricoh/bricks/gt_box_4.jpg}&\includegraphics[width=0.23\linewidth]{figures/Ricoh/poster/gt_box_4.jpg}\\
    
    \rotatebox{90}{\qquad\;\, \small NeRF~\cite{mildenhall2021nerf}} & \includegraphics[width=0.23\linewidth]{figures/OmniBlender/barbershop/NeRF_box_4.jpg} & \includegraphics[width=0.23\linewidth]{figures/OmniBlender/bistro_bike/NeRF_box_4.jpg}& \includegraphics[width=0.23\linewidth]{figures/Ricoh/bricks/NeRF_box_4.jpg}&\includegraphics[width=0.23\linewidth]{figures/Ricoh/poster/NeRF_box_4.jpg}\\
    
    \rotatebox{90}{\quad\; \small mip-NeRF 360~\cite{Barron_2022_CVPR}} & \includegraphics[width=0.23\linewidth]{figures/OmniBlender/barbershop/MipNeRF360_box_4.jpg} & \includegraphics[width=0.23\linewidth]{figures/OmniBlender/bistro_bike/MipNeRF360_box_4.jpg}& \includegraphics[width=0.23\linewidth]{figures/Ricoh/bricks/MipNeRF360_box_4.jpg}&\includegraphics[width=0.23\linewidth]{figures/Ricoh/poster/MipNeRF360_box_4.jpg}\\
    
    \rotatebox{90}{\qquad\; \small TensoRF~\cite{chen2022tensorf}} & \includegraphics[width=0.23\linewidth]{figures/OmniBlender/barbershop/TensoRF_box_4.jpg} & \includegraphics[width=0.23\linewidth]{figures/OmniBlender/bistro_bike/TensoRF_box_4.jpg}& \includegraphics[width=0.23\linewidth]{figures/Ricoh/bricks/TensoRF_box_4.jpg}&\includegraphics[width=0.23\linewidth]{figures/Ricoh/poster/TensoRF_box_4.jpg}\\
    
    \rotatebox{90}{\qquad\:\: \small DVGO~\cite{Sun_2022_CVPR}} & \includegraphics[width=0.23\linewidth]{figures/OmniBlender/barbershop/DVGO_box_4.jpg} & \includegraphics[width=0.23\linewidth]{figures/OmniBlender/bistro_bike/DVGO_box_4.jpg}& \includegraphics[width=0.23\linewidth]{figures/Ricoh/bricks/DVGO_box_4.jpg}&\includegraphics[width=0.23\linewidth]{figures/Ricoh/poster/DVGO_box_4.jpg}\\
    
    \rotatebox{90}{\qquad\quad \small EgoNeRF} & \includegraphics[width=0.23\linewidth]{figures/OmniBlender/barbershop/EgoNeRF_box.jpg} & \includegraphics[width=0.23\linewidth]{figures/OmniBlender/bistro_bike/EgoNeRF_box.jpg}& \includegraphics[width=0.23\linewidth]{figures/Ricoh/bricks/EgoNeRF_box.jpg}&\includegraphics[width=0.23\linewidth]{figures/Ricoh/poster/EgoNeRF_box.jpg}\\
    
    & \multicolumn{2}{c}{\small (a) \textit{OmniBlender}} & \multicolumn{2}{c}{\small (b) \textit{Ricoh360}} 
    
    \end{tabular}
    \vspace{-0.1in}
        
    \caption{Comparative results of novel view synthesis on the outward-looking (a) synthetic \textit{OmniBlender} dataset and (b) real-world \textit{Ricoh360} dataset. Best viewed on screen.}
    \label{fig:qualitative_comparison}
\end{figure*}
\subsection{Results}
\label{subsec:results}
\paragraph{Compared methods}
We compare EgoNeRF with state-of-the-art feature grid-based novel-view synthesis algorithms DVGO~\cite{Sun_2022_CVPR} and TensoRF~\cite{chen2022tensorf}.
Both methods accelerate the training speed of NeRF by exploiting feature grids defined in the Cartesian coordinate system with voxel and vector-matrix decomposed representation.
We evaluate our method against MLP-based methods NeRF~\cite{mildenhall2021nerf}.
We also evaluate against mip-NeRF 360~\cite{Barron_2022_CVPR} which is able to model unbounded scenes by warping space farther than a certain radius.
For all the methods, we train with the same ray batch size and the same number of feature grids (for DVGO and TensoRF) for a fair comparison.


\paragraph{Evaluation}
We report mean PSNR, SSIM~\cite{wang2004image}, and LPIPS~\cite{zhang2018unreasonable} across test images in \textit{OmniBlender} and \textit{Ricoh360} dataset in~\cref{tab:quantitative}.
Since the equirectangular images in our datasets have distortion near poles, we additionally measure weighted-to-spherically uniform PSNR and SSIM~\cite{sun2017weighted} ($\text{PSNR}^{\text{WS}}$ and $\text{SSIM}^{\text{WS}}$).
EgoNeRF outperforms all compared methods across all error metrics in the OmniBlender dataset.
Also, EgoNeRF surpasses other methods in 5k, 10k training steps, and shows comparable results in 100k steps in Ricoh360.
Both feature grid-based methods with the Cartesian coordinate system show inferior performance, especially in outdoor scenes.
This supports our main claim that the Cartesian grid is inadequate to represent large-scale scenes captured from egocentric viewpoints.
In contrast, MLP-based methods NeRF and mip-NeRF 360 achieve moderate results.
It is notable that our approach shows high performance even at the early 5k steps, which takes 10 minutes of wall-clock time, in contrast to mip-NeRF 360 needs 22.6 hours to train 100k steps to outperform our results at 5k steps.
Although our approach is not best designed to reconstruct small bounded objects from inward-facing images, EgoNeRF shows comparable results in the Synthetic-NeRF dataset.
In mip-NeRF 360 dataset, which contains inward-facing objects but has unbounded background scenes, EgoNeRF outperforms other baselines except mip-NeRF 360.

\begin{table}
    \centering
    \resizebox{\linewidth}{!}{
    \begin{tabular}{lccc|ccc}
    \toprule
    \multicolumn{1}{l|}{Method} & PSNR & LPIPS & SSIM & PSNR & LPIPS & SSIM \\
    \midrule
    \multicolumn{1}{l|}{NeRF~\cite{mildenhall2021nerf}} & 31.01 & 0.081 & 0.947 & 24.85 & 0.426 & 0.659 \\
    \multicolumn{1}{l|}{mip-NeRF~\cite{Barron_2021_ICCV}} & 32.63 & 0.047 & 0.958 & \cellcolor{tab_yellow}25.12 & \cellcolor{tab_yellow}0.414 & \cellcolor{tab_yellow}0.672 \\
    \multicolumn{1}{l|}{mip-NeRF 360~\cite{Barron_2022_CVPR}} & \cellcolor{tab_red}33.25 & 0.039 & \cellcolor{tab_orange}0.962 & \cellcolor{tab_red}29.23 & \cellcolor{tab_red}0.207 & \cellcolor{tab_red}0.844 \\
    \multicolumn{1}{l|}{TensoRF~\cite{chen2022tensorf}} & 
    \cellcolor{tab_orange}33.14 & \cellcolor{tab_red}0.027 & 
    \cellcolor{tab_red}0.963 & 22.75 & 0.619 & 0.558 \\
    \multicolumn{1}{l|}{DVGO~\cite{Sun_2022_CVPR}} & \cellcolor{tab_yellow}32.80 & \cellcolor{tab_red}0.027 & \cellcolor{tab_yellow}0.961 & 20.67 & 0.490 & 0.575 \\
    \multicolumn{1}{l|}{EgoNeRF} & 31.51 & \cellcolor{tab_yellow}0.037 & 0.952 & \cellcolor{tab_orange}25.83 & \cellcolor{tab_orange}0.320 & \cellcolor{tab_orange}0.701 \\
    \bottomrule
     & \multicolumn{3}{c}{(a) Synthetic-NeRF} & \multicolumn{3}{c}{(b) mip-NeRF 360}
    \end{tabular}
    }
    \caption{Quantitative results in inward-facing datasets (a) Synthetic-NeRF~\cite{mildenhall2021nerf} and (b) mip-NeRF 360~\cite{Barron_2022_CVPR}.}
    % The results of all compared methods on Synthetic-NeRF dataset and the results of NeRF and Mip-NeRF 360 on Mip-NeRF 360 dataset are excerpted from their original paper.}
    \label{tab:inward_facing}
\end{table}

\fi


\subsection{Qualitative Results}


The qualitative results in \textit{OmniBlender} and \textit{Ricoh360} datasets are demonstrated in~\cref{fig:qualitative_comparison}.
Our method reconstructs fine details in both close-by objects and far-away regions.
% However, when we deploy Cartesian grids with the same number of cells, many cells are wasted without being trained in far objects, while center cells might not have a sufficient resolution as depicted in \cref{fig:grid_comparison}.
However, for Cartesian grid-based methods (TensoRF and DVGO), many cells are wasted without being trained in far objects, while center cells might not have a sufficient resolution as depicted in \cref{fig:grid_comparison}.
% However, prior works with Cartesian feature grids waste many cells without being trained in far objects, while center cells might not have a sufficient resolution as depicted in \cref{fig:grid_comparison}.
It results in visible artifacts in the picture in \textit{BarberShop}, bike in \textit{BistroBike}, bricks in \textit{bricks}, and posters in \textit{poster}.
This phenomenon is predominant in large scenes, while EgoNeRF gives consistently faithful results regardless of the size of the scenes.

MLP-based approaches show better visual results than Cartesian feature grid-based methods with much longer training and rendering time.
However, mip-NeRF 360 often misses fine structures: e.g., stick of broom, thin handle and support fixture in tray, headrest attachment in chair in \textit{Barbershop}, street lamp, side mirror of bike, small colorful lightbulbs and thin wire in \textit{BistroBike}.
Some of them are also indicated with white dotted circles in \cref{fig:qualitative_comparison}. 
% This may be due to mip-NeRF 360 resample ray samples from the additional proposal MLP and they do not apply rendering loss for the proposal MLP to relieve lengthy training time to optimize large MLPs, thus the weight distribution is strongly determined by the initial guess of the proposal MLP.
This may be due to mip-NeRF 360 resample ray samples from the proposal MLP and do not apply rendering loss for the proposal MLP to relieve lengthy training time to optimize large MLPs, thus the weight distribution is strongly determined by the initial guess of the proposal MLP.
In contrast, since our approach shares the same density grid $\mathcal{G}_\sigma$ to query volume density at coarse samples and fine samples, EgoNeRF shows superior rendering results on fine details.
Also, MLP-based approaches show smoothed rendering results across all the scenes (e.g., windows are blurred, cannot see the sky through the gap between bricks, the boundaries between stepping blocks are blurred in \textit{Bricks}, two reflected lights are merged in \textit{poster}. Some of them are also highlighted with white dotted circles in \cref{fig:qualitative_comparison}.), while EgoNeRF shows high-quality images similar to ground-truth images.



\subsection{Ablation study}
\label{subsec:ablation}
\begin{figure*}[t!]
	\centering
	\includegraphics[width=1\textwidth]{Figure/imgs/Ablation.pdf}
	\caption{Results of ablation studies on CAMELYON16 dataset: (a) Effectiveness of the number of prompt blocks (b) Influence of the number of representative patches.}
	\label{fig_ablation}
\end{figure*}
\begin{figure}
       \centering
        \setlength{\tabcolsep}{1pt}
        {\scriptsize
        \begin{tabular}{c c c c c c c }
            { Original } &
            \multicolumn{2}{c}{  } &
            \multicolumn{4}{c}{$\longleftarrow$ Object level variations $\longrightarrow$} \\
            \includegraphics[width=0.185\linewidth]{images/ablation/chair.jpg} &
            \multicolumn{2}{c}{  } &
            \includegraphics[width=0.185\linewidth]{images/ablation/1_only_prompt_mixing/bench.jpg} &
            \includegraphics[width=0.185\linewidth]{images/ablation/1_only_prompt_mixing/stool.jpg} &
            \includegraphics[width=0.185\linewidth]{images/ablation/1_only_prompt_mixing/armchair.jpg} &
            \includegraphics[width=0.185\linewidth]{images/ablation/1_only_prompt_mixing/saddle.jpg} \\
            \multicolumn{3}{c}{  } &
            \multicolumn{4}{c}{ Only Prompt Mixing } \\
            \multicolumn{3}{c}{ } &
            \includegraphics[width=0.185\linewidth]{images/ablation/2_with_self_attn_injection/bench.jpg} &
            \includegraphics[width=0.185\linewidth]{images/ablation/2_with_self_attn_injection/stool.jpg} &
            \includegraphics[width=0.185\linewidth]{images/ablation/2_with_self_attn_injection/armchair.jpg} &
            \includegraphics[width=0.185\linewidth]{images/ablation/2_with_self_attn_injection/saddle.jpg} \\
            \multicolumn{3}{c}{  } &
            \multicolumn{4}{c}{ + Attention-Based Shape Localization } \\
            \multicolumn{3}{c}{ } &
            \includegraphics[width=0.185\linewidth]{images/ablation/3_background_blending/bench.jpg} &
            \includegraphics[width=0.185\linewidth]{images/ablation/3_background_blending/stool.jpg} &
            \includegraphics[width=0.185\linewidth]{images/ablation/3_background_blending/armchair.jpg} &
            \includegraphics[width=0.185\linewidth]{images/ablation/3_background_blending/saddle.jpg} \\
            \multicolumn{3}{c}{  } &
            \multicolumn{4}{c}{ + Controllable Background Preservation } \\
        \end{tabular}
        }
    \vspace{1mm}
    \captionof{figure}{
    Ablating our full object variations pipeline. Original image was crated using the prompt ``A \emph{chair} with a dog on it''. 
    }
    \vspace{-10pt}
    \label{fig:ablation}
\end{figure}

We analyze the effects of important components of EgoNeRF with ablated versions.
\Cref{tab:ablation} shows that removing any of the components in our model degrades the performance across all metrics.
The first three rows are related to the balanced spherical grid.
Using the uniform radial partition deteriorates the performance, especially in outdoor scenes.
% Without Yin-Yang grids, the angular partition exhibits in high valence grid points on two poles and coarse cells near the equator.
Without Yin-Yang grids, the angular partition exhibits high valence grid points on two poles and degrades the error metrics consequently. 
Removing both radial and angular balanced grids, which is identical to uniform spherical grids, causes the biggest drop in performance except PSNR in indoor scenes.
As shown in the first row of~\cref{fig:ablation}, the spherical feature grid has radial direction artifacts (red box) and shows blurrier rendered results for nearby objects compared to our full model.
Also, not employing resampling techniques and using a double number of ray samples reduces performance.
Lastly, removing the environment map in outdoor scenes shows blurry artifacts in infinitely far regions as shown in the second row of~\cref{fig:ablation} and reduces the performance consequently.

We provide additional analysis on the impact of hyperparameters, scene depths, and out-of-distribution testing in the supplementary material.

This paper presented a comprehensive analysis of the use of \acrfull{PINN} for power system dynamic simulations. We show that \glspl{PINN} (i) are 10 to 1'000 times faster than conventional solvers, (ii) do not face issues of numerical instability unlike conventional solvers, and, (iii) achieve a decoupling between the power system size and the required solution time. However, \glspl{PINN} are less flexible (i.e. they do not easily handle parameter changes), and require an up-front training cost. Overall, this makes \gls{PINN}-based solutions well-suited for repetitive tasks as well as task where run-time speed is crucial, such as for screening.

Besides the comparison between conventional and \gls{NN}-based methods, this paper conducts a deeper analysis on the parameters that affect the performance of the \gls{NN} solutions. In that respect, we introduce a new \gls{NN} regularisation, called dtNN, as a intermediate step between \glspl{NN} and \glspl{PINN}. We show that \glspl{PINN} achieve overall higher levels of accuracy, and more balanced error distributions thanks to the evaluation of the collocation points.

\paragraph{Acknowledgements}
This work was supported by the National Research Foundation of Korea (NRF) grant funded by the Korea government (MSIT) (No. RS-2023-00208197) and Institute of Information \& communications Technology Planning \& Evaluation (IITP) grant funded by the Korea government (MSIT) [NO.2021-0-01343, Artificial Intelligence Graduate School Program (Seoul National University)].
Young Min Kim is the corresponding author.

%%%%%%%%% REFERENCES %%%%%%%%%
{\small
\bibliographystyle{ieee_fullname}
\bibliography{EgoNeRF_ref}
}

Below we first briefly describe the selected models and then their implementation details during pre-training.

% Traditional convolutional action recognition networks before 2017 are mostly built to process single frame or multiple consecutive frames; however, such simple structures overlook the importance of long-range temporal context in action recognition, which somehow underestimates the intrinsic temporal information within videos. 
Temporal segment networks (TSN) proposes segment-based sampling to learn temporal information across frames. 
Specifically, in TSN, a video is evenly divided into several temporal segments, which one random frame is sampled from. 
Then the output from each segment will be aggregated via pooling to obtain the final prediction. 
Temporal Shift Module (TSM) shifts feature channels along the temporal axis, which facilitates information exchanged among neighboring frames. 
It can be plug-and-played in 2D networks to enable stronger temporal modeling at zero computation and zero parameters.
Thus, TSM can achieve the performance of heavy 3D CNNs while maintaining the efficiency of 2D CNNs.
% TSM introduces stronger temporal learning capacity to 2D networks while maintaining light-weight. 

Inflated 3D ConvNet (I3D) is designed to bootstrap from the corresponding 2D network since (1) the architecture of 2D network is well designed and (2) the  weights of 2D network is well pre-trained, e.g., Inception~\cite{inception} $\rightarrow$ Inception-I3D~\cite{carreira2017quo}. 
% utilize pre-trained weights from the corresponding 2D network since these 2D weights have been well-designed and trained to perceive visual concepts.
I3D initializes its 3D kernels by duplicating the 2D ones along the temporal dimension, which helps the convergence of 3D CNNs. 
Inspired by~\cite{vaswani2017attention}, non-local networks (NL) adapts the non-local operation (i.e., self-attention~\cite{vaswani2017attention}) in its building block to model long-range dependency.
For video action recognition, its goal is to relate the same object, or person-object interaction within a distant time interval in videos.
Similar to TSM, non-local block is compatible to most convolutional networks.


TimeSformer is a pure transformer-based model, which is an extension of ViT~\cite{dosovitskiy2020image} to the spatiotemporal space. 
Given the quadratic complexity of self-attention, TimeSformer compares several attention strategies when considering temporal dimention in videos.
Finally, TimeSformer introduces the divided space-time attention to greatly reduce the computation burden but achieves promising results.
% on most video action recognition datasets. 
% This structure shows both effectiveness and efficiency in their reported results. 
Continuing this modeling shift from CNNs to Transformers, VideoSwin extends Swin Transformer~\cite{liu2021swin} by adding the inductive bias of locality in video transformers. 
Simply speaking, it adapts the idea of 2D shifted window self-attention to 3D space, which results in better speed-accuracy trade-off compared to previous approaches~\cite{bertasius2021space,arnab2021vivit}.
% Similarly, VideoSwin is an extension of Swin Transformer~\cite{liu2021swin}, by adapting the 2D shifted window self-attention to 3D.
% And shifted window ensure the connection across distant regions in the spatiotemporal tensors.


\begin{figure}[t]
\centering
    \includegraphics[width=8cm]{figures/radar_new.pdf}
    \caption{The rank of the averaged performance within different data domains for the 6 models in different settings. The most outside in these radar images means the highest performance. For each domain, we average the top-1 accuracy as the scores in finetuning and average the top-1 accuracy of 16-shot results in few-shot learning. Complete results are shown in Table~\ref{tab:finetune} and Figure~\ref{fewshot}.}
    \label{radar}
\end{figure}

\end{document}
