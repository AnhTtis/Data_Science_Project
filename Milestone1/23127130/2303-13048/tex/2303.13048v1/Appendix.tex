% \documentclass[aps,twocolumn,notitlepage]{revtex4-2}
% %\usepackage[utf8]{inputenc}
% \usepackage[dvips]{graphicx}% Include figure files
% \usepackage{dcolumn}% Align table columns on decimal point
% \usepackage{bm}
% \usepackage{graphicx,color}
% \usepackage{braket}
% \usepackage{amsmath} 
% \usepackage[colorlinks=true,citecolor=blue,linkcolor=blue]{hyperref}
% \begin{document}

\onecolumngrid
\setcounter{equation}{0}
\appendix*
\renewcommand{\theequation}{A.\arabic{equation}}


\section{Appendix A: Lumped element circuit and microwave reflection probability}
\label{Appendix A}
The QD-resonator system, linearly driven at a frequency $\omega$ close to resonance, is modelled as a lumped element circuit, shown in Fig. \ref{fig:Figure1} b). The resonator is described by an inductance $L$ and capacitance $C$ and internal losses are accounted for by a resistance $R_i$. The QD has a frequency dependent complex admittance $Y(\omega)$. The resonator is coupled to an input transmisison line, with impedance $Z_0$, via a coupling capacitance $C_\text C$. The total impedance $Z(\omega)$ of the circuit is then given by
%
\begin{equation}
Z=\frac{1}{i\omega C_\text C}+\left(Y(\omega)+1/R_i+i\omega C +\frac{1}{i\omega L}\right)^{-1}\equiv Z_\text R+i Z_\text I.
\end{equation} 
%
Writing $\omega_r=1/\sqrt{LC}, \kappa_\text i=R_i/C, \kappa_{\text{QD}}=\text{Re}[Y(\omega)]/C$, and $\delta \omega_{\text{QD}}=\text{Im}[Y(\omega)]/(2C)$ and using that $\omega \approx \omega_r$ we can write the real and imaginary parts of the impedance as
%
\begin{eqnarray}
Z_\text R&=&\frac{(\kappa_{\text{QD}}+\kappa_i)/(4C)}{(\kappa_{\text{QD}}+\kappa_\text i)^2/4+(\omega-\omega_\text r+\delta \omega_\text{QD})^2} \\
Z_\text I&=&-\left(\frac{1}{\omega_\text{r} C_{\text C}}+\frac{(\omega-\omega_\text r+\delta \omega_\text{QD})/C}{(\kappa_{\text{QD}}+\kappa_\text i)^2/4+(\omega-\omega_\text r+\delta \omega_\text{QD})^2}\right). \nonumber
\end{eqnarray} 
%
The reflection probability $R$ for a coherent microwave drive tone at $\omega$ is given by 
%
\begin{equation}
R=\left| \frac{Z(\omega)-Z_0}{Z(\omega)-Z_0}\right|^2 =1-\frac{4Z_RZ_0}{(Z_R+Z_0)^2+Z_\text I^2}
\label{refl}
\end{equation}
%
and the corresponding reflection phase is
%
\begin{equation}
\phi=\mbox{atan}\left(\frac{2Z_\text IZ_0}{Z_R^2+Z_\text I^2-Z_0^2}\right)
\end{equation}
%
For the reflection probability, inserting $Z_{\text R}$ and $Z_{\text I}$ into Eq. (\ref{refl}), writing the capacitive couping rate $\kappa_{\text C}=Z_0\omega_r^2C_\text{C}^2/C$ and neglecting terms proportional to the small parameter $C_\text{C}\omega_rZ_0 \ll 1$, we arrive at
%
\begin{equation}
R=1-\frac{(\kappa_{\text{QD}}+\kappa_\text{i})\kappa_\text{C}}{(\kappa/2)^2+(\omega-\omega_\text{r}^*+\delta \omega_\text{QD})^2},
\end{equation}
%
where we introduced the total $\kappa=\kappa_{\text{QD}}+\kappa_\text{i}+\kappa_\text{C}$ and $\omega_\text{r}^*=\omega_r(1+C_\text{C}/C)$, the capacitive coupling renormalized resonance frequency. By noticing that $C_\text{C}/C\ll1$ we can put $\omega_\text{r}^*\approx \omega_\text{r}$ and we then arrive at Eq. (\ref{eq:Reflection_Coefficient}) in the main text.




\section{Appendix B: Landauer-B\"uttiker Formalism}
\label{Appendix B}
An extended sketch of the main article Fig. \ref{fig:Figure1} a) is presented in Fig. \ref{syspic}.  It includes junction capacitances and potentials.
%
\begin{figure}[h]
  \centering
  {\includegraphics[scale=0.5]{Figures/Picsysapp2.jpg}}
  \caption{Sketch of the quantum dot showing, in addition to Fig. \ref{fig:Figure1} of the main article, the tunnel junction capacitances $C_\text{L},C_\text{R}$ and the applied potential $V(t)$ on the right contact and induced potential $U(t)$ in the QD.}
\label{syspic}
\end{figure}
%
In this appendix, we calculate the QD admittance within the Landauer-Büttiker formalism. 
%is calculated theoretically in two opposite limits, namely for a life-time broadened QD resonace, $\Gamma_\text{L} + \Gamma_\text{R} \gg k_\text{B} T/\hbar$, and for a temperature broadened resonance, $\Gamma_\text{L} + \Gamma_\text{R} \ll k_\text{B} T/\hbar$. The two limits, most conveniently treated with different theoretical methods, are discussed separately below.
%\subsection{Life-time broadening}
With this approach, the QD admittance $Y(\omega)$ is evaluated within a time-dependent scattering approach, neglecting Coulomb blockade effects but fully accounting for the current conservation at the QD via the flow of dynamic screening currents. Our result is an extension of the discussion presented by Pretre, Thomas and B\"uttiker, \cite{Prtre1996DynamicAO}, here including QD-lead capacitive couplings. We therefore present only the main steps in the derivation. 

The staring point for the calculation is the energy dependent, symmetric scattering matrix $S(E)$ of the QD, assuming effectively a single transport channel, given by
%
\begin{equation}
S(E)=\left(\begin{array}{cc} r(E) & t'(E) \\ t(E) & r'(E) \end{array}\right),
\end{equation}
%
where the reflection and transmission amplitudes are given by the Breit-Wigner expressions 
%
\begin{eqnarray}
r(E)&=&1-\frac{i\Gamma_L}{E-\epsilon+i(\Gamma_L+\Gamma_R)/2} \nonumber \\ 
r'(E)&=&1-\frac{i\Gamma_R}{E-\epsilon+i(\Gamma_L+\Gamma_R)/2} \nonumber \\
 t(E)&=&t'(E)=\frac{i\sqrt{\Gamma_L\Gamma_R}}{E-\epsilon+i (\Gamma_L+\Gamma_R)/2}.
\label{smat}
\end{eqnarray}
%
Here $\epsilon=\epsilon_\text{d}-\alpha V_\text{G}$ is the energy of the discrete QD level where $\epsilon_\text{d}$ is the bare dot energy and  $\alpha= e C_\text G/(C_\text L+C_\text R+C_\text G)$ the lever arm for the gate potential $V_G$. Unprimed (primed) amplitudes correspond to particles incident from the left (right) lead.

We consider the case with a pure AC-voltage $V(t)=V\cos(\omega t)$ at contact $R$, while contact $L$ is grounded and the gate contact is kept at the constant potential $V_\text G$ corresponding to the experimental settings. The case with a pure DC-voltage bias is discussed below. As a result of the oscillating potential $V(t)$, a potential $U(t)$ is induced on the QD. The effect of the oscillating potentials is that electrons can pick up or loose quanta of energy $\hbar \omega$ when scattering at the QD. 

Our focus is on the regime of weak microwave drive, where the response is linear in the 
potentials. In this regime $V\ll \hbar \omega$ and only a single quantum can be picked up or lost. As a consequence the time dependent particle current at lead L/R has only a single Fourier component, 
%
\begin{equation}
I_\text{L/R}(t)=I_\text{L/R}(\omega)e^{i\omega t}+I_\text{L/R}^*(\omega)e^{-i\omega t}.
\end{equation}
%
The current component $I_\text{L/R}(\omega)$ can be expressed in terms of the scattering amplitudes in Eq. (\ref{smat}) and the lead Fermi distribution $f(E)$ as
%
%
\begin{equation}
I_\text L(\omega)=\frac{e^2}{h}\int dE \{-\left[1-r^*(E)r(E+\hbar \omega)\right]U(\omega) %\nonumber \\
-t'^*(E)t'(E+\hbar \omega)\left[V(\omega)-U(\omega)\right]\}F(E,\omega),
\label{IL}
\end{equation}  
%
and
%
\begin{equation}
I_\text R(\omega)=\frac{e^2}{h}\int dE \{\left[1-r'^*(E)r'(E+\hbar \omega)\right]\left[V(\omega)-U(\omega)\right] %\nonumber \\
+t^*(E)t(E+\hbar \omega)U(\omega)\}F(E,\omega) ,
\label{IR}
\end{equation}  
%
where
%
\begin{equation}
F(E,\omega)=\frac{f(E)-f(E+\hbar \omega)}{\hbar \omega}, \hspace{0.2cm} f(E)=\frac{1}{1+e^{E/k_\text B T}},
\end{equation}
%
and we have introduced the Fourier components $V(\omega)$ and $U(\omega)$ of the potentials $V(t)$ and $U(t)$. We note that $V(\omega)=V/2$, independent on $\omega$, but the frequency dependent notation is kept for convenience. 

Inserting the scattering amplitude expressions in Eq. (\ref{smat}) into the current components in Eqs. (\ref{IL}) and (\ref{IR}) we can write
%
\begin{equation}
I_\text L(\omega)=G(\omega)\left[\frac{i\hbar\omega}{\Gamma_{\text R}}U(\omega)-V(\omega)\right],
\end{equation}
%
and
%
\begin{equation}
I_\text R(\omega)=G(\omega)\left[\frac{i\hbar \omega}{\Gamma_{\text L}}U(\omega)+\left(1-\frac{i\hbar \omega}{\Gamma_{\text L}}\right)V(\omega)\right], 
\label{IRcurr}
\end{equation}
% 
where
%
\begin{equation}
G(\omega)=\frac{e^2}{h}\int dE \mathcal{T}(E,\omega)F(E,\omega),
\label{grel}
\end{equation}
%
and
%
\begin{equation}
\mathcal{T}(E,\omega)=\frac{\Gamma_L\Gamma_R}{E-\epsilon+i(\Gamma_L+\Gamma_R)/2} %\nonumber \\
\frac{1}{E+\hbar \omega-\epsilon-i(\Gamma_L+\Gamma_R)/2}.
\end{equation}
%

For non-zero frequencies, the particle currents flowing into the QD typically do not add up to zero, i.e. $I_\text L(\omega)+I_\text R(\omega) \neq 0$. As a consequence, there is nonzero charge $Q(t)$ on the QD dot which induces AC screening, or displacement, currents flowing between the QD and the leads L/R as well as the gate G. The total screening current into the QD is given by $I_\text{sc}(t)=dQ(t)/dt$, with the charge determined from classical electrostatical considerations, via the potentials $V(t)$ and $U(t)$ and the capacitances $C_\text L$, $C_\text R$, $C_\text G$. This gives the screening current Fourier component $I_\text{sc}(\omega)$ as 
%
\begin{equation}
I_\text{sc}(\omega)=-i\omega \left[-U(\omega)(C_\text L+C_\text R+C_\text G)+V(\omega)C_\text R\right].
\label{disp}
\end{equation}
%
The induced QD potential $U(\omega)$ can then be determined from the condition that the total current flowing into the dot is conserved, 
%
\begin{equation}
I_\text L(\omega)+I_\text R(\omega)+I_\text {sc}(\omega)=0,
\end{equation}
%
giving
%
\begin{equation}
U(\omega)=\frac{\Gamma_\text R(\hbar G(\omega)+C_R\Gamma_\text L)}{\hbar G(\omega)(\Gamma_\text L+\Gamma_\text R)+(C_\text L+C_\text R+C_\text G)\Gamma_\text R\Gamma_\text L}V(\omega).
\label{pot}
\end{equation}
%
We note that in the limit $\omega \rightarrow \infty$, we have $G(\omega)\rightarrow 0$ and $U(\omega)=C_\text{R}/(C_\text L+C_\text R+C_\text G) V(\omega)$, purely capacitive voltage division. The sought admittance is given by 
%
\begin{equation}
Y(\omega)=\frac{I_R(\omega)-i \omega C_R[V(\omega)-U(\omega)]}{V(\omega)},
\end{equation}
% 
as the ratio of the total current, i.e., the sum of the particle and screening current, flowing into the QD from contact R and the potential at R.
 Inserting the expression of $I_R(\omega)$ in Eq. (\ref{IRcurr}) and $U(\omega)$ in (\ref{pot}) we arrive at
%
\begin{equation}
Y(\omega)=G(\omega)-\frac{i\omega\left[\hbar G(\omega)+\Gamma_\text L C_\text R\right]\left[\hbar G(\omega)+\Gamma_\text R(C_\text L+C_\text G)\right]}{\Gamma_\text L\Gamma_\text R(C_\text L+C_\text R+C_\text G)+\hbar G(\omega)(\Gamma_\text L+\Gamma_\text R)}.
\label{GR}
\end{equation}
 %
This is the expression used for the numerical evaluations in the main text.

For the life-time broadening cases $\Gamma_\text{L} + \Gamma_\text{R} \gg k_\text{B} T$, effectively taking $T=0$, we can evaluate the integral in Eq. (\ref{grel}) giving
%
%
\begin{equation}
G(\omega)=\frac{e^2}{h}\frac{i\Gamma_L\Gamma_R}{\hbar \omega(\Gamma_L+\Gamma_R-i\hbar \omega)} %\nonumber \\
 \ln \left[\frac{\epsilon^2+([\Gamma_L+\Gamma_R]/2-i\hbar \omega)^2}{\epsilon^2+(\Gamma_L+\Gamma_R)^2/4}\right].
\end{equation}
%
We note that at $\omega \rightarrow 0$ we have $I_\text R(0)=-I_\text L(0)=G(0)V(0)$, where  
%
\begin{equation}
G(0)=\frac{e^2}{h}\frac{\Gamma_L\Gamma_R}{\epsilon^2+(\Gamma_L+\Gamma_R)^2/4}
\label{DC_Lifetime_fit}
\end{equation}
is the known DC-bias conductance, as expected. 
%
%
%
%
%
%


\section{Appendix C: Sequential Tunneling Model with Periodic Voltage Drive in the Classical Limit}
\label{Appendix C}
Another extensively used model to describe QD transport builds on sequential tunneling of the electrons. Within this model, charging effects are intrinsically accounted for, while the lifetime broadening effects are neglected. This appendix calculates the QD admittance $Y(\omega)$ within the sequential tunneling approach for an applied, time periodic voltage $V(t)=V_\text{dc}+V_{ac}\sin(\omega t)$. We follow closely the work of Bruder and Sch\"oller \cite{BruderAndShoeller}, fully accounting for Coulomb blockade effects, and presenting only the main steps in the derivation. The starting point is a rate equation for the Fourier components of the probabilities $P_0(t)$ and $P_1(t)$ to have 0 or 1 extra electron on the dot. Writing the Fourier series
%
\begin{equation}
P_j(t)=\sum_{m=-\infty}^{\infty}\tilde P_j(m)e^{-im\omega t}, \hspace{0.5cm} j=0,1
\end{equation}
%
and noting that $\tilde P_j^*(m)=\tilde P_j(-m)$, the rate equation can be written as
%
\begin{equation}
-im\hbar \omega \tilde P_0(m)=N(\Gamma_L+\Gamma_R)\tilde P_1(m) %\nonumber \\
-\sum_{\alpha=L,R}\sum_{n=-\infty}^{\infty} A_{nm}^{\alpha}\left[\tilde P_0(n)+N\tilde P_1(n)\right],
\label{MEdot}
\end{equation}
%
where
%
\begin{equation}
A_{nm}^{\alpha}=F_{n,m}^{\alpha}+\left(F_{-n,-m}^{\alpha}\right)^*,  %\\
F_{nm}^{\alpha}=\frac{i^{m-n}}{2}\sum_{k=-\infty}^{\infty}J_{k+n}\left(\frac{eV_\alpha}{\hbar \omega}\right)J_{k+m}\left(\frac{eV_\alpha}{\hbar \omega}\right)f_{\alpha}(\epsilon_k), %\nonumber
\end{equation}
%
and $J_n(x)$ is the Bessel function and $f_\alpha(\epsilon_k)$ the Fermi function of lead $\alpha=L,R$ (incorporating the dc-bias $V_\text{dc}$ at contact R) and $\epsilon_k=\epsilon+k\hbar \omega$ with $\epsilon$ the dot energy. The AC-potential drops $V_\text{L/R}$ across the L/R barriers are $V_L=-U, V_R=V_{ac}-U$ where $U=V_{ac}C_R/(C_R+C_L)$ is the amplitude of the induced dot potential. Here we have assumed $C_G \ll C_R, C_L$ as is the case for the studied devices. The integer $N$ arises from the degeneracy of the QD energy level~\cite{Hofmann2016,Barker2022}, i.e. it sets the number of possible charge states in the dot, effectively multiplying the out tunneling rates. For our spin degenerate QD we have $N = 2$. Probability conservation condition, from $P_0(t)+P_1(t)=1$ for any $t$, further gives
%
\begin{equation}
\tilde P_0(m)+\tilde P_1(m)=\delta_{m0}.
\end{equation}
%
The Fourier component of the particle current $I_\alpha(t)$ in lead $\alpha$ is given by
%
\begin{equation}
\tilde I_\alpha(m)=\frac{e\Gamma_\alpha}{\hbar}\left(N \tilde P_1(m)-\sum_{n=-\infty}^{\infty} A_{nm}^{\alpha}\left[\tilde P_0(n)+N\tilde P_1(n)\right]\right).
\end{equation}
% 
As in the life-time broadened limit, the total current is obtained by the sum of the particle currents and the displacement, or screening, currents. However, as pointed out in Ref.~\citealp{BruderAndShoeller}, the screening currents are typically very small in Coulomb blockaded systems and we neglect them here. This gives the expression for the Fourier components of the total current
%
\begin{equation}
\tilde I(m)=\frac{C_R}{C_L+C_R}\tilde I_L(m)-\frac{C_L}{C_L+C_R}\tilde I_R(m),
\label{totcurrcomp}
\end{equation}
%
valid for arbitrarily drive amplitude and frequency.

Similar to the Landauer-Büttiker theory in Appendix B, we focus on the regime of small ac-driving amplitude, $eV_{ac} \ll \hbar \omega$. Expanding the Bessel functions $J_n$ in the argument $eV_{ac}/(\hbar \omega)$ and writing the probabilities $\tilde P_j(m)=\tilde P_j^{(0)}(m)+(eV_{ac}/[\hbar \omega])\tilde P_j^{(1)}(m)+... $, we can solve Eq. (\ref{MEdot}) order by order in $eV_{ac}/(\hbar \omega)$. To evaluate the DC-current $\tilde I(0)$ (in the absence of AC-drive) as well as the first AC-component $\tilde I(1)$, we need only the non-zero probability components $\tilde P_j^{(0)}(0)=1-\tilde P_1^{(0)}(0)$ (at $V_\text{ac}=0$) and $\tilde P_j^{(1)}(1)=-\tilde P_1^{(1)}(1)$ (at $V_\text{dc}=0$), given by
%
\begin{equation}
\tilde P_0^{(0)}(0)=\frac{N\left[\Gamma_L(1-f_L)+\Gamma_R(1-f_R)\right]}{\Gamma_Lf_L+\Gamma_Rf_R+N\left[\Gamma_L(1-f_L)+\Gamma_R(1-f_R)\right]},
\end{equation}
%
where we for shortness write $f_L=f_L(\epsilon), f_R=f_R(\epsilon)$, and
%
\begin{equation}
\tilde P_0^{(1)}(1)=\frac{-ie}{2\hbar \omega}\frac{\left[\Gamma_L V_L+\Gamma_R V_R\right]}{(\Gamma_L+\Gamma_R)\left[N+(1-N)f(\epsilon)\right]-i\hbar \omega}
%\nonumber \\
 \left[N+(1-N)\tilde P_0^{(0)}(0)\right]\left[f(\epsilon_1)-f(\epsilon_{-1})\right].
\end{equation}
%
Within the same small amplitude approximation we have the current components
%
\begin{equation}
\tilde I_\alpha(0)=\frac{e\Gamma_\alpha}{\hbar}\left(N\tilde P_1^{(0)}(0)-f_{\alpha}\left[N+(1-N) \tilde P_0^{(0)}(0)\right]\right)
\end{equation}
%
and
%
\begin{equation}
\tilde I_\alpha(1)=\frac{e\Gamma_\alpha}{\hbar}\left(-\left[N+(1-N)f(\epsilon)\right]\tilde P_0^{(1)}(1) \right.  %%\\
-\left. \frac{ie V_\alpha}{2\hbar \omega}\left[N+(1-N)\tilde P_0^{(0)}(0)\right]\left[f(\epsilon_1)-f(\epsilon_{-1})\right]\right).
\end{equation}
%
Inserting the expressions for the probabilities and noting that $\tilde I_L(0)=-\tilde I_R(0)$, the DC-current becomes
%
\begin{equation}
\tilde I(0)=e\frac{N\Gamma_L\Gamma_R}{\Gamma_L+\Gamma_R}\frac{f_R-f_L}{(1-N)(f_L\Gamma_L+f_R\Gamma_R)+N(\Gamma_R+\Gamma_R)}.
\end{equation}
%
Expanding the Fermi distributions to first (linear) order in dc-bias voltage we arrive at the linear conductance
%
\begin{equation}
G=\tilde I(0)/V_{dc}=-\frac{e^2}{2\pi}\frac{N\Gamma_L\Gamma_R}{\Gamma_L+\Gamma_R}\frac{df(\epsilon)}{d\epsilon}\frac{1}{N+(1-N)f(\epsilon)}.
\label{eq:diffCondBS}
\end{equation}
which is the expression used in the plot of Fig.~\ref{fig:SymQD}~c) in the main article. 
With $N = 1$, this expression matches the standard non-degenerate result, found e.g. in Ref.~\citealp{Ihn2010}, and the Landauer-Büttiker result in the limit of $\Gamma_\text{L} + \Gamma_\text{R} \ll k_\text{B} T$ and $\omega \rightarrow 0$.


For the AC-current, we can write $\tilde I(1)=iY(\omega) V_{ac}/2$, with the admittance ($C_\Sigma=C_L+C_R$)
%
\begin{equation}
Y(\omega)= \frac{e^2}{2\pi} \frac{N\left[f(\epsilon_1)-f(\epsilon_{-1})\right]}{\hbar \omega \left[N+f(\epsilon)(1-N)\right]}
 \frac{\Gamma_L\Gamma_R\left[N+f(\epsilon)(1-N)\right]-i\hbar\omega\left[\Gamma_L\frac{C_R^2}{C_\Sigma^2}+\Gamma_R\frac{C_L^2}{C_\Sigma^2}\right]}{\left[(\Gamma_L+\Gamma_R)\left[N+(1-N)f(\epsilon)\right]-i\hbar\omega \right]},
\label{eq:BnS_ST_theory_equation}
\end{equation}
%
which is the expression used for the fit of Fig.~\ref{fig:SymQD}~d) in the main article. We note that for $\omega \rightarrow 0$ we have $Y(0)=G$, as expected.



%\bibliographystyle{apsrev4-2}
%\bibliography{referencesurl}

% \end{document} 


