% \documentclass[aps,onecolumn,notitlepage]{revtex4-2}
% %\usepackage[utf8]{inputenc}
% \usepackage[dvips]{graphicx}% Include figure files
% \usepackage{dcolumn}% Align table columns on decimal point
% \usepackage{bm}
% \usepackage{graphicx,color}
% \usepackage{braket}
% \usepackage{amsmath}
% \usepackage{etaremune}
% \usepackage{braket}
% \usepackage{amsmath} 
% \usepackage[colorlinks=true,citecolor=blue,linkcolor=blue]{hyperref}


% \begin{document}


\section{Appendix D: Sequential Tunneling Model with P(E) Theory}
\label{Appendix D}
The sequential tunneling model of Appendix C treats the voltage drive $V(t)$ as a purely classical signal. Here we consider briefly another sequential tunneling model approach, the $P(E)$ theory, that treats the voltage in the resonator quantum mechanically.
To calculate the electron and photon transport properties of the studied system within the $P(E)$ theory, we follow the formalism of Refs.~\citealp{Ingold1992,maisi2014b,Souquet2014}. In the presence of a photon environment, the tunneling rate $\Gamma_+$ into the quantum dot (QD) from an electronic reservoir and the opposite rate $\Gamma_-$ out from it~\cite{Hofmann2016,Barker2022} are convoluted with the probability $P(E)$ to absorb energy $E$ from the environment. The resulting tunneling rates for left, $i = L$, and right, $i = R$, tunnel junction are
\begin{equation}
\label{eq:GammaNQD}
\left\{
\begin{array}{ccl}
\Gamma_{i+}(\varepsilon_i) &=& \displaystyle \frac{\Gamma_i}{h} \int_{-\infty}^\infty dE\; P(E) f(\varepsilon_i + E) \vspace{3pt} \\
%
\Gamma_{i-}(\varepsilon_i) &=& \displaystyle N\; \frac{\Gamma_i}{h} \int_{-\infty}^\infty dE\; P(-E) \left[ 1-f(\varepsilon_i + E) \right].
\end{array}
\right.
\end{equation}
Here $\Gamma_i$ is the tunnel coupling strength, $\varepsilon_i$ the QD energy level position with respect to the reservoir Fermi level, $f(E)$ is the Fermi function defining the electron occupation distribution in the reservoir, and the additional pre-factor $N$ for tunneling out arises from the degeneracy of the considered QD energy level~\cite{Hofmann2016,Barker2022}, as above. For our system, we have $N=2$ as there are two electrons to choose from to tunnel out and only one vacant state to tunnel into the QD.


The $P(E)$ function is set by the environment. In our case, the resonator forms a single photon mode at frequency $\omega_r$ as the environment~\cite{Girvin2014,Souquet2014}. For the characteristic impedance $Z_0 = 53~\Omega$ of our resonator, we have $z = \pi Z_0 G_0 \ll 1$, where $G_0 = e^2/h$ is the conductance quantum. For the coherent drive used in the experiment with low enough input power and at low temperature $kT \ll \hbar \omega_r$, the resonator is in a coherent state with small photon number $\left<n\right> \ll 1/z$. In this limit, the $P(E)$ function reads~\cite{Souquet2014}
\begin{equation}
\label{eq:PEsinglephoton}
P(E) = \Big[ 1 - z \nu^2 \big(2 \left<n\right> + 1 \big)  \Big] \delta(E) + z \nu^2 \left<n\right>\, \delta(E + \hbar\omega) + z \nu^2 \big(\left<n\right> + 1 \big) \delta(E - \hbar\omega),
\end{equation}
where $\nu$ is the fraction of the resonator voltage that appears accross the tunnel barrier where the tunneling takes place~\cite{Childress2004}. We assume that the junction capacitances of the quantum dot are equal. In addition, the gate capacitance is much smaller than the junction capacitances in the studied devices. Under these conditions, the resonator voltage is divided evenly over the two tunnel barriers, and $\nu = 1/2$ and the $P(E)$ function is the same for the two tunnel junctions.

The three terms in Eq.~(\ref{eq:PEsinglephoton}) correspond to 1) no interactions with the photons, 2) absorption of a photon and 3) emission of a photon with probabilities $P_0 = 1 - z \nu^2 \left(2 \left<n\right> + 1\right)$, $P_{-1} = z \nu^2 \, \left<n\right>$ and $P_{+1} = z \nu^2 \left(\left<n\right> + 1 \right)$ respectively. Interestingly, the photon absorption or emission probability is solely set by the impedance $z$, coupling $\nu$ and photon number $\left<n\right>$ of the microwave resonator and does not depend on the tunnel coupling $\Gamma_i$. The absorption and emission rates, however, depend on $\Gamma_i$ via Eq.~(\ref{eq:GammaNQD}). Note also that Eq.~(\ref{eq:PEsinglephoton}) contains only single photon absorption and emission processes. Multiphoton processes are suppressed in the low drive limit $\left<n\right> \ll 1/(z \nu^2)$, in line with the experiments of Ref.~\citealp{Haldar2022}.


\section{Tunneling rates and electrical conduction}
With Eqs.~(\ref{eq:GammaNQD}-\ref{eq:PEsinglephoton}), we obtain the tunneling rates as
\begin{equation}
\label{eq:GammaP}
\left\{
\begin{array}{ccl}
\Gamma_{i+}(\varepsilon_i) &=&  \displaystyle \frac{\Gamma_i}{h}  \Big[ P_0 f(\varepsilon_i) + P_{-1} f(\varepsilon_i-\hbar \omega) +  P_{+1} f(\varepsilon_i+\hbar \omega)  \Big] \vspace{3pt} \\
%
\Gamma_{i-}(\varepsilon_i) &=& N  \displaystyle \frac{\Gamma_i}{h}  \Big[ P_0 [1-f(\varepsilon_i)] + P_{-1} [1-f(\varepsilon_i+\hbar \omega)] +  P_{+1} [1-f(\varepsilon_i-\hbar \omega)]  \Big] = N \Gamma_{i+}(-\varepsilon_i).
\end{array}
\right.
\end{equation}
 The energy differences across the two junctions are $\varepsilon_L = \varepsilon_d + \alpha V_G - eV_\mathrm{b}/2$ and $\varepsilon_R = \varepsilon_d + \alpha V_G + eV_\mathrm{b}/2$, where $\alpha$ is the gate lever arm, $V_g$ the voltage applied to a gate electrode and $V_b$ the bias voltage between the source and drain. Here we have again assumed that the junction capacitances are equal and the gate capacitance is small compared to the junction capacitances which divide the bias voltage $V_b$ evenly over the junctions.

Next we determine the electrical conductance $G$ by setting a rate equation to describe the probabilitiy $p$ to have an excess electron in the QD and $(1-p)$ of not having the excess electron. This rate equation reads
\begin{equation}
\label{eq:rateEq}
\frac{dp}{dt} = -\Gamma_-\, p + \Gamma_+\, (1-p),
\end{equation}
where $\Gamma_- = \Gamma_\mathrm{L-}(\varepsilon_\mathrm{L}) + \Gamma_\mathrm{R-}(\varepsilon_\mathrm{R})$ is the sum of the rates for the electron to tunnel out and  $\Gamma_+ = \Gamma_\mathrm{L+}(\varepsilon_\mathrm{L}) + \Gamma_\mathrm{R+}(\varepsilon_\mathrm{R})$ the sum of the rates to tunnel in. The steady-state solution, $dp/dt = 0$, is
\begin{equation}
\label{eq:p}
p = \frac{\Gamma_+}{\Gamma_+ + \Gamma_-}.
\end{equation}
The electrical current $I$ through the QD is given by
\begin{equation}
\label{eq:p}
I = e \Big[ \Gamma_\mathrm{L+}(\varepsilon_\mathrm{L})\, (1-p) - \Gamma_\mathrm{L-}(\varepsilon_\mathrm{L})\, p \Big] = e
\frac{\Gamma_\mathrm{L+}(\varepsilon_\mathrm{L})\Gamma_\mathrm{R-}(\varepsilon_\mathrm{R}) - 
\Gamma_\mathrm{L-}(\varepsilon_\mathrm{L}) \Gamma_\mathrm{R+}(\varepsilon_\mathrm{R})}
{\Gamma_\mathrm{L+}(\varepsilon_\mathrm{L}) + \Gamma_\mathrm{R+}(\varepsilon_\mathrm{R}) + 
\Gamma_\mathrm{L-}(\varepsilon_\mathrm{L}) + \Gamma_\mathrm{R-}(\varepsilon_\mathrm{R})}. 
\end{equation}
Finally, the differential conductance is obtained as
\begin{equation}
\label{eq:p}
G = \frac{dI}{dV_\mathrm{SD}}.
\end{equation}
For small bias voltage, $eV_\mathrm{SD} \ll kT$, and low photon absorption and emission probabilities, $P_{-1}, P_{+1} \ll P_0$, we obtain
\begin{equation}
\label{eq:p}
G = \frac{e^2}{hkT}\; \frac{\Gamma_\mathrm{L}\Gamma_\mathrm{R}}{\Gamma_\mathrm{L} + \Gamma_\mathrm{R}}\; \frac{1}{1 + \frac{1}{N} + \frac{2}{\sqrt{N}} \cosh\left((\varepsilon_d + \alpha V_G)/kT +   \frac{1}{2} \ln N \right)},
\end{equation}
which is the zero bias conductance $G$ for a QD with degeneracy $N$, and matches Eq.~(\ref{eq:diffCondBS}).
 Compared to the standard non-degenerate result with $N = 1$, the degeneracy $N \neq 1$ keeps the lineshape the same but displaces the peak position by $(kT \ln N)/2$ in energy and increases the maximum value by a factor of $4/(1+ 1/\sqrt{N})^2$. For $N = 2$, these are both small effects: $(kT \ln N)/2 \approx 0.35\; kT$ and $4/(1+ 1/\sqrt{N})^2 \approx 1.37$. With photon absorption accounted for, two side peaks appear in $G$. These are separated by $\hbar \omega$ in energy from the main conduction peak. The size of these ones are $P_{-1}/P_0$ relative to the main peak, and can therefore be neglected for our devices with $P_{-1} \ll P_0$.

\section{Photon absorption and emission rates}
In the above, we determined the low frequency electrical conductance $G$. By collecting the photon absorption terms from Eq.~(\ref{eq:GammaP}), we obtain the photon absorption rate as
\begin{equation}
\label{eq:gammaPhoton}
\Gamma_-^\mathrm{photon} = P_{-1} \Big[ \big( \Gamma_\mathrm{L}\; f(\varepsilon_L-\hbar \omega) + \Gamma_\mathrm{R}\; f(\varepsilon_R-\hbar \omega) \big) (1-p) + N \big( \Gamma_\mathrm{L}\; f(-\varepsilon_L-\hbar \omega) + \Gamma_\mathrm{R}\; f(-\varepsilon_R-\hbar \omega) \big) p \Big]/h,
\end{equation}
where we have summed the absorption terms with the corresponding weights of $(1-p)$ and $p$ of the probability to be in the right starting state of the QD. The absorption rate per photon corresponds to a resonator loss term $\kappa_\mathrm{QD}$ which is thus
\begin{equation}
\label{eq:gammaPhoton}
\kappa_\mathrm{QD} = \frac{\Gamma_-^\mathrm{photon}}{\left<n\right>} = \nu^2 z \Big[ \big( \Gamma_\mathrm{L}\; f(\varepsilon_L-\hbar \omega) + \Gamma_\mathrm{R}\; f(\varepsilon_R-\hbar \omega) \big) (1-p) + N \big( \Gamma_\mathrm{L}\; f(-\varepsilon_L-\hbar \omega) + \Gamma_\mathrm{R}\; f(-\varepsilon_R-\hbar \omega) \big) p \Big]/h.
\end{equation}
For zero bias voltage, $V_\mathrm{SD} = 0$, at low temperature, $kT \ll \hbar \omega$, and small photon number $\left<n\right>$, we have $\varepsilon_L = \varepsilon_R$ and the photon absorption takes place in two distinct regimes. The first one is with $-\hbar \omega < \varepsilon_i < 0$. Here we have $p = 1$, i.e. the quantum dot is occupied essentially at all times and $f(-\varepsilon_{i}-\hbar \omega) = 1$. With this, Eq.~(\ref{eq:gammaPhoton}) yields
\begin{equation}
\label{eq:kappan}
\kappa_\mathrm{QD} = z \nu^2 (\Gamma_\mathrm{L}+\Gamma_\mathrm{R})N/h.
\end{equation}
The other regime takes place for $0 < \varepsilon_i < \hbar \omega$. In this case we have the QD essentially always unoccupied, i.e. $p = 0$ while the energy $\varepsilon_i$ is still low enough so that $f(\varepsilon_{i}-\hbar \omega) = 1$. Now we have 
\begin{equation}
\label{eq:kappap}
\kappa_\mathrm{QD} = z \nu^2 (\Gamma_\mathrm{L}+\Gamma_\mathrm{R})/h,
\end{equation}
which is the same as above but smaller by the degeneracy factor $N$. The sum $(\Gamma_\mathrm{L}+\Gamma_\mathrm{R})$ reflects the fact that with no bias voltage applied, the two junctions are effectively in parallel. The dashed lines of Fig.~\ref{fig:SymQD} d) of the main article are the results of Eqs.~(\ref{eq:kappan}) and (\ref{eq:kappap}) and show that the measured response follows these values. The fitted full curve in the figure, based on Appendix C, is essentially the same as the result of Eq.~(\ref{eq:gammaPhoton}). This demonstrates that the two sequential tunneling models yield identical results in this regime. For a stronger coupling via larger impedance $Z_0$, deviances are expected: in the absence of the drive, the calculation of Appendix C predicts still no changes to the standard dc transport result while the $P(E)$ theory has a strong spontaneous emission process and thus a suppressed probability $P_0$ for the dc transport known as the dynamical Coulomb blockade~\cite{Altimiras2014}.

% \bibliographystyle{apsrev4-2}
% \bibliography{referencesurl}

% \end{document} 



%This could be of relevance for the main article: https://journals.aps.org/prapplied/abstract/10.1103/PhysRevApplied.16.014007