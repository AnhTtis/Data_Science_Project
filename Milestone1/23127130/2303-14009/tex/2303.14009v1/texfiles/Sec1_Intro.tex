\section{Introduction}
\label{sec:Introduction}

\IEEEPARstart{G}{raph} neural networks (GNNs) have attracted considerable attention owing to their superior performance in graph-based learning applications~\cite{kipf2016semi,hamilton2017inductive}. Researchers have successfully utilized GNNs for several electronic design automation (EDA) tasks, such as floorplanning optimization, 
estimating routing congestion, 
and assessing circuit reliability~\cite{GNN4REL,aspdacsurvey}, to name a few. 
The outstanding success of GNNs in EDA is primarily because Boolean circuits can be naturally represented as graphs. 
Recently, security researchers have incorporated GNNs into several
hardware security-related tasks~\cite{alrahis2022embracing} and have demonstrated state-of-the-art performance in the detection of hardware Trojans (HTs)~\cite{yasaei2021gnn4tj,yu2021hw2vec,GNN4TJ_Journal}, 
detection of intellectual property (IP)~\cite{yasaei2021gnn4ip,yu2021hw2vec}, reverse engineering of gate-level netlists~\cite{chowdhury2021reignn,GNNRE,bucher2022appgnn}, unlocking hardware obfuscation~\cite{gnnunlockp,omla,untangle}, and prediction of attack run-time on logic locking~\cite{chen2020estimating}.

\begin{figure}[tb]
\centering
\includegraphics[width=0.48\textwidth]{figures/1.pdf}
\caption{\textbf{Scope of this work}. 
Proposed backdoor attack against GNN-based hardware security systems. 
Malicious circuits evade detection due to the injected backdoor triggers, reducing trust in the globalized IC supply chain.}
\label{fig:gnn4tj}
\end{figure}

While GNNs offer great benefits, they create new attack vectors in the integrated circuit (IC) supply chain, especially when companies outsource GNN training to third-party entities such as \textit{machine learning as a service} (MLaaS) providers~\cite{ribeiro2015mlaas,xi2021graph,zhang2021backdoor,gu2017badnets}. 
This can also be the case, when a third-party dataset is used for training. 
For example, when a GNN is utilized for HT or IP piracy detection,\footnote{We consider the threats of HT detection and IP piracy since they are the major threat vectors identified by not only academic practitioners but also defense agencies.} as shown in Fig.~\ref{fig:gnn4tj}~\Circled{\scriptsize\textbf{1}}, an adversary might attack the employed GNN (e.g., by poisoning the training dataset) to evade detection, as illustrated in Fig.~\ref{fig:gnn4tj}~\Circled{\scriptsize\textbf{2}}.

\textit{To the best of our knowledge, the susceptibility of GNNs to poisoning attacks has been unexplored, especially in the context of hardware security-related problems, which is alarming given their increasing use.
\textit{To that end, it is imperative to identify potential security vulnerabilities before wide-scale usage and deployment.}
}



\begin{table*}[ht]
\caption{Definition of Key Terms Used in This Work}
\label{tab:terms}
\resizebox{\textwidth}{!}{%
\setlength\tabcolsep{1.9pt} % 
\begin{tabular}{l|l}
\hline
\textbf{Term} & \textbf{Description} \\ \hline
\multirow{2}{*}{Backdoor trigger} & The prediction of a backdoored model is changed for input samples that satisfy some secret, adversary-chosen property, referred to as the backdoor trigger~\cite{gu2017badnets}. \\ 
 & In the context of GNNs, the backdoor trigger is in the form of a subgraph. Backdoor triggers are not to be confused with the trigger circuitry of hardware Trojans \\ \hline
Design library & A training/testing dataset that contains a number of digital circuits either in RTL or gate-level representation \\ \hline
 Functional coverage & A measure of what functionalities of the design have been executed during simulation. It can detect hardware backdoors, which are rarely activated~\cite{fanci} \\ \hline
 Subgraph & A subgraph $\ssub{g}{t}$ of a graph $G$ is a graph whose node set and edge set are subsets of those of $G$\\ \hline 
 Sub-circuit & A self-contained circuit that appear in other larger circuits \\ \hline
 Hardware Trojans & Malicious modifications of circuits, aimed to leak secret assets on the chips or cause function disruption~\cite{tehranipoor2010survey}\\ \hline
\end{tabular}}
\end{table*}


\subsection{Motivation and Research Challenges}
\label{sec:motivation_research_challenges}


A backdoor attack \textit{stamps} chosen input samples (\textit{i.e.,} input graphs) with a \textit{backdoor trigger} (\textit{i.e.,} secret node connectivity pattern, e.g., a subgraph with a specific density/size) that causes the malicious behavior of the GNN, as seen in Fig.~\ref{fig:gnn4tj}~\Circled{\scriptsize\textbf{2}}.\footnote{Further details on backdoor attacks are included in Section~\ref{sec:backdoor_attacks_GNN}.} Existing backdoor attacks on GNNs generate (random) subgraphs with specific sizes and densities to act as backdoor triggers~\cite{xi2021graph,zhang2021backdoor}. 
However, for Boolean circuits, backdoor trigger generation cannot be randomized since the added structure: (i)~should not affect the functionality of the design and (ii)~must pass \textit{functional coverage tests} (see definition in Table~\ref{tab:terms}).


\begin{figure}[tb]
\centering
\includegraphics[width=0.475\textwidth]{figures/2.pdf}
\caption{Graph-based backdoor synthesis corrupts circuits.}
\label{fig:TJ_badgnn}
\end{figure}

\noindent\textbf{Motivational Example.} To illustrate this issue, we take the c17 ISCAS-85 benchmark as an example (Fig.~\ref{fig:TJ_badgnn}~\Circled{\scriptsize\textbf{1}}), convert it to a graph (Fig.~\ref{fig:TJ_badgnn}~\Circled{\scriptsize\textbf{2}}), and run the Erdős-Rényi (ER) model utilized in one of the state-of-the-art backdoor attacks on GNNs~\cite{zhang2021backdoor} to modify the graph, \textit{i.e.,} build a backdoor trigger with a given size and density. 
In particular, ER alters the graph's connectivity to achieve the required density. The generated backdoor trigger is depicted in Fig.~\ref{fig:TJ_badgnn}~\Circled{\scriptsize\textbf{3}}.
Such a graph structure is valid for social network graphs with no restrictions on the connection between nodes; however, such a graph represents a corrupted Boolean circuit. 
The generated backdoor trigger must meet circuit design rules. 
For example, we cannot have multiple drivers for any net in circuit design. 


\noindent\textbf{Research Challenges.} Here, we discuss the research challenges of deploying a backdoor attack against GNNs that process circuits and define the important terms in Table~\ref{tab:terms}.


\begin{enumerate}[leftmargin=*]

\item \textbf{Manipulation of the Circuit, not the Graph.} Boolean circuits in register-transfer-level (RTL) or gate-level logic are inputs to the GNN-based hardware security platform, as illustrated in Fig.~\ref{fig:gnn4tj}. 
The backdoor triggers must be injected into the circuits to poison the training dataset, and thus, the backdoor triggers would initially be in the form of \textit{sub-circuits} and later translated to subgraphs. 
As a result, direct graph-based backdoor trigger generation methods are not applicable. \textit{Thus, a technique to inject a crafted sub-circuit without affecting the functionality of a given design is required.}
\item \textbf{Selection of Nodes.} Existing backdoor attacks against GNNs
randomly select nodes in the graph and replace their connections as the backdoor trigger~\cite{zhang2021backdoor}. \blue{As demonstrated in Fig.~\ref{fig:TJ_badgnn}, performing randomized perturbations on circuits violates circuit design rules and results in corrupted circuits that cannot pass the EDA flow. For a circuit to operate electrically and to be manufactured without errors, it must be designed according to a specific set of rules. Furthermore, the selection procedure of nets (to insert the backdoor trigger) impacts the evasiveness of the backdoor trigger to possible detection. Lastly, the job of the backdoor trigger is to stamp the malicious design without affecting its functionality. Therefore, there are three requirements that restrict us from randomized trigger generation, as follows: (i)~maintaining the expected functionality, (ii)~meeting circuit design rules, and (iii)~ensuring the evasiveness of the backdoor trigger to possible detection.}
\end{enumerate}



\subsection{Our Research Contributions}
\label{sec:contribtutions}

This work aims to shed light on the vulnerabilities of GNNs, considering two case studies: hiding (i)~HTs and (ii)~IP piracy, being two of the most fundamental silicon security vulnerabilities.
To the best of our knowledge, we are the first to design, develop, and evaluate a backdoor attack on GNNs in the context of hardware security frameworks. In summary, our primary contributions are as follows.

\begin{enumerate}[leftmargin=*]

\item We develop a \textbf{backdoor attack (Section~\ref{sec:Proposed_attack})} on GNNs processing Boolean circuits (\textit{{\poisonedgnn}}), which has no restrictions on the targeted GNN architecture.


\item We implement a \textbf{sub-circuit backdoor trigger generation (Section~\ref{sec:trigger_design})} platform. 
We design stealthy backdoor triggers at the RTL or the gate level in the form of a sub-circuit without tampering the design functionality.

\item We develop a \textbf{backdoor trigger injection (Section~\ref{sec:trigger_injection})} platform. 
We automatically identify suitable nets for backdoor trigger insertion, taking into consideration the size of the original design and the functional coverage of the nets. 
This procedure applies to both RTL and gate-level designs.


\end{enumerate}

\noindent\textbf{Key Results.} We demonstrate the effectiveness of {\poisonedgnn} on GNN-based hardware security frameworks for: (i) HT detection (GNN4TJ)~\cite{yasaei2021gnn4tj}, and (ii) IP piracy detection (GNN4IP)~\cite{yasaei2021gnn4ip}.
We evaluate the efficacy of {\poisonedgnn} in hiding HTs on three datasets: (i) the advanced encryption standard (AES), (ii) the recommended standard $232$ for serial communication transmission of data (RS232), and the (iii) embedded peripheral interface controller (PIC) microcontroller, and consider HT designs from TrustHub~\cite{salmani2013design}. 
We further evaluate the efficacy of {\poisonedgnn} in hiding IP piracy considering selected ISCAS-85 designs obfuscated using different logic locking techniques from TrustHub.

Our experimental results showcase that designs with HTs can bypass detection by the GNN4TJ platform~\cite{yasaei2021gnn4tj} when exposed to our crafted backdoor triggers. 
Up to $100\%$ of the poisoned HT-infected testing samples are misclassified as HT-free. 
\textit{Please note that the goal of {\poisonedgnn} is not to design new HTs but to manipulate the HT detection model to prevent the detection of HT designs.}
Additionally, our experimental evaluation further demonstrates that up to $100\%$ of the pirated designs can bypass detection by the state-of-the-art GNN4IP tool once our backdoor triggers are injected.

\noindent\textbf{Open-Source Release.} The source code of {\poisonedgnn} and associated datasets will be released post peer-review.

\noindent\textbf{Paper Organization.} Section~\ref{sec:background} introduces fundamental concepts and surveys relevant literature. Section~\ref{sec:Proposed_attack} presents {\poisonedgnn} as the first backdoor attack on GNNs securing digital circuits. Section~\ref{sec:results} conducts an extensive experimental evaluation of {\poisonedgnn}. Section~\ref{sec:disscus} presents discussion regarding possible countermeasures. 
We provide concluding remarks in Section~\ref{sec:conclusion}.



