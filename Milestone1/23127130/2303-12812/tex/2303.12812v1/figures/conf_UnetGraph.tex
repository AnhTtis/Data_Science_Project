%\begin{tikzpicture}[scale=0.8,every node/.style={scale=0.8}]
\begin{tikzpicture}[scale=0.4]
    \begin{axis}[%colorbar/width=2.5mm,
        width=10cm,
        height=10cm,
%        xlabel={\LARGE Predicted class},
%        ylabel={\LARGE Actual class},
%        colormap={blackwhite}{gray(0cm)=(1); gray(1cm)=(0.5)},
%	colormap={bluewhite}{color=(white) color=(blue)},
%	colormap={bluewhite}{color=(white) rgb255=(0,191,255)},
	colormap={bluewhite}{color=(white) rgb255=(100,149,237)},
        xticklabels={
\texttt{AdDisplay},
\texttt{Adware},
\texttt{Benign},
\texttt{Downloader},
\texttt{Trojan}
        },
        xtick={0,...,4},
        xtick style={draw=none},
%        ytick=\empty,
	xticklabel style={anchor=east,rotate=30,yshift=-5pt,font=\tt\large},
%	xticklabel style={font=\tt},
        yticklabels={
\texttt{AdDisplay},
\texttt{Adware},
\texttt{Benign},
\texttt{Downloader},
\texttt{Trojan}
        },
        ytick={0,...,4},
        ytick style={draw=none},
        enlargelimits=false,
%        xlabel style={font=\footnotesize},
%        ylabel style={font=\footnotesize},
%        legend style={font=\footnotesize},
%        xticklabel style={font=\footnotesize\tt},
        yticklabel style={font=\tt\large},
        colorbar,
        colorbar style={
%     	  	width=0.05*\pgfkeysvalueof{/pgfplots/parent axis width},%%% added this
%     	  	height=0.5*\pgfkeysvalueof{/pgfplots/parent axis height},
%		plot graphics/node/.style={scale=1.33,anchor=south west,inner sep=0pt,}, %%% scale colorbar fill %%%
            ytick={0.0,0.2,0.4,0.6,0.8,1.0},
            yticklabels={0.0,0.2,0.4,0.6,0.8,1.0},
%            ytick={0,20,40,60,80,100},
%            yticklabels={0,20,40,60,80,100},
%            yticklabel={\pgfmathprintnumber\tick\,\%},
            yticklabel={\pgfmathprintnumber\tick},
            yticklabel style={%font=\footnotesize,
            		scale=1.33,
            		/pgf/number format/fixed,
			/pgf/number format/precision=1}
        },
%        point meta min=0,
%        point meta max=100,
        point meta min=0.0,
        point meta max=1.0,
%        nodes near coords={\pgfmathprintnumber\pgfplotspointmeta\,\%},
        nodes near coords={\pgfmathprintnumber\pgfplotspointmeta},
        % ---------------------------------------------------------------------
        % show `nodes near coords' but adapt the style so that values
        % above a threshold get another style
        % (adapted from <http://tex.stackexchange.com/a/141006/95441>)
        % #1: the THRESHOLD after which we switch to a special display.
        nodes near coords black white/.style={
            % define the style of the nodes with "small" values
            small value/.style={
%                font=\footnotesize,
%                font=\scriptsize,
                yshift=-7pt,
%                text=white,
                text=black,
                /pgf/number format/fixed,
                /pgf/number format/precision=3,
%                /pgf/number format/precision=2,
                /pgf/number format/zerofill=true,
                scale=1.2,
%                /pgf/number format/precision=0
            },
            % define the style of the nodes with "large" values
            large value/.style={
%                font=\footnotesize,
%                font=\scriptsize,
                yshift=-7pt,
%                text=black,
                text=white,
                /pgf/number format/fixed,
%                /pgf/number format/precision=2,
                /pgf/number format/precision=3,
                /pgf/number format/zerofill=true,
                scale=1.2,
%                /pgf/number format/precision=0
            },
            every node near coord/.style={
                check for zero/.code={
                    \pgfmathfloatifflags{\pgfplotspointmeta}{0}{
                        % If meta=0, make the node a coordinate
                        % (which doesn't have text)
                        \pgfkeys{/tikz/coordinate}
                    }{
                        \begingroup
                        % this group is merely to switch to FPU locally.
                        % Might be unnecessary, but who knows.
                        \pgfkeys{/pgf/fpu}
                        \pgfmathparse{\pgfplotspointmeta<#1}
                        \global\let\result=\pgfmathresult
                        \endgroup
                        %
                        % simplifies debugging:
                        %\show\result
                        %
                        \pgfmathfloatcreate{1}{1.0}{0}
                        \let\ONE=\pgfmathresult
                        \ifx\result\ONE
                            % AH: our condition 'y < #1' is met.
                            \pgfkeysalso{/pgfplots/small value}
                        \else
                            % ok, proceed as usual.
                            \pgfkeysalso{/pgfplots/large value}
                        \fi
                    }
                },
                check for zero,
            },
        },
        % asign a value to the new style which is the threshold at which
        % the two style `small value' or `large value' are used
%        nodes near coords black white=50,
%        nodes near coords black white=0.5,
        nodes near coords black white=0.5,
        % -----------------------------------------------------------------
    ]
        \addplot[
            matrix plot,
            mesh/cols=5,
            point meta=explicit,draw=gray
        ] table [meta=C] {
            x y C
0 0 0.98
1 0 0.00
2 0 0.02
3 0 0.00
4 0 0.00
0 1 0.00
1 1 0.94
2 1 0.04
3 1 0.00
4 1 0.02
0 2 0.00
1 2 0.04
2 2 0.94
3 2 0.00
4 2 0.02
0 3 0.00
1 3 0.00
2 3 0.00
3 3 0.99
4 3 0.01
0 4 0.00
1 4 0.01
2 4 0.03
3 4 0.02
4 4 0.94
         };
    \end{axis}
%\draw[black,thick] (2.625,5.8) circle(0.5);
%\draw[red,dashed,thick] (3.675,5.8) circle(0.5);
\end{tikzpicture}
