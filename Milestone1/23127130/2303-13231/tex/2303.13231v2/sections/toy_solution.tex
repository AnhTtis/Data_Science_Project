The following example demonstrates the main problem studied in the paper and gives intuition for our main results.
 Consider a game among friends. \playA, \playB and \playC play against \playboss.
\begin{enumerate}
    \item \playA, \playB, and \playC secretly agree on a list of four integers $g_1, \dots, g_8$ and put them in separate envelopes.
    \item Two players between \playA, \playB, and \playC are designated as liars, without \playboss knowing which is which. The remaining player has to be truthful.
    \item \playboss's goal is to find the sum of the numbers $g_1 + \dots + g_8$. The game is played in rounds. In each round \playbosspronoun can first \begin{enumerate*}
        \item ask the other players questions about $g_1, \dots, g_8$ and then
        \item look in any number of envelopes.
    \end{enumerate*}
    \item \playboss needs to find the correct sum every time. 
\end{enumerate}
What is the minimum number of envelopes \playbosspronoun needs to look inside?
If \playbosspronoun checks the minimum number of envelopes, how many questions does \playbosspronoun need to ask? 

If the number of questions asked were unlimited, \playbosspronoun could ask each player for all values $g_1, \dots, g_8$. With $\exwpgrad{\gind}{\wind}$ we denote the value the value player $\wind\in \{\text{\playA}, \text{\playB}, \text{\playC}\}$ claims is in the envelope for $g_\gind$, $\gind \in [8]$.
Since the players only answer questions about these numbers, and assuming the liars are smart enough to avoid contradictions so as to not be detected, \playboss cannot gain any more information by asking more questions.

Not in every case can \playboss identify the correct sum based just on these answers.
For example, \cref{tab:example_key_error_pattern} shows three cases that are indistinguishable based on the values of all $\exwpgrad{\gind}{\wind}$.
\begin{table}[h]
    \begin{minipage}{0.45\columnwidth}
    \begin{tabularx}{\columnwidth}{l*{4}{X}}
        & $\exwpgrad{1}{\wind}$ & $\exwpgrad{2}{\wind}$ & \exwpgrad{3}{\wind} & \dots \\
        \toprule
        \playA & $1$ & $3$ & $4$ & \dots \\
        \playB & $2$ & $7$ & $4$ & \dots \\
        \playC & $2$ & $3$ & $4$ & \dots \\
    \end{tabularx}
    \end{minipage}%
    \hfill
    \begin{minipage}{0.49\columnwidth}
        \begin{align*}
            \text{\emph{Case 1:  }} 
                            & g_1=2;\ g_2=3 \\
                            & \text{\playA and \playB lie} \\
                \text{\emph{Case 2:  }} 
                            & g_1=1;\ g_2=3 \\
                            & \text{\playB and \playC lie} \\
                \text{\emph{Case 3:  }}
                            & g_1=2;\ g_2=7 \\
                            & \text{\playA and \playC lie}
        \end{align*}
    \end{minipage}
    \caption{Indistinguishable Cases.}
    \label{tab:example_key_error_pattern}
\end{table}

\playboss picks any $g_i$, $i\in[8]$ on which at least two players' answers disagree and checks the corresponding envelope. \playbossPronoun is guaranteed to identify at least one liar. After eliminating the identified liar(s) \playbosspronoun repeats the process.
After opening at most two envelopes it is guaranteed that all non-eliminated players, including the honest one, agree on the sum $g_1 + \dots + g_8$ and it is therefore correct.
It can be verified, that the cases in \cref{tab:example_key_error_pattern} cannot be distinguished after opening any one envelope. Thus, using this strategy \playboss opens the smallest possible number of envelopes in the worst case.

Now, \playboss additionally wishes to minimize the number of questions asked.
With the above strategy, \playboss asks each player eight questions, for a total of $24$.

\playboss realizes that using multiple rounds, \playbosspronoun can reduce the number of questions \playbosspronoun needs to ask using the following recursive procedure.
First, \playbosspronoun asks every player for the desired sum $g_1 + \dots + g_8$ and the sum of the first half of the values, $g_1 + g_2 + g_3 + g_4$.
\playboss picks any two players, whose values for the total sum differ.
\playbossPronoun infers their values for $g_5 + \dots + g_8$ as $(g_1 + \dots + g_8) - (g_1 + \dots + g_4)$.
If the two players agree on $g_1 + \dots + g_4$, they are guaranteed to disagree on $g_5 + \dots + g_8$.
Assume they disagree on $g_5 + \dots + g_8$, then \playboss requests $g_5 + g_6$ from both players.
Again, they are guaranteed to disagree either on $g_5 + g_6$ or $g_7 + g_8$.
Assume they disagree on $g_5+g_6$. \playboss asks for $g_5$.
After having obtained a single integer, either $g_5$ or $g_6$ on which the two players disagree, \playboss opens the corresponding envelope and thus identifies and eliminates at least one liar.
In the worst case \playboss has to repeat the above procedure once more on the remaining two values.
Now, \playbosspronoun asks only $15$ questions as opposed to the $24$ from before. 





