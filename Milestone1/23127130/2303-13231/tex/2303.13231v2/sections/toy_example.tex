Consider a game among three friends \playA (A), \playB (B) and \playboss (D). A and B have $\ngrad$ private numbers $g_1,\dots, g_{\ngrad}$ whose sum should be communicated correctly to \RevisionRemove{\playboss}\Revision{D}. The problem is that one player is trying to cheat \RevisionRemove{on} D.

At the first stage, each player sends the sum of all numbers to \playbossletter. Then, the game is played in rounds. At each round, \playbossletter can first ask A and B questions about $g_1,\dots,g_{\ngrad}$. Only one player is guaranteed to reply truthfully. Then, \playbossletter can query an oracle to uncover the true value of some of the $g_i$'s. The game is repeated until \playbossletter correctly obtains the desired sum.
 
\begin{example}
  Assume w.l.o.g that A is cheating \RevisionRemove{on} \playbossletter, $\ngrad=4$, and that $g_i = i$. In a first stage, A and B send the sum of their numbers to \playbossletter. Assume that A sends the value $2$ and B sends the correct value $10$. At the first round, \playbossletter asks A and B to send the sum $g_1+g_2$. To create confusion, A acts truthfully and also sends the value $3$. Now, \playbossletter knows that $g_3+g_4$ is either $-1$ or $8$. So, \playbossletter decides not to query the oracle yet. He instead moves to the second round and asks A and B to send the value of $g_3$. Again, both players send the value $3$. Hence, \playbossletter now knows that $g_4$ has to be $-4$ if A is acting honestly or $5$ if B is acting honestly. \RevisionRemove{He then decides to query the oracle, obtain $g_4$, catch the liar A, and obtain the true sum.}
  \Revision{At this point, \playbossletter decides to query the oracle for $g_4$, thus catching the liar A and obtaining the true sum $10$.}
 
 

\label{ex:toy_problem}
\end{example}

This game is the crux of our framework. The private numbers are the \Revision{partial} gradients computed at the workers, querying the oracle represents local computations at the \master, and asking questions is the \RevisionRemove{introduced} light communication between the \master and the workers. The figures of merit of this game are: \begin{enumerate*}[label={\emph{\roman*)}}]
        \item the minimum number of queries to the oracle that \playbossletter needs; and 
        \item given that \playbossletter can only obtain the minimum number of oracle queries and can play several rounds, how many questions does \playbossletter need to ask the other players to recover the desired sum correctly.
\end{enumerate*}
For a more elaborate example see~\ARXIVonly{\cref{app:biggerexample}}\ISITonly{\cite[Appendix A]{hofmeisterBGC}}.
