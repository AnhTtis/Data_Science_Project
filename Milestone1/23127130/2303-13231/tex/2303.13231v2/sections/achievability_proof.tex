In order to show the correctness of our scheme, we recall the following facts from before. The elimination tournament runs as long as there are contradicting responses among the non-eliminated workers. In each iteration of the elimination tournament, the main node eliminates at least one malicious worker (either by majority vote or by local computation). As soon as less than $\nhonest$ malicious workers are left, the elimination terminates. Therefore, the elimination tournament is guaranteed to terminate after at most $\nmalicious+1-\nhonest$ iterations. W.l.o.g. we consider $1 \leq \nhonest \leq \nmalicious+1$ here. For $\nhonest \geq \nmalicious+1$, the elimination tournament will terminate immediately for the reason explained before. Furthermore, honest workers are never eliminated, since they always respond with a correct value and never commit to an incorrect value. Having at least $\nhonest \leq 1$ honest workers in each fractional repetition group, the \master can recover the correct group result after the elimination tournament is terminated, and hence, for the output it always holds that $\gestim=\tgrad$. After having shown that our scheme is an \BGC scheme, we show the tuple $\big( \nround,\localcomp,\replfact,\commoh \big)$ in the remainder.
    
We start with the number of locally computed gradients $\localcomp$.
As explained above, each local computation of a partial gradient eliminates at least $\nhonest$ malicious workers in our scheme. The procedure halts when there are no more discrepancies among the initial responses of the non-eliminated workers, which is at latest when all $\nmalicious$ malicious workers are identified.
Therefore, the number of gradients computed locally is
\begin{equation*}
    \localcomp \leq \compbound.
\end{equation*}

Next, we analyze the communication rate $\commoh$. Every match causes each of the two competing workers to send up to $\left\lceil \log_2\left(\frac{\ngrad}{\ngroup}\right) \right\rceil$ symbols. The reason is that a competing worker sends one symbol for each node on a specific path of the match tree. The number of nodes in such a path is upper bounded by the height of the tree, which is $\left\lceil \log_2\left(\frac{\ngrad}{\ngroup}\right) \right\rceil$.
We have at most $\nmalicious+1-\nhonest$ matches. %
That is, the total amount of symbols transmitted for all matches is at most $2 \left(\nmalicious+1-\nhonest\right) \left\lceil \log_2\left(\frac{\ngrad}{\ngroup}\right) \right\rceil$.
Finally, each committing round after a match causes an additional communication load. Every worker of the competing consistent worker subsets transmits one bit, indicating whether or not the worker commits to the response of its representative. The communication load caused by committing is maximized by; first, maximizing the number of matches; and second, maximizing the number of involved nodes per match. This is achieved at the same time in the case where there are only two large subsets of consistent workers: the first consisting of the $\nhonest$ honest workers, and the second consisting of the $\nmalicious$ malicious workers. Furthermore, the malicious workers never commit to a representative's malicious value. Hence, we have $\nmalicious+1-\nhonest$ matches. The first match $\nmalicious+\nhonest$ has one-bit voting messages. Since at least one malicious worker can be eliminated, the number of voting messages will decrease by at least one per match accordingly. Therefore, the number of voting messages in the $l$-th match is given by $\nhonest+\nmalicious-(l-1)$. Summing over all $l=1,\dots,\nmalicious+1-\nhonest$, we can bound the total number of one-bit voting messages by $\frac{\left( \nmalicious+1-\nhonest \right)\left(\nmalicious+3\nhonest\right)}{2}$.
In terms of symbols from $\galpha$, this is a load of $\frac{\left( \nmalicious+1-\nhonest \right)\left(\nmalicious+3\nhonest\right)}{2\log_2{\card{\galpha}}}$.
In total, we obtain
\begin{equation*}
    \commoh \leq \left(\nmalicious+1-\nhonest\right) \left( 2 \left\lceil \log_2\left( \frac{\ngrad}{\ngroup}\right) \right\rceil+ \frac{\nmalicious+3\nhonest}{2\log_2{\card{\galpha}}} \right).
\end{equation*}
 
Finally, we analyze the number of communication rounds $\nround$. The number of communication rounds in a match is again upper bounded by the height of the tree, i.e., $\left\lceil \log_2\left(\frac{\ngrad}{\ngroup}\right) \right\rceil$.
Matches in different fractional repetition groups can be performed concurrently in the same rounds of the interactive protocol.
As explained before, there can be up to $\nmalicious$ matches. Again, if all malicious workers are in the same group, and if there are only two consistent worker subsets in this group of size $\nhonest$ and $\nmalicious$, respectively, then all $\nmalicious+1-\nhonest$ matches have to be executed sequentially. Resolving all conflicts, thus, requires
\begin{equation*}
    \nround \leq \left(\nmalicious+1-\nhonest\right) \left\lceil \log_2\left(\frac{\ngrad}{\ngroup}\right) \right\rceil.
\end{equation*}