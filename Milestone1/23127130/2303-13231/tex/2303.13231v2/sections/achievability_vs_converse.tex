For convenience, we reproduce our achievability and converse bound on $\commoh$ from \cref{thm:scheme} and \cref{thm:converse_T}, respectively.
For fractional repetition data allocation, with $\ngroup$ groups, $\nworker = \ngroup(\nmalicious+\nhonest)$ workers, $\nmalicious$ of which are malicious, our scheme requires data transmission of at most
\begin{align*}
  \commoh[achieve] = \left(\nmalicious+1-\nhonest\right) \left( 2 \left\lceil \log_2\left( \frac{\ngrad}{\ngroup}\right) \right\rceil+ \frac{\nmalicious+3\nhonest}{2\log_2{\card{\galpha}}} \right)  
\end{align*}
during the interactive protocol as measured in symbols from $\galpha$.
Our lower bound shows, that for any computation optimal \BGC, \commoh is at least
\begin{align*}
  \commoh[bound] = {\log_{\alphasize} \binom{\ngrad/\ngroup}{\lfloor \nmalicious / \nhonest \rfloor}}.%
\end{align*}

\begin{figure}[h]
    \centering
    \resizebox{0.98\linewidth}{!}{
    \begin{tikzpicture}
\begin{axis}[
  point meta max={nan}, 
  point meta min={nan}, 
  legend cell align={left}, 
  title={}, 
  title style={at={{(0.5,1)}}, anchor={south}, font={{\fontsize{14 pt}{18.2 pt}\selectfont}}, color={rgb,1:red,0.0;green,0.0;blue,0.0}, draw opacity={1.0}, rotate={0.0}}, 
  legend columns=3, 
  legend style={
            /tikz/column 2/.style={
                column sep=3pt,
            },
      font={{\fontsize{7 pt}{10.4 pt}\selectfont}},
      cells={anchor={center}},
      at={(-0.11,1.05)},
      anchor=south west
      }, 
  axis background/.style={fill={rgb,1:red,1.0;green,1.0;blue,1.0}, opacity={1.0}}, 
  anchor={north west}, 
  xshift={1.0mm}, 
  yshift={-1.0mm}, 
width={75mm}, 
height={50mm},
  xlabel={number of samples $\ngrad$}, 
  x tick style={color={rgb,1:red,0.0;green,0.0;blue,0.0}, opacity={1.0}}, 
  x tick label style={color={rgb,1:red,0.0;green,0.0;blue,0.0}, opacity={1.0}, rotate={0}}, 
  xmode={log}, 
  log basis x={10}, 
  xmajorgrids={true}, 
  xmin={100}, 
  xmax={1.3182567385564074e6}, 
  xtick={{100.0,1000.0,10000.0,100000.0,1.0e6}}, 
  xticklabels={{$10^{2}$,$10^{3}$,$10^{4}$,$10^{5}$,$10^{6}$}}, 
  xtick align={inside}, 
  xticklabel style={font={{\fontsize{8 pt}{10.4 pt}\selectfont}}, color={rgb,1:red,0.0;green,0.0;blue,0.0}, draw opacity={1.0}, rotate={0.0}}, 
  x grid style={color={rgb,1:red,0.0;green,0.0;blue,0.0}, draw opacity={0.1}, line width={0.5}, solid}, 
  axis x line*={left}, 
  x axis line style={color={rgb,1:red,0.0;green,0.0;blue,0.0}, draw opacity={1.0}, line width={1}, solid},  
  ylabel={communication overhead $\kappa$}, 
  y tick style={color={rgb,1:red,0.0;green,0.0;blue,0.0}, opacity={1.0}}, 
  y tick label style={color={rgb,1:red,0.0;green,0.0;blue,0.0}, opacity={1.0}, rotate={0}}, 
  ymode={log}, 
  log basis y={10}, 
  ymajorgrids={true}, 
  ymin={0.33891105976772384}, 
  ymax={443.60664700624926}, 
  ytick={{1.0,10.0,100.0}}, 
  yticklabels={{$10^{0}$,$10^{1}$,$10^{2}$}}, 
  ytick align={inside}, 
  yticklabel style={font={{\fontsize{8 pt}{10.4 pt}\selectfont}}, color={rgb,1:red,0.0;green,0.0;blue,0.0}, draw opacity={1.0}, rotate={0.0}}, 
  y grid style={color={rgb,1:red,0.0;green,0.0;blue,0.0}, draw opacity={0.1}, line width={0.5}, solid}, 
  axis y line*={left}, 
  y axis line style={color={rgb,1:red,0.0;green,0.0;blue,0.0}, draw opacity={1.0}, line width={1}, solid}, 
  colorbar={false}
  ]
  \addplot[color={rgb,1:red,0.0667;green,0.4392;blue,0.6667}, 
  name path={0e242e9c-d481-452f-a98b-d4d0f32d2b0d}, 
  draw opacity={1.0}, 
  line width={1},
  solid]
        table[row sep={\\}]
        {
            \\
            100.0  128.0625  \\
            126.0  128.0625  \\
            158.0  146.0625  \\
            200.0  146.0625  \\
            251.0  146.0625  \\
            316.0  164.0625  \\
            398.0  164.0625  \\
            501.0  164.0625  \\
            631.0  182.0625  \\
            794.0  182.0625  \\
            1000.0  182.0625  \\
            1259.0  200.0625  \\
            1585.0  200.0625  \\
            1995.0  200.0625  \\
            2512.0  218.0625  \\
            3162.0  218.0625  \\
            3981.0  218.0625  \\
            5012.0  236.0625  \\
            6310.0  236.0625  \\
            7943.0  236.0625  \\
            10000.0  254.0625  \\
            12589.0  254.0625  \\
            15849.0  254.0625  \\
            19953.0  272.0625  \\
            25119.0  272.0625  \\
            31623.0  272.0625  \\
            39811.0  290.0625  \\
            50119.0  290.0625  \\
            63096.0  290.0625  \\
            79433.0  308.0625  \\
            100000.0  308.0625  \\
            125893.0  308.0625  \\
            158489.0  326.0625  \\
            199526.0  326.0625  \\
            251189.0  326.0625  \\
            316228.0  344.0625  \\
            398107.0  344.0625  \\
            501187.0  344.0625  \\
            630957.0  362.0625  \\
            794328.0  362.0625  \\
            1.0e6  362.0625  \\
        }
        ;
    \addlegendentry {scheme $\nmalicious=9$}
    \addplot[color={rgb,1:red,0.6392;green,0.6745;blue,0.7255}, name path={5c6cc1ca-1ded-4112-8f2e-02f121141a9a}, draw opacity={1.0}, line width={1}, solid]
        table[row sep={\\}]
        {
            \\
            100.0  71.35416666666667  \\
            126.0  71.35416666666667  \\
            158.0  81.35416666666666  \\
            200.0  81.35416666666666  \\
            251.0  81.35416666666666  \\
            316.0  91.35416666666666  \\
            398.0  91.35416666666666  \\
            501.0  91.35416666666666  \\
            631.0  101.35416666666666  \\
            794.0  101.35416666666666  \\
            1000.0  101.35416666666666  \\
            1259.0  111.35416666666666  \\
            1585.0  111.35416666666666  \\
            1995.0  111.35416666666666  \\
            2512.0  121.35416666666666  \\
            3162.0  121.35416666666666  \\
            3981.0  121.35416666666666  \\
            5012.0  131.35416666666666  \\
            6310.0  131.35416666666666  \\
            7943.0  131.35416666666666  \\
            10000.0  141.35416666666666  \\
            12589.0  141.35416666666666  \\
            15849.0  141.35416666666666  \\
            19953.0  151.35416666666666  \\
            25119.0  151.35416666666666  \\
            31623.0  151.35416666666666  \\
            39811.0  161.35416666666669  \\
            50119.0  161.35416666666669  \\
            63096.0  161.35416666666669  \\
            79433.0  171.35416666666669  \\
            100000.0  171.35416666666669  \\
            125893.0  171.35416666666669  \\
            158489.0  181.35416666666669  \\
            199526.0  181.35416666666669  \\
            251189.0  181.35416666666669  \\
            316228.0  191.35416666666669  \\
            398107.0  191.35416666666669  \\
            501187.0  191.35416666666669  \\
            630957.0  201.35416666666669  \\
            794328.0  201.35416666666669  \\
            1.0e6  201.35416666666669  \\
        }
        ;
    \addlegendentry {scheme $\nmalicious=7$}
    \addplot[color={rgb,1:red,0.9882;green,0.4902;blue,0.0431}, name path={dfcf77df-a2d1-4ea1-845c-33668df8d22a}, draw opacity={1.0}, line width={1}, solid]
        table[row sep={\\}]
        {
            \\
            100.0  14.3125  \\
            126.0  14.3125  \\
            158.0  16.3125  \\
            200.0  16.3125  \\
            251.0  16.3125  \\
            316.0  18.3125  \\
            398.0  18.3125  \\
            501.0  18.3125  \\
            631.0  20.3125  \\
            794.0  20.3125  \\
            1000.0  20.3125  \\
            1259.0  22.3125  \\
            1585.0  22.3125  \\
            1995.0  22.3125  \\
            2512.0  24.3125  \\
            3162.0  24.3125  \\
            3981.0  24.3125  \\
            5012.0  26.3125  \\
            6310.0  26.3125  \\
            7943.0  26.3125  \\
            10000.0  28.3125  \\
            12589.0  28.3125  \\
            15849.0  28.3125  \\
            19953.0  30.3125  \\
            25119.0  30.3125  \\
            31623.0  30.3125  \\
            39811.0  32.3125  \\
            50119.0  32.3125  \\
            63096.0  32.3125  \\
            79433.0  34.3125  \\
            100000.0  34.3125  \\
            125893.0  34.3125  \\
            158489.0  36.3125  \\
            199526.0  36.3125  \\
            251189.0  36.3125  \\
            316228.0  38.3125  \\
            398107.0  38.3125  \\
            501187.0  38.3125  \\
            630957.0  40.3125  \\
            794328.0  40.3125  \\
            1.0e6  40.3125  \\
        }
        ;
    \addlegendentry {scheme $\nmalicious=5$}
    \addplot[color={rgb,1:red,0.0667;green,0.4392;blue,0.6667}, name path={916ac44c-4455-46cd-b337-0e0654368a75}, draw opacity={1.0}, line width={1}, dashed]
        table[row sep={\\}]
        {
            \\
            100.0  2.5494268877638664  \\
            126.0  2.7440371241983943  \\
            158.0  2.9331325267517325  \\
            200.0  3.128883048004122  \\
            251.0  3.3165914111458363  \\
            316.0  3.5061901440890875  \\
            398.0  3.6955659086180153  \\
            501.0  3.8840381331444855  \\
            631.0  4.072603223780094  \\
            794.0  4.260136076116884  \\
            1000.0  4.448177546208562  \\
            1259.0  4.635755271258833  \\
            1585.0  4.8231527834983625  \\
            1995.0  5.010272100836417  \\
            2512.0  5.197609674659617  \\
            3162.0  5.384626358943232  \\
            3981.0  5.571752991561537  \\
            5012.0  5.758814679803512  \\
            6310.0  5.945840686495122  \\
            7943.0  6.13272128263218  \\
            10000.0  6.319692701362235  \\
            12589.0  6.50660156695857  \\
            15849.0  6.693532947575442  \\
            19953.0  6.880445384325474  \\
            25119.0  7.067326362661656  \\
            31623.0  7.254212751333456  \\
            39811.0  7.441092356745352  \\
            50119.0  7.627966295589302  \\
            63096.0  7.814837009150257  \\
            79433.0  8.00170443437531  \\
            100000.0  8.18856949157318  \\
            125893.0  8.375437581178968  \\
            158489.0  8.562296747476802  \\
            199526.0  8.749160108488104  \\
            251189.0  8.936024004413222  \\
            316228.0  9.122884565235324  \\
            398107.0  9.309744183688146  \\
            501187.0  9.496604285418385  \\
            630957.0  9.683464008081398  \\
            794328.0  9.870323724796744  \\
            1.0e6  10.05718326043246  \\
        }
        ;
    \addlegendentry {bound $\nmalicious=9$}
    \addplot[color={rgb,1:red,0.6392;green,0.6745;blue,0.7255}, name path={18869fbd-03e6-4fe7-a510-983c69edb319}, draw opacity={1.0}, line width={1}, dashed]
        table[row sep={\\}]
        {
            \\
            100.0  0.7670758006158959  \\
            126.0  0.8089415130101252  \\
            158.0  0.8499000935667956  \\
            200.0  0.8925300506448983  \\
            251.0  0.9335829899133037  \\
            316.0  0.9751867979102739  \\
            398.0  1.016851238605413  \\
            501.0  1.058403192366081  \\
            631.0  1.1000440133356142  \\
            794.0  1.1415107658029335  \\
            1000.0  1.1831328220284065  \\
            1259.0  1.2246861734123942  \\
            1585.0  1.2662264840691486  \\
            1995.0  1.307726420344244  \\
            2512.0  1.3492916913851953  \\
            3162.0  1.3907991857664292  \\
            3981.0  1.4323417419701476  \\
            5012.0  1.4738783451094648  \\
            6310.0  1.515413745332678  \\
            7943.0  1.5569221833427682  \\
            10000.0  1.5984550301488034  \\
            12589.0  1.6399773456751503  \\
            15849.0  1.681507335010188  \\
            19953.0  1.7230352378837055  \\
            25119.0  1.7645578365912284  \\
            31623.0  1.8060829759536532  \\
            39811.0  1.8476076713134992  \\
            50119.0  1.889131952033178  \\
            63096.0  1.9306561868386218  \\
            79433.0  1.9721802237317847  \\
            100000.0  2.0137041576156927  \\
            125893.0  2.055229101501116  \\
            158489.0  2.0967523294952826  \\
            199526.0  2.138276701715617  \\
            251189.0  2.179801361275636  \\
            316228.0  2.2213254135319436  \\
            398107.0  2.2628493626695687  \\
            501187.0  2.3043735036336517  \\
            630957.0  2.345897627428272  \\
            794328.0  2.387421803174363  \\
            1.0e6  2.4289459809970366  \\
        }
        ;
    \addlegendentry {bound $\nmalicious=7$}
    \addplot[color={rgb,1:red,0.9882;green,0.4902;blue,0.0431}, name path={a9e21dfe-363f-47d4-9a89-d46c55fb56ce}, draw opacity={1.0}, line width={1}, dashed]
        table[row sep={\\}]
        {
            \\
            100.0  0.4152410118609203  \\
            126.0  0.4360799952187448  \\
            158.0  0.45648629676106894  \\
            200.0  0.4777410118609203  \\
            251.0  0.49822147212192325  \\
            316.0  0.5189862967610689  \\
            398.0  0.5397890387839781  \\
            501.0  0.5605416745747005  \\
            631.0  0.5813435121864093  \\
            794.0  0.6020621998214348  \\
            1000.0  0.6228615177913804  \\
            1259.0  0.6436289104824385  \\
            1585.0  0.6643916953141731  \\
            1995.0  0.6851358144443568  \\
            2512.0  0.7059137968057266  \\
            3162.0  0.7266638532723981  \\
            3981.0  0.7474321972557093  \\
            5012.0  0.7681981687076163  \\
            6310.0  0.7889640181168694  \\
            7943.0  0.8097167679974888  \\
            10000.0  0.8304820237218405  \\
            12589.0  0.8512422542189988  \\
            15849.0  0.872006512204595  \\
            19953.0  0.8927698785193512  \\
            25119.0  0.9135307131567737  \\
            31623.0  0.9342929136775154  \\
            39811.0  0.9550549681273297  \\
            50119.0  0.9758168755690503  \\
            63096.0  0.9965788079587001  \\
            79433.0  1.0173406794449065  \\
            100000.0  1.0381025296523008  \\
            125893.0  1.0588649088673643  \\
            158489.0  1.079626449211309  \\
            199526.0  1.100388576814992  \\
            251189.0  1.1211508601214328  \\
            316228.0  1.1419128493349135  \\
            398107.0  1.1626747945814155  \\
            501187.0  1.1834368417718029  \\
            630957.0  1.2041988851679049  \\
            794328.0  1.2249609583449041  \\
            1.0e6  1.2457230355827609  \\
        }
        ;
    \addlegendentry {bound $\nmalicious=5$}
\end{axis}
\end{tikzpicture}

    }
    \caption{Comparison of converse and achievability for $\commoh$ over the dataset size $\ngrad$. We consider a system of $\nworker=10$ workers, $\ngroup=1$ group and an alphabet size $|\galpha|=2^{16}$. As the percentage of malicious workers rises from \SI{50}{\percent} to \SI{90}{\percent} the communication overhead of the scheme increases. }
    \label{fig:achievability_vs_converse_datacolumns}
\end{figure}

The gap between our scheme and the bound is depicted in~\cref{fig:achievability_vs_converse_datacolumns}.
It shows that for any parameter $\nmalicious$, the scheme is a constant factor away from the bound.
Typically, in distributed gradient descent applications the number of parameters $\graddim$ and the number of samples $\ngrad$ are very large, whereas the number of workers $\nworker$ and as a consequence $\nmalicious$, $\nhonest$ and $\ngroup$ are small by comparison.
For large numbers of samples $\ngrad$, the ratio $\commoh[achieve]/\commoh[bound]$ tends to 
\begin{align}
  \label{eq:conv_limit}
\lim_{\ngrad \to \infty} \frac{\commoh[achieve]}{\commoh[bound]} &= 2 \log_2(|\galpha|) \frac{(\nmalicious - \nhonest + 1)}{\lfloor \nmalicious / \nhonest 
\rfloor}.
\end{align}

The convergence behavior can be observed in \cref{fig:convergence}.
\begin{figure}[h]
    \centering
    \resizebox{0.98\linewidth}{!}{
    \begin{tikzpicture}
\begin{axis}[
  legend cell align={left}, 
  legend columns={3}, 
  title style={at={{(0.1,1)}}, anchor={south}, font={{\fontsize{14 pt}{18.2 pt}\selectfont}}, color={rgb,1:red,0.0;green,0.0;blue,0.0}, draw opacity={1.0}, rotate={0.0}}, 
    legend style={
            /tikz/column 2/.style={
                column sep=3pt,
            },
      font={{\fontsize{7 pt}{10.4 pt}\selectfont}},
      cells={anchor={center}},
      at={(-0.11,1.05)},
      anchor=south west
      }, 
  axis background/.style={fill={rgb,1:red,1.0;green,1.0;blue,1.0}, opacity={1.0}}, 
  anchor={north west}, 
  xshift={1.0mm}, 
  yshift={-1.0mm}, 
  width={75mm}, 
  height={50mm}, 
  xlabel={number of samples $\ngrad$}, 
  x tick style={color={rgb,1:red,0.0;green,0.0;blue,0.0}, opacity={1.0}}, 
  x tick label style={color={rgb,1:red,0.0;green,0.0;blue,0.0}, opacity={1.0}, rotate={0}}, 
  xmode={log}, 
  log basis x={10}, 
  xmajorgrids={true}, 
  xmin={100}, 
  xmax={1e10}, 
  xticklabel style={font={{\fontsize{8 pt}{10.4 pt}\selectfont}}, color={rgb,1:red,0.0;green,0.0;blue,0.0}, draw opacity={1.0}, rotate={0.0}}, 
  x grid style={color={rgb,1:red,0.0;green,0.0;blue,0.0}, draw opacity={0.1}, line width={0.5}, solid}, 
  axis x line*={left}, 
  x axis line style={color={rgb,1:red,0.0;green,0.0;blue,0.0}, draw opacity={1.0}, line width={1}, solid}, 
  scaled y ticks={false}, 
  ylabel={$\commoh[achieve] / \commoh[bound]$}, 
  y tick style={color={rgb,1:red,0.0;green,0.0;blue,0.0}, opacity={1.0}}, 
  y tick label style={color={rgb,1:red,0.0;green,0.0;blue,0.0}, opacity={1.0}, rotate={0}}, 
  ylabel style={at={(0, 0.5)},
   font={{\fontsize{11 pt}{14.3 pt}\selectfont}}, color={rgb,1:red,0.0;green,0.0;blue,0.0}, draw opacity={1.0}, rotate={0.0}}, 
  ymode={log}, 
  log basis y={10}, 
  ymajorgrids={true}, 
  ytick align={inside}, 
  y grid style={color={rgb,1:red,0.0;green,0.0;blue,0.0}, draw opacity={0.1}, line width={0.5}, solid}, 
  axis y line*={left}, 
  y axis line style={color={rgb,1:red,0.0;green,0.0;blue,0.0}, draw opacity={1.0}, line width={1}, solid}, 
  colorbar={false}]
    
  \addplot[color={rgb,1:red,0.0667;green,0.4392;blue,0.6667}, 
  name path={1eeefe02-17e6-4c94-a00a-326681316fae}, 
  draw opacity={1.0}, 
  line width={1}, solid]
        table[row sep={\\}]
        {
            \\
            100.0  50.23187784464185  \\
            126.0  46.66937588805779  \\
            158.0  49.797443063970725  \\
            200.0  46.68199410430874  \\
            251.0  44.03994399465005  \\
            316.0  46.792242650212465  \\
            398.0  44.39441862406194  \\
            501.0  42.24018775716199  \\
            631.0  44.704207602874185  \\
            794.0  42.73631094102282  \\
            1000.0  40.92968369825578  \\
            1259.0  43.15639810418062  \\
            1585.0  41.47961074019498  \\
            1995.0  39.93046604526758  \\
            2512.0  41.954381658003314  \\
            3162.0  40.49723889157579  \\
            3981.0  39.13714415019067  \\
            5012.0  40.99150834422307  \\
            6310.0  39.70212329034855  \\
            7943.0  38.49229226648979  \\
            10000.0  40.20171739446062  \\
            12589.0  39.04688144578651  \\
            15849.0  37.95641285250228  \\
            19953.0  39.541408266940515  \\
            25119.0  38.495816669422545  \\
            31623.0  37.504069611135954  \\
            39811.0  38.981171862093376  \\
            50119.0  38.026190567690634  \\
            63096.0  37.116896956439504  \\
            79433.0  38.49960999266156  \\
            100000.0  37.62103995295219  \\
            125893.0  36.781660303011364  \\
            158489.0  38.08119592398925  \\
            199526.0  37.26786296705972  \\
            251189.0  36.48854343262372  \\
            316228.0  37.71422268249702  \\
            398107.0  36.957245356197994  \\
            501187.0  36.23005546606724  \\
            630957.0  37.38977081939256  \\
            794328.0  36.681927573500715  \\
            1.0e6  36.00038804348399  \\
            1.258925e6  37.10083340384379  \\
            1.584893e6  36.43620911979894  \\
            1.995262e6  35.79497895914034  \\
            2.511886e6  36.841882515756005  \\
            3.162278e6  36.21555769415743  \\
            3.981072e6  35.61017335620257  \\
            5.011872e6  36.60847880764624  \\
            6.309573e6  36.01632360927727  \\
            7.943282e6  35.443020129509705  \\
            1.0e7  36.397018661122004  \\
            1.2589254e7  35.83553091978193  \\
            1.5848932e7  35.291103835461016  \\
            1.9952623e7  36.20454620521198  \\
            2.5118864e7  35.67073279485513  \\
            3.1622777e7  35.15243210067665  \\
            3.9810717e7  36.028613528596004  \\
            5.0118723e7  35.51989633552601  \\
            6.3095734e7  35.0253450987719  \\
            7.9432823e7  35.867176452423976  \\
            1.0e8  35.38131934771668  \\
            1.25892541e8  34.90844919089674  \\
            1.58489319e8  35.71851428074481  \\
            1.99526231e8  35.25356557796952  \\
            2.51188643e8  34.80056582737158  \\
            3.16227766e8  35.58116836540663  \\
            3.98107171e8  35.13541465660039  \\
            5.01187234e8  34.70069138989681  \\
            6.30957344e8  35.453893970266584  \\
            7.94328235e8  35.025823089174395  \\
            1.0e9  34.6079659594997  \\
            1.258925412e9  35.33562245455053  \\
            1.584893192e9  34.92389326404258  \\
            1.995262315e9  34.5216484632405  \\
            2.511886432e9  35.22543120976006  \\
            3.16227766e9  34.82884884642847  \\
            3.981071706e9  34.4410968407726  \\
            5.011872336e9  35.122519523880754  \\
            6.309573445e9  34.74001496234031  \\
            7.943282347e9  34.365752036937266  \\
            1.0e10  35.0261890975617  \\
        }
        ;
    \addlegendentry {scheme $\nmalicious=9$}
    \addplot[color={rgb,1:red,0.6392;green,0.6745;blue,0.7255},  draw opacity={1.0}, line width={1}, solid]
        table[row sep={\\}]
        {
            \\
            100.0  93.02101123432054  \\
            126.0  88.206830183746  \\
            158.0  95.7220351926845  \\
            200.0  91.15005887799984  \\
            251.0  87.14186906321154  \\
            316.0  93.6786335319852  \\
            398.0  89.84024722432035  \\
            501.0  86.31320022990731  \\
            631.0  92.13646493955724  \\
            794.0  88.78949695702133  \\
            1000.0  85.66592421373394  \\
            1259.0  90.92465407394606  \\
            1585.0  87.94174507298143  \\
            1995.0  85.1509650140386  \\
            2512.0  89.93916396393381  \\
            3162.0  87.25498828919135  \\
            3981.0  84.72431062418615  \\
            5012.0  89.12144418331283  \\
            6310.0  86.67874834263863  \\
            7943.0  84.36784321785724  \\
            10000.0  88.43174440353678  \\
            12589.0  86.19275567399598  \\
            15849.0  84.06396078302579  \\
            19953.0  87.84159681642109  \\
            25119.0  85.77455696156296  \\
            31623.0  83.80244356533409  \\
            39811.0  87.33140112584451  \\
            50119.0  85.41180328510632  \\
            63096.0  83.57478030869814  \\
            79433.0  86.88565304768554  \\
            100000.0  85.09401245392321  \\
            125893.0  83.3747276843791  \\
            158489.0  86.49288908162137  \\
            199526.0  84.81323606115133  \\
            251189.0  83.1975655619083  \\
            316228.0  86.14413966588103  \\
            398107.0  84.56336945068185  \\
            501187.0  83.03956210437667  \\
            630957.0  85.8324610215001  \\
            794328.0  84.33958607521396  \\
            1.0e6  82.8977541048544  \\
            1.258925e6  85.5522079182146  \\
            1.584893e6  84.13799964535005  \\
            1.995262e6  82.76978730817473  \\
            2.511886e6  85.29886289954584  \\
            3.162278e6  83.9554628949764  \\
            3.981072e6  82.65372387547478  \\
            5.011872e6  85.06872773786297  \\
            6.309573e6  83.78939956634181  \\
            7.943282e6  82.5479801722578  \\
            1.0e7  84.85875239086057  \\
            1.2589254e7  83.63767265672266  \\
            1.5848932e7  82.45123590472062  \\
            1.9952623e7  84.66639913616976  \\
            2.5118864e7  83.49850377095927  \\
            3.1622777e7  82.36238995668099  \\
            3.9810717e7  84.48953848435559  \\
            5.0118723e7  83.37039542564209  \\
            6.3095734e7  82.28051295295855  \\
            7.9432823e7  84.32637126029363  \\
            1.0e8  83.25207945755008  \\
            1.25892541e8  82.2048156496943  \\
            1.58489319e8  84.17536632011632  \\
            1.99526231e8  83.14247447216985  \\
            2.51188643e8  82.13462397810892  \\
            3.16227766e8  84.03521272321055  \\
            3.98107171e8  83.04065254291943  \\
            5.01187234e8  82.06935833654543  \\
            6.30957344e8  83.90478135456816  \\
            7.94328235e8  82.94581312045894  \\
            1.0e9  82.00851775215477  \\
            1.258925412e9  83.78309448643056  \\
            1.584893192e9  82.85726175304116  \\
            1.995262315e9  81.95166693685219  \\
            2.511886432e9  83.66930126432923  \\
            3.16227766e9  82.7743931365518  \\
            3.981071706e9  81.89842589399396  \\
            5.011872336e9  83.5626577416165  \\
            6.309573445e9  82.69667724875488  \\
            7.943282347e9  81.84846139720672  \\
            1.0e10  83.46251060976827  \\
        }
        ;
    \addlegendentry {scheme $\nmalicious=7$}
    \addplot[color={rgb,1:red,0.9882;green,0.4902;blue,0.0431}, name path={44590de5-03d4-4a3a-937b-5ede02b3c2bf}, draw opacity={1.0}, line width={1}, solid]
        table[row sep={\\}]
        {
            \\
            100.0  34.467934503525846  \\
            126.0  32.820813054771335  \\
            158.0  35.73491716124434  \\
            200.0  34.14506938907913  \\
            251.0  32.74146321017665  \\
            316.0  35.28513202426752  \\
            398.0  33.92529059362506  \\
            501.0  32.669292633583815  \\
            631.0  34.94061527169283  \\
            794.0  33.73820845425019  \\
            1000.0  32.61158286359796  \\
            1259.0  34.666714991524294  \\
            1585.0  33.583351744709894  \\
            1995.0  32.56653575772473  \\
            2512.0  34.44117413487953  \\
            3162.0  33.45769834361939  \\
            3981.0  32.528034100305526  \\
            5012.0  34.25222953117296  \\
            6310.0  33.350697111388826  \\
            7943.0  32.49593072534914  \\
            10000.0  34.09164700894587  \\
            12589.0  33.260214539016594  \\
            15849.0  32.46822082603572  \\
            19953.0  33.95331846351373  \\
            25119.0  33.181697739808754  \\
            31623.0  32.444321856927615  \\
            39811.0  33.83313115826024  \\
            50119.0  33.11328263426156  \\
            63096.0  32.42342676961588  \\
            79433.0  33.72763980962798  \\
            100000.0  33.05309352390513  \\
            125893.0  32.40498359389683  \\
            158489.0  33.63431863541976  \\
            199526.0  32.99970643561598  \\
            251189.0  32.388593981069555  \\
            316228.0  33.55115937465317  \\
            398107.0  32.95203454874345  \\
            501187.0  32.37392875367965  \\
            630957.0  33.47661295532516  \\
            794328.0  32.909212106211044  \\
            1.0e6  32.36072453387798  \\
            1.258925e6  33.40939537878277  \\
            1.584893e6  32.87053367710583  \\
            1.995262e6  32.34877925115103  \\
            2.511886e6  33.34847959590082  \\
            3.162278e6  32.83542544512263  \\
            3.981072e6  32.33791907378289  \\
            5.011872e6  33.2930190682296  \\
            6.309573e6  32.80341585166549  \\
            7.943282e6  32.32800397112567  \\
            1.0e7  33.242312378322495  \\
            1.2589254e7  32.774110814310454  \\
            1.5848932e7  32.31891480298508  \\
            1.9952623e7  33.19577350926881  \\
            2.5118864e7  32.74718198485234  \\
            3.1622777e7  32.310552844367464  \\
            3.9810717e7  33.15290873562767  \\
            5.0118723e7  32.72235149007836  \\
            6.3095734e7  32.302834162867505  \\
            7.9432823e7  33.11329953386461  \\
            1.0e8  32.69938327899996  \\
            1.25892541e8  32.29568719160354  \\
            1.58489319e8  33.07658855067729  \\
            1.99526231e8  32.67807543717949  \\
            2.51188643e8  32.289050726389824  \\
            3.16227766e8  33.04246893591277  \\
            3.98107171e8  32.65825417890546  \\
            5.01187234e8  32.28287194759399  \\
            6.30957344e8  33.01067566211962  \\
            7.94328235e8  32.639769192227135  \\
            1.0e9  32.27710509063798  \\
            1.258925412e9  32.98097864556565  \\
            1.584893192e9  32.62248974794688  \\
            1.995262315e9  32.2717102878248  \\
            2.511886432e9  32.953177184621545  \\
            3.16227766e9  32.606301635683224  \\
            3.981071706e9  32.266652660062476  \\
            5.011872336e9  32.92709540208889  \\
            6.309573445e9  32.59110463255507  \\
            7.943282347e9  32.2619015555467  \\
            1.0e10  32.90257852607314  \\
        }
        ;
    \addlegendentry {scheme $\nmalicious=5$}
    \addplot[color={rgb,1:red,0.0667;green,0.4392;blue,0.6667}, name path={233b7f50-a717-44f1-9f9e-0bb573127f34}, draw opacity={1.0}, line width={1}, thick, dashed]
        table[row sep={\\}]
        {
            \\
            100.0  32.0  \\
            126.0  32.0  \\
            158.0  32.0  \\
            200.0  32.0  \\
            251.0  32.0  \\
            316.0  32.0  \\
            398.0  32.0  \\
            501.0  32.0  \\
            631.0  32.0  \\
            794.0  32.0  \\
            1000.0  32.0  \\
            1259.0  32.0  \\
            1585.0  32.0  \\
            1995.0  32.0  \\
            2512.0  32.0  \\
            3162.0  32.0  \\
            3981.0  32.0  \\
            5012.0  32.0  \\
            6310.0  32.0  \\
            7943.0  32.0  \\
            10000.0  32.0  \\
            12589.0  32.0  \\
            15849.0  32.0  \\
            19953.0  32.0  \\
            25119.0  32.0  \\
            31623.0  32.0  \\
            39811.0  32.0  \\
            50119.0  32.0  \\
            63096.0  32.0  \\
            79433.0  32.0  \\
            100000.0  32.0  \\
            125893.0  32.0  \\
            158489.0  32.0  \\
            199526.0  32.0  \\
            251189.0  32.0  \\
            316228.0  32.0  \\
            398107.0  32.0  \\
            501187.0  32.0  \\
            630957.0  32.0  \\
            794328.0  32.0  \\
            1.0e6  32.0  \\
            1.258925e6  32.0  \\
            1.584893e6  32.0  \\
            1.995262e6  32.0  \\
            2.511886e6  32.0  \\
            3.162278e6  32.0  \\
            3.981072e6  32.0  \\
            5.011872e6  32.0  \\
            6.309573e6  32.0  \\
            7.943282e6  32.0  \\
            1.0e7  32.0  \\
            1.2589254e7  32.0  \\
            1.5848932e7  32.0  \\
            1.9952623e7  32.0  \\
            2.5118864e7  32.0  \\
            3.1622777e7  32.0  \\
            3.9810717e7  32.0  \\
            5.0118723e7  32.0  \\
            6.3095734e7  32.0  \\
            7.9432823e7  32.0  \\
            1.0e8  32.0  \\
            1.25892541e8  32.0  \\
            1.58489319e8  32.0  \\
            1.99526231e8  32.0  \\
            2.51188643e8  32.0  \\
            3.16227766e8  32.0  \\
            3.98107171e8  32.0  \\
            5.01187234e8  32.0  \\
            6.30957344e8  32.0  \\
            7.94328235e8  32.0  \\
            1.0e9  32.0  \\
            1.258925412e9  32.0  \\
            1.584893192e9  32.0  \\
            1.995262315e9  32.0  \\
            2.511886432e9  32.0  \\
            3.16227766e9  32.0  \\
            3.981071706e9  32.0  \\
            5.011872336e9  32.0  \\
            6.309573445e9  32.0  \\
            7.943282347e9  32.0  \\
            1.0e10  32.0  \\
        }
        ;
      \addlegendentry {limit $\nmalicious=9$}
    \addplot[color={rgb,1:red,0.6392;green,0.6745;blue,0.7255}, name path={f75f2af8-a259-464f-ac67-414608860373}, draw opacity={1.0}, line width={1}, thick, dashed]
        table[row sep={\\}]
        {
            \\
            100.0  80.0  \\
            126.0  80.0  \\
            158.0  80.0  \\
            200.0  80.0  \\
            251.0  80.0  \\
            316.0  80.0  \\
            398.0  80.0  \\
            501.0  80.0  \\
            631.0  80.0  \\
            794.0  80.0  \\
            1000.0  80.0  \\
            1259.0  80.0  \\
            1585.0  80.0  \\
            1995.0  80.0  \\
            2512.0  80.0  \\
            3162.0  80.0  \\
            3981.0  80.0  \\
            5012.0  80.0  \\
            6310.0  80.0  \\
            7943.0  80.0  \\
            10000.0  80.0  \\
            12589.0  80.0  \\
            15849.0  80.0  \\
            19953.0  80.0  \\
            25119.0  80.0  \\
            31623.0  80.0  \\
            39811.0  80.0  \\
            50119.0  80.0  \\
            63096.0  80.0  \\
            79433.0  80.0  \\
            100000.0  80.0  \\
            125893.0  80.0  \\
            158489.0  80.0  \\
            199526.0  80.0  \\
            251189.0  80.0  \\
            316228.0  80.0  \\
            398107.0  80.0  \\
            501187.0  80.0  \\
            630957.0  80.0  \\
            794328.0  80.0  \\
            1.0e6  80.0  \\
            1.258925e6  80.0  \\
            1.584893e6  80.0  \\
            1.995262e6  80.0  \\
            2.511886e6  80.0  \\
            3.162278e6  80.0  \\
            3.981072e6  80.0  \\
            5.011872e6  80.0  \\
            6.309573e6  80.0  \\
            7.943282e6  80.0  \\
            1.0e7  80.0  \\
            1.2589254e7  80.0  \\
            1.5848932e7  80.0  \\
            1.9952623e7  80.0  \\
            2.5118864e7  80.0  \\
            3.1622777e7  80.0  \\
            3.9810717e7  80.0  \\
            5.0118723e7  80.0  \\
            6.3095734e7  80.0  \\
            7.9432823e7  80.0  \\
            1.0e8  80.0  \\
            1.25892541e8  80.0  \\
            1.58489319e8  80.0  \\
            1.99526231e8  80.0  \\
            2.51188643e8  80.0  \\
            3.16227766e8  80.0  \\
            3.98107171e8  80.0  \\
            5.01187234e8  80.0  \\
            6.30957344e8  80.0  \\
            7.94328235e8  80.0  \\
            1.0e9  80.0  \\
            1.258925412e9  80.0  \\
            1.584893192e9  80.0  \\
            1.995262315e9  80.0  \\
            2.511886432e9  80.0  \\
            3.16227766e9  80.0  \\
            3.981071706e9  80.0  \\
            5.011872336e9  80.0  \\
            6.309573445e9  80.0  \\
            7.943282347e9  80.0  \\
            1.0e10  80.0  \\
        }
        ;
        \addlegendentry {limit $\nmalicious=7$}
    \addplot[color={rgb,1:red,0.9882;green,0.4902;blue,0.0431}, name path={6095e951-ac26-4dd8-b5f0-843e048439e3}, draw opacity={1.0}, line width={1}, thick, dashdotted]
        table[row sep={\\}]
        {
            \\
            100.0  32.0  \\
            126.0  32.0  \\
            158.0  32.0  \\
            200.0  32.0  \\
            251.0  32.0  \\
            316.0  32.0  \\
            398.0  32.0  \\
            501.0  32.0  \\
            631.0  32.0  \\
            794.0  32.0  \\
            1000.0  32.0  \\
            1259.0  32.0  \\
            1585.0  32.0  \\
            1995.0  32.0  \\
            2512.0  32.0  \\
            3162.0  32.0  \\
            3981.0  32.0  \\
            5012.0  32.0  \\
            6310.0  32.0  \\
            7943.0  32.0  \\
            10000.0  32.0  \\
            12589.0  32.0  \\
            15849.0  32.0  \\
            19953.0  32.0  \\
            25119.0  32.0  \\
            31623.0  32.0  \\
            39811.0  32.0  \\
            50119.0  32.0  \\
            63096.0  32.0  \\
            79433.0  32.0  \\
            100000.0  32.0  \\
            125893.0  32.0  \\
            158489.0  32.0  \\
            199526.0  32.0  \\
            251189.0  32.0  \\
            316228.0  32.0  \\
            398107.0  32.0  \\
            501187.0  32.0  \\
            630957.0  32.0  \\
            794328.0  32.0  \\
            1.0e6  32.0  \\
            1.258925e6  32.0  \\
            1.584893e6  32.0  \\
            1.995262e6  32.0  \\
            2.511886e6  32.0  \\
            3.162278e6  32.0  \\
            3.981072e6  32.0  \\
            5.011872e6  32.0  \\
            6.309573e6  32.0  \\
            7.943282e6  32.0  \\
            1.0e7  32.0  \\
            1.2589254e7  32.0  \\
            1.5848932e7  32.0  \\
            1.9952623e7  32.0  \\
            2.5118864e7  32.0  \\
            3.1622777e7  32.0  \\
            3.9810717e7  32.0  \\
            5.0118723e7  32.0  \\
            6.3095734e7  32.0  \\
            7.9432823e7  32.0  \\
            1.0e8  32.0  \\
            1.25892541e8  32.0  \\
            1.58489319e8  32.0  \\
            1.99526231e8  32.0  \\
            2.51188643e8  32.0  \\
            3.16227766e8  32.0  \\
            3.98107171e8  32.0  \\
            5.01187234e8  32.0  \\
            6.30957344e8  32.0  \\
            7.94328235e8  32.0  \\
            1.0e9  32.0  \\
            1.258925412e9  32.0  \\
            1.584893192e9  32.0  \\
            1.995262315e9  32.0  \\
            2.511886432e9  32.0  \\
            3.16227766e9  32.0  \\
            3.981071706e9  32.0  \\
            5.011872336e9  32.0  \\
            6.309573445e9  32.0  \\
            7.943282347e9  32.0  \\
            1.0e10  32.0  \\
        }
        ;
        \addlegendentry {limit $\nmalicious=5$}
\end{axis}
\end{tikzpicture}

    }
    \caption{Convergence of the ratio $\frac{\commoh[achieve]}{\commoh[bound]}$ to the limit given in $\eqref{eq:conv_limit}$ for large numbers of samples.
      The parameters are $\nworker=10$, $\ngroup=10$, $|\galpha|=2^{16}$.
    For $\nmalicious=5$ and $\nmalicious=9$ the limits as in \eqref{eq:conv_limit} yield the same value.}
    \label{fig:convergence}
\end{figure}
Note that depending on the alphabet the communication overhead of our scheme can be slightly improved as stated in the following.
\begin{remark}[Compression Beyond the Alphabet Size]
  If for every pair of elements $a, b \in \galpha$ there exists a
  function $f: \galpha \mapsto \mathcal{B}$, that maps from $\galpha$ to a
smaller alphabet $\mathcal{B}$, such that $f(a) \neq f(b)$ and an operation $\oplus$ with the property $\forall c, d, e, g \in \galpha: f(c+d) \neq f(e+g) \implies f(c) \neq f(d)\text{ or }f(e) \neq f(g)$, then 
our scheme can be improved to $\commoh \geq \left(\nmalicious+1-\nhonest\right) \left( 2 \lceil \log_{|\galpha|}(|\mathcal{B}|)\rceil \left\lceil \log_2\left( \frac{\ngrad}{\ngroup}\right) \right\rceil+ \frac{\nmalicious+3\nhonest}{2\log_2{\card{\galpha}}} \right)$.
At the start of each match, the main node chooses not only the index $\compind$ but also the appropriate function $f$ and communicates it to the two workers. They then use $f$ to compress their transmitted symbols during the match. 
\end{remark}



