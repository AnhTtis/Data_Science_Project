\documentclass[journal,twoside,web]{ieeecolor}
%\documentclass[journal,onecolumn,web]{ieeecolor}
\usepackage{generic}
\usepackage{cite}
\usepackage{amsmath,amssymb,amsfonts}
\usepackage{algorithmic}
\usepackage{comment,cuted,flushend}
\usepackage{graphicx,pgfplots}
\usepackage{textcomp}
\def\BibTeX{{\rm B\kern-.05em{\sc i\kern-.025em b}\kern-.08em
    T\kern-.1667em\lower.7ex\hbox{E}\kern-.125emX}}
\markboth{\journalname, VOL. XX, NO. XX, XXXX 2017}
{Mora \MakeLowercase{\textit{and}} Morris: Exponential Decay Rate of Linear Port-Hamiltonian Systems. A Multiplier Approach}



%% \documentclass[a4paper]{amsart}%[a4paper]
% %%%%% GENERAL MATH COMMANDS
% Reals
\newcommand{\R}{{\mathbb R}}
% Integers
\newcommand{\Z}{{\mathbb Z}}
% Naturals
\newcommand{\N}{{\mathbb N}}
% Expectation
\DeclareMathOperator*{\E}{\mathbb{E}}
% ^th notation
\newcommand{\tth}{^{\text{th}}}
% Small dots for integer range [a .. b]
\newcommand{\sdots}{\,..\,}
% Vectorized version of matrix
\newcommand{\matvec}{\mbox{vec}}

% := sign
\newcommand{\defeq}{\vcentcolon=}
% Zero function
\newcommand{\zf}{\mathbf{0}}
% Vector of ones
\newcommand{\ones}{\mathbf{1}}

% Argmin and argmax definitions
\DeclareMathOperator*{\argmax}{arg\,max}
\DeclareMathOperator*{\argmin}{arg\,min}


%%%%% PROBLEM STATEMENT NOTATION 
% \newcommandtwoopt{\St}[2][t][]{{S_{#1}^{#2}}} % State
\newcommand{\task}[1][i]{{\mathcal{T}_{#1}}} % Task, optionally takes index
\newcommand{\tasks}{\{ \task \}_{i=1}^N}
\newcommand{\losst}[1][i]{{l_{#1}}}
\newcommand{\lossv}[1][i]{{l_{#1}^{\textrm{val}}}}
\newcommand{\tasktarget}{{\mathcal{T}_{\textrm{target}}}}
\newcommand{\lossttarget}{l_{\textrm{target}}}
\newcommand{\lossvtarget}{l_{\textrm{target}}^{\textrm{val}}}
\newcommand{\lossttargetit}{l_{\textrm{target}}^{(k)}}
\newcommand{\losstotal}{l^{\textrm{total}}}
\newcommand{\lossopt}{l^*}

\newcommand{\thetait}[2]{\theta_{#1}^{(#2)}}
\newcommand{\phit}[1]{\phi^{(#1)}}
\newcommand{\hist}[2]{S_{#1}^{(#2)}}
\newcommand{\grad}[2]{G_{#1}^{(#2)}}

\newcommand{\Alg}{\textup{\textbf{Opt}}}
\newcommand{\MetaAlg}{\textup{\textbf{MetaOpt}}}

%%%%% Theorems
\newtheoremstyle{mytheoremstyle} % name
    {\topsep}                    % Space above
    {\topsep}                    % Space below
    {\itshape}                   % Body font
    {}                           % Indent amount
    {\scshape}                   % Theorem head font
    {.}                          % Punctuation after theorem head
    {.5em}                       % Space after theorem head
    {}  % Theorem head spec (can be left empty, meaning ‘normal’)
\theoremstyle{mytheoremstyle}
\theoremstyle{plain}
\newtheorem{theorem}{Theorem}
\newtheorem{proposition}{Proposition}
\newtheorem{assumption}{Assumption}
\newtheorem{definition}{Definition}
\newtheorem{lemma}{Lemma}
\theoremstyle{remark}
\newtheorem{remark}{Remark}

%
% \begin{document}
% \section{notation}\label{sec:notation}
For a positive integer $d$, we define $[d]:=\{1,2,\ldots,d\}$. 
The set of non-negative integers is denoted by $\NN:=\{0,1,2,\ldots\}$.
The cardinality of a set $S$ is denoted by $|S|$.
%Operations on $[d]$ cyclically.

Our \emph{graphs} are finite and undirected. We allow multiple edges and loops. A \emph{simple graph} is a graph without multiple edges or loops. 


A \emph{plane map} is a connected planar graph drawn in the plane without edge crossing, considered up to continuous deformation. 
The \emph{faces} of a plane map are the connected components of the complement of the graph. The infinite face is called \emph{outer face}, and the finite faces are called \emph{inner faces}. The vertices and edges incident to the outer face are called \emph{outer} while the other are called \emph{inner}. 
The numbers $\vv$, $\ee$ and $\ff$ of vertices, edges and faces of a plane map are related by the \emph{Euler relation}  $\vv+\ff=\ee+2$. 


We now define the class of plane maps which will be relevant for this article.
\begin{definition}\label{def:d-adapted}
A \emph{$d$-map} is a plane map such that the inner faces have degree at most $d$, and the outer face has degree $d$ and is incident to $d$ distinct vertices (in other words, the contour of the outer face is a simple cycle). 
We will assume that the outer vertices of a $d$-map are labeled $v_1,v_2,\ldots, v_d$ in clockwise order along the boundary of the outer face. %, as in Figure \ref{???}.\\
A \emph{$d$-adapted map} is a $d$-map such that any simple cycle which is not the contour of a face has length at least $d$.\\
\end{definition}
We point out that $d$-adapted maps are necessarily 2-connected (because a cut point in a $d$-map $G$ implies the existence of a simple cycle of length strictly less than the degree of an inner face of $G$, which shows that $G$ is not $d$-adapted).


In a plane map, a \emph{corner} is the sector delimited by two consecutive (half-)edges around a vertex. It is called an \emph{inner corner} if it lies in an inner face, and an \emph{outer corner} otherwise.
The \emph{degree} of a vertex or face is its number of incident corners. A  \emph{$d$-angulation} is a plane map with all faces of degree $d$. A \emph{$d$-angulation of the $k$-gon} is a plane map with inner faces of degree $d$, and outer face of degree $k$. 
A graph is \emph{bipartite} if it admits a bicoloring of its vertices such that adjacent vertices have different colors. It is known that a plane map is bipartite if and only if all its faces have even degree. For $k\geq 2$, a graph is called \emph{$k$-connected} if it is connected and the deletion of any subset of $(k-1)$ vertices does not disconnect it (loops are forbidden for $k\geq 2$, multiple edges are forbidden for $k\geq 3$). 




Let $G$ be an undirected graph. An \emph{arc} of $G$ is an edge $e$ of $G$ together with a chosen orientation of $e$ (so each edge of $G$ correspond to two arcs). The arc \emph{opposite} to an arc $a$, denoted by $-a$, is the arc corresponding to the same edge as $a$ but with the opposite direction. 
The endpoints of an arc $a$ are called the \emph{initial} and \emph{terminal} vertices of $a$ (with $a$ oriented from the initial vertex to the terminal vertex).  If $v$ is the initial (resp. terminal) vertex of the arc $a$, then we say that $a$ is an \emph{outgoing arc} (resp. \emph{ingoing arc}) at $v$. 
\\

%In a graph, a \emph{walk} (of length $k$) is a sequence $v_1,e_1,v_2,\ldots,e_k,v_{k+1}$ that alternates vertices and edges, such that $e_i$ connects $v_i$ to $v_{i+1}$ for $i\in[k]$. It is called a \emph{closed walk} if $v_1=v_{k+1}$. 
%\OB{Made a change in the def of walk (talking about arcs instead). Should we call them ``paths'' rather than ``walks''?}
A \emph{path} in an undirected graph $G$ is a sequence of arcs $a_1,a_2,\ldots,a_k$ such that the terminal vertex of $a_i$ is the initial vertex of $a_{i+1}$ for all $i\in[k-1]$. It is called a \emph{closed path} if the terminal vertex of $a_k$ is the initial vertex of $a_1$. A \emph{cycle} is a (cyclically ordered) closed path. A path or cycle is called \emph{simple} if it does not pass twice by the same vertex. The \emph{girth} of a graph is the minimum length of its simple cycles.   In a plane map, a closed path formed by the arcs around a face is called \emph{contour} of that face. It is known that face contours are simple cycles if the plane map is 2-connected. 
A simple cycle on a plane map is called \emph{counterclockwise} (resp. \emph{clockwise}) if the direction of arcs is counterclockwise (resp. clockwise) around the cycle.

Let $G$ be a graph.  Given an orientation of $G$, a \emph{directed path} (resp. \emph{directed cycle}) is a path (resp. cycle) $a_1,a_2,\ldots,a_k$ such that every arc $a_i$ is oriented according to the orientation of $G$.
A \emph{weighted orientation} of $G$ is an assignment of a non-negative integer to each arc of $G$. Given a weighted orientation $\cW$ of $G$, we call \emph{weight} of an edge the sum of the weights of the two corresponding arcs. 
Weighted orientations are a generalization of the classical (unweighted) orientations of $G$. Indeed the ``unweighted'' orientations of $G$ can be identified to the weighted orientations of $G$ such that the weight of every edge is 1 (for each edge, the arc of weight 1 is taken as the orientation of the edge). The \emph{outgoing weight} (shortly, the \emph{weight}) of a vertex $v$ is the sum of the weights of the arcs going out of $v$. Given a weighted orientation, we call \emph{positive path} (resp. \emph{positive cycle}) a path (resp. cycle) $a_1,a_2,\ldots,a_k$ such that the weight of every arc is positive (this generalizes the notion of \emph{directed path} and \emph{directed cycle}).  




A \emph{tree} is a connected, acyclic graph. For a tree $T$ with a vertex $v$ distinguished as its \emph{root}, we apply the usual ``genealogy'' vocabulary about trees, where $v$ is an \emph{ancestor} of all the other vertices, and every non-root vertex incident to $T$ has a \emph{parent} in $T$, etc. 
We say that we \emph{orient the tree $T$ toward its root} by orienting every edge from child to parent. With this orientation, every non-root vertex of $T$ is incident to one outgoing edge in $T$ (the edge leading to its parent).
%\OB{changed: calling ``subtree'' instead of ``tree''}
A \emph{subtree} of a graph $G$ is a subset of edges of $G$ such that this set of edges together with the incident vertices forms a tree. A \emph{spanning tree} of $G$ is a subtree of $G$ incident to every vertex of $G$. 





%\end{document}



\newcommand{\disp}{\displaystyle}
\newcommand{\z}{\boldsymbol{z}}
\newcommand{\x}{x}
\newcommand{\T}{\top}
\renewcommand{\L}{Q}
\renewcommand{\u}{\boldsymbol{u}}
\newcommand{\y}{\boldsymbol{y}}
\newcommand{\pd}[2]{\frac{\partial #1}{\partial #2}}
\renewcommand{\d}{{\rm d}}
\newcommand{\Ha}{\mathcal{H}}
\newcommand{\inner}[1]{\left\langle #1 \right\rangle}
\newcommand{\eig}{ {\rm eig}}
\newcommand{\highred}[1]{{\color{red} #1}}
\newcommand{\chgr}[1]{{\color{red} #1}}


\newcommand{\R}{{\mathbb R}}
\newcommand{\chgk}[1]{{\color{blue}  K:  #1}}

\newtheorem{theorem}{Theorem}
\newtheorem{lemma}{Lemma}
\newtheorem{corollary}{Corollary}


\begin{document}
\title{Exponential Decay Rate of Linear Port-Hamiltonian Systems. A Multiplier Approach}
\author{Luis A. Mora, \IEEEmembership{Member, IEEE}, and Kirsten Morris, \IEEEmembership{Fellow, IEEE}
\thanks{This research was supported by a Discovery Grant from NSERC and by a grant from the Faculty of Mathematics at the University of Waterloo.}
\thanks{Luis A. Mora and Kirsten Morris are with the Department of Applied Mathematics at University of Waterloo. 200 University Avenue West,	Waterloo, ON, Canada  N2L 3G1, (e-mails: lmora@uwaterloo.ca, kmorris@uwaterloo.ca).}
}

\maketitle

\begin{abstract}
	In this work, the multiplier method is extended  to obtain a general lower bound of the exponential decay rate in terms of the physical parameters for port-Hamiltonian systems in one space dimension with boundary dissipation. The physical parameters of the system may be spatially varying. It is shown that under assumptions of boundary or internal dissipation the system is exponentially stable. This is established through a Lyapunov function defined through a general multiplier function. Furthermore, an explicit bound on the decay rate in terms of the  physical parameters is obtained. The method is applied to a number of examples. 
\end{abstract}

\begin{IEEEkeywords}
	Boundary Dissipation,  Decay Rate, Distributed Parameter Systems, Exponential Stability, Partial Differential Equations, Port-Hamiltonian Systems, Infinite-dimensional system
\end{IEEEkeywords}

\section{Introduction}\label{sec:introduction}
	\IEEEPARstart{E}xponential stability is a desirable property  of most  systems, including those modelled by partial differential equations. As shown in such works as \cite{Komornik1994}, \cite{Russell1994}, \cite{Xu1996} and \cite{Zuazua2012}  exponential decay of a partial differential equation through boundary control or dissipation is directly related to the exact observability of the system.  Furthermore, determining not only exponential stability but also an expression for  the exponential decay rate in terms of the system parameters is of theoretical interest, and also in practical performance such as analyzing control system performance.
An  important strategy to obtain an explicit expression for the exponential decay rate of dynamical infinite dimensional systems is the multiplier method \cite{Yan2022,Cheng2021,Rivera2021,Guesmia2003,Guo2004,Wu2014}. Several approaches of the multiplier method can be found in literature. For example, in \cite{Komornik1994} the system dynamics are multiplied by $m(\x)$ and the state variables, and  integrated in the space and time to derive the exponential decay rate of the state variable norm. Alternatively, the multiplier function is used to build an auxiliary Lyapunov functional whose exponential decay is related to the decay of the system energy; see the exposition in \cite{Tucsnak2009}.
	
%P{ort}-Hamiltonian systems (PHS) have origins on the differential geometry and bond graphs machinery, making an energy-based description of the system dynamics that includes power-conjugated inputs and outputs for the interaction with the media. 
A Port-Hamiltonian formulation of boundary-controlled distributed parameter systems was initially introduced in \cite{VanderSchaft2002,LeGorrec2005} and extended to problems with internal  dissipation in \cite{Villegas2006}. Sufficient conditions for the well-posedness of   linear PHS in one-dimensional spatial domains were established in  \cite{Zwart2010}. 
%An analysis of the controllability/observability properties is presented in \cite{Jacob2019,Jacob2021,Xia2010}.Conditions for stability and stabilization are presented in works such as \cite{Augner2016,Macchelli2016,Macchelli2020,Ramirez2014,Villegas2009}, among others.
	Exponential stability of one-dimensional boundary controlled port-Hamiltonian systems has been studied in a number of works, including \cite{Villegas2009,Jacob2012,Augner2014,Macchelli2016,Trostorff2022}. In these works,  sufficient conditions that guarantee exponential stability are obtained. An explicit lower bound for the exponential decay rate of the energy  a Timoshenko beam with boundary  and internal dissipation was obtained in  \cite{Mattioni2022} .  

	In this work, we extend the multiplier method to obtain a general lower bound of the exponential decay rate in terms of the physical parameters for general port-Hamiltonian systems in one space dimension with boundary dissipation. The physical parameters of the system may be spatially varying. A  formal description of the  port-Hamiltonian systems under study is provided in Section \ref{sec:PHS}. The main results are presented in Section 3. We show that under assumptions of boundary or internal dissipation the system is exponentially stable. This is established by  considering general multiplier functions $m(\x). $   In previous work  \cite{MoraMorrisMTNS} a  linear multiplier function, $m(\x)=\x-a$, was used to obtain the an explicit bound on the  decay rate for port-Hamiltonian systems with  constant coefficients and  $P_0=0$, $G_0=0.$ This result was illustrated by obtaining  an explicit bound on the exponential decay rate of a boundary damped piezo-electric beam with magnetic effects.    The approach was extended in \cite{MoraMorrisCDC} to systems with $P_0\neq0$ and/or $G_0\neq0$, such as in a Timoshenko beam. However, the result in \cite{MoraMorrisCDC}  is restricted  to systems whose physical parameters satisfy several conditions. Here, by considering more general multiplier functions it is shown that a wider class of systems, including those with variable material parameters, are exponentially stable. Furthermore, an explicit bound on the decay rate in terms of the  physical parameters is obtained.  In Section 4, we apply the method to a number of examples. A preliminary analysis of Example A, the boundary-damped  wave equation with a linear multiplier function appeared in \cite{MoraMorrisMTNS}; here we compare the use of a linear  and exponential multiplier. The use of the linear multiplier function for the simple wave equation is well-known, here we not only compare different multiplier functions, but also, regarding the boundary damping as a control variable, the dissipation is chosen to optimize the decay rate. Example B applies our result to a wave equation with variable cross-section and material parameters.  In Example C, the Timoshenko beam, the result in \cite{MoraMorrisCDC} is extended to beams with general parameters. One lesson from these  examples is that the decay rate obtained depends on the choice of multiplier function.  Conclusions are presented on Section 5. 
	


\section{Port-Hamiltonian systems (PHS)}\label{sec:PHS}


	Consider an one-dimensional spatial domain  $\Omega=\{ \x\in [a,b]\ \} \subset\mathbb{R} .$ Denote by $\z(\x,t)$ the $n$ state variables of a system on $\Omega$.   In this work, the following class of linear boundary controlled port-Hamiltonian systems \cite{Zwart2010} is considered:
	\begin{align}
		\pd{\z(\x,t)}{t}&-\left[P_1\pd{ }{\x}+\left[P_0-G_0\right]\right]\L(\x)\z(\x,t)=0\label{eq:PHS_1}
			\end{align}
	where $P_1=P_1^{\T}\in\mathbb{R}^{n\times n}$ is invertible, $\L(\x)=\L^{\T}(\x)>0\in H^1([a,b],\mathbb{R}^{n\times n})$, $P_0=-P_0^{\T}\in\mathbb{R}^{n\times n}$, $G_0=G_0^{\T}\geq0\in\mathbb{R}^{n\times n}$, and  $\disp\frac{n}{2}\times n$ real matrices $W_1$, $W_2$ and $\tilde{W}_1.$ Defining
	\begin{align}
	\u_b(t)&=W_1 \L(b)\z(b,t), \label{eq:u:b} \\
		\y_b(t)&=\tilde{W}_1 \L(b)\z(b,t)\label{eq:y_b} \\
		\u_a(t) &=W_2\L(a)\z(a,t) \label{eq:BC_a} ,
		\end{align}
		the boundary conditions are, for  some  $K=K^{\T}>0\in\mathbb{R}^{\frac{n}{2}\times \frac{n}{2}}$
			\begin{equation}
		\u_a(t)=0,  \quad
		\u_b(t)+K\y_b(t)=0 \label{eq:BC_1}
	\end{equation}
	or equivalently, 
	$$W_2\L(a)\z(a,t) =0,$$
	$$ W_1 \L(b)\z(b,t)+ K \tilde{W}_1 \L(b)\z(b,t)= 0 .$$
	The dissipative  boundary condition at $x=b$ can arise through natural boundary dissipation \cite[e.g.]{ZLMV} or as a controlled feedback with a measurement  $\y_b $ and controlled input $\u_b .$ 
	
	It will be assumed throughout that  $W_1$, $W_2$ and $\tilde{W}_1$ satisfy the following rank conditions.
	\begin{align}
		{\rm rank} \left(\begin{bmatrix}
			0 & W_2\\
			W_1 & 0
		\end{bmatrix}\right)=&n \quad\text{and}\label{eq:rank1}\\
		{\rm rank}\left(\begin{bmatrix}
			W_1\\\tilde{W}_1
		\end{bmatrix}\right)=n .
		\label{eq:ranks}
	\end{align}
	This guarantees that system \eqref{eq:PHS_1} defines a well-posed control system  \cite[Theorem 2.4]{Zwart2010}.

	The total energy of system \eqref{eq:PHS_1}-\eqref{eq:BC_1}  is 
	\begin{align}
		\Ha(t)=&\int_{a}^{b}\frac{1}{2}\z^\T(\x,t)\L(\x)\z(\x,t)\d\x\label{eq:H} .
	\end{align}
	Since $\L (\x )>0$ for all $x,$ this defines a norm on $L^2 ([a,b],\R^n)$ equivalent to the standard norm. 
	Differentiating \eqref{eq:H} along system trajectories,  assuming that $W_1^\T\tilde{W}_1=-W_2^\T\tilde{W}_2=P_1$ for some $\tilde{W}_2\in\mathbb{R}^{\frac{n}{2}\times n}$ and defining $\eta_K=\min \eig(K),$
	%\chgk{Rewrote to avoid defining $\Sigma$ which is never used again}
	\begin{align}
		\frac{\d \Ha(t)}{\d t}=&-\int_{a}^{b}\z^\T(\x,t)\L G_0 \L\z(\x,t)\d\x \nonumber \\ & \quad +\frac{1}{2}	(\u_b^T (t)  \y_b (t)  + \y_b^T (t)   \u_b  (t)   ) 
		\nonumber\\
		\leq&-c_1\Ha(t)-\eta_K \|  \y_b  (t) \|^2 
		\label{eq:dtH_1}
	\end{align}
	where $c_1>0$ if internal dissipation $G_0 >0$  and $c_1=0$ otherwise. Exponential stability of the system when  $G_0 $ is not positive definite is not obvious. 
	
	\section{Exponential stability}

In this section the multiplier approach, see for example \cite{Tucsnak2009}, is modified and applied to the class of systems described in the previous section to obtain a explicit expression for the exponential decay rate in terms of the system parameters. 

\begin{lemma}\label{lemma:ExpSta}
	Let $\z(\x,t)\in L^2([a,b],\mathbb{R}^n)$ be the state of system \eqref{eq:PHS_1} on interval $\x\in[a,b]\subset\mathbb{R}$ and $\Ha(t)$ be the corresponding energy functional. If there exists a scalar functional $w(t)$ on $[a,b]$ of the state vector $\z(\x,t)$ such that
	\begin{align}
		|w(t)|\leq \frac{1}{\varepsilon_0}\Ha(t)
	\end{align}
	and
	\begin{align}
		\frac{\d w(t)}{\d t}\leq -\frac{1}{\varepsilon_1}\frac{\d \Ha(t)}{\d t}-c \Ha(t)
		\label{eq-wp}
	\end{align}
	for some  positive $\varepsilon_0$, $\varepsilon_1$ and $c, $ then $\Ha(t)$ decays exponentially. 
	Furthermore, defining $\disp M= \frac{\varepsilon_0+\varepsilon}{\varepsilon_0-\varepsilon}$ and decay rate $\disp \alpha=\frac{c\varepsilon\varepsilon_0}{\varepsilon_0+\varepsilon}$, $\forall \varepsilon\in[0,\min\{\varepsilon_0,\varepsilon_1\}],$
	\begin{align*}
		\Ha(t)\leq M e^{-\alpha t}\Ha(0) .
	\end{align*}
\end{lemma}

\begin{proof}
	Define $V_{\varepsilon}(t)=\Ha(t)+\varepsilon w(t)$, where $\varepsilon\in\mathbb{R}.$ %and $w(t)$ is a functional on domain $\Omega$ of the state variables $\z(\x,t)$. 	
	Note
	$$\Ha(t)-\varepsilon|w(t)|\leq V_{\varepsilon}(t)\leq \Ha(t)+\varepsilon|w(t)|.$$
	Also, since  $\displaystyle |w(t)|\leq \frac{1}{\varepsilon_0}\Ha(t),$
	\begin{align}
		\left(1-\frac{\varepsilon}{\varepsilon_0}\right) \Ha(t)\leq V_{\varepsilon}(t) \leq \left(1+\frac{\varepsilon}{\varepsilon_0}\right)\Ha(t)\label{eq:ExpDec0}
	\end{align}
	guaranteeng that $V_{\varepsilon}(t)$ is non-negative for all $\varepsilon\in[0,\varepsilon_0]$.
	
	Furthermore, using \eqref{eq-wp},  %$\frac{\d w(t)}{\d t}\leq -\frac{1}{\varepsilon_1}\frac{\d \Ha(t)}{\d t}-c \Ha(t)$ for some $\varepsilon_1>0$ and $c >0$, 
	%the time derivative of $V_{\varepsilon}(t)$ is
	\begin{align*}
		\frac{\d V_{\varepsilon}(t)}{\d t}=&\frac{\d \Ha(t)}{\d t}+\varepsilon\frac{\d w(t)}{\d t}\\
		\leq&\left(1-\frac{\varepsilon}{\varepsilon_1}\right)\frac{\d \Ha(t)}{\d t} -\varepsilon c \Ha(t)
	\end{align*}
	  For any $\varepsilon\leq \varepsilon_1 ,$ \eqref{eq:ExpDec0} implies 
	\begin{align*}
		\frac{\d V_{\varepsilon}(t)}{\d t}\leq& -\varepsilon c \Ha(t) \leq -\frac{c\varepsilon}{1+\varepsilon/\varepsilon_0}V_{\varepsilon}(t)
	\end{align*}
	obtaining that $V_{\varepsilon}(t)=V_{\varepsilon}(0)e^{-\alpha t}$ where $\alpha=\dfrac{c\varepsilon\varepsilon_0}{\varepsilon_0+\varepsilon}$. Using again \eqref{eq:ExpDec0}, we obtain that $V_{\varepsilon}(0)\leq \left(1+\frac{\varepsilon}{\varepsilon_0}\right)\Ha(0)$ and $\Ha(t)\leq \frac{1}{1-\varepsilon/\varepsilon_0}V_{\varepsilon}(t)$. As a consequence,
	\begin{align}
		\Ha(t)\leq \frac{\varepsilon_0+\varepsilon}{\varepsilon_0-\varepsilon}e^{-\alpha t}
	\end{align}
	for all $\varepsilon\in[0,\min(\varepsilon_0,\varepsilon_1)]$, completing the proof.
\end{proof}





% \chgr{Also, can the theorem be written for a general multiplier function $m$ with a condition on $m?$ This theorem could be followed by the previous lemma showing that it's always possible to choose $m$, also mentioning the usual choice of $m.$
% This might inspire others to improve the decay rate and cite our paper :) }

\begin{table}
	\centering
	\caption{System parameters}
	\label{tab:params}
	\begin{tabular}{cl}
		\hline Parameter  & Description\\ \hline
		$\disp\mu_\L$&$\disp\max_{\x\in[a,b]} \max \eig(\L(\x))$ \\
		$\disp\mu_B$&$\disp\max_{\x\in[a,b]} \max \eig(B(\x))$ \\
		$\disp\mu_{\Psi}$&$\disp\max_{\x\in[a,b]}\max \eig(\Psi(\x))$ \\
		$\disp\mu_{P_1}$&$\disp\sqrt{\max \eig(P_1^{-2})}$ \\
		$\disp \mu_m$&$\disp\max_{\x\in[a,b]} m(\x)$ \\
		$\disp\eta_\L$&$\disp\min_{\x\in[a,b]} \eig(\L(\x))$ \\
		$\disp\eta_K$&$\disp\min \eig(K)$\\
		\hline \multicolumn{2}{l}{{\footnotesize Matrices $B(\x)$ and $\Psi(\x)$ are defined in \eqref{eq:B} and \eqref{eq:Psi1}, respectively}.}
	\end{tabular}
\end{table}

\begin{theorem}\label{thm:1}
	Consider the port-Hamiltonian system with boundary dissipation given by \eqref{eq:PHS_1}-\eqref{eq:BC_1}. Define 
	\begin{align}
		\Psi(\x)=&\begin{bmatrix}
			-K\\ I
		\end{bmatrix}^\T \begin{bmatrix}
			W_1\\\tilde{W}_1
		\end{bmatrix}^{-\T}\L^{-1}(\x)\begin{bmatrix}
			W_1\\\tilde{W}_1
		\end{bmatrix}^{-1}\begin{bmatrix}
			-K\\ I
		\end{bmatrix} , \label{eq:Psi1}\\
		B(\x)=&\pd{\L(\x)}{\x}-\L(\x)(P_0+G_0)P_1^{-1}\nonumber\\&+P_1^{-1}(P_0-G_0)\L(\x)\label{eq:B} ,\\
		A_s(\x)=&¸\pd{m(\x)}{\x}\L(\x)-m(\x)B(\x)\label{eq:As1} .
	\end{align}
	Also for some $m(\x ) \in C([a,b])$ define the auxiliary function  of the state $\z$
	\begin{align}
		w(t)=\frac{1}{2}\int_{a}^{b} m(\x)\z^\T(\x,t)P_1^{-1}\z(\x,t)\d\x\label{eq:wt1}
	\end{align}
	Defining
	 $\disp\varepsilon_0=\dfrac{\eta_{\L}}{\mu_{m}\mu_{P_1}}$ and  $\disp\varepsilon_1=\frac{2\eta_{K}}{\mu_{m}\mu_{\Psi}},$
	if
	\begin{align}
		A_s(\x)>0\label{eq:As>0}
	\end{align}
	then
	%\begin{align}
	%	\alpha=\frac{c\varepsilon\varepsilon_0}{\varepsilon+\varepsilon_0}
	%\end{align}
	for all $\varepsilon\in[0,\min\{\varepsilon_0,\varepsilon_1\}]$, 
		\begin{equation}\Ha (t) \leq M e^{-\alpha t }, \quad M = \frac{\varepsilon_0+\varepsilon}{\varepsilon_0-\varepsilon}  , \quad  \alpha=\frac{c\varepsilon\varepsilon_0}{\varepsilon+\varepsilon_0} \, . \label{H-bound}
		\end{equation}
		
	
\end{theorem}
\begin{proof}
	%Note that $m(\x)\leq \mu_{m}$ for all $\x\in[a,b]$.
	Using the Cauchy-Schwarz inequality,
	\begin{align*}
		|w(t)|=&\frac{1}{2}\left|\inner{m(\x)\z(\x,t),P_1^{-1}\z(\x,t)}\right|\\
		\leq& \frac{1}{2}\|m(\x)\z(\x,t)\|_{L^2}\|P_1^{-1}\z(\x,t)\|_{L^2}\\
		\leq&\frac{\mu_{m}\mu_{P_1}}{2}\|\z(\x,t)\|_{L^2}^2\leq\frac{\mu_{m}\mu_{P_1}}{\eta_{\L}}\Ha(t)
	\end{align*}
	Thus, $|w(t)|\leq \dfrac{1}{\varepsilon_0}\Ha(t).$	
	Similarly,
	\begin{align}
		\frac{\d w(t)}{\d t}=&\int_{a}^{b}m(\x)\z^\T(\x,t)P_1^{-1}\pd{\z(\x,t)}{t} ~\d\x \nonumber \\
		=&\int_{a}^{b}m(\x)\z^\T(\x,t)P_1^{-1}(P_0-G_0)\L(\x)\z(\x,t) ~\d\x\nonumber\\&+ \int_{a}^{b}m(\x)\z^\T(\x,t)\pd{\L(\x)\z(\x,t)}{\x} ~\d\x  \label{eq-dw}
	\end{align}
	%Since term $\z^\T(\x,t)P_1^{-1}(P_0-G_0)\L(\x)\z(\x,t)$ is a quadratic product, only the symmetric part of $P_1^{-1}(P_0-G_0)\L(\x)$ is needed. 
	Using the identity
	\begin{align*}
		\frac{1}{2}\pd{}{\x}\left(m(\x)\z(\x,t)^T \L(x)\z(\x)\right)=m(\x)\z^\T(\x,t)\pd{\L(\x)\z(\x,t)}{\x}\\+\frac{1}{2}\z^{\T}(\x,t)\left(\pd{m(\x)}{\x}\L-m(\x)\pd{\L(\x)}{\x}\right)\z(\x,t)
	\end{align*}
	\eqref{eq-dw}  is rewritten as
	\begin{align*}
		\frac{\d w(t)}{\d t}
		=&\left.\frac{1}{2}m(\x)\z^\T(\x,t)\L(\x)\z(\x,t)\right|_{a}^b\\&-\frac{1}{2}\int_{a}^{b}\z^\T(\x,t)A_s(\x)\z(\x,t)~\d\x
	\end{align*}
	where $A_s$ is defined in  \eqref{eq:As1}. Since $A_s$ is assumed positive, there exists a $c>0$ such that
	$$A_s(\x)\geq c\L(\x)>0.$$
	This implies that  
	\begin{align*}
		\frac{\d w(t)}{\d t}
		\leq&\frac{1}{2}m(b)\z^\T(b,t)\L(b)\z(b,t)\\&-\frac{c}{2}\int_{a}^{b}\z^\T(\x,t)\L(\x)\z(\x,t)~\d\x . 
	\end{align*}
	
	By assumption  \eqref{eq:ranks} $\begin{bmatrix}
		W_1\\\tilde{W}_1
	\end{bmatrix}$ is full rank and so $\disp \L(b)\z(b,t)=\begin{bmatrix}
		W_1\\\tilde{W}_1
	\end{bmatrix}^{-1}\begin{bmatrix}
		\u_b(t)\\\y_b(t)
	\end{bmatrix}$. Then, including the boundary dissipation \eqref{eq:BC_1} leads to, recalling  the definition of $\Psi$ in \eqref{eq:Psi1},
	\begin{align*}
		\frac{\d w(t)}{\d t}
		\leq&\frac{1}{2}m(b)\y^\T_b(t)\Psi\y_b(t)-c\Ha(t)\\
		\leq& \frac{1}{2}\mu_m\mu_{\Psi}|\y_b|^2-c\Ha(t)\\
		\leq& -\frac{1}{\varepsilon_1}\frac{\d \Ha(t)}{\d t}-c\Ha(t) .
	\end{align*}	 
	where $\varepsilon_1=\dfrac{2\eta_K}{\mu_{m}\mu_{\Psi}}.$ Lemma \ref{lemma:ExpSta} then implies the  bound on the exponential decay of  $\Ha(t)$ in  \eqref{H-bound}.
	% bounded by $Me^{-\alpha t}\Ha(0)$ with  {\color{red}amplitude} $\disp M= \frac{\varepsilon_0+\varepsilon}{\varepsilon_0-\varepsilon}$ and decay rate $\disp \alpha=\frac{c\varepsilon\varepsilon_0}{\varepsilon_0+\varepsilon}$, $\forall \varepsilon\in[0,\min\{\varepsilon_0,\varepsilon_1\}]$. 
\end{proof}



\begin{lemma}\label{lemma:1}
	Consider the matrices $A_s(\x)$ and $B(\x)$ defined in \eqref{eq:As1} and \eqref{eq:B}, respectively. 
	%	For continuous function $m(x)$ define the symmetric matrices
	%	\begin{align}
	%		A_s (x) =&\pd{m(\x)}{\x}\L(\x)-m(\x)B(\x)\label{eq:As}\\
	%		B(\x)=&\pd{\L(\x)}{\x}-\L(\x)(P_0-G_0)P_1^{-1}+P_1^{-1}(P_0-G_0)\L(\x)\label{eq:B}
	%	\end{align}
	%	where $\L(\x)$, $G_0$, $P_0$ and $P_1$ are the matrices of the port-Hamiltonian system \eqref{eq:PHS_1}.
	%	
	Defining $\disp m(\x)=Ce^{\beta (\x-a)}, $  if $\beta$ is sufficiently large then %and $\disp \pd{m(\x)}{\x}=\beta m(\x)$. 
	\begin{align}
		A_s>0 ~,\forall \x\in[a,b]\label{eq:As>0_1} .
	\end{align}
\end{lemma}

\begin{proof}
	 With $m(\x)=Ce^{\beta (\x-a)}$, the matrix $A_s$ can be rewritten as
	\begin{align*}
		A_s=m(\x)\left(\beta \L(\x)-B(\x) \right) .
	\end{align*}
	%	where 
	%	\begin{align}
	%		B(\x)=\pd{\L(\x)}{\x}-\L(\x)(P_0-G_0)P_1^{-1}+P_1^{-1}(P_0-G_0)\L(\x)\label{eq:B}
	%	\end{align}
	
	Since $m(\x)>0$,  condition \eqref{eq:As>0_1} is satisfied if  matrix $\beta \L(\x)-B(\x)$ is  positive; that is  if $\disp \inf_{\x\in[a,b]} \eig\left(\beta \L(\x)-B(\x)\right)>0. $  
	Since $\disp\eta_{\L}=\inf_{\x\in[a,b]} \eig\left(\L(\x)\right) >0$ and recalling $\disp\mu_B=\sup_{\x\in[a,b]} \eig\left(B(\x)\right),$ if $\beta$ is chosen large enough that 
		\begin{align*}
		(\beta\eta_{\L}-\mu_B)>0 
	\end{align*}
then the required condition is satisfied. 
\end{proof}

Exponential stability of the class of systems described in section 2 now follows immediately, along with a bound on the decay rate. 
Lemma \ref{lemma:1} implies that exists at least one multiplier function, $m(\x)$, such that condition \eqref{eq:As>0_1} holds and so the system  \eqref{eq:PHS_1},\eqref{eq:BC_1} is exponentially stable.  Furthermore, Theorem \ref{thm:1} can be used to obtain a lower bound of the exponential decay rate for all systems with the form \eqref{eq:PHS_1}-\eqref{eq:BC_1} on interval $\x\in[a,b]$. 

From Lemma \ref{lemma:1}, there are  definitions for $M(\varepsilon)$ and $\alpha(\varepsilon)$ for all $\varepsilon$ on the interval $[0,\min\{\varepsilon_0,\varepsilon_1\}]$, and Theorem \ref{thm:1} provides explicit expressions of $\varepsilon_0$ and $\varepsilon_1$ for system \eqref{eq:PHS_1}-\eqref{eq:BC_1}. 
Using  the parametrization $\varepsilon=\xi\min\{\varepsilon_0,\varepsilon_1\}$ with $0<\xi<1$ leads to
\begin{align}
	M=&\begin{cases}
		\dfrac{1+\xi}{1-\xi} & \text{if}~ \varepsilon_0\leq \varepsilon_1\\
		\dfrac{\eta_{\L}\mu_{\Psi}+2\xi\eta_{K}\mu_{P_1}}{\eta_{\L}\mu_{\Psi}-2\xi\eta_{K}\mu_{P_1}} & \text{otherwise}
	\end{cases}\label{eq:M}\\
	\alpha=&\begin{cases}
		\dfrac{\xi}{\xi+1}\dfrac{c\eta_{\L}}{\mu_{P_1}\mu_{m}}& \text{if}~\varepsilon_0\leq \varepsilon_1\\
		\dfrac{2 c \eta_{K}\eta_{\L} \xi}{\mu_{m}\left(\eta_{\L}\mu_{\Psi}+2\xi\eta_{K}\mu_{P_1}\right)} & \text{otherwise}
	\end{cases}
\end{align}

Since $c$ and $\mu_{m}$ are affected by the multiplier function, an appropriate choice of $m(\x)$ improves the exponential decay rate bound obtained through Theorem \ref{thm:1}.  Considering $\disp m(\x)=Ce^{\beta (\x-a)}$ with $C>0$, as in the proof of Lemma \ref{lemma:1}, we obtain $m(a)=C$, $\mu_{m}=m(b)=Ce^{\beta (b-a)}$  and
\begin{align*}
	A_s=&m(\x)\left(\beta \L(\x)-B(\x) \right)\\ \geq& m(a)(\beta\eta_{\L}-\mu_B) \\ \geq  &c\L
\end{align*}
where $\disp c=\frac{C(\beta\eta_{\L}-\mu_B)}{\mu_{\L}}$. Then, the exponential decay rate bound is given by 
\begin{align}
	\alpha=&\begin{cases}
		\dfrac{\xi}{\xi+1}\dfrac{\eta_{\L}e^{-\beta(b-a)}(\beta\eta_{\L}-\mu_B)}{\mu_{\L}\mu_{P_1}} &\text{if}~ \varepsilon_0\leq \varepsilon_1\\
		\dfrac{2\xi\eta_{K}\eta_{\L} e^{-\beta(b-a)} (\beta\eta_{\L}-\mu_B)}{\mu_{\L}\left(\mu_{P_1}\eta_{K}+2\xi\eta_{\L}\mu_{\Psi}\right)} & \text{otherwise}
	\end{cases}\label{eq:alpha_beta}
\end{align}


\begin{theorem}The system \eqref{eq:PHS_1}-\eqref{eq:BC_1}  is exponentially stable. Furthermore, the decay rate is at least
	%$c=C\frac{\eta_{\L}}{\ell \mu_{\L}}$, $\varepsilon_0=e^{-(1+\frac{\ell\mu_B}{\eta_{\L}})}\frac{\eta_{\L}}{C\mu_{P_1}}$, $\varepsilon_1=e^{-(1+\frac{\ell\mu_B}{\eta_{\L}})}\frac{2\eta_{K}}{C\mu_{\Psi}}$
	\begin{align}
		\alpha=&\begin{cases}
			\dfrac{\xi}{\xi+1}\dfrac{\eta^2_{\L} e^{-\left(1+\frac{ \mu_B}{\eta_{\L}}(b-a)\right)}}{(b-a)\mu_{\L}\mu_{P_1}} &\text{if}~ \varepsilon_0\leq \varepsilon_1\\
			\dfrac{2\xi\eta_{K}\eta_{\L}^2 e^{-\left(1+\frac{\mu_B}{\eta_{\L}}(b-a)\right)}}{(b-a)\mu_{\L}\left(\mu_{P_1}\eta_{K}+2\xi\eta_{\L}\mu_{\Psi}\right)} & \text{otherwise}
		\end{cases}\label{eq:op_alpha}
	\end{align}
\end{theorem}

\begin{proof}
Using the exponential multiplier function from Lemma \ref{lemma:1}  along with Theorem \ref{thm:1} yields the conclusion that the system is exponentially stable, along with bounds on $M$ and $\alpha.$
	Since $M$ is independent of $\beta$, the optimal decay rate is obtained by choosing $\beta$ to maximize $\alpha$, that is 
	\begin{align}
		\beta_{op}=\arg\max_{\beta} \alpha=\frac{\eta_{\L}+(b-a)\mu_B}{(b-a)\eta_{\L}}=\frac{1}{b-a}+\frac{\mu_B}{\eta_{\L}}
	\end{align}
	Then, the optimal decay rate is obtaining substituting $\beta_{op}$ in \eqref{eq:alpha_beta}. 
\end{proof}


As shown Lemma \ref{lemma:1}, choosing $m(\x)$ as an exponential function, condition \eqref{eq:As>0} can always be satisfied. Depending on the system, other options for $m(\x)$ may be possible. 
For example, consider the case $P_0=G_0=0$, $\L(\x)=L\x+D>0$, $\forall\x\in[a,b]$, with $D$ and $L$ defined positive. Choosing $m(\x)=q\x+d$ where $\disp \frac{q}{d}\geq -\frac{1}{a},$
matrix $A_s$ becomes
\begin{align}
	A_s=&\pd{m(\x)}{\x}\L(\x)-m(\x)\pd{\L(x)}{\x}\nonumber\\=&q(L\x+D)-(q\x+d)L=qD-dL \end{align}
Then, \eqref{eq:As>0} is satisfied if $\disp \frac{q}{d}>\max\left\lbrace\frac{\mu_L}{\eta_D} ,-\frac{1}{a}\right\rbrace$.  
This point is illustrated in an example in the next section. 

\section{Examples}

%In this section we presents some practical examples. Firstly, we consider the wave equation with boundary dissipation to illustrate how the choice of the multiplier function can improve the exponential decay rate bound. In the second example, we consider two parameter sets for a Timoshenko beam to illustrate how the physical parameters affect the choice of an appropriate multiplier function.

\subsection{Wave equation with boundary dissipation}
Consider the  wave equation in an one-dimensional spatial domain
\begin{align}
	\pd{}{t}\left(\rho\pd{w(\x,t)}{t}\right)=&\pd{}{\x}\left(\tau \pd{w(\x,t)}{\x}\right) & \forall\x\in[a,b] \label{eq:ExWave}
\end{align}
with boundary conditions
\begin{align}
	\pd{w(a,t)}{t}=&0 & \forall t\geq0\label{eq:EXWaveBC1}\\
	\tau\pd{w(b,t)}{\x}+k\pd{w(b,t)}{t}=&0& \forall t\geq 0 \label{eq:EXWaveBC2}
\end{align}
and $w(\x,0)\in L^2([a,b],\mathbb{R})$, where the density and elasticity parameters, $\rho$ and $\tau$ respectively, are constant.

Defining $\disp z_1=\pd{w(\x,t)}{\x}$ and $\disp z_2=\rho\pd{w(\x,t)}{t}$, the wave equation \eqref{eq:ExWave} is expressed as the port-Hamiltonian system
\begin{align}
	\pd{\z(\x,t)}{t}=&P_1\pd{}{\x}\left(\L\z(\x,t)\right), & \forall \x\in[a,b]\label{eq:WEPHS}
\end{align}
where $\z(\x,t)=\begin{bmatrix}
	z_1(\x,t) & z_2(\x,t)
\end{bmatrix}^\T$, $P_1=\begin{bmatrix}
	0 & 1\\1&0
\end{bmatrix}$ and $\L=\begin{bmatrix}
	\tau & 0\\0 & {1}/{\rho}
\end{bmatrix}$. Similarly, choosing
$W_1=W_2=\begin{bmatrix}
	1 &0
\end{bmatrix}$ and $\tilde{W}_1=\begin{bmatrix}
	0 &1
\end{bmatrix}$, the boundary conditions \eqref{eq:EXWaveBC1}-\eqref{eq:EXWaveBC2} can be rewritten in the form  \eqref{eq:u:b}-\eqref{eq:BC_a}.
%\begin{align}
	%\frac{1}{\rho}z_1(a,t)=&0\\
	%\tau z_2(b,t)+k\frac{z_1(b,t)}{\rho}=&\u_b+k\y_b=0 \label{eq:WEPHSBC1}
%\end{align}
%and the total energy satisfies $\disp \pd{\Ha}{t}=-\frac{k}{\rho^2}z_1(b,t)=-\frac{\tau^2}{k}z_2(b,t)$.


Since $\L$ is a constant matrix and $P_0=G_0=0, $  $A_s=\pd{m(\x)}{\x}\L$. Condition \eqref{eq:As>0} holds if the multiplier function $m(\x)$ is monotonically increasing. Assuming unitary parameters, $\tau=\rho=1$, and spatial domain length, $b-a=1$, 
\begin{align*}
	\mu_B=&0&\mu_{P_1}=&\mu_{\L}=\eta_{\L}=1,&\mu_{\Psi}=&k^2+1 \\
	\varepsilon_0=&\frac{1}{\mu_m} & \varepsilon_1=&\frac{2k}{\mu_m (k^2+1)} & c=&\min_{\x\in[a,b]}\pd{m(\x)}{\x}
\end{align*}
Since $\disp \frac{2k}{k^2+1}\leq 1$ for all $k\geq 0, $ $\varepsilon_1\leq \varepsilon_0$ for any $k$.   Choosing $\disp\varepsilon=\frac{1}{2}\varepsilon_1$ we obtain $\disp M=\frac{k^2+k+1}{k^2-k+1}$ which is independent of the choice of $m(\x)$. %So, only the exponential decay rate $\alpha$ is affected by the selection of $m(\x)$.
With an exponential multiplier function, as in  Lemma \ref{lemma:1}, from \eqref{eq:op_alpha} we obtain that the decay rate $\alpha=\dfrac{ke^{-1}}{k^2+k+1}$. Alternatively, considering a linear multiplier function, $m(\x)=x-a$,    so $\mu_{m}=c=1$ and the exponential decay rate is $\alpha=\dfrac{k}{k^2+k+1}.$ This is   a better lower bound for the decay rate than the exponential multiplier function. This point is illustrated in  Figure \ref{fig:WaveEq}.
\begin{figure}
	\centering
	\includegraphics[width=\columnwidth]{fig1.pdf}
	\caption{Bound on the exponential decay rate of the wave equation as a function of the boundary dissipation $k$ for different multiplier functions}
	\label{fig:WaveEq}
\end{figure}


If the boundary dissipation comes from a control law, then the value of $k$ should be chosen to optimize the exponential decay rate. For this example the exponential decay rate $\disp \alpha=\frac{k}{k^2+k+1}$  is maximized  with $k=1. $ Note that this choice of $k$ is the same value for which no waves are reflected and the energy of the wave equation reaches zero in  finite time. 




\subsection{Wave equation with  variable cross-section and boundary dissipation}
\begin{figure*}
	\centering
	\includegraphics[width=1.95\columnwidth]{fig2.pdf}
	\caption{Vibrating string with non-uniform cross-sectional area}
	\label{fig:VS}
\end{figure*}

In this example, we consider a vibrating string with a non-uniform cross-sectional area $A(\x)$, as shown in Figure \ref{fig:VS}. The dynamics of the vertical displacements $w(\x,t)$ can be expressed as the wave equation with boundary dissipation described in \eqref{eq:ExWave}-\eqref{eq:EXWaveBC2}, where the physical parameters  $\rho(\x)=A(\x)\rho_0$ and $\tau(\x)=A(\x)\tau_0$ with $\rho_0$ and $\tau_0$ constant, and $w(\x,0)\in L^2([0,1],\mathbb{R})$. It will be assumed that $\tau_0=\rho_0=1$, boundary dissipation gain $k=0.5$ and
that cross-sectional area $$A(\x)=\dfrac{10-\x}{10}.$$
The port-Hamiltonian formulation of this vibrating string is similar to the previous analysis \eqref{eq:WEPHS} except that  
$$\L(\x)=\begin{bmatrix}
	\dfrac{(10-x)}{10} & 0\\
	0 & \dfrac{10}{(10-x)}
\end{bmatrix}.$$
As a consequence, 

%\begin{strip}
	\begin{align}
		A_s(\x)=&\pd{m(\x)}{\x}\L(\x)-m(\x)\pd{\L(\x)}{\x}\nonumber\\
		=&\begin{bmatrix}
			\dfrac{m(\x)+\pd{m(\x)}{\x}(10-x)}{10} & 0\\
			0 & \dfrac{\pd{m(\x)}{\x}(10-x)-m(\x)}{0.1(10-x)^2}
		\end{bmatrix} . 
	\end{align}
%\end{strip}

Choosing the multiplier function $m(\x)=\x$, 
\begin{align*}
	A_s(\x)	=&\begin{bmatrix}
		1 & 0\\
		0 & \dfrac{10 \left(10-2\x\right)}{(10-x)^2}
	\end{bmatrix}>0, \quad\forall \x\in[0,1] \, . 
\end{align*}




The problem of finding the maximum $c$ such that $A_s(\x)\geq c\L(\x)$ is equivalent to finding the largest $c$ so that the eigenvalues of $A_s(\x)-c\L(\x)$ are non-negative; that is so the matrix
\begin{align*}
	\begin{bmatrix}
		1-c\dfrac{10-\x}{10} & 0\\
		0 & \dfrac{10}{(10-\x)}\left(\dfrac{10-2\x}{10-\x}-c\right)
	\end{bmatrix}
\end{align*}
is positive semi-definite. The largest such value of $c$ is $c=8/9$. %Considering material density $\rho_0=8940$ kg/m$^3$ and shear elastic modulus $\tau_0=46.8\times 10^6$ N/m$^2$, copper material properties \cite{Davis2001}, 
Then,$$\disp\Psi(\x)=\frac{5}{2\left(10-\x\right)}+\frac{(10-\x)}{10},$$ 
\begin{align*}
	\mu_{\L}=&\frac{10}{9} & \eta_{\L}=&\frac{9}{10} & \mu_{\Psi}=&\frac{5}{4}\\
	\mu_{P_1}=&\mu_{m}=1 & 	\epsilon_0=&\frac{9}{10} & \varepsilon_1=&\frac{4}{5} \, . 
\end{align*}
Finally, choosing $\varepsilon=\varepsilon_1$ a bound on the exponential decay rate is 
$$\alpha=\frac{32}{85}\approx0.3765 . $$
%the material properties as $\rho_0=8790$ kg/m$^3$ and $\tau_0=117\times 10^6$ N/m$^2$ (cooper){\color{red}[reference]},

\subsection{Timoshenko Beam}

Consider a Timoshenko beam  with variable material parameters on a bar $ x\in [a,b].$ Let   $\rho(\x)$, $\epsilon(\x)$ and $\iota(\x)$ be the mass per unit length,Young’s modulus, and moment of inertia of the cross section, respectively; $\iota_{\rho}(\x)=\iota(\x)\rho(\x)$ is the mass moment of inertia of the cross section; and  $\gamma(\x)$ and $\delta(\x)$ are the viscous damping coefficients. The shear modulus $\kappa(\x)= \xi G(\x)A(\x)$, where $G(\x)$ is the modulus of elasticity in shear, $A(\x)$ is the cross sectional area, and
$\xi$ is a constant depending on the shape of the cross section.  The parameters $k_1$ and $k_2$ are  boundary damping coefficients. This leads to the following partial differential equation
\begin{subequations}
	\begin{align}
		\rho(\x)\pd{^2w(\x,t)}{t^2}=&\pd{}{\x}\left(\kappa(\x)\left(\pd{w(\x,t)}{\x}-\phi(\x,t)\right)\right)\nonumber\\&-\gamma(\x)\pd{w(\x,t)}{t}\\
		\iota_{\rho}(\x)\pd{^2\phi(\x,t)}{t^2}=&\pd{}{\x}\left(\epsilon(\x)\iota(\x)\pd{\phi(\x,t)}{\x}\right)-\delta(\x)\pd{\phi(\x,t)}{t}\nonumber\\&+\kappa(\x)\left(\pd{w(\x,t)}{\x}-\phi(\x,t)\right)
	\end{align}\label{eq:TB_1}
\end{subequations} 
with boundary conditions
\begin{subequations}
	\begin{align}
		\pd{w(a,t)}{t}=\pd{\phi(a,t)}{t}=&0\\
		\kappa(b)\left(\pd{w(b,t)}{\x}-\phi(b,t)\right)+k_1\pd{w(b,t)}{t}=&0\\
		\epsilon(b)\iota(b)\pd{\phi(b,t)}{\x}+k_2\pd{\phi(b,t)}{t}=&0 \, . 
	\end{align}\label{eq:TB_2}
\end{subequations}
Set $z_1(\x,t)=\rho(\x)\pd{w(\x,t)}{t}$, $z_2(\x,t)=\iota_{\rho}(\x)\pd{\phi(\x,t)}{t}$, $z_3(\x,t)=\pd{w(\x,t)}{\x}-\phi(\x,t), $  $z_4(\x,t)=\pd{\phi(\x,t)}{\x}$ and $\z(\x,t)=\begin{bmatrix}
	z_1(\x,t) & z_2(\x,t) &z_3(\x,t)& z_4(\x,t)
\end{bmatrix}^\T . $   System \eqref{eq:TB_1} can be rewritten in the port-Hamiltonian formulation as in \cite{Mattioni2022} to obtain
$$\pd{\z(\x,t)}{t}-P_1\pd{\L(\x)\z(\x,t)}{\x}-\left[P_0-G_0\right]\L(\x)\z(\x,t)=0$$
where
\begin{align*}
	P_1=&\begin{bmatrix}
		0 & 0 & 1 & 0\\
		0 & 0 & 0 & 1\\
		1 & 0 & 0 & 0\\
		0 & 1 & 0 & 0
	\end{bmatrix}, \quad P_0=\begin{bmatrix}
		0 & 0 & 0 & 0\\
		0 & 0 & 1 & 0\\
		0 & -1 & 0 & 0\\
		0 & 0 & 0 & 0
	\end{bmatrix}, \\
	G_0=&\begin{bmatrix}
		\gamma & 0 &0 &0\\
		0 & \delta & 0 & 0\\
		0 & 0 & 0 & 0\\
		0 & 0 & 0 & 0
	\end{bmatrix}, \quad\text{and} \\
	\L(\x)=&\begin{bmatrix}
		\frac{1}{\rho(\x)} & 0 & 0 & 0\\
		0 & \frac{1}{\iota_{\rho}(\x)} & 0 & 0\\
		0 & 0 & \kappa(\x)& 0\\
		0 & 0 & 0 & \epsilon(\x)\iota(\x)
	\end{bmatrix} \, . 
\end{align*}

Similarly, defining
\begin{align}
	W_1=&\begin{bmatrix}
		0 & 0 & 1 & 0\\
		0 & 0 & 0 & 1\\
	\end{bmatrix}, \quad \tilde{W}_1=\begin{bmatrix}
		1 & 0 & 0 & 0\\
		0 & 1 & 0 & 0
	\end{bmatrix} \quad\text{and}\nonumber\\ W_2=&\begin{bmatrix}
		1 & 0 & 0 & 0\\
		0 & 1 & 0 & 0
	\end{bmatrix}
\end{align}
 the boundary conditions \eqref{eq:TB_2} can be written  in the standard form \eqref{eq:u:b}-\eqref{eq:BC_a} .


First, consider an inviscid beam with the same  parameters as in \cite{Mattioni2022}; that is $\gamma(\x)=\delta(\x)=0$, with  $\rho=0.2$kg/m, $\epsilon\iota=1.2\times 10^{-2}$Nm$^2$,  $\kappa=4\times10^{-3}$N, $\iota_{\rho}=2\times10^{-2}$kgm , $b-a=0.1$m and $k_1=k_2=k$. This leads to 
\begin{align*}
	\L=&\begin{bmatrix}
		5 & 0 & 0 & 0\\0 & 50 & 0 & 0\\0 & 0 & \dfrac{1}{250} & 0\\ 0 & 0 & 0 & \dfrac{1}{75}
	\end{bmatrix} \\ 
	B=&\begin{bmatrix}
		0 & -50 & 0 & 0\\-50 & 0& 0&0\\0 & 0 & 0 & \dfrac{1}{250}\\ 0 & 0 & \dfrac{1}{250}& 0
	\end{bmatrix} \\ 
	\Psi=&\begin{bmatrix}
		250k^2+\dfrac{1}{5} & 0\\0 & 75k^2+\dfrac{1}{50}
	\end{bmatrix}
\end{align*}
and $\mu_{P_1}=1$, $\mu_{\L}=\mu_B=50$, $\eta_{\L}=1/250$ and $\mu_{\Psi}=250k^2+1/5$.
Choosing a linear multiplier function, $m=x-a$ with $\mu_{m}=0.1$, then
\begin{align*}
	A_s(\x)=\begin{bmatrix}
		5 & 50(x-a)& 0 & 0\\50(x-a) & 50 & 0 & 0\\0 & 0 & \dfrac{1}{250}&\dfrac{a-x}{250}\\ 0 & 0 & \dfrac{a-x}{250} & \dfrac{1}{75}
	\end{bmatrix} 
\end{align*} 
which has eigenvalues 
\begin{align*}
	\eig(A_s(\x))=\begin{cases}
		\dfrac{13\pm\sqrt{36(x-a)^2+49}}{1500}\\ \\
		\dfrac{55\pm5\sqrt{400(x-a)^2+81}}{2}
	\end{cases} \, . 
\end{align*}
Since $0\leq x-a\leq 0.1$,  $\disp \min \eig(A_s(\x))\geq \frac{65-\sqrt{1234}}{7500}>3.9 \times10^{-3}$. 
% As a consequence, $A_s(\x)$ is defined positive for all $\x\in[a,b]$ satisfying that $\disp A_s(\x)>\frac{119}{150}\times10^{-4}\L$.}  
This implies that     for sufficiently small $c>0, $  $\z^\T(\x,t) A_s(\x)\z(\x,t)\geq c\z^\T(\x,t) \L\z(\x,t)$. 
More precisely, the eigenvalues
%\begin{strip}
	\begin{align*}
		\eig(A_s-c\L)=\begin{cases}
			\dfrac{\left(13 \pm 7 \sqrt{\left(\frac{6(x-a)}{7(1-c)}\right)^2+1}\right)(1-c)}{1500}%\geq0 & \text{if~} c\leq1-\dfrac{1}{10}\sqrt{\dfrac{3}{10}}\approx0.9452
			\\ \\
			\dfrac{\left(55 \pm 45 \sqrt{\left(\frac{20(x-a)}{9(1-c)}\right)^2+1}\right)(1-c)}{2}%\geq0 & \text{if~}c\leq 1-\sqrt{\dfrac{1}{10}}\approx0.6837 
			%\dfrac{13(1-c) \pm \sqrt{36(x-a)^2+49(1-c)^2}}{1500}\geq0 & \text{if~} c\leq1-\dfrac{1}{10}\sqrt{\dfrac{3}{10}}\approx0.9452\\ \\
			%\dfrac{55(1-c) \pm 5 \sqrt{400(x-a)^2+81(1-c)^2}}{2}\geq0 & \text{if~}c\leq 1-\sqrt{\dfrac{1}{10}}\approx0.6837 . 
		\end{cases}
	\end{align*}
	need to be non-negatives. It is easy to check, through some simple calculations, that this condition is satisfied when $c\leq 1+\sqrt{\dfrac{1}{10}}$.
	
%\end{strip}
Thus, applying Theorem \ref{thm:1} with	$\varepsilon_0=\dfrac{1}{25}$, $ \varepsilon_1=\dfrac{100k}{1250k^2+1}$ and $c=0.6837$ and choosing $\varepsilon=\dfrac{1}{50}$ we obtain that $M=3$ and $\alpha=4.5 \times 10^{-3}.$ %%$\dfrac{0.6837}{150}$.
The bound on the decay rate in this example  is not improved with an exponential multiplier function. 




Now we consider normalized physical parameters as in \cite{Mattioni2022}. That is, $\rho(\x)=\iota_{\rho}(\x)=\epsilon(\x)\iota(\x)=\kappa(\x)=\gamma=\delta=1$, boundary dissipation coefficients, $k_1=k_2=1$, and beam length $b-a=1.$  We obtain that $\eta_{K}=\eta_{\L}=\mu_{\L}=\mu_{P_1}=1$, $\mu_B=\sqrt{2}$ and $\mu_{\Psi}=2$. Then, considering a linear multiplier function, 
\begin{align*}
	A_s (\x ) =\begin{bmatrix}
		1&\x-a& \x-a&0\\
		\x-a&1&0 & \x-a\\
		\x-a&0 &1& a-\x\\
		0 & \x-a&a-\x&1
	\end{bmatrix}
\end{align*}
whose eigenvalues are $1\pm\sqrt{2}(\x-a)$. As a consequence, $A_s (\x)  >0$  only for $\x<a+\dfrac{1}{\sqrt{2}}<b$ and not in the entire interval $[a,b]$. The linear multiplier function $m(\x)=\x-a$, used with the previous set of parameters, cannot be used and it is necessary to consider another function.

Choosing an exponential multiplier function, as in Lemma \ref{lemma:1}, 
\begin{align*}
	%	\eta_{K}=&\eta_{\L}=\mu_{\L}=\mu_{P_1}=1\\
	%	\mu_B=&\sqrt{2}\\
	%	\mu_{\Psi}=&2\\
	%	\beta^*_{op}=&1+\frac{1}{\sqrt{2}}\\
	\varepsilon_0=&\varepsilon_1=\frac{e^{-(1+\sqrt{2})}}{C}
\end{align*}
Then, varying $\xi$ on \eqref{eq:M} and \eqref{eq:op_alpha}, we obtain the values of $M$ and $\alpha$ shown in Table \ref{tab:1}. 

In  \cite{Mattioni2022} a Lyapunov approach is used for the stability analysis in the port-Hamiltonian formulation of a Timoshenko beam with viscous dissipation and unitary parameters, leading to an exponential decay rate of $0.0285$ with a $M=2.783.$
	Choosing $\xi=0.4713$, from \eqref{eq:op_alpha} we also obtain $M=2.783$ and $\alpha=0.0286.$
	
\begin{table}
	\caption{Values of $\alpha$ and $M$ for different choices of $\xi$}
	\label{tab:1}
	\centering
	\begin{tabular}{ccc}
		\hline $\xi$ & $M$ & $\alpha$\\\hline
		1/3 & 2 & 0.0224\\
		0.4713& 2.783 & 0.0286\\
		1/2 & 3 & 0.0298\\
		3/5 & 4 & 0.0335\\
		2/3 & 5 & 0.0358\\\hline
	\end{tabular}
\end{table}

%\chgr{ What decay rate does the old multiplier function $x-a$ give?} {\color{blue} The multiplier function $m(\x)=\x-a$ not works for the Thimoshenko beam with unitary parameter because $A_s$ is not definite positive. }
%
%\chgr{Need an example with variable parameters...}



\section{Conclusions}

An explicit formulation in terms of physical parameters for the exponential energy decay lower bound of a class of port-Hamiltonian systems with boundary dissipation on one-dimensional spatial domains have been presented. The choice of an exponential function, $m(\x)=Ce^{\beta(\x-a)}$, leads to a conclusion that provided that the boundary dissipation $K>0$  the system is exponentially stable. Furthermore, a
a lower bound on the decay rate is obtained $\alpha . $ This result applies to systems with variable physical parameters, as illustrated by several examples. 

%Note that the exponential decay rate described in Theorem \ref{thm:1} is constrained by condition \eqref{eq:As>0}. This implies an appropriated chose of the multiplier function $m(\x)$. 
For uniform systems,  $m(\x)$ is commonly chosen as a linear function; that is $m(\x)=\x-\x_0$, where $\x_0$ is chosen  so that $m(a)\geq 0$; see for example, \cite{Komornik1994,Tucsnak2009}.
 This choice of $m(\x)$ also works for uniform  port-Hamiltonian systems \eqref{eq:PHS_1}-\eqref{eq:BC_1} with $P_0=G_0=0$, as was shown in   \cite{MoraMorrisMTNS}. However, this multiplier function does not work for all port-Hamiltonian systems with the form \eqref{eq:PHS_1}, as shown by the example of a Timoshenko beam  with parameters from \cite{Mattioni2022}. In the example of a wave equation with constant coefficients, both multiplier functions can be used, but the linear function leads to a better bound on the decay rate. The selection of a multiplier function to optimize the bound on the decay rate is an open research problem. 


\bibliographystyle{IEEEtran}
\bibliography{references}

\end{document}

\begin{IEEEbiography}[{\includegraphics[width=1in,height=1.25in,clip,keepaspectratio]{a1.png}}]{First A. Author} (M'76--SM'81--F'87) and all authors may include 
biographies. Biographies are often not included in conference-related
papers. This author became a Member (M) of IEEE in 1976, a Senior
Member (SM) in 1981, and a Fellow (F) in 1987. The first paragraph may
contain a place and/or date of birth (list place, then date). Next,
the author's educational background is listed. The degrees should be
listed with type of degree in what field, which institution, city,
state, and country, and year the degree was earned. The author's major
field of study should be lower-cased. 

The second paragraph uses the pronoun of the person (he or she) and not the 
author's last name. It lists military and work experience, including summer 
and fellowship jobs. Job titles are capitalized. The current job must have a 
location; previous positions may be listed 
without one. Information concerning previous publications may be included. 
Try not to list more than three books or published articles. The format for 
listing publishers of a book within the biography is: title of book 
(publisher name, year) similar to a reference. Current and previous research 
interests end the paragraph. The third paragraph begins with the author's 
title and last name (e.g., Dr.\ Smith, Prof.\ Jones, Mr.\ Kajor, Ms.\ Hunter). 
List any memberships in professional societies other than the IEEE. Finally, 
list any awards and work for IEEE committees and publications. If a 
photograph is provided, it should be of good quality, and 
professional-looking. Following are two examples of an author's biography.
\end{IEEEbiography}

\begin{IEEEbiography}[{\includegraphics[width=1in,height=1.25in,clip,keepaspectratio]{a2.png}}]{Second B. Author} was born in Greenwich Village, New York, NY, USA in 
1977. He received the B.S. and M.S. degrees in aerospace engineering from 
the University of Virginia, Charlottesville, in 2001 and the Ph.D. degree in 
mechanical engineering from Drexel University, Philadelphia, PA, in 2008.

From 2001 to 2004, he was a Research Assistant with the Princeton Plasma 
Physics Laboratory. Since 2009, he has been an Assistant Professor with the 
Mechanical Engineering Department, Texas A{\&}M University, College Station. 
He is the author of three books, more than 150 articles, and more than 70 
inventions. His research interests include high-pressure and high-density 
nonthermal plasma discharge processes and applications, microscale plasma 
discharges, discharges in liquids, spectroscopic diagnostics, plasma 
propulsion, and innovation plasma applications. He is an Associate Editor of 
the journal \emph{Earth, Moon, Planets}, and holds two patents. 

Dr. Author was a recipient of the International Association of Geomagnetism 
and Aeronomy Young Scientist Award for Excellence in 2008, and the IEEE 
Electromagnetic Compatibility Society Best Symposium Paper Award in 2011. 
\end{IEEEbiography}

\begin{IEEEbiography}[{\includegraphics[width=1in,height=1.25in,clip,keepaspectratio]{a3.png}}]{Third C. Author, Jr.} (M'87) received the B.S. degree in mechanical 
engineering from National Chung Cheng University, Chiayi, Taiwan, in 2004 
and the M.S. degree in mechanical engineering from National Tsing Hua 
University, Hsinchu, Taiwan, in 2006. He is currently pursuing the Ph.D. 
degree in mechanical engineering at Texas A{\&}M University, College 
Station, TX, USA.

From 2008 to 2009, he was a Research Assistant with the Institute of 
Physics, Academia Sinica, Tapei, Taiwan. His research interest includes the 
development of surface processing and biological/medical treatment 
techniques using nonthermal atmospheric pressure plasmas, fundamental study 
of plasma sources, and fabrication of micro- or nanostructured surfaces. 

Mr. Author's awards and honors include the Frew Fellowship (Australian 
Academy of Science), the I. I. Rabi Prize (APS), the European Frequency and 
Time Forum Award, the Carl Zeiss Research Award, the William F. Meggers 
Award and the Adolph Lomb Medal (OSA).
\end{IEEEbiography}

