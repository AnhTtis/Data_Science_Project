\documentclass[10pt,twocolumn,letterpaper]{article}

\usepackage{iccv}
\usepackage{times}
\usepackage{epsfig}
\usepackage{graphicx}
\usepackage{amsmath}
\usepackage{amssymb}
\usepackage{algorithm}
\usepackage{algpseudocode}
\usepackage{multirow}
\usepackage{pifont}% http://ctan.org/pkg/pifont
\usepackage{xcolor}
\usepackage{bbm}
\usepackage{bm}

% Include other packages here, before hyperref.
% theo imports 

\newcommand{\tick}{\checkmark}
% \captionsetup[table]{skip=10pt}
\newcommand{\mc}[1]{\multicolumn{1}{c}{#1}}
\newcommand{\dv}{DeepLabv$3$}
\newcommand{\hrn}{HRNet}
\newcommand{\upr}{UPerNet}
\usepackage{booktabs} % cmdirule
\usepackage{subfig}

% \usepackage{color}
% \usepackage[lofdepth,lotdepth]{subfig}

% % The "axessiblity" package can be found at: https://ctan.org/pkg/axessibility?lang=en
% \usepackage[accsupp]{axessibility}  % Improves PDF readability for those with disabilities.




% If you comment hyperref and then uncomment it, you should delete
% egpaper.aux before re-running latex.  (Or just hit 'q' on the first latex
% run, let it finish, and you should be clear).
\usepackage[pagebackref=true,breaklinks=true,letterpaper=true,colorlinks,bookmarks=false]{hyperref}

% \iccvfinalcopy % *** Uncomment this line for the final submission

\def\httilde{\mbox{\tt\raisebox{-.5ex}{\symbol{126}}}}

\iccvfinaltrue
% Pages are numbered in submission mode, and unnumbered in camera-ready
\ificcvfinal\pagestyle{empty}\fi

%%%%% GENERAL MATH COMMANDS
% Reals
\newcommand{\R}{{\mathbb R}}
% Integers
\newcommand{\Z}{{\mathbb Z}}
% Naturals
\newcommand{\N}{{\mathbb N}}
% Expectation
\DeclareMathOperator*{\E}{\mathbb{E}}
% ^th notation
\newcommand{\tth}{^{\text{th}}}
% Small dots for integer range [a .. b]
\newcommand{\sdots}{\,..\,}
% Vectorized version of matrix
\newcommand{\matvec}{\mbox{vec}}

% := sign
\newcommand{\defeq}{\vcentcolon=}
% Zero function
\newcommand{\zf}{\mathbf{0}}
% Vector of ones
\newcommand{\ones}{\mathbf{1}}

% Argmin and argmax definitions
\DeclareMathOperator*{\argmax}{arg\,max}
\DeclareMathOperator*{\argmin}{arg\,min}


%%%%% PROBLEM STATEMENT NOTATION 
% \newcommandtwoopt{\St}[2][t][]{{S_{#1}^{#2}}} % State
\newcommand{\task}[1][i]{{\mathcal{T}_{#1}}} % Task, optionally takes index
\newcommand{\tasks}{\{ \task \}_{i=1}^N}
\newcommand{\losst}[1][i]{{l_{#1}}}
\newcommand{\lossv}[1][i]{{l_{#1}^{\textrm{val}}}}
\newcommand{\tasktarget}{{\mathcal{T}_{\textrm{target}}}}
\newcommand{\lossttarget}{l_{\textrm{target}}}
\newcommand{\lossvtarget}{l_{\textrm{target}}^{\textrm{val}}}
\newcommand{\lossttargetit}{l_{\textrm{target}}^{(k)}}
\newcommand{\losstotal}{l^{\textrm{total}}}
\newcommand{\lossopt}{l^*}

\newcommand{\thetait}[2]{\theta_{#1}^{(#2)}}
\newcommand{\phit}[1]{\phi^{(#1)}}
\newcommand{\hist}[2]{S_{#1}^{(#2)}}
\newcommand{\grad}[2]{G_{#1}^{(#2)}}

\newcommand{\Alg}{\textup{\textbf{Opt}}}
\newcommand{\MetaAlg}{\textup{\textbf{MetaOpt}}}

%%%%% Theorems
\newtheoremstyle{mytheoremstyle} % name
    {\topsep}                    % Space above
    {\topsep}                    % Space below
    {\itshape}                   % Body font
    {}                           % Indent amount
    {\scshape}                   % Theorem head font
    {.}                          % Punctuation after theorem head
    {.5em}                       % Space after theorem head
    {}  % Theorem head spec (can be left empty, meaning ‘normal’)
\theoremstyle{mytheoremstyle}
\theoremstyle{plain}
\newtheorem{theorem}{Theorem}
\newtheorem{proposition}{Proposition}
\newtheorem{assumption}{Assumption}
\newtheorem{definition}{Definition}
\newtheorem{lemma}{Lemma}
\theoremstyle{remark}
\newtheorem{remark}{Remark}


\begin{document}

%%%%%%%%% TITLE
\title{Stochastic Segmentation with Conditional Categorical Diffusion Models}

\author{
Lukas Zbinden$^*$ 
\qquad 
Lars Doorenbos$^*$ 
\qquad
Theodoros Pissas\\
Raphael Sznitman 
\qquad
Pablo Márquez-Neila
\vspace{0.5em}
\\
University of Bern, Bern, Switzerland\\
{\tt\small \{lukas.zbinden,lars.doorenbos,theodoros.pissas,raphael.sznitman,pablo.marquez\}@unibe.ch}
% For a paper whose authors are all at the same institution,
% omit the following lines up until the closing ``}''.
% Additional authors and addresses can be added with ``\and'',
% just like the second author.
% To save space, use either the email address or home page, not both
% \and
% Second Author\\
% Institution2\\
% First line of institution2 address\\
% {\tt\small secondauthor@i2.org}
}

\maketitle
% Remove page # from the first page of camera-ready.
\ificcvfinal\thispagestyle{empty}\fi

\def\thefootnote{*}\footnotetext{Equal contribution} % \def\thefootnote{\arabic{footnote}}

%%%%%%%%% ABSTRACT
\begin{abstract}
Semantic segmentation has made significant progress in recent years thanks to deep neural networks, but the common objective of generating a single segmentation output that accurately matches the image's content may not be suitable for safety-critical domains such as medical diagnostics and autonomous driving. Instead, multiple possible correct segmentation maps may be required to reflect the true distribution of annotation maps. In this context, stochastic semantic segmentation methods must learn to predict conditional distributions of labels given the image, but this is challenging due to the typically multimodal distributions, high-dimensional output spaces, and limited annotation data. To address these challenges, we propose a conditional categorical diffusion model (CCDM) for semantic segmentation based on Denoising Diffusion Probabilistic Models. Our model is conditioned to the input image, enabling it to generate multiple segmentation label maps that account for the aleatoric uncertainty arising from divergent ground truth annotations. Our experimental results show that CCDM achieves state-of-the-art performance on LIDC, a stochastic semantic segmentation dataset, and outperforms established baselines on the classical segmentation dataset Cityscapes.

\end{abstract}

%%%%%%%%% BODY TEXT

\section{Introduction}
\label{sec:introduction}

Semantic segmentation has significantly progressed in recent years due to powerful deep neural networks. For most methods, the key objective is to generate a single segmentation output that accurately matches the image's content.
However, this may not be suitable for safety-critical domains such as medical diagnostics and autonomous driving, as images in these applications often suffer from inherent ambiguity or annotations that have differences in opinion. In these cases, generating a single coherent segmentation may be hopeless to fully describe the set of correct labeling.

Instead, multiple possible correct segmentation maps may be required to reflect the true distribution of annotations.
For instance, Fig.~\ref{fig:intro} illustrates the task of lung nodule segmentation from CT scans where expert annotators provide multiple valid segmentation maps. In this context, stochastic semantic segmentation methods must learn to predict conditional distributions of labels given the image. Doing so is challenging, however, as the distribution is typically multimodal, the output space is high-dimensional, and annotation data is limited.

\begin{figure}
    \centering
    \includegraphics[width=0.95\linewidth]{figures/iccv_teaser.pdf}
    \caption{Examples from the LIDC dataset, where expert radiologists were asked to annotate lung nodules. Despite their expertise, they disagree significantly on many cases. Standard segmentation networks fail to capture these variations, thereby giving a false sense of confidence in model predictions. Our approach learns the distribution of possible labels, allowing us to generate realistic and diverse segmentations.}
    \label{fig:intro}
\end{figure}

Denoising Diffusion Probabilistic Models (DDPMs) appear well-suited to overcome these challenges. DDPMs have recently drawn strong interest in computer vision as a framework for learning complex distributions in high-dimensional spaces. After achieving state-of-the-art performance on image synthesis~\cite{dmsBeatGans}, they have been successfully extended to solve tasks such as text-to-image generation~\cite{saharia2022photorealistic}, counterfactual explanation generation~\cite{jeanneret2022diffusion}, inpainting ~\cite{lugmayr2022repaint}, but also image classification~\cite{zimmermann2021score} and semantic segmentation~\cite{amit2021segdiff,baranchuk2021label,wolleb2021diffusion} amongst others. 

% Problems with DDPM for segmentation
While DDPMs were originally formulated as probabilistic models able to learn high-dimensional data distributions of discrete and ordered variables (\eg,~RGB pixel values), re-formulations and modifications that allow for categorical variables (\eg,~labels)~\cite{hoogeboom2021argmax} are one of the key reasons why DDPMs are being explored in a broad range of computer vision tasks~\cite{croitoru2022diffusion}. Specifically, the ability to model the spatial distribution of categorical variables is well suited for numerous computer vision tasks, including semantic segmentation~\cite{chen2017deeplab, chen2018encoder, chu2021twins, fu2019dual, gu2022multi, harley2017segmentation, kirillov2019panoptic, li2022deep, long2015fully, zhang2022semantic, zhao2017pyramid}. Yet until now, segmentation methods using DDPMs have relied on the original discrete and ordered formulation and different heuristics to yield categorical outputs~\cite{amit2021segdiff, baranchuk2021label, wolleb2021diffusion}. Consequently, the potential advantages of adopting diffusion models of categorical variables for stochastic image segmentation are still unknown.

In light of the above, we propose a \emph{conditional categorical diffusion model}~(CCDM)~for semantic segmentation based on DDPMs, which models both the observed and the latent variables as categorical distributions. This enables the model to explicitly generate labels maps of discrete, unordered variables, thereby circumventing the need for switching between continuous and discrete domains, as in previous methods. The model is conditioned to the input image, making it possible to generate multiple segmentation label maps that account for the aleatoric uncertainty arising from image ambiguity. 
We show experimentally that our approach achieves state-of-the-art performance on LIDC, a stochastic semantic segmentation dataset, according to several performance measures. Moreover, when applied to the classical segmentation dataset Cityscapes, our method provides competitive results, outperforming established baselines.

In summary, our main contributions are the following: 
\begin{itemize}
    \item We propose a conditional categorical diffusion model capable of learning the label distribution given an input image that can be used to produce diverse segmentation samples that capture aleatoric uncertainty.
    \item For the task of learning a multi-rater semantic segmentation label distribution, our method achieves state-of-the-art performance on LIDC, being the first diffusion-based approach proposed for this task.
    \item We report competitive performance on a challenging semantic segmentation task, Cityscapes, outperforming several established baselines using a lightweight model that also leverages an off-the-shelf pre-trained feature extractor. 
\end{itemize}
%-------------------------------%
%-------------------------------%

\section{Related Work}

\textbf{Explainable AI.} The main dividing line between the different branches of explainable artificial intelligence stands between \textit{Ad-Hoc} and \textit{Post-Hoc} methods. The former promotes architectures that are interpretable by design~\cite{rymarczyk2021interpretable,Bohle_2022_CVPR,Bohle_2021_CVPR,Huang_2020_CVPR} while the latter considers analyzing existing models as they are. Since our setup lies among the Post-Hoc explainability methods, we spotlight that this branch splits into global and local explanations. The former explains the general behavior of the classifier, as opposed to a single instance for the latter. This work belongs to the latter. There are multiple local explanations methods, from which we highlight saliency maps~\cite{Jalwana_2021_CVPR,Wang_2020_CVPR_Workshops,Lee_2021_CVPR,8354201,Kim2022HIVE,zheng2022shap}, concept attribution~\cite{pmlr-v80-kim18d,NEURIPS2019_77d2afcb,kolek2022cartoon} and model distillation~\cite{tan2018learning,Ge_2021_CVPR}. Concisely, these explanations try to shed light on \emph{how} a model took a specific decision. In contrast, we focus on the on-growing branch of counterfactual explanations, which tackles the question: \emph{what}  does the model uses for a forecast? We point out that some novel methods~\cite{vandenhende2022making,pmlr-v97-goyal19a,wang2020scout,Wang_2021_CVPR} call themselves counterfactual approaches. Yet, these systems highlight regions between a pair of images without producing any modification. 


\textbf{Counterfactual Explanations.} CE have taken momentum in recent years to explain model decisions. 
Some methods rely on prototypes~\cite{looveren2021interpretable} or deep inversion~\cite{thiagarajan2021designing}, while other works explore the benefits of other classification models for CE, such as Invertible CNNs~\cite{hvilshoj2021ecinn} and Robust Networks~\cite{boreiko2022sparse,pmlr-v130-schut21a}. A common practice is using generative tools as they give multiple benefits when producing CE. In fact, using generation techniques is helpful to generate data in the image manifold. There are two modalities to produce CE using generative approaches. Many methods use conditional generation techniques~\cite{van2021conditional,Singla2020Explanation,looveren2021interpretable} to fit what a classification model learns or how to control the perturbations. Conversely, unconditional approaches~\cite{Rodriguez_2021_ICCV,nemirovsky2020countergan,Jeanneret_2022_ACCV,shih2021GANMEXOnevsoneAttributions,zhao2018GeneratingNaturalAdversarial,Khorram_2022_CVPR} optimize the latent space vectors. 

%Among the counterfactual approaches, we draw attention to Jeanneret~\etal~\cite{Jeanneret_2022_ACCV}'s work. This method uses a modified version of the guided diffusion~\cite{Dhariwal2021DiffusionMB} to steer the generation toward the target label, \edit{modifying the DDPM generation algorithm \textit{per se}.  In contrast, even when we use DDPM, we use them as a mere regularizer before the classifier. Hence,} we use adversarial attacks directly on the image space to generate semantic changes before post-processing it through the diffusion model without relying on controlling the generation process.  Finally, unlike previous methods, we use a refinement stage to perform the pertinent editings in only the regions of interest. 

We'd like to draw attention to Jeanneret~\etal~\cite{Jeanneret_2022_ACCV}'s counterfactual approach, which uses a modified version of the guided diffusion algorithm to steer image generation towards a desired label. This modification affects the DDPM generation algorithm itself. In contrast, while we also use DDPM, we use it primarily as a regularizer before the classifier. Instead of controlling the generation process, we generate semantic changes using adversarial attacks directly on the image space, and then post-process the image using a standard diffusion model.  Furthermore, we use a refinement stage to perform targeted edits only in regions of interest.


% In contrast, even when we use DDPM, we use adversarial attacks directly on the image space to generate semantic changes before post-processing it through the diffusion model without relying on controlling the generation process.


\textbf{Adversarial Attacks and their relationship with CE.} Adversarial attacks share the same main objective as counterfactual explanations: flipping the forecast of a target architecture. On the one hand, \textit{white-box} attacks~\cite{DBLP:journals/corr/GoodfellowSS14,madry2018towards,carlini2017towards,moosavi2016deepfool,croce2020reliable,Jeanneret_2021_ICCV} leverage the gradients of the input image with respect to a loss function to construct the adversary. In addition, universal noises~\cite{moosavi2017universal} are adversarial perturbations created for fooling many different instances. On the other hand, \textit{black-box} attacks~\cite{zhou2018transferable,poursaeed2018generative,ACFH2020square} restrain their attack by checking merely the output of the model. Finally, Nie~\etal~\cite{nie2022DiffPure} study DDPMs from a robustness perspective, disregarding the benefits of counterfactual explanations. 


In the context of CE for visual models, the produced noises are indistinguishable for humans when the network does not have any defense mechanism, making them useless. This lead works~\cite{NEURIPS2019_7392ea4c,akhtar2021attack,pawelczyk2022exploring} to approach the relationship between these two research fields. Compared to previous approaches, we manage to leverage adversarial attacks to create semantic changes in undefended models to explore their semantic weaknesses perceptually in the images; a difficult task due to the nature of the data.


\section{Efficient Learning of High Level Plans from Play (\alg)}
\label{sec:method}

We now present our algorithm \alg for solving long-horizon tasks with motion primitives and play-guided RL.
To address exploration, we first learn a discrete behavioral prior that eliminates infeasible actions from the set of primitives and hence prunes the search space (\Cref{ssec:learn_prior}, Figure \ref{fig:spot_priors}).
Next, in \Cref{ssec:learn_maskedQ1} we propose and motivate an integration scheme for the learned prior onto Q-learning. Consequently, our agent can focus on learning Q-values for feasible state-action pairs only, as the prior generally lifts the burden of learning to avoid infeasible actions.


\begin{figure}[b]
\begin{center}
\vspace{-1mm}
\includegraphics[width=\columnwidth]{figures/fig2_v2_small_compressed.pdf}
\end{center}
\vspace{-2mm}
\caption{The trained behavioral prior $\pi^\beta$ learns to estimate the set of feasible primitives in different environment configurations. (Left) When the agent is at the center of the space, the prior favors primitives that involve reaching elements through a free-collision path (go to \textit{door}, \textit{drawer}s or above the \textit{object}), but prevents actions that involve object manipulation. (Right) When being close to the door, the prior learns correct object affordances such as \textit{grasp} or \textit{slide}.}
\vspace{-5mm}
\label{fig:spot_priors}
\end{figure}

\subsection{Learning a Prior from Play} \label{ssec:learn_prior}

Let us start by considering the play dataset \mbox{$\mathcal{D} = \{ (s_1, a_1),  ... (s_N, a_N) \}$} introduced in Section \ref{sec:statement}. While lacking explicit exploitative behaviors, play data inherently favors actions that are \textit{feasible}: while not necessarily desirable for a given goal, those actions are likely to be successfully executed given the current state of the environment.
We aim to extract this information by estimating a (goal-independent) behavioral prior $\pi^\beta(\cdot|s)$, modeled as a conditional categorical distributions over primitives, which associates feasible primitives to higher likelihoods.

We parameterize $\pi^\beta$ through a neural network with learnable parameters $\omega$, which can be trained via standard mini-batch first order techniques to minimize the negative log-likelihood $\mathcal{L}_{NLL}(\omega)$:
\begin{equation}
    \mathcal{L}_{NLL}(\omega) = \mathop{\mathbb{E}}_{\substack{B \sim \mathcal{D}}}\Bigg[\frac{1}{|B|} \sum_{\substack{(s, a) \in B}} -\log \pi^\beta_\omega(a|s) \Bigg],
\label{eq:prior_loss}
\end{equation}
where $B$ represents a batch of state-action pairs sampled uniformly from the dataset $\mathcal{D}$.


\subsection{Selecting Feasible Actions}
Given the learned behavioral prior $\pi_\beta(\cdot|s)$, we propose to turn its soft probability distribution into hard, binary constraints on the action space. We thus define a threshold-based selection operator $\alpha: \mathcal{S} \to \mathcal{P(A)}$:
\begin{equation}
 \alpha(s) = \{ a \in \mathcal{A} \mid \pi_\beta(a|s) > \rho\},
\label{eq:mask}
\end{equation}
where $\mathcal{P(\cdot)}$ represents a powerset.
Ideally, an action $a \not \in \alpha(s)$ would not be chosen by an optimal goal-conditioned policy in state $s$. We refer to $\alpha(s)$ as the set of \textit{feasible} actions for state $s$. See Figure \ref{fig:spot_priors} for a visualization.


\subsection{Learning in a Reduced MDP}\label{ssec:learn_maskedQ1}

The learned selection operator $\alpha(s)$ enables the definition of an auxiliary MDP $\mathcal{M'}$, which we refer to as \textit{reduced} MDP. The definition and solution of the reduced MDP lay at the core of our method.
We model $\mathcal{M'}$ through a generalized definition of MDPs \cite{puterman1994} in which available actions depend on the current state: in the state $s$, the action space is restricted to a subset $\alpha(s) \subseteq \mathcal{A}$.

\begin{definition}[Reduced MDP]
  Given an MDP \mbox{$\mathcal{M}=(\mathcal{S}, \mathcal{G}, \mathcal{A}, P, R, \rho_0, \rho_g, \gamma)$} and a selection operator $\alpha: \mathcal{S} \to \mathcal{P(A)}$ such that for all $s \in \mathcal{S}$, $\alpha(s) \neq \emptyset$, the reduced MDP $\mathcal{M'}$ is defined as the 9-tuple $(\mathcal{S}, \mathcal{G}, \mathcal{A}, P, R, \rho_0, \rho_g, \alpha, \gamma)$.
\end{definition}
\vspace{2mm}
where the assumption on $\alpha(s)$ ensures that there exist a feasible action in each state and Q-values can be well-defined.
Intuitively, $\mathcal{M'}$ encodes the same environment as $\mathcal{M}$ but restricts the set of action-state pairs.

We note that in PAC RL settings, the analysis of sample complexity \cite{kakade2003sample} (i.e. the number of steps for which a learned policy is not $\epsilon$-optimal with high probability) produces upper bounds that are directly dependent on the number of state-action pairs \cite{lattimore2012pac}.
In particular, model-free PAC-MDP algorithms can attain a sample complexity that is $\tilde O(N)$, where $N \leq |\mathcal{S}||\mathcal{A}|$ is the number of state-action pairs, and $\tilde O(\cdot)$ represents $O(\cdot)$ where logarithmic factors are ignored \cite{strehl2006pac}.
Learning in $\mathcal{M'}$ instead of $\mathcal{M}$ is thus desirable and could lead to near-linear improvements in sample efficiency as the number of infeasible actions grows.
Crucially, under mild assumptions, the optimal policy for $\mathcal{M'}$ can not only be retrieved more efficiently but also attains optimality in the original MDP $\mathcal{M}$ (see \OurAppendix \ref{app:optimality}).

We thus propose a practical modified Q-learning iteration on the original MDP $\mathcal{M}$, which is equivalent to performing Q-learning directly in the reduced MDP $\mathcal{M'}$.
Given a transition $(s, a, s', g, r)$:
\begin{equation}
    Q(s, a, g) \gets (1-\delta)Q(s,a,g) + \delta(r + \gamma \max_{a' \in \alpha(s)}Q(s', a', g)),
\end{equation}
where the value of the next-state $s'$ is only computed over feasible actions and $\delta$ is the learning rate.

Under common assumptions (i.e. infinite visitation of each state-action pair and well-behaved learning rate in tabular settings \cite{bertsekas1996neuro}), this algorithm converges to $Q_{\mathcal{M'}}^*$, from which we can easily extract $\pi_{\mathcal{M}}^*(s,g) = \pi_{\mathcal{M'}}^*(s,g) = \argmax_{a \in \alpha(s)} Q_{\mathcal{M'}}(s,a,g)$. For simplicity, we will from now on refer to  $Q_{\mathcal{M'}}$ as $Q$.

In practice, following the goal-conditioned framework \cite{schaul2015universal}, we scale this algorithm by parameterizing $Q_\theta(s,a,g)$ through a neural network.
Inspired by recent success in scaling Q-learning \cite{watkins1992q} to high-dimensional spaces \cite{van2016deep, mnih2015human, fujimoto2018addressing} while reducing overestimation bias, we learn the parameters $\theta$ of the Q-function using Clipped Double Q-learning \cite{fujimoto2018addressing}, which minimizes the temporal difference (TD) loss:
\begin{gather}
    \mathcal L (\theta_j) = \mathbb E_{\substack{(s,a,s',g,r) \sim \mathcal{B} }} \big [(y_j - Q_{\theta_j}(s_t,a_t,g))^2]\label{eq:q-loss}, \text{ with } \\
    \nonumber y_j =  r + \gamma \min_{i=1,2}  Q_{\theta'_i}(s', \argmax_{a_{t+1} \in \alpha(s_{t+1})}  Q_{\theta_j}(s_{t+1}, a_{t+1}, g), g ),
\end{gather}
where $j \in \{1,2\}$, and $\theta_j, \theta_j'$ are the parameters for Q and target Q-networks respectively and where $(s,a,s',g,r)$ tuples are sampled uniformly from an experience replay buffer \cite{lin1992reinforcement} $\mathcal{B}$ exploiting the off-policy nature of Q-learning.
We collect experience following an $\eps$-greedy exploration mechanism on the feasible action set $\alpha(s) \subseteq \mathcal{A}$. We summarize our approach in Algorithm \ref{algorithm}.

\setlength{\textfloatsep}{0mm}
\begin{algorithm}[ht]
 \caption{\alg} \label{alg:algorithm}
\begin{algorithmic}
\small
    \INPUT Trained prior $\pi^\beta_w$ , randomly initialized $Q_\theta$ and Q-target $Q_{\theta'}$ with $\theta = \theta'$, probability threshold $\rho$, learning rate $\eta$, replay buffer $\mathcal{D}=\emptyset$, soft update parameter $\mu$.%, number of training episodes N, episode length T.
    \FOR{episode$=1,\ldots$ N}
        \STATE Sample $s \sim \rho_0$, $g \in \rho_g$.
        \FOR{step$=1,\ldots T$}
        \STATE Compute feasible action set $\alpha(s_t)$ in \eqref{eq:mask}, compute $Q_\theta(s_t, a, g)$ for each $a \in \mathcal{A}$.
        \STATE With probability $\epsilon$ sample $a_t \sim \mathcal{U}\{\alpha(s_t)\}$, else select $a_t = \argmax_{a \in \alpha(s_{t+1})}Q_\theta(s_t, a, g)$.
        \STATE Execute $a_t$ and store transition $(s_t, a_t, r_t, s_{t+1}, g)$ in $\mathcal{D}$.
        \STATE Sample minibatch of transitions $(s_j, a_j, r_j, s_{j+1}, g)$ uniformly from $\mathcal{D}$.
        \STATE Compute TD loss $\mathcal L_{\mathrm{Q}} (\theta)$ in \eqref{eq:q-loss}.
        \STATE Gradient step $\theta \gets \theta - \eta \nabla \mathcal L_{\mathrm{Q}}(\theta)$.
        \STATE Perform soft-update on $\theta' \gets \mu \theta + (1-\mu) \theta' $.
        \ENDFOR
    \ENDFOR
\end{algorithmic}
\label{algorithm}
\end{algorithm}
\section{Validation Experiments}\label{sec:Validation Experiments}

We compared the mean squared error (MSE) of a \textsc{cPPAP} using the three fusion methods in \Cref{sec:Proposed Method/Contextual PPAP} in predicting the \textit{normalized ISO Pleasantness} (\textsc{isoPl}) of an augmented soundscape, with \textsc{ef}/\textsc{mf} and \textsc{lf} variants respectively predicting $\widetilde{\mu}_k = \widehat{\mu}_k$ and $\widetilde{\mu}_k = \widehat{\mu}'_k$ to obtain
\vspace{-2mm}
\begin{align}
	\textup{MSE} &= \frac{1}{K}\sum_{k}\left(y_k-\widetilde{\mu}_k\right). \label{eq:MSE}
\end{align}
\vspace{-4mm}

\noindent The \textsc{isoPl} is defined in \cite{Ooi2022ProbablyAugmentation} as a value in $[-1,1]$. As a further ablation study, we investigated the MSE for each combination of including/excluding participant and/or visual information.

For ease of reference, we denote variants with participant embeddings $\bm{h}$ included/excluded as \textsc{ip}/\textsc{ep}, and with visual embeddings $\bm{r}$ included/excluded as \textsc{iv}/\textsc{ev}. The baseline model for comparison was the \textsc{aPPAP}, corresponding to the \textsc{ep}+\textsc{ev} case, with the best-performing setup from \cite{Watcharasupat2022AutonomousGain}. This setup is detailed in \Cref{sec:Validation Experiments/Model architecture and training}. 

\vspace{-2mm}

\subsection{Dataset}\label{sec:Validation Experiments/Dataset}

We used the ARAUS dataset \cite{Ooi2022ARAUS:Soundscapes}, which contains a 5-fold cross-validation set of \num{25440} unique perceptual responses to augmented urban soundscapes presented as audio-visual stimuli.
\newpage
Corresponding information on the participants rating the stimuli was also collected via a participant information questionnaire (PIQ), consisting of basic demographic information and standard psychological questionnaires.
The \textsc{isoPl} values can be computed from each unique response and were used as the target observations for our validation experiments. The base soundscapes $\bm{s}$ and accompanying images $\bm{b}$ in the ARAUS dataset were drawn from the Urban Soundscapes of the World database \cite{DeCoensel2017UrbanMind}, with images extracted from the \SI{0}{\degree}-azimuth, \SI{0}{\degree}-elevation field of view (FoV) of the 30-second video captured at the same time as the 30-second soundscape recordings. When presented to the participants, the audio-visual stimuli in the ARAUS dataset used the entirety of the 30-second video at the same FoV, synchronized to the audio. For this study, we take a random frame from the 30-second video and downsample the frame via bilinear interpolation to obtain a standard image dimension of $(H,W,C_{\text{v}}) = (240,135,3)$ to be used as raw visual input to the \textsc{cPPAP}. This corresponds to a video frame rate of $\frac{\text{1}}{\text{30}}$ \SI{}{\hertz}. More frames, or the entirety of the video, could be used to extract a time series of visual embeddings corresponding to a higher frame rate and temporal relations between them could be explored in future work.

The 30-second maskers $\bm{m}$ were drawn from the Freesound and xeno-canto repositories, and calibrated as in \cite{Ooi2021AutomationHead} to obtain accurate log-gain values $\gamma$ if non-silent. If the masker was silent, the information in $\gamma$ was irrelevant, so we drew $\gamma$ from $\mathcal{N}(\nu,\zeta^2)$, where $\nu$ and $\zeta$ are the mean and standard deviation of the log-gains of the training set samples with non-silent maskers. This prevented the trained models from varying predictions at inference time according to $\gamma$ despite the soundscape (and hence ground-truth label) staying constant regardless of the value of $\gamma$ when the masker was silent. 

To obtain the coded participant information $\bm{p}$, we normalized all PIQ responses in the ARAUS dataset to the range $[0,1]$ if they corresponded to continuous variables (e.g., age), and converted them to binary dummy variables in $\{0,1\}$ if they corresponded to unordered categorical variables (e.g., dwelling type).

As an initial study, only a subset of PIQ items was selected. This was done by using the normalized PIQ responses as additional predictor variables to the elastic net \textsc{isoPl} model in \cite{Ooi2022ARAUS:Soundscapes}. Only variables with regression coefficients significantly different from zero ($p < 0.05$) were selected for use in the \textsc{cPPAP}. There were $M=5$ such participant-linked variables: their highest education attained, whether their main residence was landed property; their satisfaction of the overall acoustic environment in Singapore; their score on a modified Weinsten Noise Sensitivity Scale \cite{Weinstein1978}; and their Positive Affect score on the Positive and Negative Affect Schedule \cite{Watson1988}.

\subsection{Model architecture and training}\label{sec:Validation Experiments/Model architecture and training}

For the \textsc{aPPAP}, the audio feature extractors $f_{\text{s}}$ and $f_{\text{m}}$ comprise 5 convolutional blocks, each with a 3-by-3 convolutional layer, batch normalization, dropout, swish activation, and 2-by-2 average pooling. The numbers of filters in each block are 16, 32, 48, 64, 64, respectively. The spectrogram parameters are $T=644$, $F=64$, and $C_{\text{s}} = 2$, so the audio feature extractors give embeddings with dimension $N=20$ and $D=128$. The attention block $f_{\text{a}}$ uses dot-product attention \cite{Luong2015EffectiveTranslation} and the output block $f_{\text{o}}$ consists of 3 dense layers in sequence, with the first two having 128 units and swish activation, and the last having 2 units and linear activation.

For the \textsc{cPPAP}, the visual feature extractor $f_{\text{v}}$ has the same 5 convolutional blocks as $f_{\text{s}}$ and $f_{\text{m}}$, but pooling is performed using square grids of width 2, 2, 2, 3, and 5, respectively, such that the visual embeddings also have dimension $D = 128$. The participant feature extractor $f_{\text{p}}$ comprises a single dense layer with 128 units and swish activation. The output adapter $A_\text{o}$ consists of 3 dense layers in sequence, with the first two having $2^{\lfloor \log_2(M) \rfloor + 1} = 8$ units and swish activation, and the last having 2 units and linear activation.

All model types in the validation experiments were trained under a 5-fold cross-validation scheme with the same 10 seeds for each validation fold, for a total of 50 runs per model type. Each model was trained for up to 100 epochs using an Adam optimizer with a learning rate of \num{e-4}. In the \textsc{ep} and \textsc{ev} scenarios for the ablation study, we set $\bm{h}$ and $\bm{r}$ respectively to zero vectors. For the \textsc{aPPAP} (\textsc{ep}+\textsc{ev} scenario), both $\bm{h}$ \textit{and} $\bm{r}$ are set to zero vectors.
\section{Discussion and Future Work} \label{sec:discussion}
\label{sec:conclusion}

We present \alg, a method that bridges motion planning and deep RL to achieve complex long-horizon manipulation tasks. 
We show that by integrating a discrete behavioral prior learned from easily collectable play data, we can achieve significant gains in sample efficiency compared to other baselines that leverage prior data.
This approach has the added benefit of largely avoiding infeasible actions during training.
By planning in a two-level hierarchy, we show how our method allows reasoning over long-horizons in a mixed decision space in an efficient manner. We finally demonstrate that within this framework, \alg can be easily transferred to physical hardware without further modifications, showing the potential of combining readily available motion planners with sample efficient RL algorithms. 

Despite showing promising results, our method assumes full observability of the state space and perfect execution of the motion primitives. These limitations could be addressed by introducing perception and by querying \alg at higher frequencies at inference time to ensure primitive completion.
 Furthermore, choosing a suitable level of abstractions for skills remains an open question, whose answer could relax the need for providing a predefined set of skills, while still maintaining a low-dimensional parametrization.
Future work may also actively learn the behavioral prior instead of leveraging a static play dataset, reaching a compromise between sample complexity and reliance on collected data.

We expect our work to enable future research directions such as a tighter coupling between the training of the high-level planner and the execution of motion primitives.
Although introducing motion planning in the training loop is time consuming, we believe that the significant gains in sample efficiency demonstrated in this work can help address this challenge.




%------------------------------------------------------------------------
%------------------------------------------------------------------------

%%%%%%%%% REFERENCES

{\small
\bibliographystyle{ieee_fullname}
\bibliography{egbib}
}

\clearpage

\section{Supplementary material}

\subsection{Metrics details}
The GED and HM-IoU metrics used in our work are computed as follows:

{\bf GED:} Let~$p_m$ be the distribution over samples generated by a model and $p_{gt}$ the distribution over possible ground-truth labels; the GED is computed as
\begin{align}
    \label{eq:ged}
    \text{GED}(p_m, p_{gt}) = &2\EX_{s\sim p_m,\hat{s}\sim p_{gt}}[d(s,\hat{s})] - \EX_{s,\hat{s}\sim p_{gt}}[d(s,\hat{s})] \nonumber\\
    & - \EX_{s,\hat{s}\sim p_m}[d(s,\hat{s})],
\end{align}
where the distance function $d(\cdot,\cdot) = 1 - \text{IoU}(\cdot,\cdot)$. 

{\bf HM-IoU:} Finds the optimal matching between ground truth and generated samples. Specifically, for $n$ generated samples, the ground-truth samples are duplicated to $n$. Then, the HM-IoU is defined as the maximum IoU possible, given that every generated sample is matched with a unique ground-truth label, found by minimizing
\begin{equation}
    \text{HM-IoU} = \min_X\sum_i\sum_j d(i,j)X_{i,j},
\end{equation}
where $X$ is a boolean matrix that assigns every row to a unique column using $d(\cdot,\cdot) = 1 - \text{IoU}(\cdot,\cdot)$.

\subsection{Sample diversity}

Sample diversity is the expected distance between generated samples, \ie,~$\EX_{s,\hat{s}\sim p_m}[d(s,\hat{s})]$, which corresponds to the last term of GED in Eq.~\eqref{eq:ged}.
We report the sample diversity for 16, 32, 50, and 100 samples for both LIDC splits in Tab.~\ref{tab:diversity1} and Tab.~\ref{tab:diversity2}. 

\begin{table}[ht]
\begin{center}
\resizebox{0.5\textwidth}{!}{
\begin{tabular}{l|cccc}
\toprule
 & \multicolumn{4}{c}{\textbf{LIDCv1}}\\
\textbf{Method} & Div$_{16}$ & Div$_{32}$ & Div$_{50}$ & Div$_{100}$\\
\hline
CCDM & 0.491\tiny{$\pm$0.001} & 0.509\tiny{$\pm$0.001} & 0.515\tiny{$\pm$0.002} & 0.519\tiny{$\pm$0.002} \\
\bottomrule
\end{tabular}
%}  
}
\end{center}
\caption{Sample diversity for our method on LIDCv1.}
\label{tab:diversity1}
\end{table}

\begin{table}[ht]
\begin{center}
\resizebox{0.5\textwidth}{!}{
\begin{tabular}{l|cccc}
\toprule
 & \multicolumn{4}{c}{\textbf{LIDCv2}} \\
\textbf{Method} & Div$_{16}$ & Div$_{32}$ & Div$_{50}$ & Div$_{100}$ \\
\hline
CCDM & 0.487\tiny{$\pm$0.003} & 0.503\tiny{$\pm$0.003} & 0.509\tiny{$\pm$0.003} & 0.515\tiny{$\pm$0.002} \\
\bottomrule
\end{tabular}
%}  
}
\end{center}
\caption{Sample diversity for our method on LIDCv2.}
\label{tab:diversity2}
\end{table}

\subsection{Model size}
While our 9M CCDM as reported in Tab.~\ref{tab:lidc} is of comparable size to most other baselines, we show in Tab.~\ref{tab:lidc_size} that by increasing the size of our CCDM from 9M to 41M, we get an increase in performance across all six metrics computed on LIDCv1. Additionally, the CCDM seems to benefit more from the increase in size than MoSE~\cite{Gao2022-zt}. While we already outperform the other baselines with our 9M model, this result suggests that we can improve the performance even further by using larger models.
\begin{table}[ht]
\begin{center}
\resizebox{0.5\textwidth}{!}{
\begin{tabular}{lc|cccccc}
\toprule
 & \multicolumn{4}{c}{\textbf{LIDCv1}}\\
\textbf{Method} & \#params & GED$_{16}$ & GED$_{32}$ & GED$_{50}$ & GED$_{100}$ & HM-IoU$_{16}$ & HM-IoU$_{32}$\\
\hline
MoSE~\cite{Gao2022-zt} & 9m & 0.219 & - & 0.195 & 0.190 & 0.620 & - \\
MoSE~\cite{Gao2022-zt} & 42m & 0.218 & - & 0.195 & 0.189 & 0.624 & - \\
\midrule
CCDM & 9m & 0.212 & 0.194 & 0.187 & 0.183 & 0.623 & 0.631\\
CCDM & 41m & 0.207 & 0.189 & 0.182 & 0.177 & 0.629 & 0.636\\
\bottomrule
\end{tabular}
%}  
}
\end{center}
\caption{Performance of CCDM and MoSE on LIDCv1 with different model sizes.}
\label{tab:lidc_size}
\end{table}

\subsection{Training settings of baselines on Cityscapes}

On Cityscapes, all baselines were trained for $500$~epochs using the optimizer, learning rate schedule, and weight decay (denoted by~$w_d$) reported in their original publications. Tab.~\ref{tab:settings_cts} details these settings for each case. All models are trained using a cross-entropy loss.

\begin{table}[htb]
\centering
\resizebox{0.5\textwidth}{!}{
\begin{tabular}{*7c}
    \toprule
    \multicolumn{2}{c}{Method}  &  \multicolumn{5}{c}{\textbf{Settings}}                     \\
    \cmidrule(r){1-2}               \cmidrule(l){3-7}
          Arch.   & Backbone                & Lr               & Decay        & $w_{d}$          & Batch Size    & Optim \\
    \midrule
    \mc{\hrn}\cite{HRNet}     & \mc{w$48$v$2$}          & $10^{-2}$        & polynomial   & $5\times10^{-5}$ & \mc{$32$}    & sgd \\ 
    \mc{\dv}\cite{DeepLabv3}      & \mc{ResNet$50/101$}     & $10^{-2}$        & polynomial   & $5\times10^{-5}$ & \mc{$32$}    & sgd \\ 
    \mc{UPerNet}\cite{UPerNet}  & \mc{ResNet$101$}        & $10^{-2}$        & polynomial   & $5\times10^{-5}$ & \mc{$32$}    & sgd \\
    \mc{UPerNet}\cite{Swin}  & \mc{Swin-T}             & $10^{-4}$        & warmup+linear       & $10^{-2}$        & \mc{$32$}    & AdamW \\
    % \mc{UNet}     & \mc{-}                         & $10^{-4}$        & linear       & \mc{-}           & \mc{$16/32$} & Adam \\
    % \mc{UNet}     & \mc{Dino-ViT-S} ($\dagger$)    & $10^{-4}$        & linear       & \mc{-}           & \mc{$16/32$} & Adam \\
    \bottomrule
\end{tabular}
} % resizebox  
\caption{Training settings of baselines on Cityscapes.
}
\label{tab:settings_cts} 
\end{table}

\subsection{Additional comparisons on Cityscapes}

\begin{table}[ht]
% \centering
\resizebox{80mm}{!}{
\begin{tabular}{*5c}
    \toprule
    \multicolumn{3}{c}{\textbf{Method}}             & \multicolumn{2}{c}{mIoU} \\
    \cmidrule(r){1-3}                                 \cmidrule(l){4-5}         
    Architecture       & \mc{Backbone}   & \#params     & \mc{$128\times256$}    &  \mc{$256\times512$}\\
    \midrule
    \mc{UNet} (CE) \cite{dmsBeatGans} & \mc{-}                      & 30m                & 48.7       &  61.0  \\ 


    \midrule
    \mc{CCDM} (ours) & \mc{-}     &  \quad          &  \quad        & \quad   \\  
    
    samples=1 & \mc{}             & 30m           &  53.2         & 60.3     \\  
    samples=5 & \mc{}             & 30m            &  55.4           & 62.0   \\  
    samples=10 & \mc{}            & 30m            & 56.2          &  62.4    \\  
  
    \midrule
    \mc{UNet (CE)} \cite{dmsBeatGans} & \mc{Dino ViT-S} ($\dagger$) & 30m + \textcolor{gray}{20M}                & 53.4       &  63.2   \\ 
    \midrule

    \mc{CCDM (ours)} & \mc{Dino ViT-S} ($\dagger$)             & \quad           &  \quad        & \quad   \\  
    samples=1 & \mc{}               & 30m + \textcolor{gray}{20M}           &  55.5         &  64.0     \\  
    samples=5 & \mc{}               & 30m + \textcolor{gray}{20M}           & \underline{56.9}           &  \underline{65.4}   \\  
    samples=10 & \mc{}              & 30m + \textcolor{gray}{20M}           & \textbf{57.3}          &  \textbf{65.8}    \\  
    \bottomrule
    
\end{tabular}}

\caption{Comparison of our method to UNet and UNet-Dino, trained with standard Cross-Entropy (CE) loss, on Cityscapes-val. \textbf{Bold} and \underline{underlined} indicate best and second best per column, respectively. ($\dagger$) indicates self-supervised pretraining of the backbone. \textcolor{gray}{Gray} indicates pretrained, non-finetuned parameters.}
\label{tab:cts_unet}
\end{table}

\begin{figure*}[]
\centering
\includegraphics[width=0.99\textwidth]{figures/fig_cts_qualitative_supp.png}
\caption[]{Qualitative comparisons of our method to competitive baselines on Cityscapes validation set.
}
\label{fig:quali_cts_2}
\end{figure*}

\begin{figure*}[h!]
\centering
\includegraphics[width=0.99\textwidth]{figures/fig_cts_forward_process_2.png}
\caption[]{Visualization of the forward diffusion process at different time steps.}
\label{fig:quali_cts_forward}
\end{figure*}

We evaluate the gains of CCDMs with respect to their backbone architectures when used as standalone segmentation models. To this end, we compare the performance of our CCDM trained as defined in Alg.~\ref{alg:training} and the UNet trained with a standard cross-entropy loss, both on the Cityscapes dataset. Similarly, we compare CCDM-Dino to its standalone backbone architecture DinoViT-S. In all cases, we adopt the same training settings as our method, namely, $800$~epochs, linearly decayed learning rate, batch size of~$32$ at $128\times256$ and $16$~at $256\times512$. As shown in Tab.~\ref{tab:cts_unet}, CCDM and CCDM-Dino outperform their respective standalone architectures.

We also provide additional qualitative comparisons of our method to competitive baselines in Fig.~\ref{fig:quali_cts_2}. Finally, Fig.~\ref{fig:quali_cts_forward} shows an example of the evolution of a Cityscapes label map under the forward diffusion process described by Eq.~\eqref{eq:q_xt_given_xt-1}.



\end{document}