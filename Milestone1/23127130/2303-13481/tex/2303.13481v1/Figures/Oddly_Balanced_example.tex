%---------------------------------

\begin{figure}
    \centering
  \begin{tikzpicture}

  %begins top row part
  \begin{scope}[scale=0.7]
  
  % link diagram
      \begin{scope}[scale=1, >=Stealth, xshift=-3.5cm, rotate around={-90:(-2.5,0)}]
       \node (a) at (-4,0){};
       \node (b) at (-4,1){};
       \node (c) at (-3,1) {};
       \node (d) at (-2,1){};
       %g=d
       \node (g) at (-2,1) {};
       \node (h) at (-1,1){};
       \node (i) at (-1,0) {};
       \node (j) at (-1,-1){};
       \node (k) at (-2,-1) {};
       %k=n
       \node (n) at (-2,-1){};
       \node (o) at (-3,-1) {};
       \node (p) at (-4,-1){};

       % middle 
       \node (middle) at (-2.5,0){};

       %OUTER ONES-----------------------

       %top row
       \draw [out=45, in=135, relative, decoration={markings, mark=at position 0.5 with {\arrow{}}}, postaction={decorate}] 
       (b.center) to (c);
       \draw [out=45, in=135, relative, decoration={markings, mark=at position 0.5 with {\arrow{}}}, postaction={decorate}] 
       (c.center) to (d);
       \draw [ out=60, in=135, relative, decoration={markings, mark=at position 0.5 with {\arrow{}}}, postaction={decorate}] 
       (g.center) to (h.center);

       %right side
       \draw [out=45, in=135, relative, decoration={markings, mark=at position 0.5 with {\arrow{>}}}, postaction={decorate}] 
       (h.center) to (i);
       \draw [out=45, in=135, relative, decoration={markings, mark=at position 0.5 with {\arrow{}}}, postaction={decorate}] 
       (i.center) to (j.center);

       %bottom row
       \draw [out=45, in=135, relative, decoration={markings, mark=at position 0.5 with {\arrow{<}}}, postaction={decorate}] 
       (j.center) to (k);
       \draw [out=45, in=135, relative, decoration={markings, mark=at position 0.5 with {\arrow{}}}, postaction={decorate}] 
       (n.center) to (o);
       \draw [out=60, in=135, relative, decoration={markings, mark=at position 0.5 with {\arrow{<}}}, postaction={decorate}] 
       (o.center) to (p.center);

       %left side
       \draw [out=45, in=135, relative, decoration={markings, mark=at position 0.5 with {\arrow{}}}, postaction={decorate}] 
       (p.center) to (a);
       \draw [out=45, in=135, relative, decoration={markings, mark=at position 0.5 with {\arrow{>}}}, postaction={decorate}] 
       (a.center) to (b.center);
  

    %END OUTER ----
    % INNER ONES--------

        %left side
        \draw [decoration={markings, mark=at position 0.5 with {\arrow{>}}}, postaction={decorate}]
       (o) to (a.center);
       \draw [decoration={markings, mark=at position 0.5 with {\arrow{<}}}, postaction={decorate}]
       (a) to (c.center);

       % top row ------    

       %right side
       \draw [decoration={markings, mark=at position 0.5 with {\arrow{<}}}, postaction={decorate}]
       (g) to (i.center);
       \draw [ decoration={markings, mark=at position 0.5 with {\arrow{>}}}, postaction={decorate}]
       (i) to (k.center);
       
       % bottom row

       %middle X
       \draw [decoration={markings, mark=at position 0.5 with {\arrow{<}}}, postaction={decorate}]
       (d.center) to (middle);
       \draw []
       (middle) to (o.center);
       \draw []
       (c) to (middle.center);
       \draw [decoration={markings, mark=at position 0.5 with {\arrow{>}}}, postaction={decorate}]
       (middle.center) to (n);

       %END INNNERS --------
       
       \end{scope}
    %end link diagram

% A-circles
      \begin{scope}[scale=1, >=Stealth, rotate around={-90:(-2.5,0)}]
       \node (a) at (-4,0){};
       \node (b) at (-4,1){};
       \node (c) at (-3,1) {};
       \node (d) at (-2,1){};
       %g=d
       \node (g) at (-2,1) {};
       \node (h) at (-1,1){};
       \node (i) at (-1,0) {};
       \node (j) at (-1,-1){};
       \node (k) at (-2,-1) {};
       %k=n
       \node (n) at (-2,-1){};
       \node (o) at (-3,-1) {};
       \node (p) at (-4,-1){};

       % middle 
       \node (middle) at (-2.5,0){};

       %OUTER ONES-----------------------

       %top row
       \draw [magenta, thick, out=45, in=135, relative, decoration={markings, mark=at position 0.5 with {\arrow{}}}, postaction={decorate}] 
       (b.center) to (c);
       \draw [cyan,thick, out=45, in=135, relative, decoration={markings, mark=at position 0.5 with {\arrow{}}}, postaction={decorate}] 
       (c.center) to (d);
       \draw [Periwinkle, thick, out=60, in=135, relative, decoration={markings, mark=at position 0.5 with {\arrow{}}}, postaction={decorate}] 
       (g.center) to (h.center);

       %right side
       \draw [Periwinkle, thick, out=45, in=135, relative, decoration={markings, mark=at position 0.5 with {\arrow{}}}, postaction={decorate}] 
       (h.center) to (i);
       \draw [Emerald, thick, out=45, in=135, relative, decoration={markings, mark=at position 0.5 with {\arrow{}}}, postaction={decorate}] 
       (i.center) to (j.center);

       %bottom row
       \draw [Emerald, thick, out=45, in=135, relative, decoration={markings, mark=at position 0.5 with {\arrow{}}}, postaction={decorate}] 
       (j.center) to (k);
       \draw [SpringGreen, thick, out=45, in=135, relative, decoration={markings, mark=at position 0.5 with {\arrow{}}}, postaction={decorate}] 
       (n.center) to (o);
       \draw [RoyalBlue, thick, out=60, in=135, relative, decoration={markings, mark=at position 0.5 with {\arrow{}}}, postaction={decorate}] 
       (o.center) to (p.center);

       %left side
       \draw [RoyalBlue, thick, out=45, in=135, relative, decoration={markings, mark=at position 0.5 with {\arrow{}}}, postaction={decorate}] 
       (p.center) to (a);
       \draw [magenta, thick, out=45, in=135, relative, decoration={markings, mark=at position 0.5 with {\arrow{}}}, postaction={decorate}] 
       (a.center) to (b.center);
  

    %END OUTER ----
    % INNER ONES--------

        %left side
        \draw [RoyalBlue, thick, decoration={markings, mark=at position 0.5 with {\arrow{}}}, postaction={decorate}]
       (o) to (a.center);
       \draw [magenta, thick, decoration={markings, mark=at position 0.5 with {\arrow{}}}, postaction={decorate}]
       (a) to (c.center);

       % top row ------    

       %right side
       \draw [Periwinkle, thick, decoration={markings, mark=at position 0.5 with {\arrow{}}}, postaction={decorate}]
       (g) to (i.center);
       \draw [Emerald, thick, decoration={markings, mark=at position 0.5 with {\arrow{}}}, postaction={decorate}]
       (i) to (k.center);
       
       % bottom row

       %middle X
       \draw [cyan, thick, decoration={markings, mark=at position 0.5 with {\arrow{}}}, postaction={decorate}]
       (d.center) to (middle);
       \draw [SpringGreen, thick]
       (middle) to (o.center);
       \draw [cyan, thick]
       (c) to (middle.center);
       \draw [SpringGreen, thick, decoration={markings, mark=at position 0.5 with {\arrow{}}}, postaction={decorate}]
       (middle.center) to (n);

       %END INNNERS --------
       
       \end{scope}
    %end A-circles

% %A-circles --- OLD VERSION
%   \begin{scope}[scale=1, >=Stealth, rotate around={-90:(-2.5,0)}]
%        \node (a) at (-4,0){};
%        \node (b) at (-4,1){};
%        \node (c) at (-3,1) {};
%        \node (d) at (-2,1){};
%        %g=d
%        \node (g) at (-2,1) {};
%        \node (h) at (-1,1){};
%        \node (i) at (-1,0) {};
%        \node (j) at (-1,-1){};
%        \node (k) at (-2,-1) {};
%        %k=n
%        \node (n) at (-2,-1){};
%        \node (o) at (-3,-1) {};
%        \node (p) at (-4,-1){};

%        %EXTRA MIDDLE NODE: 
%        \node (middle) at (-2.5,0){};

%        % EXTRA PLACEMENT NOTES FOR HIGHLIGHTING A CIRCLE SPLITTING ETC
%         \node (toprow1) at (-4.4,1){};
%         \node (toprow2) at (-0.6, 1){};
%         \node (middlerow1) at (-4.4,0){};
%         \node (middlerow2) at (-0.6,0){};
%         \node (bottomrow1) at (-4.4,-1){};
%         \node (bottomrow2) at (-0.6,-1){};
%         \node (leftcol1) at (-4,1.5){};
%         \node (leftcol2) at (-4,-1.5){};
%         \node (midlcol1) at (-3,1.5){};
%         \node (midlcol2) at (-3, -1.5){};
%         \node (midrcol1) at (-2,1.5){};
%         \node (midrcol2) at (-2,-1.5){};
%         \node (rightcol1) at (-1,1.5){};
%         \node (rightcol2) at (-1,-1.5){};
        

% %begin drawing the colors of A-circles
% \begin{scope}
%        %OUTER ONES-----------------------
       

%        %top row
%        \draw [out=45, in=135, relative, draw=magenta, double=magenta, ultra thick, cap=round] 
%        (b.center) to (c.center);
%        \draw [out=45, in=135, relative, draw=Cerulean!50, double=Cerulean!50, ultra thick, cap=round] 
%        (c.center) to (d.center);
%        \draw [out=60, in=135, relative, draw=Periwinkle!60, double=Periwinkle!60, ultra thick, cap=round] 
%        (g.center) to (h.center);

%        %right side
%        \draw [out=45, in=135, relative, draw=Periwinkle!60, double=Periwinkle!60, ultra thick, cap=round] 
%        (h.center) to (i.center);
%        \draw [out=45, in=135, relative, draw=Emerald, double=Emerald, ultra thick, cap=round] 
%        (i.center) to (j.center);

%        %bottom row
%        \draw [out=45, in=135, relative, draw=Emerald, double=Emerald, ultra thick, cap=round] 
%        (j.center) to (k.center);
%        \draw [out=45, in=135, relative, draw=SpringGreen!90, double=SpringGreen!90, ultra thick, cap=round] 
%        (n.center) to (o.center);
%        \draw [out=60, in=135, relative, draw=RoyalBlue, double=RoyalBlue, ultra thick, cap=round] 
%        (o.center) to (p.center);

%        %left side
%        \draw [out=45, in=135, relative, draw=RoyalBlue, double=RoyalBlue, ultra thick, cap=round] 
%        (p.center) to (a.center);
%        \draw [out=45, in=135, relative, draw=magenta, double=magenta, ultra thick, cap=round] 
%        (a.center) to (b.center);
  

%     %END OUTER ----
%     % INNER ONES--------

%         %left side
%         \draw [draw=magenta, double=magenta, ultra thick, cap=round]
%        (a.center) to (c.center);
%         \draw [draw=RoyalBlue, double=RoyalBlue, ultra thick, cap=round]
%        (o.center) to (a.center);
       

%        % top row ------    

%        %right side
       
%        \draw [draw=Emerald, double=Emerald, ultra thick, cap=round]
%        (i.center) to (k.center);
%        \draw [draw=Periwinkle!60, double=Periwinkle!60, ultra thick, cap=round]
%        (g.center) to (i.center);
       
%        % bottom row

%        %middle X
%        \draw [draw=Cerulean!50, double=Cerulean!50, ultra thick, cap=round]
%        (d.center) to (middle.center);
%        \draw [draw=SpringGreen!90, double=SpringGreen!90, ultra thick, cap=round]
%        (middle.center) to (o.center);
%        \draw [draw=Cerulean!50, double=Cerulean!50, ultra thick, cap=round]
%        (c.center) to (middle.center);
%        \draw [draw=SpringGreen!90, double=SpringGreen!90, ultra thick, cap=round]
%        (middle.center) to (n.center);

%        %END INNNERS --------

%        %DRAW THE LINES TO EMPHASIZE A-CIRCLES
%        \draw[white, double=white,   thick, white]
%        %(toprow1) -- (toprow2)
%        (middlerow1)--(middlerow2)
%        %(bottomrow1)--(bottomrow2)
%        %(leftcol1)--(leftcol2)
%        (midlcol1)--(midlcol2)
%        (midrcol1)--(midrcol2);
%        %(rightcol1)--(rightcol2);
%     \end{scope}
%     %end actually drawing hte A-circle colors
         
%        \end{scope}
%     %end A-circles--- OLD VERSION

%A-state graph
 \begin{scope}[xshift=0cm, thick, scale = 0.7, rotate around={-90:(0,0)}]

\node (a) at (-2,2){};
\node (b) at (0,2){};
\node (c) at (2,2){};
\node (d) at (2,0){};
\node (e) at (0,0){};
\node (f) at (-2,0){};

\draw[]
(a) -- (c) -- (d) -- (f) -- (a)
(b) -- (e);

\filldraw[color=magenta, fill=magenta!5,   thick]
(a) circle (1/3);
\filldraw[color=Cerulean, fill=Cerulean!5,   thick]
(b) circle (1/3);
\filldraw[color=Periwinkle!90, fill=Periwinkle!5,   thick]
(c) circle (1/3);
\filldraw[color=Emerald, fill=Emerald!5,   thick]
(d) circle (1/3);
\filldraw[color=SpringGreen!90, fill=SpringGreen!5,   thick]
(e) circle (1/3);
\filldraw[color=RoyalBlue, fill=RoyalBlue!5,   thick]
(f) circle (1/3);


    
\end{scope}
    %end A-state graph 

% reduced $A$-state graph 
    \begin{scope}[xshift=2.5cm, thick, scale = 0.7, rotate around={-90:(0,0)}]

\node (a) at (-2,2){};
\node (b) at (0,2){};
\node (c) at (2,2){};
\node (d) at (2,0){};
\node (e) at (0,0){};
\node (f) at (-2,0){};

\draw[]
(a) -- (c) -- (d) -- (f) -- (a)
(b) -- (e);

\filldraw[color=magenta, fill=magenta!5,   thick]
(a) circle (1/3);
\filldraw[color=Cerulean, fill=Cerulean!5,   thick]
(b) circle (1/3);
\filldraw[color=Periwinkle!90, fill=Periwinkle!5,   thick]
(c) circle (1/3);
\filldraw[color=Emerald, fill=Emerald!5,   thick]
(d) circle (1/3);
\filldraw[color=SpringGreen!90, fill=SpringGreen!5,   thick]
(e) circle (1/3);
\filldraw[color=RoyalBlue, fill=RoyalBlue!5,   thick]
(f) circle (1/3);


    
\end{scope}
%ends reduced $A$-state graph 

\end{scope}
%ends top row part 

  \end{tikzpicture}
    \caption{An Oddly Balanced diagram  of type $2$, its $A$-circles, $A$-state graph, and its reduced $A$-state graph}
    \label{fig:Oddly Balanced type 2 example}
\end{figure}

%-----------------------------------