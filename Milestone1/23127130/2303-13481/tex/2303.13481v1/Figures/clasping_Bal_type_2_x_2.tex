\begin{figure}[h]
    \centering
  \begin{tikzpicture}
      \begin{scope}[scale=0.5, >=Stealth]
       
  %      % A-state graph part 
% FIRST 
  \begin{scope}[scale = 0.6, xshift=-12cm]
  
       \node (a) at (-5,0){};
       \node (b) at (0,4){};
       \node (c) at (0,2) {};
       \node (d) at (0,0){};
       \node (e) at (0, -2) {};
       \node (f) at (0, -4) {};
       \node (g) at (5, 0) {};
        %------edges
       \draw[]
        % (c) -- (f)
        % (b) -- (g)
        (b) -- (g)
        (a) -- (b)
        (b) -- (d) 
        (d) -- (f);
        \draw[dashed]
        (a) -- (f)
        (g) -- (f);
        % \draw[out=20, in=160, relative]
        % (b) to (c)
        % (c) to (d)
        % (d) to (a)
        % (d) to (e)
        % (e) to (c)
        % (c) to (e)
        % (f) to (b)
        % (b) to (f);
        
        %---------circles
        \filldraw[color=cyan, fill=cyan!5] 
        (a) circle (1/3)
        (b) circle (1/3)
       % (c) circle (1/3)
        (d) circle (1/3)       
       % (e) circle (1/3)
        (f) circle (1/3)
        (g) circle (1/3); 
        
       \end{scope}
          % end A-state graph part 
          

          
 % SECOND
 \begin{scope}[scale = 0.6, xshift=0cm]
  
       \node (a) at (-5,0){};
       \node (b) at (0,4){};
       \node (c) at (0,2) {};
       \node (d) at (0,0){};
       \node (e) at (0, -2) {};
       \node (f) at (0, -4) {};
       \node (g) at (5, 0) {};
        %------edges
        \draw[]
        % (c) -- (f)
        % (b) -- (g)
        (b) -- (g)
        (a) -- (b)
        (b) -- (d) 
        (d) -- (f);
        \draw[dashed]
        (a) -- (f)
        (g) -- (f);
        % \draw[out=20, in=160, relative]
        % (b) to (c)
        % (c) to (d)
        % (d) to (a)
        % (d) to (e)
        % (e) to (c)
        % (c) to (e)
        % (f) to (b)
        % (b) to (f);
        \draw[out=20, in = 160, relative, thick, orange]{}
        (b) to (f);
        
        %---------circles
        \filldraw[color=cyan, fill=cyan!5] 
        (a) circle (1/3)
        (b) circle (1/3)
       % (c) circle (1/3)
        (d) circle (1/3)       
       % (e) circle (1/3)
        (f) circle (1/3)
        (g) circle (1/3); 
        
       \end{scope}


% THIRD 
 \begin{scope}[scale = 0.6, xshift=12cm]
  
       \node (a) at (-5,0){};
       \node (b) at (0,4){};
       \node (c) at (0,2) {};
       \node (d) at (0,0){};
       \node (e) at (0, -2) {};
       \node (f) at (0, -4) {};
       \node (g) at (5, 0) {};
        %------edges
    \draw[]
        % (c) -- (f)
        % (b) -- (g)
        %(b) -- (g)
        %(a) -- (b)
        %(b) -- (d) 
        (d) -- (f);
        \draw[dashed]
        (a) -- (f)
        (g) -- (f);
        % \draw[out=20, in=160, relative]
        % (b) to (c)
        % (c) to (d)
        % (d) to (a)
        % (d) to (e)
        % (e) to (c)
        % (c) to (e)
        % (f) to (b)
        % (b) to (f);
        \draw[out=30, in = 150, relative]{}
        (b) to (f)
        (f) to (a)
        (f) to (d)
        (g) to (f);
        \draw[out = 45, in =135, relative]{}
        (b) to (f);
        
        %---------circles
        \filldraw[color=cyan, fill=cyan!5] 
        (a) circle (1/3)
        (b) circle (1/3)
       % (c) circle (1/3)
        (d) circle (1/3)       
       % (e) circle (1/3)
        (f) circle (1/3)
        (g) circle (1/3); 
        
       \end{scope}
       
       \end{scope}
       %end whole thing 
       
  \end{tikzpicture}
    \caption{$x=2$: Left is a generalized $A$-state graph of a Balanced type $2$ diagram with $x=2$. The orange in the middle shows the vertices that will be clasped together to produce a diagram whose $A$-state graph looks like the figure on the right}
    \label{fig:clasping Bal type 2 x=2}
\end{figure}

