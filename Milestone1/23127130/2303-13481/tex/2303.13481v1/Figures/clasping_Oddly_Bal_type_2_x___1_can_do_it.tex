\begin{figure}[h]
    \centering
  \begin{tikzpicture}
      \begin{scope}[scale=0.9, >=Stealth]
       
  %      % A-state graph part 
% FIRST 
  \begin{scope}[scale = 0.6, xshift=-8cm]
  
       \node (a) at (-3,1){};
       \node (b) at (-1,1){};
       \node (c) at (1,1) {};
       \node (d) at (3,1){};
       \node (e) at (3, -1) {};
       \node (f) at (1, -1) {};
       \node (g) at (-1, -1) {};
       \node (h) at (-3, -1) {};
        %------edges
        \draw[]
        % (c) -- (f)
        % (b) -- (g)
        (b) -- (g)
        (a) -- (b)
        (b) -- (c) 
        (c) -- (d)
        (d) -- (e)
        (e) -- (f)
        %(f) -- (g)
        (g) -- (h);
        \draw[dashed]
        (h) -- (a)
        (f) -- (g);
        % \draw[out=20, in=160, relative]
        % (b) to (c)
        % (c) to (d)
        % (d) to (a)
        % (d) to (e)
        % (e) to (c)
        % (c) to (e)
        % (f) to (b)
        % (b) to (f);
        
        %---------circles
        \filldraw[color=cyan, fill=cyan!5] 
        (a) circle (1/3)
        (b) circle (1/3)
        (c) circle (1/3)
        (d) circle (1/3)       
        (e) circle (1/3)
        (f) circle (1/3)
        (g) circle (1/3)
        (h) circle (1/3); 
        
       \end{scope}
          % end A-state graph part 
          

        
% SECOND
  \begin{scope}[scale = 0.6, xshift=0cm]
  
       \node (a) at (-3,1){};
       \node (b) at (-1,1){};
       \node (c) at (1,1) {};
       \node (d) at (3,1){};
       \node (e) at (3, -1) {};
       \node (f) at (1, -1) {};
       \node (g) at (-1, -1) {};
       \node (h) at (-3, -1) {};
        %------edges
        \draw[]
        % (c) -- (f)
        % (b) -- (g)
        (b) -- (g)
        (a) -- (b)
        (b) -- (c) 
        (c) -- (d)
        (d) -- (e)
        (e) -- (f)
        %(f) -- (g)
        (g) -- (h);
        \draw[dashed]
        (h) -- (a)
        (f) -- (g);
        % \draw[out=20, in=160, relative]
        % (b) to (c)
        % (c) to (d)
        % (d) to (a)
        % (d) to (e)
        % (e) to (c)
        % (c) to (e)
        % (f) to (b)
        % (b) to (f);
        \draw[thick, orange]
        (c) -- (e);
        
        %---------circles
        \filldraw[color=cyan, fill=cyan!5] 
        (a) circle (1/3)
        (b) circle (1/3)
        (c) circle (1/3)
        (d) circle (1/3)       
        (e) circle (1/3)
        (f) circle (1/3)
        (g) circle (1/3)
        (h) circle (1/3); 
        
       \end{scope}

 % THIRD
\begin{scope}[scale = 0.6, xshift=8cm]
  
       \node (a) at (-3,1){};
       \node (b) at (-1,1){};
       \node (c) at (1,1) {};
       \node (d) at (3,1){};
       \node (e) at (3, -1) {};
       \node (f) at (1, -1) {};
       \node (g) at (-1, -1) {};
       \node (h) at (-3, -1) {};
        %------edges
        \draw[]
        (c) -- (f)
         (b) -- (g)
        (a) -- (b)
        (b) -- (c) 
        (c) -- (d)
        % (d) -- (e)
        % (e) -- (f)
        %(f) -- (g)
        (g) -- (h);
        \draw[dashed]
        (h) -- (a)
        (f) -- (g);
        \draw[out=20, in=160, relative]
        % (b) to (c)
        (c) to (d)
        % (d) to (e)
        (e) to (c)
        (c) to (e);
        % (f) to (b)
        % (b) to (f);
        
        %---------circles
        \filldraw[color=cyan, fill=cyan!5] 
        (a) circle (1/3)
        (b) circle (1/3)
        (c) circle (1/3)
        (d) circle (1/3)       
        (e) circle (1/3)
        (f) circle (1/3)
        (g) circle (1/3)
        (h) circle (1/3); 
        
       \end{scope}
       
       \end{scope}
       %end whole thing 
       
  \end{tikzpicture}
    \caption{$x=1$: When $k_2\geq 6$, we can clasp our $(k_1, k_2)$-Oddly Balanced diagram and obtain a $(k_1, k_2-2)$-Oddly Balanced diagram}
    \label{fig:clasping Oddly Bal type 2 x=1 can do it}
\end{figure}

