\begin{figure}[h]
    \centering
  \begin{tikzpicture}
      \begin{scope}[scale=0.53, >=Stealth]

% begin diagram part
      \begin{scope}[scale=0.8, xshift=-6cm, rotate=45]
       \node (a) at (-3,0){};
       \node (b) at (-1,0){};
       \node (c) at (0,3) {};
       \node (d) at (0,1){};
       \node (e) at (1,0) {};
       \node (f) at (3,0){};
       \node (g) at (3,-3) {};
       \node (h) at (0, -3){};
       \node (i) at (-3,-3){};
       \node (j) at (-3, -2){};
       \node (k) at (-1, -2){};
       
       %OUTER ONES-----------------------

       \draw [out=90, in=90, relative, decoration={markings, mark=at position 0.5 with {\arrow{<}}}, postaction={decorate}] 
       (a.center) to (c);
       \draw [out=90, in=90, relative, decoration={markings, mark=at position 0.5 with {\arrow{>}}}, postaction={decorate}] 
       (c.center) to (f);
       \draw [out=45, in=135, relative, decoration={markings, mark=at position 0.5 with {\arrow{}}}, postaction={decorate}] 
       (f.center) to (g.center);
       \draw [out=45, in=135, relative, decoration={markings, mark=at position 0.5 with {\arrow{<}}}, postaction={decorate}] 
       (g.center) to (h);
       \draw [out=45, in=135, relative, decoration={markings, mark=at position 0.5 with {\arrow{>}}}, postaction={decorate}] 
       (h.center) to (i.center);
       \draw [out=45, in=135, relative, decoration={markings, mark=at position 0.5 with {\arrow{}}}, postaction={decorate}] 
       (i.center) to (j.center);
       \draw [out=45, in=135, relative, decoration={markings, mark=at position 0.5 with {\arrow{>}}}, postaction={decorate}] 
       (j) to (a);

    %END OUTER ----
    % INNER ONES---------------------------
       \draw [ decoration={markings, mark=at position 0.5 with {\arrow{>}}}, postaction={decorate}] 
       (b) to (d.center);
       \draw [ decoration={markings, mark=at position 0.5 with {\arrow{>}}}, postaction={decorate}] 
       (e.center) to (d);
       \draw [out=-30, in=170, relative, decoration={markings, mark=at position 0.5 with {\arrow{<}}}, postaction={decorate}] 
       (h.center) to (e);
       \draw [decoration={markings, mark=at position 0.5 with {\arrow{<}}}, postaction={decorate}] 
       (h) to (k);
       \draw [out=40, in=140, relative, decoration={markings, mark=at position 0.5 with {\arrow{>}}}, postaction={decorate}] 
       (b.center) to (k.center);

       %-- END INNERS 

       % CLASP-Y PARTS 
       \draw [out=45, in=135, relative, decoration={markings, mark=at position 0.5 with {\arrow{>}}}, postaction={decorate}] 
       (a) to (b.center);
       \draw [out=45, in=140, relative, decoration={markings, mark=at position 0.5 with {\arrow{<}}}, postaction={decorate}] 
       (b) to (a.center);
       \draw [out=45, in=140, relative, decoration={markings, mark=at position 0.5 with {\arrow{>}}}, postaction={decorate}] 
       (d) to (c.center);
       \draw [out=45, in=135, relative, decoration={markings, mark=at position 0.5 with {\arrow{<}}}, postaction={decorate}] 
       (c) to (d.center);
       \draw [out=45, in=140, relative, decoration={markings, mark=at position 0.5 with {\arrow{>}}}, postaction={decorate}] 
       (f) to (e.center);
       \draw [out=45, in=135, relative, decoration={markings, mark=at position 0.5 with {\arrow{<}}}, postaction={decorate}] 
       (e) to (f.center);
       \draw [out=45, in=135, relative, decoration={markings, mark=at position 0.5 with {\arrow{>}}}, postaction={decorate}] 
       (j.center) to (k);
       \draw [out=45, in=135, relative, decoration={markings, mark=at position 0.5 with {\arrow{>}}}, postaction={decorate}] 
       (k.center) to (j);
%end claspy part 
       \end{scope}
    %end diagram part 

% begin A-state circles
      \begin{scope}[scale=0.8, xshift=2.5cm, rotate=45]
       \node (a) at (-3,0){};
       \node (b) at (-1,0){};
       \node (c) at (0,3) {};
       \node (d) at (0,1){};
       \node (e) at (1,0) {};
       \node (f) at (3,0){};
       \node (g) at (3,-3) {};
       \node (h) at (0, -3){};
       \node (i) at (-3,-3){};
       \node (j) at (-3, -2){};
       \node (k) at (-1, -2){};
       
       %OUTER ONES-----------------------

       \draw [red, thick, out=90, in=90, relative, decoration={markings, mark=at position 0.5 with {\arrow{<}}}, postaction={decorate}] 
       (a.center) to (c);
       \draw [blue, thick, out=90, in=90, relative, decoration={markings, mark=at position 0.5 with {\arrow{>}}}, postaction={decorate}] 
       (c.center) to (f);
       \draw [orange, thick, out=45, in=135, relative, decoration={markings, mark=at position 0.5 with {\arrow{}}}, postaction={decorate}] 
       (f.center) to (g.center);
       \draw [orange, thick, out=45, in=135, relative, decoration={markings, mark=at position 0.5 with {\arrow{<}}}, postaction={decorate}] 
       (g.center) to (h);
       \draw [purp, thick, out=45, in=135, relative, decoration={markings, mark=at position 0.5 with {\arrow{>}}}, postaction={decorate}] 
       (h.center) to (i.center);
       \draw [purp, thick, out=45, in=135, relative, decoration={markings, mark=at position 0.5 with {\arrow{}}}, postaction={decorate}] 
       (i.center) to (j.center);
       \draw [purp, thick, out=45, in=135, relative, decoration={markings, mark=at position 0.5 with {\arrow{}}}, postaction={decorate}] 
       (j) to (a);

    %END OUTER ----
    % INNER ONES---------------------------
       \draw [red, thick, decoration={markings, mark=at position 0.5 with {\arrow{}}}, postaction={decorate}] 
       (b) to (d.center);
       \draw [blue, thick, decoration={markings, mark=at position 0.5 with {\arrow{}}}, postaction={decorate}] 
       (e.center) to (d);
       \draw [orange, thick, out=-30, in=170, relative, decoration={markings, mark=at position 0.5 with {\arrow{<}}}, postaction={decorate}] 
       (h.center) to (e);
       \draw [purp, thick, decoration={markings, mark=at position 0.5 with {\arrow{<}}}, postaction={decorate}] 
       (h) to (k);
       \draw [purp, thick, out=40, in=140, relative, decoration={markings, mark=at position 0.5 with {\arrow{>}}}, postaction={decorate}] 
       (b.center) to (k.center);

       %-- END INNERS 

       % CLASP-Y PARTS 
       \draw [red, thick, out=45, in=135, relative, decoration={markings, mark=at position 0.5 with {\arrow{>}}}, postaction={decorate}] 
       (a) to (b.center);
       \draw [purp, thick, out=45, in=140, relative, decoration={markings, mark=at position 0.5 with {\arrow{<}}}, postaction={decorate}] 
       (b) to (a.center);
       \draw [red, thick, out=45, in=140, relative, decoration={markings, mark=at position 0.5 with {\arrow{>}}}, postaction={decorate}] 
       (d) to (c.center);
       \draw [blue, thick, out=45, in=135, relative, decoration={markings, mark=at position 0.5 with {\arrow{<}}}, postaction={decorate}] 
       (c) to (d.center);
       \draw [orange, thick, out=45, in=140, relative, decoration={markings, mark=at position 0.5 with {\arrow{>}}}, postaction={decorate}] 
       (f) to (e.center);
       \draw [blue, thick, out=45, in=135, relative, decoration={markings, mark=at position 0.5 with {\arrow{<}}}, postaction={decorate}] 
       (e) to (f.center);
       \draw [green, thick, out=45, in=135, relative, decoration={markings, mark=at position 0.5 with {\arrow{>}}}, postaction={decorate}] 
       (j.center) to (k);
       \draw [green, thick, out=45, in=135, relative, decoration={markings, mark=at position 0.5 with {\arrow{>}}}, postaction={decorate}] 
       (k.center) to (j);
%end claspy part 
       \end{scope}
    %end A-state circles

  %      % A-state graph part 
  \begin{scope}[scale = 1, xshift=7.75cm, yshift=0.5cm, rotate = 45]
  
       \node (a) at (-1,1){};
       \node (b) at (1,1){};
       \node (c) at (1,-1) {};
       \node (d) at (-1,-1){};
       \node (e) at (-2, -2) {};
        %------edges
        \draw[]
        (a) -- (b)
        (b) -- (c) 
        (d) -- (a)
        (c) -- (d);
        \draw[out=20, in=160, relative]
        (a) to (b)
        (b) to (c)
        (d) to (a)
        (d) to (e)
        (e) to (d);
        \draw[out=20, in=160, relative]
        (a) to (b)
        (b) to (c)
        (d) to (a);
        

        %---------circles
        \filldraw[color=red, fill=red!10] 
        (a) circle (1/3);
     \filldraw[color=blue, fill=blue!10] 
        (b) circle (1/3);
     \filldraw[color=orange, fill=orange!10] 
        (c) circle (1/3);
     \filldraw[color=purp, fill=purp!10] 
        (d) circle (1/3);
     \filldraw[color=green, fill=green!10] 
        (e) circle (1/3);
        
       \end{scope}
          % end A-state graph part 

  %     reduced A-state graph part 
  \begin{scope}[scale = 1, xshift=12cm, yshift=0.5cm, rotate = 45]
  
       \node (a) at (-1,1){};
       \node (b) at (1,1){};
       \node (c) at (1,-1) {};
       \node (d) at (-1,-1){};
       \node (e) at (-2, -2) {};
        %------edges
        \draw[]
        (a) -- (b)
        (b) -- (c) 
        (d) -- (a)
        (c) -- (d)
        (d) -- (e);
        

       %---------circles
        \filldraw[color=red, fill=red!10] 
        (a) circle (1/3);
     \filldraw[color=blue, fill=blue!10] 
        (b) circle (1/3);
     \filldraw[color=orange, fill=orange!10] 
        (c) circle (1/3);
     \filldraw[color=purp, fill=purp!10] 
        (d) circle (1/3);
     \filldraw[color=green, fill=green!10] 
        (e) circle (1/3);
        
       \end{scope}
          % end reduced A-state graph

          
    % end first row 
       
       \end{scope}
       %end whole thing 
       
  \end{tikzpicture}
  \vspace{-5pt}
     \caption{A Burdened diagram of type $1$, its $A$-circles, $A$-state graph, and reduced $A$-state graph}
    \label{fig:Burdened type 1 example}
\end{figure}

