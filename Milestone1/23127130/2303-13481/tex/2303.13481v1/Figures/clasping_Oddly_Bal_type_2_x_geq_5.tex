\begin{figure}[h]
    \centering
  \begin{tikzpicture}
      \begin{scope}[scale=0.6, >=Stealth]
       
  %      % A-state graph part 
% FIRST 
  \begin{scope}[scale = 0.6, xshift=-8cm]
  
       \node (a) at (0,5){};
       \node (b) at (0,3){};
       \node (c) at (0,1) {};
       \node (d) at (0,-1){};
       \node (e) at (0, -3) {};
       \node (f) at (0, -5) {};
       \node (g) at (3, 3){};
       \node (h) at (3, -3){};
        %------edges
        \draw[]
        % (c) -- (f)
        % (b) -- (g)
        (a.center) -- (b.center)
        (b) -- (c) 
        (c) -- (d)
        (d) -- (e)
        (a.center) -- (g.center)
        (g.center) -- (h.center);
         \draw[dashed]
          (e) -- (f)
          (h) -- (f);
          \draw[out= 60, in = 120, relative, dashed]
          (f.center) to (a.center);
        % \draw[out=20, in=160, relative]
        % (b) to (c)
        % (c) to (d)
        % (d) to (a)
        % (d) to (e)
        % (e) to (c)
        % (c) to (e)
        % (f) to (b)
        % (b) to (f);
        
        %---------circles
        \filldraw[color=cyan, fill=cyan!5] 
        (a) circle (1/3)
        (b) circle (1/3)
        (c) circle (1/3)
        (d) circle (1/3)       
        (e) circle (1/3)
        (f) circle (1/3)
        (g) circle (1/3)
        (h) circle (1/3);
        
       \end{scope}
          % end A-state graph part 
          

          
 % SECOND
 \begin{scope}[scale = 0.6, xshift=0cm]
  
       \node (a) at (0,5){};
       \node (b) at (0,3){};
       \node (c) at (0,1) {};
       \node (d) at (0,-1){};
       \node (e) at (0, -3) {};
       \node (f) at (0, -5) {};
        \node (g) at (3, 3){};
       \node (h) at (3, -3){};
        %------edges
        \draw[]
        % (c) -- (f)
        % (b) -- (g)
        (a) -- (b)
        (b) -- (c) 
        (c) -- (d)
        (d) -- (e)
        (a.center) -- (g.center)
        (g) -- (h);
         \draw[dashed]
          (e) -- (f)
          (h) -- (f);
          \draw[out= 60, in = 120, relative, dashed]
          (f) to (a);
        % \draw[out=20, in=160, relative]
        % (b) to (c)
        % (c) to (d)
        % (d) to (a)
        % (d) to (e)
        % (e) to (c)
        % (c) to (e)
        % (f) to (b)
        % (b) to (f);
        \draw[out = 35, in = 145, relative, thick, orange]
        (c.center) to (a.center);
        
        %---------circles
        \filldraw[color=cyan, fill=cyan!5] 
        (a) circle (1/3)
        (b) circle (1/3)
        (c) circle (1/3)
        (d) circle (1/3)       
        (e) circle (1/3)
        (f) circle (1/3)
         (g) circle (1/3)
        (h) circle (1/3); 
        
       \end{scope}

% THIRD 
 \begin{scope}[scale = 0.6, xshift=8cm]
  
       \node (a) at (0,5){};
       \node (b) at (0,3){};
       \node (c) at (0,1) {};
       \node (d) at (0,-1){};
       \node (e) at (0, -3) {};
       \node (f) at (0, -5) {};
        \node (g) at (3, 3){};
       \node (h) at (3, -3){};
        %------edges
        \draw[]
        % (c) -- (f)
        % (b) -- (g)
        %(a) -- (b)
        (b) -- (c) 
        (c) -- (d)
       (d) -- (e)
        %(a) -- (g)
        (g) -- (h)
        (g.center) -- (c.center);
         \draw[dashed]
          (e) -- (f)
          (h) -- (f);
          \draw[out= 60, in = 120, relative, dashed]
          (f.center) to (c.center);
        % \draw[out=20, in=160, relative]
        % (b) to (c)
        % (c) to (d)
        % (d) to (a)
        % (d) to (e)
        % (e) to (c)
        % (c) to (e)
        % (f) to (b)
        % (b) to (f);
        \draw[out = 30, in = 150, relative]
        (b.center) to (c.center)
        (c.center) to (a.center);
        \draw[out = 45, in = 135, relative]
        (c.center) to (a.center);
     
        
        %---------circles
        \filldraw[color=cyan, fill=cyan!5] 
        (a) circle (1/3)
        (b) circle (1/3)
        (c) circle (1/3)
        (d) circle (1/3)       
        (e) circle (1/3)
        (f) circle (1/3)
         (g) circle (1/3)
        (h) circle (1/3); 
        
       \end{scope}
       
       \end{scope}
       %end whole thing 
       
  \end{tikzpicture}
    \caption{$x\geq 5$: The orange in the middle shows the vertices that will be clasped together to produce a diagram whose $A$-state graph looks like the figure on the right}
    \label{fig:clasping Oddly Bal type 2 x geq 5}
\end{figure}

