\documentclass[10pt,twocolumn,letterpaper]{article}

\usepackage{ieee}
\usepackage{times}
\usepackage{epsfig}
\usepackage{graphicx}
\usepackage{amsmath}
\usepackage{amssymb}
\usepackage{scalerel}
\usepackage{siunitx}
\sisetup{detect-all=true}
\usepackage[nocompress]{cite}
\usepackage{enumitem}
\usepackage{algpseudocode}
\usepackage{algorithm}
\usepackage{multirow}
\usepackage[title]{appendix}
\usepackage{subfigure} % [FIGTOPCAP]

\usepackage[pagebackref=true,breaklinks=true,letterpaper=true,colorlinks,bookmarks=false]{hyperref}
\usepackage{xcolor}
\definecolor{deepred}{rgb}{0.80 0.07 0}
\definecolor{deepaqua}{rgb}{0.03 0.47 0.50}
\hypersetup{
    colorlinks=true,
    allcolors=deepaqua,
    linkcolor=deepred,
    filecolor=deepaqua,      
    urlcolor=deepaqua,
}

\usepackage{cleveref}

\ieeefinalcopy
\def\ieeePaperID{-}
\def\httilde{\mbox{\tt\raisebox{-.5ex}{\symbol{126}}}}
% \ifieeefinal\pagestyle{empty}\fi
\newcommand{\ourmethod}{\mbox{EfficientAD}}
\newcommand{\patchcoreens}{PatchCore\textsubscript{\scaleto{\text{Ens}}{5pt}}}
\newcommand{\studteach}{\mbox{S--T}}
\newcommand{\unet}{\mbox{U-Net}}
\newcommand{\meanwithstd}[2]{\specialcell[c]{#1 \\[-3pt] {\footnotesize ($\pm$ #2)}}}
\newcommand{\specialcell}[2][c]{%
  \begin{tabular}[#1]{@{}c@{}}#2\end{tabular}}

\begin{document}

\title{\ourmethod: Accurate Visual Anomaly Detection at Millisecond-Level Latencies}

\author{Kilian Batzner \hspace{1cm} Lars Heckler \hspace{1cm} Rebecca König \vspace{0.14cm} \\
MVTec Software GmbH\\
{\tt\small \{kilian.batzner, lars.heckler, rebecca.koenig\}@mvtec.com}
}

\maketitle

\begin{abstract}

Detecting anomalies in images is an important task, especially in real-time computer vision applications.
In this work, we focus on computational efficiency and propose a lightweight feature extractor that processes an image in less than a millisecond on a modern GPU.
We then use a student--teacher approach to detect anomalous features.
We train a student network to predict the extracted features of normal, i.e., anomaly-free training images.
The detection of anomalies at test time is enabled by the student failing to predict their features.
We propose a training loss that hinders the student from imitating the teacher feature extractor beyond the normal images.
It allows us to drastically reduce the computational cost of the student--teacher model, while improving the detection of anomalous features.
We furthermore address the detection of challenging logical anomalies that involve invalid combinations of normal local features, for example, a wrong ordering of objects.
We detect these anomalies by efficiently incorporating an autoencoder that analyzes images globally.
We evaluate our method, called \ourmethod, on 32 datasets from three industrial anomaly detection dataset collections.
\ourmethod\ sets new standards for both the detection and the localization of anomalies.
At a latency of two milliseconds and a throughput of six hundred images per second, it enables a fast handling of anomalies.
Together with its low error rate, this makes it an economical solution for real-world applications and a fruitful basis for future research.

\end{abstract}

\section{Introduction}

In the past years, deep learning methods have continued to improve the state of the art across a wide range of computer vision applications.
This progress has been accompanied by advances in making neural network architectures faster and more efficient \cite{tan2019efficientnet, tan2020efficientdet, redmon2016you, yin2022vit}.
Modern classification architectures, for example, focus on characteristics such as latency, throughput, memory consumption, and the number of trainable parameters \cite{tan2019efficientnet, tan2021efficientnetv2, liu2022convnet, yin2022vit, sandler2018mobilenetv2, liu2021swin}.
This ensures that as networks become more capable, their computational requirements remain suitable for real-world applications. 
The field of visual anomaly detection has also seen rapid progress in the recent past, especially on industrial anomaly detection benchmarks \cite{bergmann2021_mvtec_ad_ijcv, bergmann2019_mvtec_ad_cvpr, roth2022towards, rudolph2023asymmetric}.
State-of-the-art anomaly detection methods, however, often sacrifice computational efficiency for an increased anomaly detection performance.
Common techniques are ensembling, the use of large backbones, and increasing the input image resolution to up to 768$\times$768 pixels.

\begin{figure}[t]
\begin{center}
\includegraphics[width=1.0\linewidth]{figures/teaser.pdf}
\end{center}
   \caption{Anomaly detection performance vs. latency per image on an NVIDIA RTX A6000 GPU\@. Each AU-ROC value is an average of the image-level detection AU-ROC values on the MVTec AD \cite{bergmann2021_mvtec_ad_ijcv, bergmann2019_mvtec_ad_cvpr}, VisA \cite{zou2022spot}, and MVTec LOCO \cite{bergmann2021_mvtec_loco_ijcv} dataset collections.}
\label{fig:teaser}
\end{figure}


Real-world anomaly detection applications frequently put constraints on the computational requirements of a method.
There are cases where detecting an anomaly too late can cause substantial economic damage, such as metal objects in a crop field entering the interior of a combine harvester.
In other cases, even human health is at risk, for example, if a limb of a machine operator approaches a blade.
Furthermore, industrial settings commonly involve strict runtime limits caused by high production rates \cite{bailey2012machine_vision_handbook_fpgas}.
Not adhering to these limits would decrease the production rate of the respective application and thus its economic viability.
It is therefore essential to pay attention to the computational and economic cost of anomaly detection methods to keep them suitable for real-world applications.


In this work, we propose \ourmethod, a method that sets new standards for both the anomaly detection performance and the inference runtime, as shown in \Cref{fig:teaser}.
We first introduce an efficient network architecture for computing expressive features in less than a millisecond on a modern GPU\@.
To detect anomalous features, we use a student--teacher approach \cite{bergmann2020_uninformed_cvpr, rudolph2023asymmetric, wang2021student_teacher}.
We train a student network to predict the features computed by a pretrained teacher network on normal, i.e., anomaly-free training images.
Because the student is not trained on anomalous images, it generally fails to mimic the teacher on these.
A large distance between the outputs of the teacher and the student thus enables the detection of anomalies at test time.
To further increase this effect, Rudolph \etal \cite{rudolph2023asymmetric} use \textit{architectural} asymmetry between the teacher and the student.
We instead introduce \textit{loss-induced} asymmetry in the form of a training loss that hinders the student from imitating the teacher beyond the normal images.
This loss does not affect the computational cost at test time and does not restrict the architecture design.
It allows us to use our efficient network architecture for both the student and the teacher, while achieving an accurate detection of anomalous features.

Identifying anomalous local features enables the detection of anomalies that are \textit{structurally} different from the normal images, for example, contaminations or stains on manufactured products.
A challenging problem, however, are violations of \textit{logical} constraints regarding the position, size, arrangement, etc.\ of normal objects.
To address this, \ourmethod\ includes an autoencoder, as proposed by Bergmann \etal  \cite{bergmann2021_mvtec_loco_ijcv}.
This autoencoder learns the logical constraints of training images and detects violations at test time.
We simplify the autoencoder's architecture and training protocol and show how to integrate it efficiently with a student--teacher model.
Furthermore, we present a method to improve the anomaly detection performance by calibrating the detection results of the autoencoder and the student--teacher model before combining their results.

Our contributions are summarized as follows:
\begin{itemize}[topsep=0.2em]
    \itemsep-0.2em
    \item We substantially improve the state of the art for both the detection and the localization of anomalies on industrial benchmarks, at a latency of \SI{2}{\milli\second} and a throughput of more than 600 images per second.
    \item We propose an efficient network architecture to speed up feature extraction by an order of magnitude in comparison to the feature extractors used by recent methods \cite{roth2022towards, rudolph2023asymmetric, yu2021fastflow}.
    \item We introduce a training loss that significantly improves the anomaly detection performance of a student--teacher model without affecting its inference runtime.
    \item We achieve an efficient autoencoder-based detection of logical anomalies and propose a method for a calibrated combination of the detection results with those of a student--teacher model.
\end{itemize}


\section{Related Work}

\subsection{Anomaly Detection Tasks}

Visual anomaly detection is a rapidly growing area of research with a diverse range of applications, including medical imaging \cite{fernando2021_medical_ad_survey, armato2011lung, menze2015_brats_dataset}, autonomous driving \cite{hendrycks2019scaling, lis2019_iccv_resynthesis, blum2019_fishyscapes_dataset}, and industrial inspection \cite{bergmann2021_mvtec_ad_ijcv, ehret2019_ad_review_paper, pang2020_review_paper}.
Applications often have specific characteristics, such as the availability of image sequences in surveillance datasets \cite{li2013_ucsd_video_ad_dataset, zhang2016single, lu2013_avenue_video_ad_dataset} or the different modalities of medical imaging datasets (MRI \cite{bakas2017_brats_dataset}, CT \cite{armato2011lung}, X-ray \cite{irvin2019chexpert}, etc.).
This work focuses on detecting anomalies in RGB or gray-scale images without conditioning the prediction on a sequence of images.
We use industrial anomaly detection datasets to benchmark our proposed method against existing ones.


The introduction of the MVTec AD dataset by Bergmann \etal \cite{bergmann2019_mvtec_ad_cvpr, bergmann2021_mvtec_ad_ijcv} has catalyzed the development of methods for industrial applications.
It comprises 15 separate inspection scenarios, each consisting of a training set and a test set.
Each training set contains only normal images, for example, defect-free screws, while the test sets also contain anomalous images.
This represents a frequent challenge in real-world applications where the types and possible locations of defects are unknown during the development of the anomaly detection system.
Therefore, it is a challenging yet crucial requirement that methods perform well when trained only on normal images.

Recently, several new industrial anomaly detection datasets have been introduced \cite{bergmann2022_mvtec_3dad, bergmann2021_mvtec_loco_ijcv, zou2022spot, mishra2021vt, huang2018_magnetic_tile_dataset, jezek2021deep}.
The Visual Anomaly (VisA) dataset \cite{zou2022spot} and the MVTec Logical Constraints (MVTec LOCO) dataset \cite{bergmann2021_mvtec_loco_ijcv} follow the design of MVTec AD and comprise twelve and five anomaly detection scenarios, respectively.
They contain anomalies that are empirically more challenging than those of MVTec AD\@.
Furthermore, MVTec LOCO contains not only structural anomalies, such as stains or scratches but also logical anomalies.
These are violations of logical constraints, for example, a wrong ordering or a wrong combination of normal objects.
We refer to MVTec AD, VisA, and MVTec LOCO as dataset collections, as each scenario is a separate dataset consisting of a training and a test set.
All three provide pixel-precise defect segmentation masks for evaluating the anomaly localization performance of a method.

\subsection{Anomaly Detection Methods}

Traditional computer vision algorithms have been applied successfully to industrial anomaly detection tasks for several decades \cite{steger2018_mva_book}.
These algorithms commonly fulfill the requirement of processing an image within a few milliseconds.
Bergmann \etal \cite{bergmann2021_mvtec_ad_ijcv} evaluate some of these methods and find that they fail when requirements such as well-aligned objects are not met.
Deep-learning-based methods have been shown to handle such cases more robustly \cite{bergmann2021_mvtec_ad_ijcv, bergmann2021_mvtec_loco_ijcv}.


A successful approach in the recent past has been to apply outlier detection and density estimation methods in the feature space of a pretrained and frozen convolutional neural network (CNN).
If feature vectors can be mapped to input pixels, assigning their outlier scores to the respective pixels yields a 2D anomaly map of pixel anomaly scores.
Common methods include multivariate Gaussian distributions \cite{defard2021_PaDiM, rippel2021_Gaussian, li2021cutpaste}, Gaussian Mixture Models \cite{mishra2021vt, zong2018deep}, Normalizing Flows \cite{yu2021fastflow, rudolph2022_cross_flows, gudovskiy2022_CFLOW, rudolph2021_differnet, rezende2015variational}, and the k-Nearest Neighbor (kNN) algorithm \cite{napoletano2018_cnn_feature_dictionary_nanofibres, cohen2020_subimage, roth2022towards, nazare2018_pretrained_cnns_for_ad}.
A runtime bottleneck for kNN-based methods is the search for nearest neighbors during inference.
With PatchCore \cite{roth2022towards}, Roth \etal therefore perform kNN on a reduced database of clustered feature vectors.
They achieve state-of-the-art anomaly detection results on MVTec AD\@.
In our experiments, we include PatchCore and FastFlow \cite{yu2021fastflow}, a recent Normalizing-Flow-based method with a comparatively low inference runtime.


Bergmann \etal \cite{bergmann2020_uninformed_cvpr} propose a student--teacher (\studteach) framework for anomaly detection, in which the teacher is a pretrained frozen CNN\@.
They use three different teachers with different receptive field sizes.
For each teacher, they train three student networks to mimic the output of the respective teacher on the training images.
Because the students have not seen anomalous images during training, they generally fail to predict the teacher's output on these images, which enables anomaly detection.
Bergmann \etal use the variance of the students' outputs together with their distance from their teacher's output to compute anomaly scores.
Various modifications of \studteach\ have been proposed \cite{salehi2021_st_ad, wang2021student_teacher, rudolph2023asymmetric}.
Rudolph \etal \cite{rudolph2023asymmetric} reach an anomaly detection performance that is competitive to PatchCore by restricting the teacher to be an invertible neural network.
We compare our method to their Asymmetric Student Teacher (AST) approach and to the original \studteach\ method \cite{bergmann2020_uninformed_cvpr}. 


Generative models such as autoencoders \cite{bergmann2018_ssim_ae, park2020_mmnad, baur2019_ano_vae_gan, liu2020_visually_explaining_vaes, sakurada2014_aes_for_ad, bergmann2021_mvtec_loco_ijcv, gong2019_mem_ae_iccv} and GANs \cite{goodfellow2014_gans, schlegl2017_anogan, schlegl2019_fast_anogan, perera2019_cvpr_ocgan, akcay2019ganomaly} have been used extensively for anomaly detection.
Recent autoencoder-based methods rely on accurate reconstructions of normal images and inaccurate reconstructions of anomalous images \cite{bergmann2018_ssim_ae, park2020_mmnad, bergmann2021_mvtec_loco_ijcv, gong2019_mem_ae_iccv}.
This enables detecting anomalies by comparing the reconstruction to the input image.
A common problem are false-positive detections caused by inaccurate reconstructions of normal images, e.g., blurry reconstructions.
To avoid this, Bergmann \etal \cite{bergmann2021_mvtec_loco_ijcv} let an autoencoder reconstruct images in the feature space of a pretrained network.
Furthermore, they train a neural network to predict the output of the autoencoder and thus the systematic reconstruction errors on normal images.
We include their proposed method, called GCAD, in our experiments, as it outperforms other methods on the logical anomalies of MVTec LOCO by a large margin.
We also include DSR \cite{zavrtanik2022dsr}, which uses the latent space of a pretrained autoencoder and generates synthetic anomalies in it.

\section{Method}

We describe the components of \ourmethod\ in the following subsections.
It begins with the efficient extraction of features from a pretrained neural network in \cref{sec:method_pdn}.
We detect anomalous features at test time using a reduced student--teacher model, as described in \cref{sec:method_structural}.
A key challenge is to achieve a competitive anomaly detection performance while keeping the overall runtime low.
To this end, we introduce a novel loss function for the training of student--teacher pairs.
It improves the detection of anomalies without affecting the computational cost during inference.
In \cref{sec:method_logical}, we explain how to efficiently detect logical anomalies with an autoencoder-based approach.
Finally, we provide a solution for calibrating and combining the detection results of the autoencoder with those of the student--teacher model in \cref{sec:method_balancing}.

\subsection{Efficient Patch Descriptors}
\label{sec:method_pdn}

Recent anomaly detection methods commonly use the features of a deep pretrained network, such as a WideResNet-101 \cite{zagoruyko2016wideresnet_wrn, roth2022towards}.
We use a network with a drastically reduced depth as a feature extractor.
It consists of only four convolutional layers and is visualized in \Cref{fig:pdn}.
Each output neuron has a receptive field of 33$\times$33 pixels and thus each output feature vector describes a 33$\times$33 patch.
Due to this clear correspondence, we refer to the network as a patch description network (PDN).
The PDN is fully convolutional and can be applied to an image of variable size to generate all feature vectors in a single forward pass.


\begin{figure}[t]
\begin{center}
\includegraphics[width=1.0\linewidth]{figures/pdn.pdf}
\end{center}
   \caption{Patch description network (PDN) architecture of \ourmethod-S.
   Applying it to an image in a fully convolutional manner yields all features in a single forward pass.
   }
\label{fig:pdn}
\end{figure}


The \studteach\ method \cite{bergmann2020_uninformed_cvpr} also uses features from networks with only few convolutional layers.
The computational cost of these networks is nevertheless high because of the lack of downsampling in convolutional and pooling layers.
The number of parameters of the networks used by \studteach\ is comparably low (between 1.6 and 2.7 million per network).
Yet, executing a single network takes longer and requires more memory in our experiments than a \unet\ \cite{ronneberger2015_u_net} with 31 million parameters, an architecture used by the GCAD method \cite{bergmann2021_mvtec_loco_ijcv}.
This demonstrates how the number of parameters can be a misleading proxy metric for the latency, throughput, and memory footprint of a method.
Modern classification architectures typically perform downsampling early to reduce the size of feature maps and thus the runtime and memory requirements \cite{he2016_resnet_paper}.
We implement this in our PDN via strided average-pooling layers after the first and the second convolutional layer.
With the proposed PDN, we are able to obtain the features for an image of size 256$\times$256 in less than \SI{800}{\micro\second} on an NVIDIA RTX A6000 GPU.


To make the PDN generate expressive features, we distill a deep pretrained classification network into it.
For a controlled comparison, we use the same pretrained features as PatchCore \cite{roth2022towards} from a WideResNet-101.
We train the PDN on images from ImageNet \cite{russakovsky2015_alexnet} by minimizing the mean squared difference between its output and the features extracted from the pretrained network.
We provide the full list of training hyperparameters in Appendix \ref{subsec:distillation}.
Besides higher efficiency, the PDN has another benefit in comparison to the deep networks used by recent methods.
By design, a feature vector generated by the PDN only depends on the pixels in its respective 33$\times$33 patch.
The feature vectors of pretrained classifiers, on the other hand, exhibit long-range dependencies on other parts of the image.
This is shown in \Cref{fig:artifacts}, using PatchCore's feature extractors as an example.
The well-defined receptive field of the PDN ensures that an anomaly in one part of the image cannot trigger anomalous feature vectors in other, distant parts, which would impair the localization of anomalies.


\begin{figure}[t]
\begin{center}
\includegraphics[width=1.0\linewidth]{figures/artifacts_reduced.pdf}
\end{center}
   \caption{Upper row: absolute gradient of a single feature vector, located in the center of the output, with respect to each input pixel, averaged across input and output channels.
   Lower row: Average feature map of the first output channel across 1000 randomly chosen images from ImageNet \cite{russakovsky2015_alexnet}.
   The mean of these images is shown on the left.
   The feature maps of the DenseNet \cite{huang2017densely} and the WideResNet exhibit strong artifacts.
   }
\label{fig:artifacts}
\end{figure}


\subsection{Reduced Student--Teacher}
\label{sec:method_structural}

For detecting anomalous feature vectors, we reduce the \studteach\ method \cite{bergmann2020_uninformed_cvpr} to only one teacher, given by our distilled PDN, and one student.
Since we can execute the PDN in under a millisecond, we use its architecture for the student as well, resulting in a low overall latency.
This reduced student--teacher pair, however, lacks techniques used by previous methods to increase the anomaly detection performance: ensembling multiple teachers and multiple students \cite{bergmann2020_uninformed_cvpr}, using features from a pyramid of layers \cite{wang2021student_teacher}, and using architectural asymmetry between the student and the teacher network \cite{rudolph2023asymmetric}.
We therefore introduce a training loss that improves the detection of anomalies without affecting the computational requirements at test time.

We observe that in the standard \studteach\ framework, increasing the number of training images can improve the student's ability to imitate the teacher on anomalies.
This worsens the anomaly detection performance.
At the same time, deliberately decreasing the number of training images can suppress important information about normal images.
Our goal is to show the student enough data so that it can mimic the teacher sufficiently on normal images while avoiding generalization to anomalous images.
Similar to Online Hard Example Mining \cite{shrivastava2016training}, we therefore restrict the student's loss to the most relevant parts of an image.
These are the patches where the student currently mimics the teacher the least.
We propose a hard feature loss, which only uses the output elements with the highest loss for backpropagation.

Formally, we apply a teacher $T$ and a student $S$ to a training image $I$, which yields $T(I) \in \mathbb{R}^{C \times W \times H}$ and $S(I) \in \mathbb{R}^{C \times W \times H}$.
We compute the squared difference for each tuple $(c, w, h)$ as $D_{c, w, h} = (T(I)_{c, w, h} - S(I)_{c, w, h})^2$.
Based on a mining factor $p_\mathrm{hard} \in [0, 1]$, we then compute the $p_\mathrm{hard}$-quantile of the elements of $D$.
Given the $p_\mathrm{hard}$-quantile $d_\mathrm{hard}$, we compute the training loss $L_\mathrm{hard}$ as the mean of all $D_{c, w, h} \geq d_\mathrm{hard}$.
Setting $p_\mathrm{hard}$ to zero would yield the original \studteach\ loss.
In our experiments, we set $p_\mathrm{hard}$ to $0.999$, which corresponds to using, on average, ten percent of the values in each of the three dimensions of $D$ for backpropagation.
\Cref{fig:ohem} visualizes the effect of the hard feature loss for $p_\mathrm{hard} = 0.999$.
During inference, the 2D anomaly score map $M \in \mathbb{R}^{W \times H}$ is given by $M_{w, h} = C^{-1} \sum_c D_{c, w, h}$, i.e., by $D$ averaged across channels.
It assigns an anomaly score to each feature vector.
By using the hard feature loss, we avoid outliers in the anomaly scores on normal images, i.e., false-positive detections.


\begin{figure}[t]
\begin{center}
\includegraphics[width=1.0\linewidth]{figures/ohem.pdf}
\end{center}
   \caption{
   Randomly picked loss masks generated by the hard feature loss during training.
   The brightness of a mask pixel indicates how many of the dimensions of the respective feature vector were selected for backpropagation.
   The student network already mimics the teacher well on the background and thus focuses on learning the features of differently rotated screws.
   }
\label{fig:ohem}
\end{figure}


\begin{figure*}[t]
\begin{center}
\includegraphics[width=1.0\linewidth]{figures/pipeline_reduced.pdf}
\end{center}
   \caption{\ourmethod\ applied to two test images from MVTec LOCO\@.
   Normal input images contain a horizontal cable connecting the two splicing connectors at an arbitrary height.
   The anomaly on the left is a foreign object in the form of a small metal washer at the end of the cable.
   It is visible in the local anomaly map because the outputs of the student and the teacher differ.
   The logical anomaly on the right is the presence of a second cable.
   The autoencoder fails to reconstruct the two cables on the right in the feature space of the teacher.
   The student also predicts the output of the autoencoder in addition to that of the teacher.
   Because its receptive field is restricted to small patches of the image, it is not influenced by the presence of the additional red cable.
   This causes the outputs of the autoencoder and the student to differ.
   ``Diff'' refers to computing the element-wise squared difference between two collections of output feature maps and computing its average across feature maps.
   To obtain pixel anomaly scores, the anomaly maps are resized to the size of the input image using bilinear interpolation.
   }
\label{fig:gcad_architecture}
\end{figure*}


In addition to the hard feature loss, we use a loss penalty during training that further hinders the student from imitating the teacher on images that are not part of the normal training images.
In the standard \studteach\ framework, the teacher is pretrained on an image classification dataset, or it is a distilled version of such a pretrained network.
The student is not trained on that pretraining dataset but only on the application's normal images.
We propose to also use the images from the teacher's pretraining during the training of the student.
Specifically, we sample a random image $P$ from the pretraining dataset, in our case ImageNet, in each training step.
We compute the loss of the student as $L_\mathrm{ST} = L_\mathrm{hard} + (CWH)^{-1}\sum_c \|S(P)_c\|_F^2$.
This penalty hinders the student from generalizing its imitation of the teacher to out-of-distribution images.


\subsection{Logical Anomaly Detection}
\label{sec:method_logical}


We cannot rule out the possibility of logical anomalies, such as misplaced objects, and therefore want our method to be able to detect them.
As recommended by Bergmann \etal in their proposed GCAD method \cite{bergmann2021_mvtec_loco_ijcv}, we use an autoencoder for learning logical constraints of the training images and detecting violations of these constraints.
\Cref{fig:gcad_architecture} depicts the anomaly detection methodology for \ourmethod.
It consists of the aforementioned student--teacher pair and an autoencoder.
The autoencoder is trained to predict the output of the teacher.
Formally, we apply an autoencoder $A$ to a training image $I$, yielding $A(I) \in \mathbb{R}^{C \times W \times H}$, and compute the loss as $L_\mathrm{AE} = (CWH)^{-1} \sum_c \|T(I)_c - A(I)_c\|_F^2$.

In contrast to the patch-based student, the autoencoder must encode and decode the complete image through a bottleneck of 64 latent dimensions.
On images with logical anomalies, the autoencoder usually fails to generate the correct latent code for reconstructing the image in the teacher's feature space.
However, its reconstructions are also flawed on normal images, as autoencoders generally struggle with reconstructing fine-grained patterns \cite{brox2016_learned_visual_similarity_metrics, bergmann2018_ssim_ae}.
This is the case for the background grids in \Cref{fig:gcad_architecture}.
Using the difference between the teacher's output and the autoencoder's reconstruction as an anomaly map would cause false-positive detections in these cases.
Instead, we double the number of output channels of our student network and train it to predict the output of the autoencoder in addition to the output of the teacher.
Let $S'(I) \in \mathbb{R}^{C \times W \times H}$ denote the additional output channels of the student.
The student's additional loss is then $L_\mathrm{STAE} = (CWH)^{-1} \sum_c \|A(I)_c - S'(I)_c\|_F^2$.
The total training loss is the sum of $L_\mathrm{AE}$, $L_\mathrm{ST}$, and $L_\mathrm{STAE}$.

The student learns the systematic reconstruction errors of the autoencoder on normal images, e.g., blurry reconstructions.
At the same time, it does not learn the reconstruction errors for anomalies because these are not part of the training set.
This makes the difference between the autoencoder's output and the student's output well-suited for computing the anomaly map.
Analogous to the student--teacher pair, the anomaly map is the squared difference between the two outputs, averaged across channels.
We refer to the this anomaly map as the global anomaly map and to the anomaly map generated by the student--teacher pair as the local anomaly map.
We average these two anomaly maps to compute the combined anomaly map and use its maximum value as the image-level anomaly score.

Similar to our student, GCAD includes a separate \unet~\cite{ronneberger2015_u_net} model that is trained to predict the output of the autoencoder.
Sharing the student instead of using a separate \unet\ model reduces the computational cost and increases the asymmetry between the autoencoder and its imitator model.
Like the autoencoder, GCAD's \unet\ also has a bottleneck with a receptive field as large as the input image.
Although its skip connections avoid this, the \unet\ has the capacity to imitate the autoencoder even on logical anomalies.
Our student network, on the other hand, operates on 33$\times$33 patches.
As illustrated in \Cref{fig:gcad_architecture}, it is not influenced by logical anomalies that involve long-range dependencies between patches.
GCAD contains additional components which we do not include in our method.
These are the use of skip connections for training the autoencoder, a delayed activation of $L_\mathrm{STAE}$ during training, and an upsampling module that maps the output of the autoencoder to the teacher's feature space.
Removing these components simplifies the training process and improves the anomaly detection performance, as we show in our experiments.


\subsection{Anomaly Map Normalization}
\label{sec:method_balancing}


The local and the global anomaly map must be normalized to similar scales before averaging them to obtain the combined anomaly map.
This is important for cases where the anomaly is only detected in one of the maps, such as in \Cref{fig:gcad_architecture}.
Otherwise, noise in one map could make accurate detections in the other map indiscernible in the combined map.
To estimate the scale of the noise in normal images, we use validation images, i.e., unseen images from the training set.
For each of the two anomaly map types, we compute the set of all pixel anomaly scores across the validation images.
We then compute two $p$-quantiles for each set: $q_a$ and $q_b$, for $p=a$ and $p=b$, respectively.
We determine a linear transformation that maps $q_a$ to an anomaly score of $0$ and $q_b$ to a score of $0.1$.
At test time, the local and global anomaly maps are normalized with the respective linear transformation.

By using quantiles, the normalization becomes robust to the distribution of anomaly scores on normal images, which can vary between scenarios.
Whether the scores between~$q_a$ and~$q_b$ are normally distributed or a mixture of Gaussians or follow another distribution has no influence on the normalization.
Our experiments include an ablation study on the values of~$a$ and~$b$.
The choice of the mapping destination values $0$ and $0.1$ has no effect on anomaly detection metrics such as the area under the ROC curve (AU-ROC).
That is because the AU-ROC only depends on the ranking of scores, not on their scale.
Increasing the value of $0.1$ by a factor of $n$ would simply increase the scale of the two anomaly maps and the combined anomaly map by $n$.
We choose $0$ and $0.1$ because they yield maps that are suitable for a standard zero-to-one color scale.


\section{Experiments}


We compare \ourmethod\ to GCAD \cite{bergmann2021_mvtec_loco_ijcv}, \studteach\ \cite{bergmann2020_uninformed_cvpr}, PatchCore \cite{roth2022towards}, AST \cite{rudolph2023asymmetric}, FastFlow \cite{yu2021fastflow}, and DSR \cite{zavrtanik2022dsr}, using official implementations where available.
GCAD consists of an ensemble of two anomaly detection models that use different feature extractors.
We find that one of the two ensemble members performs better on average than the combined ensemble and therefore report the results for this member.
This reduces the latency reported for GCAD by a factor of two.
For PatchCore, we include two variants: the default single model variant, for which the authors report the lowest latency, and the ensemble variant, denoted by \patchcoreens.
We are able to reproduce the official results but disable the cropping of the center \SI{76.6}{\percent} of input images for a fair comparison.
In the case of MVTec AD, \SI{99.9}{\percent} of the defects lie fully or partially within this cropped area.
In real-world applications, anomalies can occur outside of this area as well.
We disable custom cropping, as it implies knowledge about the anomalies in the test set.
For FastFlow, we use the version based on the WideResNet-50-2 feature extractor, as it is similar to the WideResNet used by PatchCore and our method.
We use the implementation provided by the Intel anomalib \cite{akcay2022anomalib} but disable early stopping, i.e., the scenario-specific tuning of the training duration on test images.
We provide configuration details for all evaluated methods in Appendix \ref{sec:details_others}.


For our method, we evaluate two variants: \ourmethod-S and \ourmethod-M.
\ourmethod-S uses the architecture displayed in \Cref{fig:pdn} for the teacher and the student.
For \ourmethod-M, we double the number of kernels in the hidden convolutional layers of the teacher and the student.
Furthermore, we insert a 1$\times$1 convolution after the second pooling layer and after the last convolutional layer.
We provide a list of implementation details, such as the learning rate schedule, in Appendix \ref{subsec:training_and_evaluation}.


We evaluate each method on the 32 anomaly detection scenarios of MVTec AD, VisA, and MVTec LOCO\@.
The binary anomaly classification performance of a method is measured with the AU-ROC based on its predicted image-level anomaly scores.
We measure the anomaly localization performance using the AU-PRO segmentation metric up to a false positive rate of \SI{30}{\percent}, as recommended by \cite{bergmann2021_mvtec_ad_ijcv}.
For MVTec LOCO, we use the AU-sPRO metric \cite{bergmann2021_mvtec_loco_ijcv}, a generalization of the AU-PRO metric for evaluating the localization of logical anomalies.
Appendix \ref{sec:ad_metrics} provides the results for additional anomaly detection metrics, such as the area under the precision-recall curve and the pixel-wise AU-ROC.

\addtolength{\tabcolsep}{-2pt}  
\begin{table}
\small
\begin{center}
\begin{tabular}{ccccc}
Method & \specialcell[c]{Classif. \\ AU-ROC} & \specialcell[c]{Segment. \\ AU-PRO} & \specialcell[c]{Latency \\ {[ms]}} & \specialcell[c]{Throughput \\ {[img / s]}} \\
\hline
GCAD & 85.4 & 88.0 & 11 & 121 \\
\studteach & 88.4 & 89.7 & 75 & 16\\
FastFlow & 90.0 & 86.5 & 17 & 120\\
DSR & 90.8 & 78.6 & 17 & 104 \\
PatchCore & 91.1 & 80.9 & 32 & 76 \\
\patchcoreens & 92.1 & 80.7 & 148 & 13 \\
AST & 92.4 & 77.2 & 53 & 41 \\
\hline
\ourmethod-S & \meanwithstd{95.4}{0.06} & \meanwithstd{92.5}{0.05} & \meanwithstd{\textbf{2.2}}{0.01} & \meanwithstd{\textbf{614}}{2} \\
\ourmethod-M & \meanwithstd{\textbf{96.0}}{0.09} & \meanwithstd{\textbf{93.3}}{0.04} & \meanwithstd{4.5}{0.01} & \meanwithstd{269}{1} \\
\end{tabular}
\end{center}
\caption{Anomaly classification and anomaly localization performance in comparison to the latency and throughput. Each AU-ROC and AU-PRO percentage is an average of the mean AU-ROCs and mean AU-PROs, respectively, on MVTec AD, VisA, and MVTec LOCO.
For \ourmethod, we report the mean and standard deviation of five runs.
}
\label{tab:main}
\end{table}
\addtolength{\tabcolsep}{2pt}

\addtolength{\tabcolsep}{-3pt}  
\begin{table}
\small
\begin{center}
\begin{tabular}{ccccc||cc}
Method & MAD & LOCO & VisA & Mean & \specialcell[c]{LOCO \\ Logic.} & \specialcell[c]{LOCO \\ Struct.} \\
\hline
GCAD & 89.1 & 83.3 & 83.7 & 85.4 & 83.9 & 82.7 \\
\studteach & 93.2 & 77.4 & 94.6 & 88.4 & 66.5 & 88.3 \\
FastFlow & 96.9 & 79.2 & 93.9 & 90.0 & 75.5 & 82.9 \\
DSR & 98.1 & 82.6 & 91.8 & 90.8 & 75.0 & 90.2 \\
PatchCore & 98.7 & 80.3 & 94.3 & 91.1 & 75.8 & 84.8 \\
\patchcoreens & \textbf{99.3} & 79.4 & 97.7 & 92.1 & 71.0 & 87.7 \\
AST & 98.9 & 83.4 & 94.9 & 92.4 & 79.7 & 87.1 \\
\hline
\ourmethod-S & 98.8 & 90.0 & 97.5 & 95.4 & 85.8 & 94.1 \\
\ourmethod-M & 99.1 & \textbf{90.7} & \textbf{98.1} & \textbf{96.0} & \textbf{86.8} & \textbf{94.7} \\
\end{tabular}
\end{center}
\caption{Mean anomaly classification AU-ROC percentages per dataset collection (left) and on the logical and structural anomalies of MVTec LOCO (right). For \ourmethod, we report the mean of five runs.
\ourmethod\ matches the logical anomaly detection performance of GCAD without compromising the detection of structural anomalies.
Performing method development solely on MVTec AD (MAD) becomes prone to overfitting design choices to the few remaining misclassified test images.}
\label{tab:per_dataset}
\end{table}
\addtolength{\tabcolsep}{3pt}


\addtolength{\tabcolsep}{-0.5pt}  
\begin{table}
\small
\begin{center}
\begin{tabular}{ccccccc}
$p_\mathrm{hard}$ & 0 & 0.9 & 0.99 & \textbf{0.999} & 0.9999 & 0.99999 \\
AU-ROC & 94.9 & 94.9 & 95.7 & 96.0 & 95.8 & 95.7 \\
\hline
$a$ (for $q_a$) & 0.5 & 0.8 & \textbf{0.9} & 0.95 & 0.98 & 0.99 \\
AU-ROC & 95.9 & 95.9 & 96.0 & 95.9 & 95.9 & 95.8 \\
\hline
$b$ (for $q_b$) & 0.95 & 0.98 & 0.99 & \textbf{0.995} & 0.998 & 0.999 \\
AU-ROC & 95.8 & 95.9 & 96.0 & 96.0 & 95.9 & 95.9 \\
\end{tabular}
\end{center}
\caption{Mean anomaly classification AU-ROC of \ourmethod-M on MVTec AD, VisA, and MVTec LOCO when varying the locations of quantiles.
These are the mining factor $p_\mathrm{hard}$ and the two sampling points $a$ and $b$ of the quantile-based map normalization.
Default values used in our experiments are highlighted in bold.
}
\label{tab:hyperparams}
\end{table}
\addtolength{\tabcolsep}{-0.5pt}

When reporting the AU-ROC or AU-PRO for a dataset collection, we follow the policy of the dataset authors.
For each collection, we evaluate the respective metric for each scenario and then compute the mean across scenarios.
For MVTec LOCO, we use the official evaluation script, which gives logical and structural anomalies an equal weight in the computed metrics.
When reporting the average AU-ROC or AU-PRO on the three dataset collections, we compute the average of the three dataset means.
Thus, an overall average score weights logical anomalies and structural anomalies by roughly one-sixth and five-sixths, respectively.
We provide the evaluation results for each of the 32 anomaly detection scenarios individually in the appendix to enable an evaluation with a custom weighting.


\Cref{tab:main} reports the overall anomaly detection performance for each method.
\ourmethod\ achieves a strong detection and localization of anomalies.
Reliably localizing anomalies in an image provides explainable detection results and allows the discovery of spurious correlations in detections.
It also enables a flexible postprocessing, such as excluding defect segmentations based on their size.

\Cref{tab:per_dataset} breaks down the overall anomaly classification performance into the three dataset collections.
It shows that the lead of \ourmethod\ on MVTec LOCO is in equal parts due to its performance on logical and on structural anomalies.
In \Cref{tab:hyperparams}, we assess the robustness of \ourmethod\ to varying hyperparameters.


Furthermore, we measure the computational cost of each method during inference.
As explained above, the number of parameters can be a misleading proxy metric for the latency and throughput of convolutional architectures since it does not consider the resolution of a convolution's input feature map, i.e., how often a parameter is used in a forward pass.
Similarly, the number of floating point operations (FLOPs) can be misleading since it does not take into account how easily computations can be parallelized.
For transparency, we report the number of parameters, the number of FLOPs, and the memory footprint of each method in Appendix \ref{sec:efficiency_metrics}.
Here, we focus on the metrics that are most relevant in anomaly detection applications: the latency and the throughput.
We measure the latency with a batch size of 1 and the throughput with a batch size of 16.
\Cref{tab:main} reports the measurements for each method on an NVIDIA RTX A6000 GPU.
\Cref{fig:latency_per_gpu} shows the latency of each method on each of the GPUs in our experimental setup.
Appendix \ref{sec:efficiency_metrics} contains a detailed description of our timing methodology.

\begin{figure}[t]
\begin{center}
\includegraphics[width=1.0\linewidth]{figures/plot_per_gpu.pdf}
\end{center}
   \caption{
   Latency per GPU. The ranking of methods is the same on each GPU, except for two cases in which DSR is slightly faster than FastFlow.}
\label{fig:latency_per_gpu}
\end{figure}

\begin{table}
\small
\begin{center}
\begin{tabular}{cccc}
 & \specialcell[c]{Classif. \\ AU-ROC} & \specialcell[c]{Segment. \\ AU-PRO} & \specialcell[c]{Latency \\ {[ms]}} \\
\hline
GCAD & 85.4 & 88.0 & 10.9 \\
Shorter training & 85.6 & 87.5 & 10.9 \\
Patch-based students & 88.6 & 90.5 & 11.9 \\
\iftrue
Simplified autoencoder & 90.3 & 90.2 & 12.2 \\
Upsampling network & 91.0 & 91.0 & 12.9 \\
\else
Simplified autoencoder & 91.0 & 91.0 & 12.9 \\
\fi
PDN architecture & 91.4 & 91.0 & 2.7 \\
Hard feature loss & 92.7 & 90.9 & 2.7 \\
Pretraining penalty & 93.1 & 91.3 & 2.7 \\
Map normalization & 94.4 & 92.0  & 2.7 \\
Reduced augmentation & 95.2 & 91.8 & 2.7 \\
Dropout & 95.4 & 91.9 & 2.7 \\
Shared student & 95.2 & 91.9 & 2.2 \\
WRN distillation & 95.4  & 92.5  & 2.2 \\
\hline
\ourmethod-S & 95.4  & 92.5  & 2.2 \\
\ourmethod-M &  96.0 & 93.3  & 4.5 \\
\end{tabular}
\end{center}
\caption{Ablation study on the methodical differences between \ourmethod\ and GCAD. Each AU-ROC and AU-PRO percentage is an average of the mean AU-ROCs and mean AU-PROs, respectively, on MVTec AD, VisA, and MVTec LOCO. The latency is measured on an NVIDIA RTX A6000 GPU.}
\label{tab:ablation}
\end{table}



We examine the effects of our design choices in the ablation study shown in \Cref{tab:ablation}, using GCAD as a starting point.
We train \ourmethod\ for \num{70000} iterations, which is roughly \SI{40}{\percent} of the training duration of GCAD\@.
This allows \ourmethod-S to achieve a training time of less than twenty minutes.
For a controlled comparison of subsequent changes, we report the effect on the anomaly detection performance in a separate row called ``Shorter training.''
We then change the architecture of the two student networks of GCAD from a \unet\ to the patch-based architecture of GCAD's teacher network.
GCAD's local branch thus becomes a symmetrical student--teacher pair.
The anomaly detection performance at this point represents the performance of simply combining GCAD with a more efficient version of \studteach.
Next, we simplify the architecture and training of the autoencoder.
We replace the transposed convolutions in its decoder with bilinear upsampling and regular convolutional layers, as described in \cite{odena2016deconvolution}.
Detailed layer parameters are provided in Appendix \ref{subsec:training_and_evaluation}.
This change allows us to remove the use of skip connections during the training of the autoencoder and the delay of the training of the autoencoder's student network.
This further improves the anomaly detection performance (``Simplified autoencoder'' row), as does the removal of the upsampling network used by GCAD\@.
The previous changes, however, also increase the model's runtime.
To reduce it again, we change the architecture of the patch-based networks to that of the PDN shown in \Cref{fig:pdn}.

We evaluate the effect of the two proposed loss terms for training the student--teacher pair.
The hard feature loss causes the student's training to focus on the features with the highest loss.
These are also the most relevant features for the anomaly classification because we use the maximum pixel score as the image anomaly score.
The hard feature loss significantly improves the anomaly classification performance, which is further increased by the student's penalty on pretraining images.
Next, we replace GCAD's anomaly map normalization with our proposed quantile-based normalization.
GCAD computes normalization parameters such that anomaly scores of normal images have a mean of zero and a variance of one.
This makes it sensitive to the distribution of validation anomaly scores, which can vary between scenarios.
The quantile-based normalization is independent of how the scores between $q_a$ and $q_b$ are distributed.
It improves both the anomaly detection and the anomaly localization performance.

We switch from the scenario-specific augmentation parameters used by GCAD to a strongly reduced augmentation.
We disable augmentation in the student--teacher training to avoid teaching the student how to generalize to anomalous images.
For the autoencoder, we augment training images only with color jitter \cite{paszke2019_PyTorch} and insert dropout layers in the decoder to avoid overfitting.
To further decrease the latency of our method, we let one shared student network predict the output of both the teacher and the autoencoder.
Finally, we switch from the teacher pretraining of \studteach\ to the distillation of PatchCore's WideResNet-101, as described above, yielding \ourmethod-S.
We find that the anomaly detection performance is robust to the choice of the distillation backbone and provide a comparison in Appendix \ref{sec:backbones}.

\section{Conclusion}

In this paper, we introduce \ourmethod, a method with a strong anomaly detection performance and a high computational efficiency.
It sets new standards for the detection of structural as well as logical anomalies.
Both \ourmethod-S and \ourmethod-M outperform other methods on the classification and the localization of anomalies by a large margin.
Compared to AST, the second-best method, \ourmethod-S reduces the latency by a factor of 24 and increases the throughput by a factor of 15.
Its low latency, high throughput, and high detection rate make it suitable for real-world applications.
For future anomaly detection research, \ourmethod\ is an important baseline and a fruitful foundation.
Its efficient patch description network, for instance, can be used as a feature extractor in other anomaly detection methods as well to reduce their latency.

\paragraph{Limitations.}
The student--teacher model and the autoencoder are designed to detect anomalies of different types.
The autoencoder detects logical anomalies, while the student--teacher model detects coarse and fine-grained structural anomalies.
Fine-grained logical anomalies, however, remain a challenge -- for example a screw that is two millimeters too long.
To detect these, practitioners would have to use traditional metrology methods \cite{steger2018_mva_book}.
As for the limitations in comparison to other recent anomaly detection methods:
In contrast to kNN-based methods, our approach requires training, especially for the autoencoder to learn the logical constraints of normal images.
This takes twenty minutes in our experimental setup.

\onecolumn

{\small
% This must be in the first 5 lines to tell arXiv to use pdfLaTeX, which is strongly recommended.
\pdfoutput=1
% In particular, the hyperref package requires pdfLaTeX in order to break URLs across lines.

\documentclass[11pt]{article}

% Remove the "review" option to generate the final version.
%\usepackage[review]{ACL2023}
\usepackage{ACL2023}

% Standard package includes
\usepackage{times}
\usepackage{latexsym}

% For proper rendering and hyphenation of words containing Latin characters (including in bib files)
\usepackage[T1]{fontenc}
% For Vietnamese characters
% \usepackage[T5]{fontenc}
% See https://www.latex-project.org/help/documentation/encguide.pdf for other character sets

% This assumes your files are encoded as UTF8
\usepackage[utf8]{inputenc}

% This is not strictly necessary, and may be commented out.
% However, it will improve the layout of the manuscript,
% and will typically save some space.
\usepackage{microtype}

% This is also not strictly necessary, and may be commented out.
% However, it will improve the aesthetics of text in
% the typewriter font.
\usepackage{inconsolata}


% If the title and author information does not fit in the area allocated, uncomment the following
%
%\setlength\titlebox{10cm}
%
% and set <dim> to something 5cm or larger.

%%%%%%%%%%%%%%%%%%%%%%%%%%%%%%%%%%
\usepackage{graphicx}
\usepackage{amsfonts}
\usepackage{amsmath}
\usepackage{bigdelim}
\usepackage{diagbox}
\usepackage{amsthm}
\usepackage{makecell}
\usepackage{mathtools}
\usepackage{booktabs}
\usepackage[shortlabels]{enumitem}
\graphicspath{ {figs/} }

\theoremstyle{remark}
\newtheorem*{question}{Question}

\newcommand{\tk}[1]{\textcolor{blue}{{#1}}}
\newcommand{\sy}[1]{\textcolor{red}{{#1}}}
\newcommand{\mg}[1]{\textcolor{purple}{{#1}}}
\newcommand{\lh}[1]{\textcolor{green}{{#1}}}
\newcommand{\lc}[1]{\textcolor{green}{{#1}}}

% Rounded color box
\definecolor{light_blue}{HTML}{cfdfff}
\usepackage[most]{tcolorbox}
\tcbset{on line, 
        boxsep=1pt, left=0pt,right=0pt,top=0pt,bottom=0pt,
        colframe=white,colback=light_blue,  
        highlight math style={enhanced}
        }

\newcommand{\quash}[1]{}  %Anything in \quash is ignored
\newcommand{\gpt}{\textsc{GPT-2}}
\newcommand{\bert}{\textsc{BERT}}
\newcommand{\bertlarge}{\textsc{BERT-large}}
\newcommand{\mask}{\texttt{[MASK]}}
\newcommand{\cls}{\texttt{[CLS]}}
\newcommand{\sep}{\texttt{[SEP]}}
\newcommand{\mat}{\texttt{mat}}
\newcommand{\id}{\texttt{id}}
\newcommand{\matl}{\texttt{mat}_{\ell \rightarrow \ell'}}
\newcommand{\matattnl}{\texttt{mat\_attn}_{\ell \rightarrow \ell'}}
\newcommand{\matffl}{\texttt{mat\_ffn}_{\ell \rightarrow \ell'}}
\newcommand{\matlnl}{\texttt{mat\_ln1\_ln2}_{\ell \rightarrow \ell'}}
\newcommand{\idl}{\texttt{id}_{\ell \rightarrow \ell'}}
\newcommand{\matlL}{\texttt{mat}_{\ell \rightarrow L}}
\newcommand{\matattnlL}{\texttt{mat\_attn}_{\ell \rightarrow L}}
\newcommand{\matfflL}{\texttt{mat\_ffn}_{\ell \rightarrow L}}
\newcommand{\matlnlL}{\texttt{mat\_ln1\_ln2}_{\ell \rightarrow L}}
\newcommand{\idlL}{\texttt{id}_{\ell \rightarrow L}}

\definecolor{blue(munsell)}{rgb}{0.0, 0.5, 0.69}
%%%%%%%%%%%%%%%%%%%%%%%%%%%%%%%%%%

\title{Jump to Conclusions: Short-Cutting Transformers\\With Linear Transformations}

% Author information can be set in various styles:
% For several authors from the same institution:
% \author{Author 1 \and ... \and Author n \\
%         Address line \\ ... \\ Address line}
% if the names do not fit well on one line use
%         Author 1 \\ {\bf Author 2} \\ ... \\ {\bf Author n} \\
% For authors from different institutions:
% \author{Author 1 \\ Address line \\  ... \\ Address line
%         \And  ... \And
%         Author n \\ Address line \\ ... \\ Address line}
% To start a seperate ``row'' of authors use \AND, as in
% \author{Author 1 \\ Address line \\  ... \\ Address line
%         \AND
%         Author 2 \\ Address line \\ ... \\ Address line \And
%         Author 3 \\ Address line \\ ... \\ Address line}

\author{Alexander Yom Din$^{1}$ ~~~~~ Taelin Karidi$^{1}$ ~~~~~ Leshem Choshen$^{1}$ ~~~~~
Mor Geva$^{2}$ 
\vspace{0.2cm} \\
$^1$Hebrew University of Jerusalem ~~~ $^2$Google Research \\
\small{\texttt{\{alexander.yomdin, taelin.karidi, leshem.choshen\}@mail.huji.ac.il}}, \small{\texttt{pipek@google.com}}}

\quash{
\author{Alexander Yom Din \\
  Hebrew University of Jerusalem \\ \texttt{alexander.yomdin@mail.huji.ac.il} \\\And
  Taelin Karidi \\
  Hebrew University of Jerusalem \\
  \texttt{taelin.karidi@mail.huji.ac.il} \\\And
  Leshem Choshen \\
  Hebrew University of Jerusalem \\ \texttt{leshem.choshen@mail.huji.ac.il} \\\And
  Mor Geva \\
  Google Research \\
  \texttt{pipek@google.com} \\}
}

\begin{document}
\maketitle



\begin{abstract}
% \vspace{-1em}
The diffusion-based generative models have achieved remarkable success in text-based image generation. However, since it contains enormous randomness in generation progress, it is still challenging to apply such models for real-world visual content editing, especially in videos. 
In this paper, we propose \texttt{FateZero}, a zero-shot text-based editing method on real-world videos without per-prompt training or use-specific mask. 
\RM{Specifically, different from a pipeline of two independent inversion and then generation stages, we find the intermediate attention maps during inversions store better structure and motion information. We thus reform them to temporally casual attention and replace them in the generation progress. To further reduce the unnecessary semantic leakage of source video and enhance the editing quality, we then remix the temporally casual attentions via the cross-attention features of the source prompt as the mask.}
To edit videos consistently, we propose several techniques based on the pre-trained models. Firstly, in contrast to the straightforward DDIM inversion technique, our approach captures intermediate attention maps during inversion, which effectively retain both structural and motion information. These maps are directly fused in the editing process rather than generated during denoising. To further minimize semantic leakage of the source video, we then fuse self-attentions with a blending mask obtained by cross-attention features from the source prompt. Furthermore, we have implemented a reform of the self-attention mechanism in denoising UNet by introducing spatial-temporal attention to ensure frame consistency.
Yet succinct, our method is the first one to show the ability of zero-shot text-driven video style and local attribute editing from the trained text-to-image model. We also have a better zero-shot shape-aware editing ability based on the text-to-video model~\cite{tuneavideo}. \RM{Besides video, our unified method also achieves state-of-the-art performance in zero-shot image editing.\chenyang{Need exp or remove the zero-shot image}} Extensive experiments demonstrate our superior temporal consistency and editing capability than previous works.
% The code will be released.
% \chenyang{emphasize: our observation at inversion time} \xiaodong{replacing the bold part to the actual pipeline: \textbf{Specifically, we work on replacing and mixing the attention maps between the inversion and generation since the self-attention map keeps the structure of the original natural image and the cross-attention is semantic-related, after remixing, we replace them in the corresponding generation steps for denoising.}}
% \footnote{Since there is no general video diffusion model is publicly available, we use one-shot video generation method~(Tune-A-Video~\cite{tuneavideo}) as the pretrained video diffusion model for zero-shot video editing\xiaodong{can be removed if we actually zero-shot on video}.}.
\end{abstract}
\section{Introduction}

The ability to reason about plans is critical for performing long-horizon tasks \citep{erol1996hierarchical, sohn2018hierarchical, sharma-etal-2022-skill}, compositional generalization \citep{corona-etal-2021-modular} and generalization to unseen tasks and environments \citep{shridhar2020alfred}.
Consider a simple long-horizon planning scenario where a robot is tasked with preparing a meal and serving it on the table. 
This presents a non-trivial planning problem since the agent needs to understand the sequence of operations required to perform the task and search for the relevant objects in the unfamiliar environment by interacting with various objects. %



Large language models have been recently shown to possess commonsense knowledge about the world such as object affordances and physical dynamics \citep{ouyang2022training,chowdhery2022palm}.
Early approaches considered text based environments and fine-tuned PLMs to predict actions given the history of past observations and actions \citep{jansen-2020-visually,micheli-fleuret-2021-language,yao-etal-2020-keep}.
Recent work has used this ability to reason about plans from text instructions in simulated household environments with simplifying assumptions such as text-only environment observations or feedback \citep{huang2022language,ahn2022can,li2022pre,logeswaran-etal-2022-shot}.


We focus on \emph{visually grounded planning} with PLMs --- the ability to adapt plans based on interaction and visual feedback from the environment.
While PLMs have strong planning commonsense priors, predictions from a PLM may not be directly realizable in the environment since the observation and action spaces are unknown.
This requires \emph{grounding} the PLM in the environment and adapting it to observe visual feedback, which is highly non-trivial.
Some prior works assume the availability of a pre-trained affordance function \citep{ahn2022can} or a success detector \citep{mirchandani2021ella}.
Notably, SayCan \citep{ahn2022can} completely decouples the PLM from observation information by selecting actions that have both high affordability (through a pre-trained affordance model) and high PLM likelihood.
Although this partially addresses the grounding problem, the use of visual feedback for action affordance alone is limited.
Often an agent must choose one of many affordable actions using information from observations.
For example, a driving agent should re-navigate and possibly turn around when encountering a ``road closed'' sign, but both turning around and driving forward are indistinguishable to SayCan because they are both affordable and the PLM is blind to observations.

Another workaround explored in prior work is translating the information in the visual observations to text using a pre-trained captioning system \citep{shridhar2021alfworld,huang2022language}.
However, it can be difficult to faithfully describe an image in words and information is lost in this inherently noisy process, which limits the information available to the planner.



Recent work shows that PLMs can be adapted for various natural language tasks by inserting tunable embeddings or soft prompts at the input of the PLM (also called prompt tuning or prefix tuning)~\citep{li-liang-2021-prefix,lester-etal-2021-power}.
This approach also extends to multi-modal understanding tasks such as image captioning \citep{mokady2021clipcap} and VQA \citep{tsimpoukelli2021multimodal} where images are encoded as soft prompts and finetuned for the target task.
Transformer based architectures have also been successfully applied to offline Reinforcement Learning in recent work \citep{chen2021decision,janner2021offline,li2022pre,reid2022can}.

Taking inspiration from these works, we propose the simple approach of embedding visual observations (`visual prompts') and \textit{directly inserting them as PLM input embeddings}.
The visual encoder and PLM are jointly trained for the target task, an approach we call \textbf{\oursfull}~(\ours).
By teaching the PLM to use observations for planning in an end to end manner, we remove the dependency on external data such as captions and affordability information that was used in prior work.
We show that this simple approach performs better than prior PLM-based planning approaches on two embodied planning benchmarks based on ALFWorld~\citep{shridhar2021alfworld} and Virtualhome~\cite{puig2018virtualhome}.



\section{Related Work}

%Here we summarize prior work on transfer learning and property inference.

%\shortsection{Transfer Learning}
%%Transfer learning reuses features learned by pre-trained models for new tasks, with the pretext that inherent similarities in the generic features will be useful for the downstream tasks and hence reducing their cost of downstream training. Specifically, the downstream model trainer will use a pre-trained upstream model as the starting point for the downstream training, with inclusion of (or replacement with) the task-specific classification layer/module. The downstream model is then trained by either updating all layers of the model (including ones reused from upstream model) or freezing some earlier layers of the reused parts as the ``feature extractor'' and only updating the rest. The latter approach is more popular as the reused feature extractors can already learn useful feature representations and the training cost is also much lower and affordable for individuals with limited computational resources. We study the vulnerability of the latter transfer learning approach in this paper. 


%\shortsection{Transfer Learning} 
Several works have demonstrated risks associated with transfer learning across a variety of attack goals. Wang et al.~\cite{wang2018great} and Yao et al.~\cite{yao2019latent} consider manipulating the upstream model such that the fine-tuned downstream models contain backdoors, misclassifying test inputs that contain predefined backdoor triggers. These transfer manipulations are tailored to their particular attack goals and cannot be applied for the property inference goal considered in this paper. Zou et al.~\cite{zou2020privacy} study the threat of membership inference attacks on transfer learning, but with normally trained upstream models.  
%\dnote{its clear that the goals are different for these attacks, but how similar are the methods?} \ynote{similarity of the methods? more details about the methods? do not know what is expected here}
%In contrast, we investigate the possibility of boosting the effectiveness of property inference by manipulating the upstream model training. % Schuster et al.~\cite{schuster2020humpty} show that the attacker can modify the corpus on which the word embedding is trained such that the downstream NLP models which use that embedding will behave abnormally.

%\shortsection{Property Inference}
The risk of property inference was introduced by Ateniese et al.~\cite{ateniese2015hacking}, % introduces the threat of inferring properties of the training data from pre-trained models, 
and several subsequent works have developed property inference (also known as distribution inference) attacks~\cite{Wang2022GroupPI, suri2022formalizing, Jurez2022BlackBoxAF, Hartmann2022DistributionIR}.
% Ganju et al.~\cite{ganju2018property} and Suri and Evans~\cite{suri2022formalizing} 
These works study property inference against normally trained models, and they launch attacks using a variety of black-box and white-box attacks. All the white-box attacks use meta-classifiers, which take the permutation-invariant representation~\cite{ganju2018property} of the model parameters as the features. We use the state-of-the-art white-box attack~\cite{suri2022formalizing} in our experiments.
%We will use the state-of-the-art white-box method proposed by Ganju et al.~\cite{ganju2018property} and later extended by suri et al.~\cite{suri2022formalizing} in this paper.
%\dnote{do we use these attacks?} 
Melis et al.~\cite{melis2019exploiting} and Zhang et al.~\cite{zhang2021leakage} focus on property inference in distributed training scenarios. In their settings, the attacker is a participant in the global model training and conducts property inference using meta-classifiers that are trained on model outputs or gradients. Similarly, Suri et al.~\cite{suri2022subject} focus on federated learning settings where the attacker is a participant (or the central server) that utilizes black-box attacks for inferring membership of data from particular subjects. %\dnote{if we use black-box attacks, explain which ones, or how ours are related to previous ones} 
For our experiments, We improve the black-box meta-classifier proposed by Zhang et al.~\cite{zhang2021leakage} using the ``query tuning'' technique in Xu et al.~\cite{xu2019detecting}. 

The closest works to ours are Chase et al.~\cite{saeed} and Chaudhari et al.~\cite{Chaudhari2022SNAPEE}, which both consider a scenario where the attacker can manipulate some of the training data of the model to induce a model that significantly increases property inference risk.
% \dnote{it enables precise property inference attacks?}.
These works assume an adversary with the ability to poison the victim's training data, while the adversary in our scenario has no access to the victim's training data, and therefore, their methods are not applicable.
% \dnote{example how different from ours, and why the methods are not applicable}
%Thus, their methods are not applicable to our transfer learning scenario.
%Their methods rely on inducing certain behavior correlated with the properties to be inferred, and thus are not applicable to our transfer learning scenario. \anote{Still a bit unclear why that is the case.}
%
There are also works similar to ours that leverage ``adversarial initializations'' for attack purposes.
% \cite{grosse2019adversarial, boenisch2021curious, wen2022fishing, fowl2021robbing}.
Grosse et al.~\cite{grosse2019adversarial} focus on scenarios where the attacker can control the parameter initialization of a model, and demonstrate that the attacker can use special initializations to damage the performance of the trained model. %This attack is orthogonal to ours.
Other works \cite{boenisch2021curious, wen2022fishing, fowl2021robbing} show that the malicious central server in a federated learning protocol can reconstruct some training samples via falsifying the global model in some training rounds and then analyzing the submitted gradients. These kinds of attacks do not apply to our transfer-learning scenario since the attacker cannot access the downstream gradients, and can only manipulate the upstream training.

\iffalse %%%%%%%%%%%%%%%%%%%%%%%%%%%%%%%%

In this section, we provide the background and also the summary of prior attacks on transfer learning (Section~\ref{sec:transfer_learning}) and property inference (Section~\ref{sec:property_inference}). Then, we introduce the closely related manipulation attacks against machine learning models to boost different privacy risks in Section~\ref{sec:active_inference_attacks}.

%\anote{Do we really need a dedicated section for this? It's barely 2 paragraphs right now.}

%\dnote{the most closely related work to ours are works that attempt to amplify inference attacks by poisoning models, the two most relevant I know of are \url{https://www.computer.org/csdl/proceedings-article/sp/2022/131600b569/1CIO8nmuota} and \url{https://arxiv.org/abs/2204.00032}, but need to look thoroughly for others. We should definitely be describing this and relating it to our work, probably in the introduction. Most of what is here is Background, but should be clear what this section is for (not muddling background and related work)}

\subsection{Transfer Learning} \label{sec:transfer_learning}
Transfer learning reuses features learned by pre-trained models for new tasks, with the pretext that inherent similarities in generic features can be useful for downstream tasks, thus reducing the cost of downstream training. Specifically, the downstream model trainer uses a pre-trained upstream model as the starting point for downstream training, with the inclusion (or replacement) of task-specific classification layers/modules. The downstream model is then trained by either updating all layers of the model (including ones reused from the upstream model) or freezing some earlier layers of the reused parts as the ``feature extractor'' and only updating the rest. The latter approach is more popular as the reused feature extractors can already learn useful feature representations and the training cost is also much lower and affordable for individuals with limited computational resources. We study the vulnerability of the latter transfer learning approach in this paper. 
%mainly in two ways:  1) all the layers (including ones reused from ) and tune the full model; the other one is to freeze some earlier layers of the model as the feature extractor and only tune the rest later layers. The second update strategy could achieve better efficiency since the frozen layers can already produce meaningful feature representations~\cite{wang2018great,yao2019latent}, and we will study the transfer learning using this strategy. 

Recently, various attacks have been proposed for the transfer learning setting, but with different attack goals from ours. Wang et al.~\cite{wang2018great} generate adversarial examples against black-box student models that transfer knowledge from publicly available teacher models without repeated queries. Yao et al.~\cite{yao2019latent} propose to manipulate the upstream model such that the downstream models derived from the upstream model contain backdoors, which would misclassify test inputs that contain some predefined backdoor triggers. Zou et al.~\cite{zou2020privacy} study the threat of membership inference attacks on transfer learning and the upstream models are trained normally. In contrast, we investigate the possibility of boosting the effectiveness of property inference by manipulating the upstream model training. Schuster et al.~\cite{schuster2020humpty} show that the attacker can modify the corpus on which the word embedding is trained such that the downstream NLP models which use that embedding will behave abnormally.

%This additionally allows model trainers to achieve satisfactory performance with limited training samples, leading to reduced computational costs. The most common approach reuses parameters in the earlier layers of the pre-trained model, either by fixing them as the feature extractor or just using them for initialization, to conduct downstream training.

\subsection{Property Inference} \label{sec:property_inference}

\shortsection{Property Inference Attacks} In property inference attacks, the adversary aims to infer some sensitive properties of some data, given a model trained on it. For example, the adversary may be interested in sensitive properties like the presence of people of a specific race in the dataset~\cite{ateniese2015hacking, melis2019exploiting}), or even be curious about the 
the statistics of the training set (e.g, the ratio of people with a specific gender~\cite{saeed, ganju2018property, suri2022formalizing, zhang2021leakage}).


Ateniese et al.~\cite{ateniese2015hacking} were the first to identify the threat of inferring properties of the training data from pre-trained models. Ganju et al.~\cite{ganju2018property} and Suri and Evans~\cite{suri2022formalizing} 
study property inference against normally trained models, and they launch attacks using white-box meta-classifiers, which utilize the permutation-invariance representation~\cite{ganju2018property} of the model parameters, while other works focus on distributed training~\cite{zhang2021leakage} where the attacker is a participant in the global model training and conducts property inference using meta-classifiers trained on model outputs. Similarly, Suri et al.~\cite{suri2022subject} focus on federated learning, where the attacker is a participant (or the central server) that utilizes black-box attacks for inferring membership of data from particular subjects. Chase et al.~\cite{saeed} propose an active property inference attack for data poisoning scenarios, which we will cover and compare to in Section~\ref{sec:active_inference_attacks}.

%The closest work to ours are by Chase et al.~\cite{saeed} and Tramer et al.~\cite{tramer2022truth}. In their work, the attacker can manipulate some of the training data of the model such that a model trained (from scratch) on the poisoned data has an increased inference risk. However, their methods are not applicable to the transfer learning scenario. 
%In this work, we will focus on the property inference in transfer learning scenarios in which the attacker releases the upstream model and infer sensitive properties of the downstream models tuned from that upstream model.
% 

\shortsection{Defenses}
Defending against property inference attacks is an open problem. There are no studies in the current literature on active adversaries, and only a couple on passive ones. Ma et. al.~\cite{ma2021nosnoop} propose a defense against property inference attacks on data batches in the  collaborative learning setting. However, adversaries in the transfer-learning setting do not have access to batch-wise gradients of the downstream trainer. Chen and Ohrimenko~\cite{chen2022protecting} utilize mechanisms that add carefully-crafted noise to features to provide theoretical guarantees against inference adversaries, but focus on query-based access to the underlying dataset, not a machine learning model trained on it. These existing defenses thus do not apply to our threat model.

%propose a framework that reduces property inference to Boolean functions of individual members, posing the ratio of members satisfying the given function in a dataset as the property. These property inference attacks have since then been proposed as distribution inference attacks~\cite{suri2022formalizing}, presenting such attacks as inferring properties of the distributions used to sample datasets, differentiating them from exact inference attacks like dataset inference~\cite{maini2021dataset}. Nearly all property inference attacks use meta-classifiers to perform inference: training models on versions of datasets with and without the target property, followed by training a meta-classifier on top of these classifiers's model representations. These representations can take several forms: using model weights themselves with permutation-invariance~\cite{ganju2018property}, or model activations or logits for a generated set of query points~\cite{xu2019detecting}. However, the capability of such approaches is limited: the most that these attacks have been shown to work is medium-sized convolutional networks on the CelebA dataset~\cite{suri2022formalizing}.


\subsection{Active Privacy Attacks} \label{sec:active_inference_attacks}
% Perhaps the closely related works to ours as ones that proactively enhance the effectiveness of privacy attacks by manipulating the model training process in certain ways~\cite{saeed, melis2019exploiting, nasr2019comprehensive, tramer2022truth}. 
%shown that the adversary can, by using proactive ways, achieve stronger attacks that infer private information from deep learning systems~\cite{nasr2019comprehensive, melis2019exploiting, tramer2022truth, saeed}. In this section, we introduce the ones that are close to ours.

In the decentralized federated learning training, by submitting specially crafted gradients to the central server, malicious agents can increase membership inference risk~\cite{nasr2019comprehensive} and property inference risks~\cite{melis2019exploiting} of other benign agents' training data. However, these attacks do not apply to transfer learning scenario, as the attacker cannot control model gradients of downstream training. In the centralized setting, researchers propose attacks to poison the victim's training data such that the impacts of attribute inference and membership inference~\cite{tramer2022truth} and property inference~\cite{saeed} attacks are amplified on the poisoned model.
The ability to poison the victim's data is a threat model orthogonal to ours, since we have no access to the victim's downstream data. While there is scope to combine such approaches for stronger attacks (albeit with stronger access assumptions), we choose to focus on the scenario with no read/write access to the victim's data.

\fi %%%%%%%%%%%%%%%%%%%%%%%%%%%%%%%%

\section{Linear Shortcut Across Blocks}
\label{sec:layer_jump}

To use a hidden representation from layer $\ell<L$ as a final representation, we propose to cast it using linear regression, while skipping the computation in-between these layers. More generally, this approach can be applied to cast any $\ell$-th hidden representation to any subsequent layer $\ell'>\ell$.


\subsection{Method}
\label{subsec:methodology_linear_shortcut}

Given a source layer $\ell$ and a target layer $\ell'$ such that $0 \leq \ell < \ell' \leq L$, our goal is to learn a mapping
%$A_{\ell', \ell} \in \mathbb{R}^{d_h \times d_h}$
from hidden representations at layer $\ell$ to those at layer $\ell'$. To this end, we first collect a set of corresponding hidden representation pairs $(h^\ell, h^{\ell'})$. Concretely, we run a set $\mathcal{T}$ of input sequences through the model, and for each input $s$, we extract the hidden representations $h_{i_s}^{\ell}, h_{i_s}^{\ell'}$, where $i_s$ is a random position in $s$.
Next, we learn a matrix $A_{\ell', \ell} \in \mathbb{R}^{d_h \times d_h}$ by fitting linear regression over $\mathcal{T}$, i.e., $A_{\ell', \ell}$ is a numerical minimizer for:
$$ A \mapsto \sum_{s \in \mathcal{T}} || A \cdot h_{i_s}^\ell - h_{i_s}^{\ell'} ||^2,$$ 
and define the mapping of a representation $h$ from layer $\ell$ to layer $\ell'$ as:
\begin{equation}
\label{eq:linear_jump}
    \matl{} (h) \coloneqq A_{\ell', \ell} \cdot h.
\end{equation}


\subsection{Baseline}
\label{subsec:baseline}

We evaluate 
% our method against 
the prevalent approach of ``reading'' hidden representations directly, without any transformation. 
Namely, the propagation of a hidden representation from layer $\ell$ to layer $\ell'$ is given by the identity function, dubbed \id{}:

$$ \idl{} (h) \coloneqq h.$$

% Notably, 
This baseline 
assumes that representations at different layers operate in the same linear space.

\subsection{Quality of Fit}
\label{subsec:experiments_r2}

We first evaluate our method by measuring how well the learned linear mappings approximate the representations at the target layer. To this end, we calculate the (coordinate-averaged) $r^2$-score of our mapping's outputs with respect to the representations obtained from a full inference pass, and compare to the same for the \id{} baseline.


\paragraph{Models.}

We use \gpt{} \cite{radford2019language}, a decoder-only auto-regressive LM, with $L = 48$, $d_h = 1600$, and \bert{} \cite{devlin-etal-2019-bert}, an encoder-only model trained with masked language modeling, with $L=24$, $d_h=1024$.
% \footnote{\label{footnote:hf}We use models and data from Huggingface \cite{wolf-etal-2020-transformers,lhoest-etal-2021-datasets}.}
%For masked token prediction, we use a masked LM head pre-trained for our \bert{} model.

% \footnote{Specifically, we use the Huggingface Transformers \cite{wolf-etal-2020-transformers} implementations of all these models.}

%\sy{We use \gpt{} \cite{radford2019language}, a decoder-only auto-regressive LM, coming in four scales; $\texttt{gpt2}$ ($L = 12$, $d_h = 768$), $\texttt{gpt2-medium}$ ($L = 24$, $d_h = 1024$), $\texttt{gpt2-large}$ ($L = 36$, $d_h = 1280$) and $\texttt{gpt2-xl}$ ($L = 48$, $d_h = 1600$). Also, we use \bert{} \cite{devlin-etal-2019-bert}, an encoder-only model trained with masked language modeling, coming in two scales;  \texttt{bert-base-uncased} ($L=12$, $d_h=768$) and \texttt{bert-large-uncased} ($L=24$, $d_h=1024$). For masked token prediction, we use masked LM heads pre-trained for our models. Specifically, we use the Huggingface Transformers \cite{wolf-etal-2020-transformers} implementations of all these models. The plots presented in this section are for $48$-layered \gpt{} and $24$-layered \bert{}.}

%\sy{We use \gpt{} \cite{radford2019language}, a decoder-only auto-regressive LM, in the Huggingface \cite{wolf-etal-2020-transformers} implementation\footnote{\url{https://huggingface.co/gpt2}}, coming in four scales; $\texttt{gpt2}$ ($L = 12$, $d_h = 768$), $\texttt{gpt2-medium}$ ($L = 24$, $d_h = 1024$), $\texttt{gpt2-large}$ ($L = 36$, $d_h = 1280$) and $\texttt{gpt2-xl}$ ($L = 48$, $d_h = 1600$). Also, we use \bert{} \cite{devlin-etal-2019-bert}, an encoder-only model trained with masked language modeling, in the Hugginface implementation, coming in two scales;  \texttt{bert-base-uncased}\footnote{\url{https://huggingface.co/bert-base-uncased}} ($L=12$, $d_h=768$) and \texttt{bert-large-uncased}\footnote{\url{https://huggingface.co/bert-large-uncased}} ($L=24$, $d_h=1024$). For masked token prediction, we use the \texttt{BertForMaskedLM} heads from Huggingface, pretrained for these models. The plots presented in this section are for $48$-layered \gpt{} and $24$-layered \bert{}.}

\paragraph{Data.}
We sample random sentences from Wikipedia,
% \footref{footnote:hf} 
collecting 9,000 (resp. 3,000) sentences for the training set $\mathcal{T}$ (resp. validation set $\mathcal{V}$).\footnote{We use sentences rather than full documents to simplify the analysis.}
%\sy{We use two data sources to evaluate our method. One is Wikiepdia \cite{lhoest-etal-2021-datasets}\footnote{\url{https://huggingface.co/datasets/wikipedia}}; we use \texttt{spaCy}\footnote{\url{https://spacy.io/}} to divide documents into sentences\footnote{We use sentences rather than full documents to simplify the analysis.}\footnote{We pick randomly a Wikipedia document and then pick randomly a sentence ending in a newline character in it. \sy{[maybe this footnote is not needed?]}}, collecting 9,000 (resp. 3,000) random sentences for the training set $\mathcal{T}$ (resp. validation set $\mathcal{V}$). The second is a news article sentences dataset, the 10K English 2020 news sentences corpus
% \footnote{\url{https://downloads.wortschatz-leipzig.de/corpora/eng_news_2020_10K.tar.gz}} from the Leipzig Corpora Collection \cite{goldhahn-etal-2012-building}, which we randomly divide into a training set $\mathcal{T}$ consisting of 9,000 examples and a validation set $\mathcal{V}$ consisting of 1,000 examples.
% We truncate sentences to the maximal token length allowed by the model \mg{do we ever need to truncate? a sentence has about 10 words and the max. input len is thousands} \sy{[I surely did not need to in Leipzig, but discovered (via a transformers runtime warning) that I do need to for some (probably a minority) of the Wikipedia sentences. This probably has to do with that it is not really ``sentences" necessarily, for example, I noticed that it has some listings or something like that (bulleted items)... So some minority might get very long I guess...]}.
For each example $s$, we select a random position $i_s$ and extract the hidden representations $h_{i_s}^{\ell}$ at that position from all the layers.
For \bert{}, we first replace the input token at position $i_s$ with a \mask{} token, as our motivation is interpreting predictions, which are obtained via masked tokens in \bert{} (see \S\ref{subsec:BERT}).
Thus, in this case, the hidden representations we consider
%in the case of \bert{}
are of \mask{} tokens only.
%As we observed highly similar results for the two data sources across all our experiments, throughout the paper we will mainly report results for Wikipedia (except for \S\ref{sec:robustness}, where we cross-validate).


\begin{figure}[t]
\includegraphics[scale=0.2]{figs/r2_scores_48.pdf}
% \includegraphics[width=\columnwidth]{figs/r2_scores_48.pdf}
\caption{The coordinate-averaged $r^2$-score of $\matl{}$ (left) and $\idl{}$ (right) (\gpt{}).}
\label{fig:r2_scores}
\end{figure}


\begin{figure}[t]
\setlength{\belowcaptionskip}{-10pt}
\includegraphics[scale=0.2]{figs/bertmask_r2_scores_24.pdf}
% \includegraphics[width=\columnwidth]{figs/bertmask_r2_scores_24.pdf}
\caption{The coordinate-averaged $r^2$-score of $\matl{}$ (left) and $\idl{}$ (right) (\bert{}).}
\label{fig:bertmask_r2_scores}
\end{figure}



\paragraph{Evaluation.}
For every pair of layers $\ell, \ell'$, such that $0 \leq \ell < \ell' \leq L$, we use the training set $\mathcal{T}$ to fit linear regression as described in \S\ref{subsec:methodology_linear_shortcut}, and obtain a mapping $\matl{}$. 
Next, we evaluate the quality of $\matl{}$ as well as of $\idl{}$ using the $r^2$-coefficient, uniformly averaged over all coordinates. Concretely, we compute the $r^2$-coefficient of each of the predicted representations $\matl{} (h_{i_s}^{\ell})$ and $\idl{} (h_{i_s}^{\ell})$ versus the true representations $h_{i_s}^{\ell'}$
over all $s \in \mathcal{V}$.
%as we vary $s \in \mathcal{V}$.
%for every $s \in \mathcal{V}$.



\paragraph{Results.}
Results for \gpt{} and \bert{} are presented in Figs.~\ref{fig:r2_scores} and~\ref{fig:bertmask_r2_scores}, respectively.
In both models, \mat{} consistently yields better approximations than \id{}, as it obtains higher $r^2$-scores (in blue) across the network. 
This gap between \mat{} and \id{} is especially evident in \bert{}, where \id{} completely fails to map the representations between most layers, suggesting that hidden representations are modified  substantially by every transformer block.
Overall, this highlights the shortcoming of existing practices to inspect representations in the same linear space, and the gains from using our method to approximate future layers.
% in the network.
\section{Linear Shortcut for Language Modeling}
\label{sec:prediction}

We saw that our method approximates future hidden representations substantially better than a naive propagation. 
In this section, we will show that this improvement also translates to better predictive abilities from earlier layers. Specifically, we will use our method to estimate how often intermediate representations encode the final prediction, in the context of two fundamental LM tasks; next token prediction and masked token prediction.

\paragraph{Evaluation Metrics.}
Let $h, h' \in \mathbb{R}^{d_h}$ be a final representation and a substitute final representation obtained by some mapping, and denote by $\delta (h), \delta (h') \in \mathbb{R}^{d_v}$ their corresponding output probability distributions (obtained through projection to the output vocabulary -- see details below). 
We measure the prediction quality of $h'$ with respect to $h$ using two metrics:
\begin{itemize}
[leftmargin=*,topsep=1pt,parsep=1pt]
    \item \textbf{Precision@$k$} ($\uparrow$ is better): This checks whether the token with the highest probability according to $\delta(h')$ appears in the top-$k$ tokens according to $\delta(h)$. Namely, we sort $\delta(h)$ and assign a score of $1$ if $\arg\max(\delta(h'))$ appears in the top-$k$ tokens by $\delta(h)$, and $0$ otherwise.
    
    \item \textbf{Surprisal} ($\downarrow$ is better): We measure the minus log-probability according to $\delta(h)$, of the highest-probability token according to $\delta(h')$. Intuitively, low values mean that the model sees the substitute result as probable and hence not surprising.
\end{itemize}

\noindent We report the average Precision@$k$ and Surprisal over the validation set $\mathcal{V}$.



\subsection{Next Token Prediction}
\label{subsec:next_token_prediction_task}

Auto-regressive LMs output for every position a probability distribution over the vocabulary for the next token. Specifically, the output distribution for every position $i$ is given by $\delta (h_i^L)$, where:
\begin{equation}\label{eq:output_distribution}
    \delta (h) = \texttt{softmax} ( E^\top \cdot h) \in \mathbb{R}^{d_v}
\end{equation}
For some LMs, including \gpt{}, a layer normalization $\texttt{ln\_f}$ is applied to the final layer representation before this conversion (i.e., computing $\delta (\texttt{ln\_f}(h))$ rather than $\delta (h)$).

Recall that our goal is to measure how well this distribution can be estimated from intermediate representations, i.e. estimating $\delta (h_i^L)$ from $\delta (h_i^\ell)$ where $\ell<L$. To this end, we first run examples from the validation set through the model, while extracting for each example $s$ the hidden representation of a random position $i_s$ at every layer. Next, we apply our mappings $\matlL{}$ and the $\idlL{}$ baseline to cast the hidden representations of every layer $\ell$ to final layer substitutes (see \S\ref{sec:layer_jump}). Last, for each layer, we convert its corresponding final-layer substitute to an output distribution (Eq.~\ref{eq:output_distribution}) and compute the average Precision@$k$ (for $k=1,5,10$) and Surprisal scores with respect to the final output distribution, over the validation set.

\paragraph{Results.}
Figs.~\ref{fig:pre} and~\ref{fig:surp} show the average Precision@$k$ and Surprisal scores per layer in $48$-layered \gpt{}, respectively (the plots for the other \gpt{} models are presented in \S\ref{sec:app_scale}). Across all layers, \mat{} outperforms \id{} in terms of both scores, often by a large margin (e.g. till layer $44$ the Precision@$1$ achieved by \mat{} is bigger than that of $\id{}$ by more than $0.2$). 
This shows that linear mappings enable not just better estimation of final layer representations, but also of the predictions they induce. Moreover, the relatively high Precision@$k$ scores of \mat{} in early layers ($0.62$-$0.82$ for $k=10$, $0.52$-$0.74$ for $k=5$, and $0.28$-$0.45$ for $k=1$) suggest that early representations already encode a good estimation of the final prediction. Also, the substantially lower Surprisal scores of \mat{} compared to \id{} imply that our method allows for a more representative reading into the layer-wise prediction-formation of the model than allowed through direct projection to the vocabulary.

\begin{figure}[t]
\centering
\includegraphics[scale=0.4]{figs/pre_48.pdf}
\caption{Precision@$k$ ($k = 1,5, 10$) of $\matlL{}$ and $\idlL{}$ for next token prediction in $48$-layered \gpt{}.}
\label{fig:pre}
\end{figure}

\begin{figure}[t]
\centering
\includegraphics[scale=0.35]{figs/surp_48.pdf}
\caption{Surprisal for $\matlL$ and the baseline $\idlL{}$ ($48$-layered \gpt{} next token prediction task). A 95\% confidence interval surrounds the lines.}
\label{fig:surp}
\end{figure}

\subsection{Masked Token Prediction}
\label{subsec:BERT}

We now conduct the same experiment for the task of masked language modeling, where the model predicts a probability distribution of a masked token in the input rather than the token that follows the input. Unlike next token prediction, where the output distribution is computed from representations of varying input tokens, in masked token prediction the output is always obtained from representations of the same input token (i.e. \texttt{[MASK]}).

For this experiment, we use \bert{}, on top of which we use a pretrained masked language model head $\delta$; given a token sequence $s$, a \mask{} token inside it and its final representation $h$, $\delta (h) \in \mathbb{R}^{d_v}$
 is a probability distribution over tokens giving the model's assessment
 of the likelihood of tokens to be fitting in place of the \mask{} token in $s$.


\begin{figure}[t]
\centering
\includegraphics[scale=0.4]{figs/bertmask_pre_24.pdf}
\caption{Precision@$k$ ($k = 1,5, 10$) for  $\matlL{}$ and the baseline $\idlL{}$ ($24$-layered \bert{} masked token prediction task).}
\label{fig:bertmask_pre}
\end{figure}

\begin{figure}[t]
\centering
\includegraphics[scale=0.35]{figs/bertmask_surp_24.pdf}
\caption{Surprisal for $\matlL{}$ and the baseline $\idlL{}$ ($24$-layered \bert{} masked token prediction task). A 95\% confidence interval surrounds the lines.}
\label{fig:bertmask_surp}
\end{figure}

\paragraph{Results.}
Figs.~\ref{fig:bertmask_pre} and~\ref{fig:bertmask_surp} present the average Precision@$k$ and Surprisal scores per layer in $24$-layered \bert{} (the plots for the $12$-layered \bert{} model are presented in \S\ref{sec:app_scale}), overall showing trends similar to those observed for next token prediction in \gpt{} (\S\ref{subsec:next_token_prediction_task}). This is despite the differences between the two tasks and the considerable architectural differences between \bert{} and \gpt{}.
Notably, the superiority of \mat{} over \id{} in this setting is even more prominent; 
while \mat{}'s precision is between $0.2-0.6$ in the first ten layers (Fig.~\ref{fig:bertmask_pre}), \id{}'s precision for all values of $k$ is close to zero, again strongly indicating that our method allows for better reading into early layer hidden representations. 
More generally, \mat{} improves the Precision@$1$ of \id{} by more than $17\%$ at most layers, and unveils that a substantial amount of predictions ($>25\%$ starting from layer $3$) appear already in the very first layers.
Interestingly, the (rough) divide between the first half of layers and last half of layers for $\id{}$ in Figs.~\ref{fig:bertmask_pre},~\ref{fig:bertmask_surp} seems to align with the two-hump shape of the blue region for $\mat{}$ in Fig.~\ref{fig:bertmask_r2_scores}.

\paragraph{Analysis.}
We manually compare the predictions of our mapping $\matlL{}$ with $\idlL{}$, for a $24$-layered \bert{} model.  Concretely, we select 50 random sentences from the Leipzig dataset. Next, for each layer $\ell$, we manually analyze how many of the top-$5$ tokens according to $\matlL{}$ and $\idlL{}$ fit into context. We consider a token to fit into context if it is grammatically plausible within the sentence (see Tab.~\ref{tab:manual} for concrete examples).
In the resulting $1250$ instances (i.e. $50$ sentences $\times$ $25$ representations), we observe a substantially higher plausibility rate of $85.36\%$ for \mat{} compared to $52.8\%$ for \id{}. In fact, only in less than $4.3\%$ of the instances there are more plausible tokens among the top-$5$ tokens according to \id{} than among the top-$5$ tokens according to \mat{}, further supporting the Surprisal results above.

\begin{table*}
\footnotesize
\setlength{\belowcaptionskip}{-15pt}
\begin{tabular}{p{0.3\linewidth}ccccc}
& $\texttt{id}_{4 \rightarrow 24}$ & $\texttt{mat}_{4 \rightarrow 24}$ & $\texttt{id}_{12 \rightarrow 24}$ & $\texttt{mat}_{12 \rightarrow 24}$ & $\texttt{id}_{24 \rightarrow 24}$ \\ \midrule
\multirow{5}{=}{aldridge had shoulder surgery in \mask{}.} & fellowship & \tcbox{time} & cyclist & \tcbox{2009} & \tcbox{september} \\
& employment & \tcbox{it} & emergencies & \tcbox{2008} & \tcbox{november} \\
& agreement & her & seniors & \tcbox{2010} & \tcbox{december} \\
& \#\#ostal & them & cycling & \tcbox{2006} & \tcbox{august} \\
& \#\#com & work & \tcbox{pennsylvania} & \tcbox{2007} & \tcbox{july} \\ \midrule
\multirow{5}{=}{on your next view you will be asked to \mask{} continue reading.} & \#\#com & be & be & be & \tcbox{please} \\
& accreditation & get & undergo & \tcbox{please} & \tcbox{simply} \\ 
& $	\copyright$ & go & spartans & help & \tcbox{also} \\ 
& fellowship & \tcbox{help} & seniors & \tcbox{simply} & \tcbox{again} \\ 
& summer & have & * & say & \tcbox{immediately} \\ \bottomrule
\end{tabular}
\caption{Examples of top-$5$ predictions at layers $4$, $12$ and $24$, under the mappings $\matlL{}$ and $\idlL{}$, for a $24$-layered \bert{} model. Grammatically plausible predictions (according to a human annotator) are marked in \tcbox{blue}. Note that at layer $24$ the predictions of $\matlL{}$ and $\idlL{}$ are the same (by definition).} 
\label{tab:manual}
\end{table*}

\section{Implication to Early Exiting}
\label{sec:applications}

%The fact that it is often possible to approximate
The possibility of approximating
the final prediction already in the early layers has important implications for efficiency; applying our linear mapping instead of executing transformer blocks of quadratic time complexity, could save a substantial portion of the computation. In this section, we demonstrate this in the context of early exiting.

When 
% performing transformer model inference under 
using an early exit strategy \cite{schwartz-etal-2020-right, xin-etal-2020-deebert, schuster2022confident}, one aims at deciding dynamically at which layer to stop the computation and ``read'' the prediction from the hidden representation of that layer.
More precisely, under a confidence measure paradigm, one decides to stop the computation for a position $i$ at layer $\ell$ based on a confidence criterion, that is derived from casting the hidden representation $h_i^\ell$ as a final-layer representation and converting it to an output probability distribution. Specifically, following \citet{schuster2022confident}, a decision to exit is made if the difference between the highest and the second highest probabilities is bigger than $$ 0.9 \cdot \lambda + 0.1 \cdot {\rm exp} (-4 i / N),$$
where $N$ is the average length of the input until position $i_s$ for $s \in \mathcal{V}$, and $\lambda$ is a hyper-parameter.

\begin{figure}[t]
\setlength{\belowcaptionskip}{-10pt}
\centering
\includegraphics[width=\columnwidth]{figs/ee_gpt2bert.pdf}
\caption{Precision@$1$ with early exit and ``fixed exit'', applied to the $24$-layer \gpt{} for next token prediction (left) and the $24$-layer \bert{} for masked token prediction (right). Varying the confidence parameter $\lambda$, the $x$-coordinate is the average number of layers processed before an early exit decision is reached.}
\label{fig:ee_gpt2bert}
\end{figure}

\quash{
\begin{figure}[t]
\setlength{\belowcaptionskip}{-10pt}
\centering
\includegraphics[scale=0.35]{figs/ee_pre1_24.pdf}
\caption{Precision@$1$ for the various early exit methods, and previous ``fixed exit'' methods for comparison ($24$-layer \gpt{} next token prediction task). Varying the confidence parameter $\lambda$, the $x$-coordinate is the average number of layers processed before an early exit decision is reached.}
\label{fig:ee_pre1}
\end{figure}
}

\paragraph{Experiment.}
We assess the utility of our mapping $\matlL{}$ for early exit as a plug-and-play replacement for $\idlL{}$, through which intermediate representations are cast into final-layer representations.
We use \gpt{} for the next token prediction and \bert{} for masked token prediction (both with 24 layers).
We run each of the models over the validation set examples, while varying the confidence parameter $\lambda$ and using either $\idlL{}$ or $\matlL{}$ for casting intermediate representations.
Furthermore, we compare these early exit variants to the ``fixed exit'' strategy from \S\ref{sec:prediction}, where the computation is stopped after a pre-defined number of layers rather than relying on a dynamic decision.
We evaluate each variant in terms of both prediction's accuracy, using the Precision@$1$ metric (see \S\ref{sec:prediction}), and efficiency, measured as the average number of transformer layers processed during inference.


\paragraph{Results.}
%Figs.~\ref{fig:ee_pre1} and~\ref{fig:bertmask_ee_pre1}
Fig.~\ref{fig:ee_gpt2bert}
plots the average Precision@$1$ score against the average number of layers processed, for $24$-layer \gpt{} and $24$-layer \bert{}. For both models, under an early exit strategy our mapping \mat{} again provides a substantial improvement over \id{}.
For example, aiming at $95\%$ average precision, \mat{} saves $\sim3.3$ ($13.8$\%) layers in \gpt{} compared to only $\sim1.4$ ($5.9$\%) layers by \id{}, and $\sim4.8$ ($20$\%) layers in \bert{} versus $\sim3.5$ ($14.6$\%) layers by \id{}.
These results highlight the potential gains prominent early exit methods can obtain by using our method.
Notably, in both models and for each of the mapping methods, early exit obtains better results than fixed layer exit, as expected. 

\quash{
\begin{figure}[t]
\setlength{\belowcaptionskip}{-10pt}
\centering
\includegraphics[scale=0.35]{figs/bertmask_ee_pre1_24.pdf}
\caption{Precision@$1$ for the various early exit methods, and previous ``fixed exit'' methods for comparison ($24$-layer \bert{} masked token prediction task). Varying the confidence parameter $\lambda$, the $x$-coordinate is the average number of layers processed before an early exit decision is reached.}
\label{fig:bertmask_ee_pre1}
\end{figure}
}
\section{Linear Shortcut Across Sub-Modules}
\label{sec:submodules}

% Our experiments show that
% , despite the commonly-applied simplification by interpretability works, transformer layers do not operate in the same linear space and 
% there is a major gap in approximating future representations using an identity mapping (\S\ref{sec:layer_jump}, \S\ref{sec:prediction}).
% Here, 
In this section, we investigate whether discrepancies across layers result from specific sub-modules or are a general behaviour of all sub-modules in the network.  
This is done by extending our approach to test how well particular components in transformer blocks can be linearly approximated. 


\paragraph{Method.}

Consider \gpt{} for definiteness, then:
% we have 
$$ \texttt{b}_{\ell} = \texttt{b}_{\ell}^{\texttt{ffn}} \circ \texttt{b}_{\ell}^{\texttt{attn}}$$ 
% with
\begin{equation}\label{eq:attn} \texttt{b}^{\texttt{attn}}_{\ell} (H) = \texttt{attn}_{\ell} (\texttt{ln1}_{\ell} (H)) + H,\end{equation} 
where $\texttt{attn}_{\ell}$ is
%a multi-head self-attention
a MHSA
layer and \texttt{ln1} is a layer normalization (LN), and 
$$ \texttt{b}^{\texttt{ffn}}_{\ell} (H) = \texttt{ffn}_{\ell} (\texttt{ln2}_{\ell} (H)) + H,$$  
where $\texttt{ffn}_{\ell}$ is
%a feed-forward network
an FFN
layer and $\texttt{ln2}$ is a LN.
\quash{
Given a block $\texttt{b}_\ell$ and one of its sub-modules $\texttt{ln1}_\ell, \ \texttt{attn}_\ell, \ \texttt{ln2}_\ell$, or $\texttt{ffn}_\ell$, we fit linear regression approximating the output of the sub-module given its input and then use it in order to define mappings, as we now describe.
}
Given a block $\texttt{b}_\ell$ and one of its sub-modules $\texttt{ln1}_\ell, \ \texttt{attn}_\ell, \ \texttt{ln2}_\ell$, or $\texttt{ffn}_\ell$, we fit linear regression approximating the output of the sub-module given its input, and then use it to define mappings $\matattnl{}$, $\matlnl{}$ and $\matffl{}$.
%We provide the definition of $\matattnl{}$ below, and that of the other two in App. \ref{sec:app_submodule_skip_description}.
We provide the formal definitions of these mappings in App. \ref{sec:app_submodule_skip_description}.
\iffalse
\paragraph{$\matattnl{}$.}
%Illustrating this on $\texttt{attn}_\ell$ for definiteness,
For an input $s$, let $v^\ell_{i_s}$ be the vector at position $i_s$ in the output of $\texttt{attn}_\ell (\texttt{ln1}_\ell (H^{\ell - 1}))$. We denote by $A_\ell^{\texttt{attn}} \in \mathbb{R}^{d_h \times d_h}$ the matrix numerically minimizing 
$$ A \mapsto \sum_{s \in \mathcal{T}} || A \cdot \texttt{ln1}_\ell (h^{\ell-1}_{i_s}) - v^\ell_{i_s}||^2,$$
and define an attention sub-module replacement (Eq.~\ref{eq:attn}) by $$
\texttt{b}^{\overline{\texttt{attn}}}_\ell (h) \coloneqq A_{\ell}^{\texttt{attn}} \cdot \texttt{ln1}_\ell (h) + h. $$
We then define a mapping between two layers ${\ell \rightarrow \ell'}$ by:
$$ \matattnl{} (h) \coloneqq $$
$$ \texttt{b}^{\texttt{ffn}}_{\ell'} ( \texttt{b}^{\overline{\texttt{attn}}}_{\ell'} ( \ldots (\texttt{b}^{\texttt{ffn}}_{\ell+1} ( \texttt{b}^{\overline{\texttt{attn}}}_{\ell+1} (h)))\ldots)).$$ 
Namely, when applying each $\ell''$-th block, $\ell < \ell'' \leq \ell'$, we replace its attention sub-module $\texttt{attn}_{\ell''}$ by its linear approximation.
%In an analogous way, we consider the mappings $\matffl{}$ and $\matlnl{}$, where in the latter we perform the linear shortcut both for \texttt{ln1} and for \texttt{ln2} (see~\S\ref{sec:app_submodule_skip_description} for precise descriptions).
Importantly, unlike the original attention module, the approximation $\texttt{b}^{\overline{\texttt{attn}}}_\ell$ operates on each position independently, and therefore applying $\matattnl{}$ disables any contextualization between the layers $\ell$ and $\ell'$. Note that this is not the case for $\matffl{}$ and $\matlnl{}$, which retain the self-attention sub-modules and operate contextually.
\fi

\paragraph{Evaluation.}


We analyze the $24$-layered \gpt{}, and proceed completely analogously to \S\ref{subsec:next_token_prediction_task}, evaluating the Precision@$1$ and Surprisal metrics for the mappings $\matattnlL{}$, $\matfflL{}$ and $\matlnlL{}$.

\begin{figure}[t]
\setlength{\belowcaptionskip}{-0pt}
\centering
%\includegraphics[scale=0.2]
\includegraphics[width=\columnwidth]{figs/parts_presurp_24.pdf}
\caption{Precision@$1$ and Surprisal for the various sub-module linear mappings, and $\matlL{}$ for comparison ($24$-layer \gpt{} next token prediction task). A 95\% confidence interval surrounds the Surprisal lines.}
\label{fig:parts_presurp}
\end{figure}

\quash{
\begin{figure}[t]
\centering
\includegraphics[scale=0.4]{figs/parts_pre1_24.pdf}
\caption{Precision@$1$ for the various sub-module linear shortcut mappings, and the mapping $\matlL{}$ for comparison (\gpt{} next token prediction task).}
\label{fig:parts_pre1}
\end{figure}

\begin{figure}[t]
\centering
\includegraphics[scale=0.35]{figs/parts_surp_24.pdf}
\caption{Surprisal for the various sub-module linear shortcut mappings, and the mapping $\matlL{}$ for comparison (\gpt{} next token prediction task). A 95\% confidence interval surrounds the lines.}
\label{fig:parts_surp}
\end{figure}
}

\paragraph{Results.}
Fig.~\ref{fig:parts_presurp} shows the average Precision@$1$ and Surprisal scores per layer.
From a certain layer (\textasciitilde$7$), all sub-module mappings achieve better results than the full-block mapping $\matlL{}$. Thus, it is not just the cumulative effect of all the sub-modules in the transformer block that is amenable to linear approximation, but also individual sub-modules can be linearly approximated. 
Furthermore, the linear approximation of attention sub-modules is less harmful than that of the FFN or LN sub-modules. 
% Hypothetically, 
A possible reason is that the linear replacement of FFN or LN ``erodes'' the self-attention computation after a few layers. 
Moreover, the good performance of $\matattnlL{}$ suggests that contextualization often exhausts itself in early layers; speculatively, it is only in more delicate cases that the self-attention of late layers adds important information. Last, remark the sharp ascent of the scores for layer normalization in layers $5$-$8$, for which we do not currently see a particular reason. To conclude, we see that the possibility of linear approximation permeates
%the various
transformer components.


\section{Related Work}

Recently, there was a lot of interest in utilizing intermediate representations in transformer-based LMs, both for interpretability and for efficiency.

In the direction of interpretability, one seeks to understand the prediction construction process of the model \cite{tenney-etal-2019-bert, voita-etal-2019-bottom}.

More recent works use mechanistic interpretability and view the inference pass as a residual stream of information \cite{dar2022analyzing,geva-etal-2022-transformer}. Additionally, there are works on probing, attempting to understand what features are stored in the hidden representations \cite{adi2017finegrained, conneau-etal-2018-cram,liu-etal-2019-linguistic}. Our work is different in that it attempts to convert intermediate representations into a final-layer form, which is interpretable by design.

In the direction of efficiency, there is the thread of work on early exit, where computation is cut at a dynamically-decided earlier stage \cite{schwartz-etal-2020-right,xin-etal-2020-deebert,schuster2022confident}. Other works utilize a fixed early stage network to parallelize inference \citep{leviathan2022fast, chen2023accelerating}. However, intermediate representations are directly propagated in these works, which we show is substantially worse than our approach. Moreover, our method requires training considerably less parameters than methods such as \citet{schuster-etal-2021-consistent}, that learn a different output softmax for each intermediate layer.  

More broadly, skipping transformer layers and analyzing the linearity properties of transformer components have been discussed in prior works \cite{Zhao2021of,mickus-etal-2022-dissect,wang-etal-2022-skipbert,lamparth2023analyzing}.


\section{Conclusion and Future Work}

We present a simple and effective method for enhancing utilization of hidden representations in transformer-based LMs, that uses 
pre-fitted context-free and token-uniform linear mappings.
Through a series of experiments on different data sources, model architectures and scales, we show that our method consistently outperforms the prevalent practice of interpreting representations in the final-layer space of the model, yielding better approximations of succeeding representations and the predictions they induce, thus allowing a more faithful interpretation of the model's prediction-formation.
We demonstrate the practicality of our method for improving computation efficiency, saving a substantial amount of compute on top of prominent early exiting approaches. 
Also, by extending our method to sub-modules, 
% more specifically the attention sub-modules, 
we observe that replacing a part of the transformer inference by a non-contextual linear computation often results in a small deterioration of the prediction.
This opens new research directions for improving model efficiency,
% and parallelizability.
% including breaking the computation into several parallelizable tasks.
including breaking the computation into parallel tasks.

\section*{Limitations}

Although we see in this work that there is more linear structure to transformer inference than could be explained solely by the residual connection, we do not elucidate a reason for that. We also do not try to formulate formal criteria according to which to judge, in principle, the quality of ways of short-cutting transformer inference in-between layers. In addition, our experiments cover only English data.


%\section*{Ethics Statement}
%Scientific work published at ACL 2023 must comply with the ACL Ethics Policy.\footnote{\url{https://www.aclweb.org/portal/content/acl-code-ethics}} We encourage all authors to include an explicit ethics statement on the broader impact of the work, or other ethical considerations after the conclusion but before the references. The ethics statement will not count toward the page limit (8 pages for long, 4 pages for short papers).

\section*{Acknowledgements}

We thank Tal Schuster for constructive comments.

% Entries for the entire Anthology, followed by custom entries
\bibliography{anthology,custom}
\bibliographystyle{acl_natbib}

\appendix

\section{Descriptions of $\matattn{}$, $\matff{}$ and $\matln{}$}
\label{sec:app_submodule_skip_description}

Here we detail the definitions of the mappings $\matattnl{}$, $\matffl{}$ and $\matlnl{}$ utilized in \S\ref{sec:submodules}.

\paragraph{Description of $\matattnl{}$.}
%Illustrating this on $\texttt{attn}_\ell$ for definiteness,
For an input $s$, let $v^\ell_{i_s}$ be the vector at position $i_s$ in the output of $\texttt{attn}_\ell (\texttt{ln1}_\ell (H^{\ell - 1}))$. We denote by $A_\ell^{\texttt{attn}} \in \mathbb{R}^{d_h \times d_h}$ the matrix numerically minimizing 
$$ A \mapsto \sum_{s \in \mathcal{T}} || A \cdot \texttt{ln1}_\ell (h^{\ell-1}_{i_s}) - v^\ell_{i_s}||^2,$$
and define an attention sub-module replacement (Eq.~\ref{eq:attn}) by $$
\texttt{b}^{\overline{\texttt{attn}}}_\ell (h) \coloneqq A_{\ell}^{\texttt{attn}} \cdot \texttt{ln1}_\ell (h) + h. $$
We then define a mapping between two layers ${\ell \rightarrow \ell'}$ by:
$$ \matattnl{} (h) \coloneqq $$
$$ \texttt{b}^{\texttt{ffn}}_{\ell'} ( \texttt{b}^{\overline{\texttt{attn}}}_{\ell'} ( \ldots (\texttt{b}^{\texttt{ffn}}_{\ell+1} ( \texttt{b}^{\overline{\texttt{attn}}}_{\ell+1} (h)))\ldots)).$$ 
Namely, when applying each $\ell''$-th block, $\ell < \ell'' \leq \ell'$, we replace its attention sub-module $\texttt{attn}_{\ell''}$ by its linear approximation.
%In an analogous way, we consider the mappings $\matffl{}$ and $\matlnl{}$, where in the latter we perform the linear shortcut both for \texttt{ln1} and for \texttt{ln2} (see~\S\ref{sec:app_submodule_skip_description} for precise descriptions).
Importantly, unlike the original attention module, the approximation $\texttt{b}^{\overline{\texttt{attn}}}_\ell$ operates on each position independently, and therefore applying $\matattnl{}$ disables any contextualization between the layers $\ell$ and $\ell'$. Note that this is not the case for $\matffl{}$ and $\matlnl{}$, which retain the self-attention sub-modules and operate contextually.

\paragraph{Description of $\matffl{}$.}
Let $v^\ell_{i_s}$ be the vector at position $i_s$ in the output of $\texttt{ln2}_{\ell} (\texttt{b}_\ell^{\texttt{attn}} (H^{\ell - 1}))$, for a given input $s$. We denote by $A_\ell^{\texttt{ffn}} \in \mathbb{R}^{d_h \times d_h}$ the matrix numerically minimizing 
$$ A \mapsto \sum_{s \in \mathcal{T}} || A \cdot v^{\ell}_{i_s} - \texttt{ffn}_{\ell} (v^\ell_{i_s})||^2,$$
and define a replacement of the feed-forward sub-module $\texttt{b}_{\ell}^{\texttt{ffn}}$ by $$ \texttt{b}^{\overline{\texttt{ffn}}}_\ell (H) \coloneqq A_{\ell}^{\texttt{ffn}} \cdot \texttt{ln2}_\ell (H) + H.$$
We then define a mapping between two layers ${\ell \rightarrow \ell'}$ by:
$$ \matffl{} (H) \coloneqq $$
$$ \texttt{b}^{\overline{\texttt{ffn}}}_{\ell'} ( \texttt{b}^{\texttt{attn}}_{\ell'} ( \ldots (\texttt{b}^{\overline{\texttt{ffn}}}_{\ell+1} ( \texttt{b}^{\texttt{attn}}_{\ell+1} (H))\ldots)).$$

\paragraph{Description of $\matlnl{}$.}
Let $v^\ell_{i_s}$ be the vector at position $i_s$ in the output of $\texttt{b}^{\texttt{attn}}_{\ell} (H^{\ell - 1})$, for a given input $s$. We denote by $A_\ell^{\texttt{ln1}} \in \mathbb{R}^{d_h \times d_h}$ the matrix numerically minimizing 
$$ A \mapsto \sum_{s \in \mathcal{T}} || A \cdot h^{\ell}_{i_s} - \texttt{ln1}_{\ell} (h^\ell_{i_s})||^2$$ and we denote by $A_\ell^{\texttt{ln2}} \in \mathbb{R}^{d_h \times d_h}$ the matrix numerically minimizing $$ A \mapsto \sum_{s \in \mathcal{T}} || A \cdot v^{\ell}_{i_s} - \texttt{ln2}_{\ell} (v^\ell_{i_s})||^2.$$ We define a replacement of the block $\texttt{b}^{\texttt{attn}}_{\ell}$ by \begin{equation} \texttt{b}^{\overline{\texttt{ln1}}}_\ell (H) \coloneqq \texttt{attn}_{\ell} (A_{\ell}^{\texttt{ln1}} \cdot H) + H\end{equation} and we define a replacement of the block $\texttt{b}^{\texttt{ffn}}_{\ell}$ by \begin{equation} \texttt{b}^{\overline{\texttt{ln2}}}_\ell (H) \coloneqq \texttt{ffn}_{\ell} (A_{\ell}^{\texttt{ln2}} \cdot H) + H.\end{equation}
We then define a mapping between two layers ${\ell \rightarrow \ell'}$ by:
$$ \matlnl{} (H) \coloneqq $$
$$ \texttt{b}^{\overline{\texttt{ln2}}}_{\ell'} ( \texttt{b}^{\overline{\texttt{ln1}}}_{\ell'} ( \ldots (\texttt{b}^{\overline{\texttt{ln2}}}_{\ell+1} ( \texttt{b}^{\overline{\texttt{ln1}}}_{\ell+1} (H))\ldots)).$$


\end{document}

}
\vspace{0.2cm}
\section*{Appendices}

\appendices

We provide the following supplementary material:

\begin{itemize}[topsep=0.3em]
    \itemsep0.3em
    \item (\ref{sec:details_ours}) Implementation details for \ourmethod, including the training and inference procedure of \ourmethod\ on the dataset of an anomaly detection scenario (\ref{subsec:training_and_evaluation}) and the distillation training of the patch description network (\ref{subsec:distillation}).
    \item (\ref{sec:details_others}) Implementation and configuration details for other evaluated methods.
    \item (\ref{sec:backbones}) Evaluation of the anomaly detection performance of \ourmethod\ for different distillation backbones.
    \item (\ref{sec:ad_metrics}) Results for additional anomaly detection metrics, such as the area under the precision recall curve.
    \item (\ref{sec:efficiency_metrics}) Description of our timing methodology and results for additional computational efficiency metrics such as the number of parameters.
    \item (\ref{sec:qualitative_results}) Qualitative results in the form of anomaly maps generated by \ourmethod\ and other methods on the evaluated datasets.
    \item Anomaly detection results for each method on each of the 32 scenarios from MVTec AD, VisA, and MVTec LOCO in the \href{https://www.mydrive.ch/shares/71140/bbca60ead1194717e488313f1632504c/download/444261995-1678948469/per_scenario_results.json}{\texttt{per\_scenario\_results.json}} file \footnote{\url{https://www.mydrive.ch/shares/71140/bbca60ead1194717e488313f1632504c/download/444261995-1678948469/per_scenario_results.json}}.
\end{itemize}
\vspace{0.2cm}

\section{Implementation Details for \ourmethod}
\label{sec:details_ours}


\subsection{Training and Inference}
\label{subsec:training_and_evaluation}

\Cref{alg:training} describes the training of \ourmethod-S and \Cref{alg:inference} explains the inference procedure.
For \mbox{\ourmethod-M}, replace the architecture of \Cref{tab:arch_eads} with that of \Cref{tab:arch_eadm}.

\begin{algorithm}[h!]
  \caption{\ourmethod-S Training Algorithm} 
  \label{alg:training} 
  \begin{algorithmic} [1]
    \Require{A pretrained teacher network $T : \mathbb{R}^{3 \times 256 \times 256} \rightarrow \mathbb{R}^{384 \times 64 \times 64}$ with an architecture as given in \Cref{tab:arch_eads}}
    \Require{A sequence of training images $\mathcal{I}_\mathrm{train}$ with $I_\mathrm{train} \in \mathbb{R}^{3 \times 256 \times 256}$ for each $I_\mathrm{train} \in \mathcal{I}_\mathrm{train}$}
    \Require{A sequence of validation images $\mathcal{I}_\mathrm{val}$ with $I_\mathrm{val} \in \mathbb{R}^{3 \times 256 \times 256}$ for each $I_\mathrm{val} \in \mathcal{I}_\mathrm{val}$}
    \State{Randomly initialize a student network $S : \mathbb{R}^{3 \times 256 \times 256} \rightarrow \mathbb{R}^{768 \times 64 \times 64}$ with an architecture as given in \Cref{tab:arch_eads}}
    \State{Randomly initialize an autoencoder $A : \mathbb{R}^{3 \times 256 \times 256} \rightarrow \mathbb{R}^{384 \times 64 \times 64}$ with an architecture as given in \Cref{tab:arch_ae}}
    \For{$c \in {1, \dots, 384}$} \Comment{Compute teacher channel normalization parameters $\mu \in \mathbb{R}^{384}$ and $\sigma \in \mathbb{R}^{384}$}
    \State{Initialize an empty sequence $X \leftarrow (~)$}
    \For{$I_\mathrm{train} \in \mathcal{I}_\mathrm{train}$}
        \State{$Y'\leftarrow T(I_\mathrm{train})$}
        \State{$X \leftarrow X^\frown \mathrm{vec}(Y'_c)$} \Comment{Append the channel output to $X$}
    \EndFor
    \State{Set $\mu_c$ to the mean and $\sigma_c$ to the standard deviation of the elements of $X$}
    \algstore{alg_training}
	\end{algorithmic}
\end{algorithm}

\begin{algorithm}[h]
  \begin{algorithmic} [1]
    \algrestore{alg_training}
    \EndFor
    \State{Initialize Adam \cite{kingma2014_adam} with a learning rate of $10^{-4}$ and a weight decay of $10^{-5}$ for the parameters of $S$ and $A$}
    \For{iteration $= 1,\dots, \num{70000}$}
        \State{Choose a random training image $I_\mathrm{train}$ from $\mathcal{I}_\mathrm{train}$}
        \State{$Y'\leftarrow T(I_\mathrm{train})$} \Comment{Forward pass of the student--teacher pair}
        \State{Compute the normalized teacher output $\hat{Y}$ given by $\hat{Y}_c = (Y'_c - \mu_c) \sigma_c^{-1}$ for each $c \in \{1, \dots, 384\}$}
        \State{$Y^\mathrm{S} \leftarrow S(I_\mathrm{train})$}
        \State{Set $Y^\mathrm{ST} \in \mathbb{R}^{384 \times 64 \times 64}$ to the first 384 channels of $Y^\mathrm{S} \in \mathbb{R}^{768 \times 64 \times 64}$}
        \State{Compute the squared difference between $\hat{Y}$ and $Y^\mathrm{ST}$ for each tuple $(c, w, h)$ as $D^\mathrm{ST}_{c, w, h} = (\hat{Y}_{c, w, h} - Y^\mathrm{ST}_{c, w, h})^2$}
        \State{Compute the $0.999$-quantile of the elements of $D^\mathrm{ST}$, denoted by $d_\mathrm{hard}$}
        \State{Compute the loss $L_\mathrm{hard}$ as the mean of all $D^\mathrm{ST}_{c, w, h} \geq d_\mathrm{hard}$ }
        \State{Choose a random pretraining image $P \in \mathbb{R}^{3 \times 256 \times 256}$ from ImageNet \cite{russakovsky2015_alexnet}}
        \State{Compute the loss $L_\mathrm{ST} = L_\mathrm{hard} + (384 \cdot 64 \cdot 64)^{-1}\sum_{c=1}^{384} \|S(P)_c\|_F^2$}
        \State{Randomly choose an augmentation index $i_\mathrm{aug} \in \{1, 2, 3\}$} \Comment{Augment $I_\mathrm{train}$ for $A$ using torchvision \cite{paszke2019_PyTorch}}
        \State{Sample an augmentation coefficient $\lambda$ from the uniform distribution $U(0.8, 1.2)$}
        \If{$i_\mathrm{aug} == 1$}
        $I_\mathrm{aug} \leftarrow \mathtt{torchvision.transforms.functional\_pil.adjust\_brightness}(I_\mathrm{train}, \lambda)$
        \ElsIf{$i_\mathrm{aug} == 2$}
            $I_\mathrm{aug} \leftarrow \mathtt{torchvision.transforms.functional\_pil.adjust\_contrast}(I_\mathrm{train}, \lambda)$
        \ElsIf{$i_\mathrm{aug} == 3$}
            $I_\mathrm{aug} \leftarrow \mathtt{torchvision.transforms.functional\_pil.adjust\_saturation}(I_\mathrm{train}, \lambda)$
        \EndIf
        \State{$Y^\mathrm{A} \leftarrow A(I_\mathrm{aug})$} \Comment{Forward pass of the autoencoder--student pair}
        \State{$Y'\leftarrow T(I_\mathrm{aug})$}
        \State{Compute the normalized teacher output $\hat{Y}$ given by $\hat{Y}_c = \sigma_c^{-1}(Y'_c - \mu_c)$ for each $c \in \{1, \dots, 384\}$}
        \State{$Y^\mathrm{S} \leftarrow S(I_\mathrm{aug})$}
        \State{Set $Y^\mathrm{STAE} \in \mathbb{R}^{384 \times 64 \times 64}$ to the last 384 channels of $Y^\mathrm{S} \in \mathbb{R}^{768 \times 64 \times 64}$}
        \State{Compute the squared difference between $\hat{Y}$ and $Y^\mathrm{A}$ for each tuple $(c, w, h)$ as $D^\mathrm{AE}_{c, w, h} = (\hat{Y}_{c, w, h} - Y^\mathrm{A}_{c, w, h})^2$}
        \State{Compute the squared difference between $Y^\mathrm{A}$ and $Y^\mathrm{STAE}$ for each tuple $(c, w, h)$ as $D^\mathrm{STAE}_{c, w, h} = (Y^\mathrm{A}_{c, w, h} - Y^\mathrm{STAE}_{c, w, h})^2$}
        \State{Compute the loss $L_\mathrm{AE}$ as the mean of all elements $D^\mathrm{AE}_{c, w, h}$ of $D^\mathrm{AE}$}
        \State{Compute the loss $L_\mathrm{STAE}$ as the mean of all elements $D^\mathrm{STAE}_{c, w, h}$ of $D^\mathrm{STAE}$}
        \State{Compute the total loss $L_\mathrm{total} = L_\mathrm{ST} + L_\mathrm{AE} + L_\mathrm{STAE}$}  \Comment{Backward pass}
        \State{Update the union of the parameters of $S$ and $A$, denoted by $\phi$, using the gradient $\nabla_\phi L_\mathrm{total}$ }
        \If{iteration $>$ \num{66500}}
        \State{Decay the learning rate to $10^{-5}$}
        \EndIf
    \EndFor
    \State{Initialize empty sequences $X_\mathrm{ST} \leftarrow (~)$ and $X_\mathrm{AE} \leftarrow (~)$} \Comment{Quantile-based map normalization on validation images}
    \For{$I_\mathrm{val} \in \mathcal{I}_\mathrm{val}$}
        \State{$Y'\leftarrow T(I_\mathrm{val}),\;\;Y^\mathrm{S} \leftarrow S(I_\mathrm{val}),\;\;Y^\mathrm{A} \leftarrow A(I_\mathrm{val})$}
        \State{Compute the normalized teacher output $\hat{Y}$ given by $\hat{Y}_c = (Y'_c - \mu_c) \sigma_c^{-1}$ for each $c \in \{1, \dots, 384\}$}
        \State{Split the student output into $Y^\mathrm{ST} \in \mathbb{R}^{384 \times 64 \times 64}$ and $Y^\mathrm{STAE} \in \mathbb{R}^{384 \times 64 \times 64}$ as above}
        \State{Compute the squared difference $D^\mathrm{ST}_{c, w, h} = (\hat{Y}_{c, w, h} - Y^\mathrm{ST}_{c, w, h})^2$ for each tuple $(c, w, h)$}
        \State{Compute the squared difference $D^\mathrm{STAE}_{c, w, h} = (Y^\mathrm{A}_{c, w, h} - Y^\mathrm{STAE}_{c, w, h})^2$ for each tuple $(c, w, h)$ }
        \State{Compute the anomaly maps $M_\mathrm{ST} = 384^{-1}\sum_{c=1}^{384} D^\mathrm{ST}_c$ and $M_\mathrm{AE} = 384^{-1}\sum_{c=1}^{384} D^\mathrm{STAE}_c$}
        \State{Resize $M_\mathrm{ST}$ and $M_\mathrm{AE}$ to $256 \times 256$ pixels using bilinear interpolation}
        \State{$X_\mathrm{ST} \leftarrow {X_\mathrm{ST}}^\frown \mathrm{vec}(M_\mathrm{ST})$} \Comment{Append to the sequence of local anomaly scores}
        \State{$X_\mathrm{AE} \leftarrow {X_\mathrm{AE}}^\frown \mathrm{vec}(M_\mathrm{AE})$} \Comment{Append to the sequence of global anomaly scores}
    \EndFor
    \State{Compute the $0.9$-quantile $q_a^\mathrm{ST}$ and the $0.995$-quantile $q_b^\mathrm{ST}$ of the elements of $X_\mathrm{ST}$.}
    \State{Compute the $0.9$-quantile $q_a^\mathrm{AE}$ and the $0.995$-quantile $q_b^\mathrm{AE}$ of the elements of $X_\mathrm{AE}$.}
    \State \Return{$T$, $S$, $A$, $\mu$, $\sigma$, $q_a^\mathrm{ST}$, $q_b^\mathrm{ST}$, $q_a^\mathrm{AE}$, and $q_b^\mathrm{AE}$ }
	\end{algorithmic}
\end{algorithm}

\begin{algorithm}
  \caption{\ourmethod\  Inference Procedure} 
  \label{alg:inference} 
  \begin{algorithmic} [1]
    \Require{$T$, $S$, $A$, $\mu$, $\sigma$, $q_a^\mathrm{ST}$, $q_b^\mathrm{ST}$, $q_a^\mathrm{AE}$, and $q_b^\mathrm{AE}$, as returned by \Cref{alg:training}}
    \Require{Test image $I_\mathrm{test} \in \mathbb{R}^{3 \times 256 \times 256}$}
    \State{$Y'\leftarrow T(I_\mathrm{test}),\;\;Y^\mathrm{S} \leftarrow S(I_\mathrm{test}),\;\;Y^\mathrm{A} \leftarrow A(I_\mathrm{test})$}
    \State{Compute the normalized teacher output $\hat{Y}$ given by $\hat{Y}_c = (Y'_c - \mu_c) \sigma_c^{-1}$ for each $c \in \{1, \dots, 384\}$}
    \State{Split the student output into $Y^\mathrm{ST} \in \mathbb{R}^{384 \times 64 \times 64}$ and $Y^\mathrm{STAE} \in \mathbb{R}^{384 \times 64 \times 64}$ as above}
    \State{Compute the squared difference $D^\mathrm{ST}_{c, w, h} = (\hat{Y}_{c, w, h} - Y^\mathrm{ST}_{c, w, h})^2$ for each tuple $(c, w, h)$}
    \State{Compute the squared difference $D^\mathrm{STAE}_{c, w, h} = (Y^\mathrm{A}_{c, w, h} - Y^\mathrm{STAE}_{c, w, h})^2$ for each tuple $(c, w, h)$ }
    \State{Compute the anomaly maps $M_\mathrm{ST} = 384^{-1}\sum_{c=1}^{384} D^\mathrm{ST}_c$ and $M_\mathrm{AE} = 384^{-1}\sum_{c=1}^{384} D^\mathrm{STAE}_c$}
    \State{Resize $M_\mathrm{ST}$ and $M_\mathrm{AE}$ to $256 \times 256$ pixels using bilinear interpolation}
    \State{Compute the normalized $\hat{M}_\mathrm{ST} = 0.1 (M_\mathrm{ST} - q_a^\mathrm{ST}) (q_b^\mathrm{ST} - q_a^\mathrm{ST})^{-1}$}
    \State{Compute the normalized $\hat{M}_\mathrm{AE} = 0.1 (M_\mathrm{AE} - q_a^\mathrm{AE}) (q_b^\mathrm{AE} - q_a^\mathrm{AE})^{-1}$}
    \State{Compute the combined anomaly map $M = 0.5 \hat{M}_\mathrm{ST} + 0.5 \hat{M}_\mathrm{AE}$}
    \State{Compute the image-level score as $m_\mathrm{image} = \max_{i, j} M_{i, j}$}
    \State \Return{$M$ and $m_\mathrm{image}$}
	\end{algorithmic}
\end{algorithm}

\begin{table}
\begin{center}
\begin{tabular}{cccccc}
Layer Name & Stride & Kernel Size & Number of Kernels & Padding & Activation \\
\hline
Conv-1 & 1$\times$1 & 4$\times$4 & 128 & 3 & ReLU \\
AvgPool-1 & 2$\times$2 & 2$\times$2 & 128 & 1 & - \\
Conv-2 & 1$\times$1 & 4$\times$4 & 256 & 3 & ReLU \\
AvgPool-2 & 2$\times$2 & 2$\times$2 & 256 & 1 & - \\
Conv-3 & 1$\times$1 & 3$\times$3 & 256 & 1 & ReLU \\
Conv-4 & 1$\times$1 & 4$\times$4 & 384 & 0 & - \\
\hline
\end{tabular}
\end{center}
\caption{Patch description network architecture of the teacher network for \ourmethod-S. The student network has the same architecture, but 768 kernels instead of 384 in the Conv-4 layer. A padding value of 3 means that three rows, or columns respectively, of zeros are appended at each border of an input feature map.}
\label{tab:arch_eads}
\end{table}

\begin{table}
\begin{center}
\begin{tabular}{cccccc}
Layer Name & Stride & Kernel Size & Number of Kernels & Padding & Activation \\
\hline
Conv-1 & 1$\times$1 & 4$\times$4 & 256 & 3 & ReLU \\
AvgPool-1 & 2$\times$2 & 2$\times$2 & 256 & 1 & - \\
Conv-2 & 1$\times$1 & 4$\times$4 & 512 & 3 & ReLU \\
AvgPool-2 & 2$\times$2 & 2$\times$2 & 512 & 1 & - \\
Conv-3 & 1$\times$1 & 1$\times$1 & 512 & 0 & ReLU \\
Conv-4 & 1$\times$1 & 3$\times$3 & 512 & 1 & ReLU \\
Conv-5 & 1$\times$1 & 4$\times$4 & 384 & 0 & ReLU \\
Conv-6 & 1$\times$1 & 1$\times$1 & 384 & 0 & - \\
\hline
\end{tabular}
\end{center}
\caption{Patch description network architecture of the teacher network for \ourmethod-M. The student network has the same architecture, but 768 kernels instead of 384 in the Conv-5 and Conv-6 layers. A padding value of 3 means that three rows, or columns respectively, of zeros are appended at each border of an input feature map.}
\label{tab:arch_eadm}
\end{table}

\begin{table}
\begin{center}
\begin{tabular}{cccccc}
Layer Name & Stride & Kernel Size & Number of Kernels & Padding & Activation \\
\hline
EncConv-1 & 2$\times$2 & 4$\times$4 & 32 & 1 & ReLU \\
EncConv-2 & 2$\times$2 & 4$\times$4 & 32 & 1 & ReLU \\
EncConv-3 & 2$\times$2 & 4$\times$4 & 64 & 1 & ReLU \\
EncConv-4 & 2$\times$2 & 4$\times$4 & 64 & 1 & ReLU \\
EncConv-5 & 2$\times$2 & 4$\times$4 & 64 & 1 & ReLU \\
EncConv-6 & 1$\times$1 & 8$\times$8 & 64 & 0 & - \\
Bilinear-1 & \multicolumn{5}{c}{Resizes the 1$\times$1 input features maps to 3$\times$3}  \\
DecConv-1 & 1$\times$1 & 4$\times$4 & 64 & 2 & ReLU \\
Dropout-1 & \multicolumn{5}{c}{Dropout rate = 0.2} \\
Bilinear-2 & \multicolumn{5}{c}{Resizes the 4$\times$4 input features maps to 8$\times$8}  \\
DecConv-2 & 1$\times$1 & 4$\times$4 & 64 & 2 & ReLU \\
Dropout-2 & \multicolumn{5}{c}{Dropout rate = 0.2} \\
Bilinear-3 & \multicolumn{5}{c}{Resizes the 9$\times$9 input features maps to 15$\times$15}  \\
DecConv-3 & 1$\times$1 & 4$\times$4 & 64 & 2 & ReLU \\
Dropout-3 & \multicolumn{5}{c}{Dropout rate = 0.2} \\
Bilinear-4 & \multicolumn{5}{c}{Resizes the 16$\times$16 input features maps to 32$\times$32}  \\
DecConv-4 & 1$\times$1 & 4$\times$4 & 64 & 2 & ReLU \\
Dropout-4 & \multicolumn{5}{c}{Dropout rate = 0.2} \\
Bilinear-5 & \multicolumn{5}{c}{Resizes the 33$\times$33 input features maps to 63$\times$63}  \\
DecConv-5 & 1$\times$1 & 4$\times$4 & 64 & 2 & ReLU \\
Dropout-5 & \multicolumn{5}{c}{Dropout rate = 0.2} \\
Bilinear-6 & \multicolumn{5}{c}{Resizes the 64$\times$64 input features maps to 127$\times$127}  \\
DecConv-6 & 1$\times$1 & 4$\times$4 & 64 & 2 & ReLU \\
Dropout-6 & \multicolumn{5}{c}{Dropout rate = 0.2} \\
Bilinear-7 & \multicolumn{5}{c}{Resizes the 128$\times$128 input features maps to 64$\times$64}  \\
DecConv-7 & 1$\times$1 & 3$\times$3 & 64 & 1 & ReLU \\
DecConv-8 & 1$\times$1 & 3$\times$3 & 384 & 1 & - \\
\hline
\end{tabular}
\end{center}
\caption{Network architecture of the autoencoder for \ourmethod-S and \ourmethod-M. Layers named ``EncConv'' and ``DecConv'' are standard 2D convolutional layers.}
\label{tab:arch_ae}
\end{table}

\paragraph{Comments on \Cref{alg:training} and \Cref{alg:inference}:}
\begin{itemize}
    \item We use the default initialization method of PyTorch \cite{paszke2019_PyTorch} (version 1.12.0) for the convolutional layers.
    \item We apply the teacher and the student to both the original and the augmented training image. 
That is necessary because the student--teacher model is trained without augmentation, while the autoencoder is trained with augmentation.
During inference, we do not need these second forward passes because images are not augmented at test time.
    \item The sizes of the images of MVTec AD \cite{bergmann2021_mvtec_ad_ijcv, bergmann2019_mvtec_ad_cvpr}, VisA \cite{zou2022spot}, and MVTec LOCO \cite{bergmann2021_mvtec_loco_ijcv} differ.
We resize each input image to $256\times256$ and resize the anomaly map $M$ back to the original image size using bilinear interpolation.
    \item We use a batch size of one.
    \item We use the image normalization of the pretrained models of torchvision \cite{paszke2019_PyTorch}.
That means we subtract $0.485$, $0.456$, and $0.406$ from the R, G, and B channel, respectively, for each input image and divide the channels by $0.229$, $0.224$, and $0.225$, respectively.
We perform this normalization directly before applying a network to an image, i.e., after augmentation.
At test time, this can also be done by adjusting the weights and bias of the first convolutional layer of a network accordingly.
    \item The parameters of the autoencoder $A$ are not only affected by the gradient of $L_\mathrm{AE}$, but also by the gradient of $L_\mathrm{STAE}$.
    \item We obtain an image $P \in \mathbb{R}^{3 \times 256 \times 256}$ from ImageNet by choosing a random image, resizing it to $512\times512$, converting it to gray scale with a probability of $0.3$, and cropping the center $256\times256$ pixels.
\end{itemize}

\subsection{Distillation}
\label{subsec:distillation}

In the following, we describe how to distill the WideResNet-101 \cite{zagoruyko2016wideresnet_wrn} features used by PatchCore \cite{roth2022towards} into the teacher network $T$.
The distillation training algorithm is presented in \Cref{alg:distillation}.
The process in analogous for other pretrained feature extractors.

There are only few requirements regarding the output shape of the feature extractor.
The feature extractors used by PatchCore output features of shape $384\times64\times64$ for an input image size of $512\times512$ pixels.
Therefore, the teacher and the autoencoder also output $384$ channels (as described in \Cref{tab:arch_eads,tab:arch_eadm,tab:arch_ae}).
If a pretrained feature extractor outputs a different number of channels, this default of $384$ output channels of the teacher and the autoencoder can be adjusted flexibly.
During distillation, we resize input images to $512\times512$ for the pretrained feature extractor and to $256\times256$ for the teacher network that is being trained.
This results in an output shape of $384\times64\times64$ for the teacher network as well.
If a feature extractor outputs feature maps of a size other than $64\times64$, we can adjust its input image size to achieve an output feature map size of $64\times64$.
Alternatively, we can adjust the input image size of the teacher network because it is fully convolutional and operates separately on patches of size $33\times33$.
A feature map size of $53\times71$, for example, can be achieved by applying the teacher network to images of size $212\times284$.

We use a batch size of 16 for the distillation training and use ImageNet \cite{russakovsky2015_alexnet} as the pretraining dataset.
We use the official implementation of PatchCore \footnote{\url{https://github.com/amazon-science/patchcore-inspection/tree/6a9a281fc34cb1b13c54b318f71e6f1f371536bb}} and its default values if not stated otherwise.
We use the feature postprocessing of PatchCore as well, which includes pooling features from two layers and projecting each feature vector to a reduced dimensionality of $384$ dimensions, as described in \cite{roth2022towards}.
The features used for our distillation training are the final features used by PatchCore, i.e., those given to the coreset subsampling algorithm when training PatchCore.
We denote the WideResNet-101-based feature extractor, including the feature postprocessing, as $\Psi : \mathbb{R}^{3\times512\times512} \rightarrow \mathbb{R}^{384\times64\times64}$.

\begin{algorithm}
  \caption{Distillation Training Algorithm} 
  \label{alg:distillation} 
  \begin{algorithmic} [1]
    \Require{A pretrained feature extractor $\Psi : \mathbb{R}^{3\times W \times H} \rightarrow \mathbb{R}^{384\times64\times64}$.}
    \Require{A sequence of distillation training images $\mathcal{I}_\mathrm{dist}$}
    \State{Randomly initialize a teacher network $T : \mathbb{R}^{3 \times 256 \times 256} \rightarrow \mathbb{R}^{384 \times 64 \times 64}$ with an architecture as given in \Cref{tab:arch_eads} or \ref{tab:arch_eadm}}
    \For{$c \in {1, \dots, 384}$} \Comment{Compute feature extractor channel normalization parameters $\mu^\Psi \in \mathbb{R}^{384}$ and $\sigma^\Psi \in \mathbb{R}^{384}$}
    \State{Initialize an empty sequence $X \leftarrow (~)$}
    \For{iteration $= 1,2,\dots, \num{10000}$}
        \State{Choose a random training image $I_\mathrm{dist}$ from $\mathcal{I}_\mathrm{dist}$}
        \State{Convert $I_\mathrm{dist}$ to gray scale with a probability of $0.1$}
        \State{Compute $I^\Psi_\mathrm{dist}$ by resizing $I_\mathrm{dist}$ to $3\times W \times H$ using bilinear interpolation}
        \State{$Y^\Psi\leftarrow \Psi(I^\Psi_\mathrm{dist})$}
        \State{$X \leftarrow X^\frown \mathrm{vec}(Y^\Psi_c)$} \Comment{Append the channel output to $X$}
    \EndFor
    \State{Set $\mu^\Psi_c$ to the mean and $\sigma^\Psi_c$ to the standard deviation of the elements of $X$}
    \EndFor
    \State{Initialize the Adam \cite{kingma2014_adam} optimizer with a learning rate of $10^{-4}$ and a weight decay of $10^{-5}$ for the parameters of $T$}
    \For{iteration $= 1,\dots, \num{60000}$}
        \State{$L_\mathrm{batch} \leftarrow 0$}
        \For{batch index $= 1, \dots, 16$}
        \State{Choose a random training image $I_\mathrm{dist}$ from $\mathcal{I}_\mathrm{dist}$}
        \State{Convert $I_\mathrm{dist}$ to gray scale with a probability of $0.1$}
        \State{Compute $I^\Psi_\mathrm{dist}$ by resizing $I_\mathrm{dist}$ to $3\times W \times H$ using bilinear interpolation}
        \State{Compute $I'_\mathrm{dist}$ by resizing $I_\mathrm{dist}$ to $3\times 256 \times 256$ using bilinear interpolation}
        \State{$Y^\Psi\leftarrow \Psi(I^\Psi_\mathrm{dist})$}
        \State{Compute the normalized features $\hat{Y}^\Psi$ given by $\hat{Y}^\Psi_c = (Y^\Psi_c - \mu^\Psi_c) (\sigma^\Psi_c)^{-1}$ for each $c \in \{1, \dots, 384\}$}
        \State{$Y' \leftarrow T(I'_\mathrm{dist})$}
        \State{Compute the squared difference between $\hat{Y}^\Psi$ and $Y'$ for each tuple $(c, w, h)$ as $D^\mathrm{dist}_{c, w, h} = (\hat{Y}^\Psi_{c, w, h} - Y'_{c, w, h})^2$}
        \State{Compute the loss $L_\mathrm{dist}$ as the mean of all elements $D^\mathrm{dist}_{c, w, h}$ of $D^\mathrm{dist}$}
        \State{$L_\mathrm{batch} \leftarrow L_\mathrm{batch} + L_\mathrm{dist}$}
        \EndFor
        \State{$L_\mathrm{batch} \leftarrow 16^{-1} L_\mathrm{batch}$}
        \State{Update the parameters of $T$, denoted by $\theta$, using the gradient $\nabla_\theta L_\mathrm{batch}$ }
    \EndFor
    \State \Return{$T$}
	\end{algorithmic}
\end{algorithm}

\paragraph{Comments on \Cref{alg:distillation}:}
We use the image normalization of the pretrained models of torchvision \cite{paszke2019_PyTorch}.
That means we subtract $0.485$, $0.456$, and $0.406$ from the R, G, and B channel, respectively, for each input image and divide the channels by $0.229$, $0.224$, and $0.225$, respectively.
We perform this normalization directly before applying a network to an image, i.e., after augmentation.

\section{Implementation Details for Other Evaluated Methods}
\label{sec:details_others}

In the following, we provide the implementation and configuration details for Asymmetric Student--Teacher (AST) \cite{rudolph2023asymmetric}, DSR \cite{zavrtanik2022dsr}, FastFlow \cite{yu2021fastflow}, GCAD \cite{bergmann2021_mvtec_loco_ijcv}, PatchCore \cite{roth2022towards}, and Student--Teacher \cite{bergmann2020_uninformed_cvpr}.

\subsection{AST}
\label{sec:details_ast}

We use the official implementation of Rudolph \etal \cite{rudolph2023asymmetric} \footnote{\url{https://github.com/marco-rudolph/AST/tree/1a157973a0e2cb23b6fbb853db8ae43537ab2568}}.
We use the default configuration without modifications, but are not able to fully reproduce the results reported in the AST paper.
The AST paper reports a mean classification AU-ROC of \SI{99.2}{\percent} on MVTec AD, averaged across five runs.
We obtain an AU-ROC of \SI{98.9}{\percent} across five runs.

\subsection{DSR}
\label{sec:details_dsr}

We use the official implementation of Zavrtanik \etal \cite{zavrtanik2022dsr} \footnote{\url{https://github.com/VitjanZ/DSR_anomaly_detection/tree/672bfb81434fd2a6c5ef00db858cef8834c54f28}}.
We use the default configuration without modifications for reproducing the results on MVTec AD.
We obtain a mean classification AU-ROC of \SI{98.1}{\percent} on MVTec AD, which is close to the \SI{98.2}{\percent} reported by the authors.
On the scenarios from VisA, which contain more training images than those of MVTec AD, we change the number of epochs to 50 to keep the total number of training iterations in a similar range.

\subsection{FastFlow}
\label{sec:details_fastflow}

We use the implementation of Akcay \etal \cite{akcay2022anomalib} \footnote{\url{https://github.com/openvinotoolkit/anomalib/tree/e66a17c86489486f6bbd5099366383e8660923fd}}.
We use the FastFlow version based on the WideResNet-50-2 feature extractor, as it is similar to the WideResNet used by PatchCore and our method.
We use the default configuration, but disable early stopping, i.e., the scenario-specific tuning of the training duration on test images.
Instead, we choose a constant training duration (200 steps) that works well on average for all evaluated datasets.
The mean classification AU-ROC on MVTec AD differs from the results reported in the original work, but is in line with those reported by Akcay \etal for the released implementation in the Intel anomalib \cite{akcay2022anomalib}.

\subsection{GCAD}
\label{sec:details_gcad}

We implement GCAD as described by Bergmann \etal \cite{bergmann2021_mvtec_loco_ijcv}.
We are able to reproduce the results reported by the authors, but adapt GCAD to a configuration that performs better in our experiments.
GCAD consists of an ensemble of two anomaly detection models that use different feature extractors.
The first member uses a feature extractor that operates on patches of size $17\times17$ while the feature extractor used by the second member operates on patches of size $33\times33$.
We find that the second member performs better on average than the combined ensemble and therefore report the results for this member in the main paper.
This reduces the latency reported for GCAD by a factor of two.
Furthermore, we change the size of the autoencoder's bottleneck from 32 to 64, which slightly improves the average anomaly detection performance on MVTec AD, VisA, and MVTec LOCO.
On the logical anomalies of MVTec LOCO, the single model scores a classification AU-ROC of \SI{83.9}{\percent}, while the AU-ROC of the ensemble model used by the authors is \SI{86.0}{\percent}.
The overall anomaly classification performance on MVTec LOCO, however, stays the same.

\subsection{PatchCore}
\label{sec:details_patchcore}

We use the official implementation of Roth \etal \cite{roth2022towards} \footnote{\url{https://github.com/amazon-science/patchcore-inspection/tree/6a9a281fc34cb1b13c54b318f71e6f1f371536bb}}.
With the default configuration, we are able to reproduce the results reported for MVTec AD.
As described in the main paper, we disable the cropping of the center \SI{76.6}{\percent} of input images.
Apart from that, we do not alter the hyperparameters of PatchCore.
We report the results for two variants of PatchCore. 
For the single model variant, we use the default configuration of PatchCore for which the authors report the lowest latency.
Specifically, this means setting the coreset subsampling ratio to \SI{1}{\percent}, the image size to $224\times224$ pixels, and the feature extraction backbone to a WideResNet-101.
For the ensemble variant, we use the configuration for which the authors report the best classification AU-ROC on MVTec AD.
We use a WideResNet-101, a ResNeXT-101 \cite{xie2017resnext}, and a DenseNet-201 \cite{huang2017densely} as backbones, set the coreset subsampling ratio to \SI{1}{\percent}, and use images of size $320\times320$ pixels.

\subsection{Student--Teacher}
\label{sec:details_st}

We implement the original multi-scale Student--Teacher (\studteach) method as described by Bergmann \etal \cite{bergmann2020_uninformed_cvpr}.
We use the default hyperparameter settings without modification.
Our implementation achieves better anomaly localization results on MVTec AD than those reported by the authors but matches those reported in \cite{bergmann2021_mvtec_ad_ijcv}.

\section{Robustness to the Distillation Backbone Architecture}
\label{sec:backbones}
In the main paper, we use the features from a WideResNet-101 for training a teacher network in \Cref{alg:distillation}.
The default configuration of PatchCore uses the same features.
In \Cref{tab:distillation_backbones}, we evaluate the anomaly detection performance for other backbones.
Specifically, we evaluate the two additional backbones that \patchcoreens\ uses, i.e., a ResNeXt-101 and a DenseNet-201.
On the three evaluated dataset collections, the anomaly detection performance of \ourmethod\ is similarly robust to the choice of the backbone in comparison to the robustness of PatchCore.
On MVTec AD, both methods perform very similarly across backbones, while their performance on MVTec LOCO varies more.
On VisA, the gap between the structural anomaly detection performance of PatchCore and that of \ourmethod\ becomes evident.


\begin{table}[h]
\begin{center}
\begin{tabular}{ccccc}
& Method & WideResNet-101 & ResNeXt-101 & DenseNet-201 \\
\hline
\multirow{3}{*}{\rotatebox{90}{\specialcell[c]{MVTec \\ AD}}} & \text{PatchCore} & 98.7 & 98.8 & 98.7 \\
& \ourmethod-S & 98.8 & 98.9 & 98.8 \\
& \ourmethod-M & 99.1 & 99.0 & 99.2 \\
&&&& \\ 
\multirow{3}{*}{\rotatebox{90}{\specialcell[c]{MVTec \\ LOCO}}} & \text{PatchCore} & 80.3  & 78.9 & 76.5 \\
& \ourmethod-S & 90.0 & 90.1 & 90.6 \\
& \ourmethod-M & 90.7 & 89.9 & 88.3 \\
&&&& \\ 
\multirow{3}{*}{\rotatebox{90}{VisA}} & \text{PatchCore} & 94.3 & 95.2 & 94.8 \\
& \ourmethod-S & 97.5 & 97.3 & 97.1 \\
& \ourmethod-M & 98.1 & 98.0 & 97.7 \\
\end{tabular}
\end{center}
\caption{Mean anomaly classification AU-ROC percentages for different backbones. For \ourmethod, each listed architecture is used as the distillation backbone in \Cref{alg:distillation}. The ``WideResNet-101'' column contains the results reported in \Cref{tab:per_dataset} in the main paper.}
\label{tab:distillation_backbones}
\end{table}

\clearpage

\section{Additional Anomaly Detection Metrics}
\label{sec:ad_metrics}

In this section, we report the results for additional anomaly detection metrics.
\Cref{sec:ad_metrics_classification} evaluates image-level anomaly classification metrics.
\Cref{sec:ad_metrics_localization} evaluates pixel-level anomaly localization metrics.

For per-scenario evaluation results, see the \href{https://www.mydrive.ch/shares/71140/bbca60ead1194717e488313f1632504c/download/444261995-1678948469/per_scenario_results.json}{\texttt{per\_scenario\_results.json}} file \footnote{\url{https://www.mydrive.ch/shares/71140/bbca60ead1194717e488313f1632504c/download/444261995-1678948469/per_scenario_results.json}}.

Following the official MVTec LOCO evaluation script \footnote{\url{https://www.mvtec.com/company/research/datasets/mvtec-loco}}, we evaluate each performance metric separately on the structural and on the logical anomalies of MVTec LOCO.
Then, we compute the mean of the two scores to compute the overall performance of a method on a scenario of MVTec LOCO.


\subsection{Anomaly Classification}
\label{sec:ad_metrics_classification}

In the main paper, we evaluate the anomaly classification performance with the area under the ROC curve (AU-ROC).
Here, we report the results for the area under the precision recall curve (AU-PRC) as well.
For information on the differences between the AU-ROC and the AU-PRC, we refer to Davis and Goadrich \cite{davis2006relationship}.

\Cref{tab:classification_roc} shows the anomaly classification performance of each method measured with the AU-ROC.
This table contains the results reported in the main paper.
\Cref{tab:classification_prc} shows the results for the image-level AU-PRC.

\begin{table}[h!]
\begin{center}
\begin{tabular}{c|c|c|ccc||c}
Method & \specialcell[c]{MAD \\ Mean} & \specialcell[c]{VisA \\ Mean} & \specialcell[c]{LOCO \\ Structural} & \specialcell[c]{LOCO \\ Logical} & \specialcell[c]{LOCO \\ Mean} & \specialcell[c]{Overall \\ Mean} \\
\hline
GCAD & 89.1 & 83.7 & 82.7 & 83.9 & 83.3 & 85.4 \\
\studteach & 93.2 & 94.6 & 88.3 & 66.5 & 77.4 & 88.4 \\
FastFlow & 96.9 & 93.9 & 82.9 & 75.5 & 79.2 & 90.0 \\
DSR & 98.1 & 91.8 & 90.2 & 75.0 & 82.6 & 90.8 \\
PatchCore & 98.7 & 94.3 & 84.8 & 75.8 & 80.3 & 91.1 \\
\patchcoreens & \textbf{99.3} & 97.7 & 87.7 & 71.0 & 79.4 & 92.1 \\
AST & 98.9 & 94.9 & 87.1 & 79.7 & 83.4 & 92.4 \\
\hline
\ourmethod-S & 98.8 & 97.5 & 94.1 & 85.8 & 90.0 & 95.4 \\
\ourmethod-M & 99.1 & \textbf{98.1} & \textbf{94.7} & \textbf{86.8} & \textbf{90.7} & \textbf{96.0} \\
\end{tabular}
\end{center}
\caption{Mean anomaly classification AU-ROC percentages per dataset collection. For \ourmethod, we report the mean of five runs.}
\label{tab:classification_roc}
\end{table}

\begin{table}[h!]
\begin{center}
\begin{tabular}{c|c|c|ccc||c}
Method & \specialcell[c]{MAD \\ Mean} & \specialcell[c]{VisA \\ Mean} & \specialcell[c]{LOCO \\ Structural} & \specialcell[c]{LOCO \\ Logical} & \specialcell[c]{LOCO \\ Mean} & \specialcell[c]{Overall \\ Mean} \\
\hline
GCAD & 95.7 & 87.1 & 81.0 & 84.9 & 83.0 & 88.6 \\
\studteach & 95.7 & 94.6 & 87.9 & 70.7 & 79.3 & 89.9 \\
FastFlow & 95.3 & 94.7 & 79.5 & 76.2 & 77.9 & 89.3 \\
DSR & 98.1 & 93.8 & 88.2 & 76.6 & 82.4 & 91.4 \\
PatchCore & 98.9 & 95.2 & 84.6 & 77.7 & 81.2 & 91.8 \\
\patchcoreens & \textbf{99.0} & 97.8 & 88.3 & 74.7 & 81.5 & 92.8 \\
AST & 98.9 & 95.3 & 84.5 & 80.5 & 82.5 & 92.2 \\
\hline
\ourmethod-S & 98.7 & 97.5 & 93.6 & 86.2 & 89.9 & 95.4 \\
\ourmethod-M & 98.9 & \textbf{98.0} & \textbf{93.9} & \textbf{86.8} & \textbf{90.3} & \textbf{95.7} \\
\end{tabular}
\end{center}
\caption{Mean anomaly classification AU-PRC percentages per dataset collection. For \ourmethod, we report the mean of five runs.}
\label{tab:classification_prc}
\end{table}

\subsection{Anomaly Localization}
\label{sec:ad_metrics_localization}

To evaluate the anomaly localization performance, we use the area under the PRO curve (AU-PRO) up to a false positive rate (FPR) of \SI{30}{\percent} in the main paper, as recommended by \cite{bergmann2021_mvtec_ad_ijcv}.
The AU-PRO metric \cite{bergmann2021_mvtec_ad_ijcv} is similar to the pixel-wise AU-ROC.
The difference is that the pixel-wise AU-ROC gives each ground truth defect \textit{pixel} the same weight in its computation.
The AU-PRO gives each ground truth defect \textit{region} the same weight.
The FPR limit of \SI{30}{\percent} is due to the fact that a method that segments, on average, more than \SI{30}{\percent} of every defect-free image as anomalous is of limited use.

\Cref{tab:localization_pro_03} contains the results reported in the main paper.
Here, we report the AU-PRO for an FPR limit of \SI{5}{\percent} as well in \Cref{tab:localization_pro_005}.
For comparison, we also report the pixel-wise AU-ROC for an FPR limit of \SI{5}{\percent} in \Cref{tab:localization_roc_005}.
Furthermore, we evaluate the pixel-wise AU-PRC as an additional segmentation, and thus, localization performance metric in \Cref{tab:localization_prc}.
The \href{https://www.mydrive.ch/shares/71140/bbca60ead1194717e488313f1632504c/download/444261995-1678948469/per_scenario_results.json}{\texttt{per\_scenario\_results.json}} file \footnote{\url{https://www.mydrive.ch/shares/71140/bbca60ead1194717e488313f1632504c/download/444261995-1678948469/per_scenario_results.json}} also contains the AU-PRO and pixel-wise AU-ROC results for an FPR limit of \SI{100}{\percent}.


\begin{table}[h!]
\begin{center}
\begin{tabular}{c|c|c|ccc||c}
Method & \specialcell[c]{MAD \\ Mean} & \specialcell[c]{VisA \\ Mean} & \specialcell[c]{LOCO \\ Structural} & \specialcell[c]{LOCO \\ Logical} & \specialcell[c]{LOCO \\ Mean} & \specialcell[c]{Overall \\ Mean} \\
\hline
GCAD & 91.0 & 83.7 & 89.5 & 89.4 & 89.5 & 88.0 \\
\studteach & 92.4 & 93.0 & 90.8 & 76.4 & 83.6 & 89.7 \\
FastFlow & 92.5 & 86.8 & 84.2 & 76.5 & 80.3 & 86.5 \\
DSR & 90.8 & 68.1 & 81.3 & 72.3 & 76.8 & 78.6 \\
PatchCore & 92.7 & 79.7 & 64.3 & 76.6 & 70.4 & 80.9 \\
\patchcoreens & \textbf{95.6} & 79.3 & 62.0 & 72.6 & 67.3 & 80.7 \\
AST & 81.2 & 81.5 & 75.4 & 62.6 & 69.0 & 77.2 \\
\hline
\ourmethod-S & 93.1 & 93.1 & 92.6 & 90.1 & 91.3 & 92.5 \\
\ourmethod-M & 93.5 & \textbf{94.0} & \textbf{93.7} & \textbf{91.3} & \textbf{92.5} & \textbf{93.3} \\
\end{tabular}
\end{center}
\caption{Mean anomaly localization performance per method and dataset collection, measured with the AU-PRO up to a FPR of \SI{30}{\percent}. For \ourmethod, we report the mean of five runs.}
\label{tab:localization_pro_03}
\end{table}

\begin{table}[h!]
\begin{center}
\begin{tabular}{c|c|c|ccc||c}
Method & \specialcell[c]{MAD \\ Mean} & \specialcell[c]{VisA \\ Mean} & \specialcell[c]{LOCO \\ Structural} & \specialcell[c]{LOCO \\ Logical} & \specialcell[c]{LOCO \\ Mean} & \specialcell[c]{Overall \\ Mean} \\
\hline
GCAD & 68.8 & 52.6 & 68.8 & 67.1 & 68.0 & 63.1 \\
\studteach & 73.4 & 75.0 & 75.6 & 49.7 & 62.6 & 70.4 \\
FastFlow & 71.6 & 63.4 & 64.5 & 49.1 & 56.8 & 63.9 \\
DSR & 78.9 & 49.5 & 67.1 & 49.8 & 58.5 & 62.3 \\
PatchCore & 68.6 & 49.4 & 37.9 & 41.5 & 39.7 & 52.6 \\
\patchcoreens & \textbf{79.5} & 55.1 & 37.8 & 35.3 & 36.5 & 57.1 \\
AST & 42.1 & 48.0 & 50.1 & 35.3 & 42.7 & 44.3 \\
\hline
\ourmethod-S & 78.2 & 73.4 & 80.8 & 74.8 & 77.8 & 76.5 \\
\ourmethod-M & 78.4 & \textbf{75.9} & \textbf{83.2} & \textbf{76.5} & \textbf{79.8} & \textbf{78.0} \\
\end{tabular}
\end{center}
\caption{Mean anomaly localization performance per method and dataset collection, measured with the AU-PRO up to a FPR of \SI{5}{\percent}. For \ourmethod, we report the mean of five runs.}
\label{tab:localization_pro_005}
\end{table}


\begin{table}[h!]
\begin{center}
\begin{tabular}{c|c|c|ccc||c}
Method & \specialcell[c]{MAD \\ Mean} & \specialcell[c]{VisA \\ Mean} & \specialcell[c]{LOCO \\ Structural} & \specialcell[c]{LOCO \\ Logical} & \specialcell[c]{LOCO \\ Mean} & \specialcell[c]{Overall \\ Mean} \\
\hline
GCAD & 72.1 & 73.1 & 73.1 & 32.0 & 52.5 & 65.9 \\
\studteach & 74.3 & 82.7 & 69.8 & 20.6 & 45.2 & 67.4 \\
FastFlow & 72.1 & 78.9 & 63.4 & 33.7 & 48.6 & 66.5 \\
DSR & 76.1 & 66.5 & 66.0 & 25.5 & 45.7 & 62.8 \\
PatchCore & 74.1 & 65.0 & 43.5 & 24.1 & 33.8 & 57.6 \\
\patchcoreens & 79.4 & 65.7 & 38.7 & 20.4 & 29.6 & 58.2 \\
AST & 41.1 & 67.4 & 52.4 & 30.9 & 41.7 & 50.1 \\
\hline
\ourmethod-S & \textbf{79.7} & 86.3 & 80.6 & 33.8 & 57.2 & 74.4 \\
\ourmethod-M & 79.4 & \textbf{86.9} & \textbf{82.1} & \textbf{35.3} & \textbf{58.7} & \textbf{75.0} \\
\end{tabular}
\end{center}
\caption{Mean anomaly localization performance per method and dataset collection, measured with the AU-ROC up to a FPR of \SI{5}{\percent}. For \ourmethod, we report the mean of five runs.}
\label{tab:localization_roc_005}
\end{table}

\begin{table}[h!]
\begin{center}
\begin{tabular}{c|c|c|ccc||c}
Method & \specialcell[c]{MAD \\ Mean} & \specialcell[c]{VisA \\ Mean} & \specialcell[c]{LOCO \\ Structural} & \specialcell[c]{LOCO \\ Logical} & \specialcell[c]{LOCO \\ Mean} & \specialcell[c]{Overall \\ Mean} \\
\hline
GCAD & 59.3 & 27.8 & 41.4 & 38.7 & 40.1 & 42.4 \\
\studteach & 59.9 & 36.2 & 43.5 & 27.4 & 35.4 & 43.8 \\
FastFlow & 57.6 & 33.4 & 35.1 & 41.3 & 38.2 & 43.1 \\
DSR & \textbf{69.2} & \textbf{41.1} & 50.4 & 32.7 & 41.5 & 50.6 \\
PatchCore & 57.6 & 27.8 & 17.8 & 32.5 & 25.2 & 36.8 \\
\patchcoreens & 64.1 & 28.3 & 15.1 & 28.9 & 22.0 & 38.2 \\
AST & 29.7 & 22.9 & 17.0 & 35.6 & 26.3 & 26.3 \\
\hline
\ourmethod-S & 65.9 & 40.4 & \textbf{54.0} & 40.2 & \textbf{47.1} & \textbf{51.1} \\
\ourmethod-M & 63.8 & 40.8 & 51.9 & \textbf{42.0} & 46.9 & 50.5 \\
\end{tabular}
\end{center}
\caption{Mean anomaly localization performance per method and dataset collection, measured with the pixel-wise AU-PRC.  For \ourmethod, we report the mean of five runs.}
\label{tab:localization_prc}
\end{table}

\section{Timing Methodology and Additional Computational Efficiency Metrics}
\label{sec:efficiency_metrics}

\begin{table}[h!]
\begin{center}
\begin{tabular}{cccccccc}
Method & \specialcell[c]{Classif. \\ AU-ROC} & \specialcell[c]{Segment. \\ AU-PRO} & \specialcell[c]{Latency \\ {[ms]}} & \specialcell[c]{Throughput \\ {[img / s]}} & \specialcell[c]{Number of \\ Parameters [$\times 10^6$]} & \specialcell[c]{FLOPs \\ {[$\times 10^9$]}} &  \specialcell[c]{GPU Memory \\  {[MB]}} \\
\hline
GCAD & 85.4 & 88.0 & 11 & 121 & 65 & 416 & 555 \\
\studteach & 88.4 & 89.7 & 75 & 16 & 26 & 4468 & 1077 \\
FastFlow & 90.0 & 86.5 & 17 & 120 & 92 & 85 & 404 \\
DSR & 90.8 & 78.6 & 17 & 104 & 40 & 267 & 314 \\
PatchCore & 91.1 & 80.9 & 32 & 76 & 83 + 3 & 41 + kNN & 637 + kNN \\
\patchcoreens & 92.1 & 80.7 & 148 & 13 & 150 + 8 & 159 + kNN & 1335 + kNN \\
AST & 92.4 & 77.2 & 53 & 41 & 154 & 199 & 618 \\
\hline
\ourmethod-S & \meanwithstd{95.4}{0.06} & \meanwithstd{92.5}{0.05} & \meanwithstd{\textbf{2.2}}{0.01} & \meanwithstd{\textbf{614}}{2} & \meanwithstd{\textbf{8}}{0} & \meanwithstd{76}{0} & \meanwithstd{\textbf{100}}{0} \\
\ourmethod-M & \meanwithstd{\textbf{96.0}}{0.09} & \meanwithstd{\textbf{93.3}}{0.04} & \meanwithstd{4.5}{0.01} & \meanwithstd{269}{1} & \meanwithstd{21}{0} & \meanwithstd{235}{0} & \meanwithstd{161}{0} \\
\end{tabular}
\end{center}
\caption{
Extension of \Cref{tab:main} in the main paper by additional computational efficiency metrics measured on an NVIDIA RTX A6000 GPU.
For \ourmethod, we report the mean and standard deviation of five runs.
For PatchCore, we report the computational requirements of the feature extraction during inference separately from the nearest neighbor search.
}
\label{tab:efficiency}
\end{table}

In the following, we describe how we measure the latency and the throughput of each anomaly detection method.
Latency refers to the inference runtime, i.e., how long it takes a method to generate the anomaly detection result for a single image.
Throughput refers to how many images can be processed per second when allowing a batched processing of images.
In settings in which latency constraints are fulfilled or not present, a high throughput is relevant for using computational resources efficiently and thus for reducing the economic cost of an application.

All evaluated methods are implemented in PyTorch.
All of them, including the nearest neighbor search of PatchCore, run faster on each of the GPUs in our experimental setup than on a CPU.
We therefore execute each method on a GPU.
For a test image, our timing begins with the transfer of the image from the CPU to the GPU.
We include the transfer to regard the benefit of a method that would run exclusively on a CPU.
Our timing stops when the anomaly detection result, which for all evaluated methods is an anomaly map, is available on the CPU.
For each method, we remove unnecessary parts for the timing, such as the computation of losses during inference, and use float16 precision for all networks.
Switching from float32 to float16 for the inference of \ourmethod\ does not change the anomaly detection results for the 32 anomaly detection scenarios evaluated in this paper.
In latency-critical applications, padding in the PDN architecture of \ourmethod\ can be disabled.
This speeds up the forward pass of the PDN architecture by \SI{80}{\micro\second} without impairing the detection of anomalies.
We time \ourmethod\ without padding and therefore report the anomaly detection results for this setting in the experimental results of this paper.
We perform 1000 forward passes as warm up and report the mean runtime of the following 1000 forward passes.
For the latency, we report the average runtime of 1000 forward passes with a batch size of 1.
We compute the throughput by dividing \num{16000} by the sum of the runtimes of 1000 forward passes with a batch size of 16.
In addition to the latency and the throughput, we report the number of parameters, the number of floating point operations (FLOPs), and the GPU memory consumption for each method in \Cref{tab:efficiency}.
Analogously to the latency, we measure these metrics for the processing of one image during inference and report the mean of 1000 forward passes.
The number of parameters and the FLOPs remain constant across forward passes, while the GPU memory consumption varies slightly (less than one MB difference between forward passes).

\paragraph{Technical Details}
For methods that use features from hidden layers of a pretrained network, we exclude the layers that are not required for computing these features, i.e., classification heads etc.
We measure the number of FLOPs using the official profiling framework of PyTorch \cite{paszke2019_PyTorch} (version 1.12.0).
Specifically, we wrap the inference function of a method into a call of \texttt{with torch.profiler.profile(with\_flops=True) as prof:}.
For measuring the GPU memory consumption, we also use the official profiling framework of PyTorch.
We obtain the peak of the reserved GPU memory during inference with \texttt{torch.cuda.memory\_stats()['reserved\_bytes.all.peak']}.

\paragraph{Interpretability of Efficiency Metrics}
In the main paper, we focus on the latency and the throughput of the evaluated anomaly detection methods.
The number of parameters and the number of FLOPs are often used as proxy metrics for the runtime, but can be misleading.
For example, the number of parameters of FastFlow in \Cref{tab:efficiency} is roughly 2.5 times larger than that of S--T.
Yet, the latency of FastFlow is substantially lower and its throughput is 6.5 times higher.

With 4.5 trillion FLOPs, S--T exceeds the FLOPs of other methods by a large margin.
The high number of FLOPs, however, comes from the fact that S--T uses convolutions that operate on large feature maps.
This means that these convolutions can be parallelized well on a GPU, while implementing them naively on a CPU would indeed cause a prohibitively long runtime.
FLOPs measurements do not account for this, because they do not consider how well operations can be parallelized.
The number of FLOPs can therefore be an unreliable metric for efficiency.
For example, the number of FLOPs of S--T is more than \SI{2000}{\percent} higher than that of AST, but the latency is only \SI{42}{\percent} higher.

The GPU memory footprint of a method can theoretically be reduced drastically by freeing obsolete GPU memory segments after each layer's execution during a forward pass.
In the extreme case, one could even directly free the memory of individual input activation values directly after the output activation of a neuron in a convolutional layer has been computed.
This, however, would worsen the runtime of a forward pass, which generally improves when reserved GPU memory segments can be reused.
Therefore, the GPU memory footprint of a method needs to be reported and analyzed jointly with the latency and throughput.
We focus on the GPU memory required for achieving the reported latency and throughput and therefore measure the peak of the reserved GPU memory during a forward pass.

\paragraph{PatchCore}
For PatchCore, we distinguish between the backbones used to compute features and the kNN algorithm itself.
For example, the part of the WideResNet-101 backbone until the layer used for computing features has 83 million parameters.
During training, PatchCore computes the feature vectors of all training images.
The coreset subsampling phase of PatchCore reduces the number of feature vectors to \SI{1}{\percent} of the computed feature vectors.
These are then indexed and stored in GPU memory to enable a fast search for nearest neighbors during inference.
This, however, means that the number of parameters, the FLOPs, and the GPU memory footprint of PatchCore depend on the training images.
We therefore benchmark PatchCore on the ``cashew'' scenario of VisA, which contains 450 training images and is thus closest to the average 439 training images of the 32 scenarios of MVTec AD, VisA, and MVTec LOCO.
We do not report the FLOPs and the GPU memory consumption of the kNN search, as we were not able to measure it with the kNN library used by the official PatchCore implementation.
The number of parameters of the kNN search is given by the number of values stored in the GPU memory during inference.
In the case of \patchcoreens, for example, the search database contains 8 million values.

\clearpage

\paragraph{Anomaly Detection and Throughput}
\Cref{fig:throughput} shows the anomaly detection performance together with the throughput of each evaluated method, analogous to \Cref{fig:teaser} in the main paper. 
Apart from \ourmethod, the ranking of methods changes drastically between the image-level and the pixel-level detection of anomalies.

\begin{figure}[h!]
\hfill
\subfigure[Anomaly classification]{\includegraphics[width=.49\linewidth]{figures/throughput_classification.pdf}}
\hfill
\subfigure[Anomaly localization]{\includegraphics[width=.49\linewidth]{figures/throughput_segmentation.pdf}}
\hfill
\caption{Anomaly detection performance vs. throughput on an NVIDIA RTX A6000 GPU\@. We report the binary anomaly classification performance on the left using the image-level AU-ROC. On the right, we report the anomaly localization performance using the pixel-level AU-PRO segmentation metric up to a FPR of \SI{30}{\percent}. Each AU-ROC and AU-PRO value is an average of the values on MVTec AD, VisA, and MVTec LOCO. We measure the throughput using a batch size of 16. % Some methods are colored in blue to facilitate the association of their names and points.
%The average classification AU-ROC of GCAD is \SI{85.4}{\percent} and not plotted due to space constraints.
}
\label{fig:throughput}
\end{figure}


\paragraph{Latency per GPU}
In \Cref{tab:latency_per_gpu}, we provide the values for \Cref{fig:latency_per_gpu} in the main paper.

\begin{table}[h!]
\begin{center}
\begin{tabular}{cccccc}
Method & RTX A6000 & RTX A5000 & Tesla V100 & RTX 3080 & RTX 2080 Ti \\
\hline
\ourmethod-S & \textbf{2.2} & \textbf{2.5} & \textbf{3.9} & \textbf{3.8} & \textbf{4.5} \\
\ourmethod-M & 4.5 & 5.3 & 6.3 & 7.0 & 7.6 \\
GCAD & 10.7 & 11.7 & 12.9 & 13.7 & 18.0 \\
FastFlow & 16.5 & 17.1 & 26.1 & 27.5 & 31.0 \\
DSR & 17.2 & 18.0 & 24.8 & 24.6 & 34.5 \\
PatchCore & 32.0 & 31.5 & 47.1 & 41.1 & 53.2 \\
AST & 53.1 & 53.4 & 75.6 & 82.3 & 87.1 \\
\studteach & 74.7 & 81.0 & 82.2 & 99.6 & 121.7 \\
\patchcoreens & 147.6 & 145.0 & 229.2 & 189.0 & 216.9 \\
\end{tabular}
\end{center}
\caption{
Latency in milliseconds per GPU, as plotted in \Cref{fig:latency_per_gpu} in the main paper.
}
\label{tab:latency_per_gpu}
\end{table} 


\clearpage

\section{Qualitative Results}
\label{sec:qualitative_results}

In \Cref{fig:qualitative_1,fig:qualitative_2,fig:qualitative_3}, we display anomaly maps for each of the 32 scenarios of MVTec AD, VisA, and MVTec LOCO.
For MVTec LOCO, we show both logical and structural anomalies.
We visualize the anomaly maps using a different scale for each method, since the anomaly score scales differ between methods.
Across scenarios, however, we use the same color scale per anomaly detection method.
A consistent anomaly score scale across applications is an important requirement for a method.
Otherwise, the scale of scores on anomalies is hard to forecast if no or only few defect images are present during the development of the anomaly detection system.
Knowing the scale is important for choosing a robust threshold value that ultimately determines whether an image or a pixel is anomalous or not.
Furthermore, a consistent scale facilitates the interpretation of anomaly maps.
For the evaluated methods, we choose the start and end values of the color scales so that true positive and true negative detections become clearly visible.
For example, the color scale of AST ranges from 2 to 10.
Scores outside of this range are visualized with the minimum and maximum color value, respectively.
For PatchCore, choosing the range of the color scale is difficult.
On the one hand, scores of true positive detections are low, such as the contamination of the banana juice bottle in \Cref{fig:qualitative_1}.
On the other hand, scores of false positive detections are similarly high, such as the predictions on the breakfast box in \Cref{fig:qualitative_1}.

Overall, the evaluated anomaly detection methods succeed on the anomalies of MVTec AD, but leave room for improvement on MVTec LOCO and VisA.
\begin{itemize}
\item EfficientAD responds to both logical and structural anomalies in the images.
The strength of its response sometimes leaves room for improvement, for example, on the logical anomalies of the breakfast box and the box of pushpins in \Cref{fig:qualitative_1}.
% Furthermore, it generally struggles with the logical anomalies of the screw bag scenario of MVTec LOCO.
% This scenario, however, is very challenging for all evaluated methods.
    \item  AST detects some logical anomalies, but lacks an approach that detects logical anomalies by design.
For example, it detects that the additional blue cable connecting two splicing connectors in \Cref{fig:qualitative_1} causes unseen features.
Yet, the missing pushpin in the box of pushpins in \Cref{fig:qualitative_1} is also an unseen feature and does not cause a response in the anomaly map of AST.
This highlights the importance of a reliable approach to logical anomalies.
Furthermore, it shows the dependence of anomaly detection methods on the choice of the feature extractor.
As shown in \Cref{tab:distillation_backbones}, \ourmethod\ is robust to this choice.
\item DSR produces very precise segmentations, but also suffers from false positives, for example on the grid and the wood image of MVTec AD in \Cref{fig:qualitative_2}.
At times, it furthermore shows no response at all to defects.
\item FastFlow's anomaly maps contain a large amount of noise, i.e. false positive detections. This hinders the interpretability of its detection results.
\item GCAD succeeds at detecting logical anomalies, but has difficulty with some structural anomalies that other methods detect reliably, such as the scratches on the metal nut in \Cref{fig:qualitative_2} or the green capsules in \Cref{fig:qualitative_3}.
\item \patchcoreens\ struggles with very small defects such as those of the printed circuit boards in \Cref{fig:qualitative_3}.
Small defects are challenging, but highly relevant for practical applications.
A small contamination can cause a high economic damage if it goes unnoticed, for example, in a pharmaceutical application.
\item S--T is a patch-based anomaly detection approach and therefore can only detect anomalies if they involve patches that are anomalous per se, i.e., without putting them in the global context of the respective image.
While AST's feature vectors have a receptive field that spans across the entire image, S--T's receptive field is limited to 65$\times$65 pixels.
Therefore, it does not detect anomalies such as the missing transistor in \Cref{fig:qualitative_2}.
\end{itemize}

The qualitative results show tendencies of each method regarding the behavior on anomalous images.
While these results are informative, they should not be used exclusively for evaluating the anomaly detection performance of a method or for comparing methods.
For that, metrics such as the AU-ROC and the AU-PRO are well-suited, since they are evaluated objectively on thousands of test images across dataset collections.


\begin{figure}
\begin{center}
\includegraphics[width=1.0\linewidth]{figures/qualitative_1_reduced.pdf}
\end{center}
\caption{
Anomaly maps on anomalous images from MVTec LOCO and MVTec AD.
For MVTec LOCO, we show a logical anomaly (upper row) and a structural anomaly (lower row) for each scenario.
The receptive field of AST's features is large enough to detect some logical anomalies, while \patchcoreens\ and S--T struggle with logical anomalies.
}
\label{fig:qualitative_1}
\end{figure}

\begin{figure}
\begin{center}
\includegraphics[width=1.0\linewidth]{figures/qualitative_2_reduced.pdf}
\end{center}
\caption{
Anomaly maps on anomalous images from MVTec AD.
Almost all anomalies are detected by every method, but the separability of pixel anomaly scores varies between methods.
For example, \patchcoreens\ detects the anomaly on the capsule in the first row but the pixel anomaly scores are in a similar range as the false positive detections in the background of the screw image.
}
\label{fig:qualitative_2}
\end{figure}

\begin{figure}
\begin{center}
\includegraphics[width=1.0\linewidth]{figures/qualitative_3_reduced.pdf}
\end{center}
\caption{
Anomaly maps on anomalous images from MVTec AD and VisA.
VisA contains challenging, small anomalies, such as the defect on the non-aligned macaronis or the defect on the fryum two rows above.
}
\label{fig:qualitative_3}
\end{figure}

\end{document}
