
We verified the effectiveness of the proposed approach on two numerical examples.
%
All experiments were carried out on a PC with an Intel core i7-1165G7, Ubuntu Linux and MATLAB R2022a. Solution of the optimization problem \eqref{eq:LMICon2} was obtained by %using
YALMIP~\cite{Lofberg2004YALMIP}, while the vertices of polytopes were computed using %the
MPT3~\cite{Herceg2013MPT3} and 
YALMIP~\cite{Lofberg2004YALMIP} %\abnote{I suspect that CDD by K. Fukuda is used for vertex enumeration and that MPT/YALMIP are just a MATLAB frontend to it}.
As a Lipschitz global optimization solver, we adopted the \texttt{dDirect\_GLce} algorithm from the DIRECTGO toolbox~\cite{stripinis2022directgo}.
All the results of the CEGIS scheme were obtained in a few tens of seconds.

\subsection{Polytopic uncertainty set}
We first consider an uncertain system $\Sigma$ in \eqref{eq:Sigma} with four states and one input, where the matrix $B$ is not uncertain and $B(k)= B = \left[0 \; 0 \; 0 \; 1\right]^\top$, while { each entry of} the dynamical matrix $A$ is 
% \begin{align}
%     \Sigma_{\mathrm{poly}}: x_{k+1}= A_{\mathrm{poly}} +B_{\mathrm{poly}} u_k,
% \end{align}
% where 
% \begin{align}
%     B_{\mathrm{poly}}=\begin{bmatrix}
%    0&0&0&1
%     \end{bmatrix}',
% \end{align}
% and $A_{\mathrm{poly}}$ is 
subject { -- independently from the other entries --} to an interval uncertainty, % namely each entry $a_ij$ of $A_{\mathrm{poly}}$ is 
namely
$$
\begin{aligned}
    A(k) \geq &\left[\begin{smallmatrix}
   -0.6685&-0.8709&-0.2028&-1.5547\\
\phantom{-}1.1457&-0.5898&\phantom{-}0.5688&\phantom{-}0.8496\\
-0.7812&-0.5754&-0.8774&-0.2501\\
-1.1429&\phantom{-}0.1730&\phantom{-}0.7763&\phantom{-}0.1618\\
    \end{smallmatrix}\right]
,  \\
    %&\hspace{3.5cm}
    A(k)\leq& \left[\begin{smallmatrix}
%
-0.6295&-0.8202&-0.1910&-1.4641\\
\phantom{-}1.2166&-0.5555&\phantom{-}0.6040&\phantom{-}0.9022\\
-0.7357&-0.5419&-0.8263&-0.2355\\
-1.0763&\phantom{-}0.1837&\phantom{-}0.8243&\phantom{-}0.1718\\
   \end{smallmatrix}\right],
\end{aligned}
$$
for all $k\in\N$. Although such { an uncertain system} could be reliably stabilized using %standard
polytopic techniques such as the one presented in~\cite{kothare1996robust}, this would require %one
to solve an optimization problem with %at least
$2^{16}=65536$ LMI constraints, { each generated by taking independently} either the lower or upper bound for each of the 16 entries of the matrix $A$. While theoretically possible, this would be %terribly
{ hardly doable even with} modern hardware. %likely surpass the capabilities of modern hardware.

By using the proposed CEGIS method, instead, we %are able to 
found a quadratic control Lyapunov function with $P$ matrix:
$$
\bar P=\left[\begin{smallmatrix}
  162.4930 &  29.5126 &  82.0931 & 176.8625\\
   29.5126  & 42.9150 &  21.4642 &  50.7574\\
   82.0931  & 21.4642 &  58.6050 &  99.6742\\
  176.8625  & 50.7574 &  99.6742 & 232.9383\\
    \end{smallmatrix}\right]
$$
accompanied with linear feedback policy:
$
    \bar K=   \begin{bmatrix}
    1.7667 &   0.9014  & -0.3555    &1.0089
     \end{bmatrix}.
$
{ The soundness of the result was verified a-posteriori by checking the positive definiteness of $\Xi(A_h + B \bar K)$, built from the obtained $\bar P$, on each of the $65536$ pair of vertex matrices $(A_h,B)$ defining $\Omega$}. %Remarkably,
Such a result was obtained with a final constraint set $\mc C_i$ made of five samples (the initial point and four counter-examples). {
% \begin{remark}
 On the other hand, on our reference hardware solving the original LMI feasibility problem \eqref{eq:baseLineK} was not possible due to limitations on the available computational power. %as the SDP solver we relied on could not complete the process in a 10 minutes frame.
 In contrast, the proposed method was able to synthesize a quadratic control Lyapunov function in a few tens of seconds. 
 % \hfill$\square$
% \end{remark}
}
%\vspace{-0.4cm}
{
%\begin{remark}
%As a limitation of the proposed method, t

We note that the results shown above were obtained excluding from the set of Lyapunov candidates all those functions not satisfying the constraints dictated by our choice of $\eta$ and $\varepsilon$ in~\eqref{eq:LMICon2}. While, in our experience, tuning said quantities was not problematic, an unreasonable choice of those hyperparameters (e.g., $\eta=2 \cdot 10^3$ and $\varepsilon=10^3$) may indeed cause the proposed scheme to fail. In that case, one can always restart the procedure with more permissive values for $\eta$ and $\varepsilon$. % and, possibly,
%A different option would be only to establish \textit{bounds} on $\eta$ and $\varepsilon$ and letting them be optimized by the SDP solver itself.
Finally, when a quadratic control Lyapunov function is found, it may be characterized by a smaller $\varepsilon$ than in the verifier task. 
% \hfill$\square$
%\end{remark}
}
%, namely Algorithm~\ref{algo:OverallCegis} returned a solution in five iterations.

% \begin{remark}
% The soundness of~\eqref{eq:robustPoly} was tested verifying the positive definiteness of \eqref{eq:closedLoopCondition} on each one of the $65536$ vertexes of $\Omega$.
% \end{remark}

\vspace{-0.4cm}

\subsection{Spherical uncertainty set}
We consider now the case in which the matrix $A(k)$ is subject to spherical uncertainty, i.e.,
$
    \Omega = \{ A \in \R^{2\times 2} \mid (\mathrm{vec}(A)-c)^\top Q (\mathrm{vec}(A)-c)\ -1 \leq 0 \},
$
where the operator $\textrm{vec}(\cdot)$ stacks the columns of its argument into a single column vector, $Q=5 I$, while $B(k)= B = \left[0 \; 1\right]^\top$.
In particular, $c=\mathrm{vec}(A_\mathrm{centroid})$, where 
$
A_\mathrm{centroid}= \left[\begin{smallmatrix}
       \phantom{-} 0.6458   & 0.3852\\
   -1.4651   & 1.1183
\end{smallmatrix}\right]$. %{\color{red} For $\eta=\ldots$, $\varepsilon=\ldots$},
The synthesis of a control Lyapunov function was obtained with a counter-example set including four points, i.e., Algorithm~\ref{algo:OverallCegis} converged in three iterations, %(including the centroid $c$, also adopted for the initialization of Algorithm~\ref{algo:OverallCegis})).
yielding 
$
\bar P = \left[\begin{smallmatrix}
%
  117.4770 &  60.7593\\
   60.7593 & 130.6819
\end{smallmatrix}\right]$ 
% $$ 
% with corresponding gain matrix
and 
%
$\bar K=\begin{bmatrix}
    0.9280 &  -1.4962    
\end{bmatrix}$. 
{ We verified the soundness of these results by using a convex polytopic outer-approximation of the spherical set $\Omega$, built using the method in~\cite{yalmip2016sampleBased}}. Specifically, the polytope was built using 1000 randomly generated rays, %. The resulting polytope %(which one might have used to outerly approximate $\Omega$ and use a classical polytopic approach)
yielding 6592 vertices. This fact reinforces the findings of the previous example regarding the efficiency of our approach in comparison to purely geometric set approximation alternatives. 
Moreover, the technique in \cite{yalmip2016sampleBased} unavoidably introduces some degree of conservatism, which is avoided using our method. %(in the considered example, one of the vertices of the approximating polytope violated the constraint by over $1$)
While a spherical $\Omega$ theoretically has an infinite number of vertices, Theorem~\ref{th:convergence} guarantees that, in view of the %compactness of $\Omega$ and
assumption 
$\eta \geq \varepsilon>0$, to find a quadratic control Lyapunov function and associated linear controller is sufficient to examine only a finite number of vertices. %{\color{red} Riportare quanto segue per il primo esempio, e fare un confronto anche con un altro metodo classico, mostrando che non funge.} { This comes at the cost of excluding from the set of candidate quadratic control Lyapunov functions those that do not satisfy the constraints of the learner task \eqref{eq:LMICon2}, because of a too large choice for $\eta$ or $\varepsilon$}. {\color{red} As one extreme case, for $\eta=...$, $\varepsilon=...$, the proposed CEGIS scheme terminates by declaring infeasibility of the learner task, returning no quadratic control Lyapunov function. This highlights the importance of a proper tuning of the hyperparameters of the proposed method, in order to make its application effective.} %\abnote{Andrebbe forse commentato che pur avendo il set $\Omega$ sferico un numero infinito di vertici, il teorema 2 comunque garantisce che un numero finito di vertici \`e sufficiente. Immagino per via del fatto che $\epsilon>0$?}

%\abnote{[Nei due esempi $\Omega$ \`e convesso. \`E possibile aggiungere un esempio in cui $\Omega$ non \`e convesso facendo vedere come l'overapproximation data dal convex hull del set, che \`e quello che uno farebbe normalmente, e' molto piu' conservativa rispetto a CEGIS? Altrimenti, l'approccio si reduce ad un metodo per generare in maniera incrementale i vertici del convex-hull di $\Omega$ necessari a risolvere l'LMI robusta.]} %\ggnote{In effetti il metodo si potrebbe riassumere così: si risolve un problema più semplice in cui si considera solo un sottoinsieme di LMIs, si cerca di verificare se la soluzione trovata soddisfa anche tutte le altre LMIs, e se non è così si aggiunge al sottoinsieme di LMIs l'LMI che al momento è violata maggiormente.}

% \subsection{Experimental setup and computational times}


