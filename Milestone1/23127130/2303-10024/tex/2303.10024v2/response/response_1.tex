%\documentclass[12pt]{article}
\documentclass[12pt,fleqn]{article}

\usepackage{graphicx,fleqn,amstext,a4wide,amsmath,amssymb,color,enumitem,setspace,xr,dsfont,theorem,parskip}
%\usepackage{xc}

%----------- To include code ------------
\usepackage{fancyvrb}
\usepackage{xcolor}
\definecolor{olivegreen}{RGB}{39,128,0}
\usepackage{hyperref}
%\textwidth160mm
%\textheight 230mm
%\topmargin -15mm

%\addtolength{\oddsidemargin}{-18mm}
%\addtolength{\evensidemargin}{-25mm}

%----------- For comments ------------
\usepackage[prependcaption,colorinlistoftodos]{todonotes}
\newcommand{\todoinr}[1]{\todo[inline,color=red!60, linecolor=orange!250]{\bf\small#1}}
\newcommand{\todoinb}[1]{\todo[inline,color=blue!60, linecolor=orange!250]{\bf\small#1}}
\newcommand{\todoing}[1]{\todo[inline,color=green!60, linecolor=orange!250]{\bf\small#1}}
\newcommand{\todoiny}[1]{\todo[inline,color=yellow!80, linecolor=orange!250]{\bf\small#1}}
\newcommand{\nmr}[1]{{\color{black} #1}}

%
%%\newtheorem*{theorem*}{Theorem}
{\theorembodyfont{\itshape}\newtheorem{theorem}{Theorem}}
{\theorembodyfont{\itshape}\newtheorem{lemma}[theorem]{Lemma}}
{\theorembodyfont{\itshape}\newtheorem{proposition}[theorem]{Proposition }}
{\theorembodyfont{\itshape}\newtheorem{corollary}[theorem]{Corollary}}
{\theorembodyfont{\upshape}\newtheorem{definition}[theorem]{Definition}}
{\theorembodyfont{\itshape}\newtheorem{conjecture}[theorem]{Conjecture}}
{\theorembodyfont{\upshape}\newtheorem{remark}[theorem]{Remark}}
{\theorembodyfont{\upshape}\newtheorem{problem}[theorem]{Problem}}
{\theorembodyfont{\upshape}\newtheorem{example}[theorem]{Example}}
%{\theorembodyfont{\upshape}\newtheorem{example}{Example}}
{\theorembodyfont{\upshape}\newtheorem{assumption}{Assumption}}
\newcommand{\bs}[1]{\boldsymbol{#1}}
%\onehalfspacing

% external documents
%\externaldocument{ct_aggregative_v20}
%\externalcitedocument{ct_aggregative_v20}

%\newcommand{\fpart}[2]{\displaystyle \frac{\partial #1}{\partial #2}}
%\newcommand{\barfpart}[2] {\displaystyle \left.\frac{\partial #1}{\partial #2}\right|_-}
%\newcommand{\barfpartT}[2] {\displaystyle \left.\frac{\partial #1}{\partial #2}\right|_-^T}
%\newcommand{\wfpart}[2]{\frac{\partial #1}{\partial #2}}

\newcommand{\reviewer}[2]{\bigskip\noindent\textbf{Reviewer #1:} {\it #2}}
\newcommand{\me}[1]{\medskip\noindent\textbf{Authors' response:}{\color{blue}{#1}}}
\newcommand{\rev}[1]{{\color{blue} #1}}
%\usepackage{hyperref}

%%%----------- For comments ------------
\usepackage[prependcaption,colorinlistoftodos]{todonotes}
\newcommand{\cdpcomment}[1]{\todo[inline,color=green!20, linecolor=orange!250]{\bf\small CDP: #1}}
\newcommand{\sgcomment}[1]{\todo[inline,color=green!20, linecolor=orange!250]{\bf\small SG: #1}}
\newcommand{\pgcomment}[1]{\todo[inline,color=blue!10, linecolor=blue!250]{PG: #1}}
\newcommand{\ewinsert}[1]{ {\color{blue}#1} }
\newcommand{\cdp}[1]{ {\color{red}#1} }
\newcommand{\sg}[1]{ {\color{blue}#1} }
%\newcommand{\ones}{\mathds{1}}
\newcommand{\cmargin}[1]{\marginpar{\bf\color{blue}\tiny\ttfamily{C:} #1}}
\newcommand{\smargin}[1]{\marginpar{\bf\color{red}\tiny\ttfamily{S:} #1}}
%%%%%%%%%%%%%%%%%%%%%%%%%%%%%%%%%%%%%%%%%%%%%%%%%%%%%%%%%%%%%%%%%%%%%%%%
% Set the path for figures
%\graphicspath{figures/}

%%%%%%%%%%%%%%%%%%%%%%%%%%%%%%%%%%%%%%%%%%%%%%%%%%%%%%%%%%%%%%%%%%%%%%%%





\begin{document}
\pagestyle{myheadings}

\thispagestyle{empty}

\markright{\small IEEE L-CSS 23-0334:
	{\sl ``Counter-example guided inductive synthesis of control Lyapunov...'' \quad}}

\headsep 0.5cm

\bigskip\bigskip

\noindent{Prof. Ketan Savla\\
	University of Southern California\\
	Los Angeles, California (US)\\
	{e-mail: \texttt{ksavla@usc.edu}}
}


\bigskip\bigskip

\begin{flushright}
	Dr. Filippo Fabiani\\
	
	IMT School for Advanced Studies\\
	Lucca, IT\\
	{e-mail: \texttt{filippo.fabiani@imtlucca.it}}
\end{flushright}

\vspace*{2cm}


Lucca, \today%%
\medskip

\noindent \textbf{IEEE L-CSS 23-0334:} {\em ``Counter-example guided inductive synthesis of control Lyapunov functions for uncertain systems''.}



\bigskip

Dear Prof.~Savla,\\

Please find enclosed the revision of our manuscript ``Counter-example guided inductive synthesis of control Lyapunov functions for uncertain systems''.

We thank you for the careful handling of our submission, and we thank the Associate Editor and the Reviewers for their constructive comments and criticisms.  In the revised manuscript, we have addressed all the Reviewers' comments, and modified the paper accordingly.  The details of these modifications are in the Authors' response following this letter.   We hope that the revised manuscript meets your, the Associate Editor and the Reviewers' expectations.\\

We look forward to hearing from you.

%\vspace*{1.25cm}
\bigskip\bigskip

\begin{flushright}
	Sincerely yours,\\
	\textit{Filippo Fabiani}, \textit{Daniele Masti}, \textit{Giorgio Gnecco}, \textit{Alberto Bemporad}
\end{flushright}

\clearpage
\setcounter{page}{1}
\parindent=0pt

\newpage
%In the following, we respond to the comments raised by the Reviewers. We would like to thank the Reviewers for their constructive remarks.

All changes made in the paper in response to Reviewers' comments are highlighted in {\color{blue}blue font} so that they are easily visible for the Reviewers and the Editor(s).  We have also made a small number of minor cosmetic changes

\bigskip

\begin{center}
	{\bf \LARGE Statement of Revision}
\end{center}

\medskip

%%%%%%%%%%%%%%%%%%%%%%%%%%%%%%%%%%%%%%%%%%%%%%%%%%%%%%%%%%%%%%%%%%%%%%%%%%%%%%%%%%%
\section*{Associate Editor's Comments}
{\it
Three reviews found possible merit in the proposed method for Lyapunov
function synthesis, but all believed further discussion of the method
is required. Reviewer 2 requests comparison to sum-of-squares based
methods and more discussion about the limitations of the proposed
method. Reviewer 4 believes that there is insufficient motivation and
justification for the counter-example guided approach compared to a
direct approach. In particular, the proposed approach has its own
sources of conservatism. This reviewer suggests a systematic comparison
of the conservatism and complexity of the two approaches, preferably
with accompanying examples. Reviewer 6 also suggests the advantages of
the proposed method must be explained more definitively. All reviewer
comments should be taken into consideration in a revision. 
}

\bigskip{}

{\color{blue}
	We thank the Associate Editor for their efforts in handling our paper.

	In the revised manuscript, we have addressed all the points raised by the Reviewers and made the following substantial revisions to the paper:

	\begin{enumerate}
		\item ...
		\item ...
		\item ...
	\end{enumerate}
}

%%%%%%%%%%%%%%%%%%%%%%%%%%%%%%%%%%%%%%%%%%%%%%%%%%%%%%%%%%%%%%%%%%%%%%%%
\newpage
\section*{Response to Reviewer 2}

We would like to thank the Reviewer for their constructive criticism. We address below the comments that were raised.

\reviewer{2}{
The paper extends the construction of control Lyapunov Functions with
the CEGIS method for linear systems with uncertain parameters.
The main contributions is to characterize the Lipschitz continuity
property of the verification task if a function is a control Lyapunov
function for a given uncertain system, which allows efficient solvers.

I have the following comments, questions, and suggestions.

1. The result of the presented algorithm depends on the initial conditions
and the parameters  eta and epsilon. It may be discussed that no
statement about the non-existence of a quadratic Lyapunov function for
the uncertain system independently from this parameters can be given
(the finite time comes only to this price). In Section V. an example
should be included for the termination of the algorithm in the learner
((5) is infeasible).
}


\me{
	...
}


\reviewer{2}{
	2. There should be explicitly noted, that the soundness of the obtained
	result need to be verified separately, since the verification step via
	DIRECTGO might produce potentially unsound result due to its numerical
	nature see e.g. Example 8 in [1].
	Additionally, there should be at least one or two references with
	respect to the Sum-of-Squares techniques and the difference to the SOS
	methods that are conventionally applied to the linear parameter-varying
	systems as e.g. in [2].
	
	[1] Ahmed, D., Peruffo, A., Abate, A. (2020). Automated and Sound
	Synthesis of Lyapunov Functions with SMT Solvers.
	https://doi.org/10.1007/978-3-030-45190-5\_6
	
	[2] Fen Wu and S. Prajna, "A new solution approach to polynomial LPV
	system analysis and synthesis," Proceedings of the 2004 American
	Control Conference, Boston, MA, USA, 2004, pp. 1362-1367 vol.2, doi:
	10.23919/ACC.2004.1386764.
}


\me{
	...
}

%%%%%%%%%%%%%%%%%%%%%%%%%%%%%%%%%%%%%%%%%%%%%%%%%%%%%%%%%%%%%%%%%%%%%%%%
\newpage
\section*{Response to Reviewer 4}

We would like to thank the Reviewer for their constructive criticism. We address below the comments that were raised.

\reviewer{4}{
	The paper presents a method to compute a linear controller along with a
	quadratic Lyapunov function for discrete-time linear systems $Ax + Bu$
	with uncertainty on the matrices $A$ and $B$, i.e., $(A,B)$ belongs to some
	known set $\Omega$, but we do not know their exact values, and we want to
	compute a controller and a Lyapunov function to are valid for all
	values of $(A,B)$ in $\Omega$.
	The approach is sampled-based and counterexample-guided: from a finite
	set of samples $(A,B)$, the algorithm computes a controller and a
	Lyapunov function compatible with the samples.
	Then, it checks whether the results in valid for ALL $(A,B) \in \Omega$.
	If not, it obtains a new sample $(A,B)$ and the process is performed
	again.
	The process is guaranteed to terminate.
	It involves two user-defined parameters, which have an impact on the
	number of steps to termination as well as on the conservativeness of
	the approach (see comments below).
	
	The paper is overall well written, and I spotted no mistakes in the
	proofs (although I have some comments/questions, listed below).
}

\me{
	We thank the Reviewer for the positive evaluation.
}

\reviewer{4}{
1. My main concern is about the motivation for using a
counterexample-guided approach, instead of a direct approach to compute
the controller and the Lyapunov function.
If $\Omega$ is a complex set, then for the direct approach, one has to
resort to polyhedral approximations, ideally with few vertices.
This would inevitably introduce conservativeness (in the direct
approach) due to the approximation.
But, since the counterexample-guided approach also has conservativeness
(due to the user-provided parameters), I think that the
conservativeness (and the time complexity) of both approaches should be
compared in a more systematic way in the paper.
I reckon that building a (simple) polyhedral approximation of $\Omega$
can be challenging in itself, so that a general comparison can be
challenging to provide.
But, at least some concrete examples, for which the
counterexample-guided approach is superior to the direct approach,
would be very helpful.
I think that the paper would really benefit from such comparisons on a
few concrete problems, because in the present form, I find that just
two synthetic examples (for which the direct approach is not actually
applied) are not convincing that the new approach might be of practical
interest.
}


\me{
	...
}

\reviewer{4}{
	Other comments and questions:
	
	2. I think that it should be mentioned in the introduction that the
	counterexample-guided approach also has conservativeness due to the
	user-defined parameters.
	This is especially important since the conservativeness of the direct
	approach is mentioned has the main reason to use a
	counterexample-guided approach.
}

\me{
	...
}

\reviewer{4}{
	3. Related to the above, this is a subtle form of conservativeness that
	is present in the counterexample-guided approach: either there is not
	controller and Lyapunov function satisfying (5) with $\epsilon$ and $\eta$
	provided by the user, or the algorithm outputs a controller and a
	Lyapunov function satisfying (5) but with $\epsilon$ and $\eta$ that can be
	smaller than those provided by the user.
	I think this should be emphasized somewhere in the paper.
}

\me{
	...
}

\reviewer{4}{
	4. In Theorem 1, I think that the result should be stated in terms of
	$\Delta A$ and $\Delta B$ (instead of $\Delta A^cl$) because it is $(A,B)$ that
	are the variables of the optimization problem.
	In particular, I think that, in this case, the Lipschitz constant
	becomes a multiple of $\epsilon^2$ and not only $\epsilon$.
}

\me{
	...
}

\reviewer{4}{
	5. Question (related to the above): why Problem (8) is applied on the
	matrix defined in (7) and not on the matrix in (3)?
	I might be wrong, but I think the same argument could be used with the
	matrix in (3), and that, in that case, the Lipschitz constant is linear
	in $\eta$.
	Following this line of thoughts, I think that this could provide better
	bounds in the proof of Theorem 2: namely, something proportional to
	$\epsilon/\eta$, instead of $\epsilon/\eta^2$.
}

\me{
	...
}

\reviewer{4}{
	6. I do not agree with the affirmation in the introduction that
	considering a convexification of $\Omega$ introduces conservativeness (in
	the direct approach).
	Indeed, if (3) holds for all $(A,B) \in \Omega$, then it also holds for
	all $(A,B)$ in the convex hull of $\Omega$.
	So I think that the main argument, there is that computing the convex
	hull can be challenging, or the convex hull can be intractable
	computationally.
	Please provide clarification about this.
}

\me{
	...
}



%%%%%%%%%%%%%%%%%%%%%%%%%%%%%%%%%%%%%%%%%%%%%%%%%%%%%%%%%%%%%%%%%%%%%%%%
\newpage
\section*{Response to Reviewer 6}

We would like to thank the Reviewer for their constructive criticism. We address below the comments that were raised.

\reviewer{6}{
	The paper proposes a counter-example based control Lyapunov synthesis
	method to design state feedback controller for linear uncertain
	systems. It is an iterative learning and verification based approach,
	the
	basic idea of which is already existing in literature, The authors
	extend the idea to solve state feedback control for uncertain systems.
	Overall the paper is written well and the contributions are clear.
}

\me{
	We thank the Reviewer for the positive evaluation.
}

\reviewer{6}{
	However, the following points should be taken into consideration in the
	revised version.
	
	1. Does the proposed iterative method always yield solution if the
	problem is solvable using standard LMI method, i.e., using (3)?
	
}

\me{
	...
}

\reviewer{6}{
	2. It seems to me that in example 1, $2^16$ LMI constraints are needed to
	solve the state feedback based MPC framed in [3], but not to design a 
	fixed state feedback controller. I suggest the author to recheck it
	before writing this strong statement. 
}

\me{
	...
}

\reviewer{6}{
	3. If the problem is solvable using standard LMI, what is the use of
	proposed approach? The advantages should be clearly brought out.
}

\me{
	...
}

\reviewer{6}{
	4. It seems that the method is applicable to LPV type systems. How to
	apply the same to systems with norm-bounded uncertainties?
}

\me{
	...
}

%%%%%%%%%%%%%%%%%%%%%%%%%%%%%%%%%%%%%%%%%%%%%%%%%%%%%%%%%%%%%%%%%%%%%%%%%%%%%%%%%%%
\medskip
\end{document}
