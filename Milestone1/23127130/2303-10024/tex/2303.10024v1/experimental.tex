
We verify the effectiveness of the proposed approach on two numerical examples.
%
All experiments have been carried out on a PC using an Intel core i7-1165G7, Ubuntu Linux and MATLAB R2022a. Solution of the optimization problem \eqref{eq:LMICon2} was obtained by using YALMIP~\cite{Lofberg2004YALMIP}, while the computation of the vertexes of polytopes was carried out using %the
MPT3~\cite{Herceg2013MPT3} and 
YALMIP~\cite{Lofberg2004YALMIP} %\abnote{I suspect that CDD by K. Fukuda is used for vertex enumeration and that MPT/YALMIP are just a MATLAB frontend to it}.
As a Lipschitz global optimization solver, we adopted the \texttt{dDirect\_GLce} algorithm from the DIRECTGO toolbox~\cite{stripinis2019penalty,stripinis2022directgo}.
Note that all the results have been obtained in a few tens of seconds.

\subsection{Polytopic uncertainty set}
We first consider an uncertain system $\Sigma$ in \eqref{eq:Sigma} with four states and one input, where the matrix $B$ is deterministic and $B(k)= B = \left[0 \; 0 \; 0 \; 1\right]^\top$, while the dynamical matrix $A$ is 
% \begin{align}
%     \Sigma_{\mathrm{poly}}: x_{k+1}= A_{\mathrm{poly}} +B_{\mathrm{poly}} u_k,
% \end{align}
% where 
% \begin{align}
%     B_{\mathrm{poly}}=\begin{bmatrix}
%    0&0&0&1
%     \end{bmatrix}',
% \end{align}
% and $A_{\mathrm{poly}}$ is 
subject to an interval uncertainty, % namely each entry $a_ij$ of $A_{\mathrm{poly}}$ is 
namely
$$
\begin{aligned}
    &\left[\begin{matrix}
   -0.6685&-0.8709&-0.2028&-1.5547\\
\phantom{-}1.1457&-0.5898&\phantom{-}0.5688&\phantom{-}0.8496\\
-0.7812&-0.5754&-0.8774&-0.2501\\
-1.1429&\phantom{-}0.1730&\phantom{-}0.7763&\phantom{-}0.1618\\
    \end{matrix}\right] \\
    \leq& \hspace{0.1cm} A(k)  \\
    %&\hspace{3.5cm}
    \leq& \left[\begin{matrix}
%
-0.6295&-0.8202&-0.1910&-1.4641\\
\phantom{-}1.2166&-0.5555&\phantom{-}0.6040&\phantom{-}0.9022\\
-0.7357&-0.5419&-0.8263&-0.2355\\
-1.0763&\phantom{-}0.1837&\phantom{-}0.8243&\phantom{-}0.1718\\
   \end{matrix}\right],
\end{aligned}
$$
for all $k\in\N$. Note that such systems could be reliably stabilized using standard polytopic techniques such as the one presented in~\cite{kothare1996robust}. This would, however, require one to solve an optimization problem with %at least
$2^{16}=65536$ LMI constraints, due to all possible combinations at either the lower or upper bound for each of the 16 entries of the matrix $A$. While theoretically possible, this would be terribly inefficient and hardly doable in modern hardware. %likely surpass the capabilities of modern hardware.

By using the proposed CEGIS methodology, instead, we %are able to 
find a quadratic control Lyapunov function with $P$ matrix:
$$
    \bar P=\left[\begin{matrix}
  162.4930 &  29.5126 &  82.0931 & 176.8625\\
   29.5126  & 42.9150 &  21.4642 &  50.7574\\
   82.0931  & 21.4642 &  58.6050 &  99.6742\\
  176.8625  & 50.7574 &  99.6742 & 232.9383\\
    \end{matrix}\right]
$$
accompanied with linear feedback policy:
$
    \bar K=   \begin{bmatrix}
    1.7667 &   0.9014  & -0.3555    &1.0089
     \end{bmatrix}.
$
The soundness of the result was verified a-posteriori by checking the positive definiteness of $\Xi(A_h + B \bar K)$, constructed with the obtained $\bar P$, on each one of the $65536$ pair of vertex matrices $(A_h,B)$ defining $\Omega$.
Remarkably, such a result was obtained with a constraint set $\mc C_i$ consisting of five samples (i.e., the initial point and four counter-examples).%, namely Algorithm~\ref{algo:OverallCegis} returned a solution in five iterations.

% \begin{remark}
% The soundness of~\eqref{eq:robustPoly} was tested verifying the positive definiteness of \eqref{eq:closedLoopCondition} on each one of the $65536$ vertexes of $\Omega$.
% \end{remark}

\subsection{Spherical uncertainty set}
We consider now the case in which the matrix $A(k)$ is subject to spherical uncertainty, i.e.,
$
    \Omega = \{ A \in \R^{2\times 2} \mid (\mathrm{vec}(A)-c)^\top Q (\mathrm{vec}(A)-c)\ -1 \leq 0 \},
$
where the operator $\textrm{vec}(\cdot)$ stacks the columns of its argument into a single column vector, $Q=5 I$, while $B(k)= B = \left[0 \; 1\right]^\top$.
In particular, $c=\mathrm{vec}(A_\mathrm{centroid})$, where 
$$
A_\mathrm{centroid}= \left[\begin{matrix}
       \phantom{-} 0.6458   & 0.3852\\
   -1.4651   & 1.1183
\end{matrix}\right]. 
$$
The synthesis of a control Lyapunov function was obtained with a counter-example set including four points, i.e., Algorithm~\ref{algo:OverallCegis} converged in three iterations, %(including the centroid $c$, also adopted for the initialization of Algorithm~\ref{algo:OverallCegis})).
yielding 
$$
\bar P = \left[\begin{matrix}
%
  117.4770 &  60.7593\\
   60.7593 & 130.6819
\end{matrix}\right] \text{ and }
% $$ 
% with corresponding gain matrix
% $
\bar K=\begin{bmatrix}
    0.9280 &  -1.4962    
\end{bmatrix}.
$$
We tested these results on a polytopic outer-approximation of the spherical $\Omega$, built using the method in in~\cite{yalmip2016sampleBased}. Specifically, the polytope was build using 1000 randomly generated rays, %. The resulting polytope %(which one might have used to outerly approximate $\Omega$ and use a classical polytopic approach)
yielding 6592 vertices. This fact reinforces the findings of the previous example regarding the efficiency of our approach in comparison to purely geometric set approximation alternatives. 
Moreover, the technique in \cite{yalmip2016sampleBased} unavoidably introduces some degree of conservatism, which is avoided using our methodology. %(in the considered example, one of the vertices of the approximating polytope violated the constraint by over $1$)
While a spherical $\Omega$ theoretically has an infinite number of vertices, Theorem~\ref{th:convergence} guarantees that, in view of the compactness of $\Omega$ and
$\varepsilon>0$, to find a quadratic control Lyapunov function and associated linear controller is sufficient to examine only a finite number of vertices.  %\abnote{Andrebbe forse commentato che pur avendo il set $\Omega$ sferico un numero infinito di vertici, il teorema 2 comunque garantisce che un numero finito di vertici \`e sufficiente. Immagino per via del fatto che $\epsilon>0$?}

%\abnote{[Nei due esempi $\Omega$ \`e convesso. \`E possibile aggiungere un esempio in cui $\Omega$ non \`e convesso facendo vedere come l'overapproximation data dal convex hull del set, che \`e quello che uno farebbe normalmente, e' molto piu' conservativa rispetto a CEGIS? Altrimenti, l'approccio si reduce ad un metodo per generare in maniera incrementale i vertici del convex-hull di $\Omega$ necessari a risolvere l'LMI robusta.]} %\ggnote{In effetti il metodo si potrebbe riassumere così: si risolve un problema più semplice in cui si considera solo un sottoinsieme di LMIs, si cerca di verificare se la soluzione trovata soddisfa anche tutte le altre LMIs, e se non è così si aggiunge al sottoinsieme di LMIs l'LMI che al momento è violata maggiormente.}

% \subsection{Experimental setup and computational times}


