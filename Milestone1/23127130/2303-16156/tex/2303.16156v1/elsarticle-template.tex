\documentclass[review]{elsarticle}

\usepackage{lineno,hyperref}
\usepackage{amsfonts,amssymb}
\usepackage{amsmath}
\usepackage{ntheorem}
\biboptions{sort&compress}
\newtheorem*{Proof}{Proof}
\newtheorem{Theorem}{Theorem}


\modulolinenumbers[5]

\journal{Journal of \LaTeX\ Templates}

%%%%%%%%%%%%%%%%%%%%%%%
%% Elsevier bibliography styles
%%%%%%%%%%%%%%%%%%%%%%%
%% To change the style, put a % in front of the second line of the current style and
%% remove the % from the second line of the style you would like to use.
%%%%%%%%%%%%%%%%%%%%%%%

%% Numbered
%\bibliographystyle{model1-num-names}

%% Numbered without titles
%\bibliographystyle{model1a-num-names}

%% Harvard
%\bibliographystyle{model2-names.bst}\biboptions{authoryear}

%% Vancouver numbered
%\usepackage{numcompress}\bibliographystyle{model3-num-names}

%% Vancouver name/year
%\usepackage{numcompress}\bibliographystyle{model4-names}\biboptions{authoryear}

%% APA style
%\bibliographystyle{model5-names}\biboptions{authoryear}

%% AMA style
%\usepackage{numcompress}\bibliographystyle{model6-num-names}

%% `Elsevier LaTeX' style
\bibliographystyle{elsarticle-num}
%%%%%%%%%%%%%%%%%%%%%%%

\begin{document}

\begin{frontmatter}

\title{On the derivatives  of rational B\'{e}zier curves\tnoteref{mytitlenote}}
%\tnotetext[mytitlenote]{Fully documented templates are available in the elsarticle package on \href{http://www.ctan.org/tex-archive/macros/latex/contrib/elsarticle}{CTAN}.}

%% Group authors per affiliation:
\author{Mao Shi\fnref{myfootnote}}
\address{School of Mathematics and Statistics of Shaanxi Normal University, Xi'an China}
%\fntext[myfootnote]{Since 1880.}

%% or include affiliations in footnotes:
%\author[mymainaddress,mysecondaryaddress]{Elsevier Inc}
%\ead[url]{www.elsevier.com}

%\author[mysecondaryaddress]{Global Customer Service\corref{mycorrespondingauthor}}
%\cortext[mycorrespondingauthor]{Corresponding author}
\ead{shimao@snnu.edu.cn}

%\address[mymainaddress]{1600 John F Kennedy Boulevard, Philadelphia}
%\address[mysecondaryaddress]{360 Park Avenue South, New York}

\begin{abstract}
We first  point out the defects of the existing derivative formula on the rational B\'{e}zier curve, then propose a new recursive derivative formula, and discuss the expression of derivative formula at the endpoints.
\end{abstract}

\begin{keyword}
Derivative formulas \sep Endpoints \sep  Rational B\'{e}zier curves
\end{keyword}

\end{frontmatter}

\linenumbers

\section{Introduction}

Rational B\'{e}zier curves are essential mathematical tools in CAGD {\cite{Farin2002}}. They share the same properties as B\'{e}zier curves, such as evaluation, subdivision, the convex hull property, and degree elevation, and also exhibit non-uniform convergence and divergence as the weights $\omega_{i}$ tend to positive infinity {\cite{Shi2005}}{\cite{Shi2018}}. Finding the derivative of a rational B\'{e}zier curve is a challenging problem that differs from the derivative of a B\'{e}zier curve{\cite{Sederberg1987}}{\cite{Floater1992}}.  Although the first and second-order derivatives of rational B\'{e}zier curves have been extensively studied, there is less research on their high-order derivative formula \cite{Deng2011, Li2013, Bez2013, Jin2014, Wu2004}. On the other hand, the complexity of the expression for the derivative of a rational B\'{e}zier curve leads to more attention on the case where the degree $n$ of rational B\'{e}zier curves exceeds the order $k$ of the derivative than on the case where $n<k$ \cite{wang1995,  Shi2020}. For example, for the second order derivatives of the rational curve \eqref{eq:001} at the endpoints, neither Wang's formula \cite{wang1995} nor Lin's formula \cite{Lin2009} can lead to the following conclusion 1).

 1) When $n=1$, we have
 $${\boldsymbol{r''}}(0) = \frac{{2\omega _1 (\omega _0  - \omega _1 )\left( {{\boldsymbol{r}}_1  - {\boldsymbol{r}}_0 } \right)}}{{\omega _0^2 }},$$
 and
 $${\boldsymbol{r''}}(1) = \frac{{2\omega _0 (\omega _0  - \omega _1 )\left( {{\boldsymbol{r}}_1  - {\boldsymbol{r}}_0 } \right)}}{{\omega _1^2 }}.$$


 2) When $n \geq 2$, we have
 %$${\boldsymbol{r''}}(0) =  - \frac{{n(\omega _0 ( - \omega _2 n - 2\omega _1  + \omega _2 ) + 2n\omega _1^2 )}}{{\omega _0^2 }}\left( {{\boldsymbol{r}}_1  - {\boldsymbol{r}}_0 } \right) - \frac{{n\omega _2 (1 - n)}}{{\omega _0 }}\left( {{\boldsymbol{r}}_2  - {\boldsymbol{r}}_1 } \right),$$
 \begin{eqnarray*}
 % \nonumber to remove numbering (before each equation)
 {\boldsymbol{r''}}(0) &=&  - \frac{{n(\omega _0 ( - \omega _2 n - 2\omega _1  + \omega _2 ) + 2n\omega _1^2 )}}{{\omega _0^2 }}\left( {{\boldsymbol{r}}_1  - {\boldsymbol{r}}_0 } \right) - \frac{{n\omega _2 (1 - n)}}{{\omega _0 }}\left( {{\boldsymbol{r}}_2  - {\boldsymbol{r}}_1 } \right), \\
 \textrm{and}\\
   {\boldsymbol{r''}}(1) &=& \frac{{n\omega _{n - 2} (1 - n)}}{{\omega _n }}\left( {{\boldsymbol{r}}_{n - 1} - {\boldsymbol{r}}_{n - 2} } \right)\\
    & &- \frac{{n(\omega _n (\omega _{n - 2} n - \omega _{n - 2}  + 2\omega _{n - 1} ) - 2n\omega _{n - 1}^2 )}}{{\omega _n^2 }}\left( {{\boldsymbol{r}}_n  - {\boldsymbol{r}}_{n - 1} } \right).
 \end{eqnarray*}
 %$${\boldsymbol{r''}}(1) = \frac{{n\omega _{n - 2} (1 - n)}}{{\omega _n }}\left( {{\boldsymbol{r}}_{n - 1}  - {\boldsymbol{r}}_{n - 2} } \right) - \frac{{n(\omega _n (\omega _{n - 2} n - \omega _{n - 2}  + 2\omega _{n - 1} ) - 2n\omega _{n - 1}^2 )}}{{\omega _n^2 }}\left( {{\boldsymbol{r}}_n  - {\boldsymbol{r}}_{n - 1} } \right).$$

In this paper, we propose a new method to derive the formula for any order of derivative of a rational curve. Our method also simplifies the computation of high order derivatives at the endpoints.

The rest of this paper is organized as follows. In Section 2, we review the definition  of rational B\'{e}zier curves and some preliminary results  that will be needed later. In Section 3, we present our main results. As a example, we show the third-order derivative formula often used in engineering in Section 4. Finally, concludes the paper.

\section{Preliminary}
A rational B\'{e}zier curve of degree $n$ can be defined as
\begin{equation}\label{eq:001}
  \boldsymbol{r}\left( t \right) =\frac{\sum\limits_{i=0}^n{\omega _i\boldsymbol{r}_iB_{i}^{n}\left( t \right)}}{\sum\limits_{i=0}^n{\omega _iB_{i}^{n}\left( t \right)}},\ t\in \left[0,1\right],
\end{equation}
where $B_i^n(t)= {n \choose i}t^i(1-t)^{n-i}$ are Bernstein basis functions of degree $n$, $\boldsymbol{r}_i\in\mathbb{R}^d$ are control points and $\omega_i$ are the corresponding positive weights.

When all   weights are equal and nonzero, the rational B\'{e}zier curve is reduced to an integer B\'{e}zier curve
  \begin{equation} \label{eq:002}
 \boldsymbol{p}(t)=\sum\limits_{i=0}^{n}\boldsymbol{p}_iB_i^n(t).
 \end{equation}

The $k$th derivative of a B\'{e}zier curve can be expressed as
\begin{equation}\label{eq:003}
\boldsymbol{p}^{(k)}(t)=\frac{n!}{(n-k)!}\sum_{j=0}^{n-k}\Delta^k\boldsymbol{p}_jB_{j}^{n-k}\left( t \right),
\end{equation}
where $\triangle^k$ is the $k$th difference.

Let ${\boldsymbol{q}}(t) $ be another B\'{e}zier curve of degree $m$ with  control points$\{{\boldsymbol{q}}_i\}_{i=0}^{m}$, then the product of ${\boldsymbol{q}}(t) $ and  ${\boldsymbol{p}}(t)$ can be written as \cite{Farouki1988}
      \begin{eqnarray}\label{eq:004}
{\boldsymbol{p}}(t)\boldsymbol{q}(t)=\sum_{k=0}^{m+n}\sum\limits_{j=max(0,k-n)}^{min(m,k)}\frac{{m \choose j}{n \choose k-j}}{{m+n \choose k}}{\boldsymbol{p}}_{k-j}{\boldsymbol{q}}_{j}B_{k}^{m+n}(t).
%\label{eq:4}
\end{eqnarray}
In addition, we denote Pi notation as
\begin{equation}\label{eq:005}
  \prod\limits_{k = l}^m {x_k }  = \left\{ {\begin{array}{*{20}c}
   {x_l x_{l + 1}  \cdots x_m } & {l \le m},  \\
   1 & {otherwise}.  \\
\end{array}} \right.
\end{equation}
\section{Main results}
\begin{Theorem}
  The $k$th derivative of a rational B\'{e}zier curve can be represented as
  \begin{equation}\label{eq:006}
    \boldsymbol {r}^{(k)}({t}) =\frac{\prod\limits_{j=0}^{k-1}{\left(2^j n \right)}\sum\limits_{i=0}^{2^k n}{\boldsymbol{\hat{P}}_{i}^{\left[ k \right]}}B_{i}^{2^k n}\left( t \right)}{\sum\limits_{i=0}^{2^k n}{\omega _{i}^{\left[ k \right]}B_{i}^{2^k n}\left( t \right)}},
  \end{equation}
where
\begin{equation}\label{eq:007}
\omega_i^{[0]}=\omega_i,\  \boldsymbol{P}_i^{[0]}=\boldsymbol{\hat{P}}_{i}^{[0]}=\omega_i \boldsymbol{r}_i,
\end{equation}
\begin{equation}\label{eq:008}
  \omega _{i}^{\left[ k \right]}=\sum_{j=\max \left( 0,i-2^{k-1}n \right)}^{\min \left( i,2^{k-1}n \right)}{\frac{{
	2^{k-1}n\choose j
}{ 	2^{k-1}n\choose 	i-j
}}{{
	2^kn\choose 	i}}}\omega _{j}^{\left[ k-1 \right]}\omega _{i-j}^{\left[ k-1 \right]},
\end{equation}
\begin{equation} \label{eq:009}
\boldsymbol{\hat{P}}_{i}^{\left[ k \right]}=\sum_{j=\max \left( 0,i-1 \right)}^{\min \left(2^{k}n-1,i \right)}{\frac{{
2^{k}n-1 \choose	j
} {
	1 \choose 	i-j
}}{{
2^k n \choose	i
}}\boldsymbol{P}_{j}^{\left[ k \right]}}
\end{equation}
and
\begin{equation} \label{eq:010}
\boldsymbol{P}_{j}^{\left[ k \right]}=\sum_{h=\max \left( 0,j-2^{k-1}n \right)}^{\min \left( 2^{k-1}n-1,j \right)}{\frac{{
	2^{k-1}n-1\choose 	h
} {
	2^{k-1}n\choose 	j-h
}}{{
	2^kn-1\choose 	j
}}\left( \Delta \boldsymbol{\hat{P}}_{h}^{\left[ k-1 \right]}\omega _{j-h}^{\left[ k-1 \right]}-\Delta \omega _{h}^{\left[ k-1 \right]}\boldsymbol{\hat{P}}_{j-h}^{\left[ k-1 \right]} \right)}.
\end{equation}
\end{Theorem}
\begin{Proof}
  The proof is by induction on $k$. It is clearly true for $k=0$ and $k=1$  based on equations\eqref{eq:003}, \eqref{eq:004}, \eqref{eq:005}, \eqref{eq:007}, \eqref{eq:008}, \eqref{eq:009} and \eqref{eq:010}. Assumed that it is valid for $k > 1$, then letting
$$
\boldsymbol{\hat{P}}^{\left[ k \right]}\left( t \right) =\sum_{i=0}^{2^kn}{\boldsymbol{\hat{P}}_{i}^{\left[ k \right]}}B_{i}^{2^kn}\left( t \right),
$$
and $$
\omega ^{\left[ k \right]}\left( t \right) =\sum_{i=0}^{2^kn}{\omega _{i}^{\left[ k \right]}B_{i}^{2^kn}\left( t \right)},
$$
we have
\begin{small}
\allowdisplaybreaks[4]
\begin{align*}\label{aa}
  %\begin{split}
     & \boldsymbol{r}^{\left( k+1 \right)}\left( t \right) \\
=& \frac{\prod\limits_{j=0}^{k-1}{\left( 2^jn \right)}\left[ \left( \boldsymbol{\hat{P}}^{\left[ k \right]}\left( t \right) \right) '\omega ^{\left[ k \right]}\left( t \right) -\boldsymbol{\hat{P}}^{\left[ k \right]}\left( t \right) \left( \omega ^{\left[ k \right]}\left( t \right) \right) ' \right]}{\left( \omega ^{\left[ k \right]}\left( t \right) \right) ^2} \\
=&\frac{\prod\limits_{j=0}^k{( 2^jn )}\left[ \left( \sum\limits_{i=0}^{2^kn-1}{\varDelta \boldsymbol{\hat{P}}_{i}^{\left[ k \right]}}B_{i}^{2^kn-1}\left( t \right) \right) \omega ^{\left[ k \right]}\left( t \right) -\boldsymbol{\hat{P}}^{\left[ k \right]}\left( t \right) \left( \sum\limits_{i=0}^{2^kn-1}{\varDelta \omega _{i}^{\left[ k \right]}B_{i}^{2^kn-1}\left( t \right)} \right) \right]}{\left( \sum\limits_{i=0}^{2^kn}{\omega _{i}^{\left[ k \right]}B_{i}^{2^kn}\left( t \right)} \right) ^2}\\
=&\frac{\prod\limits_{j=0}^k{\left( 2^jn \right)}\sum\limits_{i=0}^{2^{k+1}n}{\left[ \sum\limits_{j=\max \left( 0,i-2^kn \right)}^{\min \left( 2^kn-1,i \right)}{\frac{{2^kn-1 \choose
	j}{	2^kn \choose i-j}}{{2^{k+1}n-1\choose i}}\left( \varDelta \boldsymbol{\hat{P}}_{j}^{\left[ k \right]}\omega _{i-j}^{\left[ k \right]}-\varDelta \omega _{j}^{\left[ k \right]}\boldsymbol{\hat{P}}_{i-j}^{\left[ k \right]} \right)} \right] B_{i}^{2^{k+1}n}\left( t \right)}}{ \sum\limits_{i=0}^{2^{k+1}n}{\left( \sum\limits_{j=\max \left( 0,i-2^kn \right)}^{\min \left( i,2^kn \right)}{\frac{{2^kn \choose j}{2^kn \choose i-j} }{{2^{k+1}n \choose i}}}\omega _{i}^{\left[ k \right]}\omega _{i-j}^{\left[ k \right]} \right) B_{i}^{2^{k+1}n}\left( t \right)} }\\
=&\frac{\prod\limits_{j=0}^k{\left( 2^jn \right)}\sum\limits_{i=0}^{2^{k+1}n}{\boldsymbol{\hat{P}}_{i}^{\left[ k+1 \right]}}B_{i}^{2^{k+1}n}\left( t \right)}{\sum\limits_{i=0}^{2^{k+1}n}{\omega _{i}^{\left[ k+1 \right]}B_{i}^{2^{k+1}n}\left( t \right)}},
 % \end{split}
\end{align*}
\end{small}
showing that \eqref{eq:006} holds for $k+1$. This completes the proof.
\end{Proof}

\begin{Theorem}
  The derivatives of the rational B\'{e}zier curve \eqref{eq:001} at  $t=0$ and $t=1$ are
  \begin{eqnarray}
  % \nonumber to remove numbering (before each equation)
    {\boldsymbol{r}}^{(k)} (0) &=& \frac{{2^{\frac{1}{2}k(k - 1)} n^k }}{{\left( {\omega _0 } \right)^{2^k } }}{\boldsymbol{P}}_0^{[k]} \ \ \  \ \ \ \  \ \ \ \ \ \  \ \ \  \ \  \ \ \ \ \ \ \ \ \ \ \  \ \ \ \ \  \ \   (k \ge 0) \label{eq:011} \\
     &=& \frac{{2^{\frac{1}{2}k(k - 1)} n^k }}{{\left( {\omega _0 } \right)^{2^k } }}\left( {{\boldsymbol{\hat P}}_1^{\left[ {k - 1} \right]} \omega _0^{\left[ {k - 1} \right]}  - \omega _1^{\left[ {k - 1} \right]} {\boldsymbol{\hat P}}_0^{\left[ {k - 1} \right]} } \right)\ \ \  (k \ge 1)  \label{eq:012}
  \end{eqnarray}
and
\begin{small}
\begin{eqnarray}
% \nonumber to remove numbering (before each equation)
  {\boldsymbol{r}}^{(k)} (1) &=& \frac{{2^{\frac{1}{2}k(k - 1)} n^k }}{{\left( {\omega _n } \right)^{2^k } }}{\boldsymbol{P}}_{2^k n - 1}^{[k]} {\rm{ }}\ \ \  \ \ \ \  \ \ \ \ \ \  \ \ \  \ \  \ \ \ \ \ \ \ \ \ \ \  \ \ \ \ \  \ \ \ \ \ \ \ \  (k \ge 0) \label{eq:013}  \\
   &=& \frac{{2^{\frac{1}{2}k(k - 1)} n^k }}{{\left( {\omega _n } \right)^{2^k } }}\left( {\Delta {\boldsymbol{\hat P}}_{2^{k - 1} n - 1}^{\left[ {k - 1} \right]} \omega _{2^{k - 1} n}^{\left[ {k - 1} \right]}  - \Delta \omega _{2^{k - 1} n - 1}^{\left[ {k - 1} \right]} {\boldsymbol{\hat P}}_{2^{k - 1} n}^{\left[ {k - 1} \right]} } \right)(k \ge 1) \label{eq:014}
\end{eqnarray}
\end{small}
\end{Theorem}
\begin{Proof}
We only prove ${\boldsymbol{r}}^{(k)} (0)$. Similar methods can also prove ${\boldsymbol{r}}^{(k)} (1)$.

Obviously
\begin{equation}\label{eq:015}
  \prod\limits_{j = 0}^{k - 1} {\left( {2^j n} \right)}  = 2^{{\textstyle{1 \over 2}}k(k - 1)} n^k.
\end{equation}

By \eqref{eq:008}, we get
\begin{eqnarray}
% \nonumber to remove numbering (before each equation)
  \omega _0^{\left[ k \right]}  &=& \sum\limits_{j = \max \left( {0,0 - 2^{k - 1} n} \right)}^{\min \left( {0,2^{k - 1} n} \right)} {\frac{{\binom{2^{k - 1} n}{j}\binom{2^{k - 1} n}{0 - j}}}{{\binom{2^k n}{0}}}} \omega _j^{\left[ {k - 1} \right]} \omega _{0 - j}^{\left[ {k - 1} \right]} \nonumber  \\
  &=& \omega _0^{\left[ {k - 1} \right]} \omega _{0 - 0}^{\left[ {k - 1} \right]}  = \left( {\omega _0^{\left[ {k - 1} \right]} } \right)^2  = \left( {\omega _0^{\left[ {k - 2} \right]} } \right)^{2^2 }  \nonumber \\
   &=&   \cdots  \cdots   \nonumber \\
   &=& \left( {\omega _0 } \right)^{2^k }.  \label{eq:016}
\end{eqnarray}

Based on \eqref{eq:009}and \eqref{eq:010}, it yields
\begin{eqnarray}
% \nonumber to remove numbering (before each equation)
  {\boldsymbol{\hat P}}_0^{\left[ k \right]}  &=& {\boldsymbol{P}}_0^{\left[ k \right]}  \nonumber \\
  &=& \Delta {\boldsymbol{\hat P}}_0^{\left[ {k - 1} \right]} \omega _{0 - 0}^{\left[ {k - 1} \right]}  - \Delta \omega _0^{\left[ {k - 1} \right]} {\boldsymbol{\hat P}}_{0 - 0}^{\left[ {k - 1} \right]}   \nonumber \\
  &=& {\boldsymbol{\hat P}}_1^{\left[ {k - 1} \right]} \omega _0^{\left[ {k - 1} \right]}  - \omega _1^{\left[ {k - 1} \right]} {\boldsymbol{\hat P}}_0^{\left[ {k - 1} \right]} {\rm{ }}. \label{eq:017}
\end{eqnarray}
Substituting \eqref{eq:015}, \eqref{eq:016} and \eqref{eq:017} into  ${\boldsymbol{r}}^{(k)} (0)$, we can obtain \eqref{eq:011} and \eqref{eq:012}.
\end{Proof}

Finally, using equations \eqref{eq:004}, \eqref{eq:006} and  the following inequality \cite{Kuang1989}
  \begin{equation*}%\label{eq:01la}
\frac{{\sum {a_k c_k } }}
{{\sum {b_k c_k } }} \leqslant \mathop {\max }\limits_k \frac{{a_k }}
{{b_k }},
\end{equation*}
where $a_k \in\mathbb{R}$, $b_k>0 $, $c_k>0$ and $\ 1\leq k \leq n$, we  obtain an upper bound of the $k$th derivative of the rational B\'{e}zier curve.
\begin{Theorem}
\[
\left\| {{\boldsymbol{r}}^{\left( k \right)} \left( t \right)} \right\|_{l^p} \le\mathop {\max }\limits_i \frac{ \prod\limits_{j = 0}^{k - 1} {\left( {2^j n} \right)}{\left\| {\sum\limits_{j = \max (0,i - e)}^{\min (2^k n,i)} {{{2^k n}  \choose  j }{ e  \choose
   {i - j} } {\boldsymbol{\hat P}}_j^{\left[ k \right]} } } \right\|_{l^p}}}{{\sum\limits_{j = \max (0,i - e)}^{\min (2^k n,i)} {{{2^k n}  \choose  j }{ e  \choose
   {i - j} }\omega _j^{\left[ k \right]} } }},\ \ \ (i=0,...,2^kn+e),
\]
where $e$ is the number of degree elevation and $l^ p$ is the vector norm.
\end{Theorem}

\section{Example}
The following example shows  the derivatives formulas of third of the $n$th rational B\'{e}zier curve at the endpoints. The derivative formula of other orders can also be found by applying Theorem 2.

  1) When $n=1$, we have
 $${\boldsymbol{r'''}}(0) = \frac{{6\omega _1 (\omega _0  - \omega _1 )^2 \left( {{\boldsymbol{r}}_1  - {\boldsymbol{r}}_0 } \right)}}{{\omega _0^3 }},$$
 and
 $${\boldsymbol{r'''}}(1) = \frac{{6\omega _0 ( - \omega _1  + \omega _0 )^2 \left( {{\boldsymbol{r}}_1  - {\boldsymbol{r}}_0 } \right)}}{{\omega _1^3 }}.$$

  2) When $n=2$, we have
 $${\boldsymbol{r'''}}(0) = \frac{{12(\omega _0  - 2\omega _1 )((\omega _1  + \omega _2 )\omega _0  - 2\omega _1^2 )}}{{\omega _0^3 }}\left( {{\boldsymbol{r}}_1  - {\boldsymbol{r}}_0 } \right) - \frac{{12\omega _2 (\omega _1  - \omega _0 )}}{{\omega _0^2 }}\left( {{\boldsymbol{r}}_2  - {\boldsymbol{r}}_1 } \right),$$
 and
 \[{\boldsymbol{r'''}}(1) = \frac{{12\omega _0 (\omega _2  - \omega _1 )}}{{\omega _2^2 }}\left( {{\boldsymbol{r}}_1  - {\boldsymbol{r}}_0 } \right) - \frac{{12((\omega _0  + \omega _1 )\omega _2  - 2\omega _1^2 )(2\omega _1  - \omega _2 )}}{{\omega _2^3 }}\left( {{\boldsymbol{r}}_2  - {\boldsymbol{r}}_1 } \right).\]

  3) When $n\geq3$, we have
\[
\begin{array}{l}
 {\boldsymbol{r'''}}(0) = \frac{{n((n^2 \omega _3  + (6\omega _2  - 3\omega _3 )n + 6\omega _1  - 6\omega _2  + 2\omega _3 )\omega _0^2  - 6n\omega _1 (n\omega _2  + 2\omega _1  - \omega _2 )\omega _0  + 6\omega _1^3 n^2 )}}{{\omega _0^3 }}\left( {{\boldsymbol{r}}_1  - {\boldsymbol{r}}_0 } \right) \\
  + \frac{{(n - 1)((n\omega _3  + 6\omega _2  - 2\omega _3 )\omega _0  - 3n\omega _1 \omega _2 )n}}{{\omega _0^2 }}\left( {{\boldsymbol{r}}_2  - {\boldsymbol{r}}_1 } \right) + \frac{{(n^2  - 3n + 2)n\omega _3 }}{{\omega _0 }}\left( {{\boldsymbol{r}}_3  - {\boldsymbol{r}}_2 } \right). \\
 \end{array}
\]
 and
  \[
\begin{array}{l}
 {\boldsymbol{r'''}}(1) \\
  = \frac{{n(n - 1)((n\omega _{n - 3}  + 6\omega _{n - 2}  - 2\omega _{n - 3} )\omega _n  - 3n\omega _{n - 1} \omega _{n - 2} )}}{{\omega _n^2 }}\left( {{\boldsymbol{r}}_{n - 1}  - {\boldsymbol{r}}_{n - 2} } \right) + \frac{{(n^2  - 3n + 2)n\omega _{n - 3} }}{{\omega _n }}\left( {{\boldsymbol{r}}_{n - 2}  - {\boldsymbol{r}}_{n - 3} } \right) +  \\
 \frac{{((n^2 \omega _{n - 3}  + (6\omega _{n - 2}  - 3\omega _{n - 3} )n + 6\omega _{n - 1}  - 6\omega _{n - 2}  + 2\omega _{n - 3} )\omega _n^2  - 6n\omega _{n - 1} (n\omega _{n - 2}  + 2\omega _{n - 1}  - \omega _{n - 2} )\omega _n  + 6\omega _{n - 1}^3 n^2 )}}{{\omega _n^3 }}\left( {{\boldsymbol{r}}_n  - {\boldsymbol{r}}_{n - 1} } \right). \\
 \end{array}
\]
\section{Conclusion}
  A complete and correct derivation method of rational B\'{e}zier curves is proposed, which can be extended to rational B\'{e}zier surfaces.
\section{Acknowledgements}
This paper is supported by  Natural Science Foundation of Shaanxi Province (No. 2013JM1004).
\section{Declaration of competing interest}
The authors declare that they have no known competing financial interests or personal relationships that could have appeared to influence the work reported in this paper.




%\section*{References}

\bibliography{mybibfile}


\end{document} 