\documentclass[10pt]{article} %
\usepackage[preprint]{tmlr}


\definecolor{citecolor}{HTML}{0071bc}
\usepackage[breaklinks=true,bookmarks=false,colorlinks,bookmarks=false, citecolor=citecolor]{hyperref} 
\usepackage{url}


\PassOptionsToPackage{numbers}{natbib}
\usepackage{array, multirow, tablefootnote}
\PassOptionsToPackage{dvipsnames,table}{xcolor}
\usepackage[bottom]{footmisc}
\usepackage{adjustbox}
\usepackage{amsmath, amsfonts, bm}
\usepackage{caption}
\usepackage{subcaption}
\usepackage{pifont}
\usepackage{booktabs}
\usepackage{xspace}
\usepackage{wrapfig}
\usepackage{placeins}
\usepackage[normalem]{ulem}

\let\oldcite\cite
\renewcommand{\cite}[1]{\mbox{\oldcite{#1}}}


\newcommand{\pietro}[1]{\noindent\textcolor{teal}{PA: #1}\\}
\def\ara#1{{\color{blue} Ar: #1}}
\def\jkb#1{{\color{magenta} Jkb: #1}}
\def\ars#1{{\color{olive} ARS: #1}}

\def \ours {{DA\textsubscript{IC-GAN}}\xspace}
\def \ccicgan {{CC-IC-GAN}\xspace}
\def \icgan {{IC-GAN}\xspace}
\def \allicgan {{(CC-)IC-GAN}\xspace}


\newcommand{\todo}{\noindent\textcolor{red}{\% TODO} \textcolor{red}}
\newcommand{\cmark}{\ding{51}}%
\newcommand{\cmarkg}{\textcolor{gray}{\ding{51}}}%
\newcommand{\xmark}{\ding{55}}%
\newcommand{\xmarkg}{\textcolor{gray}{\ding{55}}}%


\newcommand*{\x}{\mathbf{x}}
\newcommand*{\y}{\mathbf{y}}
\newcommand*{\z}{\mathbf{z}}
\newcommand*{\h}{\mathbf{h}}
\renewcommand*{\L}{\mathcal{L}}
\newcommand*{\D}{\mathcal{D}}
\newcommand*{\R}{\mathbb{R}}
\newcommand*{\Rimg}{\R^{3 \times H \times W}}
\DeclareMathOperator*{\argmin}{arg\,min}
\DeclareMathOperator*{\argmax}{arg\,max}
\DeclareMathOperator*{\topk}{top_{K}}
\DeclareMathOperator*{\argtopk}{arg\,\topk}

\title{Instance-Conditioned GAN Data Augmentation\\for Representation Learning}


\author{\name Pietro Astolfi\footnotemark[1] \footnotemark[2] \textsuperscript{1},
Arantxa Casanova\footnotemark[1] \textsuperscript{1,2,5},
Jakob Verbeek\textsuperscript{1},
Pascal Vincent\textsuperscript{1,2,3,4},\\
Adriana Romero-Soriano\textsuperscript{1,2,6},
Michal Drozdzal\textsuperscript{1}\\
\addr 
\textsuperscript{1}Meta AI, 
\textsuperscript{2}Mila, Quebec AI Institute,
\textsuperscript{3}Université de Montréal,
\textsuperscript{4}CIFAR,\\
\textsuperscript{5}École Polytechnique de Montréal,
\textsuperscript{6}McGill University
\\
\email \footnotemark[2] pietroastolfi@meta.com \\
\footnotemark[1] Contributed equally
      }


\newcommand{\fix}{\marginpar{FIX}}
\newcommand{\new}{\marginpar{NEW}}

\def\month{MM}  %
\def\year{YYYY} %
\def\openreview{\url{https://openreview.net/forum?id=XXXX}} %


\begin{document}


\maketitle

\begin{abstract}



Data augmentation has become a crucial component to train state-of-the-art visual representation models. However, handcrafting combinations of transformations that lead to improved performances is a laborious task, which can result in visually unrealistic samples. To overcome these limitations, recent works have explored the use of generative models as learnable data augmentation tools, showing promising results in narrow application domains, e.g., few-shot learning and low-data medical imaging. In this paper, we introduce a data augmentation module, called \ours, which leverages instance-conditioned GAN generations and can be used off-the-shelf in conjunction with most state-of-the-art training recipes. We showcase the benefits of \ours by plugging it out-of-the-box into the supervised training of ResNets and DeiT models on the ImageNet dataset, and achieving accuracy boosts up to between 1\%p and 2\%p with the highest capacity models. Moreover, the learnt representations are shown to be more robust than the baselines when transferred to a handful of out-of-distribution datasets, and exhibit increased invariance to variations of instance and viewpoints. We additionally couple \ours with a self-supervised training recipe and show that we can also achieve an improvement of 1\%p in accuracy in some settings. %
With this work, we strengthen the evidence on the potential of learnable data augmentations to improve visual representation learning, paving the road towards non-handcrafted augmentations in model training.

\end{abstract}

\section{Introduction}
\label{sec:introduction}
% \begin{itemize}
%     % Diffusion of FL
%     \item {\st{Diffusion of FL}}
%     % Security threats to FL
%     \item {\st{Security threats to FL with particular focus on model poisoning}}
%     % Limitations of existing countermeasures
%     \item {\st{Current countermeasures (e.g., KRUM) and their limitations}}
%     % Proposed method and its advantages
%     \item {\st{Intuitive description of the proposed method and its difference (i.e., advantages) w.r.t. state of the art}}
%     % Main contributions
%     \item {\st{Summary of the main contributions of this work}}
%     % Paper's structure and organization
%     \item {\st{Paper's structure and organization}}
% \end{itemize}

% Diffusion of FL
Recently, {\em federated learning} (FL) has emerged as the leading paradigm for training distributed, large-scale, and privacy-preserving machine learning (ML) systems~\cite{mcmahan2017googleai,mcmahan2017aistats}. 
The core idea of FL is to allow multiple edge clients to collaboratively train a shared, global model without disclosing their local private training data.
%Specifically, an FL system consists of a central server and many edge clients; 
A typical FL round involves the following steps: {\em(i)} the server randomly picks some clients and sends them the current, global model; {\em(ii)} each selected client locally trains its model with its own private data; then, it sends the resulting local model to the server;\footnote{Whenever we refer to global/local model, we mean global/local model {\em parameters}.} {\em(iii)} the server updates the global model by computing an \emph{aggregation function}, usually the average (FedAvg), on the local models received from clients.
% \begin{enumerate}
%     \item[{\em(i)}] the server sends the current, global model to the clients and appoints some of them for training;
%     \item[{\em(ii)}] each selected client locally trains its copy of the global model with its own private data; then, it sends the resulting local model back to the server;\footnote{Whenever we refer to global/local model, we mean global/local model {\em parameters}.}
%     \item[{\em(iii)}] the server updates the global model by computing an \emph{aggregation function} on the local models received from clients (by default, the average, also referred to as FedAvg~\cite{mcmahan2017aistats}).
% \end{enumerate}
This process goes on until the global model converges. %(e.g., after a certain number of rounds or other similar stopping criteria).
%\\
% The advantages of FL over the traditional, centralized learning paradigm are undoubtedly clear in terms of flexibility/scalability (clients can join/disconnect from the FL network dynamically), network communications (only model weights\footnote{We will use \textit{parameters} and \textit{weights} interchangeably.} are exchanged between clients and server), and privacy (each client's private training data is kept local at the client's end and not uploaded to the server).
\\
% Security threats to FL
%However, the growing adoption of FL also raises security concerns~\cite{costa2022covert}, particularly about its confidentiality, integrity, and availability.
Although its advantages over standard ML, FL also raises security concerns~\cite{costa2022covert}. %, particularly about its confidentiality, integrity, and availability~\cite{costa2022covert}.
% OLD, LONG VERSION
% Indeed, some work deals with privacy leakage that may expose the local data of some clients~\cite{melis2019sp}. 
% A large body of work, instead, investigates attacks that usually aim to detriment the predictive accuracy of the learned global model. For instance, \emph{data poisoning} attacks achieve this goal by letting an adversary pollute the training set of some corrupt FL clients with maliciously crafted examples~\cite{jagielski2018sp}.
% Similarly, in \emph{model poisoning} the attacker attempts to tweak the global model weights~\cite{bhagoji2019pmlr} by directly perturbing the local model's weights of some infected FL clients before these are sent to the central server for aggregation, usually via so-called Byzantine attacks. 
% It turns out that Byzantine model poisoning attacks severely impact standard FedAvg; therefore, more robust aggregation functions must be designed to make FL systems secure.
Here, we focus on \emph{untargeted model poisoning} attacks~\cite{bhagoji2019pmlr}, where an adversary attempts to tweak the global model weights %\footnote{We will use the terms \textit{parameters} and \textit{weights} interchangeably.} 
by directly perturbing the local model's parameters of some infected clients before these are sent to the central server for aggregation.
In doing so, the adversary aims to jeopardize the global model \textit{indiscriminately} at inference time.
Such model poisoning attacks severely impact standard FedAvg; therefore, more robust aggregation functions must be designed to secure FL systems.
\\
% In this paper, we focus on designing a novel robust aggregation scheme at the server's end to contrast the effect of Byzantine model poisoning attacks.
%
% Current countermeasures and their limitations
%Several countermeasures have been proposed in the literature to combat model poisoning attacks on FL systems.
% Some methods use simple statistics more robust than plain average to smooth the impact of malicious updates (e.g., Trimmed Mean and FedMedian~\cite{yin2018icml}). 
% Other defenses implement outlier detection techniques to discard malicious updates from the aggregation performed at the server's end. Those are either based on heuristics (e.g., Krum/Multi-Krum~\cite{blanchard2017nips} and Bulyan~\cite{mhamdi2018pmlr}) or data-driven approaches (e.g., K-means clustering~\cite{shen2016acm} or DnC via spectral analysis~\cite{shejwalkar2021ndss}). 
% Finally, some strategies rely on a centralized ``source of trust'' to spot potential malicious updates (e.g., FLTrust~\cite{cao2020fltrust}).
% Several countermeasures have been proposed in the literature to combat model poisoning attacks on FL systems, i.e., to discard possible malicious local updates from the aggregation performed at the server's end. 
% These techniques range from simple statistics more robust than plain average (e.g., Trimmed Mean and FedMedian~\cite{yin2018icml}) to outlier detection heuristics (e.g., Krum/Multi-Krum~\cite{blanchard2017nips} and Bulyan~\cite{mhamdi2018pmlr}) or data-driven approaches (e.g., spectral analysis via K-means clustering~\cite{shen2016acm} or spectral analysis), or methods based on ``source of trust'' (e.g., FLTrust~\cite{cao2020fltrust}).
% OLD, LONG VERSION
%Several countermeasures have been proposed in the literature to combat Byzantine model poisoning attacks on FL systems.
% Descriptive statistics
% For example, Trimmed Mean and FedMedian aggregate local model updates using more robust statistics than standard average~\cite{yin2018icml}.
%
% % Heuristics for outlier detection
% Many existing Byzantine-resilient strategies implement some outlier detection heuristics to discard the model updates sent by potentially malicious clients from the input of the aggregation function.
% One of the most popular heuristics is Krum~\cite{blanchard2017nips}.
% This strategy tries to mitigate the impact of Byzantine attacks by selecting as a global model the local model with the smallest sum of Euclidean distances to {\em all} the other local models.
% Although powerful, Krum requires the server to know (or, at least, estimate) the number of malicious FL clients upfront, which is generally impossible in a realistic attack scenario. %
% Moreover, Krum may become ineffective for complex, high-dimensional model parameter spaces due to the curse of dimensionality.
% Bulyan~\cite{mhamdi2018pmlr} tries to overcome this issue by combining Krum with a variant of Trimmed Mean.
% % Data-driven outlier detection
% Other strategies use data-driven outlier detection techniques -- e.g., via K-means clustering~\cite{shen2016acm} -- to spot potential malicious local model updates. 
% %For instance, Shen et al. propose to cluster local model updates with K-means and thus identify outliers.
%
% % Other techniques
% As far as the server is concerned, any local model received can be from a potential malicious client. 
% FLTrust~\cite{cao2020fltrust} assumes the server acts as a client, i.e., trains a local model on an additional {\em trustworthy} dataset at the server's end and compares it against all the local models from other clients. 
% This way, the server can rely on some ``source of trust'' when discarding potentially malicious clients.
%\\
% Limitations of existing Byzantine-resilient strategies
Unfortunately, existing defense mechanisms either rely on simple heuristics (e.g., Trimmed Mean and FedMedian by~\cite{yin2018icml}) or need strong and unrealistic assumptions to work effectively (e.g., foreknowledge or estimation of the number of malicious clients in the FL system, as for Krum/Multi-Krum~\cite{blanchard2017nips} and Bulyan~\cite{mhamdi2018pmlr}, which, however, cannot exceed a fixed threshold).
Furthermore, outlier detection methods using K-means clustering~\cite{shen2016acm} or spectral analysis like DnC~\cite{shejwalkar2021ndss} do not directly consider the temporal evolution of local model updates received.
Finally, strategies like FLTrust~\cite{cao2020fltrust} require the server to collect its own dataset and act as a proper client, thereby altering the standard FL protocol.
\\
% OLD, LONG VERSION
% Overall, existing Byzantine-resilient strategies are either simple heuristics (e.g., FedMedian) or, if they are more complex, they rely on strong and unrealistic assumptions to work effectively (e.g., knowing the number of malicious clients in the FL system in advance, as for Krum and alike).
% Furthermore, data-driven outlier detection methods do not consider the temporary evolution of local model updates received (e.g., K-means clustering). 
% Finally, strategies like FLTrust requires the server to collect its own dataset and act as a proper client, thereby altering the standard FL protocol.
%
% Description of the proposed method
This work introduces a novel pre-aggregation \textit{filter} robust to untargeted model poisoning attacks. Notably, this filter $(i)$ operates without requiring prior knowledge or constraints on the number of malicious clients and $(ii)$ inherently integrates temporal dependencies. 
The FL server can employ this filter as a preprocessing step before applying \textit{any} aggregation function, be it standard like FedAvg or robust like Krum or Bulyan.
Specifically, we formulate the problem of identifying corrupted updates as a multidimensional (i.e., matrix-valued) time series anomaly detection task. 
The key idea is that legitimate local updates, resulting from well-calibrated iterative procedures like stochastic gradient descent (SGD) with an appropriate learning rate, show \textit{higher predictability} compared to malicious updates. This hypothesis stems from the fact that the sequence of gradients (thus, model parameters) observed during legitimate training exhibit regular patterns, as validated in Section~\ref{subsec:intuition}. %until convergence. 
%This regularity may be more pronounced for smooth convex loss functions, but it can still be captured within an appropriate time window, even for more complex and convoluted loss surfaces. 
%We provide evidence of this claim in Appendix~B, where we show that the average mutual information (i.e., ``predictability''), calculated over pairs of legitimate model updates sent at different FL rounds, is significantly higher than the corresponding computation for a malicious client.
\\
Inspired by the matrix autoregressive (MAR) framework for multidimensional time series forecasting~\cite{chen2021je}, we propose the FLANDERS ({\em \textbf{F}ederated \textbf{L}earning meets \textbf{AN}omaly \textbf{DE}tection for a \textbf{R}obust and \textbf{S}ecure}) filter.
The main advantages of FLANDERS over existing strategies like FLDetector~\cite{zhao2020multivariate} are its resilience to large-scale attacks, where $50\%$ or more FL participants are hostile, and the capability of working under realistic non-iid scenarios.
We attribute such a capability to two key factors: $(i)$ FLANDERS works without knowing a priori the ratio of corrupted clients, and $(ii)$ it embodies temporal dependencies between intra- and inter-client updates, quickly recognizing local model drifts caused by evil players. Below, we summarize our main contributions:

\begin{itemize}
\item[{\em(i)}]
We provide empirical evidence that the sequence of models sent by legitimate clients is more predictable than those of malicious participants performing untargeted model poisoning attacks.
\\
\item[{\em(ii)}] 
We introduce FLANDERS, the first pre-aggregation filter for FL robust to untargeted model poisoning based on multidimensional time series anomaly detection.
\\
\item[{\em(iii)}] 
We integrate FLANDERS into Flower,\footnote{\scriptsize{\url{https://flower.dev/}}} a popular FL simulation framework for reproducibility.
\\
\item[{\em(iv)}] 
We show that FLANDERS improves the robustness of the existing aggregation methods under multiple settings: different datasets, client's data distribution (non-iid), models, and attack scenarios.
\\
\item[{\em(v)}] 
We publicly release all the implementation code of FLANDERS along with our experiments.\footnote{\scriptsize{\url{https://anonymous.4open.science/r/flanders_exp-7EEB}}}
\end{itemize}

% Paper's structure and organization
The remainder of the paper is structured as follows. %some related work and the current state-of-the-art solutions to security issues that FL entails. 
Section~\ref{sec:background} covers background and preliminaries. 
In Section~\ref{sec:related}, we discuss related work.
Section~\ref{sec:problem} and Section~\ref{sec:method} describe the problem formulation and the method proposed. % to tackle it. 
Section~\ref{sec:experiments} gathers experimental results. %, and Section~\ref{sec:limitations} discusses some limitations of this work.
Finally, we conclude in Section~\ref{sec:conclusion}.
 %discusses the limitations of this work and draws future research directions.
%reports conclusions and draws perspectives for future research directions.

%%%%%%% OLD %%%%%%%
%to overcome the resilience of Byzantine failures in distributed Stochastic Gradient Descent computations. 
% The strength of Krum is its time complexity, which is linear in the gradient dimension. 
% However, the robustness of the approach is guaranteed for gradient-based learning applications only when the majority of the clients are not compromised. 
% Besides, the aggregation mechanism of Krum, as well as that of similar methods, is robust from a coarse-grained perspective and does not provide solutions to errors and perturbations that may occur at inference time.
%A related approach to~\cite{blanchard2017nips} is the work of Su et al.~\cite{su2016dc}. Here, the authors propose an iterated approximate agreement to tackle a multi-layer scenario attacked by Byzantine agents. 
%However, the method works efficiently on the sole discrete context and it is inapplicable to continuous state environments.
%\gabri{Maybe, we should just talk about the main limitations of existing countermeasures without digging into their details (or, we can just mention Krum as this is the most popular one). I will move the description of all these methods to the Related Work section.}
\section{Related Work}
\label{sec:relw}


\paragraph{Image distortion.} %
Over the past decades, the research community has explored a plethora of simple hand-designed image distortions such as zoom, reflection, rotation, shear, color jittering, solarization, and blurring --- see \cite{shorten_survey_2019} and \cite{perez_effectiveness_2017} for an extensive survey. %
Although all these distortions induce the model to be robust to small perturbations of the input, they might lead to unrealistic images and provide only limited image augmentations. To design more powerful image distortions, the research community has started to combine multiple simple image distortions into a more powerful data augmentation schemes such as Neural Augmentation~\citep{perez_effectiveness_2017}, SmartAugment~\citep{lemley_smart_2017}, AutoAugment~\citep{cubuk_autoaugment_2019}, and RandAugment~\citep{cubuk_randaugment_2020-2}. Although these augmentation schemes oftentimes significantly improve model performance, the resulting distortions are limited by the initial set of simple distortions. Moreover, finding a good combination of simple image distortions is computationally intense since it requires numerous network trainings. 
\paragraph{Image mixing.} An alternative way to increase the diversity of augmented images is to consider multiple images and their labels. For example, CutMix~\citep{yun_cutmix_2019} creates collages of pairs of images while MixUp~\citep{zhang_mixup_2017} interpolates them pixel-wise. In both cases, the mixing factor is regulated by a hyper-parameter, which is also used for label interpolation. However, these augmentation techniques directly target the improvement of class boundaries, at the cost of producing
unrealistic  %
images. 
We argue %
that unrealistic augmentations are a sub-optimal heuristic currently adopted as a strong regularizer, which is no longer needed when larger datasets are available, as shown in~\cite{steiner_how_2021}.%


\paragraph{Data augmentation with autoencoders.} To improve the realism of augmented images, some researchers have explored applying the image augmentations in the latent space of an autoencoder (AE). \cite{devries_dataset_2017} and 
\cite{liu_data_2018} proposed to interpolate/extrapolate neighborhoods in latent space to generate new images. %
Alternatively, \cite{schwartz_delta-encoder_2018} introduced a novel way of training AE to synthesize images from a handful of samples and use them as augmentations to enhance few-shot learning.
Finally, \cite{pesteie_adaptive_2019} used a variational AE trained to synthesize clinical images for data augmentation purposes. However, most of above-mentioned approaches are limited by the quality of the reconstructed images which are oftentimes blurry. 




 



\paragraph{Data augmentation with generative models.} To improve the visual quality of augmented images, the community has studied the use of generative models in the context of both data augmentation and dataset augmentation. \cite{tritrong_repurposing_2021, mao_generative_2021} explored the use of instance-specific augmentations obtained via GAN inversion~\citep{xia_gan_2022, huh_transforming_2020, zhu_generative_2016}, which map original images into latent vectors that can be subsequently transformed to generate augmented images~\citep{jahanian_steerability_2020, harkonen_ganspace_2020}. %
However, GAN inversion is a computationally intense operation and latent space transformations are difficult to control~\citep{wang_implicit_2019, wang_regularizing_2021}. \cite{antoniou_data_2018} proposed a specific GAN model to generate a realistic image starting from an original image combined with a noise vector. However, this work was only validated on low-shot benchmarks. Researchers have also explored learning representations using samples from a trained generative model exclusively~\citep{shrivastava_learning_2017, zhang_datasetgan_2021-2, li_bigdatasetgan_2022, besnier_this_2020, li_semantic_2021, zhao_synthesizing_2022, jahanian_generative_2022} as well as combining real dataset images with samples from a pre-trained generative model~\citep{frid-adar_gan-based_2018, bowles_gan_2018, ravuri_classification_2019}, with the drawback of drastically shifting the training distribution.
Finally, the use of unpaired image-to-image translation methods to augment small datasets was explored in \cite{sandfort_data_2019, huang_auggan_2018, gao_low-shot_2018, choi_self-ensembling_2019}.
However, such approaches are designed to translate source images into target images and thus are limited by the source and target image distributions.
 
 










\paragraph{Data augmentation with latent neighbor images.} Another promising data augmentation technique uses neighbor images to create semantically-similar image pairs that can be exploited for multi-view representation learning typical of SSL. This technique was promoted in NNCLR~\citep{dwibedi_little_2021-1}, an extension of the SSL model SimCLR~\citep{chen_simple_2020} to use neighbors, with some limitations due to the restricted and dynamic subset used for neighbor retrieval. Alternatively, \cite{jahanian_generative_2022} explored the generation of neighbor pairs by using latent space transformations in conjunction with a pre-trained generative model. However, this model only uses generated samples to learn the representations and reports poor performance on a simplified ImageNet setup (training on $128\times128$ resolution images for only 20 epochs).
















\section{Method}
\label{sec:method}

% \ml{``Inconsistent'' to ``large variation''}

% In this section, we propose our methods based on the observations in Section \ref{sec:motivation}.
In this section, we propose two techniques to further enhance the strong baseline to capture the variation of activation distributions better.
We first introduce spatial re-scaling to adapt the network to pixel-to-pixel variation.
We then propose channel-wise shifting and re-scaling to better capture the channel-to-channel variation.
Meanwhile, as both of the two methods are image-dependent, the image-to-image variation can be captured naturally.
By combining the two methods with our strong baseline, we build our enhanced BNN for SR, named EBSR.

% Because the activation distributions among pixels, channels and images have large variations \red{**are highly inconsistent} in SR networks, we introduce spatial re-scaling to adapt to pixel-wise variations and channel shift and re-scaling to adapt to channel-wise variations. And both of them are image-dependent to adapt to image-wise variations, which means during inference our network re-scales and shifts the distributions of activations flexibly for different input images. Based on these methods, we build an enhanced binary neural network for image super-resolution (EBSR).

% According to [3], the difference of activation magnitudes indicates different scaling factors are needed for each pixel.

\subsection{Spatial Re-scaling}
% It is better to use different scaling factors for different pixels to reduce the quantization error and retain more detailed information for image super-resolution. 

% \ml{In the main method, we do not need to introduce the previous works but can focus on introducing our own method. Channel rescaling in Real-to-binary Net is not relevant in this context.}

% Re-scaling the output of binary convolutions was proposed at the birth of BNN in XNOR-Net \cite{rastegari2016xnor} to reduce quantization error and improve accuracy for image classification tasks.
% It is computed as below:
% \begin{equation}
% \mathcal{A} * \mathcal{W} \approx(\operatorname{sign}(\mathcal{A}) \circledast \operatorname{sign}(\mathcal{W})) \odot \mathcal{K} \alpha
% \label{eq:xnor-net rescale}
% \end{equation}
% where $\circledast$ denotes the binary convolution and $\odot$ denotes the element-wise multiplication.
% $\mathcal{A}$, $\mathcal{W}$, $\alpha$, and $\mathcal{K}$ denote the activation, weight, weight scaling factor, and activation scaling factor, respectively.
%  Later in XNOR-Net++ \cite{bulat2019xnor}, Bulat et al. fuse the activation and weight scaling factors into a single one that is learned end-to-end based on gradients and this improves the classification accuracy on ImageNet dataset.

% % It is computed as Eq.~\ref{eq:xnor-net rescale}, where $\circledast$ denotes 
% %  the binary convolution and $\odot$ denotes the element-wise multiplication. The binary convolution of $\mathcal{A}$ and $\mathcal{W}$ is rescaled by the weight scaling factor $\alpha$ and the activation scaling factor $\mathcal{K}$, both of which are calculated analytically.


% \zc{Similarly, you should explain the meaning of A, W and the operators $\circledast$ in the formula}
% Then in Real-to-binary Net \cite{martinez2020training}, Martinez et al. used a data-driven channel re-scaling module that takes the pre-convolution activations as input to predict the activation scaling factor. Unlike that in XNOR-Net++ \cite{bulat2019xnor}, these scaling factors are not fixed during inference but rather inferred from data. By doing this, they further improved the classification accuracy on ImageNet over XNOR-Net++. 
As is shown in Figure \ref{fig:pixel}, activation distributions have large pixel-to-pixel variation in SR networks
and the difference of activation magnitudes indicates different scaling factors are preferred for different pixels.
Inspired by \cite{martinez2020training}, we propose spatial re-scaling to better adapt the network to the spatial variation
of activation distributions in SR networks.
% fit the various pixel-wise distributions in SR networks.
We take the real-valued activations $A$ before convolution as input and predict pixel-wise scaling factors $S(A)$, which re-scale the binary convolution output. Spatial re-scaling process can be formulated as follows:
\begin{equation}
A * W \approx(\operatorname{sign}(A) \circledast \operatorname{sign}(W)) \odot \alpha \odot S(A)
\label{eq:spatial rescale}
\end{equation}
where $\circledast$ denotes 
the binary convolution and $\odot$ denotes the element-wise multiplication. $A$, $W$, $\alpha$, and $S\left(A\right)$ denote real-valued activations, weights, the scaling factor of weights, and the spatial-wise scaling factor of activations respectively. $S\left(A\right) \in \mathbb{R}^{1\times H\times W}$ can be calculated with a convolution and a sigmoid function.
% as $\sigma\left( CONV\left(A\right)\right)$. 
As shown in Figure \ref{fig:method}(a), real-valued activations first go through a convolution layer,
which has an input channel of $C$ and an output channel of 1, 
and then pass through a sigmoid function to produce the scaling factors $S(A)$ along the spatial dimension.
During inference, the scaling factor will change dynamically according to different input feature maps.
By re-scaling binary convolution output using $S(A)$, we can reduce the quantization error and the original pixel-wise information in FP activation
will be preserved much better.
Spatial re-scaling leads to a large PSNR improvement of 0.24 dB (from 30.30 dB to 31.54 dB) on Set5 and 0.22 dB (from 25.09 dB to 25.31 dB)
on Urban100 compared with our strong baseline. 

\subsection{Channel-wise Shifting and Re-scaling}

\begin{table}[!tb]
\centering
\caption{Comparison between whether to fuse channel-wise shifting and re-scaling or not based on our baseline with spatial re-scaling. }
\label{tab:fusing}

\scalebox{0.65}{
\begin{tabular}{c|cc|cc|cc}
\hline
\multirow{2}{*}{Method}     & \multirow{2}{*}{OPs} & \multirow{2}{*}{Params} & \multicolumn{2}{c|}{Set5} & \multicolumn{2}{c}{Urban100} \\ \cline{4-7} 
                            &                      &                         & PSNR        & SSIM        & PSNR          & SSIM         \\ \hline
Baseline + spatial re-scale & 2.16G                & 0.05M                   & 31.54       & 0.883       & 25.31         & 0.759        \\
+ channel-wise shift and re-scale             & 2.34G                & 0.09M                   & 31.61       & 0.885       & 25.35         & 0.761        \\
+ w/ fusing                   & 2.27G                & 0.08M                   & \textbf{31.64}       & \textbf{0.885}       & \textbf{25.36}         & \textbf{0.761}        \\ \hline
\end{tabular}
}
\end{table}

In SR networks, activation distributions exhibit larger channel-to-channel variation (Figure \ref{fig:chl}).
Both the mean and magnitude of the activation distributions vary significantly across channels.
% Thus we use channel-wise shifting and re-scaling to adapt to various channel-wise distributions. 
\cite{martinez2020training} has proposed the data-driven channel re-scaling, 
but our method differs from them in further introducing data-driven thresholds to handle the channel-wise variation of both mean and magnitude.
Since the blocks to generate the scaling factors and thresholds are very similar, we further propose to fuse them into one module.
% and fusing channel-wise shifting and re-scaling into one module.
We evaluate the effect of fusing the two blocks in Table \ref{tab:fusing}.
With channel-wise shifting and re-scaling fused, our models have fewer operations and parameters overhead and slightly higher performance.

For the specific process, we take the real-valued activations as input and predict different thresholds and scaling factors for each channel. They are also image dependent, e.g., $\beta_{i}$ in Eq.\ref{eq:act_binarize} is no longer fixed during inference but generated according to different input feature maps. Channel-wise shifting and re-scaling can be formulated as follows:
\begin{equation}
A * W \approx(\operatorname{sign}(A-C_s(A)) \circledast \operatorname{sign}(W)) \odot \alpha \odot C_r(A)
\label{eq:channel-wise_shift_and_rescale}
\end{equation}
where $\circledast$ denotes 
the binary convolution and $\odot$ denotes the element-wise multiplication. $C_s(A), C_r(A) \in \mathbb{R}^{C\times1\times1}$ denote the channel-wise threshold and scaling factor, respectively. 
We show the block diagram in Figure \ref{fig:method}(b).
The real-valued input feature map is first squeezed to a ${C\times1\times1}$ vector by a global average pooling (GAP) layer.
The subsequent fully connected layers and ReLU learn the channel-wise information and output a ${2C\times1\times1}$ vector.
Then the ${2C\times1\times1}$ vector is split into two ${C\times1\times1}$ vectors.
We use the first $C$ channels as the channel-wise bias and pass the last $C$ channels through a sigmoid layer 
as the channel-wise scaling factor, which are used to shift the real-valued activations and re-scale the binary convolution output, respectively. 


% \ml{We can mention previously, channel-wise re-scale has been proposed. We propose to fuse them. Add the comparison between fuse v.s. no fuse.}

\begin{figure}[!tbp]%
  \centering
    \includegraphics[width=0.4\textwidth]{fig/methods.png}
  
% \subfloat[channel-wise shifting\&re-scale]{
%     \label{subfig:channel-wise shifting and re-scale}
%     \includegraphics[width=0.2\textwidth]{fig/chl shift and rescale.png}
%   }

  \caption{Block diagram for spatial re-scaling, and channel-wise shifting and re-scaling.} 
  % Input A is the real-valued activation tensor and C, H, and W denote its dimension. GAP stands for global average pooling. The reduction ratio r is set to 16 for a better trade-off between the performance and the number of operations and parameters.}
  \label{fig:method}
\end{figure}


\subsection{Network Structure}

Combining the spatial re-scaling and the channel-wise shifting and re-scaling methods, we construct the enhanced convolution layer (E-Conv).
Then we build our EBSR model based on E-Conv.
In Figure \ref{fig:E-conv}, we compare the binary convolution layer used in the baseline network and our proposed E-Conv.
We use spatial and channel-wise scaling factors to re-scale the binary convolution output,
and use channel-wise shifting to learn appropriate thresholds for each channel before binarization.
The scaling factors and threshold used in E-Conv are learnable and depend on the real-valued input activations.
In this way, our proposed EBSR can adapt to pixel-to-pixel, channel-to-channel, and image-to-image variations
to reduce the large binarization error and preserve more details.
% In this way, our proposed E-Conv reduces the large quantization error caused by binarization and keeps the original information of input feature maps to a large extent.


\begin{figure}[!tb]%
  \centering

    \includegraphics[width=0.5\textwidth]{fig/E-conv.png}

  \caption{Comparison of (a) the binary convolution layer with a skip connection used in our baseline network and (b) the proposed E-Conv.}
  \label{fig:E-conv}
\end{figure}


Figure \ref{fig:network} shows the basic block based on the E-Conv and our EBSR composed of the basic blocks. Following existing works, the convolution layers in the head and tail modules are not binarized. We choose the lightweight EDSR which has 16 basic blocks and 64 channels, and EDSR which has 32 basic blocks and 256 channels as our backbones, which correspond to EBSR-light and EBSR, respectively.

\begin{figure}[!tb]%
  \centering
  {
    \includegraphics[width=0.35\textwidth]{fig/network.png}
  }
  
  \caption{The structure of our proposed EBSR.  Convolution layers in purple are real-valued vanilla 3x3 convolutions.}
  \label{fig:network}
\end{figure}
\section{Experimental Setup}
\label{sec:exps}

In our empirical analysis, we investigate the effectiveness of \ours in supervised and self-supervised representation learning. In the following subsections, we describe the experimental details of both scenarios.  

\subsection{Models, metrics, and datasets}

\paragraph{Models.} For supervised learning, we train ResNets~\citep{he_deep_2016} with different depths: 50, 101 and 152 layers, and widths: ResNet-50 twice wider (ResNet-50W2)~\citep{zagoruyko_wide_2016-1}, and DeiT-B~\citep{touvron_training_2021}. For self-supervised learning, we train the SwAV~\citep{caron_unsupervised_2020} model with a ResNet-50 backbone. For \ours, we employ two pre-trained generative models on ImageNet: \icgan and \ccicgan, both using the BigGAN~\citep{brock_large_2019} backbone architecture. \icgan conditions the generation process on instance feature representations, obtained with a pre-trained SwAV model\footnote{\url{https://dl.fbaipublicfiles.com/deepcluster/swav_800ep_pretrain.pth.tar}}, while \ccicgan conditions the generation process on both the instance representation obtained with a ResNet-50 trained for classification\footnote{\url{https://download.pytorch.org/models/resnet50-19c8e357.pth}} and a class label. %
Unless specified otherwise, our models use the default \icgan and \ccicgan configuration from \cite{casanova_instance-conditioned_2021}: neighborhood size of $k$=50 and $256\times256$ image resolution, trained using only horizontal flips as data augmentation\footnote{\url{https://github.com/facebookresearch/ic_gan}}. To guarantee a better quality of generations we set truncation $\sigma = 0.8$ and 1.0 for \icgan and \ccicgan respectively.
For simplicity, we will use the term \allicgan to refer to both pre-trained models hereinafter.

\paragraph{Datasets.} We train all the considered models from scratch on ImageNet \underline{(IN)}~\citep{deng09cvpr} and test them on the IN validation set. Additionally, in the supervised learning case, models are tested for robustness on a plethora of datasets, including \underline{Fake-IN}: containing 50K generated images obtained by conditioning the \icgan model on the IN validation set; \underline{Fake-IN\textsubscript{CC}}: containing 50K images generated with the \ccicgan conditioned on the IN validation set\footnote{To avoid as much as possible unrealistic generations in creating Fake-IN and Fake-IN\textsubscript{CC}, for each IN image we generate a set of 20 samples, from which we chose the one most similar (cosine similarity) to the conditioning image in the feature space.}; IN-Adversarial \underline{(IN-A)}~\citep{hendrycks_natural_2021}: composed of ResNet's adversarial examples present in IN\footnote{Although IN-A contains samples from only 200 out of the 1000 classes of IN, we compute the results without restricting the predictions to those 200 classes.}; IN-Rendition \underline{(IN-R)}~\citep{hendrycks_many_2021}: containing stylized images such as cartoons and paintings belonging to IN classes; \underline{IN-ReaL}~\citep{beyer_are_2020}: a relabeled version of the IN validation with multiple labels per image; and \underline{ObjectNet}~\citep{barbu_objectnet_2019}: containing object-centric images specifically selected to increase variance in viewpoint, background and rotation w.r.t. IN\footnote{The class mapping from ObjectNet to IN is one-to-multi -- i.e., one class is mapped to one or more classes of IN. We consider predictions pointing to any of the mapped classes as correct.}. We also consider the following datasets to study invariances in the learned representations: \underline{IN validation set} to analyze {\em instance+viewpoint} invariances; Pascal-3D+ \underline{(P3D)}~\citep{xiang_beyond_2014}, composed of $\sim$36K images from 12 categories to measure {\em instance}, and {\em instance+viewpoint} invariances; \underline{GOT}~\citep{huang_got-10k_2019}, 10K video clips with a single moving object to measure invariance to object {\em occlusion}; and \underline{ALOI}~\citep{geusebroek_amsterdam_2005}, single-object images from 1K object categories with plain dark background to measure invariance w.r.t.\ {\em viewpoint} (72 viewpoints per object), {\em illumination color} (12 color variations per object), and {\em illumination direction} (24 directions per object).

\paragraph{Metrics.} We quantify performance for classification tasks as the top-1 accuracy on a given dataset. Moreover, we analyze invariances of the learned representations by using the top-25 Representation Invariance Score (RIS) proposed by \cite{purushwalkam_demystifying_2020-1}. In particular, given a class $y$, we sample a set of object images $\mathcal{T}$ by applying a transformation $\tau$ with different parameters $\lambda_\tau$ such that $\mathcal{T} = \{\tau(x, \lambda_\tau) | \forall \lambda_\tau\}$. We then compute the mean invariance on the transformation $\tau$ of all the objects belonging to $y$ as the average firing rate of the (top-25) most frequently activating neurons/features in the learned representations. We follow the recipe suggested in \cite{purushwalkam_demystifying_2020-1} and compute the top-25 RIS only for ResNets models, extracting the learned representations from the last ResNet block ($2048$-$d$ vectors).

\paragraph{Per-class metrics.} We further stratify the results by providing class-wise accuracies and correlating them with the quality of the generated images obtained with \allicgan. We quantify the quality and diversity of generations using the Fréchet Inception Distance (FID)~\citep{heusel17nips}. We compute per-class FID by using the training samples of each class both as the reference and as the conditioning to generate the same number of synthetic images. We also measure a particular characteristic of the \icgan model: the \textit{NN corruption}, which measures the percentage of images in each datapoint's neighborhood that has a different class than the datapoint itself; this metric is averaged for all datapoints in a given class to obtain per-class NN corruption.


\subsection{Training recipes}
In this subsection, we define the training recipe for each model in both supervised and self-supervised learning; we describe which data augmentation techniques are used, how \ours is integrated and the hyper-parameters used to train the models.

\paragraph{Model selection.} In all settings, hyper-parameter search was performed with a restricted grid-search for the learning rate, weight decay, number of epochs, and \ours probability $p_G$, selecting the model with the best accuracy on IN validation.

\subsubsection{Supervised learning} 

For the ResNet models, we follow the standard procedure in Torchvision\footnote{See the \texttt{IMAGENET1K\_V1} recipe at \url{https://github.com/pytorch/vision/tree/main/references/classification}.} and apply random horizontal flips \underline{(Hflip)} with 50\% probability as well as random resized crops \underline{(RRCrop)}~\citep{krizhevsky_imagenet_2012}. We train ResNet models for 105 epochs, following the setup in VISSL~\citep{goyal2021vissl}. For DeiT-B we follow the experimental setup from~\cite{touvron_training_2021}, whose data augmentation recipe is composed of Hflip, RRCrop, RandAugment~\citep{cubuk_randaugment_2020-2}, as well as color jittering (CJ) and random erasing (RE)~\citep{zhong_random_2020} ---we refer to RandAugment + CJ + RE as \underline{RAug}---, and typical combinations of CutMix~\citep{yun_cutmix_2019} and MixUp~\citep{zhang_mixup_2017}, namely \underline{CutMixUp}. 
DeiT models are trained for the standard 300 epochs~\citep{touvron_training_2021} except when using only Hflip or RRCrop as data augmentation, where we reduce the training time to 100 epochs to mitigate overfitting. Both ResNets and DeiT-B are trained with default hyper-parameters; despite performing a small grid search, better hyper-parameters were not found. Additional details can be found in Appendix~\ref{suppl:details}.

\paragraph{Soft labels.} We note that the classes in the neighborhoods used to train the
\allicgan models are not homogeneous: a neighborhood, computed via cosine similarity between embedded images in a feature space, might contain images depicting different classes. Therefore, \allicgan samples are likely to follow the class distribution in the conditioning instance neighborhood, generating images that may mismatch with the class label from the conditioning image. To account for this mismatch when using \allicgan samples for training, we employ \textit{soft labels}, which are soft class membership distributions corresponding to each instance-specific neighborhood class distribution. More formally, considering the $i$-th datapoint, its $k$-size neighborhood in the feature space, $\mathcal{A}_i$, and its class label $y_i \in C$ one-hot encoded with the vector $\y_i$, we compute its soft label as:
\begin{equation}
    \y^\mathrm{soft}_i = \frac{1}{k} \sum_{j \in \mathcal{A}_i}{\y_j}, \quad \textrm{with} \quad \y_j \in \{0,1\}^C  \quad \textrm{and} \quad \sum_{c} {\y_{j,c} = 1}.
\end{equation}

\subsubsection{Self-supervised learning} 

We devise a straightforward use of \ours for \textit{multi-view} SSL approaches. Although we chose SwAV~\citep{caron_unsupervised_2020} to perform our experiments, \ours could also be applied to other state-of-the-art methods for \textit{multi-view} SSL like MoCov3~\citep{chen_empirical_2021}, SimCLRv2~\citep{chen_big_2020}, DINO~\citep{caron_emerging_2021} or BYOL~\citep{grill_bootstrap_2020-1}. In this family of approaches, two or more views of the same instance are needed in order to learn meaningful representations. 
These methods construct multi-view positive pairs $(\x_i', \x_i'')^+$ from an image $\x_i$ by applying two independently sampled transformations to obtain $\x'_i=T(\x_i)$ and similarly for $\x_i''$ (see orange branch in Figure~\ref{fig:method}).
To integrate \ours in such pipelines as  an alternative form of  data augmentation, we replace $\x_i'$ with a generated image $\tilde \x_i'$ with probability $p_G$. To this end, we sample an image $\tilde \x_i$ from \icgan conditioned on $\x_i$, and apply further hand-crafted data augmentations $\tilde T$ to obtain $\tilde \x_i' = \tilde T (\tilde\x_i)$.

\paragraph{SwAV pre-training and evaluation.} We follow the SwAV pre-training recipe proposed in \cite{caron_unsupervised_2020}. This recipe comprises the use of random horizontal flipping, random crops, color distortion, and Gaussian blurring for the creation of each image view. In particular, we investigate two augmentation recipes, differing in the use of the {\em multi-crop} augmentation~\citep{caron_unsupervised_2020} or the absence thereof. 
The multi-crop technique augments positive pairs $(\x_i', \x_i'')^+$ with multiple other views obtained from smaller crops of the original image: $(\x_i', \x_i'', \x_i^{\mathrm{small}'''}, \x_i^{\mathrm{small}''''}, ...)^+$. In all experiments, we pre-train SwAV for 200 epochs using the hyper-parameter settings of \cite{caron_unsupervised_2020}. 
To evaluate the learned representation we freeze the ResNet-50 SwAV-backbone and substitute the SSL SwAV head with a linear classification head, which we train supervised on IN validation set for 28 epochs with Momentum SGD and step learning rate scheduler --following the VISSL setup~\citep{goyal2021vissl}.          

\paragraph{Neighborhood augmented SwAV.} To further evaluate the impact of \ours in SSL, we devise an additional baseline, denoted as SwAV-NN, that uses real image neighbors as augmented samples instead of \icgan generations: $(\x_j', \x_i'')^+,\, \x_j \in \mathcal{A}_i$. SwAV-NN is inspired by NNCLR~\citep{dwibedi_little_2021-1}, with the main difference that neighbor images are computed off-line on the whole dataset rather than online using a subset (queue) of the dataset. The nearest neighbors are computed using cosine similarity in the same representation space used for \icgan training. With a probability $p_G$, each image in a batch is paired with a uniformly sampled neighbor in each corresponding neighborhood.

\section{Results}
\label{results}

\begin{figure*}[ht]
    \centering
    \includegraphics[scale=0.15,trim={0 2.5cm 0 5cm},clip]{images/aoi-single_burst}
    \caption{The time average peak Age of Information with burst and \gls{soa} loss values against the dynamic reliability logic for different network topologies.}
    \label{fig:aoi_burst}\vspace{-0.4cm}
\end{figure*}


This paper focuses on both transport layer and application layer metrics to determine the feasibility of dynamic reliability. For this, we have selected the session packet volume, as transmitted, retransmitted, lost and backlogged packets as \glspl{kpi} for the transport layer; while focusing on the \gls{aoi} for the application layer. The \gls{aoi} was chosen as a crucial indicator for the freshness of packets in real-time applications. More specifically, this work adopts the time average peak \gls{aoi} equation \cite{aoi_equation} depicted in Eq. \ref{aoi}, where $\Delta(r_{i+1})$ is the $i$th update at the time it was received at the server, for a session time period of $\tau$.

\begin{equation}
    \label{aoi}
    \gls{aoi}_\tau = \frac{1}{n-1}\sum_{i=1}^{n-1} \Delta(r_{i+1})
\end{equation}

We include a comparison between the vanilla QUIC implementation which does not enjoy the dynamic reliability extension, with a number of dynamic reliability policies. The tests were run a number of times for statistical significance, with the mean value of vanilla implementation used as a baseline for comparison. The topology utilised both random loss and bursty loss to explore the bounds of dynamic reliability. The \gls{soa} loss in the figures correspond to the loss values presented in Table. \ref{tab:path_char}, for ease of comparison between bursty and random loss scenarios.

\subsection{Transport-Layer KPIs}

To analyse the performance gain at the transport layer due to dynamic reliability, the volume of transmitted and backlogged packets is examined. The figures are in the form of boxplots, which take the vanilla implementation as a benchmark, depicted as the red dashed line.

As seen in Fig. \ref{fig:sent_burst}, the loss plays a crucial role in the performance of the reliability policies. The policies under random loss did incredibly well for the networks with a larger capacity, namely \gls{mmwave} and Sub-6~GHz, whereas for burst loss, the lower network capacities had a larger packet reduction. With the increase in burst loss, the behaviour of the set split reliable policies became unpredictable, if a reliable assignment happened to coincide with a burst loss, the number of transmitted packets increases, and vice versa. On the other hand, in smarter policies, such as Loss-Aware, the performance lightly matched the vanilla baseline, as the reliable assignment dominated the session to compensate for a higher burst loss. Not only that but, the burst loss also impacted the variance of the transmitted packets for the policies.

Unsurprisingly, the unreliable focused policy, 80-20 split, outperformed other policies for all topologies in random and bursty loss scenarios, with an approximate reduction of 80\%. That being said, the majority of the policies reduced the transmitted packets on the link by approximately 70\% for random loss, while the reduction started at $\approx 15\%$ and decreased as the loss increased for the burst loss scenario.

The retransmitted and lost packets, not shown due to space limitations, followed the same trend as the transmitted packets for the random loss scenarios. However, for the burst loss scenarios, the larger capacity networks had a lower reduction in the retransmitted and lost packets. This can be seen as a favorable outcome since the lower capacity networks are scarce on resources. It is important to note that the Loss-Aware policy mimicked the vanilla approach as the burst loss increased, signifying the overwhelming appointment of reliable packets in adapting to the harsh burst loss conditions.
 
Alternatively, Fig. \ref{fig:backlog_burst} clearly shows a stark comparison between the policies and loss scenario in the reduction of the backlogged packets. The Loss-Aware policy for random loss scenario reduced the backlogged packets by up to 50\%, beating all other policies by approximately 30\%. Furthermore, it is clear that the unreliability focused policies resulted in the lowest backlog for the session. In comparison, we notice that the burst loss and the backlogged frequency have a positive correlation, where the maximum reduction of the backlogged packets for the policies is at most 20\%. Much like the transmitted packets, the probability of a burst loss occurrence plays a vital role in the number of retransmissions sent and by extension the number of backlogged packets. Thus, we can conclude that the stress placed on the buffer is a result of the reliable packets which is tightly coupled with the congestion on the session. Whereas, unreliable focused policies did not encounter such a phenomenon regardless if it was experiencing a burst loss.


\subsection{Application-Layer KPIs}

The feasibility of dynamic reliability for real-time applications can be determined by the \gls{aoi}, with comparison across different topologies and policies. If we take a strict approach and consider anything below $10$~ms is real-time \cite{real-time}, then all the reliability policies passed that requirement, which is attractive for real-time applications, as shown in Fig. \ref{fig:aoi_burst}. Utilising the median as an estimate of the runs, the policies in the WLAN and Sub-6~GHz topology with random loss floated around $4-5$~ms with negligible difference, while the \gls{aoi} for \gls{mmwave} was $\approx 2-3$~ms. It is clear that the \gls{aoi} and the network capacity have a negative correlation, as the network capacity decreases, the \gls{aoi} increases. The same correlation is extended to the bursty loss scenarios, where \gls{mmwave} dominated the other topologies. That being said, it is crucial to note that the \gls{aoi} for the reliability policies is often slightly better than or equal to the \gls{aoi} of the vanilla implementation, proving that dynamic reliability reduces the congestion of the session at no cost to the \gls{aoi}.

\section{Conclusions}
We consider the phase-extraction problem, and we showed that, given a unitary $U = e^{i\pi H}$ and its inverse $U^{\dag}$, we could implement a block-encoding of $\phi(H)$ for some smooth function $\phi(x)$. The word `smooth' here means existence and continuity of the derivatives: the higher the number of continuous derivatives that a function has, the faster its Fourier sum (and thus the Laurent polynomial on the eigenphases) uniformly converges to that function. We are confident this can have many more applications beyond what is shown in this work. It is also worth remarking that Jackson showed that the convergence rate of a Fourier series is almost-optimal, in the sense that no trigonometric (or, equivalently, complex exponential) series can approximate the desired function faster, up to that $\log d$ factor~\cite[p.\ 21]{jacksonTheoryApproximation1930a}. Also remember that `smoothing' a function, i.e., replacing its derivative with a continuous function, does not give faster convergence for free in general, as its derivative will become steep in the points where we smooth out discontinuities, and this translates to a high Lipschitz constant: a~clear example is given by Eq.~\ref{eq:lipschitz-constant-recurrence-solution}, but in that case, fortunately, nothing depends on the size of the input $N$, and thus does not influence the asymptotic query complexity of Algorithm~\ref{alg:prop-sampling-qsp}, although the constant factor can become large even for $p = 20$. From a theoretical point of view, this work shows that, for any $\eta > 0$, there is an algorithm with query complexity 
$$\Tilde{\bigO}\left(\frac{1}{\bar{c}^{\frac{1}{2} + \eta}} \frac{1}{\epsilon^\eta} \right)$$
solving the proportional-sampling problem. This statement seems to suggest there exists an algorithm which directly solves the problem with $\eta = 0$, and an open question would be to find such algorithm.


It is also interesting to remark that Theorems~\ref{thm:haah-construction},~\ref{thm:haah-completion} indeed allow the construction for any $\phi$, even complex-valued, provided that its absolute value is reciprocal.

One could think that, in Section~\ref{sec:prop-sampling}, instead of using the linear function in the phase-extraction subroutine, we could approximate the square root and then apply the transformation directly on $e^{i \pi c(x)}$. However, in the case of proportional sampling this would be inconvenient, as the derivative of the square root function has a discontinuity with an infinite jump around 0, and we could not choose a constant $\delta$ if we had values of the oracle that are too close to $0$.

\bibliography{tmlr}
\bibliographystyle{tmlr}

\appendix
\section{Implementation and Practical Tips}
\label{suppl:details}

\subsection{Data augmentation and optimization hyper-parameters}
\paragraph{Handcrafted data augmentations.} We used the VISSL library~\citep{goyal2021vissl} which relies on Torchvision transformations\footnote{\url{https://pytorch.org/vision/stable/transforms.html}}, and considered the following data augmentation strategies:
\begin{itemize}
    \item \underline{Hflip}: Random horizontal flipping applied with 50\% probability.
    \item \underline{RRCrop}: Random resized cropping. The location of the crop is sampled uniformly based on the sampled crop size, which is in the range (0.08, 1). When adopted in our experiments, we apply it to all images in a batch.
    \item \underline{RAug} = RandAugment + Color Jittering (CJ) + Random erasing (RE): When enabled,  RandAugment~\citep{cubuk_randaugment_2020-2} is applied to all images with magnitude $9.0\pm0.5$ and increasing distortion severity for higher magnitude values. At each iteration RandAugment randomly chooses two types of distortions. CJ distorts brightness, contrast, saturation, and hue, each with probability 0.4. RE is applied with probability 0.25, erasing a rectangle of size sampled from (0.02, 0.33).
    \item \underline{CutMixUp}. CutMix~\citep{yun_cutmix_2019} and MixUp~\citep{zhang_mixup_2017} are never applied simultaneously; there is a 0.5 probability of choosing one or the other at each iteration. Note that Mixup is applied 80\% of the time when selected. As previously mentioned, we adopt label smoothing of 0.1 to ease convergence when using CutMixUp.  
\end{itemize}

\paragraph{Optimization hyper-parameters.} In Table~\ref{suppl:tab:in1k_training}, we list the optimization hyper-parameters explored for supervised training on ImageNet. Note that for all supervised experiments, we optimized the multi-class cross-entropy loss.

\begin{table}[ht]
    \caption{Training hyper-parameters for supervised training on ImageNet.}
    \label{suppl:tab:in1k_training}
    \resizebox{\linewidth}{!}{%
    \begin{tabular}{@{}lcccccc@{}}
    \toprule
    Model & Optimizer & Epochs & Learning rate (LR) & LR scheduler & LR scaling & Weight decay \\%& Hflip & RRCrop & RAug & Col. Jit. & RErase & CutMixUp \\
    \midrule
    ResNet-50 & Mom. SGD%
    & 105 & \{(5, 2, 1)e-1, (5, 1)e-2, 5e-3\} & Step(30,60,90,100) & Lin-256 & \{(1, 5)e-5, (1, 5)e-4, 1e-3\} \\%& .5 & (0.08, 1.0) & / & / & / & / \\ 
     ResNet(-101, -152, 50W2) & Mom. SGD%
    & 105 & 1e-1 & Step(30,60,90,100) & Lin-256 & 1e-4 \\%& .5 & (0.08, 1.0) & / & / & / & / \\ 
    DeiT-B & AdamW%
   & 100/300 & 1/5e-4 & Lin + Cosine & Lin-512 & 1e-1/5e-2 \\%& .5 & (0.08, 1.0) & magn. (9, .5) & .4 & .25, (0.02, 0.33) & .5, CM(.8) MU(1.) \\
    \bottomrule
    \end{tabular}
    }
\end{table}

\subsection{Implementation details}

\paragraph{Fixing the number of IC-GAN-augmented datapoints.} Variable batch size can cause unexpected breaks in GPU-accelerated computations, mostly due to GPU memory pre-allocation. To avoid this phenomenon, we fix the number of \icgan-augmented images in a batch to be \texttt{ceil(batch\_size * p\_G)}.

\paragraph{Computational overhead of \ours.} Adding \ours to the training recipe requires some additional space and time for the \icgan generation. For instance, in terms of space, $\sim$11GB of a single GPU memory are required to generate a batch of 64 images at $256\times 256$ resolution. In terms of time, we noticed that training ResNets with \ours and $p_G=1.0$ doubles the training time, whereas for $p_G=0.5$ the time requirement increases by roughly 50\%. However, we did not take advantage of half-precision computations nor of any other inference-only trick like \texttt{jit} scripting in PyTorch. We hypothesize that exploring such optimizations might significantly reduce the computational overhead of \ours.

\paragraph{Pre-computing dataset embeddings.} The \icgan generation step requires feature representation of the conditioning images. In order to reduce the computation needed during the training, we compute the embeddings of the entire training dataset in advance, and store them in an array which is loaded into memory at the beginning of the training. 

\paragraph{Hardware used for experiments.} For most of the experiments, we performed distributed training using cluster nodes with 8 Nvidia V100 GPUs with 32GB memory. We changed the number of nodes based on the training model and the desired batch size -- e.g., 1 node for ResNets, 4 for DeiT-B, and 8 for SwAV.













\section{Additional Results}
\label{suppl:results}


\subsection{DeiT-B per-class analysis}

Figure~\ref{suppl:fig:per_class_deit} assesses the impact of \allicgan's generation quality on the per class performance of the DeiT-B model. The exclusive use of generated samples to train DeiT-B leads to rather a low top-1 accuracy of $\sim$48\% and $\sim$51\%, when using \icgan and \ccicgan respectively.\looseness-1

Following section~\ref{sssec:class}, Figures~\ref{suppl:fig:per_class_deit} (a--b) show the per-class FID of \allicgan as a function of per-class top-1 accuracy of the vanilla baseline and the \ours models. We observe similar trends as for the ResNet-152 models, -- i.e., \ours tends to exhibit higher accuracy for classes with lower (better) FID values, and lower accuracy for classes with higher FID values, suggesting that using image generations of poorly modeled classes hurts the performance of DeiT-B. Figures~\ref{suppl:fig:per_class_deit} (c--d) highlight that the low accuracies of the model trained with generated data can be partially explained by the NN corruption.



\begin{figure}[ht]
\centering
\begin{subfigure}[b]{.43\textwidth}
    \centering
    \includegraphics[width=\textwidth]{figures/corr_icg_baseline_fid.pdf}
    \caption{FID -- DeiT-B, \icgan}
\end{subfigure}
\begin{subfigure}[b]{.43\textwidth}
    \centering
    \includegraphics[width=\textwidth]{figures/corr_ccicg_baseline_fid.pdf}
    \caption{FID -- DeiT-B, \ccicgan}
\end{subfigure}
\begin{subfigure}[b]{.43\textwidth}
    \centering
    \includegraphics[width=\textwidth]{figures/corr_icg_baseline_purity.pdf}
    \caption{NN corruption -- DeiT-B, \icgan}
\end{subfigure}
\begin{subfigure}[b]{.43\textwidth}
    \centering
    \includegraphics[width=\textwidth]{figures/corr_ccicg_baseline_purity.pdf}
    \caption{NN corruption -- DeiT-B, \ccicgan}
\end{subfigure}
\caption{
Impact of \allicgan's generation quality on per-class performance. (a-b) Per-class FID as a function of per-class top-1 accuracy of the vanilla and \ours models. We observe that higher quality \allicgan generations tend to result in improved performances. (c-d) Per-class NN corruption as a function of per-class top-1 accuracy of the vanilla and \ours models. We observe that less corrupted classes tend to result in improved performances. ImageNet validation results are shown for the DeiT-B model trained with horizontal flips, random crops, and RandAugment. We limited the FID colormap interval to 250 to aid interpretability, while we observed FID values up to 500 for certain classes.}
\label{suppl:fig:per_class_deit}
\end{figure}



\subsection{Avoiding \ours on low quality classes}
In this analysis, we try to exploit the observed correlation between per-class accuracy and per-class FID, i.e., samples quality (see Section~\ref{sssec:class}), by restricting the use of \ours only to classes that have an acceptable quality. We set an FID threshold of 150, under which we consider a class to have acceptable quality as the visual inspection of classes with FID >= 150 reveals either very poor image quality or mode-collapse (as shown in Figure~\ref{fig:mode_collapse}); this threshold value is distant around 1.5$\sigma$ and 3$\sigma$ from the average per-class FID computed on \icgan and \ccicgan samples, respectively. For this experiment, we train ResNet-152 with an augmentation recipe composed by HFlip and RRCrop applied to all classes, and \ours applied to FID-filtered classes. We report the results in Table~\ref{suppl:tab:fid_filtered_rn152}.

\begin{table}[ht]
\centering
\caption{ImageNet classification accuracy of ResNet-152 when using \ours indistinctly on all classes vs. augmenting only classes with FID < 150. For each column of results we report the mean top-1 accuracy computed over the indicated set of classes.}
\label{suppl:tab:fid_filtered_rn152}
\resizebox{.9\textwidth}{!}{%
\begin{tabular}{@{}lllccc@{}}
\toprule
\multirow{2}{*}{Method} & \multirow{2}{*}{DA base} & \multirow{2}{*}{\ours} & \multicolumn{3}{c}{Top-1 accuracy} \\
 & & & all classes & classes w/ FID < 150 & classes w/ FID >= 150 \\ \midrule
\multirow{2}{*}{ResNet-152} & \multirow{2}{*}{HFlip + RRCrop} & w/ \icgan & 77.71 & 77.44 & 80.44 \\
 & & w/ FID-filtered \icgan & \textbf{77.94} & \textbf{77.56} & \textbf{81.67} \\ \midrule
\multirow{2}{*}{ResNet-152} & \multirow{2}{*}{HFlip + RRCrop} & w/ \ccicgan & 77.96 & 77.92 & 80.29 \\
 & & w/ FID-filtered \ccicgan & 77.96 & 77.92 & \textbf{80.71} \\ \bottomrule
\end{tabular}%
}
\end{table}

Overall, we obtain, on average, a slightly better top-1 accuracy, +0.2\%p, which can be stratified into +1.2\%p considering the classes with FID >= 150 and +0.1\%p on the remaining classes. From these results we can observe that skipping the use of \ours on poorly modeled classes increases the performances on such classes, while not harming the performance on the others.






\section{Additional Visualizations}
\label{suppl:visual}

Figure~\ref{suppl:fig:seqC_mc_batch} displays images resulting from the combination of \ours with the multi-crop augmentation used in the SwAV model~\cite{caron_unsupervised_2020}. As shown in the figure, these augmentations result in significant variations of the original image, with small crop images notably differing from the \icgan generations.\looseness-1  

Figure~\ref{suppl:fig:icgan_ex} displays \allicgan generations. Note that \icgan and \ccicgan generations (a--b) tend to show slightly different viewpoints and instances of the object present in the conditioning image (left-most column).

\begin{figure}[ht]
\centering
\begin{subfigure}[b]{.2\textwidth}
    \centering
    \includegraphics[width=.9\textwidth]{figures/ssl_multicrop/ori}
    \caption{Original}
    \vspace{.13cm}
    \includegraphics[width=.9\textwidth]{figures/ssl_multicrop/icgan_gen}
    \caption{\icgan}
\end{subfigure}
\hfill
\begin{subfigure}[b]{.18\textwidth}
    \centering
    \includegraphics[width=\textwidth]{figures/ssl_multicrop/ori_big_crop}\\
    \includegraphics[width=\textwidth]{figures/ssl_multicrop/icgan_gen_big_crop}
    \caption{Main crops $224^2$}
\end{subfigure}
\hfill
\begin{subfigure}[b]{.55\textwidth}
    \centering
    \includegraphics[width=.3\textwidth]{figures/ssl_multicrop/ori_small_crop_1}
    \includegraphics[width=.3\textwidth]{figures/ssl_multicrop/ori_small_crop_2}
    \includegraphics[width=.3\textwidth]{figures/ssl_multicrop/ori_small_crop_3}\\
    \includegraphics[width=.3\textwidth]{figures/ssl_multicrop/ori_small_crop_4}
    \includegraphics[width=.3\textwidth]{figures/ssl_multicrop/ori_small_crop_5}
    \includegraphics[width=.3\textwidth]{figures/ssl_multicrop/ori_small_crop_6}
    \caption{Small crops $96^2$}
\end{subfigure}
    \caption{Example of \ours combined with multi-crop~\citep{caron_unsupervised_2020} augmentation with 2 main crops and 6 small crops: (a) depicts the original image, which is used to condition the \icgan generation process; (b) displays an \icgan generation; (c) shows the main crops of both images; and (d) presents the small crops obtained from the original image.}
    \label{suppl:fig:seqC_mc_batch}
\end{figure}


\begin{figure}[ht]
\centering
\begin{subfigure}[b]{\textwidth}
     \raggedright
     \small{``warthog''}
 \end{subfigure}
 \begin{subfigure}[b]{.1\textwidth}
    \centering
    \includegraphics[width=\textwidth]{figures/icgan_examples/img_35_class_warthog_crop}
 \end{subfigure}
\hfill
\begin{subfigure}[b]{.44\textwidth}
    \centering
    \includegraphics[width=.22\textwidth]{figures/icgan_examples/img_36_class_warthog}
    \includegraphics[width=.22\textwidth]{figures/icgan_examples/img_37_class_warthog}
    \includegraphics[width=.22\textwidth]{figures/icgan_examples/img_38_class_warthog}
    \includegraphics[width=.22\textwidth]{figures/icgan_examples/img_39_class_warthog}
\end{subfigure}
\hfill
\begin{subfigure}[b]{.44\textwidth}
    \centering
    \includegraphics[width=.22\textwidth]{figures/ccicgan_examples/img_36_class_warthog}
    \includegraphics[width=.22\textwidth]{figures/ccicgan_examples/img_37_class_warthog}
    \includegraphics[width=.22\textwidth]{figures/ccicgan_examples/img_38_class_warthog}
    \includegraphics[width=.22\textwidth]{figures/ccicgan_examples/img_39_class_warthog}
\end{subfigure}
\\
\begin{subfigure}[b]{\textwidth}
     \raggedright
     \small{``black swan''}
 \end{subfigure}
 \begin{subfigure}[b]{.1\textwidth}
    \centering
    \includegraphics[width=\textwidth]{figures/icgan_examples/img_5_class_black_swan,_Cygnus_atratus_crop}
 \end{subfigure}
\hfill
\begin{subfigure}[b]{.44\textwidth}
    \centering
    \includegraphics[width=.22\textwidth]{figures/icgan_examples/img_6_class_black_swan,_Cygnus_atratus}
    \includegraphics[width=.22\textwidth]{figures/icgan_examples/img_7_class_black_swan,_Cygnus_atratus}
    \includegraphics[width=.22\textwidth]{figures/icgan_examples/img_8_class_black_swan,_Cygnus_atratus}
    \includegraphics[width=.22\textwidth]{figures/icgan_examples/img_9_class_black_swan,_Cygnus_atratus}
\end{subfigure}
\hfill
\begin{subfigure}[b]{.44\textwidth}
    \centering
    \includegraphics[width=.22\textwidth]{figures/ccicgan_examples/img_6_class_black_swan,_Cygnus_atratus}
    \includegraphics[width=.22\textwidth]{figures/ccicgan_examples/img_7_class_black_swan,_Cygnus_atratus}
    \includegraphics[width=.22\textwidth]{figures/ccicgan_examples/img_8_class_black_swan,_Cygnus_atratus}
    \includegraphics[width=.22\textwidth]{figures/ccicgan_examples/img_9_class_black_swan,_Cygnus_atratus}
\end{subfigure}
\\
\begin{subfigure}[b]{\textwidth}
     \raggedright
     \small{``siberian husky''}
 \end{subfigure}
 \begin{subfigure}[b]{.1\textwidth}
    \centering
    \includegraphics[width=\textwidth]{figures/icgan_examples/img_25_class_Siberian_husky}
 \end{subfigure}
\hfill
\begin{subfigure}[b]{.44\textwidth}
    \centering
    \includegraphics[width=.22\textwidth]{figures/icgan_examples/img_26_class_Siberian_husky}
    \includegraphics[width=.22\textwidth]{figures/icgan_examples/img_27_class_Siberian_husky}
    \includegraphics[width=.22\textwidth]{figures/icgan_examples/img_28_class_Siberian_husky}
    \includegraphics[width=.22\textwidth]{figures/icgan_examples/img_29_class_Siberian_husky}
\end{subfigure}
\hfill
\begin{subfigure}[b]{.44\textwidth}
    \centering
    \includegraphics[width=.22\textwidth]{figures/ccicgan_examples/img_26_class_siberian_husky}
    \includegraphics[width=.22\textwidth]{figures/ccicgan_examples/img_27_class_siberian_husky}
    \includegraphics[width=.22\textwidth]{figures/ccicgan_examples/img_28_class_siberian_husky}
    \includegraphics[width=.22\textwidth]{figures/ccicgan_examples/img_29_class_siberian_husky}
\end{subfigure}
\\
\begin{subfigure}[b]{\textwidth}
     \raggedright
     \small{``tiger beetle''}
 \end{subfigure}
 \begin{subfigure}[b]{.1\textwidth}
    \centering
    \includegraphics[width=\textwidth]{figures/icgan_examples/img_15_class_tiger_beetle_crop}
 \end{subfigure}
\hfill
\begin{subfigure}[b]{.44\textwidth}
    \centering
    \includegraphics[width=.22\textwidth]{figures/icgan_examples/img_16_class_tiger_beetle}
    \includegraphics[width=.22\textwidth]{figures/icgan_examples/img_17_class_tiger_beetle}
    \includegraphics[width=.22\textwidth]{figures/icgan_examples/img_18_class_tiger_beetle}
    \includegraphics[width=.22\textwidth]{figures/icgan_examples/img_19_class_tiger_beetle}
\end{subfigure}
\hfill
\begin{subfigure}[b]{.44\textwidth}
    \centering
    \includegraphics[width=.22\textwidth]{figures/ccicgan_examples/img_16_class_tiger_beetle}
    \includegraphics[width=.22\textwidth]{figures/ccicgan_examples/img_17_class_tiger_beetle}
    \includegraphics[width=.22\textwidth]{figures/ccicgan_examples/img_18_class_tiger_beetle}
    \includegraphics[width=.22\textwidth]{figures/ccicgan_examples/img_19_class_tiger_beetle}
\end{subfigure}
\\
\begin{subfigure}[b]{\textwidth}
     \raggedright
     \small{``beer glass''}
 \end{subfigure}
 \begin{subfigure}[b]{.1\textwidth}
    \centering
    \includegraphics[width=\textwidth]{figures/icgan_examples/img_45_class_beer_glass_crop}
 \end{subfigure}
\hfill
\begin{subfigure}[b]{.44\textwidth}
    \centering
    \includegraphics[width=.22\textwidth]{figures/icgan_examples/img_46_class_beer_glass}
    \includegraphics[width=.22\textwidth]{figures/icgan_examples/img_47_class_beer_glass}
    \includegraphics[width=.22\textwidth]{figures/icgan_examples/img_48_class_beer_glass}
    \includegraphics[width=.22\textwidth]{figures/icgan_examples/img_49_class_beer_glass}
\end{subfigure}
\hfill
\begin{subfigure}[b]{.44\textwidth}
    \centering
    \includegraphics[width=.22\textwidth]{figures/ccicgan_examples/img_46_class_beer_glass}
    \includegraphics[width=.22\textwidth]{figures/ccicgan_examples/img_47_class_beer_glass}
    \includegraphics[width=.22\textwidth]{figures/ccicgan_examples/img_48_class_beer_glass}
    \includegraphics[width=.22\textwidth]{figures/ccicgan_examples/img_49_class_beer_glass}
\end{subfigure}
\\
\begin{subfigure}[b]{\textwidth}
     \raggedright
     \small{``cliff dwelling''}
 \end{subfigure}
 \begin{subfigure}[b]{.1\textwidth}
    \centering
    \includegraphics[width=\textwidth]{figures/icgan_examples/img_25_class_cliff_dwelling_crop}
 \end{subfigure}
\hfill
\begin{subfigure}[b]{.44\textwidth}
    \centering
    \includegraphics[width=.22\textwidth]{figures/icgan_examples/img_26_class_cliff_dwelling}
    \includegraphics[width=.22\textwidth]{figures/icgan_examples/img_27_class_cliff_dwelling}
    \includegraphics[width=.22\textwidth]{figures/icgan_examples/img_28_class_cliff_dwelling}
    \includegraphics[width=.22\textwidth]{figures/icgan_examples/img_29_class_cliff_dwelling}
\end{subfigure}
\hfill
\begin{subfigure}[b]{.44\textwidth}
    \centering
    \includegraphics[width=.22\textwidth]{figures/ccicgan_examples/img_26_class_cliff_dwelling}
    \includegraphics[width=.22\textwidth]{figures/ccicgan_examples/img_27_class_cliff_dwelling}
    \includegraphics[width=.22\textwidth]{figures/ccicgan_examples/img_28_class_cliff_dwelling}
    \includegraphics[width=.22\textwidth]{figures/ccicgan_examples/img_29_class_cliff_dwelling}
\end{subfigure}
\\
\begin{subfigure}[b]{\textwidth}
     \raggedright
     \small{``hook'}
 \end{subfigure}
 \begin{subfigure}[b]{.1\textwidth}
    \centering
    \includegraphics[width=\textwidth]{figures/icgan_examples/img_30_class_hook,_claw_crop}
 \end{subfigure}
\hfill
\begin{subfigure}[b]{.44\textwidth}
    \centering
    \includegraphics[width=.22\textwidth]{figures/icgan_examples/img_31_class_hook,_claw}
    \includegraphics[width=.22\textwidth]{figures/icgan_examples/img_32_class_hook,_claw}
    \includegraphics[width=.22\textwidth]{figures/icgan_examples/img_33_class_hook,_claw}
    \includegraphics[width=.22\textwidth]{figures/icgan_examples/img_34_class_hook,_claw}
\end{subfigure}
\hfill
\begin{subfigure}[b]{.44\textwidth}
    \centering
    \includegraphics[width=.22\textwidth]{figures/ccicgan_examples/img_31_class_hook,_claw}
    \includegraphics[width=.22\textwidth]{figures/ccicgan_examples/img_32_class_hook,_claw}
    \includegraphics[width=.22\textwidth]{figures/ccicgan_examples/img_33_class_hook,_claw}
    \includegraphics[width=.22\textwidth]{figures/ccicgan_examples/img_34_class_hook,_claw}
\end{subfigure}
\\
\begin{subfigure}[b]{\textwidth}
     \raggedright
     \small{``slot''}
 \end{subfigure}
 \begin{subfigure}[b]{.1\textwidth}
    \centering
    \includegraphics[width=\textwidth]{figures/icgan_examples/img_40_class_slot,_one-armed_bandit_crop}
 \end{subfigure}
\hfill
\begin{subfigure}[b]{.44\textwidth}
    \centering
    \includegraphics[width=.22\textwidth]{figures/icgan_examples/img_41_class_slot,_one-armed_bandit}
    \includegraphics[width=.22\textwidth]{figures/icgan_examples/img_42_class_slot,_one-armed_bandit}
    \includegraphics[width=.22\textwidth]{figures/icgan_examples/img_43_class_slot,_one-armed_bandit}
    \includegraphics[width=.22\textwidth]{figures/icgan_examples/img_44_class_slot,_one-armed_bandit}
\end{subfigure}
\hfill
\begin{subfigure}[b]{.44\textwidth}
    \centering
    \includegraphics[width=.22\textwidth]{figures/ccicgan_examples/img_41_class_slot,_one-armed_bandit}
    \includegraphics[width=.22\textwidth]{figures/ccicgan_examples/img_42_class_slot,_one-armed_bandit}
    \includegraphics[width=.22\textwidth]{figures/ccicgan_examples/img_43_class_slot,_one-armed_bandit}
    \includegraphics[width=.22\textwidth]{figures/ccicgan_examples/img_44_class_slot,_one-armed_bandit}
\end{subfigure}
\\
\begin{subfigure}[b]{\textwidth}
     \raggedright
     \small{``water tower''}
 \end{subfigure}
 \begin{subfigure}[b]{.1\textwidth}
    \centering
    \includegraphics[width=\textwidth]{figures/icgan_examples/img_45_class_water_tower_crop}
 \end{subfigure}
\hfill
\begin{subfigure}[b]{.44\textwidth}
    \centering
    \includegraphics[width=.22\textwidth]{figures/icgan_examples/img_46_class_water_tower}
    \includegraphics[width=.22\textwidth]{figures/icgan_examples/img_47_class_water_tower}
    \includegraphics[width=.22\textwidth]{figures/icgan_examples/img_48_class_water_tower}
    \includegraphics[width=.22\textwidth]{figures/icgan_examples/img_49_class_water_tower}
\end{subfigure}
\hfill
\begin{subfigure}[b]{.44\textwidth}
    \centering
    \includegraphics[width=.22\textwidth]{figures/ccicgan_examples/img_46_class_water_tower}
    \includegraphics[width=.22\textwidth]{figures/ccicgan_examples/img_47_class_water_tower}
    \includegraphics[width=.22\textwidth]{figures/ccicgan_examples/img_48_class_water_tower}
    \includegraphics[width=.22\textwidth]{figures/ccicgan_examples/img_49_class_water_tower}
\end{subfigure}
\\
\vspace{.3cm}
\begin{subfigure}[b]{.1\textwidth}
     \centering
     \caption*{Condit.}
\end{subfigure}
\hfill
\begin{subfigure}[b]{.44\textwidth}
     \centering
     \caption{\icgan samples}
 \end{subfigure}
 \hfill
\begin{subfigure}[b]{.44\textwidth}
     \centering
     \caption{\ccicgan samples}
 \end{subfigure}
    \caption{Visual examples of \allicgan generations. Each row shows, from left to right, the conditioning image -- i.e., central crop of ImageNet image --, followed by \icgan (a) and \ccicgan (b) generated samples. Generetad samples were obtained using the depicted image conditioning and different noise vectors.}
    \label{suppl:fig:icgan_ex}
\end{figure}







\end{document}
