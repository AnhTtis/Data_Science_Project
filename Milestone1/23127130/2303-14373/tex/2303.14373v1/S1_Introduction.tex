\label{sec:intro}
\begin{figure}[!ht]
  \centering
   \includegraphics[width=0.85\linewidth]{Fig/F1.pdf}
   \caption{The schematic illustration of proposed DoNet with decompose-and-recombined strategy, which maps each overlapping cell into the intersection, complement, and instance layers to address the overlapping issue in cytology instance segmentation.}
   \label{fig1}
\end{figure}


Cytology image has been essential for cancer screening and earlier diagnosis, such as qualitative and quantitative identification of cellular morphology, nuclei size, nuclear-cytoplasmic ratio, and other cytological features \cite{hoda2007fundamentals,nayar2015bethesda,jiang2020geometry}. However, examining tens of thousands of cells under the microscope visually is inherently tedious and suffers from inter-/intra-observer variability. Computational techniques enable efficient and accurate characterization of cells from cytology images \cite{hoda2007fundamentals,jiang2022deep}. Among all computational techniques, cell segmentation has been a fundamental and widely-studied task, since the acquisition of cell-level identification is a pre-requisition for further assessment and analysis\cite{lin2021dual,chai2022deep}. 

Deep Learning (DL) methods show promising results for cell-nuclei segmentation in the histopathology image\cite{hussain2020shape,chen2017dcan,jin2023labelefficient,chen2016dcan}. However, cytology segmentation remains challenging for the following two reasons. \textbf{Firstly}, cells in a cytology image are prone to cluster with each other, leading to the overlapping issue. In the cytology images, the translucent cytoplasm of the cell (seen in Figure \ref{fig1}) tends to occlude each other with low contrast staining, leading to ambiguous cellular boundary predictions. This phenomenon is particularly evident in cervical cell images. \textbf{Secondly}, hard mimics, are widespread in the background, along with other technical artefacts such as bubbles, which could mislead the instance segmentation models \cite{che2022learning}. Take the cervical cell image as an example, the widespread white blood cells and mucus stains lead to false predictions for nuclei. To address these challenges, several works\cite{lu2015improved, ushizima2015segmentation} propose the segment-then-refine paradigm, while others \cite{zhou2019irnet,zhou2020deep} utilize the detection-based framework, e.g., Mask R-CNN \cite{he2017mask}. However, they fail to model the interaction between intersection and complement sub-regions within the translucent cell cluster explicitly, resulting in a limited understanding of cross-region relationships.

Amodal instance segmentation tackles the occlusion problem by inferring the integral object based on the partially visible region  \cite{li2016amodal}. Based on the fact that humans can infer the occluded region of an object despite the ambiguity, these methods attempt to learn the integrated object mask (amodal mask) for better occlusion reasoning capability\cite{follmann2019learning,xiao2021amodal} via synthesizing occluded data label pairs and aggregating global information to enhance perceptual ability.
Compared to natural scenes, cell instances in cytology images are mostly semi-transparent. Therefore, an occlusion (overlapping) region exits in both the occluding and occluded instances. However, treating semi-transparent overlapping regions as general occlusion regions is not optimal, since they have different appearances compared to non-overlapping regions, and could in fact provides richer shape information than general occlusion regions.

Motivated by the amodal perception, we propose a decompose-and-recombine strategy for translucent cell instance segmentation, named De-overlapping Network (DoNet). Figure \ref{fig1} provides the schematic diagram. For each cell cluster with more than one cellular sub-region, DoNet starts from implicitly learning the hidden interaction of sub-regions by predicting instance masks from clusters. Then, it explicitly models the components and their relationships via the intersection layer, complement layer, and instance layer, to enhance its perceptual capability.

Initially, we adopt Mask R-CNN to get the coarse predictions, followed by a novel Dual-path Region segmentation Module (DRM) that combines features and coarse masks from the first stage to decompose cell clusters into intersection and complement sub-regions. Then, the semantic Consistency-guided Recombination Module (CRM) is designed to encourage consistency between the refined instances and integral sub-region predictions. Furthermore, to impose the morphological constraint that nuclei stay inside the cellular regions, we propose a Mask-guided Region Proposal Module (MRP) to encourage the model to focus on the intra-cellular area during nuclei segmentation.

The overall contributions are summarized as follows:
\begin{itemize}

\item A novel de-overlapping network for cell instance segmentation with a decompose-and-recombined strategy, decomposing the cell regions with the DRM, as well as implicitly and explicitly modeling the semantic relationship between intersection, complement, and instance (cell) components via the CRM.
These designs equip the network with enhanced perceptual capability in overlapping cellular sub-regions.  

\item A mask-guided region proposal module (MRP) that leverages the cytoplasm attention map for the intra-cellular nuclei refinement, which imposes the biology prior of cellular instances into the module, effectively mitigating the influence of mimickers widespread in the background.

\item Extensive experiments on two overlapping cytology image segmentation datasets, namely ISBI2014 \cite{lu2015improved} and CPS \cite{zhou2020deep}, demonstrating that our proposed DoNet outperforms other state-of-the-art (SOTA) methods by a large margin.   
\end{itemize}


\begin{figure*}[t]
\centering
\includegraphics[width=\textwidth]{Fig/F2.pdf} 
\caption{The flowchart of the proposed DoNet. 
It consists of four main parts: (1) baseline for coarse mask segmentation; (2) DRM to simultaneously regress intersection and complement regions which provides cues for final mask refinement; (3) CRM to explicitly refine the final mask and encourage the semantic consistency; (4) MRP to mitigate the effect of background noise.}
\label{fig2}
\end{figure*}