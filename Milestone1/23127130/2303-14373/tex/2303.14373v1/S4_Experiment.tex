\subsection{Experimental Setup}
We evaluate our proposed DoNet in two cytology image datasets for overlapping cell instance segmentation:

\noindent \textbf{ISBI2014\cite{lu2015improved}:} This is a widely-used dataset from \emph{Overlapping Cervical Cytology Image Segmentation Challenge}\footnote{~\url{https://cs.adelaide.edu.au/~carneiro/isbi14_challenge/index.html}}, which consists of 8 extended depth-of-focus (EDF) real cervical cytology images and corresponding synthetic images. It contains high-quality pixel-level annotations for both nuclei and cytoplasm with a resolution of $512 \times 512$. We follow the setting in this challenge \cite{lu2015improved} to use 45, 90 and 810 images for training, validation and testing to evaluate our proposed DoNet in a supervised setting.

\noindent \textbf{CPS\cite{zhou2020deep}:} This liquid-based cytology dataset contains 137 labeled images with a resolution of $1000 \times 1000$. In total, it contains 4439 cytoplasm and 4789 nuclei annotations. We conduct 3-fold cross-validation on this dataset.

\noindent \textbf{Evaluation metrics:} To measure the overall performance of the proposed DoNet, we utilize four commonly-used evaluation metrics in instance segmentation: aggregated Jaccard index (AJI), average Dice coefficient (Dice), F1-score (F1)\cite{kumar2017dataset}, mean of Average Precision (mAP) \cite{he2017mask}. In order to compare our result on the ISBI2014 with previous studies \cite{lu2015improved}, we further adopt evaluation metrics including Dice, object-based false negative rate ($\text{FN}_{o}$), and pixel-based true positive rate ($\text{TP}_{p}$).

\begin{table*}[!ht]
	\caption{Quantitative segmentation results of DoNet and other state-of-the-art methods on CPS and ISBI2014.} 
	\vspace{-0.2in}
	\begin{center}{\small
			\resizebox{0.95\linewidth}{!}{
				\begin{tabular}{c|cccc|cccc}
					\hline
                     \multirow{2}{*}{Methods} & \multicolumn{4}{c|}{ISBI2014} & \multicolumn{4}{c}{CPS}\\
                    \cline{2-9} 
                    & mAP$\uparrow$&Dice$\uparrow$&F1$\uparrow$&AJI$\uparrow$&mAP$\uparrow$&Dice$\uparrow$&F1$\uparrow$&AJI$\uparrow$\\
                    \hline
	            Mask R-CNN \cite{he2017mask}&59.09 &91.15 &92.54 &77.07& 48.28 $\pm$ 3.10  & 89.32 $\pm$ 0.50 &85.07 $\pm$ 2.01 & 69.20 $\pm$ 2.27  \\
                    Cascade R-CNN\cite{cai2018cascade}&62.45 &91.29 &92.51 &77.91& 47.87 $\pm$ 3.27 & 89.24 $\pm$ 0.44 &83.33 $\pm$ 1.65 &68.86 $\pm$ 3.55  \\	
                    Mask Scoring R-CNN\cite{huang2019mask} &63.56 &91.28 &91.87 &75.14 & 48.38 $\pm$ 3.13 & 89.39 $\pm$ 0.24 &82.98 $\pm$ 1.86 &67.45 $\pm$ 2.45 \\	
                    HTC\cite{chen2019hybrid}&59.62 &91.39 &88.08 &75.00& 47.60 $\pm$ 3.56 &89.08  $\pm$ 0.51 &81.30 $\pm$ 2.56 &66.35 $\pm$ 2.84  \\	
                    Occlusion R-CNN\cite{follmann2019learning}&62.35 &91.75 &93.18 &78.64& 48.14 $\pm$ 2.84 & 89.08 $\pm$ 0.28 &85.69 $\pm$ 2.28 &69.51 $\pm$  2.45  \\	
                    Xiao et al.\cite{xiao2021amodal}&57.34 &91.70 &92.75 &78.29& 48.53 $\pm$ 2.85 & 89.29 $\pm$ 0.24 &85.46 $\pm$ 2.60 &69.37 $\pm$ 2.88  \\
                    DoNet &\textbf{64.02} &\textbf{92.13} &\textbf{93.23} &\textbf{79.05}& 49.43 $\pm$ 3.83 & \textbf{89.54 $\pm$ 0.25} &85.51 $\pm$ 2.33 &70.08 $\pm$ 2.84  \\	
                    DoNet w/ Aug.&- &- &-&-& \textbf{49.65  $\pm$ 3.52}  & 89.50  $\pm$ 0.38  &\textbf{86.30  $\pm$ 2.01}   &\textbf{70.56  $\pm$ 2.34}   \\	
					\hline
		\end{tabular}}}
	\end{center}
	\label{t1}
\end{table*}

\noindent \textbf{Implementation details:}
We utilize the Mask R-CNN\cite{he2017mask} in Detectron2 \cite{wu2019detectron2} as the baseline model.
We use the ResNet-50-based FPN network in all experiments.
During training, we adopt SGD with 0.9 momentum as the optimizer. 
We set the initial learning rate to 0.001 and add the linear warm-up in the first 1k iterations. 
We train the network for 60k iterations, decreasing the learning rate by a factor of 0.1 after 50k and 55k iterations.

\subsection{Results}
We quantitatively compare the cell instance segmentation results from the ISBI2014 and CPS in Table \ref{t1} with state-of-the-art methods in the field of general instance segmentation (Mask R-CNN\cite{he2017mask}, Cascade R-CNN\cite{cai2018cascade}, Mask Scoring R-CNN\cite{huang2019mask}, HTC\cite{chen2019hybrid}) and amodal instance segmentation (Occlusion R-CNN\cite{follmann2019learning}, Xiao et al. \cite{xiao2021amodal}). Noted that the amodal instance segmentation methods and general instance segmentation methods perform differently on two datasets due to the varying degrees of overlapping. Our method achieves the highest scores among all metrics. Specifically, it gains $2.68\%$ and  $0.52\%$ improvements for mAP and AJI compared with the best of others\cite{follmann2019learning} on the ISBI2014 dataset, as well as $1.85\%$ and $1.02\%$ improvements compared with the best\cite{xiao2021amodal} on the CPS dataset. We also evaluate $TP_{p}$ and $FN_{o}$ for DoNet to compare results against winners of the ISBI2014 challenge\cite{nosrati2014variational, ushizima2015segmentation} and their following works \cite{lee2016segmentation, tareef2018multi}, which are mostly the segment-then-refine manners
(Seen in Table \ref{t2}).

Furthermore, by introducing the synthetic clusters as instance-level augmentation, DoNet further has $0.45\%$ and $0.68\%$ improvements for mAP and AJI on the CPS dataset.This is mainly because the synthetic overlapping cells can further enhance the model's occlusion reasoning capability, which is consistent with the conclusions in \cite{li2016amodal}.

\begin{table}[!h]
\vspace{-0.1in}
	\caption{Comparison with other methods on ISBI2014.} 
 	\centering
			\resizebox{0.85\linewidth}{!}{
				\begin{tabular}{c|ccc}
					\hline
                     \multirow{2}{*}{Methods} & \multicolumn{3}{c}{ISBI2014} \\
                    \cline{2-4} 
                    & Dice$\uparrow$& $TP_{p}$$\uparrow$ & $FN_{o}$$\downarrow$ \\
                    \hline
                    Ushizima et al. \cite{ushizima2015segmentation}&0.872 &0.841&0.265 \\
                    Nosrati et al. \cite{nosrati2014variational}&0.871 &0.875&0.110 \\
                    Walter et al. \cite{walter2021multistar}&0.860 &0.830&0.310 \\
                    Lu et al. \cite{lu2015improved}&0.893 &0.905&0.315 \\
                    Lee et al. \cite{lee2016segmentation}& 0.897 & 0.882 &0.137   \\	
                    Tareef et al. \cite{tareef2018multi}& 0.898& 0.946 &0.161  \\	
                    Chen et al. \cite{chen2021segmentation}& 0.920& 0.900 &\textbf{0.020}  \\	
                    DoNet & \textbf{0.921}& \textbf{0.948} &0.162\\
					\hline
		\end{tabular}}
	\label{t2}
	\vspace{-0.1in}
\end{table}

\subsection{Ablation Study}
\noindent \textbf{Effects of Network Components:} 
We perform ablation studies to investigate the effects of the different components of our proposed pipeline for DoNet. The comparison results can be seen in Table \ref{t3}. By adding DRM for explicit decomposing integral instances into intersection and complement sub-regions, we observe a $3.25\%$ increase in the average mAP of cytoplasm and nuclei on ISBI2014. However, directly adding DRM without the fusion of structural and morphological information, may mislead the model. This issue is observed in the more complex CPS dataset. To alleviate this problem, DoNet takes advantage of the decompose-and-recombined strategy by adding CRM after DRM.This strategy brings a total of $7.34\%$ and $1.74\%$ mAP improvements on two datasets by strengthening the model's perception of overlapping regions while preserving morphological information.

\begin{figure}[!ht]
% \vspace{-0.05in}
	\centering
	\includegraphics[width=0.95\linewidth]{Fig/F3.pdf}
         % \vspace{-0.1in}
	\caption{Illustration of background noise suppression in MRP: (a) the original image, (b) the origin feature map for region proposal, (c) the attention map from cytoplasm prediction, and (d) the re-weighted feature map for nuclei proposal generation.}
	%\label{fig:model}
	\label{fig:f-MRP}
	% \vspace{-0.2in}
\end{figure}

\begin{table*}[!ht]
\vspace{-0.1in}
	\caption{Effect of each proposed module on CPS and ISBI2014 datasets. \checkmark denotes adding the corresponding module.} 
 	\centering
			\resizebox{0.95\linewidth}{!}{
				\begin{tabular}{cccc|cccc|cccc}
					\hline
                        \multirow{2}{*}{Base} & \multirow{2}{*}{DRM} &\multirow{2}{*}{CRM} &\multirow{2}{*}{MRP}  & \multicolumn{4}{c|}{ISBI2014}& \multicolumn{4}{c}{CPS} \\
                    \cline{5-12} 
                    &&&& mAP$\uparrow$&Dice$\uparrow$&F1$\uparrow$&AJI$\uparrow$& mAP$\uparrow$&Dice$\uparrow$&F1$\uparrow$&AJI$\uparrow$\\
                    \hline
	              \checkmark&&&&59.09&91.15&92.54&77.07& 48.28 $\pm$ 3.10 & 89.32 $\pm$ 0.50 &85.07 $\pm$ 2.01 &69.20 $\pm$ 2.27   \\				
                    \checkmark&\checkmark&& &61.01&91.61&92.86&78.06& 48.03 $\pm$ 3.48   &89.13 $\pm$ 0.30 &84.63 $\pm$ 2.57 & 68.56 $\pm$ 2.57   \\
                     \checkmark&\checkmark&\checkmark&&63.43&91.87&\textbf{94.16}&\textbf{79.88}& 49.12  $\pm$ 3.26 &89.47  $\pm$ 0.31&84.82  $\pm$ 2.73 &69.26  $\pm$ 2.63\\
                    \checkmark&\checkmark&\checkmark&\checkmark&\textbf{64.02}&\textbf{92.13}&93.23&79.05&  \textbf{49.43 $\pm$ 3.83}  & \textbf{89.54 $\pm$ 0.25} &\textbf{85.51 $\pm$ 2.33 } &\textbf{70.08 $\pm$ 2.84 } \\	
					\hline
		\end{tabular}}
	\label{t3}
\end{table*}
\begin{table*}[!ht]
	\caption{Ablation study of DRM and CRM on the ISBI2014 dataset.\checkmark denotes adding the corresponding component or strategy. } 
	\vspace{-0.15in}
	\begin{center}{\small
			\resizebox{\linewidth}{!}{
            \begin{tabular}{p{0.2cm}p{0.2cm}p{0.2cm}p{0.3cm}p{0.7cm}|ccc|ccc|ccc|ccc}
                \hline
                \multirow{2}{*}{$H_{i}$}&\multirow{2}{*}{$H_{o}$}&\multirow{2}{*}{$H_{m}$} &\multirow{2}{*}{FU}&\multirow{2}{*}{$\mathcal{L}_{cons}$}&\multicolumn{3}{|c|}{mAP$\uparrow$} & \multicolumn{3}{c|}{Dice$\uparrow$}& \multicolumn{3}{c|}{F1$\uparrow$}& \multicolumn{3}{c}{AJI$\uparrow$}\\
                \cline{6-17} 
                &&&&& Cyt.&Nuc.&Avg.& Cyt.&Nuc.&Avg.& Cyt.&Nuc.&Avg.& Cyt.&Nuc.&Avg.\\
                \hline
                \checkmark&&&&& 50.71 & 67.46 &59.09 & 90.72 & 91.59 &91.15& 86.96 & 98.13 &92.54&70.55 & 83.60 &77.07 \\
	        \checkmark&\checkmark&&&& 54.84 & 65.94 &60.39 & 91.56 & 91.54 &91.55& 87.00 & 97.68 &92.34& 72.07 & 82.67 &77.37 \\	
               \checkmark&\checkmark&\checkmark&&& 57.41 & 66.49 &61.95& 91.79 & 91.53 &91.66& 88.67 & 97.98 &93.32& 74.00 & 83.27 &78.63 \\
               \hline
               \checkmark&\checkmark&\checkmark&\checkmark&& 58.40 & \textbf{67.82} &63.11& \textbf{92.22} & \textbf{91.74} &\textbf{91.98}& 89.21 & 98.07 &93.64& 74.94 & 83.70 &79.32 \\
               \checkmark&\checkmark&\checkmark&\checkmark&\checkmark& \textbf{59.31} & 67.56 &\textbf{63.43}& 92.03 & 91.71 &91.87& \textbf{90.13} & \textbf{98.18 }&\textbf{94.16}& \textbf{75.86} & \textbf{83.91} &\textbf{79.88} \\
               \hline
		\end{tabular}}}
	\end{center}
	\label{t4}
\end{table*}

Applying MRP for mitigating the side effects from background mimics yields a further improvement of $0.59\%$ mAP and $0.31\%$ mAP on ISBI2014 and CPS datasets. Figure \ref{fig:f-MRP} provides the visualization of MRP operation, where background instances (e.g., mucus, karyoclasis, pointed by red arrow) are suppressed with strong responses in the feature map, encouraging the RPN to concentrate on cellular instance during nuclei region proposal.

\noindent \textbf{Design Choice for DRM:} 
We provide detailed comparisons on the ISBI2014 dataset to demonstrate the effectiveness of different components in DRM: 
1) $H_{i}$: instance mask head for coarse segmentation only; 2) $H_{o}$: intersection mask head for overlapping region segmentation; 
3) $H_{m}$: complement mask head for non-overlapping region segmentation; 

As seen in Table \ref{t4}, adding $H_{o}$ yields an improvement of $2.20\%$ in  average mAP, with a further $2.64\%$ gains from the additional $H_{m}$.
We notice that cytoplasm results have a more significant improvement of $13.2\%$ in mAP, which is indeed in line with our design intent.


\begin{figure}[!ht]
	\centering
	\includegraphics[width=0.8\linewidth]{Fig/F4.pdf}
	\caption{Qualitative results from (a) standard \cite{he2017mask}, (b) multi-task (DoNet w/o CRM), (c) amodal \cite{follmann2019learning}, and (d) proposed de-overlapping instance segmentation method. }
	\label{fig:f6}
\end{figure}

\noindent \textbf{Design Choice for CRM:} 
The goal of semantic consistency regularization is to 
enhance the model's overlapping reasoning capability by learning the concept of recombined instances from sub-regions.
We provide comparisons to demonstrate the design choice (Table \ref{t4}):
1) CU + FU: integration of rich semantic features as inputs via Concatenation Unit and Fusion Unit. 
2) $\mathcal{L}_{cons}$: consistency regularization between the recombined prediction $\hat{e}^{r}_{k,i}$ and the fusion of sub-regions $\hat{o}_{k,i}$, $\hat{m}_{k,i}$.
As seen in Table \ref{t4}, by aggregating the rich semantic feature for overlapping and non-overlapping region via Fusion Unit, we observe an increase of $1.52\%$ average mAP on the ISBI2014 dataset.
Adding consistency regularization further improves $0.91\%$ mAP for cytoplasm.

\begin{figure*}[!ht]
	\centering
	\includegraphics[width=\linewidth]{Fig/F5.pdf}
	\vspace{-0.15in}
	\caption{Qualitative results of our DoNet and other SOTA methods on CPS (top) and ISBI2014 (bottom) datasets. (a) Ground Truth; (b) Mask R-CNN \cite{he2017mask}; (c) Occlusion R-CNN \cite{follmann2019learning}; (d) Xiao et al. \cite{xiao2021amodal}; (e) Cascade R-CNN \cite{cai2018cascade}; (f) Hybrid Task Cascade \cite{chen2019hybrid}; (g) Mask Scoring R-CNN \cite{huang2019mask}; (h) Our proposed DoNet.}
	\label{fig:f4}
  \vspace{-0.1in}
\end{figure*}
\begin{figure}[!ht]
	\centering
	\includegraphics[width=0.8\linewidth]{Fig/F6.pdf}
	\vspace{-0.1in}
	\caption{Heatmaps of the intersection region, the
complement region, and the integral instance on CPS (top) and ISBI2014 (bottom) datasets, including (a) the original instance, the prediction heatmap of (b) intersection regions, (c) complement regions, and (d) the integral instance.}
	\label{fig:f3}
  \vspace{-0.2in}
\end{figure}

Figure \ref{fig:f6} visualizes the results of the DoNet and other typical instance segmentation methods, including standard (Mask R-CNN), multi-task (DoNet w/o CRM), and amodal (Occlusion R-CNN) instance segmentation model.
It demonstrates the importance of adding interaction among sub-regions via CRM and the strong perceptual capability of DoNet in overlapping regions. 

\subsection{Qualitative Analysis and Discussion}
We visualize the heatmap of the intersection region, the complement region, and the integral instance in Figure \ref{fig:f3}. The proposed method successfully identifies sub-regions based on the overlapping concept, even in low-resolution areas with high transparency.

Furthermore, we provide qualitative comparisons on CPS (top) and ISBI2014 (bottom) datasets in Figure \ref{fig:f4}. Our proposed DoNet outperforms other instance segmentation methods. Specifically, red rectangles provide details and highlight the main difference among these results. In the segmentation results of the CPS data, it can be seen that overlapped cells with different staining (e.g., dark red and blue) show significant appearance inconsistency between cellular sub-regions. Previous works (e.g., Mask R-CNN\cite{he2017mask} and Occlusion R-CNN\cite{follmann2019learning}) with limited perceptual capability have difficulty capturing the relationship between pixels in the intersection and complement sub-regions, leading to ambiguous segmentation contours. In contrast, our DoNet can effectively distinguish different instance boundaries and can better perceive the integrality of cells. This generalized superiority is also observed in the ISBI2014 dataset with low-contrast cellular instances.