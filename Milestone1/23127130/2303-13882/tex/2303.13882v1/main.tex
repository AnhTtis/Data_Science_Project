\documentclass[%
reprint,
superscriptaddress,
%groupedaddress,
%unsortedaddress,
%runinaddress,
%frontmatterverbose, 
%preprint,
%preprintnumbers,
nofootinbib,
nobibnotes,
%bibnotes,
 amsmath,amssymb,
 aps,
%pra,
%prb,
%rmp,
%prstab,
%prstper,
floatfix,
prl,
]{revtex4-2}
\usepackage{graphicx}% Include figure files
\usepackage{dcolumn}% Align table columns on decimal point
\usepackage{bm}% bold math
\usepackage{hyperref}% add hypertext capabilities
%\usepackage[mathlines]{lineno}% Enable numbering of text and display math
%\linenumbers\relax % Commence numbering lineshttps://de.overleaf.com/project/63989d4d81d3700ae565b898

%\usepackage[showframe,%Uncomment any one of the following lines to test 
%scale=0.7, marginratio={1:1, 2:3}, ignoreall,% default settings
%text={7in,10in},centering,
%margin=1.5in,
%total={6.5in,8.75in}, top=1.2in, left=0.9in, includefoot,
%height=10in,a5paper,hmargin={3cm,0.8in},
%]{geometry}

\usepackage[capitalise]{cleveref}
\usepackage{tikz}
\tikzset{
  font={\fontsize{8pt}{12}\selectfont}}
\usepackage{subfigure}
\usepackage{upgreek}
%\usepackage[inkscapearea=page]{svg}
\usepackage{adjustbox}


\begin{document}

\preprint{APS/123-QED}

% \title{Turbulence suppresses bipolar charging of polydisperse particles}
%\title{Turbulence regulates bipolar charging of polydisperse particles}
\title{Suppression and control of bipolar particle charging by turbulence}

\author{Simon Jantač}
\email{simon.jantac@ptb.de}
\affiliation{Physikalisch-Technische Bundesanstalt (PTB), Braunschweig, Germany}
\author{Holger Grosshans}%
\affiliation{Physikalisch-Technische Bundesanstalt (PTB), Braunschweig, Germany}%
\affiliation{Otto von Guericke University of Magdeburg, Institute of Apparatus and Environmental Technology, Magdeburg, Germany}

% word limit 3 750
% abstract max 600 characters

\begin{abstract}
Current theoretical models predict particles of the same material but different sizes to charge bipolar upon contacts. 
However, in a wall-bounded fluid flow, we found turbulence to moderate the electrostatic charge distribution of particles. 
Aerodynamic forces skew the collision frequency and narrow the charge distribution's bandwidth. 
Especially in moderately polydisperse systems of a low Stokes number, bipolar charging reduces dramatically. 
Moreover, not the smallest but mid-sized particles collect the most negative charge.
Turbulence separates charge, producing pockets of high electric potential in low-vorticity regions.
\end{abstract}

\maketitle

%\section{introduction}

Particles suspended in an airflow, such as atmospheric dust, erupted volcanic ash, fluidized beds, or pneumatically conveyed powder, charge electrically during collisions with each other~\citep{Jari2018, James2008, Forward2009, Grosshans2022, Song201714716, Cimarelli2022, GROSSHANS2023140918}.
In these and other examples, usually, wall-bounded turbulence drives the collisions and the charge transfer between polydisperse particles of the same material.
Despite the lack of an obvious driving force for charge transfer, such as different work functions, particles in these conditions charge bipolar~\citep{Waitukaitis2014, Forward2009phys}.
Their resulting polarity depends on their size;
large particles charge positively, and small ones negatively.

Theoretical models successfully explained the bipolar charging of same-material particles in a vacuum~\citep{LIU2020199,Xie13}.
These models even revealed the relationship between the particles' charge and size distributions~\citep{KONOPKA2017150};
highly polydisperse systems charge strongly bipolar, and slightly polydisperse ones weakly.
However, the effect of wall-bounded turbulence, which drives inter-particle collisions, is not known so far.

%hg: for the first sentence 6 refs are more than enough. but for the following 3 statements we could add 1-2 refs each
%reviews: Chowdhury2021104,Lacks2019,FOTOVAT2017303
% good, but relation to text unclear: Lee2015
% you can put these refs back in, if they fit

Two theoretical models captured bipolar charging of same-material particles:
the surface state and the dipole amplification theory.
The dipole amplification theory explains charge transfer through the polarization of dielectric particles~\citep{Pahtz2010,Yoshimatsu2017}.
Charge transfer occurs when fluctuations break the electric field's symmetry.

The surface state theory, on the other hand, specifically aims to predict the charging of insulators with identical, homogeneous surface composition~\citep{Lowell1,Lowell2}.
In its initial formulation~\citep{Lacks2008}, high-energy state electrons can relax into vacant low-energy states.
The high electrical resistivity prohibits high-energy electrons from moving to the vacant low-energy states on the same surface.
Therefore, if the surface contacts another surface, high-energy electrons relax to vacant states of the other surface located at their contact point.
Bipolar charge distributions develop due to asymmetrical charge transfer between particles of different sizes during a collision.
Thus, the frequency of size-dependent collisions determines the resulting charge distribution.


To sum up, current theories explain bipolar charging by inter-particle collisions but neglect the influence of turbulence.
Our numerical model differs from previous works by including the relevant physical mechanisms:
the interaction of polydisperse particles with wall-bounded turbulence and the charge exchange in-between particles of the same material during contact.

\begin{figure*}
\centering
\subfigure[]{\begin{tikzpicture}[scale=1,]
\node at (0,0){\includegraphics[width=0.25\textwidth]{NewFigures/PSD2.pdf}\label{PSD}};
%
\node[] at (-2.05,-1.75) {$0$};
\node[] at (-.65,-1.75) {$1.08$};
\node[] at (0.7,-1.75) {$4.33$};
\node[] at (2,-1.75) {$9.75$};
\node[] at (0,-2) {$St$};
%
\node[] at (-2.05,-1.) {$0$};
\node[] at (-0.65,-1.) {$50$};
\node[] at (0.7,-1.) {$100$};
\node[] at (2,-1.) {$150$};
\node[] at (0.0,-1.3) {$d_p \ (\upmu \mathrm{m})$};

\node[right] at (-2.9,-0.9) {$10^{-2}$};
\node[right] at (-2.9,0.35) {$10^{-1}$};
\node[right] at (-2.9,1.45) {$10^{0}$};
\node[rotate=90] at (-3,0.4) {$\mathrm{pdf}_{d_\mathrm{p}} (\mathrm{\upmu m^{-1})}$};
\end{tikzpicture}}
%
%
\subfigure[]{
\begin{tikzpicture}[]
\node at (0,0){\includegraphics[trim=0cm 5.5cm 0cm 10cm,clip=true,width=0.65\textwidth]{NewFigures/CFD_new.pdf}\label{fig:CFD1}};
%
\draw [fill=white,white] (-4,-1.23) rectangle (5.2,-0.3);
%
\node[] at (-3.75,-1.1) {$0.4$};
\node[] at (-3.1,-1.1) {$2$};
\node[] at (-2.5,-1.1) {$9$};
\node[] at (-1.85,-1.1) {$42$};
\node[] at (-1.2,-1.1) {$190$};
\node[] at (-0.60,-1.1) {$860$};
\node[] at (0.1,-1.1) {$4000$};
\node[] at (-1.85,-0.58) {$|| \boldsymbol{\omega}|| \ (\mathrm{s^{-1}})$};
%
\node[] at (1.15,-1.1) {$-35$};
\node[] at (1.95,-1.1) {$-20$};
\node[] at (2.5,-1.1) {$-10$};
\node[] at (3.1,-1.1) {$0$};
\node[] at (3.65,-1.1) {$10$};
\node[] at (4.2,-1.1) {$20$};
\node[] at (5,-1.1) {$35$};
\node[] at (3.1,-0.58) {$q_p \ (\mathrm{fC})$};
%
\draw [fill=white,white] (-5.1,-1.8) rectangle (-4,-0.3);,
\node[] at (-5,-1.3) {$x$};
\node[]at (-4.4,-0.8) {$y$};
\draw[->] (-4.2,-1.5)--(-4.2,-0.7);
\draw[->] (-4.2,-1.5)-- (-5,-1.5);
%
\draw[|-|] (-5.2,1.7)--(5.18,1.7);
\node[] at (0,1.9) {$6H$};
\draw[|-|] (5.3,-0.2)--(5.3,1.55);
\node[rotate=-90] at (5.5,0.675) {$H$};
\end{tikzpicture}}
\caption{
a) Normalized particle size distributions tested in this study, from monodisperse ($d_\mathrm{p}=75~\upmu$m) to most polydisperse ($d_\mathrm{p}=10~\upmu$m--$140~\upmu$m).
The particles' mass and number are the same in all cases, their Stokes number differs slightly.
b) 
Snapshot of the simulation domain in flow ($x$) and wall-normal ($y$) direction;
most polydisperse case, the particles obtained their final charge ($t=0.26$~s). 
The vorticity $\Vert \boldsymbol{\omega} \Vert$ (note the logarithmic color map) peaks near the walls.
Pockets of high positive electric potential ($>-0.3$~V, enclosed by the red isocontour) and high negative electric potential ($<-0.3$~V, enclosed by the blue isocontour) form where $\Vert \boldsymbol{\omega} \Vert$ is low. 
}
\label{fig:PSD}
\end{figure*}
%hg: a) a legend giving the standard deviation of St would be great

%\section{numerical model}

All parts of our multiphysics tool, the gas phase, particle dynamics, and electrostatic field solvers, were extensively documented and validated earlier~\citep{Gro20d}.
The gas and particle phases are four-way coupled to each other.
That means the particles transfer momentum to the gas, and the particles' trajectories are affected by drag, lift, and collisions with other particles.
Further, the particles interact with the electric field;
charged particles generate the electric field, which then exerts a force on charged particles.

Direct numerical simulations of the Navier-Stokes equations in the Eulerian framework solve the fluid turbulence.
We simulate a wall-bounded turbulent channel flow (see \cref{fig:CFD1}) of the dimensions $6H \times 2H \times H$ (with $H=4$~cm) in streamwise~($x$), spanwise~($z$), and wall-normal~($y$) direction and a frictional Reynolds number of $Re_\tau=360$.

A grid resolution of $256 \times 144 \times 144$ cells proved to produce grid-independent particle trajectories~\citep{Gro20d}.
The domain is periodic in $x$- and $z$-direction, and the fluid sticks to the walls that confine the domain in $y$- direction.
This generic setup mimics the conditions in which particle-laden flows typically charge;
for example in pneumatic conveyors, other powder flow devices, or wall-bounded natural flows.

Particles of a number density of $4 \times 10^9$~m$^{-3}$ were seeded randomly into the turbulent gas flow.
The particles are of different sizes but identical material and density, with a particle/fluid density ratio of~1000.
To investigate the effect of polydispersity, we simulated the eight different particle size distributions depicted in \cref{fig:PSD}. 
At the same time, to isolate the effect of the size distribution, we kept all other parameters constant:
we retained the total solid mass in the system, the number of particles, and the average Stokes number, $ \langle St \rangle =\tau_p/t_0$, the ratio of the flow time-scale, $t_0$, and the particle relaxation time $\tau_0=(\rho_\mathrm{p} d^2_{32})/(18 \mu_\mathrm{p})$ (where $\rho_\mathrm{p}$ is the particle density, $d_{32}$ is the Sauter mean diameter and $\mu_\mathrm{p}$ represents the fluid dynamic viscosity).
We defined the flow time-scale as $t_0=H/u_0$, with $u_0$ being the fluid's centerline velocity.
In addition, we simulated four powders of equal polydispersity ($\sigma(St)=0.0118$), where $\sigma(St)~=~\sqrt{\sum_{n=1}^N (d_\mathrm{p}(n)-d_{32})^2/N}$) and varied the average Stokes number from 2 to 39.

The evolution of the particles is tracked in the Lagrangian framework.
Due to their turbopheretic drift, the large particles migrate towards regions of lesser turbulent intensity and small particles to the walls.
After seeding, the simulations of uncharged particles proceeded until the local average particle concentrations converged.
Then, we activated the below-described charging model.
Thus, all charging simulations presented in this Letter started with fully-developed gas and particle flow fields.

A hybrid method models the interaction of charged particles and the electric field.
Therein, the electric field, calculated by Poisson's equation, exerts far-field forces.
For high accuracy, Coulomb's law directly computes the electrostatic interaction of particles with their close-by neighbors.
The electric potential at the walls equals zero.

The charge of particle $i$ after $N$ collisions with other particles yields $q_i=\sum_{n=1}^N \delta q_i(n)$, where $\delta q_i(n)$ is the charge transferred to the particle during the $n$-th impact.
Thus, we neglect a possible charge transfer to the walls and focus on bipolar charge distributions due to collisions in-between particles.
The collisional charge exchange is computed by the surface state model.
As mentioned above, according to this model, high-energy state electrons can relax into vacant low-energy states in the surface state model.
More specifically, we followed \citet{KONOPKA2017150} who, instead of describing the transfer of electrons or ions, generalized the model to transferable charged species (TCS) irrespective of their identity.
Under these assumptions, the charge
%
\begin{equation}
\delta q_i = \epsilon (c_{\mathrm{s},j}-c_{\mathrm{s},i})A_\mathrm{c,max}^{i,j} 
\end{equation}
%
transfers from particle $j$ to $i$ during their contact.
In the above equation, $\epsilon$ denotes the charge of one TCS, which is a multiple of an elementary charge.
We assume that the TCS is either an electron or anion with a charge number of -1 (i.e., adsorbed $\mathrm{OH}^-$ \citep{Grosjean2023}, ionomer \citep{Diaz1993}, etc.)
The TCS surface concentration on particle $i$ and $j$ before their contact is $c_{\mathrm{s},i}$ and $c_{\mathrm{s},j}$, respectively.
The maximal contact area during the collision, $A_\mathrm{c,max}^{i,j}$, was calculated through the Hertzian theory.

Reflecting the scarcity of TCS on insulating surfaces, we defined an initial value of $c_\mathrm{s,0}=10~\upmu$m$^{-2}$ \citep{Lacks2019}.
According to the surface state theory, TCS relax into a stable low-energy state.
However, only TCS in high energy states can exchange during collisions; 
thus, each TCS transfers only once.
In other words, $c_{\mathrm{s},i}$ steadily decreases while $q_i$ steadily increases as the particles undergo collisions.

A flow close to saturation, which means most TCS is transferred is depicted in \cref{fig:CFD1}.
Nevertheless, most particles remain airborne instead of adhering to the grounded walls or agglomerating with other particles of opposite polarity.
Thus, all through our simulations, the fluid affects the dynamics of the particles, despite having reached their highest charge.

%\section{Results}

In our simulations, the charge of polydisperse particles of initially the same surface composition, which means identical $c_\mathrm{s,0}$, evolves in three stages:
In the first stage, particles collide for the first time with each other.
Due to their initially equal TCS density, an equal amount of TCS is transferred regardless of the particle size.
As a result, $\delta q_i(n)$ equals zero, but now the small particles have a lower average TCS density than the large ones.
In the second stage, the different TCS densities drive the charge redistribution.
Collisions between particles of different TCS densities lead to a net charge transfer, and, on average, small particles lose their charge faster than large ones.
After the particles depleted most of their TCS, the process enters its third stage.
In the third stage, the driving force for charge transfer is small until, finally, charging ultimately stops.


\begin{figure}
\centering
\subfigure[]{\includegraphics[trim=0cm 0cm 0cm 0cm,clip=true,width=0.4\textwidth]{NewFigures/DEM_CFD_Stokes.pdf}\label{fig:CFDvsDEMa}}
\subfigure[]{\includegraphics[trim=0cm 0cm 0cm 0cm,clip=true,width=0.4\textwidth]{NewFigures/DEM_CFD_Rho_lin.pdf}\label{fig:CFDvsDEMb}}
\caption{
The standard deviation of the resulting charge distribution (i.e., bipolar charging) as a function of a) the standard deviation of the Stokes number (i.e., polydispersity) while $\langle St \rangle=3.86$ and b) the average Stokes number while $\sigma(St)=0.0118$.
Turbulence suppresses bipolar charging occurring in vacuum the most for a) moderately polydisperse powder and b) small average Stokes numbers.
}
\label{fig:CFDvsDEM}
\end{figure}

Figure~\ref{fig:CFDvsDEM} summarizes the results of all our simulations.
Since the particles' relaxation times scale with their diameters ($\tau_\mathrm{p} \propto d_\mathrm{p}^2$), the standard deviation of the particles' Stokes number distribution, $\sigma(St)$, quantifies the width of the size distribution.
Even though the vacuum lacks a flow time-scale, we characterize the particle size distributions with the corresponding Stokes numbers from the turbulence cases.
Analogous, the width of the resulting charge distribution is $\sigma(q)$. 
Since, in sum, all particles in the system are always electrically neutral, $\sigma(q)$ directly quantifies bipolar charging.

The available TCS decay exponentially while the particles asymptotically reach their final charge.
Therefore, we decided to compare the simulations once the TCS density dropped to $c_{\mathrm{s},0}/\mathrm{e}$ (e is Euler's number).
On average, a particle reaches this value after 51 collisions.

As elaborated above, by neglecting turbulent dispersion so far, theoretical models predicted large particles to charge positively and small particles to charge negatively;
the broader the size distribution, the broader the resulting charge distribution.
We recovered these previous findings using our model but omitting aerodynamic forces (orange symbols in \cref{fig:CFDvsDEMa}).

Even monodisperse particles charge slightly bipolar, namely $\sigma(q)=42 \times 10^3 \ \epsilon$.
So this amount of bipolar charge is not related to the particle size but to the random collision sequence.
Particles that collide more frequently than the average become net acceptors of charged species, and particles that collide less frequently become net donors.
Those particles that collided more often charge in average negative, whereas those particles that collided less often charge in average positive.

For polydisperse particles, bipolar charging increases with the width of the size distribution up to the point where the geometrical properties of the smallest and largest particles limit the charge transfer \citep{YU2017113}.
For $\sigma(St)=0.0064$, the standard deviation of the charge distribution reaches its maximal value of $\sigma(q)=92 \times 10^3 \ \epsilon$.
Overall, these observations agree with the state-of-the-art assuming the particles move through a vacuum.

However, the charge distributions dramatically change if the surrounding turbulence affects the particle trajectories (blue symbols in \cref{fig:CFDvsDEM}).
For monodisperse particles, $\sigma(q)$ remains low because turbulence affects all particles in the same way.
But for polydisperse systems, turbulence suppresses bipolar charging.
The width of the charge distribution increases linearly with the width of the size distribution.
But turbulence reduces the gradient by one order of magnitude.
For the highest $\sigma(q)$, the particles charge only slightly more than the random bipolar charging of monodisperse particles.
At least up to $\sigma(St)=$ of~$0.0118$, size-dependent charging increases bipolarity only by less than 50\% compared to the charging of monodisperse particles.

Next, we checked whether this finding holds for other average Stokes numbers.
To do so, we started simulations keeping a constant polydispersity of $\sigma(St)=0.0118$.
But we changed the particles' material density to obtain different values for $\langle St \rangle$.
Figure~\ref{fig:CFDvsDEMb} confirms that turbulence suppresses bipolar charging for low average Stokes numbers. 
For the lowest given Stokes numbers, the width of the resulting charge distribution reduces nearly by half compared to particles in a vacuum.

Nevertheless, the difference between the vacuum and turbulence cases becomes smaller for an increasing average Stokes number.
A low Stokes number means the particles exchange little kinetic energy during collisions.
The Hertzian contact area reduces approximately proportional to the kinetic energy;
consequently, the particles exchange little charge.
If less charge is exchanged per collision, depleting the available TCS requires more collisions, which means the charge distributes homogeneously on the particles.
Therefore, in a vacuum, $\sigma(St)$ decreases if $\langle St \rangle$ decreases (see the orange symbols in \cref{fig:CFDvsDEMb}).
For $\langle St \rangle \to \infty$, inertia dominates drag, the particles decouple from the flow, and the effect of turbulence diminishes.
In other words, turbulence suppresses bipolar charging for $\langle St \rangle \ll \infty$.


\begin{figure}
\centering
\subfigure[]{\includegraphics[trim=0cm 0cm 0cm 0cm,clip=true,width=0.4\textwidth]{NewFigures/DNS_DEM_experimental.pdf}\label{fig:ChargeDistrib}
}
\subfigure[]{
\includegraphics[trim=0cm 0cm 0cm 0cm,clip=true,width=0.4\textwidth]{NewFigures/PSD_Stokes.pdf}\label{fig:PSDStokes}
}
\caption{
a) Resulting charge distribution in vacuum and turbulence for $\sigma(St)=0.0064$.
The insets give the charge distributions per particle size from $d_\mathrm{p}=30~\upmu$m (dark blue) to $120~\upmu$m (red).
The dashed line denotes neutral particles.
Turbulence suppresses small particles' collisions, thus, skewing the charge distribution.
Particles of $St \sim$ 1 obtain the highest positive charge.
b) Charge distributions of all polydisperse powders in turbulence when $c_\mathrm{s}= c_{\mathrm{s},0}/\mathrm{e}$. 
The inset shows that the charge of particles is independent of their polydispersity for $St <1$. 
}
\end{figure}

Figure~\ref{fig:ChargeDistrib} details the resulting charge distribution when turbulence suppresses bipolar charging the most, namely for $\langle St \rangle =$~3.86 and $\sigma(St)=0.0023$.
Vacuum leads to a wide bipolar distribution, with the smallest particles carrying the highest negative and the largest particles the highest positive charge.
Also in turbulence, the particles charge bipolar but, as discussed above, to a much lesser extent.
Moreover, in turbulence, the charge distribution changes its shape.
The smallest particles carry only a minuscule charge, whereas the particles of $St \sim 1$ obtain the highest positive charge.
The largest particles still carry the highest negative charge even though only a fraction of the charge of the same-sized particles in a vacuum.
Thus, turbulence changes bipolar charging not only quantitatively but qualitatively.

For $\langle St \rangle =3.86$, \cref{fig:PSDStokes} compares the charge distributions of powders of different polydispersity in turbulence.
The charge distribution of the particles of a Stokes number below unity is independent of the width of the size distribution. 
The motion of these particles is driven by turbulence, not by inertia.
That explains the slow, nearly plateau-like, increase of $\sigma (q)$ for $\sigma (St) \ge 0.0042$ (cf., \cref{fig:CFDvsDEMa}).
Many particles contained in the size distributions with $\sigma (St) \ge 0.0042$ have a Stokes number less than unity;
thus, their charge is independent of the total width of the size distribution (cf., \cref{PSD}).

Contrary, the charge of the particles of a Stokes number above unity in \cref{fig:PSDStokes} depend on the width of the size distribution, namely, the wider the size distribution the higher their charge.
The trajectories of these particles are driven by inertia.
Therefore, for $\sigma (St)<0.0042$, where even the smallest particles are or $St > 1$ (cf., \cref{fig:PSD}), $\sigma (q)$ increases in turbulence with $\sigma (St)$ similarly to vacuum cf., \cref{fig:CFDvsDEMa}).


\begin{figure}
\centering
\subfigure[]{
	\includegraphics[width=0.4\textwidth]{NewFigures/Dp_collision_rate_stokes.pdf}\label{fig:freqency}
}
\subfigure[]{
\includegraphics[width=0.4\textwidth]{NewFigures/CFD_DEM_lin_divergent_stokes.pdf}\label{fig:CollisionMatrix}
}
%hg all St italics
%hg y-axis would maybe be simpler if unscaled with tics at 0.1, 0.2, and 0.3
%sim: the limit would have to be 0.01 to 0.04 but this creates too much 
\caption{
Comparison of collision frequencies between particles of different sizes in turbulence and vacuum for $\langle St \rangle =$~3.86 and $\sigma(St)=0.0064$.
a)
Turbulence attenuates the collisions of small particles ($St=0.7$) with all other particles. 
The preferred collision partner of middle-sided particles ($St=2.1$) shifts from $St=0.4$ to $St=1.1$.
b)
The collision frequency ratio of each particle size shows that particles of $St < 2.0$ collide with any other particle in a vacuum more often than in turbulence.
Particles of $St \geq 2.0$ collide in turbulence more often with other particles of $St \geq 2.0$.
}
\label{Fig:Freq_con}
\end{figure}

As mentioned above, the suppression of bipolar charging by turbulence relates directly to the inter-particle collision frequencies, see \cref{Fig:Freq_con} for $\langle St \rangle=3.86$ and $\sigma(St)=0.0064$. 
In a vacuum, particles of all sizes collide with each other.
Their collision frequencies depend only on their collisional cross-section, which means large particles collide more often than small ones.
When modulated by turbulence, particles get separated according to their size.
In other words, polydispersity, which actually causes bipolarity, attenuates locally.
Then, collisions tend to take place between particles of a similar size.
This leads to a decrease in the collision frequency of the small particles ($St=0.7$ in \cref{fig:freqency}), which means that the charging events occur less frequently than for larger particles. 
Large particles ($St=2.1$ in \cref{fig:freqency}) collide in turbulence more with other particles of a similar size.

The critical particle diameter is 60~$\upmu$m to 70~$\upmu$m, corresponding to $St=$ 2.1 to 2.8.
Above this value, the collision frequency and charge distribution are monotonous for the Stokes number. %hg this sentence requires some explanation, can we see that in a figure? monotonous with respect to what?
This is the reason for the suppression of bipolar charging in turbulence because bipolar charging relies on the collisions between different-sized particles.
Since the rate of charge separation is roughly proportional to the diameter difference of the two collision partners, the lower collision frequency of the smallest and largest particles reduces the width of the charge distribution.


\begin{figure}
\centering
\subfigure[]{\includegraphics[trim=0cm 0cm 0cm 0cm,clip=true,width=0.4\textwidth]{NewFigures/Phi_vort_start.pdf}\label{fig:intialVort}}
\subfigure[]{\includegraphics[trim=0cm 0cm 0cm 0cm,clip=true,width=0.4\textwidth]{NewFigures/Phi_vort_end.pdf}\label{fig:PhivsOmega}}
\caption{Correlation of the electrical potential with the fluid's vorticity for $\langle St \rangle =$~3.86 and $\sigma(St)=0.0064$ at a) the initial stage ($t=0.2$~s) and b) close to saturation ($t=2.6$~s). %hg check numbers?
In flow regions of high vorticity, the potential is low.
High potential areas evolve as the charging proceeds in regions of low vorticity (indicated by the arrows).
}
\label{fig:phi-vort}
\end{figure}


To sum up the above discussion, due to turbulence, the preferred location and polarity of particles depend on their size.
Even though being overall electrically neutral, the spatial separation of equally polar particles leads to charged flow regions.
The formation of such regions is depicted in terms of the electric potential by the isocontours in \cref{fig:CFD1} and in \cref{fig:phi-vort}, which, because of its relation to the breakdown potential, associates with the explosion hazard of a flow.

During the initial stage of charging, when $c_s$ is still almost uniform, a negative potential forms.  
This negative potential forms in flow regions where the vorticity is as low as ($\Vert{\bm \omega}\Vert \approx 10^2$~s$^{-1}$), see \cref{fig:intialVort}.
During the later stage of charging, when the TCS start to deplete, in addition to the negative potential a more pronounced positive potential appears, see \cref{fig:PhivsOmega}.
From that on, the shape of the potential-vorticity correlation remains stable.
But as particles acquire more charge from collisions, the potential widens in the positive and negative directions, as indicated by the arrows in \cref{fig:PhivsOmega}.
%The spatial distribution of potential in bipolarly charged systems was reported in the flow direction (as a consequence of elutriation), but in our DNS simulations, we observed the radial distribution of potentials.
The areas of high absolute potential ($\vert \phi\vert>$ 0.5 V) remain exclusively in flow regions of low vorticity ($\Vert{\bm \omega}\Vert <500$~s$^{-1}$).

These pockets of high potential form in low vorticity regions because negatively-charged medium-sized particles and positively-charged large particles are expelled from areas of high vorticity.
Contrary, the smallest uncharged particles (cf., \cref{fig:ChargeDistrib} remain in areas of high vorticity, leading to regions of low absolute potential ($\vert \phi\vert<$ 0.3~V) for $\Vert{\bm \omega}\Vert <1000$~s$^{-1}$.

%hg: this paragraph is problemantic. you actually state we miss a physical mechanism in our model.
%This gives a potential gradient and forms an electric field between those two areas.
%In theory, this electric field can break the initial barrier preventing charge transfer from different charging mechanisms, like the dipole amplification model that requires initial polarization of colliding particles with an external electric field  \citep{Yoshimatsu2017,Yoshimatsu20166261}, in this case, formed by the macroscopical separation of positive and negative particles due to turbulences.
%Therefore, the surface state model can initialize charge separation required for other stronger charging mechanisms to take over the charging and build up a charge up to a discharge limit.


In conclusion, a multi-physics approach revealed a new picture of size-dependent bipolar particle charging:
turbulence drastically reduces the charge distribution's bandwidth.
Contrary to particles in a vacuum, even strongly polydisperse particles obtain only a slight bipolar charge in a turbulent flow.
Turbulence spatially separates particles according to their size, thus, reducing the local polydispersity and, consequently size-dependent charging.
Turbulence suppresses bipolar charging the most for particles of moderate Stokes numbers.
Furthermore, the resulting charge distribution changes qualitatively.
In a vacuum, the smallest particles collect the highest negative charge but in turbulence, the smallest particles remain nearly uncharged.
Instead, medium-sized particles gain the highest negative charge.
As a result, in an overall electrically neutral flow, pockets of high positive and negative electric potential form in flow regions of low vorticity. 
%This might explain the large potential gradients in the vulcanic plumes \citep{Cimarelli2022}; here, the gradients reach values that allow an atmospheric discharge in the form of lighting.
%Secondly, this helps to break the initial symmetry that prevents other charging mechanisms from participating in the charge build-up, namely the dipole amplification model \citep{Siu2014,Lacks2019} and refined condenser models used to describe the charging of technical flows \citep{Chowdhury2021104,GROSSHANS2023140918}.




% polarizability, current way of simulating charging, agglomeration --> evaluate coordination number (check close vicinity of each particle to get the distance)
%no aglomeration!!

%aggregation of charged particles in the core of the channel--> charge separation due to turbulence.
%particles aggregates in the core of the fluid

%how to quantify eddies and match them with particles?
 %vorticity and particle distribution


\section{Acknowledgement}
This project has received funding from the European Research Council~(ERC) under the European Union's Horizon 2020 research and innovation program~(grant agreement No.~947606 PowFEct).

%\section*{References}
%\bibliography{\string~/essentials/publications}
\bibliography{References}
\end{document}
