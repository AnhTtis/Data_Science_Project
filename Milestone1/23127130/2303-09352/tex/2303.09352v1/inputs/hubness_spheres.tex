
We will now show that hubness can be eliminated completely by embedding representations \emph{uniformly} on the hypersphere\footnote{Our results assume hyperspheres with unit radius, but can easily be extended to hyperspheres with arbitrary radii.}.

\definitionHypersphericalUniform
We then have the following propositions\footnote{The proofs for all propositions are included in the supplementary.} for random vectors with this PDF.

\propositionZeroMean
\propositionZeroGrad

These two propositions show that the hyperspherical uniform has
\begin{enumerate*}[label=(\roman*)]
    \item zero mean;
    and \item zero density gradient along all directions tangent to the hypersphere's surface, at all points on the hypersphere.
\end{enumerate*}
The hyperspherical uniform thus provably eliminates hubness, both in the sense of having a zero data mean, and having zero density gradient everywhere.
We note that the latter property is un-attainable in Euclidean space, as it is impossible to define a uniform distribution over the whole space.
It is therefore necessary to embed points on a non-Euclidean sub-manifold in order to eliminate hubness.

\begin{figure}
    \centering
    \includegraphics[width=0.85\columnwidth]{fig/nohub-new}
    \caption{Illustration of the \method embedding. Given representations \( \in \real^k \), $\Lalign$ preserves local similarities. $\Lunif$ simultaneously encourages uniformity in the embedding space $\SPHERE_d$. This feature embedding framework helps reduce hubness while improving classification performance.}
    \label{fig:noHub}
\end{figure}