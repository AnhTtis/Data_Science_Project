In this section we provide a theoretical analysis of \( \Lalign \) and \( \Lunif \).
Based on our analysis, we interpret these losses with regards to the Laplacian Eigenmaps algorithm and R\'enyi entropy, respectively.

\propositionConnectionLE
\propositionMaxEntropy
\definitionNormalizedCountingMeasure
\definitionSurfaceAreaMeasure
\definitionWeakStarConvergence
\propositionLunifMinimizer

\customparagraph{Interpretation of Proposition~\ref{prop:conLE}--\ref{prop:minLunif}}
    Proposition~\ref{prop:conLE} states an alternative formulation of \( \Lalign \), under the hyperspherical assumption.
    We recognize this formulation as the loss function in Laplacian Eigenmaps~\cite{belkinLaplacianEigenmaps2003}, which is known to produce \emph{local similarity-preserving} embeddings from graph data.
    When unconstrained, this loss has a trivial solution where the embeddings for all representations are equal.
    This is avoided in our case since \( \Lfinal \) (Eq.~\eqref{eq:Lfinal}) can be interpreted as the Lagrangian of minimizing \( \Lalign \) subject to a specified level of \emph{entropy}, by Proposition~\ref{prop:maxEntropy}.
    
    Finally, Proposition~\ref{prop:minLunif} states that the normalized counting measure associated with the set of points that minimize \( \Lunif \), converges to the normalized surface area measure on the sphere.
    Since \( \SPHEREUNIFORM \) is the density function associated with this measure, the points that minimize \( \Lunif\) will tend to be uniform on the sphere.
    Consequently, minimizing \( \Lalign \) also minimizes hubness, by Propositions~\ref{prop:zeroMean} and~\ref{prop:zeroGradient}.
