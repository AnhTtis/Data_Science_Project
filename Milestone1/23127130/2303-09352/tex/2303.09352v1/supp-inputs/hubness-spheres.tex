
\subsection*{Proof of Proposition~\ref{prop:zeroMean}}

    \def\SPHEREPOS{\SPHERE_d^{i, +}}
    \def\SPHERENEG{\SPHERE_d^{i, -}}
    \def\SPHEREZER{\SPHERE_d^{i, 0}}

    \begin{lemma}[Trisection of hypersphere]
        \label{lemma:sphereSymm}
        The trisection of the hypersphere along coordinate \( i \) is given by the three-tuple of disjoint sets \( (\SPHEREPOS, \SPHERENEG, \SPHEREZER) \) where
        \begin{align}
            & \SPHEREPOS = \{ \vec x = [x^1, \dots, x^d]\T \in \SPHERE_d \mid x^i > 0 \} \\
            & \SPHERENEG = \{ \vec x = [x^1, \dots, x^d]\T \in \SPHERE_d \mid x^i < 0 \} \\
            & \SPHEREZER = \{ \vec x = [x^1, \dots, x^d]\T \in \SPHERE_d \mid x^i = 0 \}
        \end{align}
        and
        \begin{align}
            \SPHEREPOS \cup \SPHERENEG \cup \SPHEREZER = \SPHERE_d
        \end{align}

        Then we have
        \begin{align}
            \SPHEREPOS = - \SPHERENEG = \{-\vec x \mid \vec x \in \SPHERENEG \}
        \end{align}
    \end{lemma}
    \begin{proof}
        Let \( \vec x \in \SPHEREPOS \), then
        \begin{align}
            || (-x) || = || x || = 1,
        \end{align}
        and
        \begin{align}
            -(x^i) < 0.
        \end{align}
        Hence \( \vec x \in -\SPHERENEG \), and \( \SPHEREPOS \subseteq - \SPHERENEG \).

        Similarly, let \( - \vec x \in -\SPHERENEG \), then
        \begin{align}
            ||-(-x)|| = ||x|| = 1,
        \end{align}
        and
        \begin{align}
            -(-x^i) = x^i > 0.
        \end{align}
        Hence \( \vec x \in \SPHEREPOS \), and \( -\SPHERENEG \subseteq \SPHEREPOS \).

        It then follows that \( \SPHEREPOS = - \SPHERENEG \).
    \end{proof}

    \begin{proof}[Proof (Proposition~\ref{prop:zeroMean})]
        The expectation \( \E (\vec X) \) is given by
        \begin{align}
            \E (\vec X) = \int_{\real^d} \vec x \SPHEREUNIFORM(\vec x) \dd \vec x
        \end{align}
        Since \( \SPHEREUNIFORM \) is non-zero only on the hypersphere \( \SPHERE_d \), the integral can be rewritten as a surface integral over \( \SPHERE_d \)
        \begin{align}
            \E(\vec X) = \int_{\SPHERE_d} \vec x A_d^{-1} \dd S.
        \end{align}
        Decomposing the integral over the trisection of \( \SPHERE_d \) along coordinate \( i \) gives
        \begin{align}
            & \int_{\SPHERE_d} \vec x A_d^{-1} \dd S =\\
            & A_d^{-1} \lrp{
                \int_{\SPHEREPOS} \vec x \dd S
                + \int_{\SPHERENEG} \vec x \dd S
                + \int_{\SPHEREZER} \vec x \dd S
            }.
        \end{align}
        By Lemma~\ref{lemma:sphereSymm} we have
        \begin{align}
            \SPHEREPOS = - \SPHERENEG \Rightarrow \int_{\SPHEREPOS} \vec x \dd S = - \int_{\SPHERENEG} \vec x \dd S.
        \end{align}
        Furthermore, since the set \( \SPHEREZER \) has zero width along coordinate \( i \), \( \int_{\SPHEREZER} \vec x \dd S = 0 \).
        Hence
        \begin{align}
            &\E(\vec X) = \\
            & A_d^{-1} \lrp{
                \int_{\SPHEREPOS} \vec x \dd S
                - \int_{\SPHEREPOS} \vec x \dd S
                + \int_{\SPHEREZER} \vec x \dd S
            } = 0
        \end{align}
    \end{proof}

\subsection*{Proof of Proposition~\ref{prop:zeroGradient}}
    \begin{proof}
        \( \SPHEREUNIFORM(\vec x) \) can be written in polar coordinates as
        \begin{align}
            \SPHEREUNIFORM(\vec x(r, \vec \theta)) = \SPHEREUNIFORM^{\text{Polar}}(r, \vec\theta) = A_d^{-1}\delta(r - 1)
        \end{align}
        The gradient of \( \SPHEREUNIFORM^{\text{Polar}}(r, \vec\theta) \) is then
        \begin{align}
            \nabla_{(r, \vec \theta)} \SPHEREUNIFORM^{\text{Polar}}(r, \vec\theta) =
            \begin{bmatrix}
                \frac{\partial}{\partial r} \SPHEREUNIFORM^{\text{Polar}}(r, \vec\theta) \\[0.2cm]
                0 \\
                \vdots \\
                0
            \end{bmatrix}
        \end{align}

    For an arbitrary point \( \vec p \in \SPHERE_d \), an arbitrary unit vector (direction), \( \vec \theta^{*} \), in the tangent plane \( \Pi_{\vec p} \) is given by
    \begin{align}
        \vec \theta^{*} =
        \begin{bmatrix}
            0\\
            \theta^{*}_1 \\
            \vdots \\
            \theta^{*}_{d-1}
        \end{bmatrix}
    \end{align}
    The directional derivative of \( \SPHEREUNIFORM(\vec x) \) along \( \vec \theta^{*} \) is then
    \begin{align}
        \nabla_{ \vec \theta^{*}} \SPHEREUNIFORM(\vec x) =
        \begin{bmatrix}
            \frac{\partial}{\partial r} \SPHEREUNIFORM^{\text{Polar}}(r, \vec\theta) \\[0.2cm]
            0 \\
            \vdots \\
            0
        \end{bmatrix}\T
        \cdot
        \begin{bmatrix}
            0\\
            \theta^{*}_1 \\
            \vdots \\
            \theta^{*}_{d-1}
        \end{bmatrix}
        = 0
    \end{align}
\end{proof}