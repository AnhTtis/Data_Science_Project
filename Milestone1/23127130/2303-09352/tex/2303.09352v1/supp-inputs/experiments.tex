\begin{figure}[t]
    \centering
    \begin{subfigure}{0.99\columnwidth}
        \centering
        \includegraphics[width=0.7\textwidth]{fig/kappa}
    \end{subfigure}
    \begin{subfigure}{0.99\columnwidth}
        \centering
        \includegraphics[width=0.7\textwidth]{fig/epsilon}
    \end{subfigure}
    \caption{Accuracy for different values for \( \kappa \) and \( \epsilon \). Neither \method nor \methodS are particularly sensitive the the choice of these parameters.}
    \label{fig:kappaEpsilon}
\end{figure}

\subsection{Implementation details}
    This section covers the additional implementation details not provided in the main paper.
    These include the initialization of the embeddings in Algorithm 1, hyperparameters, additional transformations wherever required, the architectures used, and a note on accessing the code, datasets, and dataset splits.

    \customparagraph{Initialization and normalization}
        Instead of a random initialization of our embeddings $Z_0$, we follow a PCA based initialization, as in~\cite{Maaten}.
        The weights are computed using the cached features from the base classes, the support and query features are then transformed using these weights.
        This procedure is also fast as we do not need to compute the PCA weights on every episode.
        To ensure that the resulting features lie on the hypersphere after each gradient update in \method and \methodS, we re-normalize the embeddings using L2 normalization.
    
    \customparagraph{Hyperparameters}
        \method and \methodS have the following hyperparameters.
        \begin{itemize}
            \item \( P \) -- perplexity for computing the \( \kappa_i \).
            \item \( T \) -- number of iterations.
            \item \( \alpha \) -- tradeoff parameter in the loss (\( \Lfinal = \alpha\Lalign + (1-\alpha) \Lunif \)).
            \item \( \eta \) -- learning rate for the Adam optimizer.
            \item \( \kappa \) -- concentration parameter for the embeddings.
            \item \( \epsilon \) -- exaggeration of similarities between supports from different classes.
            \item \( d \) -- dimensionality of embeddings.
        \end{itemize}
        All hyperparameter values used in in \method and \methodS are given in Table~\ref{tab:hyperparameters}

        \begin{table*}
            \centering
            \small
            \begin{table}
    \centering
    \caption{
        \textbf{Hyperparameters for \datavec{} and \name{}.}
        \emakefirstuc{\datavec{}} denotes our adaptation of data2vec to the point cloud modality.
        We report the best performing hyperparameters for both \datavec{} and \name{}.
        LN: layer normalization. AVG: average pooling over layers.
    }
    \label{tab:hyperparameters}
    \begin{tabular}{lll}
        \toprule
        & \textbf{\datavec}                   & \textbf{point2vec}                        \\
        \midrule
        Steps                   & $800$ epochs                               & $800$ epochs                                \\
        Optimizer                                          & AdamW                                     & AdamW                                     \\
        Learning rate        & $2 \times 10^{-3}$                                & $1 \times 10^{-3}$                                \\
        Weight decay        & $0.05$                                      & $0.05$                                      \\
        LR Schedule  & cosine                                    & cosine                                    \\
        LR Warm-Up       & $80$ epochs                       & $80$ epochs                       \\
        Batch size            & $2048$                                      & $512$                              \\
        Encoder layers         & $12$                                        & $12$                                        \\
        Encoder dimension       & $384$                                       & $384$                                       \\
        Decoder layers           & --                                         & $4$                                         \\
        Masking strategy            & random                                    & random                                    \\
        Masking ratio                       & $65\%$ & $65\%$ \\
        \arrayrulecolor{black!10}\midrule\arrayrulecolor{black}
        $\tau_0$ (EMA start)  & $0.9998$                            & $0.9998$                              \\
        $\tau_e$ (EMA end)     & $0.99999$     & $0.99999$     \\
        $\tau_n$ (EMA warm-up)  & $200$ epochs                     & $200$ epochs                       \\
        $K$ (layers to average) & $6$                                         & $6$                                         \\
        Target normalization &     \small LN$\rightarrow$AVG$\rightarrow$LN         & \small LN$\rightarrow$AVG$\rightarrow$LN         \\
        \bottomrule
    \end{tabular}
\end{table}
            \caption{Hyperparameter values used in our experiments.}
            \label{tab:hyperparameters}
        \end{table*}

    \customparagraph{Code}
        The code for our experiments is available at: \githubLink

    \customparagraph{Data splits}
        Details to access the datasets used with the requisite splits (both are consistent with~\cite{veilleuxRealisticEvaluationTransductive2021}) are available in the code repository.
    
    \customparagraph{Base feature extractors}
        \begin{itemize}
            \item \textbf{Resnet-18}: As in~\cite{boudiafTransductiveInformationMaximization2020,veilleuxRealisticEvaluationTransductive2021}, we use the  weights from~\cite{veilleuxRealisticEvaluationTransductive2021}.
                The model is trained using a cross-entropy loss on the base classes.
            \item \textbf{WideRes28-10}: Following~\cite{manglaChartingRightManifold2020,zhuEASEUnsupervisedDiscriminant2022}, we use the weights from~\cite{manglaChartingRightManifold2020}.
                The model is pre-trained using a combination of cross-entropy and rotation prediction~\cite{gidarisUnsupervisedRepresentationLearning2018}, and then fine-tuned with Manifold Mixup~\cite{vermaManifoldMixupBetter2019}.
        \end{itemize}

\subsection{Results}
    \customparagraph{FSL performance}
        The complete lists of accuracies and hubness metrics for all embeddings, classifiers, and feature extractors, are given in Tables~\ref{tab:main-tim-resnet18-1},~\ref{tab:main-s2m2-wrn-s2m2-1},~\ref{tab:main-tim-resnet18-5}, and~\ref{tab:main-s2m2-wrn-s2m2-5}.
        The exhaustive results in these tables form the basis of Table 1, Table 2 and Table 3 in the main text.
        The two proposed approaches consistently outperform prior embeddings across several classifiers, feature extractors and datasets.

    \customparagraph{Effect of the \( \kappa \) and \( \epsilon \) hyperparameters}
        The plots in Figure~\ref{fig:kappaEpsilon} show accuracy on \textit{tiered} \(5\)-shot with SIAMESE for increasing \( \kappa \) and \( \varepsilon \).
        Neither method is particularly sensitive to the choice of \( \kappa \) and \( \varepsilon \), and \methodS is less sensitive to variations in \( \kappa \), than \method.
        Choosing \( \kappa \in [0.5, 1] \) and \(\epsilon \in [3, 20] \) will result in high classification accuracy

    \begin{table*}
        {\scriptsize\centering\begin{tabular}{llllllllllll}
\toprule
 &  &  & \multicolumn{3}{c}{mini} & \multicolumn{3}{c}{tiered} & \multicolumn{3}{c}{CUB} \\
 &  &  & Acc & Skew & Hub.~Occ. & Acc & Skew & Hub.~Occ. & Acc & Skew & Hub.~Occ. \\
Arch. & Clf. & Emb. &  &  &  &  &  &  &  &  &  \\
\midrule
\multirow[c]{54}{*}{\rotatebox{90}{ResNet18}} & \multirow[c]{9}{*}{\rotatebox{90}{ILPC}} & None & \MTCWITHCONF{64.07}{0.28} & \MTCWITHCONF{1.411}{0.01} & \MTCWITHCONF{0.408}{0.001} & \MTCWITHCONF{75.5}{0.28} & \MTCWITHCONF{1.213}{0.009} & \MTCWITHCONF{0.41}{0.001} & \MTCWITHCONF{76.06}{0.27} & \MTCWITHCONF{0.886}{0.006} & \MTCWITHCONF{0.34}{0.001} \\
 &  & L2 & \MTCWITHCONF{69.28}{0.27} & \MTCWITHCONF{0.966}{0.007} & \MTCWITHCONF{0.298}{0.001} & \MTCWITHCONF{77.84}{0.28} & \MTCWITHCONF{0.811}{0.007} & \MTCWITHCONF{0.267}{0.001} & \MTCWITHCONF{79.91}{0.26} & \MTCWITHCONF{0.688}{0.006} & \MTCWITHCONF{0.236}{0.001} \\
 &  & CL2 & \MTCWITHCONF{71.48}{0.27} & \MTCWITHCONF{0.661}{0.005} & \MTCWITHCONF{0.229}{0.001} & \MTCWITHCONF{79.8}{0.27} & \MTCWITHCONF{0.679}{0.006} & \MTCWITHCONF{0.249}{0.001} & \MTCWITHCONF{80.97}{0.26} & \MTCWITHCONF{0.553}{0.005} & \MTCWITHCONF{0.203}{0.001} \\
 &  & ZN & \MTCWITHCONF{71.48}{0.27} & \MTCWITHCONF{0.677}{0.006} & \MTCWITHCONF{0.227}{0.001} & \MTCWITHCONF{79.95}{0.27} & \MTCWITHCONF{0.694}{0.006} & \MTCWITHCONF{0.263}{0.001} & \MTCWITHCONF{81.49}{0.25} & \MTCWITHCONF{0.57}{0.005} & \MTCWITHCONF{0.217}{0.001} \\
 &  & ReRep & \MTCWITHCONF{65.49}{0.28} & \MTCWITHCONF{3.688}{0.007} & \MTCWITHCONF{0.559}{0.001} & \MTCWITHCONF{76.75}{0.28} & \MTCWITHCONF{3.61}{0.01} & \MTCWITHCONF{0.55}{0.001} & \MTCWITHCONF{77.73}{0.26} & \MTCWITHCONF{3.563}{0.007} & \MTCWITHCONF{0.512}{0.001} \\
 &  & EASE & \MTCWITHCONF{71.79}{0.28} & \MTCWITHCONF{0.515}{0.005} & \MTCWITHCONF{0.157}{0.001} & \MTCWITHCONF{80.2}{0.27} & \MTCWITHCONF{0.48}{0.005} & \MTCWITHCONF{0.158}{0.001} & \MTCWITHCONF{81.88}{0.25} & \MTCWITHCONF{0.463}{0.004} & \SECONDBEST{\MTCWITHCONF{0.153}{0.001}} \\
 &  & TCPR & \MTCWITHCONF{71.77}{0.28} & \MTCWITHCONF{0.647}{0.005} & \MTCWITHCONF{0.223}{0.001} & \MTCWITHCONF{80.01}{0.28} & \MTCWITHCONF{0.652}{0.006} & \MTCWITHCONF{0.249}{0.001} & \MTCWITHCONF{81.75}{0.25} & \MTCWITHCONF{0.534}{0.004} & \MTCWITHCONF{0.203}{0.001} \\
 &  & noHub & \SECONDBEST{\MTCWITHCONF{73.18}{0.28}} & \SECONDBEST{\MTCWITHCONF{0.308}{0.005}} & \BEST{\MTCWITHCONF{0.094}{0.001}} & \SECONDBEST{\MTCWITHCONF{80.76}{0.28}} & \SECONDBEST{\MTCWITHCONF{0.296}{0.004}} & \BEST{\MTCWITHCONF{0.101}{0.001}} & \SECONDBEST{\MTCWITHCONF{82.74}{0.26}} & \SECONDBEST{\MTCWITHCONF{0.32}{0.004}} & \BEST{\MTCWITHCONF{0.112}{0.001}} \\
 &  & noHub-S & \BEST{\MTCWITHCONF{74.02}{0.28}} & \BEST{\MTCWITHCONF{0.276}{0.004}} & \SECONDBEST{\MTCWITHCONF{0.13}{0.001}} & \BEST{\MTCWITHCONF{81.34}{0.27}} & \BEST{\MTCWITHCONF{0.281}{0.004}} & \SECONDBEST{\MTCWITHCONF{0.127}{0.001}} & \BEST{\MTCWITHCONF{83.92}{0.25}} & \BEST{\MTCWITHCONF{0.296}{0.003}} & \MTCWITHCONF{0.163}{0.001} \\
\cline{2-12}
 & \multirow[c]{9}{*}{\rotatebox{90}{LaplacianShot}} & None & \MTCWITHCONF{68.92}{0.23} & \MTCWITHCONF{1.341}{0.009} & \MTCWITHCONF{0.408}{0.001} & \MTCWITHCONF{76.43}{0.25} & \MTCWITHCONF{1.214}{0.009} & \MTCWITHCONF{0.41}{0.001} & \MTCWITHCONF{79.17}{0.23} & \MTCWITHCONF{0.887}{0.006} & \MTCWITHCONF{0.34}{0.001} \\
 &  & L2 & \MTCWITHCONF{69.3}{0.23} & \MTCWITHCONF{0.945}{0.007} & \MTCWITHCONF{0.302}{0.001} & \MTCWITHCONF{77.2}{0.25} & \MTCWITHCONF{0.808}{0.007} & \MTCWITHCONF{0.265}{0.001} & \MTCWITHCONF{79.65}{0.23} & \MTCWITHCONF{0.682}{0.006} & \MTCWITHCONF{0.236}{0.001} \\
 &  & CL2 & \MTCWITHCONF{70.68}{0.23} & \MTCWITHCONF{0.661}{0.005} & \MTCWITHCONF{0.231}{0.001} & \MTCWITHCONF{77.98}{0.24} & \MTCWITHCONF{0.689}{0.006} & \MTCWITHCONF{0.248}{0.001} & \MTCWITHCONF{79.99}{0.22} & \MTCWITHCONF{0.547}{0.005} & \MTCWITHCONF{0.201}{0.001} \\
 &  & ZN & \MTCWITHCONF{70.51}{0.23} & \MTCWITHCONF{0.688}{0.006} & \MTCWITHCONF{0.233}{0.001} & \MTCWITHCONF{77.51}{0.24} & \MTCWITHCONF{0.697}{0.006} & \MTCWITHCONF{0.264}{0.001} & \MTCWITHCONF{79.86}{0.22} & \MTCWITHCONF{0.564}{0.005} & \MTCWITHCONF{0.217}{0.001} \\
 &  & ReRep & \MTCWITHCONF{72.75}{0.24} & \MTCWITHCONF{3.653}{0.007} & \MTCWITHCONF{0.548}{0.001} & \MTCWITHCONF{78.95}{0.25} & \MTCWITHCONF{3.605}{0.011} & \MTCWITHCONF{0.549}{0.001} & \MTCWITHCONF{82.38}{0.22} & \MTCWITHCONF{3.565}{0.007} & \MTCWITHCONF{0.512}{0.001} \\
 &  & EASE & \MTCWITHCONF{72.19}{0.23} & \MTCWITHCONF{0.526}{0.005} & \MTCWITHCONF{0.161}{0.001} & \MTCWITHCONF{79.34}{0.24} & \MTCWITHCONF{0.481}{0.005} & \MTCWITHCONF{0.158}{0.001} & \MTCWITHCONF{81.5}{0.22} & \MTCWITHCONF{0.459}{0.004} & \SECONDBEST{\MTCWITHCONF{0.152}{0.001}} \\
 &  & TCPR & \MTCWITHCONF{71.79}{0.23} & \MTCWITHCONF{0.654}{0.005} & \MTCWITHCONF{0.228}{0.001} & \MTCWITHCONF{78.41}{0.24} & \MTCWITHCONF{0.651}{0.005} & \MTCWITHCONF{0.249}{0.001} & \MTCWITHCONF{80.86}{0.22} & \MTCWITHCONF{0.537}{0.004} & \MTCWITHCONF{0.203}{0.001} \\
 &  & noHub & \SECONDBEST{\MTCWITHCONF{73.63}{0.25}} & \SECONDBEST{\MTCWITHCONF{0.305}{0.005}} & \BEST{\MTCWITHCONF{0.094}{0.001}} & \BEST{\MTCWITHCONF{80.84}{0.25}} & \SECONDBEST{\MTCWITHCONF{0.3}{0.005}} & \BEST{\MTCWITHCONF{0.101}{0.001}} & \SECONDBEST{\MTCWITHCONF{83.23}{0.22}} & \SECONDBEST{\MTCWITHCONF{0.318}{0.004}} & \BEST{\MTCWITHCONF{0.112}{0.001}} \\
 &  & noHub-S & \BEST{\MTCWITHCONF{73.79}{0.25}} & \BEST{\MTCWITHCONF{0.276}{0.004}} & \SECONDBEST{\MTCWITHCONF{0.13}{0.001}} & \SECONDBEST{\MTCWITHCONF{80.83}{0.25}} & \BEST{\MTCWITHCONF{0.275}{0.004}} & \SECONDBEST{\MTCWITHCONF{0.125}{0.001}} & \BEST{\MTCWITHCONF{83.47}{0.22}} & \BEST{\MTCWITHCONF{0.299}{0.003}} & \MTCWITHCONF{0.164}{0.001} \\
\cline{2-12}
 & \multirow[c]{9}{*}{\rotatebox{90}{ObliqueManifold}} & None & \MTCWITHCONF{68.89}{0.23} & \MTCWITHCONF{1.412}{0.01} & \MTCWITHCONF{0.407}{0.001} & \MTCWITHCONF{77.07}{0.25} & \MTCWITHCONF{1.21}{0.009} & \MTCWITHCONF{0.409}{0.001} & \MTCWITHCONF{79.4}{0.22} & \MTCWITHCONF{0.887}{0.006} & \MTCWITHCONF{0.341}{0.001} \\
 &  & L2 & \MTCWITHCONF{68.92}{0.23} & \MTCWITHCONF{0.964}{0.007} & \MTCWITHCONF{0.299}{0.001} & \MTCWITHCONF{77.17}{0.25} & \MTCWITHCONF{0.806}{0.007} & \MTCWITHCONF{0.266}{0.001} & \MTCWITHCONF{79.32}{0.22} & \MTCWITHCONF{0.691}{0.006} & \MTCWITHCONF{0.237}{0.001} \\
 &  & CL2 & \MTCWITHCONF{70.86}{0.24} & \MTCWITHCONF{0.66}{0.005} & \MTCWITHCONF{0.228}{0.001} & \MTCWITHCONF{78.92}{0.25} & \MTCWITHCONF{0.68}{0.006} & \MTCWITHCONF{0.249}{0.001} & \MTCWITHCONF{80.29}{0.23} & \MTCWITHCONF{0.547}{0.005} & \MTCWITHCONF{0.202}{0.001} \\
 &  & ZN & \MTCWITHCONF{71.25}{0.24} & \MTCWITHCONF{0.679}{0.006} & \MTCWITHCONF{0.227}{0.001} & \MTCWITHCONF{79.54}{0.25} & \MTCWITHCONF{0.697}{0.006} & \MTCWITHCONF{0.263}{0.001} & \MTCWITHCONF{81.38}{0.23} & \MTCWITHCONF{0.562}{0.005} & \MTCWITHCONF{0.216}{0.001} \\
 &  & ReRep & \SECONDBEST{\MTCWITHCONF{73.3}{0.25}} & \MTCWITHCONF{3.682}{0.007} & \MTCWITHCONF{0.559}{0.001} & \SECONDBEST{\MTCWITHCONF{80.26}{0.26}} & \MTCWITHCONF{3.608}{0.01} & \MTCWITHCONF{0.551}{0.001} & \BEST{\MTCWITHCONF{83.84}{0.23}} & \MTCWITHCONF{3.559}{0.008} & \MTCWITHCONF{0.513}{0.001} \\
 &  & EASE & \MTCWITHCONF{68.4}{0.24} & \MTCWITHCONF{0.516}{0.005} & \MTCWITHCONF{0.156}{0.001} & \MTCWITHCONF{77.33}{0.25} & \MTCWITHCONF{0.477}{0.004} & \MTCWITHCONF{0.158}{0.001} & \MTCWITHCONF{79.03}{0.24} & \MTCWITHCONF{0.461}{0.004} & \SECONDBEST{\MTCWITHCONF{0.152}{0.001}} \\
 &  & TCPR & \MTCWITHCONF{70.74}{0.24} & \MTCWITHCONF{0.646}{0.005} & \MTCWITHCONF{0.223}{0.001} & \MTCWITHCONF{78.92}{0.25} & \MTCWITHCONF{0.649}{0.005} & \MTCWITHCONF{0.249}{0.001} & \MTCWITHCONF{80.18}{0.23} & \MTCWITHCONF{0.537}{0.004} & \MTCWITHCONF{0.204}{0.001} \\
 &  & noHub & \MTCWITHCONF{72.55}{0.26} & \SECONDBEST{\MTCWITHCONF{0.309}{0.005}} & \BEST{\MTCWITHCONF{0.095}{0.001}} & \MTCWITHCONF{79.97}{0.26} & \SECONDBEST{\MTCWITHCONF{0.302}{0.005}} & \BEST{\MTCWITHCONF{0.102}{0.001}} & \MTCWITHCONF{82.21}{0.24} & \SECONDBEST{\MTCWITHCONF{0.319}{0.004}} & \BEST{\MTCWITHCONF{0.112}{0.001}} \\
 &  & noHub-S & \BEST{\MTCWITHCONF{74.24}{0.26}} & \BEST{\MTCWITHCONF{0.274}{0.004}} & \SECONDBEST{\MTCWITHCONF{0.13}{0.001}} & \BEST{\MTCWITHCONF{80.84}{0.26}} & \BEST{\MTCWITHCONF{0.282}{0.004}} & \SECONDBEST{\MTCWITHCONF{0.127}{0.001}} & \SECONDBEST{\MTCWITHCONF{83.67}{0.23}} & \BEST{\MTCWITHCONF{0.294}{0.003}} & \MTCWITHCONF{0.162}{0.001} \\
\cline{2-12}
 & \multirow[c]{9}{*}{\rotatebox{90}{SIAMESE}} & None & \MTCWITHCONF{20.0}{0.0} & \MTCWITHCONF{1.345}{0.009} & \MTCWITHCONF{0.407}{0.001} & \MTCWITHCONF{20.0}{0.0} & \MTCWITHCONF{1.222}{0.009} & \MTCWITHCONF{0.41}{0.001} & \MTCWITHCONF{20.0}{0.0} & \MTCWITHCONF{0.885}{0.006} & \MTCWITHCONF{0.339}{0.001} \\
 &  & L2 & \MTCWITHCONF{73.77}{0.24} & \MTCWITHCONF{0.949}{0.007} & \MTCWITHCONF{0.301}{0.001} & \MTCWITHCONF{80.46}{0.26} & \MTCWITHCONF{0.811}{0.007} & \MTCWITHCONF{0.265}{0.001} & \MTCWITHCONF{83.1}{0.23} & \MTCWITHCONF{0.691}{0.006} & \MTCWITHCONF{0.237}{0.001} \\
 &  & CL2 & \MTCWITHCONF{75.56}{0.26} & \MTCWITHCONF{0.666}{0.005} & \MTCWITHCONF{0.232}{0.001} & \MTCWITHCONF{82.1}{0.26} & \MTCWITHCONF{0.68}{0.006} & \MTCWITHCONF{0.248}{0.001} & \MTCWITHCONF{84.35}{0.24} & \MTCWITHCONF{0.549}{0.005} & \MTCWITHCONF{0.201}{0.001} \\
 &  & ZN & \MTCWITHCONF{20.0}{0.0} & \MTCWITHCONF{0.686}{0.006} & \MTCWITHCONF{0.232}{0.001} & \MTCWITHCONF{20.0}{0.0} & \MTCWITHCONF{0.69}{0.006} & \MTCWITHCONF{0.262}{0.001} & \MTCWITHCONF{20.0}{0.0} & \MTCWITHCONF{0.565}{0.005} & \MTCWITHCONF{0.217}{0.001} \\
 &  & ReRep & \MTCWITHCONF{20.0}{0.0} & \MTCWITHCONF{3.653}{0.007} & \MTCWITHCONF{0.549}{0.001} & \MTCWITHCONF{20.0}{0.0} & \MTCWITHCONF{3.616}{0.01} & \MTCWITHCONF{0.549}{0.001} & \MTCWITHCONF{20.0}{0.0} & \MTCWITHCONF{3.559}{0.007} & \MTCWITHCONF{0.512}{0.001} \\
 &  & EASE & \MTCWITHCONF{76.05}{0.27} & \MTCWITHCONF{0.529}{0.005} & \MTCWITHCONF{0.162}{0.001} & \MTCWITHCONF{82.57}{0.27} & \MTCWITHCONF{0.485}{0.005} & \MTCWITHCONF{0.159}{0.001} & \MTCWITHCONF{85.24}{0.24} & \MTCWITHCONF{0.464}{0.004} & \SECONDBEST{\MTCWITHCONF{0.153}{0.001}} \\
 &  & TCPR & \MTCWITHCONF{75.99}{0.26} & \MTCWITHCONF{0.655}{0.005} & \MTCWITHCONF{0.227}{0.001} & \MTCWITHCONF{82.65}{0.26} & \MTCWITHCONF{0.651}{0.005} & \MTCWITHCONF{0.249}{0.001} & \MTCWITHCONF{85.34}{0.23} & \MTCWITHCONF{0.535}{0.004} & \MTCWITHCONF{0.203}{0.001} \\
 &  & noHub & \SECONDBEST{\MTCWITHCONF{76.65}{0.28}} & \SECONDBEST{\MTCWITHCONF{0.308}{0.005}} & \BEST{\MTCWITHCONF{0.095}{0.001}} & \SECONDBEST{\MTCWITHCONF{82.94}{0.27}} & \SECONDBEST{\MTCWITHCONF{0.303}{0.004}} & \BEST{\MTCWITHCONF{0.101}{0.001}} & \BEST{\MTCWITHCONF{85.88}{0.24}} & \SECONDBEST{\MTCWITHCONF{0.322}{0.004}} & \BEST{\MTCWITHCONF{0.112}{0.001}} \\
 &  & noHub-S & \BEST{\MTCWITHCONF{76.68}{0.28}} & \BEST{\MTCWITHCONF{0.275}{0.004}} & \SECONDBEST{\MTCWITHCONF{0.13}{0.001}} & \BEST{\MTCWITHCONF{83.09}{0.27}} & \BEST{\MTCWITHCONF{0.281}{0.004}} & \SECONDBEST{\MTCWITHCONF{0.128}{0.001}} & \SECONDBEST{\MTCWITHCONF{85.81}{0.24}} & \BEST{\MTCWITHCONF{0.295}{0.003}} & \MTCWITHCONF{0.161}{0.001} \\
\cline{2-12}
 & \multirow[c]{9}{*}{\rotatebox{90}{SimpleShot}} & None & \MTCWITHCONF{56.14}{0.2} & \MTCWITHCONF{1.349}{0.009} & \MTCWITHCONF{0.407}{0.001} & \MTCWITHCONF{63.34}{0.23} & \MTCWITHCONF{1.211}{0.009} & \MTCWITHCONF{0.408}{0.001} & \MTCWITHCONF{64.02}{0.21} & \MTCWITHCONF{0.887}{0.006} & \MTCWITHCONF{0.341}{0.001} \\
 &  & L2 & \MTCWITHCONF{60.15}{0.2} & \MTCWITHCONF{0.937}{0.007} & \MTCWITHCONF{0.301}{0.001} & \MTCWITHCONF{68.02}{0.23} & \MTCWITHCONF{0.812}{0.007} & \MTCWITHCONF{0.265}{0.001} & \MTCWITHCONF{69.05}{0.21} & \MTCWITHCONF{0.691}{0.006} & \MTCWITHCONF{0.236}{0.001} \\
 &  & CL2 & \MTCWITHCONF{63.1}{0.2} & \MTCWITHCONF{0.667}{0.005} & \MTCWITHCONF{0.233}{0.001} & \MTCWITHCONF{69.76}{0.22} & \MTCWITHCONF{0.679}{0.006} & \MTCWITHCONF{0.249}{0.001} & \MTCWITHCONF{70.16}{0.2} & \MTCWITHCONF{0.549}{0.005} & \MTCWITHCONF{0.201}{0.001} \\
 &  & ZN & \MTCWITHCONF{63.39}{0.2} & \MTCWITHCONF{0.68}{0.005} & \MTCWITHCONF{0.231}{0.001} & \MTCWITHCONF{70.04}{0.22} & \MTCWITHCONF{0.698}{0.006} & \MTCWITHCONF{0.264}{0.001} & \MTCWITHCONF{71.03}{0.2} & \MTCWITHCONF{0.564}{0.005} & \MTCWITHCONF{0.216}{0.001} \\
 &  & ReRep & \MTCWITHCONF{66.66}{0.22} & \MTCWITHCONF{3.655}{0.007} & \MTCWITHCONF{0.548}{0.001} & \MTCWITHCONF{73.23}{0.23} & \MTCWITHCONF{3.604}{0.01} & \MTCWITHCONF{0.549}{0.001} & \MTCWITHCONF{76.8}{0.21} & \MTCWITHCONF{3.565}{0.007} & \MTCWITHCONF{0.513}{0.001} \\
 &  & EASE & \MTCWITHCONF{64.0}{0.2} & \MTCWITHCONF{0.521}{0.005} & \MTCWITHCONF{0.16}{0.001} & \MTCWITHCONF{71.0}{0.21} & \MTCWITHCONF{0.479}{0.005} & \MTCWITHCONF{0.158}{0.001} & \MTCWITHCONF{72.38}{0.2} & \MTCWITHCONF{0.466}{0.004} & \SECONDBEST{\MTCWITHCONF{0.153}{0.001}} \\
 &  & TCPR & \MTCWITHCONF{63.33}{0.2} & \MTCWITHCONF{0.651}{0.005} & \MTCWITHCONF{0.228}{0.001} & \MTCWITHCONF{69.82}{0.22} & \MTCWITHCONF{0.65}{0.005} & \MTCWITHCONF{0.25}{0.001} & \MTCWITHCONF{70.75}{0.2} & \MTCWITHCONF{0.532}{0.004} & \MTCWITHCONF{0.204}{0.001} \\
 &  & noHub & \SECONDBEST{\MTCWITHCONF{69.38}{0.22}} & \SECONDBEST{\MTCWITHCONF{0.315}{0.005}} & \BEST{\MTCWITHCONF{0.095}{0.001}} & \SECONDBEST{\MTCWITHCONF{76.72}{0.23}} & \SECONDBEST{\MTCWITHCONF{0.303}{0.004}} & \BEST{\MTCWITHCONF{0.102}{0.001}} & \SECONDBEST{\MTCWITHCONF{78.21}{0.21}} & \SECONDBEST{\MTCWITHCONF{0.32}{0.004}} & \BEST{\MTCWITHCONF{0.112}{0.001}} \\
 &  & noHub-S & \BEST{\MTCWITHCONF{71.1}{0.22}} & \BEST{\MTCWITHCONF{0.276}{0.004}} & \SECONDBEST{\MTCWITHCONF{0.13}{0.001}} & \BEST{\MTCWITHCONF{78.35}{0.23}} & \BEST{\MTCWITHCONF{0.283}{0.004}} & \SECONDBEST{\MTCWITHCONF{0.127}{0.001}} & \BEST{\MTCWITHCONF{80.31}{0.21}} & \BEST{\MTCWITHCONF{0.296}{0.003}} & \MTCWITHCONF{0.162}{0.001} \\
\cline{2-12}
 & \multirow[c]{9}{*}{\rotatebox{90}{\( \alpha \)-TIM}} & None & \MTCWITHCONF{56.39}{0.2} & \MTCWITHCONF{1.342}{0.009} & \MTCWITHCONF{0.406}{0.001} & \MTCWITHCONF{63.32}{0.23} & \MTCWITHCONF{1.216}{0.009} & \MTCWITHCONF{0.411}{0.001} & \MTCWITHCONF{64.02}{0.22} & \MTCWITHCONF{0.886}{0.006} & \MTCWITHCONF{0.341}{0.001} \\
 &  & L2 & \MTCWITHCONF{67.91}{0.23} & \MTCWITHCONF{0.942}{0.007} & \MTCWITHCONF{0.301}{0.001} & \MTCWITHCONF{74.94}{0.24} & \MTCWITHCONF{0.814}{0.007} & \MTCWITHCONF{0.266}{0.001} & \MTCWITHCONF{77.49}{0.23} & \MTCWITHCONF{0.694}{0.006} & \MTCWITHCONF{0.236}{0.001} \\
 &  & CL2 & \MTCWITHCONF{65.68}{0.21} & \MTCWITHCONF{0.665}{0.005} & \MTCWITHCONF{0.232}{0.001} & \MTCWITHCONF{73.23}{0.23} & \MTCWITHCONF{0.681}{0.006} & \MTCWITHCONF{0.248}{0.001} & \MTCWITHCONF{73.79}{0.21} & \MTCWITHCONF{0.552}{0.005} & \MTCWITHCONF{0.202}{0.001} \\
 &  & ZN & \MTCWITHCONF{63.36}{0.2} & \MTCWITHCONF{0.682}{0.005} & \MTCWITHCONF{0.232}{0.001} & \MTCWITHCONF{70.19}{0.22} & \MTCWITHCONF{0.693}{0.006} & \MTCWITHCONF{0.263}{0.001} & \MTCWITHCONF{70.85}{0.21} & \MTCWITHCONF{0.566}{0.005} & \MTCWITHCONF{0.215}{0.001} \\
 &  & ReRep & \MTCWITHCONF{66.37}{0.22} & \MTCWITHCONF{3.656}{0.007} & \MTCWITHCONF{0.55}{0.001} & \MTCWITHCONF{73.24}{0.24} & \MTCWITHCONF{3.605}{0.011} & \MTCWITHCONF{0.55}{0.001} & \MTCWITHCONF{76.86}{0.21} & \MTCWITHCONF{3.555}{0.007} & \MTCWITHCONF{0.514}{0.001} \\
 &  & EASE & \MTCWITHCONF{65.32}{0.2} & \MTCWITHCONF{0.526}{0.005} & \MTCWITHCONF{0.163}{0.001} & \MTCWITHCONF{71.88}{0.22} & \MTCWITHCONF{0.477}{0.005} & \MTCWITHCONF{0.158}{0.001} & \MTCWITHCONF{73.03}{0.21} & \MTCWITHCONF{0.459}{0.004} & \SECONDBEST{\MTCWITHCONF{0.151}{0.001}} \\
 &  & TCPR & \MTCWITHCONF{66.19}{0.21} & \MTCWITHCONF{0.65}{0.005} & \MTCWITHCONF{0.227}{0.001} & \MTCWITHCONF{73.24}{0.23} & \MTCWITHCONF{0.649}{0.005} & \MTCWITHCONF{0.25}{0.001} & \MTCWITHCONF{74.07}{0.21} & \MTCWITHCONF{0.532}{0.004} & \MTCWITHCONF{0.203}{0.001} \\
 &  & noHub & \SECONDBEST{\MTCWITHCONF{70.08}{0.23}} & \SECONDBEST{\MTCWITHCONF{0.312}{0.005}} & \BEST{\MTCWITHCONF{0.094}{0.001}} & \SECONDBEST{\MTCWITHCONF{77.39}{0.24}} & \SECONDBEST{\MTCWITHCONF{0.304}{0.004}} & \BEST{\MTCWITHCONF{0.101}{0.001}} & \SECONDBEST{\MTCWITHCONF{79.19}{0.22}} & \SECONDBEST{\MTCWITHCONF{0.319}{0.004}} & \BEST{\MTCWITHCONF{0.112}{0.001}} \\
 &  & noHub-S & \BEST{\MTCWITHCONF{72.04}{0.23}} & \BEST{\MTCWITHCONF{0.273}{0.004}} & \SECONDBEST{\MTCWITHCONF{0.13}{0.001}} & \BEST{\MTCWITHCONF{79.13}{0.24}} & \BEST{\MTCWITHCONF{0.282}{0.004}} & \SECONDBEST{\MTCWITHCONF{0.126}{0.001}} & \BEST{\MTCWITHCONF{81.42}{0.22}} & \BEST{\MTCWITHCONF{0.296}{0.003}} & \MTCWITHCONF{0.161}{0.001} \\
\cline{1-12} \cline{2-12}
\bottomrule
\end{tabular}
}
        \caption{Resnet-18: 1-shot}
        \label{tab:main-tim-resnet18-1}
    \end{table*}
    \begin{table*}
        {\scriptsize\centering\begin{tabular}{llllllllllll}
\toprule
 &  &  & \multicolumn{3}{c}{mini} & \multicolumn{3}{c}{tiered} & \multicolumn{3}{c}{CUB} \\
 &  &  & Acc & Skew & Hub.~Occ. & Acc & Skew & Hub.~Occ. & Acc & Skew & Hub.~Occ. \\
Arch. & Clf. & Emb. &  &  &  &  &  &  &  &  &  \\
\midrule
\multirow[c]{54}{*}{\rotatebox{90}{WideRes28-10}} & \multirow[c]{9}{*}{\rotatebox{90}{ILPC}} & None & \MTCWITHCONF{71.27}{0.28} & \MTCWITHCONF{1.595}{0.01} & \MTCWITHCONF{0.46}{0.001} & \MTCWITHCONF{75.01}{0.28} & \MTCWITHCONF{1.807}{0.01} & \MTCWITHCONF{0.494}{0.001} & \MTCWITHCONF{89.75}{0.19} & \MTCWITHCONF{1.072}{0.009} & \MTCWITHCONF{0.367}{0.001} \\
 &  & L2 & \MTCWITHCONF{76.41}{0.26} & \MTCWITHCONF{0.773}{0.006} & \MTCWITHCONF{0.295}{0.001} & \MTCWITHCONF{78.25}{0.27} & \MTCWITHCONF{0.731}{0.006} & \MTCWITHCONF{0.274}{0.001} & \MTCWITHCONF{90.27}{0.2} & \MTCWITHCONF{0.473}{0.004} & \MTCWITHCONF{0.228}{0.001} \\
 &  & CL2 & \MTCWITHCONF{74.13}{0.27} & \MTCWITHCONF{0.993}{0.009} & \MTCWITHCONF{0.29}{0.001} & \MTCWITHCONF{78.2}{0.27} & \MTCWITHCONF{0.815}{0.006} & \MTCWITHCONF{0.306}{0.001} & \MTCWITHCONF{90.34}{0.2} & \MTCWITHCONF{0.524}{0.004} & \MTCWITHCONF{0.267}{0.001} \\
 &  & ZN & \MTCWITHCONF{77.76}{0.26} & \MTCWITHCONF{0.728}{0.005} & \MTCWITHCONF{0.287}{0.001} & \MTCWITHCONF{79.42}{0.27} & \MTCWITHCONF{0.776}{0.006} & \MTCWITHCONF{0.302}{0.001} & \MTCWITHCONF{90.21}{0.2} & \MTCWITHCONF{0.516}{0.004} & \MTCWITHCONF{0.263}{0.001} \\
 &  & ReRep & \MTCWITHCONF{62.51}{0.34} & \MTCWITHCONF{3.56}{0.002} & \MTCWITHCONF{0.704}{0.001} & \MTCWITHCONF{60.66}{0.37} & \MTCWITHCONF{3.55}{0.002} & \MTCWITHCONF{0.776}{0.001} & \MTCWITHCONF{87.44}{0.25} & \MTCWITHCONF{3.033}{0.008} & \MTCWITHCONF{0.472}{0.001} \\
 &  & EASE & \MTCWITHCONF{78.01}{0.26} & \MTCWITHCONF{0.47}{0.004} & \MTCWITHCONF{0.176}{0.001} & \MTCWITHCONF{79.64}{0.27} & \MTCWITHCONF{0.479}{0.004} & \MTCWITHCONF{0.175}{0.001} & \MTCWITHCONF{90.76}{0.19} & \MTCWITHCONF{0.437}{0.003} & \MTCWITHCONF{0.212}{0.001} \\
 &  & TCPR & \MTCWITHCONF{78.37}{0.26} & \MTCWITHCONF{0.584}{0.005} & \MTCWITHCONF{0.237}{0.001} & \MTCWITHCONF{79.55}{0.28} & \MTCWITHCONF{0.683}{0.006} & \MTCWITHCONF{0.265}{0.001} & \MTCWITHCONF{90.77}{0.19} & \MTCWITHCONF{0.476}{0.004} & \MTCWITHCONF{0.23}{0.001} \\
 &  & noHub & \SECONDBEST{\MTCWITHCONF{78.84}{0.27}} & \SECONDBEST{\MTCWITHCONF{0.293}{0.004}} & \BEST{\MTCWITHCONF{0.112}{0.001}} & \SECONDBEST{\MTCWITHCONF{80.75}{0.28}} & \SECONDBEST{\MTCWITHCONF{0.3}{0.004}} & \BEST{\MTCWITHCONF{0.112}{0.001}} & \SECONDBEST{\MTCWITHCONF{90.91}{0.2}} & \SECONDBEST{\MTCWITHCONF{0.189}{0.004}} & \BEST{\MTCWITHCONF{0.109}{0.001}} \\
 &  & noHub-S & \BEST{\MTCWITHCONF{79.77}{0.26}} & \BEST{\MTCWITHCONF{0.262}{0.004}} & \SECONDBEST{\MTCWITHCONF{0.148}{0.001}} & \BEST{\MTCWITHCONF{81.24}{0.27}} & \BEST{\MTCWITHCONF{0.278}{0.004}} & \SECONDBEST{\MTCWITHCONF{0.135}{0.001}} & \BEST{\MTCWITHCONF{91.28}{0.19}} & \BEST{\MTCWITHCONF{0.16}{0.004}} & \SECONDBEST{\MTCWITHCONF{0.13}{0.001}} \\
\cline{2-12}
 & \multirow[c]{9}{*}{\rotatebox{90}{LaplacianShot}} & None & \MTCWITHCONF{72.56}{0.23} & \MTCWITHCONF{1.599}{0.01} & \MTCWITHCONF{0.459}{0.001} & \MTCWITHCONF{75.58}{0.25} & \MTCWITHCONF{1.795}{0.01} & \MTCWITHCONF{0.495}{0.001} & \MTCWITHCONF{88.71}{0.19} & \MTCWITHCONF{1.071}{0.009} & \MTCWITHCONF{0.369}{0.001} \\
 &  & L2 & \MTCWITHCONF{75.18}{0.23} & \MTCWITHCONF{0.777}{0.006} & \MTCWITHCONF{0.296}{0.001} & \MTCWITHCONF{77.03}{0.24} & \MTCWITHCONF{0.732}{0.006} & \MTCWITHCONF{0.274}{0.001} & \MTCWITHCONF{89.73}{0.17} & \MTCWITHCONF{0.474}{0.004} & \MTCWITHCONF{0.229}{0.001} \\
 &  & CL2 & \MTCWITHCONF{71.29}{0.24} & \MTCWITHCONF{0.987}{0.009} & \MTCWITHCONF{0.29}{0.001} & \MTCWITHCONF{75.42}{0.25} & \MTCWITHCONF{0.819}{0.006} & \MTCWITHCONF{0.309}{0.001} & \MTCWITHCONF{89.61}{0.18} & \MTCWITHCONF{0.52}{0.004} & \MTCWITHCONF{0.268}{0.001} \\
 &  & ZN & \MTCWITHCONF{75.18}{0.22} & \MTCWITHCONF{0.724}{0.005} & \MTCWITHCONF{0.286}{0.001} & \MTCWITHCONF{77.0}{0.24} & \MTCWITHCONF{0.768}{0.006} & \MTCWITHCONF{0.301}{0.001} & \MTCWITHCONF{89.22}{0.18} & \MTCWITHCONF{0.517}{0.004} & \MTCWITHCONF{0.263}{0.001} \\
 &  & ReRep & \MTCWITHCONF{75.25}{0.22} & \MTCWITHCONF{3.562}{0.002} & \MTCWITHCONF{0.704}{0.001} & \MTCWITHCONF{77.12}{0.24} & \MTCWITHCONF{3.548}{0.002} & \MTCWITHCONF{0.776}{0.001} & \MTCWITHCONF{88.98}{0.18} & \MTCWITHCONF{3.024}{0.008} & \MTCWITHCONF{0.47}{0.001} \\
 &  & EASE & \MTCWITHCONF{77.29}{0.22} & \MTCWITHCONF{0.473}{0.004} & \MTCWITHCONF{0.177}{0.001} & \MTCWITHCONF{78.97}{0.24} & \MTCWITHCONF{0.475}{0.004} & \MTCWITHCONF{0.175}{0.001} & \MTCWITHCONF{90.06}{0.17} & \MTCWITHCONF{0.435}{0.003} & \MTCWITHCONF{0.213}{0.001} \\
 &  & TCPR & \MTCWITHCONF{76.77}{0.22} & \MTCWITHCONF{0.593}{0.005} & \MTCWITHCONF{0.236}{0.001} & \MTCWITHCONF{77.49}{0.24} & \MTCWITHCONF{0.686}{0.006} & \MTCWITHCONF{0.264}{0.001} & \MTCWITHCONF{89.42}{0.17} & \MTCWITHCONF{0.475}{0.004} & \MTCWITHCONF{0.231}{0.001} \\
 &  & noHub & \BEST{\MTCWITHCONF{79.13}{0.23}} & \SECONDBEST{\MTCWITHCONF{0.29}{0.004}} & \BEST{\MTCWITHCONF{0.111}{0.001}} & \SECONDBEST{\MTCWITHCONF{80.5}{0.25}} & \SECONDBEST{\MTCWITHCONF{0.302}{0.004}} & \BEST{\MTCWITHCONF{0.112}{0.001}} & \BEST{\MTCWITHCONF{90.73}{0.18}} & \SECONDBEST{\MTCWITHCONF{0.19}{0.004}} & \BEST{\MTCWITHCONF{0.109}{0.001}} \\
 &  & noHub-S & \SECONDBEST{\MTCWITHCONF{79.13}{0.23}} & \BEST{\MTCWITHCONF{0.259}{0.004}} & \SECONDBEST{\MTCWITHCONF{0.147}{0.001}} & \BEST{\MTCWITHCONF{80.59}{0.24}} & \BEST{\MTCWITHCONF{0.277}{0.004}} & \SECONDBEST{\MTCWITHCONF{0.135}{0.001}} & \SECONDBEST{\MTCWITHCONF{90.61}{0.17}} & \BEST{\MTCWITHCONF{0.164}{0.004}} & \SECONDBEST{\MTCWITHCONF{0.13}{0.001}} \\
\cline{2-12}
 & \multirow[c]{9}{*}{\rotatebox{90}{ObliqueManifold}} & None & \MTCWITHCONF{76.02}{0.22} & \MTCWITHCONF{1.599}{0.01} & \MTCWITHCONF{0.46}{0.001} & \MTCWITHCONF{77.75}{0.25} & \MTCWITHCONF{1.801}{0.01} & \MTCWITHCONF{0.494}{0.001} & \MTCWITHCONF{90.82}{0.18} & \MTCWITHCONF{1.07}{0.009} & \MTCWITHCONF{0.368}{0.001} \\
 &  & L2 & \MTCWITHCONF{76.11}{0.22} & \MTCWITHCONF{0.779}{0.006} & \MTCWITHCONF{0.295}{0.001} & \MTCWITHCONF{77.74}{0.25} & \MTCWITHCONF{0.731}{0.006} & \MTCWITHCONF{0.274}{0.001} & \MTCWITHCONF{90.89}{0.18} & \MTCWITHCONF{0.475}{0.004} & \MTCWITHCONF{0.228}{0.001} \\
 &  & CL2 & \MTCWITHCONF{74.43}{0.24} & \MTCWITHCONF{0.985}{0.009} & \MTCWITHCONF{0.289}{0.001} & \MTCWITHCONF{77.98}{0.25} & \MTCWITHCONF{0.816}{0.007} & \MTCWITHCONF{0.307}{0.001} & \MTCWITHCONF{90.6}{0.18} & \MTCWITHCONF{0.523}{0.004} & \MTCWITHCONF{0.267}{0.001} \\
 &  & ZN & \MTCWITHCONF{77.69}{0.23} & \MTCWITHCONF{0.724}{0.005} & \MTCWITHCONF{0.286}{0.001} & \MTCWITHCONF{79.32}{0.24} & \MTCWITHCONF{0.767}{0.006} & \MTCWITHCONF{0.301}{0.001} & \MTCWITHCONF{90.73}{0.18} & \MTCWITHCONF{0.519}{0.004} & \MTCWITHCONF{0.263}{0.001} \\
 &  & ReRep & \MTCWITHCONF{78.08}{0.23} & \MTCWITHCONF{3.56}{0.002} & \MTCWITHCONF{0.703}{0.001} & \MTCWITHCONF{79.46}{0.25} & \MTCWITHCONF{3.549}{0.002} & \MTCWITHCONF{0.777}{0.001} & \SECONDBEST{\MTCWITHCONF{91.16}{0.18}} & \MTCWITHCONF{3.032}{0.008} & \MTCWITHCONF{0.471}{0.001} \\
 &  & EASE & \MTCWITHCONF{74.77}{0.23} & \MTCWITHCONF{0.472}{0.004} & \MTCWITHCONF{0.178}{0.001} & \MTCWITHCONF{77.07}{0.25} & \MTCWITHCONF{0.473}{0.004} & \MTCWITHCONF{0.174}{0.001} & \MTCWITHCONF{89.2}{0.18} & \MTCWITHCONF{0.439}{0.003} & \MTCWITHCONF{0.212}{0.001} \\
 &  & TCPR & \MTCWITHCONF{77.39}{0.23} & \MTCWITHCONF{0.587}{0.005} & \MTCWITHCONF{0.236}{0.001} & \MTCWITHCONF{78.75}{0.24} & \MTCWITHCONF{0.687}{0.006} & \MTCWITHCONF{0.265}{0.001} & \MTCWITHCONF{89.93}{0.19} & \MTCWITHCONF{0.474}{0.004} & \MTCWITHCONF{0.23}{0.001} \\
 &  & noHub & \SECONDBEST{\MTCWITHCONF{78.44}{0.24}} & \SECONDBEST{\MTCWITHCONF{0.292}{0.004}} & \BEST{\MTCWITHCONF{0.112}{0.001}} & \SECONDBEST{\MTCWITHCONF{79.99}{0.26}} & \SECONDBEST{\MTCWITHCONF{0.302}{0.004}} & \BEST{\MTCWITHCONF{0.113}{0.001}} & \MTCWITHCONF{90.59}{0.19} & \SECONDBEST{\MTCWITHCONF{0.185}{0.004}} & \BEST{\MTCWITHCONF{0.108}{0.001}} \\
 &  & noHub-S & \BEST{\MTCWITHCONF{79.89}{0.24}} & \BEST{\MTCWITHCONF{0.259}{0.004}} & \SECONDBEST{\MTCWITHCONF{0.148}{0.001}} & \BEST{\MTCWITHCONF{80.67}{0.26}} & \BEST{\MTCWITHCONF{0.279}{0.004}} & \SECONDBEST{\MTCWITHCONF{0.137}{0.001}} & \BEST{\MTCWITHCONF{91.37}{0.18}} & \BEST{\MTCWITHCONF{0.162}{0.004}} & \SECONDBEST{\MTCWITHCONF{0.13}{0.001}} \\
\cline{2-12}
 & \multirow[c]{9}{*}{\rotatebox{90}{SIAMESE}} & None & \MTCWITHCONF{45.69}{0.31} & \MTCWITHCONF{1.594}{0.009} & \MTCWITHCONF{0.459}{0.001} & \MTCWITHCONF{75.29}{0.28} & \MTCWITHCONF{1.801}{0.01} & \MTCWITHCONF{0.495}{0.001} & \MTCWITHCONF{61.36}{0.55} & \MTCWITHCONF{1.074}{0.009} & \MTCWITHCONF{0.37}{0.001} \\
 &  & L2 & \MTCWITHCONF{80.2}{0.23} & \MTCWITHCONF{0.776}{0.006} & \MTCWITHCONF{0.296}{0.001} & \MTCWITHCONF{80.89}{0.26} & \MTCWITHCONF{0.735}{0.006} & \MTCWITHCONF{0.275}{0.001} & \MTCWITHCONF{91.98}{0.18} & \MTCWITHCONF{0.476}{0.004} & \MTCWITHCONF{0.23}{0.001} \\
 &  & CL2 & \MTCWITHCONF{75.23}{0.27} & \MTCWITHCONF{0.988}{0.009} & \MTCWITHCONF{0.289}{0.001} & \MTCWITHCONF{79.59}{0.27} & \MTCWITHCONF{0.82}{0.006} & \MTCWITHCONF{0.307}{0.001} & \MTCWITHCONF{92.17}{0.18} & \MTCWITHCONF{0.518}{0.004} & \MTCWITHCONF{0.266}{0.001} \\
 &  & ZN & \MTCWITHCONF{20.0}{0.0} & \MTCWITHCONF{0.726}{0.005} & \MTCWITHCONF{0.286}{0.001} & \MTCWITHCONF{20.0}{0.0} & \MTCWITHCONF{0.775}{0.006} & \MTCWITHCONF{0.302}{0.001} & \MTCWITHCONF{20.0}{0.0} & \MTCWITHCONF{0.517}{0.004} & \MTCWITHCONF{0.264}{0.001} \\
 &  & ReRep & \MTCWITHCONF{36.69}{0.28} & \MTCWITHCONF{3.561}{0.002} & \MTCWITHCONF{0.705}{0.001} & \MTCWITHCONF{67.41}{0.29} & \MTCWITHCONF{3.55}{0.002} & \MTCWITHCONF{0.776}{0.001} & \MTCWITHCONF{57.62}{0.56} & \MTCWITHCONF{3.027}{0.008} & \MTCWITHCONF{0.472}{0.001} \\
 &  & EASE & \MTCWITHCONF{81.19}{0.25} & \MTCWITHCONF{0.474}{0.004} & \MTCWITHCONF{0.178}{0.001} & \MTCWITHCONF{82.04}{0.26} & \MTCWITHCONF{0.476}{0.004} & \MTCWITHCONF{0.176}{0.001} & \MTCWITHCONF{91.99}{0.19} & \MTCWITHCONF{0.436}{0.003} & \MTCWITHCONF{0.213}{0.001} \\
 &  & TCPR & \MTCWITHCONF{81.27}{0.24} & \MTCWITHCONF{0.582}{0.005} & \MTCWITHCONF{0.236}{0.001} & \MTCWITHCONF{81.89}{0.26} & \MTCWITHCONF{0.681}{0.006} & \MTCWITHCONF{0.264}{0.001} & \MTCWITHCONF{91.91}{0.18} & \MTCWITHCONF{0.477}{0.004} & \MTCWITHCONF{0.232}{0.001} \\
 &  & noHub & \SECONDBEST{\MTCWITHCONF{81.97}{0.25}} & \SECONDBEST{\MTCWITHCONF{0.291}{0.004}} & \BEST{\MTCWITHCONF{0.111}{0.001}} & \SECONDBEST{\MTCWITHCONF{82.8}{0.27}} & \SECONDBEST{\MTCWITHCONF{0.298}{0.004}} & \BEST{\MTCWITHCONF{0.112}{0.001}} & \SECONDBEST{\MTCWITHCONF{92.53}{0.18}} & \SECONDBEST{\MTCWITHCONF{0.189}{0.004}} & \BEST{\MTCWITHCONF{0.109}{0.001}} \\
 &  & noHub-S & \BEST{\MTCWITHCONF{82.0}{0.26}} & \BEST{\MTCWITHCONF{0.258}{0.004}} & \SECONDBEST{\MTCWITHCONF{0.148}{0.001}} & \BEST{\MTCWITHCONF{82.85}{0.27}} & \BEST{\MTCWITHCONF{0.278}{0.004}} & \SECONDBEST{\MTCWITHCONF{0.137}{0.001}} & \BEST{\MTCWITHCONF{92.63}{0.18}} & \BEST{\MTCWITHCONF{0.159}{0.004}} & \SECONDBEST{\MTCWITHCONF{0.13}{0.001}} \\
\cline{2-12}
 & \multirow[c]{9}{*}{\rotatebox{90}{SimpleShot}} & None & \MTCWITHCONF{55.66}{0.21} & \MTCWITHCONF{1.6}{0.01} & \MTCWITHCONF{0.459}{0.001} & \MTCWITHCONF{54.71}{0.22} & \MTCWITHCONF{1.81}{0.01} & \MTCWITHCONF{0.494}{0.001} & \MTCWITHCONF{70.92}{0.23} & \MTCWITHCONF{1.073}{0.009} & \MTCWITHCONF{0.369}{0.001} \\
 &  & L2 & \MTCWITHCONF{65.78}{0.2} & \MTCWITHCONF{0.781}{0.006} & \MTCWITHCONF{0.296}{0.001} & \MTCWITHCONF{68.75}{0.22} & \MTCWITHCONF{0.737}{0.006} & \MTCWITHCONF{0.275}{0.001} & \MTCWITHCONF{82.85}{0.19} & \MTCWITHCONF{0.475}{0.004} & \MTCWITHCONF{0.228}{0.001} \\
 &  & CL2 & \MTCWITHCONF{64.33}{0.2} & \MTCWITHCONF{0.981}{0.009} & \MTCWITHCONF{0.288}{0.001} & \MTCWITHCONF{67.66}{0.22} & \MTCWITHCONF{0.817}{0.006} & \MTCWITHCONF{0.307}{0.001} & \MTCWITHCONF{82.8}{0.19} & \MTCWITHCONF{0.52}{0.004} & \MTCWITHCONF{0.267}{0.001} \\
 &  & ZN & \MTCWITHCONF{67.31}{0.2} & \MTCWITHCONF{0.73}{0.005} & \MTCWITHCONF{0.287}{0.001} & \MTCWITHCONF{69.14}{0.22} & \MTCWITHCONF{0.769}{0.006} & \MTCWITHCONF{0.302}{0.001} & \MTCWITHCONF{82.79}{0.19} & \MTCWITHCONF{0.517}{0.004} & \MTCWITHCONF{0.263}{0.001} \\
 &  & ReRep & \MTCWITHCONF{67.38}{0.2} & \MTCWITHCONF{3.56}{0.002} & \MTCWITHCONF{0.704}{0.001} & \MTCWITHCONF{70.17}{0.22} & \MTCWITHCONF{3.55}{0.002} & \MTCWITHCONF{0.777}{0.001} & \MTCWITHCONF{84.86}{0.19} & \MTCWITHCONF{3.026}{0.008} & \MTCWITHCONF{0.47}{0.001} \\
 &  & EASE & \MTCWITHCONF{68.62}{0.2} & \MTCWITHCONF{0.47}{0.004} & \MTCWITHCONF{0.177}{0.001} & \MTCWITHCONF{70.26}{0.21} & \MTCWITHCONF{0.477}{0.004} & \MTCWITHCONF{0.175}{0.001} & \MTCWITHCONF{84.14}{0.18} & \MTCWITHCONF{0.437}{0.003} & \MTCWITHCONF{0.213}{0.001} \\
 &  & TCPR & \MTCWITHCONF{68.45}{0.2} & \MTCWITHCONF{0.589}{0.005} & \MTCWITHCONF{0.236}{0.001} & \MTCWITHCONF{68.68}{0.22} & \MTCWITHCONF{0.685}{0.006} & \MTCWITHCONF{0.264}{0.001} & \MTCWITHCONF{82.28}{0.19} & \MTCWITHCONF{0.477}{0.004} & \MTCWITHCONF{0.231}{0.001} \\
 &  & noHub & \SECONDBEST{\MTCWITHCONF{75.06}{0.21}} & \SECONDBEST{\MTCWITHCONF{0.29}{0.004}} & \BEST{\MTCWITHCONF{0.111}{0.001}} & \SECONDBEST{\MTCWITHCONF{76.7}{0.23}} & \SECONDBEST{\MTCWITHCONF{0.301}{0.004}} & \BEST{\MTCWITHCONF{0.111}{0.001}} & \SECONDBEST{\MTCWITHCONF{88.06}{0.18}} & \SECONDBEST{\MTCWITHCONF{0.188}{0.004}} & \BEST{\MTCWITHCONF{0.108}{0.001}} \\
 &  & noHub-S & \BEST{\MTCWITHCONF{76.86}{0.21}} & \BEST{\MTCWITHCONF{0.258}{0.004}} & \SECONDBEST{\MTCWITHCONF{0.148}{0.001}} & \BEST{\MTCWITHCONF{78.4}{0.23}} & \BEST{\MTCWITHCONF{0.274}{0.004}} & \SECONDBEST{\MTCWITHCONF{0.135}{0.001}} & \BEST{\MTCWITHCONF{89.25}{0.18}} & \BEST{\MTCWITHCONF{0.162}{0.004}} & \SECONDBEST{\MTCWITHCONF{0.13}{0.001}} \\
\cline{2-12}
 & \multirow[c]{9}{*}{\rotatebox{90}{\( \alpha \)-TIM}} & None & \MTCWITHCONF{60.31}{0.2} & \MTCWITHCONF{1.603}{0.01} & \MTCWITHCONF{0.458}{0.001} & \MTCWITHCONF{69.42}{0.25} & \MTCWITHCONF{1.811}{0.01} & \MTCWITHCONF{0.494}{0.001} & \MTCWITHCONF{73.83}{0.21} & \MTCWITHCONF{1.072}{0.009} & \MTCWITHCONF{0.369}{0.001} \\
 &  & L2 & \MTCWITHCONF{72.11}{0.22} & \MTCWITHCONF{0.778}{0.006} & \MTCWITHCONF{0.295}{0.001} & \MTCWITHCONF{74.45}{0.23} & \MTCWITHCONF{0.73}{0.006} & \MTCWITHCONF{0.275}{0.001} & \MTCWITHCONF{85.96}{0.19} & \MTCWITHCONF{0.476}{0.004} & \MTCWITHCONF{0.229}{0.001} \\
 &  & CL2 & \MTCWITHCONF{68.5}{0.21} & \MTCWITHCONF{0.988}{0.009} & \MTCWITHCONF{0.29}{0.001} & \MTCWITHCONF{72.17}{0.23} & \MTCWITHCONF{0.811}{0.006} & \MTCWITHCONF{0.306}{0.001} & \MTCWITHCONF{85.6}{0.18} & \MTCWITHCONF{0.522}{0.004} & \MTCWITHCONF{0.267}{0.001} \\
 &  & ZN & \MTCWITHCONF{67.69}{0.2} & \MTCWITHCONF{0.73}{0.005} & \MTCWITHCONF{0.287}{0.001} & \MTCWITHCONF{68.94}{0.22} & \MTCWITHCONF{0.769}{0.006} & \MTCWITHCONF{0.302}{0.001} & \MTCWITHCONF{83.03}{0.19} & \MTCWITHCONF{0.518}{0.004} & \MTCWITHCONF{0.263}{0.001} \\
 &  & ReRep & \MTCWITHCONF{73.15}{0.23} & \MTCWITHCONF{3.56}{0.002} & \MTCWITHCONF{0.704}{0.001} & \MTCWITHCONF{76.19}{0.25} & \MTCWITHCONF{3.551}{0.002} & \MTCWITHCONF{0.778}{0.001} & \MTCWITHCONF{88.55}{0.18} & \MTCWITHCONF{3.027}{0.008} & \MTCWITHCONF{0.472}{0.001} \\
 &  & EASE & \MTCWITHCONF{69.83}{0.2} & \MTCWITHCONF{0.468}{0.004} & \MTCWITHCONF{0.176}{0.001} & \MTCWITHCONF{71.54}{0.22} & \MTCWITHCONF{0.481}{0.004} & \MTCWITHCONF{0.175}{0.001} & \MTCWITHCONF{84.9}{0.19} & \MTCWITHCONF{0.436}{0.003} & \MTCWITHCONF{0.213}{0.001} \\
 &  & TCPR & \MTCWITHCONF{71.6}{0.21} & \MTCWITHCONF{0.586}{0.005} & \MTCWITHCONF{0.237}{0.001} & \MTCWITHCONF{72.71}{0.22} & \MTCWITHCONF{0.689}{0.006} & \MTCWITHCONF{0.264}{0.001} & \MTCWITHCONF{84.99}{0.19} & \MTCWITHCONF{0.479}{0.004} & \MTCWITHCONF{0.231}{0.001} \\
 &  & noHub & \SECONDBEST{\MTCWITHCONF{75.87}{0.22}} & \SECONDBEST{\MTCWITHCONF{0.29}{0.004}} & \BEST{\MTCWITHCONF{0.111}{0.001}} & \SECONDBEST{\MTCWITHCONF{77.83}{0.23}} & \SECONDBEST{\MTCWITHCONF{0.302}{0.004}} & \BEST{\MTCWITHCONF{0.112}{0.001}} & \SECONDBEST{\MTCWITHCONF{88.7}{0.17}} & \SECONDBEST{\MTCWITHCONF{0.189}{0.004}} & \BEST{\MTCWITHCONF{0.108}{0.001}} \\
 &  & noHub-S & \BEST{\MTCWITHCONF{77.76}{0.22}} & \BEST{\MTCWITHCONF{0.259}{0.004}} & \SECONDBEST{\MTCWITHCONF{0.147}{0.001}} & \BEST{\MTCWITHCONF{79.04}{0.24}} & \BEST{\MTCWITHCONF{0.276}{0.004}} & \SECONDBEST{\MTCWITHCONF{0.136}{0.001}} & \BEST{\MTCWITHCONF{89.77}{0.17}} & \BEST{\MTCWITHCONF{0.163}{0.003}} & \SECONDBEST{\MTCWITHCONF{0.13}{0.001}} \\
\cline{1-12} \cline{2-12}
\bottomrule
\end{tabular}
}
        \caption{WideRes28-10: 1-shot}
        \label{tab:main-s2m2-wrn-s2m2-1}
    \end{table*}
    \begin{table*}
            {\scriptsize\centering\begin{tabular}{llllllllllll}
\toprule
 &  &  & \multicolumn{3}{c}{mini} & \multicolumn{3}{c}{tiered} & \multicolumn{3}{c}{CUB} \\
 &  &  & Acc & Skew & Hub.~Occ. & Acc & Skew & Hub.~Occ. & Acc & Skew & Hub.~Occ. \\
Arch. & Clf. & Emb. &  &  &  &  &  &  &  &  &  \\
\midrule
\multirow[c]{54}{*}{\rotatebox{90}{ResNet18}} & \multirow[c]{9}{*}{\rotatebox{90}{ILPC}} & None & \MTCWITHCONF{76.46}{0.18} & \MTCWITHCONF{1.503}{0.01} & \MTCWITHCONF{0.421}{0.001} & \MTCWITHCONF{84.46}{0.18} & \MTCWITHCONF{1.334}{0.008} & \MTCWITHCONF{0.433}{0.001} & \MTCWITHCONF{85.86}{0.14} & \MTCWITHCONF{0.981}{0.005} & \MTCWITHCONF{0.364}{0.001} \\
 &  & L2 & \MTCWITHCONF{80.9}{0.16} & \MTCWITHCONF{1.051}{0.007} & \MTCWITHCONF{0.314}{0.001} & \MTCWITHCONF{86.23}{0.17} & \MTCWITHCONF{0.912}{0.006} & \MTCWITHCONF{0.289}{0.001} & \MTCWITHCONF{88.03}{0.13} & \MTCWITHCONF{0.808}{0.005} & \MTCWITHCONF{0.264}{0.001} \\
 &  & CL2 & \MTCWITHCONF{81.64}{0.16} & \MTCWITHCONF{0.778}{0.005} & \MTCWITHCONF{0.262}{0.001} & \MTCWITHCONF{86.88}{0.17} & \MTCWITHCONF{0.823}{0.006} & \MTCWITHCONF{0.281}{0.001} & \MTCWITHCONF{88.44}{0.13} & \MTCWITHCONF{0.695}{0.005} & \MTCWITHCONF{0.235}{0.001} \\
 &  & ZN & \MTCWITHCONF{81.61}{0.16} & \MTCWITHCONF{0.793}{0.005} & \MTCWITHCONF{0.258}{0.001} & \SECONDBEST{\MTCWITHCONF{86.9}{0.17}} & \MTCWITHCONF{0.841}{0.006} & \MTCWITHCONF{0.297}{0.001} & \MTCWITHCONF{88.44}{0.12} & \MTCWITHCONF{0.717}{0.004} & \MTCWITHCONF{0.25}{0.001} \\
 &  & ReRep & \MTCWITHCONF{74.83}{0.19} & \MTCWITHCONF{1.623}{0.003} & \MTCWITHCONF{0.871}{0.001} & \MTCWITHCONF{83.96}{0.19} & \MTCWITHCONF{1.722}{0.004} & \MTCWITHCONF{0.873}{0.001} & \MTCWITHCONF{84.54}{0.15} & \MTCWITHCONF{1.432}{0.003} & \MTCWITHCONF{0.869}{0.001} \\
 &  & EASE & \MTCWITHCONF{81.75}{0.16} & \MTCWITHCONF{0.618}{0.005} & \MTCWITHCONF{0.182}{0.001} & \MTCWITHCONF{86.84}{0.17} & \MTCWITHCONF{0.593}{0.004} & \MTCWITHCONF{0.181}{0.001} & \SECONDBEST{\MTCWITHCONF{88.85}{0.12}} & \MTCWITHCONF{0.606}{0.004} & \MTCWITHCONF{0.186}{0.001} \\
 &  & TCPR & \MTCWITHCONF{81.76}{0.16} & \MTCWITHCONF{0.766}{0.005} & \MTCWITHCONF{0.254}{0.001} & \MTCWITHCONF{86.78}{0.17} & \MTCWITHCONF{0.801}{0.005} & \MTCWITHCONF{0.284}{0.001} & \MTCWITHCONF{88.69}{0.13} & \MTCWITHCONF{0.683}{0.004} & \MTCWITHCONF{0.237}{0.001} \\
 &  & noHub & \SECONDBEST{\MTCWITHCONF{82.09}{0.16}} & \BEST{\MTCWITHCONF{0.295}{0.004}} & \SECONDBEST{\MTCWITHCONF{0.097}{0.001}} & \MTCWITHCONF{86.81}{0.17} & \BEST{\MTCWITHCONF{0.289}{0.004}} & \SECONDBEST{\MTCWITHCONF{0.102}{0.001}} & \MTCWITHCONF{88.85}{0.13} & \BEST{\MTCWITHCONF{0.333}{0.004}} & \SECONDBEST{\MTCWITHCONF{0.12}{0.001}} \\
 &  & noHub-S & \BEST{\MTCWITHCONF{82.33}{0.16}} & \SECONDBEST{\MTCWITHCONF{0.488}{0.006}} & \BEST{\MTCWITHCONF{0.086}{0.001}} & \BEST{\MTCWITHCONF{87.05}{0.17}} & \SECONDBEST{\MTCWITHCONF{0.475}{0.006}} & \BEST{\MTCWITHCONF{0.091}{0.001}} & \BEST{\MTCWITHCONF{89.12}{0.13}} & \SECONDBEST{\MTCWITHCONF{0.438}{0.006}} & \BEST{\MTCWITHCONF{0.097}{0.001}} \\
\cline{2-12}
 & \multirow[c]{9}{*}{\rotatebox{90}{LaplacianShot}} & None & \MTCWITHCONF{81.97}{0.15} & \MTCWITHCONF{1.442}{0.009} & \MTCWITHCONF{0.422}{0.001} & \MTCWITHCONF{86.17}{0.16} & \MTCWITHCONF{1.336}{0.008} & \MTCWITHCONF{0.432}{0.001} & \MTCWITHCONF{88.58}{0.12} & \MTCWITHCONF{0.985}{0.005} & \MTCWITHCONF{0.365}{0.001} \\
 &  & L2 & \MTCWITHCONF{81.89}{0.14} & \MTCWITHCONF{1.035}{0.007} & \MTCWITHCONF{0.319}{0.001} & \MTCWITHCONF{86.19}{0.16} & \MTCWITHCONF{0.913}{0.006} & \MTCWITHCONF{0.289}{0.001} & \MTCWITHCONF{88.52}{0.11} & \MTCWITHCONF{0.811}{0.005} & \MTCWITHCONF{0.264}{0.001} \\
 &  & CL2 & \MTCWITHCONF{81.93}{0.14} & \MTCWITHCONF{0.786}{0.005} & \MTCWITHCONF{0.265}{0.001} & \MTCWITHCONF{86.16}{0.16} & \MTCWITHCONF{0.82}{0.006} & \MTCWITHCONF{0.282}{0.001} & \MTCWITHCONF{88.46}{0.12} & \MTCWITHCONF{0.7}{0.005} & \MTCWITHCONF{0.235}{0.001} \\
 &  & ZN & \MTCWITHCONF{82.57}{0.14} & \MTCWITHCONF{0.803}{0.005} & \MTCWITHCONF{0.263}{0.001} & \MTCWITHCONF{86.67}{0.16} & \MTCWITHCONF{0.838}{0.006} & \MTCWITHCONF{0.296}{0.001} & \MTCWITHCONF{88.88}{0.11} & \MTCWITHCONF{0.714}{0.004} & \MTCWITHCONF{0.25}{0.001} \\
 &  & ReRep & \MTCWITHCONF{82.32}{0.14} & \MTCWITHCONF{1.633}{0.003} & \MTCWITHCONF{0.863}{0.001} & \MTCWITHCONF{86.09}{0.16} & \MTCWITHCONF{1.721}{0.004} & \MTCWITHCONF{0.873}{0.001} & \MTCWITHCONF{88.74}{0.12} & \MTCWITHCONF{1.431}{0.002} & \MTCWITHCONF{0.869}{0.001} \\
 &  & EASE & \SECONDBEST{\MTCWITHCONF{82.57}{0.14}} & \MTCWITHCONF{0.627}{0.005} & \MTCWITHCONF{0.186}{0.001} & \SECONDBEST{\MTCWITHCONF{86.82}{0.15}} & \MTCWITHCONF{0.596}{0.004} & \MTCWITHCONF{0.182}{0.001} & \MTCWITHCONF{88.94}{0.11} & \MTCWITHCONF{0.608}{0.004} & \MTCWITHCONF{0.185}{0.001} \\
 &  & TCPR & \MTCWITHCONF{82.24}{0.14} & \MTCWITHCONF{0.781}{0.005} & \MTCWITHCONF{0.259}{0.001} & \MTCWITHCONF{86.27}{0.16} & \MTCWITHCONF{0.797}{0.005} & \MTCWITHCONF{0.284}{0.001} & \MTCWITHCONF{88.63}{0.11} & \MTCWITHCONF{0.687}{0.004} & \MTCWITHCONF{0.236}{0.001} \\
 &  & noHub & \MTCWITHCONF{82.55}{0.15} & \SECONDBEST{\MTCWITHCONF{0.285}{0.004}} & \SECONDBEST{\MTCWITHCONF{0.096}{0.001}} & \MTCWITHCONF{86.75}{0.16} & \SECONDBEST{\MTCWITHCONF{0.29}{0.004}} & \SECONDBEST{\MTCWITHCONF{0.103}{0.001}} & \BEST{\MTCWITHCONF{89.08}{0.11}} & \BEST{\MTCWITHCONF{0.329}{0.004}} & \SECONDBEST{\MTCWITHCONF{0.12}{0.001}} \\
 &  & noHub-S & \BEST{\MTCWITHCONF{82.81}{0.14}} & \BEST{\MTCWITHCONF{0.25}{0.005}} & \BEST{\MTCWITHCONF{0.073}{0.001}} & \BEST{\MTCWITHCONF{87.12}{0.16}} & \BEST{\MTCWITHCONF{0.214}{0.005}} & \BEST{\MTCWITHCONF{0.077}{0.001}} & \SECONDBEST{\MTCWITHCONF{88.99}{0.11}} & \SECONDBEST{\MTCWITHCONF{0.438}{0.006}} & \BEST{\MTCWITHCONF{0.096}{0.001}} \\
\cline{2-12}
 & \multirow[c]{9}{*}{\rotatebox{90}{ObliqueManifold}} & None & \MTCWITHCONF{83.53}{0.15} & \MTCWITHCONF{1.497}{0.01} & \MTCWITHCONF{0.421}{0.001} & \MTCWITHCONF{87.85}{0.15} & \MTCWITHCONF{1.334}{0.009} & \MTCWITHCONF{0.433}{0.001} & \SECONDBEST{\MTCWITHCONF{90.28}{0.11}} & \MTCWITHCONF{0.987}{0.005} & \MTCWITHCONF{0.364}{0.001} \\
 &  & L2 & \SECONDBEST{\MTCWITHCONF{83.66}{0.15}} & \MTCWITHCONF{1.051}{0.007} & \MTCWITHCONF{0.314}{0.001} & \MTCWITHCONF{87.83}{0.15} & \MTCWITHCONF{0.922}{0.006} & \MTCWITHCONF{0.289}{0.001} & \MTCWITHCONF{90.21}{0.11} & \MTCWITHCONF{0.81}{0.005} & \MTCWITHCONF{0.263}{0.001} \\
 &  & CL2 & \MTCWITHCONF{83.62}{0.15} & \MTCWITHCONF{0.775}{0.005} & \MTCWITHCONF{0.261}{0.001} & \SECONDBEST{\MTCWITHCONF{88.1}{0.15}} & \MTCWITHCONF{0.823}{0.006} & \MTCWITHCONF{0.281}{0.001} & \MTCWITHCONF{90.09}{0.11} & \MTCWITHCONF{0.701}{0.005} & \MTCWITHCONF{0.236}{0.001} \\
 &  & ZN & \BEST{\MTCWITHCONF{83.86}{0.15}} & \MTCWITHCONF{0.795}{0.005} & \MTCWITHCONF{0.258}{0.001} & \BEST{\MTCWITHCONF{88.47}{0.15}} & \MTCWITHCONF{0.835}{0.006} & \MTCWITHCONF{0.296}{0.001} & \BEST{\MTCWITHCONF{90.47}{0.11}} & \MTCWITHCONF{0.716}{0.004} & \MTCWITHCONF{0.251}{0.001} \\
 &  & ReRep & \MTCWITHCONF{82.44}{0.15} & \MTCWITHCONF{1.62}{0.003} & \MTCWITHCONF{0.871}{0.001} & \MTCWITHCONF{86.85}{0.16} & \MTCWITHCONF{1.725}{0.004} & \MTCWITHCONF{0.872}{0.001} & \MTCWITHCONF{89.83}{0.11} & \MTCWITHCONF{1.431}{0.003} & \MTCWITHCONF{0.869}{0.001} \\
 &  & EASE & \MTCWITHCONF{82.83}{0.15} & \MTCWITHCONF{0.628}{0.005} & \MTCWITHCONF{0.185}{0.001} & \MTCWITHCONF{87.63}{0.16} & \MTCWITHCONF{0.597}{0.005} & \MTCWITHCONF{0.182}{0.001} & \MTCWITHCONF{89.74}{0.12} & \SECONDBEST{\MTCWITHCONF{0.609}{0.004}} & \MTCWITHCONF{0.186}{0.001} \\
 &  & TCPR & \MTCWITHCONF{83.51}{0.15} & \MTCWITHCONF{0.766}{0.005} & \MTCWITHCONF{0.255}{0.001} & \MTCWITHCONF{88.09}{0.15} & \MTCWITHCONF{0.795}{0.005} & \MTCWITHCONF{0.283}{0.001} & \MTCWITHCONF{90.28}{0.11} & \MTCWITHCONF{0.687}{0.004} & \MTCWITHCONF{0.236}{0.001} \\
 &  & noHub & \MTCWITHCONF{83.28}{0.15} & \BEST{\MTCWITHCONF{0.287}{0.004}} & \SECONDBEST{\MTCWITHCONF{0.096}{0.001}} & \MTCWITHCONF{87.58}{0.16} & \BEST{\MTCWITHCONF{0.288}{0.004}} & \SECONDBEST{\MTCWITHCONF{0.102}{0.001}} & \MTCWITHCONF{89.89}{0.12} & \BEST{\MTCWITHCONF{0.334}{0.004}} & \SECONDBEST{\MTCWITHCONF{0.121}{0.001}} \\
 &  & noHub-S & \MTCWITHCONF{83.25}{0.16} & \SECONDBEST{\MTCWITHCONF{0.487}{0.006}} & \BEST{\MTCWITHCONF{0.086}{0.001}} & \MTCWITHCONF{87.82}{0.16} & \SECONDBEST{\MTCWITHCONF{0.469}{0.006}} & \BEST{\MTCWITHCONF{0.091}{0.001}} & \MTCWITHCONF{89.38}{0.17} & \MTCWITHCONF{nan}{nan} & \BEST{\MTCWITHCONF{0.097}{0.001}} \\
\cline{2-12}
 & \multirow[c]{9}{*}{\rotatebox{90}{SIAMESE}} & None & \MTCWITHCONF{20.0}{0.0} & \MTCWITHCONF{1.441}{0.009} & \MTCWITHCONF{0.421}{0.001} & \MTCWITHCONF{20.0}{0.0} & \MTCWITHCONF{1.339}{0.009} & \MTCWITHCONF{0.433}{0.001} & \MTCWITHCONF{20.0}{0.0} & \MTCWITHCONF{0.984}{0.005} & \MTCWITHCONF{0.364}{0.001} \\
 &  & L2 & \MTCWITHCONF{83.14}{0.14} & \MTCWITHCONF{1.035}{0.007} & \MTCWITHCONF{0.319}{0.001} & \MTCWITHCONF{87.04}{0.16} & \MTCWITHCONF{0.912}{0.006} & \MTCWITHCONF{0.288}{0.001} & \MTCWITHCONF{89.48}{0.12} & \MTCWITHCONF{0.808}{0.005} & \MTCWITHCONF{0.264}{0.001} \\
 &  & CL2 & \MTCWITHCONF{84.04}{0.15} & \MTCWITHCONF{0.788}{0.005} & \MTCWITHCONF{0.264}{0.001} & \MTCWITHCONF{87.9}{0.16} & \MTCWITHCONF{0.816}{0.006} & \MTCWITHCONF{0.28}{0.001} & \MTCWITHCONF{90.14}{0.12} & \MTCWITHCONF{0.698}{0.005} & \MTCWITHCONF{0.235}{0.001} \\
 &  & ZN & \MTCWITHCONF{20.0}{0.0} & \MTCWITHCONF{0.8}{0.005} & \MTCWITHCONF{0.263}{0.001} & \MTCWITHCONF{20.0}{0.0} & \MTCWITHCONF{0.84}{0.006} & \MTCWITHCONF{0.296}{0.001} & \MTCWITHCONF{20.0}{0.0} & \MTCWITHCONF{0.713}{0.004} & \MTCWITHCONF{0.251}{0.001} \\
 &  & ReRep & \MTCWITHCONF{20.0}{0.0} & \MTCWITHCONF{1.633}{0.003} & \MTCWITHCONF{0.863}{0.001} & \MTCWITHCONF{20.0}{0.0} & \MTCWITHCONF{1.724}{0.004} & \MTCWITHCONF{0.872}{0.001} & \MTCWITHCONF{20.0}{0.0} & \MTCWITHCONF{1.428}{0.002} & \MTCWITHCONF{0.869}{0.001} \\
 &  & EASE & \SECONDBEST{\MTCWITHCONF{84.61}{0.15}} & \MTCWITHCONF{0.63}{0.005} & \MTCWITHCONF{0.187}{0.001} & \SECONDBEST{\MTCWITHCONF{88.33}{0.16}} & \MTCWITHCONF{0.594}{0.004} & \MTCWITHCONF{0.182}{0.001} & \MTCWITHCONF{90.42}{0.12} & \MTCWITHCONF{0.607}{0.004} & \MTCWITHCONF{0.185}{0.001} \\
 &  & TCPR & \MTCWITHCONF{84.39}{0.15} & \MTCWITHCONF{0.772}{0.005} & \MTCWITHCONF{0.259}{0.001} & \MTCWITHCONF{88.26}{0.16} & \MTCWITHCONF{0.791}{0.005} & \MTCWITHCONF{0.283}{0.001} & \SECONDBEST{\MTCWITHCONF{90.5}{0.11}} & \MTCWITHCONF{0.686}{0.004} & \MTCWITHCONF{0.235}{0.001} \\
 &  & noHub & \MTCWITHCONF{84.05}{0.16} & \SECONDBEST{\MTCWITHCONF{0.292}{0.004}} & \SECONDBEST{\MTCWITHCONF{0.096}{0.001}} & \MTCWITHCONF{87.87}{0.17} & \BEST{\MTCWITHCONF{0.291}{0.004}} & \SECONDBEST{\MTCWITHCONF{0.103}{0.001}} & \MTCWITHCONF{90.34}{0.12} & \BEST{\MTCWITHCONF{0.334}{0.004}} & \SECONDBEST{\MTCWITHCONF{0.12}{0.001}} \\
 &  & noHub-S & \BEST{\MTCWITHCONF{84.67}{0.15}} & \BEST{\MTCWITHCONF{0.247}{0.005}} & \BEST{\MTCWITHCONF{0.074}{0.001}} & \BEST{\MTCWITHCONF{88.43}{0.16}} & \SECONDBEST{\MTCWITHCONF{0.473}{0.006}} & \BEST{\MTCWITHCONF{0.092}{0.001}} & \BEST{\MTCWITHCONF{90.52}{0.12}} & \SECONDBEST{\MTCWITHCONF{0.443}{0.006}} & \BEST{\MTCWITHCONF{0.097}{0.001}} \\
\cline{2-12}
 & \multirow[c]{9}{*}{\rotatebox{90}{SimpleShot}} & None & \MTCWITHCONF{78.5}{0.14} & \MTCWITHCONF{1.436}{0.009} & \MTCWITHCONF{0.422}{0.001} & \MTCWITHCONF{83.95}{0.16} & \MTCWITHCONF{1.339}{0.008} & \MTCWITHCONF{0.432}{0.001} & \MTCWITHCONF{85.65}{0.12} & \MTCWITHCONF{0.987}{0.005} & \MTCWITHCONF{0.364}{0.001} \\
 &  & L2 & \MTCWITHCONF{79.89}{0.14} & \MTCWITHCONF{1.04}{0.007} & \MTCWITHCONF{0.318}{0.001} & \MTCWITHCONF{84.5}{0.16} & \MTCWITHCONF{0.914}{0.006} & \MTCWITHCONF{0.287}{0.001} & \MTCWITHCONF{86.46}{0.12} & \MTCWITHCONF{0.812}{0.005} & \MTCWITHCONF{0.263}{0.001} \\
 &  & CL2 & \MTCWITHCONF{80.0}{0.14} & \MTCWITHCONF{0.786}{0.005} & \MTCWITHCONF{0.264}{0.001} & \MTCWITHCONF{84.66}{0.16} & \MTCWITHCONF{0.821}{0.006} & \MTCWITHCONF{0.28}{0.001} & \MTCWITHCONF{86.3}{0.12} & \MTCWITHCONF{0.698}{0.005} & \MTCWITHCONF{0.236}{0.001} \\
 &  & ZN & \MTCWITHCONF{80.57}{0.14} & \MTCWITHCONF{0.806}{0.005} & \MTCWITHCONF{0.264}{0.001} & \MTCWITHCONF{84.97}{0.16} & \MTCWITHCONF{0.839}{0.006} & \MTCWITHCONF{0.296}{0.001} & \MTCWITHCONF{86.76}{0.12} & \MTCWITHCONF{0.716}{0.005} & \MTCWITHCONF{0.25}{0.001} \\
 &  & ReRep & \MTCWITHCONF{80.86}{0.14} & \MTCWITHCONF{1.631}{0.003} & \MTCWITHCONF{0.863}{0.001} & \MTCWITHCONF{85.05}{0.16} & \MTCWITHCONF{1.721}{0.004} & \MTCWITHCONF{0.872}{0.001} & \SECONDBEST{\MTCWITHCONF{87.83}{0.12}} & \MTCWITHCONF{1.432}{0.002} & \MTCWITHCONF{0.869}{0.001} \\
 &  & EASE & \MTCWITHCONF{80.13}{0.14} & \MTCWITHCONF{0.624}{0.005} & \MTCWITHCONF{0.186}{0.001} & \MTCWITHCONF{84.74}{0.16} & \MTCWITHCONF{0.598}{0.004} & \MTCWITHCONF{0.183}{0.001} & \MTCWITHCONF{86.76}{0.12} & \MTCWITHCONF{0.607}{0.004} & \MTCWITHCONF{0.186}{0.001} \\
 &  & TCPR & \MTCWITHCONF{80.15}{0.14} & \MTCWITHCONF{0.78}{0.005} & \MTCWITHCONF{0.259}{0.001} & \MTCWITHCONF{84.86}{0.15} & \MTCWITHCONF{0.796}{0.005} & \MTCWITHCONF{0.283}{0.001} & \MTCWITHCONF{86.8}{0.12} & \MTCWITHCONF{0.687}{0.004} & \MTCWITHCONF{0.235}{0.001} \\
 &  & noHub & \BEST{\MTCWITHCONF{82.13}{0.14}} & \SECONDBEST{\MTCWITHCONF{0.286}{0.004}} & \SECONDBEST{\MTCWITHCONF{0.096}{0.001}} & \BEST{\MTCWITHCONF{86.31}{0.16}} & \SECONDBEST{\MTCWITHCONF{0.289}{0.004}} & \SECONDBEST{\MTCWITHCONF{0.104}{0.001}} & \BEST{\MTCWITHCONF{88.46}{0.11}} & \BEST{\MTCWITHCONF{0.329}{0.004}} & \SECONDBEST{\MTCWITHCONF{0.12}{0.001}} \\
 &  & noHub-S & \SECONDBEST{\MTCWITHCONF{81.22}{0.14}} & \BEST{\MTCWITHCONF{0.25}{0.005}} & \BEST{\MTCWITHCONF{0.074}{0.001}} & \SECONDBEST{\MTCWITHCONF{86.22}{0.15}} & \BEST{\MTCWITHCONF{0.213}{0.005}} & \BEST{\MTCWITHCONF{0.078}{0.001}} & \MTCWITHCONF{87.6}{0.12} & \SECONDBEST{\MTCWITHCONF{0.433}{0.006}} & \BEST{\MTCWITHCONF{0.097}{0.001}} \\
\cline{2-12}
 & \multirow[c]{9}{*}{\rotatebox{90}{\( \alpha \)-TIM}} & None & \MTCWITHCONF{78.51}{0.15} & \MTCWITHCONF{1.45}{0.009} & \MTCWITHCONF{0.42}{0.001} & \MTCWITHCONF{83.86}{0.16} & \MTCWITHCONF{1.341}{0.009} & \MTCWITHCONF{0.433}{0.001} & \MTCWITHCONF{85.7}{0.12} & \MTCWITHCONF{0.981}{0.005} & \MTCWITHCONF{0.363}{0.001} \\
 &  & L2 & \MTCWITHCONF{80.02}{0.16} & \MTCWITHCONF{1.036}{0.007} & \MTCWITHCONF{0.318}{0.001} & \MTCWITHCONF{84.49}{0.18} & \MTCWITHCONF{0.92}{0.006} & \MTCWITHCONF{0.288}{0.001} & \MTCWITHCONF{87.88}{0.13} & \MTCWITHCONF{0.812}{0.005} & \MTCWITHCONF{0.264}{0.001} \\
 &  & CL2 & \MTCWITHCONF{80.46}{0.16} & \MTCWITHCONF{0.784}{0.005} & \MTCWITHCONF{0.264}{0.001} & \MTCWITHCONF{84.86}{0.17} & \MTCWITHCONF{0.82}{0.006} & \MTCWITHCONF{0.281}{0.001} & \MTCWITHCONF{87.53}{0.13} & \MTCWITHCONF{0.701}{0.005} & \MTCWITHCONF{0.235}{0.001} \\
 &  & ZN & \MTCWITHCONF{80.32}{0.14} & \MTCWITHCONF{0.802}{0.005} & \MTCWITHCONF{0.263}{0.001} & \MTCWITHCONF{84.93}{0.16} & \MTCWITHCONF{0.834}{0.006} & \MTCWITHCONF{0.295}{0.001} & \MTCWITHCONF{86.95}{0.12} & \MTCWITHCONF{0.715}{0.004} & \MTCWITHCONF{0.25}{0.001} \\
 &  & ReRep & \MTCWITHCONF{81.05}{0.14} & \MTCWITHCONF{1.63}{0.003} & \MTCWITHCONF{0.863}{0.001} & \MTCWITHCONF{85.18}{0.16} & \MTCWITHCONF{1.718}{0.004} & \MTCWITHCONF{0.872}{0.001} & \MTCWITHCONF{87.63}{0.12} & \MTCWITHCONF{1.43}{0.002} & \MTCWITHCONF{0.87}{0.001} \\
 &  & EASE & \MTCWITHCONF{79.13}{0.15} & \MTCWITHCONF{0.632}{0.005} & \MTCWITHCONF{0.188}{0.001} & \MTCWITHCONF{84.04}{0.17} & \MTCWITHCONF{0.596}{0.004} & \MTCWITHCONF{0.181}{0.001} & \MTCWITHCONF{86.7}{0.13} & \MTCWITHCONF{0.607}{0.004} & \MTCWITHCONF{0.186}{0.001} \\
 &  & TCPR & \MTCWITHCONF{80.52}{0.16} & \MTCWITHCONF{0.776}{0.005} & \MTCWITHCONF{0.259}{0.001} & \MTCWITHCONF{85.01}{0.17} & \MTCWITHCONF{0.796}{0.005} & \MTCWITHCONF{0.283}{0.001} & \MTCWITHCONF{87.81}{0.13} & \MTCWITHCONF{0.681}{0.004} & \MTCWITHCONF{0.234}{0.001} \\
 &  & noHub & \BEST{\MTCWITHCONF{81.39}{0.15}} & \SECONDBEST{\MTCWITHCONF{0.29}{0.004}} & \SECONDBEST{\MTCWITHCONF{0.096}{0.001}} & \SECONDBEST{\MTCWITHCONF{86.09}{0.16}} & \SECONDBEST{\MTCWITHCONF{0.292}{0.004}} & \SECONDBEST{\MTCWITHCONF{0.103}{0.001}} & \BEST{\MTCWITHCONF{88.16}{0.12}} & \BEST{\MTCWITHCONF{0.336}{0.004}} & \SECONDBEST{\MTCWITHCONF{0.121}{0.001}} \\
 &  & noHub-S & \SECONDBEST{\MTCWITHCONF{81.37}{0.15}} & \BEST{\MTCWITHCONF{0.253}{0.005}} & \BEST{\MTCWITHCONF{0.074}{0.001}} & \BEST{\MTCWITHCONF{86.14}{0.16}} & \BEST{\MTCWITHCONF{0.219}{0.005}} & \BEST{\MTCWITHCONF{0.078}{0.001}} & \SECONDBEST{\MTCWITHCONF{87.97}{0.12}} & \SECONDBEST{\MTCWITHCONF{0.437}{0.006}} & \BEST{\MTCWITHCONF{0.096}{0.001}} \\
\cline{1-12} \cline{2-12}
\bottomrule
\end{tabular}
}
        \caption{Resnet-18: 5-shot}
        \label{tab:main-tim-resnet18-5}
        \end{table*}
    \begin{table*}
        {\scriptsize\centering\begin{tabular}{llllllllllll}
\toprule
 &  &  & \multicolumn{3}{c}{mini} & \multicolumn{3}{c}{tiered} & \multicolumn{3}{c}{CUB} \\
 &  &  & Acc & Skew & Hub.~Occ. & Acc & Skew & Hub.~Occ. & Acc & Skew & Hub.~Occ. \\
Arch. & Clf. & Emb. &  &  &  &  &  &  &  &  &  \\
\midrule
\multirow[c]{54}{*}{\rotatebox{90}{WideRes28-10}} & \multirow[c]{9}{*}{\rotatebox{90}{ILPC}} & None & \MTCWITHCONF{81.93}{0.16} & \MTCWITHCONF{1.717}{0.01} & \MTCWITHCONF{0.473}{0.001} & \MTCWITHCONF{84.34}{0.17} & \MTCWITHCONF{1.927}{0.011} & \MTCWITHCONF{0.509}{0.001} & \MTCWITHCONF{93.18}{0.11} & \MTCWITHCONF{1.164}{0.008} & \MTCWITHCONF{0.396}{0.001} \\
 &  & L2 & \MTCWITHCONF{85.74}{0.14} & \MTCWITHCONF{0.888}{0.005} & \MTCWITHCONF{0.322}{0.001} & \MTCWITHCONF{86.26}{0.17} & \MTCWITHCONF{0.859}{0.005} & \MTCWITHCONF{0.306}{0.001} & \MTCWITHCONF{93.77}{0.1} & \MTCWITHCONF{0.636}{0.004} & \MTCWITHCONF{0.266}{0.001} \\
 &  & CL2 & \MTCWITHCONF{83.33}{0.16} & \MTCWITHCONF{1.12}{0.009} & \MTCWITHCONF{0.318}{0.001} & \MTCWITHCONF{85.99}{0.17} & \MTCWITHCONF{0.957}{0.006} & \MTCWITHCONF{0.338}{0.001} & \MTCWITHCONF{93.79}{0.1} & \MTCWITHCONF{0.703}{0.004} & \MTCWITHCONF{0.309}{0.001} \\
 &  & ZN & \MTCWITHCONF{85.96}{0.14} & \MTCWITHCONF{0.858}{0.005} & \MTCWITHCONF{0.32}{0.001} & \MTCWITHCONF{86.77}{0.16} & \MTCWITHCONF{0.909}{0.006} & \MTCWITHCONF{0.335}{0.001} & \MTCWITHCONF{93.73}{0.1} & \MTCWITHCONF{0.696}{0.004} & \MTCWITHCONF{0.305}{0.001} \\
 &  & ReRep & \MTCWITHCONF{72.11}{0.27} & \MTCWITHCONF{1.601}{0.003} & \MTCWITHCONF{0.819}{0.001} & \MTCWITHCONF{71.68}{0.3} & \MTCWITHCONF{1.616}{0.004} & \MTCWITHCONF{0.845}{0.001} & \MTCWITHCONF{91.52}{0.13} & \MTCWITHCONF{1.301}{0.005} & \MTCWITHCONF{0.548}{0.002} \\
 &  & EASE & \MTCWITHCONF{85.89}{0.14} & \MTCWITHCONF{0.577}{0.004} & \MTCWITHCONF{0.198}{0.001} & \MTCWITHCONF{86.83}{0.17} & \MTCWITHCONF{0.583}{0.004} & \MTCWITHCONF{0.193}{0.001} & \BEST{\MTCWITHCONF{93.87}{0.1}} & \MTCWITHCONF{0.576}{0.004} & \MTCWITHCONF{0.242}{0.001} \\
 &  & TCPR & \SECONDBEST{\MTCWITHCONF{86.29}{0.14}} & \MTCWITHCONF{0.715}{0.004} & \MTCWITHCONF{0.27}{0.001} & \SECONDBEST{\MTCWITHCONF{86.96}{0.17}} & \MTCWITHCONF{0.819}{0.005} & \MTCWITHCONF{0.295}{0.001} & \SECONDBEST{\MTCWITHCONF{93.82}{0.1}} & \MTCWITHCONF{0.634}{0.004} & \MTCWITHCONF{0.265}{0.001} \\
 &  & noHub & \MTCWITHCONF{86.07}{0.15} & \BEST{\MTCWITHCONF{0.295}{0.004}} & \SECONDBEST{\MTCWITHCONF{0.115}{0.001}} & \MTCWITHCONF{86.75}{0.17} & \BEST{\MTCWITHCONF{0.299}{0.004}} & \BEST{\MTCWITHCONF{0.115}{0.001}} & \MTCWITHCONF{93.72}{0.1} & \BEST{\MTCWITHCONF{0.2}{0.004}} & \BEST{\MTCWITHCONF{0.101}{0.001}} \\
 &  & noHub-S & \BEST{\MTCWITHCONF{86.41}{0.14}} & \SECONDBEST{\MTCWITHCONF{0.499}{0.006}} & \BEST{\MTCWITHCONF{0.104}{0.001}} & \BEST{\MTCWITHCONF{87.31}{0.17}} & \SECONDBEST{\MTCWITHCONF{0.406}{0.005}} & \SECONDBEST{\MTCWITHCONF{0.121}{0.001}} & \MTCWITHCONF{93.79}{0.1} & \SECONDBEST{\MTCWITHCONF{0.416}{0.005}} & \SECONDBEST{\MTCWITHCONF{0.126}{0.001}} \\
\cline{2-12}
 & \multirow[c]{9}{*}{\rotatebox{90}{LaplacianShot}} & None & \MTCWITHCONF{85.23}{0.13} & \MTCWITHCONF{1.711}{0.01} & \MTCWITHCONF{0.474}{0.001} & \MTCWITHCONF{86.14}{0.15} & \MTCWITHCONF{1.921}{0.011} & \MTCWITHCONF{0.509}{0.001} & \MTCWITHCONF{92.61}{0.1} & \MTCWITHCONF{1.164}{0.008} & \MTCWITHCONF{0.395}{0.001} \\
 &  & L2 & \MTCWITHCONF{85.9}{0.13} & \MTCWITHCONF{0.892}{0.006} & \MTCWITHCONF{0.321}{0.001} & \MTCWITHCONF{86.47}{0.15} & \MTCWITHCONF{0.867}{0.006} & \MTCWITHCONF{0.304}{0.001} & \MTCWITHCONF{93.17}{0.09} & \MTCWITHCONF{0.635}{0.004} & \MTCWITHCONF{0.267}{0.001} \\
 &  & CL2 & \MTCWITHCONF{82.08}{0.15} & \MTCWITHCONF{1.112}{0.009} & \MTCWITHCONF{0.318}{0.001} & \MTCWITHCONF{84.62}{0.16} & \MTCWITHCONF{0.954}{0.006} & \MTCWITHCONF{0.34}{0.001} & \MTCWITHCONF{93.01}{0.1} & \MTCWITHCONF{0.702}{0.004} & \MTCWITHCONF{0.309}{0.001} \\
 &  & ZN & \MTCWITHCONF{85.97}{0.13} & \MTCWITHCONF{0.86}{0.005} & \MTCWITHCONF{0.319}{0.001} & \MTCWITHCONF{86.67}{0.15} & \MTCWITHCONF{0.912}{0.006} & \MTCWITHCONF{0.335}{0.001} & \MTCWITHCONF{93.3}{0.1} & \MTCWITHCONF{0.698}{0.004} & \MTCWITHCONF{0.305}{0.001} \\
 &  & ReRep & \MTCWITHCONF{84.34}{0.14} & \MTCWITHCONF{1.599}{0.003} & \MTCWITHCONF{0.819}{0.001} & \MTCWITHCONF{85.61}{0.16} & \MTCWITHCONF{1.615}{0.004} & \MTCWITHCONF{0.845}{0.001} & \MTCWITHCONF{92.2}{0.1} & \MTCWITHCONF{1.304}{0.005} & \MTCWITHCONF{0.549}{0.002} \\
 &  & EASE & \SECONDBEST{\MTCWITHCONF{86.24}{0.13}} & \MTCWITHCONF{0.573}{0.004} & \MTCWITHCONF{0.198}{0.001} & \SECONDBEST{\MTCWITHCONF{86.74}{0.15}} & \MTCWITHCONF{0.582}{0.004} & \MTCWITHCONF{0.194}{0.001} & \MTCWITHCONF{93.31}{0.09} & \MTCWITHCONF{0.578}{0.004} & \MTCWITHCONF{0.243}{0.001} \\
 &  & TCPR & \MTCWITHCONF{86.16}{0.13} & \MTCWITHCONF{0.712}{0.004} & \MTCWITHCONF{0.269}{0.001} & \MTCWITHCONF{85.72}{0.16} & \MTCWITHCONF{0.813}{0.005} & \MTCWITHCONF{0.293}{0.001} & \MTCWITHCONF{92.99}{0.1} & \MTCWITHCONF{0.638}{0.004} & \MTCWITHCONF{0.264}{0.001} \\
 &  & noHub & \BEST{\MTCWITHCONF{86.25}{0.13}} & \BEST{\MTCWITHCONF{0.292}{0.004}} & \SECONDBEST{\MTCWITHCONF{0.115}{0.001}} & \BEST{\MTCWITHCONF{86.78}{0.16}} & \BEST{\MTCWITHCONF{0.299}{0.004}} & \BEST{\MTCWITHCONF{0.115}{0.001}} & \BEST{\MTCWITHCONF{93.38}{0.09}} & \BEST{\MTCWITHCONF{0.197}{0.004}} & \BEST{\MTCWITHCONF{0.1}{0.001}} \\
 &  & noHub-S & \MTCWITHCONF{85.79}{0.13} & \SECONDBEST{\MTCWITHCONF{0.494}{0.006}} & \BEST{\MTCWITHCONF{0.103}{0.001}} & \MTCWITHCONF{86.44}{0.16} & \SECONDBEST{\MTCWITHCONF{0.397}{0.005}} & \SECONDBEST{\MTCWITHCONF{0.12}{0.001}} & \SECONDBEST{\MTCWITHCONF{93.36}{0.1}} & \SECONDBEST{\MTCWITHCONF{0.42}{0.005}} & \SECONDBEST{\MTCWITHCONF{0.126}{0.001}} \\
\cline{2-12}
 & \multirow[c]{9}{*}{\rotatebox{90}{ObliqueManifold}} & None & \MTCWITHCONF{87.46}{0.13} & \MTCWITHCONF{1.712}{0.01} & \MTCWITHCONF{0.472}{0.001} & \SECONDBEST{\MTCWITHCONF{88.16}{0.15}} & \MTCWITHCONF{1.913}{0.01} & \MTCWITHCONF{0.509}{0.001} & \MTCWITHCONF{94.75}{0.09} & \MTCWITHCONF{1.161}{0.008} & \MTCWITHCONF{0.395}{0.001} \\
 &  & L2 & \MTCWITHCONF{87.61}{0.13} & \MTCWITHCONF{0.889}{0.005} & \MTCWITHCONF{0.321}{0.001} & \MTCWITHCONF{88.14}{0.15} & \MTCWITHCONF{0.862}{0.006} & \MTCWITHCONF{0.306}{0.001} & \BEST{\MTCWITHCONF{94.8}{0.09}} & \MTCWITHCONF{0.642}{0.004} & \MTCWITHCONF{0.268}{0.001} \\
 &  & CL2 & \MTCWITHCONF{86.03}{0.14} & \MTCWITHCONF{1.112}{0.009} & \MTCWITHCONF{0.317}{0.001} & \MTCWITHCONF{87.64}{0.16} & \MTCWITHCONF{0.949}{0.006} & \MTCWITHCONF{0.338}{0.001} & \MTCWITHCONF{94.67}{0.09} & \MTCWITHCONF{0.703}{0.004} & \MTCWITHCONF{0.31}{0.001} \\
 &  & ZN & \SECONDBEST{\MTCWITHCONF{87.88}{0.13}} & \MTCWITHCONF{0.852}{0.005} & \MTCWITHCONF{0.32}{0.001} & \BEST{\MTCWITHCONF{88.43}{0.15}} & \MTCWITHCONF{0.908}{0.006} & \MTCWITHCONF{0.335}{0.001} & \SECONDBEST{\MTCWITHCONF{94.77}{0.08}} & \MTCWITHCONF{0.697}{0.004} & \MTCWITHCONF{0.306}{0.001} \\
 &  & ReRep & \MTCWITHCONF{87.62}{0.12} & \MTCWITHCONF{1.599}{0.003} & \MTCWITHCONF{0.819}{0.001} & \MTCWITHCONF{88.15}{0.15} & \MTCWITHCONF{1.616}{0.004} & \MTCWITHCONF{0.845}{0.001} & \MTCWITHCONF{94.48}{0.09} & \MTCWITHCONF{1.302}{0.005} & \MTCWITHCONF{0.547}{0.002} \\
 &  & EASE & \MTCWITHCONF{86.75}{0.13} & \MTCWITHCONF{0.573}{0.004} & \MTCWITHCONF{0.198}{0.001} & \MTCWITHCONF{87.78}{0.15} & \MTCWITHCONF{0.583}{0.004} & \MTCWITHCONF{0.193}{0.001} & \MTCWITHCONF{94.16}{0.09} & \MTCWITHCONF{0.57}{0.004} & \MTCWITHCONF{0.24}{0.001} \\
 &  & TCPR & \BEST{\MTCWITHCONF{87.94}{0.12}} & \MTCWITHCONF{0.718}{0.004} & \MTCWITHCONF{0.271}{0.001} & \MTCWITHCONF{88.15}{0.15} & \MTCWITHCONF{0.816}{0.005} & \MTCWITHCONF{0.294}{0.001} & \MTCWITHCONF{94.47}{0.09} & \MTCWITHCONF{0.635}{0.004} & \MTCWITHCONF{0.265}{0.001} \\
 &  & noHub & \MTCWITHCONF{87.23}{0.13} & \BEST{\MTCWITHCONF{0.297}{0.004}} & \SECONDBEST{\MTCWITHCONF{0.115}{0.001}} & \MTCWITHCONF{87.95}{0.16} & \BEST{\MTCWITHCONF{0.296}{0.004}} & \BEST{\MTCWITHCONF{0.114}{0.001}} & \MTCWITHCONF{94.13}{0.09} & \BEST{\MTCWITHCONF{0.197}{0.004}} & \BEST{\MTCWITHCONF{0.1}{0.001}} \\
 &  & noHub-S & \MTCWITHCONF{87.13}{0.14} & \SECONDBEST{\MTCWITHCONF{0.495}{0.006}} & \BEST{\MTCWITHCONF{0.103}{0.001}} & \MTCWITHCONF{87.84}{0.16} & \SECONDBEST{\MTCWITHCONF{0.399}{0.005}} & \SECONDBEST{\MTCWITHCONF{0.12}{0.001}} & \MTCWITHCONF{94.06}{0.09} & \SECONDBEST{\MTCWITHCONF{0.421}{0.005}} & \SECONDBEST{\MTCWITHCONF{0.126}{0.001}} \\
\cline{2-12}
 & \multirow[c]{9}{*}{\rotatebox{90}{SIAMESE}} & None & \MTCWITHCONF{58.82}{0.31} & \MTCWITHCONF{1.722}{0.01} & \MTCWITHCONF{0.473}{0.001} & \MTCWITHCONF{82.56}{0.22} & \MTCWITHCONF{1.93}{0.01} & \MTCWITHCONF{0.511}{0.001} & \MTCWITHCONF{82.22}{0.37} & \MTCWITHCONF{1.154}{0.008} & \MTCWITHCONF{0.396}{0.001} \\
 &  & L2 & \MTCWITHCONF{87.11}{0.13} & \MTCWITHCONF{0.894}{0.006} & \MTCWITHCONF{0.321}{0.001} & \MTCWITHCONF{87.34}{0.15} & \MTCWITHCONF{0.861}{0.005} & \MTCWITHCONF{0.305}{0.001} & \MTCWITHCONF{94.15}{0.1} & \MTCWITHCONF{0.638}{0.004} & \MTCWITHCONF{0.266}{0.001} \\
 &  & CL2 & \MTCWITHCONF{83.99}{0.16} & \MTCWITHCONF{1.107}{0.009} & \MTCWITHCONF{0.318}{0.001} & \MTCWITHCONF{86.71}{0.16} & \MTCWITHCONF{0.953}{0.006} & \MTCWITHCONF{0.339}{0.001} & \MTCWITHCONF{94.48}{0.09} & \MTCWITHCONF{0.704}{0.004} & \MTCWITHCONF{0.31}{0.001} \\
 &  & ZN & \MTCWITHCONF{20.0}{0.0} & \MTCWITHCONF{0.856}{0.005} & \MTCWITHCONF{0.319}{0.001} & \MTCWITHCONF{20.0}{0.0} & \MTCWITHCONF{0.913}{0.006} & \MTCWITHCONF{0.334}{0.001} & \MTCWITHCONF{20.0}{0.0} & \MTCWITHCONF{0.702}{0.004} & \MTCWITHCONF{0.305}{0.001} \\
 &  & ReRep & \MTCWITHCONF{36.41}{0.3} & \MTCWITHCONF{1.597}{0.003} & \MTCWITHCONF{0.818}{0.001} & \MTCWITHCONF{76.49}{0.24} & \MTCWITHCONF{1.613}{0.004} & \MTCWITHCONF{0.846}{0.001} & \MTCWITHCONF{60.36}{0.6} & \MTCWITHCONF{1.299}{0.005} & \MTCWITHCONF{0.547}{0.002} \\
 &  & EASE & \SECONDBEST{\MTCWITHCONF{87.82}{0.13}} & \MTCWITHCONF{0.579}{0.004} & \MTCWITHCONF{0.199}{0.001} & \SECONDBEST{\MTCWITHCONF{88.06}{0.16}} & \MTCWITHCONF{0.586}{0.004} & \MTCWITHCONF{0.192}{0.001} & \MTCWITHCONF{94.36}{0.09} & \MTCWITHCONF{0.571}{0.004} & \MTCWITHCONF{0.241}{0.001} \\
 &  & TCPR & \MTCWITHCONF{87.8}{0.13} & \MTCWITHCONF{0.717}{0.004} & \MTCWITHCONF{0.27}{0.001} & \MTCWITHCONF{87.95}{0.16} & \MTCWITHCONF{0.822}{0.005} & \MTCWITHCONF{0.295}{0.001} & \MTCWITHCONF{94.25}{0.1} & \MTCWITHCONF{0.637}{0.004} & \MTCWITHCONF{0.266}{0.001} \\
 &  & noHub & \MTCWITHCONF{87.78}{0.14} & \BEST{\MTCWITHCONF{0.29}{0.004}} & \SECONDBEST{\MTCWITHCONF{0.114}{0.001}} & \MTCWITHCONF{87.99}{0.17} & \BEST{\MTCWITHCONF{0.297}{0.004}} & \BEST{\MTCWITHCONF{0.115}{0.001}} & \SECONDBEST{\MTCWITHCONF{94.56}{0.09}} & \BEST{\MTCWITHCONF{0.196}{0.004}} & \BEST{\MTCWITHCONF{0.1}{0.001}} \\
 &  & noHub-S & \BEST{\MTCWITHCONF{88.03}{0.13}} & \SECONDBEST{\MTCWITHCONF{0.492}{0.006}} & \BEST{\MTCWITHCONF{0.103}{0.001}} & \BEST{\MTCWITHCONF{88.31}{0.16}} & \SECONDBEST{\MTCWITHCONF{0.398}{0.005}} & \SECONDBEST{\MTCWITHCONF{0.12}{0.001}} & \BEST{\MTCWITHCONF{94.69}{0.09}} & \SECONDBEST{\MTCWITHCONF{0.416}{0.005}} & \SECONDBEST{\MTCWITHCONF{0.127}{0.001}} \\
\cline{2-12}
 & \multirow[c]{9}{*}{\rotatebox{90}{SimpleShot}} & None & \MTCWITHCONF{78.56}{0.14} & \MTCWITHCONF{1.709}{0.01} & \MTCWITHCONF{0.473}{0.001} & \MTCWITHCONF{80.32}{0.16} & \MTCWITHCONF{1.937}{0.01} & \MTCWITHCONF{0.51}{0.001} & \MTCWITHCONF{89.27}{0.11} & \MTCWITHCONF{1.16}{0.008} & \MTCWITHCONF{0.395}{0.001} \\
 &  & L2 & \MTCWITHCONF{83.81}{0.13} & \MTCWITHCONF{0.887}{0.005} & \MTCWITHCONF{0.322}{0.001} & \MTCWITHCONF{84.82}{0.15} & \MTCWITHCONF{0.86}{0.006} & \MTCWITHCONF{0.305}{0.001} & \MTCWITHCONF{92.06}{0.1} & \MTCWITHCONF{0.632}{0.004} & \MTCWITHCONF{0.266}{0.001} \\
 &  & CL2 & \MTCWITHCONF{81.05}{0.14} & \MTCWITHCONF{1.12}{0.009} & \MTCWITHCONF{0.318}{0.001} & \MTCWITHCONF{83.82}{0.16} & \MTCWITHCONF{0.956}{0.006} & \MTCWITHCONF{0.337}{0.001} & \MTCWITHCONF{92.19}{0.1} & \MTCWITHCONF{0.701}{0.004} & \MTCWITHCONF{0.31}{0.001} \\
 &  & ZN & \MTCWITHCONF{83.92}{0.13} & \MTCWITHCONF{0.858}{0.005} & \MTCWITHCONF{0.32}{0.001} & \MTCWITHCONF{85.1}{0.15} & \MTCWITHCONF{0.912}{0.006} & \MTCWITHCONF{0.335}{0.001} & \MTCWITHCONF{92.17}{0.1} & \MTCWITHCONF{0.699}{0.004} & \MTCWITHCONF{0.305}{0.001} \\
 &  & ReRep & \MTCWITHCONF{79.26}{0.16} & \MTCWITHCONF{1.597}{0.003} & \MTCWITHCONF{0.819}{0.001} & \MTCWITHCONF{82.7}{0.16} & \MTCWITHCONF{1.617}{0.004} & \MTCWITHCONF{0.846}{0.001} & \MTCWITHCONF{91.48}{0.11} & \MTCWITHCONF{1.299}{0.005} & \MTCWITHCONF{0.549}{0.002} \\
 &  & EASE & \MTCWITHCONF{83.65}{0.13} & \MTCWITHCONF{0.579}{0.004} & \MTCWITHCONF{0.199}{0.001} & \MTCWITHCONF{84.47}{0.15} & \MTCWITHCONF{0.585}{0.004} & \MTCWITHCONF{0.193}{0.001} & \MTCWITHCONF{92.01}{0.1} & \MTCWITHCONF{0.572}{0.004} & \MTCWITHCONF{0.241}{0.001} \\
 &  & TCPR & \MTCWITHCONF{83.77}{0.13} & \MTCWITHCONF{0.717}{0.004} & \MTCWITHCONF{0.27}{0.001} & \MTCWITHCONF{84.81}{0.15} & \MTCWITHCONF{0.815}{0.005} & \MTCWITHCONF{0.294}{0.001} & \MTCWITHCONF{91.84}{0.1} & \MTCWITHCONF{0.634}{0.004} & \MTCWITHCONF{0.264}{0.001} \\
 &  & noHub & \BEST{\MTCWITHCONF{85.73}{0.13}} & \BEST{\MTCWITHCONF{0.294}{0.004}} & \SECONDBEST{\MTCWITHCONF{0.115}{0.001}} & \BEST{\MTCWITHCONF{86.58}{0.15}} & \BEST{\MTCWITHCONF{0.298}{0.004}} & \BEST{\MTCWITHCONF{0.115}{0.001}} & \SECONDBEST{\MTCWITHCONF{93.21}{0.09}} & \BEST{\MTCWITHCONF{0.195}{0.004}} & \BEST{\MTCWITHCONF{0.1}{0.001}} \\
 &  & noHub-S & \SECONDBEST{\MTCWITHCONF{84.39}{0.13}} & \SECONDBEST{\MTCWITHCONF{0.494}{0.006}} & \BEST{\MTCWITHCONF{0.103}{0.001}} & \SECONDBEST{\MTCWITHCONF{86.38}{0.15}} & \SECONDBEST{\MTCWITHCONF{0.407}{0.005}} & \SECONDBEST{\MTCWITHCONF{0.12}{0.001}} & \BEST{\MTCWITHCONF{93.39}{0.09}} & \SECONDBEST{\MTCWITHCONF{0.421}{0.005}} & \SECONDBEST{\MTCWITHCONF{0.127}{0.001}} \\
\cline{2-12}
 & \multirow[c]{9}{*}{\rotatebox{90}{\( \alpha \)-TIM}} & None & \MTCWITHCONF{80.61}{0.15} & \MTCWITHCONF{1.711}{0.01} & \MTCWITHCONF{0.473}{0.001} & \MTCWITHCONF{83.05}{0.18} & \MTCWITHCONF{1.928}{0.01} & \MTCWITHCONF{0.51}{0.001} & \MTCWITHCONF{84.89}{0.29} & \MTCWITHCONF{1.153}{0.008} & \MTCWITHCONF{0.396}{0.001} \\
 &  & L2 & \MTCWITHCONF{83.71}{0.16} & \MTCWITHCONF{0.892}{0.005} & \MTCWITHCONF{0.323}{0.001} & \MTCWITHCONF{84.69}{0.18} & \MTCWITHCONF{0.863}{0.005} & \MTCWITHCONF{0.304}{0.001} & \MTCWITHCONF{92.88}{0.1} & \MTCWITHCONF{0.633}{0.004} & \MTCWITHCONF{0.266}{0.001} \\
 &  & CL2 & \MTCWITHCONF{82.35}{0.16} & \MTCWITHCONF{1.111}{0.009} & \MTCWITHCONF{0.318}{0.001} & \MTCWITHCONF{84.06}{0.18} & \MTCWITHCONF{0.949}{0.006} & \MTCWITHCONF{0.339}{0.001} & \MTCWITHCONF{92.81}{0.1} & \MTCWITHCONF{0.7}{0.004} & \MTCWITHCONF{0.31}{0.001} \\
 &  & ZN & \MTCWITHCONF{83.93}{0.13} & \MTCWITHCONF{0.857}{0.005} & \MTCWITHCONF{0.321}{0.001} & \MTCWITHCONF{85.07}{0.15} & \MTCWITHCONF{0.912}{0.006} & \MTCWITHCONF{0.336}{0.001} & \MTCWITHCONF{92.15}{0.1} & \MTCWITHCONF{0.698}{0.004} & \MTCWITHCONF{0.306}{0.001} \\
 &  & ReRep & \MTCWITHCONF{83.4}{0.14} & \MTCWITHCONF{1.596}{0.003} & \MTCWITHCONF{0.82}{0.001} & \MTCWITHCONF{84.4}{0.16} & \MTCWITHCONF{1.615}{0.004} & \MTCWITHCONF{0.845}{0.001} & \SECONDBEST{\MTCWITHCONF{93.19}{0.09}} & \MTCWITHCONF{1.302}{0.005} & \MTCWITHCONF{0.547}{0.002} \\
 &  & EASE & \MTCWITHCONF{82.72}{0.14} & \MTCWITHCONF{0.576}{0.004} & \MTCWITHCONF{0.2}{0.001} & \MTCWITHCONF{83.86}{0.16} & \MTCWITHCONF{0.583}{0.004} & \MTCWITHCONF{0.193}{0.001} & \MTCWITHCONF{92.31}{0.1} & \MTCWITHCONF{0.572}{0.004} & \MTCWITHCONF{0.242}{0.001} \\
 &  & TCPR & \SECONDBEST{\MTCWITHCONF{84.21}{0.15}} & \MTCWITHCONF{0.718}{0.004} & \MTCWITHCONF{0.27}{0.001} & \MTCWITHCONF{84.63}{0.18} & \MTCWITHCONF{0.814}{0.005} & \MTCWITHCONF{0.293}{0.001} & \MTCWITHCONF{92.44}{0.1} & \MTCWITHCONF{0.635}{0.004} & \MTCWITHCONF{0.265}{0.001} \\
 &  & noHub & \BEST{\MTCWITHCONF{85.56}{0.13}} & \BEST{\MTCWITHCONF{0.293}{0.004}} & \SECONDBEST{\MTCWITHCONF{0.115}{0.001}} & \BEST{\MTCWITHCONF{86.37}{0.16}} & \BEST{\MTCWITHCONF{0.3}{0.004}} & \BEST{\MTCWITHCONF{0.115}{0.001}} & \MTCWITHCONF{92.89}{0.1} & \BEST{\MTCWITHCONF{0.193}{0.004}} & \BEST{\MTCWITHCONF{0.099}{0.001}} \\
 &  & noHub-S & \MTCWITHCONF{83.96}{0.15} & \SECONDBEST{\MTCWITHCONF{0.496}{0.006}} & \BEST{\MTCWITHCONF{0.102}{0.001}} & \SECONDBEST{\MTCWITHCONF{86.01}{0.16}} & \SECONDBEST{\MTCWITHCONF{0.395}{0.005}} & \SECONDBEST{\MTCWITHCONF{0.12}{0.001}} & \BEST{\MTCWITHCONF{93.24}{0.1}} & \SECONDBEST{\MTCWITHCONF{0.422}{0.005}} & \SECONDBEST{\MTCWITHCONF{0.126}{0.001}} \\
\cline{1-12} \cline{2-12}
\bottomrule
\end{tabular}
}
        \caption{WideRes28-10: 5-shot}
        \label{tab:main-s2m2-wrn-s2m2-5}
    \end{table*}
