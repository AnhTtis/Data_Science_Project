\begin{abstract}
% Named entity recognition (NER) is an important research problem in natural language processing. 
% There are three different types of NER tasks, including flat, nested and discontinuous entity recognition. 
% Most previous sequential labeling models are task-specific, while recent years witness the rising of generative models due to the advantage of unifying all NER tasks into the seq2seq model framework. 
% Although achieving promising performance, our pilot studies demonstrate that existing generative models are not effective at detecting entity boundaries and estimating entity types. 
% In this paper, we propose a multi-task Transformer, which incorporates an entity boundary detection task into the named entity recognition task.
% More concretely, we achieve the entity boundary detection by classifying the relations between tokens within the sentence.
% To improve the accuracy of entity-type mapping during decoding, we adopt an external knowledge base to calculate the prior entity-type distributions and then incorporate the information into the model via the self and cross attention mechanisms.
% We perform experiments on an extensive set of NER benchmarks, including two flat, three nested, and three discontinuous NER datasets.
% Experimental results show that our approach improves the performance of the generative NER model considerably.
Named entity recognition (NER) is an important research problem in natural language processing. There are three types of NER tasks, including flat, nested and discontinuous entity recognition. Most previous sequential labeling models are task-specific, while recent years have witnessed the rising of generative models due to the advantage of unifying all NER tasks into the seq2seq model framework. Although achieving promising performance, our pilot studies demonstrate that existing generative models are ineffective at detecting entity boundaries and estimating entity types. This paper proposes a multi-task Transformer, which incorporates an entity boundary detection task into the named entity recognition task. More concretely, we achieve entity boundary detection by classifying the relations between tokens within the sentence. To improve the accuracy of entity-type mapping during decoding, we adopt an external knowledge base to calculate the prior entity-type distributions and then incorporate the information into the model via the self and cross-attention mechanisms. We perform experiments on an extensive set of NER benchmarks, including two flat, three nested, and three discontinuous NER datasets. Experimental results show that our approach considerably improves the generative NER model's performance.
\end{abstract}
%
\begin{keywords}    
Named Entity Recognition, Seq2seq Model, Multi-task, Attention
\end{keywords}
%