%%%%%%%%%%%%%%%%%%%%%%%%%%%%%%%%%%%%%%%%%%%%%%%%%%%%%%%%%%%%%%%%%%%%%%%%%%%%%%%%
%2345678901234567890123456789012345678901234567890123456789012345678901234567890
%        1         2         3         4         5         6         7         8

\documentclass[letterpaper, 10 pt, conference]{ieeeconf}  % Comment this line out if you need a4paper

%\documentclass[a4paper, 10pt, conference]{ieeeconf}      % Use this line for a4 paper

\IEEEoverridecommandlockouts                              % This command is only needed if 
                                                          % you want to use the \thanks command

\overrideIEEEmargins                                      % Needed to meet printer requirements.

%In case you encounter the following error:
%Error 1010 The PDF file may be corrupt (unable to open PDF file) OR
%Error 1000 An error occurred while parsing a contents stream. Unable to analyze the PDF file.
%This is a known problem with pdfLaTeX conversion filter. The file cannot be opened with acrobat reader
%Please use one of the alternatives below to circumvent this error by uncommenting one or the other
%\pdfobjcompresslevel=0
%\pdfminorversion=4

% See the \addtolength command later in the file to balance the column lengths
% on the last page of the document

% The following packages can be found on http:\\www.ctan.org
%\usepackage{graphics} % for pdf, bitmapped graphics files
%\usepackage{epsfig} % for postscript graphics files
%\usepackage{mathptmx} % assumes new font selection scheme installed
%\usepackage{times} % assumes new font selection scheme installed
\usepackage{amsmath} % assumes amsmath package installed
\usepackage{amssymb}  % assumes amsmath package installed
\usepackage{algorithm}
\usepackage{algpseudocode}
\usepackage{cite}
\usepackage{graphicx}
\usepackage{caption}
\usepackage{subcaption}
% \usepackage{array}
\usepackage{tabularx}
\usepackage{url}
\renewcommand\tabularxcolumn[1]{m{#1}} % for vertical centering text in X column

\title{\LARGE \bf
% Sim-to-real Transfer of Policies on Buoyancy Assisted Legged Robots via System Identification and Residual Dynamics Learning
Residual Physics Learning and System Identification for Sim-to-real Transfer of Policies on Buoyancy Assisted Legged Robots 
% Sim-to-real Transfer of Policies on Buoyancy Assisted Legged Robots via System Identification and Residual Dynamics Learning
}


\author{Nitish Sontakke$^{1}$ Hosik Chae$^{2}$ Sangjoon Lee$^{3}$ Tianle Huang$^{1}$ Dennis W. Hong$^{2}$ Sehoon Ha$^{1}$% <-this % stops a space
% \author{Albert Author$^{1}$ and Bernard D. Researcher$^{2}$% <-this % stops a space
% \thanks{*This work was not supported by any organization}% <-this % stops a space
\thanks{$^{1}$School of Interactive Computing, Georgia Institute of Technology, Atlanta, GA, 30308, USA. {\tt \small nitishsontakke@gatech.edu, thuang325@gatech.edu sehoonha@gatech.edu}}%
\thanks{$^{2}$Department of Mechanical and Aerospace Engineering, University of California, Los Angeles (UCLA), Los Angeles, CA, 90095, USA. {\tt\small hosikchae@ucla.edu, dennishong@ucla.edu}}%
\thanks{$^{3}$Department of Computer Science, 
University of California, Los Angeles (UCLA), Los Angeles, CA, 90095, USA. {\tt\small sangjoonlee@cs.ucla.edu}}%
}

%%%%% GENERAL MATH COMMANDS
% Reals
\newcommand{\R}{{\mathbb R}}
% Integers
\newcommand{\Z}{{\mathbb Z}}
% Naturals
\newcommand{\N}{{\mathbb N}}
% Expectation
\DeclareMathOperator*{\E}{\mathbb{E}}
% ^th notation
\newcommand{\tth}{^{\text{th}}}
% Small dots for integer range [a .. b]
\newcommand{\sdots}{\,..\,}
% Vectorized version of matrix
\newcommand{\matvec}{\mbox{vec}}

% := sign
\newcommand{\defeq}{\vcentcolon=}
% Zero function
\newcommand{\zf}{\mathbf{0}}
% Vector of ones
\newcommand{\ones}{\mathbf{1}}

% Argmin and argmax definitions
\DeclareMathOperator*{\argmax}{arg\,max}
\DeclareMathOperator*{\argmin}{arg\,min}


%%%%% PROBLEM STATEMENT NOTATION 
% \newcommandtwoopt{\St}[2][t][]{{S_{#1}^{#2}}} % State
\newcommand{\task}[1][i]{{\mathcal{T}_{#1}}} % Task, optionally takes index
\newcommand{\tasks}{\{ \task \}_{i=1}^N}
\newcommand{\losst}[1][i]{{l_{#1}}}
\newcommand{\lossv}[1][i]{{l_{#1}^{\textrm{val}}}}
\newcommand{\tasktarget}{{\mathcal{T}_{\textrm{target}}}}
\newcommand{\lossttarget}{l_{\textrm{target}}}
\newcommand{\lossvtarget}{l_{\textrm{target}}^{\textrm{val}}}
\newcommand{\lossttargetit}{l_{\textrm{target}}^{(k)}}
\newcommand{\losstotal}{l^{\textrm{total}}}
\newcommand{\lossopt}{l^*}

\newcommand{\thetait}[2]{\theta_{#1}^{(#2)}}
\newcommand{\phit}[1]{\phi^{(#1)}}
\newcommand{\hist}[2]{S_{#1}^{(#2)}}
\newcommand{\grad}[2]{G_{#1}^{(#2)}}

\newcommand{\Alg}{\textup{\textbf{Opt}}}
\newcommand{\MetaAlg}{\textup{\textbf{MetaOpt}}}

%%%%% Theorems
\newtheoremstyle{mytheoremstyle} % name
    {\topsep}                    % Space above
    {\topsep}                    % Space below
    {\itshape}                   % Body font
    {}                           % Indent amount
    {\scshape}                   % Theorem head font
    {.}                          % Punctuation after theorem head
    {.5em}                       % Space after theorem head
    {}  % Theorem head spec (can be left empty, meaning ‘normal’)
\theoremstyle{mytheoremstyle}
\theoremstyle{plain}
\newtheorem{theorem}{Theorem}
\newtheorem{proposition}{Proposition}
\newtheorem{assumption}{Assumption}
\newtheorem{definition}{Definition}
\newtheorem{lemma}{Lemma}
\theoremstyle{remark}
\newtheorem{remark}{Remark}


\begin{document}



\maketitle
\thispagestyle{empty}
\pagestyle{empty}


%%%%%%%%%%%%%%%%%%%%%%%%%%%%%%%%%%%%%%%%%%%%%%%%%%%%%%%%%%%%%%%%%%%%%%%%%%%%%%%%
\begin{abstract}

The light and soft characteristics of Buoyancy Assisted Lightweight Legged Unit (BALLU) robots have a great potential to provide intrinsically safe interactions in environments involving humans, unlike many heavy and rigid robots. However, their unique and sensitive dynamics impose challenges to obtaining robust control policies in the real world. In this work, we demonstrate robust sim-to-real transfer of control policies on the BALLU robots via system identification and our novel residual physics learning method, Environment Mimic (EnvMimic). First, we model the nonlinear dynamics of the actuators by collecting hardware data and optimizing the simulation parameters. Rather than relying on standard supervised learning formulations, we utilize deep reinforcement learning to train an external force policy to match real-world trajectories, which enables us to model residual physics with greater fidelity. We analyze the improved simulation fidelity by comparing the simulation trajectories against the real-world ones. We finally demonstrate that the improved simulator allows us to learn better walking and turning policies that can be successfully deployed on the hardware of BALLU.

\end{abstract}


%%%%%%%%%%%%%%%%%%%%%%%%%%%%%%%%%%%%%%%%%%%%%%%%%%%%%%%%%%%%%%%%%%%%%%%%%%%%%%%%
\section{Introduction}
% 1. Soft robots are cheap and safe, despite being less effective. BALLU!
Buoyancy-assisted or balloon-based robots \cite{chae2021ballu2,balloonhumanoid2015,inflatablearm2017} have great potential to offer fundamental safety in human environments. Traditional mobile robots while being able to execute a variety of tasks, tend to be rigid and heavy and may cause serious damage to their surroundings or themselves in case of control or perception errors. On the other hand, buoyancy-assisted robots (BARs) \cite{williams2021introduction} are typically designed to be lightweight, compact, and intrinsically safe. Therefore, they can be used for various applications that require close human-robot interaction, such as education, entertainment, and healthcare. For instance, Chae \etal~\cite{chae2021ballu2} present Buoyancy Assisted Lightweight Legged Unit (BALLU), which is a balloon-based robot with two legs (\figref{teaser}), and showcase that it can be deployed to various indoor and outdoor environments without any safety concerns.

% \begin{figure}
%     \centering
%     \begin{subfigure}{0.95\linewidth}
%     \includegraphics[width=\linewidth]{figures/Motion_strip_HW.jpg}
%     \label{fig:ballu_full_side}
%     % \caption{Side view}
%     \end{subfigure}
%     \hfill
%     \begin{subfigure}{0.95\linewidth}
%     \includegraphics[width=\linewidth]{figures/Motion_strip_sim.png}
%     \label{fig:ballu_knee}
%     % \caption{Motor arm, cables, knee joint}
%     \end{subfigure}
%     \caption{Teaser \nitish{Insert links to the videos}}
%     \label{fig:ballu}
% \end{figure}
\begin{figure}
    \centering
    \includegraphics[width=\linewidth]{figures/Motion_strip_HW.jpg}
    \caption{An image of the successful policy for the forward walking task, which is trained in the improved simulation using our method.}
    \label{fig:teaser}
\end{figure}

% 2. BALLU are sensitive and hard to control. MPC NG. Deep RL NG.
However, it is not straightforward to control BARs due to their unique, non-linear, and sensitive dynamics. One popular approach for robot control is model-predictive control (MPC)~\cite{shen2021novel} which plans future trajectories via models and minimizes the provided cost function. The complex dynamics of BARs, however, prevent us from developing concise and effective models and therefore rule out MPC as a control method. In contrast, deep reinforcement learning (deep RL) offers an automated approach to training a control policy from a simple reward function without robot-specific models. On the flip side, policies trained with deep RL often experience severe performance degradation when deployed to the robot due to the difference between the simulated and real-world environments, which is commonly known as the sim-to-real gap or the reality gap~\cite{jakobi1995noise}. In our experience, this gap is further compounded in the case of BALLU when we employ a vanilla rigid body simulator, such as PyBullet~\cite{coumans2021} or CoppeliaSim~\cite{coppeliaSim}, due to the unmodeled aerodynamics and low-fidelity actuators.

% 3. We propose a new technique for reducing sim-to-real
In this work, we mitigate the sim-to-real gap of the BALLU robot by identifying system parameters and modeling residual dynamics using a novel technique, \emph{EnvMimic}. First, we iteratively tune the actuator parameters in simulation based on the data collected from hardware experiments. This system identification allows us to better illustrate the nonlinear dynamics of BALLU's cable-driven actuation mechanism. Second, we learn the residual physics of the BALLU robot from the collected real-world trajectories to capture its complex aerodynamics that are difficult to model analytically. To this end, we propose a novel technique, Environment Mimic (EnvMimic), which learns to generate external forces to match the simulation and real-world trajectories via deep RL, which is different from common supervised learning formulations~\cite{ajay2018augmenting, golemo2018sim, bauersfeld2021neurobem, o2022neural}. This is similar to using pseudo-forces like centrifugal force or Coriolis force to explain the observed behavior. Our approach can also be viewed as an inside-out flipped version of the recent motion imitation frameworks~\cite{peng2018deepmimic, peng2021amp, peng2022ase, escontrela2022adversarial}, which learn internal controllers that enable the robot to imitate reference motions. In our case, we treat the real-world trajectories as a reference and learn an external residual force policy to imitate that behavior in simulation. We also observe that EnvMimic exhibits a robust generalization capability, even when we have a small number of trajectories.

% 4. Results are:
We demonstrate that the proposed techniques can successfully reduce the sim-to-real gap of the BALLU robot. Firstly, we show that modeling the actuators and capturing the aerodynamics results in a significantly improved and qualitatively richer simulation. Our augmented simulator successfully illustrates asymmetric turning behaviors, which are observed on hardware but are not captured by the vanilla version or the simulator with supervised residual dynamics learning. We also demonstrate that we can improve the sim-to-real transfer performance of the policies for two tasks, walking and turning, on the hardware of the BALLU robot. 

% % Motivation
% \sehoon{I would start with hard vs. soft robots, then move on to learning}
% Deep reinforcement learning (deep RL) has proven to be an effective tool in solving a wide array of challenging problems \cite{arulkumaran2017deep, li2017deep} that include learning controllers for robotic manipulation, locomotion, mastering classical board games like chess and Go, video games from the Atari 2600 platform as well as much more complex strategy-based games like Dota 2 \cite{berner2019dota}, and even navigating stratospheric balloons \cite{bellemare2020autonomous}. However, when it comes to robotics, and legged locomotion in particular, transferring the learned controllers to operate on hardware is one of the most challenging aspects of the pipeline.

% Insight
% \nitish{Discuss supervised learning vs RL, one-many mapping}

% \nitish{Aerodynamics modeling is of particular interest to us. Drones are a rich repository for methods that are used to tackle this issue. Various approaches have been developed that include... 
% }
% Algorithm

% Results

\section{Related Work}

% \nitish{It doesn't make sense if we have a separate section just for DRL for legged locomotion since there are a number of manipulation papers we are citing. My suggestion is to have a paragraph describing the sim-to-real landscape and then write about DR methods, hardware-only methods, sim + hardware combined methods, and system ID. Also, there are a very large number of manipulation papers that we can club together under DR. However, does it make sense to cite them?}

% Deep RL for locomotion
\subsection{Deep Reinforcement Learning}
Deep RL~\cite{pmlr-v48-mniha16,schulman2017proximal,haarnoja2018learning} has allowed researchers to make great strides in various fields of robotics, including navigation~\cite{bellemare2020autonomous,loquercio2021learning}, locomotion~\cite{miki2022learning,siekmann2021blind}, and manipulation \cite{peng2018sim,zeng2020tossingbot}. However, successfully deploying these controllers on hardware is still an active area of research~\cite{zhao2020sim, salvato2021crossing}, which is not straightforward due to the discrepancy between the simulation and the real world~\cite{jakobi1995noise}. One of the most common approaches is domain randomization ~\cite{tobin2017domain,peng2018sim,siekmann2020learning,siekmann2021sim,siekmann2021blind,li2021reinforcement}, which exposes an agent to a variety of dynamics during training. Additionally, employing extensions such as privileged learning~\cite{loquercio2021learning, miki2022learning} and adopting structured state~\cite{loquercio2021learning} and action space~\cite{kaufmann2022benchmark} representations has also enabled successful deployment of learned policies on hardware. On the other hand, researchers have developed frameworks to learn policies directly from real-world experience, which have been proven effective for both manipulators~\cite{zeng2020tossingbot} and legged robots~\cite{haarnoja2018learning,ha2020learning}. This paper discusses a sim-to-real transfer for the BALLU robot with highly sensitive dynamics inspired by these previous approaches. Drawing inspiration from these previous approaches, this paper discusses a sim-to-real transfer technique for the BALLU robot, which exhibits highly sensitive dynamics. Specifically, we train policies in simulation and enhance the sim-to-real transferability using real-world data.

% Deep RL has allowed researchers to make great strides in the fields of robotics research, having been employed to learn controllers for locomotion tasks on several classes of legged robots in simulation. However, successfully deploying these controllers on hardware is still an active area of research \cite{zhao2020sim, salvato2021crossing}. On one hand, we have numerous methods that take advantage of the speed and safety offered by simulation to train policies before transferring them to hardware. However, these set of approaches need to tackle challenges that arise from the discrepancy between the simulation and real world which is termed the \textit{reality gap} \cite{jakobi1995noise}. On the other hand, there are several approaches that ditch simulation altogether in favor of training policies directly on hardware, circumventing the reality gap while sacrificing speed and safety.

% One of the most common approaches belonging to the former class of methods is domain randomization \cite{tobin2017domain}.
% % in which the simulator parameters are randomized to expose the agent to a wide range of scenarios during the training process
%  Randomizing the dynamics \cite{peng2018sim, siekmann2020learning, siekmann2021sim, siekmann2021blind, li2021reinforcement} is one of the simplest strategies that has been employed for sim-to-real transfer. \nitish{Does it make sense to cite these 2 papers here since the rest is locomotion heavy?} Novel neural network architectures \cite{james2019sim, rusu2017sim} have been proposed for sim-to-real transfer for robot manipulation. Peng \etal \cite{peng2020learning} propose domain adaptation using advantage-weighted regression.
% % We also have methods that try to match observations in simulation and the real world. using the predicted next state in simulation and a learned inverse dynamics model to achieve the equivalent state in real world.
% Smith \etal \cite{smith2022legged} rely on pre-training policies in simulation and fine-tuning in the real world.

% While sim-to-real approaches are usually task agnostic, learning policies for locomotion directly on hardware is harder due to manual resets required after episode termination, safety concerns, and sample inefficiency. Proposed solutions involve algorithmic contributions \cite{haarnoja2018learning, bloesch2022towards}, automating resets and incorporating safety constraints \cite{ha2020learning, yang2022safe}, and model-based methods \cite{yang2020data}. Combining data from both simulation and real robots is another popular approach \cite{chebotar2019closing, kang2019generalization, di2021sim} that has proven successful.
% \nitish{How our method is different.}

\subsection{System Identification}
Our approach is also highly inspired by system identification that aims to identify model parameters from the collected experimental data. This is a well-studied problem that has been addressed by a variety of methods involving maximum likelihood estimation \cite{khosla1985parameter, gautier1988identification}, optimization-based strategies \cite{tan2016simulation, zhu2017fast, chebotar2019closing}, neural networks \cite{yu2017preparing, allevato2020tunenet, zhang2022zero} with iterative learning \cite{allevato2020iterative, du2021auto}, actuator dynamics identification~\cite{hwangbo2019learning, yu2019sim}, adversarial learning \cite{jiang2021simgan}, and learning residual physics \cite{ajay2018augmenting, golemo2018sim, zeng2020tossingbot, bauersfeld2021neurobem, o2022neural}. Combining system identification with other techniques, such as dynamics randomization, latency modeling, and noise injection \cite{tan2018sim, rodriguez2021deepwalk}, has also proven 
to be effective for successful sim-to-real transfer of learned policies. However, in our case, system identification even in combination with domain randomization, proves to be insufficient, necessitating the need for our residual dynamics learning framework, EnvMimic.

% Combining traditional physics models based on first principles and augmenting them with learning residual physics has also proven to be a highly effective strategy. CITE NeuroBEM and TossingBot. While NeuroBEM learn residual physics in a supervised manner, TossingBot relies on a combination of self-supervision and reinforcement learning. In contrast, we propose a two-stage method that affords modularity and reusability.




\subsection{Balloon-based Robots}
Balloon-based or buoyancy-assisted robots~\cite{balloonhumanoid2015,flyingactuator2018} have been investigated because of their intrinsic stability and low costs. Therefore, many researchers have investigated them in various applications, including roof cleaning~\cite{glassballoon2002}, planet exploration~\cite{nayar2019balloon}, disaster investigation~\cite{ballooncablerobot2005, yamada2018gerwalk, takeichi2017development}, social interactions~\cite{balloonfriend2015}, security~\cite{balloonsysid2022}, and many more. However, control of these balloon robots is not straightforward due to their sensitive non-linear dynamics. One common approach is to develop model-based controllers~\cite{fullstatebuav2021,finballoon2021,balloonsysid2022,inflatablearm2017}, often with system identification. This paper discusses the control of Balloon-based legged robots proposed by Chae \etal~\cite{chae2021ballu2} by leveraging deep reinforcement learning and residual physics.
% \nitish{Main points: stability, low cost, access challenging/dangerous terrain, safe to deploy around humans}
% \nitish{Cite Hosik's old paper}
% Ballu research closely follows a history of active balloon based robot researches. 


% Balloon based robot has been an active field of study due to its many useful properties. Its buoyant property has been proposed to replace high powered motor in flotation mobility researches \cite{flyingactuator2018}. Its property allowed balloon robot parts to become powerful supporters in UAV system to reduce energy consumption while providing stability \cite{solarballoonuav2020}. The balloons’ stability property has been researched in 2012 and 2017 as a passive aerodynamic stabilization as well as for locomotive standing control strategy \cite{passivestability2013}, \cite{balloonwalking2017},. Their low financial cost also has been noted by several of these research for usage in various application. In speaking of applications, many research field proposed using balloon based robot to solve variety of different tasks. As early as 2002 paper highlights the benefit of balloon based robot solution for cleaning inner side of glass roofs and atriums \cite{glassballoon2002}. 2005 paper proposes efficient balloon cable robot for information collection from sky and search strategy in major disaster \cite{ballooncablerobot2005}. Balloon based robot system called InfoBalloons by Hokkaido University demonstratively shows its effectiveness in field tests conducted at Yamakoshi village damaged from an earthquake \cite{disasterrobot2009}. In 2013, Balloon based robot research shows its usefulness as marine robot-kit, which demonstrates autonomous airborne locomotion while providing obstacle avoidance \cite{marinerobot2013}. National Sandia lab reveals Balloon based robot’s research in restricted airspace for climate and weather data collection in an unmanned aircraft system \cite{polarrobot2014}. Furthermore, fictional balloon robot such as Disney’s robotic character Baymax has inspired balloon robots in human robot interaction and medical field alike due to it safety property. In 2015 inflatable humanoid robot, King Loui, integrates MPC (model predictive control) to manipulates the full balloon upper body with neumatic actuator \cite{balloonhumanoid2015}. Likewise individual pneuamatic actuator arms can be seen used as medical tools \cite{pnumatichealth2019}. On the other hand, social robotics research develops balloon robots like “Diri” and “Ollie, the robotic blimp” which is designed as balloon based robotic friend for children \cite{balloonfriend2015}. Marketable product like remote control balloon sharks are out in the public as interactive toys for children. Finally, balloon-based robot has served many applications as enhancement in drone research such as in H-AERO and BUAV balloon supported unmanned aerial vehicle \cite{fullstatebuav2021}, \cite{inflatableuav2019}. Such unmanned aerial vehicles can serve in agricultural surveying, mining, and security etc \cite{balloonsysid2022}.

% Many control schemes are in research for non linear dynamic of balloon based robots. The early 2005 system starts with cable suspension system to transport the balloon robot \cite{ballooncablerobot2005}. Then, autonomous motor driven ballon robots are researched with traditional integrated methods of PID control to full state modeling, while control system that avoids motory device like time-state control with pectoral fin are researched in 2022 \cite{fullstatebuav2021} \cite{finballoon2021}. System Identification and MPC are another area of research in balloon based robotic control \cite{balloonsysid2022} \cite{inflatablearm2017}.





% This work aims to improve the sim-to-real transfer of machine-learned policies by identifying multiple aspects of the BALLU robot dynamics, including actuator dynamics, inertial parameters, and simplified aerodynamics. While previous approaches have learned residual physics in a supervised manner~\cite{bauersfeld2021neurobem} or in combination with deep RL~\cite{zeng2020tossingbot}, our method relies purely on deep RL and uses a two-stage approach that affords modularity and reusability.

\section{Sim-to-real of BALLU}
In this section, we will describe our techniques for reducing the `reality gap'~\cite{jakobi1995noise} of the BALLU robot~\cite{chae2021ballu2}. We approach this challenging problem by combining traditional system identification and deep residual dynamics learning. First, we improve the simulation model of cable-driven actuation by identifying non-linear relationships between motor and joint angles. Next, we use the captured real-world trajectories to model the residual dynamics of the BALLU robot, which arise from various sources such as aerodynamics, joint slackness, and inertial parameter mismatch. Our key invention is to use deep RL for building a residual dynamics model instead of the common choice of supervised learning, which offers effective generalization over a small number of trajectories.

% The weight constraints arising from the design of the BALLU robot lead to it being powered by two lightweight Lithium-Polymer batteries that allow continuous operation that is on the order of mere minutes per charge cycle. The limited runtime coupled with the sample-hungry nature of most deep RL algorithms makes it infeasible to learn policies directly on hardware. This relegates us to learning policies in simulation before we can deploy them on the robot. However, capturing the highly non-linear and sensitive dynamics of BALLU and modeling its interactions with the environment in simulation is an extremely challenging task that leads to a huge sim-to-real gap. In this section we will describe our method for reducing this `reality gap' and successfully deploying policies on the BALLU robot. Our key insight is leveraging deep RL to learn a stochastic model to fully capture BALLU's dynamics and combining it with system identification to achieve a simulation environment that more closely mirrors reality.

\subsection{Background: BALLU robot}
\begin{figure}
    \centering
    \begin{subfigure}{0.49\linewidth}
    \includegraphics[width=\linewidth]{figures/IMG_1588.jpg}
    \label{fig:ballu_full_side}
    \caption{Side view}
    \end{subfigure}
    \hfill
    \begin{subfigure}{0.49\linewidth}
    \includegraphics[width=\linewidth]{figures/IMG_1582_annotated.jpg}
    \label{fig:ballu_knee}
    \caption{Leg details}
    \end{subfigure}
    \caption{Illustration of our research platform, BALLU (Buoyancy Assisted Lightweight Legged Unit) with two passive hip joints and two active knee joints.}
    \label{fig:ballu}
\end{figure}
BALLU (Buoyancy Assisted Lightweight Legged Unit) is a novel buoyancy-assisted bipedal robot with six helium balloons, which provide enough buoyancy to counteract the gravitational force. BALLU's base is connected to helium balloons and houses a Raspberry Pi Zero W board for computing. The robot has two passive hip joints and two active knee joints, which are actuated by two Dymond D47 servo motors at the feet via cables. The overview of the robot is illustrated in \figref{ballu}. For more details, please refer to the original paper by Chae \etal colleague~\cite{chae2021ballu2}.

Due to its unique dynamics, model-free reinforcement learning can be a promising approach for developing effective controllers for BALLU without having to rely on prior knowledge or domain expertise. However, we need to mitigate the large sim-to-real gap first, which is induced by significant drag force effects and low-fidelity hardware.

% \subsection{Background: Deep Reinforcement Learning}
% Given the success of deep RL in solving challenging problems across a wide range of domains~\cite{arulkumaran2017deep, li2017deep}, we employ this as our method of choice 
\subsection{System Identification} \label{sec:system_id}
One main source of the sim-to-real gap is its cable-driven actuation mechanism. In the simulation, servo motor commands and knee joint angles maintain an ideal relationship. In reality, they are affected by friction, torque saturation, and unmodeled cable dynamics, which make the actuator dynamics noisy and nonlinear. 
% \nitish{Additionally, there is asymmetry between the left and right motors}. 
Therefore, we first perform system identification to better capture this nonlinear relationship from real-world data using optimization.

Our free variables $\vc{p}$ include knee spring parameters, motor gains, default motor angles, and default knee joint angles in simulation, which are sufficient to model various nonlinear relationships. As a result, we have eight free variables subject to optimization. 
% \nitish{Eight is total - we optimized separately per leg.}

Our objective function is to minimize the discrepancy of all four joint angles (left and right, motor arm and knee) between simulation and hardware. We sample $20$ actuation commands that constitute $\mathcal{A}$ that are uniformly distributed over the range [$0, 1$], which corresponds to motor arm angles in the range [$0^\circ$, $90^\circ$], and measure {knee and motor} joint angles in simulation and on hardware. Then we fit polynomial curves for all the joints and compute the directed Hausdorff distance between the corresponding curves. We use the {L-BFGS-B} algorithm and optimize the parameters until convergence. The entire process is summarized in Algorithm 1.



% % Brief introduction
% Given that our goal is to improve simulation fidelity, we begin by performing system identification of the robot's key components.
% % What factors we chose
% Based on a sensitivity analysis performed in simulation we observe that we need to focus on modeling knee spring parameters, motor gains, and default joint angles for both motors and knees.
% % Algorithm for how we model actuator dynamics in sim
% We begin with our initial model of the BALLU robot and proceed in a manner similar to coordinate gradient descent. Algorithm \ref{sysID} describes this process in greater detail. We found it sufficient to model both the knee and motor behavior using quadratic curves. An important point to note is that the action spaces for both simulation and hardware environments are identical.

% \nitish{Photo of actuator?} \sehoon{seems needed to describe terms.}

\begin{algorithm}
\caption{System Identification of Cable-driven Actuation}
\label{sysID}
\begin{algorithmic}[1]
\State{\textbf{Input:} the initial parameters $\vc{p}_0$}
\State{\textbf{Input:} a set of pre-defined actions $\mathcal{A}$}
% \State{Error $\epsilon \leftarrow \infty$, tolerance $\Delta$,}
\State{Measure joint angles on hardware for all actions $\mathcal{A}$}
\State{Fit polynomial curves $\overline{C}_1$, $\overline{C}_2$, $\overline{C}_3$, and $\overline{C}_4$}
\State{$\vc{p} \leftarrow \vc{p}_0$}
\While{not converged}
    \State{Update the simulation with $\vc{p}$}
    \State{Measure joint angles for all actions $\mathcal{A}$}
    \State{Fit polynomial curves $C_1$, $C_2$, $C_3$, and $C_4$}
    \State{$\epsilon \leftarrow$ directed Hausdorff distance between $C_i$ and $\overline{C}_i$}
    \State{Optimize $\vc{p}$ using {L-BFGS-B}}
\EndWhile
% \Repeat
% \For{$p \in P$}
% \For{$a \in A$}
% \State{Apply $a$ in both simulation and on hardware,}
% \State{Store both simulation and hardware actuator arm angles for each leg,}
% \State{Store both simulation and hardware knee angle for each leg,}
% \EndFor
% \State{Fit a polynomial curve for both simulation and hardware actuator arm angle data for each leg,}
% \State{Fit a polynomial curve for both simulation and hardware knee angle data for each leg,}
% \State{$\epsilon \leftarrow d$, where $d$ is the maximum of Hausdorff distances between simulation and hardware curves,}
% \State{$p \leftarrow p + \delta$, where $\delta$ is a perturbation,}
% \EndFor
% \Until{$\epsilon \leq \Delta$}
% \Return{$P$}
\end{algorithmic}
\end{algorithm}

\subsection{Residual Dynamics Learning via Reinforcement Learning} \label{sec:envmimic}
% Motivation: talk about supervised learning and one:many mapping, BALLU dynamics
Our next step is to model the residual dynamics of BALLU. Previous methods for learning residual dynamics have employed supervised learning~\cite{golemo2018sim, ajay2018augmenting, bauersfeld2021neurobem, o2022neural} or a combination of self-supervision with deep RL as part of the learning pipeline~\cite{zeng2020tossingbot}. However, off-the-shelf supervised learning, in addition to requiring a large number of real-world trajectories, is plagued by limited exploration. Even a small perturbation to the states the policy has observed during training can cause it to diverge during test time. Moreover, we observe that stochasticity in the real world often leads to multiple different state trajectories arising from the same state even when we apply the same actions. 
% This makes it difficult to effectively train a model using supervised learning. 
The framework of deep RL lends itself naturally to addressing these issues by augmenting the data with simulated trajectories, making it a suitable choice for this problem.
% Moreover, supervised learning is vulnerable to stochasticity, when the same action profiles can lead to multiple state trajectories.

% Previous methods for learning residual dynamics have employed supervised learning~\cite{golemo2018sim, bauersfeld2021neurobem} or a combination of self-supervision with deep RL as part of the learning pipeline~\cite{zeng2020tossingbot}. However, supervised learning requires a large amount of real-world data to learn an accurate model and prevent overfitting. Moreover, we observe in our experiments that the same action trajectory can lead to multiple observation trajectories. In other words, the same action applied at the same state can lead to different resulting states on hardware. This stochasticity makes it difficult to use the paradigm of supervised learning where each training sample only has a single label. Our proposed method therefore relies only on deep RL.

Our key insight is to augment the original simulation framework using a learned residual aerodynamics policy. This policy allows us to capture the complex interaction between BALLU and its environment in greater detail. We will demonstrate in the next section that learning locomotion behaviors with this aerodynamics policy in the loop translate to better transfer of our simulation policies to hardware compared to traditional techniques like domain randomization.

% Algorithm: Segue into motion imitation using deep RL + how we repurpose DeepMimic
Therefore, we design a framework to learn a policy that generates proper \emph{external} perturbation forces that can match the simulation behavior to the ground-truth trajectory collected from the hardware. We draw inspiration from motion imitation methods~\cite{peng2018deepmimic, peng2021amp, peng2022ase, escontrela2022adversarial}, which have demonstrated impressive results for learning dynamics controllers to track reference motions. The fundamental difference is that we learn a policy for external perturbations, while other motion imitation works aim to learn an internal control policy for the robot's actuators. In our experience, this deep RL approach allows us to model robust residual dynamics from a limited set of real-world trajectories compared to supervised learning.

% Data collection
\noindent \textbf{Data Collection.} The first step is to create a set of reference trajectories. We train several locomotion policies in the vanilla simulation and record their action trajectories. Next, we use the recorded actions as open loop control on hardware to collect multiple state trajectories. We use a motion capture system to obtain observation data due to the lack of onboard sensors on BALLU that can estimate its global position and orientation. We note that hand-designed action trajectories may work well for this step. 

% Describe Env DeepMimic formulation
\noindent \textbf{MDP Formulation.}
Once we have the reference dataset, we can cast learning the residual aerodynamics policy as a motion imitation problem using a Markov Decision Process (MDP). The state space consists of the balloon's position, velocity, orientation, the position and velocity of the base, and the position and velocity of the feet, at the current and last two time steps. The action space is three-dimensional and consists of x, y, and z forces that are applied to the center of mass of the balloon. The forces are in the range of $[-1, 1]$ N. The reward function is a combination of position and orientation terms and is defined as follows:
\begin{align*}
    r_t = w^{pos} r^{pos}_t + w^{orn} r^{orn}_t
    % w^{pos} = 0.7,\ w^{orn} = 0.3
\end{align*}
where the position and orientation terms respectively are computed as follows:
\begin{align*}
    &r^{pos} = \text{exp}\left[-10 \left( || \hat{\vc{p}}_t - \vc{p}_t||^2 \right) \right] \\
    % &r^{orn} = \text{exp} \left[-2 \left( 0.2 (\hat{\gamma}_t - \gamma_t)^2 + 0.4 (\hat{\beta}_t - \beta_t)^2 + 0.4 (\hat{\alpha}_t - \alpha_t)^2 \right) \right]
    &r^{orn} = \text{exp}\left[-2 \left( || \hat{\vc{r}}_t - \vc{r}_t||_W^2 \right) \right],
\end{align*}
where $\hat{\vc{p}}_t$, $\vc{p}_t$, $\hat{\vc{r}}_t$, and $\vc{r}_t$ are the desired position, the actual position, the desired orientation, and the actual orientation of the balloons, respectively. The position reward $r^{pos}_t$ encourages the simulated model's balloon to track the reference balloon position as closely as possible while the orientation reward $r^{orn}_t$ encourages it to track the reference balloon orientation. We use the Euler angle representation for orientation, which demonstrates better performance than the quaternion representation. For all experiments, we set $w^{pos} = 0.7$, $w^{orn} = 0.3$, and $W = diag(0.2, 0.4, 0.4)$.
% Training process for EnvMimic
% How we randomly sample episodes
% RSI

\noindent \textbf{Training.}
We train the residual dynamics policy using Proximal Policy Optimization~\cite{schulman2017proximal}. We use a compact network consisting of two layers with $64$ neurons each. Similar to Peng \etal~\cite{peng2018deepmimic}, we also randomize the initial state for each rollout by sampling a state uniformly at random from the selected reference trajectory. This leads to the policy being exposed to a wider initial state distribution and improves robustness, especially when transferring to hardware.

% Training with EnvMimic in the loop
\subsection{Policy Training with Improved Simulation} \label{sec:policy}
Once we improve the simulation using system identification and residual dynamics learning, we can retrain a deep RL policy for better sim-to-real transfer. Once again, we formulate the problem using a Markov Decision Process framework. The state space consists of the balloon's position, velocity, orientation, the position and velocity of the base, and the position and velocity of the feet, all measured at the current time step.
Our actions are two actuator commands, which will change the joint angles based on the identified nonlinear relationship in the previous section. 

We learn two policies - one for forward walking and one for turning left. For the forward walking task, our reward function is $x_{vel}$, whereas for turning left, it is $y_{vel}$.
% Action Space (Mention bang-bang control)

\section{Experiments and Results}
We design simulation and hardware results to answer the following two research questions. 
\begin{itemize}
\item Can we improve the fidelity of the vanilla simulator using actuator identification and residual dynamics learning?
\item Can we improve the performance on hardware by reducing the sim-to-real gap?
\end{itemize}

\subsection{Experimental Setup}
We conduct all the simulation experiments in PyBullet~\cite{coumans2016pybullet}, an open-source physics-based simulator. We use the stable baselines~\cite{raffin2021stable} implementation of Proximal Policy Optimization~\cite{schulman2017proximal} to learn the residual dynamics (Section~\ref{sec:envmimic}) and the policy in the improved simulation (Section~\ref{sec:policy}). We use the BALLU platform~\cite{chae2021ballu2} for hardware experiments while capturing all the data using a Vicon motion capture system~\cite{vicon}. 
% Talk about what research problem we are trying to solve

% Describe experiments, talk about ablations

% Data efficiency is one of our primary concerns as mentioned earlier. Mention that we use the vanilla policy that we train to act as our baseline. Our reason for using the vanilla policy is twofold. 
% For our experiments, we collected 10 observation trajectories each 20 seconds long, which took less than 10 minutes of wall clock time.

\subsection{Improved Simulation Fidelity}
This section illustrates the process for improving the simulation's fidelity. We first highlight the importance of actuator system identification in Section~\ref{sec:actuator_results} and show the learned residual dynamics using our EnvMimic technique in Section~\ref{sec:envmimic_results}.

\subsubsection{Actuator System Identification} \label{sec:actuator_results}
% Plots for actuators, 

\begin{figure}
    \centering
    \includegraphics[width=\linewidth]{figures/Sys_ID_2x2.png}
    \caption{Identified non-linear, asymmetric relationships of cable-driven mechanisms.}
    \label{fig:actuator_sys_ID}
\end{figure}

We collect the data and identify the system parameters, such as spring parameters, motor gains, and default joint angles, as described in Section~\ref{sec:system_id}. The identified relationships between the motor commands and joint angles are illustrated in Fig.~\ref{fig:actuator_sys_ID}. As shown, the identified relationships exhibit highly nonlinear behaviors compared to the simple idealized curves in simulation, which are essential to model the dynamics of the BALLU robot. 


\begin{figure}
    \centering
    \begin{subfigure}{0.4925\linewidth}
    \includegraphics[width=\linewidth]{figures/Sys_ID_overlay_front.png}
    \label{fig:sys_ID_overlay_front}
    % \caption{Front view}
    \end{subfigure}
    \hfill
    \begin{subfigure}{0.4925\linewidth}
    \includegraphics[width=\linewidth]{figures/Sys_ID_overlay_side.png}
    \label{fig:sys_ID_overlay_side}
    % \caption{Side view}
    \end{subfigure}
    % \begin{subfigure}{0.9\linewidth}
    %     \includegraphics[width=\linewidth]{figures/Action_Sys_ID_results.png}
    % \end{subfigure}
    \caption{Illustration of System Identification Results. We execute the same action trajectory and compare the final state of identified dynamics (blue) to that of the naive simulation (red), which is significantly different.}
    \label{fig:sys_ID_overlay}
\end{figure}



We highlight the importance of system identification by comparing trajectories in simulation. We run the same action sequences of periodic bang-bang control signals with and without system identification and compare the final states in Fig.~\ref{fig:sys_ID_overlay}. The two generated trajectories show a significant difference in terms of the final COM positions ($0.23$~m difference) and the joint angles ($10.15^\circ$ difference, average of all joints). 
% \nitish{I think we should skip the next sentence} In Table~\ref{table:fwd_walk} and Table~\ref{table:left_turn}, we demonstrate that system ID can greatly improve the sim-to-real performance by comparing a policy trained with and without system ID (second and third rows).
% \sehoon{This is Fig. 2. Show two different final states side-by-side, with and without system identification. Sim results are sufficient}

\subsubsection{EnvMimic: Residual Dynamics Learning} \label{sec:envmimic_results}
Next, we examine the results of residual dynamics learning using the proposed EnvMimic technique in Section~\ref{sec:envmimic}. We hypothesize that EnvMimic can learn compelling residual dynamics from a few trajectories, unlike data-hungry supervised learning approaches. We compare the x-y-yaw trajectories in four different environments: (1) a vanilla simulation, (2) a simulation with the residual dynamics learned with supervised learning, (3) a simulation with the residual dynamics learned with EnvMimic (ours), and (4) the ground-truth trajectory on hardware. For supervised learning, we use a neural work with two hidden layers of size [64, 64].
For all the trajectories, we use the same action sequences that are generated by the initial policy in a vanilla simulation. Also, please note that the testing ground-truth trajectory is unseen during training. 
% \nitish{We should mention that the EnvMimic policy is stochastic, whereas SL policy is deterministic. SL policy is also quite small [64, 64] to prevent overfitting. Going down to [32, 32] degrades performance for both. We have curves for both SL and EnvMimic for in-distribution - the overfitting is more apparent for SL. We see different trajectories in case of EnvMimic from the same starting state, but the difference is not too large so we decided to omit that.}
% \sehoon{This will be our Fig. 3.}

\begin{figure}
    \centering
    \includegraphics[width=\linewidth]{figures/EnvMimic_comparison_final.png}
    \caption{Comparison of simulation trajectories to ground truth hardware data for forward walking. Our method, EnvMimic, shows the best tracking performance, particularly in terms of the yaw angle. Note that the ground truth trajectory shown is out-of-distribution.}
    \label{fig:envmimic}
\end{figure}

% \begin{figure}
%     \centering
%     \includegraphics[width=\linewidth]{figures/EnvMimic_motion_strip_1.png}
%     \caption{Comparison of trajectories generated by our simulation augmented with System ID and EnvMimic (\textbf{top}) and a vanilla simulation (\textbf{bottom}).}
%     \label{fig:envmimic_motion_strip}
% \end{figure}
The trajectories are compared in Fig.~\ref{fig:envmimic}.
Clearly, our EnvMimic offers much-improved tracking performance compared to the vanilla simulation that fails to capture the noticeable yaw orientation changes due to the stochasticity of the hardware experiments. In our experience, the trajectory generated with supervised residual dynamics learning tends to turn less and remains on the positive Y side. We hypothesize that the robustness of our RL-based residual dynamics might be obtained due to the mix-use of real-world and simulation trajectories. On the other hand, the supervised learning baseline is trained on the pre-collected hardware trajectories without any data augmentation. We believe the comparison of SL or RL-based approaches on a wider range of scenarios will be an interesting future research direction. Please refer to the supplemental video for qualitative comparisons, highlighting the obvious benefit of our EnvMimic-based residual dynamics learning. 
% \sehoon{Compare their motions in Fig. 4. Maybe optional.}


\subsection{Improved Sim-to-real Transfer}
\begin{table}
    \centering
    \begin{tabularx}{\linewidth}{ 
    | >{\centering\arraybackslash}X
    | >{\centering\arraybackslash}X
    | >{\centering\arraybackslash}X
    | >{\centering\arraybackslash}X|}
    \hline
        \textbf{Experiment} &$(CoM^{sim}_t - CoM^{hw}_t)/T$(m) & \textbf{x-distance traveled (m)} & Total distance traveled (m) \\
        % \hline
        % Vanilla & 0 & 0 \\ 
        \hline
        Vanilla + sys ID & $2.01$ & $0.27$ & $1.24$ \\
        \hline
        Vanilla + DR & $1.24$ & $0.32$ & $0.60$ \\
        \hline
        Vanilla + sys ID + DR & $1.07$ & $0.56$ & $0.94$ \\
        \hline
        EnvMimic (Ours) & $\mathbf{0.90}$ & $\mathbf{1.12}$ & $\mathbf{1.28}$ \\
    \hline
    \end{tabularx}
    \caption{Sim-to-real Comparison for Forward Walking.}
    \label{table:fwd_walk}
\end{table}

\begin{table}
    \centering
    \begin{tabularx}{\linewidth}{ 
    | >{\centering\arraybackslash}X
    | >{\centering\arraybackslash}X
    | >{\centering\arraybackslash}X
    | >{\centering\arraybackslash}X|}
    \hline
        \textbf{Experiment} & $\alpha^{sim}_T - \alpha^{hw}_T$ & \textbf{y-distance traveled (m)} & $\Delta \alpha $ (hardware)\\
        % \hline
        % Vanilla & 0 & 0 \\ 
        \hline
        Vanilla + sys ID & $16.41^{\circ}$ & $0.20$ & $16.72^{\circ}$\\
        \hline
        Vanilla + DR & $38.18^{\circ}$ & $-0.04$ & $-4.28^{\circ}$\\
        \hline
        Vanilla + sys ID + DR & $-8.50^{\circ}$ & $0.11$ & $\mathbf{39.77^{\circ}}$ \\
        \hline
        EnvMimic (Ours) & $\mathbf{3.50^{\circ}}$ & $\mathbf{0.29}$ & $36.42^{\circ}$\\
    \hline
    \end{tabularx}
    \caption{Sim-to-real Comparison for Turning Left.}
    \label{table:left_turn}
\end{table}

\begin{figure*}
    \centering
    \setlength{\tabcolsep}{1pt}
    \renewcommand{\arraystretch}{0.7}
    \begin{tabular}{c c c c c}
    \includegraphics[width=0.195\textwidth]{figures/forward/forward_base_000001.png} &
    \includegraphics[width=0.195\textwidth]{figures/forward/forward_base_000005.png} &
    \includegraphics[width=0.195\textwidth]{figures/forward/forward_base_000011.png} &
    \includegraphics[width=0.195\textwidth]{figures/forward/forward_base_000016.png} &
    \includegraphics[width=0.195\textwidth]{figures/forward/forward_base_000024.png} \\
    \includegraphics[width=0.195\textwidth]{figures/forward/forward_ours_000001.png} &
    \includegraphics[width=0.195\textwidth]{figures/forward/forward_ours_000005.png} &
    \includegraphics[width=0.195\textwidth]{figures/forward/forward_ours_000011.png} &
    \includegraphics[width=0.195\textwidth]{figures/forward/forward_ours_000016.png} &
    \includegraphics[width=0.195\textwidth]{figures/forward/forward_ours_000024.png} \\
    $t=0$s & $t=5$s & $t=11$s & $t=16$s & $t=24$s\\
    \end{tabular}
    
    \caption{Comparison of the learned forward walking policies without (\textbf{top}) and with (\textbf{bottom}, ours) the proposed residual dynamics learning. Both policies are also trained with domain randomization and actuator system identification. Please note that the baseline (\textbf{top}) shows a significant turning, while ours (\textbf{bottom}) can walk double the distance of the baseline.
    }
    \label{fig:hardware_forward}	
\end{figure*}

\begin{figure*}
    \centering
    \setlength{\tabcolsep}{1pt}
    \renewcommand{\arraystretch}{0.7}
    \begin{tabular}{c c c c c}
    \includegraphics[width=0.195\textwidth]{figures/left/turn_left_base_000001.png} &
    \includegraphics[width=0.195\textwidth]{figures/left/turn_left_base_000004.png} &
    \includegraphics[width=0.195\textwidth]{figures/left/turn_left_base_000010.png} &
    \includegraphics[width=0.195\textwidth]{figures/left/turn_left_base_000018.png} &
    \includegraphics[width=0.195\textwidth]{figures/left/turn_left_base_000022.png} \\
    \includegraphics[width=0.195\textwidth]{figures/left/turn_left_envmimic_000001.png} &
    \includegraphics[width=0.195\textwidth]{figures/left/turn_left_envmimic_000004.png} &
    \includegraphics[width=0.195\textwidth]{figures/left/turn_left_envmimic_000010.png} &
    \includegraphics[width=0.195\textwidth]{figures/left/turn_left_envmimic_000018.png} &
    \includegraphics[width=0.195\textwidth]{figures/left/turn_left_envmimic_000022.png} \\
    $t=0$s & $t=4$s & $t=10$s & $t=18$s & $t=22$s\\
    \end{tabular}
    
    \caption{Comparison of the learned turning left policies without (\textbf{top}) and with (\textbf{bottom}, ours) the proposed residual dynamics learning. Both policies are also trained with domain randomization and actuator system identification. Both policies are able to achieve a similar change in yaw angle ($\Delta \alpha$) over the entire episode, but the baseline (\textbf{top}) takes a single step and only turns in place, rotating over the battery cover. We also observe that our method (\textbf{bottom}) covers more than twice the distance in the desired y-direction.
    }
    \label{fig:hardware_left}	
\end{figure*}

To complete the story, we investigate whether we can improve the sim-to-real transfer of policies using augmented simulation. We first train (1) a policy in the improved simulation with the system identification, learned residual dynamics, and domain randomization~\cite{tobin2017domain} (ours) and compare the performance with the selected baseline policies learned in the following settings: (2) a simulation only with system identification (Vanilla + sys ID), (3) a simulation only with domain randomization (Vanilla + DR), and (4) a simulation with both system identification and domain randomization (Vanilla + sys ID + DR). For domain randomization, we randomly sample parameters for friction and initial states. We decided not to randomize masses or the buoyancy coefficient due to the sensitivity of the policy to these parameters. We evaluate these simulation-learned policies on the hardware and measure their performance. 

For the forward walking task, the learned policy with our augmented simulation is the only one which can walk forward while the other baselines turn left significantly. Therefore, its traveled distance in the forward (x) direction is $1.12$~m, which is significantly larger than $0.27$~m, $0.32$~m, and $0.56$~m of the others. For the turning task, our approach trains the effective policy that travels the most distance in the left (y) direction, $0.29$~m, which is our objective function. On the other hand, the other policies cover a shorter distance: $0.20$~m, $-0.04$~m, and $0.11$m. We note that the change in yaw angle ($\Delta \alpha$) is slightly larger in the case of the baseline with system identification and domain randomization compared to our method. This is an artifact that arises from the policy taking a single step and rotating on the battery cover. This is evident from the y-distance traveled and can be observed clearly in the qualitative results in Figure \ref{fig:hardware_left} and the supplemental video. 

For both tasks, our augmented simulation also exhibits the least sim-to-real errors, which are defined as the average center of mass (CoM) error and final yaw angle ($\alpha$) error between simulation and hardware. The performance is summarized in Table~\ref{table:fwd_walk} and Table~\ref{table:left_turn}. Please refer to the supplemental video and Fig.~\ref{fig:hardware_forward} for qualitative comparison.  
% For the forward walking task, the learned policy with our augmented simulation is the only one which can walk forward while the other baselines turn left significantly (Fig. 5 \nitish{6, not 5?}, top). Therefore, its traveled distance in the forward (x) direction is $1.12$~m, which is significantly larger than $0.27$~m, $0.32$~m, and $0.56$~m of the others. The performance is summarized in Table 1. For the turning task, our approach trains the effective policy that turns ... while the others ... \sehoon{add some description} (Fig. 5, bottom \nitish{6, not 5?}). Please refer to the supplemental video for qualitative comparison.  




\section{Conclusion and Discussion}
% Summary
We present a learning-based method for the sim-to-real transfer of locomotion policies for the Buoyancy
Assisted Lightweight Legged Unit (BALLU) robot, which has unique and sensitive dynamics. To mitigate a large sim-to-real gap, we first identify nonlinear relationships between motor commands and joint angles. Then we develop a novel residual dynamics learning framework, \emph{EnvMimic}, which trains an external perturbation policy via deep reinforcement learning. Once we improve the simulation accuracy with the identified actuator parameters and the learned residual physics, we retrain a policy for better sim-to-real transfer. We demonstrate that using our methodology, we can train walking and turning policies that are successful on the hardware of the BALLU robot.

There exist several interesting future research directions we plan to investigate in the near future. In this work, we develop our residual dynamics model for each individual task, such as walking or turning, which limits generalization over other tasks. Therefore, it will be interesting if we collect a large dataset and train a general residual dynamics model using the proposed method. It will be possible to take some inspiration from the state-of-the-art motion imitation frameworks, which can track a large number of trajectories using a single policy~\cite{peng2021amp}. In addition, we also want to investigate various policy formulations. This paper assumes simple external forces to the center of the balloons to model aerodynamics, and it was sufficient for the locomotion tasks we tested on. However, we may need multiple forces or torques to model some sophisticated phenomena. Furthermore, the dynamics of the BALLU robot are also sensitive to time owing to the deflation of balloons. In the future, we want to introduce the concept of lifelong learning to model those gradual temporal changes.

Finally, we plan to evaluate the proposed residual dynamics learning approach, \emph{EnvMimic}, on different tasks and robotic platforms. While showing promising results, many experiments are limited to the selected walking and turning tasks and the specific hardware of BALLU. However, we believe the algorithm itself is agnostic to the problem formulation, and it has great potential to improve the sim-to-real transferability in general scenarios, even including drones and rigid robots. We intend to explore this topic further in future research.

% \addtolength{\textheight}{-12cm}   % This command serves to balance the column lengths
                                  % on the last page of the document manually. It shortens
                                  % the textheight of the last page by a suitable amount.
                                  % This command does not take effect until the next page
                                  % so it should come on the page before the last. Make
                                  % sure that you do not shorten the textheight too much.

%%%%%%%%%%%%%%%%%%%%%%%%%%%%%%%%%%%%%%%%%%%%%%%%%%%%%%%%%%%%%%%%%%%%%%%%%%%%%%%%



%%%%%%%%%%%%%%%%%%%%%%%%%%%%%%%%%%%%%%%%%%%%%%%%%%%%%%%%%%%%%%%%%%%%%%%%%%%%%%%%



%%%%%%%%%%%%%%%%%%%%%%%%%%%%%%%%%%%%%%%%%%%%%%%%%%%%%%%%%%%%%%%%%%%%%%%%%%%%%%%%

\section*{Acknowledgement}

This work is supported by the National Science Foundation under Award \#2024768.

%%%%%%%%%%%%%%%%%%%%%%%%%%%%%%%%%%%%%%%%%%%%%%%%%%%%%%%%%%%%%%%%%%%%%%%%%%%%%%%%


\bibliographystyle{ieeetr}
\bibliography{refs}


\end{document}
