\section{Related Works}
% anxiety and stress
% ----------------------------------------
Stress and anxiety-related disorders are common mental health challenges.
Such disorders can have significant negative impacts on people's lives, including higher chances of depression and suicide as well as associated comorbidities with physical health issues \cite{mayo2018_anxiety}.
Unfortunately, in many cases, these issues  remain inadequately treated due to challenges ranging from lack of viable access to therapeutic services to associated stigmas with utilization \cite{any_anxiety,apa_stress2020}.
However, even when an individual decides to seek psychotherapeutic help to alleviate these problems, challenges persist in the diagnosis and effective treatment of their disorder. 
At the inception of care, the steps to diagnose and monitor often include clinical evaluation and comparing personalized symptoms to standardized criteria, for example, the Diagnostic and Statistical Manual of Mental Disorders (DSM-5), which is commonly used for this matter.
Researchers continue to study and improve the practicality and accuracy of guidelines such as DSM-5\cite{batelaan2012mixed,zimmerman2017measuring}, but there are challenges in converting aggregated and generalized diagnostic criteria, down to episodic-level incidents of stress and anxiety.

% the disconnect between emotional and physiological stress
% ----------------------------------------
The most obvious approach to doing so leverages biometric data, extracted from wearable sensors embedded in smart devices, that measure a physiological stress response. 
However, while such data is incredibly valuable, and notably, sensing devices have become increasingly sophisticated at monitoring physiological stress, the resulting analyses are incomplete at best. 
This owes to the fact that from the standpoint of straightforward correlative analytics, it is known that there is not a direct monotonic correlation between the emotional perception of stress an individual may feel and the manifestation of the underlying physiological stress response.
A meta-analysis in the social stress domains, for instance, has recently shown that merely $25\%$ of studies in the domain demonstrated a significant correlation between physiological stress and perceived emotional stress \cite{31}.

Given that self-reports of perceived stress often do not contain information on the physiological stress response, understanding the complex relationship between the two becomes a crucial matter\cite{31}.
It is also plausible to assume that such complexity also arises from various other confounding factors (e.g., demographics, occupation, and other mental health disorders such as attention-deficit hyperactivity disorder (ADHD) can influence how prone someone is to stress).

This discrepancy has meaningful impacts on the utility of passive detection of stress based largely on physiological indicators. 
While sensors may be returning accurate readings on physiological stress, if they do not align with the user's own perceptions of stress, notably if they fail to properly account for moments when a user feels acute emotional distress, then it will demotivate further engagement with a mental health platform. 

% health monitoring in general
% ----------------------------------------
This hindrance comes in spite of considerable progress that has been made in recent decades regarding the capabilities and efficacy of personal digital devices, including smartwatches, smartphones, and wearable devices.
This fact has made such devices attract a lot of research and commercial attention, employing them for various monitoring objectives \cite{15,16}.

These monitoring approaches focus primarily on fitness and health-related aspects, resulting in a large body of research and countless commercialized applications. Examples include tracking athletes' training, detecting falls for the elderly, tracking post-surgery therapeutic and rehabilitation exercises, and posture correction \cite{darabi2017heart,vilarinho2015combined,gwak2019extra,wile2014smart}.
% mental health monitoring
% ----------------------------------------
While the central focus of health monitoring applications has undoubtedly been on physical health, a wide range of research works has focused on understanding the relationship between observations obtained leveraging digital devices and some aspects of individuals' mental health status.
% importance of "passive" sensing and why it is more beneficial
It is noteworthy that a primary goal in designing smart and automated approaches for mental health monitoring has to do with proposing meaningful passive-sensing tools so that informative observations regarding health status can be made by eliminating or diminishing the need to interfere with users' daily activities or request repeated active interactions.

As a remote mental health monitoring task, social anxiety was studied in the previous literature, and it was shown that analyzing trajectories obtained via smartphone location services can paint a comprehensive picture concerning individuals' proneness to it.
To do so, the movements and the nature of locations visited (which were obtained by cross-referencing location data with a map API) were taken into consideration, and the hypothesis of whether or not such corpus is informative for recognizing the presence of social anxiety was tested \cite{boukhechba2018predicting,chow2017using,huang2016assessing,gong2019understanding}.
Smartphones have also been helpful in developing an understanding of anxiety \cite{levine2020anxiety}.

Another choice of hardware for gathering data pertinent to health data is application-specific wearable sensors. For instance, wearable electrocardiogram (ECG) sensors were used to recognize perceived anxiety via pattern recognition \cite{king2018predicting}.

% imporance of smartwatch (a section before moving on to the related work on smartwatch, which is mainly mhdeep and our earlier work)
Smartwatches have a unique position amongst the wide range of various commonly used digital devices.
They are in close contact with the skin and, given their attachment user's wrist, which is a distal point of a major appendage, make it possible to obtain most measurements (e.g., activity) at higher accuracy, as well as enabling additional measurements such as heart-rate or pulse oximeter.
In case of the need for brief questions, interactions, or Ecological Momentary Assessments (EMAs), smartwatches can also be used to issue messages and acquire responses and entries by the user \cite{15,16,17}.
Additionally, smartwatches are prevalent, and relying on them as the hardware for health applications provides a better alternative in most cases to application-specific wearable devices in terms of cost, comfort, and user-friendliness.
% mHealth anxiety works
% ----------------------------------------
Data-driven analyses leveraging smartwatches' sensory readings have been successful at the problem of patient classification for bipolar disorder, schizoaffective disorder, and depression \cite{hassantabar2022mhdeep}.
It has also been shown that physiological readings made by basic smartwatch sensors enable efficient modeling of perceived stress response \cite{fazeli2022passive}.

% semisupervised learning, and anxiety works such as mhdeep
% ----------------------------------------
In the health analytics domain, data and human annotations are often limited.
Therefore, dealing with overfitting and memorization is a crucial matter.
Additionally, it is beneficial to go beyond the limited number of human annotations available in training efficient inference pipelines.
Less reliance on annotations by focusing on unsupervised and self-supervised approaches has received a lot of research attention in recent years \cite{chen2020simple,caron2020unsupervised,chen2021exploring,grill2020bootstrap}.
The core idea in most works in this area is that comparing and contrasting  the latent representations of examples that are expected to share certain similarities (e.g., augmented versions of the same image) can benefit the trained weights and help with regularizing the learned  decision boundaries \cite{zhang2017mixup}.

% difference
% ----------------------------------------
In short, this work is primarily focused on addressing the limitations in the previous literature on remote mental health monitoring.
The previous works do not go beyond leveraging scarcely available annotations in training network parameters and mainly rely on data augmentation to improve their performance.
They do not focus on encapsulation in embedding different modalities, which can be an obstacle in employing optimal encoders for each modality and can hinder transfer learning.
Additionally, they do not focus on the interpretability of the inference pipeline, which is crucial in health-related applications.

To address these challenges, this work proposes a framework for leveraging smartwatch-based sensor-driven data to recognize {\it perceived stress}, enabling a novel approach to remote mental health monitoring.
Our proposed inference pipeline is modular and hierarchical and is composed of modality-specific embedding branches.
The final embedding is computed via a task-specific attention-pooling mechanism, which also provides an interpretation of the estimated contribution of each modality's information to the last embedding.
During training, we leverage an inter-modality contrastive objective so as to encourage consistency among the predictions and tune all encoder branches.
Figure \ref{fig:mmssl} depicts the overview of our proposed framework.
The details of our approach are discussed in the next section.

