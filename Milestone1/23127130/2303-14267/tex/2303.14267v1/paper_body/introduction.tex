\section{Introduction}
% - the importance of mental health in general
% ---------------------------------------------
The rising epidemic of mental health disorders,  worsened by the recent COVID-19 pandemic, speaks to the growing need for effective and timely management of mental health disorders. 

The pandemic led to an increase in the need for mental health services, while concurrently, given the circumstances surrounding the outbreak, limited access to traditional modalities of care. This necessitated the explosion in the usage of alternative mechanisms to deliver mental health services, mainly through remote formats  \cite{tsamakis2020covid}.

For that reason, in spite of increased barriers to access, unprecedented levels of funding have  gone into programs to address mental health issues among the general public. 
%% government spending
For instance, in 2020 alone, the United States government spent around $280$ billion dollars on mental health services \cite{thewhitehouse_2022}.

% how therapists deal with it and treatments etc.
% ----------------------------------------
Therefore, even as the pandemic, and associated restrictions on in-person activities, have subsided, the gaps it revealed in traditional in-person-based therapeutic services persist, and demand for remote solutions remains high. 

% importance of remote monitoring of it
% ----------------------------------------
While much of the focus of remote mental health services has been around the use of video conferencing, instant messaging, and other modes of communication to facilitate interactions between therapists and patients, the use of mHealth technology and passive monitoring has the potential to be equally impactful at addressing barriers to care and gaps in monitoring.
Furthermore, by leveraging personal digital devices equipped with numerous sensors that are capable of monitoring many aspects of an individual's physiology and lifestyle (e.g., heart rate, activity level), remote health monitoring provides a novel pathway to not only monitor existing indicators of mental health, but also to improve upon our understanding mental health disorders and their impacts on one's life.


% importance of anxiety and a summary of its effects
% ----------------------------------------
Stress, commonly defined as "physical, mental, or emotional strain or tension," is a widespread problem with numerous potential causes.
According to the American Institute of Stress, $73\%$ of people suffer from acute bouts of stress to a degree of magnitude that impacts their mental well-being.
Incidents of Anxiety often manifest similarly to stress, however, it is notable that it is not always immediately tied to a specific triggering or inciting event and may take longer to resolve. 

All told, both stress and anxiety problems are very common, to the extent that most adults have been affected by at least one anxiety-related disorder \cite{any_anxiety,apa_stress2020}.
Anxiety-related disorders can have a significant negative impact on the quality of life, leading to other mental health disorders such as depression, as well as causing physical health problems \cite{mayo2018_anxiety}.

In contemplating improved means to address mental health challenges generally, and anxiety-related disorders specifically, it is notable that a critical part of modern healthcare involves accurate and efficient tracking of individuals' well-being through time.
Examples include tracking athletes and their training trajectories, and patients' rehabilitation exercises \cite{seshadri2019wearabled,gwak2019extra,valente2022multi,gwak2022internet,zhao2020design}.
Compared to physiological health, the mental health domain is less investigated in the context of remote health monitoring. 
This is largely due to a confluence of reasons.
For one, the statistical sufficiency of observations obtained via data-driven approaches is not often intuitively clear (e.g., can one draw a conclusion regarding depression from the number of phone calls?).
Another reason is that the data required for enabling the use of artificial intelligence (AI) is often not readily available or exclusive due to privacy and regulatory concerns. 

The works in this domain, therefore, have mostly focused on longer-term patient phenotyping (e.g., classifying patients into bipolar disorder vs healthy) \cite{hassantabar2022mhdeep}.

While these high-level labels are useful, they can be limited in their utility, as stress often manifests as an emotional and physiological response of an individual to a triggering event, and can occur to anyone regardless of a formal diagnosis. 
For example, arguing with someone and being anxious about a deadline are instances of interpersonal and work-related stress that are liable to occur to anyone regardless of the existence of a pre-existing mental health disorder.
Furthermore, the highly localized and temporary nature of these shorter-term episodes, which given the scarcity of data, makes making the most out of the available observations critical.

% what we propose and our contributions
% ----------------------------------------
Inspired by the advancements in the domain of self-supervised learning, we propose a multi-modal self-supervised learning framework to learn the context of stress response from continuous physiological readings.
This proposed setup addresses the following challenges and concerns  regarding data-driven monitoring of stress and anxiety:
\begin{itemize}
    \item The proposed method is inherently modular with regards to the different modalities of data, and therefore proper data-layer transforms allows leveraging various devices (e.g., smartwatches and wearable sensors different from ours) to learn efficient representations for  health monitoring.

    \item The self-supervised component allows training the network with a higher level of granularity and makes training more efficient.
    This is especially needed as the amount of labeled data available is often limited and costly to acquire, in contrast to sensor data that is generally trivially available in large quantities.

    \item The use of the attention mechanism enables a diagnostic view of the system, allowing the researchers to look into the empirical connection between various modes of data for specific monitoring tasks, counteracting the masking effect of many deep-learning frameworks on interpretability.

    \item In developing this framework, we conducted experiments on real-world data collected on perceived stress and have shown that this approach improves the performance compared to prior work leveraging early-fused embeddings of the same benchmark dataset.
\end{itemize}



