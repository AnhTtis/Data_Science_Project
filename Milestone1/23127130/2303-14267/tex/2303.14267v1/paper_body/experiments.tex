\section{Experiments}
% a discussion on the experiments
\subsection{Data}
\begin{figure}[h!]
    \centering
    \includegraphics[width=0.48\textwidth]{figs/perceived_anxiety_unified.pdf}
    \caption{A sample portion of a continuous supervision signal generated based on user inputs, from which episode stress labels can be sampled}
    \label{fig:continuous_supervision_signals}
\end{figure}

\begin{figure}[h!]
    \centering
    \includegraphics[width=0.48\textwidth]{figs/attn_interp.pdf}
    \caption{The average contribution of the four modalities to the final episode embeddings.}
    \label{fig:attn_interp}
\end{figure}

The cohort in this study consists of $14$ college students who are ex or active-duty members of the United States military.
The status of these individuals, both as military members as well as students, renders them an interesting cohort for our stress study, given that the individuals from both groups are known to be relatively more prone to experiencing stress.

\begin{table}[h]
\centering
\caption{Number of participants per category based on their duty status and service branch}
{\small
\label{tab:cohortcategories}
\resizebox{0.5\columnwidth}{!}{%
\begin{tabular}{l|l|}
\cline{2-2}
                                     & \textbf{Duty Status}    \\ \hline
\multicolumn{1}{|l|}{National Guard} & 2                          \\ \hline
\multicolumn{1}{|l|}{Active Duty}    & 3                          \\ \hline
\multicolumn{1}{|l|}{Veteran}        & 9                          \\ \hline
                                     & \textbf{Service Branch} \\ \hline
\multicolumn{1}{|l|}{Airforce}       & 2                          \\ \hline
\multicolumn{1}{|l|}{Marine}         & 3                          \\ \hline
\multicolumn{1}{|l|}{Navy}           & 4                          \\ \hline
\multicolumn{1}{|l|}{Army}           & 5                          \\ \hline
\end{tabular}%
}}
\end{table}

The smartwatch in this study was Garmin vivoactive 4S. Nevertheless, it is noteworthy that there is no component in the proposed methodology that limits the solution to this smartwatch.
The feature groups and various modalities in our configuration are shown in Table \ref{tab:modfocus}.
\begin{table}[]
\centering
\caption{Data modalities and the features they are focused on}
\label{tab:modfocus}
\begin{tabular}{|l|l|}
\hline
\textbf{Modality} & \textbf{Focus}     \\ \hline
Daily &
  \begin{tabular}[c]{@{}l@{}}heart-rate readings, \\ number of floors \\ climbed, \\ BMR kilocalories, \\ distance traveled,\\ activity levels, \\ aggregated HRV measures\end{tabular} \\ \hline
Pulse Ox          & SPO2               \\ \hline
Respiration       & Respiration rate   \\ \hline
Stress            & HRV-based readings \\ \hline
\end{tabular}
\end{table}

\subsection{Labeling}
The focus of this study has been on making predictions on {\it perceived} stress, for which the participants agreed to indicate the episodes in which they felt stressed and provide us with the intensity and timespans of these episodes.

For each record input to our system by an individual, we created a softened (via a Gaussian function) time-series per the following steps:
\begin{itemize}
    \item The peak (corresponding to the {\it mean} of this Gaussian function) is set to the given timestamp, or the midpoint of the timespan ($(t_{\text{start}} + t_{\text{end}})/2$).
    \item The standard deviation of $30$ minutes (scaled proportional to the length of time-span, if a time span of over one hour is provided).
    \item The magnitude of the peak point corresponds to the indicated  for the episode: $\{0,1,2,3\}$ for $\{\texttt {None}, \texttt{Low},\texttt{Medium}, \texttt{High}\}$, respectively.
\end{itemize}
The summation of these Gaussian signals comprises the  signal used as the primary supervision objective.
The way the labels are computed is by looking at the end-point of each episode, and its {\it stress} label is marked \texttt{True} if the value of this signal on that point is larger than a threshold of $0.5$, and \texttt{False} otherwise.
% \shayan{should we remove the subsections in the experiments section?}
\subsection{Modeling}
% separate rnn branches etc. description of the model
Our inference model is composed of a specific encoder for each modality.
In our case, each encoder is defined based on an initial mapping and normalization (via fully connected layers) followed by a bi-directional recurrent neural network (RNN) in long short-term memory (LSTM) configuration.
Specifically, the data from each modality was first projected to a $32$-dimensional vector via a multi-layer perception (MLP) with one hidden layer.
The output was then forwarded to the modality-specific bi-LSTM with the hidden-layer neuron count of $64$. The last stage for representing each modality was another fully-connected projection layer, generating a $32$-dimensional vector per modality, which were used as modality representations in our framework.

Note that the overall pipeline does not have any constraint on the local modality encoders as long as they share the final semantic space to which they project that modality's observations.
Given that we were mainly dealing with time-series data, we used RNNs to model each branch.
Nonetheless, modalities from substantially different domains and their encoders (e.g., Transformer-based Language Model for textual data) can also fit into the same system.

\subsection{Results}
Focusing on our real-world perceived stress corpus, we conducted experiments under the main settings of 1) supervised training baseline, 2) pre-training the contrastive objective and fine-tuning via supervised objective, and 3) training the supervised objective and simultaneously optimizing a scaled version of the contrastive term as a regularizing loss.
 
We observed that leveraging more features and following a late-fusion protocol for combining modality representations did lead to an improved generalization performance over the supervised setup proposed in \cite{fazeli2022passive}, which combined the features at the beginning of the pipeline. In the case of our cohort, training with contrastive regularization led to the best generalization on the unseen test data, and the results are shown in Table \ref{tab:modelresults}.
Note that, in general, it is hard to say which self-supervised setup (pre-training versus regularization) is best, as it could depend on other factors, including model complexity, optimization, data availability, and task difficulty.
That being said, our approach allows learning high-quality representations by optimizing the modality-contrastive objective via both of these setups.

Additionally, we focused on interpretability as well and leveraged the task-specific attention mechanism in our pipeline, which pools the representations from different modalities, to study the utility and contribution of observations from each feature group.
This enables the network to dynamically assign weights to each modality's latent representation (in the shared space) as it processes each instance, allowing us to study their contribution both per instance and in expectation for performing the desired task.
In Figure \ref{fig:attn_interp}, we have shown the results on this matter for the contrastive regularization setup\footnote{The label $\texttt{heart}$ in Figure \ref{fig:attn_interp} corresponds to the $\texttt{daily}$ modality's information, given that its main focus is heart-rate.}.
The results indicate that even though the contributions of the different modalities follow a non-uniform distribution as expected, none of them were ignored by the model and they all play a part in the final predictions.

\section{Results}
\label{results}

\begin{figure*}[ht]
    \centering
    \includegraphics[scale=0.15,trim={0 2.5cm 0 5cm},clip]{images/aoi-single_burst}
    \caption{The time average peak Age of Information with burst and \gls{soa} loss values against the dynamic reliability logic for different network topologies.}
    \label{fig:aoi_burst}\vspace{-0.4cm}
\end{figure*}


This paper focuses on both transport layer and application layer metrics to determine the feasibility of dynamic reliability. For this, we have selected the session packet volume, as transmitted, retransmitted, lost and backlogged packets as \glspl{kpi} for the transport layer; while focusing on the \gls{aoi} for the application layer. The \gls{aoi} was chosen as a crucial indicator for the freshness of packets in real-time applications. More specifically, this work adopts the time average peak \gls{aoi} equation \cite{aoi_equation} depicted in Eq. \ref{aoi}, where $\Delta(r_{i+1})$ is the $i$th update at the time it was received at the server, for a session time period of $\tau$.

\begin{equation}
    \label{aoi}
    \gls{aoi}_\tau = \frac{1}{n-1}\sum_{i=1}^{n-1} \Delta(r_{i+1})
\end{equation}

We include a comparison between the vanilla QUIC implementation which does not enjoy the dynamic reliability extension, with a number of dynamic reliability policies. The tests were run a number of times for statistical significance, with the mean value of vanilla implementation used as a baseline for comparison. The topology utilised both random loss and bursty loss to explore the bounds of dynamic reliability. The \gls{soa} loss in the figures correspond to the loss values presented in Table. \ref{tab:path_char}, for ease of comparison between bursty and random loss scenarios.

\subsection{Transport-Layer KPIs}

To analyse the performance gain at the transport layer due to dynamic reliability, the volume of transmitted and backlogged packets is examined. The figures are in the form of boxplots, which take the vanilla implementation as a benchmark, depicted as the red dashed line.

As seen in Fig. \ref{fig:sent_burst}, the loss plays a crucial role in the performance of the reliability policies. The policies under random loss did incredibly well for the networks with a larger capacity, namely \gls{mmwave} and Sub-6~GHz, whereas for burst loss, the lower network capacities had a larger packet reduction. With the increase in burst loss, the behaviour of the set split reliable policies became unpredictable, if a reliable assignment happened to coincide with a burst loss, the number of transmitted packets increases, and vice versa. On the other hand, in smarter policies, such as Loss-Aware, the performance lightly matched the vanilla baseline, as the reliable assignment dominated the session to compensate for a higher burst loss. Not only that but, the burst loss also impacted the variance of the transmitted packets for the policies.

Unsurprisingly, the unreliable focused policy, 80-20 split, outperformed other policies for all topologies in random and bursty loss scenarios, with an approximate reduction of 80\%. That being said, the majority of the policies reduced the transmitted packets on the link by approximately 70\% for random loss, while the reduction started at $\approx 15\%$ and decreased as the loss increased for the burst loss scenario.

The retransmitted and lost packets, not shown due to space limitations, followed the same trend as the transmitted packets for the random loss scenarios. However, for the burst loss scenarios, the larger capacity networks had a lower reduction in the retransmitted and lost packets. This can be seen as a favorable outcome since the lower capacity networks are scarce on resources. It is important to note that the Loss-Aware policy mimicked the vanilla approach as the burst loss increased, signifying the overwhelming appointment of reliable packets in adapting to the harsh burst loss conditions.
 
Alternatively, Fig. \ref{fig:backlog_burst} clearly shows a stark comparison between the policies and loss scenario in the reduction of the backlogged packets. The Loss-Aware policy for random loss scenario reduced the backlogged packets by up to 50\%, beating all other policies by approximately 30\%. Furthermore, it is clear that the unreliability focused policies resulted in the lowest backlog for the session. In comparison, we notice that the burst loss and the backlogged frequency have a positive correlation, where the maximum reduction of the backlogged packets for the policies is at most 20\%. Much like the transmitted packets, the probability of a burst loss occurrence plays a vital role in the number of retransmissions sent and by extension the number of backlogged packets. Thus, we can conclude that the stress placed on the buffer is a result of the reliable packets which is tightly coupled with the congestion on the session. Whereas, unreliable focused policies did not encounter such a phenomenon regardless if it was experiencing a burst loss.


\subsection{Application-Layer KPIs}

The feasibility of dynamic reliability for real-time applications can be determined by the \gls{aoi}, with comparison across different topologies and policies. If we take a strict approach and consider anything below $10$~ms is real-time \cite{real-time}, then all the reliability policies passed that requirement, which is attractive for real-time applications, as shown in Fig. \ref{fig:aoi_burst}. Utilising the median as an estimate of the runs, the policies in the WLAN and Sub-6~GHz topology with random loss floated around $4-5$~ms with negligible difference, while the \gls{aoi} for \gls{mmwave} was $\approx 2-3$~ms. It is clear that the \gls{aoi} and the network capacity have a negative correlation, as the network capacity decreases, the \gls{aoi} increases. The same correlation is extended to the bursty loss scenarios, where \gls{mmwave} dominated the other topologies. That being said, it is crucial to note that the \gls{aoi} for the reliability policies is often slightly better than or equal to the \gls{aoi} of the vanilla implementation, proving that dynamic reliability reduces the congestion of the session at no cost to the \gls{aoi}.





