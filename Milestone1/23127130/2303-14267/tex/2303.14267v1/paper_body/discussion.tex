\section{Discussion}
\subsection{Broader Impact}
% discussing transferrability and inter-smartwatch compatibilities etc. in the broader impact.
In the context of remote health monitoring, there are several factors addressing which is of paramount importance.
In what follows, we elaborate upon these factors and how the solution proposed in this work attempts to address them:
\begin{itemize}
\item{
{\it Affordability and Compatibility}: For the scalability of a proposed remote health monitoring framework, focusing on widely available devices that are sold at affordable price renders it easier to deploy the system.
In this work, we focused on basic physiological signals for which reading sensors are available in most commercially available smartwatches.
Nonetheless, the proposed methodology has no intrinsic limitation regarding the modalities used; thus, additional data available in often more expensive devices (e.g., galvanic skin response) can also be utilized in the same methodology, and the main requirement is providing a modality-specific encoder fit for the data domain.
Furthermore, this framework offers a more encapsulated view in representing different modalities as the observation from each can be embedded by a dedicated encoder first, and the contrastive objective encourages each local branch to optimize its parameters towards the given task as well.
This has clear advantages in terms of transferring knowledge as well, an example of which could be initializing each branch separately via pre-trained weights so as to prepare a better starting point for the model and optimization.
}
\item{
{\it Ease of use}: Optimizing a remote health monitoring with regards to minimizing the amount of required user interaction makes it easier for individuals to use the system.
This is why passive monitoring techniques are receiving more attention in the eHealth domain.
}
\item{
{\it Interpretation}: In all automated healthcare applications of machine learning, any insight and interpretation into what parts of the observation a model mostly focused on in determining the final decision, is crucial and can help experts better validate the system as well.
In this work, we incorporated a task-specific attention mechanism for pooling the representations from different modalities, which helps determine the weights assigned to each modality (per instance and in expectation) to perform the task efficiently.
}
\item{
{\it Limited Data}: The data availability for eHealth applications is often limited due to the difficulty and costs of conducting large-scale studies, the exclusivity of data, and privacy reasons.
It is, therefore, important to try to maximize the use of data in training inference pipelines.
This work combines label smoothing with inter-mode self-supervision objectives to go beyond self-reported supervision objectives.
}
\end{itemize}

\subsection{Limitations}
It is crucial to discuss the limitations of this work given the sensitive nature of dealing with health as its objective.
%% error in the data
In this work, we relied on self-reported entries to decide the supervision signals for individual timelines.
This has the issue of being prone to human error, as one might not accurately recall the time and extent to which one has felt stress.
Additionally, reports on the intensity of the felt stress are also subject to noise.
%% small data size and its challenges, use of ssl to address it to some extent
Another challenge is the small size of our dataset.
A primary reason behind our self-supervision component in this work was alleviating the negative impacts associated with the aforementioned limitations.

\subsection{Conclusion}
We proposed a remote health monitoring solution that is modular and multi-modal, thus, allowing the use of various encoders best suited for each modality. 
We proposed an instance-level attention mechanism to tune the contribution of each modality to the final representation and provide insight into the expected importance of each modality for the task at hand.
We conducted experiments with the proposed method to recognize perceived stress in short-term episodes and empirically demonstrated its superior performance over supervised training.



