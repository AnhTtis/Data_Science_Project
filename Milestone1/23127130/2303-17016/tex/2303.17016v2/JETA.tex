\documentclass[aps,prl,amsmath,lengthcheck,superscriptaddress]{revtex4-2}

\bibliographystyle{apsrev4-2}
\usepackage{graphicx}\graphicspath{ {figures/} }
\usepackage{hyperref}
\hypersetup{colorlinks,allcolors=blue,breaklinks}

%new commands

\newcommand{\be}{\begin{equation}}
\newcommand{\ee}{\end{equation}}
\newcommand{\ba}{\begin{align}}
\newcommand{\ea}{\end{align}}
\newcommand{\bi}{\begin{itemize}}
\newcommand{\ei}{\end{itemize}}
\newcommand{\R}{\mathbb{R}}
\newcommand{\bra}[1]{\langle #1|}
\newcommand{\ket}[1]{|#1\rangle}
\newcommand{\braket}[2]{\left\langle #1|#2\right\rangle}
\newcommand{\ketbrad}[1]{|#1\rangle\!\langle #1|}
\newcommand{\tr}[1]{\mathrm{tr}\left\{#1\right\}}
\newcommand{\ptr}[2]{\mathrm{tr_{#1}}\left\{#2\right\}}
\newcommand{\di}[1]{\mathrm{div}\left\{#1\right\}}
\newcommand{\la}{\left\langle}
\newcommand{\ra}{\right\rangle}
\newcommand{\pd}{\partial}
\newcommand{\de}[1]{\delta\left(#1\right)}
\newcommand{\td}{\mathrm{d}}
\newcommand{\ma}[1]{\max{\left\{#1\right\}}}
\newcommand{\mi}[1]{\min{\left\{#1\right\}}}
\newcommand{\etal}{\textit{et al.}}
\newcommand{\e}[1]{\exp{\left(#1\right)}}
\newcommand{\lo}[1]{\ln{\left(#1\right)}}
\newcommand{\id}{\mathbb{I}}
\newcommand{\com}[2]{\left[#1,\,#2\right]}
\newcommand{\acom}[2]{\left\{#1,\,#2\right\}}
\newcommand{\co}[1]{\cos{\left(#1\right)}}
\newcommand{\si}[1]{\sin{\left(#1\right)}}
\newcommand{\sh}[1]{\sinh{\left(#1\right)}}
\newcommand{\ch}[1]{\cosh{\left(#1\right)}}
\newcommand{\shi}[1]{\mathrm{shi}{\left(#1\right)}}
\newcommand{\cohi}[1]{\mathrm{chi}{\left(#1\right)}}
\newcommand{\ct}[1]{\coth{\left(#1\right)}}
\newcommand{\bla}{bla\\bla\\bla\\bla\\bla}

\newcommand{\mb}[1]{\mbox{\boldmath$#1$}}
\newcommand{\mc}[1]{\mathcal{#1}}
\newcommand{\mbb}[1]{\mathbb{#1}}
\newcommand{\mf}[1]{\mathfrak{#1}}
\newcommand{\mrm}[1]{\mathrm{#1}}

\begin{document}

\title{Jarzynski equality for time-averaged work}

\author{Pierre Naz\'e}
\email{pierre.naze@unesp.br}

\affiliation{\it Departamento de F\'isica, Instituto de Geoci\^encias e Ci\^encias Exatas, Universidade Estadual Paulista ``J\'ulio de Mesquita Filho'', 13506-900, Rio Claro, SP, Brazil}

\date{\today}

\begin{abstract}

There is evidence that taking the time average of the work performed by a thermally isolated system ``transforms'' the adiabatic process into an isothermal one. We add here one more fact to this case by showing that the time-averaged difference of Helmholtz free energy is equal to the time-averaged quasistatic work of the system. Therefore, a Jarzynski equality relating the time-averaged work and time-averaged quasistatic work is established. As an immediate consequence, the expected relation $\langle W\rangle\ge W_{\rm qs}$ is demonstrated. Numerical evidence for the equality is also presented for the classical harmonic oscillator with a driven linear equilibrium position parameter.

\end{abstract}

\maketitle

{\it Introduction.} Previous works have shown that taking the time average of the work performed by thermally isolated systems ``transforms'' the adiabatic process into an isothermal one \cite{naze2022}. In this way, important characteristics of isothermal processes, like the existence of a decorrelation time, are now present in adiabatic processes. Possible solutions for conundrums involving the existence of relaxation times, like in the Kibble-Zurek mechanism, are therefore proposed \cite{naze2023kibble}. 

In this work, I show that, for thermally isolated systems subjected to the time average work, the time-averaged difference of free energy is equal to the time-averaged quasistatic work. Because of this, the following Jarzynski equality is established
\be
\langle e^{-\beta_0 \overline{W}} \rangle = e^{-\beta_0 \overline{W}_{\rm qs}},
\label{eq:jarzynskiequality}
\ee
where $\overline{W}$ is the time-averaged work performed by the system along an adiabatic driven process, $\overline{W}_{\rm qs}$ is the time-averaged quasistatic work, and $\beta_0:=(k_B T_0)^{-1}$, with $k_B$ being Boltzmann constant, and $T_0$ the temperature of the initial thermal equilibrium of the thermally isolated system. Numerical evidence is presented for the classical harmonic oscillator with a driven linear equilibrium position parameter.

{\it Preliminaries.} Consider a system with Hamiltonian $\mathcal{H} ({\bf z}({\bf z}_0,t),\lambda(t))$, dependent on some external parameter $\lambda(t)=\lambda_0+g(t)\delta\lambda$, with $g(0)=0$ and $g(\tau)=1$, where $\tau$ is the switching time of the process. Here, ${\bf z}({\bf z}_0,t)$ is a point in the phase space $\Gamma$ of the system evolved from the initial condition ${\bf z}_0$ until the instant $t$. Initially, the system is in thermal equilibrium with a heat bath at a temperature $T_0$. When the process starts, the system is decoupled from the heat bath and adiabatically evolves in time, that is, without any source of heat. The work performed along the process is
\be
W({\bf z_0},\tau)  = \int_0^\tau \partial_\lambda \mathcal{H} ({\bf z}({\bf z_0},t),\lambda(t))\dot{\lambda}(t)dt,
\ee
or as well
\be
W({\bf z_0},\tau)  = \mathcal{H} ({\bf z}({\bf z_0},\tau),\lambda_0+\delta\lambda)-\mathcal{H} ({\bf z_0},\lambda_0).
\ee
Its average is
\be
\langle W \rangle(\tau) = \int_\Gamma\int_0^\tau \partial_\lambda \mathcal{H} ({\bf z}({\bf z}_0,t),\lambda(t))\rho_{\mathcal{H}}({\bf z}({\bf z}_0,t),t)\dot{\lambda}(t)dtd{\bf z},
\ee
or as well
\begin{multline}
\langle W \rangle(\tau) = \int_\Gamma \mathcal{H} ({\bf z}({\bf z}_0,\tau),\lambda_0+\delta\lambda)\rho_{\mathcal{H}}({\bf z}({\bf z}_0,\tau),\tau)d{\bf z}\\
-\int_\Gamma \mathcal{H} ({\bf z_0},\lambda_0)\rho_{\mathcal{H}}({\bf z}_0)d{\bf z_0},
\end{multline}
where the probability distribution $\rho_{\mathcal{H}}$ is the solution of the Liouville equation
\be
\frac{\partial\rho}{\partial t} = \mathcal{L}\rho,
\ee
where $\mathcal{L}:=-\{\cdot,\mathcal{H}\}$ is the Liouville operator. Here, $\{\cdot,\cdot\}$ is the Poisson bracket. 

In particular, we call quasistatic work the average work perform in a process made in the quasistatic regime
\begin{multline}
W_{\rm qs}(\lambda_0+\delta\lambda) = \int_\Gamma [\mathcal{H} ({\bf z}({\bf z}_0,\lambda_0+\delta\lambda),\lambda_0+\delta\lambda)\times\\
\rho_{\mathcal{H}}({\bf z}({\bf z}_0,\lambda_0+\delta\lambda),\lambda_0+\delta\lambda)]d{\bf z}-\int_\Gamma \mathcal{H} ({\bf z_0},\lambda_0)\rho_{\mathcal{H}}({\bf z}_0)d{\bf z_0},
\end{multline}
where ${\bf z}({\bf z_0},\lambda)$ is the solution of the Hamilton equations for Hamiltonian $\mathcal{H}$ taking as parameter $\lambda$ instead of $t$. 

{\it Time-averaged work.} The quantity that will interest us here is the time average of the work, given by
\be
\overline{W}({\bf z_0},\tau) = \frac{1}{\tau}\int_0^\tau W({\bf z}_{\rm ef}({\bf z_0},t),t)dt,
\ee
which can be rewritten as
\be
\overline{W}({\bf z_0},\tau)  = \int_0^\tau \partial_t \mathcal{H}^{\rm ef} ({\bf z}_{\rm ef}({\bf z_0},t),t)dt,
\ee
with the effective Hamiltonian
\begin{equation}
\mathcal{H}^{\rm ef}({\bf z_0},t) = \frac{1}{t}\int_0^t \mathcal{H}({\bf z}({\bf z_0},t'),\lambda(t'))dt'.
\end{equation}
The solutions of such an effective Hamiltonian will be represented by ${\bf z_{\rm ef}}({\bf z_0},t)$. In particular, when a quasistatic process is made, the effective Hamiltonian calculated at the parameter $\lambda$ is
\be
\mathcal{H}^{\rm ef}({\bf z_0},\lambda) = \frac{1}{\lambda-\lambda_0}\int_{\lambda_0}^{\lambda}\mathcal{H}({\bf z}({\bf z_0},\lambda),\lambda)d\lambda,
\ee
where $\mathcal{H}$ evolves accordingly with the quasistatic process, that is, solving the Hamilton equations taking the parameter $\lambda$ instead of the time. Observe that when $\lambda=\lambda_0$, the effective Hamiltonian reduces to the original Hamiltonian calculated at $\lambda_0$.

The non-equilibrium average of the time-averaged work is
\be
\langle \overline{W} \rangle(\tau) = \int_{\Gamma}\int_0^\tau \partial_t \mathcal{H}^{\rm ef} ({\bf z}_{\rm ef}({\bf z_0},t),t)\rho_{\mathcal{H}^{\rm ef}}({\bf z}_{\rm ef}({\bf z_0},t),t) dtd{\bf z},
\ee
where $\rho_{\mathcal{H}^{\rm ef}}$ is the solution of Liouville equation regarding $\mathcal{H}^{\rm ef}$. However, since both effective Hamiltonian and original Hamiltonian have the same initial canonical ensemble, we can use the Liouville theorem and involve the non-equilibrium distribution regarding $\mathcal{H}^{\rm ef}$. In this way, we have
\be
\langle \overline{W} \rangle(\tau) = \int_{\Gamma}\int_0^\tau \partial_t \mathcal{H}^{\rm ef} ({\bf z_0},t)\rho_{\mathcal{H}}({\bf z_0},\lambda_0) dtd{\bf z_0}
\ee
which leads to
\be
\langle \overline{W} \rangle(\tau) = \frac{1}{\tau}\int_0^\tau\langle W\rangle(t)dt, 
\ee
where we used the Liouville theorem to evolve the initial canonical ensemble regarding $\mathcal{H}$. Finally, considering a quasistatic process, the time-average quasistatic work will be
\begin{multline}
\overline{W}_{\rm qs}(\lambda_0+\delta\lambda) = \int_\Gamma [\mathcal{H}^{\rm ef}({\bf z}_{\rm ef}({\bf z}_0,\lambda_0+\delta\lambda),\lambda_0+\delta\lambda)\times\\
\rho_{\mathcal{H}^{\rm ef}}({\bf z}_{\rm ef}({\bf z}_0,\lambda_0+\delta\lambda),\lambda_0+\delta\lambda)]d{\bf z}_{\rm ef}-\int_\Gamma \mathcal{H} ({\bf z_0},\lambda_0)\rho_{\mathcal{H}}({\bf z}_0)d{\bf z_0},
\label{eq:TAwqs}
\end{multline}
or, as well,
\be
\overline{W}_{\rm qs} = \frac{1}{\delta\lambda}\int_{\lambda_0}^{\lambda_0+\delta\lambda}W_{\rm qs}(\lambda)d\lambda.
\ee

{\it Time-averaged difference of Helmholtz free energy.} Imagine now that the system is weakly coupled to a heat bath of Hamiltonian $\mathcal{H}_{\rm bath}({\bf z'}({\bf z'_0},t))$. Its effective Hamiltonian is
\begin{equation}
\mathcal{H}^{\rm ef}_{\rm bath}({\bf z'_0},t) = \frac{1}{t}\int_0^t \mathcal{H}_{\rm bath}({\bf z'}({\bf z'_0},t'))dt',
\end{equation}
with the solutions represented by ${\bf z'_{\rm ef}}({\bf z'_0},t)$.
The Hamiltonian of the total system is
\begin{equation}
\mathcal{H}_{\rm total}({\bf z_0},{\bf z'_0},t) =\mathcal{H}({\bf z_0},t)+\mathcal{H}_{\rm bath}({\bf z'_0},t),
\end{equation}
and its effective Hamiltonian is
\begin{equation}
\mathcal{H}^{\rm ef}_{\rm total}({\bf z_0},{\bf z'_0},t) =\mathcal{H}^{\rm ef}({\bf z_0},t)+\mathcal{H}^{\rm ef}_{\rm bath}({\bf z'_0},t).
\end{equation}
Performing a quasistatic process, the time-average difference of Helmholtz free energy will be
\be
\overline{\Delta F} = \int_{\lambda_0}^{\lambda_0+\delta\lambda} \langle \partial_\lambda \mathcal{H}^{\rm ef}_{\rm total} \rangle_\lambda(\lambda)d\lambda,
\label{eq:overlinedeltaf1}
\ee
where the operation $\langle\cdot\rangle_{\lambda}$ is the average in the canonical ensemble of the total system calculated at the parameter $\lambda$
\be
\rho_{\mathcal{H}^{\rm ef}_{\rm total}}({\bf z_0},{\bf z'_0},\lambda) = \frac{e^{-\beta_0 \mathcal{H}_{\rm total}^{\rm ef}({\bf z_0},{\bf z'_0},\lambda)}}{\mathcal{Z}_{\rm total}^{\rm ef}(\lambda)},
\ee
where $\mathcal{Z}_{\rm total}^{\rm ef}(\lambda)$ is the partition function and
\be
\mathcal{H}^{\rm ef}_{\rm total}({\bf z_0},{\bf z'_0},\lambda) = \mathcal{H}^{\rm ef}({\bf z_0},\lambda)+\mathcal{H}^{\rm ef}_{\rm bath}({\bf z'_0},\lambda),
\label{eq:overlinedeltaf2}
\ee
\be
\mathcal{H}^{\rm ef}_{\rm bath}({\bf z'_0},\lambda) = \frac{1}{\lambda-\lambda_0}\int_{\lambda_0}^{\lambda}\mathcal{H}_{\rm bath}({\bf z'}({\bf z'_0},\lambda'))d\lambda'.
\ee
Observe that the Hamiltonian inside the integrals evolves accordingly with the quasistatic driving.

{\it Isothermal process.} Using the Liouville theorem, we can rewrite Eq.~\eqref{eq:overlinedeltaf1} with an average on the initial canonical ensemble
\begin{multline}
\overline{\Delta F} = \int_{\Gamma}\int_{\lambda_0}^{\lambda_0+\delta\lambda}[\partial_\lambda \mathcal{H}^{\rm ef}_{\rm total} ({\bf z_{\rm ef}}({\bf z_0},\lambda),{\bf z'_{\rm ef}}({\bf z'_0},\lambda),\lambda)\times\\
\rho_{\mathcal{H}^{\rm ef}_{\rm total}}({\bf z_0},{\bf z'_0},\lambda_0)]d\lambda d{\bf z_0}d{\bf z'_0},
\end{multline}
Two terms come from solving this integral. The first one is the time-averaged quasistatic work, given by Eq.~\eqref{eq:TAwqs}. It remains to show what happens to the contribution of the heat bath in the time-averaged difference of Helmholtz free energy. I argue that it goes to zero. Indeed, consider that such a contribution is given by $Q$. Splitting it into $Q=Q_1+Q_2$, where
\be
Q_1 = \int_{\Gamma}\left[\frac{1}{\delta\lambda}\int_{\lambda_0}^{\lambda_0+\delta\lambda}e^{\mathcal{L}_{\rm bath}(\lambda-\lambda_0)}d\lambda \right]\mathcal{H}_{\rm bath}({\bf z'_0})\rho_{\mathcal{H}_{\rm bath}}({\bf z'_0})d{\bf z'_0}
\ee
\be
Q_2 = -\int_{\Gamma}\mathcal{H}_{\rm bath}({\bf z'_0})\rho_{\mathcal{H}_{\rm bath}}({\bf z'_0})d{\bf z'_0},
\ee
where I use the fact that the initial canonical ensemble of the effective heat bath Hamiltonian coincides with the original heat bath Hamiltonian. The quantity $Q_1$ is
\begin{equation}
\begin{split}
Q_1 &= \int_{\Gamma}\left(\frac{e^{\delta\lambda\mathcal{L}_{\rm bath}}-1}{\delta\lambda}\right) \mathcal{H}_{\rm bath}({\bf z'_0})\rho_{\mathcal{H}_{\rm bath}}({\bf z'_0})d{\bf z'_0}\\ 
&=\int_{\Gamma}\sum_{n=0}^\infty\frac{\mathcal{L}_{\rm bath}^n \delta\lambda^n}{(n+1)!} \mathcal{H}_{\rm bath}({\bf z'_0})\rho_{\mathcal{H}_{\rm bath}}({\bf z'_0})d{\bf z'_0}\\
&=\sum_{n=0}^\infty \frac{\delta\lambda^n}{(n+1)!}\int_{\Gamma} \mathcal{L}_{\rm bath}^n\mathcal{H}_{\rm bath}({\bf z'_0})\rho_{\mathcal{H}_{\rm bath}}({\bf z'_0})d{\bf z'_0}\\
&=\sum_{n=0}^\infty \frac{(-1)^n\delta\lambda^n}{(n+1)!}\int_{\Gamma} \mathcal{H}_{\rm bath}({\bf z'_0})\mathcal{L}_{\rm bath}^n\rho_{\mathcal{H}_{\rm bath}}({\bf z'_0})d{\bf z'_0}\\
&=\int_{\Gamma} \mathcal{H}_{\rm bath}({\bf z'_0})\rho_{\mathcal{H}_{\rm bath}}({\bf z'_0})d{\bf z'_0}\\
&=-Q_2
\end{split}
\end{equation}
Therefore
\be
\overline{\Delta F} = \overline{W}_{\rm qs}.
\label{eq:deltafequalwqs}
\ee
This result tells us that the under the time average, the system behaves as performing an isothermal process.

\begin{figure}[h]
    \centering
    \includegraphics[scale=0.5]{JE.eps}
    \caption{Jarzynski equality for the classical harmonic oscillator, with a driven linear equilibrium position parameter. It was used datasets with $10^5$ values of $\exp{[-\beta(\overline{W}-\overline{W}_{\rm qs})]}$ sampled accordingly with the canonical ensemble, $\beta_0=1$, $\lambda_0=1$, $\delta\lambda=0.5$.}
    \label{fig:JE}
\end{figure}

{\it Jarzynski equality.} 
From Ref.~\cite{jarzynski1997} and Eq.~\eqref{eq:deltafequalwqs}, it follows the equality
\be
\langle e^{-\beta_0 \overline{W}} \rangle = e^{-\beta_0 \overline{W}_{\rm qs}}.
\label{eq:JE1}
\ee
Using Jensen's inequality, we have
\be
\langle \overline{W} \rangle(\tau) \ge \overline{W}_{\rm qs}(\lambda_0+\delta\lambda).
\ee
To prove that $\langle W\rangle(\tau)\ge W_{\rm qs}(\lambda_0+\delta\lambda)$, suppose that, along the process, for each instant $t$ and corresponding parameter $\lambda(t)=\lambda_0+t\delta\lambda/\tau$, it holds
\be
\langle W \rangle(t) < W_{\rm qs}(\lambda_0+t\delta\lambda/\tau).
\ee
Therefore
\begin{equation}
\begin{split}
\langle \overline{W} \rangle(\tau)&< \frac{1}{\tau}\int_0^\tau W_{\rm qs}(\lambda_0+t\delta\lambda/\tau)dt\\
&=\frac{1}{\delta\lambda}\int_{\lambda_0}^{\lambda_0+\delta\lambda}W_{\rm qs}(\lambda)d\lambda\\
&=\overline{W}_{\rm qs}(\lambda_0+\delta\lambda),
\end{split}
\end{equation}
so, by absurd, $\langle W\rangle(t)\ge W_{\rm qs}(\lambda_0+t\delta\lambda/\tau)$, for each $t$. In particular the result holds for $t=\tau$.


{\it Testing.} In order to corroborate Jarzynski equality~\eqref{eq:jarzynskiequality}, we test the classical harmonic oscillator with a driven linear equilibrium position parameter
\be
\mathcal{H} = \frac{p^2}{2}+\frac{(q-\lambda(t))^2}{2},\quad \lambda(t)=\lambda_0+\frac{t}{\tau}\delta\lambda.
\ee
Figure~\ref{fig:JE} presents the result of the simulation. It was used for each switching time $\tau$ a dataset with $N=10^5$ values of $\exp{[-\beta_0(\overline{W}-\overline{W}_{\rm qs})]}$, sampled accordingly with the canonical ensemble, and parameters $\beta_0=1$, $\lambda_0=1$ and $\delta\lambda=0.5$. The results for rapid protocols deviate about $1\%$ from the expected result. I attribute that to the inherent difficulty of sampling exponentially weighted random variables, mainly in this type of process whose distributions are wider and require the occurrence of rare events \cite{jarzynski2006}.  

{\it Conclusion.} I proved that thermally isolated systems, when measured using the time-averaged work, present their time-averaged difference of free energy equal to the time-averaged quasistatic work. Because of this, a Jarzynski equality relating the fluctuations of the time-averaged work with the time-averaged quasistatic work is established. From such a result, I derived the expected inequality $\langle W\rangle\ge W_{\rm qs}$. Numerical evidence for the equality is also presented for the classical harmonic oscillator with a driven linear equilibrium position parameter. Finally, I remark that such Jarzynski equality~\eqref{eq:jarzynskiequality} seems to be a better alternative to relate the notions of work and quasistatic work of a thermally isolated system performing an adiabatic driven process than the Jarzynski equality found in \cite{naze2023jarzynski1}.

\bibliography{bibliography.bib}
\bibliographystyle{apsrev4-2}

\end{document}