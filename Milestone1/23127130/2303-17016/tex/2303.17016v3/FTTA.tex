\documentclass[aps,prl,amsmath,lengthcheck,superscriptaddress,onecolumn]{revtex4-2}

\bibliographystyle{apsrev4-2}
\usepackage{graphicx}\graphicspath{ {figures/} }
\usepackage{hyperref}
\hypersetup{colorlinks,allcolors=blue,breaklinks}

%new commands

\newcommand{\be}{\begin{equation}}
\newcommand{\ee}{\end{equation}}
\newcommand{\ba}{\begin{align}}
\newcommand{\ea}{\end{align}}
\newcommand{\bi}{\begin{itemize}}
\newcommand{\ei}{\end{itemize}}
\newcommand{\R}{\mathbb{R}}
\newcommand{\bra}[1]{\langle #1|}
\newcommand{\ket}[1]{|#1\rangle}
\newcommand{\braket}[2]{\left\langle #1|#2\right\rangle}
\newcommand{\ketbrad}[1]{|#1\rangle\!\langle #1|}
\newcommand{\tr}[1]{\mathrm{tr}\left\{#1\right\}}
\newcommand{\ptr}[2]{\mathrm{tr_{#1}}\left\{#2\right\}}
\newcommand{\di}[1]{\mathrm{div}\left\{#1\right\}}
\newcommand{\la}{\left\langle}
\newcommand{\ra}{\right\rangle}
\newcommand{\pd}{\partial}
\newcommand{\de}[1]{\delta\left(#1\right)}
\newcommand{\td}{\mathrm{d}}
\newcommand{\ma}[1]{\max{\left\{#1\right\}}}
\newcommand{\mi}[1]{\min{\left\{#1\right\}}}
\newcommand{\etal}{\textit{et al.}}
\newcommand{\e}[1]{\exp{\left(#1\right)}}
\newcommand{\lo}[1]{\ln{\left(#1\right)}}
\newcommand{\id}{\mathbb{I}}
\newcommand{\com}[2]{\left[#1,\,#2\right]}
\newcommand{\acom}[2]{\left\{#1,\,#2\right\}}
\newcommand{\co}[1]{\cos{\left(#1\right)}}
\newcommand{\si}[1]{\sin{\left(#1\right)}}
\newcommand{\sh}[1]{\sinh{\left(#1\right)}}
\newcommand{\ch}[1]{\cosh{\left(#1\right)}}
\newcommand{\shi}[1]{\mathrm{shi}{\left(#1\right)}}
\newcommand{\cohi}[1]{\mathrm{chi}{\left(#1\right)}}
\newcommand{\ct}[1]{\coth{\left(#1\right)}}
\newcommand{\bla}{bla\\bla\\bla\\bla\\bla}

\newcommand{\mb}[1]{\mbox{\boldmath$#1$}}
\newcommand{\mc}[1]{\mathcal{#1}}
\newcommand{\mbb}[1]{\mathbb{#1}}
\newcommand{\mf}[1]{\mathfrak{#1}}
\newcommand{\mrm}[1]{\mathrm{#1}}

\begin{document}

\title{Fluctuation theorem for time-averaged work}

\author{Pierre Naz\'e}
\email{pierre.naze@unesp.br}

\affiliation{\it Departamento de F\'isica, Instituto de Geoci\^encias e Ci\^encias Exatas, Universidade Estadual Paulista ``J\'ulio de Mesquita Filho'', 13506-900, Rio Claro, SP, Brazil}

\date{\today}

\begin{abstract}

There is evidence that taking the time average of the work performed by a thermally isolated system ``transforms'' the adiabatic process into an isothermal one. Also, such a measurement accesses the inherent quantities of the system, which were not available to obtain with the usual work. I add here one more fact to this case by showing that the difference of the time-averaged Helmholtz's free energies is equal to the quasistatic work of the system. I also show a fluctuation theorem relating the time-averaged work and the quasistatic work. Numerical evidence for such an equality is also presented for the classical harmonic oscillator with a driven linear equilibrium position parameter. In the end, it is proposed that a more useful way to measure the spent of energy of a thermally isolated system is made by the time-averaged work instead of the usual work.

\end{abstract}

\maketitle

\section{Introduction} 

Previous works have shown that taking the time average of the work performed by thermally isolated systems ``transforms'' the adiabatic process into an isothermal one \cite{naze2023adiabatic}. In this way, important characteristics of isothermal processes, like the existence of a decorrelation time, can now be captured in adiabatic processes. Possible solutions for conundrums involving the existence of relaxation times, like in the Kibble-Zurek mechanism, were therefore proposed \cite{naze2023quantum}. 

In this work, I show that, for thermally isolated systems subjected to the time average work, the difference of time-averaged Helmholtz's free energy is equal to the quasistatic work. Because of this, the following fluctuation theorem is established
\be
\langle e^{-\beta \overline{W}} \rangle = e^{-\beta \langle {W}_{\rm qs}\rangle},
\label{eq:jarzynskiequality}
\ee
where $\overline{W}$ is the time-averaged work performed by the system along an adiabatic driven process, $\langle{W}_{\rm qs}\rangle$ is the quasistatic work, and $\beta:=(k_B T)^{-1}$, with $k_B$ being Boltzmann constant, and $T$ the temperature of the initial thermal equilibrium of the thermally isolated system. Numerical evidence is presented for the classical harmonic oscillator with a driven linear equilibrium position parameter at the end.

\section{Preliminaries} 

Consider a system with Hamiltonian $\mathcal{H} ({\bf z}({\bf z}_0,t),\lambda_\tau(t))$, dependent on some external parameter $\lambda_\tau(t)=\lambda_0+g(t/\tau)\delta\lambda$, with $g(0)=0$ and $g(1)=1$, where $\tau$ is the switching time of the process. Here, ${\bf z}({\bf z}_0,t)$ is a point in the phase space $\Gamma$ of the system evolved from the initial condition ${\bf z}_0$ until the instant $t$. Initially, the system is in thermal equilibrium with a heat bath at a temperature $T$. When the process starts, the system is decoupled from the heat bath and adiabatically evolves in time, that is, without any source of heat. The work performed along the process is
\be
W({\bf z_0},\tau)  = \mathcal{H} ({\bf z}({\bf z_0},\tau),\lambda_0+\delta\lambda)-\mathcal{H} ({\bf z_0},\lambda_0),
\ee
or, as well,
\be
W({\bf z_0},\tau)  = \int_0^\tau \partial_\lambda \mathcal{H} ({\bf z}({\bf z_0},t),\lambda_\tau(t))\dot{\lambda}_\tau(t)dt.
\ee
Its non-equilibrium average is
\be
\langle W \rangle(\tau) = \int_\Gamma \mathcal{H} ({\bf z}({\bf z}_0,\tau),\lambda_0+\delta\lambda)\rho_{\mathcal{H}}({\bf z}({\bf z}_0,\tau),\tau)d{\bf z}-\int_\Gamma \mathcal{H} ({\bf z_0},\lambda_0)\rho_{\mathcal{H}}({\bf z_0},0)d{\bf z_0},
\ee
or, as well,
\be
\langle W \rangle(\tau) = \int_\Gamma\int_0^\tau \partial_\lambda \mathcal{H} ({\bf z}({\bf z_0},t),\lambda_\tau(t))\rho_{\mathcal{H}}({\bf z}({\bf z_0},t),t)\dot{\lambda}_\tau(t)dtd{\bf z},
\ee
where the probability distribution $\rho_{\mathcal{H}}({\bf z}({\bf z}_0,t),t)$ is the solution of the Liouville equation
\be
\frac{\partial\rho_{\mathcal{H}}}{\partial t} = \mathcal{L}_{\mathcal{H}}\rho_{\mathcal{H}},
\ee
where $\mathcal{L}_{\mathcal{H}}:=-\{\cdot,\mathcal{H}\}$ is the Liouville operator and $\rho({\bf z_0},0)$ is the canonical ensemble. Here, $\{\cdot,\cdot\}$ is the Poisson bracket. 

In particular, we call quasistatic work the average work performed in a process made in the quasistatic regime
\be
\langle W_{\rm qs}\rangle:=\lim_{\tau\rightarrow\infty}\langle W\rangle(\tau)=\lim_{\tau\rightarrow\infty}\int_\Gamma\int_0^\tau \partial_\lambda \mathcal{H} ({\bf z}({\bf z}_0,t),\lambda_\tau(t))\rho_{\mathcal{H}}({\bf z}({\bf z}_0,t),t)\dot{\lambda}_\tau(t)dtd{\bf z}.
\ee

\section{Time-averaged work} 

The quantity that will interest us here is the time-averaged work, given by~\cite{naze2023adiabatic,naze2023quantum}
\be
\langle\overline{W}\rangle(\tau) = \frac{1}{\tau}\int_0^\tau \langle W\rangle(t)dt,
\ee
or, as well,
\be
\langle\overline{W}\rangle(\tau)  = \frac{1}{\tau}\int_0^\tau \left[\int_\Gamma \mathcal{H} ({\bf z}({\bf z}_0,t),\lambda_\tau(t))\rho_{\mathcal{H}}({\bf z}({\bf z}_0,t),t)d{\bf z}-\int_\Gamma \mathcal{H} ({\bf z_0},\lambda_0)\rho_{\mathcal{H}}({\bf z_0},0)d{\bf z_0}\right].
\ee
Observe that it can be rewritten as
\be
\langle\overline{W}\rangle(\tau)  = \int_0^\tau \frac{d}{dt}\left[\frac{1}{t}\int_0^t\int_\Gamma \mathcal{H} ({\bf z}({\bf z}_0,t'),\lambda_t(t'))\rho_{\mathcal{H}}({\bf z}({\bf z}_0,t'),t')d{\bf z}dt'\right]dt.
\ee
Using Liouville's theorem, one has
\be
\langle\overline{W}\rangle(\tau)  = \int_0^\tau\int_\Gamma \frac{d}{dt}\left[\frac{1}{t}\int_0^t\mathcal{H} ({\bf z}({\bf z}_0,t'),\lambda_t(t'))dt'\right]\rho_{\mathcal{H}}({\bf z_0},0)d{\bf z_0}dt.
\ee
Observe that the time-averaged work is an averaged work of the effective Hamiltonian
\begin{equation}
\mathcal{H}'({\bf z'}({\bf z_0},t),t) = \frac{1}{t}\int_0^t \mathcal{H}({\bf z}({\bf z'}({\bf z_0},t),t'),\lambda_t(t'))dt',
\end{equation}
where ${\bf z'}({\bf z_0},t)$ is the solution of Hamilton's equations of the effective Hamiltonian. Indeed, observe that $\mathcal{H}'({\bf z'}({\bf z_0},0),0)=\mathcal{H}({\bf z_0},\lambda_0)$. Also, consider $\rho_{\mathcal{H}'}$ as the probability distribution associated with the effective Hamiltonian, where the initial probability distribution is equal to the canonical ensemble associated with $\mathcal{H}$. Using Liouville's theorem, one can show that
\be
\langle\overline{W}\rangle(\tau)  = \int_0^\tau\int_\Gamma \frac{d}{dt}\left[\frac{1}{t}\int_0^t\mathcal{H} ({\bf z}({\bf z'}({\bf z_0},t),t'),\lambda_t(t'))dt'\right]\rho_{\mathcal{H}}({\bf z_0},0)d{\bf z_0}dt.
\ee
The random variable associated to the time-averaged work is
\be
\overline{W}(\tau)  = \frac{1}{\tau}\int_0^\tau \mathcal{H} ({\bf z}({\bf z'}({\bf z_0},\tau),t),\lambda_\tau(t))dt-\mathcal{H} ({\bf z_0},\lambda_0).
\ee
Considering $\tau\rightarrow\infty$, the time-averaged quasistatic work will be
\be
\langle\overline{W}_{\rm qs}\rangle = \lim_{\tau\rightarrow\infty}\frac{1}{\tau}\int_0^\tau \langle W\rangle(t)dt=\langle W_{\rm qs} \rangle.
\label{eq:TAwqs}
\ee
Also, the difference in the time-averaged Helmholtz's free energy between the final and initial states is defined as
\be
\Delta\overline{F}:=\overline{F}(\tau)-\overline{F}(0)=-\frac{1}{\beta}\ln\left(\frac{\int_{\Gamma}e^{-\beta \frac{1}{\tau}\int_0^\tau \mathcal{H} ({\bf z}({\bf z'_\tau},t),\lambda_\tau(t))dt}d{\bf z'_\tau}}{\int_{\Gamma}e^{-\beta \mathcal{H}({\bf z_0},\lambda_0)}d{\bf z_0}}\right).
\ee

\section{Equivalence between $\overline{W}_{\rm qs}$ and $\Delta \overline{F}$}

The difference in the time-averaged Helmholtz's free energies can be rewritten as
\be
\Delta\overline{F} = \lim_{\tau\rightarrow\infty}\int_0^\tau\int_\Gamma \frac{d}{dt}\left[\frac{1}{t}\int_0^t\mathcal{H} ({\bf z}({\bf z}_t',t'),\lambda_t(t'))dt'\right]\rho^c_{\mathcal{H}'}({\bf z}_t',t)d{\bf z}_t'dt,
\label{eq:overlinedeltaf1}
\ee
where
\be
\rho^c_{\mathcal{H}'}({\bf z}_t',t) = \frac{e^{-\beta \frac{1}{t}\int_0^t\mathcal{H} ({\bf z}({\bf z}_t',t'),\lambda_t(t'))dt'}}{\int_{\Gamma}e^{-\beta \frac{1}{t}\int_0^t\mathcal{H} ({\bf z}({\bf z}_t',t'),\lambda_t(t'))dt'}d{\bf z}_t'},
\ee
is the probability distribution of the effective system and heat bath evolving in a quasistatic process. Since there is no term of the heat bath inside the derivative, the expression, when calculated at the final and initial points, does not include the corresponding part of the heat bath. This leads to
\be
\Delta\overline{F} = \lim_{\tau\rightarrow\infty}\int_\Gamma \left[\frac{1}{\tau}\int_0^\tau\mathcal{H} ({\bf z}({\bf z}'({\bf z_0},\tau),t),\lambda_\tau(t))dt\right]\rho_{\mathcal{H}}({\bf z_0},0)d{\bf z_0}-\int_\Gamma \mathcal{H}({\bf z_0},\lambda_0)\rho_{\mathcal{H}}({\bf z_0},0)d{\bf z_0},
\ee
where we use Liouville's theorem and the fact that the time evolution is a canonical transformation such that the volume of the phase space does not change.
Then the variation of the time-averaged Helmholtz's free energies between the final and initial state is nothing more than the work of the effective system performed in a quasistatic process. Therefore, $\Delta\overline{F}=\langle\overline{W}_{\rm qs}\rangle=\langle{W}_{\rm qs}\rangle$.

\section{Fluctuation theorem}

Following the usual idea used in Ref.~\cite{jarzynski1997}, observe
\begin{equation}
\begin{split}
\langle e^{-\beta \overline{W}} \rangle &= \int_{\Gamma} \exp\left(-\beta\left(\frac{1}{\tau}\int_0^\tau \mathcal{H}({\bf z}({\bf z'_\tau},t),\lambda_\tau(t))dt\right)+\beta\mathcal{H}({\bf z_0},\lambda_0)\right)\rho_{\mathcal{H}}({\bf z_0},0)d{\bf z'_\tau}\\
&=\frac{\int_{\Gamma} \exp\left(\frac{1}{\tau}\int_0^\tau \mathcal{H}({\bf z}({\bf z'_\tau},t),\lambda_\tau(t))dt\right)d{\bf z'_\tau}}{\int_{\Gamma} \exp\left(-\beta\mathcal{H}({\bf z_0},\lambda_0)\right)d{\bf z_0}}\\
&=e^{-\beta \Delta\overline{F}}\\
&=e^{-\beta \langle\overline{W}_{\rm qs}\rangle}\\
&=e^{-\beta \langle W_{\rm qs}\rangle}\\
\label{eq:JE1}
\end{split}
\end{equation}
Therefore, it holds Eq.~\eqref{eq:jarzynskiequality}. To corroborate this fluctuation theorem, I test the classical harmonic oscillator with a driven linear equilibrium position parameter
\be
\mathcal{H} = \frac{p^2}{2}+\frac{(q-\lambda(t))^2}{2},\quad \lambda(t)=\lambda_0+\frac{t}{\tau}\delta\lambda.
\ee
Figure~\ref{fig:JE} presents the result of the simulation. It was used for each switching time $\tau$ a dataset with $N=10^5$ values of $\exp{[-\beta(\overline{W}-\langle{W}_{\rm qs}\rangle)]}$, sampled accordingly with the canonical ensemble, and parameters $\beta=1$, $\lambda_0=1$ and $\delta\lambda=0.5$. The results for rapid protocols deviate about $1\%$ from the expected result. I attribute that to the inherent difficulty of sampling exponentially weighted random variables, mainly in this type of process whose distributions are wider and require the occurrence of rare events \cite{jarzynski2006}.  

\begin{figure}[h]
    \centering
    \includegraphics[scale=0.5]{JE.eps}
    \caption{Jarzynski equality for the classical harmonic oscillator, with a driven linear equilibrium position parameter. It was used datasets with $10^5$ values of $\exp{[-\beta(\overline{W}-\langle{W}_{\rm qs}\rangle)]}$ sampled accordingly with the canonical ensemble, $\beta_0=1$, $\lambda_0=1$, $\delta\lambda=0.5$.}
    \label{fig:JE}
\end{figure}

Using Jensen's inequality, from the fluctuation theorem it follows the inequality
\be
\langle \overline{W} \rangle \ge \langle W_{\rm qs}\rangle.
\ee
Such a relation is known as strong inequality, which is not true for thermally isolated systems when their spent of energy is measured with the usual work~\cite{naze2023jarzynski1}. However, with the time-averaged work, it was possible to derive it. This indicates one more time that the time-averaged work derives similar rules such as a system performing an isothermal process, but with characteristic quantities of thermally isolated systems.

\section{Conclusion} 

I proved that thermally isolated systems, when its spent of energy is measured using the time-averaged work, present their difference of time-averaged Helmholtz's free energy equal to the quasistatic work. Because of this, a fluctuation theorem relating the time-averaged work with the quasistatic work is established. Numerical evidence for such an equality is presented for the classical harmonic oscillator with a driven linear equilibrium position parameter. The strong inequality is also derived for the time-averaged work. The results found in this work corroborate that the time-averaged work ``transforms'' the thermally isolated system performing an adiabatic process into a system performing an isothermal one, with characteristic quantities related to the original system. Under such a scenario, measuring the spent of energy of thermally isolated system by means of the time-averaged work seems a better alternative than using the usual work since recognizable physical laws appear indeed.

\bibliography{bibliography.bib}
\bibliographystyle{apsrev4-2}

\end{document}