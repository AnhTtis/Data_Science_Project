
\section{Conclusion} \label{5-conclusion}
%PCM-based photonic memories are promising alternatives for electronic memories such as ReRAMs and DRAMs due to higher energy-efficiency, lower latency, higher volume and scalability, and the capability of storing multiple bits per cell. 
In this paper, we presented a design-space exploration of PCM-based photonic memories with silicon photonics using three well-known PCMs, namely GST, GSST, and Sb$_2$Se$_3$. Parameters such as optical insertion loss of the cell in the amorphous state, optical transmission contrast between amorphous and crystalline state, cell's bit capacity, and set and reset energies are explored to design an optimized photonic memory cell. We showed that for thermally controlled PCM-based photonic memory cells with GST or GSST, as the bit capacity of the cells increases, the maximum set energy also increases drastically. Finally, we presented an example of a memory array using the optimized memory cells and explored the scalability and maximum set energy in the array as the size of the array changes. Our results show the promise of PCM-based photonic memories and the critical need for cross-layer design co-optimization (material to array level) to minimize energy and latency costs in such memories. 



\section*{ACKNOWLEDGEMENTS}
This work was supported in part by the National Science Foundation under grants CCF-2006788 and CNS-2046226.
\vspace{-0.05in}