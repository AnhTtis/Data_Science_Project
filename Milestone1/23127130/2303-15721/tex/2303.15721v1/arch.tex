
\begin{figure}[t]
    \centering
    \includegraphics[width=0.48\textwidth]{bit_cap_energy_v5.pdf}
     \vspace{-0.25in}
    \caption{(a) Maximum set energy versus cell's bit capacity for two PCM-based photonic memory cells using GST and GSST. A constant 6~mW electrical pulse with various pulse widths is used. (b) Power-latency trade-off for the same cells to achieve fully crystalline state to write "111111" to the cell.}
    \label{latency}
    \vspace{-0.22in}
\end{figure}
%Due to using heaters for tuning the rings in this design, there is no need for fine-tuning the input and drop gaps, hence the same value has been considered for the cells in the rows to ensure equal splitting of the light between the cells in the memory rows. In addition, the output gaps are the same for all the cells in the memory to guarantee the maximum coupling of the light from the PCM waveguide to the output ports. 
\section{PCM-based Memory Arrays}\label{4_arch}
Leveraging the cell introduced in Section \ref{2_background} (see Fig. \ref{cell_arch}), one can realize a PCM-based photonic memory array by cascading $M$ cells per row, and for the total number of $N$ rows with the configuration depicted in Fig. \ref{arch_main}. Here, we consider the design presented in \cite{feldmann2019integrated}. Note that the original design in \cite{feldmann2019integrated} used different ring radii to induce different resonant wavelength shifts in a row, while in our work microheaters on the rings are used to realize the required resonant shift to read and write data with the PCM-based photonic memory cells. The reason for using heaters instead of different radii is to actively control the resonant shift in the rings and the spacing between the resonant peaks, which creates an additional degree of freedom when designing a memory array. Due to using heaters for tuning the rings in this design, there is no need for fine-tuning the input, drop, and output gaps in the rings associated with each PCM cell. The resonant wavelength of the rings in each row is slightly different ($\Delta \lambda$ = 850 pm \cite{feldmann2019integrated}), which is controlled by the heaters in our design.
The readout of each cell in the memory depicted in Fig.~\ref{arch_main} can be done in two steps. First, we can select the row to be read using output ports S$_1$ to S$_N$. Then, the cell to be read from within each row can be selected via the input wavelength,
due to the slight difference between the resonant wavelengths of all rings in a row \cite{feldmann2019integrated}. Finally, the optical signal transmission from each cell can be converted to an electrical signal via photodetectors (PDs) at the end of each output port, to retrieve the stored data. 

\begin{figure}[t]
    \centering
    \includegraphics[width=0.49\textwidth]{architecture_v5.pdf}
     \vspace{-0.15in}
    \caption{PCM-based photonic memory array based on the design presented in \cite{feldmann2019integrated}. PD: Photodetector.}
    \label{arch_main}
    \vspace{-0.1in}
\end{figure}

Employing the two cells designed in Section \ref{3_cells}, we explore the scalability of the memory array in Fig. \ref{arch_main}. The required laser output optical power ($P_{lsr}$) for reading from the last cell in each row in this memory array can be defined as \cite{shafiee_loci}:
\begin{equation}
     P_{lsr} \geq S_{PD} + [(N\times M)-1 + (M-1)] \times L_{p} + 2L_{d} + L_{amorphous}(\lambda).\label{penalty}
\end{equation}
Assuming $M=N$, and that all the cells contain 0 (amorphous state), PD's sensitivity of $S_{PD}=$~$-$11.7~dBm \cite{9199100PD, palmieri2020enhanced}, average ring passing/drop loss of $L_{p} =L_d=$~0.1~dB \cite{feldmann2019integrated}, and average loss in the amorphous state (for the wavelength range of $\lambda=$~1530--1565~nm) of $L_{amorphous} =$~0.35~dB for GST, and 0~dB for GSST, the required laser power to read from the last cell in the last row of the memory array (i.e., worst-case optical power consumption) is shown in Fig. \ref{laser_penalty}(a). The insertion loss in the amorphous state is considered constant due to the small spectrum spacing between the wavelengths used to read the data. Note that too much increase in the readout optical power requirement can lead to an increased number of read errors due to the change in the state of the PCM (especially for the case when the cells in the first row of the memory array have a high crystallization fraction). We can see from the results in Fig. \ref{laser_penalty}(a) that the array size plays an important role and as we increase the memory array size, the optical power at the input increases dramatically and can be as high as 30.4~dBm for a PCM-based photonic memory, regardless of the maximum number of bits per cell. Note that the effect of loss in the amorphous state is insignificant in the memory design presented in Fig. \ref{arch_main}. This is because of using different wavelengths to read different cells in a row. Therefore, each read optical signal experiences loss from a single PCM cell it is reading from, and losses from other cells will not impact this signal. However, the read signal suffers from higher optical losses as the memory array scales up, due to the increase in the number of rings it passes (and higher propagation loss, not discussed in this paper).%when using PCMs for photonic computing such as in the case when they are used to store the weights for performing matrix-vector multiplication, a small amount of insertion loss in each cell can easily accumulate with increasing the size of the network \cite{feldmann2021parallel_nature_tensor_core}.


The maximum set energy is another parameter that changes with scaling of the memory array capacity. The maximum set energy for the memory array using 6-bit GST- and GSST-based cells to write "111111" is shown in Fig. \ref{laser_penalty}(b) for different memory array capacities (i.e., total number of bits stored) with $M=N$. As can be seen, as we increase the array capacity, the maximum write energy of the entire memory increases linearly, and it can be as high as 0.6~mJ for GST and 0.4~mJ for GSST. Note that a 6~mW electrical pulse is used to write on the cells via heaters. Considering the results in Fig. \ref{laser_penalty}, we can see that scaling up a memory array to increase its capacity is infeasible without further optimization of the cell's structure due to high input optical power required to compensate for the losses. For example, to store 2400~bits (120-bits per row and 6-bits per cell when $M=N=$20), the input laser should provide at least 30.4~dBm to compensate for the losses throughout the memory array. Consequently, the 1\% margin between optical transmission levels considered in this paper may lead to unreliable readouts due to the undesired change in the state of the cells due to increased input optical power.
This motivates the need for a trade-off between the transmission level's margin and the number of levels, and therefore the cell's bit capacity. Moreover, optical transmission drift is another limitation that can lead to unreliable readout of cells. Such drifts impose a higher set pulse duration and lower bit capacity (to achieve a larger margin between the optical transmission levels) to stabilize the cell's state when writing the data \cite{survey_shafiee, li2019fast_multilevel_5bit}. Another limiting factor in scaling the design in Fig. \ref{arch_main} is the free-spectral range (FSR) of the rings. We cannot arbitrarily increase the number of rings to store more bits per array. Increasing the number of rings per row necessitates a larger number of operating wavelengths to store and read the data from the cells. This leads to the essential need for rings with a large FSR, which requires smaller rings (FSR is inversely proportional to ring radius) at the cost of increased optical loss in the rings.

 \begin{figure}[t]
    \centering
    \includegraphics[width=0.47\textwidth]{Laser_pen_last.pdf}
     \vspace{-0.1in}
    \caption{(a) Optical power consumption and (b) maximum set energy (6-bits per cell) for the memory array in Fig. \ref{arch_main}.}
    \label{laser_penalty}
    \vspace{-0.1in}
\end{figure}







