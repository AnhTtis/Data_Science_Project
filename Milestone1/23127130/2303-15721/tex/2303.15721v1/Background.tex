\section{Background}\label{2_background} 
In this section, we present an overview of the fundamentals of PCMs and how PCM-based photonic memory cells work.

\vspace{-0.1in}
\subsection{Fundamentals and Properties of PCMs}
The state of a PCM can change from the amorphous to crystalline state, and vice versa, in a nonvolatile manner, leading to different optical and electrical properties.
One important material parameter of PCMs
is the melting temperature ($T_l$). The regions of the material after absorbing the energy from a heat source that have a temperature
above $T_l$ will be melted and quenched. The quenched region will have an amorphous structure
regardless of the initial state of the material. This process is
called \textit{reset}. Another important material parameter is the
crystallization temperature ($T_g$). When a PCM in the
intermediate state (or amorphous state) absorbs energy from a heat source,
the regions of the material with temperatures larger
than $T_g$ and lower than $T_l$ will recapture the crystalline structure. This process is called \textit{set}. Note that $T_l>T_g$, making the reset the most power-hungry procedure, compared to set.

%Note that PCMs require zero static power to maintain their state over time \cite{wang2021scheme,li2019fast_multilevel_5bit}. The portion of the material in the crystalline state can be estimated via the temperature distribution which can be calculated by solving the unsteady-transient heat flow equation in the cell upon the heat transfer in the material.

 A PCM is in an intermediate state when a portion of the material has an amorphous state and the rest has a crystalline state. The crystallized area of the PCMs can be estimated by analyzing the temperature distribution of the cell when being heated \cite{youngblood2019tunable}. Temperature distribution in a PCM can be calculated by solving the unsteady-transient heat flow equation in the cell upon the heat transfer in the material \cite{li2019fast_multilevel_5bit}.
The required energy to trigger the phase transition of a PCM can be provided electrically, thermally, or optically \cite{narayan2022architecting}. For electrically (thermally) controlled PCMs, a PN junction (microheater) can be used to apply heat and initiate the phase transition \cite{rios2021ultra_phaseshifter}. When triggered optically, a laser pulse with specific power and duration will be used to set or reset the cells.  %The nonvolatile characteristic, fast switching speed, and not needing an external energy source to maintain the phase of the PCMs makes them a promising alternative to conventional DRAMs.


 \begin{figure}[t]
    \centering
    \includegraphics[width=0.40\textwidth]{cell_v5.pdf}
     %\vspace{-0.5in}
    \caption{Unit cell of a PCM-based photonic memory \cite{feldmann2019integrated}.} \label{cell_arch}
    \vspace{-0.2in}
\end{figure}


To implement PCM-based photonic memory cells, understanding the optical properties of PCMs is important. Upon a phase (state) transition, the optical refractive index of the PCM, and hence the optical transmission of the cell, will change drastically. This can be used to store data on the cell's optical transmission levels. The optical refractive index profile of three PCMs (GST, GSST, and Sb$_2$Se$_3$) is shown in Fig. \ref{Fig1_opt}. Observe the drastic contrast between the crystalline and amorphous state of the PCMs. Note that for C-band (1530–1565 nm), GST shows the highest contrast in refractive index when shifting from amorphous to the crystalline state, and vice versa. This makes GST the most suitable candidate to implement PCM-based photonic memory cells. In addition, observe that the PCMs in the crystalline state have a much higher extinction coefficient compared to their amorphous state. This leads to higher absorption of the optical power in the crystalline state compared to the amorphous state. The absorbed optical power will be converted to heat, and can be used to trigger the phase transition in PCMs. 
To estimate the optical refractive index profile of the PCMs in any intermediate state, one can use the Lorenz model in \cite{wang2021scheme, survey_shafiee}.\vspace{-0.05in} 

\begin{figure}[t]
    \centering
    \includegraphics[width=0.3\textwidth]{cell_heater.pdf}
     %\vspace{-0.15in}
    \caption{Designed PCM-based photonic memory cell. Set and reset are carried out using a microheater on top of the cell.}
    \label{cell_heater}
    \vspace{-0.23in}
\end{figure}

%\begin{equation}
%\frac{\varepsilon_{e f f}(\lambda)-1}{\varepsilon_{e f f}(\lambda)+2}=X_{f} \times \frac{\varepsilon_{c}(\lambda)-1}{\varepsilon_{c}(\lambda)+2}+\left(1-X_{f}\right) \times \frac{\varepsilon_{a}(\lambda)-1}{\varepsilon_{a}(\lambda)+2}.
%\label{lorenz}
%\end{equation}
%Here, $X_f$ is the crystalline fraction and takes a number between 0 and 1 (0 for fully amorphous state and 1 for fully crystalline states), taking into consideration the portion of the PCM which is in the crystalline state. Moreover, the wavelength-dependent dielectric permittivity function ($\varepsilon(\lambda)$) can be calculated as:
%\begin{equation}
%    \varepsilon_a = n_a^2,
%\end{equation}
%\begin{equation}
%    \varepsilon_c = n_c^2,
%\enda



%where $n_c$ and $n_a$ are the complex refractive indices of the \textcolor{black}{PCM}. 
%Finally, using (\ref{lorenz}), the real and the imaginary part of the effective refractive index---which determines the phase delay and absorption of the light in a material---of a \textcolor{black}{PCM} in an intermediate (mixed) state can be estimated as:
%\begin{equation}
%n_{e f f}=\sqrt{\frac{\sqrt{\left(\varepsilon_{1}+\varepsilon_{2}\right)^{2}}+\varepsilon_{1}}{2}},
%\label{neff}
%\end{equation}
%\begin{equation}
%    k_{e f f}=\sqrt{\frac{\sqrt{\left(\varepsilon_{1}+\varepsilon_{2}\right)^{2}}-\varepsilon_{1}}{2}}.
 %   \label{nk_eff}
 %   \end{equation}
 %In (\ref{neff}) and (\ref{nk_eff}), $\varepsilon_{2}$ and $\varepsilon_{1}$ are the real and imaginary part of $\varepsilon_{eff}(\lambda)$ in (\ref{lorenz}).
 
\begin{figure*}[t]
    \centering
    \includegraphics[width=0.9\textwidth]{Loss_amor.pdf}
     \vspace{-0.15in}
    \caption{Insertion loss of PCM-based photonic memory cells with different materials and geometries. WG: Waveguide.}
    \label{loss_amor}
   \vspace{-0.1in}
\end{figure*}

 \subsection{PCM-based Photonic Memory}
A PCM-based photonic memory cell can be realized by depositing a PCM on top of an SOI waveguide. The schematic of a PCM-based photonic memory cell is shown in Fig. \ref{cell_arch} \cite{feldmann2019integrated}. In this design, the light in the input waveguide couples to the lower ring, and then passes through the waveguide with the PCM on top of it (i.e., memory cell). The heaters on the rings are responsible for tuning the resonant wavelength in the rings
%The coupling gaps can be adjusted to control the amount of power coupled to the rings and memory cell. 
Note that in this design, both rings should have the same resonant wavelength to ensure correct operation with the same wavelength \cite{feldmann2019integrated}. %The SOI thickness of 220~nm was considered in all of the simulations in this paper.

 Because of the higher refractive index and extinction coefficient of PCMs in the crystalline state (see Fig. \ref{Fig1_opt}), as the PCM in the unit cell starts to crystallize, the optical transmission of the cell decreases due to the absorption of optical power in the PCM. The optical transmission contrast due to the absorption of light in the PCM helps realize multiple, distinct optical transmission levels between the initial and final state of the material, to store single or multiple bits per cell \cite{survey_shafiee, rios2015integrated}. As the optical transmission contrast between the initial and final state of the PCM increases, the cell is able to store a larger number of bits. However, this comes at a cost of higher power consumption or latency because a larger portion of the PCM needs to be crystallized. The light-PCM interaction reduces when using silicon (e.g., instead of silicon nitride \cite{rios2015integrated}) in PCM-based photonic memories \cite{li2020experimental_SiN_Si}. This makes the PCM-based photonic memories with SiPh more power-hungry with higher latency. However, PCM-based photonic memories with SiPh offer a more compact footprint, lower propagation loss, and compatibility with CMOS fabrication foundries \cite{li2020experimental_SiN_Si}.



 As mentioned earlier, phase transitions in PCMs can be triggered electrically, thermally, or optically. In this paper, the set and reset procedures are carried out thermally and using a microheater on top of the cell, due to the decreased light-matter interaction between silicon and PCM and, consequently, lower optical absorption in the PCM (on top of silicon waveguide) in amorphous and intermediate states. The schematic design of the cell with a microheater is shown in Fig. \ref{cell_heater}. Using this design, a low-power electrical signal with a long duration can be used to set (crystallize) the cell. A short-duration pulse with higher power (compared to the set pulse) can be used to reset (switching to the amorphous state) the cell, regardless of the initial state.
 \vspace{-0.1in}









 %\subsection{Related Prior Work}






 