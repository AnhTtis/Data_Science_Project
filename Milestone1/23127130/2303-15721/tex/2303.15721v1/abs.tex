\begin{abstract}
%Compared to electronic accelerators, integrated silicon-photonic neural networks (SP-NNs) promise higher speed and energy efficiency for emerging artificial-intelligence applications. However, a hitherto overlooked problem in SP-NNs is that the underlying silicon photonic devices suffer from intrinsic optical loss and crosstalk noise, the impact of which accumulates as the network scales up. Leveraging precise device-level models, this paper presents the first comprehensive and systematic optical loss and crosstalk modeling framework for SP-NNs. For an SP-NN case study with two hidden layers and 1380 tunable parameters, we show a catastrophic ~84\% drop in inferencing accuracy due to optical loss and crosstalk noise.

%Integrated photonic devices and systems based on silicon photonics (SiPh) have been widely deployed across multiple application domains, from light-speed chip-scale communication to energy-efficient optical computing in emerging hardware accelerators for machine learning. 
The integration of silicon photonics (SiPh) and phase change materials (PCMs) has created a unique opportunity to realize adaptable and reconfigurable photonic systems. In particular, the nonvolatile programmability in PCMs has made them a promising candidate for implementing optical memory systems. In this paper, we describe the design of an optical memory cell based on PCMs while exploring the design space of the cell in terms of PCM material choice (e.g., GST, GSST, Sb$_2$Se$_3$), cell bit capacity, latency, and power consumption. Leveraging this design-space exploration for the design of efficient optical memory cells, we present the design and implementation of an optical memory array and explore its scalability and power consumption when using different optical memory cells. We also identify performance bottlenecks that need to be alleviated to further scale optical memory arrays with competitive latency and energy consumption, compared to their electronic counterparts.
 
\end{abstract}

\begin{CCSXML}
<ccs2012>
   <concept>
       <concept_id>10010583.10010786.100108%10</concept_id>
       <concept_desc>Hardware~Emerging %optical and photonic technologies</concept_desc>
       <concept_significance>500</concept_s%ignificance>
       </concept>
 </ccs2012>
\end{CCSXML}

\ccsdesc[500]{Hardware~Emerging optical and photonic technologies}
%\vspace{-2in}
\keywords{Integrated Photonics, Phase Change Materials, Photonic Memories}
%\copyrightyear{2022} 
%\acmYear{2022} 
%\setcopyright{acmcopyright}
%\acmConference[GLSVLSI '22] {Proceedings of the Great Lakes Symposium on VLSI 2022}{June 6--8, 2022}{Irvine, CA, USA.}
%\acmBooktitle{Proceedings of the Great Lakes Symposium on VLSI 2022 (GLSVLSI '22), June 6--8, 2022, Irvine, CA, USA}
%\acmPrice{15.00}
%\acmISBN{978-1-4503-9322-5/22/06} 
%\acmDOI{10.1145/XXXXXX.XXXXXX}
% Authors, replace the red X's with your assigned DOI string during the rightsreview eform process.

