\section{PCM-based photonic memory cells }\label{3_cells}
 In this section, we present a detailed design-space exploration of PCM-based photonic memory cells for the design demonstrated in Fig. \ref{cell_heater}, using GST, GSST, and Sb$_2$Se$_3$ PCMs. \vspace{-0.1in}%Parameters such as optical insertion loss of the cell in the amorphous state, optical transmission contrast between amorphous and crystalline state, footprint, cell's capacity, and set and reset energies are explored to design an optimized photonic memory cell.


\subsection{Cell Insertion Loss}
 The insertion loss can be defined as the attenuation of the input optical signal when the cell is in an amorphous state. When a PCM is in the amorphous state (i.e., contains 0), it should not attenuate the input optical signal. The insertion loss originates from the extinction coefficient in the amorphous state (see Fig. \ref{Fig1_opt}). Note that as the cell starts to crystallize, the loss of the cell is in fact originating from the optical power absorption in the cell, which determines the optical transmission contrast being used to store data.

 
 Considering the cell in Fig. \ref{cell_heater}, Fig.\ref{loss_amor} shows the optical insertion loss for different PCMs of different geometries (width and thickness) at 1550~nm. Results are based on simulations in Lumerical MODE solver \cite{MODE}. Note that we consider the PCM's width and waveguide's width to be the same. Out of the three PCMs under test, GST shows the highest optical insertion loss in the amorphous state due to its higher extinction coefficient in the C-band (see Fig. \ref{Fig1_opt}(a)), where its loss can be as high as $\approx$0.6~dB/$\mu$m (see Fig. \ref{loss_amor}(a)). Moreover, note that the loss in the amorphous state for GST increases with its thickness, while the effect of PCM or waveguide width is insignificant. Despite GST's high insertion loss in the amorphous state, it has the highest contrast in the refractive index switching from the amorphous to crystalline state, making it the best candidate for photonic memories. Next are GSST and Sb$_2$Se$_3$ that are lossless in the amorphous state (see Figs. \ref{loss_amor}(b) and \ref{loss_amor}(c)),
 but compared to GST, have lower contrast in the refractive index between the two states. 
 Note that to realize PCM-based photonic memory cells, having low loss in the amorphous state and high refractive index contrast between crystalline and amorphous state is ideal.

 \begin{figure*}[t]
    \centering
    \includegraphics[width=0.9\textwidth]{bit_capacity4.pdf}
     \vspace{-0.2in}
    \caption{(a)--(c) Optical transmission contrast ($\Delta T$) and (d)--(f) total absorption contrast ($\Delta P$) between crystalline and amorphous state for PCM-based photonic memory cells with GST, GSST, and Sb$_2$Se$_3$.  Simulations are based on Lumerical FDTD.}
    \label{transmission}
    \vspace{-0.15in}
\end{figure*}

\vspace{-0.1in}

\subsection{Cell Capacity}
 Cell capacity in PCM-based photonic memories is a parameter that can be determined by capturing the optical transmission contrast between the amorphous and partially or fully crystallized state of the PCM \cite{li2019fast_multilevel_5bit}. As the crystallization fraction increases, the optical-transmission changes increase due to increased attenuation of the input optical signal. This leads to a higher number of separable signal levels to store data, and hence storing a larger number of bits. For example, for a 2-bit PCM-based photonic memory cell, only 4 signal levels are needed to store data (00, 10, 01, 11). %Therefore, up to $\approx$20\% of the PCM needs to be crystallized (20\% crystallization to write 11 on the cell) to realize up to 4 separable transmission levels. 
 
 The optical transmission contrast ($\Delta T$) and optical absorption contrast ($\Delta P$) between fully crystalline and fully amorphous state for 2-$\mu$m-long PCM-based photonic memory cells of different geometries and materials are shown in Fig. \ref{transmission}. Note that $\Delta T$ is not only a function of $\Delta P$ in the cells. $\Delta T$ partially originates from the optical-refractive-index mismatch between the PCM and SOI waveguide. The effect of the refractive-index contrast is more observable in Sb$_2$Se$_3$. We can see from Figs. \ref{transmission}(c) and \ref{transmission}(f) that for Sb$_2$Se$_3$, although $\Delta P$ is zero, the material unexpectedly shows some $\Delta T$ between the two states, which stems from the optical-refractive-index mismatch.  Note that such a $\Delta T$ in Sb$_2$Se$_3$ cannot be controlled actively as it is independent of the material absorption (or phase change of the material). In addition, Sb$_2$Se$_3$ shows lower refractive index contrast, and hence significantly lower $\Delta T$ compared to GST and GSST (see Fig. \ref{Fig1_opt}). These make Sb$_2$Se$_3$ not an ideal candidate to implement PCM-based photonic memories with SOI waveguides, necessitating some additional design optimization to address the optical-refractive-index mismatch.
 
%Note that in all of the simulations in this section, the phase transition from amorphous to the crystalline state has been considered uniform in the PCM's volume and the operating wavelength was considered to be $\lambda$=1.55$\mu$~m. Consequently, the refractive index profile of the PCMs in any intermediate states has been estimated using Lorenz model \cite{wang2021scheme}.

To avoid optical-refractive-index mismatch when designing GST- and GSST-based photonic memory cells, one should pick a design where both $\Delta T$ and $\Delta P$ are maximum. Doing so ensures that the $\Delta T$ is stemming from the optical power absorption. Accordingly, considering Figs. \ref{transmission}(a) and \ref{transmission}(d), for a 2-$\mu$m-long GST cell, $\Delta T$ and $\Delta P$ are at 95\% when the thickness of the cell is about 20~nm with the width of 470~nm. Note that the impact of waveguide/PCM width on $\Delta T$ and $\Delta P$ is negligible. This cell can store up to 6 bits (up to 64 separable signal levels), considering a $\approx$1\% ($0.96/64$) margin between each state of the cell. However, this cell suffers from 0.2~dB/$\mu$m insertion loss in the amorphous state (see Fig. \ref{loss_amor}(a)). Using the same approach, we can design a 2-$\mu$m-long, 40-nm thick GSST-based cell with a width of 470~nm to store 6 bits per cell, but with no insertion loss in the amorphous state.
%\vspace{-0.04in}

 The bit capacity of a cell is determined by adjusting the crystallized fraction of the cell. As mentioned in Section \ref{2_background}, the refractive index of a PCM in an intermediate state can be estimated using the Lorenz model from \cite{wang2021scheme, survey_shafiee}, and assuming a uniform phase transition in the PCM's volume from amorphous to crystalline state. Using the Lorenz model and FDTD simulations, and assuming $\approx$1\% margin to separate transmission levels \cite{li2019fast_multilevel_5bit}, to store a maximum of 2 (4) bits per cell, we found that up to 20\% (40\%) of the PCM needs to be crystallized when using GST and GSST. To store 6 bits per cell, these cells should be fully crystallized. Note that the aforementioned values are the required crystallization fraction for the extreme cases for writing "2$^n$$-$1" ($n$ is the bit capacity of the cell) to the cells. The crystalline fraction can be controlled by tuning the power and duration of the heat source being used to set the cells, and, as it was mentioned, it can be estimated by solving the unsteady-transient heat transfer equation in the PCM's volume \cite{wang2021scheme}. The reset procedure will be the same regardless of the maximum number of bits stored, as the PCM should return to its initial amorphous state. Storing multiple bits per a PCM-based photonic memory cell can be challenging due to the essential need for more complex programming and detection policies at the architectural level, when scaling the cells to implement memory arrays.\vspace{-0.1in}
 
\subsection{Latency and Power Consumption}
The latency and power consumption of a PCM-based photonic memory cell are a function of the cell's maximum bit capacity. As the cell's bit capacity increases, due to the need for a higher transmission contrast, more energy is required to reach higher levels of crystallization. Using the design in Fig. \ref{cell_heater}, a microheater is designed to set and reset PCM-based photonic memory cells. The heater material is Ti/TiN with $\rho=$~60~$\mu.\Omega$.cm and a sheet resistance of 5.5~$\Omega$/sq. The melting temperature of the heater material (Ti/TiN) is 1941 K, considered to avoid melting the heater upon heating the PCM. The thickness of the heater is 110~nm with the width and length of 2~$\mu$m, placed 600~nm above the waveguide to reduce metallic absorption due to metal-light interaction. Lumerical HEAT \cite{MODE} is used to carry out unsteady-transient heat transfer simulations to capture the temperature distribution in the PCMs (only GST and GSST; see Section 3.2), as a function of exposure time for a given electric power applied to the heater. %Note that the lower refractive-index contrast and higher optical-refractive-index mismatch effect without any absorption in the Sb$_2$Se$_3$-based cell led to not using this material to realize photonic memory cells.


Fig.\ref{latency}(a) shows the maximum set energy for the GST- and GSST-based photonic memory cells designed in Section 3.2,  when a 6~mW electrical pulse is applied to the heater with different pulses ($E=P.t$, where $t$ is the pulse duration and $P$ is the electric power applied to the heater). The $T_g$ for GST and GSST is considered to be 453~K and 423~K, respectively. In addition, the melting temperature of 890~K and 900~K is considered for GST and GSST, respectively \cite{aryana2021suppressed, li2019fast_multilevel_5bit, survey_shafiee, GSST_melting}. We can see from Fig.\ref{latency}(a) that, in general, as we increase the cell's bit capacity, the maximum energy required to set the cell (energy that is required to write "2$^n$$-$1", where $n$ is the cell's bit capacity) also increases. This is due to the essential need for larger transmission and optical absorption contrast. For example, in a 6-bit PCM-based photonic memory cell using GSST, the maximum energy of 175~nJ is required to write "111111" to the cell, while for GST, this energy can be as high as 248~nJ. An electric pulse of 40~mW with a duration of 3.5~$\mu$s is used to reset the cells by reaching the melting temperature of the PCMs, hence returning to the amorphous state. 

The power-latency trade-off for the 6-bit GST- and GSST-based cells is shown in Fig. \ref{latency}(b). As it can be seen, as we increase the maximum set power, the latency decreases and the trend is nonlinear. In addition, note that for power values lower than 6~mW, phase transition cannot be triggered, regardless of pulse duration. In other words, the required energy to trigger a specific phase transition is not always the same, and it depends on the electrical power used to write on the cells. This effect stems from the physical mechanism of the heat transfer from the heater to the PCMs given the sample's thermal properties, such as thermal conductivity, specific heat capacity, and density.



