\section{Introduction}


\begin{figure*}[t]
    \centering
    \includegraphics[width=0.9\textwidth]{Optical_properties_v2.pdf}
     \vspace{-0.1in}
    \caption{Optical refractive index ($n$) and extinction coefficient ($\kappa$) for different PCM materials and wavelengths \cite{huang2023tunable, teo2022comparison}.}
    \label{Fig1_opt}
    \vspace{-0.1in}
\end{figure*}

With the increased complexity and growth in computationally expensive and data-driven applications, 
conventional CMOS-integrated circuits fail to meet the performance demands of emerging computing and communication systems \cite{shafiee2022silicon}. To address such demands, silicon photonics (SiPh) has emerged with a promise of ultra-fast communication and high-performance computation with enhanced energy efficiency \cite{mirza_charach, SiPh_codesign}. More recently, silicon photonic devices have been integrated with PCMs to enable reconfigurable photonic systems. The phase state of PCMs can change from amorphous to crystalline, and vice versa, upon a change in the material temperature when heated. This leads to a drastic nonvolatile change in the material's optical and electrical properties, making them promising candidates to realize photonic memory cells \cite{rios2015integrated, survey_shafiee}.

%Emerging data-driven applications requires memories with higher cell densities, higher scalability, lower power consumption, and lower latency. 
Conventional CMOS-based volatile memories, such as SRAMs and DRAMs, are reaching their limits in terms of energy efficiency, volume, and speed. This has motivated the need for new memory technologies, such as PCM-based photonic memories, to be used as the main memory or storage-class memory in future computing systems. Compared to other nonvolatile memories such as ReRAMs, PCM-based photonic memories can achieve higher scalability, energy-efficiency, stability, and bandwidth \cite{narayan2022architecting,kim2018future, feldmann2021parallel_nature_tensor_core}.

A PCM-based photonic memory cell can be obtained by depositing a PCM on top of a silicon-on-insulator (SOI) waveguide \cite{li2020experimental_SiN_Si}. Once the PCM is heated, its state can change from amorphous to crystalline, and vice versa. Overall, PCMs in the crystalline state show a higher refractive index than in the amorphous state \cite{rios2015integrated}. The refractive-index contrast between the crystalline and amorphous states will affect the optical transmission of the cell, which in turn can be used to store single or multiple bits per cell. Different optical transmission levels can be realized in PCM-based photonic memory cells by controlling the heat source, and hence the crystallization and amorphization dynamics of the cells (PCMs can be partially crystallized---i.e., intermediate state). This helps to store multiple bits per cell. Nevertheless, as the number of bits per cell increases, due to the need for a larger change in the optical transmission of the cell, a higher portion of the PCM must be crystallized, leading to higher set energies per cell \cite{survey_shafiee, rios2015integrated,rios2021ultra_phaseshifter,li2019fast_multilevel_5bit, thakkar2017dyphase}.



%A high-performance PCM-based photonic memory cell require high energy-efficiency, low loss in the amorphous state, and compact footprint. 

The novel contribution of this paper is in presenting a detailed design-space exploration for PCM-based photonic memories, from material to system level while using three well-known PCMs: GST, GSST, and Sb$_2$Se$_3$. Such an exploration helps realize the optimal cell design to meet specific energy-efficiency, footprint, and maximum-tolerable-loss requirements in PCM-based photonic memory cells. By modeling optical properties of the three PCMs, we perform multi-physics simulations (FDE, FDTD, and HEAT) for PCM-based photonic memory cells, to understand the impact of changing the PCM and waveguide geometries on cell-level performance (e.g., optical loss and power required for a PCM state change). In addition, we consider a memory array design from \cite{feldmann2019integrated} to show how the power consumption and scalability of a memory array changes when using different cells explored in this paper. 

%The rest of the paper is organized as follows. Section \ref{2_background} gives an overview of the fundamentals of PCMs such as their optical and thermal properties as well as the general idea of designing an SOI-based photonic memory cell using PCMs. In Section \ref{3_cells}, we present the design-space exploration of PCM-based photonic memory cells of different geometries and materials in terms of loss in the amorphous state, optical absorption and transmission contrast between fully-amorphous and fully-crystalline state, cell capacity, and energy-efficiency. Starting from the results for a single cell, we move the analysis to the system-level in Section \ref{4_arch} to show how the energy-efficiency and insertion loss of the PCM-based photonic memory systems changes for a case-study architecture  as we scale up the memory. Lastly, Section \ref{5-conclusion} concludes this work.

\vspace{-0.1in}