
%%%%%%%%%%%%%%%%%%%%%%%%%%%%%%%%%%%%%%%%%%%%%%%%%%%%%%%%%%%%%%%%%%%%%
%% This is a (brief) model paper using the achemso class
%% The document class accepts keyval options, which should include
%% the target journal and optionally the manuscript type.
%%%%%%%%%%%%%%%%%%%%%%%%%%%%%%%%%%%%%%%%%%%%%%%%%%%%%%%%%%%%%%%%%%%%%
\documentclass[ journal=jctcce,manuscript=article]{achemso}

%%%%%%%%%%%%%%%%%%%%%%%%%%%%%%%%%%%%%%%%%%%%%%%%%%%%%%%%%%%%%%%%%%%%%
%% Place any additional packages needed here.  Only include packages
%% which are essential, to avoid problems later.
%%%%%%%%%%%%%%%%%%%%%%%%%%%%%%%%%%%%%%%%%%%%%%%%%%%%%%%%%%%%%%%%%%%%%
\usepackage{chemformula} % Formula subscripts using \ch{}
\usepackage[T1]{fontenc} % Use modern font encodings
%\usepackage{todonotes}
\usepackage{todo}
%\usepackage{minted}

%%%%%%%%%%%%%%%%%%%%%%%%%%%%%%%%%%%%%%%%%%%%%%%%%%%%%%%%%%%%%%%%%%%%%
%% If issues arise when submitting your manuscript, you may want to
%% un-comment the next line.  This provides information on the
%% version of every file you have used.
%%%%%%%%%%%%%%%%%%%%%%%%%%%%%%%%%%%%%%%%%%%%%%%%%%%%%%%%%%%%%%%%%%%%%
%%\listfiles

%%%%%%%%%%%%%%%%%%%%%%%%%%%%%%%%%%%%%%%%%%%%%%%%%%%%%%%%%%%%%%%%%%%%%
%% Place any additional macros here.  Please use \newcommand* where
%% possible, and avoid layout-changing macros (which are not used
%% when typesetting).
%%%%%%%%%%%%%%%%%%%%%%%%%%%%%%%%%%%%%%%%%%%%%%%%%%%%%%%%%%%%%%%%%%%%%
\newcommand*\mycommand[1]{\texttt{\emph{#1}}}
\definecolor{codegray}{gray}{0.9}
\newcommand{\codeinline}[1]{\colorbox{codegray}{\texttt{#1}}}
%5\setcounter{secnumdepth}{3}
\renewcommand*{\thefigure}{S\arabic{figure}}

%%%%%%%%%%%%%%%%%%%%%%%%%%%%%%%%%%%%%%%%%%%%%%%%%%%%%%%%%%%%%%%%%%%%%
%% Meta-data block
%% ---------------
%% Each author should be given as a separate \author command.
%%
%% Corresponding authors should have an e-mail given after the author
%% name as an \email command. Phone and fax numbers can be given
%% using \phone and \fax, respectively; this information is optional.
%%
%% The affiliation of authors is given after the authors; each
%% \affiliation command applies to all preceding authors not already
%% assigned an affiliation.
%%
%% The affiliation takes an option argument for the short name.  This
%% will typically be something like "University of Somewhere".
%%
%% The \altaffiliation macro should be used for new address, etc.
%% On the other hand, \alsoaffiliation is used on a per author basis
%% when authors are associated with multiple institutions.
%%%%%%%%%%%%%%%%%%%%%%%%%%%%%%%%%%%%%%%%%%%%%%%%%%%%%%%%%%%%%%%%%%%%%
\author{Jitai Yang}  
%	Zexing Qu , Hui Li
\affiliation[jlu]
{Institute of Theoretical Chemistry, College of Chemistry, Jilin University, 2519 Jiefang Road,
	Changchun 130023, P.R.China}


\author{Yang Cong}  
%	Zexing Qu , Hui Li
\affiliation[jlu]
{Institute of Theoretical Chemistry, College of Chemistry, Jilin University, 2519 Jiefang Road,
	Changchun 130023, P.R.China}

\author{You Li}  
%	Zexing Qu , Hui Li
\affiliation[jlu]
{Institute of Theoretical Chemistry, College of Chemistry, Jilin University, 2519 Jiefang Road,
	Changchun 130023, P.R.China}
\author{Hui Li}
\affiliation{Institute of Theoretical Chemistry, College of Chemistry, Jilin University, 2519 Jiefang Road,
	Changchun 130023, P.R.China}
\email{ Prof_huili@jlu.edu.cn}

%%%%%%%%%%%%%%%%%%%%%%%%%%%%%%%%%%%%%%%%%%%%%%%%%%%%%%%%%%%%%%%%%%%%%
%% The document title should be given as usual. Some journals require
%% a running title from the author: this should be supplied as an
%% optional argument to \title.
%%%%%%%%%%%%%%%%%%%%%%%%%%%%%%%%%%%%%%%%%%%%%%%%%%%%%%%%%%%%%%%%%%%%%
\title[An \textsf{achemso} demo]
  {Supporting Information for: A Deep Learning Approach Based on Range Corrected Deep Potential Model for Efficient Vibrational Frequency Computation}

%%%%%%%%%%%%%%%%%%%%%%%%%%%%%%%%%%%%%%%%%%%%%%%%%%%%%%%%%%%%%%%%%%%%%
%% Some journals require a list of abbreviations or keywords to be
%% supplied. These should be set up here, and will be printed after
%% the title and author information, if needed.
%%%%%%%%%%%%%%%%%%%%%%%%%%%%%%%%%%%%%%%%%%%%%%%%%%%%%%%%%%%%%%%%%%%%%
\abbreviations{IR,NMR,UV}
\keywords{American Chemical Society, \LaTeX}

%%%%%%%%%%%%%%%%%%%%%%%%%%%%%%%%%%%%%%%%%%%%%%%%%%%%%%%%%%%%%%%%%%%%%
%% The manuscript does not need to include \maketitle, which is
%% executed automatically.
%%%%%%%%%%%%%%%%%%%%%%%%%%%%%%%%%%%%%%%%%%%%%%%%%%%%%%%%%%%%%%%%%%%%%
\begin{document}

\section{Formic acid solution}
\begin{figure}[H]
	\includegraphics{dist_s_4in1.pdf}
	\caption{Error distribution in predicting of frequency shifts with models of \textit{ae} (a), \textit{atom} (b), \textit{mol} (c), \textit{regu} (d) results with 6~\AA\ cut range. The color of points correspond to the absolute error values.}
	\label{fgr:distS4in1}
\end{figure}

\begin{figure}[H]
	\includegraphics{fc_time_s_4in1.pdf}
	\caption{Training RMSE results of models \textit{ae} (a), \textit{atom} (b), \textit{mol} (c), \textit{regu} (d) with 6~\AA\ cut range over time.}
	\label{fgr:fcTimeS4in1}
\end{figure}

\begin{figure}[H]
	\includegraphics{fc_step_s_4in1.pdf}
	\caption{Training RMSE results of models \textit{ae} (a), \textit{atom} (b), \textit{mol} (c), \textit{regu} (d) with 6~\AA\ cut range over training steps.}
	\label{fgr:fcStepS4in1}
\end{figure}
\begin{figure}[H]
	\includegraphics{fc_time_l_4in1.pdf}
	\caption{Training RMSE results of models \textit{ae} (a), \textit{atom} (b), \textit{mol} (c), \textit{regu} (d) with 10~\AA\ cut rang over time.}
	\label{fgr:fcTimeS4in1}
\end{figure}
\begin{figure}[H]
	\includegraphics{fc_step_l_4in1.pdf}
	\caption{Training RMSE results of models \textit{ae} (a), \textit{atom} (b), \textit{mol} (c), \textit{regu} (d) with 10~\AA\ cut range over training steps.}
	\label{fgr:fcStepS4in1}
\end{figure}


\section{MeCN solution}
% distributions
\begin{figure}[H]
	\includegraphics{acn_dist_s_3in1.pdf}
	\caption{Error distribution in predicting of frequency shifts with models of \textit{ae} (a), \textit{atom} (b), \textit{mol} (c) with 6~\AA. The color of points correspond to the absolute error values.}
	\label{fgr:acnDistS3in1}
\end{figure}
\begin{figure}[H]
	\includegraphics{acn_dist_l_3in1.pdf}
	\caption{Training RMSE results of models \textit{ae} (a), \textit{atom} (b), \textit{mol} (c) with 10~\AA\ cut range over time.}
	\label{fgr:acnDistL3in1}
\end{figure}

% training curves
\begin{figure}[H]
	\includegraphics{acn_time_s_3in1.pdf}
	\caption{Training RMSE results of models \textit{ae} (a), \textit{atom} (b), \textit{mol} (c) with 6~\AA\ cut range over time.}
	\label{fgr:acnTimeS3in1}
\end{figure}
\begin{figure}[H]
	\includegraphics{acn_step_s_3in1.pdf}
	\caption{Training RMSE results of models \textit{ae} (a), \textit{atom} (b), \textit{mol} (c) with 6~\AA\ cut range over training steps.}
	\label{fgr:acnStepS3in1}
\end{figure}
\begin{figure}[H]
	\includegraphics{acn_time_l_3in1.pdf}
	\caption{Training RMSE results of models \textit{ae} (a), \textit{atom} (b), \textit{mol} (c) with 10~\AA\ cut rang over time.}
	\label{fgr:acnTimeL3in1}
\end{figure}
\begin{figure}[H]
	\includegraphics{acn_step_l_3in1.pdf}
	\caption{Training RMSE results of models \textit{ae} (a), \textit{atom} (b), \textit{mol} (c) with 10~\AA\ cut range over training steps.}
	\label{fgr:acnStepL4in1}
\end{figure}
%training data size test
\begin{figure}[H]
	\includegraphics{acn_time_size_3in1.pdf}
	\caption{Training RMSE results of model \textit{atom} with training data set sizes 1024 (a), 5120 (b), 10112 (c) with 6~\AA\ cut range over time.}
	\label{fgr:acnTimeSize3in1}
\end{figure}
\begin{figure}[H]
	\includegraphics{acn_step_size_3in1.pdf}
	\caption{Training RMSE results of model \textit{atom} with training data set sizes 1024 (a), 5120 (b), 10112 (c) with 6~\AA\ cut range over training steps.}
	\label{fgr:acnStepSize3in1}
\end{figure}
\section{Hyperparameters}
We used maximum affordable neuron networks in training. Most hyperparameters are set as default values. Main hyperparameters are: \textit{se\_e2\_a}, the descriptor type used by the smooth edition of Deep Potential. The full relative coordinates are used to construct the descriptor; 
Two hidden layers for the embedding net are with neurons ``[32, 64]''; Size of the submatrix of G (embedding matrix) is 16; Three hidden layers for the fitting net are with neurons ``[120, 120, 120]''; The start/stop learning rates are 0.01/3.51e-08, the rate decays every 5000 steps in training; The batch size for training is 4.


\end{document}
