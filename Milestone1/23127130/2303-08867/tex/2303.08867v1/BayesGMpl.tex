\documentclass[11pt]{article}
\usepackage{amssymb}
\usepackage{graphicx}
\def \uX {\underline{X}}
\newcommand{\E}[1]{\mathbb{E}\left[{#1}\right]}
\newcommand{\EQ}[1]{\mathbb{E}_Q\left[{#1}\right]}
\newcommand{\EP}[1]{\mathbb{E}_P\left[{#1}\right]}
\newcommand{\Eb}[1]{\mathbb{E}_\beta\left[{#1}\right]}
\newcommand{\V}[1]{\mathbb{V}\left[{#1}\right]}
\newcommand{\Vb}[1]{\mathbb{V}_\beta\left[{#1}\right]}
\newcommand{\VQ}[1]{\mathbb{V}_Q\left[{#1}\right]}

\begin{document}

\title{A Bayesian derivation of the square root law of market impact}
\author{%
  Louis Saddier \\
  \small{\'Ecole Normale Sup\'erieure Paris-Saclay, France} \\
  and \\
  Matteo~Marsili\thanks{marsili@ictp.it} \\
    \small{The Abdus Salam International Centre for Theoretical Physics,
   Trieste, Italy}}

\date{}
\maketitle

\begin{abstract}
We show that the main empirical findings on the impact of a sequence of transactions on prices of financial assets can be recovered within a surprisingly simple model with fully Bayesian market expectations. This includes {\em i)} the square-root behavior of the expected price variation with the total volume traded, {\em ii)}  it's crossover to a linear regime for small volumes, and {\em iii)} the impact decay back to equilibrium, after the sequence of trades is over. The simplicity of this derivation lends further support to its robustness and universality, and it sheds light on its origin. In particular, it suggests that the square-root impact law originates from the over-estimation of order flows originating from meta-orders. 
\end{abstract}

The square-root impact law (SRIL) is a well established empirical observation in high-frequency financial markets~\cite{prx}. It states that a sequence of trades -- either all buy or sell orders -- of total volume $Q$ changes, on average, the price of the traded asset by an amount proportional to 
\begin{equation}
\label{empSRIL}
\E{\Delta p_t}\simeq  \pm C\,\sigma \sqrt{\frac{Q}{V}}+\ldots
\end{equation}
where $\sigma$ and $V$ are the volatility and the total market volume measured on the same time-scale (e.g. a day), $C$ is a constant of order one, and the upper (lower) sign holds for a sequence of buy (sell) orders. Bucci {\em et al.}~\cite{bucci2019crossover} have recently found that the SRIL crosses over to a linear regime for very small $Q$. For long times, when the sequence of orders is over, the price ``decays" back towards the original price. 

%Such sequence of trades is called a meta-order.
The SRIL enjoys a remarkable universality, because it has been found to hold independently of details, such as the type of asset traded, the market mechanism and the way the sequence of trades is executed (see e.g.~\cite{prx,moro2009market,donier2015million,toth2016square}). %Such universality suggests that it should be possible to reproduce the SRIL within stylised models.
Several attempts to explain the SRIL within mechanistic models have been proposed (see e.g~\cite{gabaix2003theory,prx,barato2013impact,farmer2013efficiency,donier2015fully}). The most successful of these relates the square-root behaviour to the structure of the ``latent order book'', which encodes the propensities of traders to trade at prices close to the market price. Toth {\em et al.}~\cite{prx} and Donier {\em et al.}~\cite{donier2015fully} show that the SRIL holds provided that the density of these ``latent" orders behaves linearly close to the market price. This suggests that the SRIL arises from the expectations of market participants. 

Here we take this perspective within a simplified version of the Glosten-Milgrom model~\cite{GM} (GMM). This describes a market with a population of ``noise'' traders and one ``informed'' trader. Noise traders submit random orders, either to buy or to sell. The informed trader knows the worth of the asset and hence she either always sells, if the asset is worthless, or buys if the asset is underpriced. As a matter of fact, the informed trader behaves in a deterministic manner, so she trades exactly as a market participant who is executing a {\em meta-order} -- i.e. a sequence of buy or sell orders -- of total volume $Q$. In the GMM, the price is fixed by a market maker, based on his expectation of what the value of the asset is, given the past sequence of trades. More generally, the market maker reproduces a price mechanism which depends on the collective expectations of market participants. 
The key parameter in the GMM is the frequency $\nu$ with which the informed trader submits orders. The relevant range for this parameter is when $\nu$ is very small. Indeed, in realistic high frequency markets there are $\sim 10^4$ transactions a day for a reasonably liquid stock, few of which may be ascribed to a specific meta-order. The relevant time scales are those of the execution of a meta-order of size $Q$, hence $t\simeq Q/\nu\sim 1/\nu$. 

We show that, when $\nu\to 0$ and $t=Q/\nu\to\infty$, within a fully Bayesian approach where $\nu$ is unknown and is inferred from the sequence of trades, the empirically observed SRIL Eq.~(\ref{empSRIL}) is recovered, with $C$ which depends on the analytic behavior of the prior on $\nu$ for $\nu\to 0$, and $\sigma,V$ computed on the characteristic time $1/\nu$ between the execution of two orders in the meta-order. For an asymptotically uniform prior, the market impact takes the simple form
\begin{equation}
\label{ }
\E{\Delta p_t}\simeq \pm\frac{1}{2}{\rm erf}\!\left(\nu\sqrt{t}/{2}\right)\simeq \pm\frac{1}{2\sqrt{\pi}}\nu\sqrt{t}+\ldots
\end{equation}
which leads to $C=\sqrt{\frac{3}{\pi}}\simeq 0.9772\ldots$

The law of impact decay is recovered setting $\nu=0$ after the execution of the sequence of trades. This reveals that $\E{\Delta p_t}\sim 1/\sqrt{t}$, i.e. that the price asymptotically reverts back to its original value without leaving any permanent market impact. Finally, we show that when the prior assumes an upper bound $\bar\nu$ on the possible values of $\nu$, the market impact for small orders crosses over to a linear behavior $\E{\Delta p_t}\simeq \pm \frac{\sigma}{\sqrt{V}}Q$, as observed by Bucci {\em et al.}~\cite{bucci2019crossover}.


\section{The Glosten-Milgrom model}

The Glosten-Milgrom model (GMM) describes a financial market with only one asset. The game proceeds in discrete steps. At each step $t=1,2,\ldots$ one trader is drawn and she submits an order to buy ($x_t=1$) or sell ($x_t=0$) one share of the asset. 
There are two type of traders: {\em i)} Noise traders who trade randomly, and {\em ii)} informed traders who know the real value $Y$ of the asset. 
The asset is worth $Y=1$ with a probability $p$ and $Y=0$ with probability $1-p$. As long as the market price is in the interval $(0,1)$, informed traders will always buy the asset if $Y=1$ and always sell it if $Y=0$. The proportion of informed trader is $\nu$. 

The time series $x_{\le t}=(x_1,\ldots,x_t)$ of trades %($x_t=1$ for buy and $x_t=0$ for sell) 
is a random variable whose distribution depends on $Y$ and on the  value of $\nu$, 
\begin{equation}
\label{pxYnu}
P\{x_{\le t}|Y,\nu\}=\left(\frac{1\pm\nu}{2}\right)^{n_t}\left(\frac{1\mp\nu}{2}\right)^{t-n_t},\qquad n_t=\sum_{\tau=1}^t x_\tau
\end{equation}
where, here and in the sequel, the upper (lower) signs apply when $Y=1$ ($Y=0$) and we assumed that noise traders buy or sell with probability $1/2$.

The market maker does not know $Y$. At time $t$, he fixes the (ask) price $a_t$ at which he will sell and the (bid) price $b_t$ at which he will buy, based on his expectation of the value of $Y$, given the observed sequence $x_{\le t-1}=(x_1,\ldots,x_{t-1})$. He will do this by imposing that his expected gain in both cases is zero\footnote{Because otherwise traders will be attracted by other market makers if his expected gain is positive, or he will be driven out of the market if it is negative.}. Therefore he will set the ask price to $a_{t} = \E{Y|x_{\le t-1},x_{t}=1}$, in anticipation of a buy order ($x_t=1$). Likewise, the bid price will be set to $b_{t} = \E{Y|x_{\le t-1},x_{t}=0}$.
The realised price at time $t$, given $x_{\le t}$, is given by
\begin{eqnarray}
p_t(\nu) & = & \E{Y|x_{\le t},\nu}=P\{Y=1|x_{\le t},\nu\} \nonumber\\
 & = & \frac{P\{x_{\le t}|Y=1,\nu\}p}{P\{x_{\le t}|Y=1,\nu\}p+P\{x_{\le t}|Y=0,\nu\}(1-p)}\nonumber\\
 & = & \frac{p(1+\nu)^{n_t}(1-\nu)^{t-n_t}}{p(1+\nu)^{n_t}(1-\nu)^{t-n_t}+(1-p)(1+\nu)^{t-n_t}(1-\nu)^{n_t}}  \label{ptnu} %\\
% & = & \left[1+\frac{1-p}{p}\left(\frac{1+\nu}{1-\nu}\right)^{t-2n_t}\right]^{-1},%\qquad n_t=\sum_{\tau=1}^tx_\tau
\end{eqnarray}
where Eq.~(\ref{ptnu}) is obtained applying Bayes rule. This is the starting point of our analysis.

\subsection{Alternative interpretation of the GMM and meta-orders}

In real financial markets, some traders may want to buy or sell big amount of the same asset. Since the available liquidity at any time is much smaller than $Q$, they have to split their big order into small ``child" orders. The whole sequence of trades -- the meta-order -- is defined by a direction (buy or sell), a volume $Q$ and an horizon time $T$ to execute it. An informed trader in the GMM is, to all practical purposes, identical to a trader who executes a buy ($Y=1$) or sell ($Y=0$) meta-order. Therefore the market maker should behave setting exactly the same prices in both cases. 
The parameter $\nu$, in the case of the execution of meta-orders, becomes the frequency with which child orders are submitted. As already mentioned, we shall investigate the scaling limit $\nu\to 0$, $t\to\infty$ with $\nu t$ finite, which is the relevant one for modern financial markets where trading occurs at the millisecond time-scales. At such high frequencies, it makes sense to specialise to the case where noise traders buy or sell with the same probability and to assume that, {\em a priori}, meta-orders are equally likely to be in either direction, i.e. $p=1/2$. 

It is easy to see that $\log(1/p_t-1)$, performs a biased random walk, because the number $n_t$ of buy orders up to $t$ is a binomial random variable, with mean $\frac{1\pm\nu}{2} t$. We focus on the regime where $|\Delta p_t|=|p_t-p_0|\ll 1$, with $p_0=1/2$ being the initial price. Using $\log(1/p_t-1)\simeq -4\Delta p_t+\ldots$ we find
\begin{equation}
\label{ }
\E{\Delta p_t|\nu}\simeq \E{n_t-t/2}\nu+\ldots =\pm\frac{\nu^2}{2}t+\ldots
\end{equation}
A meta-order of size $Q=\nu t$ then has a linear impact, i.e. $\E{\Delta p_t|\nu}\simeq \pm\frac \nu 2 Q$. Notice that the drift term $\pm\frac{\nu^2}{2}t$ in the price is negligible with respect to the diffusion term (the volatility) $\sigma=\sqrt{\E{(p_t-\E{p_t})^2}}\simeq \nu \sqrt{t}$ as long as $t\ll \nu^{-2}$. The drift term becomes of the same order as the volatility on time-scales of order $t\sim \nu^{-2}$ when the value of $Y$, i.e. the direction of the meta-order, becomes statistically detectable. 
The scaling regime of interest $t\sim \nu^{-1}$ lies in a region where the contribution from the meta-order is statistically undetectable.

\section{A Bayesian market maker}

The analysis of the previous Section and the resulting linear impact behaviour assumes that the trading frequency $\nu$ of the meta-order is known to the market maker. 
We consider the case where the market maker does not know the true value of $\nu$. He only observes the stream of transactions $x_{\le t}$ and based on this he updates the distribution of $\nu$, using Bayes rule
\begin{eqnarray}
p(v|x_{\le t}) & = & \frac{P(x_{\le t}|v)\phi(v)}{\int_0^1 P(x_{\le t}|y)\phi(y)dy} \\
\nonumber
 & = & \frac{[(1+v)^{n_t}(1-v)^{t-n_t}+(1+v)^{t-n_t}(1-v)^{n_t}]\phi(v)}{\int_0^1(1+y)^{n_t}(1-y)^{t-n_t}\phi(y)dy+\int_0^1(1+y)^{t-n_t}(1-y)^{n_t}\phi(y)dy}
\end{eqnarray}
where $\phi(v)$ is the prior. Here and henceforth, $v$ denotes the inferred value of the  trading frequency of the meta-order (which is a random variable), whereas $\nu$ denotes the true value.

Using this, the market maker estimates the price $p_t$ as a function of $x_{\le t}$ (actually of $n_t$)
\begin{equation}
\label{ptnt}
p_t(n_t)=\E{Y|x_{\le t}}=\int_0^1\!dv \E{Y|x_{\le t},v}P(v|x_{\le t})
\end{equation}
where $\E{Y|x_{\le t},v}$ is given by Eq.~(\ref{ptnu}) with $\nu$ replaced by $v$. We remark that the expectation in Eq.~(\ref{ptnt}) as well as the probability $P(v|x_{\le t})$ refer to the subjective state of knowledge of the market maker, not to objective probability distributions. 

After some algebraic manipulations, this expression can be recast in the form
\begin{equation}
\label{eq:ptint}
p_t=\frac{\int_{0}^1\!dv\phi(v)e^{-tD(z||v)}}{\int_{-1}^1\!dv\phi(|v|)e^{-tD(z||v)}}
\end{equation}
where $z=(2n_t-t)/t$ and 
\begin{equation}
\label{KLD}
D(z||x) = \frac{1+z}{2}\log\frac{1+z}{1+x}+\frac{1-z}{2}\log\frac{1-z}{1-x} 
\end{equation}
is a Kullback-Leibler divergence.

We expect $z=\xi/\sqrt{t}$ to be small, where $\xi$ is a Gaussian random variable with mean $\E{\xi}=\pm\nu\sqrt{t}$, which is small in the scaling regime $1\ll t\ll \nu^{-2}$, and variance $\V{\xi}=1-\nu^2$. Since the integral on $v$ is dominated by values $z\sim 1/\sqrt{t}$, we set $x=\gamma/\sqrt{t}$. For the time being, let us assume that $\phi(v)$ has a finite limit as $v\to 0$. We shall return later on the role of the prior. To leading order
\begin{equation}
D(z||x) \simeq \frac{(x-z)^2}{2(1-z^2)}+O((x-z)^3)\simeq \frac{(\gamma-\xi)^2}{2t}+O(1/t^2)
\end{equation}
hence
\begin{equation}
p_t(\xi) \simeq  \frac{1}{\sqrt{2\pi}}\int_0^\infty\!d\gamma  e^{-\frac{1}{2}(\xi-\gamma)^2}=\frac 1 2 +\frac 1 2 {\rm erf}\!\left(\xi/\sqrt{2}\right)\,.
 \label{ptxi}
\end{equation}
Taking the expectation\footnote{The expectation here is over the objective probability distribution of $\xi$ (or $n_t$), which depends on the true parameter $\nu$.} over $\xi$, we find\footnote{In the general case $p\ne\frac{1}{2}$, Eq.~(\ref{ptxi}) becomes $p_t(\xi)\simeq\left[1+\frac{1-p}{p}\frac{1-{\rm erf}\!\left(\xi/\sqrt{2}\right)}{1+{\rm erf}\!\left(\xi/\sqrt{2}\right)}\right]^{-1}$ which can be expressed as a series in odd powers of $\xi$. Considering the leading behavior $\E{\xi^{2n+1}}\simeq \frac{2^{n+1}}{\sqrt{\pi}}\Gamma(n+3/2)\nu\sqrt{t}+O(\nu \sqrt{t})^3$ of the expected values of odd powers of $\xi$, we conclude that the SRIL holds also in this case.}
\begin{equation}
\label{Epterf}
\E{p_t}\simeq \frac 1 2 \pm\frac{1}{2}{\rm erf}\!\left(\nu\sqrt{t}/{2}\right)\simeq \frac 1 2 \pm\frac{1}{2\sqrt{\pi}}\nu\sqrt{t}+O(\nu t^{1/2})^3
\end{equation}
where, again, the $+$ sign holds for $Y=1$ and the $-$ sign for $Y=0$. Notice that, within this approximation,  $\E{p_t}\to Y$ 
as $t\to\infty$.

From Eq.~(\ref{ptxi}) one can also compute the variance of the price
\begin{equation}
\label{sigma}
\sigma=\sqrt{\V{p_t}} = \frac{1}{2\sqrt{3}} - \frac{\sqrt{3}-1}{4\sqrt{3}\pi}\nu^2t+O(\nu t^{1/2})^3
%\frac{1}{\sqrt{2\sqrt{3}}}-\frac{\sqrt{2 \sqrt{3}-3} }{4\pi }\nu^2 t+
   %\frac{\left(81-54\sqrt{3}+\left(6 \sqrt{3}-5\right) \pi\right) \nu^4 t^2}{48\sqrt{2} 3^{3/4} \pi^2}+O\left(\nu^6t^3\right)
%   O\left(\nu^4t^2\right)
\end{equation}
so that, for a meta-order of size $Q=\nu t$, Eq.~(\ref{Epterf}) can be cast in the form
\begin{equation}
\label{ }
\E{p_t}\simeq  \frac 1 2 \pm\sqrt{\frac{3}{\pi}}\sigma \sqrt{\frac{Q}{V}}+\ldots
\end{equation}
where $V=1/\nu$ is the market volume of transactions between one child order and the next one, and the coefficient $\sqrt{\frac{3}{\pi}}\simeq 0.9772\ldots$ is of the same order of those observed empirically. The true value of $\nu$ becomes statistically detectable when $t\sim \nu^{-2}$, which is also the time when the value of $Y$ is revealed. On these time scales, Eq.~(\ref{Epterf}) predicts that the impact saturates to the true value of $Y$.

\begin{figure}[!h]
\centering
\includegraphics[width=1.2\textwidth,trim = 4cm 0cm 0cm 0cm]
{market_impact_2.pdf}
\caption{\label{fig1} Price impact for a buy metaorder ($Y=1$). On the left, $\nu=0.035$ is fixed and on the right, $Q=10$ is fixed. We recover the SRIL for both the volume $Q$ traded and the speed of trade $\nu$. The dots come from Monte-Carlo numerical simulations of the GMM and dashed lines are the theoretical results from Eq.~(\ref{Epterf}) and Eq.~(\ref{sigma}). For $Q\sim1/\nu$ or $\nu\sim1/Q$, the true value of $\nu$ becomes statistically detectable and the SRIL no longer remains.}
\label{}
\end{figure}

Some insights on the origin of the SRIL can be gained by a simpler argument that assumes that the market maker relies on a Bayesian estimate $\nu$, i.e.
\begin{equation}
\label{ }
\nu = \hat\nu_{\rm Bayes}=\E{\nu|x_{\le t}}\,.
%\int_0^1\1d\nu \nu P(\nu|x_{\le t})\,.
\end{equation}
The posterior distribution of $\nu$ given $x_{\le t}$ has a width of order $1/\sqrt{t}$ and is centred around $\nu$. As long as $\nu\ll 1/\sqrt{t}$, i.e. for $t\ll \nu^{-2}$, this leads to (see Appendix)
\begin{equation}
\label{hatnubayes}
\hat\nu_{\rm Bayes}\simeq \sqrt{\frac{2}{\pi t}}
\end{equation}
%maximum likelihood estimate $\hat\nu$ of $\nu$. A simple calculation shows that 
%\begin{equation}
%\label{ }
%\hat \nu \simeq \max\left(\frac{1}{\sqrt{t}},\frac{2n_t-t}{t}\right)
%\end{equation}
%so that for $t\ll \nu^{-2}$, $\hat\nu\simeq 1/\sqrt{t}$. 
Substituting the leading behavior $\hat\nu_{\rm Bayes}\simeq \sqrt{\frac{2}{\pi t}}$ in the linear impact results 
\begin{equation}
\label{impactnubayes}
\E{\Delta p_t|\nu}\simeq \pm\frac{\hat\nu_{\rm Bayes}}{2}Q=\pm \frac{1}{\sqrt{2\pi}} \nu\sqrt{t}
\end{equation}
allows one to recover the behavior\footnote{The discrepancy by a factor $\sqrt{2}$ between Eqs.~(\ref{impactnubayes}) and (\ref{Epterf}) stems from the fact that the approach based on the Bayes estimate $\hat\nu_{\rm Bayes}$ disregards non-linear effects induced by the fluctuations of $\nu$. Indeed Eq.~(\ref{impactnubayes}) leads to the same result that one would obtain considering only the linear term in the expansion of $p_t(\xi)$ in powers of $\xi$ (Eq.~\ref{ptxi}).} of Eq.~(\ref{Epterf}). Eq.~(\ref{hatnubayes}) holds as long as $\hat\nu_{\rm Bayes}\gg\nu$, i.e. for $t\ll\nu^{-2}$. When $t\sim \nu^{-2}$ the true value $\hat\nu_{\rm Bayes}\simeq\nu$ is discovered\footnote{A maximum likelihood approach would lead to a different result. In the Appendix we show that the maximum likelihood estimate satisfies the self-consistent equation
\begin{equation}
\label{ }
\hat\nu_{\rm MLE}=\frac{\xi}{\sqrt{t}}\tanh\left[\frac{\sqrt{t}\xi}{2}\log\left(\frac{1+\hat\nu_{\rm MLE}}{1-\hat\nu_{\rm MLE}}\right)\right]
\end{equation}
which has a solution $\hat\nu_{\rm MLE}=0$ for $|\xi|\le 1$ and $\hat\nu_{\rm MLE}\sim \xi/\sqrt{t}$ for $|\xi|>1$.}. This shows that the SRIL originates from the overestimation of meta-order flow for times $t\ll \nu^{-2}$. 

\subsection{The role of the prior and the short time linear impact}

Eq.~(\ref{Epterf}) assumes a prior $\phi(\nu)$ that extends to finite values of $\nu$, e.g. a uniform prior for $\nu\in [0,1]$. This is unrealistic. Rather it makes sense to assume that the market anticipates that the contribution of meta-orders to the order flow is small, i.e. that $\nu\ll \bar\nu\ll 1$. One way to incorporate this observation is to take $\phi(y)=1/\bar \nu$ for $y\in[0,\bar \nu]$ and $\phi(y)=0$ otherwise. With this assumption, Eq.~(\ref{ptxi}) becomes
\begin{equation}
\label{ptnubar}
p_t(\xi)\simeq\frac{{\rm erf}\!\left(\frac{\xi}{\sqrt{2}}\right)+{\rm erf}\!\left(\frac{\bar\nu\sqrt{t}-\xi}{\sqrt{2}}\right)}{{\rm erf}\!\left(\frac{\bar\nu\sqrt{t}+\xi}{\sqrt{2}}\right)+{\rm erf}\!\left(\frac{\bar\nu\sqrt{t}-\xi}{\sqrt{2}}\right)}
\end{equation}
For $\bar\nu\sqrt{t}\gg 1$ (i.e. $t\gg\bar\nu^{-2}$), we recover Eq.~(\ref{ptxi}) whereas for $t\ll\bar\nu^{-2}$ the expansion of the expression above for $\bar\nu\sqrt{t}\ll 1$ leads to $p_t(\xi)\simeq \frac 1 2 +\frac 1 4 \xi\bar\nu\sqrt{t}+O(\bar\nu^2 t)$. Taking the expectation on $\xi$ we recover the linear regime found in Ref.~\cite{bucci2019crossover}
\begin{equation}
\label{linear}
\E{p_t}\simeq \frac 1 2 \pm \frac 1 4 \bar\nu\nu t+\ldots\simeq  \frac 1 2 \pm \frac 1 4 \bar\nu Q +\ldots
\end{equation}
In other words, for orders that last less than $\bar\nu^{-2}$ we find a linear impact, for orders that last a time $\bar\nu^{-2}<t<\nu^{-2}$ the SRIL sets in and for orders that are much longer than $\nu^{-2}$ the impact saturates to a finite value.

\begin{figure}[!h]
\centering
\includegraphics[width=.85\textwidth]
{exact_impact_short_time.pdf}
\caption{Market impact in the GMM by choosing a cutoff $\bar\nu$ in the prior for a buy metaorder ($Y=1$). Blue curve is derived from Monte-Carlo numerical simulations. Dashed line represents the theoretical linear impact we get from Eq.~(\ref{linear}) and dotted line represents the SRIL. The transition between the two regimes occurs at $t\sim1/\bar\nu^2\sim400$ in this case.}
\label{}
\end{figure}

The cutoff on the prior has also a consequence on the calculation of $\sigma$ in Eq.~(\ref{empSRIL}), which is the volatility over a timescale $t=1/\nu$. If $\nu\ll \bar\nu^2$ we recover the previous result $\sigma^2\simeq 1/12$. If instead $\nu\gg \bar\nu^2$, Eq.~(\ref{ptnubar}) yields, to leading order, 
\begin{equation}
\label{ }
\sigma^2\simeq \frac{1}{16}\frac{\bar\nu^2}{\nu}+\ldots
\end{equation}
which is consistent with a diffusive behaviour $\sigma^2\propto t$. Yet, within this simplified model, on the relevant time-scales where the SRIL holds ($t\sim 1/\nu\gg \bar\nu^{-2}$), the volatility $\sigma^2\simeq 1/12$ is constant.

As for the dependence of the impact on the prior, it is clear from the discussion in previous sections that the results only depend on the behavior of $\phi(v)$ in the limit $v\to 0$. Taking $\phi(v)\sim A v^{k-1}$ for $v\ll 1$, it is possible to derive the analog of Eq.~(\ref{ptxi}), which can be expressed as a series in odd powers of $\xi$: 
\begin{eqnarray}
    p_t(\xi) &\simeq& \frac{\int_0^\infty\!d\gamma \gamma^{k-1} e^{-\frac{1}{2}(\xi-\gamma)^2}}{\int_{-\infty}^\infty\!d\gamma |\gamma|^{k-1} e^{-\frac{1}{2}(\xi-\gamma)^2}} \\
    &\simeq & \frac 1 2 \pm \frac{\xi}{\sqrt{2}} \frac{\Gamma\left(\frac{1+k}{2}\right)}{\Gamma\left(\frac{k}{2}\right)}\frac{{}_1F_1\left(1-\frac k 2 ,\frac 3 2 ,-\frac{\xi^2}{2}\right)}{{}_1F_1\left(\frac{1-k}{2} ,\frac 1 2 ,-\frac{\xi^2}{2}\right)} \label{eq:SRILprior}\\
    & \simeq & \frac 1 2 \pm 
    \frac{\Gamma\left(\frac{1+k}{2}\right)}{\sqrt{2}\Gamma\left(\frac{k}{2}\right)}\left[\xi+\frac{1-2k}{6}\xi^3+\frac{(3-4k)(1-4k)}{120}\xi^5 + O(\xi^7)\right].\nonumber
\end{eqnarray}
 Considering the leading behavior 
 %$\E{\xi^{2n+1}}\simeq \frac{4}{\sqrt{2\pi}}\Gamma(2n+1)\nu\sqrt{t}+O(\nu^2 t)$ 
 $\E{\xi^{2n+1}}\simeq \frac{2^{n+1}}{\sqrt{\pi}}\Gamma(n+3/2)\nu\sqrt{t}+O(\nu \sqrt{t})^3$ 
 of the expected values of odd powers of $\xi$, we conclude that the SRIL holds also in this case, with a coefficient that depends on $k$. Numerical simulations confirm this conclusion, as shown in Fig.~\ref{fig:SRILprior}.

For small values of $k$, Eq.~(\ref{eq:SRILprior}) reads
\begin{eqnarray}
    p_t(\xi) &\simeq& \frac{1}{2} \pm k\frac{\pi}{4}\mathrm{erfi}\left(\frac{\xi}{\sqrt{2}}\right)+O(k^2) \\
    &\simeq& \frac{1}{2} \pm k\frac{\sqrt{\pi}}{2\sqrt{2}}\left[\xi+\frac{\xi^3}{6}+\frac{\xi^5}{40}+O(\xi^7)\right] \nonumber
\end{eqnarray}
where $\mathrm{erfi}(x)=\frac{2}{\sqrt{\pi}}\int_0^x e^{z^2} dz$ is the imaginary error function. In the limit of small $k$, the market impact varies linearly with $k$ and vanishes for $k\to0$. In fact for $k=0$, the prior behaves as a delta-function at $\nu=0$, that corresponds to a market maker who believes that there is no informed trader (or meta-order). Yet, at time $t\gg\nu^{-2}$ the true value of $\nu$ is revealed irrespective of what the prior is. Therefore, the behavior $\E{\Delta p_t}\propto k\nu\sqrt{t}$ at $t\ll\nu^{-2}$ should leave way to a faster growth of the market impact, as shown in Fig.~\ref{fig:SRILprior}, in the $t\sim\nu^{-2}$ regime.

\begin{figure}[!h]
\centering
\includegraphics[width=.85\textwidth]
{prior_price_impact.pdf}
\caption{Numerical simulations ($\nu=0.1$ and $Y=1$) of price impact for priors of type $\phi(v)\sim A v^{k-1}$. The case $k=1$ is the same as described in Eq.~(\ref{Epterf}). One can remark that other curves have the same slope as this one for $t\sim1/\nu$, emphasising the fact that they also follow the SRIL.}
\label{fig:SRILprior}
\end{figure}

\subsection{Impact decay}

Consider a meta-order which is active until time $Q/\nu$. The statics of $\xi$ is easily calculated taking into account that 
\begin{equation}
\label{ }
\E{n_t}=\sum_{\tau=1}^{Q/\nu}\E{x_\tau}+\sum_{\tau=Q/\nu+1}^t\E{x_\tau}=\frac t 2 \pm \frac Q 2
\end{equation}
because $\E{x_\tau}=\frac 1 2 \pm \frac \nu 2$ in the first sum and $\E{x_\tau}=\frac 1 2$ in the second. Similarly we compute the variance $\V{n_t}=\frac t 4 -\frac{\nu Q}{4}$. Hence $\xi$ is a Gaussian variable with mean $\E{\xi}=\pm \frac{Q}{\sqrt{t}}$ and variance $\V{\xi}=1-\nu Q/t$. Using this in Eq.~(\ref{ptxi}) one gets 
\begin{equation}
\label{ }
\E{p_t}\simeq\frac 1 2 \pm \frac 1 2 {\rm erf}\!\left(\frac{Q}{\sqrt{4 t-2\nu Q}}\right)\simeq \frac 1 2 \pm \frac 1 2 {\rm erf}\!\left(\frac{Q}{2\sqrt{t}}\right)
\end{equation}
Hence the theory predicts a slow decay $\E{\Delta p_t}\simeq \pm \frac{Q}{2\sqrt{\pi t}}$ of the impact to zero, and no permanent impact (see Fig.~\ref{fig:decay}).

\begin{figure}[!h]
\centering
\includegraphics[width=.85\textwidth]
{decay.pdf}
\caption{Price impact for a buy meta-order $(Y=1)$ as a function of time. The meta-order extinguishes at time $t=400$. The theoretical decay  $\sim Q/2\sqrt{\pi t}$ (dashed line) matches well with the Monte-Carlo simulation (blue full line).}
\label{fig:decay}
\end{figure}

\section{Discussion}

In summary, this paper shows that the square-root law of market impact can be derived as an exact result in the relevant scaling limit of a simple model of a market in which the price is formed through a fully Bayesian reasoning. The extreme simplicity of the model and the absence of any {\em ad-hoc} assumptions, makes it appropriate as a stylised description of a variety of market settings.  
Hence it provides further theoretical support for the empirically observed universality of this law. 
For example, the independence on the execution protocol is a consequence of the fact that the scaling regime of interest $t\sim \nu^{-1}$ lies in a region where the contribution from the meta-order to the order flow is statistically undetectable.

This model also allows to recover the crossover to a linear impact for small orders $Q< \frac{4}{\sqrt{\pi}}\frac{\nu}{\bar\nu^2}$ observed in \cite{bucci2019crossover} as well as the impact decay for times $t\gg Q/\nu$. This approach predicts that the price, on average, reverts back to the initial value. There is no clear evidence that this is true~\cite{bucci2018slow}, also because the activity of real markets is confined to a trading day, and there are theories (e.g. ~\cite{farmer2013efficiency}) that predict a permanent impact of the meta-order on prices.

Still, the derivation provided in this paper provides a transparent and simple picture of the market impact phenomenology. It's main shortcoming is the prediction of a constant volatility of prices (Eq.~(\ref{sigma}) and Fig.~\ref{fig1}), which is hard to reconcile with empirical data.
%Furthermore, the model predicts that the  bid-ask spread should decay, in expected terms, as $1/\sqrt{t}$ as the meta-order is executed. Such decrease encodes the price discovery process 


\section{Acknowledgments}
We are grateful to Jean-Philippe Bouchaud and Iacopo Mastromatteo for several discussions. 

\appendix

\section{Derivation of the estimation of $\nu$}

This Appendix details the derivation of the two estimators of $\nu$ discussed in the text. Their behavior is shown in Fig.~\ref{fig:estim}.
\begin{figure}[!h]
    \centering
    \includegraphics[width=1.3\textwidth, trim = 4.5cm 0cm 0cm 0cm]{fig_5_fus_2.pdf}
    \caption{On the left, plot of the Bayes estimator and the MLE times $\sqrt{t}$ as a function of $\xi=(2n_t-t)/\sqrt{t}$. On the right, plot of the Bayes estimator times $\sqrt{t}$ as a function of $\xi$ for priors of type $\phi(v)\sim A v^{k-1}$. In the inset, numerical simulations ($\nu=0.1$ and $Y=1$) of the expected value of the Bayes estimator as a function of time.}
    \label{fig:estim}
\end{figure}

\subsection{Bayesian estimator}

The Bayes estimator is
\begin{equation}
\label{byse}
\hat\nu_\mathrm{Bayes}=\mathbb{E}[\nu|x_{\le t}]=\int_0^1 v P(v|x_{\le t})\mathrm{d}v
\end{equation}
Using Bayes rule
\begin{equation}
\label{bysr}
\mathbb{P}(v|x_{\le t})=\frac{P(x_{\le t}|v)\phi(v)}{\int_0^1 P(x_{\le t}|v)\phi(v) \mathrm{d}v}
\end{equation}
where $\phi(v)$ is the prior and $P(x_{\le t}|v)$ is obtained from Eq.~(\ref{pxYnu}). 
%=p\left(\frac{1+\nu}{2}\right)^{n_t}\left(\frac{1-\nu}{2}\right)^{t-n_t}+(1-p)\left(\frac{1-\nu}{2}\right)^{n_t}\left(\frac{1+\nu}{2}\right)^{t-n_t}$. By taking $p=\frac{1}{2}$, and b
By injecting Eq.~(\ref{bysr}) into Eq.~(\ref{byse}), one can obtain
\begin{equation}
\label{}
\hat\nu_\mathrm{Bayes}=\frac{\int_{-1}^1\!dv\phi(|v|)|x|e^{-tD(z||v)}}{\int_{-1}^1\!dv\phi(|v|)e^{-tD(z||v)}}
\end{equation}
where $D(z||v)$ is the Kullback-Leibler divergence defined by Eq.~(\ref{KLD}). Again, we are interested in the variable $\xi=z\sqrt{t}$ where $z=(2n_t-t)/t$ and we set $v=\gamma/\sqrt{t}$. By also assuming $\phi(v)$ is finite as $v\to0$, we get to the first order in the scaling regime $1\ll t\ll \nu^{-2}$
\begin{eqnarray}
\hat\nu_\mathrm{Bayes} & \simeq&  \frac{1}{\sqrt{2\pi t}}\int_{-\infty}^\infty |\gamma| e^{-\frac{1}{2}(\xi-\gamma)^2}\mathrm{d}\gamma \\
        &=& \sqrt{\frac{2}{\pi t}}e^{-\frac{1}{2}\xi^2}+\frac{\xi}{\sqrt{t}}{\rm erf}\!\left(\xi/\sqrt{2}\right) \label{eq:nubayes} %\\
%        &=& \sqrt{\frac{2}{\pi t}}+\frac{1}{\sqrt{2\pi t}}\xi^2+o_{0}(\xi^2)\\
 %       &=& \frac{|\xi|}{\sqrt{t} }+o_{\pm\infty}(\xi)\sim z
\end{eqnarray}
which has the limiting behaviours $\hat\nu_\mathrm{Bayes}\simeq \frac{2+\xi^2}{\sqrt{2\pi t}}$ for $|\xi|\ll 1$ and $\hat\nu_\mathrm{Bayes}\simeq \frac{|\xi|}{\sqrt{t}}$ for $|\xi|\to\pm\infty$.

The expected value of $\hat\nu_\mathrm{Bayes}$ over the distribution of $\xi$ is
\begin{equation}
\label{}
\E{\hat\nu_\mathrm{Bayes}}=\frac{2}{\sqrt{\pi t}}e^{-\frac{-\nu^2 t}{4}}+\nu~\mathrm{erf}\left(\frac{\nu\sqrt{t}}{2}\right)
\end{equation}
Its limiting behaviour is $\E{\hat\nu_\mathrm{Bayes}}\sim \frac{2}{\sqrt{\pi t}}$ for $t\sim1/\nu$ and $\E{\hat\nu_\mathrm{Bayes}}\propto\nu$ for $t \sim 1/\nu^2$.

For a prior of type $\phi(v)\sim A v^{k-1}$, the same reasoning as above applies and one can show that
\begin{equation}
    \hat\nu_\mathrm{Bayes} \simeq \sqrt{\frac{2}{t}}\frac{\Gamma \left(\frac{k+1}{2}\right)}{\Gamma \left(\frac{k}{2}\right)}\frac{_1F_1\left(-\frac{k}{2},\frac{1}{2},-\frac{\xi^2}{2}\right)}{_1F_1\left(\frac{1-k}{2},\frac{1}{2},-\frac{\xi^2}{2}\right)}
\end{equation}
For $k=1$, one can recover Eq.~(\ref{eq:nubayes}). The behavior of $\hat\nu_\mathrm{Bayes}$ is plotted for several values of $k$ in Fig.~\ref{fig:estim}.


\subsection{Maximum Likelihood estimator (MLE)}
The maximum likelihood estimator is
\begin{equation}
\label{}
\nu_{\mathrm{MLE}}=\arg \max_v P(x_{\le t}|v)
\end{equation}
We have for $p=\frac{1}{2}$
\begin{eqnarray}
P(x_{\le t}|v) & \propto & (1+v^{n_t}(1-v)^{t-n_t}+(1+v)^{t-n_t}(1-v)^{n_t} \\
    &\propto& (1-v^2)^\frac{t}{2}\cosh\left[\frac{\xi\sqrt{t}}{2}\log\left(\frac{1+v}{1-v}\right)\right]
\end{eqnarray}
Hence we have
\begin{equation}
\label{}
\frac{\partial P(x_{\le t}|\nu)}{\partial \nu}=0 \Leftrightarrow \nu\sqrt{t}\cosh\left[\frac{\xi\sqrt{t}}{2}\log\left(\frac{1+\nu}{1-\nu}\right)\right]=\xi \sinh\left[\frac{\xi\sqrt{t}}{2}\log\left(\frac{1+\nu}{1-\nu}\right)\right]
\end{equation}
So that finally the MLE satisfies the self consistent equation
\begin{equation}
\label{}
\hat\nu_{\mathrm{MLE}}=\frac{\xi}{\sqrt{t}}\tanh\left[\frac{\xi\sqrt{t}}{2}\log\left(\frac{1+\hat\nu_{\mathrm{MLE}}}{1-\hat\nu_{\mathrm{MLE}}}\right)\right]
\end{equation}
For $\xi<1$, this equation has only zero as a solution. For $\xi>1$, the zero solution becomes unstable and $\hat\nu_{\mathrm{MLE}}\simeq \frac{|\xi|}{\sqrt{t}}$ for $|\xi|\gg \sqrt{t}$.

\section{Bid-ask spread}

The ask and the bid prices at time $t+1$ are given by
\begin{eqnarray}
    a_{t+1} &=& \int_0^1\!dv \E{Y|x_{\le t},x_{t+1}=1,v}P(v|x_{\le t},x_{t+1}=1)\label{askeq}\\
    b_{t+1} &=& \int_0^1\!dv \E{Y|x_{\le t},x_{t+1}=0,v}P(v|x_{\le t},x_{t+1}=0)\label{bideq}
\end{eqnarray}
With the same type of reasoning we did to get Eq.~(\ref{eq:ptint}), we can recast Eqs.~(\ref{askeq},\ref{bideq}) as
\begin{eqnarray}
    a_{t+1}(z) & = & \frac{\int_{0}^1\!dv\phi(v)(1+v)e^{-tD(z||v)}}{\int_{-1}^1\!dv\phi(|v|)(1+v)e^{-tD(z||v)}}\\
    b_{t+1}(z) & = & \frac{\int_{0}^1\!dv\phi(v)(1-v)e^{-tD(z||v)}}{\int_{-1}^1\!dv\phi(|v|)(1-v)e^{-tD(z||v)}}
\end{eqnarray}
where $z=(2n_t-t)/t$ and $D(z||x)$ is the Kullback-Leibler divergence defined by Eq.~(\ref{KLD}). Again, $z$ is typically of order $1/\sqrt{t}$, so we focus on the variable $\xi=z\sqrt{t}$ and make the change of variable $\gamma=v\sqrt{t}$. By also assuming $\phi(\nu)$ is finite as $\nu\to0$, we get to the first order in the scaling regime $1\ll t\ll \nu^{-2}$
\begin{equation}
    \frac{\int_{0}^1\!dv\phi(v)(1\pm v)e^{-tD(z||v)}}{\int_{-1}^1\!dv\phi(|v|)(1\pm v)e^{-tD(z||v)}}\simeq\frac{\int_0^\infty \left(1\pm\frac{\gamma}{\sqrt{t}}\right)e^{-\frac{1}{2}(\gamma-\xi)^2} d\gamma}{\sqrt{2\pi}(1\pm\xi/\sqrt{t})}
\end{equation}

Now, we can compute the bid-ask spread
\begin{eqnarray}
    a_{t+1}-b_{t+1} &\simeq& \sqrt{\frac{2}{\pi t}}\frac{1}{1-\frac{\xi^2}{t}}\int_0^\infty (\gamma-\xi) e^{-\frac{1}{2}(\gamma-\xi)^2} d\gamma \\
    &=& \sqrt{\frac{2}{\pi t}}e^{-\frac{\xi^2}{2}}+O(t^{-\frac{3}{2}})
\end{eqnarray}

Its expected value is
\begin{equation}
    \E{a_{t+1}-b_{t+1}} = \frac{1}{\sqrt{\pi t}}e^{-\frac{\nu^2t}{4}}=\frac{1}{\sqrt{\pi t}}+O(\nu^2t)
\end{equation}


\bibliographystyle{unsrt} 
\bibliography{BayesGM.bib}

\end{document}
