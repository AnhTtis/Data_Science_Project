\documentclass[11pt]{article}
%\begin{document}
\usepackage{geometry}                % See geometry.pdf to learn the layout options. There are lots.
\geometry{letterpaper}                   % ... or a4paper or a5paper or ... 
%\geometry{landscape}                % Activate for for rotated page geometry
%\usepackage[parfill]{parskip}    % Activate to begin paragraphs with an empty line rather than an indent
\usepackage{graphicx}
\usepackage{amssymb}
\usepackage{epstopdf}
\usepackage{color}
\usepackage{soul}
\usepackage[shortlabels]{enumitem}
\usepackage{multirow}
\usepackage{longtable}
\usepackage{subfigure}

%\title{}
%\author{}
%\date{}
%\maketitle
%
%
%\end{document}
%
%
%
%
%\documentclass[9pt,twocolumn,twoside,lineno]{pnas-new}
% Use the lineno option to display guide line numbers if required.

\newcommand{\E}[1]{\mathbb{E}\left[{#1}\right]}
\newcommand{\V}[1]{\mathbb{V}\left[{#1}\right]}

%\templatetype{pnasresearcharticle} % Choose template 
% {pnasresearcharticle} = Template for a two-column research article
% {pnasmathematics} %= Template for a one-column mathematics article
% {pnasinvited} %= Template for a PNAS invited submission
\begin{document}

\title{A Bayesian theory of market impact}

% Use letters for affiliations, numbers to show equal authorship (if applicable) and to indicate the corresponding author
\author{Louis Saddier\\
\small{\'Ecole Normale Sup\'erieure Paris-Saclay, 4 Avenue des Sciences, 91190 Gif-sur-Yvette, France}\\
and \\
Matteo~Marsili\thanks{marsili@ictp.it} \\
    \small{Quantitative Life Sciences Section}\\
    \small{The Abdus Salam International Centre for Theoretical Physics,
  34151 Trieste, Italy}}

%% Please give the surname of the lead author for the running footer
%\leadauthor{Saddier} 

% Please include corresponding author, author contribution and author declaration information
%\authorcontributions{Author contributions: M.M. designed research, L.S. and M.M. derived analytical results, L.S. performed numerical simulations, M.M. and L.S. wrote the paper.}
%\authordeclaration{The authors declare no competing interest.}
%\equalauthors{\textsuperscript{1}A.O.(Author One) contributed equally to this work with A.T. (Author Two) (remove if not applicable).}
%\correspondingauthor{\textsuperscript{1}To whom correspondence should be addressed. E-mail: marsili@ictp.it}

% At least three keywords are required at submission. Please provide three to five keywords, separated by the pipe symbol.
%\keywords{Market impact $|$ Glosten-Milgrom model $|$ Bayesian inferences} 
\maketitle

\begin{abstract}
The available liquidity at any time in financial markets falls largely short of the typical size of the orders that institutional investors would trade. In order to reduce the impact on prices due to the execution of large orders, traders in financial markets split large orders into a series of smaller ones, which are executed sequentially. The resulting sequence of trades is called a {\em meta-order}. Empirical studies have revealed a non-trivial set of statistical laws on how meta-orders affect prices, which include {\em i)} the square-root behaviour of the expected price variation with the total volume traded, {\em ii)}  its crossover to a linear regime for small volumes, and {\em iii)} a reversion of average prices towards equilibrium, after the sequence of trades is over. 
Here we recover this phenomenology within a minimal theoretical framework where the market sets prices by incorporating all information on the direction and speed of trade of the meta-order in a Bayesian manner. The simplicity of this derivation lends further support to its robustness and universality, and it sheds light on its origin. In particular, it suggests that the square-root impact law originates from the over-estimation of order flows originating from meta-orders. 
The theory also offers a number of falsifiable predictions that may be tested on empirical data.
%Please provide an abstract of no more than 250 words in a single paragraph. Abstracts should explain to the general reader the major contributions of the article. References in the abstract must be cited in full within the abstract itself and cited in the text.
\end{abstract}

%\dates{This manuscript was compiled on \today}

%\thispagestyle{firststyle}
%\ifthenelse{\boolean{shortarticle}}{\ifthenelse{\boolean{singlecolumn}}{\abscontentformatted}{\abscontent}}{}

{D}ata availability makes it possible to probe the laws that govern how financial markets process information to unprecedented precision~\cite{bouchaud2018trades}. Market impact laws are probably the best established empirical observation in high-frequency financial markets~\cite{bouchaud2018trades,prx}. They describes how a {\em meta-order}, which is a sequence of either all buy or sell orders, affect prices on average. Because of their statistical nature, these laws enjoy a remarkable level of universality, i.e. of independence on details and on context. %Universality suggests that these laws originate from simple principles, which are best unveiled by the most parsimonious theory that accounts for the observed phenomenology. 
The aim of this paper is to present a simple, parameter free theory that, by accounting for the observed phenomenology, can shed light on its origin and deliver falsifiable predictions that make it possible to dig deeper into it.

Within market impact phenomenology, the square-root impact law (SRIL) is probably the best established empirical observation in high-frequency financial markets~\cite{bouchaud2018trades,prx}. The SRIL states that a meta-order of total volume $Q$ changes, on average, the price of the traded asset by an amount proportional to 
\begin{equation}
\label{empSRIL}
\E{\Delta p_t}\simeq  \pm C\,\sigma \sqrt{\frac{Q}{V}}+\ldots
\end{equation}
where $\sigma$ and $V$ are the volatility and the volume of transactions measured on the same time-scale (e.g. a day), $C$ is a constant of order one, and the upper (lower) sign holds for a sequence of buy (sell) orders. 

The SRIL enjoys a remarkable universality, because it has been found to hold independently of details, such as the type of asset traded, the market mechanism and the way the sequence of trades is executed (see e.g.~\cite{prx,moro2009market,donier2015million,toth2016square}).
%Such universality suggests that it should be possible to reproduce the SRIL within stylised models.
Several attempts to explain the SRIL within mechanistic models have been proposed~\cite{gabaix2003theory,prx,barato2013impact,farmer2013efficiency,donier2015fully}. Some derive the SRIL from other empirical laws~\cite{gabaix2003theory,farmer2013efficiency} which however appear to have a lower degree of universality. Toth {\em et al.}~\cite{prx} relate the square-root behaviour to the structure of the ``latent order book'', which encodes the propensities of traders to trade at prices close to the market price. Toth {\em et al.}~\cite{prx} and Donier {\em et al.}~\cite{donier2015fully} show that the SRIL holds provided that the density of ``latent'' orders behaves linearly close to the market price. 

Bucci {\em et al.}~\cite{bucci2019crossover} have recently found that the SRIL crosses over to a linear regime for very small $Q$. They relate the crossover to different trading time-scales in the latent order book model. For long times, when the sequence of orders is over, the price reverts back to its original value. The law of impact decay, e.g. whether a meta-order leaves a permanent impact or not~\cite{farmer2013efficiency,brokmann2015slow,bucci2018slow}, appear to have a lower degree of universality, also due to the scarcity of data for long time-scales.

%This suggests that the SRIL arises from the expectations of market participants. Yet this theory relates the SRIL to a theoretical construct which is not directly measurable. 

This paper aims at deriving this complex phenomenology within a simple theoretical framework based on the Glosten-Milgrom model~\cite{GM} (GMM). The GMM, together with the Kyle model~\cite{kyle1985continuous} and their variants~\cite{back2004information}, is a cornerstone in the literature on market microstructure. This literature investigates how market dynamics emerges from market rules and the strategic behaviour of traders of different types. Here we abstract from strategic considerations because, as we shall see, the execution of a meta-order is equivalent to the behaviour of an informed agent in the GMM. This allows us to adopt the GMM~\cite{GM} as a stylised description of how a market responds to the statistically persistent order flow generated by a meta-order. 

The GMM describes a single-asset market with a population of traders, some of which are informed about the true value of the asset so that they either always sell, if the asset is worthless, or buy if the asset is underpriced.
%The uninformed traders submit random orders, either to buy or to sell, and will be called noise traders in what follows. 
%Informed traders know the worth of the asset and hence they either always sell, if the asset is worthless, or buy if the asset is underpriced. 
As a matter of fact, informed traders behave in a deterministic manner, so they trade exactly as a market participant who is executing a {meta-order}. In the GMM, the price is fixed by a market maker, based on his expectation of what the value of the asset is, given the past sequence of trades. %This mechanism encodes a price formation rule that captures market expectations on future order flows.
The key parameter in the GMM is the fraction $\nu$ of informed traders, which translates into the frequency with which child orders of the meta-order are executed. In high-frequency markets there are $\sim 10^4$ transactions a day for a reasonably liquid stock, few of which may be ascribed to a specific meta-order. This implies that market impact laws are related to the limit $\nu\to 0$ in the GMM. Accordingly, the relevant time scales are those of the execution of a meta-order of finite size $Q$, hence $t\simeq Q/\nu\sim 1/\nu\gg 1$. 

In this scaling limit, we show that the full market impact phenomenology described above is recovered within a fully Bayesian approach where $\nu$ is unknown and is inferred from the sequence of trades. This theory traces the origin of the crossover to a linear behaviour for small $Q$~\cite{bucci2019crossover} to a cutoff in the prior on $\nu$, and it reproduces impact decay. We finally discuss how Eq.~(\ref{empSRIL}) is recovered from a careful correspondence between the GMM and the empirical analysis leading to it. %It is important to appreciate that the empirical analysis leading to Eq.~(\ref{empSRIL}) relies on averaging market behaviour conditionally on the inception of a meta-order~\cite{prx}. Such an averaging procedure suppresses the diffusive behaviour of prices. Likewise, the GMM generates non-diffusive price behaviour 

%As we shall see, the problem is in a regime statistically undetectable so optimality is irrelevant.

We discuss the main steps of the derivation in the following sections and relegate technical details to appendices.




%We show that, when $\nu\to 0$ and $t=Q/\nu\to\infty$, within a fully Bayesian approach where $\nu$ is unknown and is inferred from the sequence of trades, the empirically observed SRIL Eq.~\ref{empSRIL} is recovered.
%with $C$ which depends on the analytic behaviour of the prior on $\nu$ for $\nu\to 0$, and $\sigma,V$ computed on the characteristic time $1/\nu$ between the execution of two orders in the meta-order. For an asymptotically uniform prior, the market impact takes the simple form
%\begin{equation}
%\label{eq:main_result}
%\E{\Delta p_t}\simeq \pm\frac{1}{2}{\rm erf}\!\left(\nu\sqrt{t}/{2}\right)\simeq \pm\frac{1}{2\sqrt{\pi}}\nu\sqrt{t}+\ldots
%\end{equation}
%which leads to $C=\sqrt{\frac{3}{\pi}}\simeq 0.9772\ldots$
%Also, an impact decay is observed setting $\nu=0$ after the execution of the sequence of trades. This reveals that $\E{\Delta p_t}\sim 1/\sqrt{t}$, i.e. that the price asymptotically reverts back to its original value without leaving any permanent market impact. We also show that when the prior assumes an upper bound $\bar\nu$ on the possible values of $\nu$, the market impact for small orders crosses over to a linear behaviour $\E{\Delta p_t}\simeq \pm \frac{\sigma}{\sqrt{V}}Q$, as observed by Bucci {\em et al.}~\cite{bucci2019crossover}. Finally, we propose to incorporate an exogenous component to the price in order to impose price diffusivity to the model.

%{T}his PNAS journal template is provided to help you write your work in the correct journal format. Instructions for use are provided below. 

%Note: please start your introduction without including the word ``Introduction'' as a section heading (except for math articles in the Physical Sciences section); this heading is implied in the first paragraphs. 

\section{Background: the Glosten-Milgrom model}

The Glosten-Milgrom model (GMM)~\cite{GM} describes a financial market with one asset, which is worth $Y=1$ with a probability $p$ and $Y=0$ with probability $1-p$. At each time $t=1,2,\ldots$ one trader is drawn at random from a population and she submits an order to buy ($x_t=1$) or sell ($x_t=0$) one share of the asset. A fraction $\nu$ of the traders is {\em informed}, i.e. they know the real value $Y$ of the asset. As long as the market price is in the interval $(0,1)$, informed traders will always buy the asset if $Y=1$ and always sell it if $Y=0$. The uninformed traders behave as {\em noise traders}, i.e. they buy with probability $1/2$ and sell otherwise. 
Hence, the time series $x_{\le t}=(x_1,\ldots,x_t)$ of trades %($x_t=1$ for buy and $x_t=0$ for sell) 
is a random variable whose distribution depends on the  value of $Y$ and $\nu$, 
\begin{equation}
P\{x_{\le t}|Y,\nu\}=\left(\frac{1\pm\nu}{2}\right)^{n_t}\left(\frac{1\mp\nu}{2}\right)^{t-n_t},~~~~ n_t=\sum_{\tau=1}^t x_\tau
\end{equation}
where, here and in the sequel, the upper (lower) signs apply when $Y=1$ ($Y=0$).

The market maker does not know $Y$. At time $t$, he fixes the (ask) price $a_t$ at which he will sell and the (bid) price $b_t$ at which he will buy, in a competitive manner\footnote{Traders will be attracted by other market makers if the market maker's expected gain is positive, or he will be driven out of the market if it is negative.} based on his expectation of the value of $Y$, given the observed sequence $x_{\le t-1}=(x_1,\ldots,x_{t-1})$. This means that he will set the ask price to $a_{t} = \E{Y|x_{\le t-1},x_{t}=1}$, in anticipation of a buy order ($x_t=1$), and the bid price to $b_{t} = \E{Y|x_{\le t-1},x_{t}=0}$. The realised price at time $t$, given $x_{\le t}$, is given by
%\begin{equation}
%    p_t(\nu) = \E{Y|x_{\le t},\nu}=P\{Y=1|x_{\le t},\nu\}
%    \label{ptnu}
%\end{equation}
\begin{eqnarray}
p_t(\nu) \!& = &\! \E{Y|x_{\le t},\nu}=P\{Y=1|x_{\le t},\nu\} \nonumber\\
 \!& = &\! \frac{P\{x_{\le t}|Y=1,\nu\}p}{P\{x_{\le t}|Y=1,\nu\}p+P\{x_{\le t}|Y=0,\nu\}(1-p)} \label{ptnu}%\nonumber\\
 %& = & \frac{p(1+\nu)^{n_t}(1-\nu)^{t-n_t}}{p(1+\nu)^{n_t}(1-\nu)^{t-n_t}+(1-p)(1+\nu)^{t-n_t}(1-\nu)^{n_t}}  \label{ptnu} %\\
% & = & \left[1+\frac{1-p}{p}\left(\frac{1+\nu}{1-\nu}\right)^{t-2n_t}\right]^{-1},%\qquad n_t=\sum_{\tau=1}^tx_\tau
\end{eqnarray}
where Eq.~(\ref{ptnu}) is obtained applying Bayes rule. %This is the starting point of our analysis.

As already mentioned, an informed trader in the GMM is, to all practical purposes, identical to a trader who executes a buy ($Y=1$) or sell ($Y=0$) meta-order, which is a sequence of buy (or sell) child orders. Hence, the market maker should behave setting prices precisely as in Eq.~(\ref{ptnu}). 
A meta-order is defined by a direction (buy or sell), a volume $Q$ and an horizon time $T$ to execute it. The parameter $\nu=Q/T$, in this interpretations, becomes the frequency with which child orders are submitted. We shall investigate the scaling limit $\nu\to 0$, $t\to\infty$ with $\nu t$ finite, which is the relevant one for modern financial markets where trading occurs at the millisecond time-scales. At such high frequencies, it makes sense to specialise to the case where noise traders buy or sell with the same probability and to assume that, {\em a priori}, meta-orders are equally likely to be in either direction, i.e. $p=1/2$. 

It is easy to see that $\log(1/p_t-1)$, performs a biased random walk, because the number $n_t$ of buy orders up to $t$ is a binomial random variable, with mean $\frac{1\pm\nu}{2} t$. We focus on the regime where $|\Delta p_t|=|p_t-p_0|\ll 1$, with $p_0=1/2$ being the initial price. Using $\log(1/p_t-1)\simeq -4\Delta p_t+\ldots$ we find
\begin{equation}
\label{eq:linear_impact}
\E{\Delta p_t|\nu}\simeq \E{n_t-t/2}\nu+\ldots =\pm\frac{\nu^2}{2}t+\ldots
\end{equation}
A meta-order of size $Q=\nu t$ then has a linear impact, i.e. $\E{\Delta p_t|\nu}\simeq \pm\frac \nu 2 Q$. Notice that the drift term $\pm\frac{\nu^2}{2}t$ in the price is negligible with respect to the diffusion term (the volatility) $\sigma=\sqrt{\E{(p_t-\E{p_t})^2}}\simeq \nu \sqrt{t}$ as long as $t\ll \nu^{-2}$. The drift term becomes of the same order as the volatility on time-scales of order $t\sim \nu^{-2}$ when the value of $Y$, i.e. the direction of the meta-order, becomes statistically detectable. 
The scaling regime of interest $t\sim \nu^{-1}$ lies in a region where the contribution from the meta-order is statistically undetectable.

\section{A Bayesian market maker}

So far we assumed that the trading frequency $\nu$ of the meta-order is known to the market maker. If the market maker does not know the true value of $\nu$, we can assume that he will update his prior belief on $\nu$ using his observation of the stream of transactions $x_{\le t}$ and using Bayes rule
\begin{eqnarray}
p(v|x_{\le t}) & = & \frac{P(x_{\le t}|v)\phi(v)}{\int_0^1 P(x_{\le t}|y)\phi(y)dy} %\\
%\nonumber
 %& = & \frac{[(1+v)^{n_t}(1-v)^{t-n_t}+(1+v)^{t-n_t}(1-v)^{n_t}]\phi(v)}{\int_0^1(1+y)^{n_t}(1-y)^{t-n_t}\phi(y)dy+\int_0^1(1+y)^{t-n_t}(1-y)^{n_t}\phi(y)dy}
\end{eqnarray}
where $\phi(v)$ is the prior. Here and henceforth, $v$ denotes the inferred value of the  trading frequency of the meta-order (which is a random variable), whereas $\nu$ denotes the true value.
Using this, the market maker estimates the price $p_t$ as a function of $x_{\le t}$ (actually of $n_t$)
\begin{equation}
\label{ptnt}
p_t(n_t)=\E{Y|x_{\le t}}=\int_0^1\!dv \E{Y|x_{\le t},v}P(v|x_{\le t})
\end{equation}
where $\E{Y|x_{\le t},v}$ is given by Eq.~(\ref{ptnu}) with $\nu$ replaced by $v$. We remark that the expectation in Eq.~(\ref{ptnt}) as well as the probability $P(v|x_{\le t})$ refer to the subjective state of knowledge of the market maker, not to objective probability distributions. 

\subsection{The square-root-impact law}

\begin{figure}[!ht]
\centering
\includegraphics[width=0.8\textwidth]
{prior.pdf}
\caption{Numerical simulations ($\nu=0.1$ and $Y=1$) of price impact for priors of type $\phi(v)\sim A v^{k-1}$. The case $k=1$ is the same as described in Eq.~(\ref{Epterf}). One can remark that other curves have the same slope as this one for $t\sim1/\nu$, emphasising the fact that they also follow the SRIL.}
\label{fig:SRILprior}
\end{figure}

Eq.~(\ref{ptnt}) is more conveniently expressed in terms of the random variable $\xi=(2n_t-t)/\sqrt{t}$, which  asymptotically follows the normal law in the scaling regime $1\ll t\ll \nu^{-2}$, with mean $\E{\xi}=\pm\nu\sqrt{t}\ll 1$ and variance $\V{\xi}=1-\nu^2$. As shown in the Appendix~\ref{app:sril}, to leading order, %\textit{SI Appendix}
\begin{equation}
p_t(\xi) \simeq  \frac{1}{\sqrt{2\pi}}\int_0^\infty\!d\gamma  e^{-\frac{1}{2}(\xi-\gamma)^2}=\frac 1 2 +\frac 1 2 {\rm erf}\!\left(\xi/\sqrt{2}\right)\,.
 \label{ptxi}
\end{equation}
Notice that $\xi\to\pm \infty$ in the limit $t\to\infty$, so that the price converges to the true value asymptotically, i.e. $p_t\to Y$.

%After some algebraic manipulations, this expression can be recast in the form
%\begin{equation}
%\label{eq:ptint}
%p_t=\frac{\int_{0}^1\!dv\phi(v)e^{-tD(z||v)}}{\int_{-1}^1\!dv\phi(|v|)e^{-tD(z||v)}}
%\end{equation}
%where $z=(2n_t-t)/t$ and 
%\begin{equation}
%\label{KLD}
%D(z||x) = \frac{1+z}{2}\log\frac{1+z}{1+x}+\frac{1-z}{2}\log\frac{1-z}{1-x} 
%\end{equation}
%is a Kullback-Leibler divergence.
%We expect $z=\xi/\sqrt{t}$ to be small, where $\xi$ is a Gaussian random variable with mean $\E{\xi}=\pm\nu\sqrt{t}$, which is small in the scaling regime $1\ll t\ll \nu^{-2}$, and variance $\V{\xi}=1-\nu^2$. Since the integral on $v$ is dominated by values $z\sim 1/\sqrt{t}$, we set $x=\gamma/\sqrt{t}$. For the time being, let us assume that $\phi(v)$ has a finite limit as $v\to 0$. We shall return later on the role of the prior. To leading order
%\begin{equation}
%D(z||x) \simeq \frac{(x-z)^2}{2(1-z^2)}+O((x-z)^3)\simeq %\frac{(\gamma-\xi)^2}{2t}+O(1/t^2)
%\end{equation}
%hence
%\begin{equation}
%p_t(\xi) \simeq  \frac{1}{\sqrt{2\pi}}\int_0^\infty\!d\gamma  e^{-\frac{1}{2}(\xi-\gamma)^2}=\frac 1 2 +\frac 1 2 {\rm erf}\!\left(\xi/\sqrt{2}\right)\,.
% \label{ptxi}
%\end{equation}
Taking the expectation over $\xi$, we find
%\footnote{In the general case $p\ne\frac{1}{2}$, Eq.~(\ref{ptxi}) becomes $p_t(\xi)\simeq\left[1+\frac{1-p}{p}\frac{1-{\rm erf}\!\left(\xi/\sqrt{2}\right)}{1+{\rm erf}\!\left(\xi/\sqrt{2}\right)}\right]^{-1}$ which can be expressed as a series in odd powers of $\xi$. Considering the leading behaviour $\E{\xi^{2n+1}}\simeq \frac{2^{n+1}}{\sqrt{\pi}}\Gamma(n+3/2)\nu\sqrt{t}+O(\nu \sqrt{t})^3$ of the expected values of odd powers of $\xi$, we conclude that the SRIL holds also in this case.}
\begin{equation}
\label{Epterf}
\E{p_t}\simeq \frac 1 2 \pm\frac{1}{2}{\rm erf}\!\left(\nu\sqrt{t}/{2}\right) = \frac 1 2 \pm\frac{1}{2\sqrt{\pi}}\nu\sqrt{t}+O(\nu t^{1/2})^3
\end{equation}
where, again, the $+$ sign holds for $Y=1$ and the $-$ sign for $Y=0$. Notice that, the true value of $\nu$ becomes statistically detectable when $t\sim \nu^{-2}$. This is also the time when Eq.~(\ref{Epterf}) predicts that the price gets close to the true value of $Y$.
Eqs.~(\ref{ptxi}) and~(\ref{Epterf}) assume that $\phi(v)$ has a finite limit for $v\to 0$.  Fig.~\ref{fig:SRILprior} shows that the leading behaviour $\E{p_t}\propto \nu\sqrt{t}+O(\nu t^{1/2})^3$ holds generally for any prior with $\phi(v)\sim v^{k-1}$ with $k>0$, as shown in Appendix~\ref{app:prior}. The square-root law indeed holds when the volume of individual orders is drawn from a distribution with finite second moment and/or in the presence of short range auto-correlations in the orders of noise traders (see Appendix~\ref{app:vol}). 

\subsection{The crossover to short time linear impact}

\begin{figure}[!ht]
\centering
\includegraphics[width=0.8\textwidth]
{crossover.pdf}
\caption{Market impact in the GMM by choosing a cutoff $\bar\nu$ in the prior for a buy metaorder ($Y=1$). Blue curve is derived from Monte-Carlo numerical simulations. Dashed line represents the theoretical linear impact we get from Eq.~(\ref{linear}) and dotted line represents the SRIL.}
\label{fig2}
\end{figure}

Eq.~(\ref{Epterf}) assumes a prior $\phi(\nu)$ that extends to finite values of $\nu$. It makes sense to assume that the market anticipates that the contribution of meta-orders to the order flow is small, i.e. that $\nu$ is at most $\bar\nu\ll 1$. We take $\phi(v)=1/\bar \nu$ for $v\in[0,\bar \nu]$ and $\phi(v)=0$ otherwise in order to incorporate this observation. With this assumption, Eq.~(\ref{ptxi}) becomes (see Appendix~\ref{app:cross})
\begin{equation}
\label{ptnubar}
p_t(\xi)\simeq\frac{{\rm erf}\!\left(\frac{\xi}{\sqrt{2}}\right)+{\rm erf}\!\left(\frac{\bar\nu\sqrt{t}-\xi}{\sqrt{2}}\right)}{{\rm erf}\!\left(\frac{\bar\nu\sqrt{t}+\xi}{\sqrt{2}}\right)+{\rm erf}\!\left(\frac{\bar\nu\sqrt{t}-\xi}{\sqrt{2}}\right)}
\end{equation}
For $\bar\nu\sqrt{t}\gg 1$ (i.e. $t\gg\bar\nu^{-2}$), we recover Eq.~(\ref{ptxi}) whereas for $t\ll\bar\nu^{-2}$ the expansion of the expression above for $\bar\nu\sqrt{t}\ll 1$ leads to $p_t(\xi)\simeq \frac 1 2 +\frac 1 4 \xi\bar\nu\sqrt{t}+O(\bar\nu^2 t)$. Taking the expectation on $\xi$ we recover the linear regime found in Ref.~\cite{bucci2019crossover}
\begin{equation}
\label{linear}
\E{p_t}\simeq \frac 1 2 \pm \frac 1 4 \bar\nu\nu t+\ldots\simeq  \frac 1 2 \pm \frac 1 4 \bar\nu Q +\ldots
~~~(t\ll\bar\nu^{-2})\end{equation}
In other words, for orders that last less than $\bar\nu^{-2}$ we find a linear impact, for orders that last a time $\bar\nu^{-2}<t<\nu^{-2}$ the SRIL sets in and for orders that are much longer than $\nu^{-2}$ the impact saturates to a finite value.

We note that adding a cutoff $\bar\nu$ to the prior is equivalent to the case where the market maker resets the prior $\tau$ time steps before the beginning of the meta-order. In the interval $t\in [-\tau,0]$ before the meta-order begins, the prior will be updated taking into account the effect of noise traders only. This results in a posterior at time $t=0$ with an effective cutoff at $\bar\nu\approx 1/\sqrt{\tau}$. The linear regime then extends for a duration which is of the same order of $\tau$. This argument provides an alternative explanation of the origin of the linear impact regime with respect to that given by Bucci {\em et al.}~\cite{bucci2019crossover}, who relate the duration of the linear regime to the separation of time-scales between two different population of traders.

\subsection{Impact decay}

Consider a meta-order which is active until time $T=Q/\nu$. The statistics of $\xi$ is easily calculated taking into account that 
\begin{equation}
\label{eq:expec_buys_decay}
\E{n_t}=\sum_{\tau=1}^{Q/\nu}\E{x_\tau}+\sum_{\tau=Q/\nu+1}^t\E{x_\tau}=\frac t 2 \pm \frac Q 2
\end{equation}
because $\E{x_\tau}=\frac 1 2 \pm \frac \nu 2$ in the first sum and $\E{x_\tau}=\frac 1 2$ in the second. The variance $\V{n_t}=\frac t 4 -\frac{\nu Q}{4}$ is computed in a similar manner. Hence $\xi$ is a Gaussian variable with mean $\E{\xi}=\pm \frac{Q}{\sqrt{t}}$ and variance $\V{\xi}=1-\nu Q/t$. Using this in Eq.~(\ref{ptxi}) one gets 
\begin{equation}
\label{eq:exp_price_decay}
\E{p_t}\simeq\frac 1 2 \pm \frac 1 2 {\rm erf}\!\left(\frac{Q}{\sqrt{4 t-2\nu Q}}\right)\simeq \frac 1 2 \pm \frac 1 2 {\rm erf}\!\left(\frac{Q}{2\sqrt{t}}\right)
\end{equation}
Hence the theory predicts a slow decay $\E{\Delta p_t}\simeq \pm \frac{Q}{2\sqrt{\pi t}}$ of the impact to zero, and no permanent impact (see Fig.~\ref{fig:decay}).

\begin{figure}[!ht]
\centering
\includegraphics[width=0.8\textwidth]
{decay.pdf}
\caption{Price impact for a buy meta-order ($Y=1$ and $\nu=0.035$) as a function of time. The meta-order extinguishes at time $t=400$. The theoretical decay  $\sim Q/2\sqrt{\pi t}$ (dashed line) matches well with the Monte-Carlo simulation (blue full line).}
\label{fig:decay}
\end{figure}

Note that $\E{\Delta p_{t>T}}/\E{\Delta p_T}\simeq\sqrt{T/t}$ for $t\gg T$, which is the same behaviour derived in the latent order books model \cite{donier2015fully} and it is compatible with the slow power-law decay predicted by the propagator model \cite{bouchaud2003fluctuations,zarinelli2015beyond} and the empirically observed power-law decay of the impact for intra-day time-scales~\cite{bucci2018slow}.
For small $t-T$ instead, our theory predicts that the impact immediately after the end of the meta-order decays as 
\[
\E{\Delta p_{t>T}}/\E{\Delta p_T}\simeq 1-(t-T)/2T\,.
\] 
This is less steep than the decay $\E{\Delta p_{t>T}}/\E{\Delta p_T}\simeq1-\sqrt{(t-T)/T}$
predicted by the latent order books model.
Also, no permanent impact is observed in our model whereas the ``fair pricing'' theory \cite{farmer2013efficiency} predicts a permanent impact equal to $2/3$ of the peak impact. An approach based on the Minority Game~\cite{barato2013impact} suggests that a permanent impact should be observed only in inefficient markets where statistical arbitrages exist. 
The empirical evidence on whether meta-orders leave a permanent impact or not~\cite{bucci2018slow} is still being debated, with some studies supporting the hypothesis of a plateau~\cite{zarinelli2015beyond,moro2009market} and some others~\cite{brokmann2015slow,gomes2015market} suggesting that the decay should tend towards the initial price. 

Another interesting prediction of our model is that, upon reverting the direction of the meta-order after a time $T=Q/\nu$, the price should revert to the original price after a time $T$ (see Appendix~\ref{app:reverse}). This contrasts with the latent order books theory, which predicts a faster return to the original price, in a time equal to $T/4$~\cite{donier2015fully}.

\subsection{An intuitive argument}

Some insights on the origin of the behaviour derived this far can be gained by a simple argument that assumes that the market maker relies on a Bayesian estimate\footnote{We show in the \textit{SI Appendix} that a maximum likelihood approach would lead to a qualitatively different result.} $\hat\nu_{\rm Bayes}=\E{\nu|x_{\le t}}$ of $\nu$. This leads to (see Appendix~\ref{app:infer})
\begin{equation}
\label{hatnubayes}
\hat\nu_{\rm Bayes}\simeq \left\{\begin{array}{cc} \bar\nu/2 & t\ll\bar\nu^{-2} \\
\sqrt{\frac{2}{\pi t}} & \bar\nu^{-2}\ll t\ll \nu^{-2}\end{array}\right.
\end{equation}

\begin{figure}[!ht]
\centering
\includegraphics[width=0.8\textwidth]{modified_intuitive.pdf}
\caption{Sketch of the behaviour of the Bayesian estimate of $\hat\nu_{\rm Bayes}$ (blue curve). The solid green curve represents the real value of $\nu$ imposed by the informed trader.}
\label{fig:sketch}
\end{figure}

In words, at short times ($t\ll\bar\nu^{-2}$) the prior is dominated by the cutoff $\bar\nu$ whereas for longer time-scales its width shrinks as $1/\sqrt{t}$. Substituting the leading behaviour of $\hat\nu_{\rm Bayes}$ in the linear impact results (Eq.~\ref{eq:linear_impact}), i.e. $\E{\Delta p_t|\nu}\simeq \pm\frac{\hat\nu_{\rm Bayes}}{2}Q$, allows one to recover the behaviour discussed thus far, as sketched in Fig.~\ref{fig:sketch}. Notice that the meta-order execution time $Q/\nu$ is typically much shorter than the time when the true value $\hat\nu_{\rm Bayes}\simeq\nu$ is discovered. Hence, the whole phenomenology of the market impact is concealed by stochastic fluctuations. Indeed, in empirical studies, it  emerges only by carefully taking  averages over many meta-orders, conditional on the inception of a meta-order. 

\section{Back to real markets}

An apparent shortcoming of our theory is that the GMM predicts that prices do not diffuse, contrary to empirical evidence. Indeed, from Eq.~(\ref{ptxi}), it is easy to compute the volatility %(see SI)
\begin{equation}
%\label{sigma}
\sigma=\sqrt{\V{\Delta p_t}} = \frac{1}{2\sqrt{3}} - \frac{\sqrt{3}-1}{4\sqrt{3}\pi}\nu^2t+O(\nu t^{1/2})^3\,.
%\frac{1}{\sqrt{2\sqrt{3}}}-\frac{\sqrt{2 \sqrt{3}-3} }{4\pi }\nu^2 t+
   %\frac{\left(81-54\sqrt{3}+\left(6 \sqrt{3}-5\right) \pi\right) \nu^4 t^2}{48\sqrt{2} 3^{3/4} \pi^2}+O\left(\nu^6t^3\right)
%   O\left(\nu^4t^2\right)
\end{equation}
This is consistent with the fact that the GMM captures the way markets respond to informed order flows. 
In the absence of informed traders, the GMM predicts a constant price, which does not diffuse at all. 

This is clearly unrealistic, but it is exactly the outcome that one expects if one averages prices over many different market conditions, as done in the empirical analysis leading to Eq.~(\ref{empSRIL}). In the absence of meta-orders we expect a constant average price whereas conditioning on the inception of the meta-order reveals the expected impact of the meta-orders. 
%This is precisely the same behaviour that one expects in empirical analysis when one takes the average over many meta-orders, conditional on the inception of the meta-order. 
In this procedure, many different market conditions and meta-order with different speeds $\nu$ are averaged, in a way which is analogous to the expectation over $\nu$ that we assumed in our analysis\footnote{It is important to acknowledge that the expectation over $\nu$ is of a different nature than the one performed by the market-maker.}. We mention, in passing, that in this averaging procedure the long range auto-correlation of order flows~\cite{bouchaud2018trades} are also likely washed out, reproducing an uncorrelated order flow generated by noise traders, as assumed in the GMM.

In order to restore diffusive price behaviour, we make the assumption that the price is the sum of an exogenous component $f_t$, with increments of order $\sigma_f=\sqrt{\E{(f_t-f_{t-1})^2}}$,
%$\E{(f_t-f_{t-1})^2}=\sigma_f^2$ 
and a endogenous component $\sigma_Y(2Y-1)$ that accounts for price variations due to the execution of the meta-order. This decomposition is similar to that used in the analysis of signal trading strategies~\cite{lehalle2019incorporating}.
Assuming that $f_t=\E{f_{t+\tau}|f_t}$ is a martingale, the price at time $t$ becomes 
\begin{eqnarray}
    p_t &=& \E{f_t+\sigma_Y(2Y-1)|x_{\le t}} \\
        &\simeq&f_{t-1}+\sigma_Y~{\rm erf}\!\left(\xi/\sqrt{2}\right) \label{eq:martingale}
\end{eqnarray}
where the last line derives from the martingale property and by using Eq.~(\ref{ptxi}). Since $\V{\Delta p_t}\simeq \sigma_f^2 t+\frac{\sigma_Y^2}{3}+O(\nu^2 t)$, diffusive price behaviour is recovered as long as 
$t\gg\sigma_Y^2/\sigma_f^2$.

Taking the expectation of Eq.~(\ref{eq:martingale}) over $\xi$ we get
\begin{equation}
\label{impact_mart}
    \E{\Delta p_t}\simeq \pm\sigma_Y~\mathrm{erf}\left(\frac{\nu\sqrt{t}}{2}\right) \simeq \pm \sqrt{\pi}\sigma_Y\nu\sqrt{t}+O(\nu t^{1/2})^3
\end{equation}
so we recover again the square root behaviour of the price impact. This can be reconciled with the empirically observed SRIL (Eq.~\ref{empSRIL}) assuming that the volatility over the time-scale used to measure $\sigma$ and $V$ 
is dominated by the exogenous component $\sigma\simeq \sigma_f\sqrt{t}$ (with $V=t$ being the traded volume). In addition, one would need to assume that $\sigma_Y=C\sigma_f/\sqrt{\pi\nu}$ depends on $\nu$. One argument in support of this assumption is that, within the GMM, $\sigma_Y$ is the magnitude of the difference between the benchmark price $f_t$ and the ``true'' price known by informed traders. Then $\sigma_Y\propto 1/\sqrt{\nu}$ amounts to assuming that this difference is of the same order of the volatility over the time-scale $1/\nu$ between two child orders, which is the only time-scale in the problem. 

The contribution of market impact to the volatility, according to our theory, should be visible in leading corrections to the volatility. One prediction %(see the SM) 
is that the leading correction to the volatility at the peak should depend on the order size as
\begin{equation}
\label{variance_mart}
    \frac{1}{T}\V{\Delta p_T}\simeq \sigma_f^2 +\frac{C^2}{3\pi Q}\,.
\end{equation}
A further prediction of our theory (see Appendix~\ref{app:bidask}) is that the average bid-ask spread, conditional on the start of the meta-order, should decrease as $1/\sqrt{t}$. 

\section{Conclusion}

In summary, this paper shows that the square-root law of market impact can be derived as an exact result in the relevant scaling limit of a simple model of a market in which the price is formed through a fully Bayesian reasoning. This approach borrows the price formation mechanism from the GMM model, by realising that the persistent order flow that originates form a meta-order has the same effect as the activity of an informed trader.

The extreme simplicity of the model and the absence of any {\em ad-hoc} assumptions, makes it appropriate as a stylised description of a variety of market settings.  
Hence it provides further theoretical support for the empirically observed universality of this law. 
The independence of impact laws on the execution protocol is a consequence of the fact that the scaling regime of interest $t\sim \nu^{-1}$ lies in a region where the contribution from the meta-order to the order flow is statistically undetectable.

This model also allows to recover the crossover to a linear impact for small orders $Q< \frac{4}{\sqrt{\pi}}\frac{\nu}{\bar\nu^2}$ observed in \cite{bucci2019crossover} as well as the impact decay for times $t\gg Q/\nu$. This approach predicts that the price, on average, reverts back to the initial value after the meta-order ends. There is no clear evidence whether this is true~\cite{bucci2018slow} or not.
The empirical laws observed in financial markets can be recovered by a careful analogy between the Bayesian setting considered in this paper and the conditional averaging procedure employed in empirical analysis. This leads us to consider an additional exogenous martingale term, as e.g. in~\cite{lehalle2019incorporating}, besides the contribution of the GMM that describes the way in which the market handles signals originating from order imbalances.

Besides providing a simple and transparent picture of market impact phenomenology, this theory also provides falsifiable predictions on the volatility and on the bid-ask spread conditional on the meta-order initiation, that could be tested in empirical data.

 
%It's main shortcoming is the prediction of a constant volatility of prices (Eq.~(\ref{sigma}) and Fig.~\ref{fig1}), which is hard to reconcile with empirical data.
%Furthermore, the model predicts that the  bid-ask spread should decay, in expected terms, as $1/\sqrt{t}$ as the meta-order is executed. Such decrease encodes the price discovery process 

%\matmethods{Please describe your materials and methods here. This can be more than one paragraph, and may contain subsections and equations as required. 

%\subsection*{Subsection for Method}
%Example text for subsection.

%\showmatmethods{} % Display the Materials and Methods section

\section{Acknowledgments}
{We are grateful to Jean-Philippe Bouchaud, Iacopo Mastromatteo, Leonardo Bargigli and Michele Vodret for several discussions.}

%\showacknow{} % Display the acknowledgments section

% Bibliography
\bibliographystyle{unsrt} 
\bibliography{pnas-sample}

\appendix

\section{Derivation of the SRIL}
\label{app:sril}

Let's recall that the time series $x_{\le t}=(x_1,\ldots,x_t)$ of trades ($x_t=1$ for buy and $x_t=0$ for sell) 
is a random variable whose distribution depends on $Y$ and on the  value of $\nu$, 
\begin{equation}
P\{x_{\le t}|Y,\nu\}=\left(\frac{1\pm\nu}{2}\right)^{n_t}\left(\frac{1\mp\nu}{2}\right)^{t-n_t},\qquad n_t=\sum_{\tau=1}^t x_\tau\,.
\end{equation}
Here and in the sequel, the upper (lower) signs apply when $Y=1$ ($Y=0$) and we assumed that noise traders buy or sell with probability $1/2$.

The price, as shown in the main text, is given by
\begin{equation}
\label{ptnt_app}
p_t(n_t)=\E{Y|x_{\le t}}=\int_0^1\!dv \E{Y|x_{\le t},v}P(v|x_{\le t})
\end{equation}
where the posterior distribution of $\nu$ is given by
\begin{eqnarray}
p(v|x_{\le t}) = \frac{P(x_{\le t}|v)\phi(v)}{\int_0^1 P(x_{\le t}|y)\phi(y)dy}
\label{bysr}
\end{eqnarray}
and $\phi(v)$ is the prior. By Bayes rule, we have
\[
\E{Y|x_{\le t},v} = \frac{P\{x_{\le t}|Y=1,v\}}{P\{x_{\le t}|Y=1,v\}+P\{x_{\le t}|Y=0,v\}} 
\]
where we took $p=1/2$. Injecting these two last equations in Eq.~(\ref{ptnt_app}) we get
\begin{equation}
\label{eq:ptint}
p_t=\frac{\int_{0}^1\!dv\phi(v)e^{-tD(z||v)}}{\int_{-1}^1\!dv\phi(|v|)e^{-tD(z||v)}}
\end{equation}
where $z=(2n_t-t)/t$ and 
\begin{equation}
\label{KLD}
D(z||x) \equiv \frac{1+z}{2}\log\frac{1+z}{1+x}+\frac{1-z}{2}\log\frac{1-z}{1-x} 
\end{equation}
is a Kullback-Leibler divergence. 

Notice that the random variable $z$ is typically very small for $t\gg 1$. Indeed $z$ is well approximated by $z=\xi/\sqrt{t}$, where $\xi$ is a Gaussian random variable with mean $\E{\xi}=\pm\nu\sqrt{t}$ and variance $\V{\xi}=1-\nu^2$. Notice also that $\E{\xi}$ is very small in the scaling regime $1\ll t\ll \nu^{-2}$ of interest. 

For $t\to\infty$, the integrals on $v$ in Eq.~(\ref{eq:ptint}) are dominated by values of $v$ such that $D(z||v)\sim t^{-1}$. Since $D(z||v)\sim (z-v)^2$, the integrals are dominated by values of $v$ such that $(v-z)\sim 1/\sqrt{t}$. This implies that the relevant range in the integrals is when $v\sim 1/\sqrt{t}$, because $z\sim 1/\sqrt{t}$. Correspondingly, the integrals probe the behaviour of the prior $\phi(v)$ in the limit $v\to 0$. We shall assume that $\phi(v)$ has a finite limit as $v\to 0$. Different singular behaviours of the prior for $v\to 0$ will be discussed in the next Section.

Summarising, to leading order we find $D(z||v) \simeq \frac{(\gamma-\xi)^2}{2t}+O(1/t^2)$, which leads to
\begin{equation}
p_t(\xi) \simeq  \frac{1}{\sqrt{2\pi}}\int_0^\infty\!d\gamma  e^{-\frac{1}{2}(\xi-\gamma)^2}=\frac 1 2 +\frac 1 2 {\rm erf}\!\left(\xi/\sqrt{2}\right)\,,
 \label{ptxi1}
\end{equation}
which is Eq.~(\ref{ptxi}) in the main text.
Taking the expectation over $\xi$, we find\footnote{In the general case $p\ne\frac{1}{2}$, Eq.~(\ref{ptxi1}) becomes $p_t(\xi)\simeq\left[1+\frac{1-p}{p}\frac{1-{\rm erf}\!\left(\xi/\sqrt{2}\right)}{1+{\rm erf}\!\left(\xi/\sqrt{2}\right)}\right]^{-1}$ which can be expressed as a series in odd powers of $\xi$. Considering the leading behavior $\E{\xi^{2n+1}}\simeq \frac{2^{n+1}}{\sqrt{\pi}}\Gamma(n+3/2)\nu\sqrt{t}+O(\nu \sqrt{t})^3$ of the expected values of odd powers of $\xi$, we conclude that the SRIL holds also in this case.} Eq.~(\ref{Epterf}) in the main text.
%\begin{equation}
%\label{Epterf_app}
%\E{p_t}\simeq \frac 1 2 \pm\frac{1}{2}{\rm erf}\!\left(\nu\sqrt{t}/{2}\right)\simeq \frac 1 2 \pm\frac{1}{2\sqrt{\pi}}\nu\sqrt{t}+O(\nu t^{1/2})^3
%\end{equation}
%where, again, the $+$ sign holds for $Y=1$ and the $-$ sign for $Y=0$. 
Notice that, within this approximation,  $\E{p_t}\to Y$ as $t\to\infty$.

From Eq.~(\ref{ptxi1}) one can also compute the variance of the price
\[
\sigma=\sqrt{\V{p_t}} = \frac{1}{2\sqrt{3}} - \frac{\sqrt{3}-1}{4\sqrt{3}\pi}\nu^2t+O(\nu t^{1/2})^3\,.
\]

\section{Square-root-law for priors $\phi(\nu){\sim}\nu^{k-1}$ as $\nu\to 0$}
\label{app:prior}

Here we generalize the derivation of the previous Section to priors which behave as $\phi(\nu){\sim}A\nu^{k-1}$ as $\nu\to 0$, with $k>0$ and $A$ a normalization constant to ensure $\int_0^1 \mathrm{d}v \phi(v)=1$. Thus, introducing again $\xi=(2n_t-t)/\sqrt{t}$ and setting $v=\gamma/\sqrt{t}$ in  Eq.~(\ref{eq:ptint}), we have
\begin{equation}
    p_t(\xi)\approx\frac{\int_0^\infty\!d\gamma \gamma^{k-1} e^{-\frac{1}{2}(\xi-\gamma)^2}}{\int_{-\infty}^\infty\!d\gamma |\gamma|^{k-1} e^{-\frac{1}{2}(\xi-\gamma)^2}}=\frac{1}{2}+\frac{\xi}{\sqrt{2}}\frac{\Gamma(\frac{1+k}{2})}{\Gamma(\frac{k}{2})}\frac{_1F_1(1-\frac{k}{2},\frac{3}{2},\frac{-\xi^2}{2})}{_1F_1(\frac{1-k}{2},\frac{1}{2},\frac{-\xi^2}{2})} 
\label{eq:price_prior}
\end{equation}
where ${_1F_1}(a,b,x)$ is the confluent hypergeometric function of parameters $a$ and $b$. The expansion of  Eq.~(\ref{eq:price_prior}) in powers of $\xi$ only contains odd powers. The leading terms are
\[
p_t(\xi)=\frac 1 2 \pm \frac{\Gamma\left(\frac{1+k}{2}\right)}{\sqrt{2}\Gamma\left(\frac{k}{2}\right)}\left[\xi+\frac{1-2k}{6}\xi^3+\frac{(3-4k)(1-4k)}{120}\xi^5 + \dots\right]\,.
\] 
Considering the leading behavior $\E{\xi^{2n+1}}\simeq \frac{2^{n+1}}{\sqrt{\pi}}\Gamma(n+3/2)\nu\sqrt{t}+O(\nu \sqrt{t})^3$ 
 of the expected values of odd powers of $\xi$, we conclude that the SRIL holds also in this case, with a coefficient that depends on $k$.
 For small values of $k$, Eq.~(\ref{eq:price_prior}) reads
\begin{eqnarray}
    p_t(\xi) &\simeq& \frac{1}{2} \pm k\frac{\pi}{4}\mathrm{erfi}\left(\frac{\xi}{\sqrt{2}}\right)+O(k^2) \\
    &\simeq& \frac{1}{2} \pm k\frac{\sqrt{\pi}}{2\sqrt{2}}\left[\xi+\frac{\xi^3}{6}+\frac{\xi^5}{40}+O(\xi^7)\right] \nonumber
\end{eqnarray}
where $\mathrm{erfi}(x)=\frac{2}{\sqrt{\pi}}\int_0^x e^{z^2} dz$ is the imaginary error function. In the limit of small $k$, the market impact varies linearly with $k$ and vanishes for $k\to0$. In fact for $k=0$, the prior behaves as a delta-function at $\nu=0$, that corresponds to a market maker who believes that there is no informed trader (or meta-order). Yet, at time $t\gg\nu^{-2}$ the true value of $\nu$ is revealed irrespective of what the prior is. Therefore, the behavior $\E{\Delta p_t}\propto k\nu\sqrt{t}$ at $t\ll\nu^{-2}$ should leave way to a faster growth of the market impact, as shown in Fig.~1 of the main paper, %\ref{fig:SRILprior}, 
in the $t\sim\nu^{-2}$ regime.

\section{Square-root law for generic order flow processes}
\label{app:vol}

In the main text, we considered that all agents trade a unit volume at each step of the game, and that orders generated by noise traders are uncorrelated. We can generalize the model by allowing both the informed trader and the noise traders to trade non unitary volumes at the same time, and by allowing correlated order flows. In order to be general, we will also assume that the value of the asset $Y$ is worth $1$ with probability $p$ and $0$ with probability $1-p$. 

The informed trader knows the real value of the asset and will take advantage of this knowledge to make profit. At each step $t$ of the game, she will buy a quantity $q>0$ of the asset if it's worth $1$ and she will sell a volume $-q<0$ it if it's worth $0$ (so that $q$ is always positive).

Noise traders trade randomly. At each step $t$, they trade a volume $v_t$ which follows a normal law $\mathcal{N}(r,\sigma_v^2)$, where $r$ quantifies the bias of the noise traders and $\sigma_v$ is the typical volume of uninformed trades. To simplify the analysis, we will study the case of unbiased noise traders so $r=0$. 

The market maker is the one who provides the liquidity of the market. At each step he fixes the price at which he sells or buys the asset in order to make zero profit on average. Hence the price is his expectation of $Y$. The only information he has is the volume imbalance $\Delta V$ at time $t$ and the probability $p$ of the asset to be equal to $1$. Let's start the analysis by considering the market maker knows the speed of trade of the informed trader $q$. 

Let us introduce $\mathcal{V}_t=\sum_{\tau=1}^t v_t$ the cumulative volume traded by noise traders until time $t$. Thus $\mathcal{V}_t$ follows the Gaussian distribution $\mathcal{N}(0,\sigma_v^2t)$. The volume imbalance at time $t$ is then $\Delta V=\mathcal{V}_t\pm qt$, which is also a Gaussian variable with distribution $P(\Delta V|Y,q)\propto\exp\left(-\frac{1}{2}\frac{(\Delta V\mp qt)^2}{\sigma_v t^2}\right)$. Therefore, the price set by the market maker will be
\begin{eqnarray}
    p_t(\Delta V) = \E{Y|\Delta V,q} &=& P(Y=1|\Delta V,q) \nonumber \\
    &=& \frac{pP(\Delta V|Y=1,q)}{pP(\Delta V|Y=1,q)+(1-p)P(\Delta V|Y=0,q)} \nonumber \\
    &=& \frac{p\exp\left(-\frac{1}{2}\frac{(\Delta V-qt)^2}{\sigma_v^2t}\right)}{p\exp\left(-\frac{1}{2}\frac{(\Delta V-qt)^2}{\sigma_v^2t}\right)+(1-p)\exp\left(-\frac{1}{2}\frac{(\Delta V+qt)^2}{\sigma_v^2t}\right)} \nonumber \\
    &=& \left[1+\frac{1-p}{p}\exp\left(-\frac{2q\Delta V}{\sigma_v^2}\right)\right]^{-1} \label{eq:q_known} \,.
\end{eqnarray}

Let's now consider the case where the speed of trade $q$ is not known. Thus we have
\begin{equation}
    p_t(\Delta V)=\E{Y|\Delta V}=\int_0^\infty dq \E{Y|\Delta V,q}P(q|\Delta V) \,. \label{eq:price_q}
\end{equation}

Knowing $\Delta V$, the probability distribution of $q$ is
\begin{eqnarray}
    P(q|\Delta V) &=& \frac{P(\Delta V|q)\phi(q)}{\int_0^\infty dq P(\Delta V|q)\phi(q)} \nonumber \\
    &=& \frac{pP(\Delta V|Y=1,q)+(1-p)P(\Delta V|Y=0,q)}{\int_0^\infty dq\phi(q)[pP(\Delta V|Y=1,q)+(1-p)P(\Delta V|Y=0,q)]} \label{eq:prob_q}
\end{eqnarray}
where $\phi(q)$ is the prior distribution of the speed of trade $q$.
By injecting Eq.~(\ref{eq:q_known}) and Eq.~(\ref{eq:prob_q}) in Eq.~(\ref{eq:price_q}) we get
\begin{equation}
    p_t(\Delta V)=\left[1+\frac{1-p}{p}\frac{\int_0^\infty dq\phi(q)\exp\left(-\frac{1}{2}\frac{(\Delta V+qt)^2}{\sigma_v^2t}\right)}{\int_{-\infty}^0 dq\phi(q)\exp\left(-\frac{1}{2}\frac{(\Delta V+qt)^2}{\sigma_v^2t}\right)}\right]^{-1} \,.
\end{equation}

If we take an uniform prior, the price reads
\begin{equation}
    p_t(\Delta V)=\left[1+\frac{1-p}{p}\frac{1-\mathrm{erf}\left(\frac{\Delta V}{\sqrt{2t}\sigma_v}\right)}{1+\mathrm{erf}\left(\frac{\Delta V}{\sqrt{2t}\sigma_v}\right)}\right]^{-1} \label{eq:price_gen}
\end{equation}
and for the special case $p=\frac{1}{2}$, this equation becomes
\begin{equation}
    p_t(\Delta V)=\frac{1}{2}+\frac{1}{2}\mathrm{erf}\left(\frac{\Delta V}{\sqrt{2t}\sigma_v}\right) \label{eq:price_1/2}
\end{equation}
which is very similar to the one obtained in the unit volume trading model (Eq.~\ref{ptxi1}). Here the analogous of $\xi$ is $\Delta V/\sigma_v\sqrt{t}\sim\mathcal{N}\left(\pm\frac{q}{\sigma_v}\sqrt{t},1\right)$. It is also easy to see that the quantity analogous to $\nu$ is $q/\sigma_v$. By taking the expectation over the transactions of noise traders, we obtain the price impact
\begin{equation}
    \E{p_t(\Delta V)}=\frac{1}{2} \pm \frac{1}{2}\mathrm{erf}\left(\frac{q\sqrt{t}}{2\sigma_v}\right)=\frac{1}{2} \pm \frac{q\sqrt{t}}{2\sqrt{\pi}\sigma_v}+O\left(\frac{q\sqrt{t}}{\sigma_v}\right)^2
    \label{eq:impact_gene}
\end{equation}
which again features the square root behavior. From Eq.~(\ref{eq:price_1/2}), one can also compute the standard deviation
\begin{equation}
\label{sigma_nonunit}
\sigma=\sqrt{\V{\Delta p_t}} = \frac{1}{2\sqrt{3}} - \frac{\sqrt{3}-1}{4\sqrt{3}\pi}\left(\frac{q}{\sigma_v}\right)^2t+O\left(\frac{q}{\sigma_v} t^{1/2}\right)^3\,.
\end{equation}

\subsection{Response of the market to order flow imbalance}

Eq.~(\ref{eq:price_gen}) and Eq.~(\ref{eq:price_1/2}) give the price as a function of the order flow imbalance $\Delta V$ at time $t$. Thus, they represent exactly the aggregated impact $\E{p_t-p_0|\Delta V}$ of the market if the initial price $p_0=p$ is subtracted. 

%Empirically, it is known that the aggregated impact, considering a time window $T$ and an order flow imbalance $\Delta V$, has the form \cite{bouchaud2018trades}
%\begin{equation}
%    \E{p_{t+T}-p_t} \simeq \mathcal{R}(1)T^\chi~\mathcal{F}\left(\frac{\Delta V}{V_D T^\kappa}\right)
%\end{equation}
%where $\mathcal{R}(1)$ is the average price variation after an unitary trade, $V_D$ is the typical volume traded during a day, $\chi\approx0.5-0.7$ and $\kappa\approx0.75-1$ are dimensionless exponents. $\mathcal{F}$ is a scaling function, linear for small arguments and concave for large ones. In the case $p=1/2$ for example, $\sqrt{2}\sigma_v$ plays the role of $V_D$ and the exponent $\kappa=1/2$. As the standard deviation of the price is of order of the unity, $\mathcal{R}(1)=1/2$ is coherent. However, we find an exponent $\chi=0$. The scaling function is the error function, it is linear for small arguments and concave for large ones as expected.

An interesting parameter that can be extracted from the aggregated impact is the Kyle lambda $\Lambda$. It is the slope of the aggregated impact at $\Delta V=0$ and it quantifies the response of the market to a small order flow imbalance. By taking the derivative of Eq.~(\ref{eq:price_gen}) at $\Delta V=0$, we obtain
\begin{equation}
    \Lambda=2\sqrt{\frac{2}{\pi t}}\frac{p(1-p)}{\sigma_v}
\end{equation}
so the Kyle lambda in our model decreases as a function of time, as observed empirically \cite{bouchaud2018trades}. This contrasts with the prediction of the Kyle model where $\Lambda$ is a constant~\cite{kyle1985continuous}. Notice also that the response of the market increases with $\V{Y}=p(1-p)$, i.e. with the uncertainty on the value of $Y$. 

By noticing that the Bayesian estimator of the speed of trade of the informed trader reads
$q^\mathrm{Bayes}=\int_0^\infty \mathrm{d}q~qP(q|\Delta V)=\sqrt{\frac{2}{\pi t}}\sigma_v+O(\Delta V)$, one can recast the Kyle lambda as
\begin{equation}
\label{KyleBayes}
    \Lambda=2q^\mathrm{Bayes}\frac{\V{Y}}{\sigma_v^2} \,,
\end{equation}
emphasizing the fact that the response of the market maker to the volume imbalance is proportional to his estimate of the speed of trade of the informed trader. This is not surprising because if we expand Eq.~(\ref{eq:q_known}), which describes the response of the market maker when $q$ is known, we find
\begin{equation}
    p_t(\Delta V)=p+2q\frac{\V{Y}}{\sigma_v^2}\Delta V+O(\Delta V^2)\,.
\end{equation}
Eq.~(\ref{KyleBayes}) is consistent with this equation where the value of $q$ is replaced by its Bayes estimator. 

\subsection{Fat tailed distribution for the noise trading volumes}

The derivation of the previous Section can be generalised to the case where noise traders submit orders with a fat tailed volume distribution. Let's assume that the volume $v_t$ traded by noise traders at time $t$ follows an even distribution with $P(v)\sim\frac{C}{|v|^{1+\alpha}}$ for $v\to\infty$, where $C$ is a constant and $\alpha$ is the exponent of the fat tails distribution. For $\alpha\ge2$, the variance $\sigma_v$ of $v_t$ is finite. Therefore, the central limit theorem (CLT) applies and the cumulative noise volume $\mathcal{V}_t$ follows the normal law distribution $\mathcal{N}(0,\sigma_v^2t)$. In summary, the behavior of $\Delta V$ is asymptotically the same than for Gaussian noise trades when $t\to\infty$. Thus we recover the square root law for all $\alpha\ge 2$. 

For $\alpha<2$, we can not apply the CLT because the variance of $v_t$ is not finite. Here, we will assume that the $v_t$ follow Levy stable distribution $L_{\alpha,\sigma_v}$, with $\alpha$ the stability parameter, $\sigma_v$ the scale parameter and both the location parameter and skewness parameter equal to zero in order to have an even distribution. This distribution can be seen as a generalisation of the Gaussian distribution which is recover for $\alpha=2$. Also, this distribution has the asymptotic behaviour $L_{\alpha,\sigma_v}(x)\sim\frac{C(\alpha)}{|x|^{1+\alpha}}$. It has the nice property that, if $\mathcal{V}_t$ are independent random variables with distribution $L_{\alpha,\sigma_v}$ for all $t$, then $\sum_{\tau=1}^t \mathcal{V}_\tau$ follows the $L_{\alpha,\sigma_v t^\frac{1}{\alpha}}$ Levy distribution.

% so $\mathcal{V}_t$ follows $L(\alpha,\sigma_v t^\frac{1}{\alpha})$ and by definition of the scale parameter, we also have $L_{a,c}(x)=\frac{1}{c}L_{a,1}\left(\frac{x}{c}\right)$.

Thanks to these properties, and by injecting $\mathcal{V}_t$ in Eq.~(\ref{eq:q_known}) to (\ref{eq:prob_q}) (remember that $\Delta V=\mathcal{V}_t\pm qt$), for $p=1/2$ we obtain
\begin{equation}
    p_t(\Delta V)=\int_{-\infty}^\frac{\Delta V}{\sigma_v t^{1/\alpha}} L_{\alpha,1}(x)dx \,.
\end{equation}

Introducing $\mathcal{L}_{a,c}(x)=\int_0^x L_{a,c}(x)dx$, as $\lim\limits_{x\to-\infty}\mathcal{L}_{a,c}(x)=-\frac{1}{2}=-p_0$, we can rewrite the last expression
\begin{equation}
    p_t(\Delta V)-p_0=\mathcal{L}_{\alpha,1}\left(\frac{\Delta V}{\sigma_v t^{1/\alpha}}\right) \,.
\end{equation}

We can give some examples of Alpha-L\'evy distributions. For $\alpha=2$ we recover the Gaussian distribution and $\mathcal{L}_{2,1}(x)=\frac{1}{2}\mathrm{erf}(x)$. For $\alpha=1$, we have the Cauchy distribution and $\mathcal{L}_{1,1}(x)=\frac{1}{\pi}\arctan(x)$.

Let us now compute the price impact. If we consider $qt\ll \mathcal\sigma_v t^{1/\alpha}$, we will have
\begin{eqnarray}
    \mathcal{L}_{\alpha,1}\left(\frac{\Delta V}{\sigma_v t^{1/\alpha}}\right) &=& \mathcal{L}_{\alpha,1}\left(\frac{\mathcal{V}_t+qt}{\sigma_v t^{1/\alpha}}\right) \nonumber \\
    &=& \mathcal{L}_{\alpha,1}\left(\frac{\mathcal{V}_t}{\sigma_v t^{1/\alpha}}\right)+L_{\alpha,1}\left(\frac{\mathcal{V}_t}{\sigma_v t^{1/\alpha}}\right)\frac{qt}{\sigma_v t^{1/\alpha}}+O\left(\frac{q}{\sigma_v}t^{1-\frac{1}{\alpha}}\right)^2
\end{eqnarray}

So the expected price reads
\begin{eqnarray}
    \E{p_t} &=& \int_\mathbb{R} du~L_{\alpha,\sigma_v t^{1/\alpha}}(u)\mathcal{L}_{\alpha,1}\left(\frac{u+qt}{\sigma_v t^{1/\alpha}}\right) \nonumber \\
    &=& \int_\mathbb{R} du~L_{\alpha,\sigma_v t^{1/\alpha}}(u)\mathcal{L}_{\alpha,1}\left(\frac{u}{\sigma_v t^{1/\alpha}}\right) \\
    & ~ & + \frac{qt}{\sigma_v t^{1/\alpha}}\int_\mathbb{R} du~L_{\alpha,\sigma_v t^{1/\alpha}}(u)L_{\alpha,1}\left(\frac{u}{\sigma_v t^{1/\alpha}}\right)+O\left(\frac{q}{\sigma_v}t^{1-\frac{1}{\alpha}}\right)^2 \nonumber \\
    &=& \frac{1}{2}+K(\alpha)\frac{qt^{1-\frac{1}{\alpha}}}{\sigma_v}+O\left(\frac{q}{\sigma_v}t^{1-\frac{1}{\alpha}}\right)^2
\end{eqnarray}
where $K(\alpha)=\int_\mathbb{R} du~L_{\alpha,1}(u)^2$. For $\alpha=2$, $K(2)=1/(2\sqrt{\pi})$ so we recover Eq.~(\ref{eq:impact_gene}).

In financial market, the exponent of the fat tail is generally reported to be $\alpha\approx 5/2>2$ \cite{bouchaud2018trades} but this exponent can vary from study to study \cite{lillo2005theory,gopikrishnan2000statistical,farmer2004origin}.

\subsection{Square-root law for correlated order flow}
%\label{app:corr}

In order to add correlations between the orders, let us assume that the sequence $(v_1,\dots,v_t)$ follows a multivariate normal distribution $\mathcal{N}(\mathbf{0},\mathbf{C})$. $\mathbf{0}$ is the null vector, it indicates that the mean of each $v_t$ is zero. $\mathbf{C}$ is the correlation matrix defined by $(\mathbf{C})_{t,t'}=\E{v_t v_{t'}}=f(|t-t'|)$, where $f(0)=\sigma_v^2$ is the variance of $v_t$ and $f$ is a decreasing function. Thus, the cumulative noise volume $\mathcal{V}_t$ follows the normal distribution $\mathcal{N}(0,\Sigma_t^2)$ where $\Sigma_t^2=\sum_{t'=1}^t\sum_{t''=1}^t f(|t''-t'|)$.

Now, the equivalent of Eq.~(\ref{eq:price_1/2}) is
\begin{equation}
    p_t(\Delta V)=\frac{1}{2}+\frac{1}{2}\mathrm{erf}\left(\frac{\Delta V}{\sqrt{2}\Sigma_t}\right)
\end{equation}
and its expectancy reads
\begin{equation}
    \E{p_t(\Delta V)}=\frac{1}{2}+\frac{1}{2}\mathrm{erf}\left(\frac{qt}{2\Sigma_t}\right) \,.
\end{equation}

Let's consider the case where we have a fast decay of the order correlations. For example we can take $(\mathbf{C})_{t,t'}=\sigma_v^2 \exp\left(-\frac{|t-t'|}{\tau_c}\right)$ where $\tau_c$ is a correlation time. We can now compute the variance of $\mathcal{V}_t$
\begin{eqnarray}
    \Sigma_t^2 &=& t\sigma_v^2+2\sigma_v^2\sum_{1\le t'<t''\le t}\exp\left(-\frac{t''-t'}{\tau_c}\right) \nonumber \\
    &\sim& \left(1+\frac{2}{e^{1/\tau_c}-1}\right)\sigma_v^2 t \,.
\end{eqnarray}

\begin{figure}
\centering
\includegraphics[width=0.8\textwidth]{correl.pdf}
\caption{Numerical simulation of the price impact for correlated noise trades in the non unit volume model ($q=0.1$ and $\sigma_v=1$). The blue curve shows the price impact in the case of uncorrelated noise trades.}
\label{fig:correl}
\end{figure}

For $\tau_c\to0$, $\Sigma_t^2=\sigma_v^2 t$ so we recover the case where the $v_t$ are i.i.d. random variables. One can also remark that the correlated case does not differ so much from the independent case as they both have a linear time dependency. Hence, the square root impact law should hold in the case of short range auto-correlations in the orders of noise traders.

Let's now consider the case where we have long range correlations between market orders, by choosing \footnote{We use this particular correlation matrix because its Fourier transform can be expressed analytically, which allows to generate quality samples for numerical simulations (see \cite{makse1996method}).}
\begin{equation}
    (\mathbf{C})_{t,t+\tau}=\frac{\sigma_v^2}{(1+\tau^2)^\frac{\eta}{2}}{\sim}\frac{\sigma_v^2}{\tau^\eta}~~~({\tau \to +\infty})\,.
\end{equation}

For $\eta=1$, one can show that $\Sigma_t\sim\sigma_v\sqrt{2t\log(t)}$ and for $0<\eta<1$, $\Sigma_t\sim\sqrt{\frac{2}{(1-\eta)(2-\eta)}}\sigma_vt^{1-\frac{\eta}{2}}$. Thus for $0<\eta<1$, we get a price impact law
\begin{equation}
    \E{p_t}-p_0\simeq\frac{1}{2}\sqrt{\frac{(1-\eta)(2-\eta)}{2\pi}}\frac{q}{\sigma_v}t^{\eta/2}+O\left(\frac{q}{\sigma_v}t^{\eta/2}\right)^2
\end{equation}
with a different exponent with respect to the SRIL.
The price impact is plotted for several value of $\eta$ in Fig.~\ref{fig:correl}.

\section{Crossover to linear behaviour}
\label{app:cross}

It makes sense to assume that the market anticipates that the contribution of meta-orders to the order flow is small, i.e. that $\nu\ll \bar\nu\ll 1$. One way to incorporate this observation is to take $\phi(y)=1/\bar \nu$ for $y\in[0,\bar \nu]$ and $\phi(y)=0$ otherwise. With this assumption, Eq.~(\ref{eq:ptint}) becomes
\begin{equation}
p_t=\frac{\int_{0}^{\bar\nu}\!dv~e^{-tD(z||v)}}{\int_{-\bar\nu}^{\bar\nu}\!dv~e^{-tD(z||v)}} \,.
\end{equation}

Setting $v=\gamma/\sqrt{t}$ and introducing as before $\xi=z\sqrt{t}=(2n_t-t)/\sqrt{t}$, we obtain 
\begin{equation}
    p_t(\xi) \simeq \frac{\int_0^{\bar\nu\sqrt{t}}\mathrm{d}\gamma~e^{-\frac{1}{2}(\gamma-\xi)^2}}{\int_{-\bar\nu\sqrt{t}}^{\bar\nu\sqrt{t}}\mathrm{d}\gamma~e^{-\frac{1}{2}(\gamma-\xi)^2}}
    \simeq \frac{{\rm erf}\!\left(\frac{\xi}{\sqrt{2}}\right)+{\rm erf}\!\left(\frac{\bar\nu\sqrt{t}-\xi}{\sqrt{2}}\right)}{{\rm erf}\!\left(\frac{\bar\nu\sqrt{t}+\xi}{\sqrt{2}}\right)+{\rm erf}\!\left(\frac{\bar\nu\sqrt{t}-\xi}{\sqrt{2}}\right)} \,.
\end{equation}

For $\bar\nu\sqrt{t}\gg 1$ (i.e. $t\gg\bar\nu^{-2}$), we recover $p_t(\xi) \simeq \frac 1 2 +\frac 1 2 {\rm erf}\!\left(\xi/\sqrt{2}\right)$ whereas for $t\ll\bar\nu^{-2}$ the expansion of the expression above for $\bar\nu\sqrt{t}\ll 1$ leads to $p_t(\xi)\simeq \frac 1 2 +\frac 1 4 \xi\bar\nu\sqrt{t}+O(\bar\nu^2 t)$. Taking the expectation on $\xi$ we find a linear regime, Eq.~(\ref{linear}) in the main text.

\section{Impact decay upon reversing the direction of the meta-order}
\label{app:reverse}

In this section, we will be interested in the case where a meta-order is executed until time $T=Q/\nu$ and is immediately followed by another meta-order in the opposite direction. The statistics of $\xi$ is easily calculated taking into account that 
\begin{equation}
\E{n_t}=\sum_{\tau=1}^{Q/\nu}\E{x_\tau}+\sum_{\tau=Q/\nu+1}^t\E{x_\tau}=\frac t 2 \pm \left(Q-\frac{\nu t}{2}\right)
\label{eq:sum_decay}
\end{equation}
because $\E{x_\tau}=\frac 1 2 \pm \frac \nu 2$ in the first sum and $\E{x_\tau}=\frac 1 2 \mp \frac \nu 2$ in the second. Similarly we compute the variance $\V{n_t}=\frac{1-\nu^2}{4}t$. Hence $\xi$ is asymptotically a Gaussian variable with mean $\E{\xi}=\pm\frac{2Q-\nu t}{\sqrt{t}}$ and variance $\V{\xi}=1-\nu^2$. Using this in Eq.~(\ref{ptxi1}), one gets 
\begin{equation}
\E{p_t}\simeq\frac 1 2 \pm \frac 1 2 {\rm erf}\!\left(\frac{Q}{\sqrt{t}}-\frac{\nu\sqrt{t}}{2}\right) \,.
\end{equation}

From this last equation, one can see that the time needed to go back to the original price after the end of the first meta-order is $t-T=T$. Note that for large values of $t$, we recover the square root impact law but in the opposite direction. For small $t-T$ instead, the impact immediately after the end of the first meta-order decays as $\E{\Delta p_{t>T}}/\E{\Delta p_T}\simeq 1-\frac{3}{2}\frac{t-T}{T}$. This decay is steeper than the case where no meta-order is executed after the first one. In the latter case we find $\E{\Delta p_{t>T}}/\E{\Delta p_T}\simeq 1-\frac{1}{2}\frac{t-T}{T}$.

\section{Inference of $\nu$}
\label{app:infer}

This section details the derivation of the Bayes estimator and of the maximum likelihood estimator (MLE) of $\nu$ discussed in the main text. Their behaviour is shown in Fig.~\ref{fig:bayes_mle}.
\begin{figure}
\centering
\includegraphics[width=0.8\textwidth]{bayes_mle.pdf}
\caption{Plot of estimators of $\nu$ times $\sqrt{t}$ as a function of $\xi=(2n_t-t)/\sqrt{t}$.}
\label{fig:bayes_mle}
\end{figure}

\subsection{Bayesian estimator}

The Bayes estimator is
\begin{equation}
\label{byse}
\hat\nu_\mathrm{Bayes}=\mathbb{E}[\nu|x_{\le t}]=\int_0^1 v P(v|x_{\le t})\mathrm{d}v \,.
\end{equation}

By injecting Eq.~(\ref{bysr}) into Eq.~(\ref{byse}), one can obtain
\begin{equation}
\label{eq:nubayes_exa}
\hat\nu_\mathrm{Bayes}=\frac{\int_{-1}^1\!dv\phi(|v|)|v|e^{-tD(z||v)}}{\int_{-1}^1\!dv\phi(|v|)e^{-tD(z||v)}}
\end{equation}

Again, the asymptotic behaviour for $t\to\infty$ is derived with the change of variables $\xi=z\sqrt{t}$ and $v=\gamma/\sqrt{t}$. 
By assuming that $\phi(v)$ is finite as $v\to0$, we get to the first order in the scaling regime $1\ll t\ll \nu^{-2}$
\begin{eqnarray}
\hat\nu_\mathrm{Bayes} & \simeq&  \frac{1}{\sqrt{2\pi t}}\int_{-\infty}^\infty |\gamma| e^{-\frac{1}{2}(\xi-\gamma)^2}\mathrm{d}\gamma \nonumber\\
        &=& \sqrt{\frac{2}{\pi t}}e^{-\frac{1}{2}\xi^2}+\frac{\xi}{\sqrt{t}}{\rm erf}\!\left(\xi/\sqrt{2}\right) \label{eq:nubayes} %\\
%        &=& \sqrt{\frac{2}{\pi t}}+\frac{1}{\sqrt{2\pi t}}\xi^2+o_{0}(\xi^2)\\
 %       &=& \frac{|\xi|}{\sqrt{t} }+o_{\pm\infty}(\xi)\sim z
\end{eqnarray}
which has the limiting behaviours $\hat\nu_\mathrm{Bayes}\simeq \frac{2+\xi^2}{\sqrt{2\pi t}}$ for $|\xi|\ll 1$ and $\hat\nu_\mathrm{Bayes}\simeq \frac{|\xi|}{\sqrt{t}}$ for $|\xi|\to\pm\infty$.

The expected value of $\hat\nu_\mathrm{Bayes}$ over the distribution of $\xi$ is
\[
\E{\hat\nu_\mathrm{Bayes}}=\frac{2}{\sqrt{\pi t}}e^{-\frac{-\nu^2 t}{4}}+\nu~\mathrm{erf}\left(\frac{\nu\sqrt{t}}{2}\right) \,.
\]
Its limiting behaviour is $\E{\hat\nu_\mathrm{Bayes}}\simeq \frac{2}{\sqrt{\pi t}}$ for $t\sim1/\nu$ and $\E{\hat\nu_\mathrm{Bayes}}\propto\nu$ for $t \sim 1/\nu^2$.

{For a prior of type $\phi(v)\sim A v^{k-1}$,} the same reasoning as above applies and one can show that
\begin{equation}
    \hat\nu_\mathrm{Bayes} \simeq \sqrt{\frac{2}{t}}\frac{\Gamma \left(\frac{k+1}{2}\right)}{\Gamma \left(\frac{k}{2}\right)}\frac{_1F_1\left(-\frac{k}{2},\frac{1}{2},-\frac{\xi^2}{2}\right)}{_1F_1\left(\frac{1-k}{2},\frac{1}{2},-\frac{\xi^2}{2}\right)} 
\end{equation}
For $k=1$, one can recover Eq.~(\ref{eq:nubayes}). The behavior of $\hat\nu_\mathrm{Bayes}$ is plotted for several values of $k$ in Fig.~\ref{fig:bayes_prior}. 
\\
\begin{figure}
\centering
\includegraphics[width=0.8\textwidth]{prior_inset.pdf}
\caption{Plot of the Bayes estimator times $\sqrt{t}$ as a function of $\xi$ for priors of type $\phi(v)\sim A v^{k-1}$. In the inset, numerical simulations ($\nu=0.1$) of the expected value of the Bayes estimator as a function of time.}
\label{fig:bayes_prior}
\end{figure}

In the case where the prior has a cutoff, i.e.  $\phi(y)=1/\bar \nu$ for $y\in[0,\bar \nu]$ and $\phi(y)=0$ otherwise, Eq.~(\ref{eq:nubayes_exa}) becomes
\begin{eqnarray}
    \hat\nu_\mathrm{Bayes} &=& \frac{\int_{-\bar\nu}^{\bar\nu}\!dv\phi(|v|)|v|e^{-tD(z||v)}}{\int_{-\bar\nu}^{\bar\nu}\!dv\phi(|v|)e^{-tD(z||v)}} \nonumber \\
    &\simeq& \frac{1}{\sqrt{t}}\frac{\int_{-\bar\nu\sqrt{t}}^{\bar\nu\sqrt{t}}\mathrm{d}\gamma~|\gamma|e^{-\frac{1}{2}(\gamma-\xi)^2}}{\int_{-\bar\nu\sqrt{t}}^{\bar\nu\sqrt{t}}\mathrm{d}\gamma~e^{-\frac{1}{2}(\gamma-\xi)^2}} \nonumber \\
%    &\simeq& \sqrt{\frac{2}{\pi t}}\frac{2e^{-\frac{\xi^2}{2}}-e^{-\frac{1}{2}(\xi-\bar\nu\sqrt{t})^2}-e^{-\frac{1}{2}(\xi+\bar\nu\sqrt{t})^2}+\sqrt{2\pi}\xi{\rm erf}\!\left(\frac{\xi}{\sqrt{2}}\right)-\sqrt{\frac{\pi}{2}}\xi{\rm erf}\!\left(\frac{\xi-\bar\nu\sqrt{t}}{\sqrt{2}}\right)-\sqrt{\frac{\pi}{2}}\xi{\rm erf}\!\left(\frac{\xi+\bar\nu\sqrt{t}}{\sqrt{2}}\right)}{{\rm erf}\!\left(\frac{\bar\nu\sqrt{t}+\xi}{\sqrt{2}}\right)+{\rm erf}\!\left(\frac{\bar\nu\sqrt{t}-\xi}{\sqrt{2}}\right)}\\
    &\simeq& \frac{\sqrt{\frac{2}{\pi t}}}{{\rm erf}\!\left(\frac{\bar\nu\sqrt{t}+\xi}{\sqrt{2}}\right)+{\rm erf}\!\left(\frac{\bar\nu\sqrt{t}-\xi}{\sqrt{2}}\right)}\left[
    2e^{-\frac{\xi^2}{2}}-e^{-\frac{1}{2}(\xi-\bar\nu\sqrt{t})^2}-e^{-\frac{1}{2}(\xi+\bar\nu\sqrt{t})^2}\right.\nonumber\\
    &~&~~~~~~~~~~~~~~\left.+\sqrt{2\pi}\xi{\rm erf}\!\left(\frac{\xi}{\sqrt{2}}\right)-\sqrt{\frac{\pi}{2}}\xi{\rm erf}\!\left(\frac{\xi-\bar\nu\sqrt{t}}{\sqrt{2}}\right)-\sqrt{\frac{\pi}{2}}\xi{\rm erf}\!\left(\frac{\xi+\bar\nu\sqrt{t}}{\sqrt{2}}\right)\right]\nonumber
\end{eqnarray}

Note that we recover Eq.~(\ref{eq:nubayes}) in the limit $\bar\nu\sqrt{t}\gg 1$. However for $\bar\nu\sqrt{t}\ll 1$, the expansion of the expression leads to $\hat\nu_\mathrm{Bayes} \simeq \bar\nu/2+\frac{1}{\sqrt{t}}O(\bar\nu\sqrt{t})^2$.

\subsection{Maximum likelihood estimator (MLE)}

The maximum likelihood estimator is
\[
\nu_{\mathrm{MLE}}=\arg \max_v P(x_{\le t}|v) \,.
\]
We have for $p=\frac{1}{2}$
\begin{eqnarray}
P(x_{\le t}|v) & \propto & (1+v)^{n_t}(1-v)^{t-n_t}+(1+v)^{t-n_t}(1-v)^{n_t} \nonumber\\
    &\propto& (1-v^2)^\frac{t}{2}\cosh\left[\frac{\xi\sqrt{t}}{2}\log\left(\frac{1+v}{1-v}\right)\right] \,.
\end{eqnarray}
Hence the condition for the maximum likelihood reads
\[
\frac{\partial P(x_{\le t}|\nu)}{\partial \nu}=0 \Leftrightarrow \nu\sqrt{t}\cosh\left[\frac{\xi\sqrt{t}}{2}\log\left(\frac{1+\nu}{1-\nu}\right)\right]=\xi \sinh\left[\frac{\xi\sqrt{t}}{2}\log\left(\frac{1+\nu}{1-\nu}\right)\right]\,.
\]
Therefore the MLE satisfies the self consistent equation
\begin{equation}
\label{}
\hat\nu_{\mathrm{MLE}}=\frac{\xi}{\sqrt{t}}\tanh\left[\frac{\xi\sqrt{t}}{2}\log\left(\frac{1+\hat\nu_{\mathrm{MLE}}}{1-\hat\nu_{\mathrm{MLE}}}\right)\right] \,.
\end{equation}
For $\xi<1$, this equation has only zero as a solution. For $\xi>1$, the zero solution becomes unstable and $\hat\nu_{\mathrm{MLE}}\simeq \frac{|\xi|}{\sqrt{t}}$ for $|\xi|\gg \sqrt{t}$.

\section{Bid-ask spread}
\label{app:bidask}

The ask and the bid prices at time $t+1$ are given by
\begin{eqnarray}
    a_{t+1} &=& \int_0^1\!dv \E{Y|x_{\le t},x_{t+1}=1,v}P(v|x_{\le t},x_{t+1}=1)\label{askeq}\\
    b_{t+1} &=& \int_0^1\!dv \E{Y|x_{\le t},x_{t+1}=0,v}P(v|x_{\le t},x_{t+1}=0)\label{bideq}
\end{eqnarray}
which can be recast, using Bayes rule, as 
\begin{eqnarray}
    a_{t+1}(z) & = & \frac{\int_{0}^1\!dv\phi(v)(1+v)e^{-tD(z||v)}}{\int_{-1}^1\!dv\phi(|v|)(1+v)e^{-tD(z||v)}}\\
    b_{t+1}(z) & = & \frac{\int_{0}^1\!dv\phi(v)(1-v)e^{-tD(z||v)}}{\int_{-1}^1\!dv\phi(|v|)(1-v)e^{-tD(z||v)}}
\end{eqnarray}
where $z=(2n_t-t)/t$ and $D(z||x)$ is a Kullback-Leibler divergence.  Again, $z$ is typically of order $1/\sqrt{t}$, so we focus on the variable $\xi=z\sqrt{t}$ and make the change of variable $\gamma=v\sqrt{t}$. By also assuming $\phi(\nu)$ is finite as $\nu\to0$, we get to the first order in the scaling regime $1\ll t\ll \nu^{-2}$
\begin{equation}
    \frac{\int_{0}^1\!dv\phi(v)(1\pm v)e^{-tD(z||v)}}{\int_{-1}^1\!dv\phi(|v|)(1\pm v)e^{-tD(z||v)}}\simeq\frac{\int_0^\infty \left(1\pm\frac{\gamma}{\sqrt{t}}\right)e^{-\frac{1}{2}(\gamma-\xi)^2} d\gamma}{\sqrt{2\pi}(1\pm\xi/\sqrt{t})}
\end{equation}

Now, we can compute the bid-ask spread
\begin{eqnarray}
    a_{t+1}-b_{t+1} &\simeq& \sqrt{\frac{2}{\pi t}}\frac{1}{1-\frac{\xi^2}{t}}\int_0^\infty (\gamma-\xi) e^{-\frac{1}{2}(\gamma-\xi)^2} d\gamma \nonumber \\
    &=& \sqrt{\frac{2}{\pi t}}e^{-\frac{\xi^2}{2}}+O(t^{-\frac{3}{2}})
\end{eqnarray}

Its expected value is
\begin{equation}
    \E{a_{t+1}-b_{t+1}} = \frac{1}{\sqrt{\pi t}}e^{-\frac{\nu^2t}{4}}=\frac{1}{\sqrt{\pi t}}+O(\nu^2t)
\end{equation}


\end{document}
