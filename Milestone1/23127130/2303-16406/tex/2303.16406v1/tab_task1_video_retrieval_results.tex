\begin{table}[t]
    \centering
    \resizebox{0.95\columnwidth}{!}{
    \begin{tabular}{l c c c c c}
        \toprule
        Model & Frames & FT & R@1 & R@5 & R@10 \\
        \midrule

        CLIP-B/32 & 1 & & 11.4 & 20.7 & 27.3 \\
        CLIP-B/32 & 4 & & 12.5 & 28.8 & 37.4 \\
        CLIP-B/32 & 10 & & 13.0 & 31.7 & 39.9 \\
        CLIP-B/32 & 20 & & 13.0 & 33.3 & 41.2 \\
        CLIP-B/32 & 32 & & 12.6 & 33.0 & 41.8 \\

        Frozen-in-Time & 4 & & 7.0 & 19.4 & 26.7 \\
        MIL-NCE (S3D) & 32 &  & 13.9 & 31.1 & 41.4 \\

        \midrule
        
        
        CLIP-B/32 & 1 & \cmark & 11.5 & 22.7 & 27.1 \\
        CLIP-B/32 & 4 & \cmark & 13.9 & 29.5 & 39.4 \\
        CLIP-B/32 & 10 & \cmark & 11.4 & 31.3 & 41.4 \\
        CLIP-B/32 & 20 & \cmark & 12.3 & 31.7 & 41.6 \\
        CLIP-B/32 & 32 & \cmark & 13.0 & 32.1 & 41.9 \\

        \midrule
        
        EVA-CLIP-G/14 & 1 & & 18.9 & 32.6 & 37.5 \\
        EVA-CLIP-G/14 & 4 & & 20.7 & 43.6 & 53.7 \\
        EVA-CLIP-G/14 & 10 & & 26.0 & 48.5 & 58.8\\
        EVA-CLIP-G/14 & 20 & & \textbf{26.4} & \textbf{51.1} & \textbf{61.5} \\
        EVA-CLIP-G/14 & 32 & & 26.0 & 50.0 & 61.4\\

        
        \bottomrule
    \end{tabular}
    }
    \caption{
        Video retrieval results on \dataname{} test split.
        CLIP/EVA-CLIP results are based on temporal average pooling.
        \textit{FT: finetuning on \dataname{}, R@k: Recall@k.}
        MIL-NCE was trained on the HowTo100M dataset, which is the video source of \dataname{}.
    }
    \label{tab:video_retrieval}
\end{table}
