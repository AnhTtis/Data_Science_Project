\section{Related Work} \label{related_work}
\vspace{-1mm}


\noindent \textbf{Document Shadow Removal and Upsampling} Earlier works based on intrinsic images \cite{yang2012shadow,brown2006geometric} and hand-crafted methods \cite{jung2018water,bako2016removing,kligler2018document} are built on simplifying assumptions and fail to perform effectively. Most recently, a deep-learning based solution BEDSR is proposed \cite{lin2020bedsr}. BEDSR assumes document images have a dominant background color, and explicitly estimates this via a background estimation module. This is used to create an attention map of background/foreground pixels, which acts as a shadow mask. Input image and the attention map are fed to an encoder/decoder to remove the shadows. On the other hand, several document specific super-resolution methods, based on CNNs \cite{pandey2017language} or GANs \cite{peng2020building}, are shown to be effective and also improve OCR performance. 

\noindent \textbf{Document Shadow Removal Datasets.} Existing real-life document shadow removal datasets are quite small  \cite{bako2016removing,jung2018water,kligler2018document,lin2020bedsr} and thus not suitable for training purposes. Synthetically casting shadows on shadow free images is arguably more practical than creating large-scale real-life datasets, as synthetic datasets can eliminate intrinsic data errors (on non-shadow regions), simulate various lighting/occluder settings and can automate labelling process entirely. SDSRD is the largest synthetic dataset formed of 8.3K triplets, created from 970 unique images \cite{lin2020bedsr}, however, it is not publicly available and even larger datasets are still desirable.




In comparison to the current state-of-the-art BEDSR \cite{lin2020bedsr}, our work has several key advantages: we i) do not need the background color labels during training, ii) address high resolution output requirement explicitly, iii) enlarge data distribution via our new datasets and iv) do not use large, complex architectures. Our pipeline is also a prime candidate for optimizations via pruning/quantization as it uses simple building blocks, whereas BEDSR requires gradient information \cite{selvaraju2017grad} in inference, which can be hard to obtain on resource-constrained environments like mobile phones. Such optimizations are interesting for future work, but not within our scope.

\vspace{-2mm}

