\section{Introduction} \label{introduction}
\vspace{-1mm}



\noindent  The wide spread use of mobile phone cameras has made document digitization significantly practical. Using mobile phone cameras often leads to issues like distortion, blur, noise and shadows cast on documents. Document shadow removal task aims to remove shadows cast on document images in a visually pleasant manner. Despite the recent advances  \cite{bako2016removing,jung2018water,kligler2018document,lin2020bedsr}, there are issues with existing methods. First, the majority of the existing methods do not often aim for a lightweight solution. Second, most methods do not operate at high resolutions. Third, document images come in many forms, such as text-only colorless documents and colored/figure-heavy documents, which necessitates good performance in-the-wild. A recent work \cite{lin2020bedsr} addresses document shadow removal in an end-to-end manner, however, it does so with a large model that does not operate at high resolutions.


The contributions of our work are as follows: we propose i) IOANet, a document shadow removal network with input/output attention for real-time operation, ii) a lightweight upsampling module that encapsulates IOANet, letting us operate at high resolutions, iii) three new datasets which cover various lighting conditions, document types and viewpoints and iv) a two-stage training pipeline that lets us leverage any low-resolution dataset for improved generalization. Our LP-IOANet comfortably outperforms the state-of-the-art, runs in real-time on a mobile device in 4 times the resolution of the state-of-the-art. The state-of-the-art method runs out of memory, thus can not be run, even on a 24GB VRAM desktop GPU. The diagram of LP-IOANet is shown in Figure \ref{fig:overall_diagram}.






\begin{figure*}[!ht]
  \centering
    \includegraphics[width=\textwidth]{figures/fig1_g2_att_cropped.pdf}
  \vspace{-8mm}
  \caption{Our LP-IOANet pipeline. Following the training of our shadow removal network (red dashed lines) in low-resolution (see Figure \ref{fig:networks}), we freeze it and train our lightweight upsampling module (dashed blue lines) on our proposed A-BSDD dataset.}
  \label{fig:overall_diagram}
  \vspace{-4mm}
\end{figure*}


\vspace{-3mm}


