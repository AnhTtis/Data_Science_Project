% !TeX spellcheck = en_US
\documentclass[letterpaper, 10 pt, conference]{ieeeconf} 
\pdfoutput=1 
%\IEEEoverridecommandlockouts
%\overrideIEEEmargins
\usepackage[english]{babel}
\usepackage{cite}
\usepackage{amsmath,amssymb,amsfonts}
\usepackage{algorithm}
\usepackage{algorithmic}
\usepackage{graphicx}
\usepackage{textcomp}
\usepackage{amsmath}
\usepackage{dsfont}
\usepackage{xcolor}
\usepackage{units}
\usepackage{booktabs}
\usepackage{comment}
\usepackage{url}
% Leo added -------
\usepackage[normalem]{ulem}
\usepackage[hidelinks]{hyperref}
% ---------

%\usepackage[noend]{algpseudocode}

%\makeatletter
%\def\BState{\State\hskip-\ALG@thistlm}
%\makeatother

\let\proof\relax
\let\endproof\relax
\let\theorem\relax
\let\endtheorem\relax
\usepackage{amsthm}
\newcounter{thm}
\newtheorem{prob}[thm]{Problem}
\newtheorem{lemma}{Lemma}[section]
\newtheorem{theorem}{Theorem}[section]
\newtheorem{definition}{Definition}[section]
\newtheorem{remark}{Remark}[section]
\newtheorem{approximation}{Approximation}[section]
\def\BibTeX{{\rm B\kern-.05em{\sc i\kern-.025em b}\kern-.08em
    T\kern-.1667em\lower.7ex\hbox{E}\kern-.125emX}}
\section{Approximate Leximin Optimality}\label{sec:approx-leximin-def}
In this section, we present our definition of leximin approximation in the presence of multiplicative and additive errors, in the context of multi-objective optimization problems.
% Section \ref{} discusses other potential definitions that might be considered intuitive and the reasons we made a different choice.


% \yonatan{I think the following discussion is too detailed for the introduction - where the actual definition is not provided.}
\subsection{Motivation: Unsatisfactory  Definitions}

Which solutions should be considered approximately-optimal in terms of  leximin? 
Several definitions appear intuitive at first glance.
As an example, suppose we are interested in approximations with an allowable multiplicative error of $0.1$.
Denote the utilities in the leximin-optimal solution by $(u_1,\ldots,u_n)$.
A first potential definition is that any solution in which the sorted utility vector is at least $(0.9\cdot u_1,\ldots,0.9\cdot u_n)$ should be considered approximately-optimal.
For example, if the utilities in the optimal solution are $(1,2,3)$, then a solution with utilities $(0.9, 1.8, 2.7)$ is approximately-optimal.
However, allowing the smallest utility to take the value $0.9$ may substantially increase the maximum possible value of the second (and third) smallest utility --- e.g.~a solution that yields utilities $(0.9, 1000,1000)$ might exist. In that case, a solution with utilities  $(0.9, 1.8, 2.7)$ is very far from optimal.
We expect a good approximation notion to consider the fact that an error in one utility might change the optimal value of the others.

The following, second attempt at a definition, captures this requirement.
An approximately-optimal solution is one that yields utilities at least $(0.9\cdot m_1, 0.9 \cdot m_2, \dots, 0.9 \cdot m_n)$, where $m_1$ is the maximum value of the smallest utility, $m_2$ is the maximum value of the second-smallest utility \emph{among all solutions whose smallest utility is at least $0.9 \cdot m_1$};
$m_3$ is the maximum value of the third-smallest utility among all solutions whose smallest utility is at least $0.9 \cdot m_1$ and their second-smallest utility is at least $0.9\cdot m_2$; and so on. 
In the above example, to be considered approximately-optimal, the smallest utility should be at least $0.9$ and the second-smallest should be at least $900$.
Thus, a solution with utilities $(0.9, 1.8, 2.7)$ is not considered approximately-optimal. Unfortunately, according to this definition, even the leximin-optimal solution --- with utilities $(1,2,3)$  --- is not considered approximately-optimal.
We expect a good approximation notion to be a relaxation of leximin-optimality.
% \eden{is this paragraph relevant here? if so, to write about algorithms as well?\\
% \textit{Is there an algorithm that can be used when the single-objective solver can only approximate the optimal value?}}
% \erel{I like the examples. I am not sure where is the best place to mention the algorithms.}



\subsection{Our Definition}

\paragraph{The approximate leximin order} 
The first step in the defining  the following  strict \textit{partial} order on solutions.
%(a partial order allows two solutions with different utilities such that no one is preferred over the other).
Here, a maximal element is one over which no other solution is preferred; note that it is not equivalent to one that is preferred over all others (as in total order).

We focus on maximization problems. Given $\DEFmultApprox\in (0,1]$ and $\DEFadditiveApprox \geq 0$, 
%a value $v_1 \in \mathbb{R}$ is considered a $(\DEFmultApprox,\DEFadditiveApprox)$-approximation of a value $v_2 \in \mathbb{R}$ if $v_1 \geq \DEFmultApprox\cdot v_2 - \DEFadditiveApprox$.
value %Therefore, we say that 
$v_2$ is \emph{$(\DEFmultApprox,\DEFadditiveApprox)$-preferred} over $v_1$ if $v_2 > \frac{1}{\DEFmultApprox}(v_1 + \DEFadditiveApprox)$.

The approximate leximin order can now be described.%
\footnote{A proof that the approximate leximin order is a strict partial order can be found in appendix \ref{sec:approx-order-is-strict-partial}}
% It is determined according to the following preferences relation: we say that
A solution $y$ is \emph{$(\DEFmultApprox,\DEFadditiveApprox)$-leximin-preferred} over a solution $x$ if there exists an integer $k \in [n]$ such that the smallest $(k-1)$ objective values of $y$ are \emph{at least} those of $x$, and the $k$'th smallest objective value of $y$ is $(\DEFmultApprox,\DEFadditiveApprox)$-preferred over the $k$'th smallest objective value of $x$, that is:
\begin{align*}
    \forall j < k \colon \quad &\valBy{j}{y} \geq \valBy{j}{x}\\
    &\valBy{k}{y} > \frac{1}{\DEFmultApprox} \left( \valBy{k}{x} + \DEFadditiveApprox\right)
\end{align*}
This relation is denoted by $y \alphaBetaPreferred x$.
The corresponding relation set is defined as follows:
\begin{align*}
    \relationSetAlphaBeta = \{(y,x) \mid \forall x,y \in S \colon y \alphaBetaPreferred x\}
\end{align*}

Before describing the approximation definition, we present some observations of this new relation that will be useful later.
The proofs are straightforward and are omitted. 

\iffalse
Consider the case were $\DEFmultApprox = 1$ and $\DEFadditiveApprox=0$. 
The relations $\leximinPreferred$ and $\alphaBetaPreferredParams{1}{0}$ may look similar at first glance (as the requirement for $k$ is the same), but they are different.
In the first relation, the requirement for $j<k$ says that the values of $y$ are \emph{equal} to those of $x$, while in the second relation it says that they are \emph{at least} as good. 
In spite of this, Lemma \ref{lemma:approx-relation-prop1} proves that these two relation are equivalent.
\fi

\begin{lemma}\label{lemma:approx-relation-prop1}
    Let $x,y \in S$. Then $y \leximinPreferred x \iff y \alphaBetaPreferredParams{1}{0} x$.
\end{lemma}

\iffalse
\begin{proof}
    % The first direction $y \leximinPreferred x \Rightarrow y \alphaBetaPreferredParams{0}{0} x$ is almost trivial.
For the first direction,
    assume that $y \leximinPreferred x$. 
    By definition there exists an integer $k \in [n]$ such that $\valBy{j}{y} = \valBy{j}{x}$ for any $j < k$, and $\valBy{k}{y} > \valBy{k}{x}$.
    It is easy to verify that the same $k$ also implies that $y \alphaBetaPreferredParams{1}{0} x$. 
    
For the second direction,
    assume that $y \alphaBetaPreferredParams{1}{0} x$. 
    By definition there exists an integer $k \in [n]$ such that $\valBy{j}{y} \geq \valBy{j}{x}$ for any $j < k$, and $\valBy{k}{y} > \frac{1}{1} \left(\valBy{k}{x}+0\right) = \valBy{k}{x}$.
    Let $k'$ be the smallest integer for which $\valBy{k'}{y} > \valBy{k'}{x}$; 
    % (such a $k'$ must exist since it is true in particular for $k$) . 
    Note that $k'\leq k$.
    This means that $\valBy{j'}{y} = \valBy{j'}{x}$  for any $j' < k'$, and $\valBy{k'}{y} > \valBy{k'}{x}$, so $y \leximinPreferred x$.
\end{proof}
\fi

% \eden{is this ok?} 
% \erel{Yes}
Throughout the remainder of this section, we denote the difference between $\DEFmultApprox$ and $1$ by $\DEFmultError = 1-\DEFmultApprox$; in the context of approximations, $\DEFmultError$ can be viewed as the multiplicative \emph{error} factor.

Another important property of this relation arises from the following observation.

\begin{observation}\label{obs:approx-relation-prop2}
     If $0 \leq \DEFmultErrorOf{\DEFmultApprox_1} \leq  \DEFmultErrorOf{\DEFmultApprox_2} < 1$ and $0 \leq \DEFadditiveError_1 \leq \DEFadditiveError_2$. 
     Then:
     \begin{align*}
         y \alphaBetaPreferredParams{\DEFmultApprox_2}{\DEFadditiveApprox_2} x \Rightarrow y \alphaBetaPreferredParams{\DEFmultApprox_1}{\DEFadditiveApprox_1} x
     \end{align*}
\end{observation}
% \begin{proof}
%     Assume that $y \xPreferred{\gamma} x$.
%     By definition this means that there exists an integer $k\in [n]$ such that:
%     \begin{align*}
%         \forall j < k \colon \quad &\valBy{j}{y} \geq \valBy{j}{x}\\
%         &\valBy{k}{y} > \gamma \cdot \valBy{k}{x} 
%     \end{align*}
%     However, since $\gamma \geq \delta$, this $k$ also implies that $y \alphaBetaPreferred x$ as $\valBy{k}{y} > \gamma \cdot \valBy{k}{x} \geq \delta \cdot \valBy{k}{x}$, and for $j<k$ it is the same as for $\gamma$. 
% \end{proof}
One can easily verify that it follows directly from the definition as
% $\DEFmultApprox_1 \geq \DEFmultApprox_2$ and therefore,
$\frac{1}{\DEFmultApprox_2} \geq \frac{1}{\DEFmultApprox_1}$. 
% Lemma \ref{lemma:approx-relation-prop2} connects the different relations generated by different parameters, that is, the relations $\xPreferred{\gamma}$ and $\alphaBetaPreferred$, arising from $\gamma \geq \delta \geq 1$.
Accordingly, by considering the relation sets $\relationSetParams{\DEFmultApprox_1}{\DEFadditiveApprox_1}$ and $\relationSetParams{\DEFmultApprox_2}{\DEFadditiveApprox_2}$, we can conclude that $\relationSetParams{\DEFmultApprox_2}{\DEFadditiveApprox_2} \subseteq \relationSetParams{\DEFmultApprox_1}{\DEFadditiveApprox_1}$.
This means that as the \emph{error} parameters $\DEFmultError$ and $\DEFadditiveApprox$ increase,
the relation becomes \emph{more partial}:
when $\DEFmultError = 0$ and $\DEFadditiveApprox = 0$ it is a total order, any two elements that yield different utilities appear as a pair in $\relationSetParams{1}{0}$; but as they increase, the set $\relationSetAlphaBeta$ potentially becomes smaller, as fewer pairs are comparable.

To illustrate, consider the following example with three solutions $x,y,z$ and sorted utility vectors $u_x=(1,10,15), u_y =(1,40,60), u_z=(2,20,30)$.
It is easy to see that the maximum element according to the leximin order is $z$ and that $\relationSetParams{1}{0} = \{(z,x),(z,y),(y,x)\}$.
However, the relation set either stays the same or becomes smaller as $\DEFmultError$ increases (and the approximation factor $\DEFmultApprox$ decreases), for example $\relationSetParams{0.75}{0} = \{(z,x),(z,y),(y,x)\}$,
$\relationSetParams{0.5}{0} = \{(y,x)\}$, and $\relationSetParams{0.25}{0} = \emptyset$.
The same applies when $\DEFadditiveError$ increases, for example $\relationSetParams{1}{1} = \{(z,x),(z,y),(y,x)\}$, $\relationSetParams{1}{15} = \{(y,x)\}$ and $\relationSetParams{1}{45} = \emptyset$.
Similarly, when both $\DEFmultError$ and $\DEFadditiveError$ increase, for example $\relationSetParams{0.9}{0.5} = \{(z,x),(z,y),(y,x)\}$, $\relationSetParams{0.75}{1} = \{(y,x)\}$ and $\relationSetParams{0.75}{20} = \emptyset$.
% \eden{for us for checking my calculation:
% \begin{align*}
%     &\relationSetParams{0.25}{0} = \{(z,x),(z,y),(y,x)\} \Hquad \frac{1}{1-0.25} \approx 1.33\\
%     & \quad (z,x), (z,y) \text{ since } 2 > \frac{1}{1-0.25}1 \approx 1.33, \Hquad (y,x)  \text{ since } 40 > \frac{1}{1-0.25}10 \approx 13.33\\
%     &\relationSetParams{0.5}{0} = \{(y,x)\}  \Hquad \frac{1}{1-0.5} = 2\\
%     & \quad (y,x) \text{ since } 40 > \frac{1}{1-0.5}10\approx 1.33, \Hquad \text{NOT }(z,x), (z,y)  \text{ since the ratio is} <2\\
%      &\relationSetParams{0.75}{0} = \emptyset,  \Hquad \frac{1}{1-0.75} = 4, \Hquad \text{ the ratio is always} \leq 4\\ 
%      &\relationSetParams{0}{1} = \{(z,x),(z,y),(y,x)\}\\
%      &\relationSetParams{0}{15} = \{(y,x)\}\\
%      &\relationSetParams{0}{45} = \emptyset,\Hquad \text{ the difference is always} \leq 45\\ \\
%      &\relationSetParams{0.1}{0.5} = \{(z,x),(z,y),(y,x)\}\\
%      &\relationSetParams{0.25}{1} = \{(y,x)\}\\
%      &\relationSetParams{0.25}{20} = \emptyset
% \end{align*}
% }


The leximin approximation can now be defined.

\paragraph{Leximin approximation}
% \eden{should we say again that it is a generalization of Henzinger et al?}
We say that a solution $x\in S$ is an \emph{$(\DEFmultApprox,\DEFadditiveError)$-approximately-optimal} if it is a maximum element of the order $\alphaBetaPreferred$ in $S$ for $\DEFmultApprox\in (0,1]$ and $\DEFadditiveError \geq 0$.
That is, there is \emph{no} solution in $S$ that is $(\DEFmultApprox,\DEFadditiveError)$-leximin-preferred over it --- $y \nAlphaBetaPreferred x$ for any $y\in S$.


% \eden{==== stopped here}
This definition has some important properties.
Lemma \ref{lemma:absence-of-errors} proves that in the absence of errors ($\DEFmultError = \DEFadditiveError = 0$) it is equivalent to the exact leximin optimal definition. 
Then, Lemma \ref{lemma:beta1-beta2-approx} shows that a $(\DEFmultApprox_1,\DEFadditiveError_1)$-approximation is also a $(\DEFmultApprox_2,\DEFadditiveError_2)$-approximation when $0 \leq \DEFmultErrorOf{\DEFmultApprox_1} \leq  \DEFmultErrorOf{\DEFmultApprox_2} < 1$ and $0 \leq \DEFadditiveError_1 \leq \DEFadditiveError_2$.
Finally, Lemma \ref{lemma:exact-is-always-optimal} proves that a leximin optimal solution is always approximately-optimal (for any error factors).
% And third, the definition preserves the leximin nature according to which a solution that hurts the poorest is never preferred.

\begin{lemma}\label{lemma:absence-of-errors}
 In the absence of errors, $\DEFmultError = \DEFadditiveError = 0$, a solution is approximately-leximin-optimal if and only if it is leximin-optimal.
\end{lemma}
% \eden{should it be a corollary?}
\begin{proof}
    % We will show that $x^*$ is a leximin optimal solution if and only if it is approximately-optimal for $\DEFmultError = 0$.
    % First, assume that $x^*$ is a leximin optimal solution. 
    % From Lemma \ref{lemma:exact-is-always-optimal} it is also approximately-optimal for any $\DEFmultError \in [0,1)$, in particular for  $\DEFmultError = 0$.
    % Now, assume that $x^*$ is approximately-optimal for $\DEFmultError = 0$.
    % By definition, $x \nxPreferred{1} x^*$ for any solution $x \in S$. 
    % By Observation \ref{obs:approx-relation-prop2} we can conclude that $x \nLeximinPreferred x^*$ and therefore it is 
    %
    % The claim follows almost directly from Lemma \ref{lemma:approx-relation-prop1}, which implies that $y \nLeximinPreferred x \iff y \nAlphaBetaPreferredParams{0}{0} x$.
    By definition, a solution $x^*$ is approximately-optimal for $\DEFmultError = \DEFadditiveError = 0$ if and only if $x \nAlphaBetaPreferredParams{1}{0} x^*$ for any solution $x \in S$.
    This holds if and only if $x \nLeximinPreferred x^*$ for any solution $x \in S$ (by Lemma \ref{lemma:approx-relation-prop1}).
    This means, by definition, that $x^*$ is a leximin-optimal solution.
\end{proof}




\begin{lemma}\label{lemma:beta1-beta2-approx}
    Let $x \in S$, $0 \leq \DEFmultErrorOf{\DEFmultApprox_1} \leq  \DEFmultErrorOf{\DEFmultApprox_2} < 1$, and $0 \leq \DEFadditiveError_1 \leq \DEFadditiveError_2$. If $x$ is $(\DEFmultApprox_1,\DEFadditiveError_1)$-approximately-optimal then it is also $(\DEFmultApprox_2,\DEFadditiveError_2)$-approximately-optimal.
\end{lemma}

\begin{proof}
    Assume that $x$ is $(\DEFmultApprox_1,\DEFadditiveError_1)$-approximately-optimal.
    By definition, $y \nAlphaBetaPreferredParams{\DEFmultApprox_1}{\DEFadditiveError_1} x$ for any solution $y \in S$.
    Observation \ref{obs:approx-relation-prop2} implies
        % \footnote{Observation \ref{obs:approx-relation-prop2} says that $y \alphaBetaPreferredParams{\DEFmultError_2}{\DEFadditiveError_2} x \Rightarrow y \alphaBetaPreferredParams{\DEFmultError_1}{\DEFadditiveError_1} x$, which implies that $y \nAlphaBetaPreferredParams{\DEFmultError_1}{\DEFadditiveError_1} x \Rightarrow y \nAlphaBetaPreferredParams{\DEFmultError_2}{\DEFadditiveError_2} x$.} 
        that
        % $y \nAlphaBetaPreferredParams{\DEFmultError_1}{\DEFadditiveError_1} x \Rightarrow 
        $y \nAlphaBetaPreferredParams{\DEFmultApprox_2}{\DEFadditiveError_2} x$ 
    % Therefore, $y \nAlphaBetaPreferredParams{\DEFmultError_2}{\DEFadditiveError_2} x$ 
    for any solution $y \in S$. This means, by definition, that $x$ is $(\DEFmultApprox_2,\DEFadditiveError_2)$-approximately-optimal.
\end{proof}


\begin{lemma}\label{lemma:exact-is-always-optimal}
    Let $x^* \in S$ be a leximin optimal solution. Then $x^*$ is also $(\DEFmultApprox,\DEFadditiveError)$-approximately-optimal for any $\DEFmultError \in [0,1)$  and $\DEFadditiveError \geq 0$.
\end{lemma}

% \eden{maybe we should say somewhere that for brevity when we say "any other solution" we mean any solution that has different sorted utility vector; where is the right place to write it?}
\begin{proof}
    % Let $\DEFmultError \in [0,1)$.
    By Lemma \ref{lemma:absence-of-errors}, the solution $x^*$ is also approximately-optimal for $\DEFmultError = \DEFadditiveError = 0$.
    But this means, according to Lemma \ref{lemma:beta1-beta2-approx}, that $x^*$ is also $(\DEFmultApprox_2,\DEFadditiveError_2)$-approximately-optimal for any $0 \leq \DEFmultErrorOf{\DEFmultApprox_2} < 1$ and $\DEFadditiveError_2 \geq 0$.
    %
    % definition of a leximin optimal solution,  $x \nLeximinPreferred x^*$ for any solution $x \in S$.
    % However, as $\frac{1}{1-\DEFmultError} \geq 1$, Observation \ref{obs:approx-relation-prop2} implies\footnote{Observation \ref{obs:approx-relation-prop2} says that $y \xPreferred{\gamma} x \Rightarrow y \alphaBetaPreferred x$ for $\gamma \geq \delta \geq 1$, which implies that $y \nDeltaPreferred x \Rightarrow y \nxPreferred{\gamma} x$} that $x \nLeximinPreferred x^* \Rightarrow x \nxPreferred{\frac{1}{1-\DEFmultError}} x^*$.
    % Therefore, there is no solution that is  $\frac{1}{1-\DEFmultError}$-preferred over $x^*$ and so, by definition, $x^*$ is also $(1-\DEFmultError)$ approximately-optimal.
\end{proof}


Using the example given previously, we shall now demonstrate that as the error parameters $\DEFmultError$ and $\DEFadditiveError$ increase, the quality of the \emph{approximation} decreases.
Given solutions $x,y,z$ with sorted utility vectors $u_x=(1,10,15), u_y =(1,40,60), u_z=(2,20,30)$, we saw that $\relationSetParams{1}{0} = \{(z,x),(z,y),(y,x)\}$. 
In this case, the only solution over which no solution is preferred is $z$.
Therefore, when $\DEFmultError = \DEFadditiveError = 0$, the only approximately-optimal solution is $z$ which is also the only leximin optimal one.
We also saw that $\relationSetParams{0.75}{0} = \relationSetParams{1}{1} = \relationSetParams{0.9}{0.5} = \{(z,x),(z,y),(y,x)\}$; here, similarly, the only approximately-optimal solution for these parameters is $z$.
However, $\relationSetParams{0.5}{0} =\relationSetParams{1}{15} = \relationSetParams{0.75}{1} = \{(y,x)\}$. 
According to the relation set $\{(y,x)\}$, both $z$ and $y$ are solutions over which no solution is preferred, and therefore, they are both approximately optimal for these parameters.
Lastly, as $\relationSetParams{0.25}{0} =\relationSetParams{1}{45} = \relationSetParams{0.75}{20} = \emptyset$, \emph{all} three solutions are approximately-optimal for these parameters.



% let $\gamma \geq \delta \geq 1$ if $x \succ_{\gamma} y$ then also $x \succ_{\delta} y$.
% \eden{it is easy to see that $x \succ_{\delta} y \Rightarrow x \succ_{\delta'} y$ for $\delta \geq \delta'$}
% \eden{therefore, we can also notice the following relation: $x \succ_{\delta} y \Rightarrow x \succ y$, which also implies $x \nsucc y \Rightarrow x \nsucc_{\delta} y$}
% To illustrate, l
% In particular, this implies that for any $\delta > 1$ 



% \paragraph{Characteristics}
% \eden{need to rewrite}
% \begin{itemize}
%     \item In the absence of errors ($\DEFmultError = 1$) the approximate definition is identical to the exact definition.

%     \item For any $\DEFmultError$ the Leximin optimal solution is approximately optimal as well.

%     \item Preserves the Leximin nature/semantics.
    
% \end{itemize}
\pdfobjcompresslevel=0

%\newif\ifmargincomments %A quick way of turning off margin comments for, say, arXiv submission
%\margincommentstrue

%\ifmargincomments
%\newcommand{\fpmargin}[2]{{\color{blue}#1}\marginpar{\color{blue}\raggedright\footnotesize [FP]:#2}}
%\newcommand{\msmargin}[2]{{\color{red}#1}\marginpar{\color{red}\raggedright\footnotesize [MaS]:\\ #2}}
%\else
%\newcommand{\fpmargin}[2]{#1}
%\newcommand{\msmargin}[2]{#1}
%\fi

\begin{document}

\title{\LARGE \bf A Time-invariant Network Flow Model\\ for Two-person Ride-pooling Mobility-on-Demand
%	Joint Optimization of Fleet Size, Battery Capacity and Operations of an Electric Autonomous Mobility-on-Demand System
%	Joint Design and Operation of an Electric Autonomous Mobility-on-Demand System
}


\author{Fabio Paparella, Leonardo Pedroso, Theo Hofman, Mauro Salazar % <-this % stops a space
%\thanks{Eindhoven University of Technology}% <-this % stops a space
\thanks{Mechanical Engineering, Control System Techonology, Eindhoven University of Technology, PO Box 513 5600 MB Eindhoven, The Netherlands
        {\tt\small \{f.paparella,l.pedroso,t.hofman,m.r.u.salazar\} @tue.nl}}%
%\thanks{P. Misra is with the Department of Electrical Engineering, Wright State University,
      %  Dayton, OH 45435, USA
       % {\tt\small pmisra@cs.wright.edu}}%
}

%\author{\IEEEauthorblockN{1\textsuperscript{st} Fabio Paparella}
%\IEEEauthorblockA{\textit{Mechanical Engineering} \\
%\textit{Eindhoven University of Techonology}\\
%Eindhoven, Netherlands \\
%f.paparella@tue.nl}
%\and
%\IEEEauthorblockN{2\textsuperscript{nd} Theo Hofman}
%\IEEEauthorblockA{\textit{Mechanical Engineering} \\
%\textit{Eindhoven University of Techonology}\\
%Eindhoven, Netherlands \\
%t.hofman@tue.nl}
%\and
%\IEEEauthorblockN{3\textsuperscript{rd} Mauro Salazar}
%\IEEEauthorblockA{\textit{Mechanical Engineering} \\
%\textit{Eindhoven University of Techonology}\\
%Eindhoven, Netherlands \\
%m.r.u.salazar@tue.nl}
%}	

\maketitle
\thispagestyle{empty}
\pagestyle{empty}

\begin{abstract}
%\msmargin{Urban mobility is currently being pervaded by mobility-on-demand services providing ride-pooling options without the need of a private vehicle.}{Do we even need this sentence?}
This paper presents a time-invariant network flow model capturing two-person ride-pooling that can be integrated within design and planning frameworks for Mobility-on-Demand systems.
In these type of models, the arrival process of travel requests is described by a Poisson process, meaning that there is only statistical insight into request times, including the probability that two requests may be pooled together.
Taking advantage of this feature, we devise a method to capture ride-pooling from a stochastic mesoscopic perspective.
This way, we are able to transform the original set of requests into an equivalent set including pooled ones which can be integrated within standard network flow problems, which in turn can be efficiently solved with off-the-shelf LP solvers for a given ride-pooling request assignment.
Thereby, to compute such an assignment, we devise a polynomial-time algorithm that is optimal w.r.t.\ an approximated version of the problem.
Finally, we perform a case study of Sioux Falls, South Dakota, USA, where we quantify the effects that waiting time and experienced delay have on the vehicle-hours traveled.
Our results suggest that the higher the demands per unit time, the lower the waiting time and delay experienced by users. In addition, for a sufficiently large number of demands per unit time, with a maximum waiting time and experienced delay of 5 minutes, more than 90\% of the requests can be pooled. 
\end{abstract}

%\begin{IEEEkeywords}
%Mobility-as-a-Service, Smart Mobility, Autonomous Mobility-on-Demand
%\end{IEEEkeywords}
\section{Introduction}
In \cite{bondal2001reconstruction}, Bondal and Orlov showed that if $X$ is a smooth projective variety over $\mathbb{C}$ with ample (anti-)canonical bundle then its bounded derived category $\der$ completely recovers the space. More precisely, they showed that
\begin{thm}\cite[Theorem 2.5]{bondal2001reconstruction}\label{thm:bonorvreconstruction}
Let X be an irreducible smooth projective variety with ample (anti-)canonical bundle. If $\der\simeq D^{b}(Y)$ for some other smooth algebraic variety Y, then $X \cong Y$.
\end{thm}
This theorem came in contrast with the discovery by Mukai (\cite{mukai1987fourier}) that for an abelian variety $A$, there exists an equivalence as triangulated categories $D^{b}(A)\simeq D^{b}(\hat{A})$ between the bounded derived category of $A$ and the bounded derived category of its dual $\hat{A}$. \\
This observation sparked the study of what is now called Fourier-Mukai partners of a given variety $X$, those varieties which are triangulated equivalent to the bounded derived category of $X$. \\
Bondal and Orlov's reconstruction pointed out that a (birational) geometric condition on the variety can introduce some control on these derived equivalences and with this in mind Kawamata generalized this theorem for varieties with big (anti-)canonical bundle clarifying from a geometric point of view what is the role of this condition on the possible equivalence of derived categories. Namely he showed:
\begin{thm}\cite[Theorem 1.4]{kawamata2002d}
Let $X,Y$ be smooth projective varieties such that there is an equivalence \[\mathcal{F}:D^{b}(X)\overset{\simeq}{\longrightarrow} D^{b}(Y)\] as triangulated categories, then
\begin{enumerate}
    \item dim X = dim Y.
    \item If the canonical divisor $K_{X}$ is nef, so is $K_{Y}$ and there is an equality in the numerical Kodaira dimensions $\nu(X)$ and $\nu(Y)$. 
    \item If X is of general type, then X and Y are birational and furthermore, there is a smooth projective variety $p:Z\to X$, $q:Z\to Y$ such that $p^{\ast}K_{X} \simeq q^{\ast}K_{Y}$.
\end{enumerate}
\end{thm}
This theorem should be understood as a strong indication of a relationship between the birational geometry of a variety and its derived category. \\
On the other hand, Balmer showed in \cite{balmer2002presheaves,bondal2001reconstruction} that when equipped with the derived tensor product $\dtee$, the derived category of perfect complexes $Perf(X)$ of any coherent scheme $X$ can recover the space $X$ by what is now known as the Balmer spectrum $Spc(Perf(X),\dtee)$. The Balmer spectrum can be constructed for a general tensor triangulated category, a triangulated category equipped with a compatible monoidal structure, and produce a locally ringed space. \\
The existence of non isomorphic Fourier-Mukai partners $Y$ for a smooth variety $X$ implies using the Balmer spectrum construction that the bounded derived category $D^{b}(X)$ can be equipped with at least as many tensor triangulated category structures as non-isomorphic Fourier-Mukai partners, up to monoidal equivalence. \\
In other words, if $FM(X)$ is the set of isomorphism classes of Fourier-Mukai partners of $X$ and $TTS(X)$ is the set of equivalence classes of tensor triangulated category structures on the bounded derived category $D^{b}(X)$ there exists an injection
\begin{align*}
FM(X)&\rightarrow TTS(X)\\
Y&\mapsto (\otimes_{Y}^{\mathbb{L}}, \Ox_{Y})
\end{align*}
Where the pair $(\otimes_{Y}^{\mathbb{L}}, \Ox_{Y})$ denotes the tensor triangulated category structure given by the derived tensor product $\otimes_{Y}^{\mathbb{L}}$ with unit $\Ox_{Y}$. \\
Our main interest in this work is the study of this function, its surjectivity and the properties that one can deduce about possible tensor triangulated category structures outside of the image of this injection, all under the condition that the (anti-)canonical bundle of $X$ is big. \\
In Section~\ref{sec2} we give a brief general overview of the results we will need about general derived categories of quasi-coherent sheaves on a smooth projective variety, together with a reminder of the Balmer spectrum construction through Thomason's classification theorem. \\
In Section\ref{section3}, given a tensor triangulated category structure $(\der, \boxtimes,\mathbbm{1})$ with unit $\mathbbm{1}$ on a bounded derived category $\der$, we introduce the notion of almost spanning class with respect to a thick subcategory $I$ (Definition \ref{defn:almostspanning}) and we show (Theorem \ref{thm:almostspanning}) that if $X$ is a smooth projective variety of general type then there exists a proper tensor ideal $I_{X^{\ast}}$ of $(\der,\dtee,\Ox_{X})$ such that the set of tensor powers of $\omega_{X}$ forms an almost spanning sequence with respect to this ideal $I_{X^{\ast}}$.
This result is meant to highlight the more general behavior of almost spanning classes through the use of Thomason's classification theorem and properties of the Balmer spectrum. We see that this collection of objects can be used to prove the following theorem:
\begin{lemma}(Lemma \ref{lemma:picardlemma})
Suppose $X$ is a smooth projective variety of general type. If $\boxtimes$ is a tensor triangulated structure on $D^{b}(X)$ with unit $\Ox_{X}$, and $U$ is a $\boxtimes$-invertible object  such that $U\boxtimes I_{X^{\ast}} \subseteq I_{X^{\ast}}$. Then there is a natural equivalence between the functors induced by $U\boxtimes \_ $ and $U\dtee\_$ in $D^{b}(X)/I_{X^{\ast}}$.
\end{lemma}
When the $\dtee$-tensor ideal $I_{X^{\ast}}$ is also a $\boxtimes$-tensor ideal for a tensor triangulated category structure as described in the previous lemma, then we obtain that the Picard group of $\boxtimes$-invertible objects is a subgroup of the Picard group of $\dtee$-invertible objects (Corollary \ref{cor:bigsubideal}). This hypothesis holds true in particular when the (anti-)canonical bundle of $X$ is ample. \\
With this observation, our main corollary is the following monoidal version of the Bondal-Orlov reconstruction theorem:
\begin{cor}(Corollary \ref{cor:monoidalbondalorlov})
Let $X$ be a smooth projective variety with ample (anti-)canonical bundle, then if $\omega_{X}[n]$ is an invertible object for a tensor triangulated structure $\boxtimes$ on $\der$ with unit $\Ox_{X}$ then $\boxtimes$ and $\dtee$ coincide on objects.
\end{cor}
The results in this work were obtained as part of the author's PhD thesis at the Laboratoire J.A. Dieudonné at the Université Côte d'Azur. The author would like to thank his advisor Carlos Simpson for many discussions and to Ivo Dell'Ambrogio and Bertrand Toën for their careful and valuable comments on the thesis manuscript. The PhD thesis was partially financed by the CONACyT-Gobierno Francés 2018 doctoral scholarship. 

% !TeX spellcheck = en_US
\section{Ride-pooling Network Flow Model}\label{sec:model}
In this section, we first introduce the standard network   traffic flow model~\cite{PavoneSmithEtAl2012}. Then, we extend the formulation to take into account ride-pooling, and finally present a brief discussion on the model.
\subsection{Time-invariant Network Flow Model} 
We model the mobility system as a multi-commodity network flow model, similar to the approaches of~\cite{SalazarLanzettiEtAl2019,LukeSalazarEtAl2021,PaparellaChauhanEtAl2023,RossiIglesiasEtAl2018b,PaparellaSripanhaEtAl2023}. The transportation network is a directed graph $\mathcal{G} = (\mathcal{V}, \mathcal{A})$. {It consists of a set of vertices $\mathcal{V} := \{1,2,...,\abs{\mathcal{V}}\}$,} representing the location of intersections on the road network, and {a set of arcs $\mathcal{A} \subseteq \mathcal{V} \times \mathcal{V}$}, representing the road links between {intersections}. We indicate ${B \in \{-1,0,1\}^{\abs{\cV} \times \abs{\cA}}}$ as the incidence matrix~\cite{Bullo2018} of the road network $\mathcal{G}$. Consider an arbitrary arc indexing of natural numbers $\{1,\ldots,\abs{\cA}\}$, then $B_{ip} = -1$ if the arc indexed by $p$ is directed towards vertex $i$, $B_{ip} = 1$ if the arc indexed by $p$ leaves vertex $i$,  and $B_{ip}=0$ otherwise. We denote $t$ as the vector whose entries are the travel time $t_{a}$ required to traverse each arc $a\in \cA$, ordered in accordance with the arc ordering of $B$, which we assume to be constant. Similarly to \cite{PavoneSmithEtAl2012}, we define travel requests as follows:
\begin{definition}[Requests]
	A travel request is defined as the tuple $r = (o,d,\alpha) \in \mathcal{V} \times \mathcal{V} \times \mathbb{R}_{>0}$, in which $\alpha$ is the number of users traveling from the origin $o$ to the destination $d \neq o$ per unit time.  Define the set of requests as $\mathcal{R} := \{r_m\}_{m\in \mathcal{M}}$, where $\mathcal{M} = \{1,\ldots,M\}$.
\end{definition} 

We assume, without any loss of generality, that the origin-destination pairs of the requests $r_m \in \mathcal{R}$ are distinct. 
In this paper, we distinguish between active vehicle flows, which correspond to the flows of vehicles serving users whether they are ride-pooling or not, and rebalancing flows which correspond to the flows of empty vehicles between the drop-off and pick-up vertices of consecutive requests. We define the active vehicle flow induced by all the requests that share the same origin $i \in {\mathcal{V}}$ as vector $x^{i}$, where element $x^{i}_{a}$ is the flow on arc $a \in \cA$, ordered in accordance with the arc ordering of $B$. The overall active vehicle flow is a {matrix $X \in \mathbb{R}^{\abs{\cA} \times \abs{\cV}}$ defined as $X := \left[x^{1}\  x^2 \, \dots \,x^{\abs{\cV}}\right]$}. The rebalancing flow across the arcs is denoted by $x^{\mathrm{r}} \in \mathbb{R}^{\abs{\cA}}$. In the following, we define the network flow problem.
\begin{prob}[Multi-commodity Network Flow Problem]\label{prob:main}
	Given a road graph $\cG$ and a demand matrix {$D$}, the active vehicle flows $X$ and rebalancing flow $x^\mathrm{r}$ that minimize the cost in terms of overall travel time result from
	\begin{equation*}
		\begin{aligned}
			\min_{X}\; &J(X) =t^\top ( X \mathds{1} + x^\mathrm{r} )  \\
			\mathrm{s.t. }\; & BX = D \\
			&B ( X \mathds{1}+ x^\mathrm{r} )=0 \\
			& X, x^\mathrm{r} \geq 0,
		\end{aligned}
	\end{equation*}
	where the demand matrix $D \in \mathbb{R}^{\abs{\mathcal{V}} \times \abs{\mathcal{V}}}$ represents the requests between every pair of vertices, whose entries are
	\begin{equation}\label{eq:def_D}
		\!\!\!\!D_{ij} = \begin{cases}
			\alpha_m, &  \exists m \in \mathcal{M} : o_m = j \land d_m = i\\
			-\sum_{k\neq j} D_{kj}, & i  = j \\ %\sum \limits_{\substack{k = 1 \\ k\neq i}}^{\abs{\mathcal{V}}} D_{ik}
			0, &   \mathrm{otherwise}.%i\neq j \land \nexists m \in \mathcal{M} : o_m = i \land d_m = j\\
		\end{cases}\!\!\!
	\end{equation}
\end{prob}
Since Problem~\ref{prob:main} is totally unimodular, $X$ and $x^\mathrm{r}$ can be decoupled and computed separately~\cite{Rossi2018}. The objective function can also be interpreted as the minimum fleet size required to implement the flows~\cite{PavoneSmithEtAl2012}. % represents both the minimum fleet size to implement the flows and the overall travel time of vehicles~\cite{PavoneSmithEtAl2012}.

We utilise a minimal extension of the SM, with the inclusion of three right-handed neutrinos, which are singlets of the SM gauge group: $\textrm{SU}(3)_c \times \textrm{SU}(2)_L \times \textrm{U}(1)_Y$, and have lepton number $\textrm{L} = 1$. Given this additional particle content and quantum number assignment, the SM is extended through the additional Lagrangian terms
\begin{align}
    \mathcal{L}_{\nu_R} =i\overline{\nu}_R\slashed{\partial}\nu_R -\left( \overline{L}\,\h^\nu\tilde{\Phi}\,\nu_R + \frac{1}{2}\overline{\nu}^C_R\,\m_M\nu_R + {\rm H.c.\, }\right)\label{eq:seesaw_lagrangian}.
\end{align}
Here, $L_i = \left(\nu_{L, i}, e_{L, i} \right)^{\sf T}$, with $i=1,2,3$, are left-handed lepton doublets; $\nu_{R, \alpha}$, with $\alpha=1,2,3$, are right-handed neutrino singlet fields; and $\tilde{\Phi}$ is the isospin conjugate Higgs doublet. The matrices $\mathbf{h}_{i\alpha}^\nu$ and $(\mathbf{m}_M)_{\alpha\beta}$ are the neutrino Yukawa couplings and Majorana mass matrix, respectively. It is worth pointing out that the inclusion of the Majorana mass matrix explicitly breaks lepton number conservation by two units, $\Delta L = 2$, satisfying one of the three Sakharov conditions for the generations of appreciable lepton asymmetry.

Without loss of generality, we may select a basis for the singlet neutrino sector such that the Majorana mass term is diagonalised, \textit{i.e} $\m_M = \textrm{diag}(m_{N_1}, m_{N_2}, m_{N_3})$. In this basis, the Lagrangian in the unbroken phase takes the form
\begin{align}
\mathcal{L}_{\nu_R} =i\overline{N}\slashed{\partial}N-\left( \overline{L}\,\h^\nu\tilde{\Phi}\,P_R N + {\rm H.c.}\right) - \frac{1}{2}\overline{N}\,\m_MN.\label{eq:seesaw_lagrangian_mass}
\end{align}
In this expression, $N_\alpha = \nu_{R, \alpha} + \nu_{R, \alpha}^C$, and $P_{R/L} = \frac{1}{2}\left(\mathds{1}_4 \pm \gamma^5 \right)$ are right/left-chiral projection operators.

In the broken phase, the addition of a Dirac mass term from the Yukawa sector results in the mixing between left- and right-chiral neutrinos, with the mass basis in the broken phase a particular combination of left- and right-chiral neutrinos
\begin{align}
P_R \begin{pmatrix}
\nu \\
N
\end{pmatrix}=
\begin{pmatrix}
U_{\nu\nu_L^C} & U_{\nu \nu_R}\\
U_{N\nu_L^C} & U_{N \nu_R}
\end{pmatrix}
\begin{pmatrix}
\nu_L^C\\
\nu_R
\end{pmatrix}\;.
\end{align}
In the above, we have defined $\nu_i$ as light neutrino mass eigenstates and $N_i$ as heavy neutrino mass eigenstates. Furthermore, the unitary matrix, $U$, diagonalises the full neutrino mass matrix. To leading order in the quantity $\xi_{i\alpha} = (\mathbf{m}_D \mathbf{m}_M^{-1})_{i\alpha}$~\cite{Pilaftsis:1991ug}, the light neutrino mass matrix may be written as
\begin{align}
\m^\nu = -\m_D \m^{-1}_M \m^{\sf T}_D \; ,\label{eq:tree_level_mass}
\end{align}
with $\m_D = \mathbf{h}^\nu v /\sqrt{2}$ the Dirac mass matrix, with Higgs VEV, $v \simeq 246\; \GeV$~\cite{GellMann:1980vs}. By virtue of this relation, it is clear that to satisfy observed neutrino data, the Majorana mass matrix would have to be GUT scale if the Dirac matrix is of electroweak scale ($|\!|\m_D|\!| \sim v$), and of general structure. As a result, the impact of singlet neutrinos on experimental signatures would be minimal as the charged current interactions are suppressed through the mixing parameter $B_{i\alpha} = \xi_{i\alpha}$~\cite{Pilaftsis:1991ug}
\begin{align}
\mathcal{L}^W_{\rm int} = -\frac{g_w}{\sqrt{2}}W^-_\mu
  \overline{e}_{iL} B_{i\alpha}\gamma^\mu P_L N_\alpha + {\rm H.c.}\,.
\end{align}
Consequently, there is a motivation to identify models which allow for low-scale heavy neutrino masses whilst remaining in alignment with the observed neutrino data.

One approach which may be taken to address this problem is to assume the existence of a symmetry on the flavour structure of the Yukawa couplings, $\mathbf{h}_0^\nu$, which would render the light neutrino eigenstates massless
\begin{equation}
    -\m_D \m^{-1}_M \m^{\sf T}_D = -\frac{v^2}{2}\mathbf{h}_0^\nu \m^{-1}_M (\mathbf{h}_0^\nu)^{\sf T} = \mathbf{0}_3.\label{eq:zeromassconstraint}
\end{equation}
From this, small neutrino masses may be generated through perturbations about the symmetric Yukawa couplings
\begin{align}
    (\h^\nu_0 + \delta\h^\nu)\,\m_M^{-1}\,(\h^\nu_0 + \delta\h^\nu)^{\sf
  T}\, =\, \frac{2}{v^2}\,\m^\nu\;. 
    \label{eq:numassconstraint}
\end{align}

In the case of a near degenerate heavy neutrino mass spectrum, the condition on the symmetric Yukawa couplings given in (\ref{eq:zeromassconstraint}) may be approximately satisfied by
\begin{equation}
    \mathbf{h}_0^\nu (\mathbf{h}_0^\nu)^{\sf T} =\mathbf{0}_3.\label{eq:NilPotent}
\end{equation}
This motivates a nil-potent structure of the Yukawa matrix. In particular, we have identified the structure
\begin{align}
    \h^\nu_0=\begin{pmatrix}
        a & a\,\omega & a\,\omega^2\\
        b & b\,\omega & b\,\omega^2\\
        c & c\,\omega & c\,\omega^2
    \end{pmatrix}\;,
    \label{eq:z6yukawa}
\end{align}
with $a, \, b, \,c \in \mathbb{C}$, and $\omega = \exp\left( \frac{2\pi i}{6} \right)$ the generator of the $\mathbb{Z}_6$ group. This structure is not unique in satisfying the constraint given in (\ref{eq:NilPotent}). Other similar structures, such as $\mathbb{Z}_3$, with generators $\omega^\prime = \exp\left( \frac{2\pi i}{3} \right)$ would also produce a vanishing light neutrino mass spectrum at leading order. Most interestingly, this symmetry-motivated structure offers large CP-violating phases which contribute significantly to the generation of appreciable BAU.
Instead, this possibility is not easily achievable in bi-resonant models, where the CP-odd phases are strongly correlated to the light-neutrino masses.

As a further insight, since this symmetry exists within the flavour structure, any additional contributions to the light neutrino mass matrix with an identical flavour structure will vanish. In particular, the first-order loop correction to the light neutrino mass matrix~\cite{Pilaftsis:1991ug} may be incorporated into the zero mass condition of the symmetric Yukawa matrix
\begin{align}
    \frac{v^2}{2}\h^\nu_0\left[\m^{-1}_M - \frac{\alpha_w}{16\pi M^2_W}\m^{\dagger}_Mf(\m_M\m^\dagger_M)\right]\h^{\nu\sf
  T}_0=\,\mathbf{0}_3\;,
    \label{eq:one_loop_zero_mass}
\end{align}
where
\begin{align}
    f(\m_M\m^\dagger_M)=\frac{M^2_H}{\m_M\m^\dagger_M -
M^2_H\mathds{1}_3}\ln\left(\frac{\m_M\m^\dagger_M}{M^2_H}\right) +
  \frac{3M^2_Z}{\m_M\m^\dagger_M -
M^2_Z\mathds{1}_3}\ln\left(\frac{\m_M\m^\dagger_M}{M^2_Z}\right)\;. 
    \label{eq:loop_factor_f_def}
\end{align}
In the above, $\alpha_w\equiv g_w^2/(4\pi)^2$ is the electroweak gauge-coupling parameter, and $M_W$, $M_Z$, and $M_H$ are the masses of the $W$, $Z$, and Higgs bosons, respectively.
% !TeX spellcheck = en_US

\subsubsection{Temporal Analysis of Ride-pooling}\label{sec:TD}
In this section, we analyze the temporal alignment of two requests for ride-pooling.  %Specifically, we define $\bar{t}$ as the maximum waiting time since the event of a request and the event of starting to serve that request. 
We derive the probability of two requests taking place within the maximum waiting time, $\bar{t}$. As common in traffic flow models~\cite{PavoneSmithEtAl2012}, we consider that the arrival rate of a request $r_m \in \cR$ follows a Poisson process with parameter $\alpha_m$. Consider two requests $r_m, r_n \in \cR$. In the following lemma, we indicate the probability of %being possible to ride-pool those requests in a temporal perspective, i.e., the probability of 
the two events occurring within a maximum time window $\bar{t}$.
%Note that we consider that a request which can be ride-pooled starts being served when the ride-poolling configuration starts, no matter its origin node.
%%In this second part, we show the \msmargin{mathematical formulation to compute the probability}{why?} of two events (the OD pairs) happening within a maximum waiting time, $\bar{t}$. 
%
%In traffic flow models~\cite{PavoneSmithEtAl2012}, the arrival rate of a request follows a Poisson process with parameter $\alpha$.
%\msmargin{We consider two events 1 and 2 with parameters $\alpha_1,\alpha_2$. It has to be computed the probability that, within a maximum time window, one occurrence of each event happens so that they are matched together for pooling. }{unclear}
%We announce the following lemma, that we prove in Appendix 2.
\begin{lemma}\label{lemma:finalprob}
Let $r_m, r_n \in \cR$ be two requests whose arrival rate follow a Poisson process with parameters $\alpha_m$ and $\alpha_n$, respectively. The probability of each having an occurrence within a maximum time interval $\bar{t}$ is
\begin{equation}\label{eq:lem}
	P(\alpha_m,\alpha_n)  := 1-\frac{\alpha_m e^{-\alpha_n\bar{t}}+\alpha_n e^{-\alpha_m\bar{t}}}{\alpha_m+\alpha_n}.
\end{equation}
\end{lemma}
\begin{proof}
	The proof can be found in Appendix~\ref{app:1}.
\end{proof}

\subsubsection{Expected Number of Pooled Rides}%Stochastic Assignment in Ride-pooling} 
\label{sec:STF}
In Section~\ref{sec:SD}, we analyzed the spatial dimension of the ride-pooling problem, whereby we computed the best feasible pooling path given two requests. In Section~\ref{sec:TD}, we analyzed the temporal dimension of the ride-polling problem, whereby we derived the probability of two requests happening within a time window.  By lifting the temporary assumptions made in Section~\ref{sec:SD}, we formulate the ride-pooling demand matrix given a certain pooling assignment, defined in what follows.

A fraction of the demand of every request $r_m\in \cR$ can be assigned to be pooled with a request $r_n \in \cR$. Let ${\beta \in \mathbb{R}_{\geq0}^{|\cR|\times |\cR|}}$ denote the assignment matrix, whose entry $(m,n)$ is the demand of $r_m$ that is assigned to be pooled with $r_n$. %Matrix $\beta$ defines the ride-pooling assignment. 
For the remainder of this subsection we assume that $\beta$ is given. In Section~\ref{sec:alg}, we propose an algorithm to compute the optimal value of $\beta$ under Approximation~\ref{approx}.

From the analysis in Section~\ref{sec:TD}, it is noticeable that only a fraction of the allocated ride-pooling demand $\beta_{mn}$ can actually be pooled due to the aforementioned temporal constraints. Specifically, the probability of pooling is given by $P(\beta_{mn},\beta_{nm})$ according to Lemma~\ref{lemma:finalprob}. Moreover, given that we only consider pooling between two requests, at most, the maximum pooled demand between $r_m,r_n\in \cR$ is  $\min(\beta_{mn},\beta_{nm})$. Therefore, the effective expected pooled demand between two requests $r_m,r_n\in \cR$ is given by ${\gamma_{nm} = \gamma_{mn} := \min(\beta_{mn},\beta_{nm})P(\beta_{mn},\beta_{nm})}$.  As a result, according to the spatial analysis in Section~\ref{sec:SD}, this pooled demand is portrayed by the demand matrix $\gamma_{mn}D^{mn,\star}$. Note that the effective expected pooling demand follows $\sum_{n\in \cM} \gamma_{mn} \leq \alpha_m, \, \forall r_m \in \mathcal{R}$ with equality if the full demand of $r_m$ is pooled.
The full ride-pooling demand matrix $D^\mathrm{rp}$ is made up of two contributions: i)~the sum of the expected pooled active vehicle flows of the form $\gamma_{mn}D^{mn,\star}$ for ${r_m,r_n\in \cR}$; and ii)~the requested demands that were not ride-pooled. Thus, the entry $(i,j)$ of $D^\mathrm{rp}$ can be written as
\begin{equation*}\label{eq:b}
		\!D_{ij}^{\mathrm{rp}} = \begin{cases}
			\sum\limits_{\substack{p,q\in \cM\\ p\geq q}} \gamma_{pq}D_{ij}^{pq,\star} + \left(D_{ij} - \sum\limits_{p\in\cM}\gamma_{mp}D_{ij}^{mp,\star} \right), \\ 
			 \quad \quad  \quad \quad  \quad \quad  \quad \quad \;\; \exists m \in \mathcal{M} : d_m \!= i \land o_m\! = j\\
			- \sum_{k\neq j} D_{kj}^{\mathrm{rp}},  \quad \quad \quad  i  = j \\ %\sum \limits_{\substack{k = 1 \\ k\neq i}}^{\abs{\mathcal{V}}} D_{ik}
				\sum\limits_{\substack{p,q\in \cM\\ p\geq q}} \gamma_{pq}D_{ij}^{pq,\star}, \quad \quad   \text{otherwise}.%i\neq j \land \nexists m \in \mathcal{M} : o_m = i \land d_m = j\\
		\end{cases}
\end{equation*}
%In Sec.~\ref{sec:TD} we found the probability of events 1 and 2 happening within a maximum waiting time $\bar{t}$. 
%\mbox{In Sec.~\ref{sec:SD},} given two travel requests $ij$ and $kl$, \msmargin{and a maximum delay $\bar{\delta}$}{il delay lo metterei dopo}, we computed the best feasible pooling path. 
%Combining together the properties of space and time,  if pooling between $ij$ (event 1) and $kl$ (event 2) is the best feasible option, the probability of it happening is equal to \msmargin{$P_{ijkl}(\alpha'_{ijkl},\alpha'_{klij})$}{why $\alpha'$? Data $P(\alpha)$ estrai la parte compatibile, $\alpha P(\alpha)$, no?}. This probability depends on $\alpha'_{ijkl},\alpha'_{klij}$, which are the fractions of demands dedicated to ride pooling with the corresponding pair. In other words, every OD pair has a given $\alpha_{ij}$ that can partially be assigned to other request $kl$. We define $\alpha'^{\abs{ \mathcal{V}} \times \abs{\mathcal{V}} \times \abs{\mathcal{V}} \times \abs{\mathcal{V}}} $ as an assignment four-dimensional matrix, where element $\alpha'_{ijkl}$ is the amount assigned to pooling from request $ij$ to request $kl$. %This can result in either a non linear cooperative game with shared resources, or a set of non linear equations that must be added to the original traffic flow problem.
%We then define the pooling matrix, where element $\beta_{ijkl}$ is the expected number of pooled rides between $ij$ and $kl$, and is computed as \msmargin{follows}{$\min$ si scrive roman}:
%\begin{equation}\label{eq:gamma}
%	\beta_{ijkl}=\beta_{klij}= min(\alpha'_{ijkl} ,\alpha'_{klij})   P_{ijkl}(\alpha'_{ijkl},\alpha'_{klij}).
%\end{equation}
%The term $min(\alpha'_{ijkl} ,\alpha'_{klij})$ assures that the number of pooled rides $\beta_{ijkl}$ is equal to $\beta_{klij}$, because $P_{ijkl}=P_{klij}$.
%The corresponding OD matrix $b^{ijkl}$ is computed by scaling $b^{ijkl,\mathrm{best}}$  by the number of requests pooled $\beta_{ijkl}$:
%\begin{equation}
%	b^{ijkl}=\beta_{ijkl}   b^{ijkl,\mathrm{best}},
%\end{equation}
%where $b^{ijkl,\mathrm{best}}$ is the matrix of the best ride pooling option between $ij$ and $kl$, computed in Sec.\ref{sec:SD}.
%Each request $ij$ can be assigned to be pooled with every other feasible requests $kl$. However, the remaining part of the demand $\alpha'_{ijkl}-\beta_{ijkl}$ that has been assigned but not pooled, is re-assigned back to the original demands $\alpha_{ij}$ and $\alpha_{kl}$. Then, they can be allocated again to other requests. %The re-allocation cannot be assigned twice to the same $ijkl$ pool. If that would be the case, the loop would reflect in the probability of being pooled equal to 1. 
\begin{comment}
\begin{equation}
	%\alpha_{ijkl} 
	\alpha_{ij} = \sum_{kl } \alpha_{ijkl} - \sum_{kl } (\alpha_{ijkl}-\beta_{ijkl}),
\end{equation}
that can be reformulated in
\begin{equation}
		\alpha_{ij} = \sum_{kl } \alpha_{ijkl}    P_{ijkl}(\alpha_{ijkl},\alpha_{klij}).
\end{equation}
Every non pooled flow, has to be reallocated over all the other OD pairs.
 \begin{equation}
 \alpha_{ijkl} = \alpha_{ij} - \sum_{nm} \alpha_{ijnm} + \sum_{ nm} E_{ijklijnm}   (\alpha_{ijnm}-\beta_{ijnm}),
 \end{equation}
where $F_{ijklijnm}$ is the fraction of $(\alpha_{ijnm}-\beta_{ijnm})$ requests that is re-allocated to $\alpha_{ijkl}$.
\end{comment}
%The elements of the final OD matrix result from 
%\begin{equation}\label{eq:b}
%	b^\mathrm{rp}_{nm} = \sum_{i,j,k,l} b^{ijkl}_{nm} %+ (\alpha_{nmkl}-\beta_{nmkl})   b^{ijkl,\mathrm{np}}.
%	+ (\alpha_{nm} - \sum_{kl} \beta_{nmkl}).
%\end{equation}

Finally, one can input $D^\mathrm{rp}$ to Problem~\ref{prob:rides}, which yields an LP, given a pooling assignment $\beta$.
% !TeX spellcheck = en_US


\subsubsection{Optimal Ride-pooling Assignment}\label{sec:alg}
In this section, we will compute the optimal ride-pooling assignment matrices $\beta^\star$ and $\gamma^\star$, under Approximation~\ref{approx}, leveraging an iterative approach, which is described in what follows.
For every pair of requests $r_m,r_n\in \cR$, we can compute the unitary improvement of the objective function of Problem~\ref{prob:rides}, denoted by $\Delta \tilde{J}_{mn}$, w.r.t. the no-pooling scenario. Specifically, it amounts to the difference  between $\tilde{J}_{nm}$, which denotes the cost with $D^\mathrm{rp} = D^{mn,\star}$, and $\tilde{J}_n+\tilde{J}_m$, which again denotes the cost with $D^\mathrm{rp} = D^{mn,0}$. Let $\alpha_m^\prime, m\in \cM$ stand for an auxiliary variable throughout the iterations and represent the demand of request $r_m$ that has not yet been assigned, and which is initialized as $\alpha_m^\prime = \alpha_m$. Further, the pair of requests with the highest improvement is prioritized with the highest possible pooling demand assignment. That is, in each iteration, if $r_m,r_n\in \cR$  is the pair of requests with the highest $\Delta \tilde{J}_{mn}$, we set $\beta_{mn} = \alpha_m^\prime$ and $\beta_{nm}=\alpha_n^\prime$. Moreover, the rides that have been assigned but not pooled, are added back to the original requests, i.e., we set $\alpha_m^\prime =  \beta_{mn}-\gamma_{mn}$ and $\alpha_n^\prime =\beta_{nm}-\gamma_{nm}$. Let $\Delta \tilde{J}_{mn}^\prime, m,n\in \cM$ denote another auxiliary variable  throughout the iterations, initialized as $\Delta \tilde{J}_{mn}^\prime=  \Delta \tilde{J}_{mn}$. At the end of every iteration, $\Delta \tilde{J}^\prime_{mn}$ is set to  0.   
This procedure is repeated until convergence is achieved, i.e., $\max_{m,n} (\Delta \tilde{J}^\prime_{mn}) \leq 0$. The pseudocode of this procedure is presented in Algorithm~\ref{alg:one}. In the following theorem, we establish the convergence and optimality of Algorithm~\ref{alg:one}.
	
\begin{algorithm}[ht]
	\caption{Compute optimal assignment matrices $\beta^\star, \gamma^\star$.}\label{alg:one}
	\begin{algorithmic}
		\STATE $\tilde{J}_{mn} \leftarrow  \mathrm{input} \;\; D^{mn, \star} \; \text{to Problem}~\ref{prob:rides}, \;\; \forall {m,n\in \cM}$
		\STATE $\tilde{J}_{m} + \tilde{J}_n  \leftarrow  \mathrm{input} \;\; D^{mn, 0} \; \text{to Problem}~\ref{prob:rides}, \;\; \forall {{m,n\in \cM}}$
%		\STATE $ \forall m,n \; \mathrm{input} \; D^{mn,0} \; to \; Problem~\ref{prob:rides} \rightarrow \tilde{J}_{m},\tilde{J}_{n}$
		\STATE $\Delta \tilde{J}_{mn} \;\leftarrow\; \tilde{J}_{m} + \tilde{J}_{n} - \tilde{J}_{mn} $
		\STATE $\Delta \tilde{J}_{mn}^\prime \;\leftarrow\; \Delta \tilde{J}_{mn}, \; \forall m,n\in \cM$
		\STATE $\alpha_{m}^\prime \;\leftarrow\; \alpha_{m}, \; \forall m\in \cM $
%		\STATE $\;D^\mathrm{rp} = \mathbb{0}$
		\WHILE {$ \mathrm{max}_{m,n}(\Delta \tilde{J}^\prime_{mn}) > 0 $}
		\STATE $ (m,n) \in \mathrm{argmax}_{m,n} (\Delta \tilde{J}^\prime_{mn})$
		\IF{$o_n = o_m \; \mathrm{and} \; d_n=d_m$}
		\STATE $ \beta_{mn}  \;\leftarrow\; \alpha_{m}^\prime, \;\beta_{nm} \;\leftarrow\; \beta_{mn}$
		\STATE $ \gamma_{mn} \;\leftarrow\; \beta_{mn}  P(\beta_{mn},\beta_{nm})/2, \;\gamma_{nm} \;\leftarrow\; \gamma_{mn}$
%		\STATE $D^\mathrm{rp}_{ii} = b^\mathrm{rp}_{ii} - \beta_{ijkl} , \; b^\mathrm{rp}_{ij} = b^\mathrm{rp}_{ij} + \beta_{ijkl}  $
		\ELSE
		\STATE $\beta_{nm}  \;\leftarrow\;  \alpha_{n}^\prime, \; \beta_{mn}  \;\leftarrow\;  \alpha_{m}^\prime$
		\STATE $\gamma_{mn}  \;\leftarrow\;  \min(\beta_{nm},\beta_{mn}) P(\beta_{mn},\beta_{nm})$
		\STATE $\gamma_{nm} \;\leftarrow\;  \gamma_{mn}$
		\ENDIF
%		\STATE $D^\mathrm{rp} = D^\mathrm{rp} + \gamma_{mn}  D^{mn \star}$
		\STATE$\alpha_{m}^\prime  \;\leftarrow\;  \alpha_{m}^\prime - \gamma_{mn},  \; \alpha_{n}^\prime  \;\leftarrow\; \alpha_{n}^\prime - \gamma_{nm} $
		\STATE$\Delta \tilde{J}^\prime_{mn}  \;\leftarrow\; 0, \;   \Delta \tilde{J}^\prime_{nm} \leftarrow \Delta \tilde{J}^\prime _{mn}  $
		\ENDWHILE
	\end{algorithmic}
\end{algorithm}
%We define the optimal objective function of Problem~\ref{prob:rides} with parametric entry $\gamma$ as $\tilde{J}(X^\star_\gamma)$.


\begin{theorem}\label{theorem:one}
Let $X^\star_\gamma$ denote the optimal solution of Problem~\ref{prob:rides}, under Approximation~\ref{approx}, for the effective ride-pooling demand matrix $\gamma$. Then, in $\abs{\cM}(\abs{\cM}-1)$ iterations at most, Algorithm~\ref{alg:one} converges to $\beta = \beta^\star$ and $\gamma= \gamma^\star$, which is a minimizer of $\tilde{J}(X^\star_\gamma)$ among all valid effective ride-pooling matrices.
\end{theorem} 
\begin{proof}
		The proof can be found in Appendix~\ref{app:2}. 
\end{proof}
%	Algorithm~\ref{alg:one} leads to the parameter $\beta^\star$ so that \mbox{$J(X^\star_{\gamma^\star}) \leq J(X^\star{_\gamma})$}, where $\gamma = \gamma(\beta)$, and $\gamma^\star = \gamma(\beta^\star)$.
%We prove Theorem~\ref{theorem:one} in Appendix~\ref{app:2}. 



\subsection{Discussion}
A few comments are in order.
The mobility system is analyzed at steady-state, which is unsuitable for an online implementation, but it is appropriate for planning and design~\cite{LukeSalazarEtAl2021,SalazarLanzettiEtAl2019}. 
Then, Problems~\ref{prob:main} and \ref{prob:rides} allow for fractional flows, which is acceptable because of the mesoscopic perspective of the work~\cite{PaparellaChauhanEtAl2023,LukeSalazarEtAl2021,SalazarLanzettiEtAl2019}.
We consider the travel time of each arc to be constant, meaning that the routing strategies do not impact travel time and congestion. Finally, $D^\mathrm{rp}$ is not optimal w.r.t. the objective function of Problem~\ref{prob:rides}, but it is w.r.t. its relaxed version, enabling a polynomial-time computation.
\section*{Results}
We started by assembling a dataset derived from public hikes. This process included an iterative data cleaning process to remove erroneous/false data, identify and remove breaks (e.g. Fig \ref{Fig2}) to give us a final usable dataset containing 7,636 GPS tracks, with over 1.4 million individual data points and covering almost 88,000 km of travel in the U.K. 

Our curated hike dataset allowed us to create a data-driven model which we can directly compare with existing walking speed algorithms. The model formulation was selected using a small-scale exploratory study which considered data from Scotland (see \nameref{S3_Appendix}). In this exploratory study, multiple different model types were explored which could fit the data, and which matched existing knowledge about walking speeds. Cross-validation methods showed that there was very little difference in performance of the best models, therefore the final model was a Generalised Linear Model (GLM), which was chosen as it was the simplest of those tested (we had no evidence that a more complex model would be superior). This choice also meant that our model was both easy to interpret, and simple to apply to future work.

This final GLM model included all three of the variables suggested by Arnet \cite{Arnet2009ArithmeticalJapan}:

\begin{equation}
    v = exp(a+b\phi+c\theta+d\theta^2)
\end{equation}
where
\begin{quote}
$v = \text{walking speed (km/h)}$\\
$\phi = \text{hill slope angle (degrees)}$\\
$\theta = \text{walking slope angle (degrees)}$
\end{quote}

Terrain obstruction level was included as a factor variable, while we considered the road types as both factor variables and interaction terms. Not all terms had a significant effect on all variables; we therefore created a model with all possible terms, and removed them one at a time (in order of least significance) until all remaining terms were significant to at least 95\% confidence  level (using Wald test). The final values for a, b, c and d are given in Table \ref{tab:2ROUK model variable values} for each of the terrain obstruction levels and road types. The critical gradient for this model is between 14 -- 16 degrees when walking uphill and -16 -- -18 degrees when walking downhill (depending on road and obstruction conditions), which is in line with previous findings. 

Fig \ref{Fig3} shows the predicted walking speeds under different conditions. The importance of including both the hill slope and terrain obstruction variables can be clearly seen when looking at the Off Road Light Obstruction speed predictions. When directly ascending or descending a slope, the walking speed is comparable to walking on a road. However, when traversing a slope while off road, the walking speed is comparable to traversing a slope of double the gradient while on a road or path. Similarly, comparing the walking speed predictions of Off Road Light Obstruction and Off Road Heavy Obstruction reveals that just 10 cm of vegetation (our cutoff point for heavy obstruction) can reduce the walking speed by more than 0.5 km/h.

\begin{table}[!ht]
\begin{adjustwidth}{-0.5in}{0in}
    \centering
    \caption{Final walking speed model variable coefficients}
    \begin{tabular}{|l+c|c|c|c|}
    \hline
    & $a$ & $b$ & $c$  & $d$ \\ 
    \thickhline
    Paved road & 1.580 & -0.00389 & -0.00726 & -0.00218 \\ 
    \hline
    Unpaved road & 1.580 & -0.00389 & -0.00965 & -0.00248 \\
    \hline
    Off-road (obstruction unknown) & 1.536 & -0.00731 & -0.00965 & -0.00187 \\
    \hline
    Off-road (light obstruction) & 1.580 & -0.00731 & -0.00965 & -0.00187 \\ 
    \hline
    Off-road (heavy obstruction) & 1.400 & -0.00731 & -0.00965 & -0.00187 \\ 
    \hline
    \end{tabular}
    \label{tab:2ROUK model variable values}
\end{adjustwidth}
\end{table}

\begin{figure}[!h]
\begin{adjustwidth}{-2.25in}{0in} 
    \includegraphics[width=\linewidth]{Images/Paper/Fig3.eps}
    \captionsetup{width=1\linewidth}
    \caption[width=\textwidth]{{\bf Walking speed predictions under different terrain conditions.}  When: (A) travelling directly up or down hills of varying slope, (B) traversing across hills of varying slope.}
    \label{Fig3}
    \end{adjustwidth}
\end{figure}

Fig \ref{Fig4} compares the Paved Road and Off Road Heavy Obstruction speed predictions from our model against the existing functions from Naismith, Tobler and Campbell et al. When looking at the walking slope, the largest areas of deviation between our model and Naismith's rule occurs when descending a slope, as Naismith's rule does not predict a reduced speed in this scenario. For both Tobler's and Campbell et al.'s functions, the shape of the walking slope component is relatively similar to our new model, with the main distinction being the peak predicted speed on flat ground. None of the existing functions account for the hill slope, which leads to large disparities when predicting the walking speed for slope traversals. A further example of this can be seen in \nameref{S6_Appendix}, which shows the walking speeds for a simulated off-road route which encounters the full range of hill and walking slopes.

\begin{figure}[!h]
\begin{adjustwidth}{-2.25in}{0in} 
    \includegraphics[width=\linewidth]{Images/Paper/Fig4.eps}
    \captionsetup{width=1\linewidth}
    \caption[width=\textwidth]{{\bf Comparison of new model and existing hiking functions.}  Predicted walking speeds of the new model, Naismith's rule, Tobler's function and Campbell et al.'s function when: (A, C, E) travelling directly up or down hills of varying slope, (B, D, F) traversing across hills of varying slope.}
    \label{Fig4}
\end{adjustwidth}
\end{figure}

When comparing the performances of each of the models (Table \ref{tab:2comparison}), the predicted speeds for individual 50 m sections had a lower RMSE and percentage error, and a higher R squared value using our new model than in the existing ones. To isolate the impact of each of the slope variables, we filtered the results to look at the data where a slope was being directly climbed or traversed. Figs \ref{Fig5}A, B and \ref{Fig6}A, B show the RMSE and mean residuals for each of the models, for data which was within 5 degrees of directly climbing (A) or traversing (B) hills of varying slope. From this we can clearly see that Naismith's rule consistently overestimates walking speeds when descending a slope, and underestimates speeds when climbing a slope. When ascending or descending a slope, the RMSE of our GLM is similar to that of Tobler's hiking function. However, one of the main areas where we see an improvement using our model is on slight declines. Tobler's hiking function suggests that walking speed increases on mild descents up to a maximum of 6 km/h. It is clear from Fig \ref{Fig5}A, that Tobler's function overestimates the walking speed in this region. Campbell et al.'s function has a slightly lower RMSE value than our new model on the steepest walking slopes, however it underestimates the walking speeds on flat ground and mild slopes. Previous research has found that most walking takes place on low walking slopes \cite{Proffitt1995PerceivingSlant}, and this is evidenced by our data ($\sim$98\% of our data was from walking slopes of under 10 degrees). Improved walking speed predictions in this region therefore have the greatest impact in real-world situations. Within this region our model consistently has a lower RMSE than the existing functions, and a mean residual error close to 0 km/h. 

\begin{table}[!ht]
\centering
\caption{Comparison of new model against existing methods to calculate walking speeds.}
\begin{tabular}{|l|c|c|c|c|}
\hline
& New Model & Naismith & Tobler & Campbell\\
\hline
Average \% error & 23.68 & 26.36 & 26.17 & 25.33\\
\hline
MSE & 1.20 & 1.61 & 1.53 & 1.58\\
\hline
RMSE & 1.10 & 1.27 & 1.24 & 1.26\\
\hline
R\textsuperscript{2}  & 0.09 & -0.22 & -0.16 & -0.19\\
\hline
\end{tabular}
\label{tab:2comparison}  
\end{table}

\begin{figure}[!h]    
\begin{adjustwidth}{-2.25in}{0in} 
    \includegraphics[width=\linewidth]{Images/Paper/Fig5.eps}
    \captionsetup{width=1\linewidth}
    \caption[width=\textwidth]{{\bf Comparing RMSE values for the new model, Naismith's rule, Tobler's function and Campbell et al.'s function.} When: (A) travelling directly up or down hills of varying slope (all data), (B) traversing across hills of varying slope (all data), (C) travelling directly up or down hills of varying slope (off-road data only), (D) traversing across hills of varying slope (off-road data only). Campbell et al.'s function does not provide off-road speed estimates, so was not included in the off-road data comparisons.}
    \label{Fig5}
\end{adjustwidth}
\end{figure}

\begin{figure}[!h]
    \begin{adjustwidth}{-2.25in}{0in} 
    \includegraphics[width=\linewidth]{Images/Paper/Fig6.eps}
    \captionsetup{width=1\linewidth}
    \caption[width=\textwidth]{{\bf Comparing mean residual values for the new model, Naismith's rule, Tobler's function and Campbell et al.'s function.} When: (A) travelling directly up or down hills of varying slope, (B) traversing across hills of varying slope, (C)  travelling directly up or down hills of varying slope (off-road data only), (D) traversing across hills of varying slope (off-road data only). Campbell et al.'s function does not provide off-road speed estimates, so was not included in the off-road data comparisons.}
    \label{Fig6}
\end{adjustwidth}
\end{figure}

 We also see an improvement in RMSE when using our model to predict speeds for hill traversals (Fig \ref{Fig5}B). We can note from Fig \ref{Fig6}B that both Naismith's rule and Tobler's hiking function consistently overestimate the walking speed when traversing a slope, as they do not take into account the impact that the hill slope has on reducing walking speeds. The performance of Campbell et al's model improves as the hill slope increases, although we suggest this is more due to it underestimating the speed on shallow slopes. We do see that the average error in our model increases as the hill slope increases, but we believe that this is due to limited volumes of data at high hill slopes ($\sim$0.5\% of our data occurs on hill slopes steeper than 40 degrees). 

As well as looking at the overall performance of our new model, we looked to explore how well our model performed in off-road conditions, compared to the off-road adjustments for the existing functions (Naismith's reduced base speed of 4 km/h, and Tobler's correction factor of 0.6). Figs \ref{Fig5}C, D and \ref{Fig6}C, D show the RMSE and mean residuals, only considering data which was recorded in off-road conditions. From Figs \ref{Fig5}C and \ref{Fig6}C it is clear that Tobler's function consistently underestimates the walking speed when off-road. The factor of 0.6 is a larger reduction in walking speed than is observed in practice. As we found when looking at our data as a whole, Naismith's rule underestimates the walking speed when climbing a slope and overestimates when descending a slope. Our new model does not suffer from these problems, with both a lower RMSE and lower absolute mean residual value across all walking slopes. Both of these existing models also consistently underestimate walking speeds when traversing a slope, unlike our new model which has a mean residual of less than 0.4 km/h on slopes of up to 35 degrees. The error in predictions of our new model does increase as the hill slope increases, though the RMSE is generally lower than seen in the existing models. On the steepest hill slopes our model appears to perform less well than the existing ones, though only 0.2\% of our off-road data occurred on a hill slope steeper than 40 degrees. 

Although we have shown an improvement in walking speed predictions over short sections of routes, this did not translate to similar results when looking at predicted walking times for routes as a whole. Our model and all of the existing models which we have explored here had an average percentage error of 13.5\% - 15.5\% when predicting the time taken for a complete route. However, based on the errors seen in Figs \ref{Fig5} and \ref{Fig6}, we believe that this is a result of errors cancelling out over the course of a hike. For example while ascending a hill, Naismith's rule will underestimate the walking speed (and thus overestimate the walking time), but it will then overestimate the walking speed on the subsequent descent, leading to a relatively accurate total time estimate. The results here suggest that Naismith's rule, and other existing functions, are still a good rule of thumb to calculate route times as a whole, but time estimates for individual sections of a route will be less accurate than when using the new model found here.



\section{Conclusions}
\label{sec:conclusions}

We have demonstrated that the fraction of negative event weights in
existing large high-multiplicity samples can be reduced by more than
an order of magnitude, whilst preserving predictions for observables
within statistical uncertainties. Concretely, we have employed the cell
resampling method proposed in~\cite{Andersen:2021mvw} with NLO event
samples for Z boson production with up to three jets
and W boson production with five jets produced with \textsc{Sherpa}
and \textsc{BlackHat}.

For the first time, cell resampling has been applied to samples with
up to several billions of events. This was made possible by
algorithmic improvements leading to a speed-up by several orders of
magnitude. Our updated implementation can be retreived from
\url{https://cres.hepforge.org/}.

The advances in the development of the cell resampling method
presented in this work pave the way for future applications to processes with
high-multiplicities, in particular including parton showered
predictions. It will be necessary to quantify the uncertainty
introduced by the weight smearing. Variations in the maximum cell size
parameter and different prescriptions for weight redistribution within
a cell can serve as handles to assess this uncertainty. Another
promising avenue for further exploration is the analysis of the
information on weight distribution within phase space collected during
cell resampling. Regions with insufficient Monte Carlo statistics
could be identified by their accumulated negative weight, thereby
guiding the event generation. We leave the investigation of these
questions to future work.

\section*{Acknowledgements}

AM thanks Zahari Kassabov for encouragement to reconsider the use of nearest
neighbour search trees. The work of JRA and DM is supported by the STFC under
grant ST/P001246/1.

%%% Local Variables:
%%% mode: latex
%%% TeX-master: "main"
%%% End:


\section*{Statement of Code Availability}
A MATLAB implementation of the methods presented is available in an open-source repository at {\small \url{https://github.com/fabiopaparella/ride-pooling-MoD}}.

\section*{Acknowledgments}\label{Sec:akn}
We thank Dr. I. New, F. Vehlhaber, and J. Kampen for proofreading the paper. This publication is part of the
project NEON with number 17628 of the research
program Crossover, partly financed by the Dutch
Research Council.


\bibliographystyle{IEEEtran}
\bibliography{main.bib,SML_papers.bib}
% !TeX spellcheck = en_US
\appendices
\section{Proof of Lemma~\ref{lemma:finalprob}}\label{app:1}
%Since the two events are independent, to compute the probability of both happening \textcolor{blue}{within a time window $\bar{t}$}, \textcolor{blue}{one} can \textcolor{blue}{take the product of } multiply the probability of the two separate events and integrate it over the desired interval.

%the overall probability of the two events happening within a time interval $\bar{t}$ is the integral from $t=0$ to $t = \infty$ of one of the two functions, multiplied by the integral of the second function. However the extremes of the latter integral, are from $t-\bar{t}$ to $t-\bar{t}$.   

Recall that the exponential distribution, whose probability density function is given by $f(x)= \alpha e^{-\alpha x}$, models the time between events in a Poisson process of parameter $\alpha$. Since the two Poisson processes are independent,
\begin{equation*}
	\begin{split}
		P(\alpha_m,\alpha_n) = & \int_0^{\bar{t}} \alpha_n  e^{-\alpha_n  t_n} \left( \int_0^{t_n+\bar{t}} \alpha_m  e^{-\alpha_m t_m} \mathrm{d}t_m \right) \mathrm{d}t_n + \\
		& \int_{\bar{t}}^{\infty} \alpha_n e^{-\alpha_n t_n} \left(\int_{t_n-\bar{t}}^{t_n+\bar{t}} \alpha_m e^{-\alpha_m t_m}  \mathrm{d}t_m \right)  \mathrm{d}t_n,
	\end{split}
\end{equation*}
where the presence of two terms arises from the fact that the time interval $[0,+\infty)$ is considered. Making use of standard integral calculus techniques, it can be rewritten as \eqref{eq:lem}.
\begin{comment}
Evaluating the first integral we get
\begin{align*}
	\int_0^{\bar{t}} \int_0^{t_n+\bar{t}} \alpha_n   e^{-\alpha_n  t_n}  \alpha_m   e^{-\alpha_m  t_2}  \mathrm{d}t_2  \mathrm{d}t_n 
	%=\\ 
	%\int_0^{\bar{t}} \alpha_n  e^{-\alpha_n   t_n}  %\left(1 - e^{-\alpha_m   (t_n+\bar{t})}\right)  %\mathrm{d}t_n 
	=\\ 
	\int_0^{\bar{t}} \left( \alpha_n  e^{-\alpha_n   t_n} - \alpha_n  e^{-t_n \cdot(\alpha_n +\alpha_m) -\alpha_m   \bar{t}} \right)  \mathrm{d}t_n  
	=\\
	\left(1 - e^{-\alpha_n  \bar{t}}\right) - \frac{\alpha_n}{\alpha_n+\alpha_m} e^{-\alpha_m  \bar{t}}  \left(1 - e^{-(\alpha_n+\alpha_m) \bar{t}}\right),
\end{align*}
whilst from the second integral we obtain
\begin{align*}
	\int_{\bar{t}}^{\infty} \int_{t_n-\bar{t}}^{t_n+\bar{t}} \alpha_n e^{-\alpha_n  t_n}  \alpha_m  e^{-\alpha_m  t_2}  \mathrm{d}t_2  \mathrm{d}t_n 
	%=\\
	%\int_{\bar{t}}^{\infty} \alpha_n e^{-\alpha_n  t_n}  \left(e^{-\alpha_m  (t_n-\bar{t})} - e^{-\alpha_m  (t_n + \bar{t})}\right)   \mathrm{d}t_n 
	=\\ 
	\int_{\bar{t}}^{\infty} \alpha_n   e^{-(\alpha_n+\alpha_m)  t_n}   \left(e^{\alpha_m   \bar{t}}-e^{-\alpha_m   \bar{t}}\right)  \mathrm{d}t_n 
	=\\
	\frac{\alpha_n}{\alpha_n+\alpha_m}  \left(e^{\alpha_m   \bar{t}}-e^{-\alpha_m   \bar{t}}\right)  \left(e^{-(\alpha_n+\alpha_m) \bar{t}}\right).
\end{align*}
\end{comment}
%We obtain
\begin{comment}
\begin{align*}
	P_{kl} = 	\left(1 - e^{-\alpha_n  \bar{t}}\right) 
	-\\- \frac{\alpha_n}{\alpha_n+\alpha_m} e^{-\alpha_m  \bar{t}}  \left(1 - e^{-(\alpha_n+\alpha_m) \bar{t}}\right) 
	+\\+ 
	\frac{\alpha_n}{\alpha_n+\alpha_m}  \left(e^{\alpha_m   \bar{t}}-e^{-\alpha_m   \bar{t}}\right)  \left( e^{-(\alpha_n+\alpha_m) \bar{t}}\right) 
	=\\=
	\left(1 - e^{-\alpha_n  \bar{t}}\right) 
	-
	\frac{\alpha_n}{\alpha_n+\alpha_m} \cdot
	e^{-\alpha_m  \bar{t}}
	+\\+
	\frac{\alpha_n}{\alpha_n+\alpha_m}  e^{\alpha_m   \bar{t}}  \left(e^{-(\alpha_n+\alpha_m) \bar{t}}\right) 
	=\\=
	\end{align*}
\end{comment}
%\begin{align*}
%	P_{mn}(\alpha_m,\alpha_n) 
%	%&= 
%	%\left(1 - e^{-\alpha_n  \bar{t}}\right) - \frac{\alpha_n}{\alpha_n+\alpha_m}   \left(e^{-\alpha_m  \bar{t}} -  e^{-\alpha_n   \bar{t}}\right)\\
%	&= 1-\frac{\alpha_n e^{-\alpha_m\bar{t}}+\alpha_m e^{-\alpha_n\bar{t}}}{\alpha_n+\alpha_m}.
%\end{align*}
%We do a quick sanity check of the obtained results. If \mbox{$\alpha_n=0$} or $\alpha_m=0$, $P_{nm} =0$, meaning flows have 0\% probability of being matched. If $\alpha_n = \alpha_m$ the formula coincides with the original formulation with 1 event.

\section{Proof of Theorem~\ref{theorem:one}}\label{app:2}

The convergence of Algorithm~\ref{alg:one} in at most  ${\!\abs{\cM}(\abs{\cM}\!-\!1)\!}$ iterations is immediate. In fact, since for each pair $(m,n)$ chosen in each iteration we set $\Delta \tilde{J}^\prime_{mn} \!= \!\Delta \tilde{J}^\prime_{nm} \!= \!0$, neither $(n,m)$ nor $(m,n)$ will be chosen again. The optimality of the solution $\beta^\star$ and associated $\gamma^\star$ is carried out making use of an analogy with the continuous Knapsack problem,  which can be solved by a well-known polynomial-time greedy algorithm~\cite{Dantzig1957}. Recall that such algorithm consists in, every iteration, allocating the maximum amount of the resource with the highest improvement in the objective function per unit of  the resource, which is intuitively evident.  Similarly to the continuous Knapsack problem, the goal is to minimize $\tilde{J}(X^\star_\gamma)$ by allocating ${\gamma_{mn} \geq 0}$ with $m,n \in \cM$. First, borrowing the notation from Section~\ref{sec:SD}, if $\gamma_{mn}$ is assigned, then the corresponding decrease in the cost function amounts to $\tilde{J}(\gamma_{nm}X^{mn,0})\! -\!\tilde{J}(\gamma_{mn}X^{mn,\star}) =\gamma_{mn}(\tilde{J}(X^{mn,0})\! -\!\tilde{J}(X^{mn,\star})) = \gamma_{mn}\Delta{\tilde{J}}_{mn}$, where the linearity of $\tilde{J}$ played a key role. Thus, the allocation of $\gamma_{mn}$ leads to a relative improvement on the cost that amounts to $\Delta{\tilde{J}}_{mn}$. Second, as pointed out in Section~\ref{sec:alg}, throughout the algorithm, $\alpha^\prime_m$ corresponds to the demand of $r_m$ which has not yet been ride-pooled with another request. Thus, the value of $\gamma_{mn}$ that can be allocated has an upper bound given by $\gamma_{mn} \leq \min(\alpha_m^\prime, \alpha_n^\prime)P(\alpha_m^\prime,\alpha_n^\prime)$. Note that Algorithm~\ref{alg:one} corresponds to allocating the maximum amount of $\gamma_{mn}$, where $m$ and $n$ are such that, at each iteration, the highest positive relative improvement in the objective function is achieved, i.e., $(m,n) \in \mathrm{argmax}_{m,n} (\Delta \tilde{J}^\prime_{mn})$, which shows its optimality.




%We prove Theorem~\ref{theorem:one} by contradiction. If $\beta$ would not be an optimal parameter, there would exist an optimal parameter $\beta'$, so that the objective of Problem~\ref{prob:rides} $\tilde{J}(X^\star_{\gamma'}) < J(X^\star_{\gamma})$, where $X^\star_{\gamma}$ is the optimal solution of Problem~\ref{prob:rides} with demand matrix $D^\mathrm{rp}$, that, leveraging~\eqref{eq:b}, is a function of the expected number of pooled rides $\gamma$, that in turn is a function of the assignment matrix $\beta$.  To have a more compact notation, we define $\gamma_{nm}=\gamma_{nm}(\beta_{nm},\beta_{mn})$, and $\gamma'_{nm}=\gamma_{nm}(\beta'_{nm},\beta'_{mn})$. Moreover, we define $\Delta J _{nm} = \Delta J _{nm}(X^\star_{\gamma_{nm}})$, $\Delta J'_{nm} = \Delta J_{nm}(X^\star_{\gamma'_{nm}})$, and $\Delta J_{nm}^1 = \Delta J_{nm}(X^\star_{D^{nm,\star}})$.
%We recall that $ \Delta J _{nm}(X^\star_{\gamma_{nm}})$ is linear w.r.t. $\gamma_{nm}$. In turn, $\gamma_{nm}$ is monotonically increasing w.r.t. $\beta_{nm}$ and $\beta_{mn}$.
%For the sake of simplicity, let's consider a simple case with only 2 demands $n$ and $m$. They can both pool with themselves---pooling options $nn$ and $mm$---and with the other one---pooling options $nm$ and $mn$---. The algorithm prioritizes the pooling of a requests with the highest improvement in the partial objective function. In this case, options $nn$ and $mm$ will be prioritized because they have no delay.
%This results in
%\begin{equation}\label{eq:lin2}
%	\Delta J^1 _{nm} < \Delta J^1 _{mm}, \;\; \Delta J^1_{mn} < \Delta J^1_{mm},
%\end{equation} 
%\begin{equation*}
%	\Delta J^1_{nm} < \Delta J^1 _{nn}, \;\; 	\Delta J^1_{mn} < \Delta J^1 _{nn}. 
%\end{equation*}
%If $\beta$ is not the optimal parameter, then $\beta'$ would exist so that:
%\begin{equation} \label{eq:monot}
%	\beta_{nn} - \beta'_{nn}  \geq 0, \;\;\beta_{nm} - \beta'_{nm} \leq 0,
%\end{equation} 
%\begin{equation*}
%	\beta_{mm} -  \beta'_{mm} \geq 0, \;\; 	\beta_{mn} - \beta'_{mn} \leq 0.
%\end{equation*} 
%Since travel time is constant, it follows that
%\begin{equation}
%	\Delta J_{nn}+ \Delta J_{mm} + \Delta J_{nm} + \Delta J_{mn} <  
%\end{equation}
%\begin{equation*}
%	< \Delta J'_{nn} + \Delta J'_{mm}+ \Delta J'_{nm} + \Delta J'_{mn}. 
%\end{equation*}
%The problem linear in $\gamma$. It means that \begin{equation}\label{eq:deltaj}
%	(\gamma_{nn} -\gamma'_{nn} )\Delta J^1_{nn} +
%  (\gamma_{mm} -\gamma'_{mm} )\Delta J^1_{mm} -
%\end{equation}
%\begin{equation*}
%	-(\gamma_{nm} -\gamma'_{nm} )\Delta J^1_{nm}- (\gamma_{mn} -\gamma'_{mn} )\Delta J^1_{mn} < 0.
%\end{equation*}
%By leveraging Eq.~\eqref{eq:lin2},\eqref{eq:monot} and recalling that $\gamma$ is monotone w.r.t. $\beta$, Eq.~\eqref{eq:deltaj} never holds, resulting in $J(X^\star_{\gamma'}) \geq J(X^\star_{\gamma})$.  This concludes the proof.

\end{document}
