% !TeX spellcheck = en_US
\section{Case Study}\label{sec:res}
This section showcases our modeling and optimization framework in a real-world case study of Sioux Falls, South Dakota, USA, with data obtained from the Transportation Networks for Research repository~\cite{ResearchCoreTeam}. The road network is shown in Fig.~\ref{fig:road}.
\begin{figure}[t]
	\centering
	\includegraphics[trim={0cm 300 0 80},clip,width=\linewidth]{Figures/Sioux-Falls-Network.pdf}
		\caption{Road network of Sioux Falls, South Dakota, USA. Prepared by Hai Yang and Meng Qiang, Hong Kong University of Science and Technology.}
	\label{fig:road}
\end{figure}
The computations were performed on an Intel core i7-10850H, 32GB RAM. 
Problems~\ref{prob:rides} was parsed with YALMIP~\cite{Loefberg2004} and solved with Gurobi 9.5~\cite{GurobiOptimization2021}.
The computation of the matrices $D^{mn,\star}$ with $m,n \in \cM$ required less than 10 wall-clock minutes. Each instance of Problems~\ref{prob:rides} took less than 1 minute to solve. We compute it for a varying amount of hourly demands, waiting times and experienced delays considering the optimal ride-pooling assignment of the relaxed problem, obtained as described in Section~\ref{sec:alg}. Then, we compare the objectives of Problem~\ref{prob:main} and \ref{prob:rides}, i.e. the overall travel time.  Fig.~\ref{fig:WD} shows that ride-pooling always contributes to lowering the overall travel time. In particular, the larger the number of hourly demands, the larger the difference with respect to the no-pooling scenario. The reason is that the probability function in~\eqref{eq:lem} is monotonically increasing w.r.t. $\beta_m$ and $\beta_n$ that in turn, are monotonically increasing with the number of demands. 
\begin{figure}[t]
	\centering
	%\includegraphics[trim={0cm 10 0 10},clip,width=0.9\linewidth]{Figures/PooledPerc.eps}
	\includegraphics[width=\linewidth]{Figures/WaitDelay-eps-converted-to.pdf}
	\caption{Percentage of pooled rides, objective function of Problem~\ref{prob:rides}, and improvement w.r.t. no pooling as a function of the overall number of hourly demands, waiting time, and experienced delay.}
	\label{fig:WD}
\end{figure}
\begin{figure}[t]
	\centering
	\includegraphics[width=\linewidth]{Figures/Surface2-eps-converted-to.pdf}
	\caption{Improvement of $J$ with ride-pooling w.r.t. no ride-pooling,  as a function of maximum waiting time and delay.}
%		 Improvement of the objective function of Problem~\ref{prob:rides} w.r.t. the objective of Problem~\ref{prob:main}, that is the base case with no ride pooling, as a function of maximum waiting time and delay.}
	\label{fig:surf}
\end{figure}%
We also note that the percentage of rides that are pooled is strongly influenced by the number of demands, to a lower extent by the maximum waiting time, and marginally by the maximum delay. In fact, for large demands, both the waiting time and the delay have a minor impact on the percentage of rides being pooled and on the costs, as shown in Fig.~\ref{fig:WD}. This phenomenon resembles the Mohring Effect~\cite{FielbaumTirachiniEtAl2021}, stating that the more people use a mobility service, the lower the waiting time they experience. 
We can also find the same effect in Fig.~\ref{fig:surf}, which shows that the higher the number of demands per unit time, the lower the delay and waiting time to obtain the same improvement with respect to the no-pooling scenario. Moreover, it is usually more beneficial to increase the waiting time of users rather than increase the delay, so that less distance is driven by the fleet, reducing the costs.

%\textcolor{blue}{Last, we see experimentally that the part of the objective function of rebalancing flow $x^\mathrm{r}$ accounts for approximately $3\%$ on the total objective function, thus justifying the relaxation of Problem~\ref{prob:rides}. }