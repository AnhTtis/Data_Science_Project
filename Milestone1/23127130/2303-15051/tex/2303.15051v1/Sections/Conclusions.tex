% !TeX spellcheck = en_US
\section{Conclusions}\label{sec:conc}
This paper presented a framework to capture ride-pooling in a time-invariant network flow model. Specifically, we proposed a framework wherein we devise an equivalent set of requests w.r.t. to the original set so that the structure of the traffic flow  problem remains unchanged. This allows to still obtain an LP problem that can be efficiently solved with off-the-shelf solvers in polynomial-time. Additionally, we proposed a method to compute a ride-pooling request assignment, that is optimal w.r.t. a relaxed version of the minimum travel time problem. Our case study of Sioux Falls, USA, quantitatively showed that the overall number of requests per unit time is a crucial factor to assess the benefit of ride-pooling in mobility-on-demand system. In fact, we achieved average improvements from 25\% to 45\% for an increasing number of requests.  We also showed that, for a large number of requests, more than 90\% of them could be pooled with a relatively short waiting and delay time.


%We also obtained a quantitative insight on the percentage of rides that can pooled. In fact, similar to the previous case, e in waiting time, delay and number of hourly demands, all determine a higher percentage of rides that are pooled. 


%At the same time, both maximum waiting time and delay play a key role, whereby an increase in their thresholds benefits the system.

In the future, we would like to analyze the results with respect to the granularity of the road graph. Moreover, we would like to build on this research by including endogenous traffic congestion and by applying this method to other problems that can be approached with linear time-invariant traffic flow models. 

%Few examples are intermodal autonomous mobility-on-demand ~\cite{SalazarLanzettiEtAl2019}, where the system is operated in parallel with public transport, and electric-AMoD~\cite{PaparellaChauhanEtAl2023}, where energy consumption and recharging of an electric fleet is considered.