% !TeX spellcheck = en_US


\subsection{Ride-pooling Time-invariant Network Flow Model}

In this paper, we propose a formulation to take into account ride-pooling without the need to change the original structure of the problem. We transform the original set of requests, portrayed by $D$, into an equivalent set of requests accounting for ride-pooling, portrayed by $D^\mathrm{rp}$. We define the ride-pooling network flow problem as follows:
\begin{prob}[Ride-pooling Network Flow Problem]\label{prob:rides}
	Given a road graph $\cG$ and a demand matrix $D^\mathrm{rp}$, the active vehicle flows $X$ and rebalancing flow $x^\mathrm{r}$ that minimize the cost in terms of overall travel time result from
	\begin{equation*}
		\begin{aligned}
			\min_{X}\; &J(X) = {t^\top (X \mathds{1} + x^r)} \\
			\mathrm{s.t. }\; & BX = D^\mathrm{rp} \\
			&B ( X \mathds{1}+ x^\mathrm{r} )= 0 \\
			& X, x^\mathrm{r} \geq 0.
		\end{aligned}
	\end{equation*}
\end{prob}

The ride-pooling demand matrix $ D^\mathrm{rp}$ in Problem~\ref{prob:rides}, which describes the pooling pattern, has to be determined according to four key conditions. First, the individual requests, described by $D$, must be served. Second, ride-pooling two requests is only spatially feasible if the detour travel time is not greater than a threshold $\bar{\delta} \in \mathbb{R}_{>0}$. Third, ride-pooling two requests is only temporally feasible if the waiting time for a request to start being served does not exceed a threshold $\bar{t} \in \mathbb{R}_{>0}$. Fourth, the requests are pooled to minimize the cost function of Problem~\ref{prob:rides} at its solution. Due to the combinatorial nature of such an endeavor, we relax the problem in order to attain a computationally tractable algorithm, according to the following approximation.

%such that: i)~the individual requests, described by $D$, are served, and ii)~the requests are pooled to minimize the cost function of Problem~\ref{prob:rides} at its solution. 

\begin{approximation}\label{approx}
For the purpose of computing the demand matrix $D^\mathrm{rp}$,  the cost function of Problem~\ref{prob:rides} is approximated by $\tilde{J}(X):=t^\top X \mathds{1}$.
\end{approximation}
This approximation makes sense in the context of the problem, since the active vehicle travel time $t^\top X \mathds{1}$ is usually dominant over the rebalancing travel time $t^\top x^\mathrm{r}$. In fact, in the simulations performed in Section~\ref{sec:res}, $t^\top x^\mathrm{r}$ accounts for less than $3\%$ of $J$ for every scenario studied. Crucially, leveraging Approximation~\ref{approx}, we can devise a polynomial-time algorithm to compute $D^\mathrm{rp}$ that is optimal w.r.t. the approximated version of the problem.

%allows    which we show in the sequel that can be computed in ploynomia--ime .  , as shown in the sequel.
%Thereby, to compute such an assignment, we devise a polynomial-time algorithm that is optimal w.r.t.\ an approximated version of the problem.
%Note that it is not exact, but it makes sense
%the cost function of Problem~\ref{prob:rides} to $\tilde{J}(X) := {t^\top X \mathds{1}}$. 
%say that it makes sense (teh relaxation)
%unlike Problem~\ref{prob:main},
%%Compared to Problem~\ref{prob:main}, in Problem~\ref{prob:rides}, each vehicle can transport more than one user at the same time. Thus, the vehicles' travel time is lower or equal to the users'. By using $J(X,x^\mathrm{r}) =t^\top (X  \mathds{1} +  x^\mathrm{r})$, we obtain the minimum fleet size.
%Polynomial time, we show in the sequel. 
%Makes sense

\subsection{Approximate  Computation of the Demand Matrix}

In this section, we present a framework to compute the demand matrix $D^\mathrm{rp}$ under Approximation~\ref{approx}.  

% we ormalize the four conditions and present a alg which is optimal in relation to j tilde

\subsubsection{Spatial Analysis of Ride-pooling}\label{sec:SD}
%TO BE ADDED BEFORE THE TIME|SPACE sections\msmargin{In a ride pooling environment, both space and time are relevant to understand if \textcolor{blue}{two} requests can be pooled. However, in Problem~\ref{prob:main}, the elements of the OD matrix $b$ are coefficients of a Poisson process (i.e., \textcolor{blue}{it models} requests per unit time). For this reason, we must move from a deterministic point of view to a probabilistic one. In particular, we decouple time properties of the requests from the spacial ones: We first compute the optimal ride pooling paths, and then, we evaluate the probability of them happening. We define the following ride pooling problem.}{dobbiamo rendere questo paragrafo pi\`u chiaro: \`e cruciale per il paper}
In this section, we analyze the feasibility and optimal configuration of ride-pooling two requests from a spatial perspective. First, we define $\delta$ as the delay experienced by each user, representing the time required to travel the additional detour distance w.r.t. the scenario without ride-pooling. If the delay experienced by any of the two users is higher than the threshold $\bar{\delta}$, then pooling the two requests is unfeasible. Second, 
given the feasible pooling itineraries, we analyze which one is the optimal, i.e., the best itinerary to serve the requests, and whether pooling is advantageous w.r.t. no pooling.
Consider two requests $r_m,r_n\in \cR$. To restrict this analysis to the spatial dimension, we temporarily make two key considerations, that we lift in Section~\ref{sec:STF}: i)~the requests $r_m$ and $r_n$ are made at the same time; and ii)~both requests have the same demand, which we set, without any loss of generality, to $\alpha=1$. %$\alpha \in \mathbb{R}^+$.
\begin{figure}[t]
	\centering
	\includegraphics[width = 0.65\linewidth]{Figures/serve_conf-eps-converted-to.pdf}
	\caption{Distinct configurations for serving two requests $r_m,r_n \in \cR$. Each arrow represents a flow of $\alpha = 1$ vehicles. The dashed arrows represent a flow with two users, whilst the solid ones represent a flow with one user.}
	\label{fig:serve_conf}
\end{figure}

There are five different ways of serving two requests $r_m, r_n \in \cR$, as depicted in Fig.~\ref{fig:serve_conf}.
The goal is to assess whether it is feasible to ride-pool $r_m$ and $r_n$ and which is the best configurations among the five. Index each configuration with number $c \in \{0,\ldots,4\}$, with $c=0$ corresponding to no pooling. Each configuration can be split into either two or three equivalent travel requests, as shown in Fig.~\ref{fig:serve_conf}, each corresponding to an arrow. Denote the set of such equivalent requests for configuration $c$ as $\cR_{mn}^c$ ($\cR_{nm}^c = \cR_{mn}^c$) and we define $\Pi(\cR_{mn}^c)$ as the order of visited nodes. For each configuration $c$, one can now solve Problem~\ref{prob:rides}, under Approximation~\ref{approx},  with a simplified demand matrix $D^\mathrm{rp} = D^{mn,c}$ which is obtained from the set of requests $\cR_{mn}^c$ with \eqref{eq:def_D}, obtaining a flow $X^{mn,c} \in \mathbb{R}^{\abs{\mathcal{V}}\times \abs{\mathcal{V}}}$. %From the flows for each configuration, $X^c, c \in \{0,\ldots,4\}$, the questions that we set out to answer in this subsection follow immediately. 
The delay $\delta^{m,c}$ of request $r_m$ for a configuration $c$, is
\begin{equation*}
	\delta^{m,c} = \sum_{p \in {\pi^c_{mn}} } [t^\top X^{mn,c}]_p   - [t^\top X^{mn,0}]_{o_m},
\end{equation*}
where ${\pi^c_{mn}} \subseteq \Pi(\cR_{mn}^c)$ is the ordered set of nodes $\Pi(\cR_{mn}^c)$ from $o_m$ to the node before $d_m$. 
The feasible configurations are those whose delay of both users is below the threshold $\bar{\delta}$. Then, among the feasible ones, comprehending also the no pooling option, the optimal configuration is the one whose flow $X^{mn,c}$ achieves the lowest cost $\tilde{J}(X^{mn,\star})$. Henceforth, the simplified demand matrix of the optimal configuration for ride-pooling $r_m$ and $r_n$ is denoted by $D^{mn,\star}$.

\begin{remark} The demand matrix $D^{mn,c}$ contains either two or three equivalent travel requests. To reduce the computational load, Problem~\ref{prob:rides} can be computed for each equivalent request and stored separately. It amounts to solve Problem~\ref{prob:rides} $\abs{\cV}^2$ times. Since each has a computational complexity of $\mathcal{O} (\abs{\cV}^2)$, the overall computational complexity is $\mathcal{O} (\abs{\cV}^4)$. The procedure depends on the graph $\mathcal{G}$, meaning that the computations have to be performed only once.
\end{remark}
\begin{remark}
The procedure can be extended to account for the possibility of pooling three or more requests. In this case, the number of possible configurations increases, but the overall complexity remains $\mathcal{O} (\abs{\cV}^4)$.
\end{remark}
