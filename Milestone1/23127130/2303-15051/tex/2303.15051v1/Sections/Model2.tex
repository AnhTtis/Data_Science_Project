% !TeX spellcheck = en_US

\subsubsection{Temporal Analysis of Ride-pooling}\label{sec:TD}
In this section, we analyze the temporal alignment of two requests for ride-pooling.  %Specifically, we define $\bar{t}$ as the maximum waiting time since the event of a request and the event of starting to serve that request. 
We derive the probability of two requests taking place within the maximum waiting time, $\bar{t}$. As common in traffic flow models~\cite{PavoneSmithEtAl2012}, we consider that the arrival rate of a request $r_m \in \cR$ follows a Poisson process with parameter $\alpha_m$. Consider two requests $r_m, r_n \in \cR$. In the following lemma, we indicate the probability of %being possible to ride-pool those requests in a temporal perspective, i.e., the probability of 
the two events occurring within a maximum time window $\bar{t}$.
%Note that we consider that a request which can be ride-pooled starts being served when the ride-poolling configuration starts, no matter its origin node.
%%In this second part, we show the \msmargin{mathematical formulation to compute the probability}{why?} of two events (the OD pairs) happening within a maximum waiting time, $\bar{t}$. 
%
%In traffic flow models~\cite{PavoneSmithEtAl2012}, the arrival rate of a request follows a Poisson process with parameter $\alpha$.
%\msmargin{We consider two events 1 and 2 with parameters $\alpha_1,\alpha_2$. It has to be computed the probability that, within a maximum time window, one occurrence of each event happens so that they are matched together for pooling. }{unclear}
%We announce the following lemma, that we prove in Appendix 2.
\begin{lemma}\label{lemma:finalprob}
Let $r_m, r_n \in \cR$ be two requests whose arrival rate follow a Poisson process with parameters $\alpha_m$ and $\alpha_n$, respectively. The probability of each having an occurrence within a maximum time interval $\bar{t}$ is
\begin{equation}\label{eq:lem}
	P(\alpha_m,\alpha_n)  := 1-\frac{\alpha_m e^{-\alpha_n\bar{t}}+\alpha_n e^{-\alpha_m\bar{t}}}{\alpha_m+\alpha_n}.
\end{equation}
\end{lemma}
\begin{proof}
	The proof can be found in Appendix~\ref{app:1}.
\end{proof}

\subsubsection{Expected Number of Pooled Rides}%Stochastic Assignment in Ride-pooling} 
\label{sec:STF}
In Section~\ref{sec:SD}, we analyzed the spatial dimension of the ride-pooling problem, whereby we computed the best feasible pooling path given two requests. In Section~\ref{sec:TD}, we analyzed the temporal dimension of the ride-polling problem, whereby we derived the probability of two requests happening within a time window.  By lifting the temporary assumptions made in Section~\ref{sec:SD}, we formulate the ride-pooling demand matrix given a certain pooling assignment, defined in what follows.

A fraction of the demand of every request $r_m\in \cR$ can be assigned to be pooled with a request $r_n \in \cR$. Let ${\beta \in \mathbb{R}_{\geq0}^{|\cR|\times |\cR|}}$ denote the assignment matrix, whose entry $(m,n)$ is the demand of $r_m$ that is assigned to be pooled with $r_n$. %Matrix $\beta$ defines the ride-pooling assignment. 
For the remainder of this subsection we assume that $\beta$ is given. In Section~\ref{sec:alg}, we propose an algorithm to compute the optimal value of $\beta$ under Approximation~\ref{approx}.

From the analysis in Section~\ref{sec:TD}, it is noticeable that only a fraction of the allocated ride-pooling demand $\beta_{mn}$ can actually be pooled due to the aforementioned temporal constraints. Specifically, the probability of pooling is given by $P(\beta_{mn},\beta_{nm})$ according to Lemma~\ref{lemma:finalprob}. Moreover, given that we only consider pooling between two requests, at most, the maximum pooled demand between $r_m,r_n\in \cR$ is  $\min(\beta_{mn},\beta_{nm})$. Therefore, the effective expected pooled demand between two requests $r_m,r_n\in \cR$ is given by ${\gamma_{nm} = \gamma_{mn} := \min(\beta_{mn},\beta_{nm})P(\beta_{mn},\beta_{nm})}$.  As a result, according to the spatial analysis in Section~\ref{sec:SD}, this pooled demand is portrayed by the demand matrix $\gamma_{mn}D^{mn,\star}$. Note that the effective expected pooling demand follows $\sum_{n\in \cM} \gamma_{mn} \leq \alpha_m, \, \forall r_m \in \mathcal{R}$ with equality if the full demand of $r_m$ is pooled.
The full ride-pooling demand matrix $D^\mathrm{rp}$ is made up of two contributions: i)~the sum of the expected pooled active vehicle flows of the form $\gamma_{mn}D^{mn,\star}$ for ${r_m,r_n\in \cR}$; and ii)~the requested demands that were not ride-pooled. Thus, the entry $(i,j)$ of $D^\mathrm{rp}$ can be written as
\begin{equation*}\label{eq:b}
		\!D_{ij}^{\mathrm{rp}} = \begin{cases}
			\sum\limits_{\substack{p,q\in \cM\\ p\geq q}} \gamma_{pq}D_{ij}^{pq,\star} + \left(D_{ij} - \sum\limits_{p\in\cM}\gamma_{mp}D_{ij}^{mp,\star} \right), \\ 
			 \quad \quad  \quad \quad  \quad \quad  \quad \quad \;\; \exists m \in \mathcal{M} : d_m \!= i \land o_m\! = j\\
			- \sum_{k\neq j} D_{kj}^{\mathrm{rp}},  \quad \quad \quad  i  = j \\ %\sum \limits_{\substack{k = 1 \\ k\neq i}}^{\abs{\mathcal{V}}} D_{ik}
				\sum\limits_{\substack{p,q\in \cM\\ p\geq q}} \gamma_{pq}D_{ij}^{pq,\star}, \quad \quad   \text{otherwise}.%i\neq j \land \nexists m \in \mathcal{M} : o_m = i \land d_m = j\\
		\end{cases}
\end{equation*}
%In Sec.~\ref{sec:TD} we found the probability of events 1 and 2 happening within a maximum waiting time $\bar{t}$. 
%\mbox{In Sec.~\ref{sec:SD},} given two travel requests $ij$ and $kl$, \msmargin{and a maximum delay $\bar{\delta}$}{il delay lo metterei dopo}, we computed the best feasible pooling path. 
%Combining together the properties of space and time,  if pooling between $ij$ (event 1) and $kl$ (event 2) is the best feasible option, the probability of it happening is equal to \msmargin{$P_{ijkl}(\alpha'_{ijkl},\alpha'_{klij})$}{why $\alpha'$? Data $P(\alpha)$ estrai la parte compatibile, $\alpha P(\alpha)$, no?}. This probability depends on $\alpha'_{ijkl},\alpha'_{klij}$, which are the fractions of demands dedicated to ride pooling with the corresponding pair. In other words, every OD pair has a given $\alpha_{ij}$ that can partially be assigned to other request $kl$. We define $\alpha'^{\abs{ \mathcal{V}} \times \abs{\mathcal{V}} \times \abs{\mathcal{V}} \times \abs{\mathcal{V}}} $ as an assignment four-dimensional matrix, where element $\alpha'_{ijkl}$ is the amount assigned to pooling from request $ij$ to request $kl$. %This can result in either a non linear cooperative game with shared resources, or a set of non linear equations that must be added to the original traffic flow problem.
%We then define the pooling matrix, where element $\beta_{ijkl}$ is the expected number of pooled rides between $ij$ and $kl$, and is computed as \msmargin{follows}{$\min$ si scrive roman}:
%\begin{equation}\label{eq:gamma}
%	\beta_{ijkl}=\beta_{klij}= min(\alpha'_{ijkl} ,\alpha'_{klij})   P_{ijkl}(\alpha'_{ijkl},\alpha'_{klij}).
%\end{equation}
%The term $min(\alpha'_{ijkl} ,\alpha'_{klij})$ assures that the number of pooled rides $\beta_{ijkl}$ is equal to $\beta_{klij}$, because $P_{ijkl}=P_{klij}$.
%The corresponding OD matrix $b^{ijkl}$ is computed by scaling $b^{ijkl,\mathrm{best}}$  by the number of requests pooled $\beta_{ijkl}$:
%\begin{equation}
%	b^{ijkl}=\beta_{ijkl}   b^{ijkl,\mathrm{best}},
%\end{equation}
%where $b^{ijkl,\mathrm{best}}$ is the matrix of the best ride pooling option between $ij$ and $kl$, computed in Sec.\ref{sec:SD}.
%Each request $ij$ can be assigned to be pooled with every other feasible requests $kl$. However, the remaining part of the demand $\alpha'_{ijkl}-\beta_{ijkl}$ that has been assigned but not pooled, is re-assigned back to the original demands $\alpha_{ij}$ and $\alpha_{kl}$. Then, they can be allocated again to other requests. %The re-allocation cannot be assigned twice to the same $ijkl$ pool. If that would be the case, the loop would reflect in the probability of being pooled equal to 1. 
\begin{comment}
\begin{equation}
	%\alpha_{ijkl} 
	\alpha_{ij} = \sum_{kl } \alpha_{ijkl} - \sum_{kl } (\alpha_{ijkl}-\beta_{ijkl}),
\end{equation}
that can be reformulated in
\begin{equation}
		\alpha_{ij} = \sum_{kl } \alpha_{ijkl}    P_{ijkl}(\alpha_{ijkl},\alpha_{klij}).
\end{equation}
Every non pooled flow, has to be reallocated over all the other OD pairs.
 \begin{equation}
 \alpha_{ijkl} = \alpha_{ij} - \sum_{nm} \alpha_{ijnm} + \sum_{ nm} E_{ijklijnm}   (\alpha_{ijnm}-\beta_{ijnm}),
 \end{equation}
where $F_{ijklijnm}$ is the fraction of $(\alpha_{ijnm}-\beta_{ijnm})$ requests that is re-allocated to $\alpha_{ijkl}$.
\end{comment}
%The elements of the final OD matrix result from 
%\begin{equation}\label{eq:b}
%	b^\mathrm{rp}_{nm} = \sum_{i,j,k,l} b^{ijkl}_{nm} %+ (\alpha_{nmkl}-\beta_{nmkl})   b^{ijkl,\mathrm{np}}.
%	+ (\alpha_{nm} - \sum_{kl} \beta_{nmkl}).
%\end{equation}

Finally, one can input $D^\mathrm{rp}$ to Problem~\ref{prob:rides}, which yields an LP, given a pooling assignment $\beta$.