% !TeX spellcheck = en_US
\section{Introduction}
Ride-sharing is a service that is revolutionizing urban transportation.
Within this service, ride-pooling is the concept of having multiple users traveling at the same time on a single vehicle at lower costs, e.g., emissions, energy consumption, fleet size, and also the cost of the ride charged to the user. Nevertheless, these improvements come at the expense of additional waiting time and delays caused by detours.
Ride-pooling is a difficult problem to deal with due to its combinatorial nature. However, sometimes it is enough to study a ride-sharing system from a macroscopic point of view, especially when dealing with mobility planning or design~\cite{SalazarLanzettiEtAl2019,ZardiniLanzettiEtAl2020b,LukeSalazarEtAl2021}. For this reason, the microscopic nature of ride-pooling  is, at first sight, incompatible with an approach on a different scale. In this paper, we propose a framework to deal with ride-pooling from a mesoscopic point of view by moving from a deterministic to a stochastic approach. In particular, we devise a framework to easily incorporate ride-pooling into a linear time-invariant multi-commodity network flow model, also known as traffic flow model, that is a mesoscopic modeling framework usually used for mobility planning and design. 

\textit{Related Literature:} 
This paper pertains to the research streams of traffic flow models and ride-pooling, that we review in the following.
One of the approaches to characterize and control ride-sharing systems is the multi-commodity network flow model, that is suited for easy implementation of many constraints of different nature and can be efficiently solved with commercial solvers.  This model has been used for multiple purposes, from minimizing electricity costs~\cite{RossiIglesiasEtAl2018b,BoewingSchifferEtAl2020} to joint optimization with public transport~\cite{SalazarLanzettiEtAl2019,Wollenstein-BetechSalazarEtAl2021} and the power grid~\cite{RossiZhangEtAl2017,SpieserTreleavenEtAl2014,IglesiasRossiEtAl2018}. For example, Luke et al.~\cite{LukeSalazarEtAl2021} proposed a joint optimization framework for the siting and sizing of the charging infrastructure for an electric ride-sharing system, while in~\cite{PaparellaChauhanEtAl2023} we proposed a simplified, yet more tractable version of the same problem. Yet in all these models the assumption of one person per vehicle is made.

Ride-pooling has been extensively studied. 
Alonso-Mora et al.~\cite{Alonso_Mora_2017} conceived the vehicle group assignment algorithm, which optimally solves the ride-sharing problem with high capacity vehicles in a microscopic setting.
In~\cite{SantiRestaEtAl2013,JintaoHaiEtAl2020} the benefits of vehicle pooling and the pricing and equilibrium in on-demand ride-pooling markets were analyzed, respectively.
Fieldbaum et al.~\cite{FielbaumBaiEtAl2021} studied ride-pooling considering that users can be picked-up and dropped-off within a walkable distance, while in~\cite{FielbaumKucharskiEtAl2022} they examine how to split costs between users that share the same ride. In~\cite{TsaoMilojevicEtAl2019} a time-expanded network flow model is leveraged to compute the optimal routes of a mobility system that allows for ride-pooling. However, in all of these papers the ride-pooling problem has been studied from a microscopic perspective, whereby each request is considered individually.

To the best of the authors' knowledge, a mesoscopic time invariant network flow model that accounts for ride-pooling has not yet been proposed.

\textit{Statement of Contributions:} The main contributions of this paper are threefold. First, we propose a framework to capture ride-pooling, a microscopic combinatorial phenomenon, in a time-invariant network flow model, whereby the arrival process in stochastic. {Second, within the proposed framework, we devise a method to compute a ride-pooling request assignment that is optimal w.r.t. a relaxed version of the minimum fleet size problem.} Third, we showcase our framework with a case study of Sioux Falls, South Dakota, USA, where we show that the inclusion of ride-pooling can significantly benefit such ride-sharing mobility systems. 



\textit{Organization:} The remainder of this paper is structured as follows: Section \ref{sec:model} introduces the multi-commodity traffic flow problem and the framework to capture ride-pooling. Section \ref{sec:res} details the case study of Sioux Falls.
Last, in Section \ref{sec:conc}, we draw the conclusions from our findings and provide an outlook on future research endeavors.

\emph{Notation:} Throughout this paper, we denote the vector of ones, of appropriate dimensions, by $\mathds{1}$. The $i$th component of a vector $v$ is denoted by $v_i$ and the entry $(i,j)$ of a matrix $A$ is denoted by $A_{ij}$. The cardinality of set $\mathcal{S}$ is denoted by $\abs{\mathcal{S}}$.