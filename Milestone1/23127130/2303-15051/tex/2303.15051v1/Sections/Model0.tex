% !TeX spellcheck = en_US
\section{Ride-pooling Network Flow Model}\label{sec:model}
In this section, we first introduce the standard network   traffic flow model~\cite{PavoneSmithEtAl2012}. Then, we extend the formulation to take into account ride-pooling, and finally present a brief discussion on the model.
\subsection{Time-invariant Network Flow Model} 
We model the mobility system as a multi-commodity network flow model, similar to the approaches of~\cite{SalazarLanzettiEtAl2019,LukeSalazarEtAl2021,PaparellaChauhanEtAl2023,RossiIglesiasEtAl2018b,PaparellaSripanhaEtAl2023}. The transportation network is a directed graph $\mathcal{G} = (\mathcal{V}, \mathcal{A})$. {It consists of a set of vertices $\mathcal{V} := \{1,2,...,\abs{\mathcal{V}}\}$,} representing the location of intersections on the road network, and {a set of arcs $\mathcal{A} \subseteq \mathcal{V} \times \mathcal{V}$}, representing the road links between {intersections}. We indicate ${B \in \{-1,0,1\}^{\abs{\cV} \times \abs{\cA}}}$ as the incidence matrix~\cite{Bullo2018} of the road network $\mathcal{G}$. Consider an arbitrary arc indexing of natural numbers $\{1,\ldots,\abs{\cA}\}$, then $B_{ip} = -1$ if the arc indexed by $p$ is directed towards vertex $i$, $B_{ip} = 1$ if the arc indexed by $p$ leaves vertex $i$,  and $B_{ip}=0$ otherwise. We denote $t$ as the vector whose entries are the travel time $t_{a}$ required to traverse each arc $a\in \cA$, ordered in accordance with the arc ordering of $B$, which we assume to be constant. Similarly to \cite{PavoneSmithEtAl2012}, we define travel requests as follows:
\begin{definition}[Requests]
	A travel request is defined as the tuple $r = (o,d,\alpha) \in \mathcal{V} \times \mathcal{V} \times \mathbb{R}_{>0}$, in which $\alpha$ is the number of users traveling from the origin $o$ to the destination $d \neq o$ per unit time.  Define the set of requests as $\mathcal{R} := \{r_m\}_{m\in \mathcal{M}}$, where $\mathcal{M} = \{1,\ldots,M\}$.
\end{definition} 

We assume, without any loss of generality, that the origin-destination pairs of the requests $r_m \in \mathcal{R}$ are distinct. 
In this paper, we distinguish between active vehicle flows, which correspond to the flows of vehicles serving users whether they are ride-pooling or not, and rebalancing flows which correspond to the flows of empty vehicles between the drop-off and pick-up vertices of consecutive requests. We define the active vehicle flow induced by all the requests that share the same origin $i \in {\mathcal{V}}$ as vector $x^{i}$, where element $x^{i}_{a}$ is the flow on arc $a \in \cA$, ordered in accordance with the arc ordering of $B$. The overall active vehicle flow is a {matrix $X \in \mathbb{R}^{\abs{\cA} \times \abs{\cV}}$ defined as $X := \left[x^{1}\  x^2 \, \dots \,x^{\abs{\cV}}\right]$}. The rebalancing flow across the arcs is denoted by $x^{\mathrm{r}} \in \mathbb{R}^{\abs{\cA}}$. In the following, we define the network flow problem.
\begin{prob}[Multi-commodity Network Flow Problem]\label{prob:main}
	Given a road graph $\cG$ and a demand matrix {$D$}, the active vehicle flows $X$ and rebalancing flow $x^\mathrm{r}$ that minimize the cost in terms of overall travel time result from
	\begin{equation*}
		\begin{aligned}
			\min_{X}\; &J(X) =t^\top ( X \mathds{1} + x^\mathrm{r} )  \\
			\mathrm{s.t. }\; & BX = D \\
			&B ( X \mathds{1}+ x^\mathrm{r} )=0 \\
			& X, x^\mathrm{r} \geq 0,
		\end{aligned}
	\end{equation*}
	where the demand matrix $D \in \mathbb{R}^{\abs{\mathcal{V}} \times \abs{\mathcal{V}}}$ represents the requests between every pair of vertices, whose entries are
	\begin{equation}\label{eq:def_D}
		\!\!\!\!D_{ij} = \begin{cases}
			\alpha_m, &  \exists m \in \mathcal{M} : o_m = j \land d_m = i\\
			-\sum_{k\neq j} D_{kj}, & i  = j \\ %\sum \limits_{\substack{k = 1 \\ k\neq i}}^{\abs{\mathcal{V}}} D_{ik}
			0, &   \mathrm{otherwise}.%i\neq j \land \nexists m \in \mathcal{M} : o_m = i \land d_m = j\\
		\end{cases}\!\!\!
	\end{equation}
\end{prob}
Since Problem~\ref{prob:main} is totally unimodular, $X$ and $x^\mathrm{r}$ can be decoupled and computed separately~\cite{Rossi2018}. The objective function can also be interpreted as the minimum fleet size required to implement the flows~\cite{PavoneSmithEtAl2012}. % represents both the minimum fleet size to implement the flows and the overall travel time of vehicles~\cite{PavoneSmithEtAl2012}.
