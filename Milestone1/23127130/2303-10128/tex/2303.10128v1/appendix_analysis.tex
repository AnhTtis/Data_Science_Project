We here present complementary analyses, tables and plots.

\subsection{The impact of the unsupervised filter}
\label{app:no_filter}

\autoref{tab:coll_comparison_pud} and \autoref{tab:coll_comparison_cv} show the impact of the unsupervised filter in the optional filter. PUD is a controlled setting for the impact of the filter because it is a collection where tokens are of high quality compared to CV. Thus we expect that the impact of the optional filter is low in PUD. Unexpectedly, the number of tokens reduces substantially (a reduction of the order of thousands) in Chinese, Japanese and Korean. An additional drastic reduction in the observed alphabet size in these languages strongly suggests that the optional filter is not adequate for them.  
For these reasons, we believe we should not apply the unsupervised filter to these languages because their writing system is essentially a syllabary. We suspect that the actual need for the exclusion could be a combination of sampling problems relating to a large alphabet size (compared to the Latin script) and a heavy- tailed rank distribution that breaks the optional filter. It is well-known that the rank distribution of Chinese characters is long-tailed, spanning two orders of magnitude \parencite{Deng2014a}, while that of phonemes (the counterpart of letters in many languages using the Latin script) is exponential-like \parencite{Naranan1993,Balasubrahmanyan1996}.
However, that issue should be the subject of future research. 

In CV, we find that the optional filter has a similar impact in languages concerning the reduction in the number of tokens but higher impacts concerning the reduction of the alphabet sizes, suggesting that presence of strings with strange characters. The three languages with the most marked reduction  in alphabet size are French, Spanish, German and Italian, with an alphabet size greater then 100.

\begin{table}[H]
\centering
\caption{The impact of the unsupervised filter in the PUD collection. For every language, we show its linguistic family, the writing system (namely script name according to ISO-15924) and various numeric parameters after applying the mandatory filter but before applying the unsupervised filter, that are $A$, the observed alphabet size (number of distinct characters),
$n$,  the number of types, and, $T$, the number of tokens.
$A'$, $n'$ and $T'$ are the respective values of $A$, $n$ and $T$ after applying the unsupervised filter. 
% 'Conlang' stands for 'constructed language', that is an artificially created language. This is not a family in the proper sense as Conlang languages are not related in the common linguistic family sense.
} 
\label{tab:coll_comparison_pud}
\begin{tabular}{lllrrrrrr}
\hline
Language & Script & Family & $A$ & $A'$ & $n$ & $n'$ & $T$ & $T'$ \\ 
\hline
 Arabic & Arabic & Afro-Asiatic & 47 & 39 & 6600 & 6596 & 18214 & 18201 \\ 
Indonesian & Latin & Austronesian & 39 & 23 & 4596 & 4501 & 16819 & 16702 \\ 
Russian & Cyrillic & Indo-European & 61 & 31 & 7358 & 7113 & 15870 & 15588 \\ 
Hindi & Devanagari & Indo-European & 84 & 50 & 4920 & 4716 & 21184 & 20796 \\ 
Czech & Latin & Indo-European & 49 & 33 & 7360 & 7073 & 15700 & 15331 \\ 
English & Latin & Indo-European & 39 & 25 & 5082 & 5001 & 18135 & 18028 \\ 
French & Latin & Indo-European & 48 & 26 & 5593 & 5214 & 21084 & 20407 \\ 
German & Latin & Indo-European & 39 & 28 & 6215 & 6116 & 18446 & 18331 \\ 
Icelandic & Latin & Indo-European & 43 & 32 & 6175 & 6035 & 16385 & 16209 \\ 
Italian & Latin & Indo-European & 42 & 24 & 5944 & 5606 & 21815 & 21266 \\ 
Polish & Latin & Indo-European & 47 & 31 & 7329 & 7188 & 15386 & 15191 \\ 
Portuguese & Latin & Indo-European & 47 & 38 & 5678 & 5661 & 21873 & 21855 \\ 
Spanish & Latin & Indo-European & 39 & 32 & 5765 & 5750 & 21083 & 21067 \\ 
Swedish & Latin & Indo-European & 39 & 25 & 5842 & 5624 & 16653 & 16378 \\ 
Japanese & Japanese & Japonic & 1549 & 609 & 4990 & 3345 & 24899 & 22538 \\ 
Japanese-strokes & Japanese & Japonic & 1549 & 609 & 4852 & 3345 & 24737 & 22538 \\ 
Japanese-romaji & Latin & Japonic & 23 & 19 & 4984 & 4860 & 24892 & 24743 \\ 
Korean & Hangul & Koreanic & 1002 & 401 & 8031 & 6424 & 14475 & 12540 \\ 
Thai & Thai & Kra-Dai & 89 & 52 & 3818 & 3599 & 21642 & 21121 \\ 
Chinese & Han (Traditional variant) & Sino-Tibetan & 2038 & 814 & 5224 & 3154 & 18129 & 15436 \\ 
Chinese-strokes & Han (Traditional variant) & Sino-Tibetan & 2038 & 814 & 4970 & 3154 & 17845 & 15436 \\ 
Chinese-pinyin & Latin & Sino-Tibetan & 49 & 44 & 5224 & 5038 & 18129 & 17885 \\ 
Turkish & Latin & Turkic & 42 & 28 & 6793 & 6587 & 14092 & 13799 \\ 
Finnish & Latin & Uralic & 39 & 24 & 7076 & 6938 & 12853 & 12701 \\ 
  \hline

\end{tabular}
\end{table}

\begin{table}[H]
\centering
\caption{The impact of the unsupervised filter in the CV collection. The content is the same as in \autoref{tab:coll_comparison_pud}.
% For every language we show its linguistic family, the writing system (namely script name according to ISO-15924) and various numeric parameters after applying the mandatory filter but before applying the unsupervised filter, that are $A$, the observed alphabet size (number of distinct characters), $n$, the number of types, and, $T$, the number of tokens. $A'$, $n'$ and $T'$ are the respective values of $A$, $n$ and $T$ after applying the unsupervised filter. 
'Conlang' stands for 'constructed language', that is an artificially created language. This is not a family in the proper sense as Conlang languages are not related in the common linguistic family sense.
} 
\label{tab:coll_comparison_cv}
\begin{tabular}{lllrrrrrr}
\hline
Language & Script & Family & $A$ & $A'$ & $n$ & $n'$ & $T$ & $T'$ \\ 
\hline
         Arabic & Arabic & Afro-Asiatic & 44 & 31 & 7497 & 6397 & 49448 & 45825 \\ 
        Maltese & Latin & Afro-Asiatic & 40 & 31 & 8148 & 8058 & 44272 & 44112 \\ 
        Vietnamese & Latin & Austroasiatic & 86 & 41 & 574 & 370 & 1300 & 938 \\ 
        Indonesian & Latin & Austronesian & 28 & 22 & 3817 & 3768 & 44336 & 44210 \\ 
        Esperanto & Latin & Conlang & 38 & 27 & 27932 & 27759 & 406725 & 406261 \\ 
        Interlingua & Latin & Conlang & 27 & 20 & 5552 & 5126 & 31428 & 30504 \\ 
        Tamil & Tamil & Dravidian & 44 & 29 & 1525 & 1210 & 7580 & 6439 \\ 
        Persian & Arabic & Indo-European & 105 & 38 & 13240 & 13115 & 1665428 & 1662508 \\ 
        Assamese & Assamese & Indo-European & 60 & 43 & 1115 & 971 & 2000 & 1813 \\ 
        Russian & Cyrillic & Indo-European & 54 & 32 & 31921 & 31827 & 638782 & 637686 \\ 
        Ukrainian & Cyrillic & Indo-European & 44 & 34 & 14399 & 14337 & 120984 & 120760 \\ 
        Panjabi & Devanagari & Indo-European & 48 & 37 & 95 & 84 & 110 & 98 \\ 
        Modern Greek & Greek & Indo-European & 46 & 33 & 5834 & 5813 & 37926 & 37880 \\ 
        Breton & Latin & Indo-European & 41 & 28 & 4322 & 4228 & 38493 & 38237 \\ 
        Catalan & Latin & Indo-European & 67 & 39 & 79213 & 79112 & 3294506 & 3294206 \\ 
        Czech & Latin & Indo-European & 44 & 33 & 16032 & 15518 & 150312 & 147582 \\ 
        Dutch & Latin & Indo-European & 41 & 23 & 10666 & 10225 & 320992 & 316498 \\ 
        English & Latin & Indo-European & 97 & 28 & 173522 & 173023 & 9829660 & 9828713 \\ 
        French & Latin & Indo-European & 244 & 49 & 162740 & 160243 & 3732822 & 3729370 \\ 
        German & Latin & Indo-European & 152 & 30 & 150362 & 148436 & 4235094 & 4230565 \\ 
        Irish & Latin & Indo-European & 31 & 23 & 2311 & 2251 & 22751 & 22593 \\ 
        Italian & Latin & Indo-European & 110 & 34 & 55480 & 54996 & 812604 & 811783 \\ 
        Latvian & Latin & Indo-European & 35 & 27 & 7792 & 7251 & 30358 & 29456 \\ 
        Polish & Latin & Indo-European & 38 & 32 & 25365 & 25340 & 595613 & 595411 \\ 
        Portuguese & Latin & Indo-European & 41 & 27 & 13049 & 11509 & 295042 & 283048 \\ 
        Romanian & Latin & Indo-European & 36 & 29 & 6449 & 6423 & 33370 & 33341 \\ 
        Romansh & Latin & Indo-European & 40 & 26 & 9801 & 9614 & 44192 & 43792 \\ 
        Slovenian & Latin & Indo-European & 28 & 24 & 5994 & 5937 & 26402 & 26304 \\ 
        Spanish & Latin & Indo-European & 186 & 33 & 75617 & 75010 & 1843646 & 1842474 \\ 
        Swedish & Latin & Indo-European & 30 & 25 & 4454 & 4371 & 63282 & 62951 \\ 
        Welsh & Latin & Indo-European & 43 & 22 & 11488 & 11143 & 547345 & 539621 \\ 
        Western Frisian & Latin & Indo-European & 42 & 30 & 8419 & 8383 & 63127 & 63073 \\ 
        Oriya & Odia & Indo-European & 59 & 41 & 921 & 764 & 1929 & 1700 \\ 
        Dhivehi & Thaana & Indo-European & 40 & 27 & 155 & 111 & 1388 & 1284 \\ 
        Georgian & Georgian & Kartvelian & 34 & 25 & 7945 & 6505 & 15481 & 12958 \\ 
        Basque & Latin & Language isolate & 28 & 21 & 24998 & 24748 & 460188 & 458071 \\ 
        Mongolian & Mongolian & Mongolic & 36 & 31 & 14844 & 14608 & 70638 & 70217 \\ 
        Kinyarwanda & Latin & Niger-Congo & 96 & 26 & 135328 & 133815 & 1945038 & 1939810 \\ 
        Abkhazian & Cyrillic & Northwest Caucasian & 37 & 28 & 150 & 119 & 189 & 156 \\ 
        Hakha Chin & Latin & Sino-Tibetan & 28 & 23 & 2515 & 2499 & 17806 & 17776 \\ 
        Chuvash & Cyrillic & Turkic & 36 & 22 & 5565 & 4311 & 16270 & 13583 \\ 
        Kirghiz & Cyrillic & Turkic & 38 & 30 & 10497 & 10130 & 62687 & 61844 \\ 
        Tatar & Cyrillic & Turkic & 47 & 34 & 22313 & 21823 & 145458 & 144356 \\ 
        Yakut & Cyrillic & Turkic & 42 & 28 & 8041 & 7904 & 22795 & 22577 \\ 
        Turkish & Latin & Turkic & 37 & 31 & 8957 & 8926 & 107910 & 107686 \\ 
        Estonian & Latin & Uralic & 34 & 23 & 30135 & 28691 & 123895 & 121549 \\ 
          \hline

\end{tabular}
\end{table}


% optimality scores tables
\subsection{Mean word length and the law of abbreviation}
\label{sec:opt_scores}

In \autoref{tab:opt_scores_pud}, \autoref{tab:opt_scores_cv_characters} and \autoref{tab:opt_scores_cv_meadianDuration}, we show the mean word length ($L$) and the random baseline ($L_r$) as well as the outcome of the correlation test between length and frequency for PUD and for CV when length is measured in characters and also in duration, respectively.

%% PUD
\begin{table}[H]
\centering
\caption{Mean word length and the correlation between frequency and length in PUD. Word length is measured in number of characters. Mean word length ($L$) is followed by the random baseline ($L_r$). Each correlation statistic (Kendall $\tau$ or Pearson $r$) is followed by \textit{p}-values after applying Holm-Bonferroni correction (rather than being the direct output of the correlation test).
}  
\label{tab:opt_scores_pud}
\begin{tabular}{lllrrrlrl}
  \hline
language & family & script & $L$ & $L_r$ & $\tau$ & $\tau_{pvalue}$ & $r$ & $r_{pvalue}$\\  
  \hline
% latex table generated in R 4.2.1 by xtable 1.8-4 package
% Fri Feb 10 12:26:50 2023
 Arabic & Afro-Asiatic & Arabic & 4.03 & 5.54 & -0.13 & $8.32 \times 10^{-32}$ & -0.13 & $1.12 \times 10^{-20}$ \\ 
  Czech & Indo-European & Latin & 5.44 & 7.27 & -0.22 & $1.20 \times 10^{-113}$ & -0.15 & $2.47 \times 10^{-36}$ \\ 
  English & Indo-European & Latin & 4.87 & 7.00 & -0.20 & $2.52 \times 10^{-66}$ & -0.12 & $6.98 \times 10^{-17}$ \\ 
  French & Indo-European & Latin & 4.81 & 7.47 & -0.16 & $2.44 \times 10^{-49}$ & -0.12 & $4.24 \times 10^{-19}$ \\ 
  German & Indo-European & Latin & 5.74 & 8.56 & -0.23 & $1.25 \times 10^{-108}$ & -0.12 & $3.85 \times 10^{-21}$ \\ 
  Indonesian & Austronesian & Latin & 5.96 & 7.35 & -0.11 & $6.37 \times 10^{-21}$ & -0.12 & $6.53 \times 10^{-15}$ \\ 
  Italian & Indo-European & Latin & 4.85 & 7.64 & -0.16 & $4.09 \times 10^{-54}$ & -0.13 & $8.45 \times 10^{-23}$ \\ 
  Polish & Indo-European & Latin & 6.07 & 8.00 & -0.19 & $1.12 \times 10^{-80}$ & -0.13 & $2.78 \times 10^{-26}$ \\ 
  Portuguese & Indo-European & Latin & 4.35 & 7.47 & -0.20 & $9.96 \times 10^{-67}$ & -0.12 & $1.12 \times 10^{-17}$ \\ 
  Russian & Indo-European & Cyrillic & 6.04 & 8.08 & -0.19 & $4.58 \times 10^{-88}$ & -0.13 & $4.85 \times 10^{-26}$ \\ 
  Spanish & Indo-European & Latin & 4.83 & 7.59 & -0.16 & $4.10 \times 10^{-51}$ & -0.11 & $1.89 \times 10^{-17}$ \\ 
  Swedish & Indo-European & Latin & 5.41 & 7.99 & -0.23 & $3.99 \times 10^{-101}$ & -0.13 & $6.28 \times 10^{-21}$ \\ 
  Turkish & Turkic & Latin & 6.43 & 7.94 & -0.24 & $4.26 \times 10^{-124}$ & -0.12 & $4.20 \times 10^{-23}$ \\ 
   \hline

\end{tabular}
\end{table}

%% CV characters
\begin{table}[H]
\centering
\caption{Mean word length and the correlation between frequency and length in CV. Word length is measured in number of characters. Content is the same as in \ref{tab:opt_scores_pud}. 
'Conlang' stands for 'constructed language', that is an artificially created language. This is not a family in the proper sense, and Conlang languages are not related in the common family sense. 
}
\label{tab:opt_scores_cv_characters}
\begin{tabular}{lllrrrlrl}
  \hline
language & family & script & $L$ & $L_r$ & $\tau$ & $\tau_{pvalue}$ & $r$ & $r_{pvalue}$\\ 
  \hline
\input{tables/cv_opt_scores_characters}
\end{tabular}
\end{table}

%% CV duration
\begin{table}[H]
\centering
\caption{Mean word length and the correlation between frequency and length in CV. Word length is measured in duration. Content is the same as in \ref{tab:opt_scores_cv_characters}.} 
\label{tab:opt_scores_cv_meadianDuration}
\begin{tabular}{lllrrrlrl}
  \hline
language & family & script & $L$ & $L_r$ & $\tau$ & $\tau_{pvalue}$ & $r$ & $r_{pvalue}$\\ 
  \hline
\input{tables/cv_opt_scores_medianDuration}
\end{tabular}
\end{table}


