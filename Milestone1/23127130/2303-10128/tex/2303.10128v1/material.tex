% This chapter (Material, Corpus, Data\ldots) should contain a detailed description of the corpus/data. The data should be freely available to all for use unless there is a serious reason not to do so (this reason should be explained). The data should be ideally available on some online data repository such as GitHub, CLARIN, etc.

\subsection{General information about corpora and languages}

We investigate the relationship between the frequency of a word and its length in languages from two collections: Common Voice Forced Alignments (\autoref{CVFA}), hereafter CV, and Parallel Universal Dependencies (\autoref{PUD}), hereafter PUD. 

All the preprocessed files used to produce the results from the original collections are available in the repository of the article\footnote{In the \textit{data} folder of \url{\repository}.}.

PUD comprises 20 distinct languages from 7 linguistic families and 8 scripts (\autoref{tab:coll_summary_pud}).
CV comprises 46 languages from 14 linguistic families (we include 'Conlang', i.e. 'constructed languages', as a family for Esperanto and Interlingua) and 10 scripts (\autoref{tab:coll_summary_cv}).
Both PUD and CV are biased towards the Indo-European family and the Latin script. The typological information (language family) is obtained from Glottolog 4.6\footnote{\url{https://glottolog.org/}}. %(hence Turkish is from the Turkic family, not Altaic as in the  \href{https://wals.info/}{World Atlas of Language Structures}).
The writing systems are determined according to ISO-15924 codes\footnote{\url{https://unicode.org/iso15924/iso15924-codes.html}}. In \autoref{tab:coll_summary_pud} and \autoref{tab:coll_summary_cv}, we show the scripts using their standard English names. For example, most languages from the Indo-European family are written in Latin scripts. We also categorize Chinese Pinyin and Japanese Romaji as Latin scripts.

% SUMMARY TABLES

\begin{table}[H]
\centering
\caption{Summary of the main characteristics of the languages in the PUD collection. For each language, we show the linguistic family, the writing system (namely script name according to ISO-15924) and various numeric parameters: $A$, the observed alphabet size (number of distinct characters), $n$, the number of word types, and $T$, the number of word tokens.}
\label{tab:coll_summary_pud}
\begin{tabular}{lllrrr}
\hline
Language & Family & Script & $A$ & $n$ & $T$ \\ 
\hline
\input{tables/coll_summary_pud}
\end{tabular}
\end{table}


\begin{table}[H]
\centering
\caption{Summary of the main characteristics of the languages in the CV collection. For every language we show its linguistic family, the writing system (namely script name according to ISO-15924) and various numeric parameters: $A$, the observed alphabet size (number of distinct characters), $n$,  the number of word types, and, $T$, the number of word tokens.
'Conlang' stands for 'constructed language', that is an artificially created language. This is not a family in the proper sense as Conlang languages are not related in the common linguistic family sense.
} 
\label{tab:coll_summary_cv}
\begin{tabular}{lllrrr}
\hline
Language & Family & Script & $A$ & $n$ & $T$ \\ 
\hline
\input{tables/coll_summary_cv}
\end{tabular}
\end{table}

\subsection{The datasets}

We measure word length in two main ways: \textit{duration in time} and \textit{length in characters}. Concerning Chinese and Japanese, we additionally consider the number of strokes and the number of characters of their romanization (i.e. Pinyin for Chinese and Romaji for Japanese). 
% Their traditional writing systems yield very short word lengths in characters, while the number of distinct characters is very large, especially compared to Western languages with mostly alphabetic writing systems \parencite{Chen2015a, Joyce2012a} \footnote{See \autoref{tab:coll_summary_pud} for alphabet size and \autoref{tab:opt_scores_pud} for average word length in Chinese and Japanese.}. We want to test whether these differences are reflected in optimality scores. 

%This poses questions on the correct methodology to capture the studied phenomenon. \textcolor{violet}{is this previous part supposed to be here?}
 
 Given these datasets, word durations are obtained only from CV. Word lengths in characters are obtained from both CV as well as from PUD. Word lengths in strokes, and word lengths in characters after romanization, are obtained only from PUD.

\subsubsection{Common Voice Forced Alignments} \label{CVFA}

The Common Voice Corpus\footnote{\url{https://commonvoice.mozilla.org/en/datasets}} is an open source dataset of recorded voices uttering sentences in many different languages. The amount of data, as well as the source and topic of each sentence, depends considerably on the language and the corpus version. Specifically, the Common Voice Corpus 5.1 contains information on 54 languages and dialects.

Common Voice Forced Alignments (CVFA)\footnote{\url{https://github.com/JRMeyer/common-voice-forced-alignments}} were created by Josh Meyer using the Montreal Forced Aligner\footnote{\url{https://github.com/MontrealCorpusTools/Montreal-Forced-Aligner}} on top of the Common Voice Corpus 5.1. Kabyle, Upper Sorbian and Votic were left out of the alignments for an undocumented reason. Therefore, CVFA contains information on 51 languages.

In our analyses, Japanese and the three Chinese dialects were excluded as the forced aligner failed to correctly extract words from sentences. In addition, both Romansh dialects were fused into a single Romansh language. 
% resulting in a single occurrence of each remaining language.
Indeed, given the nature of this corpus, all languages are likely to be represented by more than one dialect.

Notice that Abkhazian, Panjabi, and Vietnamese have a critically low number of tokens ($T<1000$ in \autoref{tab:coll_summary_cv}). However, we decided to include them in the analyses so as to understand their limitations related to corpus size.

\subsubsection{Parallel Universal Dependencies} \label{PUD}

The Universal Dependencies (UD)\footnote{\url{https://universaldependencies.org/}} collection is an open source dataset of annotated sentences, in which the amount of data depends on each language. The Parallel Universal Dependencies (PUD) collection is a parallel subset of 20 languages from the UD collection, consisting of 1000 sentences. % It allows for a cross-language comparison that takes into account content and annotation style. 
It allows for a cross-language comparison, controlling for content and annotation style.

% The collection was already tokenized, and we thus took all the available tokens excluding those with the POS tag 'PUNCT'. Moreover, we filtered out the tokens not tagged as 'PUNCT' but containing ASCII punctuation and digits, and additional symbols that were still retained in some languages a posteriori. 
In \autoref{tab:coll_summary_pud}, we show the characteristics of the languages in PUD. For traditional Chinese and Japanese, we also include word lengths in romanizations (Pinyin and Romaji respectively), as well as word lengths measured in strokes,
resulting in a total of 24 language files. 
% Chinese here refers to traditional Chinese. 
Notice that three Japanese words that are hapax legomena  could not be romanized and thus the number of tokens and types varies slightly with respect to the original Japanese characters (\autoref{tab:coll_summary_pud}). 
% Thus Japanese in strokes follows the setting of the weak recoding problem approximately. 







