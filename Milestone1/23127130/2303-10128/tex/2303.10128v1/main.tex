\documentclass[leqno,11pt]{article}
\usepackage{CJKutf8}
\usepackage[utf8]{inputenc}
\usepackage[OT2, T1]{fontenc}
\usepackage[russian, english]{babel}
\usepackage{glottometrics}
\usepackage{float}
\usepackage{multirow}
\usepackage{subcaption}

\usepackage{csquotes}
\captionsetup[subfigure]{position=top, labelfont=bf,textfont=normalfont,singlelinecheck=off,justification=raggedright}
% %%% -*-BibTeX-*-
%%% Do NOT edit. File created by BibTeX with style
%%% ACM-Reference-Format-Journals [18-Jan-2012].

\begin{thebibliography}{52}

%%% ====================================================================
%%% NOTE TO THE USER: you can override these defaults by providing
%%% customized versions of any of these macros before the \bibliography
%%% command.  Each of them MUST provide its own final punctuation,
%%% except for \shownote{}, \showDOI{}, and \showURL{}.  The latter two
%%% do not use final punctuation, in order to avoid confusing it with
%%% the Web address.
%%%
%%% To suppress output of a particular field, define its macro to expand
%%% to an empty string, or better, \unskip, like this:
%%%
%%% \newcommand{\showDOI}[1]{\unskip}   % LaTeX syntax
%%%
%%% \def \showDOI #1{\unskip}           % plain TeX syntax
%%%
%%% ====================================================================

\ifx \showCODEN    \undefined \def \showCODEN     #1{\unskip}     \fi
\ifx \showDOI      \undefined \def \showDOI       #1{#1}\fi
\ifx \showISBNx    \undefined \def \showISBNx     #1{\unskip}     \fi
\ifx \showISBNxiii \undefined \def \showISBNxiii  #1{\unskip}     \fi
\ifx \showISSN     \undefined \def \showISSN      #1{\unskip}     \fi
\ifx \showLCCN     \undefined \def \showLCCN      #1{\unskip}     \fi
\ifx \shownote     \undefined \def \shownote      #1{#1}          \fi
\ifx \showarticletitle \undefined \def \showarticletitle #1{#1}   \fi
\ifx \showURL      \undefined \def \showURL       {\relax}        \fi
% The following commands are used for tagged output and should be
% invisible to TeX
\providecommand\bibfield[2]{#2}
\providecommand\bibinfo[2]{#2}
\providecommand\natexlab[1]{#1}
\providecommand\showeprint[2][]{arXiv:#2}

\bibitem[\protect\citeauthoryear{Albrecht and Stone}{Albrecht and
  Stone}{2017}]%
        {Albrecht2017ReasoningAH}
\bibfield{author}{\bibinfo{person}{Stefano~V. Albrecht} {and}
  \bibinfo{person}{P. Stone}.} \bibinfo{year}{2017}\natexlab{}.
\newblock \showarticletitle{Reasoning about Hypothetical Agent Behaviours and
  their Parameters}. In \bibinfo{booktitle}{\emph{AAMAS}}.
\newblock


\bibitem[\protect\citeauthoryear{Andrejczuk, Berger, Rodriguez-Aguilar, Sierra,
  and Mar{\'\i}n-Puchades}{Andrejczuk et~al\mbox{.}}{2018}]%
        {andrejczuk2018composition}
\bibfield{author}{\bibinfo{person}{Ewa Andrejczuk}, \bibinfo{person}{Rita
  Berger}, \bibinfo{person}{Juan~A Rodriguez-Aguilar}, \bibinfo{person}{Carles
  Sierra}, {and} \bibinfo{person}{V{\'\i}ctor Mar{\'\i}n-Puchades}.}
  \bibinfo{year}{2018}\natexlab{}.
\newblock \showarticletitle{The composition and formation of effective teams:
  computer science meets organizational psychology}.
\newblock \bibinfo{journal}{\emph{The Knowledge Engineering Review}}
  \bibinfo{volume}{33} (\bibinfo{year}{2018}), \bibinfo{pages}{e17}.
\newblock


\bibitem[\protect\citeauthoryear{Arjona-Medina, Gillhofer, Widrich,
  Unterthiner, Brandstetter, and Hochreiter}{Arjona-Medina
  et~al\mbox{.}}{2019}]%
        {arjona2019rudder}
\bibfield{author}{\bibinfo{person}{Jose~A Arjona-Medina},
  \bibinfo{person}{Michael Gillhofer}, \bibinfo{person}{Michael Widrich},
  \bibinfo{person}{Thomas Unterthiner}, \bibinfo{person}{Johannes
  Brandstetter}, {and} \bibinfo{person}{Sepp Hochreiter}.}
  \bibinfo{year}{2019}\natexlab{}.
\newblock \showarticletitle{Rudder: Return decomposition for delayed rewards}.
\newblock \bibinfo{journal}{\emph{NeurIPS}}  \bibinfo{volume}{32}
  (\bibinfo{year}{2019}).
\newblock


\bibitem[\protect\citeauthoryear{Beal, Changder, Norman, and Ramchurn}{Beal
  et~al\mbox{.}}{2020}]%
        {beal2020learning}
\bibfield{author}{\bibinfo{person}{Ryan Beal}, \bibinfo{person}{Narayan
  Changder}, \bibinfo{person}{Timothy Norman}, {and} \bibinfo{person}{Sarvapali
  Ramchurn}.} \bibinfo{year}{2020}\natexlab{}.
\newblock \showarticletitle{Learning the value of teamwork to form efficient
  teams}. In \bibinfo{booktitle}{\emph{Proceedings of the AAAI Conference on
  Artificial Intelligence}}, Vol.~\bibinfo{volume}{34}.
  \bibinfo{pages}{7063--7070}.
\newblock


\bibitem[\protect\citeauthoryear{Beetz, Hoyningen-Huene, Bandouch,
  Kirchlechner, Gedikli, and Maldonado}{Beetz et~al\mbox{.}}{2006}]%
        {beetz2006camera}
\bibfield{author}{\bibinfo{person}{Michael Beetz}, \bibinfo{person}{Nico~v
  Hoyningen-Huene}, \bibinfo{person}{Jan Bandouch}, \bibinfo{person}{Bernhard
  Kirchlechner}, \bibinfo{person}{Suat Gedikli}, {and} \bibinfo{person}{Alexis
  Maldonado}.} \bibinfo{year}{2006}\natexlab{}.
\newblock \showarticletitle{Camera-based observation of football games for
  analyzing multi-agent activities}. In \bibinfo{booktitle}{\emph{Proceedings
  of the fifth international joint conference on Autonomous agents and
  multiagent systems}}. \bibinfo{pages}{42--49}.
\newblock


\bibitem[\protect\citeauthoryear{Bialkowski, Lucey, Carr, Yue, Sridharan, and
  Matthews}{Bialkowski et~al\mbox{.}}{2014}]%
        {bialkowski2014large}
\bibfield{author}{\bibinfo{person}{Alina Bialkowski}, \bibinfo{person}{Patrick
  Lucey}, \bibinfo{person}{Peter Carr}, \bibinfo{person}{Yisong Yue},
  \bibinfo{person}{Sridha Sridharan}, {and} \bibinfo{person}{Iain Matthews}.}
  \bibinfo{year}{2014}\natexlab{}.
\newblock \showarticletitle{Large-scale analysis of soccer matches using
  spatiotemporal tracking data}. In \bibinfo{booktitle}{\emph{2014 IEEE
  international conference on data mining}}. IEEE, \bibinfo{pages}{725--730}.
\newblock


\bibitem[\protect\citeauthoryear{Bouveret and Lang}{Bouveret and Lang}{2014}]%
        {bouveret2014manipulating}
\bibfield{author}{\bibinfo{person}{Sylvain Bouveret} {and}
  \bibinfo{person}{J{\'e}r{\^o}me Lang}.} \bibinfo{year}{2014}\natexlab{}.
\newblock \showarticletitle{Manipulating picking sequences.}. In
  \bibinfo{booktitle}{\emph{ECAI}}, Vol.~\bibinfo{volume}{14}.
  \bibinfo{pages}{141--146}.
\newblock


\bibitem[\protect\citeauthoryear{Brams and Straffin~Jr}{Brams and
  Straffin~Jr}{1979}]%
        {brams1979prisoners}
\bibfield{author}{\bibinfo{person}{Steven~J Brams} {and}
  \bibinfo{person}{Philip~D Straffin~Jr}.} \bibinfo{year}{1979}\natexlab{}.
\newblock \showarticletitle{Prisoners' dilemma and professional sports drafts}.
\newblock \bibinfo{journal}{\emph{The American Mathematical Monthly}}
  \bibinfo{volume}{86}, \bibinfo{number}{2} (\bibinfo{year}{1979}),
  \bibinfo{pages}{80--88}.
\newblock


\bibitem[\protect\citeauthoryear{Bransen and Van~Haaren}{Bransen and
  Van~Haaren}{2020}]%
        {bransen2020player}
\bibfield{author}{\bibinfo{person}{Lotte Bransen} {and} \bibinfo{person}{Jan
  Van~Haaren}.} \bibinfo{year}{2020}\natexlab{}.
\newblock \showarticletitle{Player chemistry: Striving for a perfectly balanced
  soccer team}.
\newblock \bibinfo{journal}{\emph{Sports Analytics Conference}}
  (\bibinfo{year}{2020}).
\newblock


\bibitem[\protect\citeauthoryear{Dafoe, Bachrach, Hadfield, Horvitz, Larson,
  and Graepel}{Dafoe et~al\mbox{.}}{2021}]%
        {DafoeNature2021}
\bibfield{author}{\bibinfo{person}{Allan Dafoe}, \bibinfo{person}{Yoram
  Bachrach}, \bibinfo{person}{Gillian Hadfield}, \bibinfo{person}{Eric
  Horvitz}, \bibinfo{person}{Kate Larson}, {and} \bibinfo{person}{Thore
  Graepel}.} \bibinfo{year}{2021}\natexlab{}.
\newblock \showarticletitle{Cooperative {AI}: machines must learn to find
  common ground}.
\newblock \bibinfo{journal}{\emph{Nature}}  \bibinfo{volume}{593}
  (\bibinfo{year}{2021}), \bibinfo{pages}{33--36}.
\newblock


\bibitem[\protect\citeauthoryear{Derks and Peters}{Derks and Peters}{1993}]%
        {derks1993shapley}
\bibfield{author}{\bibinfo{person}{Jean Derks} {and} \bibinfo{person}{Hans
  Peters}.} \bibinfo{year}{1993}\natexlab{}.
\newblock \showarticletitle{A Shapley value for games with restricted
  coalitions}.
\newblock \bibinfo{journal}{\emph{International Journal of Game Theory}}
  \bibinfo{volume}{21}, \bibinfo{number}{4} (\bibinfo{year}{1993}),
  \bibinfo{pages}{351--360}.
\newblock


\bibitem[\protect\citeauthoryear{Durugkar, Liebman, and Stone}{Durugkar
  et~al\mbox{.}}{2020}]%
        {Durugkar2020BalancingIP}
\bibfield{author}{\bibinfo{person}{Ishan Durugkar}, \bibinfo{person}{E.
  Liebman}, {and} \bibinfo{person}{P. Stone}.} \bibinfo{year}{2020}\natexlab{}.
\newblock \showarticletitle{Balancing Individual Preferences and Shared
  Objectives in Multiagent Reinforcement Learning}. In
  \bibinfo{booktitle}{\emph{IJCAI}}.
\newblock


\bibitem[\protect\citeauthoryear{Elitzur}{Elitzur}{2020}]%
        {elitzur2020data}
\bibfield{author}{\bibinfo{person}{Ramy Elitzur}.}
  \bibinfo{year}{2020}\natexlab{}.
\newblock \showarticletitle{Data analytics effects in major league baseball}.
\newblock \bibinfo{journal}{\emph{Omega}}  \bibinfo{volume}{90}
  (\bibinfo{year}{2020}), \bibinfo{pages}{102001}.
\newblock


\bibitem[\protect\citeauthoryear{Ellis}{Ellis}{1983}]%
        {ellis1983similarities}
\bibfield{author}{\bibinfo{person}{M Ellis}.} \bibinfo{year}{1983}\natexlab{}.
\newblock \showarticletitle{Similarities and differences in games: A system for
  classification}. In \bibinfo{booktitle}{\emph{International association for
  physical education in higher education Conference}}.
\newblock


\bibitem[\protect\citeauthoryear{Fern{\'a}ndez, Bornn, and
  Cervone}{Fern{\'a}ndez et~al\mbox{.}}{2021}]%
        {fernandez2021framework}
\bibfield{author}{\bibinfo{person}{Javier Fern{\'a}ndez}, \bibinfo{person}{Luke
  Bornn}, {and} \bibinfo{person}{Daniel Cervone}.}
  \bibinfo{year}{2021}\natexlab{}.
\newblock \showarticletitle{A framework for the fine-grained evaluation of the
  instantaneous expected value of soccer possessions}.
\newblock \bibinfo{journal}{\emph{Machine Learning}} \bibinfo{volume}{110},
  \bibinfo{number}{6} (\bibinfo{year}{2021}), \bibinfo{pages}{1389--1427}.
\newblock


\bibitem[\protect\citeauthoryear{Fisac, Bronstein, Stefansson, Sadigh, Sastry,
  and Dragan}{Fisac et~al\mbox{.}}{2019}]%
        {fisac2019hierarchical}
\bibfield{author}{\bibinfo{person}{Jaime~F Fisac}, \bibinfo{person}{Eli
  Bronstein}, \bibinfo{person}{Elis Stefansson}, \bibinfo{person}{Dorsa
  Sadigh}, \bibinfo{person}{S~Shankar Sastry}, {and} \bibinfo{person}{Anca~D
  Dragan}.} \bibinfo{year}{2019}\natexlab{}.
\newblock \showarticletitle{Hierarchical game-theoretic planning for autonomous
  vehicles}. In \bibinfo{booktitle}{\emph{ICRA}}. IEEE,
  \bibinfo{pages}{9590--9596}.
\newblock


\bibitem[\protect\citeauthoryear{Garner, Humphrey, and Simkins}{Garner
  et~al\mbox{.}}{2016}]%
        {garner2016business}
\bibfield{author}{\bibinfo{person}{Jacqueline Garner},
  \bibinfo{person}{Phillip~R Humphrey}, {and} \bibinfo{person}{Betty Simkins}.}
  \bibinfo{year}{2016}\natexlab{}.
\newblock \showarticletitle{The business of sport and the sport of business: A
  review of the compensation literature in finance and sports}.
\newblock \bibinfo{journal}{\emph{International Review of Financial Analysis}}
  \bibinfo{volume}{47} (\bibinfo{year}{2016}), \bibinfo{pages}{197--204}.
\newblock


\bibitem[\protect\citeauthoryear{Goes, Kempe, Meerhoff, and Lemmink}{Goes
  et~al\mbox{.}}{2019}]%
        {goes2019not}
\bibfield{author}{\bibinfo{person}{Floris~R Goes}, \bibinfo{person}{Matthias
  Kempe}, \bibinfo{person}{Laurentius~A Meerhoff}, {and}
  \bibinfo{person}{Koen~APM Lemmink}.} \bibinfo{year}{2019}\natexlab{}.
\newblock \showarticletitle{Not every pass can be an assist: a data-driven
  model to measure pass effectiveness in professional soccer matches}.
\newblock \bibinfo{journal}{\emph{Big data}} \bibinfo{volume}{7},
  \bibinfo{number}{1} (\bibinfo{year}{2019}), \bibinfo{pages}{57--70}.
\newblock


\bibitem[\protect\citeauthoryear{Hu, Xie, Liang, and Chang}{Hu
  et~al\mbox{.}}{2022}]%
        {hu2022policy}
\bibfield{author}{\bibinfo{person}{Siyi Hu}, \bibinfo{person}{Chuanlong Xie},
  \bibinfo{person}{Xiaodan Liang}, {and} \bibinfo{person}{Xiaojun Chang}.}
  \bibinfo{year}{2022}\natexlab{}.
\newblock \showarticletitle{Policy diagnosis via measuring role diversity in
  cooperative multi-agent {RL}}. In \bibinfo{booktitle}{\emph{ICML}}.
  \bibinfo{pages}{9041--9071}.
\newblock


\bibitem[\protect\citeauthoryear{Le, Yue, Carr, and Lucey}{Le
  et~al\mbox{.}}{2017}]%
        {le2017coordinated}
\bibfield{author}{\bibinfo{person}{Hoang~M Le}, \bibinfo{person}{Yisong Yue},
  \bibinfo{person}{Peter Carr}, {and} \bibinfo{person}{Patrick Lucey}.}
  \bibinfo{year}{2017}\natexlab{}.
\newblock \showarticletitle{Coordinated multi-agent imitation learning}. In
  \bibinfo{booktitle}{\emph{International Conference on Machine Learning}}.
  PMLR, \bibinfo{pages}{1995--2003}.
\newblock


\bibitem[\protect\citeauthoryear{Ledezma, Aler, Sanchis, and Borrajo}{Ledezma
  et~al\mbox{.}}{2009}]%
        {ledezma2009ombo}
\bibfield{author}{\bibinfo{person}{Agapito Ledezma}, \bibinfo{person}{Ricardo
  Aler}, \bibinfo{person}{Araceli Sanchis}, {and} \bibinfo{person}{Daniel
  Borrajo}.} \bibinfo{year}{2009}\natexlab{}.
\newblock \showarticletitle{OMBO: An opponent modeling approach}.
\newblock \bibinfo{journal}{\emph{{AI} Communications}} \bibinfo{volume}{22},
  \bibinfo{number}{1} (\bibinfo{year}{2009}), \bibinfo{pages}{21--35}.
\newblock


\bibitem[\protect\citeauthoryear{Lewis}{Lewis}{2004}]%
        {lewis2004moneyball}
\bibfield{author}{\bibinfo{person}{Michael Lewis}.}
  \bibinfo{year}{2004}\natexlab{}.
\newblock \bibinfo{booktitle}{\emph{Moneyball: The art of winning an unfair
  game}}.
\newblock \bibinfo{publisher}{WW Norton \& Company}.
\newblock


\bibitem[\protect\citeauthoryear{Liemhetcharat and Luo}{Liemhetcharat and
  Luo}{2015}]%
        {liemhetcharat2015applying}
\bibfield{author}{\bibinfo{person}{Somchaya Liemhetcharat} {and}
  \bibinfo{person}{Yicheng Luo}.} \bibinfo{year}{2015}\natexlab{}.
\newblock \showarticletitle{Applying the Synergy Graph Model to Human
  Basketball.}. In \bibinfo{booktitle}{\emph{AAMAS}}.
  \bibinfo{pages}{1695--1696}.
\newblock


\bibitem[\protect\citeauthoryear{Liu, Schulte, Poupart, Rudd, and Javan}{Liu
  et~al\mbox{.}}{2020}]%
        {liu2020learning}
\bibfield{author}{\bibinfo{person}{Guiliang Liu}, \bibinfo{person}{Oliver
  Schulte}, \bibinfo{person}{Pascal Poupart}, \bibinfo{person}{Mike Rudd},
  {and} \bibinfo{person}{Mehrsan Javan}.} \bibinfo{year}{2020}\natexlab{}.
\newblock \showarticletitle{Learning agent representations for ice hockey}.
\newblock \bibinfo{journal}{\emph{Advances in Neural Information Processing
  Systems}}  \bibinfo{volume}{33} (\bibinfo{year}{2020}),
  \bibinfo{pages}{18704--18715}.
\newblock


\bibitem[\protect\citeauthoryear{Ljung, Carlsson, and Lambrix}{Ljung
  et~al\mbox{.}}{2018}]%
        {Ljung2018PlayerPV}
\bibfield{author}{\bibinfo{person}{Dennis Ljung}, \bibinfo{person}{Niklas
  Carlsson}, {and} \bibinfo{person}{P. Lambrix}.}
  \bibinfo{year}{2018}\natexlab{}.
\newblock \showarticletitle{Player Pairs Valuation in Ice Hockey}. In
  \bibinfo{booktitle}{\emph{MLSA@PKDD/ECML}}.
\newblock


\bibitem[\protect\citeauthoryear{Lucey, Bialkowski, Carr, Foote, and
  Matthews}{Lucey et~al\mbox{.}}{2012}]%
        {lucey2012characterizing}
\bibfield{author}{\bibinfo{person}{Patrick Lucey}, \bibinfo{person}{Alina
  Bialkowski}, \bibinfo{person}{Peter Carr}, \bibinfo{person}{Eric Foote},
  {and} \bibinfo{person}{Iain Matthews}.} \bibinfo{year}{2012}\natexlab{}.
\newblock \showarticletitle{Characterizing multi-agent team behavior from
  partial team tracings: Evidence from the english premier league}. In
  \bibinfo{booktitle}{\emph{Proceedings of the AAAI Conference on Artificial
  Intelligence}}, Vol.~\bibinfo{volume}{26}. \bibinfo{pages}{1387--1393}.
\newblock


\bibitem[\protect\citeauthoryear{Pourmehr and Dadkhah}{Pourmehr and
  Dadkhah}{2011}]%
        {pourmehr2011overview}
\bibfield{author}{\bibinfo{person}{Shokoofeh Pourmehr} {and}
  \bibinfo{person}{Chitra Dadkhah}.} \bibinfo{year}{2011}\natexlab{}.
\newblock \showarticletitle{An overview on opponent modeling in RoboCup soccer
  simulation 2D}.
\newblock \bibinfo{journal}{\emph{Robot Soccer World Cup}}
  (\bibinfo{year}{2011}), \bibinfo{pages}{402--414}.
\newblock


\bibitem[\protect\citeauthoryear{Raabe, Nabben, and Memmert}{Raabe
  et~al\mbox{.}}{2022}]%
        {raabe2022graph}
\bibfield{author}{\bibinfo{person}{Dominik Raabe}, \bibinfo{person}{Reinhard
  Nabben}, {and} \bibinfo{person}{Daniel Memmert}.}
  \bibinfo{year}{2022}\natexlab{}.
\newblock \showarticletitle{Graph representations for the analysis of
  multi-agent spatiotemporal sports data}.
\newblock \bibinfo{journal}{\emph{Applied Intelligence}}
  (\bibinfo{year}{2022}), \bibinfo{pages}{1--21}.
\newblock


\bibitem[\protect\citeauthoryear{Radke, Brecht, and Radke}{Radke
  et~al\mbox{.}}{2022a}]%
        {radke2022identifying}
\bibfield{author}{\bibinfo{person}{David Radke}, \bibinfo{person}{Tim Brecht},
  {and} \bibinfo{person}{Daniel Radke}.} \bibinfo{year}{2022}\natexlab{a}.
\newblock \showarticletitle{Identifying Completed Pass Types and Improving
  Passing Lane Models}. In \bibinfo{booktitle}{\emph{Link{\"o}ping Hockey
  Analytics Conference}}. \bibinfo{pages}{71--86}.
\newblock


\bibitem[\protect\citeauthoryear{Radke, Larson, and Brecht}{Radke
  et~al\mbox{.}}{2022b}]%
        {Radke2022Exploring}
\bibfield{author}{\bibinfo{person}{David Radke}, \bibinfo{person}{Kate Larson},
  {and} \bibinfo{person}{Tim Brecht}.} \bibinfo{year}{2022}\natexlab{b}.
\newblock \showarticletitle{Exploring the Benefits of Teams in Multiagent
  Learning}. In \bibinfo{booktitle}{\emph{IJCAI}}.
\newblock


\bibitem[\protect\citeauthoryear{Radke, Larson, and Brecht}{Radke
  et~al\mbox{.}}{2022c}]%
        {radke2022importance}
\bibfield{author}{\bibinfo{person}{David Radke}, \bibinfo{person}{Kate Larson},
  {and} \bibinfo{person}{Tim Brecht}.} \bibinfo{year}{2022}\natexlab{c}.
\newblock \showarticletitle{The Importance of Credo in Multiagent Learning}.
\newblock \bibinfo{journal}{\emph{ALA Workshop at AAMAS}}
  (\bibinfo{year}{2022}).
\newblock


\bibitem[\protect\citeauthoryear{Radke, Radke, Brecht, and Pawelczyk}{Radke
  et~al\mbox{.}}{2021}]%
        {Radke2021Passing}
\bibfield{author}{\bibinfo{person}{D.~T. Radke}, \bibinfo{person}{D.~L. Radke},
  \bibinfo{person}{T. Brecht}, {and} \bibinfo{person}{A. Pawelczyk}.}
  \bibinfo{year}{2021}\natexlab{}.
\newblock \showarticletitle{Passing and Pressure Metrics in Ice Hockey}.
\newblock \bibinfo{journal}{\emph{Workshop of AI for Sports Analytics}}
  (\bibinfo{year}{2021}).
\newblock


\bibitem[\protect\citeauthoryear{Rahimian and Toka}{Rahimian and Toka}{2022}]%
        {rahimian2022optical}
\bibfield{author}{\bibinfo{person}{Pegah Rahimian} {and}
  \bibinfo{person}{Laszlo Toka}.} \bibinfo{year}{2022}\natexlab{}.
\newblock \showarticletitle{Optical tracking in team sports}.
\newblock \bibinfo{journal}{\emph{Journal of Quantitative Analysis in Sports}}
  \bibinfo{volume}{18}, \bibinfo{number}{1} (\bibinfo{year}{2022}),
  \bibinfo{pages}{35--57}.
\newblock


\bibitem[\protect\citeauthoryear{Rahwan, Michalak, Wooldridge, and
  Jennings}{Rahwan et~al\mbox{.}}{2015}]%
        {rahwan2015coalition}
\bibfield{author}{\bibinfo{person}{Talal Rahwan}, \bibinfo{person}{Tomasz~P
  Michalak}, \bibinfo{person}{Michael Wooldridge}, {and}
  \bibinfo{person}{Nicholas~R Jennings}.} \bibinfo{year}{2015}\natexlab{}.
\newblock \showarticletitle{Coalition structure generation: A survey}.
\newblock \bibinfo{journal}{\emph{Artificial Intelligence}}
  \bibinfo{volume}{229} (\bibinfo{year}{2015}), \bibinfo{pages}{139--174}.
\newblock


\bibitem[\protect\citeauthoryear{Rashid, Samvelyan, Schroeder, Farquhar,
  Foerster, and Whiteson}{Rashid et~al\mbox{.}}{2018}]%
        {rashid2018qmix}
\bibfield{author}{\bibinfo{person}{Tabish Rashid}, \bibinfo{person}{Mikayel
  Samvelyan}, \bibinfo{person}{Christian Schroeder}, \bibinfo{person}{Gregory
  Farquhar}, \bibinfo{person}{Jakob Foerster}, {and} \bibinfo{person}{Shimon
  Whiteson}.} \bibinfo{year}{2018}\natexlab{}.
\newblock \showarticletitle{Qmix: Monotonic value function factorisation for
  deep multi-agent reinforcement learning}. In
  \bibinfo{booktitle}{\emph{ICML}}. \bibinfo{pages}{4295--4304}.
\newblock


\bibitem[\protect\citeauthoryear{Rein and Memmert}{Rein and Memmert}{2016}]%
        {rein2016big}
\bibfield{author}{\bibinfo{person}{Robert Rein} {and} \bibinfo{person}{Daniel
  Memmert}.} \bibinfo{year}{2016}\natexlab{}.
\newblock \showarticletitle{Big data and tactical analysis in elite soccer:
  future challenges and opportunities for sports science}.
\newblock \bibinfo{journal}{\emph{SpringerPlus}} \bibinfo{volume}{5},
  \bibinfo{number}{1} (\bibinfo{year}{2016}), \bibinfo{pages}{1--13}.
\newblock


\bibitem[\protect\citeauthoryear{Ritchie, Harell, and Shreeves}{Ritchie
  et~al\mbox{.}}{2022}]%
        {ritchie2022pass}
\bibfield{author}{\bibinfo{person}{Robyn Ritchie}, \bibinfo{person}{Alon
  Harell}, {and} \bibinfo{person}{Phillip Shreeves}.}
  \bibinfo{year}{2022}\natexlab{}.
\newblock \showarticletitle{Pass Evaluation in Women's Olympic Ice Hockey}. In
  \bibinfo{booktitle}{\emph{Proceedings of the 5th International ACM Workshop
  on Multimedia Content Analysis in Sports}}. \bibinfo{pages}{65--73}.
\newblock


\bibitem[\protect\citeauthoryear{Sampaio, McGarry, Calleja-Gonz{\'a}lez,
  Jim{\'e}nez~S{\'a}iz, Schelling i~del Alc{\'a}zar, and Balciunas}{Sampaio
  et~al\mbox{.}}{2015}]%
        {sampaio2015exploring}
\bibfield{author}{\bibinfo{person}{Jaime Sampaio}, \bibinfo{person}{Tim
  McGarry}, \bibinfo{person}{Julio Calleja-Gonz{\'a}lez},
  \bibinfo{person}{Sergio Jim{\'e}nez~S{\'a}iz}, \bibinfo{person}{Xavi
  Schelling i~del Alc{\'a}zar}, {and} \bibinfo{person}{Mindaugas Balciunas}.}
  \bibinfo{year}{2015}\natexlab{}.
\newblock \showarticletitle{Exploring game performance in the National
  Basketball Association using player tracking data}.
\newblock \bibinfo{journal}{\emph{PloS one}} \bibinfo{volume}{10},
  \bibinfo{number}{7} (\bibinfo{year}{2015}), \bibinfo{pages}{e0132894}.
\newblock


\bibitem[\protect\citeauthoryear{Santos, Santos, Pacheco, and Levin}{Santos
  et~al\mbox{.}}{2021}]%
        {Santos2021SocialNI}
\bibfield{author}{\bibinfo{person}{F. Santos}, \bibinfo{person}{F.~C. Santos},
  \bibinfo{person}{J. Pacheco}, {and} \bibinfo{person}{S. Levin}.}
  \bibinfo{year}{2021}\natexlab{}.
\newblock \showarticletitle{Social Network Interventions to Prevent
  Reciprocity-driven Polarization}. In \bibinfo{booktitle}{\emph{AAMAS}}.
\newblock


\bibitem[\protect\citeauthoryear{Schr{\"o}der, Hoey, and Rogers}{Schr{\"o}der
  et~al\mbox{.}}{2016}]%
        {schroder2016modeling}
\bibfield{author}{\bibinfo{person}{Tobias Schr{\"o}der}, \bibinfo{person}{Jesse
  Hoey}, {and} \bibinfo{person}{Kimberly~B Rogers}.}
  \bibinfo{year}{2016}\natexlab{}.
\newblock \showarticletitle{Modeling dynamic identities and uncertainty in
  social interactions: Bayesian affect control theory}.
\newblock \bibinfo{journal}{\emph{American Sociological Review}}
  \bibinfo{volume}{81}, \bibinfo{number}{4} (\bibinfo{year}{2016}),
  \bibinfo{pages}{828--855}.
\newblock


\bibitem[\protect\citeauthoryear{Schuckers}{Schuckers}{2011}]%
        {schuckers2011s}
\bibfield{author}{\bibinfo{person}{Michael~E Schuckers}.}
  \bibinfo{year}{2011}\natexlab{}.
\newblock \showarticletitle{What's An NHL Draft Pick Worth? A Value Pick Chart
  for the National Hockey League}.
\newblock \bibinfo{journal}{\emph{Statistical Sports Consulting}}
  (\bibinfo{year}{2011}).
\newblock


\bibitem[\protect\citeauthoryear{Schulte, Khademi, Gholami, Zhao, Javan, and
  Desaulniers}{Schulte et~al\mbox{.}}{2017}]%
        {schulte2017markov}
\bibfield{author}{\bibinfo{person}{Oliver Schulte}, \bibinfo{person}{Mahmoud
  Khademi}, \bibinfo{person}{Sajjad Gholami}, \bibinfo{person}{Zeyu Zhao},
  \bibinfo{person}{Mehrsan Javan}, {and} \bibinfo{person}{Philippe
  Desaulniers}.} \bibinfo{year}{2017}\natexlab{}.
\newblock \showarticletitle{A Markov Game model for valuing actions, locations,
  and team performance in ice hockey}.
\newblock \bibinfo{journal}{\emph{Data Mining and Knowledge Discovery}}
  \bibinfo{volume}{31}, \bibinfo{number}{6} (\bibinfo{year}{2017}),
  \bibinfo{pages}{1735--1757}.
\newblock


\bibitem[\protect\citeauthoryear{Schwind, Demirovic, Inoue, and
  Lagniez}{Schwind et~al\mbox{.}}{2021}]%
        {schwind2021partial}
\bibfield{author}{\bibinfo{person}{Nicolas Schwind}, \bibinfo{person}{Emir
  Demirovic}, \bibinfo{person}{Katsumi Inoue}, {and}
  \bibinfo{person}{Jean-Marie Lagniez}.} \bibinfo{year}{2021}\natexlab{}.
\newblock \showarticletitle{Partial Robustness in Team Formation: Bridging the
  Gap between Robustness and Resilience.}. In
  \bibinfo{booktitle}{\emph{AAMAS}}, Vol.~\bibinfo{volume}{21}.
  \bibinfo{pages}{20th}.
\newblock


\bibitem[\protect\citeauthoryear{Simon}{Simon}{1990}]%
        {simon1990bounded}
\bibfield{author}{\bibinfo{person}{Herbert~A Simon}.}
  \bibinfo{year}{1990}\natexlab{}.
\newblock \showarticletitle{Bounded rationality}.
\newblock In \bibinfo{booktitle}{\emph{Utility and probability}}.
  \bibinfo{publisher}{Springer}, \bibinfo{pages}{15--18}.
\newblock


\bibitem[\protect\citeauthoryear{Spearman}{Spearman}{2018}]%
        {spearman2018beyond}
\bibfield{author}{\bibinfo{person}{William Spearman}.}
  \bibinfo{year}{2018}\natexlab{}.
\newblock \showarticletitle{Beyond expected goals}. In
  \bibinfo{booktitle}{\emph{Proceedings of the 12th MIT sloan sports analytics
  conference}}. \bibinfo{pages}{1--17}.
\newblock


\bibitem[\protect\citeauthoryear{Stone, Riley, and Veloso}{Stone
  et~al\mbox{.}}{2000}]%
        {stone2000defining}
\bibfield{author}{\bibinfo{person}{Peter Stone}, \bibinfo{person}{Patrick
  Riley}, {and} \bibinfo{person}{Manuela Veloso}.}
  \bibinfo{year}{2000}\natexlab{}.
\newblock \showarticletitle{Defining and using ideal teammate and opponent
  agent models}. In \bibinfo{booktitle}{\emph{AAAI/IAAI}}.
  \bibinfo{pages}{1040--1045}.
\newblock


\bibitem[\protect\citeauthoryear{Tuyls, Omidshafiei, Muller, Wang, Connor,
  Hennes, Graham, Spearman, Waskett, Steel, et~al\mbox{.}}{Tuyls
  et~al\mbox{.}}{2021}]%
        {tuyls2021game}
\bibfield{author}{\bibinfo{person}{Karl Tuyls}, \bibinfo{person}{Shayegan
  Omidshafiei}, \bibinfo{person}{Paul Muller}, \bibinfo{person}{Zhe Wang},
  \bibinfo{person}{Jerome Connor}, \bibinfo{person}{Daniel Hennes},
  \bibinfo{person}{Ian Graham}, \bibinfo{person}{William Spearman},
  \bibinfo{person}{Tim Waskett}, \bibinfo{person}{Dafydd Steel},
  {et~al\mbox{.}}} \bibinfo{year}{2021}\natexlab{}.
\newblock \showarticletitle{Game Plan: What {AI} can do for Football, and What
  Football can do for AI}.
\newblock \bibinfo{journal}{\emph{Journal of Artificial Intelligence Research}}
   \bibinfo{volume}{71} (\bibinfo{year}{2021}), \bibinfo{pages}{41--88}.
\newblock


\bibitem[\protect\citeauthoryear{Van Der~Hoek, Jamroga, and Wooldridge}{Van
  Der~Hoek et~al\mbox{.}}{2005}]%
        {van2005logic}
\bibfield{author}{\bibinfo{person}{Wiebe Van Der~Hoek},
  \bibinfo{person}{Wojciech Jamroga}, {and} \bibinfo{person}{Michael
  Wooldridge}.} \bibinfo{year}{2005}\natexlab{}.
\newblock \showarticletitle{A logic for strategic reasoning}. In
  \bibinfo{booktitle}{\emph{Proceedings of the fourth international joint
  conference on Autonomous agents and multiagent systems}}.
  \bibinfo{pages}{157--164}.
\newblock


\bibitem[\protect\citeauthoryear{Vats, Fani, Clausi, and Zelek}{Vats
  et~al\mbox{.}}{2022}]%
        {vats2022evaluating}
\bibfield{author}{\bibinfo{person}{Kanav Vats}, \bibinfo{person}{Mehrnaz Fani},
  \bibinfo{person}{David~A Clausi}, {and} \bibinfo{person}{John~S Zelek}.}
  \bibinfo{year}{2022}\natexlab{}.
\newblock \showarticletitle{Evaluating deep tracking models for player tracking
  in broadcast ice hockey video}.
\newblock \bibinfo{journal}{\emph{arXiv preprint arXiv:2205.10949}}
  (\bibinfo{year}{2022}).
\newblock


\bibitem[\protect\citeauthoryear{Visser, Dr{\"u}cker, H{\"u}bner, Schmidt, and
  Weland}{Visser et~al\mbox{.}}{2000}]%
        {visser2000recognizing}
\bibfield{author}{\bibinfo{person}{Ubbo Visser}, \bibinfo{person}{Christian
  Dr{\"u}cker}, \bibinfo{person}{Sebastian H{\"u}bner}, \bibinfo{person}{Esko
  Schmidt}, {and} \bibinfo{person}{Hans-Georg Weland}.}
  \bibinfo{year}{2000}\natexlab{}.
\newblock \showarticletitle{Recognizing formations in opponent teams}. In
  \bibinfo{booktitle}{\emph{Robot Soccer World Cup}}. Springer,
  \bibinfo{pages}{391--396}.
\newblock


\bibitem[\protect\citeauthoryear{Williamson and Cox}{Williamson and
  Cox}{2014}]%
        {williamson2014distributed}
\bibfield{author}{\bibinfo{person}{Kellie Williamson} {and}
  \bibinfo{person}{Rochelle Cox}.} \bibinfo{year}{2014}\natexlab{}.
\newblock \showarticletitle{Distributed cognition in sports teams: Explaining
  successful and expert performance}.
\newblock \bibinfo{journal}{\emph{Educational Philosophy and Theory}}
  \bibinfo{volume}{46}, \bibinfo{number}{6} (\bibinfo{year}{2014}),
  \bibinfo{pages}{640--654}.
\newblock


\bibitem[\protect\citeauthoryear{Yan, Kroer, and Peysakhovich}{Yan
  et~al\mbox{.}}{2020}]%
        {Yan2020EvaluatingAR}
\bibfield{author}{\bibinfo{person}{Tom Yan}, \bibinfo{person}{Christian Kroer},
  {and} \bibinfo{person}{A. Peysakhovich}.} \bibinfo{year}{2020}\natexlab{}.
\newblock \showarticletitle{Evaluating and Rewarding Teamwork Using Cooperative
  Game Abstractions}.
\newblock \bibinfo{journal}{\emph{NeurIPS}} (\bibinfo{year}{2020}).
\newblock


\end{thebibliography}
 % we had to discard this solution as it created new problems


% \usepackage{amsmath}
% \usepackage{amssymb}

\DeclareMathOperator{\expect}{\mathbb{E}}
\newcommand \expe[1] {\expect\left[\,#1\,\right]}
\newcommand \cexpe[2] {\expect\left[\,#1 \;|\; #2\,\right]}
\newcommand \expei[2] {\expect_{#1}\left[\,#2\,\right]}
\newcommand \cexpei[3] {\expect_{#1}\left[\,#2 \;|\; #3\,\right]}

\DeclareMathOperator{\varies}{\mathbb{V}}
\newcommand \var[1] {\varies\left[\,#1\,\right]}
\newcommand \vari[2] {\varies_{#1}\left[\,#2\,\right]}

% \newcommand{\sgn}{\mbox{sgn}}
\DeclareMathOperator{\sgn}{sgn}

% -----------------------------------------------------------------------------

\let\openbox\undefined % trick to avoid a compilation error borrowed from % https://tex.stackexchange.com/questions/386739/clash-between-newtxmath-and-amsthm-packages

\usepackage{amsthm}

\newtheorem{theorem}{Theorem}[section]
\newtheorem{property}{Property}[section]

\newcommand{\repository}[0]{https://github.com/IQL-course/IQL-Research-Project-21-22}

\usepackage[toc,page,title]{appendix} % this is for complex appendices as ours; the template of the journal only offers a plain appendix.
\usepackage{etoolbox}

\newtoggle{anonymous}
\togglefalse{anonymous}

% To be filled by editors
\glottovol{XX}
\glottoyear{20XX}
\glottodoi{Here will be added DOI number by Glottometrics.}

% Author names and short title in the header

\iftoggle{anonymous}{
\runningauthors{}
}
{
\runningauthors{Petrini, Casas-i-Muñoz, Cluet-i-Martinell, Wang, Bentz \& Ferrer-i-Cancho}
}

\runningtitle{Direct and indirect evidence of compression of word lengths.}

%==================================================
% Title page

% Title
\title{Direct and indirect evidence of compression of word lengths. \\ 
Zipf's law of abbreviation revisited.}


\iftoggle{anonymous}{
\addauthor{1}{Some author}{Some ORCID}
\affil{1}{Some affiliation.}

}{
% Authors
\addauthor{1}{Sonia Petrini}{0000-0002-0514-6223}
\addauthor{2}{Antoni Casas-i-Muñoz}{0000-0001-5690-316X}
\addauthor{2}{Jordi Cluet-i-Martinell}{0000-0003-4188-6728}
\addauthor{2}{Mengxue Wang}{0000-0002-8262-9333}
\addauthor{3}{Christian Bentz}{0000-0001-6570-9326}
\addauthor{1*}{Ramon Ferrer-i-Cancho}{0000-0002-7820-923X}

% Affiliation 
% We prefer only University/Institution. Department, Faculty/College is optional
\affil{1}{% Complexity and Quantitative Linguistics Lab, LARCA Research Group, 
Quantitative, Mathematical and Computational Linguistics Research Group. Departament de Ci\`encies de la Computaci\'o, Universitat Polit\`ecnica de Catalunya (UPC), Barcelona, Catalonia, Spain.}
\affil{2}{Universitat Polit\`ecnica de Catalunya (UPC), Barcelona School of Informatics, Barcelona, Catalonia, Spain.}
\affil{3}{Department of Linguistics, University of Tübingen, Tübingen, Germany.}

% Corresponding author
\correspond{*}{Corresponding author's email: rferrericancho@cs.upc.edu}

}

\begin{document}
\maketitle

%==================================================
% Abstract

\begin{abstract}
% The article should contain an abstract within 200 words. The abstract states briefly the purpose of the research, methodology, key results, and major conclusions. The abstract should be a citation-free single paragraph with running sentences. Use the \texttt{\textbackslash maketitle} command before the text of your abstract as shown in this .tex file.
Zipf's law of abbreviation, the tendency of more frequent words to be shorter, is one of the most solid candidates for a linguistic universal, in the sense that it has the potential for being exceptionless or with a number of exceptions that is vanishingly small compared to the number of languages on Earth.  
Since Zipf's pioneering research, this law has been viewed as a manifestation of a universal principle of communication, i.e. the minimization of word lengths, to reduce the effort of communication. 
Here we revisit the concordance of written language with the law of abbreviation. Crucially, we provide wider evidence that the law holds also in speech (when word length is
measured in time), in particular in 46 languages from 14 linguistic families. Agreement with the law of abbreviation provides indirect evidence of compression of languages via the theoretical argument that the law of abbreviation is a prediction of optimal coding. Motivated by the need of direct evidence of compression, we derive a simple formula for a random baseline
indicating that word lengths are systematically below chance, across linguistic families and writing systems, and independently of the unit of measurement (length in characters or duration in time). 
Our work paves the way to measure and compare the degree of optimality of word lengths in  languages. 
\end{abstract}

%==================================================
% Keywords

\begin{keywords}
% C3--5 keywords separated by commas.
word length, compression, law of abbreviation
\end{keywords}

%==================================================
% Introduction

\section{Introduction}
\label{sec:introduction}

\section{Introduction}
\label{sec:introduction}
% \begin{itemize}
%     % Diffusion of FL
%     \item {\st{Diffusion of FL}}
%     % Security threats to FL
%     \item {\st{Security threats to FL with particular focus on model poisoning}}
%     % Limitations of existing countermeasures
%     \item {\st{Current countermeasures (e.g., KRUM) and their limitations}}
%     % Proposed method and its advantages
%     \item {\st{Intuitive description of the proposed method and its difference (i.e., advantages) w.r.t. state of the art}}
%     % Main contributions
%     \item {\st{Summary of the main contributions of this work}}
%     % Paper's structure and organization
%     \item {\st{Paper's structure and organization}}
% \end{itemize}

% Diffusion of FL
Recently, {\em federated learning} (FL) has emerged as the leading paradigm for training distributed, large-scale, and privacy-preserving machine learning (ML) systems~\cite{mcmahan2017googleai,mcmahan2017aistats}. 
The core idea of FL is to allow multiple edge clients to collaboratively train a shared, global model without disclosing their local private training data.
%Specifically, an FL system consists of a central server and many edge clients; 
A typical FL round involves the following steps: {\em(i)} the server randomly picks some clients and sends them the current, global model; {\em(ii)} each selected client locally trains its model with its own private data; then, it sends the resulting local model to the server;\footnote{Whenever we refer to global/local model, we mean global/local model {\em parameters}.} {\em(iii)} the server updates the global model by computing an \emph{aggregation function}, usually the average (FedAvg), on the local models received from clients.
% \begin{enumerate}
%     \item[{\em(i)}] the server sends the current, global model to the clients and appoints some of them for training;
%     \item[{\em(ii)}] each selected client locally trains its copy of the global model with its own private data; then, it sends the resulting local model back to the server;\footnote{Whenever we refer to global/local model, we mean global/local model {\em parameters}.}
%     \item[{\em(iii)}] the server updates the global model by computing an \emph{aggregation function} on the local models received from clients (by default, the average, also referred to as FedAvg~\cite{mcmahan2017aistats}).
% \end{enumerate}
This process goes on until the global model converges. %(e.g., after a certain number of rounds or other similar stopping criteria).
%\\
% The advantages of FL over the traditional, centralized learning paradigm are undoubtedly clear in terms of flexibility/scalability (clients can join/disconnect from the FL network dynamically), network communications (only model weights\footnote{We will use \textit{parameters} and \textit{weights} interchangeably.} are exchanged between clients and server), and privacy (each client's private training data is kept local at the client's end and not uploaded to the server).
\\
% Security threats to FL
%However, the growing adoption of FL also raises security concerns~\cite{costa2022covert}, particularly about its confidentiality, integrity, and availability.
Although its advantages over standard ML, FL also raises security concerns~\cite{costa2022covert}. %, particularly about its confidentiality, integrity, and availability~\cite{costa2022covert}.
% OLD, LONG VERSION
% Indeed, some work deals with privacy leakage that may expose the local data of some clients~\cite{melis2019sp}. 
% A large body of work, instead, investigates attacks that usually aim to detriment the predictive accuracy of the learned global model. For instance, \emph{data poisoning} attacks achieve this goal by letting an adversary pollute the training set of some corrupt FL clients with maliciously crafted examples~\cite{jagielski2018sp}.
% Similarly, in \emph{model poisoning} the attacker attempts to tweak the global model weights~\cite{bhagoji2019pmlr} by directly perturbing the local model's weights of some infected FL clients before these are sent to the central server for aggregation, usually via so-called Byzantine attacks. 
% It turns out that Byzantine model poisoning attacks severely impact standard FedAvg; therefore, more robust aggregation functions must be designed to make FL systems secure.
Here, we focus on \emph{untargeted model poisoning} attacks~\cite{bhagoji2019pmlr}, where an adversary attempts to tweak the global model weights %\footnote{We will use the terms \textit{parameters} and \textit{weights} interchangeably.} 
by directly perturbing the local model's parameters of some infected clients before these are sent to the central server for aggregation.
In doing so, the adversary aims to jeopardize the global model \textit{indiscriminately} at inference time.
Such model poisoning attacks severely impact standard FedAvg; therefore, more robust aggregation functions must be designed to secure FL systems.
\\
% In this paper, we focus on designing a novel robust aggregation scheme at the server's end to contrast the effect of Byzantine model poisoning attacks.
%
% Current countermeasures and their limitations
%Several countermeasures have been proposed in the literature to combat model poisoning attacks on FL systems.
% Some methods use simple statistics more robust than plain average to smooth the impact of malicious updates (e.g., Trimmed Mean and FedMedian~\cite{yin2018icml}). 
% Other defenses implement outlier detection techniques to discard malicious updates from the aggregation performed at the server's end. Those are either based on heuristics (e.g., Krum/Multi-Krum~\cite{blanchard2017nips} and Bulyan~\cite{mhamdi2018pmlr}) or data-driven approaches (e.g., K-means clustering~\cite{shen2016acm} or DnC via spectral analysis~\cite{shejwalkar2021ndss}). 
% Finally, some strategies rely on a centralized ``source of trust'' to spot potential malicious updates (e.g., FLTrust~\cite{cao2020fltrust}).
% Several countermeasures have been proposed in the literature to combat model poisoning attacks on FL systems, i.e., to discard possible malicious local updates from the aggregation performed at the server's end. 
% These techniques range from simple statistics more robust than plain average (e.g., Trimmed Mean and FedMedian~\cite{yin2018icml}) to outlier detection heuristics (e.g., Krum/Multi-Krum~\cite{blanchard2017nips} and Bulyan~\cite{mhamdi2018pmlr}) or data-driven approaches (e.g., spectral analysis via K-means clustering~\cite{shen2016acm} or spectral analysis), or methods based on ``source of trust'' (e.g., FLTrust~\cite{cao2020fltrust}).
% OLD, LONG VERSION
%Several countermeasures have been proposed in the literature to combat Byzantine model poisoning attacks on FL systems.
% Descriptive statistics
% For example, Trimmed Mean and FedMedian aggregate local model updates using more robust statistics than standard average~\cite{yin2018icml}.
%
% % Heuristics for outlier detection
% Many existing Byzantine-resilient strategies implement some outlier detection heuristics to discard the model updates sent by potentially malicious clients from the input of the aggregation function.
% One of the most popular heuristics is Krum~\cite{blanchard2017nips}.
% This strategy tries to mitigate the impact of Byzantine attacks by selecting as a global model the local model with the smallest sum of Euclidean distances to {\em all} the other local models.
% Although powerful, Krum requires the server to know (or, at least, estimate) the number of malicious FL clients upfront, which is generally impossible in a realistic attack scenario. %
% Moreover, Krum may become ineffective for complex, high-dimensional model parameter spaces due to the curse of dimensionality.
% Bulyan~\cite{mhamdi2018pmlr} tries to overcome this issue by combining Krum with a variant of Trimmed Mean.
% % Data-driven outlier detection
% Other strategies use data-driven outlier detection techniques -- e.g., via K-means clustering~\cite{shen2016acm} -- to spot potential malicious local model updates. 
% %For instance, Shen et al. propose to cluster local model updates with K-means and thus identify outliers.
%
% % Other techniques
% As far as the server is concerned, any local model received can be from a potential malicious client. 
% FLTrust~\cite{cao2020fltrust} assumes the server acts as a client, i.e., trains a local model on an additional {\em trustworthy} dataset at the server's end and compares it against all the local models from other clients. 
% This way, the server can rely on some ``source of trust'' when discarding potentially malicious clients.
%\\
% Limitations of existing Byzantine-resilient strategies
Unfortunately, existing defense mechanisms either rely on simple heuristics (e.g., Trimmed Mean and FedMedian by~\cite{yin2018icml}) or need strong and unrealistic assumptions to work effectively (e.g., foreknowledge or estimation of the number of malicious clients in the FL system, as for Krum/Multi-Krum~\cite{blanchard2017nips} and Bulyan~\cite{mhamdi2018pmlr}, which, however, cannot exceed a fixed threshold).
Furthermore, outlier detection methods using K-means clustering~\cite{shen2016acm} or spectral analysis like DnC~\cite{shejwalkar2021ndss} do not directly consider the temporal evolution of local model updates received.
Finally, strategies like FLTrust~\cite{cao2020fltrust} require the server to collect its own dataset and act as a proper client, thereby altering the standard FL protocol.
\\
% OLD, LONG VERSION
% Overall, existing Byzantine-resilient strategies are either simple heuristics (e.g., FedMedian) or, if they are more complex, they rely on strong and unrealistic assumptions to work effectively (e.g., knowing the number of malicious clients in the FL system in advance, as for Krum and alike).
% Furthermore, data-driven outlier detection methods do not consider the temporary evolution of local model updates received (e.g., K-means clustering). 
% Finally, strategies like FLTrust requires the server to collect its own dataset and act as a proper client, thereby altering the standard FL protocol.
%
% Description of the proposed method
This work introduces a novel pre-aggregation \textit{filter} robust to untargeted model poisoning attacks. Notably, this filter $(i)$ operates without requiring prior knowledge or constraints on the number of malicious clients and $(ii)$ inherently integrates temporal dependencies. 
The FL server can employ this filter as a preprocessing step before applying \textit{any} aggregation function, be it standard like FedAvg or robust like Krum or Bulyan.
Specifically, we formulate the problem of identifying corrupted updates as a multidimensional (i.e., matrix-valued) time series anomaly detection task. 
The key idea is that legitimate local updates, resulting from well-calibrated iterative procedures like stochastic gradient descent (SGD) with an appropriate learning rate, show \textit{higher predictability} compared to malicious updates. This hypothesis stems from the fact that the sequence of gradients (thus, model parameters) observed during legitimate training exhibit regular patterns, as validated in Section~\ref{subsec:intuition}. %until convergence. 
%This regularity may be more pronounced for smooth convex loss functions, but it can still be captured within an appropriate time window, even for more complex and convoluted loss surfaces. 
%We provide evidence of this claim in Appendix~B, where we show that the average mutual information (i.e., ``predictability''), calculated over pairs of legitimate model updates sent at different FL rounds, is significantly higher than the corresponding computation for a malicious client.
\\
Inspired by the matrix autoregressive (MAR) framework for multidimensional time series forecasting~\cite{chen2021je}, we propose the FLANDERS ({\em \textbf{F}ederated \textbf{L}earning meets \textbf{AN}omaly \textbf{DE}tection for a \textbf{R}obust and \textbf{S}ecure}) filter.
The main advantages of FLANDERS over existing strategies like FLDetector~\cite{zhao2020multivariate} are its resilience to large-scale attacks, where $50\%$ or more FL participants are hostile, and the capability of working under realistic non-iid scenarios.
We attribute such a capability to two key factors: $(i)$ FLANDERS works without knowing a priori the ratio of corrupted clients, and $(ii)$ it embodies temporal dependencies between intra- and inter-client updates, quickly recognizing local model drifts caused by evil players. Below, we summarize our main contributions:

\begin{itemize}
\item[{\em(i)}]
We provide empirical evidence that the sequence of models sent by legitimate clients is more predictable than those of malicious participants performing untargeted model poisoning attacks.
\\
\item[{\em(ii)}] 
We introduce FLANDERS, the first pre-aggregation filter for FL robust to untargeted model poisoning based on multidimensional time series anomaly detection.
\\
\item[{\em(iii)}] 
We integrate FLANDERS into Flower,\footnote{\scriptsize{\url{https://flower.dev/}}} a popular FL simulation framework for reproducibility.
\\
\item[{\em(iv)}] 
We show that FLANDERS improves the robustness of the existing aggregation methods under multiple settings: different datasets, client's data distribution (non-iid), models, and attack scenarios.
\\
\item[{\em(v)}] 
We publicly release all the implementation code of FLANDERS along with our experiments.\footnote{\scriptsize{\url{https://anonymous.4open.science/r/flanders_exp-7EEB}}}
\end{itemize}

% Paper's structure and organization
The remainder of the paper is structured as follows. %some related work and the current state-of-the-art solutions to security issues that FL entails. 
Section~\ref{sec:background} covers background and preliminaries. 
In Section~\ref{sec:related}, we discuss related work.
Section~\ref{sec:problem} and Section~\ref{sec:method} describe the problem formulation and the method proposed. % to tackle it. 
Section~\ref{sec:experiments} gathers experimental results. %, and Section~\ref{sec:limitations} discusses some limitations of this work.
Finally, we conclude in Section~\ref{sec:conclusion}.
 %discusses the limitations of this work and draws future research directions.
%reports conclusions and draws perspectives for future research directions.

%%%%%%% OLD %%%%%%%
%to overcome the resilience of Byzantine failures in distributed Stochastic Gradient Descent computations. 
% The strength of Krum is its time complexity, which is linear in the gradient dimension. 
% However, the robustness of the approach is guaranteed for gradient-based learning applications only when the majority of the clients are not compromised. 
% Besides, the aggregation mechanism of Krum, as well as that of similar methods, is robust from a coarse-grained perspective and does not provide solutions to errors and perturbations that may occur at inference time.
%A related approach to~\cite{blanchard2017nips} is the work of Su et al.~\cite{su2016dc}. Here, the authors propose an iterated approximate agreement to tackle a multi-layer scenario attacked by Byzantine agents. 
%However, the method works efficiently on the sole discrete context and it is inapplicable to continuous state environments.
%\gabri{Maybe, we should just talk about the main limitations of existing countermeasures without digging into their details (or, we can just mention Krum as this is the most popular one). I will move the description of all these methods to the Related Work section.}

\section{A random baseline revisited}
\label{sec:baselines}

% \begin{table*}
% \centering
% \small
% \caption{Comparison of our best classifier (RoBERTa) with Random Classifier (results in \%).}
% \label{table:compare}
% \begin{tabular}{lllllll} 
% \hline
%                & \multicolumn{3}{l}{Our best classifier (RoBERTa)} & \multicolumn{3}{l}{Random Classifier} \\ 
%             \cline{2-3} 
%                & Precision & Recall & F1-score           & Precision & Recall & F1-score              \\ 
% \hline
% Classification & 86.64         & 95.31    & 90.77           & 29    & 50    & 36            \\ 
% \hline
% Improvement    & -         & -      & -            &    2.9x         &    1.9x  &    2.5x         \\
% \hline
% \end{tabular}
% \end{table*}

\begin{table*}[]
\centering
\small
\caption{Comparison of our best classifier (RoBERTa) with Random Classifier (results in \%).}
\label{table:compare}
\begin{tabular}{lllllll}
\hline
\multirow{2}{*}{} & \multicolumn{3}{c}{Our best classifier (RoBERTa)} & \multicolumn{3}{c}{Random Classifier} \\ \cline{2-7} 
                  & Precision        & Recall        & F1-score       & Precision    & Recall    & F1-score   \\ \hline
Classification    & 86.64            & 95.31         & 90.77          & 29.00        & 50.00     & 36.00      \\ \hline
Improvement       & -                & -             & -              & 2.9x         & 1.9x      & 2.5x       \\ \hline
\end{tabular}
\end{table*}

% \section{The optimality scores}
% \label{sec:theory}

% % !TEX root = ./CauchyCombination.tex
\section{Multiple Hypothesis Testing} \label{secPrelims}

This section introduces the notation on multiple hypothesis testing and the benchmark procedures for addressing the multiple testing problem. 

\subsection{Setting}
Let $H_{i}$ denote the $i^{\text{th}}$ null hypothesis of interest, with $i=1,...,d$,  
and $d$ being the total number of individual hypotheses. To test the $d$ hypotheses, we can use the associated vector of test statistics $\bm{X}=(X_{1},X_{2},\ldots,X_{d})^{^{\prime }}$, one for each hypothesis being tested, or the corresponding raw $p$-values $p_{1},\ldots ,p_{d}$. The test statistics can be independent or % corrected.
correlated. 

%In some cases, like in Section \ref{secApplDriftBurst}, the test statistics are constructed from rolling windows and are extremely serially correlated. 

The first task is to test the global null hypothesis. Let $\mathcal{H}_{0}$ be the collection of null hypotheses of interest. 
The strategy of a classical global test is to abandon the multiplicity issue altogether and replace multiple tests with the global null hypothesis that all elementary hypotheses are true.  The alternative is that at least one elementary hypothesis is false. For example, in high-frequency financial econometrics,  we often need to monitor the presence of certain events (e.g., jumps or drift bursts) within a fixed time period (e.g., within a day). The global null is that there is no occurrence of such an event at all (e.g., 
% in Example 2, none of the stocks has a significant alpha
in Example 1, there is no drift burst within the day or in Example 2, none of the stocks has a significant alpha).
The goal is to get $\alpha$-level control under this global null, i.e., $P_{\mathcal{H}_0} [\text{reject}\, \mathcal{H}_0] \leq \alpha$. The test is conservative when $P_{\mathcal{H}_0} [\text{reject}\, \mathcal{H}_0]$ is strictly less than the theoretical upper bound $\alpha$ and ideal when it is equal to  $\alpha$. 

When, by any  test, the global null $\mathcal{H}_0$ is rejected, the second task is to identify which of the elementary hypotheses $H_{i}$ should be rejected. The set of true hypotheses $\mathcal{T}$, the set of false hypotheses $\mathcal{F}$ and the set of rejected hypotheses $\mathcal{R}$ are defined as: 
\begin{align}
	\begin{split}  \label{eqLocalHypothesis}
		\mathcal{T} &= \{ H_{i}\in \mathcal{H}_0: H_{i} \, \text{is true}\}, \\
		\mathcal{F} &= \{ H_{i}\in \mathcal{H}_0: H_{i} \, \text{is false}\}, \text{and} \\
		\mathcal{R} &= \{H_{i}\in \mathcal{H}_0: H_{i} \, \text{is rejected}\}.
	\end{split}%
\end{align}
The set of true and false hypotheses are unknown. We choose a set of hypotheses to reject. 
on the basis of our data. 
The set of discoveries $\mathcal{R}$ 
should coincide with the set of false hypotheses $\mathcal{F}$ as much as possible.

The goal of various multiple testing corrections is to control the familywise error rate (FWER), defined as the probability of at least one false
rejection in the family, $P[\mathcal{T} \cap \mathcal{R} \neq \varnothing]$,
while retaining the reasonable power in detecting false hypotheses. We want procedures for which the FWER is less than or equal to the upper bound $\alpha$ and ideally as close as possible to the upper bound. We focus on strong control of the FWER, meaning that some of the hypotheses we are testing can be false ($\mathcal{F} \neq \varnothing$), as opposed to the weak FWER control where all hypotheses of interest are true, i.e., $\mathcal{H}_0=\mathcal{T}$.

The probability of falsely rejecting a single hypothesis that is true (i.e., false positive or Type I error) is usually controlled at a nominal $\alpha$-level. However, when the number of tested hypotheses is large, the problem of multiplicity arises: the probability of having at least one false positive conclusion rises well above $\alpha$ if the Type I error of each individual test is controlled at the $\alpha$-level. Numerous controlling procedures have been proposed to deal with this problem. 
We review two classes of controlling procedures: one based on statistical inequalities (Section \ref{ssecOrderdPvals}) and one based on the maximum of the test statistics (Section \ref{ssecMaxTest}).


\subsection{Procedures based on statistical inequalities}
\label{ssecOrderdPvals}

Let us denote by $0 < p_{(1)}\leq p_{(2)}\leq \ldots\leq p_{(d)} < 1$ the set of $d$  ordered (in ascending order) raw $p$-values and $H_{(1)},H_{(2)},\ldots,H_{(d)}$ their corresponding null hypotheses. A single-stage method uses the same rejection 
criterion for all individual hypotheses, like the conservative Bonferroni threshold, while a multi-stage method examines the ordered $p$-values sequentially and adjusts the rejection criterion for each of the individual tests  \citep[e.g.,][]{holm1979simple,hochberg1988sharper,hommel1988stagewise}. 

The Bonferroni method rejects the elementary null hypothesis $H_{(i)}$ if 
$p_{(i)}\leq\alpha/d$ 
for $i=1,\ldots,d$. \citet{holm1979simple} and \citet{hochberg1988sharper} use the same critical values $ \alpha / (d - i + 1)$ depending on the rank of the $p$-value, but reject differently depending on whether they ``step up" or ``step down". The terminologies (``step up" or ``step down") were originally formulated in terms of test statistics which can be confusing when discussing $p$-values.  \citet{holm1979simple} proposes a step-down method that ``steps up" from the smallest $p$-value to the largest one. It is a pessimistic approach: it scans forward and stops as soon as a $p$-value fails to clear its threshold. \citet{hochberg1988sharper} suggests a step-up method that ``steps down" from the largest $p$-value to the	smallest one. It is an optimistic approach: it scans backward and stops as soon as a $p$-value succeeds in clearing its threshold. By construction, Hochberg's procedure will reject as many hypotheses as Holm's procedure. 

\cite{hommel1988stagewise} proposes a more complicated procedure which applies  \citet{simes1986improved}' global test to the $p$-value subset $\left\{ p_{\left(k\right) }\right\} _{ k = i }^{d}$, instead of relying  only on $p_{\left(i\right)}$ to draw inference on $H_{(i)}$ and thus borrows power across hypotheses. 
Hommel's procedure is shown to have higher power than Hochberg's method \citep{hommel1989}. We refer to Appendix \ref{AppOrderedPVals} for more details on the practical implementation of these procedures.


Bonferroni and \citet{holm1979simple}'s method are based on the first-order Bonferroni inequality, which states that given any set of events, the probability of their union is smaller than or equal to the sum of their probabilities. 
Under the null hypothesis, 
the probability that there is at least one hypothesis $H_{(i)}$  for which its raw $p$-value $p_{(i)} \leq \alpha / d$ 
% is not greater than $\alpha$ 
is bounded by $\alpha$: 
\begin{align}  \label{eqIneqPval}
	\Pr\left( \min_{i} p_{(i)} \leq \frac{\alpha}{d} \right) 
	= \Pr\left(\bigcup_{i =
		1}^{d} \left\{ p_{(i)} \leq \frac{\alpha}{d} \right\} \right) 
	&\leq
	\sum^{d}_{i = 1} \Pr \left( p_{(i)} \leq \frac{\alpha}{d}\right) \leq d
	\frac{\alpha}{d} \leq \alpha.
\end{align}
The Bonferroni \eqref{eqIneqPval} inequality 
makes no specific assumption on the dependence between the $p$-values, but protects against the so-called ``worst-case", in which all events are independent and the rejection regions are disjoint (the right half of Equation \eqref{eqIneqPval}) . 
The inequality becomes an equality when all test statistics are independent, and a strict inequality when the hypotheses are dependent. 
In other words, the Bonferroni correction is  conservative when the $p$-values are correlated. 

The methods of \cite{hochberg1988sharper} and \cite{hommel1988stagewise} are based on  \cite{simes1986improved}'s inequality. If a set of hypotheses $H_{(1)}, ...,H_{(d)}$ are all true, the probability of the joint event is: 
\begin{equation}  \label{eqSimes}
	\Pr\left( p_{\left( i\right) }> \frac{i\alpha}{d}, \text{ for all } i=1,\ldots
	,d\right) \geq 1-\alpha.
\end{equation}
\citet{simes1986improved}' inequality was developed for independent uniform $p$-values, and it is applicable for a large family of multivariate distributions. The simulations of \citet{simes1986improved} do show, however, that the test is very conservative for highly correlated multivariate normal statistics, but less so than the classical Bonferroni correction. 



\subsection{Procedures based on the maximum of test statistics}
\label{ssecMaxTest}

Another class of controlling procedures uses the maximum in a group of test statistics: $X_m = \max_{i} \abs{X_{i}}$, with $i = 1, \ldots, d$, to set a stringent critical value. The same critical value can be used for each elementary hypothesis and will control the familywise error rate. In particular, when the individual test statistics are independent and follow the standard normal distribution under the null, the maximum of the test statistics follows a Gumbel distribution when $d$ is large. Quantiles of the Gumbel distribution were used as critical values of the individual tests as a multiple testing correction when, for example, conducting jump tests in high-frequency asset returns \citep[][]{lee2007jumps}. Unfortunately, if the sequence of test statistics exhibits strong correlation, the number of tests severely overstates the effective number of independent copies in a given sample, which makes the Gumbel critical values too conservative \citep[see e.g.,][]{christensen2018drift}. We refer to Appendix \ref{AppOrderedPVals} for more details on the Gumbel distribution. 

Resampling-based methods account for the dependence structure that is specific to the considered dataset, leading to less conservative testing outcomes than the Gumbel-based methods and the inequality-based procedures. Depending on the empirical problem of interest, the resampling can be carried out by bootstrap, permutation, simulation, or randomization \citep[see e.g.,][for detailed discussions on resampling methods and testing procedures]{white2000,romano2005exact,romano2005stepwise,lehmann2005testing}. We refer to Section \ref{secApplDriftBurst} for an example of the resampling-based approach for the drift burst test. 


\section{Cauchy Combination Tests}
\label{secSeqCauchy}

In this section, we first review the global Cauchy combination (GCC) test of \citet{liu2020cauchy} and present our sequential Cauchy combination (SCC) test. While the global test tests the global null hypothesis $\mathcal{H}_0 = \bigcap_{i=1}^{d} H_{i}$ against the alternative hypothesis that at least one of the elementary null hypotheses is false, the sequential test aims at identifying the violations of the elementary null hypotheses while controlling the global error rate. 

\subsection{Global Cauchy combination test}
\label{sec:CC}

The GCC test statistic is constructed from the raw $p$-values of the test statistics $X_i$, which follow a uniform distribution between $0$ and $1$ under the  null hypothesis. The idea of this test is first to transform the uniformly distributed $p$-values into standard Cauchy variates using the formula $\tan \{(0.5-p_{i})\pi \}$ and then construct a new test statistic as the weighted sum of these transformed $p$-values. The new test statistic is denoted by $\tilde{T}$ and defined as: 
\begin{equation}
	\label{eqCauchyStatistic}
	{\normalsize \tilde{T}=\sum_{i=1}^{d}w_{i}\tan \{(0.5-p_{i})\pi \},} 
\end{equation}
in which the $w_{i}$'s are non-negative weights summing to 1. Throughout the paper, the weights $w_{i}$ are set to $1/d$ for $i=1,\ldots,d$ as in \citet{liu2020cauchy}. 

When the raw and hence the transformed $p$-values are independent (resp. perfectly correlated), % under the null hypothesis, 
the new test statistic $\tilde{T}$ is a linear combination of independent (resp. perfectly correlated) Cauchy variates and therefore follows a standard Cauchy distribution because the family of Cauchy densities is closed under convolutions. Although the correlation structure can affect the null distribution of $\tilde{T}$ in the case of general dependence, \citet{liu2020cauchy} show that the impact on the tail is very limited because of the heaviness of the Cauchy tail. Specifically, they prove that: 
\begin{equation}
	\lim_{h\rightarrow \infty }\frac{\Pr\left( \tilde{T}>h\right) }{\Pr\left(
		C>h\right) }=1,  \label{eq:tail}
\end{equation}
in which $C$ is a standard Cauchy random variable, under the null hypothesis $\mathcal{H}_0$ and Assumption \ref{ass1} which requires the test statistics to follow a bivariate zero mean normal distribution.
\begin{assumption}
	\label{ass1} (1) The original test statistics $(X_{i},X_{j})$, for any $1\leq
	i<j\leq d$, follow a bivariate normal distribution; (2) $E\left( \bm{X}%
	\right) =0$, with $\bm{X}=(X_{1},X_{2},\ldots,X_{d})^{^{\prime }}$. 
\end{assumption}
The bivariate normal requirement of Assumption \ref{ass1} is a condition weaker than joint normality, making the procedure applicable for high-dimensional settings. When the dimension $d$ increases at a certain rate with the sample size, the test statistics $\bm{X}$ may not jointly converge to a multivariate normal distribution due to its slower rate of convergence \citep[see][and references therein]{liu2020cauchy} and thus a joint normality assumption is not realistic for those settings. In contrast, the bivariate normality assumption is much weaker and more realistic. There are, of course, applications for which the test statistics are not normally distributed. Through simulations, \citet[][]{liu2020cauchy} show the Cauchy approximation is still accurate when the normality assumption is violated and follows a multivariate Student-$t$ distribution (with 4 degrees of freedom) instead. 
We refer to Section \ref{secApplFan} for a showcase example in finance with test statistics being Student-$t$ distributed. 

The result in  \eqref{eq:tail} suggests that, under the null hypothesis $\mathcal{H}_0$, the tail of the Cauchy combination test statistic is approximately Cauchy under arbitrary dependence structures, so that a $p$-value of the Cauchy combination test, denoted 
$\widetilde{p}$, can  be calculated from the standard Cauchy distribution. Suppose that we observe $\tilde{T}=t_{0}$, then: 
\begin{equation}
	\label{eqCauchyPval}
	\widetilde{p}=\frac{1}{2}-\frac{\arctan t_{0}}{\pi }. 
\end{equation}

Using the GCC $p$-values, the tail result in \eqref{eq:tail} can be equivalently stated as the actual size converging to the nominal size $\alpha$ as the significance level tends to zero:  % \textit{i.e.}, 
\begin{equation}
	\lim_{\alpha\rightarrow 0 }\frac{\Pr\left( \widetilde{p} \leq \alpha\right) }{\alpha}=1, \label{eq:wFWER}
\end{equation} 
The approximation should be particularly accurate for small $\alpha$'s, which are of particular interest in large-scale problems as in Examples 1 and 2. The simulations in \citeauthor{liu2020cauchy} show that when the significance level is moderately small ($\alpha = 10^{-1}, 10^{-2},10^{-3},10^{-4},10^{-5}$), the $p$-value calculation is accurate:  the ratio of the empirical size to the significance level is close to 1 for different types of correlations. 
Put differently, the GCC test achieves the weak familywise error rate control as the empirical size is very close to the nominal size $\alpha$ regardless of the correlation structure. 


Figure \ref{figCauchyPvalsAR} illustrates the fact that while the dependence between the individual test statistics $X_i$ can affect the null distribution of the GCC test statistic, the impact of the dependence is marginal on the tail. We simulate a vector of $d$ test statistics  $\bm{X}$ from a $d$-variate normal distribution with correlation matrix $\bm{\Sigma}$, \textit{i.e.}, $N_d(\bm{0}, \bm{\Sigma})$ with $\bm{\Sigma} = (\sigma_{ij})$ and $d = 300$. The diagonal elements $\sigma_{ii}=1$ for all $i=1,\ldots,d$ and the off-diagonal elements $\sigma_{ij} = \theta^{\abs{i-j}}$ for $i \neq j$, with $\theta = 0.2, 0.4, 0.6,$ $0.8, 0.90, 0.95$. The simulation is repeated $10^7$ times. For each draw, we calculate the GCC test statistic \eqref{eqCauchyStatistic} and its corresponding $p$-value \eqref{eqCauchyPval}. The histogram of the $10^7$ GCC $p$-values is displayed in Figure \ref{figCauchyPvalsAR}. For a low level of autocorrelation (i.e., $\theta=0.2$), the distribution of the $p$-values is close to a uniform distribution. When the level of autocorrelation is higher, there is a pothole in the middle and a bump at the end of the histogram, but whatever the strength of the autoregressive parameter, the percentage of the GCC $p$-values in the first bin is always around $5$\% as is ensured by the limit result in \eqref{eq:wFWER}. 


\begin{figure}[p]
	\caption{The impact of dependence on the tail of the GCC test statistic}
	\label{figCauchyPvalsAR}
	\centering
	
	\par
	
	\subfloat[${\theta} = 0.2$ ]{{\includegraphics[width=.40\textwidth,angle =
			-90,scale=0.70]{Sample_pval_dim_300_rho_0.2_alpha0.05.eps} }} 
	\subfloat[$\theta = 0.4$ ]{{\includegraphics[width=.40\textwidth,angle =
			-90,scale=0.70]{Sample_pval_dim_300_rho_0.4_alpha0.05.eps} }} 
	
	\vspace{0.4cm}
	
	\subfloat[$\theta = 0.6$ ]{{\includegraphics[width=.40\textwidth,angle =
			-90,scale=0.70]{Sample_pval_dim_300_rho_0.6_alpha0.05.eps} }} 
	\subfloat[$\theta = 0.8$ ]{{\includegraphics[width=.40\textwidth,angle =
			-90,scale=0.70]{Sample_pval_dim_300_rho_0.8_alpha0.05.eps} }} 
	
	\vspace{0.4cm}	
	
	\subfloat[$\theta = 0.90$]{{\includegraphics[width=.40\textwidth,angle =
			-90,scale=0.70]{Sample_pval_dim_300_rho_0.9_alpha0.05.eps} }} 
	% 
	\subfloat[$\theta = 0.95$]{{\includegraphics[width=.40\textwidth,angle =
			-90,scale=0.70]{Sample_pval_dim_300_rho_0.95_alpha0.05.eps} }} 
	
	\par
	\begin{minipage}{1.0\linewidth}
		\begin{tablenotes}
			\small
			\item {
				\medskip
				Note: We plot histograms of GCC $p$-values \eqref{eqCauchyPval} for various correlation strengths. The individual test statistics are drawn from a $d$-variate normal distribution $N_d(\bm{0}, \bm{\Sigma})$ with $\bm{\Sigma}= (\sigma_{ij})$ and $d=300$. The diagonal elements of the covariance matrix $\sigma_{ii}=1$ for all $i=1,\ldots,d$ and the off-diagonal elements  $\sigma_{ij} = \theta^{\vert i-j \vert}$ for $i\neq j$, with $\theta = 0.2, 0.4, 0.6,$ $0.8, 0.90, 0.95$. We compute the GCC $p$-value from the test statistic sequence. The simulation is repeated $10^7$ times. The simulated GCC $p$-values are sorted into bins with the bin edges being a sequence of edges from 0 to 1 with a width of  0.05. 				Each bin includes the right edge (right-closed) but does not include the left edge (left-open). We highlight the first bin in black and we
				also add a text note with the probability of $p$-values being in the first bin. }
		\end{tablenotes}
	\end{minipage}
\end{figure}

Interestingly,  \citet{liu2020cauchy} show that the tail property \eqref{eq:tail}  also holds when the number of hypotheses $d$ diverges to infinity at a rate of $o\left(h^{\eta }\right) \ $with $0<\eta <1/2$ and the following additional assumption is satisfied.
\begin{assumption}
	\label{ass2} Let $\mathbf{\Sigma }=corr\left( \bm{X}\right) $. 
	(1) The	largest eigenvalue of the correlation matrix $\lambda _{\max}\left( \mathbf{%
		\Sigma }\right) \leq C_0$ for some constant $C_0>0$; 
	(2) $\max_{1\leq i<j\leq
		d}\left\{ \sigma _{i,j}^{2}\right\} \leq \sigma _{\max }^{2}<1$ for some
	constant $0<\sigma _{\max }^{2}<1$, where $\sigma _{i,j}$ is the $\left(
	i,j\right) $ element of $\mathbf{\Sigma }$.
\end{assumption}
The additional assumptions on the correlation matrix are mild and standard in high dimensional settings and are general enough to incorporate a large class of tests. 


\subsection{Sequential Cauchy Combination Test}

The main contribution of this paper is the sequential Cauchy combination (SCC) test, which unravels the GCC test to make statements on the elementary hypotheses. The raw $p$-values are sorted in ascending order so that  $p_{(1)}\leq p_{(2)}\leq \ldots \leq p_{(d)}$, which is standard for step-down and step-up sequential procedures (see Section \ref{ssecOrderdPvals}). For the inference on hypothesis $H_{\left( i\right) }$ we compute a Cauchy combination test statistic ${\normalsize \tilde{T}}_{\left( i\right) }$ from a subset of $p$-values, running from $p_{(i)}$ to $p_{(d)}$ as:  
\begin{equation}
{\normalsize \ \tilde{T}%
		_{\left( i\right) }=\sum_{j=i}^{d}w_{j}\tan \{(0.5-p_{(j)})\pi \}
	}.
	\label{eq:CC_mt}
\end{equation}
The corresponding $p$-value is: $$ \widetilde{p}_{(i)}=\frac{1}{2}-\frac{\arctan \tilde{T}_{\left(i\right) }}{\pi }.$$ We reject the null hypothesis $H_{(i)}$ if  $\widetilde{p}_{(i)}\leq\alpha$. Like the step-up procedure of  \citet{hommel1988stagewise}, the SCC test also borrows power across hypotheses: the test statistic $\tilde{T}_{(i)}$ is computed from the raw $p$-values associated with $\mathcal{H}_0^{(i)}=\bigcap_{j=i}^{d} H_{(j)}$.


\subsubsection*{Theoretical Properties}
The SCC testing procedure can be viewed as a sequential rejection procedure. Let $\mathcal{R}^{(s)}$ be the collection of rejected hypothesis after step $s$, with $s=\left\{1,2,\ldots,d\right\}$. The hypothesis of interest and decision rules in each step  are illustrated in Table \ref{tabDecisionRule}.
\begin{table}[H]
	\caption{Decision rule in the sequential Cauchy combination test}
	\label{tabDecisionRule}
	\centering
	\begin{tabular}{p{1.cm}p{3.8cm}p{10.5cm}}
		\hline
		Step & Hypothesis & Decision\\ 
		$s=1$ & $\mathcal{H}_0^{\left( d\right) }=H_{(d)}$ & 
		If $\widetilde{p}_{(d)}\leq\alpha$ then reject $\mathcal{H}_0^{\left( d\right) }$ and include $H_{(d)}$ in $\mathcal{R}^{(1)}$
		\\
		$s=2$ & $\mathcal{H}_0^{\left( d-1\right) }=\bigcap_{j=d-1}^d H_{(j)}$  
		& 
		If $\widetilde{p}_{(d-1)}\leq\alpha$ then reject $\mathcal{H}_0^{\left( d-1\right) }$  and include $H_{(d-1)}$ in $\mathcal{R}^{(2)}$  
		\\
		$\ldots$  & $\ldots$  & $\ldots$  \\ 
		$s=d$ & $\mathcal{H}_0^{\left( 1\right) }=\bigcap_{j=1}^d H_{(j)}$ & 
		If $\widetilde{p}_{(1)}\leq\alpha$ then reject $\mathcal{H}_0^{\left( 1\right) }$ and include $H_{(1)}$ in $\mathcal{R}^{(d)}$ \\
		\hline
	\end{tabular}
\end{table}
Let $\mathcal{N}\left(\mathcal{R}^{(s)}\right)$ be the successor function, representing hypotheses to be rejected in the next step given that $\mathcal{R}^{(s)}$ has been rejected. For the SCC test, the successor function is defined as: 
\[
\mathcal{N}\left(\mathcal{R}^{(s)}\right)=\left\{H_{(d-s)} :  \widetilde{p}_{(d-s)} \leq \alpha_{\mathcal{R}^{(s)}}=\alpha\right\}.
\]
The cut-off value is fixed (i.e., $\alpha_{\mathcal{R}^{(s)}}=\alpha$) instead of depending on the rejection set $\mathcal{R}^{(s)}$ like in many other sequential procedures. According to the sequential rejection principle of \cite{goeman2010sequential},  the SCC test  achieves a strong family-wise error rate control if the following two conditions are satisfied. 
\begin{condition}[Monotonicity]
	For every $\mathcal{R}^{(s)}\subseteq \mathcal{R}^{(l)} \subset \mathcal{H}_{0}$, 
	\[
	\mathcal{N}(\mathcal{R}^{(s)}) \subseteq \mathcal{N}(\mathcal{R}^{(l)}) \cup \mathcal{R}^{(l)}
	\]
	almost surely. 
\end{condition}
% By construction, 
The transformed $p$-values of the SCC test are monotonic by construction, with $\widetilde{p}_{(d)}$ being the largest for the smallest set of global null hypotheses $\mathcal{H}_0^{(d)} = H_{(d)}$ and $\widetilde{p}_{(1)}$ being the smallest for the largest set of global nulls $\mathcal{H}_0^{(1)} = \bigcap_{j=1}^{d} H_{(j)}$ (see Figure \ref{figSequentialCauchyIllustration}(e) for an illustration of the monotonic $p$-values). Note that the largest set of global null hypotheses has the same null specification as the GCC test \eqref{eqCauchyStatistic}. It follows that that $\widetilde{p}_{(s)}\geq \widetilde{p}_{(l)}$. Since the cut-off value is fixed, the monotonicity condition of the successor function is satisfied.

\begin{condition}[Single-step condition] \label{SS} 
	When $\mathcal{H}_{0}^{(i)} =\mathcal{T}$, 
	$\Pr\left( \widetilde{p}_{(i)} \leq \alpha\right) \leq \alpha. $
\end{condition}
Condition \ref{SS} requires FWER control of the underlying test of SCC (i.e., the Cauchy combination test) at the ``critical case" in which all hypotheses of interest are true: $\mathcal{H}_{0}^{(i)} =\mathcal{T}$. The condition can be rewritten as $\Pr{\mathcal{N}(\mathcal{F})\subseteq \mathcal{F}} \geq 1-\alpha$ and has been shown to be satisfied by \cite{liu2020cauchy}. In fact,  when $\alpha$ is very small, the familywise false rejection probability of the GCC test under the null is not only bounded by $\alpha$ but also approaches the nominal size $\alpha$, as stated in \eqref{eq:wFWER}, which implies that it is less conservative than tests based on statistical inequalities or tests which impose independence in the presence of correlation. 

The theorem below follows directly from \cite[Theorem 1]{goeman2010sequential} for general sequential rejection procedures, so that Type I control in the critical case is sufficient for overall familywise error control of the sequential procedure. 
\begin{theorem}\label{thm}
	The SCC testing procedure satisfies both the monotonicity and the single-step condition and achieves the strong FWER control:  
	\[
	\lim_{\alpha\rightarrow 0}\Pr\left\{\mathcal{R}^{(d)} \subseteq \mathcal{F} \right\} \geq 1-\alpha, 
	\]
	under Assumption \ref{ass1} if $d$ is fixed and under Assumptions \ref{ass1} and \ref{ass2} if $d\rightarrow \infty$.
\end{theorem}


\subsubsection*{An Illustration}

A more prescriptive description of the SCC testing procedure is as follows: 
\begin{enumerate}
	
	\item Obtain raw $p$-values $p_1, p_2,\ldots, p_d$ corresponding to the null hypotheses $H_{1}, H_{2},\ldots, H_{d} $;%
	
	\item Order the raw $p$-values in ascending order, 	$p_{(1)},p_{(2)},\ldots,p_{(d)}$, with corresponding null ordered hypotheses $H_{(1)},H_{(2)},\ldots,H_{(d)}$;
	
	\item Calculate the SCC test statistic $\tilde T_{(i)}$ and the transformed Cauchy $p$-values $\widetilde{p}_{(i)}$ from a subset of the ordered $p$-values $\left\{p_{(j)}\right\} _{j=i}^{d}$ using \eqref{eq:CC_mt} for $i=1,\ldots,d$;
	
	\item Obtain the rejection set $\mathcal{R}=\left\{H_{\left(i\right)} : \widetilde{p}_{(i)}\leq \alpha\right\}$. 
\end{enumerate}

Figure \ref{figSequentialCauchyIllustration} illustrates the sequential Cauchy combination procedure on a simulated sequence of test statistics. The top row shows the simulated test statistics and their corresponding $p$-values, of which some hypotheses are under the null (grey dots) and some are under the alternative (black dots). The data-generating process is the same as that in Figure \ref{figCauchyPvalsAR} with $\theta=0.9$ and $d=100$. We add constant signals for $5$ out of 100 hypotheses,  with a signal strength equal to $\pm2.806$. The sign of the signal is the same as the sign of the test statistic under the null, such that the signal always amplifies the magnitude of the test statistic. 
The GCC test rejects the global null at $\alpha = 5\%$ for this sequence of $p$-values, which tells us there is at least one signal in the sequence.  
%The estimated first-order autocorrelation of the simulated test statistics is equal to $0.7910$ under the null and is equal to $0.4987$ under the alternative. 

\begin{figure}[p]
	\caption{Rejection procedure of the sequential Cauchy Combination test}
	\label{figSequentialCauchyIllustration}\centering

	\par
	
	\subfloat[Raw test statistics]{{\includegraphics[width=.31\textwidth,angle =
			-90]{1_tstat_d_100_rho_0.9_signal_5_5} }} 
	\subfloat[Raw
	$p$-values]{{\includegraphics[width=.31\textwidth,angle = -90]{2_pvals_d_100_rho_0.9_signal_5_5} }}
	
	\vspace{0.4cm}	
	
	\subfloat[Ordered raw $p$-values]{{\includegraphics[width=.31\textwidth,angle =
			-90]{3_spvals_d_100_rho_0.9_signal_5_5} }} 
	
	\vspace{0.4cm}	
	
	\subfloat[SCC test statistics
	]{{\includegraphics[width=.31\textwidth,angle = -90]{4_ctstats_d_100_rho_0.9_signal_5_5}}}
	\subfloat[SCC 
	$p$-values]{{\includegraphics[width=.31\textwidth,angle = -90]{4_cpvals_d_100_rho_0.9_signal_5_5}}}
	
	
	\begin{minipage}{1.0\linewidth}
		\begin{tablenotes}
			\small
			\item {
				\medskip
				Note: We illustrate the mechanics of the SCC procedure on a simulated test statistic sequence with sparse signals. The top row shows raw test statistics and $p$-values of which some hypotheses are under the null and some are under the alternative. The test statistics are simulated from $N_{d}(\bm{0},\bm{\Sigma})$ as in Figure \ref{figCauchyPvalsAR}. We set $d=100$, $\theta=0.9$ and add $5\%$ signals. The strength of the signal is $\pm2.806$, with its sign identical to that of the test statistic under the null. The horizon line in panel (e) is the 5\% significance level.
			}
		\end{tablenotes}
	\end{minipage}
\end{figure}

The SCC test can tell us which individual $p$-values trigger the rejection of the GCC test. The middle row plots the raw $p$-values in ascending order and the bottom row plots its sequential Cauchy combination test statistics and $p$-values. Specifically, the bottom right panel shows that the SCC $p$-values $\widetilde{p}_{(i)}$ decrease as $i$ moves from $d$ to $1$. In this example, the SCC test rejects three out of the five alternative hypotheses and does not reject under the null hypothesis. The rejections correspond to the 4$^\text{th}$, 29$^\text{th}$ and  46$^\text{th}$ hypotheses in the top row. Note that the smallest SCC $p$-value corresponds to the $p$-value of the GCC test of \citet{liu2020cauchy}, which performs the test on the largest set of hypotheses. 

\section{Material}
\label{sec:material}

% This chapter (Material, Corpus, Data\ldots) should contain a detailed description of the corpus/data. The data should be freely available to all for use unless there is a serious reason not to do so (this reason should be explained). The data should be ideally available on some online data repository such as GitHub, CLARIN, etc.

\subsection{General information about corpora and languages}

We investigate the relationship between the frequency of a word and its length in languages from two collections: Common Voice Forced Alignments (\autoref{CVFA}), hereafter CV, and Parallel Universal Dependencies (\autoref{PUD}), hereafter PUD. 

All the preprocessed files used to produce the results from the original collections are available in the repository of the article\footnote{In the \textit{data} folder of \url{\repository}.}.

PUD comprises 20 distinct languages from 7 linguistic families and 8 scripts (\autoref{tab:coll_summary_pud}).
CV comprises 46 languages from 14 linguistic families (we include 'Conlang', i.e. 'constructed languages', as a family for Esperanto and Interlingua) and 10 scripts (\autoref{tab:coll_summary_cv}).
Both PUD and CV are biased towards the Indo-European family and the Latin script. The typological information (language family) is obtained from Glottolog 4.6\footnote{\url{https://glottolog.org/}}. %(hence Turkish is from the Turkic family, not Altaic as in the  \href{https://wals.info/}{World Atlas of Language Structures}).
The writing systems are determined according to ISO-15924 codes\footnote{\url{https://unicode.org/iso15924/iso15924-codes.html}}. In \autoref{tab:coll_summary_pud} and \autoref{tab:coll_summary_cv}, we show the scripts using their standard English names. For example, most languages from the Indo-European family are written in Latin scripts. We also categorize Chinese Pinyin and Japanese Romaji as Latin scripts.

% SUMMARY TABLES

\begin{table}[H]
\centering
\caption{Summary of the main characteristics of the languages in the PUD collection. For each language, we show the linguistic family, the writing system (namely script name according to ISO-15924) and various numeric parameters: $A$, the observed alphabet size (number of distinct characters), $n$, the number of word types, and $T$, the number of word tokens.}
\label{tab:coll_summary_pud}
\begin{tabular}{lllrrr}
\hline
Language & Family & Script & $A$ & $n$ & $T$ \\ 
\hline
\input{tables/coll_summary_pud}
\end{tabular}
\end{table}


\begin{table}[H]
\centering
\caption{Summary of the main characteristics of the languages in the CV collection. For every language we show its linguistic family, the writing system (namely script name according to ISO-15924) and various numeric parameters: $A$, the observed alphabet size (number of distinct characters), $n$,  the number of word types, and, $T$, the number of word tokens.
'Conlang' stands for 'constructed language', that is an artificially created language. This is not a family in the proper sense as Conlang languages are not related in the common linguistic family sense.
} 
\label{tab:coll_summary_cv}
\begin{tabular}{lllrrr}
\hline
Language & Family & Script & $A$ & $n$ & $T$ \\ 
\hline
\input{tables/coll_summary_cv}
\end{tabular}
\end{table}

\subsection{The datasets}

We measure word length in two main ways: \textit{duration in time} and \textit{length in characters}. Concerning Chinese and Japanese, we additionally consider the number of strokes and the number of characters of their romanization (i.e. Pinyin for Chinese and Romaji for Japanese). 
% Their traditional writing systems yield very short word lengths in characters, while the number of distinct characters is very large, especially compared to Western languages with mostly alphabetic writing systems \parencite{Chen2015a, Joyce2012a} \footnote{See \autoref{tab:coll_summary_pud} for alphabet size and \autoref{tab:opt_scores_pud} for average word length in Chinese and Japanese.}. We want to test whether these differences are reflected in optimality scores. 

%This poses questions on the correct methodology to capture the studied phenomenon. \textcolor{violet}{is this previous part supposed to be here?}
 
 Given these datasets, word durations are obtained only from CV. Word lengths in characters are obtained from both CV as well as from PUD. Word lengths in strokes, and word lengths in characters after romanization, are obtained only from PUD.

\subsubsection{Common Voice Forced Alignments} \label{CVFA}

The Common Voice Corpus\footnote{\url{https://commonvoice.mozilla.org/en/datasets}} is an open source dataset of recorded voices uttering sentences in many different languages. The amount of data, as well as the source and topic of each sentence, depends considerably on the language and the corpus version. Specifically, the Common Voice Corpus 5.1 contains information on 54 languages and dialects.

Common Voice Forced Alignments (CVFA)\footnote{\url{https://github.com/JRMeyer/common-voice-forced-alignments}} were created by Josh Meyer using the Montreal Forced Aligner\footnote{\url{https://github.com/MontrealCorpusTools/Montreal-Forced-Aligner}} on top of the Common Voice Corpus 5.1. Kabyle, Upper Sorbian and Votic were left out of the alignments for an undocumented reason. Therefore, CVFA contains information on 51 languages.

In our analyses, Japanese and the three Chinese dialects were excluded as the forced aligner failed to correctly extract words from sentences. In addition, both Romansh dialects were fused into a single Romansh language. 
% resulting in a single occurrence of each remaining language.
Indeed, given the nature of this corpus, all languages are likely to be represented by more than one dialect.

Notice that Abkhazian, Panjabi, and Vietnamese have a critically low number of tokens ($T<1000$ in \autoref{tab:coll_summary_cv}). However, we decided to include them in the analyses so as to understand their limitations related to corpus size.

\subsubsection{Parallel Universal Dependencies} \label{PUD}

The Universal Dependencies (UD)\footnote{\url{https://universaldependencies.org/}} collection is an open source dataset of annotated sentences, in which the amount of data depends on each language. The Parallel Universal Dependencies (PUD) collection is a parallel subset of 20 languages from the UD collection, consisting of 1000 sentences. % It allows for a cross-language comparison that takes into account content and annotation style. 
It allows for a cross-language comparison, controlling for content and annotation style.

% The collection was already tokenized, and we thus took all the available tokens excluding those with the POS tag 'PUNCT'. Moreover, we filtered out the tokens not tagged as 'PUNCT' but containing ASCII punctuation and digits, and additional symbols that were still retained in some languages a posteriori. 
In \autoref{tab:coll_summary_pud}, we show the characteristics of the languages in PUD. For traditional Chinese and Japanese, we also include word lengths in romanizations (Pinyin and Romaji respectively), as well as word lengths measured in strokes,
resulting in a total of 24 language files. 
% Chinese here refers to traditional Chinese. 
Notice that three Japanese words that are hapax legomena  could not be romanized and thus the number of tokens and types varies slightly with respect to the original Japanese characters (\autoref{tab:coll_summary_pud}). 
% Thus Japanese in strokes follows the setting of the weak recoding problem approximately. 









%==================================================
% Methodology

\section{Methodology}
\label{sec:methodology}

\section{Method}
\label{s:method}

We consider the 3D euclidean space $\Real^3$ with points $p=(x,y,z)\in\Real^3$. We discretize the unit cube $\gC=[0,1]^3$ with a 3D voxel grid $\gG=\set{p_I}$, with nodes $p_I$ indexed by $I=(i,j,k)$, $i,j,k\in [n]=\set{1,\ldots,n}$, \ie, $p_I=(x_{ijk},y_{ijk},z_{ijk})$. We denote by $h=n^{-1}$, and by $N=n^3$ the total number of nodes.   
We represent our reconstructed surface as a zero level of a scalar function $f$ defined over the cube $\gC$. $f$ is defined by prescribing its values at the grid's nodes $f_I\in\Real$ and trilinear interpolating in each voxel. We will denote by $f(p)$ the interpolated value at point $p$. 

Given an input point cloud consisting of $m$ points $q_k\in\Real^3$ with or without (unit norm) normals $n_k\in \Real^3$, $k\in [m]$, our goal is to compute $f$ so that its zero level set approximates the unknown surface, \ie, 
\begin{equation}
    \gS = \set{p\in\gC \ \vert \ f(p)=0}.
\end{equation}
Our approach to compute $f$ is to minimize a loss function of the form
\begin{equation}
    \gL = \gL_{\text{data}} + \gL_{\text{prior}}
\end{equation}
where 
\begin{equation}\label{e:loss_data}
    \gL_{\text{data}} = \frac{\lambda_{\text{p}}}{m}\sum_{k=1}^m \abs{f(q_k)}^2 + \frac{\lambda_{\text{n}}}{m}\sum_{k=1}^m \norm{\nabla f(q_k) - n_k}^2
\end{equation}
where $\norm{\cdot}$ is the standard euclidean norm in $\Real^3$, $\nabla f(p) \in \Real^3$ is the gradient of $f$ sampled at point $p$. Note that $\nabla f$ is defined in interior of voxels, which is generically where the input points $q_k$ resides. $\gL_{\text{data}}$ is the standard data loss encouraging the zero level to pass through the input points $q_k$, and its normals (defined by gradients of $f$) to coincide with input normals $n_k$. 

The prior, $\gL_{\text{prior}}$, is the main contribution of this work, where we combine two novel losses,
\begin{equation}
    \gL_{\text{prior}} = \lambda_{\text{v}} \gL_{\text{viscosity}} + \lambda_{\text{c}} \gL_{\text{coarea}}
\end{equation}
Intuitively, the viscosity loss optimizes for a smooth Signed Distance Function (SDF) solutions, avoiding auxiliary bad minima of the Eikonal equation, while the coarea loss strives to minimize the area of the zero level surface. Our loss has $4$ hyper-parameters $\lambda_{\text{p}},\lambda_{\text{n}},\lambda_{\text{v}},\lambda_{\text{c}}$. We provide more details on these priors next. 


\subsection{Viscosity Loss}\label{ss:viscosity_loss}
The goal of the viscosity loss is to make $f$ approximate an SDF over $\gC$. Given boundary conditions asking $f$ to vanish on some closed compact surface $\gS$, the SDF solves the Eikonal equation PDE, \ie, $\norm{\nabla f(p)}=1$, in a certain well defined sense (viscosity). This motivated some previous work to directly optimize the Eikonal loss \citep{gropp2020implicit,sitzmann2020implicit}
\begin{equation}\label{e:loss_eikonal}
    \gL_{\text{eikonal}} = \int_\gC \Big (\norm{\nabla f(p)}-1\Big )^2 dp
\end{equation}
\begin{wrapfigure}[14]{r}{0.28\textwidth}\vspace{-15pt}
  \begin{center}
    \includegraphics[width=0.25\textwidth]{figs/illustrations/eikonl_1d.png}
  \end{center}
  \caption{Two global minimizers of the Eikonal loss over a grid in 1D. Top solution is not an SDF. }\label{fig:eikonal_1d}
\end{wrapfigure}
Unfortunately, the Eikonal loss has many undesirable minima which are not SDFs. Figure \ref{fig:eikonal_1d} shows a 1D example: both depicted solutions (denoted $f$) vanish at the input points $q_1,q_2$ (black points) and globally minimize the Eikonal loss over the grid (grid points are shown in blue). The INR works mentioned above use neural networks for representing $f$ which injects an inductive bias avoiding these bad minima, however on grids, minimizing \eqref{e:loss_eikonal} cannot avoid these solutions. See, \eg, middle column in Figure \ref{fig:teaser}. 

More classical Eikonal solvers do work with grids however use mostly fast marching or sweeping methods \citep{osher1988fronts,sethian1996fast,zhao2005fast,chacon2012fast}. Namely, use a special discretization of the Eikonal equation favoring the viscosity  solution of the Eikonal \cite{rouy1992viscosity}, and update node values according to a moving front \cite{sethian1996fast}. Since this discretization is up-wind (will only propagate values in one direction) and requires choosing the maximal among its solution, its success in adaptation to a loss is not clear. 

We use a different approach to build a loss favoring SDF solutions over grids motivated by the vanishing viscosity method \cite{crandall1983viscosity}. Namely, adding to the Eikonal PDE a small perturbation of the Laplacian of $f$ (denoted by $\Delta f$), \ie, $\norm{\nabla f(p)}-1 - \eps\Delta f(p)=0$, makes the PDE semi-linear elliptic \citep{calder2018lecture}, and hence with suitable boundary conditions it is uniquely solvable inside $\gS$ with a smooth solution, approaching the viscosity positive distance function to the boundary as $\eps\too 0$. Similarly, for $1-\norm{\nabla f(p)} - \eps \Delta f(p)=0$ the solution approaches the negative distance function inside the domain. 
Motivated by the vanishing viscosity principle we suggest the following viscosity loss:
\begin{equation}\label{e:loss_viscosity_eikonal}
\gL_{\text{viscosity}} = \int_\gC \Big((\norm{\nabla f (p)}-1)\mathrm{sign}(f(p)) - \eps \Delta f(p)\Big)^2 dp
\end{equation}
We discretize this loss over the grid $\gG$ by replacing the first order derivatives and second order derivatives with symmetric finite  differences, \ie,
\begin{align*}
    D_x f_I=D_x f_{i,j,k} = \frac{f_{i+1,j,k}-f_{i-1,j,k}}{2h}, \quad D^2_x f_I = D^2_x f_{i,j,k}=\frac{f_{i+1,j,k}-2f_{i,j,k}+f_{i-1,j,k}}{h^2}
\end{align*}
and similarly for $D_y$ and $D_z$. We use these discrete operators to approximate the gradient $\widehat{\nabla} f(p_I) = (D_x f_I, D_y f_I, D_z f_I)$ and Laplacian $\widehat{\Delta}f(p_I) = D_x^2f_I + D_y^2 f_I + D_z^2 f_I$. The discretized viscosity loss now takes the form
\begin{equation}
    \widehat{\gL}_{\text{viscosity}} = \frac{1}{N}\sum_{I} \Big((\|\widehat{\nabla} f (p_I)\|-1)\mathrm{sign}(f(p_I)) - \eps \widehat{\Delta} f(p_I)\Big)^2
\end{equation}



\subsection{Coarea loss}\label{ss:coarea_loss}
The coarea loss is approximating the area of the zero level set, and therefore incorporating it in the optimization pushes the reconstructed surface to be economic in area. 

First, similarly to  \citep{yariv2021volume} we use the centered Laplace CDF
\begin{equation}
   \Psi\beta(s)= \begin{cases}
   \frac{1}{2}\exp\parr{\frac{s}{\beta}} & s\leq 0 \\ 1-\frac{1}{2}\exp\parr{-\frac{s}{\beta}} & s\geq  0
   \end{cases}
\end{equation} to transform the SDF $f$ to a smooth approximation of the indicator function:
\begin{equation}
    \chi_\beta(p)=\Psi\beta (-f(p))
\end{equation}
As $\beta\too 0$, $\chi_\beta$ converges to an indicator function leading to $1$ inside $\gS$ and $0$ outside. The coarea loss is defined as 
\begin{equation}
    \gL_{\text{coarea}} = \int_\gC \norm{\nabla \chi_\beta (p)} dp
\end{equation}
To understand why this loss approximates the area of $\gS$ we can use the coarea formula \citep{rindler2018calculus}:
\begin{equation}\label{e:coarea}
    \int_\gC \norm{\nabla \chi_\beta(p)}dp = \int_{-\infty}^{\infty} \mathrm{area}(\chi_\beta^{-1}(s))ds,
\end{equation}
where $\chi_\beta^{-1}(s)=\set{p\ \vert \ \chi_\beta(p)=s}$ is the preimage of the value $s$. Since $\chi_x(p)\in [0,1]$ the r.h.s.~integral can be restricted to the interval $[0,1]$, and therefore the coarea loss averages the area of the level sets of $\chi_\beta$. Next,  $$\chi_\beta^{-1}(s)= \set{p\ \vert \ \Psi\beta (-f(p)) = s } = \{p\ \vert \ f(p) = -\Psi\beta^{-1} (s) \} = f^{-1}(-\Psi\beta^{-1} (s)),$$
\begin{wrapfigure}[11]{r}{0.28\textwidth}\vspace{-20pt}
  \begin{center}
  \includegraphics[width=0.25\textwidth]{figs/semi.png}
  \end{center}
  \caption{Reconstruction of a semisphere point cloud (white dots) without (left) and with (right) coarea loss. }\label{fig:coarea_semisphere}
\end{wrapfigure}

which shows that the level set $s\in (0,1)$ of $\chi_\beta$ is the level set $-\Psi\beta^{-1}(s)$ of the SDF $f$. As $\beta\too 0$, $-\Psi\beta^{-1}(s)\too 0$ for all $s\in (0,1)$ (and uniformly in $(\eps,1-\eps)$ for fixed $\eps>0$). Therefore the average of the level set area (\ie, the r.h.s.~of \eqref{e:coarea}) converges to the area of $f^{-1}(0)=\gS$. Figure \ref{fig:teaser} (right) shows how removing the coarea loss introduces an extraneous zero level set, and hence results in an undesired surface part. Figure \ref{fig:coarea_semisphere} shows a comparison of a reconstruction of semisphere with and without coarea. In the experiments section we provide more ablation tests with the coarea and viscosity losses.

To discretize the coarea loss we let $w_I$ denote the centers of grid's voxels, and note that $\nabla \chi_\beta(w_I) = \Phi_\beta(-f(w_I))\nabla f(w_I)$, where 
\begin{equation*}
    \Phi_\beta(s) = \frac{1}{2\beta}\exp\parr{\frac{\abs{s}}{\beta}}
\end{equation*}
is the PDF of the Laplace distribution, and $\nabla f(w_I)$ is computed as a linear combination of the voxel's corner values $f_{I_1},\ldots,f_{I_8}$, see more details in the Appendix. We end up with the discretized loss:
\begin{equation}
    \widehat{\gL}_{\text{coarea}} = \frac{1}{N}\sum_{I}\Phi_\beta(-f(w_I))\norm{\nabla f(w_I)}
\end{equation}
This loss is usually incorporated with a small hyper-parameter $\lambda_{\text{c}}$ with the purpose of eliminating redundant surface parts.



\section{Results}
\label{sec:results}

\section{Results}
\label{results}

\begin{figure*}[ht]
    \centering
    \includegraphics[scale=0.15,trim={0 2.5cm 0 5cm},clip]{images/aoi-single_burst}
    \caption{The time average peak Age of Information with burst and \gls{soa} loss values against the dynamic reliability logic for different network topologies.}
    \label{fig:aoi_burst}\vspace{-0.4cm}
\end{figure*}


This paper focuses on both transport layer and application layer metrics to determine the feasibility of dynamic reliability. For this, we have selected the session packet volume, as transmitted, retransmitted, lost and backlogged packets as \glspl{kpi} for the transport layer; while focusing on the \gls{aoi} for the application layer. The \gls{aoi} was chosen as a crucial indicator for the freshness of packets in real-time applications. More specifically, this work adopts the time average peak \gls{aoi} equation \cite{aoi_equation} depicted in Eq. \ref{aoi}, where $\Delta(r_{i+1})$ is the $i$th update at the time it was received at the server, for a session time period of $\tau$.

\begin{equation}
    \label{aoi}
    \gls{aoi}_\tau = \frac{1}{n-1}\sum_{i=1}^{n-1} \Delta(r_{i+1})
\end{equation}

We include a comparison between the vanilla QUIC implementation which does not enjoy the dynamic reliability extension, with a number of dynamic reliability policies. The tests were run a number of times for statistical significance, with the mean value of vanilla implementation used as a baseline for comparison. The topology utilised both random loss and bursty loss to explore the bounds of dynamic reliability. The \gls{soa} loss in the figures correspond to the loss values presented in Table. \ref{tab:path_char}, for ease of comparison between bursty and random loss scenarios.

\subsection{Transport-Layer KPIs}

To analyse the performance gain at the transport layer due to dynamic reliability, the volume of transmitted and backlogged packets is examined. The figures are in the form of boxplots, which take the vanilla implementation as a benchmark, depicted as the red dashed line.

As seen in Fig. \ref{fig:sent_burst}, the loss plays a crucial role in the performance of the reliability policies. The policies under random loss did incredibly well for the networks with a larger capacity, namely \gls{mmwave} and Sub-6~GHz, whereas for burst loss, the lower network capacities had a larger packet reduction. With the increase in burst loss, the behaviour of the set split reliable policies became unpredictable, if a reliable assignment happened to coincide with a burst loss, the number of transmitted packets increases, and vice versa. On the other hand, in smarter policies, such as Loss-Aware, the performance lightly matched the vanilla baseline, as the reliable assignment dominated the session to compensate for a higher burst loss. Not only that but, the burst loss also impacted the variance of the transmitted packets for the policies.

Unsurprisingly, the unreliable focused policy, 80-20 split, outperformed other policies for all topologies in random and bursty loss scenarios, with an approximate reduction of 80\%. That being said, the majority of the policies reduced the transmitted packets on the link by approximately 70\% for random loss, while the reduction started at $\approx 15\%$ and decreased as the loss increased for the burst loss scenario.

The retransmitted and lost packets, not shown due to space limitations, followed the same trend as the transmitted packets for the random loss scenarios. However, for the burst loss scenarios, the larger capacity networks had a lower reduction in the retransmitted and lost packets. This can be seen as a favorable outcome since the lower capacity networks are scarce on resources. It is important to note that the Loss-Aware policy mimicked the vanilla approach as the burst loss increased, signifying the overwhelming appointment of reliable packets in adapting to the harsh burst loss conditions.
 
Alternatively, Fig. \ref{fig:backlog_burst} clearly shows a stark comparison between the policies and loss scenario in the reduction of the backlogged packets. The Loss-Aware policy for random loss scenario reduced the backlogged packets by up to 50\%, beating all other policies by approximately 30\%. Furthermore, it is clear that the unreliability focused policies resulted in the lowest backlog for the session. In comparison, we notice that the burst loss and the backlogged frequency have a positive correlation, where the maximum reduction of the backlogged packets for the policies is at most 20\%. Much like the transmitted packets, the probability of a burst loss occurrence plays a vital role in the number of retransmissions sent and by extension the number of backlogged packets. Thus, we can conclude that the stress placed on the buffer is a result of the reliable packets which is tightly coupled with the congestion on the session. Whereas, unreliable focused policies did not encounter such a phenomenon regardless if it was experiencing a burst loss.


\subsection{Application-Layer KPIs}

The feasibility of dynamic reliability for real-time applications can be determined by the \gls{aoi}, with comparison across different topologies and policies. If we take a strict approach and consider anything below $10$~ms is real-time \cite{real-time}, then all the reliability policies passed that requirement, which is attractive for real-time applications, as shown in Fig. \ref{fig:aoi_burst}. Utilising the median as an estimate of the runs, the policies in the WLAN and Sub-6~GHz topology with random loss floated around $4-5$~ms with negligible difference, while the \gls{aoi} for \gls{mmwave} was $\approx 2-3$~ms. It is clear that the \gls{aoi} and the network capacity have a negative correlation, as the network capacity decreases, the \gls{aoi} increases. The same correlation is extended to the bursty loss scenarios, where \gls{mmwave} dominated the other topologies. That being said, it is crucial to note that the \gls{aoi} for the reliability policies is often slightly better than or equal to the \gls{aoi} of the vanilla implementation, proving that dynamic reliability reduces the congestion of the session at no cost to the \gls{aoi}.


\section{Discussion}

\label{sec:discussion}

We provide some comments on the growth conditions which constituted the majority of our analysis in sections \ref{sec:Hmixing} and \ref{sec:Hsigma}. In the simplest cases of Lemma \ref{lemma:unstableGrowth}, growth was established in an analogous fashion to the old one-step expansion condition (\ref{eq:oldOneStepExpansion}), finding the relevant Jacobians $M_j$ and checking that their expansion factors $K(M_j)$ satisfy
\begin{equation}
    \label{eq:discussionOneStep}
    \sum_j \frac{1}{K(M_j)} <1.
\end{equation}
For the more complicated cases, the inductive method used to establish growth near the accumulation points in Lemma \ref{lemma:unstableGrowth} and the weakened one-step expansion condition (\ref{eq:oneStep}) both address the same fundamental issue: the splitting of unstable curves by singularities into an unbounded number of small components. They circumvent this obstacle in rather different ways, however. While (\ref{eq:oneStep}) generalises (\ref{eq:discussionOneStep}) to ensure an growth of unstable curves `on average' (see \cite{chernov_statistical_2009} for a precise statement), our inductive method is a more direct adaptation of (\ref{eq:discussionOneStep}), using it to generate contradictory geometric conditions which a hypothetical non-growing unstable curve must satisfy. It may be possible to prove Theorem \ref{sec:Hmixing} using (\ref{eq:oneStep}) as the basis for growth. Since we required (\ref{eq:oneStep}) anyway for proving Theorem \ref{thm:HsigmaExp}, this could potentially condense our analysis, but only to a minor extent. A convenience of the method used in section \ref{sec:Hmixing} is that, by way of the `simple intersection' property, it naturally gives geometric information on the images of manifolds, useful for proving the property \textbf{(M)} of Theorem \ref{thm:katok-strelcyn}.

We expect that essentially analogous analysis can be applied to establish mixing properties in a wide class of piecewise linear non-uniformly hyperbolic maps, including those (like the OTM) which sit on the boundary of ergodicity and beyond. While we have relied on the precise partition structure of $H_\sigma$, its fundamental feature (self-similar sequences of elements $A^k$, sharing boundaries with its neighbours $A^{k-1},A^{k+1}$ and accumulating onto some point $p$) is quite typical to return map systems. See, for example, those of various stadium billiards \cite{chernov_chaotic_2006,chernov_improved_2008,chernov_statistical_2009} and LTMs \cite{springham_polynomial_2014}. Indeed, the same method can be used to prove the Bernoulli property for non-monotonic LTMs \cite{myers_hill_mixing_2022}, where monotonicity of the manifold images cannot be assumed and the classical argument \cite{sturman_mathematical_2006} fails. The OTM is the pointwise limit of these maps as the boundary shrinks to null measure. It further has utility in proving growth conditions for maps which are uniformly hyperbolic but possess regions $A_j$ where the hyperbolicity is very weak, signified by $K(M_j) \approx 1$, so that (\ref{eq:discussionOneStep}) fails. Typically this leads to suboptimal bounds on mixing windows, see e.g. \cite{wojtkowski_model_1981,przytycki_ergodicity_1983,myers_hill_family_2022}. The map $H_{(\eta,\eta)}$ for $\eta \approx 1/2$ is another example, possessing weak hyperbolicity over $A_2, A_3$. Letting $\varepsilon = |\eta-1/2|>0$, there is an upper bound $N = N(\varepsilon)$ on escape times from the intersections $A_2\cap \sigma, A_3 \cap \sigma$. The growth lemma then follows by applying the inductive step roughly $N$ times and can be established for arbitrarily small $\varepsilon$, opening the door to establishing optimal mixing windows.

The above gives two examples of piecewise linear perturbations to $H$ where mixing with respect to Lebesgue is preserved and our methods can be applied. Nonlinear perturbations to the shear profiles complicate the analysis in several ways. Firstly as the map's Jacobians takes on a broader range of values, cone invariance becomes an increasingly harder condition to establish. Cones must be widened, giving looser bounds on expansion factors, which may already be weak due to new regions of weaker stretching. This, together with the change from polygonal to curvilinear return time partition elements and nonlinear local manifolds, adds some complexity to showing growth conditions. This does not rule out certain (small) nonlinear perturbations however. There is some leeway in the inequalities which govern cone invariance and growth of local manifolds, the latter of which is not too dissimilar from the piecewise linear setting (see Lemmas \ref{lemma:piecewiseApprox}, \ref{lemma:componentLength}). Certain small perturbations would not alter the \emph{topological} structure of the return time partition, i.e. which elements share boundaries, the key information needed for setting up the induction. Finally while the partition elements would no longer be polygonal, only coarse geometric information is required for verifying each inductive step. Following the above, a potential perturbation could be to replace the linear portions of each shear by a cubic, perturbing the tent profile
\[  f(t) = \begin{cases} 2t & 0 \leq t \leq 1/2, \\ 2(1-t) & 1/2 \leq t \leq 1 ,\end{cases} \]
of the OTM shears to
\[  f_a(t) = \begin{cases} \frac{1}{8} t \left(16 - a + 6at - 8at^{2} \right) & 0 \leq t \leq 1/2, \\ \frac{1}{8}\left(1-t\right)\left( 16 - a + 6a\left(1-t\right) - 8a\left(1-t\right)^{2}\right)  & 1/2 \leq t \leq 1, \end{cases}   \]
for $a>0$. For small enough $a$ the gradient range $f'(t)$ is restricted to small neighbourhoods of $\{ 2, -2\}$ and the escape time partition retains a similar structure. We illustrate this in Figure \ref{fig:perturbations}, showing escapes from the square $S_3$ under the map $G \circ F$, equivalent to escapes from the perturbed $A_3$ under the $G \circ F$, but with a cleaner geometry for comparison. When $a$ is too large the analogy to the OTM breaks down. At $a=16$ the map is twice differentiable everywhere and features a new source of slowed mixing, the Jacobian is the identity at the corner points $x,y \in \{  0, 1/2 \}$ giving locally parabolic behaviour (visible in the escape time partition). 

\begin{figure}
    \centering
    \includegraphics[width=0.24 \linewidth]{0.png}
    \includegraphics[width=0.24 \linewidth]{4.png}
    \includegraphics[width=0.24 \linewidth]{8.png}
    \includegraphics[width=0.24 \linewidth]{16.png}
    \caption{Partition of escape times from $S_3$ under the mapping $F \circ G$ for $a= 0,4,8,16$. }
    \label{fig:perturbations}
\end{figure}



\iftoggle{anonymous}{}{
\section*{Authors' contributions}
SP: Conceptualization, Formal Analysis, Investigation, Software, Supervision, Validation, Visualization, Writing-original draft, Writing-review \& editing;
ACM: Data curation, Resources, Software;
JCM: Writing-review \& editing;
MW: Writing-review \& editing, Software, Visualization;
CB: Conceptualization, Writing-review \& editing;
RFC: Conceptualization, Formal analysis, Funding acquisition, Methods, Project Administration, Supervision, Writing-original draft, Writing-review \& editing.
}

\section*{Acknowledgments}
% This part is dedicated mainly to funding information. For example: This research is supported by the Scientific Grant Agency of the Ministry of Education, Science, Research and Sport of the Slovak Republic and of the Slovak Academy of Sciences (VEGA 2/0054/18).

This article is one of the products of the research project of the 1st edition of the master degree course ``Introduction to Quantitative Linguistics'' at Universitat Politècnica de Catalunya. We are specially grateful to two students of that course: L. Alemany-Puig for helpful discussions and computational support, and M. Michaux for comments on early versions of the manuscript. We thank M. Farrús and A. Hernández-Fernández for advice on voice datasets. We also thank S. Komori for advice on Japanese, Y. M. Oh for advice on Korean and S. Semple for helping us to improve English. Last but not least, we are very grateful to an anonymous reviewer for very helpful comments.

SP is funded by the grant 'Thesis abroad 2021/2022' from the University of Milan. CB was partly funded by the \textit{Deutsche Forschungsgemeinschaft} (FOR 2237:  Words, Bones, Genes, Tools - Tracking Linguistic, Cultural and Biological Trajectories of the Human Past), and the \textit{Schweizerischer Nationalfonds zur Förderung der
Wissenschaftlichen Forschung} (Non-randomness in Morphological Diversity: A Computational Approach Based on Multilingual Corpora, 176305).



% 
\section{Conclusion and Future Work}
\label{sec:future}
% Whenever a new solution concept is defined, one of the most intuitive questions is how it can be related to other properties? 
We presented a practical solution to the problem of leximin optimization when only an approximate single-objective solver is available. 
The algorithm is guaranteed to terminate in polynomial time, and its approximation ratio degrades gracefully as a function of the approximation ratio of the single-objective solver.

Currently, our algorithm handles two main settings. First, when inaccuracies in the single-objective solver stem from numeric errors.
Second, when the problem is convex and satisfy several assumptions.
It may be interesting to study more settings in which the inaccuracies stem from computational hardness of the single-objective problem.
% Currently, our algorithm handles settings in which the inaccuracies in the single-objective solver stem from numeric errors.
% It may be interesting to study settings in which the inaccuracies stem from computational hardness of the single-objective problem.
%
%

% identify problems in which an appropriate approximate solver can be designed. 
In particular, to approximate the egalitarian welfare, it is common to model the problem as an integer program or as an exponential sized linear program (e.g., \cite{bansal2006santa, kawase_max-min_2020}) and then approximate the program using different techniques.
% rounding techniques or methods for convex optimization (such as the ellipsoid method).
Can these algorithms be generalized to consider the additional constraints described in Section \ref{sec:algo-short}? This will allow approximating leximin using the approach in this paper.
% In particular, in the problem of stochastic allocations (in Section \ref{sec:app}), to extend the approximation algorithm for the egalitarian welfare, we had to change some steps within.
% What if an algorithm for egalitarian welfare is provided as a black box --- could it be used to design the appropriate procedure to approximate leximin?

% In the context of fair division, this study assumes that there is an access to the true valuations of the agents involved. 
% In reality, people may lie about their valuations.
% Can our definition of approximate-leximin be related to some approximate version of truthfulness?

Another question is whether it is possible to obtain a better approximation factor for leximin, given an $(\multApprox, \additiveApprox)$-approximation algorithm for the single-objective problem.
Specifically, can an $(\multApprox, \additiveApprox)$-approximation to leximin can be obtained in polynomial time? 
If not, what would be the best possible approximation in this case?
% \erel{Mention the tightness of our results}


\iffalse % EREL: removed for the submission. To clarify later
Further, the algorithm suggested in Section \ref{sec:algo-short} tend to work very well if the single-objective optimization problems are convex, and in particular if they are linear programs. 
\erel{Why? the algorithm of \textcite{Ogryczak2004TelecommunicationsND} works even for non-convex programs.}
Can we find an algorithm that works with general approximation algorithms? For example, naive algorithms such as \emph{Next-fit}?
\fi

% \begin{itemize}
%     \item Meaning of approximately leximin in the context of other characteristics like truthfulness.
    
%     \item Solving more problems.
    
%     \item Is it possible to obtained a better approximation factor for leximin maximization in polynomial time? given that $(1-\beta)$ is the best possible for the egalitarian maximization, is it possible to obtain a $(1-\beta)$ approximately leximin optimal solution? what is the best possible approximation  in this case?
    
%     \item The algorithm works very well if the single-objective problem is convex, in particular if it is a linear programs, but can it be modify to work with general approximation algorithms? such as algorithms for makespan minimization?
% \end{itemize}

%==================================================
% References

\printbibliography

%==================================================
% Appendix

% \appendix % this is not from the original template of the journal

\begin{appendices} % this is not from the original template of the journal

% https://tex.stackexchange.com/questions/559218/use-appendix-letter-in-figure-and-table-captions
\counterwithin{figure}{section}
\counterwithin{table}{section}
\renewcommand\thefigure{\thesection\arabic{figure}}
\renewcommand\thetable{\thesection\arabic{table}}

\section{Theory}
\label{app:theory}

% \subsection{The relationship between $\Psi$ and Pearson correlation}
% \subsection{The relationship between \texorpdfstring{$L$}{L}, \texorpdfstring{$L_r$}{Lr} and Pearson correlation}

% \label{app:Pearson_correlation}

Here we review the relationship between \texorpdfstring{$L$}{L}, \texorpdfstring{$L_r$}{Lr} and Pearson correlation

Given two random variables $x$ and $y$ and a sample of $n$ points, $\left\{(x_1, y_1),...,(x_i, y_i),...,(x_n,y_n) \right\}$,
the sample covariance is defined as 
$$s_{xy} = \frac{1}{n-1}\left(\sum_{i=1}^n x_i y_i - n \bar{x}\bar{y}\right),$$
where $\bar{x}$ is the sample mean of $x$ and $\bar{y}$ is the sample mean for $y$, i.e. 
\begin{eqnarray*}
\bar{x}= \frac{1}{n} \sum_{i=1}^n x_i \\
\bar{y}= \frac{1}{n} \sum_{i=1}^n y_i.
\end{eqnarray*}
Now consider than the random variables are $p$ (the probability of a type) and $l$ (the length/duration of a type) instead of $x$ and $y$. 
Then our sample of $n$ points is $\left\{(p_1, l_1),...,(p_i, l_i),...,(p_n,l_n)\right\}$, one point per type. 
Accordingly, the covariance between $p$ and $l$ in a sample of points is 
$$s_{pl} = \frac{1}{n-1}\left(\sum_{i=1}^n p_i l_i - n \bar{p}\bar{l}\right).$$
Recalling the definition of $L$ (\autoref{eq:mean_type_length}) and noting that $\bar{p} = \frac{1}{n}$ and $\bar{l} = M = L_r$ (recall Property \ref{prop:random_baseline}), we finally obtain
$$s_{pl} = \frac{1}{n-1}(L - L_r).$$

The sample Pearson correlation is  
$$r = \frac{s_{xy}}{s_x s_y},$$
where $s_x$ and $s_y$ are the sample standard deviation of $x$ and $y$, i.e.
\begin{eqnarray*}
s_x = \sqrt{\frac{1}{n-1} \sum_{i=1}^n (x_i - \bar{x})^2} \\
s_y = \sqrt{\frac{1}{n-1} \sum_{i=1}^n (y_i - \bar{y})^2}.
\end{eqnarray*}
Proceeding as we did for the covariance, we find that the Pearson correlation between $p$ and $l$ is 
$$r = \frac{L - L_r}{(n-1)s_p s_l}.$$
Then it is easy to see that $L$ is a linear function of the Pearson correlation $r$ or $s_{pl}$. For instance, 
$$L = ar + b,$$
where 
\begin{eqnarray*}
a = (n-1)s_p s_l \\
b = L_r.
\end{eqnarray*}
Other linear relationships are left as an exercise. 


\section{Analysis}
\label{app:analysis}

We here present complementary analyses, tables and plots.

\subsection{The impact of the unsupervised filter}
\label{app:no_filter}

\autoref{tab:coll_comparison_pud} and \autoref{tab:coll_comparison_cv} show the impact of the unsupervised filter in the optional filter. PUD is a controlled setting for the impact of the filter because it is a collection where tokens are of high quality compared to CV. Thus we expect that the impact of the optional filter is low in PUD. Unexpectedly, the number of tokens reduces substantially (a reduction of the order of thousands) in Chinese, Japanese and Korean. An additional drastic reduction in the observed alphabet size in these languages strongly suggests that the optional filter is not adequate for them.  
For these reasons, we believe we should not apply the unsupervised filter to these languages because their writing system is essentially a syllabary. We suspect that the actual need for the exclusion could be a combination of sampling problems relating to a large alphabet size (compared to the Latin script) and a heavy- tailed rank distribution that breaks the optional filter. It is well-known that the rank distribution of Chinese characters is long-tailed, spanning two orders of magnitude \parencite{Deng2014a}, while that of phonemes (the counterpart of letters in many languages using the Latin script) is exponential-like \parencite{Naranan1993,Balasubrahmanyan1996}.
However, that issue should be the subject of future research. 

In CV, we find that the optional filter has a similar impact in languages concerning the reduction in the number of tokens but higher impacts concerning the reduction of the alphabet sizes, suggesting that presence of strings with strange characters. The three languages with the most marked reduction  in alphabet size are French, Spanish, German and Italian, with an alphabet size greater then 100.

\begin{table}[H]
\centering
\caption{The impact of the unsupervised filter in the PUD collection. For every language, we show its linguistic family, the writing system (namely script name according to ISO-15924) and various numeric parameters after applying the mandatory filter but before applying the unsupervised filter, that are $A$, the observed alphabet size (number of distinct characters),
$n$,  the number of types, and, $T$, the number of tokens.
$A'$, $n'$ and $T'$ are the respective values of $A$, $n$ and $T$ after applying the unsupervised filter. 
% 'Conlang' stands for 'constructed language', that is an artificially created language. This is not a family in the proper sense as Conlang languages are not related in the common linguistic family sense.
} 
\label{tab:coll_comparison_pud}
\begin{tabular}{lllrrrrrr}
\hline
Language & Script & Family & $A$ & $A'$ & $n$ & $n'$ & $T$ & $T'$ \\ 
\hline
 Arabic & Arabic & Afro-Asiatic & 47 & 39 & 6600 & 6596 & 18214 & 18201 \\ 
Indonesian & Latin & Austronesian & 39 & 23 & 4596 & 4501 & 16819 & 16702 \\ 
Russian & Cyrillic & Indo-European & 61 & 31 & 7358 & 7113 & 15870 & 15588 \\ 
Hindi & Devanagari & Indo-European & 84 & 50 & 4920 & 4716 & 21184 & 20796 \\ 
Czech & Latin & Indo-European & 49 & 33 & 7360 & 7073 & 15700 & 15331 \\ 
English & Latin & Indo-European & 39 & 25 & 5082 & 5001 & 18135 & 18028 \\ 
French & Latin & Indo-European & 48 & 26 & 5593 & 5214 & 21084 & 20407 \\ 
German & Latin & Indo-European & 39 & 28 & 6215 & 6116 & 18446 & 18331 \\ 
Icelandic & Latin & Indo-European & 43 & 32 & 6175 & 6035 & 16385 & 16209 \\ 
Italian & Latin & Indo-European & 42 & 24 & 5944 & 5606 & 21815 & 21266 \\ 
Polish & Latin & Indo-European & 47 & 31 & 7329 & 7188 & 15386 & 15191 \\ 
Portuguese & Latin & Indo-European & 47 & 38 & 5678 & 5661 & 21873 & 21855 \\ 
Spanish & Latin & Indo-European & 39 & 32 & 5765 & 5750 & 21083 & 21067 \\ 
Swedish & Latin & Indo-European & 39 & 25 & 5842 & 5624 & 16653 & 16378 \\ 
Japanese & Japanese & Japonic & 1549 & 609 & 4990 & 3345 & 24899 & 22538 \\ 
Japanese-strokes & Japanese & Japonic & 1549 & 609 & 4852 & 3345 & 24737 & 22538 \\ 
Japanese-romaji & Latin & Japonic & 23 & 19 & 4984 & 4860 & 24892 & 24743 \\ 
Korean & Hangul & Koreanic & 1002 & 401 & 8031 & 6424 & 14475 & 12540 \\ 
Thai & Thai & Kra-Dai & 89 & 52 & 3818 & 3599 & 21642 & 21121 \\ 
Chinese & Han (Traditional variant) & Sino-Tibetan & 2038 & 814 & 5224 & 3154 & 18129 & 15436 \\ 
Chinese-strokes & Han (Traditional variant) & Sino-Tibetan & 2038 & 814 & 4970 & 3154 & 17845 & 15436 \\ 
Chinese-pinyin & Latin & Sino-Tibetan & 49 & 44 & 5224 & 5038 & 18129 & 17885 \\ 
Turkish & Latin & Turkic & 42 & 28 & 6793 & 6587 & 14092 & 13799 \\ 
Finnish & Latin & Uralic & 39 & 24 & 7076 & 6938 & 12853 & 12701 \\ 
  \hline

\end{tabular}
\end{table}

\begin{table}[H]
\centering
\caption{The impact of the unsupervised filter in the CV collection. The content is the same as in \autoref{tab:coll_comparison_pud}.
% For every language we show its linguistic family, the writing system (namely script name according to ISO-15924) and various numeric parameters after applying the mandatory filter but before applying the unsupervised filter, that are $A$, the observed alphabet size (number of distinct characters), $n$, the number of types, and, $T$, the number of tokens. $A'$, $n'$ and $T'$ are the respective values of $A$, $n$ and $T$ after applying the unsupervised filter. 
'Conlang' stands for 'constructed language', that is an artificially created language. This is not a family in the proper sense as Conlang languages are not related in the common linguistic family sense.
} 
\label{tab:coll_comparison_cv}
\begin{tabular}{lllrrrrrr}
\hline
Language & Script & Family & $A$ & $A'$ & $n$ & $n'$ & $T$ & $T'$ \\ 
\hline
         Arabic & Arabic & Afro-Asiatic & 44 & 31 & 7497 & 6397 & 49448 & 45825 \\ 
        Maltese & Latin & Afro-Asiatic & 40 & 31 & 8148 & 8058 & 44272 & 44112 \\ 
        Vietnamese & Latin & Austroasiatic & 86 & 41 & 574 & 370 & 1300 & 938 \\ 
        Indonesian & Latin & Austronesian & 28 & 22 & 3817 & 3768 & 44336 & 44210 \\ 
        Esperanto & Latin & Conlang & 38 & 27 & 27932 & 27759 & 406725 & 406261 \\ 
        Interlingua & Latin & Conlang & 27 & 20 & 5552 & 5126 & 31428 & 30504 \\ 
        Tamil & Tamil & Dravidian & 44 & 29 & 1525 & 1210 & 7580 & 6439 \\ 
        Persian & Arabic & Indo-European & 105 & 38 & 13240 & 13115 & 1665428 & 1662508 \\ 
        Assamese & Assamese & Indo-European & 60 & 43 & 1115 & 971 & 2000 & 1813 \\ 
        Russian & Cyrillic & Indo-European & 54 & 32 & 31921 & 31827 & 638782 & 637686 \\ 
        Ukrainian & Cyrillic & Indo-European & 44 & 34 & 14399 & 14337 & 120984 & 120760 \\ 
        Panjabi & Devanagari & Indo-European & 48 & 37 & 95 & 84 & 110 & 98 \\ 
        Modern Greek & Greek & Indo-European & 46 & 33 & 5834 & 5813 & 37926 & 37880 \\ 
        Breton & Latin & Indo-European & 41 & 28 & 4322 & 4228 & 38493 & 38237 \\ 
        Catalan & Latin & Indo-European & 67 & 39 & 79213 & 79112 & 3294506 & 3294206 \\ 
        Czech & Latin & Indo-European & 44 & 33 & 16032 & 15518 & 150312 & 147582 \\ 
        Dutch & Latin & Indo-European & 41 & 23 & 10666 & 10225 & 320992 & 316498 \\ 
        English & Latin & Indo-European & 97 & 28 & 173522 & 173023 & 9829660 & 9828713 \\ 
        French & Latin & Indo-European & 244 & 49 & 162740 & 160243 & 3732822 & 3729370 \\ 
        German & Latin & Indo-European & 152 & 30 & 150362 & 148436 & 4235094 & 4230565 \\ 
        Irish & Latin & Indo-European & 31 & 23 & 2311 & 2251 & 22751 & 22593 \\ 
        Italian & Latin & Indo-European & 110 & 34 & 55480 & 54996 & 812604 & 811783 \\ 
        Latvian & Latin & Indo-European & 35 & 27 & 7792 & 7251 & 30358 & 29456 \\ 
        Polish & Latin & Indo-European & 38 & 32 & 25365 & 25340 & 595613 & 595411 \\ 
        Portuguese & Latin & Indo-European & 41 & 27 & 13049 & 11509 & 295042 & 283048 \\ 
        Romanian & Latin & Indo-European & 36 & 29 & 6449 & 6423 & 33370 & 33341 \\ 
        Romansh & Latin & Indo-European & 40 & 26 & 9801 & 9614 & 44192 & 43792 \\ 
        Slovenian & Latin & Indo-European & 28 & 24 & 5994 & 5937 & 26402 & 26304 \\ 
        Spanish & Latin & Indo-European & 186 & 33 & 75617 & 75010 & 1843646 & 1842474 \\ 
        Swedish & Latin & Indo-European & 30 & 25 & 4454 & 4371 & 63282 & 62951 \\ 
        Welsh & Latin & Indo-European & 43 & 22 & 11488 & 11143 & 547345 & 539621 \\ 
        Western Frisian & Latin & Indo-European & 42 & 30 & 8419 & 8383 & 63127 & 63073 \\ 
        Oriya & Odia & Indo-European & 59 & 41 & 921 & 764 & 1929 & 1700 \\ 
        Dhivehi & Thaana & Indo-European & 40 & 27 & 155 & 111 & 1388 & 1284 \\ 
        Georgian & Georgian & Kartvelian & 34 & 25 & 7945 & 6505 & 15481 & 12958 \\ 
        Basque & Latin & Language isolate & 28 & 21 & 24998 & 24748 & 460188 & 458071 \\ 
        Mongolian & Mongolian & Mongolic & 36 & 31 & 14844 & 14608 & 70638 & 70217 \\ 
        Kinyarwanda & Latin & Niger-Congo & 96 & 26 & 135328 & 133815 & 1945038 & 1939810 \\ 
        Abkhazian & Cyrillic & Northwest Caucasian & 37 & 28 & 150 & 119 & 189 & 156 \\ 
        Hakha Chin & Latin & Sino-Tibetan & 28 & 23 & 2515 & 2499 & 17806 & 17776 \\ 
        Chuvash & Cyrillic & Turkic & 36 & 22 & 5565 & 4311 & 16270 & 13583 \\ 
        Kirghiz & Cyrillic & Turkic & 38 & 30 & 10497 & 10130 & 62687 & 61844 \\ 
        Tatar & Cyrillic & Turkic & 47 & 34 & 22313 & 21823 & 145458 & 144356 \\ 
        Yakut & Cyrillic & Turkic & 42 & 28 & 8041 & 7904 & 22795 & 22577 \\ 
        Turkish & Latin & Turkic & 37 & 31 & 8957 & 8926 & 107910 & 107686 \\ 
        Estonian & Latin & Uralic & 34 & 23 & 30135 & 28691 & 123895 & 121549 \\ 
          \hline

\end{tabular}
\end{table}


% optimality scores tables
\subsection{Mean word length and the law of abbreviation}
\label{sec:opt_scores}

In \autoref{tab:opt_scores_pud}, \autoref{tab:opt_scores_cv_characters} and \autoref{tab:opt_scores_cv_meadianDuration}, we show the mean word length ($L$) and the random baseline ($L_r$) as well as the outcome of the correlation test between length and frequency for PUD and for CV when length is measured in characters and also in duration, respectively.

%% PUD
\begin{table}[H]
\centering
\caption{Mean word length and the correlation between frequency and length in PUD. Word length is measured in number of characters. Mean word length ($L$) is followed by the random baseline ($L_r$). Each correlation statistic (Kendall $\tau$ or Pearson $r$) is followed by \textit{p}-values after applying Holm-Bonferroni correction (rather than being the direct output of the correlation test).
}  
\label{tab:opt_scores_pud}
\begin{tabular}{lllrrrlrl}
  \hline
language & family & script & $L$ & $L_r$ & $\tau$ & $\tau_{pvalue}$ & $r$ & $r_{pvalue}$\\  
  \hline
% latex table generated in R 4.2.1 by xtable 1.8-4 package
% Fri Feb 10 12:26:50 2023
 Arabic & Afro-Asiatic & Arabic & 4.03 & 5.54 & -0.13 & $8.32 \times 10^{-32}$ & -0.13 & $1.12 \times 10^{-20}$ \\ 
  Czech & Indo-European & Latin & 5.44 & 7.27 & -0.22 & $1.20 \times 10^{-113}$ & -0.15 & $2.47 \times 10^{-36}$ \\ 
  English & Indo-European & Latin & 4.87 & 7.00 & -0.20 & $2.52 \times 10^{-66}$ & -0.12 & $6.98 \times 10^{-17}$ \\ 
  French & Indo-European & Latin & 4.81 & 7.47 & -0.16 & $2.44 \times 10^{-49}$ & -0.12 & $4.24 \times 10^{-19}$ \\ 
  German & Indo-European & Latin & 5.74 & 8.56 & -0.23 & $1.25 \times 10^{-108}$ & -0.12 & $3.85 \times 10^{-21}$ \\ 
  Indonesian & Austronesian & Latin & 5.96 & 7.35 & -0.11 & $6.37 \times 10^{-21}$ & -0.12 & $6.53 \times 10^{-15}$ \\ 
  Italian & Indo-European & Latin & 4.85 & 7.64 & -0.16 & $4.09 \times 10^{-54}$ & -0.13 & $8.45 \times 10^{-23}$ \\ 
  Polish & Indo-European & Latin & 6.07 & 8.00 & -0.19 & $1.12 \times 10^{-80}$ & -0.13 & $2.78 \times 10^{-26}$ \\ 
  Portuguese & Indo-European & Latin & 4.35 & 7.47 & -0.20 & $9.96 \times 10^{-67}$ & -0.12 & $1.12 \times 10^{-17}$ \\ 
  Russian & Indo-European & Cyrillic & 6.04 & 8.08 & -0.19 & $4.58 \times 10^{-88}$ & -0.13 & $4.85 \times 10^{-26}$ \\ 
  Spanish & Indo-European & Latin & 4.83 & 7.59 & -0.16 & $4.10 \times 10^{-51}$ & -0.11 & $1.89 \times 10^{-17}$ \\ 
  Swedish & Indo-European & Latin & 5.41 & 7.99 & -0.23 & $3.99 \times 10^{-101}$ & -0.13 & $6.28 \times 10^{-21}$ \\ 
  Turkish & Turkic & Latin & 6.43 & 7.94 & -0.24 & $4.26 \times 10^{-124}$ & -0.12 & $4.20 \times 10^{-23}$ \\ 
   \hline

\end{tabular}
\end{table}

%% CV characters
\begin{table}[H]
\centering
\caption{Mean word length and the correlation between frequency and length in CV. Word length is measured in number of characters. Content is the same as in \ref{tab:opt_scores_pud}. 
'Conlang' stands for 'constructed language', that is an artificially created language. This is not a family in the proper sense, and Conlang languages are not related in the common family sense. 
}
\label{tab:opt_scores_cv_characters}
\begin{tabular}{lllrrrlrl}
  \hline
language & family & script & $L$ & $L_r$ & $\tau$ & $\tau_{pvalue}$ & $r$ & $r_{pvalue}$\\ 
  \hline
\input{tables/cv_opt_scores_characters}
\end{tabular}
\end{table}

%% CV duration
\begin{table}[H]
\centering
\caption{Mean word length and the correlation between frequency and length in CV. Word length is measured in duration. Content is the same as in \ref{tab:opt_scores_cv_characters}.} 
\label{tab:opt_scores_cv_meadianDuration}
\begin{tabular}{lllrrrlrl}
  \hline
language & family & script & $L$ & $L_r$ & $\tau$ & $\tau_{pvalue}$ & $r$ & $r_{pvalue}$\\ 
  \hline
\input{tables/cv_opt_scores_medianDuration}
\end{tabular}
\end{table}




% \section{Desirable properties of the scores}
% \label{app:desirable}

% \input{appendix_desirable_properties}

\end{appendices}

\end{document}
