% \subsection{The relationship between $\Psi$ and Pearson correlation}
% \subsection{The relationship between \texorpdfstring{$L$}{L}, \texorpdfstring{$L_r$}{Lr} and Pearson correlation}

% \label{app:Pearson_correlation}

Here we review the relationship between \texorpdfstring{$L$}{L}, \texorpdfstring{$L_r$}{Lr} and Pearson correlation

Given two random variables $x$ and $y$ and a sample of $n$ points, $\left\{(x_1, y_1),...,(x_i, y_i),...,(x_n,y_n) \right\}$,
the sample covariance is defined as 
$$s_{xy} = \frac{1}{n-1}\left(\sum_{i=1}^n x_i y_i - n \bar{x}\bar{y}\right),$$
where $\bar{x}$ is the sample mean of $x$ and $\bar{y}$ is the sample mean for $y$, i.e. 
\begin{eqnarray*}
\bar{x}= \frac{1}{n} \sum_{i=1}^n x_i \\
\bar{y}= \frac{1}{n} \sum_{i=1}^n y_i.
\end{eqnarray*}
Now consider than the random variables are $p$ (the probability of a type) and $l$ (the length/duration of a type) instead of $x$ and $y$. 
Then our sample of $n$ points is $\left\{(p_1, l_1),...,(p_i, l_i),...,(p_n,l_n)\right\}$, one point per type. 
Accordingly, the covariance between $p$ and $l$ in a sample of points is 
$$s_{pl} = \frac{1}{n-1}\left(\sum_{i=1}^n p_i l_i - n \bar{p}\bar{l}\right).$$
Recalling the definition of $L$ (\autoref{eq:mean_type_length}) and noting that $\bar{p} = \frac{1}{n}$ and $\bar{l} = M = L_r$ (recall Property \ref{prop:random_baseline}), we finally obtain
$$s_{pl} = \frac{1}{n-1}(L - L_r).$$

The sample Pearson correlation is  
$$r = \frac{s_{xy}}{s_x s_y},$$
where $s_x$ and $s_y$ are the sample standard deviation of $x$ and $y$, i.e.
\begin{eqnarray*}
s_x = \sqrt{\frac{1}{n-1} \sum_{i=1}^n (x_i - \bar{x})^2} \\
s_y = \sqrt{\frac{1}{n-1} \sum_{i=1}^n (y_i - \bar{y})^2}.
\end{eqnarray*}
Proceeding as we did for the covariance, we find that the Pearson correlation between $p$ and $l$ is 
$$r = \frac{L - L_r}{(n-1)s_p s_l}.$$
Then it is easy to see that $L$ is a linear function of the Pearson correlation $r$ or $s_{pl}$. For instance, 
$$L = ar + b,$$
where 
\begin{eqnarray*}
a = (n-1)s_p s_l \\
b = L_r.
\end{eqnarray*}
Other linear relationships are left as an exercise. 
