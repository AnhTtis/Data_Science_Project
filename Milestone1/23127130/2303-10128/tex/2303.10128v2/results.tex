In \autoref{sec:introduction}, we highlighted the importance of distinguishing % between principles and their manifestations 
between direct and indirect evidence of compression. Against this theoretical backdrop, here we first investigate the presence of Zipf's law of abbreviation in languages. Then we investigate direct evidence of compression with the help of the new random baseline. 

\subsection{The law of abbreviation revisited}

\label{sec:law_of_abbrebiation}

We investigate the presence of the law of abbreviation by means of left-sided correlation tests for the association between frequency and length. We use both Kendall correlation, as suggested by theory on the origins of the law \parencite{Ferrer2019c}, and Pearson's. 
For each language, we show the significance level of the relationship, color-coded by the value of the correlation coefficient. \autoref{fig:corr_significance} (a,b) indicates that the law holds in all languages -- regardless of the definition of word length -- %%%%%%%%%%%%
when Kendall $\tau$ correlation is used. 
In both collections, we find Kendall $\tau$ correlation coefficients significant at the 99\% confidence level, except for Dhivehi in the CV collection when length is measured in characters, and Abkhazian, Dhivehi, Panjabi and Vietnamese when length is measured in duration. However, note that these are all still significant at the 95\% confidence level. When Pearson correlation is used instead, \autoref{fig:corr_significance} (c) shows that the picture remains the same in PUD.
%, \textcolor{red}{with a slight drop of p-value in Japanese}. 
The main findings are the same also in CV (\autoref{fig:corr_significance} (d)),
%, but with some minor changes: Dhivehi gains significance reaching 99\%, while
 but when length is measured in duration Panjabi ceases to be significant at the 95\% confidence level. Overall, we only fail to find the law of abbreviation in Panjabi given word durations, and using Pearson correlation. This is most probably related to undersampling, as this particular language only features 98 tokens (\autoref{tab:coll_summary_cv}). 

\begin{figure}[H]
    \centering
    \begin{subfigure}{.45\textwidth}
        \caption{}
        \includegraphics[width=8cm, height=10cm]{figures/corr_significance_pud_kendall.pdf}
    \end{subfigure}
    \hskip\baselineskip
    \begin{subfigure}{0.45\textwidth}
        \caption{}
        \includegraphics[width=8cm, height=10cm]{figures/corr_significance_cv_kendall.pdf}
    \end{subfigure}
        \begin{subfigure}{.45\textwidth}
        \caption{}
        \includegraphics[width=8cm, height=10cm]{figures/corr_significance_pud_pearson.pdf}
    \end{subfigure}
    \hskip\baselineskip
    \begin{subfigure}{0.45\textwidth}
        \caption{}
        \includegraphics[width=8cm, height=10cm]{figures/corr_significance_cv_pearson.pdf}
    \end{subfigure}
    \caption{\label{fig:corr_significance} 
     The correlation between frequency and length across languages.
     '***' indicates a Holm-Bonferroni corrected $p$-value lower than or equal to 0.01, '**' indicates lower than or equal to 0.05 but smaller than 0.1 and '*' indicates lower than or equal to 0.1. Here '*' symbols are not used to indicate significance but p-value ranges. 
     (a) Kendall $\tau$ correlation in PUD (word length in characters). (b) Kendall $\tau$ correlation in CV 
     (left: word length in characters; right: word length in duration).
     (c) Same as (a) with Pearson $r$ correlation. (d) Same as (b) with Pearson $r$ correlation. 
     }
\end{figure}

\subsection{Real word lengths versus the random baseline}


We investigate the relationship between the actual mean word length ($L$) and the random baseline ($L_r$). We find that $L < L_r$ for all languages in every collection (\autoref{fig:mean_word_length_versus_random_baseline} and Tables \ref{tab:opt_scores_pud}, \ref{tab:opt_scores_cv_characters}, \ref{tab:opt_scores_cv_meadianDuration}). Interestingly, there is a large gap between $L$ and $L_r$ in the majority of languages, which is more compelling in CV with word durations (\autoref{fig:mean_word_length_versus_random_baseline}).  
% This pattern emerges more clearly in the CV collection, thanks to the greater variability in terms of $L$ 
Exceptions to the large gap -- as in the case of Panjabi and Abkhazian when length is measured in duration -- mainly concern languages with reduced sample sizes. The result holds even when alternative units of measurement are considered for Chinese and Japanese. 


\autoref{fig:mean_word_length_versus_random_baseline} is reminiscent of Figure 4 of \textcite{Pimentel2021a} but our setting is much simpler (it only involves $L$ and $L_r$).


\begin{figure}[H]
    \centering
    \begin{subfigure}{.4\textwidth}
        \caption{}
        \includegraphics[scale=0.4]{figures/mean_vs_random_L_pud_characters.pdf}
    \end{subfigure}
    \hskip\baselineskip
    \begin{subfigure}{0.4\textwidth}
        \caption{}
        \includegraphics[scale=0.4]{figures/mean_vs_random_L_cv_characters.pdf}
    \end{subfigure}
        \begin{subfigure}{.45\textwidth}
        \caption{}
        \includegraphics[scale=0.4]{figures/mean_vs_random_L_cv_medianDuration.pdf}
    \end{subfigure}
    \caption{\label{fig:mean_word_length_versus_random_baseline} Mean word length ($L$) as a function of the random baseline ($L_r$) in languages. Every point stands for a language. The diagonal (long dashed line) indicates the line $L=L_r$. Languages with $L < L_r$ are located below the diagonal. (a) Languages in PUD with word length measured in characters (or strokes for Chinese and Japanese). (b) Languages in CV with word length measured in characters. (c) Languages in CV with word length measured in duration (seconds). 
    }
\end{figure}




\subsection{Impact of disabling the filter of words that contain ``foreign'' characters }

All results presented in this section have been obtained after applying the new method to filter out highly unusual characters and words described in \autoref{sec:filter}. If the filter is disabled, we obtain some slight changes in the values, but the qualitative results summarized above remain the same. 


