\section{Experimental Results}

\paragraph{2-Dimensional Holograms}
We validate our simulation results by capturing holograms in experiment. The SLM patterns were optimized for a propagation distance of \SI{120}{\milli\metre} using our perceptual loss function laid out in Section \ref{sec:perceptional-loss}.  A white border was added to each target image to improve the color fidelity by encouraging a proper white balance. After each hologram is captured, debayering is performed and a homography is applied to map from camera space to SLM space. The homography also downsamples the captured holograms to the same resolution as the SLM.  The captured results are shown in Figure ~\ref{fig:Stills}.  The images match simulation well, validating our simultaneous color algorithm, although experimental results are noisier with lower color fidelity due to model-mismatch. 

\paragraph{3-Dimensional Holograms}
As mentioned earlier a major appeal of holography is the ability to solve the vergence-accommodation conflict without the need for eye tracking.  Consequently, we also validate our method for 3D scenes.  A 4-plane focal stack was rendered with 0.5 pixels blur radius per millimeter depth.  Holograms were captured at distance from \SI{90}{\milli\metre} to \SI{120}{\milli\metre} in \SI{10}{\milli\metre} increments.  The results are displayed in Fig. \ref{fig:Stack}, and quality is similar to the 2D case. These experimental results demonstrate the ability to form 3D color holograms with natural defocus blur from a single SLM frame.