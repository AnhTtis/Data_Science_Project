\clearpage 
\begin{figure*}
    \centering
    \includegraphics[clip, trim = 0in 8in 0in 0in, width=\textwidth]{figures/Stills.pdf}
    \caption{\textbf{Experimentally captured 2D holograms.}
     This figure depicts experimentally captured holograms at a depth of \SI{120}{\milli\metre}.  Row one contains the experimentally captured images.  Row two is the simulation output of the optimized SLM pattern.  Row three contains the target images. While most the of captured holograms have good color fidelity, our method is least effective on highly saturated images with low texture, such as the cat paws in column 4, which represents a limitation or our method (see Sec. ~\ref{sec:limits}).}  
    \label{fig:Stills}
\end{figure*}



\begin{figure*}
    \centering
    \vspace{10mm}
    \includegraphics[clip, trim = 0in 7.5in 0in 0in, width=\textwidth]{figures/FocalStack.pdf}
    \caption{\textbf{Experimentally captured focal stack.}  This figure depicts a focal stack captured from $\SI{90}{\milli\metre}$ to $\SI{120}{\milli\metre}$ in $\SI{10}{\milli\metre}$ increments. Row one contains the experimentally captured images.  Row two is the simulation output of the optimized SLM pattern.  Row three contains the target images.}  
    \label{fig:Stack}
\end{figure*}

\begin{figure*}
    \centering
    \includegraphics[clip, trim = 0in 6.125in 0in 0in, width=\textwidth]{figures/ModelAblation.pdf}
    \caption{\textbf{Comparison of different propagation methods for suppressing higher diffraction orders.} The first column shows the results obtained using the traditional angular spectrum method (ASM) which doesn't model higher diffraction orders. The second column shows the results obtained using HOASM which reduces the visibility of higher orders but fails to completely suppress them. The third column shows the results obtained using our proposed learned propagation method that includes a U-net, which largely suppresses the higher diffraction orders and results in a hologram with the fewest artifacts, suggesting the learned propagation model best matches the physical propagation.
    }
    \label{fig:Ablation}
    \end{figure*}



\begin{figure*}
    \centering
    \vspace{10mm}
    \includegraphics[clip, trim = 0in 7.75in 0in 0in, width=.9\textwidth]{figures/Setup.pdf}
    \caption{\textbf{Experimental setup.} (A) Top view of our system with labeled components and approximate beam path is drawn in green.  (B) Side-view of the system.
    }
    \label{fig:Setup}
    \end{figure*}


\clearpage