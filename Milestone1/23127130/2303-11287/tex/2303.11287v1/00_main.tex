
\documentclass[acmtog,anonymous=false,review=false,nonacm]{acmart}

% \documentclass[12pt]{article}
\AtBeginDocument{%
  \providecommand\BibTeX{{%
    \normalfont B\kern-0.5em{\scshape i\kern-0.25em b}\kern-0.8em\TeX}}}

%%%%%%%%%%%%%%%%%%%%%%%%%%%%%%%%%%%%%%%%%%%%%%%
% Packages and definitions
 
\usepackage{xcolor}
\usepackage{siunitx}

\DeclareMathOperator*{\argmax}{argmax}
\DeclareMathOperator*{\argmin}{argmin}

\citestyle{acmauthoryear}

%%
%% end of the preamble, start of the body of the document source.
\begin{document}

\title{Simultaneous Color Holography}

\author{Eric Markley}
\email{emarkley@berkeley.edu}
 \affiliation{%
   \institution{Reality Labs Research, Meta}
%   \streetaddress{9805 Willows Road NE}
%   \city{Redmond}
%   \state{Washington}
  \country{USA}
%   \postcode{98052}
 }

\author{Nathan Matsuda}
 \affiliation{%
   \institution{Reality Labs Research, Meta}
%   \streetaddress{9805 Willows Road NE}
%   \city{Redmond}
%   \state{Washington}
  \country{USA}
%   \postcode{98052}
   }
% \email{nathan.matsuda@meta.com}

\author{Florian Schiffers}
% \email{fschiffers@meta.com}
 \affiliation{%
   \institution{Reality Labs Research, Meta}
%   \streetaddress{9805 Willows Road NE}
%   \city{Redmond}
%   \state{Washington}
  \country{USA}
%   \postcode{98052}
   }

\author{Oliver Coissart}
 \affiliation{%
   \institution{Reality Labs Research, Meta}
%   \streetaddress{9805 Willows Road NE}
%   \city{Redmond}
%   \state{Washington}
  \country{USA}
%   \postcode{98052}
   }
% \email{ocoissart@meta.com}



\author{Grace Kuo}
 \affiliation{%
   \institution{Reality Labs Research, Meta}
%   \streetaddress{9805 Willows Road NE}
%   \city{Redmond}
%   \state{Washington}
  \country{USA}
%   \postcode{98052}
   }
% \email{gracekuo@meta.com}

%%
%% By default, the full list of authors will be used in the page
%% headers. Often, this list is too long, and will overlap
%% other information printed in the page headers. This command allows
%% the author to define a more concise list
%% of authors' names for this purpose.
\renewcommand{\shortauthors}{Markley, et al.}

\makeatletter
\let\@authorsaddresses\@empty
\makeatother

%%
%% The abstract is a short summary of the work to be presented in the
%% article.
\begin{abstract}
%%%%%%%%% ABSTRACT



\begin{abstract}

% Version 1:
% Modern depth sensors such as LiDAR operate by sweeping laser-beams across the scene, resulting in a point cloud with notable 1D curve-like structures. However, most existing point cloud backbones  discard the rich, 1D traversal patterns and rely mainly on Euclidean operations.
% In this work, we present a novel point cloud processing scheme and backbone, \textbf{CurveCloudNet}, that exploits the curve-like structure of modern depth sensors. Concretely, %instead of treating each point independently, 
% we parameterize the point cloud as a collection of polylines and thus establish a local surface-level ordering on the points. 
% We then devise curve-specific operations to process the ``curve clouds:'' (1) a \textit{symmetrical 1D convolution}, 2) a \textit{ball grouping} operation for merging points along curves, and (3) an efficient \textit{1D furthest-point-sampling} algorithm on curves. \textbf{CurveCloudNet} combines these curve operations with existing point-based operations, resulting in an efficient, scalable, and expressive backbone that uses little GPU memory. We evaluate \textbf{CurveCloudNet} on the ShapeNet, Kortx, Audi Driving, and nuScenes datasets, showcasing state-of-the-art segmentation and classification performance across {\em both} object-level and large outdoor scene datasets, the first reported 3D point backbones to do so. 

% Version 2:
% In this work we introduce a new point cloud processing scheme and backbone, called CurveCloudNet, which takes advantage of the curve-like structure inherent in modern depth sensors such as LiDAR. While traditional point cloud backbones discard the rich, 1D laser-traversal patterns and rely on Euclidean operations, CurveCloudNet parameterizes the point cloud as a collection of polylines. This parameterization establishes a local surface-level ordering on the points. Our method applies curve-specific operations to process the ``curve clouds," including symmetrical 1D convolution, ball grouping for merging points along curves, and an efficient 1D furthest-point-sampling algorithm on curves. Combining these curve operations with existing point-based operations results in an efficient, scalable, and expressive backbone that uses little GPU memory. We evaluate CurveCloudNet on several datasets, including ShapeNet, Kortx, Audi Driving, and nuScenes, and report state-of-the-art segmentation and classification performance across \textbf{both} object-level and large outdoor scene datasets, making CurveCloudNet the first 3D point backbone to achieve such results.
% \vspace{-1em}

% Version 3
Modern depth sensors such as LiDAR operate by sweeping laser-beams across the scene, resulting in a point cloud with notable 1D curve-like structures. In this work, we introduce a new point cloud processing scheme and backbone, called \arch, which takes advantage of the curve-like structure inherent to these sensors. While existing backbones discard the rich 1D traversal patterns and rely on generic 3D operations, \arch parameterizes the point cloud as a collection of polylines (dubbed a ``curve cloud”), establishing a local surface-aware ordering on the points. By reasoning along curves, \arch captures lightweight curve-aware priors to efficiently and accurately reason in several \textbf{diverse} 3D environments. 
% , including a symmetric 1D convolution, a ball grouping for merging points along curves, and an efficient 1D farthest point sampling algorithm on curves.
We evaluate \arch on multiple synthetic and real datasets that exhibit distinct 3D size and structure.
%, including: ShapeNet, Audi Driving, nuScenes, Kitti, and a new dataset we name KortX.
We demonstrate that \arch outperforms both point-based and sparse-voxel backbones in various segmentation settings, notably scaling to large scenes better than point-based alternatives while exhibiting improved single-object performance over sparse-voxel alternatives.
In all, \arch is an efficient and accurate backbone that can handle a larger variety of 3D environments than past works. 
%In all, \arch is an off-the-shelf trainable and performant backbone that is ready for the diverse environments faced in open-world applications such as robotics. 

% in various segmentation settings, notably scaling better to large scenes than point-based alternatives while exhibiting better single object performance than sparse-voxel alternatives. 

% CurveCloudNet applies a mix of curve-specific operations and Euclidean point-based operations, resulting in an efficient and accurate backbone that can flexibly reason on \textit{many} different types of 3D scenes. 
% , including a symmetric 1D convolution, a ball grouping for merging points along curves, and an efficient 1D farthest point sampling algorithm on curves.
% By combining these curve operations with existing point-based operations, CurveCloudNet is an efficient and accurate backbone that can flexibly reason on \textit{many} different types of 3D scenes. 
% CurveCloudNet achieves state-of-the-art segmentation performance on the ShapeNet, Kortx, Audi Autonomous Driving, and nuScenes datsets, which include both individual objects and large outdoor scenes captured with various sensor scanning patterns. These evaluations demonstrate that \arch scales to large scenes better than existing point-based backbones while improving object-level semantic segmentation compared to sparse-voxel backbones.
% We evaluate semantic segmentation on four datasets - two common (ShapeNet and nuScenes) and two less common (KortX and Audi Driving). Taken together, these datasets patterns -
% We evaluate semantic segmentation the ShapeNet, Kortx, Audi Driving, and nuScenes datasets. 

% demonstrate that \arch outperforms both point-based and sparse-voxel backbones in various segmentation settings, notably scaling better to large scenes than point-based alternatives while exhibiting better single object performance than sparse-voxel alternatives. 

% Evaluations on ShapeNet, Kortx, Audi Driving, and nuScenes demonstrate that \arch outperforms point-based methods on both individual objects and large-scale scenes, outperforms sparse-voxel backbones on individual objects, and closes the gap between point-based and sparse-voxel backbones on large-scale scenes while requiring significantly less GPU memory.

% CurveCloudNet is evaluated on several datasets that include both individual objects and large
% outdoor scenes captured with various sensor scanning patterns. These evaluations demonstrate that our model can
% outperform point-based and sparse-voxel backbones at both
% object and scene level, achieving state-of-the-art performance on segmentation tasks.
\vspace{-1em}
\end{abstract}

\end{abstract}


\begin{teaserfigure}
  \includegraphics[clip, trim=0in 9.25in 0in 0in, width=\textwidth]{figures/Teaser.pdf}
  \caption{\textbf{Simultaneous color holograms captured in experiment.}  Traditionally, color holograms are illuminated sequentially with a unique spatial light modulator (SLM) pattern for each color channel.  In this work we outline a flexible framework that enables the use of a single SLM pattern for red-green-blue (RGB) holograms using simultaneous RGB illumination.  We validate this framework experimentally on a simple and compact optical setup.
  }
  \label{fig:teaser}
\end{teaserfigure}

%% This command processes the author and affiliation and title
%% information and builds the first part of the formatted document.
\maketitle

%%%%%%%%%%%%%%%%%%%%%%%%%%%%%%%%%%%%%%%%%%%
% Main Text





Figures of speech include metaphors, similes, and idioms that allow language to be expressive, to convey abstract ideas that might otherwise be difficult to visualize, and to evoke emotion \cite{why-do-people-use-figurative-language, SusanAndMallie1994}. A metaphor is a comparison between two unrelated concepts that enable us to think of the target concept in terms of the source concept. For example, in the sentence ``You’re a peach!'', the person being addressed is equated with a peach, with the suggestion that the person is pleasing or delightful. A simile is a figure of speech that compares two things and is often introduced by ``like'' or ``as'' \cite{Paul-1970}. A simile is called ``open'' when the shared properties are not explicitly revealed, like ``Her heart is like a stone'', and ``closed'' when they are explicitly revealed, like ``Her heart is hard as stone''. An idiom is a group of words with a figurative, non-literal meaning that can not be interpreted by looking at its individual words. For example, the idiom ``We're on the same page'' means ``Agreeing about something (such as how things should be done)''.
\begin{figure}[tp]
\includegraphics[width=0.45\textwidth ,height=\textheight,keepaspectratio]{figures/let_the_cat_out_of_the_bag_big.jpg}
\includegraphics[width=0.45\textwidth ,height=\textheight,keepaspectratio]{figures/blanket_of_snow_bigger.jpg}
\includegraphics[width=0.45\textwidth ,height=\textheight,keepaspectratio]{figures/car_cheetah_bigger.JPG}
\caption{Examples of the figurative understanding task for idiom, metaphor, and simile in corresponding order. The figurative phrase is displayed in the top section, and the bottom section displays four candidates from which the correct answer (orange) has been selected. Idiom tasks also display the idiom definitions below the idiom.}
\label{fig:first-task-idiom-figurative}
\end{figure}
Understanding metaphors and similes require the cognitive ability to map between domains, and depending on the source and target concept, it can require commonsense, association abilities, and general knowledge. Understanding idioms requires profound language, and cultural knowledge 
\cite{Paul-1970, philip2011colouring}. Humans intuitively understand these figures and employ them in everyday communication \cite{LakoffandJohnson1980, Hoffman1987WhatCR}. These figurative forms are often conveyed through multiple modes, such as text and images, and frequently appear in advertising, news, social media, etc. \\
%Figure~\ref{fig:multimodal} presents an image of a luxury car posted on social media with the simile ``As fast as a cheetah'' and an advertisement for Toyota Corolla with the idiom ``get your hands on''. \\
%\yonatan{so far you have a table and an image for explaining existing concepts. I wonder whether it should be in the Appendix rather than taking important space at the start of the paper, because it's not something new you present in the work.}\ron{@Dafna}
% Multimodal metaphors are metaphors conveyed through multiple modes ``whose target and source are each represented exclusively or predominantly in different modes'' \cite{Forceville2016}. Multimodal information from different modes, such as language and vision, can contribute to metaphorical conceptualization and comprehension \cite{PerspectivesMultimodalityBook, multimodalMetaphorBook}. The comprehension of multimodal metaphors takes cognitive efforts like decoding metaphorical messages and understanding the relationships between domains, analyzing the emotion metaphors convey, and interpreting authorial intent [5–7] \cite{conceptual-integration-and-metaphor, YANG2013312, FAUCONNIER1998133}. Since similes are often referred to as metaphors in the cognitive linguistic and rhetorical literature \cite{culler1981pursuit, lou2021multimodal}, the topic of multimodal similes has not been thoroughly investigated. Recently, Adrian Lou \cite{lou2021multimodal} proposed a novel and robust framework for the analysis of both verbal and multimodal similes. Multimodal idioms have yet to be studied cognitively, in part due to a lack of multimodal datasets.  \cite{LakoffandJohnson1980, Hoffman1987WhatCR}. 
% \begin{figure}[h]
% \includegraphics[width=0.48\textwidth,height=12cm,keepaspectratio]{figures/Figure 1 - multimodal example.JPG}
% \caption{An advertisement for Toyota Corolla from the 60s with the idiom "get your hands on" and an image posted on a social media platform with the caption ``As fast as a cheetah''.}
% \label{fig:multimodal}
% \end{figure}
% There has been an increase in the use of text and vision modes over the past few years due to the growing usage of social media and mass media. The increased use of multimodality has created new challenges, which sometimes expand existing challenges from monomodality to multimodality.
\newline\noindent Due to its integral part in human communication, the detection and comprehension of multimodal figurative language is an important aspect of various multimodal challenges. Among these challenges are hate speech detection in memes \cite{detecting-hate-speech-in-multi-modal-memes}, fact-checking \cite{multimodal-fact-checking}, sentiment analysis \cite{SOLEYMANI20173}, humor recognition \cite{REYES20121, detecting-sarcasm-in-multimodal-social-platforms}, and identifying depression in social media posts \cite{yadav-etal-2020-identifying, multimodal-time-aware-attention-networks}. Figure \ref{fig:figurative-language-social-media} shows two photos posted on social media with metaphoric captions. In the left image, the caption reads, ``Jumped off the sinking ship just in time'', as this player left Chelsea - ``the sinking ship'', which is having a bad year, to join the leading team of the premier league, Arsenal. The right image was posted with the caption ``A performing clown'', as the person who is getting hit is a famous YouTuber who lost in a boxing match against a professional boxer. Multimodal figurative understanding is required to comprehend the metaphorical message being conveyed in these two posts. % In the task of sentiment analysis, the sentiment of the right post should be classified as ``negative'', but this cannot be achieved solely by text or image.
% In light of this rapidly growing trend, figurative language studies must be expanded from monomodality to multimodality.
Vision and Language Pre-Trained Models’ (VL-PTMs) understanding of figurative language combined with vision has not been thoroughly examined, if at all, partly due to the absence of large-scale datasets with ground truth labels of multimodal smilies, idioms, metaphors, etc.\\
%\begin{table}[ht]
\begin{center}
\scalebox{0.6}{
\begin{tabularx}{380pt}{| c X X |}
 \hline
 Figure & Example & Explanation \\ [0.5ex] 
 \hline\hline
 Metaphor & "You're a peach!" & The person being addressed is being equated with a peach, with the suggestion that the person is pleasing or delightful. The target concept is "person" and the source concept is "peach".\\ 
 \hline
 Open Simile & "Her heart is like stone" &  Inflexible and unfriendly or unkind disposition. The shared properties of "her heart" and "stone" are not explicitly revealed. The target concept is "her heart" and the source concept is "stone".  \\
 \hline
 Closed Simile & "The old man walk as slow as a snail" &  The old man's movement is compared to that of a snail, the shared property ("slow") is explicitly revealed. The target concept is "old man" and the source concept is "snail".\\
 \hline
 Idiom & "We're on the same page" &  Agreeing about something (such as how things should be done).  \\
 \hline
\end{tabularx}}

\end{center}
\caption{\ron{@Dafna Should we delete this/move it to appendix? If not, I think merging it with Figure 2 could be a good idea}Examples of simile, metaphor and idiom with a corresponding explanation.}
\label{table:figures-of-speech}
\end{table}%
\begin{figure}[tp]
\begin{center}
\includegraphics[width=0.5\textwidth,height=12cm,keepaspectratio]{figures/figure-figurative_language_in_social_media-bigger.jpg}
\end{center}
\caption{Two photos posted on social media. The left photo depicts football player Jorge Luiz Frello Filho Cavaliere wearing an Arsenal football club uniform. The right photo shows famous YouTuber Jake Paul taking a hit from professional boxer Tommy Fury during their boxing fight.}
\label{fig:figurative-language-social-media}
\end{figure}
\newline \noindent In this work, we introduce the IRFL dataset of idioms, metaphors, and similes with matching figurative and literal images. We leveraged a textual dataset of idioms and an extensive pipeline we developed to find possible figurative and literal idiom images. We annotated these images via Amazon Mechanical Turk using the UI seen in (Appendix~\ref{sec:annotation_ui}) to create a large-scale dataset of idioms' figurative and literal images. In addition, we collected metaphors and similes' figurative and literal images. We then used the IRFL dataset to create two novel tasks of figurative understanding and figurative preference to examine the figurative understanding of Vision and Language models. The figurative understanding task evaluates VL-PTMs' ability to understand the relation between an image and a figurative phrase. The task is to choose the image that best visualizes the figurative phrase out of X candidates.  Figure~\ref{fig:first-task-idiom-figurative} shows an example of the task for idiom, metaphor, and simile. The preference task examines VL-PTMs' preference for figurative images. In this task, the model needs to rank figurative images of different categories correctly. Figure~\ref{fig:second-task-idiom-figurative-vs-caption} shows the expected order versus the actual order of the idiom ``ruffle someone's feathers'' images based on the model scores. Finally, we experiment with generative models such as Dall-E and Stable Diffusion to examine their ability to generate figurative images for idioms. %\ron{Should the result be in intro?}%
% \yonatan{change quotation marks to ``THIS''} - Did not understand this
% \yonatan{figures in seperate files?} - Didn't understand this
% The figurative understanding task evaluates VL-PTMs' ability to understand the relation between an image and a figurative phrase. The task is to choose the image that best visualizes the figurative phrase out of X candidates.  Figure~\ref{fig:first-task-idiom-figurative} shows an example of the task for idiom, metaphor, and simile.
% The figurative understanding task evaluates VL-PTMs' ability to understand the relation between an image and a figurative phrase. The task is to choose the image that best visualizes the figurative phrase out of X candidates. Figure~\ref{fig:first-task-idiom-figurative} shows an example for the idiom ``Let the cat out of the bag'' - to disclose a secret, and examples with difficult distractors for the metaphor "Sea of bees" and the simile "The child is as proud as a peacock". Using partially literal images as distractors, the best model ($22\%$) fails to match human performance ($92\%$). This task goes beyond object detection and scene understanding, it requires a profound and rich language and cultural knowledge (idioms) in addition to commonsense, abstraction, general knowledge and the ability to decode the domains of figurative phrases and understand their relationships (metaphors and similes).
% \newline \noindent The preference task examines VL-PTMs' preference of figurative images over partially literal images. In this task, the model needs to rank figurative phrase images of different categories correctly. We suggest that Vision and language models should prioritize figurative images over partial literal images. Meaning that an image of someone disclosing a secret should receive a higher matching score for the idiom  ``let the cat out of the bag'' than an image of a bag. Figure~\ref{fig:second-task-idiom-figurative-vs-caption} shows the expected order versus the actual order of the idiom ``ruffle someone's feathers'' images based on the model scores. Assuming that the model understands the idiom and sees a figurative connection to an image, the task purpose is to measure how well the model comprehends it. This is done by comparing the figurative images' matching score to other images with a weaker relationship (partially literal). We find that the best model receives a $F_1$ score of $35-50$ depending on the figure of speech.
% \noindent Finally, we experiment with generative models such as Dall-E and Stable Diffusion to examine their ability to generate figurative images for idioms. We provide these models with idioms and their definitions as prompts and compare the results to our automatic pipeline. Our findings show that Stable Diffusion fails to generate figurative images even when given the definitions as input, while Dall-E succeeds in generating figurative images for the idioms' definitions.
\begin{figure}[h]
\includegraphics[width=0.45\textwidth,height=\textheight,keepaspectratio]{figures/Figure_5_ranking_task.JPG}
\caption{The Figurative and Partial Objects images of the idiom ``ruffle someone's feathers'' - ``To unease, cause discomfort to someone'' sorted from right to left by the CLIP-VIT-L/14 score. The images with the letter F are figurative, and the images with the letter P are partially literal. Green indicates correct rank, red indicates incorrect rank. The first row shows the ideal ranking order, while the second row shows the actual one. Figurative images with a green letter will appear in the first \#F image from the right, and Partial Objects images with a green letter will appear in the last \#P image. In this example, all of the models except LiT received 0 $F_1$ score.}
\label{fig:second-task-idiom-figurative-vs-caption}
\end{figure}



\begin{figure*}
    \centering
    \includegraphics[clip, trim = 0in 8.5in 0in 0in, width=\textwidth]{figures/ForwardModel.pdf}
    \caption{\textbf{Hologram optimization framework.}
    This figure illustrates the three key components of the simultaneous color optimization framework: an SLM model, a propagation model, and a perceptual loss function. The SLM model maps voltage values to a complex field using a learned cross-talk kernel and a linear lookup table. The complex wavefront from the SLM is then propagated to the sensor plane using a modified version of the model proposed by \citet{Gopakumar2021UnfilteredDisplays}, which separates the zeroth and first diffraction orders and combines them through a U-Net. The output is then fed into the perceptual loss function, and gradients are calculated using Pytorch's autograd implementation. The SLM voltages are then updated using these gradients. Rubik's cube source image by Iwan Gabovitch (CC BY 2.0).}  
    \label{fig:ForwardModel}
\end{figure*}

\section{Related Works}
\label{RelatedWorks}

\paragraph{Field Sequential Color}
The vast majority of color holographic displays use field sequential color in which the SLM is sequentially illuminated by red, green, and blue sources while the SLM pattern is updated accordingly \cite{maimone2017holographic, jang2018holographic, Chakravarthula2019WirtingerDisplays, Chakravarthula2020LearnedDisplays, chakravarthula2022pupil,  Peng2020NeuralTraining, Peng2021Speckle-freeCalibration,Choi2021NeuralDisplays, choi2021optimizing, shi2021towards, yang2022diffraction, li2016holographic}. Field sequential color is effective at producing full color holograms but reduces framerate by a factor of $3\times$. This is a challenge for LCoS SLMs where framerate is severely limited by the LC response time \cite{zhang2014fundamentals}. Although, SLMs based on MEMS technology can run at high framerates in the kilohertz range \cite{MEMS}, so far these modulators are maximum 4-bit displays, with most being binary \cite{Choi2022Time-multiplexedModulators, kim2022accommodative, lee2022high}. Even with high framerate modulators, it may be worthwhile to maintain the full temporal bandwidth, since the extra bandwidth can be used to address other holography limitations. For example, speckle can be reduced through temporal averaging \cite{Choi2022Time-multiplexedModulators, kim2022accommodative, lee2022high}, and limited etendue can be mitigated through pupil scanning \cite{jang2018holographic, kim2022holographic}.

\paragraph{Spatial Multiplexing} An alternate approach is spatial multiplexing, which maintains the native SLM framerate by using different regions of the SLM for each color. Most prior works in this area use three separate SLMs and an array of optics to combine the wavefronts~\cite{Yaras2009Real-timeIllumination, Shiraki2009SimplifiedLinks, Nakayama2010Real-timePanels}. 
Although this method produces high quality holograms, the resulting systems are  bulky, expensive, and require precise alignment, making them poorly suited for near-eye displays. Spatial multiplexing can also be implemented with a single SLM split into sub-regions~\cite{Makowski2011SimpleColor, Makowski2009ExperimentalDisplay}; while less expensive, this approach still requires bulky combining optics and sacrifices space-bandwidth product (SBP), also known as etendue. Etendue is already a limiting factor in holographic displays \cite{kuo2020high}, and further reduction is undesirable as it limits the range of viewing angles or display field-of-view.

\paragraph{Frequency Multiplexing} Rather than split the physical extent of the SLM into regions, frequency multiplexing assigns each color a different region in the frequency domain, and the colors are separated with a physical color filter at the Fourier plane of a 4$f$ system \cite{Makowski2010ColorHolograms, Lin17, Lin19}. A variation on this idea uses different angles of illumination for each color so that the physical filter in Fourier space is not color-specific \cite{Xue:14}. Frequency multiplexing can also be implemented with white light illumination, which reduces speckle noise at the cost of resolution \cite{Kozacki16, yang2019full}. However, all of these techniques involve filtering in Fourier space, which sacrifices system etendue and requires a bulky 4$f$ system.

\paragraph{Depth Division and Bit Division for Simultaneous Color} The prior methods most closely related to our work also use simultaneous RGB illumination over the SLM, maintain the full SLM etendue, and don't require a bulky 4$f$ system \cite{Pi2022ReviewDisplay}. We refer to the first method as \textit{depth division multiplexing} which takes advantage of the ambiguity between color and propagation distance (explained in detail in Sec. \ref{sec:color-depth-ambiguity}) and assigns each color a different depth~\cite{Makowski2008ColorfulHologram, Makowski2010ColorHolograms}. After optimizing with a single color for the correct multiplane image, the authors show they can form a full color 2D hologram when illuminating in RGB. However,  this approach does not account for wavelength dependence of the SLM response, and since it explicitly defines content at multiple planes, it translates poorly to 3D.

Another similar approach is \textit{bit division multiplexing}, which takes advantage of the extended phase range of LCoS SLMs \cite{Jesacher2014ColourRange}. The authors calibrate an SLM lookup-table consisting of phase-value triplets (for RGB) as a function of digital SLM input, and they note that SLMs with extended phase range (up to $10\pi$) can create substantial diversity in the calibrated phase triplets. After pre-optimizing a phase pattern for each color separately, the lookup-table is used on a per-pixel basis to find the digital input that best matches the desired phase for all colors. In our approach, we also use an extended SLM phase range for the same reason, but rather than using a two-step process, we directly optimize the output hologram. This flexible framework also allows us to incorporate a perceptual loss function to further improve perceived image quality.

\paragraph{Algorithms for Hologram Generation}
Our work builds on a body of literature applying iterative optimization algorithms to holographic displays. Perhaps most popular is the Gerchberg-Saxton (GS) method \cite{gerchberg1972practical}, which is effective and easy to implement, but does not have an explicitly defined loss function, making it challenging to adapt to specific applications. \citet{zhang20173d} and \citet{Chakravarthula2019WirtingerDisplays} were the first to explicitly formulate the hologram generation problem in an optimization framework. This framework has been very powerful, enabling custom loss functions \cite{Choi2022Time-multiplexedModulators} and flexible adaptation to new optical configurations \cite{choi2021optimizing, Gopakumar2021UnfilteredDisplays}. In particular, perceptual loss functions can improve the perceived image by taking aspects of human vision into account, such as human visual acuity \cite{kuo2020high}, foveated vision \cite{walton2022metameric}, and sensitivity to spatial frequencies during accommodation \cite{kim2022accommodative}. Like these prior works, we use an optimization-based framework which we adapt to account for the wavelength-dependence of the SLM; this also enables our  new perceptual loss function for color, which is based on visual acuity difference between chrominance and luminance channels.


\paragraph{Camera-Calibration of Holographic Displays}
When the model used for hologram generation does not match the physical system, deviations cause artifacts in the experimental holograms. Recently, several papers have addressed this issue using measurements from a camera in the system for calibration.  \citet{Peng2020NeuralTraining} proposed using feedback from the camera to update the SLM pattern for a particular image; although a single image can be improved, it does not extend to new content.
A more flexible approach uses pairs of SLM patterns and camera captures to estimate the learnable parameters in a model, which is then used for offline hologram generation. Learnable parameters can be physically-based \cite{Peng2020NeuralTraining, kavakli2022learned, Chakrabarti2016LearningBack-propagation}, black box CNNs \cite{Choi2021NeuralDisplays}, or a combination of both \cite{Choi2022Time-multiplexedModulators}. The choice of learnable parameters effects the ability of the model to match the physical system; we introduce a new parameter for modeling SLM cross talk and tailor the CNN architecture for higher diffraction orders from the SLM.  

\section{Simultaneous Color Holography}
\label{ProblemStatement}

A holographic image is created by a spatially coherent illumination source incident on an SLM. The SLM imparts a phase delay on the electric field; after light propagates some distance, the intensity of the electric field forms an image. Our goal in this work is to compute a single SLM pattern that simultaneously creates a three color RGB hologram. For instance, when the SLM is illuminated with a red source, the SLM forms a hologram of the red channel of an image; with a green source the same SLM pattern forms the green channel; and with the blue source it creates the blue channel. 

We propose a flexible optimization-based framework (Fig.~\ref{fig:ForwardModel}) for generating simultaneous color holograms. We start with a generic model for estimating the hologram from the digital SLM pattern, $s$, as a function of illumination wavelength, $\lambda$:
\begin{align}
    g_{\lambda} &= e^{i\phi_{\lambda}\left(s\right)} \label{eq:SLM_to_field} \\
    I_{z, \lambda} &= \left| f_{\text{prop}} \left( g_{\lambda}, z, \lambda \right) \right|^2. \label{eq:field_to_intensity}
\end{align}
Here, $\phi_{\lambda}$ is a wavelength-dependent function that converts the 8 bit digital SLM pattern to a phase delay, $g_{\lambda}$ is the electric field coming off the SLM, $f_{\text{prop}}$ represents propagation of the electric field, and $I_{z, \lambda}$ is the intensity a distance $z$ from the SLM.

To calculate the SLM pattern, $s$, we can solve the following optimization problem
\begin{align} \label{eq:loss_func_rgb}
    \argmin_{s} \sum_z \mathcal{L}\left(\hat{I}_{z, \lambda_r}, I_{z, \lambda_r}\right) + \mathcal{L}\left(\hat{I}_{z, \lambda_g}, I_{z, \lambda_g}\right) + \mathcal{L}\left(\hat{I}_{z, \lambda_b}, I_{z, \lambda_b}\right),
\end{align}
where $\hat{I}$ is the target image, $\mathcal{L}$ is a pixel-wise loss function such as mean-square error, and $\lambda_r, \lambda_g, \lambda_b$ are the wavelengths corresponding to red, green, and blue respectively. Since the model is differentiable, we solve Eq. \ref{eq:loss_func_rgb} with gradient descent.

\begin{figure}
    \centering
    \includegraphics[clip, trim = 0in 7.7in 4.25in 0in ,width=.45\textwidth]{figures/Replicates.pdf}
    \caption{
    %
    \textbf{Extended phase range reduces depth replicas in simulation.}
    (A) Using an SLM with a uniform $2\pi$ phase range across all channels leads to strong depth replicas (top row), which reduce image quality at the target plane compared to the target (bottom row) and add in-focus content at depths that should be defocused. By using the extended phase Holoeye Pluto-2.1-Vis-016 SLM (with  Red: $2.4\pi$, Green: $5.9\pi$, Blue: $7.4\pi$ phase ranges), depth replicas are significantly reduced (middle row), improving the quality of target plane holograms and creating defocused content at other depths. (B) Schematic illustrating the positions of the replicate planes and target plane. Note, this simulation was generated using RGB images and three color channels, but only the green and blue channels are displayed for clarity. (Rubik's cube source image by Iwan Gabovitch (CC BY 2.0).}
    \label{fig:replicates}
\end{figure}


\subsection{Color-Depth Ambiguity} \label{sec:color-depth-ambiguity}
A common model for propagating electric fields is Fresnel propagation\footnote{Fresnel propagation is the paraxial approximation to the popular angular spectrum method (ASM). Since most commercials SLMs have pixel pitch greater than \SI{3}{\micro\metre}, resulting in a maximum diffraction angle under $5^\circ$ (well within the small angle approximation), Fresnel and ASM are almost identical for holography.} \cite{goodmap2005fourier}, which can be written in Fourier space as
\begin{align}
    f_{\text{fresnel}}(g, z, \lambda) &= \mathcal{F}^{-1} \left\{ \mathcal{F}\{g\} \cdot H(z, \lambda)\right\} \label{eq:fresnel_convolution}\\
    H(z, \lambda) &= \exp \left(i \pi \lambda z \left(f_x^2 + f_y^2\right) \right) \label{eq:fresnel_kernel}% Fresnel propagation kernel
\end{align}
where $\mathcal{F}$ is a 2D Fourier transform, $H$ is the Fresnel propagation kernel, and $f_x$, $f_y$ are the spatial frequency coordinates. In Eq.~\ref{eq:fresnel_kernel}, note that $\lambda$ and $z$ appear together, creating an ambiguity between wavelength and propagation distance.

To see how this ambiguity affects color holograms, consider the case where $\phi_{\lambda}$ in Eq. \ref{eq:SLM_to_field} is independent of wavelength ($\phi_{\lambda} = \phi$). For example, this would be the case if the SLM had a linear phase range from 0 to $2\pi$ at every wavelength.
%
Although this is unrealistic for most off-the-shelf SLMs, it is a useful thought experiment.
%
Note that if $\phi$ is wavelength-independent, then so is the electric field off the SLM ($g_\lambda = g$). In this scenario, assuming $f_\text{prop} = f_\text{fresnel}$, the Fresnel kernel is the only part of the model affected by wavelength.

%

Now assume that the SLM forms an image at distance $z_0$ under red illumination. From the ambiguity in the Fresnel kernel, we have the following equivalence:
\begin{align}
    H(z_0, \lambda_r) = H\left(\tfrac{\lambda_g}{\lambda_r} z_0, \lambda_g\right) = H\left(\tfrac{\lambda_b}{\lambda_r} z_0, \lambda_b\right).
\end{align}
This means the \textit{same} image formed in red at $z_0$ will also appear at $z = z_0 \lambda_g/\lambda_r$ when the SLM is illuminated with green and at $z = z_0 \lambda_b / \lambda_r$ when the SLM is illuminated with blue. We refer to these additional copies as ``depth replicas,'' and this phenomena is depicted in Fig. ~\ref{fig:replicates}.
%
Note that depth replicas do not appear in sequential color holography, since the SLM pattern optimized for red is never illuminated with the other wavelengths.

If we only care about the hologram at the target plane $z_0$, then the depth replicas are not an issue, and in fact, we can take advantage of the situation for hologram generation:
%
The SLM pattern for an RGB hologram at $z_0$ is equivalent to the pattern that generates a three-plane red hologram where the RGB channels of the target are each at a different depth ($z0$,  $z_0 \lambda_r/\lambda_g$, and  $z_0\lambda_r/\lambda_b$ for RGB respectively).
%
This is the basis of the depth division multiplexing approach of \citet{Makowski2008ColorfulHologram, Makowski2010ColorHolograms}, where the authors optimize for this three-plane hologram in red, then illuminate in RGB.
%
Although this makes the assumption that $\phi$ does not depend on $\lambda$, this connection between simultaneous color and multi-plane holography suggests simultaneous color should be possible for a single plane, since multi-plane holography has been successfully demonstrated in prior work.

However, the ultimate goal of holography is to create 3D imagery, and the depth replicas could prevent us from placing content arbitrarily over the 3D volume.
%
In addition, in-focus images can appear at depths that should be out-of-focus, which may prevent the hologram from successfully driving accommodation \cite{kim2022accommodative}.
%
We propose taking advantage of SLMs with extended phase range to mitigate the effects of depth replicas.


\subsection{SLM Extended Phase Range} \label{sec:extended-phase}
In general, the phase $\phi_\lambda$ of the light depends on its wavelength, which was not considered in Sec. \ref{sec:color-depth-ambiguity}.
% 
Perhaps the most popular SLM technology today is LCoS, in which rotation of birefringent LC molecules causes a change in refractive index.
%
The phase of light traveling through the LC layer is delayed by
\begin{align}
    \phi_\lambda = \frac{2 \pi d}{\lambda}n(s, \lambda), \label{eq:phase-LC}
\end{align}
where $d$ is the thickness of the LC layer, and its refractive index, $n$, is controlled with the digital input $s$.
%
$n$ also depends on $\lambda$ due to dispersion \cite{Jesacher2014ColourRange}.
% 
The wavelength dependence of $\phi_\lambda$ presents an opportunity to reduce or remove the depth replicas.
%
Even if the propagation kernel $H$ is the same for several $(\lambda, z)$ pairs, if the phase, and therefore the electric field off the SLM, changes with $\lambda$, then the output image intensity at the replica plane will also be different.
%
As the wavelength-dependence of $\phi_\lambda$ increases, the replicas are diminished.

We can quantify the degree of dependence on $\lambda$ by looking at the derivative $d\phi / {d\lambda}$ which informs us that larger $n$ will give $\lambda$ more influence on the SLM phase.  
%
However, the final image intensity depends only on relative phase, not absolute phase;
%
therefore, for the output image to have a stronger dependence on $\lambda$, we desire larger $\Delta n = n_{\text{max}} - n_{\text{min}}$.
%
 In addition, $d\phi / {d\lambda}$ increases with $-dn / {d\lambda}$, suggesting that more dispersion is helpful for simultaneous color. Although $d\phi / {d\lambda}$ also depends on the absolute value of $\lambda$, we have minimal control over this parameter since there are limited wavelengths corresponding to RGB. In summary, this means we can reduce depth replicas in simultaneous color with larger phase range on the SLM and higher dispersion.

However, there is a trade-off: 
%
As the range of phase increases, the limitations of the bit depth of the SLM become more noticeable, leading to increased quantization errors.
%
We simulate the effect of quantization on hologram quality and find that PSNR and SSIM are almost constant for 6 bits and above.
%
This suggests that each $2\pi$ range should have at least 6 bits of granularity.
%
Therefore, we think that using a phase range of around $8\pi$ for an 8-bit SLM will be the best balance between replica reduction and maintaining accuracy for hologram generation.
%
Figure ~\ref{fig:replicates} simulates the effect of extended phase range on depth replica removal. While holograms were calculated on full color images, only two color channels are shown for simplicity.
%

%
In the first row of Fig. ~\ref{fig:replicates}, we simulate an SLM
with no wavelength dependence to $\phi$ (i.e. 0 - $2\pi$ phase range for each color). Consequently, perfect copies appear at the replica planes.
%
In the second row, we simulate using the specifications from an extended phase range SLM (Holoeye Pluto-2.1-Vis-016), which has $2.4\pi$ range in red, $5.9\pi$ range in green, and $7.4\pi$ range in blue demonstrating that replicas are substantially diminished with an extended phase range.
%
By reducing the depth replicas, the amount of high frequency out of focus light at the sensor plane is reduced leading to improved hologram quality.

\begin{figure}
    \centering
    \includegraphics[clip, trim = 0in 8.125in 4.27in 0in, width=.45\textwidth]{figures/PerceptualLoss.pdf}
    \caption{
    \textbf{Perceptual loss improves color fidelity and reduces noise in simulation.}
    %
    The first column of this figure depicts simulated holograms that were optimized with an RGB loss function (A) and our perceptual loss function (B).  The same filters for the perceptual loss function then were applied to both of these simulated holograms as well as the target image.  Image metrics were calculated between the filtered holograms and the filtered target image (D). All image metrics are better for the perceptually optimized hologram (B). One should also note that the filtered target (D) and original target (C) are indistinguishable suggesting our perceptual loss function only removes information imperceptible by the human visual system as intended.}
    \label{fig:perceptual_loss}
\end{figure}

\begin{figure*}
    \centering
    \includegraphics[clip, trim = 0in 8.35in 0in 0in, width=\textwidth]{figures/SimulationComparison.pdf}
    \caption{
    %
    \textbf{Comparison of bit division, depth division and our method of simultaneous color holography in simulation.}
    %
    Bit division (Col. 1) is noisier than our method but achieves comparable color fidelity, although more washed out.  The depth division method (Col. 2) is also noisier than our method and has inferior color fidelity.  Our method matches the target image well.  Our method uses our perceptual loss function and a high order angular spectrum propagation model with no learned components. Further implementation details for each method are available in the supplement.
    }
    \label{fig:Simulation}
\end{figure*}

\subsection{Perceptual Loss Function} \label{sec:perceptional-loss}

% taking out the statement about 3D being harder since we use a mask so there are the same number of target outputs with 3D as 2D.

Creating an RGB hologram with a single SLM pattern is an overdetermined problem as there are $3\times$ more output pixels than degrees of freedom of the SLM. As a result, it may not be possible to exactly match the full RGB image, which can result in color deviations and de-saturation. To address this, we take advantage of color perception in human vision. There's evidence that the human visual systems converts RGB images into a luminance channel (a grayscale image) and two chrominance channels, which contain information about the color \cite{wandell1995foundations}. The visual system is only sensitive to high resolution features in the luminance channel, so the chrominance channels can be lower resolution with minimal impact on the perceived image \cite{wandell1995foundations}. This observation is used in JPEG compression \cite{pennebaker1992jpeg} and subpixel rendering \cite{platt2000optimal}, but to our knowledge, it has never been applied to holographic displays. By allowing the unperceived high frequency chrominance and extremely high frequency luminance features to be unconstrained, we can better use the the degrees of freedom on the SLM to faithfully represent the rest of the image.

Our flexible optimization framework allows us to easily change the RGB loss function in Eq. \ref{eq:loss_func_rgb} to a perceptual loss. For each depth, we transform the RGB intensities of both $\hat{I}$ (the target image) and $I$ (the simulated image from the SLM) into opponent color space as follows: 
\begin{equation}
\begin{split}
    O_{1} &= 0.299 \cdot I_{\lambda_r} + 0.587 \cdot I_{\lambda_g} + 0.114 \cdot I_{\lambda_b} \\
    O_{2} &= I_{\lambda_r} - I_{\lambda_g}\\
    O_{3} &= I_{\lambda_b} - I_{\lambda_r} - I_{\lambda_g}
\end{split}
\end{equation}
where $O_1$ is the luminance channel, and $O_2$, $O_3$ are the red-green and blue-yellow chrominance channels, respectively. We can then update Eq. \ref{eq:loss_func_rgb} to
\begin{equation} \label{eq:loss_func_opponent}
\begin{split}
    \argmin_{s} \sum_z \Big[ &\mathcal{L}\left(\hat{O}_{1} * k_1, {O}_{1} * k_1 \right) + 
    \mathcal{L}\left(\hat{O}_{2} * k_2, {O}_{2} * k_2 \right) + \\
    &\mathcal{L}\left(\hat{O}_{3} * k_3, {O}_{3} * k_3 \right) \Big],
\end{split}
\end{equation}
where $*$ represents a 2D convolution with a low pass filter ($k_1 \hdots k_3$) for each channel in opponent color space . $\hat{O}_i$ and $O_i$ are the $i$-th channel in opponent color space of $\hat{I}$ and $I$, respectively. In order to mimic the contrast sensitivity functions of the human visual system, we implement filters in the Fourier domain by applying a low-pass filter of 45\% of the width of Fourier space to the chrominance channels ($O_2$, $O_3$) and a filter of 75\% of the width of Fourier space to the luminance channel ($O_1$).  These filter widths were heuristically determined.

By de-prioritizing high frequencies in chrominance and extremely high frequencies in luminance, the optimizer is able to better match the low frequency color. This low frequency color is what is perceivable by the human visual system. Figure \ref{fig:perceptual_loss}  depicts the hologram quality improvement by optimizing with our perceptual loss function.  

The first column of Fig. ~\ref{fig:perceptual_loss} shows the perceptually filtered versions of simulated holograms generated using an RGB loss function (Fig \ref{fig:perceptual_loss}A) and our perceptual loss function (Fig \ref{fig:perceptual_loss}B). The second column displays the original target image (Fig \ref{fig:perceptual_loss}C) and the perceptually filtered target image (Fig \ref{fig:perceptual_loss}D). It can be observed that the two targets are indistinguishable, indicating that our perceptual filter choices align well with the human visual system. The PSNR and SSIM values are higher for the perceptually optimized hologram, and it also appears less noisy and with better color fidelity. This suggests that the loss function has effectively shifted most of the error into imperceptible regions of the opponent color space.  We see an average PSNR increase of 6.4 dB and average increase of 0.266 in SSIM across a test set of 294 images.


\subsection{Simulation Comparisons}

In Figure ~\ref{fig:Simulation} we compare the performance of our method to the depth and bit division approach to simultaneous color holography.  Depth and bit division use only a single SLM, make use of the full space-time-bandwidth product of the SLM, and contain no bulky optics or filters making these methods the most similar to our method.  The holograms simulated with depth and bit division are much noisier and have lower color fidelity than our proposed method.  The depth division simulated hologram has the worst color fidelity due to to the replica planes discussed in Sec. \ref{sec:color-depth-ambiguity} contributing defocused light at the target plane.  Our method uses a perceptual loss function and the HOASM outlined by ~\citet{Gopakumar2021UnfilteredDisplays} to directly optimize the simultaneous color hologram, while comparison methods optimize indirectly.  This direct approach produces less noisy holograms with better color fidelity.


\section{Camera-Calibrated Model}




%


We've demonstrated that our algorithm can generate simultaneous color holograms in simulation. However, experimental holograms frequently do not match the quality of simulations due to mismatch between the physical system and the model used in optimization (Eqs. \ref{eq:SLM_to_field}, \ref{eq:field_to_intensity}). Therefore, to demonstrate simultaneous color experimentally, we need to calibrate the model to the experimental system. 

To do this, we design a model based on our understanding of the system's physics, but we include several learnable parameters representing unknown elements. To fit the parameters, we capture a dataset of SLM patterns and camera captures and use gradient descent to estimate the learnable parameters based on the dataset. Next we explain the model which is summarized in Fig. \ref{fig:ForwardModel}.

\subsection{Learnable Parameters for Offline Calibration} \label{sec:learnable-model}

\paragraph{Lookup Table} 
A key element in our optimization is $\phi_\lambda$ which converts the digital SLM input into the phase coming off the SLM, and it's important that this function accurately matches the behavior of the real SLM. Many commercial SLMs ship with a lookup-table (LUT) describing $\phi_\lambda$; however, the manufacturer LUT is generally only calibrated at a few discrete wavelengths.  These wavelengths may not be accurate for the source used in the experiment. Therefore, we learn a LUT for each color channel as part of the model. Based on a pre-calibration of the LUT using the approach of \citet{yang2015nonlinear}, we observe the LUT is close to linear; we therefore parameterize the LUT with a linear model to encourage physically realistic solutions.

\paragraph{SLM Crosstalk}
SLMs are usually modeled as having a constant phase over each pixel with sharp transitions at boundaries. However, in LCoS SLMs, 
elastic forces in the LC layer prevent sudden spatial variations, and the electic field that drives the pixels changes gradually over space. As a result, LCoS SLMs suffer from crosstalk, also called field-fringing, in which the phase is blurred \cite{apter2004fringing, moser2019model, persson2012reducing}. We model crosstalk with a convolution on the SLM phase. Combined with our linear LUT described above, we can describe the phase off the SLM as
\begin{align} \label{eq:learned-model-01-LUT}
    \phi_\lambda(s) = k_{\text{xt}} * (a_1 \cdot s + a_2)
\end{align}
where $a_1, a_2$ are the learn parameters of the LUT, and $k_{\text{xt}}$ is a learned $5 \times 5$ convolution kernel representing crosstalk. Separate values of these parameters are learned for each color channel.

\paragraph{Propagation with Higher Diffraction Orders}
The discrete pixel structure on the SLM creates higher diffraction orders that are not modeled well with ASM or Fresnel propagation. A physical aperture at the Fourier plane of the SLM can be used to block higher orders. However, accessing the Fourier plane requires a 4$f$ system, which adds significant size to the optical system, reducing the practicality for head-mounted displays. Therefore, we chose to avoid additional lenses after the SLM and instead account for higher orders in the propagation model.

% 
We adapt the higher-order angular spectrum model (HOASM) of \citet{Gopakumar2021UnfilteredDisplays}. 

The zero order diffraction, $G(f_x, f_y)$, and first order diffraction, $G_{\text{1st order}}$, patterns are propagated with ASM to the plane of interest independently. Then the propagated fields are stacked and passed into a U-net, which combines the zero and first orders and returns the image intensity:
\begin{align} 
f_{\text{ASM}}(G, z) &= \mathcal{F}^{-1}\left\{ G \cdot H_{\text{ASM}}(z)\right\} \label{eq:learned-model-03-propagation} \\ 
I_z &= \text{Unet}\left(f_\text{ASM}(G, z), \: f_\text{ASM}(G_{\text{1st order}}, z)\right), \label{eq:learned-model-04-unet}
\end{align}
where $H_{\text{ASM}}(z)$ is the ASM kernel. The U-Net architecture is detailed in the supplement; a separate U-net for each color is learned from the data. The U-Net helps to address any unmodeled aspects of the system that may affect the final hologram quality such as source polarization, SLM curvature, and beam profiles, and the U-net better models superposition of higher orders, allowing for more accurate compensation in SLM pattern optimization. Figure~\ref{fig:Ablation} compares ASM, HOASM, and our modified version with the U-Net.

\section{Implementation}

\paragraph{Experimental Setup}
Our system starts with a fiber coupled RGB source, collimated with a $\SI{400}{\milli\metre}$ lens. The beam is aligned using two mirrors, passes through a linear polarizer and beamsplitter, reflects off the SLM (Holoeye-2.1-Vis-016), and passes through the beamsplitter a second time before directly hitting the color camera sensor with Bayer filter (FLIR GS3-U3-123S6C). As seen in Fig.~\ref{fig:Setup}, there's no bulky 4$f$ system between the SLM and camera sensor, which allows the setup to be compact, but requires modeling of higher diffraction orders. The camera sensor is on a linear motion stage, enabling a range of propagation distances from $z = \SI{80}{\milli\metre}$ to $z = \SI{130}{\milli\metre}$.    

For our source, we use a superluminescent light emitting diode (SLED, Exalos EXC250011-00) rather than a laser due to its lower coherence, which has been demonstrated to reduce speckle in holographic displays \cite{Deng2017CoherenceDisplays}. Although previous work showed state-of-the-art image quality by modeling the larger bandwidth of the SLED as a summation of coherent sources \cite{Peng2021Speckle-freeCalibration}, we found the computational cost to be prohibitively high for our application due to GPU memory constraints. We achieved sufficient image quality while assuming a fully coherent model, potentially due to the U-net which is capable of simulating the additional blur we expect from a partially coherent source.

\paragraph{Calibration Procedure}
We fit the learned parameters in our model (Eqs. \ref{eq:learned-model-01-LUT} - \ref{eq:learned-model-04-unet}) using a dataset captured on the experimental system. We pre-calculate 882 SLM patterns from a personally collected dataset of images using a naive ASM propagation model. Each SLM pattern is captured in $\SI{5}{\milli\metre}$ increments from $z = \SI{90}{\milli\metre}$ to $\SI{120}{\milli\metre}$, resulting in a total of 6174 paired entries. The raw camera data is debayered and an affine transform is applied to align the image with the SLM (see Supplement for details). Model fitting is implemented in Pytorch using an L1 loss function between the model output and camera capture. To account for the camera color balance, we additionally learn a $3 \times 3$ color calibration matrix from the RGB simulated intensities to the captured color image. We train until convergence, which is typically reached between 50 and 100 epochs (2-3 days on Nvidia A6000 GPU).

\paragraph{Hologram Generation}
After training, we can generate holograms by solving Eq. \ref{eq:loss_func_opponent} using the trained model for $I_{z,\lambda}$, implemented with Pytorch's native auto-differentiation. The SLM pattern, $s$, is constrained to the range where the LUT is valid (for example, 0 - 255); the values outside that range are wrapped after every optimization step.  On the Nvidia A6000 GPU, it takes about two minutes to optimize a 2D hologram.  Computation time for the optimization of a 3D hologram scales proportional to the number of depth planes.

\section{Experimental Results}

\paragraph{2-Dimensional Holograms}
We validate our simulation results by capturing holograms in experiment. The SLM patterns were optimized for a propagation distance of \SI{120}{\milli\metre} using our perceptual loss function laid out in Section \ref{sec:perceptional-loss}.  A white border was added to each target image to improve the color fidelity by encouraging a proper white balance. After each hologram is captured, debayering is performed and a homography is applied to map from camera space to SLM space. The homography also downsamples the captured holograms to the same resolution as the SLM.  The captured results are shown in Figure ~\ref{fig:Stills}.  The images match simulation well, validating our simultaneous color algorithm, although experimental results are noisier with lower color fidelity due to model-mismatch. 

\paragraph{3-Dimensional Holograms}
As mentioned earlier a major appeal of holography is the ability to solve the vergence-accommodation conflict without the need for eye tracking.  Consequently, we also validate our method for 3D scenes.  A 4-plane focal stack was rendered with 0.5 pixels blur radius per millimeter depth.  Holograms were captured at distance from \SI{90}{\milli\metre} to \SI{120}{\milli\metre} in \SI{10}{\milli\metre} increments.  The results are displayed in Fig. \ref{fig:Stack}, and quality is similar to the 2D case. These experimental results demonstrate the ability to form 3D color holograms with natural defocus blur from a single SLM frame.

\section{Discussion}
\label{discussion}
We have presented an algorithm that achieves state-of-the-art performance even though it is much simpler in terms of theoretical interpretation, architecture and training strategy than other methods. In this section, we detail aspects of RGI and its advantages with respect to other methods.

\textbf{Augmentation-free.} Most previous works train an encoder with the invariance via data augmentation principle \cite{thakoor2021bootstrapped}, \cite{Zhu:2020vf}, \cite{zhang2021canonical}, \cite{icml2020_1971}, \cite{NEURIPS2021_ff1e68e7}, \cite{peng2020graph}. It has been stated that transformations that drop information may modify the semantics of the graph so the invariance assumption may be incorrect and not hold for all graph domains. Other works propose alternative views leveraging the rich structure of graph data \cite{icml2020_1971}, \cite{NEURIPS2021_ff1e68e7}, \cite{peng2020graph}, \cite{lee2021augmentation}, \cite{https://doi.org/10.48550/arxiv.2204.04874}, but they also require carefully designed strategies to obtain the views. Alternatively, more recent methods also propose augmentation-free solutions \cite{anonymous2023localized}, \cite{https://doi.org/10.48550/arxiv.2204.04874} by incorporating training tricks. RGI falls into this category and require no transformations, however, it is much more intuitive and only involves graph convolution-like operations.

\textbf{Non-contrastive.} Graph contrastive learning is the most popular approch to avoid the collapse of the representations by relying on negative pair sampling \cite{velickovic2018deep}, \cite{sun2019infograph}, \cite{Zhu:2020vf}, \cite{NEURIPS2021_ff1e68e7}, \cite{https://doi.org/10.48550/arxiv.2204.04874}. BGRL \cite{thakoor2021bootstrapped} and CCA-SSG \cite{zhang2021canonical}, instead, do not need negative samples. The former avoids collapse with an asymmetric architecture and the later, regularizing the covariance matrix of the representations. In this work, we also adopt a regularized solution since it is more interpretable and naturally avoids the collapse whereas it is still an open problem to theoretically demonstrate that bootstrapped methods avoid trivial solutions.

\textbf{Single branch architecture.} Current state of the art algorithms usually rely on a joint embedding architecture that require multiple forward passes at each training step. For example, BGRL \cite{thakoor2021bootstrapped} forwards the graph through the encoder four times at each iteration. On the contrary, RGI is much more efficient and only performs one single forward pass while achieving similar performance to BGLR and other methods.

\textbf{Dropout regularization.}
Optionally, RGI can add noise to the input graph for example with dropout and edge sampling. However, this noise acts as regularization rather than data augmentation. First, we do not train the encoder to be invariant to these transformations since we only compute one forward pass. Secondly, no invariance is assumed, the loss function does not include any term to make the encoder invariant to the transformations, instead, we compute the iterations with noisy versions of the graph. Finally, node attribute masking transformation usually drops the same feature for all nodes of the graph whereas we are employing standard dropout on the input features.


\clearpage 

    \begin{figure*}
    \centering
    \includegraphics[clip, trim = 0in 7.1in 0in 0in, width=0.95\textwidth]{figures/Stills.pdf}
    \caption{\textbf{Experimentally captured 2D holograms.}
    For each target image (A), we show (B) the experimental capture with sequential pseudo-color, (C) our experimental capture with full simultaneous color, and (D) the simulated model output for simultaneous color. Recall that our simultaneous color results (C) use 3$\times$ fewer degrees of freedom than the sequential capture (B). Although some color fidelity is lost in experiment (C), the simulated model output (D) shows good color quality, demonstrating that accurate color is possible with our method and improvements to the calibration.
    % and the which was captured in pseudo-color using the red light source only. 
    % %
    % This figure shows experimentally captured holograms at a depth of \SI{120}{\milli\metre}. Row one: target images. Row two: experimentally captured sequential color holograms. Row three: experimentally captured simultaneous color holograms. Row four: simulation output of the optimized simultaneous color SLM pattern. While most captured simultaneous color holograms have good color fidelity, our method is least effective on highly saturated images with low texture, such as the cat paws in column 4, representing a limitation of our method (see Sec. ~\ref{sec:limits}).
    }
    \label{fig:Stills}
    \end{figure*}

    \begin{figure*}
    \centering
    \includegraphics[clip, trim = 0in 6.42in 0in 0in, width=0.95\textwidth]{figures/FocalStack.pdf}
    \caption{\textbf{Experimentally captured focal stack.} This figure displays a focal stack, with the target shown in (A), captured from $\SI{90}{\milli\metre}$ to $\SI{120}{\milli\metre}$ in $\SI{10}{\milli\metre}$ increments. We compare (B) the sequential pseudo-color experimental capture with (C) the experimental capture of the simultaneous full color hologram and (D) the simulated model output for simultaneous color. Although model mismatch creates some deviations between the simultaneous capture (C) and the target (A), the simulated model (D) is representative of the color fidelity we expect from our method with improvements to the system calibration.
    }
    \label{fig:Stack}
    \end{figure*}

\begin{figure*}
    \centering
    \includegraphics[clip, trim = 0in 6.125in 0in 0in, width=\textwidth]{figures/ModelAblation.pdf}
    \caption{\textbf{Comparison of different propagation methods for suppressing higher diffraction orders.} The first column shows the results obtained using the traditional angular spectrum method (ASM) which doesn't model higher diffraction orders. The second column shows the results obtained using HOASM which reduces the visibility of higher orders but fails to completely suppress them. The third column shows the results obtained using our proposed learned propagation method that includes a U-net, which largely suppresses the higher diffraction orders and results in a hologram with the fewest artifacts, suggesting the learned propagation model best matches the physical propagation.
    }
    \label{fig:Ablation}
    \end{figure*}

\begin{figure*}
    \centering
    \includegraphics[clip, trim = 0in 7.75in 0in 0in, width=.9\textwidth]{figures/Setup.pdf}
    \caption{\textbf{Experimental setup} A top view of our system with labeled components and an approximate beam path drawn in green is depicted in (A).  A side-view of the system is provided by (B). \revised{Note that the hologram is formed directly on the bare camera sensor with no lens or eyepiece between. This configuration allows us to validate our method, but for a human-viewable system, an eyepiece must be added between the hologram plane and the user's eye.}
    }
    \label{fig:Setup}
    \end{figure*}


\clearpage
\clearpage
\bibliographystyle{ACM-Reference-Format}
\bibliography{references_local}
\clearpage

\newcommand{\beginsupplement}{%
    \setcounter{table}{0}
    \renewcommand{\thetable}{S\arabic{table}}%
    \setcounter{figure}{0}
    \renewcommand{\thefigure}{S\arabic{figure}}%
    \setcounter{section}{0}
    \renewcommand{\thesection}{S\arabic{section}}%
 }

\beginsupplement{}
\onecolumn
\textbf{\huge Supplementary Material -- Simultaneous Color Holography}
\section{Additional Implementation Details}

\paragraph{Spatial Light Modulator}
For all simulations, a spatial light modulator (SLM) with $1920 \times 1080$ pixels and a pixel size of $\SI{8}{\micro\metre} \times \SI{8}{\micro\metre}$ is used. The phase ranges of the red, green, and blue channels are $2.4\pi$, $5.9\pi$, and $7.4\pi$, respectively, unless otherwise noted. These values were experimentally calibrated for the Holoeye Pluto-2.1-Vis-016 SLM. The propagation distance of all simulated holograms is $\SI{100}{\milli\metre}$ unless otherwise noted.

\paragraph{Modified High Order Angular Spectrum Method (HOASM)}
We implement a modified version of the High Order Angular Spectrum Method (HOASM) as described by \citet{Gopakumar2021UnfilteredDisplays}. Instead of propagating the zero- and first-order together, we propagate them separately. The zero-order is propagated by performing the traditional angular spectrum method (ASM). To propagate the first-order, we pattern the zero-padded Fourier transform of the complex field to be propagated into a $3 \times 3$ grid. The center Fourier transform of the grid is then zeroed out. The Fourier representation of the first-order is then weighted with a sinc function and propagated to the sensor plane using ASM. The field is then down-sampled and cropped. The complex fields of the zero- and first-orders are then split into real and imaginary parts and stacked before being fed into a U-Net. The U-Net consists of 4 downsampling layers, the number of channels increases from 4 to 32 during the first downsampling layer and doubles in each of the next 3 downsampling layers until there are 256 channels. Four upsampling layers are then applied, producing a single-channel output representing the intensity of the propagated wavefront.

\paragraph{ Camera Space to SLM Space Homography}
To perform either offline or active camera-in-the-loop optimization, the captured wavefront and SLM must be in the same space. This requires a transform and downsampling of the captured image to place it in the same coordinate system as the SLM pattern used to generate it. We opt to use an affine transform to perform this mapping. The affine transform is calculated as follows: First, an SLM pattern is calculated that produces a grid of dots. The dots are then detected on the sensor, and their centers are estimated in camera space coordinates. The centers of the dots are known in SLM space since the target image containing the dots is in SLM space for optimization. Finally, Python's OpenCV package is used to produce the affine transform matrix that maps the captured dots to the SLM coordinate space. A unique homography is calculated for each depth location and color channel.

\paragraph{Source Power Optimization}
Correctly setting the power of each color channel of the SLED for a given hologram is an important step to achieving good color fidelity. To achieve this, we use an active camera-in-the-loop based approach to optimize the power of the color channels. First, the power of the source is set to an arbitrary value less than 100\% across all three color channels. A baseline reference image is captured, debayered, and mapped to the SLM space. Three learnable weighting parameters, one for each color channel, are initialized to unity and applied to the captured reference image. These weighting parameters serve as a proxy to optimizing the source power. An iterative process is then undertaken, where an image is captured on the camera, debayered, and mapped to the SLM space. The loss between this image and the target image is calculated, and then backpropagated using the computational graph of the weighting parameters applied to the reference image. The initial source power is then multiplied by the updated weighting parameters, and a new image is captured, restarting the iterative loop. This is done until the color weighting parameters have converged, usually taking between 15-30 iterations. If the process fails to converge or the initial source power multiplied by the weighting function becomes greater than 100\%, the exposure time is increased, and the source power optimization is restarted.

Although we use camera feedback in this process, we note that the information needed for source power optimization is contained in the color balance of the image itself. We believe this step could be replaced with a precomputed source power that's dependent on the image content.

\section{Active camera-in-the-Loop (CiTL)}
Active CiTL \cite{Peng2020NeuralTraining} is a special case of camera-calibrated models in which an image is displayed on the SLM, and camera captures are used to improve that particular image using the difference between the experimental capture and target image. While active CiTL is incompatible for real time displays, it does provide a useful proof of achievable hologram quality since. Consequently, we implemented active CiTL for our system as follows.

First, an SLM pattern is optimized using our learned simulation model and the computational graph is retained. This SLM pattern is then displayed and the resulting hologram is captured. A homography is applied to the captured hologram for each color channel to map it from camera space to simulation space. Our perceptual loss function is applied to the remapped captured hologram and target image. Backpropagation is performed using a computational graph saved during the forward pass, but the experimentally captured hologram is used in the loss function (instead of the simulated model output). This is the first time to our knowledge that active CiTL has been combined with a deep component to the forward model. Figure \ref{fig:CiTL} shows reduced noise and improved color fidelity for holograms generated with active CiTL. Since active CiTL uses the difference between the experimental capture and the target, the alignment between the two must be precise. We find that improved alignment using a piecewise affine homography, rather than a global affine homography, dramatically improves color fidelity. A comparison of this case is shown in Figure \ref{fig:PWA}.


\begin{figure*}
    \centering
    \includegraphics[clip, trim = 0in 3in 0in 0in, width=0.85\textwidth]{figures/CiTL.pdf}
    \caption{\textbf{Active camera-in-the-Loop (CiTL) reduces noise and improves color fidelity.} The first column of this image depicts experimentally captured color holograms.  The second columns shows images that were iteratively improved with a camera in the system using the active CiTL algorithm of ~\citet{Peng2020NeuralTraining}.
    }  
    \label{fig:CiTL}
\end{figure*}

\begin{figure*}
    \centering
    \includegraphics[clip, trim = 0in 9.275in 0in 0in, width=\textwidth]{figures/PWA.pdf}
    \caption{\textbf{Piecewise affine homography improves color fidelity for active CiTL.} The first column shows the target image. The second column shows the experimentally captured hologram optimized using active CiTL with a global affine homography. The third column depicts active CiTL with a piecewise affine homography, which reduces color artifacts and noise due to better alignment during optimization. Cat source image by Chris Erwin (CC-BY-2.0).
    }  
    \label{fig:PWA}
\end{figure*}

\clearpage
\section{Additional Experimental Results and Failure Cases}

Figure \ref{fig:AddResults} depicts additional captured results, which are intended to showcase a wider variety of scenes and include failure cases of our method. Our method has the most difficulty when the target has large, flat areas (i.e. textureless) of saturated color. Textureless targets lack high frequency information that can be leveraged by our loss function, leading to substantial artifacts such as color non-uniformity and ringing . These artifacts are particularly apparent in the image of colored bars in Fig. \ref{fig:AddResults}. Highly saturated images or ``unnatural'' images (like the colored bars) often fail due to disparate color channels, resulting in a single SLM pattern having to produce three holograms at the same plane with substantially different structures. In contrast, natural images typically have similarly structured color channels.

\begin{figure*}[!b]
    \centering
    \includegraphics[clip, trim = 0in 2.875in 0in 0in, width=.97\textwidth]{figures/AddResults.pdf}
    \caption{ \textbf{Additional simultaneous color holograms captured in experiment.}  The first column depicts holograms captured in experiment.  The second column shows the simulation output.  The third column depicts the target image. Although our system performs well on most natural scenes, unnatural images such as the bars in the center row are more challenging for our algorithm.
    }  
    \label{fig:AddResults}
\end{figure*}

\clearpage
\section{Additional Perceptual Loss Function Details}
The filter sizes for our perceptual loss function were chosen heuristically, such that no visible change was noticeable between the target image and the target image with the perceptual filter applied. These filter sizes were kept constant regardless of the scene being optimized.  These filters can be viewed in Fig. \ref{fig:Filters} To test the effectiveness of our perceptual loss function, we applied it to a personally captured dataset of 294 images. For each target image, an SLM pattern was optimized using both the traditional RGB loss function and our perceptual loss function. The resulting hologram was then captured, and the perceptual filter was applied. The PSNR, SSIM, and NMSE were calculated for the filtered simulated holograms and the perceptually filtered target image. The average metrics over the entire dataset are provided in Table \ref{loss_metrics}.


\begin{table}[ht]
    \centering
        \begin{tabular}[t]{lccc}
        & PSNR & SSIM & NMSE \\
        \hline
        RGB Loss Function & 20.11 & 0.603 & 0.010  \\
        \hline
        Perceptual Loss Function & 26.58 & 0.869 & 0.003  \\
        \hline
        \end{tabular}
        \vspace{3mm}
        \caption{ A comparison of the average PSNR, SSIM, and NMSE for holograms optimized with the traditional RGB loss function and perceptual loss function.  The metrics were calculated between the perceptually filtered simulated holograms and the perceptually filtered target.  The data set used was a personally captured set of 294 images of natural scenes.}
        \label{loss_metrics}
\end{table}


\begin{figure}
    \centering
    \includegraphics[clip, trim = 0in 9.25in 0in 0in, width=\textwidth]{figures/Filters.pdf}
    \caption{\textbf{Perceptual loss function filters in Fourier opponent color space.} The white areas of the filters pictured represents the pass band of the filter.  The luminance channel has a filter width of 75\% of Fourier space.  Both chrominance channels (Red-Green, Blue-Yellow) have filter widths of 45\% of Fourier space.
    }  
    \label{fig:Filters}
\end{figure}

\clearpage
\section{The Effect of Bit Depth on Hologram Quality}
The effect of quantization on hologram quality is an important consideration when choosing an extended phase SLM. We define the effective bit depth as the number of bits contained in a $2\pi$ interval of the extended range. For example, the effective bit depth of an 8-bit SLM with a phase range of $8\pi$ is 6 bits as each $2\pi$ interval contains 64 discrete samples i.e. 6 bits. To determine the minimum bit depth required for adequate image quality, we simulated holograms using an SLM with a $2\pi$ phase range and bit depths from 2 bits to 8-bits. Simulations are done by optimizing the hologram with gradient descent, then quantizing to the target bit depth. A significant drop off in both PSNR and SSIM was observed between 5 and 6 bits, as depicted in Fig. \ref{fig:Bits}. This suggests that the minimum effective bit depth required for an extended phase SLM is 6 bits. Since most commercially available SLMs are 8 bits, this suggests that the maximum phase range in any channel should be $8\pi$, which aligns well with the SLM used in our experiments (maximum phase range of $7.4\pi$ in the blue channel).

\begin{figure*}[!b]
    \centering
    \includegraphics[clip, trim = 0in 2.875in 0in 0in, width=0.95\textwidth]{figures/BitDepth.pdf}
    \caption{ \textbf{An analysis of SLM bit depth on hologram quality in simulation}  We simulate holograms using SLMs of 2 to 8 bits.  The target image is pictured in the top left of the figure.  One should note the rapidly increasing drop off in both PSNR and SSIM between 5 and 6 bits.
    }  
    \label{fig:Bits}
\end{figure*}

\clearpage
\section{Bit and Depth Division Implementation Details and Analysis}
In this section we provide our implementation details of bit and depth division holography.  Additionally, we analyze the methods for SLMs of various phase ranges.  We implement bit division largely as laid out by \citet{Jesacher2014ColourRange}.  First we calculated the three color channels SLM patterns using a modified Gerchberg-Saxton approach assuming a $2\pi$ phase range in each color channel.  Instead of using the Fourier transform for propagation as in \citet{Jesacher2014ColourRange}, we use ASM match our other results.  This is run until convergence, and 3 unique SLM patterns are produced.  These SLM patterns are then combined via an optimization problem as described by \cite{Jesacher2014ColourRange}. We then used the combined SLM pattern to simulate a color hologram at the sensor plane.

We choose to implement the depth division method using gradient descent-based optimzation rather than a modified Gerchberg-Saxton (GS) algorithm for multiplane holograms originally proposed by \citet{Makowski2008ColorfulHologram, Makowski2010ColorHolograms} for depth division holography. Since we use gradient descent in our approach, we determined this was a fairer comparison. In our implementation the SLM pattern is first converted to a complex field.  The complex field is then propagated to $z =\SI{68}{\milli\metre}, \SI{80}{\milli\metre}, \SI{100}{\milli\metre}$ using the ASM kernel for the red color channel. These correspond to the replica planes. The intensity of the of the fields are then calculated at each target plane and compared to the blue, green, and red channels, respectively, using an L2 loss function.  Backpropagation is then used to calculate the gradients of the loss function with respect the SLM voltage values and then update these voltages.  

We implement both the bit and depth division holography methods for 3 simulated SLMs.  The first SLM has a uniform $2\pi$ phase range in each color channel.  This phase range is optimal for depth division, but performs the worst of the simulated SLMs for bit division, demonstrating how bit division relies on extended SLM phase range. The next simulated SLM is an arbitrary standard SLM i.e. not extended phase.  We model this SLM to have $2\pi$ phase range in red, $2.7\pi$ in green, and $3.4\pi$ in blue.  The simulated holograms increase in quality from the $2\pi$ SLM for the bit division method, but decrease in quality for depth division.  Finally we simulate the Holoeye Pluto SLM used in our experimental setup.  This SLM has a $2.4\pi$ phase range in red, $5.9\pi$ phase range in green, and $7.4\pi$ phase range in blue.  The results for depth division continue to degrade with this SLM, since the depth division algorithm does not take into account the wavelength-dependent response of the SLM. The results improve for bit division with the additional extended phase.  This suggests that that phase diversity across channels provides the best performance for bit division holography, while phase uniformity across channels provides the best performance for depth division holography.  The results of the outlined experiment can be found in Figs. \ref{fig:ExtenedBits} and \ref{fig:ExtenedDepth}.

\begin{figure*}
    \centering
    \includegraphics[clip, trim = 0in 4.575in 0in 0in, width=\textwidth]{figures/ExtendedBits.pdf}
    \caption{ \textbf{SLM phase range affects hologram quality for bit division holography.} Bit division takes advantage of the extended phase range of the SLM, so does not perform well with an SLM with only $2\pi$ phase range per channel (left column). With a `` standard'' SLM with realistic wavelength dependence to the phase, bit division performs better. It works best with the extended phase range of the simulated Holoeye Pluto that we use for our experiments.}  
    \label{fig:ExtenedBits}
\end{figure*}

\begin{figure*}
    \centering
    \includegraphics[clip, trim = 0in 4.575in 0in 0in, width=\textwidth]{figures/ExtendedDepth.pdf}
    \caption{ \textbf{SLM phase range affects hologram quality for depth division holography.} The depth division approach assumes no wavelength dependence of the SLM, which is simulated in the first column. With standard SLM with $2\pi$ phase in red and realistic wavelength dependence (second column) the results are slightly degraded due to inaccurate modeling of the SLM response. Finally, with the extended phase range of the simulated Holoeye Pluto SLM, the results show significant color artifacts and noise.}  
    \label{fig:ExtenedDepth}
\end{figure*}




%%%%%%%%%%%%%%%%%%%%%%%%%%%%%%%%%%%%%%%%%%%



\end{document}



