\newcommand{\beginsupplement}{%
    \setcounter{table}{0}
    \renewcommand{\thetable}{S\arabic{table}}%
    \setcounter{figure}{0}
    \renewcommand{\thefigure}{S\arabic{figure}}%
    \setcounter{section}{0}
    \renewcommand{\thesection}{S\arabic{section}}%
 }

\beginsupplement{}
\onecolumn
\textbf{\huge Supplementary Material -- Simultaneous Color Holography}
\section{Additional Implementation Details}

\paragraph{Spatial Light Modulator}
For all simulations, a spatial light modulator (SLM) with $1920 \times 1080$ pixels and a pixel size of $\SI{8}{\micro\metre} \times \SI{8}{\micro\metre}$ is used. The phase ranges of the red, green, and blue channels are $2.4\pi$, $5.9\pi$, and $7.4\pi$, respectively, unless otherwise noted. These values were experimentally calibrated for the Holoeye Pluto-2.1-Vis-016 SLM. The propagation distance of all simulated holograms is $\SI{100}{\milli\metre}$ unless otherwise noted.

\paragraph{Modified High Order Angular Spectrum Method (HOASM)}
We implement a modified version of the High Order Angular Spectrum Method (HOASM) as described by \citet{Gopakumar2021UnfilteredDisplays}. Instead of propagating the zero- and first-order together, we propagate them separately. The zero-order is propagated by performing the traditional angular spectrum method (ASM). To propagate the first-order, we pattern the zero-padded Fourier transform of the complex field to be propagated into a $3 \times 3$ grid. The center Fourier transform of the grid is then zeroed out. The Fourier representation of the first-order is then weighted with a sinc function and propagated to the sensor plane using ASM. The field is then down-sampled and cropped. The complex fields of the zero- and first-orders are then split into real and imaginary parts and stacked before being fed into a U-Net. The U-Net consists of 4 downsampling layers, the number of channels increases from 4 to 32 during the first downsampling layer and doubles in each of the next 3 downsampling layers until there are 256 channels. Four upsampling layers are then applied, producing a single-channel output representing the intensity of the propagated wavefront.

\paragraph{ Camera Space to SLM Space Homography}
To perform either offline or active camera-in-the-loop optimization, the captured wavefront and SLM must be in the same space. This requires a transform and downsampling of the captured image to place it in the same coordinate system as the SLM pattern used to generate it. We opt to use an affine transform to perform this mapping. The affine transform is calculated as follows: First, an SLM pattern is calculated that produces a grid of dots. The dots are then detected on the sensor, and their centers are estimated in camera space coordinates. The centers of the dots are known in SLM space since the target image containing the dots is in SLM space for optimization. Finally, Python's OpenCV package is used to produce the affine transform matrix that maps the captured dots to the SLM coordinate space. A unique homography is calculated for each depth location and color channel.

\paragraph{Source Power Optimization}
Correctly setting the power of each color channel of the SLED for a given hologram is an important step to achieving good color fidelity. To achieve this, we use an active camera-in-the-loop based approach to optimize the power of the color channels. First, the power of the source is set to an arbitrary value less than 100\% across all three color channels. A baseline reference image is captured, debayered, and mapped to the SLM space. Three learnable weighting parameters, one for each color channel, are initialized to unity and applied to the captured reference image. These weighting parameters serve as a proxy to optimizing the source power. An iterative process is then undertaken, where an image is captured on the camera, debayered, and mapped to the SLM space. The loss between this image and the target image is calculated, and then backpropagated using the computational graph of the weighting parameters applied to the reference image. The initial source power is then multiplied by the updated weighting parameters, and a new image is captured, restarting the iterative loop. This is done until the color weighting parameters have converged, usually taking between 15-30 iterations. If the process fails to converge or the initial source power multiplied by the weighting function becomes greater than 100\%, the exposure time is increased, and the source power optimization is restarted.

Although we use camera feedback in this process, we note that the information needed for source power optimization is contained in the color balance of the image itself. We believe this step could be replaced with a precomputed source power that's dependent on the image content.

\section{Active camera-in-the-Loop (CiTL)}
Active CiTL \cite{Peng2020NeuralTraining} is a special case of camera-calibrated models in which an image is displayed on the SLM, and camera captures are used to improve that particular image using the difference between the experimental capture and target image. While active CiTL is incompatible for real time displays, it does provide a useful proof of achievable hologram quality since. Consequently, we implemented active CiTL for our system as follows.

First, an SLM pattern is optimized using our learned simulation model and the computational graph is retained. This SLM pattern is then displayed and the resulting hologram is captured. A homography is applied to the captured hologram for each color channel to map it from camera space to simulation space. Our perceptual loss function is applied to the remapped captured hologram and target image. Backpropagation is performed using a computational graph saved during the forward pass, but the experimentally captured hologram is used in the loss function (instead of the simulated model output). This is the first time to our knowledge that active CiTL has been combined with a deep component to the forward model. Figure \ref{fig:CiTL} shows reduced noise and improved color fidelity for holograms generated with active CiTL. Since active CiTL uses the difference between the experimental capture and the target, the alignment between the two must be precise. We find that improved alignment using a piecewise affine homography, rather than a global affine homography, dramatically improves color fidelity. A comparison of this case is shown in Figure \ref{fig:PWA}.


\begin{figure*}
    \centering
    \includegraphics[clip, trim = 0in 3in 0in 0in, width=0.85\textwidth]{figures/CiTL.pdf}
    \caption{\textbf{Active camera-in-the-Loop (CiTL) reduces noise and improves color fidelity.} The first column of this image depicts experimentally captured color holograms.  The second columns shows images that were iteratively improved with a camera in the system using the active CiTL algorithm of ~\citet{Peng2020NeuralTraining}.
    }  
    \label{fig:CiTL}
\end{figure*}

\begin{figure*}
    \centering
    \includegraphics[clip, trim = 0in 9.275in 0in 0in, width=\textwidth]{figures/PWA.pdf}
    \caption{\textbf{Piecewise affine homography improves color fidelity for active CiTL.} The first column shows the target image. The second column shows the experimentally captured hologram optimized using active CiTL with a global affine homography. The third column depicts active CiTL with a piecewise affine homography, which reduces color artifacts and noise due to better alignment during optimization. Cat source image by Chris Erwin (CC-BY-2.0).
    }  
    \label{fig:PWA}
\end{figure*}

\clearpage
\section{Additional Experimental Results and Failure Cases}

Figure \ref{fig:AddResults} depicts additional captured results, which are intended to showcase a wider variety of scenes and include failure cases of our method. Our method has the most difficulty when the target has large, flat areas (i.e. textureless) of saturated color. Textureless targets lack high frequency information that can be leveraged by our loss function, leading to substantial artifacts such as color non-uniformity and ringing . These artifacts are particularly apparent in the image of colored bars in Fig. \ref{fig:AddResults}. Highly saturated images or ``unnatural'' images (like the colored bars) often fail due to disparate color channels, resulting in a single SLM pattern having to produce three holograms at the same plane with substantially different structures. In contrast, natural images typically have similarly structured color channels.

\begin{figure*}[!b]
    \centering
    \includegraphics[clip, trim = 0in 2.875in 0in 0in, width=.97\textwidth]{figures/AddResults.pdf}
    \caption{ \textbf{Additional simultaneous color holograms captured in experiment.}  The first column depicts holograms captured in experiment.  The second column shows the simulation output.  The third column depicts the target image. Although our system performs well on most natural scenes, unnatural images such as the bars in the center row are more challenging for our algorithm.
    }  
    \label{fig:AddResults}
\end{figure*}

\clearpage
\section{Additional Perceptual Loss Function Details}
The filter sizes for our perceptual loss function were chosen heuristically, such that no visible change was noticeable between the target image and the target image with the perceptual filter applied. These filter sizes were kept constant regardless of the scene being optimized.  These filters can be viewed in Fig. \ref{fig:Filters} To test the effectiveness of our perceptual loss function, we applied it to a personally captured dataset of 294 images. For each target image, an SLM pattern was optimized using both the traditional RGB loss function and our perceptual loss function. The resulting hologram was then captured, and the perceptual filter was applied. The PSNR, SSIM, and NMSE were calculated for the filtered simulated holograms and the perceptually filtered target image. The average metrics over the entire dataset are provided in Table \ref{loss_metrics}.


\begin{table}[ht]
    \centering
        \begin{tabular}[t]{lccc}
        & PSNR & SSIM & NMSE \\
        \hline
        RGB Loss Function & 20.11 & 0.603 & 0.010  \\
        \hline
        Perceptual Loss Function & 26.58 & 0.869 & 0.003  \\
        \hline
        \end{tabular}
        \vspace{3mm}
        \caption{ A comparison of the average PSNR, SSIM, and NMSE for holograms optimized with the traditional RGB loss function and perceptual loss function.  The metrics were calculated between the perceptually filtered simulated holograms and the perceptually filtered target.  The data set used was a personally captured set of 294 images of natural scenes.}
        \label{loss_metrics}
\end{table}


\begin{figure}
    \centering
    \includegraphics[clip, trim = 0in 9.25in 0in 0in, width=\textwidth]{figures/Filters.pdf}
    \caption{\textbf{Perceptual loss function filters in Fourier opponent color space.} The white areas of the filters pictured represents the pass band of the filter.  The luminance channel has a filter width of 75\% of Fourier space.  Both chrominance channels (Red-Green, Blue-Yellow) have filter widths of 45\% of Fourier space.
    }  
    \label{fig:Filters}
\end{figure}

\clearpage
\section{The Effect of Bit Depth on Hologram Quality}
The effect of quantization on hologram quality is an important consideration when choosing an extended phase SLM. We define the effective bit depth as the number of bits contained in a $2\pi$ interval of the extended range. For example, the effective bit depth of an 8-bit SLM with a phase range of $8\pi$ is 6 bits as each $2\pi$ interval contains 64 discrete samples i.e. 6 bits. To determine the minimum bit depth required for adequate image quality, we simulated holograms using an SLM with a $2\pi$ phase range and bit depths from 2 bits to 8-bits. Simulations are done by optimizing the hologram with gradient descent, then quantizing to the target bit depth. A significant drop off in both PSNR and SSIM was observed between 5 and 6 bits, as depicted in Fig. \ref{fig:Bits}. This suggests that the minimum effective bit depth required for an extended phase SLM is 6 bits. Since most commercially available SLMs are 8 bits, this suggests that the maximum phase range in any channel should be $8\pi$, which aligns well with the SLM used in our experiments (maximum phase range of $7.4\pi$ in the blue channel).

\begin{figure*}[!b]
    \centering
    \includegraphics[clip, trim = 0in 2.875in 0in 0in, width=0.95\textwidth]{figures/BitDepth.pdf}
    \caption{ \textbf{An analysis of SLM bit depth on hologram quality in simulation}  We simulate holograms using SLMs of 2 to 8 bits.  The target image is pictured in the top left of the figure.  One should note the rapidly increasing drop off in both PSNR and SSIM between 5 and 6 bits.
    }  
    \label{fig:Bits}
\end{figure*}

\clearpage
\section{Bit and Depth Division Implementation Details and Analysis}
In this section we provide our implementation details of bit and depth division holography.  Additionally, we analyze the methods for SLMs of various phase ranges.  We implement bit division largely as laid out by \citet{Jesacher2014ColourRange}.  First we calculated the three color channels SLM patterns using a modified Gerchberg-Saxton approach assuming a $2\pi$ phase range in each color channel.  Instead of using the Fourier transform for propagation as in \citet{Jesacher2014ColourRange}, we use ASM match our other results.  This is run until convergence, and 3 unique SLM patterns are produced.  These SLM patterns are then combined via an optimization problem as described by \cite{Jesacher2014ColourRange}. We then used the combined SLM pattern to simulate a color hologram at the sensor plane.

We choose to implement the depth division method using gradient descent-based optimzation rather than a modified Gerchberg-Saxton (GS) algorithm for multiplane holograms originally proposed by \citet{Makowski2008ColorfulHologram, Makowski2010ColorHolograms} for depth division holography. Since we use gradient descent in our approach, we determined this was a fairer comparison. In our implementation the SLM pattern is first converted to a complex field.  The complex field is then propagated to $z =\SI{68}{\milli\metre}, \SI{80}{\milli\metre}, \SI{100}{\milli\metre}$ using the ASM kernel for the red color channel. These correspond to the replica planes. The intensity of the of the fields are then calculated at each target plane and compared to the blue, green, and red channels, respectively, using an L2 loss function.  Backpropagation is then used to calculate the gradients of the loss function with respect the SLM voltage values and then update these voltages.  

We implement both the bit and depth division holography methods for 3 simulated SLMs.  The first SLM has a uniform $2\pi$ phase range in each color channel.  This phase range is optimal for depth division, but performs the worst of the simulated SLMs for bit division, demonstrating how bit division relies on extended SLM phase range. The next simulated SLM is an arbitrary standard SLM i.e. not extended phase.  We model this SLM to have $2\pi$ phase range in red, $2.7\pi$ in green, and $3.4\pi$ in blue.  The simulated holograms increase in quality from the $2\pi$ SLM for the bit division method, but decrease in quality for depth division.  Finally we simulate the Holoeye Pluto SLM used in our experimental setup.  This SLM has a $2.4\pi$ phase range in red, $5.9\pi$ phase range in green, and $7.4\pi$ phase range in blue.  The results for depth division continue to degrade with this SLM, since the depth division algorithm does not take into account the wavelength-dependent response of the SLM. The results improve for bit division with the additional extended phase.  This suggests that that phase diversity across channels provides the best performance for bit division holography, while phase uniformity across channels provides the best performance for depth division holography.  The results of the outlined experiment can be found in Figs. \ref{fig:ExtenedBits} and \ref{fig:ExtenedDepth}.

\begin{figure*}
    \centering
    \includegraphics[clip, trim = 0in 4.575in 0in 0in, width=\textwidth]{figures/ExtendedBits.pdf}
    \caption{ \textbf{SLM phase range affects hologram quality for bit division holography.} Bit division takes advantage of the extended phase range of the SLM, so does not perform well with an SLM with only $2\pi$ phase range per channel (left column). With a `` standard'' SLM with realistic wavelength dependence to the phase, bit division performs better. It works best with the extended phase range of the simulated Holoeye Pluto that we use for our experiments.}  
    \label{fig:ExtenedBits}
\end{figure*}

\begin{figure*}
    \centering
    \includegraphics[clip, trim = 0in 4.575in 0in 0in, width=\textwidth]{figures/ExtendedDepth.pdf}
    \caption{ \textbf{SLM phase range affects hologram quality for depth division holography.} The depth division approach assumes no wavelength dependence of the SLM, which is simulated in the first column. With standard SLM with $2\pi$ phase in red and realistic wavelength dependence (second column) the results are slightly degraded due to inaccurate modeling of the SLM response. Finally, with the extended phase range of the simulated Holoeye Pluto SLM, the results show significant color artifacts and noise.}  
    \label{fig:ExtenedDepth}
\end{figure*}


