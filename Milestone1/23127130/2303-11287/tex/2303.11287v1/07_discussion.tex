\section{Limitations}
While our method improves hologram quality for simultaneous illumination and is compatible with VR and AR displays, it does have limitations.  First, our method is not equally effective for all images.  Natural images with high levels of texture work best, as they have similarly structured color channels and contain high frequency color information that is perceptually suppressible by our loss function. 
However, images with large flat areas may show noticeable artifacts such as the cat paws image in Figure \ref{fig:Stills}.  Unnatural images often have more saturated color, creating the more difficult task of finding an SLM pattern that can produce three largely unique holograms.

Additionally, our method takes on the order of minutes to calculate a single SLM pattern for a 2D image using a Nvidia A6000.  This is incompatible with real time displays.  Recent work has shown that neural nets can produce SLM patterns for holography in near real time while maintaining hologram quality ~\cite{shi2021towards, eybposh2020high, yang2022diffraction}. 
A neural SLM pattern generator for simultaneous color holography is likely feasible, but has been left for future work.
\label{sec:limits}

\section{Conclusion}

In summary, we have developed a comprehensive framework for generating high-quality color holograms using simultaneous RGB illumination and a simple, compact optical setup. Our framework features a camera-calibrated, differentiable forward model that reduces model mismatch and allows for the use of custom loss functions. By employing a perceptual loss function, we have successfully addressed the difficult challenge of simultaneous color holography, as validated by experimental testing in 2D and 3D. Our work brings us closer to creating holographic near-eye displays.

 
\clearpage