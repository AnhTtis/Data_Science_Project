\section{Introduction}
\label{Intro}
Holographic displays are a promising technology for augmented and virtual reality (AR/VR). Such displays use a spatial light modulator (SLM) to shape an incoming coherent wavefront so that it appears as though the wavefront came from a real, three-dimensional (3D) object. The resulting image can have natural defocus cues, providing a path to resolve the vergence-accommodation conflict of stereoscopic displays \cite{kim2022accommodative}.  Additionally, the fine-grain control offered by holography can also correct for optical aberrations, provide custom eyeglass prescription correction in software, and enable compact form-factors \cite{maimone2017holographic}, while improving light efficiency compared traditional LCD or OLED displays \cite{Yin2022AdvancedApplications}. Recent publications have demonstrated significant improvement in hologram image quality \cite{maimone2017holographic, Peng2020NeuralTraining, Choi2021NeuralDisplays} and computation time \cite{shi2021towards, eybposh2020high}, bringing holographic displays one step closer to practicality. However, color holography for AR/VR has remained an open problem.

Traditionally, red-green-blue (RGB) holograms are created through \textit{field sequential color}, where a separate hologram is computed for each of the three wavelengths and displayed in sequence and synchronized with the color of the illumination source. Due to persistence of vision, this appears as a single full color image if the update is sufficiently fast, enabling color holography for static displays. However, in a head-mounted AR/VR system displaying world-locked content, framerate requirements are higher to prevent noticeable judder \cite{van2016asynchronous}. Furthermore, field sequential color can lead to visible spatial separation of the colors, particularly when the user rotates their head while tracking a fixed object with their eyes \cite{riecke2006selected}. Although these negative effects can be mitigated with high framerate displays, the most common SLM technology for holography, liquid-crystal-on-silicon (LCoS), is quite slow due to the physical response time of the liquid crystal (LC) layer \cite{zhang2014fundamentals}. Although most commercial LCoS SLMs can be driven at 60 Hz, at that speed the SLM will have residual artifacts from the prior frames \cite{haist2015holography}. Micro-electro-mechanical system (MEMS) SLMs can be much faster (in the kilohertz range) but so far have larger pixels and limited bit depth~\cite{MEMS, Choi2022Time-multiplexedModulators}.

In this work, we aim to display RGB holograms using only a single SLM pattern, enabling a $3\times$ increase in framerate compared to sequential color and removing color fringing artifacts in the presence of head motion. Our compact setup does not use a physical filter in the Fourier plane or bulky optics to combine color channels. Instead, the full SLM is simultaneously illuminated by an on-axis RGB source, and we optimize the SLM pattern to form the three color image. We design a flexible framework for end-to-end optimization of the digital SLM input from the target RGB intensity, allowing us to optimize for SLMs with extended phase range, and we develop a color-specific perceptual loss function which further improves color fidelity. Our method is validated experimentally on 2D and 3D content.

Specifically, we make the following contributions:

\begin{itemize}
    \item We introduce a novel algorithm for generating simultaneous color holograms which takes advantage of the extended phase range of the SLM in an end-to-end manner and uses a new loss function based on human color perception.
    \item We analyze the ``depth replicas'' artifact in simultaneous color holography and demonstrate how these replicas can be mitigated with extended phase range.
    \item We demonstrate high quality experimental simultaneous color holograms in both 2D and 3D using a custom camera-calibrated model.
\end{itemize}