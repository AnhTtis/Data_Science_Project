\section{Simultaneous Color Holography}
\label{ProblemStatement}

A holographic image is created by a spatially coherent illumination source incident on an SLM. The SLM imparts a phase delay on the electric field; after light propagates some distance, the intensity of the electric field forms an image. Our goal in this work is to compute a single SLM pattern that simultaneously creates a three color RGB hologram. For instance, when the SLM is illuminated with a red source, the SLM forms a hologram of the red channel of an image; with a green source the same SLM pattern forms the green channel; and with the blue source it creates the blue channel. 

We propose a flexible optimization-based framework (Fig.~\ref{fig:ForwardModel}) for generating simultaneous color holograms. We start with a generic model for estimating the hologram from the digital SLM pattern, $s$, as a function of illumination wavelength, $\lambda$:
\begin{align}
    g_{\lambda} &= e^{i\phi_{\lambda}\left(s\right)} \label{eq:SLM_to_field} \\
    I_{z, \lambda} &= \left| f_{\text{prop}} \left( g_{\lambda}, z, \lambda \right) \right|^2. \label{eq:field_to_intensity}
\end{align}
Here, $\phi_{\lambda}$ is a wavelength-dependent function that converts the 8 bit digital SLM pattern to a phase delay, $g_{\lambda}$ is the electric field coming off the SLM, $f_{\text{prop}}$ represents propagation of the electric field, and $I_{z, \lambda}$ is the intensity a distance $z$ from the SLM.

To calculate the SLM pattern, $s$, we can solve the following optimization problem
\begin{align} \label{eq:loss_func_rgb}
    \argmin_{s} \sum_z \mathcal{L}\left(\hat{I}_{z, \lambda_r}, I_{z, \lambda_r}\right) + \mathcal{L}\left(\hat{I}_{z, \lambda_g}, I_{z, \lambda_g}\right) + \mathcal{L}\left(\hat{I}_{z, \lambda_b}, I_{z, \lambda_b}\right),
\end{align}
where $\hat{I}$ is the target image, $\mathcal{L}$ is a pixel-wise loss function such as mean-square error, and $\lambda_r, \lambda_g, \lambda_b$ are the wavelengths corresponding to red, green, and blue respectively. Since the model is differentiable, we solve Eq. \ref{eq:loss_func_rgb} with gradient descent.

\begin{figure}
    \centering
    \includegraphics[clip, trim = 0in 7.7in 4.25in 0in ,width=.45\textwidth]{figures/Replicates.pdf}
    \caption{
    %
    \textbf{Extended phase range reduces depth replicas in simulation.}
    (A) Using an SLM with a uniform $2\pi$ phase range across all channels leads to strong depth replicas (top row), which reduce image quality at the target plane compared to the target (bottom row) and add in-focus content at depths that should be defocused. By using the extended phase Holoeye Pluto-2.1-Vis-016 SLM (with  Red: $2.4\pi$, Green: $5.9\pi$, Blue: $7.4\pi$ phase ranges), depth replicas are significantly reduced (middle row), improving the quality of target plane holograms and creating defocused content at other depths. (B) Schematic illustrating the positions of the replicate planes and target plane. Note, this simulation was generated using RGB images and three color channels, but only the green and blue channels are displayed for clarity. (Rubik's cube source image by Iwan Gabovitch (CC BY 2.0).}
    \label{fig:replicates}
\end{figure}


\subsection{Color-Depth Ambiguity} \label{sec:color-depth-ambiguity}
A common model for propagating electric fields is Fresnel propagation\footnote{Fresnel propagation is the paraxial approximation to the popular angular spectrum method (ASM). Since most commercials SLMs have pixel pitch greater than \SI{3}{\micro\metre}, resulting in a maximum diffraction angle under $5^\circ$ (well within the small angle approximation), Fresnel and ASM are almost identical for holography.} \cite{goodmap2005fourier}, which can be written in Fourier space as
\begin{align}
    f_{\text{fresnel}}(g, z, \lambda) &= \mathcal{F}^{-1} \left\{ \mathcal{F}\{g\} \cdot H(z, \lambda)\right\} \label{eq:fresnel_convolution}\\
    H(z, \lambda) &= \exp \left(i \pi \lambda z \left(f_x^2 + f_y^2\right) \right) \label{eq:fresnel_kernel}% Fresnel propagation kernel
\end{align}
where $\mathcal{F}$ is a 2D Fourier transform, $H$ is the Fresnel propagation kernel, and $f_x$, $f_y$ are the spatial frequency coordinates. In Eq.~\ref{eq:fresnel_kernel}, note that $\lambda$ and $z$ appear together, creating an ambiguity between wavelength and propagation distance.

To see how this ambiguity affects color holograms, consider the case where $\phi_{\lambda}$ in Eq. \ref{eq:SLM_to_field} is independent of wavelength ($\phi_{\lambda} = \phi$). For example, this would be the case if the SLM had a linear phase range from 0 to $2\pi$ at every wavelength.
%
Although this is unrealistic for most off-the-shelf SLMs, it is a useful thought experiment.
%
Note that if $\phi$ is wavelength-independent, then so is the electric field off the SLM ($g_\lambda = g$). In this scenario, assuming $f_\text{prop} = f_\text{fresnel}$, the Fresnel kernel is the only part of the model affected by wavelength.

%

Now assume that the SLM forms an image at distance $z_0$ under red illumination. From the ambiguity in the Fresnel kernel, we have the following equivalence:
\begin{align}
    H(z_0, \lambda_r) = H\left(\tfrac{\lambda_g}{\lambda_r} z_0, \lambda_g\right) = H\left(\tfrac{\lambda_b}{\lambda_r} z_0, \lambda_b\right).
\end{align}
This means the \textit{same} image formed in red at $z_0$ will also appear at $z = z_0 \lambda_g/\lambda_r$ when the SLM is illuminated with green and at $z = z_0 \lambda_b / \lambda_r$ when the SLM is illuminated with blue. We refer to these additional copies as ``depth replicas,'' and this phenomena is depicted in Fig. ~\ref{fig:replicates}.
%
Note that depth replicas do not appear in sequential color holography, since the SLM pattern optimized for red is never illuminated with the other wavelengths.

If we only care about the hologram at the target plane $z_0$, then the depth replicas are not an issue, and in fact, we can take advantage of the situation for hologram generation:
%
The SLM pattern for an RGB hologram at $z_0$ is equivalent to the pattern that generates a three-plane red hologram where the RGB channels of the target are each at a different depth ($z0$,  $z_0 \lambda_r/\lambda_g$, and  $z_0\lambda_r/\lambda_b$ for RGB respectively).
%
This is the basis of the depth division multiplexing approach of \citet{Makowski2008ColorfulHologram, Makowski2010ColorHolograms}, where the authors optimize for this three-plane hologram in red, then illuminate in RGB.
%
Although this makes the assumption that $\phi$ does not depend on $\lambda$, this connection between simultaneous color and multi-plane holography suggests simultaneous color should be possible for a single plane, since multi-plane holography has been successfully demonstrated in prior work.

However, the ultimate goal of holography is to create 3D imagery, and the depth replicas could prevent us from placing content arbitrarily over the 3D volume.
%
In addition, in-focus images can appear at depths that should be out-of-focus, which may prevent the hologram from successfully driving accommodation \cite{kim2022accommodative}.
%
We propose taking advantage of SLMs with extended phase range to mitigate the effects of depth replicas.


\subsection{SLM Extended Phase Range} \label{sec:extended-phase}
In general, the phase $\phi_\lambda$ of the light depends on its wavelength, which was not considered in Sec. \ref{sec:color-depth-ambiguity}.
% 
Perhaps the most popular SLM technology today is LCoS, in which rotation of birefringent LC molecules causes a change in refractive index.
%
The phase of light traveling through the LC layer is delayed by
\begin{align}
    \phi_\lambda = \frac{2 \pi d}{\lambda}n(s, \lambda), \label{eq:phase-LC}
\end{align}
where $d$ is the thickness of the LC layer, and its refractive index, $n$, is controlled with the digital input $s$.
%
$n$ also depends on $\lambda$ due to dispersion \cite{Jesacher2014ColourRange}.
% 
The wavelength dependence of $\phi_\lambda$ presents an opportunity to reduce or remove the depth replicas.
%
Even if the propagation kernel $H$ is the same for several $(\lambda, z)$ pairs, if the phase, and therefore the electric field off the SLM, changes with $\lambda$, then the output image intensity at the replica plane will also be different.
%
As the wavelength-dependence of $\phi_\lambda$ increases, the replicas are diminished.

We can quantify the degree of dependence on $\lambda$ by looking at the derivative $d\phi / {d\lambda}$ which informs us that larger $n$ will give $\lambda$ more influence on the SLM phase.  
%
However, the final image intensity depends only on relative phase, not absolute phase;
%
therefore, for the output image to have a stronger dependence on $\lambda$, we desire larger $\Delta n = n_{\text{max}} - n_{\text{min}}$.
%
 In addition, $d\phi / {d\lambda}$ increases with $-dn / {d\lambda}$, suggesting that more dispersion is helpful for simultaneous color. Although $d\phi / {d\lambda}$ also depends on the absolute value of $\lambda$, we have minimal control over this parameter since there are limited wavelengths corresponding to RGB. In summary, this means we can reduce depth replicas in simultaneous color with larger phase range on the SLM and higher dispersion.

However, there is a trade-off: 
%
As the range of phase increases, the limitations of the bit depth of the SLM become more noticeable, leading to increased quantization errors.
%
We simulate the effect of quantization on hologram quality and find that PSNR and SSIM are almost constant for 6 bits and above.
%
This suggests that each $2\pi$ range should have at least 6 bits of granularity.
%
Therefore, we think that using a phase range of around $8\pi$ for an 8-bit SLM will be the best balance between replica reduction and maintaining accuracy for hologram generation.
%
Figure ~\ref{fig:replicates} simulates the effect of extended phase range on depth replica removal. While holograms were calculated on full color images, only two color channels are shown for simplicity.
%

%
In the first row of Fig. ~\ref{fig:replicates}, we simulate an SLM
with no wavelength dependence to $\phi$ (i.e. 0 - $2\pi$ phase range for each color). Consequently, perfect copies appear at the replica planes.
%
In the second row, we simulate using the specifications from an extended phase range SLM (Holoeye Pluto-2.1-Vis-016), which has $2.4\pi$ range in red, $5.9\pi$ range in green, and $7.4\pi$ range in blue demonstrating that replicas are substantially diminished with an extended phase range.
%
By reducing the depth replicas, the amount of high frequency out of focus light at the sensor plane is reduced leading to improved hologram quality.

\begin{figure}
    \centering
    \includegraphics[clip, trim = 0in 8.125in 4.27in 0in, width=.45\textwidth]{figures/PerceptualLoss.pdf}
    \caption{
    \textbf{Perceptual loss improves color fidelity and reduces noise in simulation.}
    %
    The first column of this figure depicts simulated holograms that were optimized with an RGB loss function (A) and our perceptual loss function (B).  The same filters for the perceptual loss function then were applied to both of these simulated holograms as well as the target image.  Image metrics were calculated between the filtered holograms and the filtered target image (D). All image metrics are better for the perceptually optimized hologram (B). One should also note that the filtered target (D) and original target (C) are indistinguishable suggesting our perceptual loss function only removes information imperceptible by the human visual system as intended.}
    \label{fig:perceptual_loss}
\end{figure}

\begin{figure*}
    \centering
    \includegraphics[clip, trim = 0in 8.35in 0in 0in, width=\textwidth]{figures/SimulationComparison.pdf}
    \caption{
    %
    \textbf{Comparison of bit division, depth division and our method of simultaneous color holography in simulation.}
    %
    Bit division (Col. 1) is noisier than our method but achieves comparable color fidelity, although more washed out.  The depth division method (Col. 2) is also noisier than our method and has inferior color fidelity.  Our method matches the target image well.  Our method uses our perceptual loss function and a high order angular spectrum propagation model with no learned components. Further implementation details for each method are available in the supplement.
    }
    \label{fig:Simulation}
\end{figure*}

\subsection{Perceptual Loss Function} \label{sec:perceptional-loss}

% taking out the statement about 3D being harder since we use a mask so there are the same number of target outputs with 3D as 2D.

Creating an RGB hologram with a single SLM pattern is an overdetermined problem as there are $3\times$ more output pixels than degrees of freedom of the SLM. As a result, it may not be possible to exactly match the full RGB image, which can result in color deviations and de-saturation. To address this, we take advantage of color perception in human vision. There's evidence that the human visual systems converts RGB images into a luminance channel (a grayscale image) and two chrominance channels, which contain information about the color \cite{wandell1995foundations}. The visual system is only sensitive to high resolution features in the luminance channel, so the chrominance channels can be lower resolution with minimal impact on the perceived image \cite{wandell1995foundations}. This observation is used in JPEG compression \cite{pennebaker1992jpeg} and subpixel rendering \cite{platt2000optimal}, but to our knowledge, it has never been applied to holographic displays. By allowing the unperceived high frequency chrominance and extremely high frequency luminance features to be unconstrained, we can better use the the degrees of freedom on the SLM to faithfully represent the rest of the image.

Our flexible optimization framework allows us to easily change the RGB loss function in Eq. \ref{eq:loss_func_rgb} to a perceptual loss. For each depth, we transform the RGB intensities of both $\hat{I}$ (the target image) and $I$ (the simulated image from the SLM) into opponent color space as follows: 
\begin{equation}
\begin{split}
    O_{1} &= 0.299 \cdot I_{\lambda_r} + 0.587 \cdot I_{\lambda_g} + 0.114 \cdot I_{\lambda_b} \\
    O_{2} &= I_{\lambda_r} - I_{\lambda_g}\\
    O_{3} &= I_{\lambda_b} - I_{\lambda_r} - I_{\lambda_g}
\end{split}
\end{equation}
where $O_1$ is the luminance channel, and $O_2$, $O_3$ are the red-green and blue-yellow chrominance channels, respectively. We can then update Eq. \ref{eq:loss_func_rgb} to
\begin{equation} \label{eq:loss_func_opponent}
\begin{split}
    \argmin_{s} \sum_z \Big[ &\mathcal{L}\left(\hat{O}_{1} * k_1, {O}_{1} * k_1 \right) + 
    \mathcal{L}\left(\hat{O}_{2} * k_2, {O}_{2} * k_2 \right) + \\
    &\mathcal{L}\left(\hat{O}_{3} * k_3, {O}_{3} * k_3 \right) \Big],
\end{split}
\end{equation}
where $*$ represents a 2D convolution with a low pass filter ($k_1 \hdots k_3$) for each channel in opponent color space . $\hat{O}_i$ and $O_i$ are the $i$-th channel in opponent color space of $\hat{I}$ and $I$, respectively. In order to mimic the contrast sensitivity functions of the human visual system, we implement filters in the Fourier domain by applying a low-pass filter of 45\% of the width of Fourier space to the chrominance channels ($O_2$, $O_3$) and a filter of 75\% of the width of Fourier space to the luminance channel ($O_1$).  These filter widths were heuristically determined.

By de-prioritizing high frequencies in chrominance and extremely high frequencies in luminance, the optimizer is able to better match the low frequency color. This low frequency color is what is perceivable by the human visual system. Figure \ref{fig:perceptual_loss}  depicts the hologram quality improvement by optimizing with our perceptual loss function.  

The first column of Fig. ~\ref{fig:perceptual_loss} shows the perceptually filtered versions of simulated holograms generated using an RGB loss function (Fig \ref{fig:perceptual_loss}A) and our perceptual loss function (Fig \ref{fig:perceptual_loss}B). The second column displays the original target image (Fig \ref{fig:perceptual_loss}C) and the perceptually filtered target image (Fig \ref{fig:perceptual_loss}D). It can be observed that the two targets are indistinguishable, indicating that our perceptual filter choices align well with the human visual system. The PSNR and SSIM values are higher for the perceptually optimized hologram, and it also appears less noisy and with better color fidelity. This suggests that the loss function has effectively shifted most of the error into imperceptible regions of the opponent color space.  We see an average PSNR increase of 6.4 dB and average increase of 0.266 in SSIM across a test set of 294 images.


\subsection{Simulation Comparisons}

In Figure ~\ref{fig:Simulation} we compare the performance of our method to the depth and bit division approach to simultaneous color holography.  Depth and bit division use only a single SLM, make use of the full space-time-bandwidth product of the SLM, and contain no bulky optics or filters making these methods the most similar to our method.  The holograms simulated with depth and bit division are much noisier and have lower color fidelity than our proposed method.  The depth division simulated hologram has the worst color fidelity due to to the replica planes discussed in Sec. \ref{sec:color-depth-ambiguity} contributing defocused light at the target plane.  Our method uses a perceptual loss function and the HOASM outlined by ~\citet{Gopakumar2021UnfilteredDisplays} to directly optimize the simultaneous color hologram, while comparison methods optimize indirectly.  This direct approach produces less noisy holograms with better color fidelity.
