\section{Introduction}
\label{Intro}
Holographic displays are a promising technology for augmented and virtual reality (AR/VR). Such displays use a spatial light modulator (SLM) to shape an incoming coherent wavefront so that it appears as though the wavefront came from a real, three-dimensional (3D) object. The resulting image can have natural defocus cues, providing a path to resolve the vergance-accommodation conflict of stereoscopic displays \cite{kim2022accommodative}.  Additionally, the fine-grain control offered by holography can also correct for optical aberrations, provide custom eyeglass prescription correction in software, and enable compact form-factors \cite{maimone2017holographic}, while improving light efficiency compared to traditional LCD or OLED displays \cite{Yin2022AdvancedApplications}. Recent publications have demonstrated significant improvement in hologram image quality \cite{maimone2017holographic, Peng2020NeuralTraining, Choi2021NeuralDisplays} and computation time \cite{shi2021towards, eybposh2020high}, but color holography for AR/VR has remained an open problem.

Traditionally, red-green-blue (RGB) holograms are created through \textit{field sequential color}, where a separate hologram is computed for each of the three wavelengths; these are displayed in sequence and synchronized with the color of the illumination source. Due to persistence of vision, this appears as a single full color image if the update is sufficiently fast, enabling color holography for static displays. However, in a head-mounted AR/VR system displaying world-locked content, frame rate requirements are higher to prevent noticeable judder \cite{van2016asynchronous}. In fact, all modern VR displays are ``low persistance'' meaning the image content is only displayed for a fraction of the frame time (usually about 10\%) and no content is shown during the rest of the frame \cite{zielinski2015exploring}. This is usually achieved by strobing the illumination, but if one wished to display three sequential color frames all within a 10\% persistence time, it would require the display to update $30\times$ faster than the effective frame rate. Without low persistence, field sequential color leads to strong color fringing (visible spatial separation of the colors) particularly when the user rotates their head while tracking a fixed object with their eyes \cite{riecke2006selected}.

Low frame rate displays exacerbate these artifacts, and the most common SLM technology for holography, liquid-crystal-on-silicon (LCoS), is quite slow due to the physical response time of the liquid crystal (LC) layer \cite{zhang2014fundamentals}. Although most commercial LCoS SLMs can be driven at 60 Hz, at that speed the SLM will have residual artifacts from the prior frames \cite{haist2015holography}. High speed SLMs based on micro-electro-mechanical system (MEMS)~\cite{MEMS, Choi2022Time-multiplexedModulators} or dual-frequency LCoS~\cite{serati2003highresolution} are becoming more widely available, but even with these devices, simultaneous color is desirable since it eliminates color fringing, enables low persistence, and frees temporal bandwidth for other uses, such as increasing the effective etendue by scanning the field of view or eyebox position \cite{lee2020wide}.

% SLMs can be much faster (in the kilohertz range) but so far have larger pixels and limited bit depth~\cite{MEMS, Choi2022Time-multiplexedModulators}.

%Dual-frequency LCoS technology has recently enabled sub-millisecond response time for LCoS SLMs of standard phase range~\cite{serati2003highresolution}.  This technology is compatible with an extended phase range, and the authors hope to see this commercially available in the near future.

In this work, we aim to display RGB holograms using only a single SLM pattern, enabling a $3\times$ increase in frame rate compared to sequential color and completely removing color fringing artifacts. Our compact setup does not use a physical filter in the Fourier plane or bulky optics to combine color channels. Instead, the full SLM is simultaneously illuminated by an on-axis RGB source, and we optimize the SLM pattern to form the full color image. We design a flexible framework for end-to-end optimization of the digital SLM input from the target RGB intensity, allowing us to optimize for SLMs with extended phase range, and we develop a color-specific perceptual loss function which further improves color fidelity. Our method is validated experimentally on 2D and 3D content.
%
% Rather than improve upon existing SLM hardware, this work aims to increase the effective frame rate of current LCOS SLMs for color holography through the use of simultaneous illumination.  To achieve this in a simple and compact form factor, an unfiltered holography setup with on axis illumination is used. The compact setup results in a complex optical forward model involving higher diffraction orders and neural networks requiring calibration.  The calibrated, differentiable forward model allows a gradient based approach to hologram optimization to be taken.  By using a gradient based method, a custom, perceptual loss function can be used helping making the optimization problem better posed.  Finally, camera-based calibration and active camera-in-the-loop are used to achieve state-of-the-art image quality.
  
Specifically, we make the following contributions:

\begin{itemize}
    \item We introduce a novel algorithm for generating simultaneous color holograms which takes advantage of the extended phase range of the SLM in an end-to-end manner and uses a new loss function based on human color perception.
    \item We analyze the ``depth replicas'' artifact in simultaneous color holography and demonstrate how these replicas can be mitigated with extended phase range.
    \item We demonstrate experimental simultaneous color holograms in both 2D and 3D using a custom camera-calibrated model.
\end{itemize}



% Holographic displays have long promised to be the future of AR/VR displays~\cite{Yin2022AdvancedApplications}. In theory, a single SLM and a coherent light source can perfectly replicate a three dimensional wavefront.
% %
% \fs{unreferenced claim}
% %
% Such a wavefront can provide natural defocus cues eliminating the need for eye-tracking and variafocal displays while still providing a natural and immersive experience for the user.
% %
% Holographic displays are also far more light efficient than traditional LCD or OLED panel based displays in turn reducing power consumption\cite{Yin2022AdvancedApplications}.
% %
% While the promise of holography for AR/VR displays is strong, several hurdles have long prevented the realization of this promise.
% %
% While many of these hurdles have been overcome in the recent years particularly in the improvement of hologram quality~\cite{Peng2021Speckle-freeCalibration,Peng2020NeuralTraining} and computation time~\cite{Peng2020NeuralTraining, Choi2021NeuralDisplays}, color holography for AR/VR still remains elusive. 

% Traditionally, color CGH has been through sequential illumination and updating the SLM display pattern for each illumination channel.
% %
% \fs{Some more references backing up the challenges of color-sequential might be helpful here}
% %
% While this works well for static displays, users move AR/VR displays through space while viewing world locked content. \gk{persistence of vision}
% %
% This leads to substantial judder and color fringing when viewing world locked AR and VR content due to the vestibulo-ocular reflex. \gk{[citation]}
% %
% %
% Unfortunately, the widely available LCOS SLMs typically have a frame rate around 60 Hz ~\cite{FasterSLM} and sequential color reduces the effective frame rate of the display by a factor of three making the system too slow to provide an adequate user experience.

% %
% One solution is MEMS based SLMs~\cite{MEMS, Choi2022Time-multiplexedModulators}.
% %
% MEMS based SLMs can have frame rates upwards of 1 kHz but are not well suited for holography (i.e. large pixel sizes or tilt rather than piston actuation)~\cite{MEMS}.
% %
