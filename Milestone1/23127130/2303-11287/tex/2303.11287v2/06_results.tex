\section{Experimental Results}

\paragraph{2-Dimensional Holograms}
We validate our simulation results by capturing holograms in experiment. For simultaneous color, the SLM patterns were optimized for a propagation distance of \SI{120}{\milli\metre} using our perceptual loss function described in Section \ref{sec:perceptional-loss}. A white border was added to each target image to improve the color fidelity by encouraging a proper white balance. After each hologram is captured, debayering is performed and a homography is applied to map from camera space to SLM space.

Figure~\ref{fig:Stills} compares the simultaneous color capture using a single frame (B) to sequential color using 3 frames (C). Unlike the simultaneous color version, which was captured in one shot with RGB illumination, the sequential color was captured with only the red light source (due to a failure of the green channel in the SLED), and the correct color was assigned in software. Although the sequential captures are higher contrast than our simultaneous results, we'd like to emphasize that our approach uses 3$\times$ fewer degrees of freedom and can still produce full color images. In addition, the simulation output from our model (D) shows color fidelity on par with the sequential capture; the difference between the simulation output and experimental capture can be attributed to model mismatch. This suggests improvements to the calibration pipeline could enable experimental results with the  quality of the simultaneous model.

\paragraph{3-Dimensional Holograms}
A major appeal of holography is the ability to solve the vergence-accommodation conflict, so we also validate our method for 3D scenes.  A 4-plane focal stack was rendered with 0.5 pixels blur radius per millimeter depth.  Holograms were captured at distance from \SI{90}{\milli\metre} to \SI{120}{\milli\metre} in \SI{10}{\milli\metre} increments. The results are displayed in Fig. \ref{fig:Stack}, and once again pseudo-color sequential images (B), which use $3\times$ the number of frames, are shown for comparison. \revised{Although model mismatch creates some color shift in the experimental captures (C), the simultaneous model output (D) shows what the results could look like with improved calibration. We note that 3D hologram generation is not as well-posed as 2D;} despite this, our results demonstrate the ability to form 3D color holograms with natural defocus blur from a single SLM frame.

%Color sequential images, which use $3\times$ the number of frames, are are included as a point of comparison and are captured in the same manner as the 2D case. 