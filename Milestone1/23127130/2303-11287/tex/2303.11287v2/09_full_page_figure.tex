\clearpage 

    \begin{figure*}
    \centering
    \includegraphics[clip, trim = 0in 7.1in 0in 0in, width=0.95\textwidth]{figures/Stills.pdf}
    \caption{\textbf{Experimentally captured 2D holograms.}
    For each target image (A), we show (B) the experimental capture with sequential pseudo-color, (C) our experimental capture with full simultaneous color, and (D) the simulated model output for simultaneous color. Recall that our simultaneous color results (C) use 3$\times$ fewer degrees of freedom than the sequential capture (B). Although some color fidelity is lost in experiment (C), the simulated model output (D) shows good color quality, demonstrating that accurate color is possible with our method and improvements to the calibration.
    % and the which was captured in pseudo-color using the red light source only. 
    % %
    % This figure shows experimentally captured holograms at a depth of \SI{120}{\milli\metre}. Row one: target images. Row two: experimentally captured sequential color holograms. Row three: experimentally captured simultaneous color holograms. Row four: simulation output of the optimized simultaneous color SLM pattern. While most captured simultaneous color holograms have good color fidelity, our method is least effective on highly saturated images with low texture, such as the cat paws in column 4, representing a limitation of our method (see Sec. ~\ref{sec:limits}).
    }
    \label{fig:Stills}
    \end{figure*}

    \begin{figure*}
    \centering
    \includegraphics[clip, trim = 0in 6.42in 0in 0in, width=0.95\textwidth]{figures/FocalStack.pdf}
    \caption{\textbf{Experimentally captured focal stack.} This figure displays a focal stack, with the target shown in (A), captured from $\SI{90}{\milli\metre}$ to $\SI{120}{\milli\metre}$ in $\SI{10}{\milli\metre}$ increments. We compare (B) the sequential pseudo-color experimental capture with (C) the experimental capture of the simultaneous full color hologram and (D) the simulated model output for simultaneous color. Although model mismatch creates some deviations between the simultaneous capture (C) and the target (A), the simulated model (D) is representative of the color fidelity we expect from our method with improvements to the system calibration.
    }
    \label{fig:Stack}
    \end{figure*}

\begin{figure*}
    \centering
    \includegraphics[clip, trim = 0in 6.125in 0in 0in, width=\textwidth]{figures/ModelAblation.pdf}
    \caption{\textbf{Comparison of different propagation methods for suppressing higher diffraction orders.} The first column shows the results obtained using the traditional angular spectrum method (ASM) which doesn't model higher diffraction orders. The second column shows the results obtained using HOASM which reduces the visibility of higher orders but fails to completely suppress them. The third column shows the results obtained using our proposed learned propagation method that includes a U-net, which largely suppresses the higher diffraction orders and results in a hologram with the fewest artifacts, suggesting the learned propagation model best matches the physical propagation.
    }
    \label{fig:Ablation}
    \end{figure*}

\begin{figure*}
    \centering
    \includegraphics[clip, trim = 0in 7.75in 0in 0in, width=.9\textwidth]{figures/Setup.pdf}
    \caption{\textbf{Experimental setup} A top view of our system with labeled components and an approximate beam path drawn in green is depicted in (A).  A side-view of the system is provided by (B). \revised{Note that the hologram is formed directly on the bare camera sensor with no lens or eyepiece between. This configuration allows us to validate our method, but for a human-viewable system, an eyepiece must be added between the hologram plane and the user's eye.}
    }
    \label{fig:Setup}
    \end{figure*}


\clearpage