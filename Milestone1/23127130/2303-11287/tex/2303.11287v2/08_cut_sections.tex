% I don't think we need the partial coherence in the lit review since it's not that important to the method and we site them later

\paragraph{Incoherent and Partially Coherent Holography}
\label{Coherence}
Traditional holography has been performed with strongly coherent light sources leading to distracting speckle in the holograms formed.
%
Additionally, these coherent light sources are typically lasers which pose danger to the eyes
%
Recent work has investigated the feasibility of using partially coherent and incoherent light sources such as SLEDs and LEDs \cite{Deng2017CoherenceDisplays}.
%
\fs{Can LED really be called incoherent? Dont they still have a big spatial coherence, just poor temporal coherence?}
%
\cite{Peng2021Speckle-freeCalibration} models the partially coherent light source simply as the summation of coherent light sources.
%
Using this model produces holograms with significantly reduced speckle leading to sharp, high contrast images.
%
Speckle reduction is of particular importance for color holography as the speckle from each color channel is not aligned. 
%
This leads to mismatching of color balancing and the production of sepia toned holograms.
%
\fs{It's not really clear why you mention partial coherent holography here. Can you maybe explain a bit what implications this might have for your algorithm/work?}

\subsection{Active Camera-in-the-Loop}

\todo{This section is on hold, pending if we use active CiTL in image capture tomorrow. Below is text already written about CiTL which needs to be condensed into a few sentences if we're using it.}

% \gk{ from related work:
% As previously mentioned, small model mismatch often leads to large artifacts in CGH\cite{Peng2020NeuralTraining}.
% %
% While including a neural network in the forward model makes dramatic improvement, it still leaves room for image quality improvement.
% %
% In order to further reduce this detrimental model mismatch, SLM patterns can be optimized with what is known as a camera in the loop strategy. 
% %
% This strategy consists of both computationally and physically propagating the modified wave front. 
% %
% By propagating the wavefront computationally, the computational graph is created allowing for backpropagation to be used for SLM pattern optimization later. 
% %
% The result of the computational propagation is replaced, however, with the captured physical propagation before backpropagation is performed.
% %
% While this does not change the underlying calculation for the gradients, it can change the scale and even the sign of the gradients \cite{Peng2020NeuralTraining, Peng2021Speckle-freeCalibration,Chakravarthula2020LearnedDisplays, Mosleh2020Hardware-in-the-LoopPipelines, Chen2022Off-axisGeneration} .
% %
% This gradient modification allows for the experimentally captured hologram to be iteratively improved upon achieving excellent image quality.}

% \gk{this section should be compressed}


