% CVPR 2023 Paper Template
% based on the CVPR template provided by Ming-Ming Cheng (https://github.com/MCG-NKU/CVPR_Template)
% modified and extended by Stefan Roth (stefan.roth@NOSPAMtu-darmstadt.de)

\documentclass[10pt,twocolumn,letterpaper]{article}

%%%%%%%%% PAPER TYPE  - PLEASE UPDATE FOR FINAL VERSION
% \usepackage[review]{cvpr}      % To produce the REVIEW version
\usepackage{cvpr}              % To produce the CAMERA-READY version
%\usepackage[pagenumbers]{cvpr} % To force page numbers, e.g. for an arXiv version

% Include other packages here, before hyperref.
\usepackage{graphicx}
\usepackage{amsmath}
\usepackage{amssymb}
\usepackage{booktabs}
\usepackage[dvipsnames]{xcolor}
% New packages
\usepackage{makecell}
\usepackage{amssymb}% http://ctan.org/pkg/amssymb
\usepackage{pifont}% http://ctan.org/pkg/pifont
\usepackage{diagbox}
\usepackage{multirow}
\usepackage{amsmath}
\usepackage{bbm}
\usepackage{fancyhdr}
\usepackage[accsupp]{axessibility}  % Improves PDF readability for those with disabilities.
% \usepackage{ulem}

\definecolor{citecolor}{HTML}{0071bc}
\definecolor{abbrcolor}{HTML}{990000}
\usepackage[breaklinks,colorlinks,citecolor=citecolor,urlcolor=citecolor,bookmarks=false]{hyperref}

% New commands
\newcommand{\ourmodel}{ZOOM\xspace}
\DeclareMathOperator*{\argmax}{argmax}
\DeclareMathOperator*{\argmin}{argmin}

% It is strongly recommended to use hyperref, especially for the review version.
% hyperref with option pagebackref eases the reviewers' job.
% Please disable hyperref *only* if you encounter grave issues, e.g. with the
% file validation for the camera-ready version.
%
% If you comment hyperref and then uncomment it, you should delete
% ReviewTempalte.aux before re-running LaTeX.
% (Or just hit 'q' on the first LaTeX run, let it finish, and you
%  should be clear).

% \usepackage[pagebackref,breaklinks,colorlinks]{hyperref}

% Support for easy cross-referencing
\usepackage[capitalize]{cleveref}
\crefname{section}{Sec.}{Secs.}
\Crefname{section}{Section}{Sections}
\Crefname{table}{Table}{Tables}
\crefname{table}{Tab.}{Tabs.}


%%%%%%%%% PAPER ID  - PLEASE UPDATE
\def\cvprPaperID{3083} % *** Enter the CVPR Paper ID here
\def\confName{CVPR}
\def\confYear{2023}


\begin{document}

%%%%%%%%% TITLE - PLEASE UPDATE
\title{Zero-shot Model Diagnosis}

\author{Jinqi Luo\footnotemark[1] \qquad Zhaoning Wang\footnotemark[1] \qquad Chen Henry Wu \qquad Dong Huang \qquad Fernando De la Torre\\
Carnegie Mellon University\\
{\tt\small \{jinqil, zhaoning, chenwu2, dghuang, ftorre\}@cs.cmu.edu}
% For a paper whose authors are all at the same institution,
% omit the following lines up until the closing ``}''.
% Additional authors and addresses can be added with ``\and'',
% just like the second author.
% To save space, use either the email address or home page, not both
}

\maketitle

{
  \renewcommand{\thefootnote}%
    {\fnsymbol{footnote}}
  \footnotetext[1]{Equal contribution.}
  \footnotetext[2]{The code is publicly available at the project page: \href{https://zero-shot-model-diagnosis.github.io/}{https://zero-shot-model-diagnosis.github.io/}. }
}

%%%%%%%%% ABSTRACT
\begin{abstract}
\vspace{-3mm}
When it comes to deploying deep vision models, the behavior of these systems must be explicable to ensure confidence in their reliability and fairness. A common approach to evaluate deep learning models is to build a labeled test set with attributes of interest and assess how well it performs. However, creating a balanced test set (i.e., one that is uniformly sampled over all the important traits) is often time-consuming, expensive, and prone to mistakes. The question we try to address is: can we evaluate the sensitivity of deep learning models to arbitrary visual attributes \textbf{without an annotated test set}? 

This paper argues the case that \textcolor{abbrcolor}{\textbf{Z}}er\textcolor{abbrcolor}{\textbf{o}}-sh\textcolor{abbrcolor}{\textbf{o}}t \textcolor{abbrcolor}{\textbf{M}}odel Diagnosis (ZOOM) is possible without the need for a test set nor labeling. To avoid the need for test sets, 
our system relies on a generative model and CLIP. The key idea is enabling the user to select a set of prompts (relevant to the problem) and our system will automatically search for semantic counterfactual images (i.e., synthesized images that flip the prediction in the case of a binary classifier) using the generative model.  We evaluate several visual tasks (classification, key-point detection, and segmentation) in multiple visual domains to demonstrate the viability of our methodology. Extensive experiments demonstrate that our method is capable of producing counterfactual images and offering sensitivity analysis for model diagnosis without the need for a test set. 
\end{abstract}

%%%%%%%%% BODY TEXT
\section{Introduction}
\label{sec:introduction}
% \begin{itemize}
%     % Diffusion of FL
%     \item {\st{Diffusion of FL}}
%     % Security threats to FL
%     \item {\st{Security threats to FL with particular focus on model poisoning}}
%     % Limitations of existing countermeasures
%     \item {\st{Current countermeasures (e.g., KRUM) and their limitations}}
%     % Proposed method and its advantages
%     \item {\st{Intuitive description of the proposed method and its difference (i.e., advantages) w.r.t. state of the art}}
%     % Main contributions
%     \item {\st{Summary of the main contributions of this work}}
%     % Paper's structure and organization
%     \item {\st{Paper's structure and organization}}
% \end{itemize}

% Diffusion of FL
Recently, {\em federated learning} (FL) has emerged as the leading paradigm for training distributed, large-scale, and privacy-preserving machine learning (ML) systems~\cite{mcmahan2017googleai,mcmahan2017aistats}. 
The core idea of FL is to allow multiple edge clients to collaboratively train a shared, global model without disclosing their local private training data.
%Specifically, an FL system consists of a central server and many edge clients; 
A typical FL round involves the following steps: {\em(i)} the server randomly picks some clients and sends them the current, global model; {\em(ii)} each selected client locally trains its model with its own private data; then, it sends the resulting local model to the server;\footnote{Whenever we refer to global/local model, we mean global/local model {\em parameters}.} {\em(iii)} the server updates the global model by computing an \emph{aggregation function}, usually the average (FedAvg), on the local models received from clients.
% \begin{enumerate}
%     \item[{\em(i)}] the server sends the current, global model to the clients and appoints some of them for training;
%     \item[{\em(ii)}] each selected client locally trains its copy of the global model with its own private data; then, it sends the resulting local model back to the server;\footnote{Whenever we refer to global/local model, we mean global/local model {\em parameters}.}
%     \item[{\em(iii)}] the server updates the global model by computing an \emph{aggregation function} on the local models received from clients (by default, the average, also referred to as FedAvg~\cite{mcmahan2017aistats}).
% \end{enumerate}
This process goes on until the global model converges. %(e.g., after a certain number of rounds or other similar stopping criteria).
%\\
% The advantages of FL over the traditional, centralized learning paradigm are undoubtedly clear in terms of flexibility/scalability (clients can join/disconnect from the FL network dynamically), network communications (only model weights\footnote{We will use \textit{parameters} and \textit{weights} interchangeably.} are exchanged between clients and server), and privacy (each client's private training data is kept local at the client's end and not uploaded to the server).
\\
% Security threats to FL
%However, the growing adoption of FL also raises security concerns~\cite{costa2022covert}, particularly about its confidentiality, integrity, and availability.
Although its advantages over standard ML, FL also raises security concerns~\cite{costa2022covert}. %, particularly about its confidentiality, integrity, and availability~\cite{costa2022covert}.
% OLD, LONG VERSION
% Indeed, some work deals with privacy leakage that may expose the local data of some clients~\cite{melis2019sp}. 
% A large body of work, instead, investigates attacks that usually aim to detriment the predictive accuracy of the learned global model. For instance, \emph{data poisoning} attacks achieve this goal by letting an adversary pollute the training set of some corrupt FL clients with maliciously crafted examples~\cite{jagielski2018sp}.
% Similarly, in \emph{model poisoning} the attacker attempts to tweak the global model weights~\cite{bhagoji2019pmlr} by directly perturbing the local model's weights of some infected FL clients before these are sent to the central server for aggregation, usually via so-called Byzantine attacks. 
% It turns out that Byzantine model poisoning attacks severely impact standard FedAvg; therefore, more robust aggregation functions must be designed to make FL systems secure.
Here, we focus on \emph{untargeted model poisoning} attacks~\cite{bhagoji2019pmlr}, where an adversary attempts to tweak the global model weights %\footnote{We will use the terms \textit{parameters} and \textit{weights} interchangeably.} 
by directly perturbing the local model's parameters of some infected clients before these are sent to the central server for aggregation.
In doing so, the adversary aims to jeopardize the global model \textit{indiscriminately} at inference time.
Such model poisoning attacks severely impact standard FedAvg; therefore, more robust aggregation functions must be designed to secure FL systems.
\\
% In this paper, we focus on designing a novel robust aggregation scheme at the server's end to contrast the effect of Byzantine model poisoning attacks.
%
% Current countermeasures and their limitations
%Several countermeasures have been proposed in the literature to combat model poisoning attacks on FL systems.
% Some methods use simple statistics more robust than plain average to smooth the impact of malicious updates (e.g., Trimmed Mean and FedMedian~\cite{yin2018icml}). 
% Other defenses implement outlier detection techniques to discard malicious updates from the aggregation performed at the server's end. Those are either based on heuristics (e.g., Krum/Multi-Krum~\cite{blanchard2017nips} and Bulyan~\cite{mhamdi2018pmlr}) or data-driven approaches (e.g., K-means clustering~\cite{shen2016acm} or DnC via spectral analysis~\cite{shejwalkar2021ndss}). 
% Finally, some strategies rely on a centralized ``source of trust'' to spot potential malicious updates (e.g., FLTrust~\cite{cao2020fltrust}).
% Several countermeasures have been proposed in the literature to combat model poisoning attacks on FL systems, i.e., to discard possible malicious local updates from the aggregation performed at the server's end. 
% These techniques range from simple statistics more robust than plain average (e.g., Trimmed Mean and FedMedian~\cite{yin2018icml}) to outlier detection heuristics (e.g., Krum/Multi-Krum~\cite{blanchard2017nips} and Bulyan~\cite{mhamdi2018pmlr}) or data-driven approaches (e.g., spectral analysis via K-means clustering~\cite{shen2016acm} or spectral analysis), or methods based on ``source of trust'' (e.g., FLTrust~\cite{cao2020fltrust}).
% OLD, LONG VERSION
%Several countermeasures have been proposed in the literature to combat Byzantine model poisoning attacks on FL systems.
% Descriptive statistics
% For example, Trimmed Mean and FedMedian aggregate local model updates using more robust statistics than standard average~\cite{yin2018icml}.
%
% % Heuristics for outlier detection
% Many existing Byzantine-resilient strategies implement some outlier detection heuristics to discard the model updates sent by potentially malicious clients from the input of the aggregation function.
% One of the most popular heuristics is Krum~\cite{blanchard2017nips}.
% This strategy tries to mitigate the impact of Byzantine attacks by selecting as a global model the local model with the smallest sum of Euclidean distances to {\em all} the other local models.
% Although powerful, Krum requires the server to know (or, at least, estimate) the number of malicious FL clients upfront, which is generally impossible in a realistic attack scenario. %
% Moreover, Krum may become ineffective for complex, high-dimensional model parameter spaces due to the curse of dimensionality.
% Bulyan~\cite{mhamdi2018pmlr} tries to overcome this issue by combining Krum with a variant of Trimmed Mean.
% % Data-driven outlier detection
% Other strategies use data-driven outlier detection techniques -- e.g., via K-means clustering~\cite{shen2016acm} -- to spot potential malicious local model updates. 
% %For instance, Shen et al. propose to cluster local model updates with K-means and thus identify outliers.
%
% % Other techniques
% As far as the server is concerned, any local model received can be from a potential malicious client. 
% FLTrust~\cite{cao2020fltrust} assumes the server acts as a client, i.e., trains a local model on an additional {\em trustworthy} dataset at the server's end and compares it against all the local models from other clients. 
% This way, the server can rely on some ``source of trust'' when discarding potentially malicious clients.
%\\
% Limitations of existing Byzantine-resilient strategies
Unfortunately, existing defense mechanisms either rely on simple heuristics (e.g., Trimmed Mean and FedMedian by~\cite{yin2018icml}) or need strong and unrealistic assumptions to work effectively (e.g., foreknowledge or estimation of the number of malicious clients in the FL system, as for Krum/Multi-Krum~\cite{blanchard2017nips} and Bulyan~\cite{mhamdi2018pmlr}, which, however, cannot exceed a fixed threshold).
Furthermore, outlier detection methods using K-means clustering~\cite{shen2016acm} or spectral analysis like DnC~\cite{shejwalkar2021ndss} do not directly consider the temporal evolution of local model updates received.
Finally, strategies like FLTrust~\cite{cao2020fltrust} require the server to collect its own dataset and act as a proper client, thereby altering the standard FL protocol.
\\
% OLD, LONG VERSION
% Overall, existing Byzantine-resilient strategies are either simple heuristics (e.g., FedMedian) or, if they are more complex, they rely on strong and unrealistic assumptions to work effectively (e.g., knowing the number of malicious clients in the FL system in advance, as for Krum and alike).
% Furthermore, data-driven outlier detection methods do not consider the temporary evolution of local model updates received (e.g., K-means clustering). 
% Finally, strategies like FLTrust requires the server to collect its own dataset and act as a proper client, thereby altering the standard FL protocol.
%
% Description of the proposed method
This work introduces a novel pre-aggregation \textit{filter} robust to untargeted model poisoning attacks. Notably, this filter $(i)$ operates without requiring prior knowledge or constraints on the number of malicious clients and $(ii)$ inherently integrates temporal dependencies. 
The FL server can employ this filter as a preprocessing step before applying \textit{any} aggregation function, be it standard like FedAvg or robust like Krum or Bulyan.
Specifically, we formulate the problem of identifying corrupted updates as a multidimensional (i.e., matrix-valued) time series anomaly detection task. 
The key idea is that legitimate local updates, resulting from well-calibrated iterative procedures like stochastic gradient descent (SGD) with an appropriate learning rate, show \textit{higher predictability} compared to malicious updates. This hypothesis stems from the fact that the sequence of gradients (thus, model parameters) observed during legitimate training exhibit regular patterns, as validated in Section~\ref{subsec:intuition}. %until convergence. 
%This regularity may be more pronounced for smooth convex loss functions, but it can still be captured within an appropriate time window, even for more complex and convoluted loss surfaces. 
%We provide evidence of this claim in Appendix~B, where we show that the average mutual information (i.e., ``predictability''), calculated over pairs of legitimate model updates sent at different FL rounds, is significantly higher than the corresponding computation for a malicious client.
\\
Inspired by the matrix autoregressive (MAR) framework for multidimensional time series forecasting~\cite{chen2021je}, we propose the FLANDERS ({\em \textbf{F}ederated \textbf{L}earning meets \textbf{AN}omaly \textbf{DE}tection for a \textbf{R}obust and \textbf{S}ecure}) filter.
The main advantages of FLANDERS over existing strategies like FLDetector~\cite{zhao2020multivariate} are its resilience to large-scale attacks, where $50\%$ or more FL participants are hostile, and the capability of working under realistic non-iid scenarios.
We attribute such a capability to two key factors: $(i)$ FLANDERS works without knowing a priori the ratio of corrupted clients, and $(ii)$ it embodies temporal dependencies between intra- and inter-client updates, quickly recognizing local model drifts caused by evil players. Below, we summarize our main contributions:

\begin{itemize}
\item[{\em(i)}]
We provide empirical evidence that the sequence of models sent by legitimate clients is more predictable than those of malicious participants performing untargeted model poisoning attacks.
\\
\item[{\em(ii)}] 
We introduce FLANDERS, the first pre-aggregation filter for FL robust to untargeted model poisoning based on multidimensional time series anomaly detection.
\\
\item[{\em(iii)}] 
We integrate FLANDERS into Flower,\footnote{\scriptsize{\url{https://flower.dev/}}} a popular FL simulation framework for reproducibility.
\\
\item[{\em(iv)}] 
We show that FLANDERS improves the robustness of the existing aggregation methods under multiple settings: different datasets, client's data distribution (non-iid), models, and attack scenarios.
\\
\item[{\em(v)}] 
We publicly release all the implementation code of FLANDERS along with our experiments.\footnote{\scriptsize{\url{https://anonymous.4open.science/r/flanders_exp-7EEB}}}
\end{itemize}

% Paper's structure and organization
The remainder of the paper is structured as follows. %some related work and the current state-of-the-art solutions to security issues that FL entails. 
Section~\ref{sec:background} covers background and preliminaries. 
In Section~\ref{sec:related}, we discuss related work.
Section~\ref{sec:problem} and Section~\ref{sec:method} describe the problem formulation and the method proposed. % to tackle it. 
Section~\ref{sec:experiments} gathers experimental results. %, and Section~\ref{sec:limitations} discusses some limitations of this work.
Finally, we conclude in Section~\ref{sec:conclusion}.
 %discusses the limitations of this work and draws future research directions.
%reports conclusions and draws perspectives for future research directions.

%%%%%%% OLD %%%%%%%
%to overcome the resilience of Byzantine failures in distributed Stochastic Gradient Descent computations. 
% The strength of Krum is its time complexity, which is linear in the gradient dimension. 
% However, the robustness of the approach is guaranteed for gradient-based learning applications only when the majority of the clients are not compromised. 
% Besides, the aggregation mechanism of Krum, as well as that of similar methods, is robust from a coarse-grained perspective and does not provide solutions to errors and perturbations that may occur at inference time.
%A related approach to~\cite{blanchard2017nips} is the work of Su et al.~\cite{su2016dc}. Here, the authors propose an iterated approximate agreement to tackle a multi-layer scenario attacked by Byzantine agents. 
%However, the method works efficiently on the sole discrete context and it is inapplicable to continuous state environments.
%\gabri{Maybe, we should just talk about the main limitations of existing countermeasures without digging into their details (or, we can just mention Krum as this is the most popular one). I will move the description of all these methods to the Related Work section.}
\section{Related work}
% There is extensive recent work on speeding up analytical queries due to the need for consistent execution times in the face of the explosive growth in the volume of available data.
% In this section, we divide existing work into two categories: maintaining data freshness in MVs (\Cref{sec:server_side}) and optimizations for minimizing ad-hoc query latency (\Cref{sec:client_side}).

% \subsection{Maintaining Data Freshness in MVs}
% \label{sec:server_side}
% There exists a variety of data warehousing applications aimed at supporting low-latency analytical queries on fresh data.
% In particular, these applications require efficiency in the propagation of newly ingested data into downstream MVs.
 
\mypara{Efficient MV Refresh}
Incremental view maintenance (IVM) aims to update MVs to reflect newly ingested data, taking advantage of already computed results to perform the update in a manner more efficient than computing from scratch (full refresh)
~\cite{ahmad2012dbtoaster,mcsherry2013differential,armbrust2013generalized,zeng2016iolap, palpanas2002incremental, griffin1995incremental, agiwal2021napa, braun2015analytics}. 
There is an abundance of work in IVM, including incremental updates on duplicate values~\cite{griffin1995incremental}, non-distributive aggregate functions~\cite{palpanas2002incremental}, higher-order views~\cite{ahmad2012dbtoaster}, and sliding windows~\cite{braun2015analytics}. 
More recent works also investigate the scalability aspect of IVM, proposing scale-independent updates~\cite{armbrust2013generalized} and sampled views~\cite{zeng2016iolap}. Since \system is applicable to arbitrary SQL statements, \system is orthogonal to and is fully compatible with existing IVM techniques.

\mypara{MV Refresh Scheduling}
There exist works on scheduling the refresh of a MV set focusing on resolving cyclic dependencies~\cite{folkert2005optimizing}, minimizing weighted average staleness~\cite{golab2009scheduling}, and prioritizing MVs with the highest speedups on predicted future queries~\cite{ahmed2020automated}.
\system's scheduling to speed up the end-to-end refresh of the MV set is not addressed in existing works.

\mypara{DAG Workflow Scheduling}
The execution of workloads consisting of individual jobs with acyclic dependencies is a well-studied topic~\cite{apacheoozie,sparkdag,marchal2018parallel,bathie2020revisiting,baruah2022ilp}; many of these techniques can be applied to MV refresh runs studied in this paper.
Existing workflow scheduling systems such as Apache Oozie~\cite{apacheoozie}, Apache Airflow~\cite{airflow}, and Spark DAG scheduler~\cite{sparkdag} automate the execution of user-defined workflows following a topological order.
There are recent works aimed at finding more optimal execution orders in terms of peak memory usage~\cite{marchal2018parallel, bathie2020revisiting} and execution time on parallel platforms~\cite{baruah2022ilp}.
While \system is designed for use with MV refresh runs/workloads, our technique on joint scheduling and optimization can be reasonably applied to general workloads as a possible future direction.

% \paragraph{Incremental MV indexing}
% Update-optimized indices such as the log-structured merge-trees (LSM)~\cite{o1996log} are used for indexing MVs due to frequent updates induced by data ingestion~\cite{gupta2016mesa,agiwal2021napa}.
% \system is orthogonal to indexing: \system is capable of efficiently performing MV refresh runs regardless of whether the individual MVs are indexed or not.

% \subsection{Ad-hoc Query Latency Reduction}
% \label{sec:client_side}

% The minimization of ad-hoc analytical query response times is a well-studied topic due to latency being negatively correlated with the productivity of a data analyst during a data analysis session~\cite{liu2014effects}.
% Sessions are commonly conducted within visualization systems that contain a variety of optimization techniques to ensure that query response times fall within a certain latency tolerance.

% \mypara{Data prefetching}
% Data is often loaded into memory on a by-need basis in visualization systems to minimize interference with user-issued query computations~\cite{mani2017effective,xin2021enhancing,galakatos2017revisiting, yan2020auto, battle2016dynamic, crotty2016case, jalaparti2018netco}. 
% Query-time data retrieval can be significantly expedited by anticipating the data usage of the user in future queries and pre-loading the data into memory during the downtime between user queries (`think time'). SMART~\cite{mani2017effective} prefetches data for modified versions of current user-issued queries with different filters and dimensions. A-WARE~\cite{crotty2016case} maintains a sample store constantly refined through ingesting data based on speculations of future plots.
% ForeCache~\cite{battle2016dynamic} uses an SVM to predict the user's current analysis phase and accordingly prefetches data tiles partitioned based on different numerical values. NetCo predicts future queries via log analysis, and solves an ILP formulation to prefetch data to maximize the number of SLO-meeting queries~\cite{jalaparti2018netco}.
% In the case of MV refresh workloads, `think time' is nonexistent as individual MVs are refreshed back-to-back, rendering data prefetching techniques non-applicable.

\mypara{Intermediate Data Caching}
Some existing data visualization systems cache user-defined variables to support the typical incremental construction of data visualizations~\cite{zgraggen2016progressive, eichmann2020idebench} during data analysis sessions~\cite{jupyter, rstudio, colab}. 
Recent work proposes a management scheme for these cached variables under a memory constraint that greedily keeps variables with the highest estimated time savings based on predicted future user behavior via neural networks~\cite{xin2021enhancing}.
While useful for data visualization, a greedy approach to memory management fails to achieve satisfactory results compared to \system.

\mypara{Intermediate Result Reuse}

There exist works on storing intermediate results from computations to speedup future computations~\cite{yang2018intermediate, dursun2017revisiting, nagel2013recycling, michiardi2019memory, galakatos2017revisiting}.
Studied topics include the identification of reuse opportunities by finding overlaps in computation graphs of successive jobs~\cite{yang2018intermediate, michiardi2019memory},
selective storage under a space constraint with heuristics such as reuse probability~\cite{dursun2017revisiting}, expected savings~\cite{yang2018intermediate}, and recompute-storage cost difference~\cite{nagel2013recycling},
and rewriting incoming jobs to take advantage of stored intermediates~\cite{galakatos2017revisiting}.
These works share similarity with \system in their selection of items to store under a memory constraint, however, \system's problem setting requires it to uniquely consider the joint (re)ordering of job executions along with the selection of items.

% work that considers both job execution (re)order as well as intermediate result caching with a bounded amount of memory. but notably lack the joint aspect of \system and cannot be used to achieve immediate speedup on an incoming MV refresh run if no intermediates are stored beforehand. 

\mypara{Incremental Query Processing} Incremental processing (IQP) is useful for cases where not all data required for a query is immediately available. Similar to online aggregation~\cite{hellerstein1997online}, initial results of a query are computed on a subset of required data and progressively refined as the rest of the required data arrives in a predictable pattern~\cite{tang2019intermittent,wangtempura}. Tang et al. propose a dynamic programming formulation to pick intermediate states to store in memory given a limited memory budget~\cite{tang2019intermittent}. Tempura rewrites the query plan for more efficient execution based on predicted data arrival patterns~\cite{wangtempura}. While similarities exist between the problem setting of IQP and \system, such as management of bounded memory, \system notably includes additional joint optimization for the order of MV updates.

% \paragraph{Sampling}
% Sampling has seen wide use in visualization systems for reducing the computation time of ad-hoc queries by computing an approximate result over a subset of data as exact results are not always required by the user~\cite{crotty2016case, mani2017effective, zgraggen2014panoramicdata, kraska2021northstar, galakatos2017revisiting, kandula2016quickr}. 
% Commonly studied topics in sampling for ad-hoc queries include complex query sampling~\cite{kandula2016quickr}, rare event aggregation~\cite{kraska2021northstar, galakatos2017revisiting}, and maintaining consistency between related sampled visualizations~\cite{zgraggen2014panoramicdata}.
% Sampling server-side at the MV level compromises the assumptions of downstream applications and is thus not considered in \system.

% \paragraph{Progressive visualization}
% The latency tolerance for time-consuming queries can be circumvented by presenting a partially-computed visualization to the user within the tolerance, which is then incrementally refined until it is fully accurate~\cite{rahman2017ve, zgraggen2016progressive, crotty2015vizdom, kraska2021northstar, kamat2017infiniviz}.
% Example plots which benefit from progressive visualization include bar charts~\cite{kamat2017infiniviz} and heatmaps~\cite{rahman2017ve}.
% Similar to sampling, study on this topic is orthogonal to \system as pushing out partially-updated MVs compromises downstream assumptions.
\section{Our method}
\label{sec:our_method}
In this section, we supply an overview of our method.
Next, we offer a detailed explanation that goes through the motivation and each part of our algorithm. 
Finally, we present an efficient alternative for our method.


\begin{figure*}[th]
    \centering
    \includegraphics[width=\textwidth]{images/method.pdf}
    \caption{
      \textbf{A Comparison Between the Classifier and \AlgoName Decision Rules}
    The background color of the image describes the classifier's classification rules, and the intensity describes the classifier's certainty.
    The clean image (green dot) is attacked (red dot), leading to a wrong classification.
    In contrast, \AlgoName predicts based on the shortest transformation.
    It operates in two steps: First, it class conditioned transform (dotted arrow) the attacked image towards each one of the datasets's classes (blue dots). 
    Next, prediction is made based on the shortest distance between the attacked image and the transformed images, highlighted by a red dotted circle.
    }
    \label{fig:method}
\end{figure*}

\subsection{Method Overview}
We propose a test-time classification method that enhances an AT classification model for seen and unseen attacks.
Our method does not operate in the standard classification methodology $f : x \rightarrow y$, where $f(\cdot)$ is the classifier, $x\in R^{H\text{x}W\text{x}3}$ is the input image, $y \in [0,1]^N$ is the prediction probability vector, and $N$ is the number of classes of the dataset.

Instead, our method operates through two phases, as illustrated in \cref{fig:method}: 
First, we get an input image (which can be either clean or attacked) and transform it $N$ times, each time to a different class of the dataset.
The transformation utilizes the PAG property of the AT classifier and, hence, does not require additional architecture or additional training. 
The transformation is performed through an iterative class-conditioned regularized optimization process.
For every class, we perform $M$ such optimization steps, in each step we follow two objectives.
First, maximizing the classifier probability towards the current class.
Second, regularizing the transformation so the transformed image will stay close to the input image $x$.



\begin{algorithm*}[t!]
    \caption{CODIP}
    \label{alg:CODIP}
    \hspace*{\algorithmicindent}\textbf{Input} \text{   }classifier $f(\cdot)$, input image $x$, step size $\alpha$, regularization coefficient $\gamma$, \\ \hspace*{\algorithmicindent}\hspace*{\algorithmicindent} \hspace*{\algorithmicindent} number of iterations $M$, number of dataset's classes $N$ 
    \begin{algorithmic}[1]
    \Procedure{CODIP}{}
        \For{\texttt{$i \in  C = \{1,\dots, N\}$}} \Comment{Iterate over \# classes} \label{alg:loop_over_classes}
            \State $\mathbf{T}_{i 0} \gets 0$  \Comment{Initialize the $i^{th}$ transformation} \label{alg:CODIP_init}
            \For{\texttt{$j$ in $0:M-1$}} \Comment{Iterate over \# steps}
                \State $\mathbf{G}_{ij} \gets \nabla_{\mathbf{T}_{ij}}\left[ L_{CE}\left( f \left( x+ \mathbf{T}_{ij} \right), i \right) + \gamma \, \| \mathbf{T}_{ij} \|_2^2 \right]$ \label{alg:CODIP_calc_grad}
                \State $\mathbf{T}_{ij+1} \gets \Pi \left( \mathbf{T}_{ij} - \alpha \, \frac{\mathbf{G}_{ij}}{\| \mathbf{G}_{ij} \|_2 } \right)$ \label{alg:CODIP_update_trans}
            \EndFor
        \EndFor
    \State $\mathbf{d} \gets \|\mathbf{T}_{1:N M}\|_{2}$ \Comment{Update the transformation distances for each class} \label{alg:CODIP_clac_distance}
    \State $\hat{y} = \mathrm{arg}\min_{i=1,\dots,N} \mathbf{d}_i$ \Comment{Classify based on the shortest distance} \label{alg:CODIP_prediction}
    \State \Return $\hat{y}$
    \EndProcedure
    
    \end{algorithmic}
    The operator $\Pi(z)$ projects $z$ onto the image domain $\mathbb{R}^{H \times W \times 3} \in [0,1]^{H  \times W \times 3}$ by clipping its values. 
\end{algorithm*}


\subsection{\AlgoName}
In what follows, we detail our method summarized in \cref{alg:CODIP}:
We perform $N$ transformations towards each one of the dataset's classes, and we initialize the $i^{th}$ class transformation to zero $\mathbf{T}_{i0}$, in \cref{alg:CODIP_init}. 
Next, we start the transformation, which operates through $M$ gradient descent steps. 
In each iteration, we calculate the gradient of our objective and store it in $\mathbf{G}_{ij}$ in \cref{alg:CODIP_calc_grad}. 
Next, in \cref{alg:CODIP_update_trans}, we perform a gradient descent step in the direction of $\mathbf{G}_{ij}$, the step of size $\alpha$ which is a hyper-parameter. 
It is essential to normalize the gradient step since we want to make an even progress in each step, regardless of the gradient size.
Without normalization, the step size might be very small, which will prevent us from making progress during the transformation.
Before finishing the update step, in \cref{alg:CODIP_update_trans}, we perform a projection back to the image domain to $R^{H \times W \times 3} \in [0,1]^{H \times W \times 3}$. 
When finishing the transformation's iterations, we calculate the transformation distance using $\ell_2$ over the $M^{th}$ column of the transformation matrix, and we store it in the distance vector $\mathbf{d}$, in \cref{alg:CODIP_clac_distance}. 
Finally, when the process is over we predict according to the shortest transformation. 



\begin{table*}[hb!]
\begin{center}
\vspace{0pt}
\caption{
    \textbf{CIFAR10 and CIFAR100 Results} The performance of test-time methods over two datasets: CIFAR10 and CIFAR100.  
    The results are grouped per AT model, evaluating different test-time methods: `Base' in which we do not use any test-time method, followed by other test-time methods.
    % \vspace{-30pt}
}
\resizebox{\textwidth}{!}{%

\begin{tabular}{lllllcccccc}
\hline\noalign{\smallskip}\hline

\rowcolor{gray!5}  &  &  &  &  &  & \multicolumn{4}{c}{Attack}\\
\rowcolor{gray!5} & & & & & & \multicolumn{2}{c}{$L_{\infty}$} & \multicolumn{2}{c}{$L_{2}$}\\
\rowcolor{gray!5} \multirow{-3}{*}{Dataset} & \multirow{-3}{*}{AT Method} & \multirow{-3}{*}{\makecell{Trained Threat \\ Model}} & \multirow{-3}{*}{Architecture}  & \multirow{-3}{*}{\makecell{Test-Time \\ Method}}  & \multirow{-3}{*}{Clean} & $8/255$ & $16/255$ & $0.5$ & $1.0$\\

\hline\noalign{\smallskip}\hline\noalign{\smallskip}

\multirow{17}{*}{CIFAR10} & PAT \cite{laidlaw2020perceptual} & & RN50 & & $71.60\%$ & $28.70\%$ & $-$ & $33.30\%$ & $-$\\


\cline{2-10}\noalign{\smallskip}

& \multirow{5}{*}{AT \cite{madry2017towards}} & \multirow{5}{*}{$L_{2}, \epsilon=0.5$} & \multirow{5}{*}{RN50} & Base & $90.83\%$ & $29.04\%$ & $00.93\%$ & $69.24\%$ & $36.21\%$\\

& & & &  RSmooth \cite{cohen2019certified} & $89.43\%$ &  $30.84\%$ & $01.23\%$ & $68.94\%$ & $38.53\%$\\

& & & &  TTE \cite{perez2021enhancing} & $\textbf{90.99}\%$ &  $36.41\%$ & $02.40\%$ & $71.90\%$ & $41.18\%$\\

& & & & DRQ \cite{schwinn2022improving} & $88.79\%$ & $45.37\%$ & $07.09\%$ & $77.56\%$ & $51.28\%$\\

& & & & \cellcolor{golden!10} \AlgoName  & \cellcolor{golden!10} $87.40\%$ & \cellcolor{golden!10} $\textbf{51.66\%}$ & \cellcolor{golden!10} $\textbf{14.96\%}$ & \cellcolor{golden!10} $\textbf{78.66\%}$ & \cellcolor{golden!10} $\textbf{59.82\%}$\\



\cline{2-10}\noalign{\smallskip}


& \multirow{4}{*}{Rebuffi \emph{et al.} \cite{rebuffi2021fixing}} & \multirow{4}{*}{$L_{2}, \epsilon=0.5$} & \multirow{4}{*}{WRN28-10} & Base & $\textbf{91.79\%}$ & $47.85\%$ & $05.00\%$ & $78.80\%$ & $54.73\%$\\

& & & &  TTE \cite{perez2021enhancing} & $91.59\%$ & $50.49\%$ & $06.79\%$ & $79.18\%$ & $55.38\%$\\


& & & &  DRQ \cite{schwinn2022improving} & $90.99\%$ & $58.66\%$ & $\textbf{13.69\%}$ & $84.12\%$ & $64.69\%$\\

& & & &  \cellcolor{golden!10} \AlgoName & \cellcolor{golden!10} $88.23\%$ & \cellcolor{golden!10} $\textbf{59.99\%}$ & \cellcolor{golden!10} $11.45\%$ & \cellcolor{golden!10} $\textbf{85.56\%}$ & \cellcolor{golden!10} $\textbf{66.15\%}$\\


\cline{2-10}\noalign{\smallskip}

& \multirow{4}{*}{Rebuffi \emph{et al.} \cite{rebuffi2021fixing}} & \multirow{4}{*}{$L_{\infty}, \epsilon=8/255$} & \multirow{4}{*}{WRN28-10} & Base & $\textbf{87.33\%}$ & $60.77\%$ & $25.44\%$ & $66.72\%$ & $35.01\%$\\

 & & & & TTE \cite{perez2021enhancing} & $87.30\%$ & $61.52\%$ & $27.50\%$ & $66.88\%$ & $36.07\%$\\

& & & & DRQ \cite{schwinn2022improving} & $87.17\%$ & $66.23\%$ & $33.62\%$ & $72.24\%$ & $44.56\%$\\

& & & & \cellcolor{golden!10} \AlgoName & \cellcolor{golden!10} $85.00\%$ & \cellcolor{golden!10} $\textbf{66.86\%}$ & \cellcolor{golden!10} $\textbf{34.88\%}$ & \cellcolor{golden!10} $\textbf{74.24\%}$ & \cellcolor{golden!10} $\textbf{53.02\%}$\\


\cline{2-10}\noalign{\smallskip}

& \multirow{3}{*}{Gowal \emph{et al.} \cite{gowal2020uncovering}} & \multirow{3}{*}{$L_{\infty}, \epsilon=8/255$}  & \multirow{3}{*}{WRN70-16} & Base & $\textbf{91.09\%}$ & $65.88\%$ & $25.95\%$ & $66.43\%$ & $27.21\%$\\

& & & & DRQ \cite{schwinn2022improving} & $90.77\%$ & $71.00\%$ & $35.89\%$ & $72.87\%$ & $39.51\%$\\

 & & & & \cellcolor{golden!10} \AlgoName  & \cellcolor{golden!10} $88.18\%$ & \cellcolor{golden!10} $\textbf{72.02}\%$ & \cellcolor{golden!10} $\textbf{40.30\%}$ & \cellcolor{golden!10} $\textbf{75.90\%}$ & \cellcolor{golden!10} $\textbf{49.21\%}$\\

\hline\noalign{\smallskip}



\multirow{9}{*}{CIFAR100} & \multirow{5}{*}{Rebuffi \emph{et al.} \cite{rebuffi2021fixing}} & \multirow{5}{*}{$L_{\infty}, \epsilon=8/255$} & \multirow{5}{*}{WRN28-10}  & Base & $\textbf{62.40}\%$ & $32.06\%$ & $12.47\%$ & $38.32\%$ & $18.86\%$\\

& & & & TTE \cite{perez2021enhancing} &  $62.35\%$ &  $33.25\%$ & $13.84\%$ & $39.14\%$ & $20.22\%$\\

& & & & DRQ \cite{schwinn2022improving} & $61.32\%$ & $38.22\%$ & $19.41\%$ & $44.58\%$ & $26.78\%$\\

& & & & \cellcolor{golden!10} \AlgoName & \cellcolor{golden!10} $55.18\%$ & \cellcolor{golden!10} $37.61\%$ & \cellcolor{golden!10} $19.72\%$ & \cellcolor{golden!10} $45.86\%$ & \cellcolor{golden!10} $32.79\%$\\

& & & & \cellcolor{golden!10} \AlgoNameTop  & \cellcolor{golden!10} $57.95\%$ & \cellcolor{golden!10} $\textbf{38.73\%}$ & \cellcolor{golden!10} $\textbf{19.87\%}$ & \cellcolor{golden!10} $\textbf{47.56\%}$ & \cellcolor{golden!10} $\textbf{33.01\%}$\\


\cline{2-10}\noalign{\smallskip}


& & & & Base & $\textbf{69.15\%}$ & $36.90\%$ & $13.64\%$ & $40.86\%$ & $17.20\%$\\


& & & & DRQ \cite{schwinn2022improving} & $69.12\%$ & $43.96\%$ & $20.25\%$ & $48.95\%$ & $25.43\%$\\

& & & & \cellcolor{golden!10} \AlgoName & $59.83\%$ & \cellcolor{golden!10} $44.66\%$ & \cellcolor{golden!10} $\textbf{23.81\%}$ & \cellcolor{golden!10} $51.32\%$ & \cellcolor{golden!10} $\textbf{37.02\%}$\\

& \multirow{-4}{*}{Gowal \emph{et al.} \cite{gowal2020uncovering}}  & \multirow{-4}{*}{$L_{\infty}, \epsilon=8/255$} & \multirow{-4}{*}{WRN70-16}  & \cellcolor{golden!10} \AlgoNameTop  & \cellcolor{golden!10} $62.47\%$ & \cellcolor{golden!10} $\textbf{46.09\%}$ & \cellcolor{golden!10} $23.48\%$ & \cellcolor{golden!10} $\textbf{53.06\%}$ & \cellcolor{golden!10} $36.19\%$ \\



\hline\noalign{\smallskip} \hline\noalign{\smallskip}



\end{tabular}
}
% \vspace{-20pt}
\label{table:cifar10_and_cifar100}
\end{center}
\end{table*}


\begin{figure}[h!]
    
  \begin{center}
    \includegraphics[width=0.4\textwidth]{images/gamma_effect.pdf}
  \end{center}
  \vspace{-15pt}
  \caption{
    \textbf{Impact of $\gamma$ on Clean-Robust Accuracy Trade-off } We present three $\alpha$ working points on the ImageNet dataset using an AT model $L_2,\epsilon = 3.0$.
    }
  \label{fig:gamma_tradeoff}
\end{figure}

\AlgoNameNoSpace's objective, which appears in \cref{alg:CODIP_calc_grad}, weights two targets: 
The first term, $L_{CE} \left(f \left(x + \mathbf{T}_{ij}\right), i \right)$, is a cross entropy loss between the classifier's prediction $f(\cdot)$ over the $j^{th}$ step of the $i^{th}$ transformation $x + \mathbf{T}_{ij}$, and the current class $i$. 
The goal of this term is to measure the classifier's error towards class $i$ and to transform the image so it semantically looks like class $i$.
The goal of the second term, $\gamma \| \mathbf{T}_{ij} \|_2 $, is to regularize the transformation to stay close to the input image $x$. 
This regularization is needed since we desire to change the input image toward class $i$ with minimal semantic changes.
Without this term, the transformation could turn the image into any instance of class $i$. 
Even in the case where $i$ is the true class, we can change the image into a completely different instance of the class, ultimately nullifying the efficacy of our method.
To conclude, balancing between the two terms, which is determined by the hyper-parameter $\gamma$, influences the class-similarity transformation balance, hence, it is essential.%for the success.% of the classification.
We demonstrate the impact of changing this value in \cref{fig:gamma_tradeoff}, and discuss it further in Appendix H.









\subsection{\AlgoNameTopNoSpace}
While \AlgoName is an effective algorithm that significantly enhances the classifier robustness, its inference time grows with the number of datasets' classes. 
To mitigate this issue, we propose \AlgoNameTopNoSpace, which reduces the inference time significantly by filtering out the classes with the lowest probability.


The classifier predicts the probability of the input image belonging to each class $f(x) \rightarrow \mathbb{R}^N \in [0,1]^N$. 
One way to speed up our algorithm is by limiting our prediction only to the most probable classes predicted by the classifier.
More specifically, we choose a group of $k$ classes $C_k \subset C=\{1, \dots, N\}$ containing the top-$k$ most probable prediction made by the classifier.
Next, we modify our method, which is presented in \cref{alg:CODIP}, by adapting \cref{alg:loop_over_classes} to transform only towards the most probable classes $C_k$.



\section{Experimental Results}
\label{sec:experimental_results}

This section describes the experimental validations on the effectiveness and reliability of \ourmodel. First, we describe the model setup in Sec.~\ref{sec:experiment_setups}. Sec.~\ref{sec:single_attr_diagnosis} and Sec.~\ref{sec:validation_diagnosis} visualize and validate the model diagnosis results for the single-attribute setting. In Sec.~\ref{sec:multiple_attr_diagnosis}, we show results on synthesized multiple-attribute counterfactual images and apply them to counterfactual training.

\subsection{Model Setup}
\label{sec:experiment_setups}
{\bf Pre-trained models:} We used Stylegan2-ADA \cite{Karras2020ada} pretrained on FFHQ \cite{2019stylegan} and AFHQ \cite{choi2020starganv2} as our base generative networks, and the pre-trained CLIP model \cite{CLIP}  which is parameterized by ViT-B/32. We followed StyleCLIP \cite{2021StyleCLIP} setups to compute the channel relevance matrices $\mathcal{M}$.

{\bf Target models:} Our classifier models are ResNet50 with single fully-connected head initialized by TorchVision\footnote{https://pytorch.org/blog/how-to-train-state-of-the-art-models-using-torchvision-latest-primitives/}. In training the binary classifiers, we use the Adam optimizer with learning rate 0.001 and batch size 128. We train binary classifiers for \textit{Eyeglasses, Perceived Gender, Mustache, Perceived Age} attributes on CelebA and for \textit{cat/dog} classification on AFHQ. For the 98-keypoint detectors, we used the HR-Net architecture~\cite{WangSCJDZLMTWLX19} on WFLW~\cite{wayne2018lab}. %Unless explicitly mentioned, our approach samples 1000 images from StyleGAN for each diagnosis by histogram.

\subsection{Visual Model Diagnosis: Single-Attribute}
\label{sec:single_attr_diagnosis}
Understanding where deep learning model fails is
an essential step towards building trustworthy models. Our proposed work allows us to generate counterfactual images (Sec.~\ref{sec:Counterfactual_Synthesis}) and provide insights on sensitivities of the target model (Sec.~\ref{sec:Attribute_Sensitivity_Analysis}). This section visualizes the counterfactual images in which only one attribute is modified at a time. 

Fig. \ref{fig:age_classifier_single} shows the single-attribute counterfactual images. Interestingly (but not unexpectedly), 
we can see that reducing the hair length or joyfulness causes the age classifier more likely to label the face to an older person. Note that our approach is extendable to multiple domains, as we change the cat-like pupil to dog-like, the dog-cat classification tends towards the dog. 
Using the counterfactual images, we can conduct model diagnosis with the method mentioned in Sec.~\ref{sec:Attribute_Sensitivity_Analysis}, on which attributes the model is sensitive to. In the histogram generated in model diagnosis, a higher bar means the model is more sensitive toward the corresponding attribute.

Fig.~\ref{fig:histograms_vanilla} shows the model diagnosis histograms on regularly-trained classifiers. For instance, the cat/dog classifier histogram shows outstanding sensitivity to green eyes and vertical pupil.
The analysis is intuitive since these are cat-biased traits rarely observed in dog photos. Moreover, the histogram of eyeglasses classifier shows that the mutation on bushy eyebrows is more influential for flipping the model prediction. 
It potentially reveals the positional correlation between eyeglasses and bushy eyebrows. The advantage of single-attribute model diagnosis is that the score of each attribute in the histogram are independent from other attributes, enabling unambiguous understanding of exact semantics that make the model fail. Diagnosis results for additional target models can be found in Appendix B.

\subsection{Validation of Visual Model Diagnosis} 
\label{sec:validation_diagnosis}
Evaluating whether our zero-shot sensitivity histograms (Fig.~\ref{fig:histograms}) explain the true vulnerability is a difficult task, since we do not have access to a sufficiently large and balanced test set fully annotated in an open-vocabulary setting. To verify the performance, we create synthetically imbalanced cases where the model bias is known. We then compare our results with a supervised diagnosis setting~\cite{sia}. In addition, we will validate the decoupling of the attributes by CLIP. 

\vspace{-2mm}
\subsubsection{Creating imbalanced classifiers}
\label{sec:creating_imbalance_classifiers}
\vspace{-1mm}
In order to evaluate whether our sensitivity histogram is correct, we train classifiers that are highly imbalanced towards a known attribute and see whether \ourmodel can detect the sensitivity w.r.t the attribute. For instance, when training the perceived-age classifier (binarized as Young in CelebA), we created a dataset on which the trained classifier is strongly sensitive to Bangs (hair over forehead). The custom dataset is a CelebA training subset that consists of $20,200$ images. More specifically, there are $10,000$ images that have both young people that have bangs, represented as $(1,1)$, 
and $10,000$ images of people that are not young and have no bangs, represented as $(0,0)$. The remaining combinations of $(1,0)$ and $(0,1)$ have only 100 images.
With this imbalanced dataset, bangs is the attribute that dominantly correlates with whether the person is young, and hence the perceived-age classifier would be highly sensitive towards bangs.
% will learn that bangs is the most sensitive attribute to predict age. 
See Fig.~\ref{fig:histogram_attgan} (the right histograms) for an illustration of the sensitivity histogram computed by our method for the case of an age classifier with bangs (top) and lipstick (bottom) being imbalanced. 
\begin{figure}[t]
    \begin{subfigure}[b]{\linewidth}
        \label{fig:histogram_attgan_1}
         \centering
         \includegraphics[width=\linewidth]{images/histograms/attgan_histogram_1.pdf}\\
    \end{subfigure}
    \begin{subfigure}[b]{\linewidth}
    \label{fig:histogram_attgan_2}
         \includegraphics[width=\linewidth]{images/histograms/attgan_histogram_2.pdf}
    \end{subfigure}
        \vspace{-6mm}
         \caption{ The sensitivity of the age classifier is evaluated with \ourmodel (right) and AttGAN (left), achieving comparable results. }
         \label{fig:histogram_attgan}
         \vspace{-1mm}
    %  \end{subfigure}
\end{figure}

 We trained multiple imbalanced classifiers with this methodology,  and visualize the model diagnosis histograms of these imbalanced classifiers in Fig.~\ref{fig:histograms_unbalanced}. We can observe that the \ourmodel histograms successfully detect the synthetically-made bias, which are shown as the highest bars in histograms. See the caption for more information. 

\begin{figure}[t]
    \begin{subfigure}[b]{0.49\linewidth}
        \centering
        \includegraphics[width=\linewidth]{images/matrix/confusion-matrix-Mustache.pdf}
        \caption{Mustache classifier}
        \label{fig:matrix_CLIP_Score_a}
    \end{subfigure}
    \begin{subfigure}[b]{0.49\linewidth}
        \centering
        \includegraphics[width=\linewidth]{images/matrix/confusion-matrix-Young.pdf}
        \caption{Perceived age classifier}
        \label{fig:matrix_CLIP_Score_b}
    \end{subfigure}
    \vspace{-2mm}
    \caption{Confusion matrix of CLIP score variation (vertical axis) when perturbing attributes (horizontal axis). This shows that attributes in \ourmodel are highly decoupled. }
    \label{fig:matrix_CLIP_Score}
    \vspace{-3mm}
\end{figure}

\begin{figure*}[ht]
    \centering
    \includegraphics[width=\linewidth]{images/multi_attr_human.pdf}
    \caption{Multi-attribute counterfactual in faces. The model probability is documented in the upper right corner of each image.}
    \label{fig:human_classifier_multiattr}
    \vspace{-4mm}
\end{figure*}

\vspace{-2mm}
\subsubsection{Comparison with supervised diagnosis}
\vspace{-1mm}
We also validated our histogram by comparing it with the case in which we have access to a generative model that has been explicitly trained to disentangle attributes.  We follow the work on~\cite{sia} and used AttGAN~\cite{attGAN} trained on the CelebA training set over $15$ attributes\footnote{\textit{Bald, Bangs, Black\_Hair, Blond\_Hair, Brown\_Hair, Bushy\_Eyebrows, Eyeglasses, Male, Mouth\_Slightly\_Open, Mustache, No\_Beard, Pale\_Skin, Young, Smiling, Wearing\_Lipstick.}}.
After the training converged, we performed adversarial learning in the attribute space of AttGAN and create a sensitivity histogram using the same approach as Sec.~\ref{sec:Attribute_Sensitivity_Analysis}. Fig.~\ref{fig:histogram_attgan} shows the result of this method on the perceived-age classifier which is made biased towards bangs.  As anticipated, the AttGAN histogram (left) corroborates the histogram derived from our method (right). Interestingly, unlike \ourmodel, AttGAN show less sensitivity to remaining attributes. This is likely because AttGAN has a latent space learned in a supervised manner and hence attributes are better disentangled than with StyleGAN. Note that AttGAN is trained with a fixed set of attributes; if a new attribute of interest is introduced, the dataset needs to be re-labeled and AttGAN retrained. ZOOM, however, merely calls for the addition of a new text prompt.  More results in Appendix B.

\vspace{-2mm}
\subsubsection{Measuring disentanglement of attributes}
\vspace{-1mm}
Previous works demonstrated that the StyleGAN's latent space can be entangled~\cite{interfacegan, EditinginStyle}, adding undesired dependencies when searching single-attribute counterfactuals. This section verifies that our framework can disentangle the attributes and mostly edit the desirable attributes.

We use CLIP as a super annotator to measure attribute changes during single-attribute modifications. For $1,000$ images, we record the attribute change after performing adversarial learning in each attribute, and average the attribute score change. Fig.~\ref{fig:matrix_CLIP_Score} shows the confusion matrix during single-attribute counterfactual synthesis. The horizontal axis is the attribute being edited during the optimization, and the vertical axis represents the CLIP prediction changed by the process. For instance, the first column of Fig.~\ref{fig:matrix_CLIP_Score_a} is generated when we optimize over bangs for the mustache classifier. We record the CLIP prediction variation. It clearly shows that bangs is the dominant attribute changing during the optimization. From the main diagonal of matrices, it is evident that the \ourmodel mostly perturbs the attribute of interest. The results indicate reasonable disentanglement among attributes.



\subsection{Visual Model Diagnosis: Multi-Attributes}
\label{sec:multiple_attr_diagnosis}
In the previous sections, we have visualized and validated single-attribute model diagnosis histograms and counterfactual images. 
In this section, we will assess \ourmodel's ability to produce counterfactual images by concurrently exploring multiple attributes within $\mathcal{A}$, the domain of user-defined attributes.  The approach conducts multi-attribute counterfactual searches across various edit directions, identifying distinct semantic combinations that result in the target model's failure. By doing so, we can effectively create more powerful counterfactuals images (see Fig.~\ref{fig:multiple_attribute_is_more_powerful}).


\begin{figure}[t]
    \centering
    \includegraphics[width=\linewidth]{images/multi_attr_dog_cut.pdf}
    \caption{Multi-attribute counterfactual on Cat/Dog classifier. The number in each image is the predicted probability of being a dog.}
    \label{fig:dog_classifier_multiattr}
    \vspace{-2mm}
\end{figure}

\begin{figure}[t]
    \centering
    \includegraphics[width=\linewidth]{images/multi_attr_is_more_powerful/multi_attr_is_more_powerful_2.pdf}\\
    \vspace{-1mm}
    \includegraphics[width=\linewidth]{images/multi_attr_is_more_powerful/multi_attr_is_more_powerful_1.pdf}
    \vspace{-8mm}
    \caption{ Multiple-Attribute Counterfactual (MAC, Sec.~\ref{sec:multiple_attr_diagnosis}) compared with Single-Attribute Counterfactual (SAC, Sec.~\ref{sec:single_attr_diagnosis}). We can see that optimization along multiple directions enable the generation of more powerful counterfactuals.}
    \label{fig:multiple_attribute_is_more_powerful}
    \vspace{-4mm}
\end{figure}

Fig.~\ref{fig:human_classifier_multiattr} and Fig.~\ref{fig:dog_classifier_multiattr} show examples of multi-attribute counterfactual
images generated by \ourmodel, against human and animal face classifiers. 
It can be observed that multiple face attributes such as lipsticks or hair color are edited in Fig.~\ref{fig:human_classifier_multiattr}, and various cat/dog attributes like nose pinkness, eye shape, and fur patterns are edited in Fig.~\ref{fig:dog_classifier_multiattr}. 
These attribute edits are blended to affect the target model prediction. Appendix B further illustrates \ourmodel counterfactual images for semantic segmentation, multi-class classification, and a church classifier. By mutating semantic representations, \ourmodel reveals semantic combinations as outliers where the target model underfits.


In the following sections, we 
will use the Flip Rate (the percentage of counterfactuals that flipped the model prediction) and Flip Resistance (the percentage of counterfactuals for which the model successfully withheld its prediction) to evaluate the multi-attribute setting. 
\begin{figure}[t]
    \centering
    \begin{subfigure}[b]{\linewidth}
    \includegraphics[width=0.495\linewidth]{images/histograms/multi_attr_eyeglasses.pdf}
    \includegraphics[width=0.495\linewidth]{images/histograms/multi_attr_age_biased_beard.pdf}
    \caption{Sensitivity histograms generated by \ourmodel on attribute combinations.}
    \label{fig:histograms_combination}
    \end{subfigure}\\
    \begin{subfigure}[b]{\linewidth}
    \includegraphics[width=\linewidth]{images/histograms/grand_histogram.pdf}
    \caption{Model diagnosis by \ourmodel over $19$ attributes. Our framework is generalizable to analyze facial attributes of various domains.}
    \label{fig:histograms_grand}
    \end{subfigure}
    \vspace{-6mm}
    \caption{Customizing attribute space for \ourmodel.}
    \label{fig:multiple_attribute_histogram}
    \vspace{-4mm}
\end{figure}
\vspace{-3mm}
\subsubsection{Customizing attribute space}
\vspace{-2mm}
\looseness=-1

In some circumstances,  users may finish one round of model diagnosis and proceed to another round by adding new attributes, or trying a new attribute space.
The linear nature of attribute editing (Eq.~\ref{eq:total_edit}) in \ourmodel makes it possible to easily add or remove attributes. 
Table~\ref{tab:model_flip_rate} shows the flip rates results when adding new attributes into $\mathcal{A}$ for perceived age classifier and big lips classifier.  We can observe that a different attribute space will results in different effectiveness of counterfactual images. Also, increasing the search iteration will make counterfactual more effective (see last row). 
 Note that neither re-training the StyleGAN nor user-collection/labeling of data is required at any point in this procedure.  Moreover, Fig.~\ref{fig:histograms_combination} shows the model diagnosis histograms generated with combinations of two attributes. Fig.~\ref{fig:histograms_grand} demonstrates the capability of \ourmodel in a rich vocabulary setting where we can analyze attributes that are not labeled in existing datasets~\cite{liu2015celeba,MAAD}.
 
\vspace{-4mm}
\subsubsection{Counterfactual training results}
\label{sec:ct_result}
\vspace{-1mm}


This section evaluates regular classifiers trained on CelebA~\cite{liu2015celeba} and counterfactually-trained (CT) classifiers on a mix of CelebA data and counterfactual images as described in Sec.~\ref{sec:ct}. Table \ref{tab:ct_acc_table} presents accuracy and flip resistance (FR) results. CT outperforms the regular classifier. FR is assessed over 10,000 counterfactual images, with FR-25 and FR-100 denoting Flip Resistance after 25 and 100 optimization iterations, respectively. Both use $\eta=0.2$ and $\epsilon=30$. We can observe that the classifiers after CT are way less likely to be flipped by counterfactual images while maintaining a decent accuracy on the CalebA testset. Our approach robustifies the model by increasing the tolerance toward counterfactuals. Note that CT slightly improves the CelebA classifier when trained on a mixture of CelebA images (original images) and the counterfactual images generated with a generative model  trained in the FFHQ~\cite{2019stylegan} images (different domain).  


\begin{table}[t]
  \centering
  \footnotesize
  \begin{tabular}{@{}lccc@{}}
     \toprule
     Method & \makecell{AC Flip Rate (\%)} & \makecell{BC Flip Rate (\%)} \\
     \midrule
     Initialize \ourmodel by $\mathcal{A}$                        & 61.95 &  83.47\\
     + Attribute: Beard                                           &  72.08 & 90.07\\
     + Attribute: Smiling                                        &  87.47 &  \textbf{96.27}\\
     + Attribute: Lipstick                                         &  90.96 &  94.07\\
     + Iterations increased to 200                                &  \textbf{92.91} &  94.87\\
     \bottomrule
  \end{tabular}
  \caption{\label{tab:model_flip_rate} Model flip rate study. The initial attribute space $\mathcal{A} =$ \{Bangs, Blond Hair, Bushy Eyebrows, Pale Skin, Pointy Nose\}. AC is the perceived age classifier and BC is the big lips classifier.} 
  \vspace{-3mm}
\end{table}


\begin{table}[t]
    \centering
    \footnotesize
    \begin{tabular}{ccccc}
        \toprule
         Attribute & \makecell{Metric} & \makecell{Regular (\%)} & \makecell{CT (Ours) (\%)} \\

\midrule
        \multirow{3}{*}{Perceived Age} & CelebA Accuracy   & 86.10 & \textbf{86.29}   \\
        & \ourmodel FR-25  & 19.54 & \textbf{97.36}  \\
        & \ourmodel FR-100  & 9.04 & \textbf{95.65}  \\
        \midrule
        \multirow{3}{*}{Big Lips} & CelebA Accuracy   & 74.36 & \textbf{75.39}    \\
        & \ourmodel FR-25  & 14.12 & \textbf{99.19}  \\
        & \ourmodel FR-100  & 5.93 & \textbf{88.91}  \\
        \bottomrule
    \end{tabular}
    \caption{\label{tab:ct_acc_table} Results of network inference on CelebA original images and \ourmodel-generated counterfactual. The CT classifier is significantly less prone to be flipped by counterfactual images, while test accuracy on CelebA remains performant.}
    \vspace{-6mm}
\end{table}

\vspace{-2mm}

\section{Conclusion and Discussion} \label{conclusion_and_future}
\looseness=-1
\vspace{-2mm}

In this paper, we present \ourmodel, a zero-shot model diagnosis framework that generates sensitivity histograms based on 
user's input of natural language attributes. 
\ourmodel assesses failures and generates corresponding sensitivity histograms for each attribute.  A significant advantage
of our technique is its ability to analyze the failures of a target deep model without the need for laborious collection and annotation of test sets. \ourmodel effectively visualizes the correlation between attributes and model outputs, elucidating model behaviors and intrinsic biases.

Our work has three primary limitations. First, users should possess domain knowledge as their input (text of attributes of interest) should be relevant to the target domain.  Recall that it is a small price to pay for model evaluation without an annotated test set. Second, StyleGAN2-ADA struggles with generating out-of-domain samples. Nevertheless, our adversarial learning framework can be adapted to other generative models (e.g., stable diffusion), and the generator can be improved by training on more images. We have rigorously tested our generator with various user inputs, confirming its effectiveness for regular diagnosis requests. Currently, we are exploring stable diffusion models to generate a broader range of classes while maintaining the core concept. Finally, we rely on a pre-trained model like CLIP which we presume to be free of bias and capable of encompassing all relevant attributes.

{\bf Acknowledgements: }We would like to thank George Cazenavette, Tianyuan Zhang, Yinong Wang, Hanzhe Hu, Bharath Raj for suggestions in the presentation and experiments. We sincerely thank Ken Ziyu Liu, Jiashun Wang, Bowen Li, and Ce Zheng for revisions to improve this work.

%------------------------------------------------------------------------

%%%%%%%%% REFERENCES
{\small
\bibliographystyle{ieee_fullname}
\bibliography{PaperForArXiv}
}

% %------------------------------------------------------------------------
\clearpage
\newpage
\appendix
\section{Validation for CLIP-guided Editing}
\label{sec:appendix_sectiona}

Our methodology relies on CLIP-guided fine-grained image editing to provide adequate model diagnostics. It is critical to verify CLIP's ability to link language and visual representations. This section introduces two techniques for validating CLIP's capabilities.


\subsection{Visualization for edited images}
In this section, we analyze the decoupling of attribute editing used in StyleCLIP~\cite{2021StyleCLIP} in our domain. 


\textbf{Effect of $\lambda$.} Fig.~\ref{fig:appendix_lambda} shows the effect of $\lambda$ in Equation 2 of the main text~\cite{2021StyleCLIP} . 
Originally in StyleCLIP, this filter parameter (denoted as $\beta$ in~\cite{2021StyleCLIP}) helps the style disentanglement for editing. 
As we have normalized the edit vectors, which contributes to disentanglement in our framework, the impact of $\lambda$ on style disentanglement is reduced. Consequently, $\lambda$ primarily influences intensity control and denoising.


\textbf{Single-attribute editing.} Fig.~\ref{fig:appendix_attribute_editing_afhq} and Fig.~\ref{fig:appendix_attribute_editing_ffhq} show a set of images of different object categories by editing different attributes extracted with the global edit directions method (as described in Section 3.2 of the main text). By analyzing the user's input attribute string, we can see that the modified image only alters in the attribute direction while maintaining the other attributes. 

\textbf{Multiple-attribute editing.} 
We demonstrate the validity of our method for editing multiple attributes through linear combination (as outlined in Equation 3 of the main text) by presenting illustrations of combined edits in Figure ~\ref{fig:appendix_interpolation}.


\subsection{User study for edited images}
To validate that our counterfactual image synthesis preserve fine-grained details and authenticity, we conducted a user study validating two aspects: synthesis fidelity and attribute consistency. 

\textbf{User study for synthesis fidelity.} 
The classification of the counterfactual synthesis image vs real images by the user is employed to confirm that no unrealistic artifacts are introduced throughout the process of our model Fig.~\ref{fig:user_study_visual_fidelity} shows sample questions of this study.  In theory, the worst-case scenario is that users can accurately identify the semantic modification and achieve a user recognition rate of $100\%$. Conversely, the best-case scenario would be that users are unable to identify any counterfactual synthesis and make random guesses, resulting in a user recognition rate of $50\%$.


\textbf{User study for attribute consistency.} We ask users whether they agree that the counterfactual and original images are consistent on the ground truth w.r.t. the target classifier. For example, during the counterfactual synthesis for the cat/dog classifier, a counterfactual cat image should stay consistent as a cat. Fig.~\ref{fig:user_study_attribute_consistency} shows another sample questions. The worst case is that the counterfactual changes the ground truth label to affect the target model, which makes the user agreement rate very low (even to zero). 

The user study statistics are presented in Table ~\ref{tab:user_study}. The study involved $34$ participants with at least an undergraduate level of education, who were divided into two groups using separate collector links. The participants themselves randomly selected their group (i.e., the link clicked), and their responses were collected.

\begin{figure*}[t]
    \centering
    \includegraphics[width=0.49\linewidth]{images/single_attr_dog_2.pdf}
    \includegraphics[width=0.49\linewidth]{images/single_attr_Male.pdf}
    \vspace{-3mm}
    \caption{Effect of progressively generating counterfactual images on the Cat/Dog classifier (0-Cat / 1-Dog), and the Perceived Gender classifier (0-Female / 1-Male). Model probability prediction during the process is attached at the top right corner.}
    \label{fig:appendix_classifier_single}
\end{figure*}

The production of high-quality counterfactual images is supported by the difficulty users had in differentiating them. Additionally, the majority of users concurred that the counterfactual images do not change the ground truth concerning the target classifier, confirming that our methodology generates meaningful counterfactuals. However, it should be noted that due to the nature of our recognition system, human volunteers are somewhat more responsive to human faces. As a result, we observed a slightly higher recognition rate in the human face (FFHQ) domain than in the animal face (AFHQ) domain.

\subsection{Stability across CLIP phrasing/wording:} 

It is worth noting that the resulting counterfactual image is affected by the wording of the prompt used. In our framework, we subtract the neutral phrase (such as "a face") after encoding in CLIP space to ensure that the attribute edit direction is unambiguous enough. Our experimentation has shown that as long as the prompt accurately describes the object, comparable outcomes can be achieved. For instance, we obtained similar sensitivity results on the perceived-age classifier using prompts like "a picture of a person with X," "a portrait of a person with X," or other synonyms. Examples of this are presented in Figure ~\ref{fig:stability_histogram}.


\begin{figure*}[t]
  \centering
  % \fbox{\rule{0pt}{0.5in} \rule{0.9\linewidth}{0pt}}
  \includegraphics[width=0.9\linewidth]{Rebuttal_Images/rebuttal_more_models.pdf}
   \caption{\ourmodel counterfactuals on more tasks (segmentation, multi-class classifier) and additional visual domains (cars, churches). Zoom in for better visibility.}
   \label{fig:more_models}
   \vspace{-3mm}
\end{figure*}

\begin{table}[h]
   \centering
   \small	
   \begin{tabular}{@{}lccc@{}}
     \toprule
    \multicolumn{1}{c}{Name of Study} & Domain & Group 1    & Group 2\\
     \midrule
    \multicolumn{1}{c}{\multirow{2}{*}{\makecell{Synthesis Fidelity (\\Recognition Rate $\downarrow$, \%)}}} & FFHQ &62.12 & 71.79 \\
                                                                & AFHQ & 51.30 & 50.55  \\
                        \midrule
     \multicolumn{1}{c}{\multirow{2}{*}{\makecell{Attribute Consistency (\\Agreement Rate $\uparrow$, \%)}}} & FFHQ &94.12 & 90.76  \\
                                                          & AFHQ & 89.92 & 88.26 \\


     \bottomrule
   \end{tabular}
   %\end{adjustbox}
   \caption{User study results. We can see from the table that our counterfactual synthesis preserves the visual quality and maintains the ground truth labels from the user's perspective.}
   \label{tab:user_study}
   \vspace{-3mm}
\end{table}
% Appendix for Histograms and Counterfactual demo
\section{Additional Results of Model Diagnosis}

\subsection{Additional counterfactual images}
Fig.~\ref{fig:appendix_classifier_single} shows more examples of single-attribute counterfactual images on the Cat/Dog and Perceived Gender classifiers. The output prediction is shown in the top-right corner. It shows that the model prediction is flipped without changing the actual target attribute. In addition to binary classification and key-point detection in our manuscript, we further illustrate the extension of \ourmodel counterfactuals on semantic segmentation, multi-class classification, and binary church classifier (BCC) in Fig.~\ref{fig:more_models}. Fig.~\ref{fig:appendix_classifier_multi} shows more examples of multiple-attribute counterfactual images.

\subsection{Additional histograms}
Fig.~\ref{fig:apendix_histograms} shows more histograms on the classifiers trained on CelebA (top) and the classifiers that are intentionally biased  (bottom). The models and datasets are created using the same method described in Section 4 of the main text.
% Details of using other optimization appraoches (e.g., linear approximation
\section{Ablation of the Adversarial Optimization Method}
When there are multiple attributes (i.e., $N>1$) to optimize, linearizing the cost function as grid in high dimensional space will help to efficiently approximate convergence in limited epochs. Specifically, we have the option to adopt PGD \cite{madry2018towards} (i.e., update using $\eta \cdot\operatorname{sign} (\nabla_{\mathbf{w}} \mathcal{L})$) for efficient optimization. We compared generating counterfactuals with and without  projected gradients. Table~\ref{tab:appendix_pgd} shows the visual quality and flip rate of the generated counterfactuals. We can observe that \ourmodel-PGD image quality is finer under Structured Similarity Indexing Method (SSIM) \cite{SSIM}, while \ourmodel-SGD has a higher flip rate. The images from \ourmodel-PGD is finer since the signed method stabilizes the optimization by eliminating problems of gradient vanishing and exploding. 

 \begin{table}[h]
   \small
   \centering
   \begin{tabular}{@{}llcc@{}}
     \toprule
    \multicolumn{1}{c}{Optimization} & Classifier &  SSIM ($\uparrow$) & Flip Rate (\%, $\uparrow$)\\
     \midrule
     \multicolumn{1}{c}{\multirow{3}{*}{SGD}} & Perceived Age & 0.5732 & 67.24  \\
                                              & Perceived Gender & 0.5815 & 49.40 \\
                                              & Mustache & 0.5971 & 36.33 \\
                        \midrule
     \multicolumn{1}{c}{\multirow{3}{*}{PGD}} & Perceived Age & 0.8065 & 50.19\\
                                              & Perceived Gender & 0.7035 & 42.84 \\
                                              & Mustache & 0.7613 & 25.10 \\
     \bottomrule
   \end{tabular}
   %\end{adjustbox}
   \caption{The comparison of counterfactuals generated with stochastic gradient descent (SGD) and projected gradient descent (PGD) method. We can observe that \ourmodel-PGD image quality is finer under SSIM (Structured Similarity Indexing Method) \cite{SSIM} metrics, while \ourmodel-SGD has a higher flip rate. }
   \label{tab:appendix_pgd}
\end{table}

Our empirical observation during the experiment is that \ourmodel-PGD frequently oscillates around a local minima of edit weights and fails to reach an optimal counterfactual. We hypothesize that the reason of lower flip rates from the signed method is that the edit weight search is constrained on nodes of a grid space (the grid unit length is step-size $\eta$), which loses precision and underperforms during counterfactual search. 



\begin{figure*}[t]
  \centering
  % \fbox{\rule{0pt}{0.5in} \rule{0.9\linewidth}{0pt}}
  \includegraphics[width=\linewidth]{Rebuttal_Images/grand_histogram_rebuttal.pdf}
  \vspace{-8mm}
   \caption{ Sensitivity histograms when using four instances of phrases with a similar concept. Zoom in for better visibility.}
   \label{fig:stability_histogram}
   \vspace{-4mm}
\end{figure*}

\begin{figure*}[t]
    \centering
    \includegraphics[width=0.33\linewidth]{Appendix_Images/Appendix_B/histograms/attractiveness.pdf}
    \includegraphics[width=0.33\linewidth]{Appendix_Images/Appendix_B/histograms/lipstick.pdf}
    \includegraphics[width=0.33\linewidth]{Appendix_Images/Appendix_B/histograms/gender.pdf}\\
    \includegraphics[width=0.33\linewidth]{Appendix_Images/Appendix_B/histograms/blond_hair_biased_pale_skin.pdf}
    \includegraphics[width=0.33\linewidth]{Appendix_Images/Appendix_B/histograms/gender_biased_pale_skin.pdf}
    \includegraphics[width=0.33\linewidth]{Appendix_Images/Appendix_B/histograms/gender_biased_smiling.pdf}
    % \includegraphics[width=0.24\linewidth]{Appendix_Images/Appendix_B/histograms/smiling.pdf}
    \vspace{-5mm}
    \caption{The above histograms show \ourmodel on three regularly trained classifiers on CelebA, and the bottom histograms show \ourmodel successfully detects the bias in the manually-crafted imbalanced classifiers.}
    \label{fig:apendix_histograms}
    \vspace{-1mm}
\end{figure*}

\begin{figure*}[h]
    \centering
    \begin{subfigure}[t]{0.37\linewidth}
        \centering
        \captionsetup{margin={4pt,4pt}}
        \includegraphics[width=\linewidth]{Appendix_Images/Appendix_A/user_study_visual_fidelity.pdf}
        \caption{Evaluating visual fidelity. We show two images and let users choose the one that they think is edited. The counterfactuals are generated on Eyeglasses classifier and Cat/Dog classifier. }
        \label{fig:user_study_visual_fidelity}
    \end{subfigure}
    \begin{subfigure}[t]{0.32\linewidth}
        \centering
        \captionsetup{margin={4pt,4pt}}
        \includegraphics[width=\linewidth]{Appendix_Images/Appendix_A/user_study_attribute_consistency.pdf}
        \caption{Evaluating attribute consistency. The user classifies whether the ground truth is flipped. Example of counterfactual images on Cat/Dog classifier and Eyeglasses classifier is shown above.}
        \label{fig:user_study_attribute_consistency}
    \end{subfigure}
\caption{Sample questions in the user study. Each user answers $10$ questions for each of the two user studies. }
\end{figure*}

\begin{figure*}[!t]
\vspace{-3mm}
    \centering
    \begin{subfigure}[b]{\linewidth}
    \includegraphics[width=\linewidth]{Appendix_Images/Appendix_B/multi_dog.pdf}
    \caption{Multiple-attribute counterfactual for cat/dog classifier.}
    \end{subfigure}\\
    \begin{subfigure}[b]{\linewidth}
    \includegraphics[width=\linewidth]{Appendix_Images/Appendix_B/multi_Eyeglasses.pdf}
    \caption{Multiple-attribute counterfactual for eyeglasses classifier.}
    \end{subfigure}\\
    \begin{subfigure}[b]{\linewidth}
    \includegraphics[width=\linewidth]{Appendix_Images/Appendix_B/multi_Male.pdf}
    \caption{Multiple-attribute counterfactual for perceived gender classifier.}
    \end{subfigure}\\
    \begin{subfigure}[b]{\linewidth}
    \includegraphics[width=\linewidth]{Appendix_Images/Appendix_B/multi_Mustache.pdf}
    \caption{Multiple-attribute counterfactual for mustache classifier.}
    \end{subfigure}\\
    \begin{subfigure}[b]{\linewidth}
    \includegraphics[width=\linewidth]{Appendix_Images/Appendix_B/multi_Young.pdf}
    \caption{Multiple-attribute counterfactual for perceived age classifier.}
    \end{subfigure}\\
    \caption{Multi-attribute counterfactual in the human face and animal face domain. The right-up corner of each image records the model output prediction.}
    \label{fig:appendix_classifier_multi}
\end{figure*}

\begin{figure*}[h]
    \centering
    \captionsetup{justification=centering}
    \begin{subfigure}[t]{0.60\linewidth}
        \includegraphics[width=\linewidth]{Appendix_Images/Appendix_A/lambda/lambda_beard_2.pdf}
    \caption{Effect of $\lambda$ values for editing beard.}
    \label{fig:appendix_lambda_1}
    \end{subfigure}\\
    \begin{subfigure}[t]{0.60\linewidth}
        \centering
        \captionsetup{justification=centering}
        \includegraphics[width=\linewidth]{Appendix_Images/Appendix_A/lambda/lambda_pale_skin_20.pdf}
        \caption{Effect of $\lambda$ values for editing pale skin.}
        \label{fig:appendix_lambda_2}
    \end{subfigure}
\caption{Visualization of the effect of different $\lambda$ values.}
\label{fig:appendix_lambda}
\end{figure*}


% Appendix B


\begin{figure*}[t]
\centering
    \begin{subfigure}[b]{0.48\linewidth}
        \centering
        \includegraphics[width=\linewidth]{Appendix_Images/Appendix_A/a_cat_with_green_eyes.pdf}
        \caption{Attribute editing: a cat with green eyes.}
    \end{subfigure}
    \begin{subfigure}[b]{0.48\linewidth}
        \centering
        \includegraphics[width=\linewidth]{Appendix_Images/Appendix_A/a_cute_cat.pdf}
        \caption{Attribute editing: a cute cat.}
    \end{subfigure}\\
    \begin{subfigure}[b]{0.48\linewidth}
        \centering
        \includegraphics[width=\linewidth]{Appendix_Images/Appendix_A/a_dog_with_round_face.pdf}
        \caption{Attribute editing: a dog with round face.}
    \end{subfigure}
    \begin{subfigure}[b]{0.48\linewidth}
        \centering
        \includegraphics[width=\linewidth]{Appendix_Images/Appendix_A/a_cute_dog.pdf}
        \caption{Attribute editing: a cute dog.}
    \end{subfigure}\\
    \begin{subfigure}[b]{0.48\linewidth}
        \centering
        \includegraphics[width=\linewidth]{Appendix_Images/Appendix_A/a_cat_with_round_face.pdf}
        \caption{Attribute editing: a cat with round face.}
    \end{subfigure}
    \begin{subfigure}[b]{0.48\linewidth}
        \centering
        \includegraphics[width=\linewidth]{Appendix_Images/Appendix_A/a_cat_with_pointed_ears.pdf}
        \caption{Attribute editing: a cat with pointed ears.}
    \end{subfigure}\\
    \begin{subfigure}[b]{0.48\linewidth}
        \centering
        \includegraphics[width=\linewidth]{Appendix_Images/Appendix_A/a_dog_with_open_mouth.pdf}
        \caption{Attribute editing: a dog with open mouth.}
    \end{subfigure}
    \begin{subfigure}[b]{0.48\linewidth}
        \centering
        \includegraphics[width=\linewidth]{Appendix_Images/Appendix_A/a_Black_dog.pdf}
        \caption{Attribute editing: a black dog.}
    \end{subfigure}
    
    \caption{Visualization of global edit directions by utilizing the StyleCLIP channel relevance matrix. Images are sampled from the AFHQ domain using StyleGAN2-ADA. Every column demonstrates an edited image from edit weight $w = -30$ to $w = 30$. Weights of five images are linearly interpolated as $\{-30, -15, 0, 15, 30\}$. We can see that global edit directions are generalizable on multiple images.}
    \label{fig:appendix_attribute_editing_afhq}
\end{figure*}


\begin{figure*}[h]
    \centering
    \begin{subfigure}[b]{0.48\linewidth}
        \label{fig:appendix_A_visual_an_Angry_face}
        \centering
        \includegraphics[width=\linewidth]{Appendix_Images/Appendix_A/an_Angry_face.pdf}
        \caption{Attribute editing: an angry face.}
    \end{subfigure}
    \begin{subfigure}[b]{0.48\linewidth}
        \label{fig:appendix_A_visual_a_face_with_Eyeglasses}
        \centering
        \includegraphics[width=\linewidth]{Appendix_Images/Appendix_A/a_face_with_Eyeglasses.pdf}
        \caption{Attribute editing: a face with eyeglasses.}
    \end{subfigure}
    \begin{subfigure}[b]{0.48\linewidth}
        \label{fig:appendix_A_visual_a_Cute_face}
        \centering
        \includegraphics[width=\linewidth]{Appendix_Images/Appendix_A/a_Cute_face.pdf}
        \caption{Attribute editing: a cute face.}
    \end{subfigure}
    \begin{subfigure}[b]{0.48\linewidth}
        \label{fig:appendix_A_visual_a_face_with_Blond_Hair}
        \centering
        \includegraphics[width=\linewidth]{Appendix_Images/Appendix_A/a_face_with_Blond_Hair.pdf}
        \caption{Attribute editing: a face with blond hair.}
    \end{subfigure}\\
    \begin{subfigure}[b]{0.48\linewidth}
        \label{fig:appendix_A_visual_a_face_with_Bangs}
        \centering
        \includegraphics[width=\linewidth]{Appendix_Images/Appendix_A/a_face_with_Bangs.pdf}
        \caption{Attribute editing: a face with bangs.}
    \end{subfigure}
    \begin{subfigure}[b]{0.48\linewidth}
        \label{fig:appendix_A_visual_a_Smiling_face}
        \centering
        \includegraphics[width=\linewidth]{Appendix_Images/Appendix_A/a_Smiling_face.pdf}
        \caption{Attribute editing: a smiling face.}
    \end{subfigure}\\
    \begin{subfigure}[b]{0.48\linewidth}
        \label{fig:appendix_A_visual_a_Happy_face}
        \centering
        \includegraphics[width=\linewidth]{Appendix_Images/Appendix_A/a_Happy_face.pdf}
        \caption{Attribute editing: a happy face.}
    \end{subfigure}
    \begin{subfigure}[b]{0.48\linewidth}
        \label{fig:appendix_A_visual_a_face_with_Curly_Hair}
        \centering
        \includegraphics[width=\linewidth]{Appendix_Images/Appendix_A/a_face_with_Curly_Hair.pdf}
        \caption{Attribute editing: a face with curly hair.}
    \end{subfigure}
    \end{figure*}

    \begin{figure*}[h]
    \centering
    \ContinuedFloat 
    \begin{subfigure}[b]{0.48\linewidth}
        \label{fig:appendix_A_visual_a_face_with_Beard}
        \centering
        \includegraphics[width=\linewidth]{Appendix_Images/Appendix_A/a_face_with_Beard.pdf}
        \caption{Attribute editing: a face with beard.}
    \end{subfigure}
    \begin{subfigure}[b]{0.48\linewidth}
        \label{fig:appendix_A_visual_a_face_with_Lipstick}
        \centering
        \includegraphics[width=\linewidth]{Appendix_Images/Appendix_A/a_face_with_Lipstick.pdf}
        \caption{Attribute editing: a face with lipstick.}
    \end{subfigure}\\
    \begin{subfigure}[b]{0.48\linewidth}
        \label{fig:appendix_A_visual_a_Male_face}
        \centering
        \includegraphics[width=\linewidth]{Appendix_Images/Appendix_A/a_Tired_face.pdf}
        \caption{Attribute editing: a tired face.}
    \end{subfigure}
    \begin{subfigure}[b]{0.48\linewidth}
        \label{fig:appendix_A_visual_a_Surprised_face}
        \centering
        \includegraphics[width=\linewidth]{Appendix_Images/Appendix_A/a_Skinny_face.pdf}
        \caption{Attribute editing: a skinny face.}
    \end{subfigure}\\
    \begin{subfigure}[b]{0.48\linewidth}
        \label{fig:appendix_A_visual_a_Male_face}
        \centering
        \includegraphics[width=\linewidth]{Appendix_Images/Appendix_A/a_Male_face.pdf}
        \caption{Attribute editing: a male face.}
    \end{subfigure}
    \begin{subfigure}[b]{0.48\linewidth}
        \label{fig:appendix_A_visual_a_Surprised_face}
        \centering
        \includegraphics[width=\linewidth]{Appendix_Images/Appendix_A/a_Surprised_face.pdf}
        \caption{Attribute editing: a surprised face.}
    \end{subfigure}\\
    \begin{subfigure}[b]{0.48\linewidth}
        \label{fig:appendix_A_visual_a_face_with_Long_Hair}
        \centering
        \includegraphics[width=\linewidth]{Appendix_Images/Appendix_A/a_face_with_Long_Hair.pdf}
        \caption{Attribute editing: a face with long hair.}
    \end{subfigure}
    \begin{subfigure}[b]{0.48\linewidth}
        \label{fig:appendix_A_visual_a_face_with_Pale_Skin}
        \centering
        \includegraphics[width=\linewidth]{Appendix_Images/Appendix_A/a_face_with_Pale_Skin.pdf}
        \caption{Attribute editing: a face with pale skin.}
    \end{subfigure}
    \caption{Visualization of global edit directions by utilizing the StyleCLIP channel relevance matrix. Images are sampled from the FFHQ domain using StyleGAN2-ADA. Every column demonstrates an edited image from edit weight $w = -30$ to $w = 30$. Weights of five images are linearly interpolated as $\{-30, -15, 0, 15, 30\}$. We can see that global edit directions are generalizable on multiple images.}
    \label{fig:appendix_attribute_editing_ffhq}
    
\end{figure*}



\begin{figure*}[h]
    \centering
    \begin{subfigure}[b]{\linewidth}
        \includegraphics[width=0.49\linewidth]{Appendix_Images/Appendix_A/interpolation/smiling_lipstick_0.pdf}
        \includegraphics[width=0.49\linewidth]{Appendix_Images/Appendix_A/interpolation/smiling_lipstick_1.pdf}
        \caption{Combination of smiling ($w_1$) and lipstick ($w_2$).}
        \label{fig:interpolation_similing_lipstick}
    \end{subfigure}
    \begin{subfigure}[b]{\linewidth}
        \includegraphics[width=0.49\linewidth]{Appendix_Images/Appendix_A/interpolation/paleskin_blondhair_0.pdf}
        \includegraphics[width=0.49\linewidth]{Appendix_Images/Appendix_A/interpolation/paleskin_blondhair_1.pdf}
        \caption{Combination of pale skin ($w_1$) and blond hair ($w_2$).}
        \label{fig:interpolation_paleskin_blondhair}
    \end{subfigure}
    \caption{Visualization of traversing on directional (attribute) style vectors to validate the effectiveness of multiple attribute editing.}
    \label{fig:appendix_interpolation}
\end{figure*}

\end{document}
