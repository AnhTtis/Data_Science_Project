\section{Conclusion}
We have presented the potential of employing comments to better form visual-lingual understanding on social media, where 27M tweets with comments are contributed for the related study.
A novel framework is proposed to retrieve comments from similar posts and explore comments' hinting capabilities via self-training. 
Experimental results on four social media benchmarks show the universal benefit of leveraging retrieved comments and conduct comment-aware self-training on various multi-modal classification tasks and architectures. 

%proposed a retrieval-based comment-aware self-training framework for social media multimodal classification, where the retrieved comments are utilized to bridge the semantic gap between the images and texts while the retrieved semantically similar posts are employed in the self-training framework to solve the limitation of the scale of labeled data and improve the performance. Additionally, about 27M multimodal social meida tweets with comments are collected. Extensive experimental results on four multimodal social media tasks have shown the effectiveness of the proposed method.