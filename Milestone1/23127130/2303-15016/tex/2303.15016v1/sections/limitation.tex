\section*{Limitations}
Here we point out more limitations in addition to what we have discussed in $\S$\ref{ssec:exp:qualitative}.

First, the post and comment retrieval should be built upon a large-scale corpus (wild dataset).
Substantial efforts might be needed to gather a likewise corpus, if applying our work to other a different social media platform. 
However, for our follow-up work exploring Twitter as well, the pre-trained retrieval module, based on the Faiss library, can work in an efficient manner.
%Although the retrieval efficiency is acceptable. 
For example, we test the retrieval module to search similar posts for 5,000 cross-media posts by using the Faiss library on one single 2080Ti GPU, and it would cost 231.66s for image modality and 77.68s for text modality.\footnote{The reason of different retrieval time is due to the different dimension of extracted features (i.e., the dimension is 2048 for image feature while 768 for text)} %Therefore, the retrieval efficiency is acceptable benefit from the Faiss library.

%The main limitation is the 
Second, the time and size of the retrieval corpus would result in another limitation.  
%constructed retrieval dataset. 
As shown in Table \ref{tab:retrieval_dataset_analysis}, we build the dataset with multimodal tweets 
%we collect the multimodal tweet data 
posted from 2014 to 2019. 
While a timely update might be needed if the task requires fresher data, e.g., the research of COVID-19 because the event becomes trendy in 2020.
%This means that no similar data might be retrieved from the retrieval dataset if the time of query data are not included in 2014 and 2019. 
%For example, it's impossible to retrieve the multimodal data about the COVID-19 due to the time limitation. 
%Additionally, not enough similar data can be retrieved for some unique multimodal tweets. 
%To solve the problem, the direct and simple way is to expand the size of the retrieval dataset and update the dataset with recent multimodal tweet data.
Nevertheless, dynamic dataset update might also explosively scale up the data quantity, and how to enable feasible real-time dataset management calls for another research question, which is beyond the scope of this paper and is valuable to be explored in future studies.

% Another limitation is the retrieval efficiency. We test the retrieval module to search similar posts for 5,000 cross-media posts by using the Faiss library on one single 2080Ti GPU, and it would cost 231.66s for image modality and 77.68s for text modality \footnote{The reason of different retrieval time is due to the different dimension of extracted features (i.e., the dimension is 2048 for image feature while 768 for text)}. Therefore, the retrieval efficiency is acceptable benefit from the Faiss library.