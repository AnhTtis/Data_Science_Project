\pdfoutput=1
\documentclass[11pt]{article}

% Remove the "review" option to generate the final version.
% \usepackage[review]{acl}
\usepackage{EMNLP2022}

\usepackage{times}
\usepackage{latexsym}
\usepackage{graphicx}

\usepackage[T1]{fontenc}
\usepackage{booktabs}

\usepackage[utf8]{inputenc}
\urlstyle{same}
\usepackage{hyperref}
\usepackage{microtype}
\usepackage{multirow}
\newcommand{\jing}[1]{[\textcolor{red}{JL: #1}]}
\newcommand{\xu}[1]{[\textcolor{blue}{xu: #1}]}
\usepackage{tablefootnote}
\usepackage{inconsolata}
\title{
Borrowing Human Senses: Comment-Aware Self-Training for \\Social Media Multimodal Classification
%Self-Training with Retrieved Comments for Social Media Multimodal Classification
%\jing{Learning from Readers: Social Media Multimodal Classification via Comment Retrieval and Self-Training}
}


\author{Chunpu Xu, Jing Li\thanks{~~~Corresponding author}
\\ % All authors must be in the same font size and format. Use \Large and \textbf to achieve this result when breaking a line
Department of Computing, The Hong Kong Polytechnic University, China\\
\texttt{chun-pu.xu@connect.polyu.hk}\\ \texttt{jing-amelia.li@polyu.edu.hk}}


% \author{First Author \\
%   Affiliation / Address line 1 \\
%   Affiliation / Address line 2 \\
%   Affiliation / Address line 3 \\
%   \texttt{email@domain} \\\And
%   Second Author \\
%   Affiliation / Address line 1 \\
%   Affiliation / Address line 2 \\
%   Affiliation / Address line 3 \\
%   \texttt{email@domain} \\}

\begin{document}
\maketitle
\begin{abstract}
%Millions of cross-media posts consisting of image-text pairs are created on social platform every day to convey users' ideas and feelings.
%\jing{Social media is daily exhibiting huge volume of cross-media posts in the form of  image-text pairs.}
% Thus, amounts of multimodal classification tasks are derived to analyse social media for different applications.
% \jing{Social media is daily exhibiting massive multimedia content with pair-wised image and text.
% It hence presents the pressing need to automate the vision and language understanding with various multimodal classification tasks.}
Social media is daily creating massive multimedia content with 
paired image and text, presenting the pressing need to automate the vision and language understanding for various multimodal classification tasks.
%The advance of multimodal classification would crucially automate the vision and language understanding 
%on social media exhibiting massive multimedia content every day.
% However, different from traditional  multimodal classification tasks (i.g., visual entailment and visual question answering) where strong semantic interrelation are in the image-text pair, weak semantic correlation are contained in the cross-media posts. 
% Most existing methods focus on exploring the cross-modal interactions, despite of their implicit and intricate nature on social media.
% As a result, it's hard for the multimodal model to learn the interactions between images and texts since the content of images and texts are not aligned.  
% To tackle the problem, we use the comments obtained from similar posts as the bridge to connect the image modality and text modality.
Compared to the commonly researched visual-lingual data, social media posts tend to exhibit more implicit image-text relations.
To better glue the cross-modal semantics therein, 
% \jing{which tend to be implicit on social media}, 
%which tend to be implicit on social media, 
we capture hinting features from user comments, which are retrieved via jointly leveraging visual and lingual similarity.
% Additionally, the size of most datasets of multimodal social media tasks are small due to the expensive annotation cost. To alleviate the problem, we present a self-training method, which employs pseudo-labeled data constructed from the  retrieved similar image-text pairs to improve learning. 
Afterwards, the classification tasks are explored via self-training in a teacher-student framework,
%based on pseudo-labeling, 
motivated by the usually limited labeled data scales in existing benchmarks.  
%of labeled multimodal data for social media tasks. 
%multimodal classification. 
%labeled data in most multimodal social media datasets.
%We evaluate our approach 
Substantial experiments are conducted on four multimodal social media benchmarks for image-text relation classification, 
%multimodal 
sarcasm detection, %multimodal 
sentiment classification, and 
%multimodal 
hate speech detection.
%In the experiments, 
The results show that our method further advances the performance of
%enables better results of 
previous state-of-the-art models, which do not employ comment modeling or self-training.
\end{abstract}


\section{Introduction}
\label{sec:introduction}
% \begin{itemize}
%     % Diffusion of FL
%     \item {\st{Diffusion of FL}}
%     % Security threats to FL
%     \item {\st{Security threats to FL with particular focus on model poisoning}}
%     % Limitations of existing countermeasures
%     \item {\st{Current countermeasures (e.g., KRUM) and their limitations}}
%     % Proposed method and its advantages
%     \item {\st{Intuitive description of the proposed method and its difference (i.e., advantages) w.r.t. state of the art}}
%     % Main contributions
%     \item {\st{Summary of the main contributions of this work}}
%     % Paper's structure and organization
%     \item {\st{Paper's structure and organization}}
% \end{itemize}

% Diffusion of FL
Recently, {\em federated learning} (FL) has emerged as the leading paradigm for training distributed, large-scale, and privacy-preserving machine learning (ML) systems~\cite{mcmahan2017googleai,mcmahan2017aistats}. 
The core idea of FL is to allow multiple edge clients to collaboratively train a shared, global model without disclosing their local private training data.
%Specifically, an FL system consists of a central server and many edge clients; 
A typical FL round involves the following steps: {\em(i)} the server randomly picks some clients and sends them the current, global model; {\em(ii)} each selected client locally trains its model with its own private data; then, it sends the resulting local model to the server;\footnote{Whenever we refer to global/local model, we mean global/local model {\em parameters}.} {\em(iii)} the server updates the global model by computing an \emph{aggregation function}, usually the average (FedAvg), on the local models received from clients.
% \begin{enumerate}
%     \item[{\em(i)}] the server sends the current, global model to the clients and appoints some of them for training;
%     \item[{\em(ii)}] each selected client locally trains its copy of the global model with its own private data; then, it sends the resulting local model back to the server;\footnote{Whenever we refer to global/local model, we mean global/local model {\em parameters}.}
%     \item[{\em(iii)}] the server updates the global model by computing an \emph{aggregation function} on the local models received from clients (by default, the average, also referred to as FedAvg~\cite{mcmahan2017aistats}).
% \end{enumerate}
This process goes on until the global model converges. %(e.g., after a certain number of rounds or other similar stopping criteria).
%\\
% The advantages of FL over the traditional, centralized learning paradigm are undoubtedly clear in terms of flexibility/scalability (clients can join/disconnect from the FL network dynamically), network communications (only model weights\footnote{We will use \textit{parameters} and \textit{weights} interchangeably.} are exchanged between clients and server), and privacy (each client's private training data is kept local at the client's end and not uploaded to the server).
\\
% Security threats to FL
%However, the growing adoption of FL also raises security concerns~\cite{costa2022covert}, particularly about its confidentiality, integrity, and availability.
Although its advantages over standard ML, FL also raises security concerns~\cite{costa2022covert}. %, particularly about its confidentiality, integrity, and availability~\cite{costa2022covert}.
% OLD, LONG VERSION
% Indeed, some work deals with privacy leakage that may expose the local data of some clients~\cite{melis2019sp}. 
% A large body of work, instead, investigates attacks that usually aim to detriment the predictive accuracy of the learned global model. For instance, \emph{data poisoning} attacks achieve this goal by letting an adversary pollute the training set of some corrupt FL clients with maliciously crafted examples~\cite{jagielski2018sp}.
% Similarly, in \emph{model poisoning} the attacker attempts to tweak the global model weights~\cite{bhagoji2019pmlr} by directly perturbing the local model's weights of some infected FL clients before these are sent to the central server for aggregation, usually via so-called Byzantine attacks. 
% It turns out that Byzantine model poisoning attacks severely impact standard FedAvg; therefore, more robust aggregation functions must be designed to make FL systems secure.
Here, we focus on \emph{untargeted model poisoning} attacks~\cite{bhagoji2019pmlr}, where an adversary attempts to tweak the global model weights %\footnote{We will use the terms \textit{parameters} and \textit{weights} interchangeably.} 
by directly perturbing the local model's parameters of some infected clients before these are sent to the central server for aggregation.
In doing so, the adversary aims to jeopardize the global model \textit{indiscriminately} at inference time.
Such model poisoning attacks severely impact standard FedAvg; therefore, more robust aggregation functions must be designed to secure FL systems.
\\
% In this paper, we focus on designing a novel robust aggregation scheme at the server's end to contrast the effect of Byzantine model poisoning attacks.
%
% Current countermeasures and their limitations
%Several countermeasures have been proposed in the literature to combat model poisoning attacks on FL systems.
% Some methods use simple statistics more robust than plain average to smooth the impact of malicious updates (e.g., Trimmed Mean and FedMedian~\cite{yin2018icml}). 
% Other defenses implement outlier detection techniques to discard malicious updates from the aggregation performed at the server's end. Those are either based on heuristics (e.g., Krum/Multi-Krum~\cite{blanchard2017nips} and Bulyan~\cite{mhamdi2018pmlr}) or data-driven approaches (e.g., K-means clustering~\cite{shen2016acm} or DnC via spectral analysis~\cite{shejwalkar2021ndss}). 
% Finally, some strategies rely on a centralized ``source of trust'' to spot potential malicious updates (e.g., FLTrust~\cite{cao2020fltrust}).
% Several countermeasures have been proposed in the literature to combat model poisoning attacks on FL systems, i.e., to discard possible malicious local updates from the aggregation performed at the server's end. 
% These techniques range from simple statistics more robust than plain average (e.g., Trimmed Mean and FedMedian~\cite{yin2018icml}) to outlier detection heuristics (e.g., Krum/Multi-Krum~\cite{blanchard2017nips} and Bulyan~\cite{mhamdi2018pmlr}) or data-driven approaches (e.g., spectral analysis via K-means clustering~\cite{shen2016acm} or spectral analysis), or methods based on ``source of trust'' (e.g., FLTrust~\cite{cao2020fltrust}).
% OLD, LONG VERSION
%Several countermeasures have been proposed in the literature to combat Byzantine model poisoning attacks on FL systems.
% Descriptive statistics
% For example, Trimmed Mean and FedMedian aggregate local model updates using more robust statistics than standard average~\cite{yin2018icml}.
%
% % Heuristics for outlier detection
% Many existing Byzantine-resilient strategies implement some outlier detection heuristics to discard the model updates sent by potentially malicious clients from the input of the aggregation function.
% One of the most popular heuristics is Krum~\cite{blanchard2017nips}.
% This strategy tries to mitigate the impact of Byzantine attacks by selecting as a global model the local model with the smallest sum of Euclidean distances to {\em all} the other local models.
% Although powerful, Krum requires the server to know (or, at least, estimate) the number of malicious FL clients upfront, which is generally impossible in a realistic attack scenario. %
% Moreover, Krum may become ineffective for complex, high-dimensional model parameter spaces due to the curse of dimensionality.
% Bulyan~\cite{mhamdi2018pmlr} tries to overcome this issue by combining Krum with a variant of Trimmed Mean.
% % Data-driven outlier detection
% Other strategies use data-driven outlier detection techniques -- e.g., via K-means clustering~\cite{shen2016acm} -- to spot potential malicious local model updates. 
% %For instance, Shen et al. propose to cluster local model updates with K-means and thus identify outliers.
%
% % Other techniques
% As far as the server is concerned, any local model received can be from a potential malicious client. 
% FLTrust~\cite{cao2020fltrust} assumes the server acts as a client, i.e., trains a local model on an additional {\em trustworthy} dataset at the server's end and compares it against all the local models from other clients. 
% This way, the server can rely on some ``source of trust'' when discarding potentially malicious clients.
%\\
% Limitations of existing Byzantine-resilient strategies
Unfortunately, existing defense mechanisms either rely on simple heuristics (e.g., Trimmed Mean and FedMedian by~\cite{yin2018icml}) or need strong and unrealistic assumptions to work effectively (e.g., foreknowledge or estimation of the number of malicious clients in the FL system, as for Krum/Multi-Krum~\cite{blanchard2017nips} and Bulyan~\cite{mhamdi2018pmlr}, which, however, cannot exceed a fixed threshold).
Furthermore, outlier detection methods using K-means clustering~\cite{shen2016acm} or spectral analysis like DnC~\cite{shejwalkar2021ndss} do not directly consider the temporal evolution of local model updates received.
Finally, strategies like FLTrust~\cite{cao2020fltrust} require the server to collect its own dataset and act as a proper client, thereby altering the standard FL protocol.
\\
% OLD, LONG VERSION
% Overall, existing Byzantine-resilient strategies are either simple heuristics (e.g., FedMedian) or, if they are more complex, they rely on strong and unrealistic assumptions to work effectively (e.g., knowing the number of malicious clients in the FL system in advance, as for Krum and alike).
% Furthermore, data-driven outlier detection methods do not consider the temporary evolution of local model updates received (e.g., K-means clustering). 
% Finally, strategies like FLTrust requires the server to collect its own dataset and act as a proper client, thereby altering the standard FL protocol.
%
% Description of the proposed method
This work introduces a novel pre-aggregation \textit{filter} robust to untargeted model poisoning attacks. Notably, this filter $(i)$ operates without requiring prior knowledge or constraints on the number of malicious clients and $(ii)$ inherently integrates temporal dependencies. 
The FL server can employ this filter as a preprocessing step before applying \textit{any} aggregation function, be it standard like FedAvg or robust like Krum or Bulyan.
Specifically, we formulate the problem of identifying corrupted updates as a multidimensional (i.e., matrix-valued) time series anomaly detection task. 
The key idea is that legitimate local updates, resulting from well-calibrated iterative procedures like stochastic gradient descent (SGD) with an appropriate learning rate, show \textit{higher predictability} compared to malicious updates. This hypothesis stems from the fact that the sequence of gradients (thus, model parameters) observed during legitimate training exhibit regular patterns, as validated in Section~\ref{subsec:intuition}. %until convergence. 
%This regularity may be more pronounced for smooth convex loss functions, but it can still be captured within an appropriate time window, even for more complex and convoluted loss surfaces. 
%We provide evidence of this claim in Appendix~B, where we show that the average mutual information (i.e., ``predictability''), calculated over pairs of legitimate model updates sent at different FL rounds, is significantly higher than the corresponding computation for a malicious client.
\\
Inspired by the matrix autoregressive (MAR) framework for multidimensional time series forecasting~\cite{chen2021je}, we propose the FLANDERS ({\em \textbf{F}ederated \textbf{L}earning meets \textbf{AN}omaly \textbf{DE}tection for a \textbf{R}obust and \textbf{S}ecure}) filter.
The main advantages of FLANDERS over existing strategies like FLDetector~\cite{zhao2020multivariate} are its resilience to large-scale attacks, where $50\%$ or more FL participants are hostile, and the capability of working under realistic non-iid scenarios.
We attribute such a capability to two key factors: $(i)$ FLANDERS works without knowing a priori the ratio of corrupted clients, and $(ii)$ it embodies temporal dependencies between intra- and inter-client updates, quickly recognizing local model drifts caused by evil players. Below, we summarize our main contributions:

\begin{itemize}
\item[{\em(i)}]
We provide empirical evidence that the sequence of models sent by legitimate clients is more predictable than those of malicious participants performing untargeted model poisoning attacks.
\\
\item[{\em(ii)}] 
We introduce FLANDERS, the first pre-aggregation filter for FL robust to untargeted model poisoning based on multidimensional time series anomaly detection.
\\
\item[{\em(iii)}] 
We integrate FLANDERS into Flower,\footnote{\scriptsize{\url{https://flower.dev/}}} a popular FL simulation framework for reproducibility.
\\
\item[{\em(iv)}] 
We show that FLANDERS improves the robustness of the existing aggregation methods under multiple settings: different datasets, client's data distribution (non-iid), models, and attack scenarios.
\\
\item[{\em(v)}] 
We publicly release all the implementation code of FLANDERS along with our experiments.\footnote{\scriptsize{\url{https://anonymous.4open.science/r/flanders_exp-7EEB}}}
\end{itemize}

% Paper's structure and organization
The remainder of the paper is structured as follows. %some related work and the current state-of-the-art solutions to security issues that FL entails. 
Section~\ref{sec:background} covers background and preliminaries. 
In Section~\ref{sec:related}, we discuss related work.
Section~\ref{sec:problem} and Section~\ref{sec:method} describe the problem formulation and the method proposed. % to tackle it. 
Section~\ref{sec:experiments} gathers experimental results. %, and Section~\ref{sec:limitations} discusses some limitations of this work.
Finally, we conclude in Section~\ref{sec:conclusion}.
 %discusses the limitations of this work and draws future research directions.
%reports conclusions and draws perspectives for future research directions.

%%%%%%% OLD %%%%%%%
%to overcome the resilience of Byzantine failures in distributed Stochastic Gradient Descent computations. 
% The strength of Krum is its time complexity, which is linear in the gradient dimension. 
% However, the robustness of the approach is guaranteed for gradient-based learning applications only when the majority of the clients are not compromised. 
% Besides, the aggregation mechanism of Krum, as well as that of similar methods, is robust from a coarse-grained perspective and does not provide solutions to errors and perturbations that may occur at inference time.
%A related approach to~\cite{blanchard2017nips} is the work of Su et al.~\cite{su2016dc}. Here, the authors propose an iterated approximate agreement to tackle a multi-layer scenario attacked by Byzantine agents. 
%However, the method works efficiently on the sole discrete context and it is inapplicable to continuous state environments.
%\gabri{Maybe, we should just talk about the main limitations of existing countermeasures without digging into their details (or, we can just mention Krum as this is the most popular one). I will move the description of all these methods to the Related Work section.}

\section{Related work}
% There is extensive recent work on speeding up analytical queries due to the need for consistent execution times in the face of the explosive growth in the volume of available data.
% In this section, we divide existing work into two categories: maintaining data freshness in MVs (\Cref{sec:server_side}) and optimizations for minimizing ad-hoc query latency (\Cref{sec:client_side}).

% \subsection{Maintaining Data Freshness in MVs}
% \label{sec:server_side}
% There exists a variety of data warehousing applications aimed at supporting low-latency analytical queries on fresh data.
% In particular, these applications require efficiency in the propagation of newly ingested data into downstream MVs.
 
\mypara{Efficient MV Refresh}
Incremental view maintenance (IVM) aims to update MVs to reflect newly ingested data, taking advantage of already computed results to perform the update in a manner more efficient than computing from scratch (full refresh)
~\cite{ahmad2012dbtoaster,mcsherry2013differential,armbrust2013generalized,zeng2016iolap, palpanas2002incremental, griffin1995incremental, agiwal2021napa, braun2015analytics}. 
There is an abundance of work in IVM, including incremental updates on duplicate values~\cite{griffin1995incremental}, non-distributive aggregate functions~\cite{palpanas2002incremental}, higher-order views~\cite{ahmad2012dbtoaster}, and sliding windows~\cite{braun2015analytics}. 
More recent works also investigate the scalability aspect of IVM, proposing scale-independent updates~\cite{armbrust2013generalized} and sampled views~\cite{zeng2016iolap}. Since \system is applicable to arbitrary SQL statements, \system is orthogonal to and is fully compatible with existing IVM techniques.

\mypara{MV Refresh Scheduling}
There exist works on scheduling the refresh of a MV set focusing on resolving cyclic dependencies~\cite{folkert2005optimizing}, minimizing weighted average staleness~\cite{golab2009scheduling}, and prioritizing MVs with the highest speedups on predicted future queries~\cite{ahmed2020automated}.
\system's scheduling to speed up the end-to-end refresh of the MV set is not addressed in existing works.

\mypara{DAG Workflow Scheduling}
The execution of workloads consisting of individual jobs with acyclic dependencies is a well-studied topic~\cite{apacheoozie,sparkdag,marchal2018parallel,bathie2020revisiting,baruah2022ilp}; many of these techniques can be applied to MV refresh runs studied in this paper.
Existing workflow scheduling systems such as Apache Oozie~\cite{apacheoozie}, Apache Airflow~\cite{airflow}, and Spark DAG scheduler~\cite{sparkdag} automate the execution of user-defined workflows following a topological order.
There are recent works aimed at finding more optimal execution orders in terms of peak memory usage~\cite{marchal2018parallel, bathie2020revisiting} and execution time on parallel platforms~\cite{baruah2022ilp}.
While \system is designed for use with MV refresh runs/workloads, our technique on joint scheduling and optimization can be reasonably applied to general workloads as a possible future direction.

% \paragraph{Incremental MV indexing}
% Update-optimized indices such as the log-structured merge-trees (LSM)~\cite{o1996log} are used for indexing MVs due to frequent updates induced by data ingestion~\cite{gupta2016mesa,agiwal2021napa}.
% \system is orthogonal to indexing: \system is capable of efficiently performing MV refresh runs regardless of whether the individual MVs are indexed or not.

% \subsection{Ad-hoc Query Latency Reduction}
% \label{sec:client_side}

% The minimization of ad-hoc analytical query response times is a well-studied topic due to latency being negatively correlated with the productivity of a data analyst during a data analysis session~\cite{liu2014effects}.
% Sessions are commonly conducted within visualization systems that contain a variety of optimization techniques to ensure that query response times fall within a certain latency tolerance.

% \mypara{Data prefetching}
% Data is often loaded into memory on a by-need basis in visualization systems to minimize interference with user-issued query computations~\cite{mani2017effective,xin2021enhancing,galakatos2017revisiting, yan2020auto, battle2016dynamic, crotty2016case, jalaparti2018netco}. 
% Query-time data retrieval can be significantly expedited by anticipating the data usage of the user in future queries and pre-loading the data into memory during the downtime between user queries (`think time'). SMART~\cite{mani2017effective} prefetches data for modified versions of current user-issued queries with different filters and dimensions. A-WARE~\cite{crotty2016case} maintains a sample store constantly refined through ingesting data based on speculations of future plots.
% ForeCache~\cite{battle2016dynamic} uses an SVM to predict the user's current analysis phase and accordingly prefetches data tiles partitioned based on different numerical values. NetCo predicts future queries via log analysis, and solves an ILP formulation to prefetch data to maximize the number of SLO-meeting queries~\cite{jalaparti2018netco}.
% In the case of MV refresh workloads, `think time' is nonexistent as individual MVs are refreshed back-to-back, rendering data prefetching techniques non-applicable.

\mypara{Intermediate Data Caching}
Some existing data visualization systems cache user-defined variables to support the typical incremental construction of data visualizations~\cite{zgraggen2016progressive, eichmann2020idebench} during data analysis sessions~\cite{jupyter, rstudio, colab}. 
Recent work proposes a management scheme for these cached variables under a memory constraint that greedily keeps variables with the highest estimated time savings based on predicted future user behavior via neural networks~\cite{xin2021enhancing}.
While useful for data visualization, a greedy approach to memory management fails to achieve satisfactory results compared to \system.

\mypara{Intermediate Result Reuse}

There exist works on storing intermediate results from computations to speedup future computations~\cite{yang2018intermediate, dursun2017revisiting, nagel2013recycling, michiardi2019memory, galakatos2017revisiting}.
Studied topics include the identification of reuse opportunities by finding overlaps in computation graphs of successive jobs~\cite{yang2018intermediate, michiardi2019memory},
selective storage under a space constraint with heuristics such as reuse probability~\cite{dursun2017revisiting}, expected savings~\cite{yang2018intermediate}, and recompute-storage cost difference~\cite{nagel2013recycling},
and rewriting incoming jobs to take advantage of stored intermediates~\cite{galakatos2017revisiting}.
These works share similarity with \system in their selection of items to store under a memory constraint, however, \system's problem setting requires it to uniquely consider the joint (re)ordering of job executions along with the selection of items.

% work that considers both job execution (re)order as well as intermediate result caching with a bounded amount of memory. but notably lack the joint aspect of \system and cannot be used to achieve immediate speedup on an incoming MV refresh run if no intermediates are stored beforehand. 

\mypara{Incremental Query Processing} Incremental processing (IQP) is useful for cases where not all data required for a query is immediately available. Similar to online aggregation~\cite{hellerstein1997online}, initial results of a query are computed on a subset of required data and progressively refined as the rest of the required data arrives in a predictable pattern~\cite{tang2019intermittent,wangtempura}. Tang et al. propose a dynamic programming formulation to pick intermediate states to store in memory given a limited memory budget~\cite{tang2019intermittent}. Tempura rewrites the query plan for more efficient execution based on predicted data arrival patterns~\cite{wangtempura}. While similarities exist between the problem setting of IQP and \system, such as management of bounded memory, \system notably includes additional joint optimization for the order of MV updates.

% \paragraph{Sampling}
% Sampling has seen wide use in visualization systems for reducing the computation time of ad-hoc queries by computing an approximate result over a subset of data as exact results are not always required by the user~\cite{crotty2016case, mani2017effective, zgraggen2014panoramicdata, kraska2021northstar, galakatos2017revisiting, kandula2016quickr}. 
% Commonly studied topics in sampling for ad-hoc queries include complex query sampling~\cite{kandula2016quickr}, rare event aggregation~\cite{kraska2021northstar, galakatos2017revisiting}, and maintaining consistency between related sampled visualizations~\cite{zgraggen2014panoramicdata}.
% Sampling server-side at the MV level compromises the assumptions of downstream applications and is thus not considered in \system.

% \paragraph{Progressive visualization}
% The latency tolerance for time-consuming queries can be circumvented by presenting a partially-computed visualization to the user within the tolerance, which is then incrementally refined until it is fully accurate~\cite{rahman2017ve, zgraggen2016progressive, crotty2015vizdom, kraska2021northstar, kamat2017infiniviz}.
% Example plots which benefit from progressive visualization include bar charts~\cite{kamat2017infiniviz} and heatmaps~\cite{rahman2017ve}.
% Similar to sampling, study on this topic is orthogonal to \system as pushing out partially-updated MVs compromises downstream assumptions.

\section{Method}
\label{sec:method}

% \ml{``Inconsistent'' to ``large variation''}

% In this section, we propose our methods based on the observations in Section \ref{sec:motivation}.
In this section, we propose two techniques to further enhance the strong baseline to capture the variation of activation distributions better.
We first introduce spatial re-scaling to adapt the network to pixel-to-pixel variation.
We then propose channel-wise shifting and re-scaling to better capture the channel-to-channel variation.
Meanwhile, as both of the two methods are image-dependent, the image-to-image variation can be captured naturally.
By combining the two methods with our strong baseline, we build our enhanced BNN for SR, named EBSR.

% Because the activation distributions among pixels, channels and images have large variations \red{**are highly inconsistent} in SR networks, we introduce spatial re-scaling to adapt to pixel-wise variations and channel shift and re-scaling to adapt to channel-wise variations. And both of them are image-dependent to adapt to image-wise variations, which means during inference our network re-scales and shifts the distributions of activations flexibly for different input images. Based on these methods, we build an enhanced binary neural network for image super-resolution (EBSR).

% According to [3], the difference of activation magnitudes indicates different scaling factors are needed for each pixel.

\subsection{Spatial Re-scaling}
% It is better to use different scaling factors for different pixels to reduce the quantization error and retain more detailed information for image super-resolution. 

% \ml{In the main method, we do not need to introduce the previous works but can focus on introducing our own method. Channel rescaling in Real-to-binary Net is not relevant in this context.}

% Re-scaling the output of binary convolutions was proposed at the birth of BNN in XNOR-Net \cite{rastegari2016xnor} to reduce quantization error and improve accuracy for image classification tasks.
% It is computed as below:
% \begin{equation}
% \mathcal{A} * \mathcal{W} \approx(\operatorname{sign}(\mathcal{A}) \circledast \operatorname{sign}(\mathcal{W})) \odot \mathcal{K} \alpha
% \label{eq:xnor-net rescale}
% \end{equation}
% where $\circledast$ denotes the binary convolution and $\odot$ denotes the element-wise multiplication.
% $\mathcal{A}$, $\mathcal{W}$, $\alpha$, and $\mathcal{K}$ denote the activation, weight, weight scaling factor, and activation scaling factor, respectively.
%  Later in XNOR-Net++ \cite{bulat2019xnor}, Bulat et al. fuse the activation and weight scaling factors into a single one that is learned end-to-end based on gradients and this improves the classification accuracy on ImageNet dataset.

% % It is computed as Eq.~\ref{eq:xnor-net rescale}, where $\circledast$ denotes 
% %  the binary convolution and $\odot$ denotes the element-wise multiplication. The binary convolution of $\mathcal{A}$ and $\mathcal{W}$ is rescaled by the weight scaling factor $\alpha$ and the activation scaling factor $\mathcal{K}$, both of which are calculated analytically.


% \zc{Similarly, you should explain the meaning of A, W and the operators $\circledast$ in the formula}
% Then in Real-to-binary Net \cite{martinez2020training}, Martinez et al. used a data-driven channel re-scaling module that takes the pre-convolution activations as input to predict the activation scaling factor. Unlike that in XNOR-Net++ \cite{bulat2019xnor}, these scaling factors are not fixed during inference but rather inferred from data. By doing this, they further improved the classification accuracy on ImageNet over XNOR-Net++. 
As is shown in Figure \ref{fig:pixel}, activation distributions have large pixel-to-pixel variation in SR networks
and the difference of activation magnitudes indicates different scaling factors are preferred for different pixels.
Inspired by \cite{martinez2020training}, we propose spatial re-scaling to better adapt the network to the spatial variation
of activation distributions in SR networks.
% fit the various pixel-wise distributions in SR networks.
We take the real-valued activations $A$ before convolution as input and predict pixel-wise scaling factors $S(A)$, which re-scale the binary convolution output. Spatial re-scaling process can be formulated as follows:
\begin{equation}
A * W \approx(\operatorname{sign}(A) \circledast \operatorname{sign}(W)) \odot \alpha \odot S(A)
\label{eq:spatial rescale}
\end{equation}
where $\circledast$ denotes 
the binary convolution and $\odot$ denotes the element-wise multiplication. $A$, $W$, $\alpha$, and $S\left(A\right)$ denote real-valued activations, weights, the scaling factor of weights, and the spatial-wise scaling factor of activations respectively. $S\left(A\right) \in \mathbb{R}^{1\times H\times W}$ can be calculated with a convolution and a sigmoid function.
% as $\sigma\left( CONV\left(A\right)\right)$. 
As shown in Figure \ref{fig:method}(a), real-valued activations first go through a convolution layer,
which has an input channel of $C$ and an output channel of 1, 
and then pass through a sigmoid function to produce the scaling factors $S(A)$ along the spatial dimension.
During inference, the scaling factor will change dynamically according to different input feature maps.
By re-scaling binary convolution output using $S(A)$, we can reduce the quantization error and the original pixel-wise information in FP activation
will be preserved much better.
Spatial re-scaling leads to a large PSNR improvement of 0.24 dB (from 30.30 dB to 31.54 dB) on Set5 and 0.22 dB (from 25.09 dB to 25.31 dB)
on Urban100 compared with our strong baseline. 

\subsection{Channel-wise Shifting and Re-scaling}

\begin{table}[!tb]
\centering
\caption{Comparison between whether to fuse channel-wise shifting and re-scaling or not based on our baseline with spatial re-scaling. }
\label{tab:fusing}

\scalebox{0.65}{
\begin{tabular}{c|cc|cc|cc}
\hline
\multirow{2}{*}{Method}     & \multirow{2}{*}{OPs} & \multirow{2}{*}{Params} & \multicolumn{2}{c|}{Set5} & \multicolumn{2}{c}{Urban100} \\ \cline{4-7} 
                            &                      &                         & PSNR        & SSIM        & PSNR          & SSIM         \\ \hline
Baseline + spatial re-scale & 2.16G                & 0.05M                   & 31.54       & 0.883       & 25.31         & 0.759        \\
+ channel-wise shift and re-scale             & 2.34G                & 0.09M                   & 31.61       & 0.885       & 25.35         & 0.761        \\
+ w/ fusing                   & 2.27G                & 0.08M                   & \textbf{31.64}       & \textbf{0.885}       & \textbf{25.36}         & \textbf{0.761}        \\ \hline
\end{tabular}
}
\end{table}

In SR networks, activation distributions exhibit larger channel-to-channel variation (Figure \ref{fig:chl}).
Both the mean and magnitude of the activation distributions vary significantly across channels.
% Thus we use channel-wise shifting and re-scaling to adapt to various channel-wise distributions. 
\cite{martinez2020training} has proposed the data-driven channel re-scaling, 
but our method differs from them in further introducing data-driven thresholds to handle the channel-wise variation of both mean and magnitude.
Since the blocks to generate the scaling factors and thresholds are very similar, we further propose to fuse them into one module.
% and fusing channel-wise shifting and re-scaling into one module.
We evaluate the effect of fusing the two blocks in Table \ref{tab:fusing}.
With channel-wise shifting and re-scaling fused, our models have fewer operations and parameters overhead and slightly higher performance.

For the specific process, we take the real-valued activations as input and predict different thresholds and scaling factors for each channel. They are also image dependent, e.g., $\beta_{i}$ in Eq.\ref{eq:act_binarize} is no longer fixed during inference but generated according to different input feature maps. Channel-wise shifting and re-scaling can be formulated as follows:
\begin{equation}
A * W \approx(\operatorname{sign}(A-C_s(A)) \circledast \operatorname{sign}(W)) \odot \alpha \odot C_r(A)
\label{eq:channel-wise_shift_and_rescale}
\end{equation}
where $\circledast$ denotes 
the binary convolution and $\odot$ denotes the element-wise multiplication. $C_s(A), C_r(A) \in \mathbb{R}^{C\times1\times1}$ denote the channel-wise threshold and scaling factor, respectively. 
We show the block diagram in Figure \ref{fig:method}(b).
The real-valued input feature map is first squeezed to a ${C\times1\times1}$ vector by a global average pooling (GAP) layer.
The subsequent fully connected layers and ReLU learn the channel-wise information and output a ${2C\times1\times1}$ vector.
Then the ${2C\times1\times1}$ vector is split into two ${C\times1\times1}$ vectors.
We use the first $C$ channels as the channel-wise bias and pass the last $C$ channels through a sigmoid layer 
as the channel-wise scaling factor, which are used to shift the real-valued activations and re-scale the binary convolution output, respectively. 


% \ml{We can mention previously, channel-wise re-scale has been proposed. We propose to fuse them. Add the comparison between fuse v.s. no fuse.}

\begin{figure}[!tbp]%
  \centering
    \includegraphics[width=0.4\textwidth]{fig/methods.png}
  
% \subfloat[channel-wise shifting\&re-scale]{
%     \label{subfig:channel-wise shifting and re-scale}
%     \includegraphics[width=0.2\textwidth]{fig/chl shift and rescale.png}
%   }

  \caption{Block diagram for spatial re-scaling, and channel-wise shifting and re-scaling.} 
  % Input A is the real-valued activation tensor and C, H, and W denote its dimension. GAP stands for global average pooling. The reduction ratio r is set to 16 for a better trade-off between the performance and the number of operations and parameters.}
  \label{fig:method}
\end{figure}


\subsection{Network Structure}

Combining the spatial re-scaling and the channel-wise shifting and re-scaling methods, we construct the enhanced convolution layer (E-Conv).
Then we build our EBSR model based on E-Conv.
In Figure \ref{fig:E-conv}, we compare the binary convolution layer used in the baseline network and our proposed E-Conv.
We use spatial and channel-wise scaling factors to re-scale the binary convolution output,
and use channel-wise shifting to learn appropriate thresholds for each channel before binarization.
The scaling factors and threshold used in E-Conv are learnable and depend on the real-valued input activations.
In this way, our proposed EBSR can adapt to pixel-to-pixel, channel-to-channel, and image-to-image variations
to reduce the large binarization error and preserve more details.
% In this way, our proposed E-Conv reduces the large quantization error caused by binarization and keeps the original information of input feature maps to a large extent.


\begin{figure}[!tb]%
  \centering

    \includegraphics[width=0.5\textwidth]{fig/E-conv.png}

  \caption{Comparison of (a) the binary convolution layer with a skip connection used in our baseline network and (b) the proposed E-Conv.}
  \label{fig:E-conv}
\end{figure}


Figure \ref{fig:network} shows the basic block based on the E-Conv and our EBSR composed of the basic blocks. Following existing works, the convolution layers in the head and tail modules are not binarized. We choose the lightweight EDSR which has 16 basic blocks and 64 channels, and EDSR which has 32 basic blocks and 256 channels as our backbones, which correspond to EBSR-light and EBSR, respectively.

\begin{figure}[!tb]%
  \centering
  {
    \includegraphics[width=0.35\textwidth]{fig/network.png}
  }
  
  \caption{The structure of our proposed EBSR.  Convolution layers in purple are real-valued vanilla 3x3 convolutions.}
  \label{fig:network}
\end{figure}

\section{Experimental Setup}

\begin{table}[t]
	 \centering
{\renewcommand{\arraystretch}{0.9}
\resizebox{0.9\columnwidth}{!}
{
	\begin{tabular}[b]{l rrrr}
		\toprule
		\textbf{$\textbf{Dataset}$}        & \textbf{\#Train} & \textbf{\#Val} & \textbf{\#Test} & \textbf{\#All}  \\
		\midrule

	    MVSA & 3,611  & 451   & 451   & 4,511  \\
	    ITR & 3,575  & 447   & 449   & 4,471 \\
	    MSD & 19,816  & 2,410   & 2,409   & 24,635  \\
	    MHP & 3,998  & 500   & 502   & 5,000  \\
		\bottomrule	
		\end{tabular}}}
	\vspace{-0.5em}
	\caption{Statistics of the evaluation datasets. \label{tab:four_dataset_analysis}
	} 
	\vspace{-1em}

	%\label{tab:results}
\end{table}

\subsection{Evaluation Datasets}
%\begin{table}[t]
	 \centering
{\renewcommand{\arraystretch}{0.9}
\resizebox{0.9\columnwidth}{!}
{
	\begin{tabular}[b]{l rrrr}
		\toprule
		\textbf{$\textbf{Dataset}$}        & \textbf{\#Train} & \textbf{\#Val} & \textbf{\#Test} & \textbf{\#All}  \\
		\midrule

	    MVSA & 3,611  & 451   & 451   & 4,511  \\
	    ITR & 3,575  & 447   & 449   & 4,471 \\
	    MSD & 19,816  & 2,410   & 2,409   & 24,635  \\
	    MHP & 3,998  & 500   & 502   & 5,000  \\
		\bottomrule	
		\end{tabular}}}
	\vspace{-0.5em}
	\caption{Statistics of the evaluation datasets. \label{tab:four_dataset_analysis}
	} 
	\vspace{-1em}

	%\label{tab:results}
\end{table}

Our evaluation is conducted on four Twitter classification benchmarks on  
%We demonstrate our experiments on four public multimodal social media datasets: 
multimodal sentiment classification (MVSA) \cite{DBLP:conf/mmm/MVSA}, image-text relation (ITR) \cite{DBLP:conf/acl/VempalaP19}, multimodal sarcasm detection (MSD) \cite{DBLP:conf/acl/sarcasm}, and multimodal hate speech detection (MHP) \cite{DBLP:conf/acl/BotelhoHV21}. 
% ITR contains four classes data  which is annotated based on whether text or image provide additional content outside the other modality. MHP focus on the online hate scenario and  the label of the  
%The data in all datasets are 
Each data instance is an image-text pair 
%collected from Twitter 
and it is annotated with
%, and consists of a image-text pair and 
a single class label. 
For MVSA, MHP and MSD, we adopt the same 
%train, validation and test 
dataset split 
%for training, validation, and text 
as their original papers 
%\cite{DBLP:conf/acl/BotelhoHV21} and \cite{DBLP:conf/acl/sarcasm} 
for fair comparisons. 
For ITR, 
%lacking such setup details, 
we randomly
split 80\%, 10\% and, 10\% for training, validation, and test instead of their 10-fold cross-validation setup for the concern of experimental efficiency.

The statistics of evaluation datasets are shown in Table \ref{tab:four_dataset_analysis}, where we observe the small-scale training data in MVSA, ITR, and MHP.
For MVSA, though relatively larger in scales, its automatic labeling under hashtag-based distant supervision, may result in noisy labels, which further require larger data scales for robustness. 
These imply the annotation difficulties and potential benefits from self-training.


\subsection{Implementation and Evaluation Details}

All experimental models 
%for the experiments 
are implements with PyTorch\footnote{\url{https://pytorch.org/}} and HuggingFace Transformers\footnote{\url{https://github.com/huggingface/transformers}}. 
%The maximum length is 50 for both 

Both text and comment are capped at 50 words for encoding. 
The batch size is set to 8, 8, 16, and 16 for ITR, MHP, MVSA, MSD. 
The learning rate is set to 1e-5 with a warm-up rate to 0.1. 
Classifiers are trained with an AdamW optimizer. 
The maximum of consensus comments ($N$) is set to 5.
We run the self-training for three iterations. At each iteration,
the teacher model is fine-tuned for 10 epochs on the labeled training data. 
The teacher model performing the best in validation is adopted to predict the pseudo-labels for the unlabeled retrieved data. 
The student model is then 
fine-tuned for 10 epochs. After that, the student model is used as teacher for the next iteration.
%with the labeled training dataset and pseudo-labeled data. 

For evaluation metrics, we follow the benchmark practice to use precision (pre), recall (rec), and F1-score (F1) for
%to evaluate the performance of models for 
ITR, MVSD, and MHP, 
%tasks 
and accuracy (acc) and F1 
%is used 
for MSVA task.






\subsection{Baselines and Comparisons}

To investigate our universal benefits over different classification tasks varying in SOTA methods, we integrate our comment-aware self-training module into the BERT-based SOTA architectures and examine the results following baselines and comparisons employed in the original paper for fair comparison.
%For fair comparisons with previous work, different baselines are adopted for the four tasks due to the characteristics of different tasks 
%\footnote{I.g., the attributes of images are usually adopted for MSD while the texts in the images are extracted by using OCR for MHP.}.

%\paragraph{Baselines for MVSA.}





\paragraph{MVSA Comparisons.} This benchmark presents baselines of
MultiSentiNet \cite{DBLP:conf/cikm/MultiSentiNet} (a deep semantic network with the visual clues guided attention),
%mechanism 
%for multimodal sentiment classification
CoMN \cite{DBLP:conf/sigir/Co-MN} 
%is 
(a co-memory network to learn image-text interactions),
%the interactions between image features and text features
%for multimodal sentiment analysis. 
MMMU-BA \cite{DBLP:conf/emnlp/MMMU-BA} 
%proposes 
(enriching context for cross-modal fusion), and
%a fusion method by utilizing the context from neighboring utterances to generate richer multi-modal representation)
Self-MM \cite{DBLP:conf/aaai/Self-MM} (joint training of uni-modal and multi-modal tasks to explore cross-modal consistency).
%jointly learns the uni-modal tasks and multimodal sentiment analysis to capture the the consistency and difference among different modalities.) 
CoMN-BERT is a SOTA architecture combining CoMN and the pre-trained BERT \cite{DBLP:conf/naacl/BERT}, which will later be combined with our module for comparison.

%which employs the CoMN architecture and uses BERT \cite{DBLP:conf/naacl/BERT} 
%to encode text is used as our base model for MVSA.

%\paragraph{Baselines for ITR} 
\paragraph{ITR Comparisons.}
In the original paper \cite{DBLP:conf/acl/VempalaP19}, LSTM-CNN performs the best via combining CNN-encoded  visual features \cite{DBLP:conf/cvpr/InceptionNet} and LSTM-encoded lingual features \cite{LSTM}.
It is compared with the baseline ablations CNN and LSTM using uni-modal features only.
To line up with the SOTA, we implement BERT-CNN to employ pre-trained BERT for text encoding instead of LSTM, which is likewise compared to a BERT classifier using lingual features only.
Based on BERT-CNN, we integrate in our module to examine its effectiveness over SOTA. 
%Following \citet{DBLP:conf/acl/VempalaP19}, LSTM \cite{LSTM} and CNN \cite{DBLP:conf/cvpr/InceptionNet} are adopted to encode text and image, respectively.
%are adopted as the text-based method and image-based method for image-text relation classification, separately. 
%LSTM-CNN \cite{DBLP:conf/acl/VempalaP19} combines the image features and text features to jointly learn the interaction of image-text pairs. 
%BERT, a pretrained model for text, is also taking as the text-modality approach. BERT-CNN is utilized as our base model for ITR.

%\paragraph{Baselines for MSD}
\paragraph{MSD Comparisons.} Here the MMSD baseline is introduced in the original paper \cite{DBLP:conf/acl/sarcasm}, which employs a hierarchical fusion model to explore visual and lingual features with optical characters.
%\footnote{
%Here visual features include 
%Image attributes refer to optical characters in images.
%are usually adopted for MSD while 
%the texts in the images are extracted by using OCR for MHP.
%}
%introduces a hierarchical fusion model to learn the joint representation of text features, image features and image attributes. 
We also compare with the following more advanced models on the benchmark: D\&R Net \cite{DBLP:conf/acl/D_R} (using decomposition and relation network to learn visual-lingual interactions),
%to capture the semantic interactions between image modality and text modality.
Res-BERT \cite{DBLP:conf/emnlp/Pan_sarcasm} (concatenating visual features from ResNet (cite) and lingual features from BERT),
%(concates the image features and text features for sarcasm classification) 
Att-BERT \cite{DBLP:conf/emnlp/Pan_sarcasm} (with attention mechanism to capture image-text semantic consistency), and
%attention mechanism to capture the inconsistency between images and texts),
CMGCN \cite{DBLP:conf/acl/CMGCN} (building a  graph to explore cross-modal interactions).
%constructs a cross-modal graph to utilize the inconsistent implications between different modalities. 
MMSD-BERT is based on MMSD with a pre-trained BERT to encode texts, where we will later architect with our proposed module.

%which uses the NMSD architecture and uses BERT to encode text is used as our base model for MSD.

\begin{table}[t]
	 \centering
{\renewcommand{\arraystretch}{0.8}
\resizebox{0.7\columnwidth}{!}
{
	\begin{tabular}[b]{
	%l c c c c c c
	lcc
	}
		\toprule
		\textbf{Methods}        & \textbf{Acc} & \textbf{F1} \\
		\midrule

		MultiSentiNet
		& 69.84   & 69.63 \\
		%GRN (
        CoMN
        & 70.51 & 70.01 \\
        MMMU-BA
        & 68.72 & 68.35 \\
        Self-MM
        & 72.37 & 71.96 \\
        CoMN-BERT
        & 71.33 & 70.66 \\
        \midrule
        
        CoMN-BERT (full)
        &  \textbf{73.71}  & \textbf{72.83} \\
		\bottomrule	\end{tabular}}}
	\vspace{-0.5em}
	\caption{
	Comparison results 
	%of the baselines and our model 
	on the MVSA dataset. 
	}
	\label{tab:MVSA_results}
\end{table}
\begin{table}[t]
	 \centering
{\renewcommand{\arraystretch}{0.8}
\resizebox{0.8\columnwidth}{!}
{
	\begin{tabular}[b]{
	%l c c c c c c
	lccc
	}
		\toprule
		\textbf{Methods}      & \textbf{Pre}  & \textbf{Rec} & \textbf{F1} \\
		\midrule

		LSTM
		&42.33  & 48.55  & 38.77 \\
		%GRN (
        CNN
        & 37.11 & 47.22 & 35.99 \\
        LSTM-CNN
        & 48.21 & 50.78 & 44.58 \\
        BERT
        & 44.65 & 48.78 & 40.39  \\
        BERT-CNN
        & 50.31 & 50.60 & 49.72 \\
        \midrule
        
        BERT-CNN (full)
        & \textbf{53.69} & \textbf{54.42} & \textbf{53.38} \\
		\bottomrule	\end{tabular}}}
	\vspace{-0.5em}

	\caption{
	Comparison results 
	%of the baselines and our model 
	on the ITR dataset. 
	}
	\vspace{-1em}
	\label{tab:ITR_results}
	
\end{table}
\begin{table}[t]
	 \centering
{\renewcommand{\arraystretch}{0.8}
\resizebox{0.8\columnwidth}{!}
{
	\begin{tabular}[b]{
	%l c c c c c c
	lccc
	}
		\toprule
		\textbf{Methods}      & \textbf{Pre}  & \textbf{Rec} & \textbf{F1} \\
		\midrule

		MMSD
		& 76.57 & 84.15  & 80.18 \\
		%GRN (
         D\&R Net
        & 77.97 &  83.42 & 80.60 \\
        Res-BERT 
        & 78.87 & 84.46 & 81.57  \\
        Att-BERT 
        & 80.87 & 85.08 & 82.92  \\
         CMGCN
        & 83.63 & 84.69 & 84.16 \\
        MMSD-BERT 
        & 83.57 & 84.52 & 84.04 \\
        \midrule
        
        MMSD-BERT (full)
        & \textbf{85.50} & \textbf{85.92} & \textbf{85.70} \\
		\bottomrule	\end{tabular}}}
	\vspace{-0.5em}
	\caption{
	Comparison results 
	%of the baselines and our model 
	on the MSD dataset. 
	}
	\label{tab:MSD_results}
\end{table}
\begin{table}[t]
	 \centering
{\renewcommand{\arraystretch}{0.8}
\resizebox{0.7\columnwidth}{!}
{
	\begin{tabular}[b]{
	%l c c c c c c
	lccc
	}
		\toprule
		\textbf{Methods}      & \textbf{Pre}  & \textbf{Rec} & \textbf{F1} \\
		\midrule

		Xception
		& 56.0 & 54.5 & 54.4  \\
		%GRN (
        LSTM
        & 70.7 & 73.7 & 71.9  \\
        RoBERTa
        & 75.9 & 76.5 & 75.4  \\
        MMBT
        & 76.3 & 78.5 & 77.1 \\
        \midrule
        
        MMBT (full)
        & \textbf{79.15} & \textbf{79.88} & \textbf{78.76} \\
		\bottomrule	\end{tabular}}}
	\vspace{-0.5em}
	\caption{
	Comparison results 
	%of the baselines 
	%and our model 
	on the MHP dataset.\protect\footnotemark}
	\label{tab:MHP_results}
	\vspace{-1em}
\end{table}



%\paragraph{Baselines for MHP}
\paragraph{MHP Comparisons.}
Following the setup in  \citet{DBLP:conf/acl/BotelhoHV21}, we consider the Xception \cite{DBLP:conf/cvpr/Xception} baseline using visual features only.
%, which only encodes image content for classification, 
For the text-only comparison, 
%is used as the unimodal-image model while 
LSTM and RoBERTa \cite{DBLP:journals/corr/RoBERTa} are adopted.
%are utilized as the context-based unimodal-text models. 
MMBT \cite{DBLP:conf/acl/BotelhoHV21} is the SOTA model learning cross-modal representations with pre-trained MultiModal BiTransformers and will be employed as the base to experiment with our module.

%our base model, integrate the image features with text tokens to the 
%MultiModal BiTransformers, initialized with pre-trained
%BERT weights, for classification.


\footnotetext{
The baseline results are copied from the original paper, where the numbers are rounded to one decimal place.
%The results of baselines only have three significant figures, and the results of the full model also employ the same format for consistency.
}

\paragraph{Integrating our Comment-aware Self-training.}
Based on aforementioned SOTA architectures ( \textbf{base classifiers} --- CoMN-BERT, BERT-CNN, MMSD-BERT, and MMBT, selected for the four benchmarks), we further employ comment-aware self-training in their training and
%integrate our comment-aware self-training module with the 
%base classifiers,
%, which yield 
%The newly combined models are 
therefore result in CoMN-BERT (full), BERT-CNN (full), MMSD-BERT (full), and MMBT (full). 
%The model CoMN-BERT (full), BERT-CNN (full), NMSD-BERT (full), and MMBT (full) represent the corresponding base model adding the comment-aware self-training for MVSA, ITR, MSD, and MHP, separately. 

%To demonstrate the effects of different modules in the proposed retrieval-based comment-aware self-attention method, 
To further examine the relative contributions of each sub-module in comment-aware self-training, the following ablations are considered in comparison:
%we conduct experiments on four variants based on the base model:
(1) Base+Com,
integrating  BERT-encoded comment features in the base classifiers.
%utilize the retrieved comments to add context with the base model (Base+Com); 
(2) Base+ST,  self-training with retrieved tweets yet without comments. 
%with retrieved similar image-text pairs based on the base models
(3) Base+Com+ST, the full model without randomly dropping the retrieved comments in student model training.
%The full model
%use retrieved comments and self-training framework (Base+Com+ST);
%(4) randomly drop the retrieved comments for the student model in the self-training framework (Base+Com+ST+Drop).




\section{Experimental Discussions}

\subsection{Main Comparison Results}\label{ssec:exp:main}



\begin{figure}
       \centering
        \setlength{\tabcolsep}{1pt}
        {\scriptsize
        \begin{tabular}{c c c c c c c }
            { Original } &
            \multicolumn{2}{c}{  } &
            \multicolumn{4}{c}{$\longleftarrow$ Object level variations $\longrightarrow$} \\
            \includegraphics[width=0.185\linewidth]{images/ablation/chair.jpg} &
            \multicolumn{2}{c}{  } &
            \includegraphics[width=0.185\linewidth]{images/ablation/1_only_prompt_mixing/bench.jpg} &
            \includegraphics[width=0.185\linewidth]{images/ablation/1_only_prompt_mixing/stool.jpg} &
            \includegraphics[width=0.185\linewidth]{images/ablation/1_only_prompt_mixing/armchair.jpg} &
            \includegraphics[width=0.185\linewidth]{images/ablation/1_only_prompt_mixing/saddle.jpg} \\
            \multicolumn{3}{c}{  } &
            \multicolumn{4}{c}{ Only Prompt Mixing } \\
            \multicolumn{3}{c}{ } &
            \includegraphics[width=0.185\linewidth]{images/ablation/2_with_self_attn_injection/bench.jpg} &
            \includegraphics[width=0.185\linewidth]{images/ablation/2_with_self_attn_injection/stool.jpg} &
            \includegraphics[width=0.185\linewidth]{images/ablation/2_with_self_attn_injection/armchair.jpg} &
            \includegraphics[width=0.185\linewidth]{images/ablation/2_with_self_attn_injection/saddle.jpg} \\
            \multicolumn{3}{c}{  } &
            \multicolumn{4}{c}{ + Attention-Based Shape Localization } \\
            \multicolumn{3}{c}{ } &
            \includegraphics[width=0.185\linewidth]{images/ablation/3_background_blending/bench.jpg} &
            \includegraphics[width=0.185\linewidth]{images/ablation/3_background_blending/stool.jpg} &
            \includegraphics[width=0.185\linewidth]{images/ablation/3_background_blending/armchair.jpg} &
            \includegraphics[width=0.185\linewidth]{images/ablation/3_background_blending/saddle.jpg} \\
            \multicolumn{3}{c}{  } &
            \multicolumn{4}{c}{ + Controllable Background Preservation } \\
        \end{tabular}
        }
    \vspace{1mm}
    \captionof{figure}{
    Ablating our full object variations pipeline. Original image was crated using the prompt ``A \emph{chair} with a dog on it''. 
    }
    \vspace{-10pt}
    \label{fig:ablation}
\end{figure}


Table \ref{tab:MVSA_results}$\sim$\ref{tab:MHP_results} shows the main comparison results on MVSA, ITR, MSD, and MHP, respectively. 
%The observations are summarized as follows: 
%(1) The 

%We first observe 
The full model significantly outperforms
%improve the performance compared with 
all baselines and advances their base ablation on all test benchmarks (measured by paired t-test; $p-value<0.05$). 
This indicates that our
%the retrieved comments and 
comment-aware self-training can universally benefit varying tasks  
and classification architectures.
%retrieval-based comment-aware self-training method 
%can be employed on different models and different multimodal social media tasks. 
%Specially, our method could obtain gains 
It enables performance gains on both the simple architecture (e.g., BERT-CNN for ITR) and other more complicated models. 
%(i.e., CoMN-BERT for MVSA, NMSD-BERT for MSD, and MMBT for MHP). 
The possible reasons are twofold. 
%for the improvement 
First, retrieved comments, carrying viewpoints from human readers, may provide
%could provide 
complementary context hinting the cross-modal semantic understanding for weakly-connected image-text pairs.
%for the image-text pairs 
Second, our self-training may enable the models to leverage both labeled data and unlabeled retrieved data, potentially mitigating the overfitting issue caused by insufficient labeled data scales.
%more data are included for training by using the self-training framework; 


We also observe the models with BERT encoders 
%BERT-based models 
consistently outperform their counterparts with LSTM encoders, either in 
%LSTM-based models 
%on both 
multimodal or unimodal architectures. 
These demonstrate the benefit of pre-training on large-scale text, where the gained generic language understanding capability may enable models to well induce cross-modal meanings.
%enable promising results on downstream visual-lingual social media tasks.
%This demonstrates that pre-trained language models have the excellent ability to capture the semantics of texts.  

\subsection{Ablation study}\label{ssec:exp:ablation}
The general superiority of our method has been demonstrated in $\S$\ref{ssec:exp:main} compared to previous benchmark results.
Here we conduct an ablation study to further probe the relative contribution of varying components and 
%To examine the effects of different components of the proposed comment-aware self-training method, we conduct an ablation study and 
show the results 
%for the four tasks 
in Table \ref{tab:ablation_results}. 
%Both comments and self-training module contribute greatly to the model. 

The obvious performance drop of Base+Com and Base+ST, compared to the Full Model, together suggest the positive effects individually from retrieved comments and self-training.
These strengthen our previous findings: the comments may enrich context with human hints to bridge visual and lingual semantics and self-training may enrich the data scales with semantically related posts and comments to allow better robustness.  
%This demonstrates that comments could provide necessary context and bridge the images and texts while utilizing the self-training module to add semantically similar data into the training set is helpful for the four tasks. 

For the results of Base+Com+ST, though better than other ablations, are slightly worse than the Full Model.
It implies the extra benefit of modeling retrieved comments in self-training, while randomly dropping some of them may enable the student model to better catch up with the teacher, mitigating the least favorable perturbation phenomena \cite{DBLP:conf/cvpr/XieLHL20} in teacher-student alignment.

%Furthermore, from the results of Base+Com+ST and Base+Com+ST, adopting the strategy of randomly dropping comments could further boost the performance. 
%The reason might be that the student model with fewer comments would have more difficulty to align with the predictions of teacher model, which leads to favorable perturbation for the training step.



\subsection{Quantitative Analysis}

$\S$\ref{ssec:exp:ablation} shows the crucial roles self-training and comment retrieval play in our method. 
Here we further quantify the effects of varying unlabeled data scales on self-training and those of individual modality (images or text) on comment retrieval.

\begin{figure}[t]
\centering
\includegraphics[width=0.8\columnwidth]{figures/self_train_num_fig.pdf}
\vspace{-0.5em}
\caption{Performance gain observed from self-training given varying number of unlabeled retrieved posts (and their associated comments).
%the retrieved image-text pairs used in the self-training. 
X-axis: within each dataset, the bars from left to right indicate self-training with varying number of posts ($K$); y-axis: the difference in F1 between our Full Model and Base Classifier.
%the different value of $K$. 
%Y-axis represents the performance gain compared with related base models.
}
\label{fig:self_train_num}
\end{figure}
\begin{figure}[t]
\centering
\includegraphics[width=0.8\columnwidth]{figures/modality_fig.pdf}
\vspace{-0.5em}
\caption{F1 gain compared to the Base Classifier (y-axis) over varying datasets. For each dataset, the bars from left to right indicate the retrieval with image only, text only, and both image and text (Image+Text).  
%Full model performance compared with varying modality ablations for the four tasks. 
}
\vspace{-1em}
\label{fig:modality}
\end{figure}

%\paragraph{The effects of the num of retrieved image-text pairs used in the self-training}
\paragraph{Self-training w/ Varying Unlabeled Data Scales.}
%To examine model sensitivities to varying num of image-text pairs used for the step of self-training, 
Here we train our full model via self-training with varying number of retrieved posts ($K$) and show the performance gain compared to Base Classifier in Figure \ref{fig:self_train_num} (F1 difference of the Full Model and Base Classifier).
%setting the num $k$ of similar posts to $1,3,5,7,9$. 
%As shown in Figure \ref{fig:self_train_num}, 
%We observe that most models obtain the best performance when $k$ is 3 or 5. And the performance world decrease with more data (i.e., $k$=7 or 9). 
We observe the results peak at $K=3$ or $5$, implying  self-training may benefit from some similar unlabeled data while further retrieving more data may result in noise as well.

%The reason might be that there are not enough semantically similar posts in the retrieval dataset, and it's might be hard for the model to learn from the pseudo-labeled posts which are not similar to the original labeled data.

%\paragraph{The efftects of the modality used for the retrieval}
\paragraph{Modality Effects on Comment Retrieval.}
Recall that in comment retrieval, we balance the visual and lingual similarity to retrieve similar posts (and obtain their comments). 
To further study how features in varying modalities affect comment retrieval results, we examine two ablations relying on the similarity in image (Only Image) and text (Only Text) in comparison to the full model trading-off image and text similarities (Image+Text).

%To investigate the effectiveness of the proposed comment-aware self-training method when used with different retrieval modalities, we conduct experiments with only using image features, only using text features, and using image and text features for the retrieval process. 
The results (F1 gain compared to Base Classifier) are shown in Figure \ref{fig:modality}.
Varying tasks might prefer the similarity measure with image or text semantics,
whereas the full model leveraging posts' visual and lingual features achieves the best results.

%We can observe that the model with both modalities could consistently obtain the best performance on all tasks compared with the unimodality. This demonstrates that semantic gaps exists in the image-text pairs, and it's necessary to consider the both modalities for the understanding.


\subsection{Qualitative Analysis}\label{ssec:exp:qualitative}

The discussions above are from a quantitative view. 
To provide more insight, a case study will be presented here, followed by analyses for error cases.

In this section, we revisit the full problem (introduced in \cref{s.model}) and conduct numerical experiments to evaluate the performance of $\DPI$ for its intended use case. Our analysis is motivated by our partner organization, Keheala, which operates a digital health platform to support medication adherence among TB patients in highly resource-constraint settings. Here, we first summarize the state of the global TB epidemic and the Keheala behavioral intervention (\cref{ss.keheala}). We then describe our data sources and the validated simulation model we have developed to test outreach policies (\cref{ss.data} and \cref{ss.sim}). Next, we discuss how our policy, as well as some benchmark policies, can be implemented using Keheala's data (\cref{ss.policies}), before presenting our numerical results (\cref{ss.results}).

\subsection{The global TB epidemic and the Keheala intervention}\label{ss.keheala}
TB remains one of the deadliest communicable diseases in the world, causing 1.6 million deaths in 2021. This is remarkable in light of the fact that effective treatment has been available for over 80 years, with the current WHO guidelines recommending a 6 month regimen of antibiotics for drug-susceptible TB and a more intense regiment for drug-resistant strains \citep{World22Global}. A key limiting factor for curbing the epidemic is lack of patient adherence to these treatment regiments, which increases the probability of infection spreading, drug resistance, and poor health outcomes \citep{garfein2019synchronous}. 

Keheala was designed to provide treatment adherence support to TB patients in resource-limited settings. Their platform operates on the Unstructured Supplementary Service Data (USSD) mobile phone protocol, which importantly allows phones without smart capabilities to access the service. Once a patient has enrolled with Keheala, they are meant to self-verify treatment adherence every day, using their mobile phone. In addition, they have access to a range of services. Some are on-demand, for example educational material about TB or leaderboards for verification rates. Others are automatic, such as adherence reminders, which are sent to patients daily (at their pre-determined medication time) in the absence of verification. In addition, the Keheala protocol is to escalate outreach interventions when patients do not self-verify adherence. It states that after one day of non-adherence patients should receive a customized message to encourage resumed adherence and after two days of non-adherence patients should receive a phone call from a support sponsor. While these support sponsors are full-time employees, they are not healthcare professionals. They are members of the local community who have experience with TB treatment and are therefore familiar with the many contributing factors associated with low treatment adherence, such as side-effects, societal stigma against TB patients, and challenges with refilling prescriptions.

The overall effectiveness of Keheala's combination of services was evaluated in a randomized controlled trial (RCT) in Nairobi, Kenya. The trial demonstrated that Keheala reduced unsuccessful TB treatment outcomes—a composite of loss to follow-up, treatment failure, and death—by roughly two-thirds, as compared to a control group that received the standard of care \citep{Yoeli19Digital}. Given this success, Keheala's primary practical objective is to ensure that enrolled patients remain engaged with the platform through adherence verification.

In this paper, we focus on the final level in Keheala's escalation protocol---support sponsors making phone calls to patients. This part of the outreach was organized through populating a daily list of patients who had not verified treatment adherence for 48 hours. Support sponsors had many responsibilities in operating the platform, but would make phone calls to as many patients on the list as possible on a given day. Since hiring support sponsors is a costly aspect of operating the service, Keheala is interested in implementing a more personalized and targeted approach for prioritizing which patients should receive a phone call on a given day. Being able to maintain a similar performance with fewer support sponsors (or equivalently, serve more patients with the same number of support sponsors) is desirable for any future scale-up of the system.








\subsection{Data sources.}\label{ss.data}
Based on the success of the first RCT, the effectiveness of Keheala was further evaluated in a second RCT\footnote{The trial was approved by the institutional review board of Kenyatta National Hospital and the University of Nairobi. Trial participants or their parents or guardians provided written informed consent. The trial’s protocol and statistical analysis plan were registered in advance with ClinicalTrials.gov (\#NCT04119375).} during 2018-2020. The RCT was conducted in partnership with 902 health clinics distributed across each of Kenya's eight regions, representing a mix of rural and urban clinics. The study included four treatment arms and enrolled over 15,000 patients. We obtained data for 5,433 patients enrolled in the Keheala intervention arm (other arms aimed to independently test specific components of the Keheala intervention). 

As part of the RCT, the study team collected socio-demographic information from all patients. This information includes static covariates such as age, gender, language preferences, location, as well as limited clinical history (see \cref{s.app.list_features} for a full list). In addition, Keheala collected engagement data about each patient during their enrollment in the service. This includes whether a patient verified on a given day, how many reminders they received, and whether they were contacted by a support sponsor. 

After filtering out patients with missing information or not enough data, we conducted all our analysis on 3594 patients. The average patient was enrolled on the platform for 118 days. On an average day, 608 patients were enrolled and 210 of those were eligible for a support sponsor call according to the protocol (i.e., having not verified treatment adherence for the preceding 48 hours). The support sponsors, employed by Keheala, had a range of responsibilities in operating the platform, including making outreach phone calls to the eligible patients. 
The average number of calls made per day was 25.5. Hence, in our analysis, we use a budget of $B$ = 26 as our main point of comparison.
	

\subsection{Simulation Model.}\label{ss.sim}
We build a simulation model that we use to estimate the counterfactual outcomes of different outreach approaches.
The simulator is effectively represented by a single function, $f(S, A) \in [0, 1]$, which denotes the probability that a patient in state $S$ with action $A$ verifies in the next time step.
This function is used to simulate one step transitions for every patient.
We first describe the state space of the patients, describe the exact simulation procedure, and then discuss how we learn $f$ from data and validate the simulator.

\subsubsection{Patient state space.}\label{ss.simstatespace}
For patient $i$, let $X_i \in \bR^{13}$ be their static covariates. 
Let $V_{it} \in \{0, 1\}$ denote whether patient $i$ verified at time $t$, and let $A_{it}\in \{0, 1\}$ denote whether the patient received the intervention at time $t$.
Let $H_{it} = (V_{i1}, A_{i1}, \dots, V_{i, t-1}, A_{i,t-1}, V_{it}) \in \bR^{2t-1}$ be the history of verifications and interventions up to time $t$.
We define a \textit{condensed} history $\tH_{it} \in \bR^{21}$ by summarizing the history $H_{it}$ into 21 features, aiming to capture as much relevant information as possible.
The condensed history contains the patient's recent and overall behavior.
For statistics such as the number of times the patient verified and the number of interventions they have received, we aggregate them over the past week, as well as in total.
We also include information on their verification and non-verification streaks, as well as how long they have been in the platform.
See \cref{s.app.list_features} for a full list of these features.
Then, we define the state of patient $i$ at time $t$ to be $S_{it} = (X_i, \tH_{it}) \in \bR^{34}$.

\subsubsection{Simulation procedure.} 
Given $f$ and an intervention policy $\pi$, we `mimic' the RCT by simulating patient behavior day by day. 
In total, we simulate $T=700$ time steps, where each $t \in [T]$ corresponds to one day between April 2018 to March 2020. We let $T_{s}(i)$ and $T_{e}(i)$ denote the starting and ending time steps that patient $i$ was enrolled in Keheala. Each patient $i$ is then introduced into the system at time $t=T_s(i)$, and removed at time $t=T_e(i)$. We use the observed data from the RCT for their first 7 days in the system to initialize their state. 
Then, in each time period, given the set of patients that were active in the RCT for more than 7 days, we use a policy $\pi$ on these patients to determine who receives sponsor outreach. If a patient $i$ was in state $S_{it}$ at time $t$ and the policy $\pi$ chose action $A_{it}$, we let $V_{i, t+1}$ be 1 with probability $f(S_{it}, A_{it})$, and 0 otherwise (where the randomness is independent across patients and time steps).
Finally, we use $V_{i, t+1}$ to update their state for the next time step. 

\subsubsection{Estimating the $f$ function.} \label{s.learnf} 
Using the state space $\cS$ as described above, we construct the function $f: \cS \times \{0, 1\} \to [0, 1]$ using data from the RCT. Specifically, we learn the two functions $f(S, 0)$ and $f(S, 1)$ separately. For $f(S, 0)$, we train a gradient boosting classifier on the dataset $\{(S_{it}, V_{i,t+1})\}_{i \in N, t \in [T] : A_{it} = 0}$, using $V_{it}$ as the outcome variable. For $f(S, 1)$, we write the function as $f(S, 1) = f(S, 0) + \tau(S)$ and we learn $\tau(S)$ using the double machine learning method of estimating heterogeneous treatment effects \citep{chernozhukov2018double}. See \cref{s.app.simulation} for details on implementing this method.

\subsubsection{Train and test split.} \label{s.traintestsplit}
Importantly, we use a different set of patients to estimate the $f$ function (and to run our simulations) from the set of patients we use to train our policies. In particular, we randomly split all patients from the RCT into two groups, which we call \textit{train} and \textit{test}. We use the \textit{test} set to estimate the $f$ function that forms the basis for the simulation model. We keep the \textit{train} set of patients separate and use it to train policies (see \cref{ss.policies}). This ensures that the policies we evaluate are not learned off of the same dataset that was used to learn the simulator. 
The simulation itself uses the test patients, and we duplicated each patient so that we maintain a similar total number of patients as in the original study.


\subsubsection{Simulation validation.}
We validate the performance of the simulator on a \textit{different} intervention policy than the simulator was trained on, by leveraging the fact that there was variability in the number of interventions given throughout the RCT.
In particular, the average number of interventions given during the first half of the RCT was around double of that of the latter half (45.9 vs. 21.4),
and this variation induces a change in the intervention assignment policy.
Then, dividing the data into halves produces two datasets that are generated using effectively different intervention policies.

To validate the simulator, we use the method from \cref{s.learnf} to learn $f$ using the first dataset, and then validate its performance on the second dataset. 
Using this procedure, the AUCs on the second dataset for $f(S, 0)$ and $f(S, 1)$ were 0.918 and 0.745, respectively.
We also check the calibration of both of these functions, by grouping the samples into bins based on their predicted probability of verifying the next day, and checking whether their actual verification rates.


We group the samples based on the simulated probability of verification into bins with a 10\% range, and we compute the expected calibration error (ECE) \citep{naeini2015obtaining}. For bin $i$, let $o_i$ be the true fraction of positive instances in bin $i$, $e_i$ be the mean of the predicted probabilities of the instances in bin $i$, and $N_i$ be the number of samples in bin $i$. Then, the ECE is defined as
\begin{align}
\text{ECE}	 = \frac{1}{N} \sum_{i=1}^{10} N_i |o_i - e_i|,
\end{align}
where $N$ is the total number of samples.
The expected calibration error was 0.0066 for $f(S, 0)$ and 0.0308 for $f(S, 1)$.
Figure~\ref{f.calibration} displays these bins.

These results demonstrate that the simulator has good performance in mimicking patient behavior. 
As expected, the AUC and the ECE is worse for $f(S, 1)$ compared to $f(S, 0)$; 
this is due to the \textit{significantly} fewer number samples with an intervention in the training data, as well as the increase in variance of doing off-policy estimation.
The training data used for $f(S, 0)$ had 300K samples, while the one used for $f(S, 1)$ had 4.5K samples.
For $f(S, 1)$, the calibration is slightly off for samples with a high probability of verification (bins 0.7-0.9); however, we note that the 0.7-0.9 bins only contain 11.3\% of all samples for $f(S, 1)$.


\begin{figure}[h]%
	\centering
	\subfloat[Calibration for $f(S, 0)$]{{\includegraphics[width=0.48\linewidth]{figs/bar_pred0} }}%
	\subfloat[Calibration for $f(S, 1)$]{{\includegraphics[width=0.48\linewidth]{figs/bar_pred1} }}%
	\vspace{2mm}
	\caption{Calibration plots for $f(S, 0)$ and $f(S, 1)$ for simulation validation. 
		We group the samples based on the simulated probability of verification
		into bins with a 10\% range, which we label by the lower number. 
		For example, the 0.3 bin on the x-axis represents the samples whose probability of verification according to $f$ is in $[0.3, 0.4)$; hence we should expect the actual number of verifications of those samples to be close to 0.35.}
	\label{f.calibration}%
	\vspace{-2mm}
\end{figure}



\subsection{Outreach Policies and Experimental Design}\label{ss.policies}
Using the simulation model described above, we compare the performance of three main policies. For each policy, we vary the budget for outreach interventions per day between 10 and 40. As mentioned before, the average number of sponsor outreaches during a given day of the trial was 26. Importantly, we restrict all policies so that they can only provide an outreach to patients who have not verified for at least two days in a row. This is because that was what was done in the RCT, hence there is no data for how an outreach affects behaviors for those who do not meet this criterion (thus we would not be able to accurately evaluate policies that do not follow this restriction).

We note that attaining improved performance with lower outreach capacity is particularly important for the resource-limited setting at hand as it speaks to the performance achievable during a future scale-up of the system, in which the ratio of patients to support sponsors is likely to be much higher. 
Next, we describe the implemented policies.

\subsubsection{$\DPI$ for Keheala.} 
The first step in operationalizing $\DPI$ is defining the state space for each patient. For this, we use the same condensed state space as described in \cref{ss.simstatespace}, i.e., we define the state of patient $i$ at time $t$ to be $S_{it} = (X_i, \tH_{it}) \in \bR^{34}$ (a full list of these features is included in \ref{s.app.list_features}). Importantly, we note that all of the features of this state space are observable to Keheala at any time $t$, once a patient has been enrolled on the platform for seven days. 

The second step is estimating the $\hz_{it}(S_{it})$ score for each patient at each time period, which is ultimately used to prioritize patients. 
As before, we let $T_{s}(i)$ and $T_{e}(i)$ be the starting and ending time steps that patient $i$ was enrolled in Keheala. Using this notation, we can represent the future verification \emph{rate} for patient $i$ at time $t$ by $y_{it} = \frac{1}{T_{\text{e}}(i)-t} \sum_{r=t+1}^T V_{ir}$.
Then, the data from the RCT can be written in the form $\{(S_{it}, A_{it}, y_{it})\}_{i \in [N], t \in \{T_{s}(i), \dots, T_{e}(i)\}}$, and we can estimate the function $q_{it}^{\baseline}(S, A)$ using this data.
In our implementation, we use a linear function approximation for the verification rate, assuming the form 
\begin{align*}
q_{it}^{\baseline}(S, A) = \langle \theta_A, S \rangle \cdot (T_{\text{e}}(i)-t),
\end{align*}
for each of the two actions $A \in \{0, 1\}$.
The $\langle \theta_A, S \rangle$ term represents the future verification rate, and $T_{\text{e}}(i)-t$ represents the number of days left; combined, $q_{it}^{\baseline}(S, A)$ represents the total number of future verifications.
We note that the state contains information regarding the number of days the patient has been enrolled in Keheala, hence the verification rate is also a function of the time step.

We estimate $\theta_a$ using least squares with an $\ell_2$ regularizer:
\begin{align} \label{eq:leastsquares}
	\hat{\theta}_a &\in \argmin_{\theta \in \bR^{34}} \bigg( \sum_{i \in N} \sum_{t=T_{\text{s}}(i)}^{T_{\text{e}}(i)}  \bI(A_{it} = A)(y_{it} - \theta^\top S_{it})^2 + ||\theta||^2_2 \bigg).
\end{align}

Finally, we compute a patient's estimate of their intervention value at time $t$ as
\begin{align*}
	\hz_{it}(S_{it}) = \langle \htheta_1 - \htheta_0,S_{it} \rangle \cdot (T_{\text{e}}(i)-t), 
\end{align*}
and the resulting policy is to give the intervention to up to $B$ patients with the highest positive $\hz_{it}(S_{it})$ values.





\subsubsection{Bandit.}
The bandit policy aims to choose patients with the highest increase in the probability of next-day verification, using a linear contextual bandit model. In terms of the two-state model from \cref{ss.2statemodel}, the goal is to choose patients with the highest value of $\tau$.
We essentially use the same state space and linear model as was used for $\DPI$, except that the outcome variable is next-day verification, rather than total future verifications.
We first learn a prior using the offline data, and then we run a Thompson sampling style policy, which continually updates the policy with online data.

Specifically, we assume the linear form $V_{i,t+1} = \langle \beta_a, S_{it} \rangle$ for action $a \in \{0, 1\}$, with unknown parameters $\beta_0, \beta_1 \in \bR^{34}$.
The prior on $(\beta_0, \beta_1)$ is initialized as the output of a least-squares regression using the offline data, the same data that was used to train $\DPI$.
At each time step, $(\tilde{\beta_0}, \tilde{\beta_1})$ is sampled from the posterior. 
Then, the policy chooses the $B$ patients with the highest value of $\langle \tilde{\beta_1}, S_{it} \rangle - \langle \tilde{\beta_0}, S_{it} \rangle$.
After the outcome is observed at each time step, the posterior is updated accordingly.
The detailed description on the algorithm can be found in Section~\ref{sec:app:bandit}.

This policy makes use of strictly more data than $\DPI$, since $\DPI$ only uses the offline data. 
In the results, we confirm that this policy indeed learns myopic rewards correctly.
Therefore, this is a very strong benchmark algorithm for optimizing myopic rewards.


\subsubsection{Whittle's index (QWI).} \label{sec:qwi_description}
The next benchmark is the Whittle's index.
The advantage of this method compared to the bandit benchmark is that is is non-myopic.
However, the downside is that computing the Whittle's index requires the model to be known.
To implement Whittle's index in our setting where the model is unknown, we leverage the recent work of \cite{avrachenkov2022whittle} who propose a Q-learning approach to learn the Whittle's index, which we refer to as $\QWI$.
$\QWI$ is an online learning method that simultaneously learns the Q-values as well as the Whittle's index for each state.

There are two main challenges in implementing $\QWI$ in our setting. The first is that the algorithm is an online learning method, and the second is that it requires a finite state space as it learns the Whittle's index for each state separately. 
For the first point, we adapt the algorithm from \cite{avrachenkov2022whittle} to an offline setting so that it can use the same data that is used to train $\DPI$.
For the second point, we define a smaller, finite state space so that $\QWI$ can be implemented.
We define a patient's state at a point in time to be a 3-tuple $(s_1, s_2, s_3)$, where $s_1$ represents the number of times the patient verified in the last week, $s_2$ is the patient's historical total verification rate, and $s_3$ is the number of times that the patient received an intervention in the last week. The values of each of these terms are bucketed into a small number of bins (3 bins for $s_1$ and $s_3$, 5 bins for $s_2$), resulting in 45 states in total. Specifically, the bins for both $s_1$ and $s_2$ are $0, 1$ and $2-7$. For $s_2$, the bins are $0-1\%$, $1-5\%$, $5-20\%$, $20-45\%$, and $45-100\%$. The bin values were chosen to balance the number of samples in each bin. 
This results in 45 states in total.


Based on this state space, we learn the Whittle's index, $\lambda(s) \in \bR$, for each state $s$.
Then, at each point in time, $\QWI$ chooses the patients in states with the highest Whittle's index to give the intervention to.
Further details of the learning algorithm is deferred to Appendix~\ref{sec:app:qwi}. 

\added{
We note that the state space for $\QWI$ is different from that of $\DPI$, due to the computational limitation of $\QWI$. In \cref{ss.state_space_results}, we run simulations where we modify the state space for $\DPI$ to be the same as $\QWI$, so that we can isolate the performance difference to the algorithm rather than the state space.
That said, we believe that the ease of working with an infinite state space is a substantial advantage of $\DPI$.
}




\subsubsection{Baseline.}
The $\baseline$ policy approximates the heuristic followed by Keheala in the two RCTs that have been implemented. In both cases, the protocol was that patients were added to the support sponsor call queue after not verifying for 48 hours. As a result, the order of patients in the queue is effectively random, determined by a combination of their designated medication time (which prompts automated reminders to take the medicine and verify) and the timing of their self-verification. We approximate the resulting outreach policy by selecting $B$ patients out of all those who have not verified for 48 hours, at random. 

\added{
The $\RAND$ policy used in the theoretical results is aimed to be an approximation of $\baseline$. 
The discrepancy between these policies is solely for technical convenience, 
as $\RAND$ is easier to analyze due to the independence across patients.
}


\subsubsection{Null policy.}
\added{
Lastly, we simulate the $\NULL$ policy which does not give any interventions. 
We note that this policy does not depend on the budget parameter $B$.
}

\subsection{Results}\label{ss.results}

The results are shown in Figure~\ref{fig:main_results}. 

\begin{figure}[h]
\begin{center}
\vspace{-3mm}
  \includegraphics[width=1\linewidth]{figs/results_June14_whittle}
  \caption{Average overall verification rate over 50 runs for each policy and budget. 
  The overall verification rate for the $\NULL$ policy was 54.2\%.
  The shaded region indicates a 95\% confidence interval. The star represents the operating point for Keheala.}
  \label{fig:main_results}
\vspace{-6mm}
\end{center}
\end{figure}

\subsubsection{Overall performance.}
The average performance for each budget and policy over 50 runs are shown in Figure~\ref{fig:main_results}, which shows that $\DPI$ clearly outperforms the other policies over a wide range of budget values.
For a practical interpretation of the results, consider \textsf{Baseline} at a budget of 26, the policy and budget that Keheala was operating during the RCT, which results in an overall verification rate of 62.0\%.
By using less than \textit{half} of the budget, $B=12$, $\DPI$ achieves the same verification rate at 62.2\%.
As the costliest aspect of Keheala's system is in hiring staff to provide the interventions, these results imply that they can cut these costs by half to achieve the same outcome.
\added{
The $\NULL$ policy (no interventions) results in a verification rate of 54.2\% (we did not plot this for readability of the figure).
One can interpret this number as a reference benchmark to compare the effectiveness of interventions.
When the budget is 26, $\baseline$ improves over $\NULL$ by 14.6\%, while $\DPI$ improves over $\NULL$ by 20.3\%. 
Therefore, $\DPI$ improves the effectiveness of the interventions over $\baseline$ by 38.3\%.
}

Additionally, we observe that the improvement of $\DPI$ compared to the other policies is especially substantial for smaller budgets. 
This implies that when the number of patients that can be targeted is small, $\DPI$ can correctly identify the set of patients to target that result in the largest gains.
This is especially valuable for scaling up the system.
Indeed, if Keheala wanted to expand to include more patients without linearly increasing their staff costs, then the ratio of budget to the number of patients would decrease, resulting in the regime where $\DPI$ offers major improvements.

The fact that the performance of $\bandit$ policy improves over $\baseline$ as the budget increases is caused by the increase in relevant data.
Note that the number of interventions is small ($\sim 26$) relative to the number of patients in the system at once ($\sim 600$), implying that the number of data points with $A=1$ is much smaller than that of $A=0$. 
Therefore, the main bottleneck in estimation is learning patient behaviors after receiving an intervention.
As the budget increases, the $\bandit$ has access to more data from patients with an intervention, and hence is able to improve its learning. 

$\QWI$ has a slightly inconsistent performance curve relative to the other policies.
Its performance is always better than or similar to $\baseline$, but compared to $\bandit$, it over-performs in the mid-budget regime, but under-performs as the budget increases.
We dive deeper into the types of patients each policy targets in \cref{sss.targetedpatients}, where we provide an explanation for this behavior.  


\added{
One factor that may be contributing to the poor performance of $\QWI$ is the state space that is used.
$\QWI$ uses a discretized state space as described in \cref{sec:qwi_description}, different than the infinitely-sized state space used by $\DPI$.
In \cref{ss.state_space_results}, we run additional experiments where we run $\DPI$ with the same state space as $\QWI$, so that the performance differences can be purely attributed to the algorithm rather than the state space. 
}


\subsubsection{Patient-level verification rates.} 
The overall number of verifications increases under $\DPI$, but how do these rates get impacted at the patient-level?
Fixing the budget to be 26, we compute the verification rate of \textit{each} patient, and we examine the distribution of these patient-level rates. \added{
In Figure~\ref{fig:diff_vrates_all}, we plot how the distribution of patient verification rates shift under the $\DPI$, $\bandit$, and $\QWI$ algorithms, compared to $\baseline$.
We see that under $\DPI$, the distribution shifts in a way that there are fewer patients with verification rates under 50\%, and more patients with a verification rate higher than 50\%.
We see a similar phenomenon for $\QWI$, but the magnitude of the shift is smaller.
$\bandit$ also observes an increase in $>50\%$ verification rates, but there is also an increase of those with very low (0-10\%) verification rates.
These results for $\DPI$ represent a desirable type of shift, where main improvement of $\DPI$ comes from an increase in the number of patients with a high verification rate.
We also provide absolute numbers in \cref{tab.verification_rates}, where we show the percentage of patients whose verification rate is higher than 50\% and 70\% for each of the four algorithms.
Under $\DPI$, the number of patients whose verification rate is above 50\% and 70\% increased relatively by 6.7\% and 5.2\% respectively compared to $\baseline$.
}


\begin{figure}
\centering
\begin{subfigure}{.47\textwidth}
  \centering
  \includegraphics[width=1\linewidth]{figs/2024_dpi-keheala}
  \caption{Comparing $\DPI$ to $\baseline$.}
\end{subfigure}%
\begin{subfigure}{.47\textwidth}
  \centering
  \includegraphics[width=1\linewidth]{figs/2024_bandit-keheala}
  \caption{Comparing $\bandit$ to $\baseline$.}
\end{subfigure} \\
\begin{subfigure}{.47\textwidth}
  \centering
  \includegraphics[width=1\linewidth]{figs/2024_qwi-keheala}
  \caption{Comparing $\QWI$ to $\baseline$.}
\end{subfigure}
\caption{
\added{
Differences in the distribution of patient verification rates compared to $\baseline$. 
  The bins represent the difference in the number of patients whose overall verification rate is between $0-10\%$, $10-20\%, \dots, 90-100\%$.
For example, the first bin in (a) shows that there were 28 fewer patients whose verification rate was between 0 and 10\% under $\DPI$, compared to $\baseline$.
 There were 3594 patients in total, and the budget was fixed at 26.
 }
}
  \label{fig:diff_vrates_all}
\end{figure}


\begin{table}[h]
\TableSpaced %
\caption{\added{
The average percentage of patients whose verification rate was over 50\% and 70\% across the four algorithms.
 There were 3594 patients in total, and the budget was fixed at 26.
}} \label{tab.verification_rates}
\vspace{2mm}
\begin{center}
\begin{tabular}{@{}c|cccc@{}}
\toprule
\% patients with \\ verification rate      & \quad $\baseline$ \;  & $\DPI$ & $\bandit$& $\QWI$ \\ \midrule
$\ge 50\%$  &  61.7\%   & 66.1\%  & 63.8\% & 63.9\% \\ 
$\ge 70\%$  &  38.2\%  & 40.2\%  & 39.7\%  & 39.5\% \\ \bottomrule
\end{tabular}
\end{center}
\end{table}



\subsubsection{Description of the targeted patients.} \label{sss.targetedpatients}
In \cref{tab.stats}, we fix the budget to be 26 and we show statistics regarding the state of the targeted patients for each of the four policies.
For example, under $\baseline$, on average, the patient that received an intervention had a treatment effect of 8.8\% with respect to the probability that they will verify the next day.
8.8\% is the `true' average treatment effect, in the sense that the numbers that are averaged are taken directly from the simulation model.


\begin{table}[h]
\TableSpaced %
\caption{
Average statistics of the state of patients who were given an intervention, across the three policies that were run for $B=26$.
(a) is the average value of $f(x, 1) - f(x, 0)$, the increase in probability that the patient verifies the next day when they are given an intervention. 
(b) is the average $f(x, 0)$, the probability that a patient verifies the next day \textit{without} an intervention. 
(c) is the average number of remaining days the patient will be on TB treatment for.
} \label{tab.stats}
\vspace{2mm}
\begin{center}
\begin{tabular}{@{}ccccc@{}}
\toprule
                                                 & $\baseline$ & $\DPI$ & $\bandit$ & $\QWI$ \\ \midrule
\multicolumn{1}{l}{(a) \added{$f(x, 1) - f(x, 0)$}}  & 8.8\%   & 10.6\%    & 13.2\%    & 6.9\%    \\ 
\multicolumn{1}{l}{(b) \added{$f(x, 0)$}}   & 18.2\%  & 22.2\%  & 35.2\%   & 12.2\%      \\ 
\multicolumn{1}{l}{(c) Days on TB treatment remaining} & 68.3     & 109.3   & 92.4    & 69.7      \\ \bottomrule
\end{tabular}
\end{center}
\end{table}



Statistic (a) represents exactly what the $\bandit$ policy optimizes for, the increase in probability of the patient verifying the next day.
The fact that $\bandit$ yields the highest value confirms that indeed, the policy correctly learns what it is supposed to learn.
$\DPI$ chooses patients with a higher one-step treatment effect than $\baseline$, but lower than that of $\bandit$.
Then, the fact $\DPI$ outperforms $\bandit$ in terms of overall verification implies that a myopic strategy of looking only one step ahead is not sufficient.
The next two statistics shed light on why this may be.


Statistic (b) represents the probability that the targeted patient would have verified anyway without an intervention, and we see that the $\bandit$ targets patients with a much higher verify probability than the other policies. 
We plot the entire distribution of this quantity in Figure~\ref{fig:base_probs}, where we see that $\bandit$ often targets those with a relatively high probability ($>45\%$), while $\DPI$ targets those with a relatively low probability ($<15\%$).
This may contribute to the improved performance of $\DPI$, and the reasoning for this can be seen through the two-state model from \cref{ss.2statemodel}, where statistic (b) corresponds to the parameter $p$.
If two patients have the same values of the parameters $g$ and $\tau$ but differing values for $p$, the intervention value is higher when $p$ is smaller (see \cref{prop:z_clean_form}).
This is because the patient with a high value of $p$ is more likely to switch to state 1 at the current time step as well as all future time steps.
As an extreme example, for a patient with $p=0$, they \textit{need} an intervention to switch to state 1, whereas a patient with $p>0$ may switch to state 1 (either now or in the future), without an intervention.
Therefore, an intervention is more likely to be helpful for those with a smaller value of $p$, which $\DPI$ targets.

On the other hand, $\QWI$ takes the above strategy to an extreme, where it targets those with a very low probability of verifying (12.2\%), but on average these patients also do not have a high next-day treatment effect (6.9\%).
This may explain the inconsistent behavior of $\QWI$ as the budget increases. 
The strategy of targeting these patients with a low verify probability and a low treatment effect is reasonably effective in the mid-budget regime; however, as the budget increases, one may also need to judiciously target other types of patients, which $\QWI$ does not do. 





\begin{figure}[h]
\begin{center}
\vspace{-5mm}
  \includegraphics[width=0.54\linewidth]{figs/base_probs_June2024_2}
\vspace{-2mm}
  \caption{
  Histogram of the value of $f(x, 0)$ of targeted patients, the probability that the patients would verify without an intervention.
  This is the entire distribution of the statistic (b) in \cref{tab.stats} for $\bandit$ and $\DPI$.}
  \label{fig:base_probs}
\vspace{-6mm}
\end{center}
\end{figure}

Lastly, statistic (c) is the average number of days a targeted patient has remaining on the platform.
If an intervention positively affects patients for all of their future time steps, then targeting those with longer time left in the system would result in higher benefits. 
The results show that $\DPI$ targets those with the longest days of treatment left.

\subsubsection{Prominent features for $\DPI$.}
\cref{tab.coefs} displays the five most predictive features of the intervention values that $\DPI$ uses to target patients.
These features were found by using Lasso regression with a tuned parameter -- see Section~\ref{s.app.coefficients} for details on the method used.
The results show that the intervention value is lower when the number of previous interventions is higher (first two features), which is intuitive since patients may become fatigued and less receptive when there are too many interventions.
The intervention value is lower when the patient's past verifications is higher (third and fourth features). This is consistent with the analysis in \cref{tab.stats}, where $\DPI$ targets those with a smaller value of $f(x, 0)$.
Lastly, the intervention value increases for patients who are older.


\begin{table}[h]
\TableSpaced %
\caption{
The most predictive features of higher intervention values for $\DPI$, as well as the sign of their coefficient. 
} \label{tab.coefs}
\vspace{2mm}
\begin{center}
\begin{tabular}{@{}lc@{}}
\toprule
\; Feature & Sign of Coefficient  \\ \midrule
\; Interventions: total number & $-$ \\ 
\; Interventions: \# previous week &  $-$ \\
\; Verifications: overall percentage &  $-$ \\
\; Verifications: \# previous week & $-$ \\
\; Age & $+$ \\
\bottomrule
\end{tabular}
\end{center}
\end{table}



\subsection{\added{Robustness of State Space Representation}} \label{ss.state_space_results}
\added{One of the factors attributing to the performance gap between $\DPI$ and $\QWI$ is the differences in state space representation. $\QWI$ requires a finite state space and its computation time scales with the number of states. Hence, we use a relatively small state space for $\QWI$ for our numerical experiments. The ease of using a larger and infinite size state space is an inherent advantage of $\DPI$ over $\QWI$; however, in order to isolate the performance difference caused by the algorithm itself, we run $\DPI$ using the same state space as $\QWI$.

We try two variants of $\DPI$ that differ in the state space used:
\begin{itemize}
	\item \textsf{DecompPI-3}: This uses the same three features used for $\QWI$ (number of times verified in the last week, total historical verification rate, and number of interventions received in the last week), but these features are \textit{not discretized}, and hence the size of the state space is still infinite.
	\item \textsf{DecompPI-3-discrete}: This uses the exact same state space as $\QWI$ --- three features that are discretized in the same way, resulting in 45 states.
\end{itemize}


\begin{figure}[h]
\begin{center}
\vspace{-3mm}
  \includegraphics[width=1\linewidth]{figs/2024_May13_results3D}
  \caption{Average overall verification rate over 50 runs for each policy and budget. The shaded region indicates a 95\% confidence interval. The star represents the operating point for Keheala.}
  \label{fig:DPI3}
\vspace{-6mm}
\end{center}
\end{figure}

The performance of these two algorithms, along with the original $\DPI$ and $\QWI$ policies are shown in Figure~\ref{fig:DPI3}.
The policies \textsf{DecompPI-3-discrete} and $\QWI$ use the exact same state space, and we see that the former consistently outperforms the latter.
This comparison isolates the performance gap induced by the \emph{algorithms}, and the results provide robust evidence on the strength of $\DPI$.
Lastly, we see that \textsf{DecompPI-3} consistently has a strong performance, comparable to that of $\DPI$.
This demonstrates the robustness of $\DPI$ with respect to the feature space, and it also exemplifies the benefit of employing an infinite state space compared to a discretized one.  
}















\paragraph{Case Study.}
% To analyze how comments hint cross-modal understanding,  attention maps over comments  are shown in Figure \ref{fig:case_study}, where a case is sampled from each benchmark. 
To analyze how comments hint at cross-modal understanding,  attention maps over comments  are shown in Figure \ref{fig:case_study}, where the case is sampled from the MSD benchmark. 
% \footnote{More cases could be found in Appendix \ref{appendix:case}.}
As can be seen, models tend to capture the salient comments mentioning the key visual objects, e.g., ``owl'' and ``traffic camera'' in the case, helpfully connecting visual semantics to lingual.
It is probably because human readers are likely to echo crucial points in their comments in response to what they viewed from a post, which inspires models to explicitize the weakly-connected visual-lingual semantics.


%We can observe that the retrieved comments could provide the necessary context from human views to fill the semantic gap between images and texts. Taking the third case for example, the text ``snowy owl is being ticketed for not reading the signs'', the model might be fused to distinguish the label of the post. However, the comment ``a traffic camera caught this amazing pic of an owl in flight'' could add external background context of the post. Therefore, the model can reasoning from the comments and  predict the true label.

\begin{table}[htbp]
\begin{center}
\caption{\textit{Specialized} Model Error Counts by MTC Type. For example, of the 10 labeled consistency MTC (type 6) extraction errors, 7 are hallucinations. The other 3 have undetermined error sources.}
\begin{tabular}{|l|r|r|r|r|}
\hline
\thead{MTC Type} &\thead{Hallucination} &\thead{Semantic \\ Overlap} &\thead{Nonvalidity} & \thead{Other} \\
\hline
\makecell{Definitive \\ Dependency (1)} &3	&2	&1	&4 \\
\hline
\makecell{Frequency (2)} &0	&0	&4	&6 \\
\hline
\makecell{Interval (3)} &5	&1	&3	&1 \\
\hline
\makecell{Imprecise \\ Dependency (4)} &4	&4	&0	&2 \\
\hline
\makecell{Consistency (6)} &7	&0	&0	&3 \\
\hline
\makecell{Time-of-Day (7)} &7	&2	&1	&0 \\
\hline
\end{tabular}
\label{table:error_analysis}
\end{center}
\end{table}


\paragraph{Error Analysis.}
%\begin{table}[htbp]
\begin{center}
\caption{\textit{Specialized} Model Error Counts by MTC Type. For example, of the 10 labeled consistency MTC (type 6) extraction errors, 7 are hallucinations. The other 3 have undetermined error sources.}
\begin{tabular}{|l|r|r|r|r|}
\hline
\thead{MTC Type} &\thead{Hallucination} &\thead{Semantic \\ Overlap} &\thead{Nonvalidity} & \thead{Other} \\
\hline
\makecell{Definitive \\ Dependency (1)} &3	&2	&1	&4 \\
\hline
\makecell{Frequency (2)} &0	&0	&4	&6 \\
\hline
\makecell{Interval (3)} &5	&1	&3	&1 \\
\hline
\makecell{Imprecise \\ Dependency (4)} &4	&4	&0	&2 \\
\hline
\makecell{Consistency (6)} &7	&0	&0	&3 \\
\hline
\makecell{Time-of-Day (7)} &7	&2	&1	&0 \\
\hline
\end{tabular}
\label{table:error_analysis}
\end{center}
\end{table}

%As mentioned above, the main gain of the proposed comment-aware self-training method come from the utilization of retrieval results. 
%Here we analyze error cases, which are mostly caused by 
The potential benefit has been potentially observed in varying cross-modal learning scenarios; however, many errors are also related to the retrieved comments and posts used in self-training.
Figure \ref{fig:error_analysis} summarizes the two major error types.
%Therefore, the quality of retrieval results could greatly influence the performance. 
%Here we summarize two main types of bad retrieval results which 
%are shown in : 

First, general comments, e.g., ``thank you'', are retrieved, useless in learning specific meanings in social media posts. This calls for a future direction for comment selection in a more effective manner.
%is general and useless for understanding the post. 
%the potential solution is 
%to design special rules to filter the retrieved comments. 
%
Second, semantically unrelated posts might be retrieved due to the misunderstanding to the query and hence result in irrelevant comments. 
For example, the posts concerning ``tomato'' are wrongly retrieved because of its similar color to ``beet sause''.
Future work may consider the advance in similarity measurement of cross-modal posts and the detection of high-quality unlabeled data 
(e.g., selecting the 
%highest-confidence 
pseudo-labeled data by confidence) for self-training.
 
%for self-training.

%The retrieved image-text pairs used for self-training are not related to the query post. 
%This reason might be that the similar posts are not included in the retrieval dataset, and the image encoder couldn't capture accurate objects. 
%Expanding the scale of the retrieval dataset and using more advanced image encoder could improve the quality of retrieved similar posts. Additionally, selecting the highest-confidence pseudo-labeled data instead of all retrieved similar posts might solve the dissimilar problem and improve the performance, which could be tried in future.



\section{Conclusion}\label{sec:conclusion}
In this work, we focus on addressing the fundamental challenge of OOD detection tasks, which is how to fully understand the semantic discrepancy between the ID/OOD samples. We reveal that the key to success in the realistic SCOOD task is to allocate as many ID samples in the unlabeled set correctly as possible. To this end, we propose a novel uncertainty-aware optimal transport scheme that introduces class-specific energy scores as guidance for effective label assignment. Experimental results show that our method achieves better performance than previous state-of-the-art methods on SCOOD benchmarks.

\textbf{Limitations.} In addition to temperature scaling, other techniques such as feature clipping applied in ReAct~\cite{sun2021react} also enhance the performance of energy score, so how to obtain an OOD score that best fits the SCOOD task can be further explored. Moreover, a setting highly related to SCOOD has been proposed in \cite{katz2022training} and formulated as a constrained optimization problem. We will also theoretically analyze these practical OOD settings in our feature work.

% \section*{Acknowledgments}
\textbf{Acknowledgments.} 
This work is supported by National Key R\&D Program of China under Grant 2020AAA0105701, National Natural Science Foundation of China (NSFC) under Grants 61872327, Major Special Science and Technology Project of Anhui, National Natural Science Foundation of China (62033012) and Ant Group through Ant Research Intern Program.

\section*{Acknowledgments}

Preprocessing of the GPX files made use of the resources provided by the Edinburgh Compute and Data Facility (ECDF) \cite{Eddie2022website}.



\section{Generalization, Limitation and Future Work}
The Matcha framework exhibits a high degree of generalizability thanks to the commonsense knowledge inside LLMs.
Without LLMs, a control algorithm, e.g. one trained with reinforcement learning \cite{Li23InternallyRewarded, Singh20COGConnecting}, may require massive datasets/interactions to learn
the common sense \cite{Singh20COGConnecting} of collaborating different modalities, yet being less efficient and generalizable.

However, interpreting the real world with language can be limited to the complexity of the task and the environment dynamics.
For example, advanced reasoning techniques such as decomposing may be required to deal with a complicated task,
where the task is decomposed into several sub-tasks to tackle separately. 
This automatic operation highlights the flexibility of LLMs but also poses challenges to the static language expression of a complex world
--- The vision-to-language module should be called multiple times with flexible queries.
This brings the requirement of vision-enabled LLMs \cite{Zhu23MiniGPT4Enhancing, Brohan23RT2Visionlanguageaction}, 
built on which the reasoning can be malleable. But multimodal LLMs are yet less controllable and accurate in terms of describing the scene
compared with a templated module.

Despite current limitations, multimodal LLMs gain increasing attention due to their great potential and flexibility.
Future work will explore the multimodal models \cite{Tong22VideoMAEMasked, Brohan23RT2Visionlanguageaction} to leverage unified features.
\section{Compliance with ethical standards}
This research study was conducted retrospectively using human subject data made available in open access by the Genomic Data Commons (GDC) provided by the National Cancer Institute of the National Institues of Health (NIH/NCI). Ethical approval was not required as confirmed by the license attached with the open access data.
% Entries for the entire Anthology, followed by custom entries
\bibliography{anthology}
\bibliographystyle{acl_natbib}
\clearpage
% \section{Appendix for Proofs}

\paragraph{Proof of Theorem \ref{thm:main}.}

\begin{proof}
\label{proof:main}
Our proof has two steps. In Step 1, we will show that SimCLR is equivalent to minimizing the cross entropy loss defined in Eqn.~(\ref{eqn:cross-entropy}). 
In Step 2, we will show  that minimizing the cross-entropy loss 
is equivalent to spectral clustering on $\bfpi$. 
Combining the two steps together, we have proved our theorem. 

\textbf{Step 1: } SimCLR is equivalent to minimizing the cross entropy loss.

The cross-entropy loss takes expectation over 
$\bfW_\bfX\sim \mathbb{P}(\cdot ; \bfpi)$, 
which means $\bfW_\bfX$ has exactly one non-zero entry in each row $i$. By Lemma~\ref{lem:multinomial}, we know every row $i$ of $\bfW_\bfX$ is independent of other rows. Moreover, 
$\bfW_{\bfX,i}\sim \mathcal{M}(1, \bfpi_i/\sum_j \bfpi_{i,j})=\mathcal{M}(1, \bfpi_i)$, because $\bfpi_i$ itself is a probability distribution.
Similarly, we know $\bfW_\bfZ$ also has the row-independent property by sampling over $\mathbb{P}(\cdot;\bfK_\bfZ)$.
Therefore, by Lemma~\ref{lem:cross_split}, we know Eqn.~(\ref{eqn:cross-entropy}) is equivalent to:
\[
 -\sum_{i=1}^n \mathbb{E}_{\bfW_{\bfX,i}}[\log \mathbb{P}(\bfW_{\bfZ,i}=\bfW_{\bfX,i};\bfK_\bfZ)],
\]

This expression takes expectation over $\bfW_{\bfX,i}$ for the given row $i$. Notice that 
$\bfW_{\bfX,i}$ has exactly one non-zero entry, which equals $1$ (same for $\bfW_{\bfZ,i}$). 
As a result
we expand the above expression to be:
\begin{equation}
 -\sum_{i=1}^n \sum_{j\neq i} \Pr(\bfW_{\bfX,i,j}=1)\log \Pr(\bfW_{\bfZ,i,j}=1).
\label{eqn:detailed-expansion}    
\end{equation}


By Lemma~\ref{lem:multinomial}, $\Pr(\bfW_{\bfZ,i,j}=1)=\bfK_{\bfZ,i,j}/\|\bfK_{\bfZ,i}\|_1$ for $j\neq i$. Recall that $\bfK_\bfZ=(k(\bfZ_i-\bfZ_j))_{(i,j)\in[n]^2}$, which means 
$\bfK_{\bfZ,i,j}/\|\bfK_{\bfZ,i}\|_1=\frac{\exp(-\|\bfZ_i-\bfZ_j\|^2/{2\tau})}{\sum_{k\neq i}
\exp(-\|\bfZ_i-\bfZ_k\|^2/{2\tau})
}$ for $j\neq i$, when $k$ is the Gaussian kernel with variance $\tau$. 

Notice that $\bfZ_i=f(\bfX_i)$, so we know
\begin{equation}
-\log \Pr(\bfW_{\bfZ,i,j}=1)=
-\log \frac{\exp(-\|f(\bfX_i)-f(\bfX_j)\|^2/{2\tau})}{\sum_{k\neq i}
\exp(-\|f(\bfX_i)-f(\bfX_k)\|^2/{2\tau}),
}
\label{eqn:infonce-equivalence}    
\end{equation}


The right hand side is exactly the InfoNCE loss defined in Eqn.~(\ref{eqn:infonce}).
Inserting Eqn.~(\ref{eqn:infonce-equivalence}) into Eqn.~(\ref{eqn:detailed-expansion}), we get the SimCLR algorithm, which first samples augmentation pairs $(i,j)$ with $\Pr(\bfW_{\bfX,i,j}=1)$ for each row $i$, and then optimize the InfoNCE loss. 

\textbf{Step 2: } minimizing the cross entropy loss 
is equivalent to spectral clustering on $\bfpi$.


By Lemma~\ref{lem:convert_to_spectral}, we may further convert the loss to 
\begin{equation}
\label{eqn:main-theorem-repul-attr}
\min_{\bfZ}
-\sum_{(i,j)\in [n]^2} \mathbf{P}_{i,j}
\log k (\bfZ_i-\bfZ_j)+\log \mathbf{R}(\bfZ).
\end{equation}
Since $k$ is the Gaussian kernel, this reduces to \[
\min_\bfZ \mathrm{tr}(\bfZ^\top \mathbf{L}(\bfpi) \bfZ)
+\log \mathbf{R}(\bfZ),
\]

where we use the fact that $\mathbb{E}_{\bfW_\bfX\sim \mathbb{P}(\cdot; \bfpi)}[\mathbf{L}(\bfW_\bfX)]
=\mathbf{L}(\bfpi)
$, because the Laplacian operator is linear and $
\mathbb{E}_{\bfW_\bfX\sim \mathbb{P}(\cdot; \bfpi)}(\bfW_\bfX)=\bfpi
$.
\end{proof}

\paragraph{Proof of Theorem \ref{thm:clip}.}
\begin{proof}
Since $\bfW_\bfX\sim \mathbb{P}(\cdot;\bfpi_{\mathbf{A}, \mathbf{B}})$, we know 
$\bfW_\bfX$ has exactly one non-zero entry in each row, denoting the pair that got sampled. 
A notable difference compared to the previous proof is we now have $n_\mathcal{A}+n_\mathcal{B}$ objects in our graph. CLIP deals with this by taking a mini-batch of size $2N$, 
such that $n_\mathcal{A}=n_\mathcal{B}=N$, and adding the $2N$ InfoNCE losses together. We label the objects in $\mathcal{A}$ as $[n_\mathcal{A}]$, and the objects in $\mathcal{B}$ as $\{n_\mathcal{A}+1, \cdots, n_\mathcal{A}+n_\mathcal{B}\}$. 

Notice that $\bfpi_{\mathbf{A}, \mathbf{B}}$ is a bipartite graph, so the edges of objects in $\mathcal{A}$ will only connect to object in $\mathcal{B}$ and vice versa. We can define the similarity matrix in $\cZ$ as $\bfK_\bfZ$, 
where $\bfK_\bfZ(i, j+n_\mathcal{A})=\bfK_\bfZ(j+n_\mathcal{A},i)= k(\bfZ_i-\bfZ_j)$ for $i\in [n_\mathcal{A}], j\in [n_\mathcal{B}]$, and otherwise we set $\bfK_\bfZ(i,j)=0$. 
The rest is same as the previous proof. 
\end{proof}

\paragraph{Proof of Theorem \ref{thm:exponential}.}

\begin{proof}
\label{proof:exponential}
Since the objective function consists of a linear term combined with an entropy regularization, which is a strongly concave function, the maximization problem is a convex optimization problem. Owing to the implicit constraints provided by the entropy function, the problem is equivalent to having only the equality constraint. We then introduce the Lagrangian multiplier $\lambda$ and obtain the following relaxed problem:

$$
\widetilde{E}(\boldsymbol{\alpha})=\psi_{1}-\sum_{i=1}^n \alpha_{i} \psi_{i}+\tau \sum_{i=1}^n \alpha_{i}\log \alpha_{i}+\lambda\left(\boldsymbol{\alpha}^{\top} \mathbf{1}_n-1\right).
$$

As the relaxed problem is unconstrained, taking the derivative with respect to $\alpha_{i}$ yields

$$
\frac{\partial \widetilde{E}(\boldsymbol{\alpha})}{\partial \alpha_{i}}=-\psi_{i}+\tau\left(\log \alpha_{i}+\alpha_{i} \frac{1}{\alpha_{i}}\right)+\lambda=0.
$$

Solving the above equation implies that $\alpha_{i}$ takes the form
$
\alpha_{i}=\exp \left(\frac{1}{\tau} \psi_{i}\right) \exp \left(\frac{-\lambda}{\tau}-1\right).
$ Since $\alpha_{i}$ lies on the probability simplex, the optimal $\alpha_{i}$ is explicitly given by
$
\alpha^{*}_{i}=\frac{\exp \left(\frac{1}{\tau} \psi_{i}\right)}{\sum_{i^{\prime}=1}^n \exp \left(\frac{1}{\tau} \psi_{i^{\prime}}\right)} .
$ Substituting the optimal point into the objective function, we obtain
$$
\begin{aligned}
E\left(\boldsymbol{\alpha}^*\right)  &=\psi_1-\sum_{i=1}^n \frac{\exp \left(\frac{1}{\tau} \psi_{i}\right)}{\sum_{i^{\prime}=1}^n \exp \left(\frac{1}{\tau} \psi_{i^{\prime}}\right)} \psi_{i}+\tau \sum_{i=1}^n \frac{\exp \left(\frac{1}{\tau} \psi_{i}\right)}{\sum_{i^{\prime}=1}^n \exp \left(\frac{1}{\tau} \psi_{i^{\prime}}\right)}\log \frac{\exp \left(\frac{1}{\tau} \psi_{i}\right)}{\sum_{i^{\prime}=1}^n \exp \left(\frac{1}{\tau} \psi_{i^{\prime}}\right)} \\
& =\psi_1 - \tau \log \left(\sum_{i=1}^n \exp \left(\frac{1}{\tau} \psi_{i}\right)\right).
\end{aligned}
$$
Thus, the Lagrangian dual function is given by
\begin{equation*}
-E\left(\boldsymbol{\alpha}^*\right)= -\tau \log \frac{\exp \left(\frac{1}{\tau} \psi_{1}\right)}{\sum_{i=1}^n \exp \left(\frac{1}{\tau} \psi_{i}\right)}.\qedhere
\end{equation*}
\end{proof}



\section{More on Experiments} \label{section: experiment_details}

\paragraph{CIFAR-10 and CIFAR-100} CIFAR-10 ~\citep{krizhevsky2009learning} and CIFAR-100 ~\citep{krizhevsky2009learning} are well-known classic image classification datasets. Both CIFAR-10 and CIFAR-100 contain a total of 60k $32 \times 32$ labeled images of different classes, with 50k for training and 10k for testing. CIFAR-10 is similar to CIFAR-100, except there are 10 different classes in CIFAR-10 and 100 classes in CIFAR-100.

\paragraph{TinyImageNet} TinyImageNet ~\citep{le2015tiny} is a subset of ImageNet ~\citep{deng2009imagenet}. There are 200 different object classes in TinyImageNet, with 500 training images, 50 validation images, and 50 test images for each class. All the images in TinyImageNet are colored and labeled with a size of $64 \times 64$.

\textbf{Pseudo-code.} Algorithm \ref{alg:Training Procedure} presents the pseudo-code for our empirical training procedure.

\begin{algorithm}[!htbp]
\caption{Training Procedure}
\label{alg:Training Procedure}
\begin{algorithmic}[1]
\REQUIRE trainable encoder network $f$, batch size $N$, augmentation strategy \textit{aug}, loss function $L$ with hyperparameters \textit{args}
\FOR {sampled minibatch ${x_i}_{i=1}^N$}
\FORALL{$i \in { 1, ..., N }$}
\STATE draw two augmentations $t_i = \textit{aug}\left(x_i\right) $, $t_i' = \textit{aug}\left(x_i\right) $
\STATE $z_i = f\left(t_i\right)$, $z_i' = f\left(t_i'\right)$
\ENDFOR
\STATE compute loss $\mathcal{L} = L(N, z, z', \textit{args})$
\STATE update encoder network $f$ to minimize $\mathcal{L}$
\ENDFOR
\STATE \textbf{Return} encoder network $f$
\end{algorithmic}
\end{algorithm}

We also provide the pseudo-code for our core loss function used in the training procedure in Algorithm \ref{alg:Core loss}. The pseudo-code is almost identical to SimCLR's loss function, with the exception of an extra parameter $\gamma$.

\begin{algorithm}[!htbp]
\caption{Core loss function $\mathcal{C}$}
\label{alg:Core loss}
\begin{algorithmic}[1]
\REQUIRE batch size $N$, two encoded minibatches $z_1, z_2$, $\gamma$, temperature $\tau$
\STATE $z = \textit{concat}\left(z_1, z_2\right)$
\FOR {$i \in {1, ..., 2N }, j \in {1, ..., 2N}$ }
\STATE $s_{i,j} = \Vert z_i - z_j \Vert_2^{\gamma}$
\ENDFOR
\STATE \textbf{define} $l(i, j)$ \textbf{as} $l(i, j) = - \log \frac{exp\left(s_{i,j}/\tau \right)}{\sum_{k=1}^{2N} \mathbf{1}{[k \ne i]} exp\left(s{i, j} / \tau \right)} $
\STATE \textbf{Return} $\frac{1}{2N} \sum_{k=1}^N\left[l(i, i+N) + l(i+N, i)\right]$
\end{algorithmic}
\end{algorithm}

Utilizing the core loss function $\mathcal{C}$, we can define all kernel loss functions used in our experiments in Table \ref{table: loss definition}. For all $z_i \in z$ with even dimensions $n$, we define $z_{L_i} = z_i\left[0:n/2\right]$ and $z_{R_i} = z_i\left[n/2:n\right]$.

\begin{table}[ht]
\centering
\begin{tabular}{{@{}l|l@{}}}
Kernel  &  Loss function \\ \midrule
Laplacian & $\mathcal{C}\left(N, z, z', \gamma=1, \tau\right)$\\ \midrule
Sum       & $\lambda * \mathcal{C}\left(N, z, z', \gamma=1, \tau_1\right) + (1-\lambda) * \mathcal{C}\left(N, z, z', \gamma=2, \tau_2\right)$  \\ \midrule
Concatenation Sum&$\lambda * \mathcal{C}\left(N, z_L, z'_L, \gamma=1, \tau_1\right) + (1-\lambda) * \mathcal{C}\left(N, z_R, z'_R, \gamma=2, \tau_2\right)$\\ \midrule
$\gamma = 0.5$ & $\mathcal{C}\left(N, z, z', \gamma=0.5, \tau\right)$          \\ 

\end{tabular}

\caption{Definition of kernel loss functions in our experiments}
\label {table: loss definition}
\end{table}

\textbf{Baselines.} We reproduce the SimCLR algorithm using PyTorch Lightning~\citep{PytorchLightning}.

\textbf{Encoder details.}
The encoder $f$ consists of a backbone network and a projection network. We employ ResNet50~\citep{ResNet} as the backbone and a 2-layer MLP (connected by a batch normalization~\citep{ioffe2015batch} layer and a ReLU \cite{nair2010rectified} layer) with hidden dimensions 2048 and output dimensions 128 (or 256 in the concatenation kernel case).

\textbf{Encoder hyperparameter tuning.}
For each encoder training case, we randomly sample 500 hyperparameter groups (sample details are shown in Table \ref{table: Hyperparameter sample}) and train these samples simultaneously using Ray Tune ~\citep{RayTune}, with the ASHA scheduler~\citep{li2018massively}. Ultimately, the hyperparameter group that maximizes the online validation accuracy (integrated in PyTorch Lightning) within 5000 validation steps is chosen for the given encoder training case.

\begin{table}[ht]
\centering

\begin{tabular}{@{}l|l|l@{}}
\midrule
Hyperparameter  & Sample Range & Sample Strategy \\ \midrule
start learning rate & $\left[10^{-2}, 10\right]$ & log uniform \\ \midrule
$\lambda$       & $\left[0, 1\right]$ & uniform \\ \midrule
$\tau$, $\tau_1$, $\tau_2$ & $\left[0, 1\right]$ & log uniform \\ \midrule
\end{tabular}

\caption{Hyperparameters sample strategy}
\label {table: Hyperparameter sample}
\end{table}

\textbf{Encoder training.} 
We train each encoder using the LARS optimizer~\citep{LARSOptimizer}, LambdaLR Scheduler in PyTorch, momentum 0.9, weight decay $10^{-6}$, batch size 256, and the aforementioned hyperparameters for 400 epochs on a single A-100 GPU.

\textbf{Image transformation.} The image transformation strategy, including augmentation, is identical to the default transformation strategy provided by PyTorch Lightning.

\textbf{Linear evaluation.}
The linear head is trained using the SGD optimizer with a cosine learning rate scheduler, batch size 64, and weight decay $10^{-6}$ for 100 epochs. The learning rate starts at $0.3$ and ends at $0$.

\textbf{Moco Experiments.} We also tested our method based on MoCo~\citep{he2019moco}. The results are summarized in Table \ref{tab:results-moco}. Here we choose ResNet18~\citep{ResNet} as the backbone and set a temperature of $0.1$ as default. For our simple sum kernel, we set $\lambda=0.8$. The results show that our method outperforms the original MoCo method.

\begin{table}[thb]
\centering
\caption{MoCo Experiment Results on CIFAR-10 and CIFAR-100.}
\label{tab:results-moco}
\resizebox{\textwidth}{!}{%
\begin{tabular}{@{}c|ccc|ccc@{}}
\toprule
\multirow{3}{*}{Method} & \multicolumn{3}{c|}{CIFAR-10} & \multicolumn{3}{c}{CIFAR-100} \\ \cmidrule(lr){2-4} \cmidrule(lr){5-7} 
                        & 200 epochs & 400 epochs    & 1000 epochs   & 200 epochs & 400 epochs & 1000 epochs         \\ \midrule
MoCo (repro.)         & $76.41 \pm 0.12$    & $80.01 \pm 0.15$          & $84.45 \pm 0.08$    & $\mathbf{47.02 \pm 0.11}$ & $52.50 \pm 0.07$ & $57.62 \pm 0.15$            \\
\midrule
Laplacian Kernel        & ${78.09 \pm 0.10}$    & $\mathbf{83.85 \pm 0.09}$          & $\mathbf{88.34 \pm 0.16}$    & $46.12 \pm 0.22$   & $53.44 \pm 0.17$ & $59.10 \pm 0.14$        \\
Simple Sum Kernel & $\mathbf{78.12 \pm 0.15}$   & $83.23 \pm 0.18$ & $87.50 \pm 0.20$ & $46.65 \pm 0.06$ & $\mathbf{53.62 \pm 0.19}$ & $\mathbf{59.83 \pm 0.12}$\\
\bottomrule
\end{tabular}
}
\end{table}



\section{More Experiments on Synthetic Data}


Consider a scenario with $n$ clusters, each containing $k$ vertices. Let the probability of vertices $u$ and $v$ from the same cluster belonging to $\bfpi$ be $p$. Conversely, for vertices $u$ and $v$ from different clusters, let the probability of belonging to $\pi$ be $q$. We generate the graph $\bfpi$ randomly, based on $p$ and $q$. We experiment with values of $k=100$ and $n=6$ for ease of visualization, embedding all points in a two-dimensional space. Each vertex's initial position originates from a normal distribution. In each iteration, we sample a subgraph of $\bfpi$ uniformly, ensuring each vertex has an out-degree of $1$. We then optimize the corresponding vectors using InfoNCE loss with an SGD optimizer and iterate until convergence. Our experimental setup consists of an SGD learning rate of $1$, an InfoNCE loss temperature of $0.5$, and a batch size of $50$. We evaluate two scenarios with different $p$ and $q$ values: $p=1$, $q=0$, and $p=0.75$, $q=0.2$. The results of these experiments are visualized in Figure \ref{fig:vis-spectral-cluster}. The obtained embeddings exhibit the hallmark pattern of spectral clustering of graph $\bfpi$.

\begin{figure}[!tb]
\centering
\subfigure{
\includegraphics[width=1\textwidth]{Figures/cluster_pi.png}
\label{fig:vis-cluster}
}
\subfigure{
\includegraphics[width=1\textwidth]{Figures/noised_cluster_pi.png}
\label{fig:vis-noised-cluster}
}
\caption{Visualizations of the optimization process using InfoNCE Loss on the vectors corresponding to $\bfpi$. Points of identical color belong to the same cluster within $\bfpi$. To showcase the internal structure of $\bfpi$, we randomly select 10 vertices from each cluster to display the edge distribution of $\bfpi$.}
\label{fig:vis-spectral-cluster}
\end{figure}





\end{document}