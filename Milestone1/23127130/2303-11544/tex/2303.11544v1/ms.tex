%% Beginning of file 'sample631.tex'
%%
%% Modified 2022 May  
%%
%% This is a sample manuscript marked up using the
%% AASTeX v6.31 LaTeX 2e macros.
%%
%% AASTeX is now based on Alexey Vikhlinin's emulateapj.cls 
%% (Copyright 2000-2015).  See the classfile for details.

%% AASTeX requires revtex4-1.cls and other external packages such as
%% latexsym, graphicx, amssymb, longtable, and epsf.  Note that as of 
%% Oct 2020, APS now uses revtex4.2e for its journals but remember that 
%% AASTeX v6+ still uses v4.1. All of these external packages should 
%% already be present in the modern TeX distributions but not always.
%% For example, revtex4.1 seems to be missing in the linux version of
%% TexLive 2020. One should be able to get all packages from www.ctan.org.
%% In particular, revtex v4.1 can be found at 
%% https://www.ctan.org/pkg/revtex4-1.

%% The first piece of markup in an AASTeX v6.x document is the \documentclass
%% command. LaTeX will ignore any data that comes before this command. The 
%% documentclass can take an optional argument to modify the output style.
%% The command below calls the preprint style which will produce a tightly 
%% typeset, one-column, single-spaced document.  It is the default and thus
%% does not need to be explicitly stated.
%%
%% using aastex version 6.3
%\documentclass[manuscript,linenumbers]{aastex631}
%\documentclass[twocolumn,linenumbers]{aastex631}
\documentclass[twocolumn]{aastex631}

%% The default is a single spaced, 10 point font, single spaced article.
%% There are 5 other style options available via an optional argument. They
%% can be invoked like this:
%%
%% \documentclass[arguments]{aastex631}
%% 
%% where the layout options are:
%%
%%  twocolumn   : two text columns, 10 point font, single spaced article.
%%                This is the most compact and represent the final published
%%                derived PDF copy of the accepted manuscript from the publisher
%%  manuscript  : one text column, 12 point font, double spaced article.
%%  preprint    : one text column, 12 point font, single spaced article.  
%%  preprint2   : two text columns, 12 point font, single spaced article.
%%  modern      : a stylish, single text column, 12 point font, article with
%% 		  wider left and right margins. This uses the Daniel
%% 		  Foreman-Mackey and David Hogg design.
%%  RNAAS       : Supresses an abstract. Originally for RNAAS manuscripts 
%%                but now that abstracts are required this is obsolete for
%%                AAS Journals. Authors might need it for other reasons. DO NOT
%%                use \begin{abstract} and \end{abstract} with this style.
%%
%% Note that you can submit to the AAS Journals in any of these 6 styles.
%%
%% There are other optional arguments one can invoke to allow other stylistic
%% actions. The available options are:
%%
%%   astrosymb    : Loads Astrosymb font and define \astrocommands. 
%%   tighten      : Makes baselineskip slightly smaller, only works with 
%%                  the twocolumn substyle.
%%   times        : uses times font instead of the default
%%   linenumbers  : turn on lineno package.
%%   trackchanges : required to see the revision mark up and print its output
%%   longauthor   : Do not use the more compressed footnote style (default) for 
%%                  the author/collaboration/affiliations. Instead print all
%%                  affiliation information after each name. Creates a much 
%%                  longer author list but may be desirable for short 
%%                  author papers.
%% twocolappendix : make 2 column appendix.
%%   anonymous    : Do not show the authors, affiliations and acknowledgments 
%%                  for dual anonymous review.
%%
%% these can be used in any combination, e.g.
%%
%% \documentclass[twocolumn,linenumbers,trackchanges]{aastex631}
%%
%% AASTeX v6.* now includes \hyperref support. While we have built in specific
%% defaults into the classfile you can manually override them with the
%% \hypersetup command. For example,
%%
%% \hypersetup{linkcolor=red,citecolor=green,filecolor=cyan,urlcolor=magenta}
%%
%% will change the color of the internal links to red, the links to the
%% bibliography to green, the file links to cyan, and the external links to
%% magenta. Additional information on \hyperref options can be found here:
%% https://www.tug.org/applications/hyperref/manual.html#x1-40003
%%
%% Note that in v6.3 "bookmarks" has been changed to "true" in hyperref
%% to improve the accessibility of the compiled pdf file.
%%
%% If you want to create your own macros, you can do so
%% using \newcommand. Your macros should appear before
%% the \begin{document} command.
%%
\newcommand{\vdag}{(v)^\dagger}
\newcommand\aastex{AAS\TeX}
\newcommand\latex{La\TeX}


\newcommand{\chioh}{\chi_\mathrm{oh}}
\newcommand{\chico}{\chi_\mathrm{co}}
\newcommand{\Pathchp}{Path$_\mathrm{CH^+}$}
\newcommand{\Pathoh}{Path$_\mathrm{OH}$}
\newcommand{\Pathwoh}{Path$_\mathrm{woH}$}

\newcommand{\taumid}{\tau_\mathrm{mid}}

\newcommand{\fracdg}{f_\mathrm{dg}}
\newcommand{\fracdgfid}{f_\mathrm{dg,fid}}
\newcommand{\fracdgcri}{f_\mathrm{dg,cri}}

%\newcommand{\NCtaumid}{N_\mathrm{C,\tau_\mathrm{mid}=1}}
%\newcommand{\NCtaumidfid}{N_\mathrm{C,\tau_\mathrm{mid}=1}^\mathrm{fid}}
\newcommand{\NCtaumidfid}{ {\cal N}_\mathrm{C,\tau_\mathrm{mid}=1}^\mathrm{fid}}
\newcommand{\NCtaumid}{ {\cal N}_\mathrm{C,\tau_\mathrm{mid}=1}}

\newcommand{\nHfid}{n_\mathrm{H,fid}}
\newcommand{\nCfid}{n_\mathrm{C,fid}}
\newcommand{\nCcri}{n_\mathrm{C,cri}}
\newcommand{\nH}{n_\mathrm{H}}
\newcommand{\nHminoh}{n_\mathrm{H,min}^\mathrm{OH}}
\newcommand{\nHmincop}{n_\mathrm{H,min}^\mathrm{CO^+}}
\newcommand{\nHcri}{n_\mathrm{H,cri}}
\newcommand{\NH}{N_\mathrm{H}}
\newcommand{\nO}{n_\mathrm{O}}
\newcommand{\nC}{n_\mathrm{C}}
\newcommand{\NC}{N_\mathrm{C}}
\newcommand{\pcc}{\mathrm{cm}^{-3}}
\newcommand{\psc}{\mathrm{cm}^{-2}}
\newcommand{\Ac}{ {\cal A}_\mathrm{C}}
\newcommand{\Ao}{ {\cal A}_\mathrm{O}}
\newcommand{\Acism}{ {\cal A}_\mathrm{C,ism}}
\newcommand{\CO}{\mathrm{CO}}
\newcommand{\nCp}{n(\mathrm{C^+})}
\newcommand{\nOH}{n(\mathrm{OH})}
\newcommand{\nCOp}{n(\mathrm{CO^+})}
\newcommand{\nCHp}{n(\mathrm{CH^+})}
\newcommand{\nCO}{n(\mathrm{CO})}
\newcommand{\nCOana}{n(\mathrm{CO})_\mathrm{ana}}
\newcommand{\nCIana}{n(\mathrm{C^0})_\mathrm{ana}}
\newcommand{\NCO}{N(\mathrm{CO})}
\newcommand{\nCI}{n(\mathrm{C^0})}
\newcommand{\NCI}{N(\mathrm{C^0})}
\newcommand{\NHm}{N(\mathrm{H_2})}
\newcommand{\nHm}{n(\mathrm{H_2})}
\newcommand{\nHmv}{n(\mathrm{H_2}^*)}
\newcommand{\hfid}{h_\mathrm{fid}}
\newcommand{\Qave}{\langle Q\rangle}

\newcommand{\HmOp}{\mathrm{H_2O^+}}
\newcommand{\CHp}{\mathrm{CH^+}}
\newcommand{\HCOp}{\mathrm{HCO^+}}
\newcommand{\CmH}{\mathrm{C_2H}}

\newcommand{\FH}{F_\mathrm{H}}

\newcommand{\amin}{a_\mathrm{min}}
\newcommand{\amax}{a_\mathrm{max}}

\newcommand{\COr}{\mathrm{C/O}}

\newcommand{\kcoph}{k_\mathrm{cop,h}}
\newcommand{\kcpo}{k_\mathrm{cp,o}}
\newcommand{\kohc}{k_\mathrm{oh,c}}
%\newcommand{\koho}{k_\mathrm{oh,o}}
\newcommand{\aCOp}{\alpha_\mathrm{cop}}
\newcommand{\kohcp}{k_\mathrm{oh,cp}}
\newcommand{\aOH}{\alpha_\mathrm{oh}}
\newcommand{\koh}{k_\mathrm{o,h}}
\newcommand{\kcope}{k_\mathrm{cop,e}}
\newcommand{\aCO}{\alpha_\mathrm{co}}

\newcommand{\kcph}{k_\mathrm{cp,h}}
\newcommand{\aCHp}{\alpha_\mathrm{chp}}
\newcommand{\aCt}{\alpha_\mathrm{c2}}
\newcommand{\kcphm}{k_\mathrm{cp,h2}}
\newcommand{\kchpo}{k_\mathrm{chp,o}}
\newcommand{\kchpe}{k_\mathrm{chp,e}}
\newcommand{\kchph}{k_\mathrm{chp,h}}
\newcommand{\kcto}{k_\mathrm{c2,o}}
\newcommand{\kcc}{k_\mathrm{c,c}}
\newcommand{\kco}{k_\mathrm{c,o}}

\newcommand{\reviewstrike}[1]{\sout{#1}}


%% Reintroduced the \received and \accepted commands from AASTeX v5.2
%\received{March 1, 2021}
%\revised{April 1, 2021}
%\accepted{\today}

%% Command to document which AAS Journal the manuscript was submitted to.
%% Adds "Submitted to " the argument.
%\submitjournal{PSJ}

%% For manuscript that include authors in collaborations, AASTeX v6.31
%% builds on the \collaboration command to allow greater freedom to 
%% keep the traditional author+affiliation information but only show
%% subsets. The \collaboration command now must appear AFTER the group
%% of authors in the collaboration and it takes TWO arguments. The last
%% is still the collaboration identifier. The text given in this
%% argument is what will be shown in the manuscript. The first argument
%% is the number of author above the \collaboration command to show with
%% the collaboration text. If there are authors that are not part of any
%% collaboration the \nocollaboration command is used. This command takes
%% one argument which is also the number of authors above to show. A
%% dashed line is shown to indicate no collaboration. This example manuscript
%% shows how these commands work to display specific set of authors 
%% on the front page.
%%
%% For manuscript without any need to use \collaboration the 
%% \AuthorCollaborationLimit command from v6.2 can still be used to 
%% show a subset of authors.
%
%\AuthorCollaborationLimit=2
%
%% will only show Schwarz & Muench on the front page of the manuscript
%% (assuming the \collaboration and \nocollaboration commands are
%% commented out).
%%
%% Note that all of the author will be shown in the published article.
%% This feature is meant to be used prior to acceptance to make the
%% front end of a long author article more manageable. Please do not use
%% this functionality for manuscripts with less than 20 authors. Conversely,
%% please do use this when the number of authors exceeds 40.
%%
%% Use \allauthors at the manuscript end to show the full author list.
%% This command should only be used with \AuthorCollaborationLimit is used.

%% The following command can be used to set the latex table counters.  It
%% is needed in this document because it uses a mix of latex tabular and
%% AASTeX deluxetables.  In general it should not be needed.
%\setcounter{table}{1}

%%%%%%%%%%%%%%%%%%%%%%%%%%%%%%%%%%%%%%%%%%%%%%%%%%%%%%%%%%%%%%%%%%%%%%%%%%%%%%%%
%%
%% The following section outlines numerous optional output that
%% can be displayed in the front matter or as running meta-data.
%%
%% If you wish, you may supply running head information, although
%% this information may be modified by the editorial offices.
%\shorttitle{AASTeX v6.3.1 Sample article}
%\shortauthors{Schwarz et al.}
%%
%% You can add a light gray and diagonal water-mark to the first page 
%% with this command:
%% \watermark{text}
%% where "text", e.g. DRAFT, is the text to appear.  If the text is 
%% long you can control the water-mark size with:
%% \setwatermarkfontsize{dimension}
%% where dimension is any recognized LaTeX dimension, e.g. pt, in, etc.
%%
%%%%%%%%%%%%%%%%%%%%%%%%%%%%%%%%%%%%%%%%%%%%%%%%%%%%%%%%%%%%%%%%%%%%%%%%%%%%%%%%
%\graphicspath{{./}{figures/}}
%% This is the end of the preamble.  Indicate the beginning of the
%% manuscript itself with \begin{document}.

\begin{document}

\title{
A Constraint on the Amount of Hydrogen from the CO Chemistry in Debris Disks
}

%% LaTeX will automatically break titles if they run longer than
%% one line. However, you may use \\ to force a line break if
%% you desire. In v6.31 you can include a footnote in the title.

%% A significant change from earlier AASTEX versions is in the structure for 
%% calling author and affiliations. The change was necessary to implement 
%% auto-indexing of affiliations which prior was a manual process that could 
%% easily be tedious in large author manuscripts.
%%
%% The \author command is the same as before except it now takes an optional
%% argument which is the 16 digit ORCID. The syntax is:
%% \author[xxxx-xxxx-xxxx-xxxx]{Author Name}
%%
%% This will hyperlink the author name to the author's ORCID page. Note that
%% during compilation, LaTeX will do some limited checking of the format of
%% the ID to make sure it is valid. If the "orcid-ID.png" image file is 
%% present or in the LaTeX pathway, the OrcID icon will appear next to
%% the authors name.
%%
%% Use \affiliation for affiliation information. The old \affil is now aliased
%% to \affiliation. AASTeX v6.31 will automatically index these in the header.
%% When a duplicate is found its index will be the same as its previous entry.
%%
%% Note that \altaffilmark and \altaffiltext have been removed and thus 
%% can not be used to document secondary affiliations. If they are used latex
%% will issue a specific error message and quit. Please use multiple 
%% \affiliation calls for to document more than one affiliation.
%%
%% The new \altaffiliation can be used to indicate some secondary information
%% such as fellowships. This command produces a non-numeric footnote that is
%% set away from the numeric \affiliation footnotes.  NOTE that if an
%% \altaffiliation command is used it must come BEFORE the \affiliation call,
%% right after the \author command, in order to place the footnotes in
%% the proper location.
%%
%% Use \email to set provide email addresses. Each \email will appear on its
%% own line so you can put multiple email address in one \email call. A new
%% \correspondingauthor command is available in V6.31 to identify the
%% corresponding author of the manuscript. It is the author's responsibility
%% to make sure this name is also in the author list.
%%
%% While authors can be grouped inside the same \author and \affiliation
%% commands it is better to have a single author for each. This allows for
%% one to exploit all the new benefits and should make book-keeping easier.
%%
%% If done correctly the peer review system will be able to
%% automatically put the author and affiliation information from the manuscript
%% and save the corresponding author the trouble of entering it by hand.

%\correspondingauthor{August Muench}
%\email{greg.schwarz@aas.org, gus.muench@aas.org}

\author[0000-0002-2707-7548]{Kazunari Iwasaki}
\affiliation{
Center for Computational Astrophysics, National Astronomical Observatory of Japan, Osawa, Mitaka, Tokyo 181-8588, Japan,
kazunari.iwasaki@nao.ac.jp
}

\author[0000-0001-8808-2132]{Hiroshi Kobayashi}
\affiliation{
Department of Physics, Nagoya University, Furo-cho, Chikusa-ku, Nagoya, Aichi 464-8602, Japan
}

%\collaboration{20}{(AAS Journals Data Editors)}

\author[0000-0002-9221-2910]{Aya E. Higuchi}
\affiliation{
    Division of Science, School of Science and Engineering, Tokyo Denki University,
    Ishizaka, Hatoyama-machi, Hiki-gun, Saitama 350-0394, Japan
}
%\affiliation{AAS Journals Associate Editor-in-Chief}

\author{Yuri Aikawa}
%\altaffiliation{AASTeX v6+ programmer}
\affiliation{Department of Astronomy, Graduate School of Science, The University of Tokyo, 7-3-1 Hongo, Bunkyo-ku, Tokyo 113-0033, Japan}

%\author{Julie Steffen}
%\affiliation{AAS Director of Publishing}
%\affiliation{American Astronomical Society \\
%1667 K Street NW, Suite 800 \\
%Washington, DC 20006, USA}

%% Note that the \and command from previous versions of AASTeX is now
%% depreciated in this version as it is no longer necessary. AASTeX 
%% automatically takes care of all commas and "and"s between authors names.

%% AASTeX 6.31 has the new \collaboration and \nocollaboration commands to
%% provide the collaboration status of a group of authors. These commands 
%% can be used either before or after the list of corresponding authors. The
%% argument for \collaboration is the collaboration identifier. Authors are
%% encouraged to surround collaboration identifiers with ()s. The 
%% \nocollaboration command takes no argument and exists to indicate that
%% the nearby authors are not part of surrounding collaborations.

%% Mark off the abstract in the ``abstract'' environment. 
\begin{abstract}

 The faint CO gases in debris disks are easily dissolved into C by UV
 irradiation, while CO can be reformed via reactions with hydrogen.
 The abundance ratio of C/CO could thus be a probe of the amount of hydrogen in the debris disks.
 We conduct radiative transfer calculations with chemical reactions for debris disks. 
 For a typical dust-to-gas mass ratio of debris disks, 
 CO formation proceeds without the involvement of H$_2$ 
 because a small amount of dust grains makes H$_2$ formation inefficient.
 We find that the CO to C number density ratio depends on a 
 combination of $n_\mathrm{H}Z^{0.4}\chi^{-1.1}$, where $n_\mathrm{H}$ is the hydrogen nucleus number density, 
 $Z$ is the metallicity, and $\chi$ is the FUV flux normalized by the Habing flux. 
 Using an analytic formula for the CO number density, we give constraints on the amount of hydrogen and metallicity for debris disks.
 CO formation is accelerated by excited H$_2$ either when the dust-to-gas mass ratio is increased or 
 the energy barrier of chemisorption of hydrogen on the dust surface is decreased.
 This acceleration of CO formation occurs only when the shielding effects of CO are insignificant. In shielded regions, the CO fractions are almost 
 independent of the parameters of dust grains.

\end{abstract}

%% Keywords should appear after the \end{abstract} command. 
%% The AAS Journals now uses Unified Astronomy Thesaurus concepts:
%% https://astrothesaurus.org
%% You will be asked to selected these concepts during the submission process
%% but this old "keyword" functionality is maintained in case authors want
%% to include these concepts in their preprints.
\keywords{}

%% From the front matter, we move on to the body of the paper.
%% Sections are demarcated by \section and \subsection, respectively.
%% Observe the use of the LaTeX \label
%% command after the \subsection to give a symbolic KEY to the
%% subsection for cross-referencing in a \ref command.
%% You can use LaTeX's \ref and \label commands to keep track of
%% cross-references to sections, equations, tables, and figures.
%% That way, if you change the order of any elements, LaTeX will
%% automatically renumber them.
%%
%% We recommend that authors also use the natbib \citep
%% and \citet commands to identify citations.  The citations are
%% tied to the reference list via symbolic KEYs. The KEY corresponds
%% to the KEY in the \bibitem in the reference list below. 

%------------------------------------------
\section{Introduction} \label{sec:intro}
%------------------------------------------
Planets are believed to be formed in protoplanetary disks (PPDs) containing gas and dust.
The dissipation timescale of PPDs, especially that of gaseous component, is of
special importance for planetary system formation.
The gas giant planets need to be formed prior to the
significant dispersal of gas in disks \citep{Mizuno1978,Tanigawa2007}
and the gas dispersal may then lead to long term orbital instabilities of
protoplanets that trigger their impacts \citep{Iwasaki2001}.
The giant impacts between protoplanets are believed to form the terrestrial
planets in the solar system.

Infrared emission from PPDs decreases in several million
years \citep{Heisch2001}.
Older faint disks around main-sequence stars
are called debris disks \citep[][for review]{Hughes2018},
which might be explained by dust production due to collisional cascades starting from
disruptive collisions of kilometer or larger bodies.
That may be induced by planet formation events after
gas dispersal such as giant impacts for terrestrial planet formation
\citep[e.g.,][]{Genda2015} or planetary accretion in outer disks
\citep{Kobayashi2014}.
%%%%%%%%
Sub-mm observations
showed some debris disks with ages of several 10\,Myrs still have
molecular CO gas \citep{Hughes2008,Dent2014,Moor2019}.
In the present work we aim to investigate if the gas in debris disks is a primary
origin, i.e., the gas of PPDs is not fully dispersed, or secondary origin.
This question is directly linked to how and when the disk gas dissipates.

In the secondary scenario, the observed CO is produced from solid bodies via collisions
\citep{Kral2016,Kral2017,Matra2015,Matra2017,Matra2018}.
However, CO is photo-dissociated rapidly.
For $\beta$
Pictoris whose disk has CO with $\sim 3 \times 10^{-5} M_\oplus$, the
dissociation timescale is about $\sim 100$ years so that the mass
production of CO during the stellar age is estimated to be $\sim 3
M_\oplus$, where $M_\oplus$ is the Earth mass \citep[cf.][]{Kral2016}.  The IR
observations for comets by AKARI infer the fraction of CO to H$_2$O
$\sim 0.1$ \citep{Ootsubo2012}.
    Total solid bodies required for the observed gas
      is simply estimated from the CO mass and fraction to be $30 M_\oplus$,
    which is comparable to or larger than that of the solar system.

The gas production due to sublimation strongly depends on the
temperature of parent bodies.
Using the condensation temperature, the sublimation timescale
is estimated to be 0.003~yrs for kilometer-sized CO bodies with 50~K,
while it may be somehow longer for mixed ices. However,
the timescale is much shorter than the ages of debris disks.
It would thus be difficult to keep $\sim 3M_\oplus$ of CO in the icy bodies, 
unless there was a steady inflow from the outer colder regions.
Carbon dioxide, CO$_2$, can be another reservoir.
%so that CO in most disks is lost from solid bodies or
%changes into CO$_2$.
CO observed around comets in our solar system is mainly explained by the photo-dissociation of
CO$_2$ \citep{Weaver2011} so that
CO might be less abundant than CO$_2$ in comets.
  Outgassing CO$_2$ changes into CO via photo-dissociation.
Collisional vaporization occurs due to shock
heating \citep{Ahrens1972,Kurosawa2010}, so that high speed impacts
($\la 5$\,km/s) are required even for vaporization of volatile materials
\citep[e.g.,][]{Okeefe1982,Kraus2011}.
The collisional velocities are estimated to be
$~1~\mathrm{km~s^{-1}}(e/0.1) (M_*/M_\odot)^{1/2} (r/10~\mathrm{au})^{1/2}$,
where $e$ and $r$ are the orbital eccentricities and semimajor axes of colliding bodies, respectively,
$M_*$ is the mass of the central star, and $M_\odot$ is the solar mass
(e.g., Kobayashi \& L\"ohne 2014).
The collisional velocities of bodies with $e\sim 0.1$ beyond 10\,au are much slower than the vaporization velocity.
The collisional outgassing of CO$_2$ from cometary bodies is therefore difficult beyond 10 au,
where CO gas is observed in debris disks.
In summary, in the secondary-origin scenario, a large amount of reservoir ice is required,
while it should be sublimated efficiently beyond $>10~$au.

If the gas in debris disks is a primary origin, or is a mixture of primary and secondary
components, CO can be reformed from C$^+$ and C atoms in the gas phase.
The required CO reservoir is then significantly reduced compared with the case of
purely secondary gas. \citet{Higuchi2017} showed that the abundance ratio of atomic carbon to
CO can be a probe of the abundance of H$_2$ relative to carbon, i.e. the metallicity of gas in debris disks.
If the gas is a remnant of PPDs , the metallicity is expected to be close
to that of the interstellar medium (ISM).
On the other hand, if gas is supplied from solid bodies, the metallicity is
much larger than the ISM value while hydrogen is provided by H$_2$O desorbed from solid bodies.

Although the CO chemistry in debris disks was investigated by many authors
\citep{Kamp2000,Kamp2003,Gorti2004,Hughes2008,Roberge2013}, the elemental abundances in their models are not different
from those in the ISM significantly.
The metallicity can be different from the ISM value, depending on the origin of gas.
Thus, in this paper, we examine how the CO chemistry depends on metallicity
by calculating detailed chemical reactions and thermal processes,
and the radiative transfer including exact line transfers in the debris disks.

This paper is organized as follows:
one-dimensional plane-parallel PDR calculations.
The results of the PDR calculations are described 
and an analytical formula for the amount of CO is developed on the basis of 
the plane-parallel PDR calculations in Section \ref{sec:result}.
%In Section \ref{sec:analyticmodel}, we perform detailed analyses for the abundances of C, C$^+$, and CO 
%at the optically-thin limit.
%Shielding effects of CO are investigated in Section \ref{sec:shielding}.
Astrophysical implications are discussed in Section \ref{sec:discuss}.
%{\color{blue}
%We apply our analytical formula for the amount of CO to the chemical structure in debris disks, 
%and  compare our predictions with some observational results.
%}
Finally, our results are summarized in Section \ref{sec:summary}.

%---------------------------------
\section{Methods and Models}\label{sec:model}
%---------------------------------

%-----------------------------
\subsection{Numerical Methods}\label{sec:method}
%-----------------------------
The Meudon PDR code version 1.5.4 ({\ttfamily http://pdr.obspm.fr/}) is a publicly
available photon dominated region (PDR) code which is designed to model
a stationary plane-parallel slab of gas and dust controlled by
far-ultraviolet (FUV) photons \citep{LeP2006}.
%Figure \ref{fig:setting} shows the schematic picture of our setting.  
A plane-parallel slab is illuminated by a radiation field penetrating from
the left-side of the slab.  

The Meudon code solves the radiative
transfer at each point in the slab, taking into account dust extinction
and line absorption of gaseous species.
We use the exact method described in \citet{Goi2007} 
which allows us to take into account overlapping of the H, H$_2$, and CO UV absorption lines.
For the H$_2$ line transfer, we define a value of $J_\mathrm{max}=3$
under which the exact method is used.
The FGK approximation proposed by \citet{Fed1979}
is used for all electric transitions for $J_l\ge J_\mathrm{max}$, where $J_l$ is 
the lower level of Lyman and Werner transitions.
We do not take into account the exact line transfer for $^{13}$CO and C$^{18}$O.

Coupled with the radiative transfer, 
the full level population system of a number of species, including H$_2$ and CO, is computed 
from the detailed balance between 
radiative transitions, collisional transitions, transitions due to the cosmic microwave background, 
formation and destruction processes 
both in the gas phase and on dust grains \citep{LeP2006}.

This treatment allows us to evaluate the H$_2$ self-shielding effect more accurately than
using the fitting formula provided by \citet{DraineBertoldi1996}, which 
has been used in most studies.
Furthermore, the accurate calculation of the level populations provides 
the accurate rates of chemical reactions associated with excited H$_2$, 
which will be discussed in Section \ref{sec:parametersurvey}.
Similarly, the shielding effect of CO is calculated more accurately 
than using the table from \citet{Visser2009}.

The Meudon code calculates the thermal equilibrium of gas and dust grains.
For gas temperature, heating processes (including the
photo-electric heating on dust, exothermic chemical reactions, and cosmic
rays) are balanced with cooling processes (including infrared and
millimeter emission from ions, atoms, and molecules).
The treatment of dust grains is explained in
Section \ref{sec:dust}.
Chemical and thermal equilibria determine the abundances of each chemical species at
each position.


In order to investigate 
the basic properties of the chemical 
structure of debris disks, 
in Section \ref{sec:result}, we conduct 
simple plane-parallel PDR calculations 
of a semi-infinite gas slab 
with a uniform density illuminated by stellar radiation fields and 
the interstellar radiation field (ISRF).


We should note that the Meudon PDR code is not 
applicable directly to the multi-dimensional 
disk geometry because 
it is designed for the one-dimensional plane-parallel geometry.
In debris disks, we should take into account the geometrical dilution of the 
stellar radiation and vertical ISRF.
Nevertheless, we will find that the main 
underlying physics of debris disks 
can be approximated by the findings from the one-dimensional plane-parallel PDR calculations.
In Section \ref{sec:observation}, 
we compare the observational results with the predictions obtained 
by applying the findings of the plane-parallel models to disk models.


%In Section \ref{sec:observation}, we will consider  
%realistic disk structures for the debris disks around $\beta$ Pictoris and 49~Ceti, and 
%will see the main underlying physics is approximated in 
%the findings in the one-dimensional plane-parallel PDR calculations. 


%From plane-parallel PDR calculations, 
%an analytical formula for the CO fractions is developed by extracting important chemical reactions to form CO. 
%In Section \ref{sec:COdiskana}, 
%the analytical formula will applied into the disk geometry where 
%we consider the geometrical dilution of the stellar radiation propagating radially and 
%the vertical ISRF striking top and bottom of the disk.



%In Section \ref{sec:analyticmodel},
%an analytic formulation of the CO fraction 
%will be derived as a function of FUV fluxes, gas densities, 
%gas metallicities where shielding effects of CO are not significant.
%This analytic formulation is applicable in the deeper 
%interior of the gas slab (Section \ref{sec:compana}).
%
%Two important effects are not considered in the above plane-parallel PDR calculations.
%One is the geometrical dilution of the stellar radiation, and the other is the 
%vertical ISRF striking top and bottom of the disk.
%Our analytical formula derived from the plane-parallel PDR calculations is applied into 
%the disk geometry in Section \ref{sec:COdiskana}. 


%We note that the plane-parallel geometry adopted in this paper is different from
%a disk configuration, which is actually two- or three-dimensional.
%In Sections \ref{sec:result}-\ref{sec:thin}, we will derive an analytic formula of the CO fraction as a function of FUV fluxes, gas densities, 
%metallicities at the low $\NC$ limit, which is expected not to depend on the geometry.
%Section \ref{sec:shielding} will show that the effect of the C$^0$ attenuation, which is one of the important shielding effects of CO, 
%can be taken into account in the analytic formula.
%The other factor that
%is unrealistic is that, in fact, the disk is thinner than the radial extent, the incident stellar
%flux at 912-1100 is fairly weak, so that in the outer disk it is the incident ISRF striking top
%and bottom of the disk that dominates. Given these very significant effects, these
%results are only presented to give the reader an idea of the chemistry and temperatures
%involved. They are also useful in finding the very low tau solution at the very surface
%facing the star, which you then approximate with analytic solutions.

%paper you will consider a more realistic disk solution for Beta Pic and 49 Ceti, but that
%the main underlying physics is approximated in this 1D solution. T
%Using the knowledge from Sections \ref{sec:result}-\ref{sec:shielding}, 
%the CO formation in the disk geometry will be discussed in Section \ref{sec:observation}.






%However, it may be justified because
%the chemical abundances are expected to be determined mainly by the 
%local physical conditions in
%debris disks, which are optically thin.

%\begin{figure}[htpb]
%        \centering
%        \includegraphics[width=9.5cm,clip]{Figure1.ps}
%\caption{
%Schematic picture of our setting.
%}
%\label{fig:setting}
%\end{figure}

\subsection{Modifications to the Meudon Code}\label{sec:mod}
We made some modifications to the Meudon code because 
it is not designed for debris disks.

%Firstly, since the size of dust grains is much larger than 
%that of the ISM dust, 
%the computational cost for calculating the dust charge becomes too high.
%We modified the module for calculating the dust charge as 
%is explained in Section \ref{sec:charge} and Appendix \ref{sec:approx}.

Firstly, we modified the calculation of the photo-dissociation of CO$^+$.
In calculating the photo-dissociation rate of species, the Meudon code 
divides the species into two groups.
For one group, 
the photo-dissociation rate is estimated from 
a fitting formula, which is a form of $\propto \exp(-\beta A_\mathrm{V})$, 
assuming the radiation field 
%to be the interstellar standard radiation field (ISRF) \citep{vanDishoeck2006}.
to be the ISRF \citep{vanDishoeck2006}, where $\beta$ is the fitting parameter.
The rate is scaled by the ratio of the total UV flux to the ISRF flux.
While this formulation is efficient and is often used in PDR calculations of debris disks,
it can be inaccurate when the SED of the radiation field is different from the ISRF.
In addition, the possible shielding of photo-cross sections that resonate with absorption lines 
is not considered in the exponential form.
Since dust grains in debris disks are larger than those found in the ISM,
$\beta$ is expected to be inaccurate.

For the other group, the photo-dissociation rate is computed 
directly from $\int \sigma_\lambda I_\lambda/h\nu d\lambda$;
$\sigma_\lambda$ is the photo-dissociation cross section and $I_\lambda$ is the radiation intensity 
summed over all incidence angles. 
This treatment allows us to evaluate the photo-dissociation 
consistent with the local radiation field at each position.
It is however computationally expensive and used only for important species. 


In the Meudon PDR code, CO$^+$
was included in the former group where the 
fitting formula is used
although CO$^+$ is one of the most important 
species to form CO in debris disks.
In this work, the photo-dissociation rate of CO$^+$ is calculated from the numerical integration 
of $\int \sigma_\lambda I_\lambda/h\nu d\lambda$,
where $\sigma_\lambda$ is taken from 
\citet{Lavendy1993} \citep[also see][]{Heays2017}, and tabulated as a function of the wavelength.

One of the important molecules to form CO in our results is the hydroxyl radical OH, whose 
photo-dissociation rate is computed directly from the integration of photo-reaction cross sections.
It is dissociated by photons with wavelengths longer than 1100~\AA.
In the adopted radiation fields, the photo-dissociation rate of OH is 
mainly determined by the $1 ^2\Sigma^-$ channel, which 
absorbs photons with 
$\sim 1600~$\AA \citep{vanDishoeck1983}.
We will define a normalized UV flux dissociating OH in Section \ref{sec:fuv}.

Secondly, we modified the rate coefficient of an endothermic reaction 
of $\mathrm{O+H_2\rightarrow OH + H}$, which is one of the most important 
reactions to form CO.
The Meudon PDR code treats endothermic reactions of C$^{+}$, S$^+$, O, and OH with H$_2$,
taking into account excited states of H$_2$.
Except for reactions of 
$\mathrm{C^++H_2}$ and $\mathrm{S^++H_2}$ \footnote{
To estimate the enthothermic chemical reaction rates,
the Meudon PDR code uses
the results of the theoretical studies done by 
\citet{Zanchet2013} and \citet{Herrez-Aguilar2014} 
for $\mathrm{C^++H_2\rightarrow CH^+ + H}$
and by \citet{Zanchet2013_Sp} 
for $\mathrm{S^++H_2\rightarrow SH^+ + H}$.
}
the rate coefficients were calculated 
assuming that all the internal energies are used to overcome 
an activation barrier or an endothermicity.
%The reaction between O and H$_2$ is endothermic by $\sim 900~$K and 
%has an activation barrier of $\sim 4800$~K if H$_2$ is in the ground state.
%\citet{Sultanov2005} found that the reaction is suppressed 
%by an activation barrier even for H$_2$ in the $v=3$ state, where 
%$v$ is the vibrational quantum number.
%We use the fitting formulae for the rate coefficient of $\mathrm{O+H_2}$ derived 
%by \citet{Agndez2010} based on the theoretical calculations by 
%\citet{Sultanov2005}; the population of H$_2$ in $v=0$, 1, 2, and 3 states 
%is assumed to be thermal.
In order to calculate the reaction rate of $\mathrm{O+H_2\rightarrow OH + H}$ more accurately,
we use the fitting formulae for the rate coefficient of 
$\mathrm{O+H_2}(v)\rightarrow \mathrm{OH} + \mathrm{H}$ for $v=0$, 1, 2, and 3 
derived by \citet{Agndez2010} based on the theoretical calculations by 
\citet{Sultanov2005}. 
The reaction rates for $v>3$ are replaced by that for $v=3$, 
where $v$ is the vibrational quantum number.
As a result, the total reaction rate may be underestimated.
Recently, precise rates especially for $v>3$ 
have been estimated by \citet{Veselinova2021}\footnote{
O and H$^*_2$ with $v>3$ does not affect the CO formation because 
the H$^*_2~(v>3)$ fractions are too small to affect the CO formation.
}.

Thirdly, we do not consider heating owing to the di-electronic recombination.
In the Meudon code, for simplicity, the electron captured in an upper electronic level
is assumed to be de-exited only through collisions, resulting in gas heating.
The density range considered in this paper is so low that 
most of transitions in the cascade of the captured electron will occur through 
spontaneous emissions. 
We only consider gas cooling owing to the removal of the kinetic energy of recombined electron.


%--------------------------------
\subsection{Incident Radiation Fields}\label{sec:fuv}
%--------------------------------

We consider two kinds of radiation fields which are incident on the edge 
of a semi-infinite gas slab.

One is the ISRF 
\citep{Habing1968,Draine1978,Mat1983}.
The ISRF is given by the summation of four components.
One is the ISRF from far- to near-ultraviolet;
an expression of the first component is given by
\begin{eqnarray}
I_\mathrm{ISRF}(\lambda) &=& 107.192 \left( \frac{\lambda}{1~\mathrm{\AA}}\right)^{-2.89}\nonumber\\
&& \times \left[
\mathrm{tanh}\left( 4.07\times10^{-3}\left( \frac{\lambda}{1~\mathrm{\AA}} \right)
- 4.5991 \right)+1  \right] \nonumber\\
&& \mathrm{erg~cm^{-2}~s^{-1}~\AA^{-1}~sr^{-1}}
\label{ISRF}
\end{eqnarray}
for $\lambda \ge 912~$\AA
%for $912~\mathrm{\AA}\le \lambda\le 8000~\mathrm{\AA}$ and
%\begin{equation}
%        I(\lambda) = 2\times 107.192 \left( \frac{\lambda}{1~\mathrm{\AA}} \right)^{-2.89}
%\mathrm{erg~cm^{-2}~s^{-1}~\AA^{-1}~sr^{-1}}
%\end{equation}
%for $\lambda > 8000~\mathrm{\AA}$
\citep{Mat1983}\footnote{Equation (\ref{ISRF}) is a fitting function of the results of \citep{Mat1983} 
as shown in the manual of the Meudon PDR code.}.
The second ISRF component is supplied by cold stars and
is expressed as a combination of three black bodies at 6184, 6123, and 2539~K.
The third component comes from the dust thermal emission which is estimated 
by the DustEM code \citep{Compigne2011}.
The forth ISRF component is the cosmic microwave background radiation given 
by a black body at 2.73~K.

The other radiation field is the stellar emission taken from
ATLAS9 \citep{Kurucz1992,Howarth2011}.
We consider two spectral types of A5V and A1V.
The effective temperatures $T_\mathrm{eff}$ of the A5V and A1V stars are
8250~K and 9000~K, respectively.
The metallicities of the stars are set to the solar value.
The surface gravity is fixed to $\log_{10}g = 4$.
Examples of the A5V and A1V stars are $\beta$~Pictoris 
\citep{Lecavelier2001}
and 49~Ceti \citep{Roberge2013}, respectively.

As the stellar radiation incident into the slab, 
the radiation field at a representative 
distance of 50~au from the star is adopted.
The mean intensities at $50~$au are shown in  Figure \ref{fig:spec}.

%Figure \ref{fig:spec} shows the intensities emitted from
%the A5V and A1V stars at a radius of $r=50~\mathrm{au}$ as a function of wavelength.
%The A1V star emits stronger emission than the A5V star
%because the effective temperature is higher and the stellar radius is larger.

%The distance between the center of the star and the left edge of the slab is
%denoted by $r$ (Figure \ref{fig:setting}).
%In this paper, $r$ is fixed to 50~au.
%The stellar radiation which enters the slab is diluted by the geometrical effect from the central star to the left edge as will be shown
%in Equation (\ref{chi}).
%As mentioned in Section \ref{sec:method}, the geometrical dilution of the stellar radiation inside the gas slab and 
%the ISRF flux coming from the vertical directions are not taken into account in the PDR calculations.
%In Section \ref{sec:observation}, the disk geometry will be taken into account approximately to 
%compare our results with observational results.
%In Section \ref{sec:result}, we mainly focus on the PDR structure in the low optical depth limit.

\begin{figure}[htpb]
        \centering
        \includegraphics[width=8cm]{Fig1.pdf}
\caption{
Mean intensities of the stellar radiation of A5V (red) and A1V (blue) stars 
at a representative distance of 50~au from the central star 
as a function of wavelength.
%  Intensities of the stellar emission of A5V (red) and A1V (blue) stars
%  a function of wavelength. 
  The FUV photons within the gray region induces photo-dissociation of H$_2$ and CO,
  and photo-ionization of C$^0$.
}
\label{fig:spec}
\end{figure}


We parameterize the UV fluxes that dissociate species which are important to form CO.
The following two ranges of wavelengths are focused on.

One wavelength range is $912~\mathrm{\AA}<\lambda<1100~\mathrm{\AA}$, which is shown by
the gray region in Figure \ref{fig:spec}.
The photons in this wavelength range photo-dissociate CO and H$_2$, and photo-ionize C$^0$.
The FUV photon number flux integrated over the wavelength range
is denoted by $F_\mathrm{FUV,CO}$, and is often measured in units of the \citet{Habing1968} field
flux, $\FH\equiv 1.2\times 10^7$~cm$^{-2}$~s$^{-1}$ 
\citep{Bertoldi1996}.
The normalized incident FUV photon number 
flux $\chico \equiv F_\mathrm{FUV,CO}/F_\mathrm{H}$\footnote{
In this paper, $\chi_\mathrm{co}=1$ refers to an FUV field whose 
energy density at the surface of the gas slab is equal to that 
of the Habing field considering all $4\pi$ steradian in the free-space.
}
is given by
\begin{equation}
        \chico = \chi_\mathrm{co,star} + \chi_\mathrm{co,ISRF}/2,
      \label{chi}
\end{equation}
where 
$\chi_\mathrm{co,star}$ indicates the normalized stellar FUV flux at a distance of 
50~au from the central star, and 
$\chi_\mathrm{co,ISRF}=1.3$ is the normalized ISRF flux derived from Equation (\ref{ISRF}).  
%$R_*$ is the stellar radius (Figure \ref{fig:setting}).
%A factor of $0.5$ in the first term in the right-hand side of Equation (\ref{chi}) 
%comes from the fact that the ISRF penetrates the gas slab only from the left-edge in Figure \ref{fig:setting}.
%The factor of $(R_*/r)^2$ indicates geometrical dilution from the central star to the left edge.

%The other wavelength range is 
%$1600~\mathrm{\AA}\le \lambda \le 1700~\mathrm{\AA}$, where 
%OH is photo-dissociated.
{
The other wavelength range is considered to 
characterize the OH photo-dissociation rate.
The cross section of photo-dissociation of OH adopted in the Meudon PDR 
code has the two broad peaks centered around $\lambda\sim 1090~\mathrm{\AA}$ ($3~^2\Pi$)
and $\lambda\sim 1600~\mathrm{\AA}$ ($1~^2\Sigma^-$) \citep{vanDishoeck1983,vanDishoeck1984,Heays2017}.
The latter peak mainly contributes to the OH photo-dissociation while the contribution from the former peak
is negligible. This is because 
as shown in 
Figure \ref{fig:spec} the intensities of the stellar radiation increase rapidly 
with wavelength in the corresponding wavelength range.
Considering that the intensities of the stellar radiation  
are an increasing function of wavelength within the 
latter peak,
we set a wavelength range of
$1600~\mathrm{\AA}\le \lambda \le 1700~\mathrm{\AA}$,
which corresponds to the 
long wavelength half of the latter peak.
The OH photo-dissociation 
rate is roughly proportional to the FUV photon number flux 
integrated over this wavelength range, which 
is denoted by $F_\mathrm{FUV,OH}$.
}
%The FUV photon number flux integrated over the wavelength range is 
%denoted by $F_\mathrm{FUV,OH}$.
For convenience, we use the normalized FUV flux $\chioh \equiv F_\mathrm{FUV,OH}/10^{12}~\mathrm{cm^{-2}~s^{-1}}$.


From the stellar spectra of the A5V and A1V stars shown in 
Figure \ref{fig:spec}, one derives that 
$\chico$ are $1.4$ and $31$, respectively.
The contribution of the ISRF to $\chioh$ is negligible, and 
$\chioh=2.4$ for A5V and $24$ for A1V, respectively. 

In this paper, the models with ($\chico=1.4$, $\chioh=2.4$) and 
those with ($\chico=31$ and $\chioh=24$) are referred to 
"weak-FUV" and "strong-FUV", respectively.

%The normalized FUV photon number fluxes at the stellar surfaces
%are $\chi_\mathrm{co,star,A5V}=3.0\times 10^7$ and $\chi_\mathrm{co,star,A1V}=6.6\times 10^8$. 
%At $r=50~$au, $\chico$ is $1.4$ for the A5V star and $\chico$ is $31$ for the A1V star,
%where we use $R_*=1.8~R_\odot$ (A5V) and $R_*=2.3~R_\odot$ (A1V).
%For $\chioh$, the contribution of the ISRF is negligible, and 
%$\chioh=2.4$ and $24$ for A5V and A1V at $r=50~$au, respectively. 

%We should note that we do not parametrize the FUV flux dissociating 
%CH$^+$ and CO$^+$, both of which are important species in the CO chemistry, because 
%they become effective only when the CO abundance is extremely low.
%We are not interested in such a situation.

%-----------------------------
\subsection{Dust Properties}\label{sec:dust}
%-----------------------------
Dust grains play important roles in at least three processes.
First, they are responsible for absorption and scattering of photons.
Second, they work as a catalyst in some chemical reactions.
For instance, H$_2$ forms on the surface of dust grains rather than by gas phase reactions.
Third, they contribute to thermal processes through photo-electric heating and
collision with the gas.
These processes depend significantly on the dust properties:
%\replaced{the size distribution and chemical compositions.}
{the size distribution, dust-to-gas mass ratio, and chemical compositions.}

%------------------------------------
\subsubsection{Size Distribution}\label{sec:size}
%------------------------------------

One of the differences of the debris disk from the ISM is
absence of small dust grains.
The small dust grains with radii smaller than $\sim 1~\mu$m are blown out by
the radiation pressure of stellar photons
\citep{Burns1979,Kobayashi2008,Kobayashi2009}.
Absence of the small dust grains reduces the visual extinction,
the H$_2$ formation rate, and the photo-electric heating rate 
for a given dust-to-gas mass ratio.
For simplicity, we consider spherical dust grains whose radii are denoted by $a$.
%Since $\amax$ has not been well constrained,
%we consider two maximum dust radii $a_\mathrm{max}=10~\mu $m and  
%$a_\mathrm{max}=1$~cm.
The power-law grain size distribution $n(a)\propto a^{-3.5}$ is assumed, where
$n(a)da$ is the number density of the dust grains
with the radius range from $a$ to $a+da$ \citep{Mathis1977}.
This power-law size distribution is controlled by
the collisional cascade \citep{Dohnanyi1969,Tanaka1996},
although the power-law index could be modulated due to the size dependence of collisional strength
\citep{Kobayashi2010}.
In this work, the minimum and maximum radii of the dust grains are fixed to
$a_\mathrm{min}=1~\mu$m and $a_\mathrm{max}=10~\mu$m.
We will discuss that our results do not depend 
directly on the size distribution in Section \ref{sec:plane}.

%------------------------------------
\subsubsection{Composition}
%------------------------------------

The dust grains are assumed to be a mixture of graphite and silicon.
The absorption and scattering coefficients of dust grains 
are calculated by averaging those of the graphite and silicon 
in a ratio of $7:3$, 
taking into account the size distribution.
The heat capacity of the dust grains, 
which is used to determine the dust temperature at each 
dust size and position, is computed 
assuming the mixed composition.
The absorption/scattering coefficients and heat capacity at each dust size
are computed by \citet{LaorDraine1993} and \citet{DraineLi2001}.



%----------------------------------------
\subsubsection{Charge of Dust Grains}\label{sec:charge}
%----------------------------------------
In addition to the FUV flux, 
the charge of dust grains characterizes the photo-electric yield on 
the dust grain.
The Meudon code takes into account the probability distribution function 
of the dust charge for each dust size $f(a,Q)$, which is determined by 
collisional charging and photo-electric ionization \citep{BakesTielens1994}, where
$Q$ is the dust charge.


As the grain size increases, the photo-electric ionization rate increases faster than 
the recombination rate. 
As a result, the charge of dust grains increases with $a$.
Since the maximum grain size considered in this work is up to 1~cm,
a great number of the charge grid is required.


To reduce the computational cost,
we approximate the charge distribution function at each grain size $f(a,Q)$ 
by a delta function of $\delta(Q-\Qave_a)$, where $Q$ is the 
dust charge and $\Qave_a$ is the mean charge for the grain size $a$.
The validity of the approximation $f(a,Q)=\delta(Q-\Qave_a)$ is confirmed.
%in Appendix \ref{sec:approx}.


%------------------------------------
\subsubsection{Dust-to-gas Mass Ratio}\label{sec:fracdg}
%------------------------------------
  Since the dust-to-gas mass ratio $\fracdg$ of debris disks is unknown observationally and
  theoretically, a wide range of $\fracdg$ is considered in the present work.
%  Supposing that $n(a)=n_\mathrm{d}\tilde{n}(a)da$, where $n_\mathrm{d} \equiv 
%  \int_{\amin}^{\amax} n(a) da$,
      Using the size distribution $n(a)$, 
  the dust-to-gas mass ratio is expressed as 
  \begin{equation}
      \fracdg \equiv \frac{1}{\mu_\mathrm{C}m_\mathrm{C}\nC} 
      \frac{4\pi}{3}\rho_\mathrm{gr}\langle a^3\rangle n_\mathrm{d},
      \label{fdg0}
  \end{equation}
  where 
  \begin{equation}
      \langle g(a) \rangle \equiv \frac{1}{n_\mathrm{d}} 
      \int_{\amin}^{\amax}g(a) {n}(a)da,
  \end{equation}
  $n_\mathrm{d} = \int_{\amin}^{\amax} {n}(a)da$ is the total number density of dust grains,
  $m_\mathrm{x}$ and $n_\mathrm{x}$ are the mass and number density of 
  element ``x'', respectively,
  $\mu_\mathrm{x}$ is the mean molecular weight per element ``x'' nucleus, and 
  $\rho_\mathrm{gr}=2.62~\mathrm{g~\pcc}$ is the internal density of the dust grains. 
  We should note that the gas mass density is expressed as $\mu_\mathrm{C}m_\mathrm{C}\nC$
  because the gas component is considered on the basis of carbon.
  The gas metallicity is denoted by $Z$, and 
  the elemental abundances of the gas phase used in the present work are shown in Table \ref{tab:abundance} 
  for the solar metallicity $Z=1$.
  When $Z$ changes, the relative abundances among metals are fixed.   
  From Table \ref{tab:abundance}, $\mu_\mathrm{C}$ is given by 
  \begin{equation}
%      \mu_\mathrm{C}= 890\left( Z^{-1} + 6.6\times 10^{-3}\right).
          \mu_\mathrm{C}(Z)=  883\left( Z^{-1} + 6.6\times 10^{-3}\right).
      \label{muC}
  \end{equation}


\begin{table}[htpb]
	\centering
	\begin{tabular}{|c|c|c|}
		\hline
        atom & relative abundance 
        of the gas phase & reference \\
		\hline
		\hline
    He & $0.1$ & \\
		\hline
    C & {$1.32\times 10^{-4}$} & 1 \\
		\hline
    N & {$7.5\times 10^{-5}$}& 2 \\
		\hline
    O & {$3.2\times 10^{-4}$}& 3 \\
		\hline
    Ne & {$6.9\times 10^{-5}$} & 4 \\
		\hline
    Si & {$8.2\times 10^{-7}$}& 5 \\
		\hline
    S & {$1.0\times10^{-8}$}& 6 \\
		\hline
    Ar & {$3.29\times 10^{-6}$}& 7 \\
		\hline
    Fe & {$1.5\times 10^{-8}$} & 1 \\
		\hline
	\end{tabular}
	\caption{
        Relative elemental abundances  of the gas phase with respect to hydrogen for $Z=1$.
    (1) \citet{Savage1996}, (2) \citet{Meyer1997}, (3) \citet{Meyer1998},
      (4) \citet{Adamkovics2011}, (5) \citet{Morton1975}, (6) \citet{Tieftrunk1994},
      (7) \citet{Lodders2008}.
  }
	\label{tab:abundance}
\end{table}

  In the context of PPDs, $n_\mathrm{d}$ is often determined from Equation (\ref{fdg0}) at a given $\fracdg$.
  By contrast, we use the fact that in debris disks, the amount of dust grains is constrained by 
  the optical depth at {\it V} band,
  which is often estimated from the fraction of the disk luminosity to stellar luminosity.
  The most practical criterion for debris disks is 
  $\tau < 8\times 10^{-3}$ \citep{Hughes2018}, where $\tau$ denotes a typical vertical optical depth of debris disks.
  A typical optical depth along the mid-plane 
  $\taumid$ is given by $\tau \theta^{-1}$, where 
  $\theta$ is the aspect ratio of debris disks.
%  In the model setup shown in Figure \ref{fig:setting}, 
%  The mid-plane optical depth $\taumid \sim \tau \theta^{-1}$ should be used, where 
%  $\theta$ is the aspect ratio of debris disks.

  The mean free path $\ell$ of a photon for dust absorption 
  depends on the grain size distribution as follows:
  \begin{equation}
      \ell = \left(n_\mathrm{d} \langle Q_\mathrm{abs} \pi a^2\rangle\right)^{-1}. 
      \label{meanfreepath}
  \end{equation} 
  In the integration, the absorption coefficient $Q_\mathrm{abs}$ 
  can be set to be unity because
  $2\pi a/\lambda > 1$ is satisfied for $a\ge 1~\mu$m at the {\it V} band.
  Eliminating $n_\mathrm{d}$ using Equations (\ref{fdg0}) and (\ref{meanfreepath}), 
  one obtains 
  \begin{equation}
  \fracdg = 1.74\times 10^{23}~\mathrm{cm}^{-3}~\frac{\langle a^3\rangle}{\langle a^2\rangle}
  \mu_\mathrm{C}^{-1}  {\cal N}_\mathrm{C,\taumid=1}^{-1},
    \label{fdg}
  \end{equation}
  where ${\cal N}_\mathrm{C,\taumid=1} = \nC \ell$ 
  is the mid-plane carbon nucleus column density  of the gas phase
  at $\taumid=1$ of the dust opacity at {\it V} band.
  The values of $\fracdg$ adopted in our PDR calculations will be shown in
  Section \ref{sec:modelparameter}.

    As mentioned in Section \ref{sec:size}, 
    the size distribution $n(a)\propto a^{-3.5}$ is used, 
    and $\langle a^3\rangle /\langle a^2\rangle = \sqrt{\amin\amax}$.
    
%----------------------------------------
\subsubsection{Temperature of Dust Grains}\label{sec:Tgr}
%----------------------------------------


In the Meudon PDR code, the temperature of dust grains $T_\mathrm{gr}$ is 
computed from the energy balance between absorption of the radiation and 
thermal emission at each grain radius bin.
For reference, we here show the grain temperature given by 
\citet{Hughes2008} as follows:
\begin{equation}
    T_\mathrm{gr} = 10^2~\mathrm{K}\left( \frac{L_*}{10L_\odot} \right)^{0.2} 
    \left( \frac{a}{1~\mu\mathrm{m}}\right)^{-0.2} 
    \left( \frac{r}{50~\mathrm{au}} \right)^{-0.4},
    \label{Tgr}
\end{equation}
where $L_*$ is the stellar luminosity, 
$a$ is the grain radius, and 
$r$ is the distance from the central star.
The dust temperatures obtained the PDR calculations 
are  consistent with Equation (\ref{Tgr}).





%--------------------------------------------------------------
\subsubsection{Grain Surface Chemistry}\label{sec:dustsurface}
%--------------------------------------------------------------

The Meudon code takes into account H$_2$ and HD formation with the Langmuir-Hinshelwood (LH) 
and the Eley-Rideal (ER) mechanisms \citep{LeBourlot2012}.
The grain-surface chemistry for other species heavier than H$_2$ and HD 
is not included in this study because photo-desorption is so efficient that heavy species are difficult to freeze out on the grain surface 
\citep[e.g.,][]{Grigorieva2007}.
%as shown in Appendix \ref{app:freeze}. 



In debris disks, the most important H$_2$ formation mechanism on dust grains 
is the ER mechanism involved by chemisorption of H atoms.
This is because the dust temperatures, which are as high as 
$\sim$100~K as shown in Equation (\ref{Tgr}),
are so high that a hydrogen atom physisorbed on the grain surface 
evaporates before it combines with another hydrogen atom.



In the ER mechanism, there are several free parameters 
in the Meudon PDR code.
One is a sticking coefficient $\alpha(T)$, which approaches zero for high temperatures 
because the hydrogen atom cannot stick on the dust grains owing to the excess kinetic energy.
The Meudon PDR code adopts the empirical form 
$\alpha(T)=1/(1+(T/T_\mathrm{stick})^\beta)$, 
where $T_\mathrm{stick}=464~$K and $\beta=3/2$ are adopted \citep{LeBourlot2012}.

Another parameter is $T_\mathrm{chem}$ corresponding to the 
energy barrier that a hydrogen atom overcomes 
to reach a chemisorption site on the grain surface.
The H$_2$ formation rate is highly sensitive to $T_\mathrm{chem}$
because the probability of overcoming the energy 
barrier is proportional to $\exp(-T_\mathrm{chem}/T)$.
Nonetheless, $T_\mathrm{chem}$ is highly uncertain.  
Its value depends on the surface condition and composition of dust grains.
The energy barrier of chemisorption onto a perfect graphite surface 
is as high as $\sim 2000$~K
\citep[e.g.,][by using quantum mechanics]{Sha2002}.
This has been confirmed by 
the experimental study of a highly ordered pyrolytic graphite 
in the laboratory done by \citet{Zecho2002}.
By contrast, topological defects on the carbonaceous surface 
decrease $T_\mathrm{chem}$, and can make chemisorption 
almost barrierless 
($T_\mathrm{chem}\sim 10-100~$K), 
depending on structures \citep{Ivanovskaya2010}.
Experiments of chemisorption of H on porous, defective, 
aliphatic carbon surfaces have 
found that the activation energy is 
as low as $\sim 70$~K \citep{Mennella2006}. 
By contrast, in the cases with silicate, 
quantum chemical calculations revealed 
$T_\mathrm{chem}\sim 300~$K both for crystalline silicate 
\citep{Navarro-Ruiz2014} and amorphous silicate \citep{Navaroo-Ruiz2015},
but they have not been confirmed experimentally yet.



Uncertainty of the H$_2$ formation rate on 
warm dust grains also has been 
discussed in PDRs illuminated by strong radiation 
from massive stars.
Except for the absence of small dust grains, 
the environments are similar to debris disks.
In such PDRs, \citet{Habart2011} observationally found that 
rotationally-excited H$_2$ is more abundant than the predictions of PDR models, 
suggesting that the H$_2$ production rate on warm dust grains 
needs to be higher than a typical value \citep[e.g,][]{Jura1974}.
This result suggests that there is significant uncertainty in H$_2$ formation rates 
on warm dust grains.

As will be mentioned in Section \ref{sec:modelparameter}, 
considering the uncertainty of $T_\mathrm{chem}$, 
$T_\mathrm{chem}$ is changed as one of the 
model parameters in our PDR calculations.



%--------------------------
\subsection{Model Parameters}\label{sec:modelparameter}
%--------------------------

  We summarize the model parameters in our models and show their parameter spaces.
  The model parameters are divided into ones related to the radiation field,
  the dust, gas components, the H$_2$ formation on the grain surface.
  The model parameters are tabulated in Table \ref{tab:param}.

%-----------------------------------------
\subsubsection{The Parameters Related to the Radiation Fields, $(\chi_\mathrm{co},\chi_\mathrm{oh})$}
%-----------------------------------------

As mentioned in Section \ref{sec:fuv}, 
in order to investigate how the CO chemistry depends on radiation fields,
the weak-FUV and strong-FUV models are considered.
They are characterized by $\chico$ and $\chioh$.

%  For the radiation field,
%  we consider the two kinds of spectral types (A5V and A1V) as mentioned in Section \ref{sec:fuv}.
%  They are characterized by the normalized FUV flux $\chico$.

%   The size distribution of dust grains are characterized by $\amax$ and $\amin$. 
%   As mentioned in Section \ref{sec:size}, 
%   {\color{blue}
%   $\amin$ is fixed to $1~\mu$m.
%   Although $\amax$ is highly uncertain, 
%   $\amax$ is set to 10~$\mu$m.
%   As will be discussed in Section \ref{sec:plane}, our results are 
%   independent of $\amax$ as long as $\fracdg/\fracdgfid$ is fixed.
%   }
%   $\amin$ is fixed to be $1~\mu$m, and 
%   two maximum grain radii $a_\mathrm{max}=10~\mu $m and $a_\mathrm{max}=1$~cm are considered.


%  The properties of dust grains are characterized by $\amax$ and $\fracdg$.
%  As mentioned in Section \ref{sec:size}, $\amin$ is fixed to be $1~\mu$m, and 
%  two maximum grain radii $a_\mathrm{max}=10~\mu $m and $a_\mathrm{max}=1$~cm are considered.
%  For the dust-to-gas mass ratio, 
%  we firstly determine a fiducial value of $\fracdg$ by the following argument.

%-----------------------------------------
\subsubsection{The Parameter Related to the Amount of Dust Grains, $f_\mathrm{dg}$}
%-----------------------------------------

   The amount of dust grains is characterized by $\NCtaumidfid$.
   We here estimate  $\NCtaumidfid$ from observation results of $\beta$ Pictoris and 49 Ceti, which 
  are examples of the stars having gas-poor and gas-rich debris disks, respectively.
      Since both the debris disks are almost edge-on, the observed column densities are comparable to 
      those along the mid-plane.
      The total column density of species ``A'' integrated along the mid-plane is denoted by ${\cal N}(A)$.
  
  For $\beta$ Pictoris, several authors derive the C$^0$ column densities from the [C{\sc I}] $^3P_1$-$^3P_0$ emission line
  \citep{Higuchi2017,Cataldi2018} and $^3P$ absorption lines \citep{Roberge2000}.
  Their values range from ${\cal N}(\mathrm{C}^0)\sim 2\times 10^{16}$ to $7\times 10^{16}~\psc$.
  The C$^+$ column densities are estimated to be $\sim 2\times 10^{16}-12\times 10^{16}~\psc$ from the [C{\sc I$\!$I}] 
  158~$\mu$m emission line \citep{Cataldi2014,Cataldi2018} and 
  absorption lines \citep{Roberge2006}.
  The CO column densities are estimated to be $3\times 10^{14}-5
  \times 10^{14}~\psc$ from emission lines \citep{Higuchi2017} 
  and $\sim 6\pm 0.3\times 10^{14}~\psc$ from
  absorption lines \citep{Roberge2000}.
  These observations suggest that ${\cal N}(\mathrm{C}^0)$ is comparable to ${\cal N}(\mathrm{C}^+)$, and 
   ${\cal N}(\mathrm{CO})$ is roughly two orders of magnitude smaller than ${\cal N}(\mathrm{C}^0)$ and  ${\cal N}(\mathrm{C}^+)$.
  The carbon nucleus column density ${\cal N}_\mathrm{C}$ is estimated to be $\sim {\cal N}(\mathrm{C}^0) + {\cal N}(\mathrm{C}^+)\sim 10^{17}~\psc$. 
  The spatial distribution of the dust surface area obtained by HST observations of \citet{Heap2000} is fitted by \citet{Fernandez2006}.
  The mid-plane optical depth is about $\taumid \sim 10^{-2}$, 
  where the fitting formula is integrated over $50~\mathrm{au}\le R\le 120~\mathrm{au}$, which 
  is the gas extent of the best fit model in
  \citet{Cataldi2018}.
  Finally, $\NCtaumid^\mathrm{\beta Pic}\sim {\cal N}_\mathrm{C}/\taumid \sim 10^{19}~\psc$ is obtained for $\beta$ Pictoris.

  For 49 Ceti, the C$^0$ and CO column densities are estimated to 
  be ${\cal N}(\mathrm{C}^0)\sim 
  2\times 10^{18}~\psc$ \citep{Higuchi2019} and 
  ${\cal N}(\mathrm{CO})
  \sim 1.8\times 10^{17}-5.9 \times 10^{17}~\psc$ \citep{Higuchi2020}, 
  respectively. 
  The C$^+$ column density is not well constrained because only a lower limit of the C$^+$ mass, 
  which is smaller than the CO mass, was obtained \citep{Roberge2013}.
  We however expect that  ${\cal N}_\mathrm{C}$ 
  is comparable to ${\cal N}(\mathrm{C}^0)$ because 
  ${\cal N}(\mathrm{C}^+)$
  should not be much larger than 
   ${\cal N}(\mathrm{C}^0)$ if all CO are formed 
  through chemical reactions.
%  The carbon nucleus surface number density $\NC$ is larger than $\sim 2\times 10^{18}~\psc$ if the contribution from C$^+$ is considered.
  The vertical 
  optical depth estimated from the fractional luminosity is 
  $\tau \sim  10^{-3}$ \citep{Jura1993,Jura1998}.
  Assuming that the aspect ratio of the dust distribution is $\sim 0.1$, the mid-plane optical depth is $\taumid \sim 10^{-2}$.
  Finally, we obtain $\NCtaumid^\mathrm{49Ceti}\sim 2\times 10^{20}~\psc$ for 49 Ceti.


  From $\NCtaumid^\mathrm{\beta Pic}=10^{19}~\psc$
  and $\NCtaumid^\mathrm{49Ceti}=2\times 10^{20}~\psc$, 
  a fiducial value of $\NCtaumid={\cal N}_\mathrm{C}/\tau_\mathrm{mid}$ is set to $ \NCtaumidfid=2\times 10^{19}~\psc$
  ($\NCtaumid^\mathrm{\beta Pic}=0.5\NCtaumidfid$, $\NCtaumid^\mathrm{49Ceti}=10\NCtaumidfid$) 
  because the behavior of the CO formation changes around 
  ${\cal N}_\mathrm{C,\taumid=1}\sim \NCtaumidfid$ 
  for weak-FUV and strong-FUV 
  as will be shown in Section \ref{sec:result}.
  
  


%  \citet{Higuchi2017} dectected the   from which the CI column density is evaluated to be $2.2\pm 0.2\times 10^{17}~\psc$, which 
%  is consistent with the value measured in absorption by \citet{Roberge2000}.
%  With a model fitting, \citet{Cataldi2018} estimated the CI column density of $6-7\times 10^{16}~\psc$ 
%  using ALMA observations and the CI$\!$I column density of $5-12\times 10^{16}~\psc$ 
%  using the Herschel observations \citep[also see][]{Roberge2006,Cataldi2014}. 
%  Observationally, the C$^0$ column densities were estimated to
%  $(2.1\pm 0.2)\times 10^{17}~$cm$^{-2}$ for 49 Ceti and
%  $(2.4\pm 0.5)\times 10^{16}~$cm$^{-2}$ for $\beta$ Pictoris \citep{Higuchi2017} 

%  although these values contain a lot of uncertainty because the C$^0$ emissions are not 
%  spatially resolved \citep[also see][]{Higuchi2019}.
%  The carbon nucleus column density $\NC$ is set to $3\times 10^{16}$~cm$^{-2}$
%  because the C$^0$ masses are an order of magnitude higher than that of the CO masses.
%  we set $N_\mathrm{C,\taumid=1}^\mathrm{fid}$ to $1.67\times 10^{19}$~cm$^{-2}$
%  because the C$^0$ masses are an order of magnitude higher than that of the CO masses.
%  We should note that only C$^0$ and CO are considered to determine the 
%  $N_\mathrm{C,\taumid=1}^\mathrm{fid}$ because their spatial distributions 
%  have a good correlation \citep{Cataldi2018,Higuchi2019}.
%  The effect of C$^+$ will be discussed in Section \ref{sec:observation}.



In this paper, instead of $\NCtaumid$, a normalized dust-to-gas mass ratio is used as the 
parameter indicating the amount of dust grains. Using Equation (\ref{fdg}), one obtains 
\begin{equation}
%    \frac{\fracdg}{\fracdgfid} =  \left( \frac{N_\mathrm{C,\taumid=1}}
%    {2\times 10^{19}~\mathrm{cm}^{-2}} \right)^{-1}.
\frac{\fracdg}{\fracdgfid} =  \left( \frac{\NCtaumid}{\NCtaumidfid} \right)^{-1},
    \label{fdg_tau_Nc}
\end{equation}
where $\fracdgfid$ is the dust-to-gas mass ratio at  $\NCtaumid=\NCtaumidfid$ and 
given by 
  \begin{eqnarray}
    \fracdgfid(\amin,\amax,Z) &=& 3\times 10^{-3}
    \left\{ Z^{-1} + 6.6\times 10^{-3} \right\}^{-1} \nonumber \\
 &&   \times \left( \frac{\amin}{1~\mathrm{\mu m}} \right)^{1/2}
    \left( \frac{\amax}{10~\mathrm{\mu m}} \right)^{1/2},
    \label{fdgfid}
  \end{eqnarray}
where $n(a)\propto a^{-3.5}$ is used.
We should note that although both $\fracdg$ and $\fracdgfid$ depend on $\amin$ and $\amax$, 
the ratio $\fracdg/\fracdgfid$ is independent of the size distribution for a given  $\NCtaumid$.
%The three values of {\color{blue} $\NCtaumid$ can be rewritten in terms of
%$\fracdg/\fracdgfid$ as listed in Table \ref{tab:param}}.
%%%

To study how the amount of dust grains affects the thermal and chemical 
structures, 
the PDR calculations are performed with three different  $\fracdg$ as 
listed in Table \ref{tab:param}.
For weak-FUV, since a typical $\fracdg$ is about $\sim 2\fracdgfid$
for $\beta$ Pictoris, the cases with 
($0.1\fracdgfid$, $\fracdgfid$, $10\fracdgfid$) are considered.
For strong-FUV, since a typical $\fracdg$ is about $\sim 0.1\fracdgfid$, 
the cases with ($10^{-2}\fracdgfid$, $0.1\fracdgfid$, $\fracdgfid$)  
are considered.


%  The grain-surface chemistry and photo-electric heating 
%  from dust grains  are characterized by 
%  the total geometric cross section of dust grains $n_\mathrm{d} \langle \pi a^2\rangle$.
%  Equations (\ref{meanfreepath}) and (\ref{fdg_tau_Nc}) give 
%  \begin{equation}
%      n_\mathrm{d}\langle \pi a^2\rangle
%      = \frac{n_\mathrm{C}}{ {\color{blue} \NCtaumid}}
%      = \left( \frac{\fracdg}{\fracdgfid} \right) \frac{\nC}{ {\color{blue} \NCtaumidfid}},
%      \label{crosssec}
%  \end{equation}
%  where we use the fact that $Q_\mathrm{abs}=1$ because $Q_\mathrm{abs}\sim 1$ for $a > 1~\mu$m.
%%  Equation (\ref{crosssec}) shows that $n_\mathrm{d}\langle \pi a^2\rangle$ 
%%  is controlled by $\fracdg/\fracdgfid$. 
%  Equation (\ref{crosssec}) shows that $n_\mathrm{d}\langle \pi a^2\rangle$ 
%  is controlled by $\nC$ and $\fracdg/\fracdgfid \propto {\color{blue} \NCtaumid^{-1}}$. 
%  In this paper, since {\color{blue} $\NCtaumid$} is treated as a quantity given by observations, $n_\mathrm{d}\langle \pi a^2\rangle$ 
%  does not depend on the size distribution of dust grains {\color{blue} if $\nC$ and $\fracdg/\fracdgfid$ are fixed.}

%-----------------------------------------
\subsubsection{The Parameters Related to the Gas Component, ($\nC, Z$)}
%-----------------------------------------

The parameters associated with the gas component are $\nC$ and metallicity $Z$.
Given $\nC$ and $Z$, the corresponding hydrogen nucleus number density is 
 $\nH = \nC (\Ac Z)^{-1}$,
where $\Ac=1.32\times 10^{-4}$ is the carbon elemental abundance of the gas for $Z=1$ 
(Table \ref{tab:abundance}).
The hydrogen gas density is difficult to constrain 
in most debris disks observationally
although there are some successful exceptions  which will be described in Section \ref{sec:caveat}.
%\citep[e.g.,][]{Freudling1995,Lecavelier2001,Matra2017}.


%A fiducial carbon nucleus density is defined as $\nC^\mathrm{fid}=1.32\times 10^2~\pcc$;
%the corresponding hydrogen nucleus density is $\nH=10^6~\pcc$ for $Z=1$.
Fiducial carbon nucleus number densities are defined as $\nC=1.32\times 10^2~\pcc$ for weak-FUV %the A5V star
and $\nC=1.32\times 10^3~\pcc$ for strong-FUV.
%In this work, $\nC$ is varied from $\nC = 10^{-1}\nC^\mathrm{fid}$ to $10\nC^\mathrm{fid}$.
%This is because the mean densities of the debris disks around $\beta$ Pictoris and 49Ceti 
%take values between $\nC^\mathrm{fid}$ and $10\nC^\mathrm{fid}$, as shown below.
The debris disk around $\beta$ Pictoris has a spatial extent of $\sim 70~\mathrm{au}$ in the best-fit model in 
\citet[][also see Section \ref{sec:betaPic}]{Cataldi2018}.
A mean density is $\nC\sim 10^{17}~\psc/70~\mathrm{au}\sim 100~\pcc$, which is comparable to the fiducial $\nC$ for weak-FUV.
For 49 Ceti, because the radial extent is $\sim 100~$au, 
which corresponds to a cut-off radius of the surface density distribution 
\citep[][also see Section \ref{sec:49Ceti}]{Hughes2018},
the mean density is $\nC \sim 2\times 10^{18}~\psc/100~\mathrm{au} 
\sim 10^3~\pcc$, which is comparable to 
the fiducial $\nC$ for strong-FUV.
%For comparison, the case $\nC=0.1\nC^\mathrm{fid}$ is also considered.
For comparison, the densities 10 times larger than the typical $\nC$ are also 
considered for both the weak-FUV and strong-FUV models.


%The reason why the fiducial density is chosen is as follows:
%In this paper, we will show that CO formation proceeds without H$_2$ 
%for $\fracdg=0.1\fracdgfid$, and H$_2$ accelerates CO formation for $\fracdg=10\fracdgfid$.
%The $\nC=\nC^\mathrm{fid}$ case shows both behaviors clearly. 
%We note that $\nC^\mathrm{fid}$ is just a representative value.
%A limitation of our models is that the gas density is assumed to be spatially constant.
%Since in reality the gas density could vary spatially in a debris disk, 
%it is worth considering a wide range of the gas densities.
%Yet the constant density could still be a reasonable approximation because 
%the chemical abundances are expected to be determined mainly by the local quantities
%owing to the optically-thin nature of debris disks.

Next, we consider a range of the gas metallicity.
The gas metallicity in debris disks depends on its origin. 
If the gas is a remnant of PPDs, the gas metallicity is 
expected not to be significantly different from the solar metallicity $Z=1$. 
It is maximized when all the gas comes from the secondary processes; hydrogen is provided by photo-dissociation of H$_2$O outgassed from 
dust grains. In this case, the number of hydrogen nuclei becomes comparable to that of oxygen nuclei, 
and the gas metallicity reaches $1.6\times 10^{3}$ since ${\cal A}_\mathrm{O} = 3.2\times 10^{-4}$ at $Z=1$ (Table \ref{tab:abundance}).
In this paper, a range of $1\le Z\le 10^3$ is considered. 

%-----------------------------------------
\subsubsection{The Parameter Related to the H$_2$ Formation on the Grain Surface, $T_\mathrm{chem}$}
%-----------------------------------------

In order to investigate how the uncertainty of the H$_2$ formation affects CO formation (Section \ref{sec:dustsurface}),  
the cases with $T_\mathrm{chem}=300$~K and $10$~K are considered.
The fiducial case $T_\mathrm{chem}=300~$K leads 
to a moderate H$_2$ formation rate, corresponding to situations where 
chemisorption efficiently occurs on grains with plenty of surface defects \citep{LeBourlot2012}. 
The cases with $T_\mathrm{chem}=10~$K give an upper limit of the H$_2$ formation rate.
Other parameters $T_\mathrm{stick}$ and $\beta$ are fixed to $464$~K and 1.5, respectively, because
the gas temperatures do not exceed $\sim 500$~K in most cases.

\begin{table*}[htbp]
    \begin{center}
    \begin{tabular}{|l|c|c|}
        \hline
        parameter  & weak-FUV & strong-FUV \\
        \hline
        \hline
        FUV flux (CO, H$_2$, C$^+$) $\chico$ & $1.4$ & $31$ \\
        \hline
         FUV flux (OH) $\chioh$ & 
        $2.4$ & $24$  \\
        \hline
        dust-to-gas mass ratio $\fracdg$ & 
        $0.1\fracdgfid$, $\fracdgfid$, $10\fracdgfid$ &
        $0.01\fracdgfid$, $0.1\fracdgfid$, $\fracdgfid$  
        \\
        \hline
        gas density $\nC~[\pcc]$ 
%        gas density $\nC~[\pcc]$ & $0.1 n_\mathrm{C,fid}, n_\mathrm{C,fid}, 10n_\mathrm{C,fid}$ \\
%        \begin{minipage}{0.3\textwidth}
%            \vspace{1mm}
%        \begin{center}
%            {\color{blue}
%%        $n_\mathrm{C,fid}=1.32\times 10^2~\pcc$ (A5V)\\
%%        $n_\mathrm{C,fid}=1.32\times 10^3~\pcc$ (A1V)
%        $n_\mathrm{C,fid}=1.32\times 10^2~\pcc$ (weak-FUV)\\
%        $n_\mathrm{C,fid}=1.32\times 10^3~\pcc$ (strong-FUV)
%    }
%        \end{center}
%            \vspace{1mm}
%        \end{minipage} 
        & 
        $1.32\times 10^2~\pcc, 1.32\times 10^3~\pcc$ 
        &
        $1.32\times 10^3~\pcc, 1.32\times 10^4~\pcc$ \\
%        $n_\mathrm{C,fid}, 10n_\mathrm{C,fid}$ \\
        \hline
        the gas metallicity $Z$ & \multicolumn{2}{c|}{$1,10,10^2,10^3$} \\
        \hline
        {
        \begin{minipage}{5cm}
        $T_\mathrm{chem}$~[K]$^{(1)}$
        \end{minipage}
        } & 
         \multicolumn{2}{c|}{ 300, 10} \\
        \hline
    \end{tabular}
    \caption{ 
   {  $^{(1)}$The energy barrier of  chemisorption.}
    List of the model parameters considered in the plane-parallel PDR calculations}.
    \label{tab:param}
    \end{center}
\end{table*}

%-----------------------------
\section{Results}\label{sec:result}
%-----------------------------

\begin{figure*}[htpb]
        \centering
%       \includegraphics[width=17cm]{Figure3.eps}
%       \includegraphics[width=17cm]{plane-parallel-fdg.eps}
%       \includegraphics[width=18cm]{plane-parallel-fdg_CI_rev4.eps}
       \includegraphics[width=18cm]{Fig2.pdf}
\caption{
    Results of the PDR calculations where the incident flux 
    corresponds to the summation of ISRF and 
    stellar radiation 
    flux at a representative distance of 50~au from the central star.
    Profiles of the gas temperature 
    (left column), $2\nHm/\nH$ (middle column), and  $\nCO/\nC$ (right column) as 
    a function of $\NC$.
        The top and bottom panels show the results for weak-FUV and strong-FUV models, respectively.
        The gas metallicity is fixed to $Z=1$, and 
%        the gas densities are fixed to be $\nC=1.32\times 10^2~\pcc$ for the A5V case and $\nC=1.32\times 10^3~\pcc$ for the A1V case. 
        the gas densities are fixed to $\nC=1.32\times 10^2~\pcc$ for weak-FUV and $\nC=1.32\times 10^3~\pcc$ for strong-FUV. 
%        The top and bottom panels show the results for the A5V and A1V cases, respectively.
        The colors represent $\fracdg$ which are $(0.1\fracdgfid,\fracdgfid, 10\fracdgfid)$ for weak-FUV and
        $(0.01\fracdgfid,0.1\fracdgfid,\fracdgfid)$ for strong-FUV. 
        The thick solid and dotted lines correspond to the results with $T_\mathrm{chem}=300~$K and $10~$K, respectively.
        The thin solid line indicates the results without the H$_2$ formation on  the grain surface. 
%    For each panel, we show the results with 
%    $2\times 3$ combinations of the dust parameters of  
%    $\amax = (10^{-3}~\mathrm{cm}, \mathrm{1~cm})$ and 
%    $\fracdg=(0.1\fracdgfid,\fracdgfid, 10\fracdgfid)$. 
%    The thin solid lines show the results without H$_2$ formation on grain surfaces.
}
   \label{fig:plane}
\end{figure*}
%---------------------------------------------------------------------------------

\subsection{Overall Behaviors of the Models with the Solar Metallicity
}\label{sec:plane}
%----------------------------------------------------------------------------------------


%In debris disks, the amount of dust grains and the total area of the surface of 
%dust grains  per unit gas mass are much smaller than those for the ISM and PPDs.
In this section, we investigate how the results of
the fiducial models 
($\nC=1.32\times 10^2~\pcc$ for weak-FUV and 
$\nC=1.32\times 10^3~\pcc$ for strong-FUV)
depend on the dust parameters $(\fracdg/\fracdgfid,  T_\mathrm{chem})$ 
and incident FUV fluxes ($\chico$, $\chioh$).
The gas metallicity is set to $Z=1$.



We should note that 
the results with different $\amin$ and 
$\amax$ are not presented
because our results are independent of 
both $\amin$ and $\amax$ as long as 
$\fracdg/\fracdgfid$ and $\nC$ 
remain unchanged.
This is explained as follows.
The grain-surface chemistry and photo-electric heating 
from dust grains  are characterized by 
the total geometric cross section of dust grains $n_\mathrm{d} \langle \pi a^2\rangle$.
  Equations (\ref{meanfreepath}) and (\ref{fdg_tau_Nc}) give 
  \begin{equation}
      n_\mathrm{d}\langle \pi a^2\rangle
      = \frac{n_\mathrm{C}}{ \NCtaumid}
      = \left( \frac{\fracdg}{\fracdgfid} \right) \frac{\nC}{ \NCtaumidfid},
      \label{crosssec}
  \end{equation}
  where we use the fact that $Q_\mathrm{abs}=1$ because $Q_\mathrm{abs}\sim 1$ for $a > 1~\mu$m.
%  Equation (\ref{crosssec}) shows that $n_\mathrm{d}\langle \pi a^2\rangle$ 
%  is controlled by $\fracdg/\fracdgfid$. 
  Equation (\ref{crosssec}) shows that $n_\mathrm{d}\langle \pi a^2\rangle$ 
  is controlled by $\nC$ and $\fracdg/\fracdgfid$. 
  Since $\NCtaumid^\mathrm{fid}$
  is a given quantity, 
  $n_\mathrm{d}\langle \pi a^2\rangle$ 
  does not depend on the size distribution of 
  dust grains 
  if $\nC$ and $\fracdg/\fracdgfid$ are fixed.
Thus, the thermal and chemical processes related to dust grains, i.e., 
H$_2$ formation on dust grains and photo-electric heating,
are independent of the size distribution for fixed $\nC$ and $\fracdg/\fracdgfid$.

The results are shown in Figure \ref{fig:plane}.
In this paper, instead of the optical depth, $\NC$ is used 
as a measure of the spatial coordinate,
where $\NC$ is the carbon nucleus column density integrated from the slab edge to a given position.
In Figure \ref{fig:plane}, the maximum column densities are set 
to $\NC=10^{17}~\psc$ for weak-FUV and $\NC=2\times 10^{18}~\psc$ for strong-FUV 
according to the observational constraints shown in Section \ref{sec:modelparameter}.
%It is useful to derive the relation between $\fracdg/\fracdgfid$, $\tau$, and $\NC$.the
%Substituting $\NC = \tau N_\mathrm{C,\tau=1}$ in Equation (\ref{fdg_tau_Nc}), one obtains 
%\begin{equation}
%    \left(\frac{\NC}{2\times 10^{17}~\mathrm{cm^{-2}}}\right) = 
%  \left(\frac{\fracdg}{\fracdgfid}\right)^{-1}
%  \left(\frac{\tau_\mathrm{mid}}{10^{-2}}\right).
%  \label{tau}
%\end{equation}



%Figures \ref{fig:plane} show the profiles of the gas temperatures, 
%and abundances of H$_2$ and CO for the A5V and A1V cases with
%the abundances of H$_2$ and CO {\color{blue}for the weak-FUV and strong-FUV models}. 
%various dust parameters; the $2\times 3$ combinations of the dust parameters 
%$\amax = (10^{-3}~\mathrm{cm}, \mathrm{1~cm})$ and $\fracdg=(0.1\fracdgfid, \fracdgfid, 10\fracdgfid)$. 


%Figures \ref{fig:plane} clearly show that 
%all the quantities are independent of $\amax$ for a fixed $\fracdg/\fracdgfid$
%although $\amax$ varies by three orders of magnitude. 
%This is simply because the total surface area of dust grains per volume
%remains unchanged when $\fracdg/\fracdgfid$ and $\nC$ are fixed (Equation \ref{crosssec}).
%The thermal and chemical processes related to dust grains, i.e., 
%H$_2$ formation on dust grains and photo-electric heating,
%are characterized by $\fracdg/\fracdgfid$ and $\nC$.

%Since it has been confirmed that the results are independent of 
%$\amax$ as long as $\fracdg$ is changed so that $\fracdg/\fracdgfid$ remains unchanged, $\amax$ is fixed to be $10^{-3}~$cm hereafter.
%The minimum dust radius $\amin$ is fixed in this section, but our results are independent of both $\amin$ and $\amax$ as long as $\fracdg/\fracdgfid$, or {\color{blue} $\NCtaumid$}, remain unchanged.
%Thus, $\amax$ is fixed to be $10^{-3}~$cm hereafter.
%



%%%------------------------------------------------------------
\subsubsection{Gas temperature}\label{sec:planeT}
%%%------------------------------------------------------------
%%
Figures \ref{fig:plane}a and \ref{fig:plane}b show 
that the gas temperatures increase with $\fracdg$ for 
both the weak-FUV and strong-FUV models
because the photo-electric heating rate from dust grains, which 
is one of the dominant heating processes, increases in proportion to $\fracdg/\fracdgfid$.
A weak positive dependence of $T$ on $T_\mathrm{chem}$ 
is attributed to enhancement of the heating rate owing to the 
H$_2$ formation on the grain surface.

Comparison between Figures \ref{fig:plane}a and \ref{fig:plane}b shows that 
the gas temperatures increase with $\chico$ because 
the photo-electric heating rate increases with $\chico$.
%\deleted{
%We however should note that the photo-electric heating 
%rate is not proportional to $\chico$.
%The efficiency of photo-electric heating is determined not only by $\chico$ but also 
%by the charge of dust grains.
%The charge increases with $\chico$, making the ionization potential of dust grains deeper.
%Then the critical photon energy, above which the photon can contribute to 
%the photo-electric heating, increases.
%The positive effect of increased $\chico$ is partly compensated by the negative effect of dust charge.
%}

%----------------------------------------------
\subsubsection{Molecular Hydrogen}\label{sec:planeH2}
%----------------------------------------------
Before presenting the results, we introduce the analytic formation rate 
of H$_2$ on grain surfaces, 
\begin{equation}
    F_\mathrm{H_2,dust} = 3\times 10^{-3}R_\mathrm{H_2,ISM}~Z
    \left( \frac{\fracdg}{\fracdgfid} \right) \sqrt{T} \kappa(T) \nH n(\mathrm{H})
  \label{RER}
\end{equation}
\citep{LeBourlot2012},
where Equation (\ref{crosssec}) is used, $R_\mathrm{H_2,ISM}=3\times 10^{-17}~$cm$^{3}$~s$^{-1}$ 
is a typical rate coefficient in the ISM \citep{Jura1974}, and 
$\kappa(T)$ is the chemisorption efficiency\footnote{
We should note that the Meudon code does not solve Equation (\ref{RER}) directly, but 
solves a set of rate equations considering chemisorbed hydrogen atoms and the dust size distribution
\citep[see][for details]{LeBourlot2012}.
In addition to the chemisorbed H atoms, the Meudon code treats H$_2$ formation with physisorbed H atoms.
}.
%Equation (\ref{RER}) is an analytic formula of the ER mechanism,
%which is dominant at high dust and gas temperatures.
%The dust temperatures, which are as high as $\sim$100~K for both the A5V and A1V cases, 
%are so high that a hydrogen atom physisorbed on the grain surface 
%evaporates before it can combine with another hydrogen atom.
\citet{LeBourlot2012} derived 
%Although there is a large uncertainty,
%the Meudon code adopts
\begin{equation}
    %    \kappa(T) = \frac{\exp\left( -300/T \right)}{ \exp\left( -300/T \right) + 1+ (T/464~\mathrm{K})^{1.5}}
%        \kappa(T) = \frac{\exp\left( -T_1/T \right)}{ \exp\left( -300/T \right) + 1+ (T/464~\mathrm{K})^{1.5}}
        \kappa(T) = \frac{\alpha(T)\exp\left( -T_\mathrm{chem}/T \right)}
        {1+\alpha(T)\exp\left( -T_\mathrm{chem}/T \right)},
    \label{kappa}
\end{equation}
where $\alpha(T)$ and $T_\mathrm{chem}$ are defined in Section \ref{sec:dustsurface}.
%$\alpha(T)=1/(1+(T/T_\mathrm{stick})^\beta)$ 
%is a sticking coefficient, which approaches zero for high temperatures 
%because the hydrogen atom cannot stick on the dust grains owing to the excess kinetic energy,
%and $T_\mathrm{chem}$ corresponds to the energy barrier that a hydrogen atom must overcome 
%to reach a chemisorption site on the grain surface.
%The adopted parameters are $T_\mathrm{stick}=464~$K, $\beta=3/2$, and $T_\mathrm{chem}=300~$K, which are 
%fiducial values of the Meudon code.
%We will discuss the uncertainty of the H$_2$ formation rate of the ER mechanism 
%in Section \ref{sec:ER}, and the parameter dependence of the CO fraction in Sections \ref{sec:betaPic} and \ref{sec:49Ceti}.
For gas temperatures lower than $T_\mathrm{stick}=464~$K, which is 
satisfied in all the cases shown in Figure \ref{fig:plane}, 
the H$_2$ formation rate is proportional to $\exp(-T_\mathrm{chem}/T)$, and 
it depends sensitively on $T$.

First, the cases with $T_\mathrm{chem}=300~$K are considered.
Although H$_2$ mainly forms on grain surfaces for the ISM, this may not be the case 
for debris disks because the H$_2$ formation rate is extremely small.
In order to examine the contribution of the grain-surface chemistry to H$_2$ formation, 
we performed the additional PDR calculations where the grain-surface chemistry 
is switched off and plotted the results by the thin lines in 
the middle column of Figure \ref{fig:plane}.
%The H$_2$ fractions are independent of $\fracdg$ 
%when the grain-surface chemistry is switched-off 
%as is expected.
For $\fracdg\le\fracdgfid$ (weak-FUV) 
and $\fracdg\le 0.1\fracdgfid$ (strong-FUV), 
there are almost no changes in the H$_2$ fractions with and without
the grain-surface chemistry (Figures \ref{fig:plane}c and \ref{fig:plane}d), 
indicating that the grain-surface chemistry does not contribute to H$_2$ formation.
In this case, H$_2$ is mainly formed in the gas phase,
and the dominant chemical reaction is 
$\mathrm{CH^+ + H\rightarrow H_2 + C^+}$,
where CH$^+$ is formed by the radiative association between C$^+$ and H\footnote{
Especially for metal-poor environments, the so-called H$^-$ channel is an important H$_2$ formation path \citep[e.g.,][]{Sternberg2021}.
Additional PDR calculations were performed with chemical reactions associated with H$^-$, which 
had not been activated in the default setting of the Meudon code.
The photo-detachment of H$^-$ is computed directly from $\int \sigma_\lambda I_\lambda/h\nu d\lambda$ (Section \ref{sec:mod}),
where the cross section obtained in \citet{Wishart1979} is used while the Meudon code calculates
it by using a fitting formula.
We confirmed that the H$^-$ channel does not affect the CO chemistry.
}. 


The H$_2$ formation rate of the grain-surface chemistry 
is enhanced in proportion to $\fracdg$ (Equation (\ref{RER})) while
the rate in the gas phase does not change.
As shown in Figures \ref{fig:plane}c and \ref{fig:plane}d, 
$\fracdg$ is high enough for H$_2$ to be formed mainly through 
the grain-surface chemistry when $\fracdg=10\fracdgfid$ for weak-FUV and 
$\fracdg=\fracdgfid$ for strong-FUV.

For $T_\mathrm{chem}=300~$K, 
the H$_2$ fractions stay low level even after 
they increase rapidly around $\sim 10^{14}~\psc$ owing to self-shielding \citep{DraineBertoldi1996}. 
This is because the H$_2$ formation rate is so low that the destruction 
reactions in the gas phase and H$_2$ photo-dissociation keep the H$_2$ fractions low.


%For $\fracdg \le\fracdgfid$, 
%$2\nHm/\nH$ does not reach unity even when the self-shielding effect 
%becomes important because of inefficient H$_2$ formation.  



%Note that the H$_2$ abundances are more than ten times larger for $\fracdg = 10\fracdgfid$ than 
%for $\fracdg = \fracdgfid$.
%This comes from the sensitive $T$ dependence of $F_\mathrm{H_2,dust}$ shown in Equation (\ref{RER}).

%All the lines in Figures \ref{fig:plane}c and \ref{fig:plane}d
%have the common feature; $\nHm$ takes a constant value for small $\NC$, and 
%begins to increase rapidly when $\NC$ exceeds the point shown by the filled circle, 
%which corresponds to $\NHm = 10^{14}~\psc$. 
%The rapid increases in $\nHm$ are caused by self-shielding \citep{vanDishoeck1988}.
%For each spectral type, the self-shielding effect becomes effective at lower $\NC$ for larger
%$\fracdg$ since H$_2$ is accumulated at a faster rate for larger $\fracdg$.


As mentioned in Section \ref{sec:dustsurface}, there 
is large uncertainty in H$_2$ formation on the warm dust grains with 
$T_\mathrm{gr}\sim 100~$K.
A decrease in $T_\mathrm{chem}$ from 300~K to 10~K 
significantly enhances the H$_2$ fractions
in Figures \ref{fig:plane}c and \ref{fig:plane}d 
because $\kappa(T)$ depends sensitively on $T_\mathrm{chem}$.
Considering the gas temperatures shown in Figures \ref{fig:plane}a and \ref{fig:plane}b, 
the H$_2$ formation rate increases by a factor of 
$\sim 1.6\times 10^4$ at $T=30$~K and $\sim 1.3\times 10^2$ at $T=60$~K.


%For the strong FUV models with $\fracdg=0.01\fracdgfid$, 
%the H$_2$ fractions do not increase significantly by reducing $T_\mathrm{chem}$
%because $\fracdg=0.01\fracdgfid$ is too small 
%for $F_\mathrm{H_2,dust}$ to be much larger than the H$_2$ formation rate in  the gas phase.
%This indicates that when $\fracdg \ll \fracdgfid$, the H$_2$ fractions 
%do not increase significantly even if there is no energy barrier in chemisorption.



%The models with $T_\mathrm{chem}=10~$K yield a much larger amount of H$_2$ than 
%those with $T_\mathrm{chem}=300$~K.
%From Figures \ref{fig:plane}c and \ref{fig:plane}d, 
%comparison between the results with $T_\mathrm{chem}=300~$K and $10~$K 
%shows that the H$_2$ fraction increases significantly.

%The exception is found in the strong-FUV models 
%with $\fracdg=0.01\fracdgfid$.
%Since the gas temperature is about 40~K, the boost factor of 
%the H$_2$ formation rate is $\sim \exp(290/40.0) = 1.4\times 10^3$.

%The exception is found in the weak-FUV model with $\fracdg=10\fracdgfid$ where 
%the H$_2$ fraction increases only by 10\%. 
%There are two reasons for this behavior.
%One is that the gas temperatures ($\sim 150$~K) are so high that the H$_2$ formation rate increases only by 7.
%The other is that the $2n(\mathrm{H}_2)/\nH$ is as high as $\sim$ 0.6 owing to large $\fracdg$ for small $\NC$ 
%even for the fiducial case ($T_\mathrm{chem}=300$~K), indicating that $2n(\mathrm{H}_2)/\nH$ is insensitive to the H$_2$ formation rate.



%\deleted{
%The H$_2$ abundances drop when $\NC>3\times 10^{16}~\psc$ for the A5V case 
%and $\NC> 2\times 10^{17}~\psc$ for the A1V case. 
%The decrease in $\nHm$ is caused by a lack of C$^+$, which initiates H$_2$ formation in the gas phase,
%because most carbon nuclei are locked in CO in such deep interiors
%as shown in Figures \ref{fig:plane}e and \ref{fig:plane}f.
%The $\fracdg = 10\fracdgfid$ cases do not show such a decrease in $\nHm$ in the deep interior 
%where $\nCO\sim \nC$ because the dust-surface reaction dominates.
%}

%%%%------------------------------------------------------------
%\subsubsection{ Atomic and Ionized Carbon}\label{sec:CI_CII}
%%%%------------------------------------------------------------
%
%Figures \ref{fig:plane}e and \ref{fig:plane}f show $n(\mathrm{C}^0,\mathrm{C}^+)/\nC$ for the A5V and A1V cases.
%Both the C$^0$ and C$^+$ fractions are almost independent of $\fracdg$ and whether the H$_2$ formation on the dust surface is included.
%For the A1V case, the strong FUV flux makes C$^+$ more abundant than C$^0$ at the low $\NC$ limit while 
%for the A5V case, most carbon nuclei are present in atomic form.
%Once $\NC$ exceeds $\sim 10^{16-17}~\psc$, the C$^0$ (C$^+$) fraction begins to increase (decrease) rapidly.
%
%The behaviors shown in Figures \ref{fig:plane}e and \ref{fig:plane}f can be explained by the following 
%simple argument.
%The C$^0$ and C$^+$ fractions are determined by the balance between photo-ionization of 
%C$^0$ and radiative recombination of C$^+$ \citep{Kamp2000} as follows:
%\begin{equation}
%        \alpha_\mathrm{C} \chico e^{ -a_\mathrm{C}N(\mathrm{C}^0) }F_\mathrm{H} n(\mathrm{C^0})
%        \sim  k_\mathrm{rec} n(\mathrm{C^+}) n({e}^{-}),
%\label{ionC}
%\end{equation}
%where 
%$a_\mathrm{C}=1.777\times 10^{-17}~\mathrm{cm}^2$ \citep{Heays2017}\footnote{
%Note that the Meudon code takes into account the frequency-dependent radiative flux 
%to calculate the photo-ionization rate although $a_\mathrm{C}\chico F_\mathrm{H}$ gives a good estimate.
%}
%is the photo-ionization cross section \citep{vanDishoeck2006},
%$k_\mathrm{rec}(T)=1.7\times 10^{-11}~\mathrm{cm^3~s^{-1}}(T/100~\mathrm{K})^{-0.82}$\footnote{
%The expression of $k_\mathrm{rec}$ is a fitting formula
%of the recombination coefficient taking into account 
%radiative recombination \citep{Badnell2006} and 
%di-electronic recombination \citep{Badnell2003} in 
%$10~\mathrm{K}\le T\le 10^3~\mathrm{K}$.
%}
%is the recombination coefficient, and 
%the attenuation of FUV photons owing to atomic carbons is taken into account.
%%Figures \ref{fig:CI_CII}c and \ref{fig:CI_CII}d show that $n({e}^-)$ 
%%is approximately equal to $n(\mathrm{C^+})$ for most cases. 
%%For the A5V cases with $\nC>10^2~\pcc$, however, $n(e^-)$ is larger than $\nCp$. 
%%In such a high density, the abundance of C$^+$ 
%%becomes smaller than that of Si$^+$, and electrons are mainly supplied by 
%%photo-ionization of Si, which has a lower ionization energy of 8~eV than carbon atom.
%%With $n(e^-) \sim \nCp$,
%%Equation (\ref{ionC}) is rewritten as
%%\begin{equation}
%%    \frac{n(\mathrm{C^0})}{\nC} \sim f_\mathrm{c}(\xi) \equiv 1- \frac{1}{2\xi} \left[- 1+\sqrt{1+4\xi} \right],
%%\label{ionCana}
%%\end{equation}
%From Equation (\ref{ionC}) with $n(e^-) = \nCp$ and $\nC = n(\mathrm{C}^0) + \nCp$, 
%one obtains analytic formulae of the C$^0$ fraction given by 
%\begin{equation}
%     \frac{n(\mathrm{C}^0)_\mathrm{ana}}{\nC} = \frac{2\xi+1-\sqrt{1+4\xi}}{2\xi},
%    \label{nCp}
%\end{equation}
%%\begin{equation}
%%        \frac{n(\mathrm{C}^+)_\mathrm{ana}}{\nC} = \frac{ -1 + \sqrt{1+4\xi}}{2\xi}\;\;\;
%%    \mathrm{and}\;\;
%%%    \frac{n_\mathrm{ana}(\mathrm{C}^0)}{\nC} = 1 - \frac{n(\mathrm{C^+})}{\nC},
%%    \frac{n(\mathrm{C}^0)_\mathrm{ana}}{\nC} = \frac{2\xi}{2\xi+1+\sqrt{1+4\xi}}
%%    \label{nCp}
%%\end{equation}
%where $\xi$ is the ratio of the recombination to the ionization coefficients, and is given by 
%\begin{eqnarray}
%   \xi &\equiv& \frac{k_\mathrm{rec}n_\mathrm{C}}
%   {\alpha_\mathrm{C} \chico  e^{ -\alpha_\mathrm{C}N(\mathrm{C}^0) } F_\mathrm{H}} \nonumber \\
%   &=& \left(\chico e^{-\alpha_\mathrm{C}N(\mathrm{C}^0)}\right)^{-1}
%   \left(\frac{\nC}{11~\mathrm{cm}^{-3}}\right)
%    \left( \frac{T}{100~\mathrm{K}} \right)^{-0.82}.
%     \label{xi1}
%\end{eqnarray}
%The assumption $\nC=n(\mathrm{C}^0) + \nCp$ is valid when $n(\mathrm{C}^0) \gg \nCO$ because it is unlikely 
%that $\nCp > n(\mathrm{C}^0)$ becomes comparable to $\nCO$.
%Recently, the debris disk HD32297 where the CO mass is larger than the CI mass was discovered \citep{Cataldi2020}.
%An iterative method is used to derive the spatial distribution of $n(\mathrm{C}^0)_\mathrm{ana}/\nC$ 
%because $\xi$ depends on the C$^0$ column density integrated from the slab edge.
%The C$^+$ fraction is estimated from $\nCp_\mathrm{ana} = \nC-n(\mathrm{C}^0)_\mathrm{ana}$.
%
%From Figures \ref{fig:plane}e and \ref{fig:plane}f,
%our results are consistent with the predictions from the analytic formulae $n(\mathrm{C}^0)_\mathrm{ana}$ and $\nCp_\mathrm{ana}$.
%This indicates that the C$^0$ and C$^+$ fractions are determined by the simple chemical balance as shown above, and 
%the rapid increases in $n(\mathrm{C}^0)$ arise from the C$^0$ attenuation.
%Small differences in the C$^0$ and C$^+$ fractions between the models with different $\fracdg/\fracdgfid$ come from 
%differences in the gas temperatures (the first column of Figure \ref{fig:plane}).
%
%Equations (\ref{nCp}) and (\ref{xi1}) show that the C$^0$ fraction is independent of $\nH$ and $Z$ but
%depend on $\nC$, $\chico$, and $T$ because hydrogen does not 
%involved in the chemical balance between C$^0$ and C$^+$ \citep{Kamp2000}.
%We confirmed that the C$^0$ fractions obtained from our results are consistent with Equation (\ref{nCp}) quantitatively.
%

%---------------------------------------
\subsubsection{Carbon Monoxide, CO}\label{sec:planeCO}
%---------------------------------------

%First, we focus on the CO frations at the low $\NC$ limit.


First, the models with $T_\mathrm{chem}=300~$K are considered.
Figures \ref{fig:plane}e and \ref{fig:plane}f show that 
$\nCO/\nC$ does not depend on $\fracdg$ significantly 
as long as 
$\fracdg\le \fracdgfid$ for both weak-FUV and strong-FUV. 
%Figures \ref{fig:plane}e and \ref{fig:plane}f show that 
%for each $\chico$, $\nCO/\nC$ does not depend on $\fracdg$ although 
%$\nHm$ varies more than an order of magnitude.
The CO fractions decrease with increasing $\chico$ because 
stronger FUV flux destroys CO more efficiently.


When $\fracdg$ is increased from $\fracdgfid$ to $10\fracdgfid$, 
the weak-FUV models show that 
the CO fractions increase by a factor of 2 or 3.
%By contrast, for the strong-FUV models, the CO fraction with $\NC<10^{17}~\psc$ 
%for $\fracdg=\fracdgfid$ is comparable to that for $\fracdg=0.1\fracdgfid$.

Acceleration of the CO formation owing to H$_2$ is more significant 
for $T_\mathrm{chem}=10~$K than for $T_\mathrm{chem}=300$~K since 
a decrease in $T_\mathrm{chem}$ increases the H$_2$ fraction (Section \ref{sec:planeH2}).
The CO fractions monotonically increase with $\fracdg$. 

The $\fracdg$-dependence of the CO fractions disappears 
when the grain-surface chemistry is omitted from the PDR calculations.
This clearly indicates that 
the enhancement of the CO fraction for larger $\fracdg$ is related to 
efficient H$_2$ formation on the grain surface.


How much H$_2$ is needed to accelerate CO formation?
Comparing between the middle and right columns of Figure \ref{fig:plane},
one can see that H$_2$ fraction must be well above 0.1 for the CO fractions 
to increase by more than on order of magnitude.
When the H$_2$ fraction is comparable to 0.1 as in the strong-FUV models 
with ($\fracdg=0.01\fracdgfid$, $T_\mathrm{chem}=10~$K)
and ($\fracdg=\fracdgfid$, $T_\mathrm{chem}=300~$K), 
CO formation is not accelerated significantly.


In the deep interior $\NC>5\times 10^{16}~\psc$, 
the CO fractions increase owing to shielding effects, 
which will be discussed in Section \ref{sec:shielding}.


%\deleted{
%We cannot distinguish $\nCO$ with and without the 
%dust-surface chemistry at the low $\NC$ limit for all $\chico$ and $\fracdg$, indicating 
%that an amount of H$_2$ does not affect CO chemistry.
%}
%
%\deleted{
%As will be shown in Section 4 in detail, 
%in the H$_2$-poor environments, the CO formation starts from the radiative association between 
%O and H which produces OH. The product OH reacts with C$^+$ to form CO$^+$.
%Finally, CO forms through the reaction between CO$^+$ and H.
%CO formation also occurs by reacting OH with C.
%}
%
%\deleted{
%Let us move on to the deeper interior $\NC\sim 10^{13}-10^{15}~\psc$, where 
%the self-shielding of H$_2$ becomes significant.
%For $\fracdg=\fracdgfid$, $\nCO$ remains constant 
%because the amount of H$_2$ is too low for H$_2$ to enhance CO formation. 
%By contrast, for $\fracdg=10\fracdgfid$, 
%he A1V case shows that $\nCO$ increases slightly around $\NC\sim 5\times 10^{14}~\psc$ 
%where $\nHm$ rapidly increases owing to self-shielding.
%Abundant H$_2$ activates another formation path of OH that starts from 
%$\mathrm{O^++H_2\rightarrow OH^+ + H}$, where O$^+$ forms through 
%the charge exchange between H$^+$ and O.
%The product OH$^+$ reacts with H$_2$ to form H$_2$O$^+$. 
%The dissociative recombination of H$_2$O$^+$ forms OH, which is an ingredient of CO formation.
%By contrast, a similar increase in $\nCO$ 
%is not seen in the A5V case with $\fracdg=10\fracdgfid$ 
%although there is a small wavy structure at the HI-H$_2$ transition.
%This is because the low temperatures for the A5V case suppress
%the reaction $\mathrm{O+H^+\rightarrow O^+ + H}$, which 
%is endothermic by 227~K ().%\citep{Stancil1999}. 
%}
%\deleted{
%The photodissociation rates become half of the unshielded values at 
%$\NC = 6.3\times 10^{15}~\psc$ 
%($\NCO = 2\times 10^{14}~\psc$, $N(\mathrm{C^0}) = 5\times 10^{15}~\psc$, 
%$\NHm = 1.7\times 10^{17}~\psc$) for the A5V case  and 
%at $\NC = 6.5\times 10^{16}~\psc$ ($\NCO = 10^{14}~\psc$, $N(\mathrm{C^0}) = 2\times 10^{16}~\psc$, 
%$\NHm = 8.8\times 10^{18}~\psc$) for the A1V case. 
%The CO shielding by H$_2$ does not work because it becomes effective for 
%$\NHm > 10^{20}~\psc$ ()%\citep{vanDishoeck1988, Visser2009}.
%Attenuation owing to C$^0$ ionization decreases the dissociating photon flux by a factor of two 
%at $N(\mathrm{C}^0)\sim 4\times 10^{16}~\psc$, but $N(\mathrm{C}^0)$ does not reach this value
%for both A5V and A1V cases.
%The rapid increases in $\nCO/\nC$ are attributed to the self-shielding effect. 
%}
%
%

%%---------------------------------------------------------------------
%\subsection{ Overall Behaviors of the Models in Metal-rich Environments
%}\label{sec:planeZ}
%%---------------------------------------------------------------------
%
%In Section \ref{sec:plane}, we have seen that 
%for the solar metallicity, the H$_2$ formation on grain surfaces is 
%strongly suppressed due to low grain surface area and low gas temperatures. 
%As long as $\nHm$ is low, the CO formation proceeds without H$_2$, and 
%$\nCO$ does not depend on $\fracdg$ significantly.
%When $\nHm$ becomes sufficiently high, H$_2$ enhances the CO formation.
%
%In this section, we investigate how the thermal and chemical properties 
%depend on $Z$.
%{\color{blue}
%We focus on the weak-FUV models with $\fracdg=\fracdgfid$ where 
%the CO formation proceeds in the H$_2$-poor environments.
%The strong-FUV models provide qualitatively similar results.
%The parameter survey will be presented in Section \ref{sec:parametersurvey}.
%}
%It should be noted that $Z$ and $\fracdg/\fracdgfid$ are independent parameters in our models.
%The carbon nucleus density is fixed to $\nC=132~\pcc$, indicating that 
%$\nH$ decreases with increasing $Z$.
%
%
%\subsubsection{Gas temperatures}\label{sec:planeTZ}
%
%%First, the $Z$-dependence of the gas temperatures is investigated.
%%As a reference, we focus on the A5V cases with $\fracdg=\fracdgfid$
%%As a reference, we focus on the {\color{blue} strong-FUV models} with $\fracdg=\fracdgfid$
%%to see how the gas temperature varies with $Z$.
%Figure \ref{fig:planeT} shows the gas temperatures as a function of $\NC$ 
%at the metallicities from $Z=1$ to $10^3$. 
%The gas temperatures increase with $Z$.
%
%This can be understood by considering the dominant heating and cooling processes.
%The dominant heating process is photo-electric heating from dust grains, which 
%is independent of $\nH$ at a fixed $\nC$ because 
%its heating rate is proportional to the dust number density, which 
%is constant at a fixed $\nC$. 
%%The dominant heating processes are electron-recombination followed by collisional 
%%de-excitation and photo-electric heating from dust grains. 
%%{\color{blue}
%%We should note that the electron-recombination works as a heating 
%%process when an electron is recombined to an excited energy level and is
%%then deexcited by collisions.
%%In the Meudon PDR code, 
%%the de-excitation is assumed to occur only through collisions instead of computing the full recombination spectrum of ions \citep{LeP2006}.
%%The heating rate may be thus overestimated.
%%%In order to check the effect of the assumption, we show the results assuming that 
%%%all the de-excitation occurs through spontaneous emissions.
%%}
%%The heating rates of these two processes are both independent of $\nH$
%%at a fixed $\nC$ for the reasons given below.
%%The heating rate is independent of $\nH$ at a fixed $\nC$ for the reasons given below.
%%The recombination reaction $\mathrm{C}^+ + e\rightarrow \mathrm{C}$
%%is the most important process to heat the gas. 
%%Its rate is proportional to $\nCp n(e^-)$, which 
%%are constant at a fixed $\nC$ because $\nCp$ and $n(e^-)$ are independent of $\nH$ 
%%as described in Appendix \ref{sec:CI_CII}.
%By contrast, the cooling rate provided by the O fine-structure transitions 
%($^3$P$_2$, $^3$P$_1$, $^3$P$_0$) depends on $\nH$ because  
%oxygen atoms are excited mainly by collision with hydrogen atoms.
%A decrease in $\nH$ reduces the cooling rate if $T$ is fixed.
%In order for the cooling rate to be balanced with the heating rate, 
%the gas temperatures need to increase monotonically as $Z$ increases
%as shown in Figure \ref{fig:planeT}.
%
%%\deleted{
%%The gas temperatures drop when 
%%CO is shielded and most carbon nuclei become CO as seen in Figure \ref{fig:planeCO}a.
%%There are two reasons for this.
%%Firstly, the electron-recombination heating rate decreases 
%%owing to a lack of C$^+$.
%%Secondly, the photo-electric heating rate also decreases because 
%%a decline of $n(e^-)$ enhances grain charges.
%%}
%
%
%\begin{figure}[htpb]
%        \centering
%%        \includegraphics[width=8.5cm]{Figure4.eps}
%        \includegraphics[width=8.5cm]{plane-parallel-T-nCconst-A5V.eps}
%\caption{
%      Profiles of the gas temperatures as a function of $\NC$
%      for various metallicities,
%      $Z=1$ (red), $Z=10$ (green), $Z=10^2$ (blue), and $Z=10^3$ (magenta).
%      The carbon nucleus density is fixed to $\nC=1.32\times 10^2~\pcc$.
%      The dust-to-gas mass ratio is fixed to $\fracdgfid$.
%%      The dust-to-gas mass ratio and 
%%      the spectral type of the central star are fixed to be $\fracdgfid$ and A5V, respectively.
%}
%   \label{fig:planeT}
%\end{figure}
%
%%----------------------------------------
%\subsubsection{Molecular Hydrogen}\label{sec:planeH2Z}
%%----------------------------------------
%
%Figure \ref{fig:planeH2} shows that 
%the H$_2$ abundances at the low $\NC$ limit are almost independent of $Z$ 
%{\color{blue} for weak-FUV } with $\fracdg=\fracdgfid$. 
%As shown in Section \ref{sec:planeH2}, 
%H$_2$ mainly forms through the gas-phase chemistry ($\mathrm{CH^+ + H\rightarrow H_2 + C^+}$) and dust-surface chemistry.
%The H$_2$ fraction is determined by 
%\begin{equation}
%    k_\mathrm{ch+,h} n(\mathrm{CH^+})n(\mathrm{H}) + F_\mathrm{H_2,dust} = \alpha_\mathrm{H_2} \nHm,
%    \label{H2formreact}
%\end{equation}
%where $k_\mathrm{ch+,h}$ is the H$_2$ formation rate and $\alpha_\mathrm{H_2}$ is the photo-dissociation rate of H$_2$.
%In order to examine the contributions from the two main H$_2$ formation processes, 
%we also plot the H$_2$ abundances in models without H$_2$ formation on grain surfaces in Figure \ref{fig:planeH2}.
%
%Figure \ref{fig:planeH2} shows that the contribution from the gas-phase reaction (the dust-surface reaction)
%decreases (increases) with increasing $Z$ at a fixed $\nC$, {\color{blue} indicating that 
%    $\nH$ decreases with increasing $Z$.
%}
%This behavior can be understood as follows:
%Using the fact that $F_\mathrm{H_2,dust}$ is expressed as $k_\mathrm{dust}(T) n_\mathrm{C}n(\mathrm{H})$, where 
%$k_\mathrm{dust}(T)$ is a function of the gas temperature (Equation (\ref{RER})), 
%the H$_2$ fraction is given by 
%\begin{equation}
%    \frac{\nHm}{n(\mathrm{H})} = \frac{k_\mathrm{ch+,h} \nCHp + k_\mathrm{dust}(T)\nC}{\alpha_\mathrm{H_2}}. 
%    \label{H2frac}
%\end{equation}
%For $Z\le 10$, $\nCHp$ does not depend on $Z$ at a fixed $\nC$
%because it is determined by the balance between $\mathrm{C}^++\mathrm{H}\rightarrow \mathrm{CH^+}$ 
%and $\mathrm{CH^+} + \mathrm{H} \rightarrow \mathrm{H_2}+\mathrm{C^+}$.
%As $Z$ increases from $Z\sim 10$, $\nCHp$ decreases because $\mathrm{CH^+}+e^-\rightarrow \mathrm{C}+\mathrm{H}$ dominates over 
%$\mathrm{CH^+} + \mathrm{H} \rightarrow \mathrm{H_2}+\mathrm{C^+}$. 
%By contrast, as $Z$ increases, the contribution from the grain-surface chemistry in Equation (\ref{H2frac}) increases
%because the gas temperature increases with $Z$ as shown in Figure \ref{fig:planeT} and 
%$k_\mathrm{dust}\propto \kappa(T)$ increases exponentially with $T$ for $T\lesssim 300~\mathrm{K}$ (Equation (\ref{kappa})).
%The apparent $Z$-independence of the H$_2$ fraction at the low $\NC$ limit is coincidental.
%
%%Figure \ref{fig:planeH2} shows that 
%%the H$_2$ abundances at the low $\NC$ limit increase with $Z$ 
%%for the A5V cases with $\fracdg=\fracdgfid$. 
%%The cases with different $\chico$ and $\fracdg$ also show a similar trend.
%%In order to understand it, 
%%We can see that 
%%the dust~surface chemistry mainly forms H$_2$ at $Z\ge 10$ 
%%for $\fracdg=\fracdgfid$.
%
%\begin{figure}[htpb]
%        \centering
%%        \includegraphics[width=8.5cm]{Figure5.eps}
%        \includegraphics[width=8.5cm]{plane-parallel-H2-nCconst-A5V.eps}
%\caption{
%      The same as Figure \ref{fig:planeT} but for the H$_2$ abundance.
%      The thin lines indicate the results without H$_2$ formation on grain surfaces, 
%      and their colors indicate the metallicities.
%      The horizontal black dashed lines indicate $2\nHm/\nH=1$.
%}
%   \label{fig:planeH2}
%\end{figure}
%
%
%%The positive dependence of $\nHm/\nH$ on $Z$ comes from the fact that 
%%$F_\mathrm{H_2,dust}$ increases with $T$ for $T<T_\mathrm{chem}=300~\mathrm{K}$ 
%%(Equation (\ref{RER}) and Figure \ref{fig:planeT}).
%%The positive dependence of $\nHm/\nH$ on $Z$ shown in Figure \ref{fig:planeH2} 
%%can be explained as follows.
%%Considering the chemical balance between 
%%H$_2$ formation on grain surfaces and H$_2$ photo-dissociation 
%%(the rate coefficient $\alpha_\mathrm{H_2}$), one obtains 
%%\begin{equation}
%%    \frac{\nHm}{\nH} = \frac{F_\mathrm{H_2,dust}}{\alpha_\mathrm{H_2}\nH} \propto \kappa(T) \nC,
%%    \label{H2Z}
%%\end{equation}
%%where we use Equation (\ref{RER}) and $n(\mathrm{H}) \sim \nH$.
%%Equation (\ref{H2Z}) clearly shows that the positive dependence of $\nHm/\nH$ on $Z$ comes from 
%%the fact that the gas temperatures increase with $Z$ (Figure \ref{fig:planeT}) since
%%$\nC$ is fixed.
%
%%The H$_2$ formation rate per hydrogen atom $F_\mathrm{H_2,dust}/n(\mathrm{H})$, 
%%which is proportional to $\nHm/\nH$ because the H$_2$ formation is balanced with 
%%H$_2$ photo-dissociation proportional to $\nHm$, 
%%increases in proportion to $Z\nH$. 
%%A combination of $Z\nH$ ($\propto \nC$) in Equation (\ref{RER}) does not change since 
%%$\nC$ is fixed.
%
%%\deleted{
%%The critical $\NC$ beyond which H$_2$ is self-shielded increases with $Z$ although 
%%$\nHm/\nH$ increases with $Z$. This is because $\nHm$ itself decreases with $Z$.
%%Self-shielding of H$_2$ works in deeper interior for larger $Z$.
%%}
%
%
%%-------------------------------------------------------------
%\subsubsection{Carbon Monoxide, CO}\label{sec:planeCOZ}
%%-------------------------------------------------------------
%
%
%Figures \ref{fig:planeCO} show that the CO fractions
%decrease with increasing $Z$ at fixed $\fracdg/\fracdgfid$ and $\nC$.
%This behavior was pointed out by \citet{Higuchi2017}; hydrogen is involved in CO formation. 
%
%%\deleted{
%%In order to examine how the CO abundance depends on the H$_2$ abundance, 
%%we over-plot $\nCO/\nC$ obtained from the PDR calculations without H$_2$ formation 
%%on grain surfaces in Figure \ref{fig:planeCO}.
%%The dust-surface chemistry yields a large amount of H$_2$ for most cases 
%%(Figures \ref{fig:plane}c, \ref{fig:plane}d, and \ref{fig:planeH2}). 
%%If the CO fraction depends on the H$_2$ abundance, 
%%it should vary between models with and without the dust-surface chemistry.
%%The models with $\fracdg=\fracdgfid$ exhibit that 
%%the CO abundances remain unchanged whether H$_2$ 
%%formation on grain surfaces is considered or not.
%%This clearly shows that CO formation proceeds without H$_2$.
%%}
%%\deleted{
%%The $\fracdg=10\fracdgfid$ cases produce H$_2$ enough to affect CO formation
%%since H$_2$ formation on dust grains become efficient for larger $Z$ and $\fracdg$. 
%%The lower metallicities with $Z=1$ and $10$ however do not show clear differences between 
%%the results with and without the dust-surface chemistry although there are 
%%wiggles in $\nCO/\nC$ around the HI-H$_2$ transitions as seen in Figures \ref{fig:plane}e.
%%For high metallicties with $Z=10^2$ and $10^3$, 
%%efficient H$_2$ formation on grain surfaces yields CO more than 
%%obtained without the dust-surface chemistry at the low $\NC$ limit.
%%}
%%\deleted{
%%In deep interiors $\NC>10^{16}~\psc$, the C$^0$/CO transitions appear owing to shielding effects.
%%For low metallicites with $Z=1$ and $Z=10$, 
%%CO is self-shielded at higher $\NC$ for $Z=10$ than for $Z=1$ 
%%since the CO fractions at the low $\NC$ limit decrease with increasing $Z$.
%%For metallicities with $Z\ge 10^2$, 
%%CO self-shielding is not the cause of a rapid increase in $\nCO$ around $\NC\sim 10^{16-17}~\psc$
%%in Figures \ref{fig:planeCO}a and \ref{fig:planeCO}b because 
%%the corresponding CO column density is too small.
%%In these cases, C$^0$ attenuation is attributed to the C$^0$/CO transition. 
%%A detailed discussion for shielding effects is described in Section \ref{sec:shielding}.
%%}
%
%\begin{figure}[htpb]
%        \centering
%        \includegraphics[width=8.5cm]{plane-parallel-CO-nCconst-A5V.eps}
%%        \includegraphics[width=8.5cm]{Figure6.eps}
%\caption{
% The same as Figure \ref{fig:planeH2} but for the CO abundance at 
% (a) $\fracdg=\fracdgfid$ and (b) $10\fracdgfid$.
%}
%   \label{fig:planeCO}
%\end{figure}
%
%Figure \ref{fig:planeCO}a shows that 
%at $\fracdg=\fracdgfid$ there is almost no difference in the CO fractions between the models 
%with and without the H$_2$ formation on grains surfaces while
%$\nHm/\nH$ differs by {\color{blue} over one order of magnitude} for $Z=10^3$ (Figure \ref{fig:planeH2}).
%%$\nHm/\nH$ differs by several orders of magnitude for $Z=10^3$ (Figure \ref{fig:planeH2}).
%This suggests that the CO formation proceeds regardless of the amount of H$_2$.
%As will be shown in Figure \ref{fig:COnetwork}, 
%H$_2$ is not involved in the most important chemical networks to form CO.
%
%%When $\fracdg$ is increased from $\fracdgfid$ to $10\fracdgfid$,
%%a sufficient amount of H$_2$ is formed on the grain surface to enhance the CO formation
%%(Figure \ref{fig:planeCO}b).
%%Without the H$_2$ formation on grain surfaces, the CO fractions stay in a low level and 
%%are almost consistent with 
%%those with $\fracdg=\fracdgfid$.
%
%%A detailed discussion for shielding effects is described in Section \ref{sec:shielding}.
%
%%%In deep interiors $\NC>10^{16}~\psc$, the C$^0$/CO transitions appear owing to shielding effects.
%%the CO fractions at the low $\NC$ limit are too small to make the self-shielding effective.
%%%For instance, CO self-shielding is not the cause of a rapid increase in $\nCO$ 
%%%around $\NC\sim 10^{16-17}~\psc$ for $Z=10^3$ in Figure \ref{fig:planeCO}a because the 
%%%corresponding CO column density $\sim 10^{10-11}~\psc$ is too small.
%%%In these cases, CO is shielded by C$^0$. 
%
%


%---------------------------------------------------------------------------
%\section{Chemical Properties at Optically-thin Limit }\label{sec:thin}
%\section{The Carbon Monoxide Fractions at Optically-thin Limit }\label{sec:thin}
\subsection{CO Formation in H$_2$-poor Environments}\label{sec:analyticmodel}
%---------------------------------------------------------------------------

%In Sections \ref{sec:plane} and \ref{sec:planeZ}, we have investigated 
%the chemical and thermal properties of the gas slabs with $\nC=132~\pcc$
%for various $Z$, $\fracdg$, and $\chico$.
%In this section, we change the remaining parameter, $\nC$.
%
%We focus on the optically-thin, low $\NC$ limit, where 
%the parameter-dependence of the chemical and thermal properties 
%is rather simple. 
%%$\nHm$ and $\nCO$ derived in the low $\NC$ limit is applicable as long as the shielding effects
%%are not effective.
%We note that the $\fracdg = 10\fracdgfid$ cases show small variations in the $\nCO$ profiles 
%where $\nHm/\nH$ increases rapidly owing to H$_2$ self-shielding.
%The variations are however within a factor of $\sim5$ in most cases
%(Figure \ref{fig:planeCO}b).
%In this section, the results at $\tau=10^{-5}$ are shown.
%
%
%%\deleted{
%%The main purposes of this section are to find an analytic formula of the CO fraction
%%when H$_2$ is not involved in CO formation, and to investigate how 
%%}
%
%%\deleted{
%%In order to achieve the purposes, this section is organized as follows:
%%In Section \ref{sec:CI_CII}, we investigated the dominant C-bearing species, 
%%C$^0$ and C$^+$, both of which are important in CO formation.
%%}
%
%%%----------------------------------------------------------------------
%%\subsection{Neutral Atomic Carbon and Ionized Carbon}\label{sec:CI_CII}
%%%----------------------------------------------------------------------
%%
%%Before investigating the CO fractions, we focus on 
%%the main C-bearing species, C$^0$ and C$^+$, both of which 
%%are important in CO formation.
%%For most situations, 
%%the relation $n(\mathrm{C^0})+n(\mathrm{C^+})\sim \nC$ is satisfied
%%because the CO fraction is negligible. 
%%
%%\begin{figure*}[htpb]
%%        \centering
%%        \includegraphics[width=14cm]{Figure7.eps}
%%\caption{
%%    C$^0$ fractions at $\tau=10^{-5}$ 
%%    as a function of $\nC$ at the low $\NC$ limit for (a) $\fracdg=\fracdgfid$ 
%%and (b) $10\fracdgfid$.
%%The red and blue lines indicate the results for the A5V and A1V cases, respectively.
%%The difference in the linetypes indicates the difference of the metallicities, 
%%$Z=1$ (solid), $10$ (dashed), $10^2$ (dotted), and $10^3$ (dot-dashed).
%%For each spectral type,
%%the black solid and dashed lines in the top panels correspond to 
%%the analytic estimations given by Equation (\ref{ionCana}) at $Z=1$ and $Z=10^3$, 
%%respectively.
%%The gas temperatures substituted into Equation (\ref{ionCana}) use a fitting function of $T=a \nC^b$, 
%%where $a$ and $b$ are the fitting parameters.
%%The ratios of $n(e)$ to $\nCp$ as a function of $\nC$ for (c) $\fracdg=\fracdgfid$ 
%%and (d) $10\fracdgfid$.
%%}
%%        \label{fig:CI_CII}
%%\end{figure*}
%%
%%
%%Figures \ref{fig:CI_CII}a and \ref{fig:CI_CII}b show the C$^0$ fractions at the low $\NC$ limit 
%%as a function of $\nC$ for $\fracdg=\fracdgfid$ and $10\fracdgfid$, respectively.
%%First, we focus on the results with $Z=1$.
%%Both the spectral types show a similar $\nC$-dependence of the C$^0$ fraction.
%%For larger $\nC$, radiative recombination of C$^+$ becomes more efficient,
%%leading to a higher fraction of C$^0$ in the equilibrium state.
%%Comparing Figures \ref{fig:CI_CII}a and \ref{fig:CI_CII}b shows that 
%%there is almost no difference between the C$^0$ fractions with 
%%$\fracdg=\fracdgfid$ and $10\fracdg$ at $Z=1$ for each spectral type.
%%At a fixed $\nC$, the C$^0$ fraction
%%is lower for the A1V case than for the A5V case because the A1V star emits
%%stronger ionizing FUV flux (Figure \ref{fig:spec}).
%%
%%To examine the $Z$-dependence of $n(\mathrm{C^0})/\nC$, 
%%the C$^0$ fractions at a range of metallicities $Z=1-10^3$
%%are plotted in the top panels of Figure \ref{fig:CI_CII}.
%%The C$^0$ fractions do not depend on $Z$ significantly for the A5V case although 
%%$n(\mathrm{C^0})/\nC$ gradually decrease with increasing $Z$ at a fixed $\nC$ 
%%for the A1V case especially with $\fracdg = 10\fracdgfid$.
%%
%%The parameter-dependence 
%%of the C fraction is explained by the following 
%%simple analytic estimate given by \citet{Kamp2000}.
%%The C$^0$ fraction is roughly determined by
%%the balance between the photo-ionization rate of C$^0$ and radiative recombination rate of C$^+$ as 
%%follows:
%%\begin{equation}
%%        \alpha_\mathrm{C} \chico F_\mathrm{H} n(\mathrm{C^0})
%%        \sim  k_\mathrm{rec} n(\mathrm{C^+}) n({e}^{-}),
%%\label{ionC}
%%\end{equation}
%%where 
%%$\alpha_\mathrm{C}=1.6\times 10^{-17}~\mathrm{cm}^2$ 
%%\footnote{
%%    Note that the Meudon code takes into account the frequency-dependent radiative flux 
%%    to calculate the photo-ionization rate although $\alpha_\mathrm{C}\chico F_\mathrm{F}$ 
%%    gives a good estimate.
%%}
%%is the photo-ionization cross section
%%\citep{vanDishoeck2006}
%%and $k_\mathrm{rec}(T)=1.7\times 10^{-11}~\mathrm{cm^3~s^{-1}}(T/100~\mathrm{K})^{-0.82}$
%%\footnote{
%%The expression of $k_\mathrm{rec}$ is a fitting formula
%%of the recombination coefficient taking into account 
%%radiative recombination \citep{Badnell2006} and 
%%dielectric recombination \citep{Badnell2003}. 
%%in 
%%$10~\mathrm{K}\le T\le 10^3~\mathrm{K}$.
%%}
%%is the recombination coefficient.
%%Figures \ref{fig:CI_CII}c and \ref{fig:CI_CII}d show that $n({e}^-)$ 
%%is approximately equal to $n(\mathrm{C^+})$ for most cases. 
%%For the A5V cases with $\nC>10^2~\pcc$, however, $n(e^-)$ is larger than $\nCp$. 
%%In such a high density, the abundance of C$^+$ 
%%becomes smaller than that of Si$^+$, and electrons are mainly supplied by 
%%photo-ionization of Si, which has a lower ionization energy of 8~eV than carbon atom.
%%With $n(e^-) \sim \nCp$,
%%Equation (\ref{ionC}) is rewritten as
%%\begin{equation}
%%    \frac{n(\mathrm{C^0})}{\nC} \sim f_\mathrm{c}(\xi) \equiv 1- \frac{1}{2\xi} \left[- 1+\sqrt{1+4\xi} \right],
%%\label{ionCana}
%%\end{equation}
%%where $\nC \sim n(\mathrm{C^0}) + n(\mathrm{C}^+)$ is used
%%and $\xi$ is defined as 
%%\begin{equation}
%%   \xi \equiv \frac{k_\mathrm{rec}n_\mathrm{C}}
%%   {\alpha_\mathrm{C} \chico F_\mathrm{H}}
%%   = \chico^{-1}
%%   \left(\frac{\nC}{11~\mathrm{cm}^{-3}}\right)
%%    \left( \frac{T}{100~\mathrm{K}} \right)^{-0.82},
%%     \label{xi1}
%%\end{equation}
%%which indicates the ratio of the recombination to the ionization coefficients.
%%We here show the asymptotic behavior of Equation (\ref{ionCana}) at $\nC\rightarrow 0$ or $\xi\ll 1$.
%%Using the Taylor expansion $\sqrt{1+4\xi} = 1+2\xi - \xi^2/2 + O(\xi^3)$, 
%%one obtains $n(\mathrm{C}^0)/\nC \sim \xi/4\propto \nC/\chico$ for $\xi\ll1$.
%%As $\nC$ increases, the C$^0$ fraction increases in proportion to $\nC$, and approaches 
%%unity. For $Z=1$, 
%%the analytic predictions are consistent with the results of the PDR calculations 
%%(Figures \ref{fig:CI_CII}a and \ref{fig:CI_CII}b).
%%
%%The weak dependence of the C$^0$ fraction on $Z$ and $\fracdg$ seen 
%%in the top panels of Figure \ref{fig:CI_CII} 
%%comes from the fact that hydrogen and dust grains are not involved in 
%%the chemical balance shown in Equation (\ref{ionC}) \citep{Kamp2000}.
%%We note that the C$^0$ fraction gradually decreases with increasing $Z$ especially for 
%%the A1V cases because $T$ increases with $Z$ as shown in Figure \ref{fig:planeT}.
%%An increase in $T$ makes the radiative 
%%recombination of C$^+$ inefficient, resulting in lower C$^0$ fractions.
%%In order to confirm this, we plot the predictions from Equation (\ref{ionCana}) with 
%%a fitting function of the gas temperatures at 
%%$Z=10^3$ by the black dashed lines in the top panels of Figure \ref{fig:CI_CII}.
%%One can see that the $Z$-dependence of $n(\mathrm{C^0})/\nC$ is explained by the 
%%change of gas temperature.
%%   
%%In order to quantify the parameter-dependence of the C$^0$ fraction,
%%we derive the critical $\nC$ above which more than half of 
%%carbon nuclei becomes neutral carbon atoms.
%%This is achieved when $\xi=2$.
%%If $\xi\gg 2$, or
%%\begin{equation}
%%    \nC \gg \nCcri = 
%%    5.5~\mathrm{cm}^{-3}~ \chico \left( \frac{T}{100~\mathrm{K}} \right)^{0.82},
%%  \label{nCcri}
%%\end{equation}
%%the C$^+$ fraction is negligible.
%%It is useful to describe the $Z$-dependence of the C$^0$ fraction at a fixed $\nH$.
%%The behavior of the C$^0$ fraction is characterized by the critical metallicity 
%%given by 
%%\begin{equation}
%%Z_\mathrm{cri} = \frac{\nCcri}{\nH \Ac} = 
%%\left( \frac{\nH}{4.2\times 10^4~\pcc} \right)^{-1} \chico \left( \frac{T}{100~\mathrm{K}} \right)^{0.82}.
%%\end{equation}
%%If $Z$ increases from $Z=1$ at a fixed $\nH$, the C$^0$ fraction increases in proportion to $Z$ 
%%as long as $Z\ll Z_\mathrm{cri}$.
%%Once $Z$ exceeds $Z_\mathrm{cri}$, the C$^0$ fraction approaches unity. 
%%
%%----------------------------------------------------------
%%\subsection{Carbon Monoxide, CO}\label{sec:CO}
%%----------------------------------------------------------
%
%As mentioned in Section \ref{sec:planeCOZ}, as long as 
%the amount of H$_2$ is negligible, CO formation proceeds without H$_2$.
%When a sufficient amount of H$_2$ is provided by increasing $Z$ and $\fracdg$, 
%H$_2$ accelerates CO formation significantly.
%
%We construct an analytical model for CO formation 
%in the H$_2$-free gas in Section \ref{sec:COana}, 
%and compare our results of the PDR calculations with 
%the predictions from the analytical model in Section \ref{sec:COcomp}.
%We also investigate how H$_2$ promotes CO formation for larger $\fracdg$ and $Z$ in 
%Section \ref{sec:CO_H2}.
%
%


In Section \ref{sec:result}, we show that the CO formation is independent of the amount of H$_2$ 
when H$_2$ formation is inefficient in 
situations where $T_\mathrm{chem}$ is high and/or 
$\fracdg$ is small.
In this section, we 
investigate how CO forms in such 
H$_2$-poor environment, and 
develop an analytic formula for the CO fraction that is 
applicable in a wide range of $\NC$ 
in Sections \ref{sec:COana} and \ref{sec:shielding}.

%In this section, we will construct an analytical model for the CO fraction
%on the basis of the chemical network that is extracted from the full chemical network considered in the Meudon PDR code.

%--------------------------------------------------------
\subsubsection{Dependence of CO Fractions on Gas Metallicity}\label{sec:COZ}
%--------------------------------------------------------


Figure \ref{fig:COshield} shows the spatial distributions 
of the CO fractions for various $\chi$, $\nC$, and $Z$ listed in Table \ref{tab:param}.
The dust-to-gas mass ratios are fixed to 
$\fracdg=0.1\fracdgfid$ for weak-FUV and 
$\fracdg=0.01\fracdgfid$ for strong-FUV, where 
the CO formation proceeds, regardless of H$_2$ (Figure \ref{fig:plane}).

We first focus on the CO fractions for $\NC\lesssim 10^{17}~\psc$.
For given $\nC$ and the FUV fluxes ($\chico,\chioh$), 
the CO fractions decrease with increasing $Z$.
Since $Z$ is changed keeping $\nC$ constant, $\nH$ decreases with increasing $Z$. 
The CO fractions thus decrease with decreasing $\nH$.
This behavior has been pointed out by \citet{Higuchi2017}; 
hydrogen is involved in CO formation. 

Comparing between the panels of Figure \ref{fig:COshield}, 
one can see that the CO fractions increase as $\nC$ increases and/or $\chico$ decreases.
This behavior was qualitatively expected because the CO formation rate will
increase with increasing $\nC$ and the CO destruction rate will decrease with decreasing $\chico$.

For all the models shown in Figure \ref{fig:COshield}, 
the CO fractions begin to increase in deep interiors
owing to shielding effects, 
which will be investigated in Section \ref{sec:shielding}.



\begin{figure*}[htpb]
\centering
\includegraphics[width=15.0cm]{Fig3.pdf}
\caption{
            The CO fractions as a function of $N_\mathrm{C}$ for 
            (a) weak-FUV ($\nC=1.32\times 10^2~\pcc$), 
            (b) weak-FUV ($\nC=1.32\times 10^3~\pcc$),
            (c) strong-FUV ($\nC=1.32\times 10^3~\pcc$), and  
            (d) strong-FUV ($\nC=1.32\times 10^4~\pcc$).
            The difference in the gas metallicites is shown by the red solid ($Z=1$), green dashed
            ($Z=10$), blue dotted ($Z=10^2$), and 
            magenta dot-dashed ($Z=10^3$) lines.
           The horizontal black dashed lines indicate
           $(\nCO/\nC)_\mathrm{cri}$ (Equation (\ref{COshield_cri})).
           The thick lines with lighter colors show $\nCOana$ as 
           a function of $N_\mathrm{C}$
           taking into account both the C$^0$ and CO shielding effects.
           The dashed lines with lighter colors 
           correspond to $\nCOana$ 
           without the CO self-shielding effect.
%         Profiles of the CO fractions as a function of $N(\mathrm{C}^0)$ for the (a) A5V and (b) A1V 
%         cases with $\fracdg=\fracdgfid$.
%         For each panel, we show the results with 
%         the $4\times 4$ combinations of the parameters of  
%         $\nC= (1.32, 1.32\times 10, 1.32\times 10^2, 1.32\times 10^3)~\mathrm{cm^{-3}}$ and 
%         $Z=(1,10,10^2,10^3)$. 
}
\label{fig:COshield}
\end{figure*}

%--------------------------------------------------------
\subsubsection{An Analytical Formula for the CO Fraction}\label{sec:COana}
%--------------------------------------------------------

In this section, we develop an analytic formula 
for the CO fractions from
the most important chemical reactions associated with CO formation 
of the H$_2$-free gas in Figure \ref{fig:COnetwork}.
There are three main paths to form CO.
At the high density limit (Equation (\ref{nHreq}))
, the CO formation is intermediated mainly by OH  ({\Pathoh}). 
For lower densities, the dominant CO formation path depends on $Z$.
At $Z\le 10$, the CO formation is intermediated by CH$^+$ (\Pathchp).
For higher $Z$, the CO formation proceeds without hydrogen as shown by the green arrows
in Figure \ref{fig:COnetwork} (\Pathwoh).

%The figure shows that O-bearing species mainly contribute to CO formation.
%Why are C-bearing species not involved in CO formation although 
%the rate coefficient of $\mathrm{C^+ + H \rightarrow CH^+ + h\nu}$ is roughly 20 times larger than 
%that of $\mathrm{O + H \rightarrow OH + h\nu}$?
%This comes from the difference in the abundances of the reaction partner in 
%the destruction reactions.
%One of the important destruction processes of CH$^+$ is the reaction with H 
%$(\mathrm{CH^+ + H \rightarrow H_2 + C^+})$
%while that of OH is the reaction with C$^+$ 
%$(\mathrm{OH+C^+\rightarrow CO + H^+})$.
%Since the H abundance is much larger than the C$^+$ abundance, CH$^+$ 
%is destructed efficiently so that it is not involved in CO formation.

\begin{figure}[htpb]
        \centering
        \includegraphics[width=9cm]{Fig4.pdf}
\caption{
     The most important reactions to form CO in the H$_2$-free gas.
     The red and blue arrows show the CO formation paths intermediated by CH$^+$ and OH, respectively.
     The green arrows correspond to the CO formation paths not associated with hydrogen.
     The photo-dissociation of OH and CO and photo-ionization of C are shown by the dashed arrows.
     The chemical reaction rate coefficients are summarized in Table \ref{tab:reaction}.
}
\label{fig:COnetwork}
\end{figure}

%On the basis of the extracted chemical network illustrated in Figure \ref{fig:COnetwork},
%we derive an analytic formula of the CO fraction as a function of $\nH$, $Z$, 
%$\chico$, $\chioh$.
%The detailed derivation is presented in Appendix \ref{app:COana}.

The analytic formula taken into account 
all the chemical reactions shown in Figure \ref{fig:COnetwork} is complicated and is not useful.
In Appendix \ref{app:COana}, we construct a fitting formula of the CO fraction 
based on {\Pathoh}, which is dominated for higher densities,
as follows:
\begin{equation}
    \frac{\nCOana}{\nC} = \frac{\Ao}{ {\cal A}_\mathrm{O,ism}} 
    \left(
        10^{-14} \eta^{1.8} + 6.0\times 10^{-11} \eta 
    \right),
   \label{fitting}
\end{equation}
where 
\begin{equation}
    \eta = \nH Z^{0.4}\chi^{-1.1},
    \label{eta}
\end{equation}
$\chi \equiv \sqrt{\chico \chioh}$, 
$\Ao$ is the relative oxygen abundance with respect to hydrogen 
in the gas phase, and 
${\cal A}_\mathrm{O,ism}=3.2\times 10^{-4}Z$ corresponds to 
$\Ao$ shown in Table \ref{tab:abundance}.



We should note that Equation (\ref{fitting}) is based on the chemical reactions in regions 
where shielding effects are not important. 
However, 
by considering the shielding effect in $\chi$,
we will show Equation (\ref{fitting}) 
is applicable also in regions where shielding effects
are effective in Section \ref{sec:shielding}.


%In the high density limit, the path 
%$\mathrm{O}\rightarrow \mathrm{OH}\rightarrow \mathrm{CO}$ dominates over 
%$\mathrm{O}\rightarrow \mathrm{OH}\rightarrow \mathrm{CO^+} \rightarrow \mathrm{CO}$, and 
%the CO fraction can be easily derived analytically as follows.
%The OH abundance is determined by the balance between the radiative association 
%$\mathrm{O+H\rightarrow OH+h\nu}$ 
%(the rate coefficient $\koh$) and the OH photo-dissociation (the rate coefficient $\aOH$).
%Eventually, OH is combined with C to form CO with a rate coefficient of $\kohc$.
%Considering the CO photo-dissociation (the rate coefficient $\aCO$), 
%the analytical form of the CO fraction is given by 
%\begin{eqnarray}
%    \frac{n(\mathrm{CO})}{\nC} &=& \frac{\koh\kohc}{\aCO\aOH}\nO\nH \nonumber \\
%&=& 7.4\times 10^{-17}~T_{300}^{-0.38}
%\frac{\Ao}{ {\cal A}_\mathrm{O,ism}} \left( \frac{\nH Z^{0.5} }{\chi} \right)^2,
%\label{COana0}
%\end{eqnarray}
%where $T_{300}=T/300~\mathrm{K}$, ${\cal A}_\mathrm{O}$ is the oxygen elemental abundance 
%at $Z=1$, ${\cal A}_\mathrm{O,ism} = 3.2\times 10^{-4}$ 
%(Table \ref{tab:abundance}),  
%and we use the fact that $\aCO\propto \chico$ and $\aOH \propto \chioh$.
%Equation (\ref{COana0}) shows that 
%the CO fraction depends on $\nH Z^{0.5}\chi^{-1}$, which is quite similar to $\eta$.
%The difference between $\eta$ and $Z^{0.5}\nH \chi^{-1}$ comes from the 
%temperature dependence shown in Equation (\ref{COana0}).


%\begin{figure*}[htpb]
%  \centering
%  \includegraphics[width=18.0cm]{Figure8.eps}
%\caption{
%CO fractions at the small $\NC$ limit ($\tau=10^{-5}$) as a function of $Z^{0.4}\nH$ for 
%metallicities $Z=1$ (red), $10$ (green), $10^2$ (blue), and $10^3$ (magenta). 
%Panels correspond to the $2\times 3$ combinations of (A5V and A1V) and 
%$\fracdg=(0.1\fracdgfid,\fracdgfid, 10\fracdgfid)$. 
%In each panel, the gray line correspond to the fitting formula 
%shown in Equation (\ref{fitting}).
%}
%        \label{fig:COfit}
%\end{figure*}
%
%
%
%%------------------------------------------------------------------
%\subsection{Comparison 
%of the CO Fractions with Those Predicted from the Analytical Model}\label{sec:COcomp}
%%------------------------------------------------------------------
%
%Figures \ref{fig:COfit}a and \ref{fig:COfit}b compare the CO fractions at the small $\NC$ limit 
%with those predicted from the fitting formula (Equation \ref{fitting}) as a function of 
%$Z^{0.4}\nH$ for the A5V  and A1V cases, respectively.
%For $\fracdg=0.1\fracdgfid$, the CO fractions are consistent 
%with those predicted from the fitting formula, indicating that H$_2$ does not play 
%an important role in the CO formation.
%
%We found that the predictions from Equation (\ref{fitting}) overestimate the CO fractions 
%at $Z^{0.4}\nH< 10^4~\pcc$ for $\fracdg=0.1\fracdgfid$ in Figures \ref{fig:COfit}a and \ref{fig:COfit}b.
%This is caused by the photo-dissociation of CH$^+$, which 
%is not included in the extracted chemical network shown in Figure \ref{fig:COnetwork} for simplicity.
%This is because this study is not interested in the density range where the CO fractions are 
%extremely low $\nCO/\nC< 10^{-8}$.
%
%As was observed in Section \ref{sec:planeCOZ},
%%Figures \ref{fig:COfit}c and \ref{fig:COfit}d show that 
%Figure \ref{fig:COfit} shows that 
%increasing $\fracdg$ from $0.1\fracdgfid$ to $10\fracdgfid$ enhances the CO fractions more significantly 
%than predicted from the fitting formula at higher $\nH$ and $Z$.
%The acceleration of the CO formation reflects efficient H$_2$ formation. 
%
%There is a critical value of $\fracdg$ between $0.1\fracdgfid$ and $10\fracdgfid$, above which the CO fractions deviate from the predictions.
%The critical $\fracdg$ depends on the model parameters, FUV fluxes, $Z$, and $\nH$.
%For A5V, the CO fractions significantly deviate from the predictions only for the highest density at $Z=10^3$ (Figure \ref{fig:COfit}c).
%Figure \ref{fig:COfit}d shows that most models have such large deviations for A1V.
%
%In summary, the predictions from Equation (\ref{fitting}) are valid when $\fracdg=0.1\fracdgfid$ and $Z^{0.4}\nH>10^4~\pcc$.
%For models with smaller FUV fluxes, such as A5V ($\chi\sim 1$), Equation (\ref{fitting}) is valid even with $\fracdg=\fracdgfid$.
%
%
%
%
%
%%--------------------------------------------------------------
%\subsection{
%Accelerated CO Formation by H$_2$ 
%}\label{sec:CO_H2}
%%--------------------------------------------------------------
%
%For $\fracdg = 0.1 \fracdgfid$, CO formation proceeds without H$_2$, and the CO fractions agree with the 
%predictions from the analytical model.
%By contrast, the models with $\fracdg \ge \fracdgfid$ yield
%CO more than predicted from the analytical model 
%especially for larger $\nH$, $Z$, and $\chi$.
%
%What chemical reactions contribute to the acceleration of the CO formation?
%%For the models with $\fracdg=10\fracdgfid$, a sufficient amount of 
%%H$_2$ is produced to promote CO formation.
%We found that ro-vibrationally excited H$_2$ plays an important role, and 
%it accelerates the formation of CH$^+$ and OH, which initiate the CO formation.
%The internal energy of excited H$_2$ is available to overcome the 
%energy barriers of the endothermic reactions $\mathrm{C^+ + H_2\rightarrow CH^+ + H}$ and 
%$\mathrm{O + H_2 \rightarrow OH + H}$.\footnote{
%The reaction $\mathrm{C^+ + H_2\rightarrow CH^+ + H}$ is known to be endothermic by 
%$\sim 4300~$K, and the reaction $\mathrm{O + H_2 \rightarrow OH + H}$ also has endothermicity of 900~K and 
%has an activation barrier of $\sim 4800~$K 
%if H$_2$ is in the ground state.
%Such high energy gaps prevent these reactions from proceeding in low temperature environments.}
%%H$_2$ is excited in ro-vibrational states by cascades following UV pumping.
%The efficiency of the mechanism is investigated in \citet{Zanchet2013} and \citet{Herrez-Aguilar2014}.
%In PDRs of the ISM, the importance of excited H$_2$ was pointed out in 
%\citet{Agndez2010}, \citet{Goicoechea2016}, and \citet{Joblin2018}.
%
%%In order to identify the chemical reactions that enhance the CO fraction, 
%%we extract the important chemical reactions from the full chemical network 
%%considered in the Meudon code, and illustrate them in Figure \ref{fig:COnetworkH2}.
%%Although CO forms through two paths intermediated by OH and CO$^+$ 
%%in a similar manner as shown in 
%%Figure \ref{fig:COnetwork}, 
%%H$_2$ becomes involved in the OH and CO$^+$ formation.
%
%%\begin{figure}[htpb]
%%  \centering
%%  \includegraphics[width=8cm]{Figure11.eps}
%%  \caption{
%%  The most important reactions to form CO in situations where there is 
%%  a non-negligible amount of H$_2$.
%%  The dashed arrows indicate photo-dissociation.
%%}
%%        \label{fig:COnetworkH2}
%%\end{figure}
%
%
%\begin{figure*}[htpb]
%  \centering
%  \includegraphics[width=16cm]{Figure9.eps}
%  \caption{
%      CO fractions at the low $\NC$ limit ($\tau=10^{-5}$)  as a function of $\nH$ 
%      for the cases with (dashed) and without (solid) the CH$^+$ and OH formation 
%      reactions with excited H$_2$ for the (a) A5V and (b) A1V cases.
%The gray lines correspond to the fitting formula 
%shown in Equation (\ref{fitting}).
%}
%\label{fig:CO_fit_noex}
%\end{figure*}
%
%
%
%In order to investigate the effect of excited H$_2$ on the CO formation, 
%we performed additional PDR calculations without the CH$^+$ 
%and OH formation reactions with excited H$_2$.
%Figures \ref{fig:CO_fit_noex}a and \ref{fig:CO_fit_noex}b show that
%%compare the CO fractions in the models with and without the CH$^+$ and OH reactions of excited H$_2$
%%for the A5V and A1V cases, respectively.
%The CO fractions without the reactions of excited H$_2$ do not show any enhancement compared with the 
%predictions from the fitting formula except for $Z=10^3$.
%This clearly shows that 
%the over-production of the CO fraction seen for $\fracdg\ge \fracdgfid$ is attributed to 
%acceleration of the CH$^+$ and OH formation owing to excited H$_2$.
%
%We should note that at $Z=10^3$ for the A5V and A1V cases the CO abundance is higher than predicted from the 
%analytical model even if the CH$^+$ and OH formation reactions of excited H$_2$ are not considered.
%This is because the gas temperatures are high enough to overcome the reaction barriers of 
%the CH$^+$ and OH formation at high rates.
%As shown in Sections \ref{sec:planeT} and \ref{sec:planeTZ}, 
%the gas temperatures increase with increasing either $Z$ and $\fracdg$.
%At $Z=10^3$, the gas temperatures reach  $270~$K for the A5V case with $\nH=10^4~\pcc$, and 
%$900~$K for the A1V cases with $\nH=10^4~\pcc$.


%%------------------------------------------------------------------------------------
%\section{Shielding Effects}\label{sec:shielding}
%%------------------------------------------------------------------------------------
%%------------------------------------------------------------------------------------
\subsubsection{Application of the Analytical Formula 
to Regions Where Shielding Effects Work}\label{sec:shielding}
%%------------------------------------------------------------------------------------

The analytic formula $\nCOana$ at a given position depends on $\nH$, $Z$, and $\chi=\sqrt{\chico\chioh}$.
Although $\nH$ and $Z$ are given locally, $\chi$ is 
determined by the column densities integrated from sources of light.
%Since OH is destroyed by the photons with $1600~\mathrm{\AA}\le \lambda \le 1700~\mathrm{\AA}$,
%$\chioh$ is not decreased by attenuation.
The FUV radiation that destroys CO can be attenuated by dust grains, C$^0$, CO, and H$_2$.
In debris disks, the dust extinction is negligible.

Which of C$^0$, CO, and H$_2$ is more important 
for the shielding effects?
In each shielding effect, 
one can define the critical column density at which $\chico$ is halved. 
Their critical densities are 
\begin{eqnarray}
&&N(\mathrm{H_2})_\mathrm{shld} \sim  4\times 10^{20}~\psc,\nonumber \\
&&N(\mathrm{C^0})_\mathrm{shld} \sim 4\times 10^{16}~\psc \\
&& N(\mathrm{CO})_\mathrm{shld} \sim 10^{14}~\psc\nonumber.
\label{COsh}
\end{eqnarray}
$N(\mathrm{H_2})_\mathrm{shld}$ is given by the results in \citet{vanDishoeck1988};
we derive the H$_2$ column density at which the shielding factor 
becomes 0.5 by the linear interpolation using Table 5 of their paper at $\log_{10} N(\mathrm{CO})=0$ \citep[also see][]{Visser2009}. 
$N(\mathrm{C}^0)_\mathrm{shld}$ is given by $(\ln 2)\alpha_\mathrm{C}^{-1}$,
where $\alpha_\mathrm{C}=1.777\times 10^{-17}~\mathrm{cm}^2$ is 
the cross section of photo-ionization of C$^0$ \citep{Heays2017},  
because the shielding factor owing to C$^0$ is proportional to $\exp(-\alpha_\mathrm{C}N(\mathrm{C}^0))$. 
$N(\mathrm{CO})_\mathrm{shld}$ is taken from 
the CO column density at which the shielding factor is 0.5 at $\log_{10} N(\mathrm{H}_2)=0$
in Table 5 of \citet[][also see Equation (\ref{fshield})]{Visser2009}.

Let us estimate the importance of the shielding 
effect owing to H$_2$.
Considering the abundance ratio between hydrogen and carbon, 
one finds that the H$_2$ column density is expressed as 
$N(\mathrm{H}_2) = 3.8\times 10^3 Z^{-1} N(\mathrm{C}^0)$, 
where it is assumed that all hydrogen is in H$_2$ ($n(\mathrm{H}_2)=\nH/2$) 
and all carbon is in C$^0$ ($n(\mathrm{C}^0)=\nC$). 
By contrast, in order for the shielding effect owing to H$_2$ to be more important than 
the C$^0$ attenuation, $N(\mathrm{H}_2)$ should be larger than 
$10^4 N(\mathrm{C}^0)$, 
where the numerical factor $10^4$ corresponds to 
$N(\mathrm{H}_2)_\mathrm{shld}/N(\mathrm{C}^0)_\mathrm{shld}$.
This condition is not satisfied even for $Z=1$, 
and the shielding effect owing to H$_2$ is not important 
compared with the C$^0$ attenuation.
For simplicity, 
the shielding by H$_2$ is neglected in 
the analytic formula.
We should note that the photo-dissociation rate 
may be overestimated in significantly shielded regions with 
$N(\mathrm{H_2})>N(\mathrm{H_2})_\mathrm{shld}$.




In debris disks, C$^0$ and CO contribute mainly to a decrease in $\chico$.
%also does not work significantly
%because it becomes effective for $\NHm > 10^{21}~\psc$ \citep{vanDishoeck1988, Visser2009}.
%C$^0$ and CO contribute mainly to a decrease in $\chico$.
The problem is that the C$^0$ and CO fractions both depend on local $\chico$,
which depends on their column densities.
In this section, we construct a 
procedure to determine the  
the spatial distributions of $\nCI$, $\nCO$, and $\chico$ consistently.

%At given $\nH$ and $Z$,
%$\nCOana$ depends on $\chi=\sqrt{\chico\chioh}$ that is decreased by the shielding effects in deeper interiors.  
%In this section, in order to obtain the spatial profile of $\nCOana$, 
%$\chi$ is expressed by an analytic expression.

Before presenting a way to evaluate the 
shielding effects, 
we should mention about the analytic formulae for $\nCO$ and $\nCI$.
The analytic formula for $\nCO$ shown in 
Equation (\ref{fitting}) is not applicable directly 
to a situation where carbon nuclei are mostly in CO 
because $\nCOana$ increases without limit as 
$\eta$ increases. To address this issue, 
Equation (\ref{fitting}) is simply modified 
so that $\nCO_\mathrm{ana}$ approaches $\nC$ smoothly 
in $\eta\rightarrow \infty$ as follows:
\begin{eqnarray}
    && \frac{\nCOana}{\nC}  =  \nonumber \\
    &&  \left[1 + \left\{\frac{\Ao}{ {\cal A}_\mathrm{O,ism}} 
    \left(
        10^{-14} \eta^{1.8} + 6.0\times 10^{-11} \eta 
\right)\right\}^{-1}\right]^{-1}.
\label{nCOana}
\end{eqnarray}
This modification may be ad hoc, but it predicts the CO fractions consistent with those obtained from the PDR calculations
as will be shown in  Figure \ref{fig:COshield}.


An analytic formula for the C$^0$ fraction is presented. 
The C$^0$ fraction is determined by the balance between photo-ionization of 
C$^0$ and radiative recombination of C$^+$ \citep{Kamp2000} as follows:
\begin{equation}
    \frac{n(\mathrm{C}^0)_\mathrm{ana}}{\nC-\nCO_\mathrm{ana}} = \frac{2\xi+1-\sqrt{1+4\xi}}{2\xi},
\label{nCIana}
\end{equation}
where the denominator $\nC-\nCO_\mathrm{ana}$ guarantees that the sum of 
the C$^0$, C$^+$, and CO number densities is equal to $\nC$,
and $\xi$ is the ratio of the recombination to the ionization coefficients that is given by 
\begin{eqnarray}
    \xi &\equiv& \frac{k_\mathrm{rec}(n_\mathrm{C}-\nCO_\mathrm{ana})}
   {\alpha_\mathrm{C} \chico F_\mathrm{H}} \nonumber \\
   &=& 
   \chico^{-1}
   \left(\frac{\nC -\nCO_\mathrm{ana}}{11~\mathrm{cm}^{-3}}\right)
    \left( \frac{T}{100~\mathrm{K}} \right)^{-0.82}.
     \label{xi1}
\end{eqnarray}
We confirmed that Equation (\ref{nCIana}) reproduced the C$^0$ fractions obtained from 
the PDR calculations.
%The assumption $\nC=n(\mathrm{C}^0) + \nCp$ is valid 
%when $n(\mathrm{C}^0) \gg \nCO$ because it is unlikely 
%that $\nCp > n(\mathrm{C}^0)$ becomes comparable to $\nCO$.
%Recently, the debris disk HD32297 where the CO mass is larger than 
%the CI mass was discovered \citep{Cataldi2020}.



The normalized FUV flux $\chico$ is expressed 
in terms of the unattenuated 
value $\chi_\mathrm{co,thin}$ as follows:
 \begin{equation}
     \chico = \chi_\mathrm{co,thin} f_\mathrm{shield}(N(\mathrm{C}^0)_\mathrm{ana},\NCO_\mathrm{ana}),
     \label{chico_shield}
 \end{equation}
 where $f_\mathrm{shield}$ corresponds to the shielding factor, which 
 depends on the C$^0$ and CO column densities ($N(\mathrm{C^0})_\mathrm{ana}$ 
 and $N(\mathrm{CO})_\mathrm{ana}$) computed from 
 $n(\mathrm{C}^0)_\mathrm{ana}$ and $\nCOana$, respectively.
 As the shielding factor, 
 we adopt the following expression,
 \begin{equation}
     f_\mathrm{shld} = e^{ - \alpha_\mathrm{C} N(\mathrm{C}^0)_\mathrm{ana} }
     \left[  1 + 
     \left(\frac{N(\mathrm{CO})_\mathrm{ana}}{10^{14}~\mathrm{cm}^{-2}} \right)^{0.6} \right]^{-1},
     \label{fshield}
 \end{equation}
 where the first factor on the right-hand side corresponds to the C$^0$ attenuation, 
 and the second factor is a fitting function of the self-shielding factor at $N(\mathrm{H}_2)=0$ tabulated in \citet{Visser2009}.
 

The spatial distributions of $\chico$, $n(\mathrm{C}^0)_\mathrm{ana}$, and $\nCOana$ are 
determined consistently in a iterative manner by using 
Equations (\ref{nCOana}), (\ref{nCIana}), and (\ref{chico_shield}).
% At a given distance from the slab edge, 
% Equation (\ref{chico_shield}) shows that 
% the local $\chico$ depends on $n(\mathrm{C}^0)_\mathrm{ana}(\nC,Z,\chico)$ and $\nCOana(\nH,Z,\chioh,\chico)$ 
% from the slab edge to the position $L$.
% The spatial distributions of $\chico$, $n(\mathrm{C}^0)_\mathrm{ana}$, and $\nCOana$ are 
% determined consistently in a iterative manner.


It is useful to derive $\nH$ required to produce a specific value of the CO fraction.
In the high density limit where Path$_\mathrm{OH}$ is dominated, since the second term in the right-hand side of Equation (\ref{fitting}) is negligible,
Equation (\ref{fitting}) is solved for $\nH$ as follows:
\begin{eqnarray}
    n_\mathrm{H,req} &=& 6\times 10^7~\pcc~Z^{-0.4}\chi^{1.1} \nonumber \\
    && \;\;\;\;
    \times \left( \frac{\Ao}{ {\cal A}_\mathrm{O,ism}} \right)^{-0.56} 
    \left( \frac{n(\mathrm{CO)}}{\nC} \right)^{0.56},
    \label{nHconstrained}
\end{eqnarray}
which is valid when
\begin{equation}
    n_\mathrm{H,req} > 5.3\times 10^4~\pcc~Z^{-0.4}\chi^{1.1}.
    \label{nHreq}
\end{equation}
Equation (\ref{nHconstrained}) is rewritten as 
\begin{eqnarray}
    n_\mathrm{C,req} &=& 8\times 10^3~\pcc Z^{0.6}\chi^{1.1} \nonumber \\
    && \;\;\;\;
    \times \left( \frac{\Ao}{ {\cal A}_\mathrm{O,ism}} \right)^{-0.56} 
    \left( \frac{n(\mathrm{CO})}{\nC} \right)^{0.56},
    \label{nCreq}
\end{eqnarray}
which corresponds to the carbon nucleius density 
required to reproduce a specific value of $\nCO/\nC$.
From Equation (\ref{nCreq}), an important conclusion can be drawn that $\nCO/\nC$ decreases with increasing $Z$
for fixed $\nC$ and $\chi$.
In other words, the upper limit of the CO fraction is obtained at $Z=1$.

%%------------------------------------------------------------------------------------
\subsubsection{Comparison of the CO Fractions with Those Predicted from the Analytical Formula}\label{sec:compana}
%%------------------------------------------------------------------------------------

%If there is a sufficient amount of the gas, the CO fraction can be enhanced by the 
%shielding effects: CO self-shielding and shielding by 
%carbon atoms and H$_2$. 

%Figure \ref{fig:COshield} shows the spatial distributions 
%of the CO fractions for  various $\chi$, $\nC$, and $Z$ listed in Table \ref{tab:param}.
%The mass-to-flux ratios are fixed to 
%$\fracdg=0.1\fracdgfid$ for weak-FUV and 
%$\fracdg=0.01\fracdgfid$ for strong-FUV, where 
%the CO formation proceeds without H$_2$.


Figure \ref{fig:COshield} compares 
the results of the PDR calculations 
with the predictions from the analytic formula. 
$\nCOana$ reproduces the spatial profiles of $\nCO$ reasonably well 
from the unattenuated regions ($\NC\ll 10^{17}~\psc$) to 
the shielded regions ($\NC\gtrsim 10^{17}~\psc$)
for all the parameter sets.
%the CO fractions are reproduced by Equation (\ref{fitting}) both for 
%A5V and A1V (Figure \ref{fig:COfit}).
%For each line, the CO fraction takes a constant value 
%for low $N(\mathrm{C}^0)$, and begins to increase rapidly when 
%$N(\mathrm{C}^0)$ exceeds a certain value.
%The C$^0$ attenuation is expected to play an important role when $N(\mathrm{C}^0)$ becomes larger than 
%$\sim 10^{17}~\mathrm{cm}^{-2}$ because the cross section of the C$^0$ attenuation is 
%$a_\mathrm{C}=1.777\times 10^{-17}~$cm$^{-2}$ \citep{Heays2017}.


%The analytic formula $\nCOana$ taken into account the shielding effects (Section \ref{sec:shielding})
%are overplotted by the thick solid lines in Figure \ref{fig:COshield}.
%The difference between the calculated and predicted values 
%is within a factor of several.
%For low $N(\mathrm{C}^0)$ where the shielding effects are not important, 
%the difference between the calculated and predicted values is within a factor of several.


%{\color{blue}
%In order to examine the effect of the C$^0$ attenuation, 
%$\nCOana$ without the CO self-shielding effect is plotted 
%by the thick dashed lines.
%The predicted CO fractions $\nCOana$ with and without the CO self-shielding 
%are consistent as long as the CO fractions are lower than $10^{-3}$ at the low $N(\mathrm{C}^0)$ limit. 
%}
%One can see that rapid increases in the CO fractions are roughly 
%explained {\color{blue} only} by the C$^0$ attenuation
%in Equation (\ref{fitting}) as long as the CO fractions are lower than $10^{-3}$ at the low $N(\mathrm{C}^0)$ limit. 
%It should be noted, however, that the CO fraction begins to increase at slightly lower $N(\mathrm{C}^0)$ than predicted because 
%chemical reactions different from those shown in Figure \ref{fig:COnetwork} are important.
%The difference between the calculated and predicted values is within a factor of several for $N(\mathrm{C}^0) \le 10^{17}~\psc$ 
%and $\nCO/\nC<10^{-3}$ at the low $N(\mathrm{C}^0)$ limit.

For lower $\nC$ shown in Figures \ref{fig:COshield}a 
and \ref{fig:COshield}c, the CO fractions begin to 
increase around $\NC\sim 10^{16-17}~\psc$, regardless of $Z$ and $\chi$.
By contrast, Figures \ref{fig:COshield}b and \ref{fig:COshield}d show that 
the CO fractions with $Z=1$ begin to increase at lower $\NC$.
This comes from the fact that CO self-shielding only works if the CO fraction is high enough; otherwise, C$^0$ attenuation 
becomes more important.

%We here derive a condition under which the 
%CO self-shielding  effect dominates over the  C$^0$ attenuation.
%First $\NCO_\mathrm{shld}$ and $\NCI_\mathrm{shld}$ are defined as 
%the C$^0$ and CO column densities at which $\chico$ is reduced 
%by half owing to the CI attenuation and CO self-shielding effect, respectively.
%From Equation (\ref{fshield}), we obtain  $N(\mathrm{C}^0)_\mathrm{shield} = (\ln 2)\alpha_\mathrm{C}^{-1}=4\times 10^{16}~\psc$ and 
%$\NCO_\mathrm{shield} =  10^{14}~\psc$.
A critical CO fraction above which 
CO self-shielding effect works is given by 
\begin{equation}
    \left(\frac{\nCO}{\nC}\right)_\mathrm{cri} \sim \frac{\NCO_\mathrm{shld}}{N(\mathrm{C}^0)_\mathrm{shld}}
    \sim  3\times 10^{-3}.
    \label{COshield_cri}
\end{equation}
The horizontal dashed lines in Figure \ref{fig:COshield} 
correspond to $(\nCO/\nC)_\mathrm{cri}$. 
In order to illustrate the effect of the 
CO self-shielding 
in Figure \ref{fig:COshield}, 
we overplot the profiles of $\nCO_\mathrm{ana}$ without 
considering the CO self-shielding effect in Equation (\ref{fshield}) by 
the thick dashed lines.
It is clearly seen that the CO self-shielding effect 
increases the CO fractions at
$N_\mathrm{C}<N(\mathrm{C}^0)_\mathrm{shld}$
only when $\nCO$ exceeds 
$\nCO_\mathrm{cri}$.
%where we use the fact that $n(\mathrm{C}^0)$ and $\nCO$ are 
%roughly constant until  the shielding becomes effective.

The importance of C$^0$ attenuation in CO formation has been pointed 
out by \citet{Kral2019}. 

%\begin{figure*}[htpb]
%        \centering
%        \includegraphics[width=16.0cm]{CO_fit_fdg_rev4.eps}
%        \caption{
%        \color{blue}
%        (Upper panels)
%        The CO column density ratios $N(\mathrm{CO}/N_\mathrm{C}$ at (circle) $N_\mathrm{C}=10^{16}~\psc$
%        and (triangle) $N_\mathrm{C}=10^{16}~\psc$ for the weak-FUV models as 
%        a function of $Z$.
%        From left to right, the dust-to-gas mass ratios are $\fracdg=0.1\fracdgfid$,
%        $\fracdg=\fracdgfid$, and $\fracdg=10\fracdgfid$. 
%        The solid and dashed lines show the results with $T_\mathrm{chem}=300$~K and $T_\mathrm{chem}=10~$K,
%        respectively.
%        The red and blue lines show the results with 
%        $n_\mathrm{C}=1.32\times 10^{2}~\pcc$ and  $n_\mathrm{C}=1.32\times 10^{3}~\pcc$, respectively.
%        The light red and blue lines correspond to the predictions from $\nCO_\mathrm{ana}$ 
%        with  $n_\mathrm{C}=1.32\times 10^{2}~\pcc$ and  $n_\mathrm{C}=1.32\times 10^{3}~\pcc$, respectively.
%        (Lower panels)
%        The same as the upper panels but for strong-FUV models.
%        The red and blue lines show the results with 
%        $n_\mathrm{C}=1.32\times 10^{3}~\pcc$ and  $n_\mathrm{C}=1.32\times 10^{4}~\pcc$, respectively.
%        The left, middle, and right panels correspond to $\fracdg=0.01\fracdgfid$,
%        $\fracdg=0.1\fracdgfid$,
%        $\fracdg=\fracdgfid$, respectively.
%	    }
%\label{fig:COfit}
%\end{figure*}


%

%Figure \ref{fig:COshield}b shows that 
%the CO fraction in the $Z=1$ case begins to increase at much lower $N(\mathrm{C}^0)$ than predicted by the light red line.
%This is attributed to the CO self-shielding effect, which reduces $\chico$ by half 
%when $\NCO$ reaches $\NCO_\mathrm{shield} = 10^{14}-2\times 10^{14}~\psc$.
%We investigate a condition under which the CO fraction is increased due to the CO self-shielding effect rather than C$^0$ attenuation.
%Using $N(\mathrm{C}^0)_\mathrm{shield} = (\ln 2)a_\mathrm{C}^{-1}$, where $\chico$ is reduced by half due to the C$^0$ attenuation, 
%one obtains the critical CO fraction above which the CO self-shielding works  as follows:
%%The C$^{0}$ attenuation reduces the CO photo-dissociation rate by a factor of two at 
%%$N(\mathrm{C}^0)_\mathrm{shield} \sim 4\times 10^{16}~\psc$.
%%Using $\NCO_\mathrm{shield}$ and $N(\mathrm{C}^0)_\mathrm{shield}$, one obtains 
%%the critical CO fraction below which the C$^0$ attenuation increases the CO fraction before 
%%CO self-shielding works as follows:
%\begin{equation}
%    \left(\frac{\nCO}{\nC}\right)_\mathrm{cri} \sim \frac{\NCO_\mathrm{shield}}{N(\mathrm{C}^0)_\mathrm{shield}}
%    \sim  3\times 10^{-3},
%    \label{COshield_cri}
%\end{equation}
%where we use the fact that $n(\mathrm{C}^0)$ and $\nCO$ are roughly constant until 
%the shielding becomes effective.
%The importance of C$^0$ attenuation in CO formation was pointed out by \citet{Kral2019}.
%The horizontal lines in Figure \ref{fig:COshield} show $(\nCO/\nC)_\mathrm{cri}$.
%In Figure \ref{fig:COshield}, it is confirmed that the CO shielding effect works only when the CO fraction at the low $N(\mathrm{C}^0)$ limit 
%exceeds $(\nCO/\nC)_\mathrm{cri}$.



%In order to confirm that the critical CO fraction characterises the CO shieldings,
%$\NCO/\NC$ is plotted as a function of $N(\mathrm{C}^0)$ for various $\nC$ and $Z$ 
%in Figures \ref{fig:COshield}.
%For $\NCO/\NC < (\nCO/\nC)_\mathrm{cri,\tau=0}$,
%the CO fractions begin to increase around $N(\mathrm{C}^0)\sim N(\mathrm{C}^0)_\mathrm{shield}$ for 
%both the A5V and A1V cases.
%    In Figure \ref{fig:COshield}, only for the model with $Z=1$ and $\nC=1.32\times 10^3$~cm$^{-3}$, 
%$\NCO/\NC$ at the low $\NC$ limit exceeds the critical CO fraction.
%In this case, the transition form $\mathrm{C}^0$ to CO occurs for 
%$N(\mathrm{C}^0)<N(\mathrm{C}^0)_\mathrm{shield}$ owing to the CO self-shielding effect.



%%------------------------------------------------------------------------------------
\subsection{Parameter Survey}\label{sec:parametersurvey}
%%------------------------------------------------------------------------------------

In Section \ref{sec:analyticmodel}, we developed 
the analytic formula which reproduces the results of the PDR calculations reasonably 
well when the H$_2$ fraction is too small to affect the CO formation.
By contrast, in Section \ref{sec:plane}, 
we found that the models with 
lower $T_\mathrm{chem}$ and/or higher $\fracdg$ 
yield a sufficient amount of H$_2$ to  accelerate the CO formation. 
In this section, we investigate how the CO fractions depend on $\fracdg$, $T_\mathrm{chem}$, and 
$\NC$ for various $n_\mathrm{C}$, $Z$, and $\chico$.

\begin{figure*}[htpb]
        \centering
        \includegraphics[width=18.0cm]{Fig5.pdf}
        \caption{
        CO column density ratios $N(\mathrm{CO})/N_\mathrm{C}$ as a function of $\fracdg/\fracdgfid$ 
        for ({\it top panels}) the weak-FUV models with $\NC=10^{17}~\psc$,
        ({\it middle panels}) the strong-FUV models with $\NC=10^{17}~\psc$, and 
        ({\it bottom panels}) the strong-FUV models with $\NC=2\times 10^{18}~\psc$. 
        The results with $2\times 2$ combinations of 
        $\nC=(1.32\times 10^2~\pcc, 1.32\times 10^3~\pcc)$
        and $T_\mathrm{chem}=(300~\mathrm{K},10~\mathrm{K})$ for weak-FUV and 
        $\nC=(1.32\times 10^3~\pcc, 1.32\times 10^4~\pcc)$
        and $T_\mathrm{chem}=(300~\mathrm{K},10~\mathrm{K})$ for strong-FUV. 
        In each panel, the color represents the gas metallicities,
        (red) $Z=1$, (green) $Z=10$, (blue) $Z=10^2$, and (magenta) $Z=10^3$.
        The horizontal thick lines correspond to the predictions from $\nCO_\mathrm{ana}$ with 
        four different $Z$ ($Z=1,10,10^2,10^3$).
        The horizontal dashed black lines show the critical CO fraction $(\nCO/\nC)_\mathrm{cri}$ (Equation (\ref{COshield_cri}))
        above which the CO is shielded mainly by self-shielding.
        }
\label{fig:COfit}
\end{figure*}

%----------------------------------------------------------------------
\subsubsection{The CO Fractions at $\NC = 10^{17}~\psc$}\label{sec:CO1e17}
%----------------------------------------------------------------------


First, we investigate the CO fractions at $\NC=10^{17}~\psc$, which 
is the mid-plane column density inferred from the observational results 
of $\beta$ Pictoris
(Section \ref{sec:modelparameter}).
At this column density, CO is not shielded by C$^0$ attenuation significantly while
CO self-shielding can work if the CO fraction is sufficiently high (Section \ref{sec:compana}).
For reference, we also investigate the CO fractions at 
$\NC=10^{17}~\psc$ for strong-FUV.
The top and middle panels of Figure \ref{fig:COfit} 
show the CO fractions as a function of $\fracdg/\fracdgfid$
for various $\nC$, $T_\mathrm{chem}$, and $\chico$.
%The models with $T_\mathrm{chem}=300~$K and 
%the lowest $\fracdg$ show that the CO fractions 
%are consistent with the predictions 
%from the analytic formula as shown in Figure \ref{fig:COshield}.
Overall, as $T_\mathrm{chem}$ decreases  and/or 
$\fracdg$ increases,
the CO fractions increase and deviate from the predictions from the analytic formula.

%The top and middle panels of Figure \ref{fig:COfit}
%show $\NCO/\NC$ as a function of $\fracdg$ at $\NC=10^{17}~\pcc$
%for the weak-FUV and strong-FUV models, respectively.

First, we investigate the results with $T_\mathrm{chem}=300$~K 
(Figures \ref{fig:COfit}a, \ref{fig:COfit}c, \ref{fig:COfit}e, and 
\ref{fig:COfit}g).
There is a critical $\fracdg/\fracdgfid$ ($(\fracdg/\fracdgfid)_\mathrm{cri}$) 
above which the CO formation is accelerated.
$(\fracdg/\fracdgfid)_\mathrm{cri}$ 
depends on the parameters. 
For the fiducial weak-FUV models ($\nC=1.32\times 10^2~\pcc$),
$(\fracdg/\fracdgfid)_\mathrm{cri}$ is around 1, 
regardless of $Z$ (Figure \ref{fig:COfit}a).
For the high-density weak-FUV models ($\nC=1.32\times 10^3~\pcc$), $(\fracdg/\fracdgfid)_\mathrm{cri}$ depends on $Z$ (Figure \ref{fig:COfit}c).
The $Z\ge 10^2$ models show acceleration of the CO formation at $\fracdg/\fracdgfid=1$ while 
$(\fracdg/\fracdgfid)_\mathrm{cri}$ is around 1 for $Z\le 10$.

The $\fracdg$ dependence of the CO fractions for the fiducial 
strong-FUV models ($\nC=1.32\times 10^3~\pcc$) is 
similar to that for the high-density weak-FUV models ($\nC=1.32\times 10^3~\pcc$) (see Figures \ref{fig:COfit}c and \ref{fig:COfit}e);
enhancement of the CO fractions is seen only for $Z\gtrsim 10^2$ when $\fracdg/\fracdgfid$ is increased from 0.1 to 1.
For the high-density strong-FUV model ($\nC=1.32\times 10^4~\pcc$), $(\fracdg/\fracdgfid)_\mathrm{cri}\sim 0.1$ for all $Z$.

The short summary is that $(\fracdg/\fracdgfid)_\mathrm{cri}\sim 1$ for $\nC=1.32\times 10^2~\pcc$,
$(\fracdg/\fracdgfid)_\mathrm{cri}\sim 0.1$ for $\nC=1.32\times 10^4~\pcc$, and 
the models with $\nC=1.32\times 10^3~\pcc$ show the intermediate behavior where 
$(\fracdg/\fracdgfid)_\mathrm{cri}$ takes values between 0.1 and 1 depending on $Z$.
The dependence of $\chico$ on $(\fracdg/\fracdgfid)_\mathrm{cri}$ appears to be weak.


\begin{figure}[htpb]
        \centering
        \includegraphics[width=8.0cm]{Fig6.pdf}
        \caption{
        Scatter plots of $f_\mathrm{boost}$ versus $2N(\mathrm{H}^*_2)/\NH$, where 
        $f_\mathrm{boost}$ shows the degree of the enhancement of the CO fraction owing to H$^*_2$
        for (a) weak-FUV and (b) strong-FUV.
        The column densities are measured at $\NC=10^{17}~\psc$.
        The ranges of the parameters ($\nC, T_\mathrm{chem}, Z, \fracdg$) are shown in Table \ref{tab:param}.
        The data points are distinguished by $\nC$, $T_\mathrm{chem}$, and $Z$.
        The shapes of the markers represent the difference in ($\nC$, $T_\mathrm{chem}$).
        The colors of the markers represent different gas metallicities,
        (red) $Z=1$, (green) 10, (blue) 10$^2$, and (magenta) $10^3$.
        The data points with different $\fracdg$ are illustrated without distinction, but 
        $2N(\mathrm{H^*_2})/\NH$ increase with $\fracdg$.
        The vertical dashed black lines correspond to $2n(\mathrm{H_2^*})_\mathrm{cri}/\nH$ defined in Equation (\ref{nH2cri}).
	 }
%        \caption{
%        \color{blue}
%        Scatter plots of ({\it left column}) $2N(\mathrm{H}^*_2)/\NH$ versus $f_\mathrm{boost}$ and 
%        ({\it right column}) $2N(\mathrm{H}_2)/\NH$ versus $f_\mathrm{boost}$, where 
%        $f_\mathrm{boost}$ shows the degree of the enhancement of the CO fraction owing to H$^*_2$
%        for ({\it top row}) weak-FUV and ({\it bottom row}) strong-FUV.
%        The column densities are measured at $\NC=10^{17}~\psc$.
%        The ranges of the parameters ($\nC, T_\mathrm{chem}, Z, \fracdg$) are shown in Table \ref{tab:param}.
%        The data points are distinguished by $\nC$, $T_\mathrm{chem}$, and $Z$.
%        The shapes of the markers represent the difference in ($\nC$, $T_\mathrm{chem}$).
%        The colors of the markers represent different gas metallicities,
%        (red) $Z=1$, (green) 10, (blue) 10$^2$, and (magenta) $10^3$.
%        The data points with different $\fracdg$ are illustrated without distinction, but 
%        both $2N(\mathrm{H^*_2})/\NH$ and $2N(\mathrm{H_2})/\NH$ increase with $\fracdg$.
%        In the left panels, the vertical dashed black lines correspond to $2n(\mathrm{H_2^*})_\mathrm{cri}/\nH$ defined in Equation (\ref{nH2cri}).
%	 }
\label{fig:H2CO}
\end{figure}

A decrease in $T_\mathrm{chem}$ reduces 
$(\fracdg/\fracdgfid)_\mathrm{cri}$ significantly.
Figures \ref{fig:COfit}b, \ref{fig:COfit}d, \ref{fig:COfit}f, 
and \ref{fig:COfit}h show that 
$(\fracdg/\fracdgfid)_\mathrm{cri}<0.1$ for weak-FUV and 
$(\fracdg/\fracdgfid)_\mathrm{cri}\sim 10^{-2}$ for strong-FUV.
The CO fractions monotonically increase with 
$\fracdg/\fracdgfid$ for most cases.

We should note that increases in the CO fractions appear to saturate for 
the weak-FUV models with ($n_\mathrm{C}=1.32\times 10^2~\pcc$, $\fracdg\ge \fracdgfid$, $Z > 1$, $T_\mathrm{chem}=10~$K) and 
($n_\mathrm{C}=1.32\times 10^3~\pcc$, $\fracdg\ge \fracdgfid$, $Z > 10$, $T_\mathrm{chem}=10~$K). 
This is because the H$_2$ fractions are close to unity in these models, and 
the enhancement of the H$_2$ formation rate does not lead to further acceleration of the CO fractions.

The CO fractions at $\NC=10^{17}~\psc$ are affected by CO self-shielding when the CO fractions exceed 
$(\nCO/\nC)_\mathrm{cri}$ (Equation (\ref{COshield_cri})), which is shown by the horizontal dashed black lines.
When $\nCO/\nC>(\nCO/\nC)_\mathrm{cri}$, a small increase in $\nCO/\nC$ for $\NC<10^{17}~\psc$ results in 
a large increase in $\NCO/\NC$ at $\NC=10^{17}~\psc$.
This effect is clearly seen in 
Figures \ref{fig:COfit}b and \ref{fig:COfit}d. 

What chemical reactions contribute to the acceleration of the CO formation?
We found that vibrationally excited H$_2$ (hereafter H$^*_2$) plays an important role. 
As a result, different reaction paths to CO are activated than the chemical reactions shown in 
Figure \ref{fig:COnetwork}.
It starts from $\mathrm{C^+ + H_2\rightarrow CH^+ + H}$ 
which is known to be endothermic by $\sim 4300~$K.
%The product CH$^+$ is combined with O to form CO$^+$.
%Finally, CO forms through $\mathrm{CO^+ + H \rightarrow CO + H^+}$.
The internal energy of H$^*_2$ is available to 
overcome its energy barrier. 
The reaction between C$^+$ and H$^*_2$ with $v>0$ 
proceeds at almost the Langevin collision rate 
($k_\mathrm{cp,h2*}=1.6\times 10^{-9}~\mathrm{cm}^3~\mathrm{s}^{-1}$) \citep{Hierl1997}.

The efficiency of 
the mechanism is investigated in \citet{Zanchet2013} and \citet{Herrez-Aguilar2014}.
In PDRs of the ISM, the importance of H$^*_2$ was also 
pointed out in many references \citep{Agndez2010,Goicoechea2016,Joblin2018,Veselinova2021,Goicoechea2022}.
%One may consider that $\mathrm{O+H_2^* \rightarrow  OH + H}$ can yield a more 
%abundant OH, which initiates the CO formation as shown in Figure \ref{fig:COnetwork}.
%This is the case only for $\NC$ is significantly low.
%As $\NC$ increases, The number density of the vibrationaly-excited H$_2$ decreases rapidly.

%As a necessary condition for H$_2^*$ to accelerate the CO formation,

How much H$^*_2$ is required to accelerate CO formation?
Figure \ref{fig:H2CO} shows the scatter plots of $2N(\mathrm{H}^*_2)/\NH$ versus a boost factor $f_\mathrm{boost}$, which 
is defined as the CO fraction divided by that at the H$_2$-poor environments where 
only $\fracdg$ is changed to $0.1\fracdgfid$ for weak-FUV 
and $0.01\fracdgfid$ for strong-FUV, keeping the other parameters unchanged.

%Roughly speaking, 
Figure \ref{fig:H2CO} show that 
$f_\mathrm{boost}$ is correlated with 
$2N(\mathrm{H}^*_2)/\NH$ for both weak-FUV and strong-FUV, 
indicating that H$^*_2$ accelerates the CO formation.
$f_\mathrm{boost}$ is larger than unity only when $2N(\mathrm{H}_2^*)/\NH$ exceeds a critical 
H$_2^*$ fraction of $\sim 10^{-7}$ although there are large scatters.
%The correlation between $f_\mathrm{boost}$ and the H$_2^*$ fraction 
%clearly suggests that the CO formation is enhanced by the excited H$_2$.

The critical H$^*_2$ fraction can be roughly understood as follows.
A necessary condition affecting the CO formation by H$^*_2$ is that 
the reaction rate of $\mathrm{C^+ + H_2^*\rightarrow CH^+ + H}$ 
must be larger than that of $\mathrm{C^++H \rightarrow CH^+ + h\nu}$. 
The condition becomes 
\begin{eqnarray}
n(\mathrm{H}_2^*) > 
    n(\mathrm{H}_2^*)_\mathrm{cri} &=& \frac{k_\mathrm{cp,h}}{k_\mathrm{cp,h2*}}\nH \nonumber \\
   & \sim & 3\times 10^{-8}\left(\frac{T}{50~\mathrm{K}}\right)^{-0.42}\nH,
    \label{nH2cri}
\end{eqnarray}
where  $k_\mathrm{cp,h} =
2.29\times 10^{-17}(T/300)^{-0.42}~\mathrm{cm^3~s^{-1}}$ is the reaction rate of 
$\mathrm{C^+ + H\rightarrow CH^+ + h\nu}$.
Figures \ref{fig:H2CO} show that 
$2N(\mathrm{H_2^*})/\NH > 2n(\mathrm{H_2^*})_\mathrm{cri}/\nH$
can distinguish whether to accelerate the CO formation or not. 

The correlation between $f_\mathrm{boost}$ 
and $2N(\mathrm{H_2^*})/\NH$ 
depends on $Z$; the models with lower $Z$ 
tend to show larger $f_\mathrm{boost}$
at a fixed $2N(\mathrm{H_2^*})/\NH$.
This is because 
for smaller $Z$, the efficient 
formation of CH$^+$ activates more reaction pathways involved by hydrogen to produce CO 
since $\nH$ increases with decreasing $Z$ at a given $\nC$.

%How much H$_2$ is required to accelerate CO formation?
%The right panels of Figure \ref{fig:H2CO} show the scatter plot of $2N(\mathrm{H}_2)/\NH$ versus $f_\mathrm{boost}$. 
%For weak-FUV, in order to obtain $f_\mathrm{boost}$ greater than 2, 
%the H$_2$ fractions must be larger than $10^{-0.1}\sim 0.8$.
%
%For strong-FUV, a striking feature not seen in weak-FUV is that 
%the scatter plot of $f_\mathrm{boost}$ and $2N(\mathrm{H}^*_2)/\NH$ has a large scatter depending on $Z$ (Figure \ref{fig:H2CO}d).
%The $Z\lesssim 10$ models behave a similar way as those for weak-FUV, but the H$_2$ fractions are shifted toward lower values.
%The reason why the H$^*_2$ fractions in the strong-FUV and weak-FUV models are comparable, even though the 
%H$_2$ fractions in the strong-FUV models are smaller than those in the weak-FUV models, can be explained as follows.
%Considering the chemical balance between the H$_2$ formation rate on the grain surface and H$_2$ photo-dissociation, 
%one obtains 
%\begin{equation}
%   \frac{n(\mathrm{H}_2)}{\nH}  \propto \nC \left(\frac{\fracdg}{\fracdgfid}\right)\chi_\mathrm{H_2}^{-1},
%\end{equation}
%where Equation (\ref{RER}) is used and $\chi_\mathrm{H_2}$ is the normalized FUV flux for H$_2$ dissociating photons.
%In unattenuated regions, $\chi_\mathrm{H_2}$ is equal to $\chico$, but it decreases owing to H$_2$ self-shielding.
%Using the fact that 
%$n(\mathrm{H}^*_2)/n(\mathrm{H}_2)$ is roughly in 
%proportion to $\chi_\mathrm{H_2}$,
%the $\chi_\mathrm{H_2}$-dependence of the amount of H$^*_2$ disappears as follows:
%\begin{equation}
%   \frac{n(\mathrm{H}^*_2)}{\nH}  \propto \nC \left(\frac{\fracdg}{\fracdgfid}\right).
%\end{equation}
%This is a reason why the critical $\fracdg/\fracdgfid$ does 
%not depend on $\chico$ significantly (Figure \ref{fig:COfit}).
%
%The strong-FUV models with $Z\gtrsim 10^2$ show that 
%$f_\mathrm{boost}$ takes large values for $2N(\mathrm{H}_2)/\NH\gtrsim 10^{-4}$.
%This is attributed to the following two effects.
%One is that $n(\mathrm{H^*_2})/n(\mathrm{H}_2)$ rapidly 
%decreases when the H$_2$ self-shielding effect becomes important.
%The other is that $\NH$ at $\NC=10^{17}~\psc$ decreases inversely with $Z$,
%indicating that the H$_2$ self-shielding effect is less significant for 
%higher $Z$. Thus, $n(\mathrm{H^*_2})/n(\mathrm{H}_2)$ increases 
%with $Z$ at a fixed $\NC$.
%The reason why the weak-FUV models do not have such features 
%in Figure \ref{fig:H2CO}c is that 
%the H$_2$ fractions for weak-FUV are large enough for the H$_2$ self-shielding 
%effect to be significant at $\NC=10^{17}~\psc$.
%
%%Combination of these two effects results in 
%%that $n(\mathrm{H^*_2})/n(\mathrm{H_2})$
%
%%This can be understood as follows.
%%The H$_2$ number density is expressed as 
%%$n(\mathrm{H}_2)=3.8\times 10^3 Z^{-1}\nC x_\mathrm{H_2}$, where 
%%$x_\mathrm{H_2}=2n(\mathrm{H}_2)/\nH$ is the H$_2$ fraction.
%%Assuming that $x_\mathrm{H_2}$ is constant, 
%%the H$_2$ column density at $\NC = 10^{17}~\psc$ is given by 
%%$N(\mathrm{H}_2) = 3.8\times 10^{20}Z^{-1}x_\mathrm{H_2}$.
%%The H$_2$ self-shielding effect becomes significant when 
%%$x_\mathrm{H_2} > 2.6\times 10^{-4}(Z/10^3)^{-1}$.
%
%
%
%
%
%
%
%
%
%
%
%
%
%
%
%
%
%
%
%
%
%%The product CH$^+$ is combined with O to form CO$^+$.
%%Finally, CO forms through $\mathrm{CO^+ + H \rightarrow CO + H^+}$.
%
%
%%The internal energy of excited H$_2$ is available to overcome the 
%%energy barriers of the endothermic reactions $\mathrm{C^+ + H_2\rightarrow CH^+ + H}$ and 
%%$\mathrm{O + H_2 \rightarrow OH + H}$.\footnote{
%%The reaction $\mathrm{C^+ + H_2\rightarrow CH^+ + H}$ is known to be endothermic by 
%%$\sim 4300~$K, and the reaction $\mathrm{O + H_2 \rightarrow OH + H}$ 
%%also has endothermicity of 900~K and has an activation barrier of $\sim 4800~$K 
%%if H$_2$ is in the ground state.
%%Such high energy gaps prevent these reactions from proceeding in low temperature environments.}
%%H$_2$ is excited in ro-vibrational states by cascades following UV pumping.
%%The efficiency of the mechanism is investigated in \citet{Zanchet2013} and \citet{Herrez-Aguilar2014}.
%%In PDRs of the ISM, the importance of excited H$_2$ was pointed out in 
%%\citet{Agndez2010}, \citet{Goicoechea2016}, and \citet{Joblin2018}.
%
%
%
%


 

%----------------------------------------------------------------------
\subsubsection{The CO Fractions at $\NC = 2\times 10^{18}~\psc$}\label{sec:CO2e18}
%----------------------------------------------------------------------

We investigate the CO fractions for strong-FUV at $\NC=2\times 10^{18}~\psc$, which 
is the mid-plane column density inferred from the observational results 
of 49 Ceti (Section \ref{sec:modelparameter}).
At this column density, CO is shielded significantly.

%For $\nC=1.32\times 10^3~\pcc$
Interestingly the CO fractions 
at $\NC=2\times 10^{18}~\psc$ show the opposite
$\fracdg$-dependence from that at $\NC=10^{17}~\psc$.
For $T_\mathrm{chem}=300~$K, 
increases in $\fracdg$ reduce the CO fractions 
(Figures \ref{fig:COfit}i and \ref{fig:COfit}k).
The production of CH$^+$ through the reaction between C$^+$ and H$^*_2$ 
does not contributes to the CO formation because the  C$^0$ attenuation 
makes the C$^+$ fractions extremely low. 
That is why efficient formation of H$_2$ no longer promotes the CO formation.
Conversely, the presence of H$_2$ negatively affects CO formation.
In the chemical network shown in Figure \ref{fig:COnetwork}, 
an increase in the amount of H$_2$ 
reduces the amount of available H atoms, 
leading to a decrease in the CO formation rate.


A similar behavior is seen for $T_\mathrm{chem}=10~$K.
As long as $\fracdg\le 0.1\fracdgfid$, 
increasing $\fracdg$ reduces the amount of CO.
When $\fracdg$ is increased from $0.1\fracdgfid$ to $\fracdgfid$, 
the CO fractions turn to increase 
in Figures \ref{fig:COfit}j and \ref{fig:COfit}l.
This is because most hydrogen exists as H$_2$ for $\fracdg=\fracdgfid$. 
The chemical reactions shown in Figure \ref{fig:COnetwork} 
no longer work, and different chemical 
reactions associated with H$_2$ become 
important to form CO.

%----------------------------------------------------------------------
%\subsubsection{Short Summary}\label{sec:shortsummary}
%----------------------------------------------------------------------
%Let us briefly summarise Section \ref{sec:parametersurvey}.












%-------------------------------
\section{Discussion}\label{sec:discuss}
%-------------------------------

%------------------------------------------------------------------------
%\subsubsection{ Setup of the PDR Calculations Taking Into Disk Structures}
\subsection{ Predictions of the Spatial Distributions of CO in a Disk Structure}\label{sec:COdiskana}
%------------------------------------------------------------------------



The Meudon PDR code is not designed to investigate the chemical and thermal structure of disks.
However, in Section \ref{sec:analyticmodel}, we developed the numerical procedure to determine the CO fractions 
on the basis of the findings 
of the plane-parallel PDR calculations.
In this section,
we present the method to derive the spatial distributions of 
$\nCIana$ and $\nCOana$ in a given disk structure using a similar method as shown in Section \ref{sec:shielding}.  

%To confirm whether our results are reasonable, we conduct the PDR calculations using local $\nH$, $Z$, and radiation field at the midplane, 
%and compare the PDR calculation results with $\nCIana$ and $\nCOana$ in Section \ref{sec:}


\begin{figure}[htpb]
        \centering
        \includegraphics[width=8cm]{Fig7.pdf}
        \caption{
            Schematic picture of our setting 
            where the three rays are considered.
	}
\label{fig:obs}
\end{figure}




Figure \ref{fig:obs} shows the disk structure considered in this paper.
The disk extends from $R=R_\mathrm{in}$ to $R_\mathrm{out}$, 
where the cylindrical coordinate $(R,z)$ is used and the central star locates at the origin.
The density distribution of carbon nuclei is denoted as $n_\mathrm{C}(R,z)$,
and $Z$ and $\fracdg$ are assumed to be uniform throughout the disk.
At a certain position $(R,z)$, 
the local radiation field is determined by the three rays, 
the stellar radiation that 
penetrates the disk from the inner edge to $(R,z)$ 
and the ISRF that propagates vertically 
from above and below the disk toward $(R,z)$.



%As mentioned in Section \ref{sec:shielding}, 
%the local FUV flux $\chico$ at $(R,z)$, which determines 
%both $\nCIana(R,z)$ and $\nCOana(R,z)$,
%depends on the spatial integrals of 
%$\nCIana(R,z)$ and $\nCOana(R,z)$ along the three rays shown in Figure \ref{fig:obs}.
%Thus, $\chico(R,z)$, $\nCIana(R,z)$, and $\nCOana(R,z)$ should be determined consistently.

Applying Equation (\ref{chico_shield}) into the cases with the disk geometry, 
%We here show the expressions of $\chico$
%Considering the shielding effects of the three rays as they propagate 
%through the disk, the normalized FUV flux $\chico$ at $(R,z)$ 
one obtains  $\chico(R,z)$ as follows:
\begin{eqnarray}
    \chico(R,z) &=& \chi_\mathrm{co,star} \left( \frac{\sqrt{R^2+z^2}}{r_0} \right)^{-2} 
    f_{\mathrm{shield},r}(R,z) \nonumber  \\
    &+& \frac{\chi_\mathrm{co,ISRF}}{2} f_{\mathrm{shield},+z}(R,z) \nonumber  \\
    &+& \frac{\chi_\mathrm{co,ISRF}}{2} f_{\mathrm{shield},-z}(R,z), 
    \label{chicoRz}
\end{eqnarray}
%\begin{eqnarray}
%    \chico(R,z) &=& \chi_\mathrm{co,star} \left( \frac{r}{R_*} \right)^{-2} 
%    f_\mathrm{C^0}(N_r(\mathrm{C}^0)) 
%    f_\mathrm{CO}(N_r(\mathrm{CO})) \nonumber  \\
%    &+& \frac{\chi_\mathrm{co,ISRF}}{2} f_\mathrm{C^0}(N_+(\mathrm{C}^0)) 
%    f_\mathrm{CO}(N_+(\mathrm{CO})) \nonumber  \\
%    &+& \frac{\chi_\mathrm{co,ISRF}}{2} f_\mathrm{C^0}(N_-(\mathrm{C}^0)) 
%    f_\mathrm{CO}(N_-(\mathrm{CO})),
%    \label{chicoRz}
%\end{eqnarray}
where $\chi_\mathrm{co,star}$ and $\chi_\mathrm{co,ISRF}$ are 
the normalized unattenuated 
FUV fluxes of the stellar radiation at a reference radius of $r_0=50~\mathrm{au}$ and 
the ISRF, respectively. 
The geometrical dilution is considered in the first term on the right-hand side of Equation (\ref{chicoRz}).
A shielding factor is defined for each of the three rays;
$f_{\mathrm{shield},r}$ is the shielding factor of the stellar radiation, and 
$f_{\mathrm{shield},+z}$ ($f_{\mathrm{shield},-z}$) 
is the shielding factor of the ISRF coming from below (above) the disk.
They are computed using Equation (\ref{fshield}), but the column densities 
$\NCO_\mathrm{ana}$ and $\NCI_\mathrm{ana}$ 
are computed by integrating 
$\nCO_\mathrm{ana}$ and $\nCI_\mathrm{ana}$ 
along the corresponding rays, respectively.

The normalized FUV flux $\chioh$ is given by 
\begin{eqnarray}
    \chi_\mathrm{oh}(R,z) &=& \chi_\mathrm{oh,star} \left( \frac{\sqrt{R^2+z^2}}{r_0} \right)^{-2},
    \label{chiohRz}
\end{eqnarray}
where $\chi_\mathrm{oh,star}$ 
is the normalized unattenuated 
FUV flux of the stellar radiation at $r_0=50~\mathrm{au}$. 
Shielding effects are not effective in $\chioh$ 
as mentioned in Section \ref{sec:fuv}

%We here present the procedure to obtain the spatial distributions of $\nCIana$ and 
%$\nCOana$.
%First, the shielding factors $f_{\mathrm{shield},r}(R,z)$, $f_{\mathrm{shield},+z}(R,z)$,
%and  $f_{\mathrm{shield},-z}(R,z)$ are set to unity throughout the disk.
%Substituting $\chico(R,z)$, $\chioh(R,z)$, $\nC(R,z)$, and $Z$ 
%into Equations (\ref{nCIana}) and (\ref{nCOana}), $\nCIana(R,z)$ and $\nCOana(R,z)$ are
%derived.
%The shielding factors are revised using $\nCIana(R,z)$ and $\nCOana(R,z)$ 
%from Equation (\ref{chicoRz}) to calculate $\chico(R,z)$, which 

From Equations  (\ref{nCIana}), (\ref{nCOana}), (\ref{chicoRz}), and (\ref{chiohRz}),
$\nCIana(R,z)$, $\nCOana(R,z)$, and $\chico(R,z)$
are determined consistently in an iterative manner.


%The vertical and mid-plane column densities of carbon nuclei are given by 
%\begin{equation}
%    N_\mathrm{C,\perp}(R) = \int_{-\infty}^{\infty} n_\mathrm{C}(R,z) dz
%\end{equation}
%and
%\begin{equation}
%    N_\mathrm{C,mid}(R) = \int_{R_\mathrm{in}}^R n_\mathrm{C}(R,z=0)dR,
%\end{equation}
%respectively.


%A method similar to that proposed in \citet{Kral2016} is adopted.
%Figure \ref{fig:obs} displays the schematic picture of our setting.
%At a fixed radius $R_i$, we consider a uniform slab extended along the $z$-axis with 
%$n_\mathrm{C,mid} = n_\mathrm{C}(R_i,z=0)$.
%{\color{blue}
%At a given $N_\mathrm{C,\perp}(R_i)$, 
%the thickness of the slab at $R=R_i$ is $N_\mathrm{C,\perp}(R_i)/n_\mathrm{C,mid}$.
%}
%Although the dust extinction is negligible in debris disks, 
%the C$^0$ attenuation of the stellar radiation can be significant 
%\citep[Section \ref{sec:shielding} and ][]{Kral2019}.
%In the PDR calculation at $R=R_i$, 
%the radiation intensity incident at one edge of the vertical slab is 
%\begin{eqnarray}
%    F_\mathrm{inc}(R,\lambda) &=& \left( \frac{R_*}{R_i} \right)^{2} \int_{912\mathrm{\AA}}^\infty 
%    e^{-\sigma_\mathrm{C}(\lambda) N_\mathrm{C^0,mid}(R_i)}F_\mathrm{star}(\lambda) d\lambda 
%    \nonumber \\
%    & + & F_\mathrm{ISRF}(\lambda),
%    \label{stellarflux}
%\end{eqnarray}
%where the first and second terms in the right-hand side show 
%the stellar radiation flux and the ISRF illuminated from the vertical direction, respectively,
%$F_\mathrm{star}$ is the stellar radiation flux at the stellar surface, 
%and $F_\mathrm{ISRF}$ is the radiation flux of the ISRF.
%The cross section of the C$^0$ ionization, $\sigma_\mathrm{C}(\lambda)$, takes 
%a constant value of $\sigma_\mathrm{C}(\lambda) = a_\mathrm{C} = 1.777\times 10^{-17}~\mathrm{cm^2}$ 
%in $912\mathrm{\AA}\le \lambda \le 1100\mathrm{\AA}$ and 
%becomes zero in $\lambda > 1100~\mathrm{\AA}$ \citep{Heays2017}.
%In Equation (\ref{stellarflux}), 
%$N_\mathrm{C^0,mid}(R_i)$ is defined as
%\begin{equation}
%%    N_\mathrm{C^0,mid}(R_i) = \int_{R_\mathrm{in}}^{R_i} n_\mathrm{C,mid}(R)f_\mathrm{c}(\xi)dR,
%    N_\mathrm{C^0,mid}(R_i) = \int_{R_\mathrm{in}}^{R_i} n_\mathrm{C,mid}(R)f_\mathrm{c}(\xi)dR,
%    \label{NC0mid}
%\end{equation}
%where $f_\mathrm{c}=n(\mathrm{C}^0)_\mathrm{ana}/\nC$ (Equation (\ref{nCp})), 
%and $\xi$ is a function of $n_\mathrm{C,mid}$ and $\chico$ at each radius (Equation (\ref{xi1})).
%
%
%We should note that $N_\mathrm{C^0,mid}(R_i)$ depends on $\chico$ obtained from $F_\mathrm{inc}(R,\lambda)$ in 
%$R_\mathrm{in}\le R\le R_i$.
%Thus, $N_\mathrm{C^0,mid}(R_i)$ and $\chico$ are determined self-consistently so that 
%the following equation is satisfied; 
%\begin{equation}
%    \chi_\mathrm{co}(R_i) = 
%    \chi_\mathrm{co,star} \left( \frac{R_*}{R_i} \right)^{2}e^{- a_\mathrm{C} 
%    N_\mathrm{C^0,mid}(R_i) } + \chi_\mathrm{co,ISRF}.
%    \label{chico}
%\end{equation}
%Finally, with the obtained $N_\mathrm{C^0,mid}$, 
%the incident flux $F_\mathrm{inc}(\lambda)$ is derived from Equation (\ref{stellarflux}).
%
%%For simplicity, we only consider the contribution of C$^0$ 
%%to gas attenuation of the stellar radiation.
%%An important shielding of the stellar radiation is expected to be effective in 
%%the Lyman-Werner bands, which extend to wavelengths longer than the C$^0$ ionzation energy.
%%Instead of calculating the H$_2$ attenuation in $R_\mathrm{in}\le R\le R_i$, 
%%we remove the H$_2$ photo-dissociations owing to 
%%the Lyman-Werner bands with $\lambda_{ul}\ge 1100\mathrm{\AA}$
%
%The CO fraction predicted from Equation (\ref{fitting}) depends not only on $\chico$ but also on $\chioh$, whose 
%radial dependence is given by 
%\begin{eqnarray}
%    \chi_\mathrm{oh}(R_i) &=& \chi_\mathrm{oh,star} \left( \frac{R_*}{R_i} \right)^{2}. %+ \chi_\mathrm{oh,ISRF}.
%    \label{chioh}
%\end{eqnarray}
%Shielding effects are not effective in $\chioh(R_i)$ owing to long wavelengths 
%($1600~\mathrm{\AA}\le \lambda \le 1700~\mathrm{\AA}$).
%In addition, the ISRF does not contribute to $\chi_\mathrm{oh}$ 
%significantly because the ISRF spectrum rapidly decreases in the corresponding wavelength range.
%
%
%Once a disk model is given, 
%at $R_i$, the PDR calculation of the gas slab is conducted with 
%the radiation spectrum shown in Equation (\ref{stellarflux}), 
%the uniform density $n_\mathrm{C,mid}(R_i)$, and metallicity $Z$.
%We measure the chemical abundances at $\NC = N_\mathrm{C,\perp}(R_i)/2$, which corresponds to the mid-plane.
%
%We should note that we cannot take into account 
%the difference of the directions of the stellar radiation and ISRF shown in Figure \ref{fig:obs} in the Meudon PDR code.
%One of the edges of the gas slab is illuminated by both the stellar radiation and ISRF.
%This causes the stellar radiation to be attenuated vertically.
%However, this does not affect our results because the attenuation is more significant in the radial direction.
%

%the PDR calculations using the Meudon PDR code at several radii $R_i$.
%The number of $R_i$ is determined to resolve the spatial variation of $\nCp$, $\n(\mathrm{C})$, and $\nCO$.



%----------------------------------------------------------------------
\subsection{Comparison with Observations}\label{sec:observation}
%----------------------------------------------------------------------
%Our results show that the behavior of the CO fraction depends on 
%whether $\fracdg$ exceeds  $\fracdgfid$ 
%although it is possible that a sufficient amount of H$_2$ is produced to 
%accelerate the CO formation for large $Z$, $\nH$, and $\chi$ even when $\fracdg\le\fracdgfid$.
%Referring to Equation (\ref{fdg_tau_Nc}), the dust-to-gas mass ratio is 
%calculated as follows:
%{\color{blue}
%\begin{equation}
%    \frac{\fracdg}{\fracdgfid} = 
%    \left( \frac{{\cal N}_\mathrm{C}}{2\times 10^{17}~\psc} \right)^{-1}
%    \left( \frac{\taumid}{10^{-2}} \right).
%    \label{fdg_determ}
%\end{equation}
%}
%%Although $\tau$ and $\NC$ are both estimated from observable quantities, 
%%we should note that the observable quantities 
%%do not always indicate the amounts of material integrated along the same direction.
%%{\color{blue}
%%The mid-plane optical depth is estimated by dividing the fraction of the disk luminosity 
%%to the stellar luminosity by the aspect ratio.
%%}
%%Since $\tau$ is often derived by the fraction of the disk luminosity to stellar luminosity,
%%it is a measure of the amount of dust grains 
%%integrated over the disk radius.
%%By contrast, $\NC$ is estimated from $N_\mathrm{C,obs} = N(\mathrm{C}^0)_\mathrm{obs} + 
%%N(\mathrm{C}^+)_\mathrm{obs}  + N(\mathrm{CO})_\mathrm{obs}$, 
%%which indicate the amount of carbon nuclei along the line of sight. 
%%Either $\tau$ or $N_\mathrm{C,obs}$ needs to be corrected
%%so that the direction of integration is the same as that of the other by using a disk model.
%%Thus, $\fracdg/\fracdgfid$ obtained by substituting $N_\mathrm{C,obs}$ 
%%into $\NC$ would give an upper limit of the actual value since $\NC$ 
%%is larger than $N_\mathrm{C,obs}$.

%
%The most important C-bearing species are C$^+$, C$^0$, and CO.
%The C$^+$ and C$^0$ fractions are predicted 
%{\color{blue}
%from Equation (\ref{nCIana}) and 
%$\nCp_\mathrm{ana} = \nC-n(\mathrm{C}^0)_\mathrm{ana}$.
%}
%We here summarize the analytical formula of the 
%CO number densities.

%%First, the C$^+$ and C$^0$ fractions are considered. 
%%In most debris disks, $n(\mathrm{C}^0) + n(\mathrm{C}^+) \sim \nC$ is satisfied 
%%since $\nCO < n(\mathrm{C}^0), n(\mathrm{C}^+)$. 
%%Some debris disks, such as HD32297, the CO mass is much larger than the C$^0$ mass \citep{Cataldi2020}.
%%Their number densities are determined by the balance between photo-ionization of 
%%C$^0$ and radiative recombination of C$^+$ as follows:
%%\begin{equation}
%%   \alpha_\mathrm{C} \chico F_\mathrm{H} n(\mathrm{C^0})
%%        \sim  k_\mathrm{rec} n(\mathrm{C^+}) n({e}^{-}),
%%\label{ionC}
%%\end{equation}
%%where 
%%$\alpha_\mathrm{C}=1.777\times 10^{-17}~\mathrm{cm}^2$ \citep{Heays2017} 
%%\footnote{
%%    Note that the Meudon code takes into account the frequency-dependent radiative flux 
%%    to calculate the photo-ionization rate although $\alpha_\mathrm{C}\chico F_\mathrm{F}$ 
%%    gives a good estimate.
%%}
%%is the photo-ionization cross section
%%\citep{vanDishoeck2006}
%%and $k_\mathrm{rec}(T)=1.7\times 10^{-11}~\mathrm{cm^3~s^{-1}}(T/100~\mathrm{K})^{-0.82}$
%%\footnote{
%%The expression of $k_\mathrm{rec}$ is a fitting formula
%%of the recombination coefficient taking into account 
%%radiative recombination \citep{Badnell2006} and 
%%di-electronic recombination \citep{Badnell2003}. 
%%in 
%%$10~\mathrm{K}\le T\le 10^3~\mathrm{K}$.
%%}
%%is the recombination coefficient.
%%%Figures \ref{fig:CI_CII}c and \ref{fig:CI_CII}d show that $n({e}^-)$ 
%%%is approximately equal to $n(\mathrm{C^+})$ for most cases. 
%%%For the A5V cases with $\nC>10^2~\pcc$, however, $n(e^-)$ is larger than $\nCp$. 
%%%In such a high density, the abundance of C$^+$ 
%%%becomes smaller than that of Si$^+$, and electrons are mainly supplied by 
%%%photo-ionization of Si, which has a lower ionization energy of 8~eV than carbon atom.
%%%With $n(e^-) \sim \nCp$,
%%%Equation (\ref{ionC}) is rewritten as
%%%\begin{equation}
%%%    \frac{n(\mathrm{C^0})}{\nC} \sim f_\mathrm{c}(\xi) \equiv 1- \frac{1}{2\xi} \left[- 1+\sqrt{1+4\xi} \right],
%%%\label{ionCana}
%%%\end{equation}
%%From Equation (\ref{ionC}) with $n(e^-) \sim \nCp$, one obtains analytic formulae of the C$^0$ fraction given by 
%%\begin{equation}
%%        \frac{n(\mathrm{C}^0)_\mathrm{ana}}{\nC} = \frac{2\xi+1-\sqrt{1+4\xi}}{2\xi}
%%    \label{nCp}
%%\end{equation}
%%%\begin{equation}
%%%        \frac{n(\mathrm{C}^+)_\mathrm{ana}}{\nC} = \frac{ -1 + \sqrt{1+4\xi}}{2\xi}\;\;\;
%%%    \mathrm{and}\;\;
%%%%    \frac{n_\mathrm{ana}(\mathrm{C}^0)}{\nC} = 1 - \frac{n(\mathrm{C^+})}{\nC},
%%%    \frac{n(\mathrm{C}^0)_\mathrm{ana}}{\nC} = \frac{2\xi}{2\xi+1+\sqrt{1+4\xi}}
%%%    \label{nCp}
%%%\end{equation}
%%where $\nC = n(\mathrm{C^0}) + n(\mathrm{C}^+)$ is used, 
%%and $\xi$ is defined as 
%%\begin{equation}
%%   \xi \equiv \frac{k_\mathrm{rec}n_\mathrm{C}}
%%   {\alpha_\mathrm{C} \chico F_\mathrm{H}}
%%   = \chico^{-1}
%%   \left(\frac{\nC}{11~\mathrm{cm}^{-3}}\right)
%%    \left( \frac{T}{100~\mathrm{K}} \right)^{-0.82},
%%     \label{xi1}
%%\end{equation}
%%which indicates the ratio of the recombination to the ionization coefficients.
%%Equations (\ref{nCp}) and (\ref{xi1}) show that the C$^0$ fraction is independent of $\nH$ and $Z$ but
%%depend on $\nC$, $\chico$, and gas temperature  because hydrogen does not 
%%involved in the chemical balance between C$^0$ and C$^+$.
%%We confirmed that the C$^0$ fractions obtained from our results are consistent with Equation (\ref{nCp}) quantitatively.

%%Analytic formulae of the C$^+$ and C$^0$ fractions are given by 
%%\begin{equation}
%%    \frac{n(\mathrm{C}^+)}{\nC} = \frac{ -1 + \sqrt{1+4\xi}}{2\xi}\;\;\;
%%    \mathrm{and}\;\;
%%    \frac{n(\mathrm{C}^0)}{\nC} = 1 - \frac{n(\mathrm{C^+})}{\nC},
%%    \label{nCp}
%%\end{equation}
%%respectively,
%%where $\xi=\xi(T,\chico,\nC)$ is the ratio of the recombination to the ionization coefficients,
%%and its definition is given by Equation (\ref{xi1}).
%%The detailed derivation of Equation (\ref{nCp}) is presented in Appendix \ref{sec:CI_CII}.

%Our findings in Sections \ref{sec:result}, \ref{sec:thin}, and \ref{sec:shielding} show 
%that the CO fraction is predicted from Equation (\ref{fitting}).
%Equation (\ref{fitting}) enables us to derive $\nH$ required to produce a specific value of the CO fraction.
%In the high density limit, Equation (\ref{fitting}) can be further simplified.
%Since the second term in the right-hand side of Equation (\ref{fitting}) is negligible,
%Equation (\ref{fitting}) is solved for $\nH$ as follows:
%\begin{eqnarray}
%    n_\mathrm{H,req} &=& 6\times 10^7~\pcc~Z^{-0.4}\chi^{1.1} \nonumber \\
%    && \;\;\;\;
%    \times \left( \frac{\Ao}{ {\cal A}_\mathrm{O,ism}} \right)^{-0.56} 
%    \left( \frac{n(\mathrm{CO)}}{\nC} \right)^{0.56},
%    \label{nHconstrained}
%\end{eqnarray}
%which is valid when
%\begin{equation}
%    n_\mathrm{H,req} > 5.3\times 10^4~\pcc~Z^{-0.4}\chi^{1.1}.
%\end{equation}
%Equation (\ref{nHconstrained}) is rewritten as 
%\begin{eqnarray}
%    n_\mathrm{C,req} &=& 8\times 10^3~\pcc Z^{0.6}\chi^{1.1} \nonumber \\
%    && \;\;\;\;
%    \times \left( \frac{\Ao}{ {\cal A}_\mathrm{O,ism}} \right)^{-0.56} 
%    \left( \frac{n(\mathrm{CO})}{\nC} \right)^{0.56},
%    \label{nCreq}
%\end{eqnarray}
%which is the carbon nucleus number density required to reproduce a specific value of $\nCO/\nC$.
%From Equation (\ref{nCreq}), an important conclusion can be drawn that $n_\mathrm{C,req}$ increases with $Z$.
%In other words, at fixed $n_\mathrm{C}$ and $\chi$, the upper limit of the CO fraction is obtained at $Z=1$.

%
%In order to validate our analytic formulae (Equations (\ref{fitting}) and (\ref{nCp})) and 
%investigate the impact of uncertainty in the ER mechanism on the CO fractions (Section \ref{sec:ER}), 
%additional PDR calculations are conducted by taking into account 
%the gas structures of the debris disks around $\beta$ Pictoris and 49 Ceti.

%show that $\nCO$ and $n(\mathrm{C^0})$ are both roughly independent of $\NC$ 
%until the shielding effects become effective.
%from Equations (\ref{ionCana}) and (\ref{fitting}), one obtains the C/CO column density ratio, 
%\begin{eqnarray}
%    \frac{\mathrm{C}}{\mathrm{CO}} &\equiv& 
%    \frac{N(\mathrm{C}^0)}{\NCO^\mathrm{fit}} \nonumber \\
%    &=& 
%    \left(\frac{\Ao}{ {\cal A}_\mathrm{O,ism}} \right)^{-1} f_\mathrm{c}(\xi)
%    \left(
%        10^{-14} \eta^{1.8} + 6.0\times 10^{-11} \eta 
%    \right)^{-1},
%  \label{fittingN}
%\end{eqnarray}
%where the definition of $f_\mathrm{c}(\xi) = n(\mathrm{C^0})/\nC$ is given by Equation (\ref{ionCana}) 
%and $\eta = \nH Z^{0.4}\chi^{-1.1}$.





%If all $N(\mathrm{C}^+)$, $N(\mathrm{C}^0)$, and $N(\mathrm{CO})$ are 
%observationally constrained and the stellar spectrum is determined, 
%combining Equations (\ref{NCp}) and (\ref{fittingN}) can predict 
%the parameters of the gas component, $\nC$, $Z$, and $\nH$.
%The C$^0$ and CO gases are spatially resolved 
%and their emission distributions are spatially correlated \citep{Cataldi2018,Higuchi2019}.
%However, C$^+$ gas is not spatially resolved \citep{Roberge2013,Brandeker2016}.
%The C$^+$ spatial distribution in $\beta$ Pictoris 
%is investigated by resolving spectrally with {\it Herschel}/HIFI \citep{Cataldi2014}.
%the C$^+$ gas is extended beyond $\sim 150$~au where 
%there is almost no CO gas \citep{Dent2014}.
%
%To avoid uncertainty of the correlation between 
%the spatial distributions of C$^+$ and the neutral components (C$^0$, CO),  
%we do not consider the C$^+$ component in Sections \ref{sec:betaPic} and \ref{sec:49Ceti}
%where 
%our results are compared with the observational results of $\beta$ Pictoris and 49Ceti.
%We should note that $\fracdg/\fracdgfid$ is 
%overestimated in Sections \ref{sec:betaPic} and \ref{sec:49Ceti} 
%because $\NC$ is underestimated in Equation (\ref{fdg_determ}).
%

Our analytical model shown in Section \ref{sec:COdiskana} is 
applied to the debris disks around $\beta$ Pictoris and 49 Ceti
in Sections \ref{sec:betaPic} and \ref{sec:49Ceti}, respectively.

Although $\nCIana(R,z)$ and $\nCOana(R,z)$ are obtained throughout the disk,
additional PDR calculations are conducted at various radii along the mid-plane considering
the stellar radiation and vertical ISRF.
There are two reasons for this.
One is to verify whether $\nCIana$ and $\nCOana$ reproduce 
the results of the PDR calculations with $T_\mathrm{chem}=300~$K.
The other is to investigate how the H$_2$ formation accelerated 
by setting $T_\mathrm{chem}=10$~K affects the CO formation.
Section \ref{sec:pdrdisk} will describe how the PDR calculations 
are conducted taking into account the disk geometry.

%------------------------------------------------------------------------
\subsubsection{PDR Calculations Using Disk Geometry and Structure}
\label{sec:pdrdisk}
%\subsubsection{ \color{blue} PDR Calculations Using Disk Geometry and Structure}\label{sec:COdiskPDR}
%------------------------------------------------------------------------




In the PDR calculation at $(R=R_i,z=0)$, supposing that 
$I_\mathrm{star}(\lambda)$ is the stellar radiation intensity 
and $I_\mathrm{ISRF}(\lambda)$ is the ISRF intensity, 
the mean intensity $J(\lambda)$ at $(R_i,0)$ is given by 
\begin{eqnarray}
    J(\lambda) &=& 
        I_\mathrm{star}(\lambda)W( R_i)
        f_{\mathrm{shield},r}(R_i,0) \nonumber \\
        &+& I_\mathrm{ISRF}(\lambda)
        f_{\mathrm{shield},+z}(R_i,0)
    \label{radPDR}
\end{eqnarray}
for $\lambda \le 1100~$\AA, 
where $W(R_i)$ is the geometrical dilution 
factor at $R=R_i$, and 
we use the fact that $f_{\mathrm{shield},-z}(R_i,0) = f_{\mathrm{shield},+z}(R_i,0)$.
For the spectrum with $\lambda > 1100~$\AA, 
the shielding factors in Equation (\ref{radPDR}) are set to unity.


%\begin{equation}
%    F(\lambda) = \left\{
%    \begin{array}{ll}
%        F_\mathrm{star}(\lambda)\left( R_i/R_* \right)^{-2} 
%        f_{\mathrm{shield},r}(R_i,0) 
%        + F_\mathrm{ISRF}(\lambda)
%        f_{\mathrm{shield},+z}(R_i,0) 
%        & \mathrm{for\;\;\lambda \le 1100~\mathrm{\AA}} \\
%        F_\mathrm{star}(\lambda)\left( R_i/R_* \right)^{-2} + F_\mathrm{ISRF}(\lambda) & 
%        \mathrm{for\;\;\lambda > 1100~\mathrm{\AA}} \\
%    \end{array}
%    \right.,
%    \label{radPDR}
%\end{equation}


For each $R_i$, we perform the PDR calculation with 
$\nC(R_i,0)$, $Z$, $\fracdg/\fracdgfid$, and 
the radiation field shown in Equation (\ref{radPDR}). 
Since the geometrical dilution and shielding effects are already taken into 
account in the incident radiation, 
the chemical abundances at the optically thin limit ($N_\mathrm{C}=10^{14}~\psc$) 
are considered as those at the position $(R_i,0)$.



%\begin{eqnarray}
%    \chi_\mathrm{co,mid}(R_i) &=& \chi_\mathrm{co,star} \left( \frac{R_i}{R_*} \right)^{-2} e^{- \alpha_\mathrm{C} N_\mathrm{C,mid}(R_i) } \nonumber \\
%    &+& \chi_\mathrm{co,ISRF}e^{- \alpha_\mathrm{C} N_\mathrm{C,\perp}(R_i)/2},
%    \label{chico}
%\end{eqnarray}
%and 
%\begin{eqnarray}
%    \chi_\mathrm{oh,mid}(R_i) &=& \chi_\mathrm{oh,star} \left( \frac{R_i}{R_*} \right)^{-2} + \chi_\mathrm{oh,ISRF}.
%    \label{chioh}
%\end{eqnarray}










%The $Z$-dependence of the C/CO ratio appears in $f_\mathrm{c}(\xi)$.
%The behavior of Equation (\ref{fittingN}) is characterized by 
%the critical hydrogen nucleus density 
%\begin{equation}
%  n_\mathrm{H,cri} = 4.2\times 10^4~\pcc~Z^{-1} \chi \left( \frac{T}{100~\mathrm{K}} \right)^{0.82},
%\end{equation}
%where $n_\mathrm{H,cri} \equiv \nCcri Z$ from Equation (\ref{nCcri}). 
%For $\nH\ll n_\mathrm{H,cri}$, since $f_\mathrm{c}$ approaches $\xi/4$, one obtains 
%\begin{eqnarray}
%    \frac{\mathrm{C}}{\mathrm{CO}} &=& 
%    \left( \frac{\nH Z}{8\times 10^{4}~\pcc} \right)
%    \chi^{-1} \left( \frac{T}{100~\mathrm{K}} \right)^{-0.82}\nonumber \\
%    && 
%    \left(\frac{\Ao}{ {\cal A}_\mathrm{O,ism}} \right)^{-1} f_\mathrm{c}(\xi)
%    \left(
%        10^{-14} \eta^{1.8} + 6.0\times 10^{-11} \eta 
%    \right)^{-1}.
%  \label{fittingN1}
%\end{eqnarray}
%}
%For low densities, the C/CO ratio increases with $Z$ because the C$^0$ fraction increases.
%For $\nH\gg n_\mathrm{H,cri}$, Equation (\ref{fittingN}) becomes 
%\begin{equation}
%    \frac{\mathrm{C}}{\mathrm{CO}} = 
% \left( \frac{\mathrm{C/O}}{0.4} \right)  \left( f_\infty(\nH, \chi)^{-1/2} 
% + f_0(\nH,\chi)^{-1/2}  \right)^{2}.
%  \label{fittingN2}
%\end{equation}
%Equation (\ref{fittingN2}) shows that the C/CO ratio is independent of $Z$ for $\nH\gg n_\mathrm{H,cri}$.
%In other words, $\nH$ is determined directly from the C/CO ratio, regardless of $Z$.

%%If $N(\mathrm{C}^0)$ is larger than $N(\mathrm{C}^0)_\mathrm{shield}$,
%%the CO abundance is enhanced by C$^0$ attenuation of the FUV flux.
%
%In the comparison between our results and observational results, 
%we approximate the whole debris disk as a plane-parallel gas slab with a uniform density.
%%Although we cannot evaluate how much uncertainty this approximation brings 
%%at this moment, we believe that the constrained parameters capture 
%%the global properties of the debris disk.
%
%Even when the global structure of debris disks is considered,
%we should emphasize that Equation (\ref{fitting}) is applicable to 
%debris disks by using local $\nH$, $\chi$, and $Z$ as long as 
%the CO shielding effects are not important.


%------------------------------------
\subsubsection{$\beta$ Pictoris}\label{sec:betaPic}
%------------------------------------

%{\color{blue}
%To construct the disk structure, 
%a given parameter is 
%the carbon nucleus column density integrated from $R=R_\mathrm{in}$ to 
%$R_\mathrm{out}$ along the mid-plane, which is denoted by ${\cal N}_\mathrm{C}$.
%In addition, we assume the radial dependence of the vertical carbon nucleus
%surface density ${\cal N}_\mathrm{C\perp}(R)$ and gas temperature $T(R)$, 
%where we assume that the gas temperature is independent of $z$.
%The carbon nucleus number density at the mid-plane $n_\mathrm{C,mid}(R) \equiv 
%n_\mathrm{C}(R,0)$ 
%}

    On the basis of the disk models shown in \citet{Cataldi2018}, 
    we here consider a ring with a uniform carbon nucleus vertical column density.
    With the C$^0$ and C$^+$ line data 
    \citep{Cataldi2014,Cataldi2018},
    \citet{Cataldi2018} derive the best fit parameters 
    of $R_\mathrm{in}=50~\mathrm{au}$ 
    and $R_\mathrm{out}=120~$au. 
    The stellar parameters of the central star 
    adopted here are the same 
    as those of the A5V star shown in Section \ref{sec:fuv}.
    The scale height is 
    $h(R) = c_\mathrm{s}(R)/\Omega(R)$, 
    where $c_\mathrm{s}$ is the sound speed with 
    the constant temperature 
    $T=75~\mathrm{K}$ and $\Omega(R) = \sqrt{G1.75M_\odot/R^3}$.

    
%    They assume that the gas is secondary origin and $Z$ is as high as $10^3$.
%    Since we consider the case with $Z=1$, the scale height increases due to smaller mean molecular weight.

    Since ${\cal N}_\mathrm{C}\sim 10^{17}~\psc$ for $\beta$ Pictoris
    (Section \ref{sec:modelparameter}), 
    an average mid-plane carbon nucleus number density 
    is estimated to be 
    $\sim {\cal N}_\mathrm{C}/
    (R_\mathrm{out}-R_\mathrm{in})\sim 100~\pcc$.
    In order to investigate how the CO fractions depend on the mid-plane density,  
    we consider two different values of
    $n_\mathrm{C,mid0}=\nC(R_\mathrm{in},0)$ at the inner edge,
    $n_\mathrm{C,mid0}=75~\pcc$ (low-density model) 
    and $n_\mathrm{C,mid0}=190~\pcc$ (high-density model) as listed in Table \ref{tab:betaPic}.
    
    \begin{table}
    \begin{center}
    \begin{tabular}{|l|c|c|}
        \hline
        & low-density & high-density \\
        \hline
        $n_\mathrm{C,mid0}$ [$\pcc$]$^{(1)}$ & 75 & 190  \\
        \hline
        ${\cal N}_\mathrm{C}$ [$\psc$]$^{(2)}$ & $4\times 10^{16}$ & $10^{17}$ \\
        \hline
        ${\cal N}_\mathrm{C\perp}$ [$\psc$]$^{(3)}$ & $1.8\times 10^{16}$ & $4.4\times 10^{16}$ \\
        \hline
        $f_\mathrm{dg}/f_\mathrm{dg,fid}$ & 5.0 & 2.0 \\
        \hline
    \end{tabular}
    \end{center}
    \caption{
    List of the model parameters for $\beta$ Pictoris.
    $^{(1)}$The mid-plane carbon nucleus number densities at $R=50~$au. 
    $^{(2)}$The mid-plane carbon nucleus column densities integrated from $R_\mathrm{in}$ to $R_\mathrm{out}$.
    $^{(3)}$The vertical carbon nucleus number densities with $Z=1$. 
    }
    \label{tab:betaPic}
    \end{table}

    The radial distribution of the mid-plane density is given by 
    \begin{equation}
        n_\mathrm{C}(R,0) = 75~\mathrm{cm}^{-3} 
        \left( \frac{ n_\mathrm{C,mid0}}{75~\mathrm{\pcc}} \right) 
        \left( \frac{R}{50~\mathrm{au}} \right)^{-3/2}.
        \label{nCbeta}
    \end{equation}
    where the mean molecular weight $\mu$ 
    is assumed to be spatially constant.
    
    
    
    The mid-plane column densities ${\cal N}_\mathrm{C}$ integrated from $R=R_\mathrm{in}$ to 
    $R_\mathrm{out}$ for both the models are 
    shown in Table \ref{tab:betaPic}.
%    In order to investigate how the CO fractions depend on the amount of gas,  
%    we consider two different values of ${\cal N}_\mathrm{C}$,
%    ${\cal N}_\mathrm{C,low}=4\times 10^{16}~\psc$ (low-${\cal N}_\mathrm{C}$ model) 
%    and ${\cal N}_\mathrm{C,high}=10^{17}~\psc$ 
%    (high-${\cal N}_\mathrm{C}$ model).
    They correspond to the minimum and maximum values of the 
    observed C$^0$ column densities multiplied by two 
    \citep{Higuchi2017,Cataldi2018}, where 
    the factor two comes from the contribution from C$^+$.
    
     
     
%     It is noted that $n_\mathrm{C}(R,0)$ is independent of the absolute values of 
%     ${\cal N}_\mathrm{C\perp}$, $T$ and $Z$ as long as their radial dependence is fixed.
     The vertical column density ${\cal N}_\mathrm{C\perp}$ is 
     determined by multiplying $n_\mathrm{C}(R,0)$ by
     the scale height, which depends on 
     $T$ and $\mu$. Although $\mu$ depends 
     on the chemical state of the species, for simplicity, 
     we assume that all gases are neutral atoms, and 
     $\mu$ is given by $
     1.27(1 + 6.6\times 10^{-3}Z)/
     (1 + 5.5\times 10^{-4}Z)$.
     Even when the gas is fully molecular, 
     our results do not change significantly because 
     $\mu$ will be about a factor of two larger and 
     the vertical column density decreases by about 30\%.
     
%     When most gases are molecular, $\mu$
%     is underestimated in this paper, and 
%     thus ${\cal N}_\mathrm{C\perp}$ is 
%     overestimated for a given mid-plane density.
%     The vertical surface densities at $Z=1$ are listed in Table \ref{tab:betaPic}.
    
%    we consider two different vertical column densities, 
%    $N_\mathrm{C0,low}=1.8\times 10^{16}C(Z)~\psc$ (low-$\NC$ model) 
%    and $N_\mathrm{C0,high}=4.0\times 10^{16}C(Z)~\psc$ 
%    (high-$\NC$ model),
%    where $C(Z)=\sqrt{\mu_\mathrm{C}(Z=1)/\mu_\mathrm{C}(Z)}$ corresponds to 
%    the scale height divided by that at $Z=1$\footnote{
%       In $C(Z)$, we neglect the temperature dependence of the scale height, which 
%       is proportional to $\sqrt{T/\mu_\mathrm{C}(Z)}$ because 
%       comparison between Figure \ref{fig:plane} and Equation (\ref{muC}) shows that 
%       $\mu_\mathrm{C}(Z)$ has a stronger dependence on $Z$ than $T$.
%    },
%    and it is required for $N_\mathrm{C,mid}$ to be independent of $Z$.
%    $N_\mathrm{C0,low}$ and $N_\mathrm{C0,high}$ are determined so that the corresponding $N_\mathrm{C,mid}$ 
%    are equal to the minimum and maximum values of the observed $N_\mathrm{mid}(\mathrm{C}^0)$ 
%    multiplied two, respectively \citep{Higuchi2017,Cataldi2018}, where 
%    the factor two comes from the contribution from C$^+$.

Referring to Equation (\ref{fdg_tau_Nc}), 
the dust-to-gas mass ratio is expressed in terms of 
${\cal N}_\mathrm{C}$ and $\tau_\mathrm{mid}$
as follows:
\begin{equation}
    \frac{\fracdg}{\fracdgfid} = 
   \left( \frac{{\cal N}_\mathrm{C}}{2\times 10^{17}~\psc} \right)^{-1}
    \left( \frac{\taumid}{10^{-2}} \right).
    \label{fdg_determ}
\end{equation}
From Equation (\ref{fdg_determ}), 
the dust-to-gas mass ratios are 
given by $\fracdg/\fracdgfid = 5.0$ for the 
low-density model and $2.0$ for the  high-density model, where  $\taumid=10^{-2}$ is used (Table \ref{tab:betaPic}).

%    The mid-plane density is given by 
%    \begin{equation}
%        n_\mathrm{C,mid}(R) = 80~\mathrm{cm}^{-3} \left( \frac{N_\mathrm{C0}}{N_\mathrm{C0,low}} \right) \left( \frac{R}{50~\mathrm{au}} \right)^{-3/2},
%        \label{nCbeta}
%    \end{equation}
%    which is comparable to $\nCfid$ at the inner edge (Section \ref{sec:modelparameter}).
%    Equating the mid-plane density and the required density to produce a given CO fraction (Equation \ref{nCreq}), 
%    the CO fraction is expected to be 
%    \begin{eqnarray}
%        \left(\frac{\nCO}{\nC}\right)_\mathrm{\beta Pic} &=& 8.6\times 10^{-5}\left( \frac{\NC}{N_\mathrm{C,low}} \right)^{1.8}
%        \left( \frac{R}{50~\mathrm{au}} \right)^{-0.7} \nonumber \\
%        & & \hspace{5mm} \times Z^{-1.1} \left( \frac{\chico}{2} \right)^{-1} \left( \frac{ {\cal A}_\mathrm{O}}{ 
%        {\cal A}_\mathrm{O,ism}} \right),
%        \label{nCOpred_beta}
%     \end{eqnarray}
%     where Equation (\ref{chioh}) is used.


Using the analytical model 
presented in Section \ref{sec:COdiskana}, 
the spatial distribution of $\chico$ is obtained.
Figures \ref{fig:betadisk}a and \ref{fig:betadisk}b show 
$\chico$ at the mid-plane as a fuction of $R$. 
$\chico$ is almost constant because the ISRF gives 
dominant contribution in most radii except near the 
inner edge.
Since the vertical column density for the high-density model 
is around $N(\mathrm{C}^0)_\mathrm{shld}$, 
$\chi_\mathrm{co}$ is slightly attenuated.
By contrast, $\chioh$ at the mid-plane decreases with $R$ 
owing to the geometrical dilution.

    Before showing the results of the PDR calculations, 
    we present the approximate formula of the predictions from the analytic formula.
    Equation (\ref{nHreq}) can be rewritten as 
    $n_\mathrm{C,req}>7Z^{0.6}\chi^{-1.1}~\pcc$.
    Since $\nC\sim 100~\pcc$ and $\chi\sim O(1)$, 
    Equation (\ref{nCreq}) is valid for $Z\lesssim 10^2$.
    Equating the mid-plane density and the density required 
    to produce a given 
    CO fraction (Equation (\ref{nCreq})), 
    the CO fraction is expected to be 
    \begin{eqnarray}
        \left(\frac{\nCO}{\nC}\right)_\mathrm{\beta Pic} &=& 
        10^{-4}\left( \frac{n_\mathrm{C,mid0}}
        {75~\mathrm{cm}^{-3}} \right)^{1.8}
        \left( \frac{R}{50~\mathrm{au}} \right)^{-0.7} \nonumber \\
        & & \hspace{5mm} \times Z^{-1.1} 
        \chico^{-1} \left( \frac{ {\cal A}_\mathrm{O}}{ 
        {\cal A}_\mathrm{O,ism}} \right),
        \label{nCOpred_beta}
     \end{eqnarray}
     where Equation (\ref{chiohRz}) is used for $\chioh$,
     and $\chico$ is the attenuated local value.
     Integrating $\nCO$ over the disk extent with 
     Equations  (\ref{nCbeta}) and (\ref{nCOpred_beta}), 
     one obtains the predicted mid-plane CO column density,
    \begin{eqnarray}
        {\cal N}(\mathrm{CO})_\mathrm{\beta Pic} &=& 
        3\times 10^{12}~\psc\left( \frac{n_\mathrm{C,mid0}}
        {75~\mathrm{cm}^{-3}} \right)^{2.8} \nonumber \\
        & & \hspace{5mm} \times Z^{-1.1} 
        \langle \chico^{-1} \rangle \left( \frac{ {\cal A}_\mathrm{O}}{ 
        {\cal A}_\mathrm{O,ism}} \right),
        \label{NCOpred_beta}
     \end{eqnarray}
     where $\langle \chico^{-1}\rangle$ is the average of $\chico^{-1}$ weighted by 
     $R^{-2.2}$ over the disk extent.



The PDR calculations are conducted at four radii, $50~$au, $75~$au, 100~au, and 120~au.


%    The mid-plane density is given by 
%    \begin{equation}
%    n_\mathrm{C,mid}(R) = 10^{2}~\pcc~\left( \frac{N_\mathrm{C0}}{N_\mathrm{C0,low}} \right)
%    \left( \frac{R}{50~\mathrm{au}} \right)^{-3/2}.
%    \end{equation}
%
%    The column densities along the mid-plane are $4.2\times 10^{16}~\psc$ for model low $\NC$ and 
%    $10^{17}~\psc$ for model high $\NC$, respectively.

%    The disk model is shown in Figure \ref{fig:betadisk}a.
%    The mid-plane column density of the carbon nuclei reaches $\sim 10^{17}~\psc$ 
%    at the ring outer edge, which is consistent with the 
%    observational facts that $N(\mathrm{C}^0)\sim 2-7\times 10^{16}~\psc$ and 
%    $N(\mathrm{C}^+)\sim 2-12\times 10^{16}~\psc$ (Section \ref{sec:modelparameter}).

%    Figures \ref{fig:betadisk}a and \ref{fig:betadisk}b display 
%    the radial profiles of the normalized UV fluxes at the slab edge for the low- and high-$\NC$ models, respectively
%    (Equations (\ref{chico}) and (\ref{chioh})).
%    The contribution of the stellar radiation to $\chico$ rapidly decreases with radius, 
%    and the ISRF dominates at most radii. 
%    By contrast, $\chioh$ decreases in proportion to $R^{-2}$ 
%    because the ISRF does not contribute to $\chioh$.
%    As a result, $\chi=\sqrt{\chico\chioh}$ is roughly propoertional to $R^{-2}$.

\begin{figure}[htpb]
        \centering
%        \includegraphics[width=9.0cm]{betaPic_disk.eps}
        \includegraphics[width=9.0cm]{Fig8.pdf}
        \caption{
             Results of the PDR calculations with $Z=1$ 
             for $\beta$ Pictoris.
             As the stellar radiation field, A5V star is adopted.
             The left and right columns show the results 
             of the low-density and high-density models, respectively.
             ({\it Top panels})  The normalized 
             UV fluxes (solid) $\chico$ and (dashed) $\chioh$. 
             The horizontal dotted lines correspond to $\chi_\mathrm{co,ISRF}$.
             ({\it Middle panels})
             The radial distributions of (red) $n(\mathrm{C}^0)/\nC$ and (blue) $\nCO/\nC$.
             The circles, triangles, and boxes correspond to the results 
             for $T_\mathrm{chem}=300~$K, $100~$K, and $10~$K, respectively.
             The C$^0$ and CO fractions predicted from 
             the analytic model (Section \ref{sec:COdiskana})
             are shown by the light red and blue solid lines, respectively.
             ({\it Bottom panels})
             The radial column densities of (red) atomic carbon and (blue) CO 
             integrated from the inner edge (50~pc) to $R$ 
             as a function of $R$.
%             The light blue dashed lines correspond to the predicted CO fractions at the mid-plane.
}
\label{fig:betadisk}
\end{figure}





%    The red lines in Figure \ref{fig:betadisk}a show the radial distributions of 
%    $n_\mathrm{C,req}$ with three different values of $\nCO/\nC$ under the 
%    radiation fields shown in Figure \ref{fig:betadisk}b.
%    The mid-plane density of the disk model is consistent with $n_\mathrm{C,req}$ 
%    at $\nCO/\nC=2\times 10^{-4}$.

    Firstly, the cases with $Z=1$ are considered.
    The results of the PDR calculations of the low-density
    and high-density models
    are shown in Figures \ref{fig:betadisk}c 
    and \ref{fig:betadisk}d, respectively.
     For $T_\mathrm{chem}=300$~K, the radial profiles 
    of $\nCO/\nC$ are consistent with the predictions from the analytic model (Section \ref{sec:COdiskana})
    for both the models 
    because H$_2$ formation is inefficient.
%    At the low $N_\mathrm{C}$ limit, the radial profiles 
%    of $\nCO/\nC$ are consistent with the predictions from Equation (\ref{fitting}) 
%    for both the models.
    Although $\fracdg/\fracdgfid$ is slightly larger than unity, 
    the effect of excited H$_2$ is limited 
    as shown in Figure \ref{fig:COfit}a.
%    The CO fractions at the mid-plane 
%    ($N_\mathrm{C} = N_\mathrm{C,\perp}(R)/2$) are enhanced by about factor of three than 
%    those at the low $\NC$ limit for both the models.
%    In order to examine whether the increases in the CO fractions can be explained only by the vertical attenuation of $\chico$,
%    we plot the predictions from Equation (\ref{fitting}) by replacing $\chi_\mathrm{co}$ with $\chi_\mathrm{co} \exp\left( -
%    a_\mathrm{C}N_\mathrm{C^0,\perp}(R_i)\right)$ taking into account vertical attenuation
%    \begin{eqnarray}
%        \chi_\mathrm{co,mid}(R_i) &=& 
%    \chi_\mathrm{co,star} \left( \frac{R_i}{R_*} \right)^{-2}e^{- \alpha_\mathrm{C} 
%    N_\mathrm{C^0,mid}(R_i) } \nonumber \\
%    & + & \chi_\mathrm{co,ISRF}e^{-\alpha_\mathrm{C}N_\mathrm{C^0,\perp}(R_i)}
%    \end{eqnarray}
%    in Figures \ref{fig:betadisk}c and \ref{fig:betadisk}d.
%    The predictions slightly underestimate the CO fractions at the mid-plane where 
%    H$_2$ is self-shielded, and the abundant H$_2$ activates other pathways to form CO.

%    We also investigate how the uncertainty of the parameters in the ER mechanism influences our results 
%    (Section \ref{sec:dustsurface}).
    When $T_\mathrm{chem}$
%    , which corresponds to the energy  barrier of chemisorption, 
    is decreased from $300$~K to $10~$K,
%    \footnote{
%     We fix $T_\mathrm{stick}$ to be 464~K because the gas temperatures are low enough for the H$_2$ formation rate in our models 
%     not to be limited by the sticking probability.
%    }, 
    the H$_2$ formation rate increases by about four 
    orders of magnitude at $T\sim 40~$K,
    making a large amount of hydrogen nuclei being molecular.
    As a result, H$^*_2$ increases the 
    CO fraction by about an order of magnitude 
    in both the models (Figures \ref{fig:betadisk}c and \ref{fig:betadisk}d).
     This is consistent with what we found in Figure \ref{fig:COfit}b.
%    An decrease in $T_\mathrm{chem}$ from 100~K to 10~K does not increase the amount of CO significantly
%    because most hydrogen is already molecular at $T_\mathrm{chem}=100~$K.

    The mid-plane column densities of C$^0$ and CO integrated 
    from $R=R_\mathrm{in}$ to $R$ are shown in 
    Figures \ref{fig:betadisk}e and \ref{fig:betadisk}f as a function of $R$.
    After a rapid increase in $\NCO$ near the inner edge, 
    they become almost constant outside from $R\sim 75~$au.
    This clearly shows that the CO mid-plane column densities are determined near the inner edge, and 
    CO in the outer disk does not contribute to the total column density.
    

\begin{figure}[htpb]
        \centering
%        \includegraphics[width=8cm]{betaPic_CO_C.eps}
        \includegraphics[width=8cm]{Fig9.pdf}
        \caption{
            Results of the PDR calculations for $\beta$ Pictoris.
            (a) The mid-plane column densities of (red) C$^+$, (green) C$^0$, and (blue) CO 
            along the mid-plane as a function of $T_\mathrm{chem}$
            for (circles) the low- and (triangles) high-density models with $Z=1$.
            The green and blue rectangle regions indicate the observational constraints on 
            the C$^0$ and CO column densities obtained from 
            \citet{Higuchi2017,Cataldi2018} and \citet{Higuchi2017}, respectively.
            (b) The mid-plane CO column densities as a function of $Z$  for 
            (red) the low- and (blue) high-density models.
            The circles and triangles show the results with $T_\mathrm{chem}=300$~K 
            and $T_\mathrm{chem}=10$~K, respectively.
            The red and blue thick dashed lines correspond to the predictions from the analytic model (Section \ref{sec:COdiskana}) with 
            $T_\mathrm{chem}=300~$K and 10~K, respectively.
            The blue rectangle region is the same as that in Panel (a).
            As a reference, ${\cal N}(\mathrm{CO}) \propto Z^{-1.1}$ is plotted 
            (Equation (\ref{nCOpred_beta})).
}
\label{fig:beta_comp}
\end{figure}



In order to compare our results with the observational results, 
the mid-plane column  densities of C$^+$, C$^0$, and CO 
integrated from $R=R_\mathrm{in}$ to $R_\mathrm{out}$ are shown as 
a function of $T_\mathrm{chem}$ in Figure \ref{fig:beta_comp}a.
The C$^0$ column densities of the low- and 
high-density
models are both consistent with the observational results, 
and do not depend on $T_\mathrm{chem}$.
    
The CO column densities increase as 
the mid-plane gas density increases and/or 
$T_\mathrm{chem}$  decreases.
The high-density models produce 
about an order of magnitude larger 
CO column densities than the low-density model for all $T_\mathrm{chem}$.
This comes from the fact that 
${\cal N}(\mathrm{CO}) \propto n_\mathrm{C,mid0}^{2.8}$ 
(Equation (\ref{NCOpred_beta})).
The contribution of the attenuation of $\chico$ to 
an increase in ${\cal N}(\mathrm{CO})$ is limited since 
the vertical column density ${\cal N}_\mathrm{C,\perp}$ 
is comparable to or smaller than 
$N(\mathrm{C^0})_\mathrm{shld}$ (Table \ref{tab:betaPic}).
The CO column density of the high-density model with $T_\mathrm{chem}=300$~K
does not reach the lower limit of the observational constraints \citep{Higuchi2017}.
%The CO column densities increase with decreasing $T_\mathrm{chem}$.
%The low-{\color{blue} density} model cannot produce 
%a sufficient amount of CO compared with the observed CO column densities
%even when $T_\mathrm{chem}$ is as low as 10~K.
%The CO column density of the fiducial model ($T_\mathrm{chem}=300$~K)
%does not reach the lower limit of the observational constraints \citep{Higuchi2017}.
If $T_\mathrm{chem}$ is lower than 100~K,
the high-density models yield the CO column density comparable to the observational constraints. 

The $Z$ dependence of ${\cal N}(\mathrm{CO})$
is shown in Figure \ref{fig:beta_comp}b.
For the fiducial models ($T_\mathrm{chem}=300$~K), 
 ${\cal N}(\mathrm{CO})$ decreases with increasing $Z$, 
roughly following 
the predictions from Equation (\ref{NCOpred_beta}).
When $T_\mathrm{chem}$ decreases to $10~$K, 
${\cal N}(\mathrm{CO})$
is increased by an order of magnitude, keeping 
the $Z$ dependence almost unchanged  (also see Figure \ref{fig:COfit}b).
Figure \ref{fig:beta_comp}b shows that metallicities larger than $Z=1$ cannot reproduce the observational CO column densities 
even when the H$_2$ formation is enhanced by decreasing $T_\mathrm{chem}$.




%\begin{figure}[htpb]
%        \centering
%        \includegraphics[width=7cm]{betaPic_49Ceti_CO_C.eps}
%%        \includegraphics[width=7cm]{plane-parallel-CO-49Ceti.eps}
%        \caption{
%            Dependence of $\mathrm{C/CO}=N_\mathrm{mid}(\mathrm{C}^0)/N_\mathrm{mid}(\mathrm{CO})$ on $T_\mathrm{chem}$ 
%            for the disk models of $\beta$ Pictoris and 49 Ceti (strong and weak UV models).
%            The red and blue regions indicate the observed values derived from \citet{Higuchi2017} ($\beta$ Pictoris) and 
%            \citet{Higuchi2019,Higuchi2020} (49 Ceti).
%	}
%\label{fig:beta49_comp}
%\end{figure}



%Since the debris disk is inclined close to edge-on \citep{Dent2014}, 
%the observed $N(\mathrm{C}^0)\sim 2\times 10^{16}~\psc$ is 
%substituted directly into $\NC$ of Equation (\ref{fdg_determ}).
% As a result, one obtains $\fracdg/\fracdgfid\sim 8$.
%This indicates that the CO over-production by excited H$_2$ 
%is expected for $\beta$ Pictoris on average.
%}
%Recently, \citet{Higuchi2017} detected the [CI] $^3P_1 - ^3P_0$
%line and derived $\mathrm{C/CO}\sim 69\pm 42$ \citep[also see][]{Cataldi2018}.
%
%Because the observed CO column density is around $3\times 10^{14}~\psc$ which is close to 
%$\NCO_\mathrm{shield}$,
%the CO self-shielding is marginally important but not significant.
%Thus, Equation (\ref{fittingN}) can be used to predict the required $\nH$.
%We found that in order for $\mathrm{C/CO}$ to be $69\pm 42$, $\nH$ needs to be large enough 
%for Equation (\ref{nHconstrained}) to be valid.
%Substituting $\mathrm{C/CO}=69\pm 42$, $\chico=1.4$, and $\chioh=2.4$ into 
%Equation (\ref{nHconstrained}), 
%one obtains 
%\begin{equation}
%    \nH = 6_{-2}^{+4}\times 10^6~\pcc~Z^{-0.4}.
%    \label{nHbetaPic}
%\end{equation}
%We confirm that 
%this number density is too low for the excited H$_2$ to proceed the CO formation 
%(the upper panels of Figure \ref{fig:COfit}).
%%From Equation (\ref{fittingN}),  the hydrogen density is derived to be 
%%$\nH\sim 5_{-2}^{+5} \times 10^5~$cm$^{-3}$ at $r=50$~au.
%
%The column density of hydrogen is constrained by several observations.
%\citet{Freudling1995} obtained an upper limit of the HI column density, $N_\mathrm{H}<
%2-5\times 10^{19}~\mathrm{cm}^{-2}$ from non-detection of the 21~cm line.
%More recently, \citet{Wilson2017} detected the Lyman $\alpha$ line and
%derived an HI column density of
%$(3-5)\times 10^{18}~\mathrm{cm}^{-2}$.
%\citet{Lecavelier2001} found that the H$_2$ column density should be smaller than
%$\sim 10^{18}~\mathrm{cm}^{-2}$ because H$_2$ absorption lines are not detected.
%Using a C$^0$ column density of $\sim (2.5\pm 0.7)\times 10^{16}~$cm$^{-2}$
%\citep{Higuchi2017}, $\NC/N_\mathrm{H}$ is estimated to be larger than
%$0.025$, which is much larger than ${\cal A}_\mathrm{C,ism}$.
%Recently \citet{Matra2017} constrained the hydrogen volume density by the line
%ratio of CO ($J=2-1$) and
%CO ($J=3-2$); their result suggests that $n(\mathrm{H}_2)$
%should be smaller than $\sim 10^4~\mathrm{cm}^{-3}$.
%The observed C/CO ratio cannot be explained by such a small value of $\nH$.
%The combination of our results and the observational constraints
%rules out the primordial origin ($Z=1$) of the gas in $\beta$ Pictoris.
%
%
%Even for $Z\sim 10^3$, which is the maximum metallicity to be achieved in debris disks, 
%the required $\nH$ is as high as $4_{-1.5}^{+2}\times 10^5~\pcc$.
%%Since $\nH$ is larger than $n_\mathrm{H,cri} \sim 3\times 10^4 Z^{-1}~\pcc$, 
%%the C$^0$ fraction is almost unity.
%This suggests that higher metallicities also cannot bring the required 
%$\nH$ down to the observational constraint $\sim 10^4~\pcc$.
%
%From the point of view of the spatial width of the gas slab $L\sim N(\mathrm{C}^0)/\nC$, 
%$L$ may be comparable to or larger than the scale hight of the disks so that $L$ 
%should not be much larger or much smaller than $R$.
%From Equation (\ref{nHbetaPic}), $L$ is given by 
%\begin{equation}
%    L \sim 2_{-1}^{+1} Z^{-0.6}~\mathrm{au},
%    \label{LbetaPic}
%\end{equation}
%where we use $N(\mathrm{C^0})\sim 2\times 10^{16}~\psc$.
%The required spatial width of the gas slab decreases with increasing $Z$, and 
%becomes $L\sim 0.03~\mathrm{au}\ll R$ for $Z=10^3$.
%Models with higher metallicities require unphysically 
%small spatial extents of the gas slabs because the required $\nC$ increases with $Z$
%(Equation \ref{nCconstrained}).
%
%
%
%From the above discussion, we conclude that 
%steady-state chemical reactions alone cannot explain 
%the observed C/CO ratio with any metallicities.
%Temporal supply of CO, for instance  
%through photo-dissociation of CO$_2$ caused by outgassing from dust grains, 
%might be important.

%------------------------------------
\subsubsection{49 Ceti}\label{sec:49Ceti}
%------------------------------------

    Following \citet{Hughes2017} and \citet{Higuchi2019},
    we adopt the disk model which has a power-law distribution with 
    the inner edge at $R_\mathrm{in}=20~\mathrm{au}$
    and with a exponential cut-off at $R=R_\mathrm{c}$, 
    \begin{equation}
        {\cal N}_\mathrm{C\perp}(R) \propto 
        \left( \frac{R}{R_\mathrm{c}} \right)^{-\gamma} 
        \exp\left[ - \left( \frac{R}{R_\mathrm{c}} \right)^{2-\gamma} \right],
        \label{49disk}
    \end{equation}
    where $R_\mathrm{in}=20~$au, $\gamma=-0.5$, 
    and $R_\mathrm{c}=140~$au are the 
    best fit parameters.
    %, and $N_\mathrm{C0}$ corresponds a typical vertical column density of the disk. 
    The outer edge of the disk $R_\mathrm{out}$ is at infinity.
    The scale height is $h(R) = c_\mathrm{s}(R)/\Omega$, 
    where $c_\mathrm{s}$ is the sound speed with
    $T=23~\mathrm{K}(R/100~\mathrm{au})^{-1/2}$ and
    $\Omega = \sqrt{GM_\mathrm{star}/R^3}$,
    where $M_\mathrm{star}$ is  the central star mass
    \citep[][also see Table \ref{tab:49Ceti}]{Hughes2018}.
    
    As discussed in Section \ref{sec:modelparameter}, a typical mid-plane number density of carbon nuclei 
    is given by $\sim 10^3~\pcc$, where 
    ${\cal N}_\mathrm{C} \sim 10^{18}~\psc$ is divided by the disk extent 
    $R_\mathrm{c} - R_\mathrm{in}$.
    We consider two different values of the mid-plane carbon nucleus 
    density at $R=R_\mathrm{c}/2$, 
    $n_\mathrm{C,mid0}=4.9\times 
    10^2~\pcc$ (low-density model) 
    and $n_\mathrm{C,mid0}=2.0\times 10^3~\pcc$ (high-density model).
    
    
    \begin{table}
    \begin{center}
    \begin{tabular}{|l|c|c|}
        \hline
        & low-density & high-density \\
        \hline
        $n_\mathrm{C,mid0}$ [$\pcc$]$^{(1)}$ & $4.9\times 10^2$ & 
        $2.0\times 10^3$ \\
        \hline
        ${\cal N}_\mathrm{C}$ [$\psc$]$^{(2)}$ & $8.4\times 10^{17}$ & $3.3\times 10^{18}$ \\
        \hline
        ${\cal N}_\mathrm{C\perp}$ [$\psc$]$^{(3)}$ & $8.2 \times 10^{16}$ & $3.5\times 10^{17}$ \\
        \hline
        $f_\mathrm{dg}/f_\mathrm{dg,fid}$ & 0.24 & 0.06 \\
        \hline
    \end{tabular}
    \end{center}
    \caption{
    List of the model parameters for 49 Ceti.
    $^{(1)}$The mid-plane carbon nucleus number densities at $R=R_\mathrm{c}/2=60~$au. 
    $^{(2)}$The mid-plane carbon nucleus column densities integrated from the inner edge to infinity.
    $^{(3)}$The vertical carbon nucleus number densities with $Z=1$ at $R=60~$au.
%    outer edge of the disk.
    }
    \label{tab:49Ceti_num}
    \end{table}
    
    
    
   
   The mid-plane density is given by 
     \begin{eqnarray}
        n_\mathrm{C}(R,0) &=& 5.6\times 10^2 ~\mathrm{cm}^{-3} 
        \left( \frac{n_\mathrm{C,mid0}}{4.9\times 10^2~\pcc} \right) 
        \nonumber \\
        &  & \hspace{2mm} 
        \times \left( \frac{R}{60~\mathrm{au}} \right)^{-3/4} \exp\left[ - \left( \frac{R}{140~\mathrm{au}} \right)^{2.5} \right],
        \label{nCmid49}
    \end{eqnarray}
    where $\mu$ is assumed to be spatially constant.
    The corresponding mid-plane column densities are listed in Table \ref{tab:49Ceti_num}.
    The high-density model gives the mid-plane column density comparable to or 
    slightly larger than that predicted in \citet{Higuchi2019}.
    As in the case of $\beta$ Pictoris, the vertical column densities 
    are determined by assuming all the gas is neutral atomic, and 
    the values with $Z=1$ at $R=60~$au are 
    listed in Table \ref{tab:49Ceti_num}.
    


    Substituting ${\cal N}_\mathrm{C}$ into 
    Equation (\ref{fdg_determ}) gives 
    $\fracdg/\fracdgfid = 0.24$ for the low-density models and $0.06$ for the high-density models, where 
    $\taumid=10^{-2}$ is used. 
    Since $\fracdg/\fracdgfid$ is much lower than unity, 
    the acceleration of the CO formation owing to H$^*_2$
    is not effective (Section \ref{sec:CO1e17}).


%    The mid-plane density is given by 
%    \begin{eqnarray}
%    n_\mathrm{C,mid}(R) &=& 2\times 10^{3}~\pcc~\left( \frac{N_\mathrm{C0}}{N_\mathrm{C0,low}} \right)
%    \left( \frac{R}{20~\mathrm{au}} \right)^{-3/4} \nonumber \\
%     && \;\;\;\times \exp\left[ -\left( \frac{R}{120~\mathrm{au}} \right)^{2.5} \right].
%    \end{eqnarray}



%    \begin{figure}[htpb]
%        \centering
%        \includegraphics[width=8cm]{stellar_spec.eps}
%        \caption{
%            \color{blue}
%            Spectral Energy Density of models (red) SSUV and (blue) WSUV.
%            The distance to 49~Ceti is assumed to be 61~pc.
%        }
%        \label{fig:stellarspec}
%    \end{figure}

    \begin{table}[htpb]
        \centering
        \begin{tabular}{|c|c|c|}
            \hline
            & SSUV & WSUV \\
            \hline
        $M_\mathrm{star}~[M_\odot]^{(1)}$ & $2.1$ & $2.0$ \\
            \hline
            $T_\mathrm{eff}~[\mathrm{K}]^{(2)}$ & $10,000$ & $9,000$ \\
            \hline
            $\log_{10} g^{(3)}$ & $4.5$ & $4.3$ \\
            \hline
        \end{tabular}
        \caption{
            $^{(1)}$ Stellar mass. $^{(2)}$ Effective temperature.  $^{(3)}$ Surface gravity.
            The stellar models of SSUV and WSUV 
            are taken from \citet{Hughes2008} and \citet{Roberge2013}, respectively.
            In both models, the metallicities [M/H] are assumed to be the solar one.
        }
        \label{tab:49Ceti}
    \end{table}

Different stellar parameters are used in the literature 
\citep{Chen2006,Hughes2008,Montesinos2009,Roberge2013}.
Among the different stellar models, 
the two extreme stellar models that provide the smallest and largest 
$\chi = \sqrt{\chi_\mathrm{co}\chi_\mathrm{oh}}$ are considered.
The stellar parameters of these models are summarized in Table \ref{tab:49Ceti}.
The strong and weak stellar UV models are called SSUV and WSUV, respectively.
%As shown in Figure \ref{fig:stellarspec}, the spectral energy distributions (SEDs) of these models
%are almost identical for wavelengths larger than that of the Balmer jump ($\sim 3600~\mathrm{\AA}$) since 
%the SEDs are well constrained by observations.
%By contrast, the two models differ in the SEDs by about an order of magnitude in the shorter wavelengths.

\begin{figure*}[htpb]
        \centering
%        \includegraphics[width=18.0cm]{49Ceti_disk.eps}
        \includegraphics[width=18.0cm]{Fig10.pdf}
        \caption{
             The same as Figure \ref{fig:betadisk} but for 49 Ceti.
             The vertical dashed lines show the position of the disk inner edge $R_\mathrm{in}=20$~au.
	}
\label{fig:49disk}
\end{figure*}


    Equating Equations (\ref{nCmid49}) and  Equation (\ref{nCreq}), 
    one obtains the expected CO fraction,
    \begin{eqnarray}
        \left(\frac{\nCO}{\nC}\right)_\mathrm{49Ceti} &=& 5.2 
        \times 10^{-4}
        \left( \frac{n_\mathrm{C,mid0}}{490~\pcc} \right)^{1.8}
        \left( \frac{R}{60~\mathrm{au}} \right)^{0.62} \nonumber \\
        & & \hspace{0mm} \times Z^{-1.1} \chico^{-1} 
        \left(\frac{\chioh}{\chi_\mathrm{oh,SSUV}}\right)^{-1} 
        \left( \frac{ {\cal A}_\mathrm{O}}{ 
        {\cal A}_\mathrm{O,ism}} \right) \nonumber \\
        && \hspace{0mm} \times \exp\left[ -1.8\left( \frac{R}{R_\mathrm{c}} \right)^{2.5} \right],
        \label{nCO49}
     \end{eqnarray}
     where $\chi_\mathrm{oh,SSUV}$ shows $\chioh$ of the SSUV model.



Models SSUV (WSUV) with the lower and higher mid-plane densities
are called 
SSUV$_\mathrm{low}$ (WSUV$_\mathrm{low}$) and 
SSUV$_\mathrm{high}$ (WSUV$_\mathrm{high}$), respectively. 


First, the low-density models are focused on.
Figures \ref{fig:49disk}a and \ref{fig:49disk}b show the radial 
distributions of the normalized UV fluxes 
for models WSUV$_\mathrm{low}$ and SSUV$_\mathrm{low}$, respectively.
%Decreases in $\chico$ around the inner edge of the disk 
%arise from the shielding effects. 
Since the C$^0$ fractions are smaller for SSUV owing to the 
stronger FUV flux, $\chico$ decreases 
more slowly for SSUV$_\mathrm{low}$ than for WSUV$_\mathrm{low}$.
As a result, more outer parts of the disk is exposed to 
intense FUV radiation for SSUV$_\mathrm{low}$.
The FUV fluxes $\chico$ take minimum values in $R\sim 60$~au 
for WUV$_\mathrm{low}$ 
and in $R\sim 10^2~$au for SUV$_\mathrm{low}$, 
and then increases to $\sim \chi_\mathrm{co,ISRF}$ 
around $R\sim 200~$au.
For WSUV$_\mathrm{low}$,
the vertical attenuation of $\chico$ is clearly seen 
in $40~\mathrm{au}<R<200~$au because 
${\cal N}_\mathrm{C,\perp}$ is larger than $N(\mathrm{C}^0)_\mathrm{shld}$ (Table \ref{tab:49Ceti_num}).



%$\chico$ decreases with $R$ more rapidly than $\chioh$ because of the C$^0$ attenuation,
%and reaches the constant value $\chi_\mathrm{co,ISRF}$ at a certain radius.
%The region where the stellar radiation is important extends farther outward for model SSUV$_\mathrm{low}$ than 
%for WSUV$_\mathrm{low}$.

%   Equating the mid-plane density, which is given by 
%     \begin{eqnarray}
%        n_\mathrm{C,mid}(R) &=& 8\times 10^2 ~\mathrm{cm}^{-3} \left( \frac{N_\mathrm{C0}}{N_\mathrm{C0,low}} \right) \left( \frac{R}{50~\mathrm{au}} \right)^{-3/2} 
%        \nonumber \\
%        &  & \hspace{5mm} \times \exp\left[ - \left( \frac{R}{120~\mathrm{au}} \right)^{2.5} \right],
%    \end{eqnarray}
%    and Equation (\ref{nCreq}), 
%    the CO fraction is expected to be 
%    \begin{eqnarray}
%        \left(\frac{\nCO}{\nC}\right)_\mathrm{49Ceti} &=& 1.0 \times 10^{-3}\left( \frac{\NC}{N_\mathrm{C,low}} \right)^{1.8}
%        \left( \frac{R}{50~\mathrm{au}} \right)^{0.7} \nonumber \\
%        & & \hspace{5mm} \times Z^{-1.1} \left( \frac{\chico}{1.3} \right)^{-1} \left( \frac{ {\cal A}_\mathrm{O}}{ 
%        {\cal A}_\mathrm{O,ism}} \right) \nonumber \\
%        && \hspace{5mm} \times \exp\left[ -1.7\left( \frac{R}{120~\mathrm{au}} \right)^{2.5} \right].
%        \label{nCO49}
%     \end{eqnarray}

The radial profiles of the C$^0$ and CO fractions in 
the low-density models with are shown in 
Figures \ref{fig:49disk}e and \ref{fig:49disk}f.  
 For $T_\mathrm{chem}=300$~K,
%At the low $\NC$ limit, 
both the models clearly show 
that CO forms efficiently only in the region where 
the stellar radiation flux is low because 
$(\nCO/\nC)$ is in proportion to $\chico^{-1}$.
Their radial distributions agree with the predictions from 
the analytic model (Section \ref{sec:COdiskana}).
%Equation (\ref{fitting}).
%Comparison between the CO fractions at the mid-plane and the predictions from Equation (\ref{fitting}) with 
%$\chi_\mathrm{co,mid}$ show that 
%increases in the CO fractions at the mid-plane come from the vertical attenuation of $\chico$.
%the CO fraction ($\sim 10^{-3}$) at the low $N_\mathrm{C}$ limit 
%is comparable to that the critical value, above which the CO self-shielding becomes effective. 
%As is the models of $\beta$ Pictoris (Figures \ref{fig:betadisk}c and \ref{fig:betadisk}d), 
%the CO fractions increase by an order of magnetude when $T_\mathrm{chem}$ is decreased from 300~K to 10~K.

A decrease in $T_\mathrm{chem}$ does not affect the CO 
fractions significantly, compared with the results with $\beta$ Pictoris
(Figures \ref{fig:49disk}e and \ref{fig:49disk}f).
The CO fractions are increased only by a factor of 2 or 3.
This comes from the fact that $\fracdg/\fracdgfid$ for 49 Ceti is more than one order of magnitude smaller than that for $\beta$ Pictoris.
Roughly speaking, 
the inner (outer) regions dominated by the stellar radiation (the vertical ISRF) correspond to the strong-FUV (weak-FUV) models
at $\NC \sim 10^{17}~\psc$.
As seen in the first and middle panels of Figure \ref{fig:COfit}, 
when $\fracdg < 0.24\fracdgfid$, the existence of H$_2^*$ increases the CO fractions only by a factor of 2 or 3.


%The debris disk around 49 Ceti cannot produce a sufficient amount of H$_2$ to enhance the CO fractions significantly since 
%the H$_2$ formation rate on the dust surface is proportional to $\fracdg/\fracdgfid$ (Equation \ref{RER}).


When the mid-plane density increases,
the behaviors of the CO fractions change significantly.
The stellar radiation in the wavelength range of $\chico$ is
almost attenuated in very inner regions even for model SSUV$_\mathrm{high}$.
Since the vertical column density 
${\cal N}_\mathrm{C,\perp}=3.5\times 10^{17}~\psc$ at $R=60~$au (Table \ref{tab:49Ceti_num}) is 
much larger than $N(\mathrm{C^0})_\mathrm{shld}$,
the vertical attenuation of the ISRF is significant in 
$30~\mathrm{au}<R<200~\mathrm{au}$ (Figures \ref{fig:49disk}c and \ref{fig:49disk}d).

Figures \ref{fig:49disk}g  and \ref{fig:49disk}h show that 
both the high-density models yield large amounts of CO where $\chico$ is almost attenuated.
The CO fractions in the attenuated regions are almost independent of $T_\mathrm{chem}$.
This is consistent with the results shown in Section \ref{sec:CO2e18}. 
In a such attenuated region, 
$\mathrm{C^++H^*_2\rightarrow CH^+ + H}$ does not accelerate the CO formation because of depletion of C$^+$.
We should note that the CO fractions slightly decrease with decreasing $T_\mathrm{chem}$ 
in the attenuated regions
because the H atoms available for the CO chemistry shown in Figure \ref{fig:COnetwork} are reduced by enhancement of the H$_2$ formation (Section \ref{sec:CO2e18})

%Instead, the abundant H$_2$ slightly decreases the CO fractions as shown in Figures \ref{fig:COfit}i and \ref{fig:COfit}j.
%Since the difference is small, the CO fractions are almost independent of $T_\mathrm{chem}$ (Figures \ref{fig:49disk}g  and \ref{fig:49disk}h).

The bottom panels of Figures \ref{fig:49disk} show 
the radial distributions of $\NCI$ and $\NCO$ integrated from the inner edge to $R$.
Increases in the CO column densities are saturated around the peak radius of $\nCO$.
Thus, ${\cal N}_\mathrm{C}$ is determined in the amount of CO inside the peak radius of $\nCO$.


%The dependence of the stellar model on $\chico$ disappears at most 
%radii as shown in Figures \ref{fig:49disk}c and \ref{fig:49disk}d.
%The mid-plane density is high enough for the CO fractions 
%to exceed the critical fraction 
%(Equation \ref{COshield_cri})
%and the CO self-shielding effect enhances the CO fractions significantly.  
%As a result, in the models with {\color{blue} ${\cal N}_\mathrm{C,high}$},
%the CO fractions become larger than the C$^0$ fractions at the mid-plane 
%(Figures \ref{fig:49disk}g and \ref{fig:49disk}h).
%The PDR calculations with lower $T_\mathrm{chem}$ 
%were not conducted for {\color{blue} ${\cal N}_\mathrm{C}=
%{\cal N}_\mathrm{C,high}$} because the CO fractions are already 
%high at $T_\mathrm{chem}=300~$K.











%Increases in the CO fractions in inner regions of the disk 
%arise from the rapid decreases in $\chico$ owing to the C$^0$ attenuation (Figures \ref{fig:beta49}e and \ref{fig:beta49}h).
%The CO fractions are maximized when $\chico$ reaches the floor value $\chi_\mathrm{co,ISRF}$, indicating that 
%the contribution of the stellar radiation to $\chico$ becomes lower than $\chi_\mathrm{co,ISRF}$.
%Outside of the cutoff radius ($R_\mathrm{c}=120~$au), the CO fractions decrease as the mid-plane densities decrease rapidly 
%in both the models.
%The C$^0$ and CO fractions are consistent with those predicted from Equations (\ref{nCp}) and (\ref{fitting}), 
%respectively (Figures \ref{fig:beta49}f and \ref{fig:beta49}i).
%At the mid-plane, the CO fractions are enhanced through the C$^0$ attenuation and the H$_2$-accelerated CO chemistry.
%The latter is effective only in model WSUV.


\begin{figure}[htpb]
\centering
\includegraphics[width=7cm]{Fig11.pdf}
\caption{
    Results of the PDR calculations for 49 Ceti.
(a) The mid-plane column densities of (red) $\mathrm{C}^+$, (green) $\mathrm{C}^0$, and (blue) CO 
as a function of ${\cal N}_\mathrm{C}$ for 
the models with the weak and strong stellar radiation models are shown by 
the circles and triangles, respectively.
The results with $T_\mathrm{chem}=300~$K are shown.
The results with the low and high density models are displayed by 
the filled and open markers.
For reference, the results with the intermediate density 
$n_\mathrm{C,mid0}=1.2\times 10^3~\pcc$ (${\cal N}_\mathrm{C}=2\times 10^{18}~\psc$)
are shown by the markers filled by gray.
At each ${\cal N}_\mathrm{C}$, 
the data points with different stellar models are slightly shifted horizontally for better visibility.
The green and blue rectangle regions indicate the observational constraints
on the C$^0$ and CO column densities obtained from \citet{Higuchi2019} and 
\citet{Higuchi2020}, respectively.
(b) The mid-plane CO column densities ${\cal N}(\mathrm{CO})$ as a function of $Z$ for (blue circles) WSUV$_\mathrm{low}$, 
(red circles) SSUV$_\mathrm{low}$, (blue triangles) WSUV$_\mathrm{high}$, and (red triangles) SSUV$_\mathrm{high}$.
For each model, the prediction from the analytic formula is shown by the thin line.
The blue rectangle region is the same as that in
Panel (a). As a reference, 
$\NCO\propto Z^{-1.1}$ is plotted
(Equation (\ref{nCO49})).
}
\label{fig:49_comp}
\end{figure}


The obtained C$^+$, C$^0$, and CO column densities integrated along the mid-plane are 
compared with the observational results in Figure \ref{fig:49_comp}a for $Z=1$.
The low-density models show that
the CO fractions does not reach the observed values 
for both the stellar models.
By contrast, the high-density models
produce sufficient amounts of CO to explain the observational results.
We thus expect that there is a best solution to explain the observed 
CO column densities in $4.9\times 10^2~\pcc<n_\mathrm{C,mid0}<2.0\times 10^3~\pcc$ 
(Table \ref{tab:49Ceti_num}).
For reference, we conducted additional PDR calculations with 
$n_\mathrm{C,mid0}=1.2\times 10^3~\pcc$ 
(${\cal N}_\mathrm{C}=2\times 10^{18}~\psc$).
The mid-plane density is between those in the low-density and high-density models.
Figure \ref{fig:49_comp}a show that a best solution is located 
in $1.2\times 10^3~\pcc<n_\mathrm{C,mid0}<2.0\times 10^3~\pcc$, and 
both ${\cal N}(\mathrm{C}^0)$
and ${\cal N}(\mathrm{CO})$ of a best solution 
will be consistent with the observational results.


The $Z$ dependence of ${\cal N}(\mathrm{CO})$ is shown in Figure \ref{fig:49_comp}b.
For all the models, the CO column densities decrease with increasing $Z$ as 
$ {\cal N}(\mathrm{CO}) \propto Z^{-1.1}$, and 
are consistent with the predictions from Equation (\ref{nCO49}). 
%are consistent with the predictions from Equation (\ref{nCO49}). 
%although 
%$N_\mathrm{mid}(\mathrm{CO})$ is enhanced from the prediction by an order of magnitude for $Z\le 10^1$ because of the CO self-shielding effect.
Figure \ref{fig:49_comp}b shows that
WSUV$_\mathrm{high}$ (SSUV$_\mathrm{high}$) provides ${\cal N}(\mathrm{CO})$ 
consistent with 
the observed CO column densities if $1< Z < 10$ ($Z\sim 1$).
This suggests that if the gas is secondary-origin ($Z\sim 10^{3}$), 
the CO column densities cannot be explained only by steady-state chemical reactions.



%%----------------------------------------------------------------------------
\subsection{ Uncertainty of the H$_2$ Formation on Dust Grains}\label{sec:ER}
%%----------------------------------------------------------------------------

One of the main results of this paper is that 
the CO formation is affected strongly by $T_\mathrm{chem}$, which is one of the uncertain 
parameters in the ER mechanism, when the shielding effects are not significant.
In this section, other uncertainties of the H$_2$ formation are discussed.
%This result however relies on the H$_2$ formation rate on dust grains. 
%The Meudon code adopts the method of \citet{LeBourlot2012} where the 
%LH and ER mechanisms are  taken into account in the rate equation formulations.



%In the ER mechanism, which is the most important H$_2$ formation mechanism on warm grains in debris disks, 
%there are several free parameters, some of which are shown in Equations (\ref{RER}) and 
%(\ref{kappa}).   
%One of the most critical parameters is $T_\mathrm{chem}$ corresponding to the 
%energy barrier that a hydrogen atom must overcome 
%to reach a chemisorption site on the grain surface.
%It is highly uncertain and depends on the surface condition and composition of dust grains.
%Our adopted value of $T_\mathrm{chem}=300~$K leads 
%to {\color{blue} a moderate} H$_2$ formation, corresponding to situations where 
%chemisorption efficiently occurs on grains with plenty of surface defects \citep{LeBourlot2012}. 
%The energy barrier of chemisorption onto a perfect graphite surface is 
%as high as $\sim 2000$~K \citep[e.g.,][]{Sha2002}.
%By contrast, topological defects on the grain surface 
%decrease $T_\mathrm{chem}$ and can make chemisorption 
%almost barrierless 
%($T_\mathrm{chem}\sim 10-100~$K) depending on structures \citep{Ivanovskaya2010}.
%We thus need to keep in mind that 
%the H$_2$ formation rate is very sensitive to $T_\mathrm{chem}$
%because of its exponential dependence $\exp(-T_\mathrm{chem}/T)$.
%In this paper, we found that $\fracdg=\fracdgfid$ is a critical dust-to-gas mass ratio 
%above which a sufficient amount of H$_2$ can be produced to proceed CO formation.
%The critical dust-to-gas mass ratio should depend on $T_\mathrm{chem}$.
%If $T_\mathrm{chem}$ is lower than the fiducial value $300~$K, 
%the H$_2$ formation on the grain surface becomes more efficient, 
%leading to an increase in the CO fraction.
%In Section \ref{sec:observation} where our models are compared with observational results, 
%the $T_\mathrm{chem}$ dependence of the CO fraction is investigated.


Another factor not considered in this paper is stochastic effects.  
Fluctuations of the number of H atoms on the grain surface 
should be take into account when 
few reactive H atoms are on the grain surface \citep{LePetit2009}.
We confirmed that the number density of the H atoms physisorbed on the grain surface is much 
smaller than the number density of dust grains.
This indicates that the master equation should be used instead of the rate equation 
to accurately estimate the H$_2$ formation rate of the LH mechanism.
\citet{Bron2014} found that the LH mechanism becomes an efficient H$_2$ formation mechanism 
in unshielded PDR regions even when the dust temperature is high for the interstellar environment.
It is still unknown how the fluctuations of the number of physisorbed H atoms affect H$_2$ formation 
in debris disks.

Fluctuations of the dust temperature also affects the H$_2$ formation 
for small grains with $a\le 0.02~\mu$m \citep{Cuppen2006}.
If only large dust grains exist in debris disks, dust temperature fluctuations are negligible 
because of their large heat capacities.
If small dust grains exist in debris disks, 
their temperature fluctuations may affect 
the H$_2$ formation rate.




%---------------------------------------------------
\subsection{Caveats}\label{sec:caveat}
%---------------------------------------------------

%In Sections \ref{sec:betaPic} and \ref{sec:49Ceti}, 
%it is confirmed that once a disk model is given, the radial distributions of 
%the C$^0$ and CO fractions are consistent with the predictions from 
%the analytic formulae given in Equations (\ref{ionCana}) and (\ref{fitting}).
%Even at the mid-plane, the CO fractions can be roughly predicted by Equation (\ref{fitting}) 
%if the vertical attenuation of the ISRF is taken into account.
%
%For $\beta$ Pictoris, since the amount of dust grains is high, 
%the CO fractions are sensitive to $T_\mathrm{chem}$.
%
%the CO fractions at the mid-plane are slightly 
%larger than or comparable to to those at the low $\NC$ limit.
%Since the amount of dust grains is high related to that of gas, 
%the excited-H$_2$ can accelerate CO formation.
%
%In gas-rich debris disks like 49~Ceti, if the CO fraction predicted from Equation (\ref{fitting}) exceeds 
%$(\nCO/\nC)_\mathrm{cri}=3\times 10^{-3}$, the CO self-shielding effect is likely to work, and 
%a large amount of CO is formed at the mid-plane.
%If the mid-plane density is sufficiently large, 
%the CO column density can be comparable or larger than the C$^0$ column density.

Our analytic formula of $\nCO$ enables us to predict the CO spatial distributions 
at a given $Z$ and stellar radiation field (Section \ref{sec:COdiskana}).
Comparison of our results with observational results gives a constraint on the gas metallicity. 
In order to confirm whether the steady-state chemistry is a promising mechanism to produce the 
observed amount of CO, it is crucial to constrain the gas metallicity by other methods.

We should note that the column density of hydrogen is 
constrained by several observations for the debris disk around $\beta$ Pictoris.
\citet{Freudling1995} obtained an upper limit of the 
HI column density, ${\cal N}(\mathrm{H^0})<
2-5\times 10^{19}~\mathrm{cm}^{-2}$ from non-detection of the 21~cm line.
%More recently, \citet{Wilson2017} detected the Lyman $\alpha$ line and
%derived an HI column density of
%$(3-5)\times 10^{18}~\mathrm{cm}^{-2}$.
\citet{Lecavelier2001} found that the H$_2$ column density should be smaller than
$\sim 10^{18}~\mathrm{cm}^{-2}$ because H$_2$ absorption lines are not detected.
Using a C$^0$ column density of $\sim (2.5\pm 0.7)\times 10^{16}~$cm$^{-2}$
\citep{Higuchi2017}, 
${\cal N}_\mathrm{C}/{\cal N}_\mathrm{H}$ is estimated to be larger than
$0.025$, which is much larger than ${\cal A}_\mathrm{C,ism}$.
Recently \citet{Matra2017} constrained the hydrogen number density by the line
ratio of CO ($J=2-1$) and CO ($J=3-2$); their result suggests that 
$\nH$ should be smaller than $\sim 10^4~\mathrm{cm}^{-3}$.
If this is the case, the observed CO fractions may not be explained by such a small value of 
$\nH$ even if $T_\mathrm{chem}$ is sufficiently low.
%Recently \citet{Matra2017} constrained the hydrogen volume density by the line
%ratio of CO ($J=2-1$) and
%CO ($J=3-2$); their result suggests that $n(\mathrm{H}_2)$
%should be smaller than $\sim 10^4~\mathrm{cm}^{-3}$.
%Assuming the gas density is given by Equation (\ref{nCbeta}),
%the metallicity is need to be larger than $\sim 60$.
%The observed CO column densities cannot be explained by such a large metallicity as shown in Figure \ref{fig:beta_comp}.
The combination of our results and the observational constraints
may rule out the primordial origin ($Z=1$) of the gas in $\beta$ Pictoris.


Although we found that there is a parameter set which may explain the amount of CO 
by considering only steady-state chemical reactions in Sections \ref{sec:betaPic} and \ref{sec:49Ceti}, 
we need to compare our models with observational results in more detail by 
performing synthetic observations of our disk models.

For 49 Ceti, \citet{Higuchi2020} showed that the excitation temperatures need to be as low as $\sim 10$~K
by conducting non-LTE analysis of the rotational spectral lines of CO considering 
isotopologues line ratios \citep[also see][]{Kospal2013}.
In our PDR calculations of 49 Ceti, the gas temperatures do not fall below $\sim 20~$K.
This low excitation temperature may suggest that level populations 
are not thermalized because of low $\nH$.

We will address direct comparisons between the chemical model and observational results 
in forthcoming papers.


%The estimated FUV fluxes at $R=R_\mathrm{in}$ have about an order of variation of
%$\chi=\sqrt{\chi_\mathrm{co}\chi_\mathrm{oh}}$ from 80 (weak stellar radiation model) to 580 (strong stellar radiation model).
%We consider the range of $\chi$ as an uncertainty of the stellar parameter of the central star.


%\begin{table}[htpb]
%\begin{tabular}{|c|c|c|}
%    \hline
%      & strong UV model & weak UV model \\
%    \hline
%      $T_0=300~\mathrm{K}$ & 0.003 & 0.03 \\
%    \hline
%      $T_0=100~\mathrm{K}$ & 0.005 & 0.08 \\
%    \hline
%      $T_0=10~\mathrm{K}$ & 0.02 & 0.4 \\
%\end{tabular}
%\end{table}



%The 49 Ceti debris disk is also well-studied \citep{Sadakane1986}.
%First, we estimate $\fracdg/\fracdgfid$.
%Since the debris disk is also inclined close to edge-on ($\sim 80^\circ$) \citep{Hughes2017}, 
%the observed $N(\mathrm{C}^0)\sim 2\times 10^{17}~\psc$ \citep{Higuchi2017} is 
%substituted directly into $\NC$ of Equation (\ref{fdg_determ}).
%One obtains $\fracdg/\fracdgfid \sim 0.3$, where 
%the optical depth is estimated to be $\tau\sim 3\times 10^{-3}$ from the fractional 
%luminosity, indicating that the CO over-production is not expected.
%Since $N(\mathrm{C}^0)\sim 2\times 10^{17}~\psc$ is larger than $N(\mathrm{C}^0)_\mathrm{shield}$, 
%the C$^0$ attenuation is expected to enhance the CO fraction.
%
%Different stellar parameters are used in the literature 
%\citep{Chen2006,Hughes2008,Montesinos2009,Roberge2013}.
%The estimated FUV fluxes at 50~au have about an order of variation of
%$\chi=\sqrt{\chi_\mathrm{co}\chi_\mathrm{oh}}$ from 13 to 93.
%We consider the range of $\chi$ as an uncertainty of the stellar parameter of the central star.

%\begin{figure}[htpb]
%        \centering
%        \includegraphics[width=7cm]{Figure11.eps}
%%        \includegraphics[width=7cm]{plane-parallel-CO-49Ceti.eps}
%        \caption{
%        C/CO column density ratios as a function of $N(\mathrm{C}^0)$ 
%        the (a) $\chi=13$ and (b) $\chi=93$ cases with 
%        (red) $n_\mathrm{H}=10^7~\pcc$, (green) $10^6~\pcc$, (blue) $10^5~\pcc$.
%        The dashed horizontal lines show the predictions from Equation (\ref{fittingN}).
%        The gray regions indicate the observed C/CO ratio, and 
%        The vertical lines indicate the observationally estimated $N(\mathrm{C}^0)$
%        \citep{Higuchi2017}.
%	}
%\label{fig:49Ceti}
%\end{figure}

%The debris disk around 49 Ceti has a sufficient amount of C$^0$ for CO to be enhanced.
%Since Equation (\ref{fittingN}) cannot be used when shielding is effective,  
%we perform the PDR calculations with $\chi=13$ and $\chi=93$,
%which are the minimum and maximum FUV fluxes adopted in the literature, respectively.
%Firstly, the solar metallicity is assumed.
%Figures \ref{fig:49Ceti}a and \ref{fig:49Ceti}b
%show the C/CO ratios as a function of $N(\mathrm{C}^0)$ for $\chi=13$ and $\chi=93$,
%respectively, with different hydrogen nucleus densities.
%At the low $N(\mathrm{C}^0)$ limit, the CO/C ratios are consistent 
%with the predictions from Equation (\ref{fittingN}), indicating that 
%the analytical model is applicable for a wide range of $\chi$.
%
%
%%At the observed $N(\mathrm{C}^0)$, CO is shielded
%%(Figures \ref{fig:49Ceti}a and \ref{fig:49Ceti}b).
%\citet{Higuchi2017} derived $\mathrm{C/CO}\sim 54\pm 19$, which 
%are shown by the gray regions in Figure \ref{fig:49Ceti} \citep[also see][]{Higuchi2019}.
%The $\chi=13$ case show that the observed C/CO ratio can be explained for $\nH\sim 10^7~\pcc$ while
%for $\chi=93$, $\nH$ needs to be larger than $10^7~\pcc$ to reproduce the observed C/CO rato.
%
%%The corresponding range of $\nC$ is $10 < \nC/\pcc< 10^3$ for the solar metallicity.
%The spatial width of the gas slab $L\sim N(\mathrm{C}^0)/n_\mathrm{C}$ 
%is around $\sim 10$~au for $\chi=13$ and $< 10$~au for $\chi=93$.
%The $\chi=93$ model thus may be ruled out if all CO gas is formed by chemical reactions.
%If a stellar model with lower FUV flux is valid as a model of 49 Ceti, 
%the solar metallicity i.e., the primordial gas may not contradict with the observed C/CO ratio. 
%
%In a similar way as the discussion in Section \ref{sec:betaPic}, 
%the required spatial extent of the gas slab decreases with increasing $Z$.
%If most CO forms through chemical reactions, 
%higher metallictity is ruled out because an extremely high carbon density is required.

%Although there is the parameter space where CO can form only through 
%chemical reactions for the solar metallicity,
%a parameter space with $Z>1$ can explain the observed C/CO ratio.
%In our models, since $Z$ controls the spatial extent of the gas slab at fixed 
%$\nH$ and $N(\mathrm{C}^0)$, a model taking into account the global disk structure of debris disks 
%should be considered to constrain $Z$.
%In addition, the $Z$-dependence of $T$ may be used to constrain $Z$ by comparing 
%the intensities of the emission lines with those predicted from the models.
%\citet{Roberge2013}, for instance, pointed out that 
%the primordial-origin model is
%difficult to explain the CO emission, the C$^+$ emission, 
%and the lack of O$^0$ emission simultaneously but they consider only the solar metallicity 
%case.


%%%------------------------------------------------------------------------------------
%\subsection{ Dust-to-Gas Mass Ratios Smaller Than $0.1\fracdgfid$}\label{sec:0.1fdg}
%%%------------------------------------------------------------------------------------
%The maximum dust-to-gas mass ratio is set to $\fracdg=10\fracdgfid$
%because too large $\fracdg$ is not allowed by the condition of $\tau<8\times 10^{-3}$.
%Although we did not show the results with $\fracdg<0.1\fracdgfid$, 
%the behaviors of the CO fractions do not depend on $\fracdg$ in this range.
%This is because even when $\fracdg=0.1\fracdgfid$, an amount of H$_2$ 
%is too low for the CO fraction to be accelerated by excited H$_2$.
%A further decrease in $\fracdg$ improves the approximation of CO formation of the H$_2$-free gas.
%

%%---------------------------------------------------
%\subsection{Influence of Carbon-to-Oxygen Ratio on CO Chemistry}\label{sec:O_C}
%%---------------------------------------------------
%
%In this paper, an interstellar C/O ratio of 0.4 is used (Table \ref{tab:abundance}).
%Recently \citet{Kama2016} and \citet{Bergin2016} however 
%found evidence that C/O ratios could exceed unity
%in PPDs, indicating that C/O ratios
%change from the interstellar value during the evolution of PPDs.
%If a part of the gas in a debris disk comes from a remnant PPD,
%the C/O ratio could exceed unity also in a debris disk.
%
%In order to investigate how the CO fractions depend on the C/O ratio,
%we perform additional PDR calculations with $\COr=2$, keeping $\Ac=1.32\times 10^{-4}$ 
%constant. The oxygen nucleus number density in the $\COr=2$ case
%is five times smaller than that in the interstellar value at a fixed $\nC$.
%Depletion of oxygen nuclei is expected to reduce the CO fractions from Equation (\ref{fitting}).

%\begin{figure}[htpb]
%        \centering
%       \includegraphics[width=8cm]{CO_fit_fdg_o.eps}
%%        \includegraphics[width=8cm]{Figure12.eps}
%        \caption{
%        The same as Figure \ref{fig:COfit} but the results with $\mathrm{C/O=2}$ are shown.
%        The vertical axes are the compensated CO fraction $(\Ao/{\cal A}_\mathrm{O,ism})^{-1}\left( \eta/10^6 \right)^{-2} 
%        \nCO/\nC$
%        For comparison, the results with $\mathrm{C/O=0.4}$ are plotted by the dashed lines.
%        The gray solid lines indicate the predictions from Equation (\ref{fitting}) divided by $(\Ao/{\cal A}_\mathrm{O,ism})$
%        to eliminate the $\Ao$ dependence.
%	}
%\label{fig:cor}
%\end{figure}


%For all the models shown Figure \ref{fig:cor}, the compensated CO fractions do not show a 
%$\Ao$-dependence at each $Z$ and $\chi$ for $\nH Z^{0.4}<10^{3-4}~\pcc$, indicating that the CO fractions is proportional to 
%$\Ao$ as predicted from Equation (\ref{fitting}).
%
%%For $\fracdg=\fracdgfid$,
%%Figures \ref{fig:cor}a and \ref{fig:cor}b show that the CO fractions are consistent with 
%%the predictions from Equation (\ref{fitting}) for small metallicities and lower densities.
%%A decrease in $\Ao$ thus reduce the CO fraction in proportion to $\Ao$ as expected.
%%However, unlike for $\COr=0.4$, 
%%CO over-productions compared with the analytical model 
%%are found at higher metallicities and higher densities 
%%even for $\fracdg=\fracdgfid$.
%%The $\fracdg=10\fracdgfid$ cases show more significant enhancement of CO for both the spectral types.
%%The CO fractions become larger for $\mathrm{C/O=2}$ than for $\mathrm{C/O=0.4}$ although 
%%the oxygen elemental abundance is smaller for $\mathrm{C/O=2}$.
%For $\nH Z^{0.4}>10^4~\pcc$, the $\mathrm{C/O}=2$ cases show larger compensated CO fractions than 
%the $\mathrm{C/O}=0.4$ cases.
%This is attributed to an increase in the gas temperature.
%Because the fine-structure transition of neutral oxygen atom is main coolant in higher densities with 
%$\nC>10-100~\pcc$, 
%a decrease in $\Ao$ makes the O cooling inefficient 
%while the heating rate is independent of $\Ao$.  For $\COr=2$,
%gas becomes hotter. A similar discussion was done by \citet{Roberge2013}.
%Higher temperatures promote OH formation significantly, 
%resulting in more efficient CO formation for the $\COr=2$ cases.
%
%---------------------------------------------------

%%-------------------------------------------------------
%\subsection{Effect of Different Dust Size Distributions}\label{sec:smalldust}
%%-------------------------------------------------
%
% The PDR calculations are conducted assuming a MRN-like distribution 
% with $\amin=1$~$\mu$m and $\amax=(10~\mu\mathrm{m},~1~\mathrm{cm})$.
% In Section  \ref{sec:plane}, we confirmed that our results are independent of 
% $\amax$ when $\fracdg/\fracdgfid$ is kept constant 
% because $\fracdg/\fracdgfid$ characterize the total surface area of dust grains 
% as shown in Equation (\ref{crosssec}).
% We would like to emphasis that our results are expected not to depend on any 
% parameters of the dust size distribution significantly as long as $\fracdg/\fracdgfid$ is fixed
% because the derivation of Equation (\ref{fdg_tau_Nc}) do not need to specify the size distribution 
% of dust grains.
%
%% As described in Sections \label{sec:fracdg} and \ref{sec:modelparameter},
%% the dust-to-gas mass ratio is defined on the basis on the optical depth, which is estimated observationally.
%% If the size of dust grains is larger than the wavelength (ultraviolet and optical) of the incomping radiation field,
%% the absorption efficiency of dust grains is unity.
%% The optical depth is thus directly related to the geometrical cross section of dust grains,
%% which is characterized by the parameter $\fracdg/\fracdgfid$ and controls the grain surface chemistry.
%
% In several debris disks, small sub-micron dust grains were observationally inferred from 
% the color in scattered light \citep[e.g.,][]{Kalas2005} and 
% solid-state features \citep[e.g.,][]{Chen2006,Fujiwara2012}.
% Such small dust grains have the absorption efficiency $Q_\mathrm{abs}$ much smaller than unity in the 
% incoming radiation field.
% Equation (\ref{crosssec}) is however expected to be still valid even if small dust grains exist.
% This is justified by the fact that the geometrical cross section of small dust grains does not contribute 
% to the total geometrical cross section of dust grains significantly in debris disks around A-type stars \citep{Lohne2008,Krivov2010,Thebault2019}.
%
%






%-------------------------------------------------------
\subsection{Dissipation of Protoplanetary Disks}
%-------------------------------------------------

If the gas in debris disks contains a significant amount of H$_2$, the gas of PPDs
still remains, at least partially, in debris disks; i.e.
the timescale of PPD gas dispersal is longer than the age of debris disks.
Many authors investigated the gas dispersal of PPDs
through magnetically-driven winds \citep{Suzuki2009,Suzuki2014} or/and
photo-evaporation by UV/X-rays \citep{Hollenbach1994,GortiHollenbach2009,Owen2012,Nakatani2018}.
EUV photo-evaporation \citep{Hollenbach1994} 
is expected to be inefficient 
in debris disks around A-type 
stars because they emit less EUV than either lower 
and higher mass stars.
FUV photo-evaporation would be important at large radii $>100~$au 
as long as a large amount of small dust grains exist \citep{GortiHollenbach2009}.
During the evolution from a PPD to a debris disk, an amount of dust grains decreases and 
the grain size increases. 
\citet{Nakatani2021} recently found that 
this evolution of dust grains suppress 
photo-electric heating, making FUV photo-evaporation ineffective.
As a result, the lifetime of the gas component 
is determined by EUV photo-evaporation. 
Suppose that the EUV flux of an A type star is about $\sim 10^{39}~\mathrm{s}^{-1}$, the photo-evaporation is at least 
$3\times 10^{-11}~M_\odot~\mathrm{yr^{-1}}$ \citep{Gorti2009}.
If the initial disk mass at the point 
small grains have been depleted is 
$\sim 10^{-3}~M_\odot$, the lifetime is about $30~$Myr, 
indicating that the lifetime can be longer than $10~$Myr.
However, the EUV flux in young A-type stars is highly uncertain, and 
how protoplanetary disks evolve into small-grain-depleted disks are 
unknown. Further investigations are needed.
%As a result, the lifetime of the gas component 
%of disks around A-type stars can be longer than $10~$Myr.
%These mechanisms depend on the strength of magnetic
%fields, the dust size distribution, etc, which are still uncertain.
{The future high resolution observation for the ${\rm C}/{\rm CO}$ ratio
combined with Equation~(\ref{fitting}) would give the distribution of
$n_\mathrm{H}$ in depleting PPDs, which may reveal
the physics of disk dispersal.
}

%}
%------------------------------------
\section{Summary}\label{sec:summary}
%------------------------------------
In this paper, we investigated the CO chemistry in optically thin PDRs with larger 
dust grains than the ISM as a model of debris disks.
Because hydrogen is involved in CO formation, the CO abundance should depend on the hydrogen 
density.
This may allow us to constrain the amount of hydrogen if
we know how much carbon nuclei becomes CO \citep{Higuchi2017}.

We investigated CO formation only through steady chemical reactions without
considering any secondary processes.
For simplicity, we consider a stationary plane-parallel semi-infinite uniform slab which
is illuminated by the interstellar standard radiation and
stellar radiation from its edge.
We compute the chemical and thermal equilibria taking into account detailed
radiative transfer using the Meudon PDR code \citep{LeP2006,Goi2007,LeBourlot2012}.
Although the plane-parallel geometry is far from the disk geometry, 
we found that the chemical structure of debris disks 
can be approximated by the findings from the one-dimensional plane-parallel PDR calculations.

The model parameters are the carbon nucleus density $\nC$,
gas metallicity $Z$, FUV photon number flux $\chico$ that dissociates CO normalized by the Habing flux,
the dust-to-gas mass ratio $\fracdg$,  and 
$T_\mathrm{chem}$, which corresponds to the energy barrier of chemisorption.
The model parameters are tabulated in Table \ref{tab:param}.
We call the models with weaker and strong FUV incident fluxes "weak-FUV" and "strong-FUV", respectively.
%, maximum dust radius $\amax$, where $\amin$ is fixed to be $1~\mu$m as shown in Table \ref{tab:param}.
The dust-to-gas mass ratio is expressed in terms of the carbon nucleus column density at $\taumid=1$ ($\NCtaumid$), where
$\taumid$ is the mid-plane optical depth (Equation (\ref{fdg})).
Observations of debris disks can constrain $\NCtaumid$.
As a fiducial value of $\NCtaumid$, $\NCtaumidfid=2\times 10^{19}~\psc$ is adopted.
The corresponding dust cross section is 80 times smaller than that in the ISM. 
A fiducial dust-to-gas mass ratio $\fracdgfid$ is defined as $\fracdg$ at ${\cal N}_\mathrm{C}=\NCtaumidfid$.
The ratio $\fracdg/\fracdgfid$ is equal to  $(\NCtaumid/\NCtaumidfid)^{-1}$, and determines the dust surface area (Equation (\ref{crosssec})).
%We change $\fracdg$ in units of $\fracdgfid$.
Considering large uncertainty in H$_2$ formation on the grain surface, 
we explore a range of H$_2$ formation rates on dust grains with
a dust temperature of $\sim 100~$K %and gas temperatures of $40-100$~K 
by adopting $T_\mathrm{chem}=300$~K and 10~K.
The $T_\mathrm{chem}=10~$K case gives an upper limit of the H$_2$ formation rate since there is almost no energy barrier in chemisorption.



Our results are summarized as follows:


\begin{enumerate}

%    \item The results of the PDR calculations do not depend on $\amax$ at a fixed
%            $\fracdg/\fracdgfid$ because 
%            the total surface area of dust grains remain unchanged even if $\amax$ changes.
%    \item The results of the PDR calculations do not depend on $\amax$ when
%            $\amax$ changes keeping $\fracdg/\fracdgfid$ unchanged (Equation \ref{crosssec}) 
%            because $\fracdg/\fracdgfid\propto {\color{blue} {\cal N}^{-1}_\mathrm{C,\tau_{mid}=1}}$ 
%            determines the total surface area of dust grains, which 
%            control the H$_2$ formation on the grain surface and photo-electric heating process.

%     \item For $\fracdg < \fracdgfid$, CO formation proceeds 
%     in the H$_2$-poor environments
%     \item {\color{blue} For $\fracdg \le \fracdgfid$ in weak-FUV models and 
          \item  For higher $T_\mathrm{chem}$ and  
           smaller $\fracdg$,  
           we found that CO formation proceeds in the H$_2$-poor
           environments without the influence of H$_2$. 
           Our environments tend to 
           have a very low H$_2$
    abundance because the low grain surface area, 
           high dust temperatures, and low gas temperatures make H2 
           formation extremely inefficient if the activation barrier 
           is as high as $T_\mathrm{chem}=300$~K.
           We developed an analytical formula for the CO fractions 
           (Section \ref{sec:analyticmodel}).
           The analytic formula is applicable even in 
           shielded regions by using the local flux attenuated 
           by C$^0$ and CO, and reproduces 
           the spatial distributions of the 
           CO fractions obtained from the PDR 
           calculations reasonably well (Figure \ref{fig:COshield}).
     
%           An analytic formula for $\nCO$ shown in Equation (\ref{fitting}) reproduces 
%           the CO fraction at the low $\tau$ limit reasonably well (Section \ref{sec:COcomp}).

\item From the high density limit of
the analytic formula for $\nCO/\nC$ (Equation (\ref{fitting})), 
we obtain the hydrogen nucleus number
density required to produce a given $\nCO/\nC$ 
as a function of the FUV flux and gas metallicity
(Equation (\ref{nHconstrained})). 
If the amount of carbon nuclei and 
local FUV flux are fixed, 
lower metallicity produces a larger amount of CO
because there are more H atoms to start the 
chemistry that produces CO.

\item 
       The H$_2$ formation rate depends sensitively on $T_\mathrm{chem}$
       since it is proportional to $\exp(-T_\mathrm{chem}/T)$.
       The smaller $T_\mathrm{chem}$,
       the more active the H$_2$ formation on the grain surface.
       Increases in $\fracdg$ also enhances the H$_2$ formation.
          The CO formation is accelerated by vibrationally-excited H$_2$, whose internal energy is used to 
          overcome an endothermicity of $\mathrm{C^++H_2\rightarrow CH^+ + H}$.
          The enhancement of the CO fractions owing to the excited H$_2$ is occurred 
          only when the shielding 
          effects are insignificant.
          If the shielding effects are important, 
          the CO fractions show different dependence on 
          $T_\mathrm{chem}$ and $\fracdg$.
          As $T_\mathrm{chem}$ decreases  and/or 
          $\fracdg$ increases, the CO fractions decrease although the dependence is weak.
          This is because in such a shielded region, the C$^+$ fractions is too low for 
          $\mathrm{C^++H_2\rightarrow CH^+ + H}$ to accelerate the CO formation.
          A decrease in the number of H atoms available for CO formation decreases the CO fractions.
       
       \item 
           A critical dust-to-gas mass ratio 
           $(\fracdg/\fracdgfid)_\mathrm{cri}$ 
           above which the CO formation is accelerated depends 
           sensitively on $T_\mathrm{chem}$. 
           For $T_\mathrm{chem}=300~$K, roughly speaking,
           $(\fracdg/\fracdgfid)_\mathrm{cri}$ is about $\fracdgfid$
           although it depends on $\nC$ and $Z$ 
           (Section \ref{sec:parametersurvey}).
           A decrease of $T_\mathrm{chem}$ from 300~K to 10~K 
           reduces $(\fracdg/\fracdgfid)_\mathrm{cri}$ significantly;
           $(\fracdg/\fracdgfid)_\mathrm{cri} \lesssim 0.1$ for weak-FUV and $\sim 0.01$ for strong-FUV.
       
%             If a sufficient amount of H$_2$ is produced, H$_2$ accelerates the CO formation. 
%             The CO over-production 
%           is triggered by endothermic reactions accelerated by 
%           excited H$_2$ (Section \ref{sec:CO_H2}).
%           As a result, CO is yielded more than predicted from the analytical model.

       \item 
       CO is self-shielded but also shielded by C$^0$.
Which shielding effect of CO 
           is effective depends on $\nCO/\nC$ at the low $\tau$ limit (Section \ref{sec:shielding}).
           For $\nCO/\nC > (\nCO/\nC)_\mathrm{cr} = 3\times 10^{-3}$ at $\tau\sim 0$, CO is self-shielded for
           $\NCO > 10^{14}~\psc$.
           When the CO fraction is sufficiently smaller than $(\nCO/\nC)_\mathrm{cri}$, 
           the C$^0$ attenuation increases the CO fraction for 
           $N(\mathrm{C}^0)>4\times 10^{16}~\psc$ before CO self-shielding becomes important.
%               For $\nCO/\nC<3\times 10^{-3}$ at the low $\tau$ limit, the spatial profiles of the CO fractions are predicted 
%               from Equation (\ref{fitting}) by taking into account the C$^0$ attenuation in the UV flux of the CO-dissociating photons.

   
       \item
       On the basis of the analytic formula shown in Section \ref{sec:analyticmodel}, 
       we developed a method to determine the spatial distributions of the CO number density in a given disk 
       structure by taking into account shielding effects 
       in the radiation field consistently in Section \ref{sec:COdiskana}.
%               From the analytic formula for $\nCO/\nC$ (Equation \ref{fitting}), we obtain the hydrogen nucleus number density required 
%               to produce a given $\nCO/\nC$ as a function of the FUV flux and metallicity (Equation \ref{nHconstrained}).
%               If the amount of carbon nuclei is fixed, lower metallicity produces a larger amount of CO.


\end{enumerate}

The chemical structure of debris disks have not been studied by 
taking into account detailed chemical processes consistent with 
radiation transfer. Particularly, it may be important to consider 
level populations of H$_2$ accurately 
because for 
lower $T_\mathrm{chem}$ and/or higher $\fracdg$ 
excited H$_2$ can promote endothermic 
formation reactions of CH$^+$ to accelerate the  CO formation. 
%We would like to emphasize that the H$_2$ formation rate of 
%Eley-Rideal mechanism is highly uncertain, and the uncertainty 
%affects the chemical structures especially when the dust-to-gas 
%mass ratio is larger (Section \ref{sec:observation}).  
%The H$_2$ formation rate of the Langmuir-Hinshelwood mechanism 
%can be much larger than that computed by the rate equations
%used in this paper if the stochastic H$_2$ formation process 
%is considered (Section \ref{sec:ER}).
%If the H$_2$ formation rate is larger than that in our study, 
%the CO formation accelerated by excited H$_2$ will become 
%important furthermore, and vice versa.

In this paper, an interstellar C/O ratio of 0.4 is used (Table \ref{tab:abundance}).
Recently \citet{Kama2016} and \citet{Bergin2016} however 
found evidence that C/O ratios could exceed unity
in PPDs, indicating that C/O ratios
change from the interstellar value during the evolution of PPDs.
If a part of the gas in a debris disk comes from a remnant PPD,
the C/O ratio could exceed unity also in a debris disk.
From our analytic formula for the CO fraction (Equation (\ref{fitting})),
it is expected that the CO fraction changes in proportion to ${\cal A}_\mathrm{O}$.


Our results were compared with the observational results of the 
debris disks around $\beta$ Pictoris and 49 Ceti 
in Sections \ref{sec:betaPic} and \ref{sec:49Ceti}, respectively.
%We take into account the disk structure approximately, 
%and confirm that the analytic formula is applicable.

For $\beta$ Pictoris as an example of gas-depleted debris disks, 
the steady chemical model is difficult to produce a sufficient amount of CO to fit the observational results 
if the fiducial parameter $T_\mathrm{chem}=300~$K 
of the Eley-Rideal mechanism 
for the formation of H$_2$ on warm dust grains
is adopted. 
The higher the metallicity, the more inefficient 
CO formation becomes, due to the lack of H atoms to 
facilitate its formation.
The CO column densities decrease as $\propto Z^{-1}$.
If $T_\mathrm{chem}$ is lower than 100~K, 
the steady chemical model may explain the observed amount of CO 
due to the H$_2$-accelerated CO formation 
only if $Z\sim 1$ (Figure \ref{fig:beta_comp}).
If $Z>1$, the steady chemical model cannot produce a sufficient amount of CO even when the H$_2$ formation on dust surfaces are efficient.
%Unless $T_\mathrm{chem}\le 100~$K and $Z=1$ are simultaneously satisfied, non-steady processes to provide CO are required.

For 49 Ceti as an example of gas-rich debris disks, 
the CO fraction is sensitive to the column density of carbon nuclei (Figure \ref{fig:49_comp}).
Our results show that there is a solution that may explain 
both the observed CO and C$^0$ column densities \citep{Higuchi2020} 
in a reasonable range of ${\cal N}_\mathrm{C}$ 
for $Z=1$ \citep{Higuchi2019}.
Since the dust-to-gas mass ratio 
is extremely small and 
shielding effects are significant, 
the CO fraction is insensitive to $T_\mathrm{chem}$.
As in the case of $\beta$ Pictoris, the observed CO column density 
cannot be reproduced for 49 Ceti if $Z>10$.



We should note that 
steady state chemical models can explain $\beta$ Pictoris and 49~Ceti only
under a very restricted set of parameters,
including enhanced production of H$_2$ on grains and $Z=1$. 
However, for $\beta$ Pictoris, 
complementary published observations mentioned in Section \ref{sec:caveat}
indicated much lower H or H$_2$ densities than are required. 
The combination of this work and the
constraints on the H and H$_2$ reveals that steady state chemistry cannot explain the
observations.
%, and that very likely C$^0$, C$^+$ and CO are produced by continuing
%secondary processes such as sublimation or collisions of icy bodies and that the
%chemistry of these injections has not had time to reach steady state for $\beta$ Pictoris.

As mentioned in Section \ref{sec:intro},
if the gas in $\beta$ Pictoris is secondary-origin and 
CO is supplied from solid bodies through collisions,
a required injection rate of CO is estimated by the observed CO mass divided by 
the CO photo-dissociation timescale.
Our results clearly show that the CO formation rate with $Z\gtrsim 1$ 
is too small to explain the observed amount of CO.
The total solid bodies required to provide CO during the stellar age 
is about $30~M_\otimes$ taking into  account an expected fraction of CO in the solid bodies
\citep[also see][]{Dent2014}.
As pointed out in Section \ref{sec:intro}, 
we need to consider detailed outgassing processes from solid bodies 
in order to investigate if the required amount of CO can be supplied.
In addition, the gas should be removed at a timescale comparable to the CO photo-dissociation 
timescale because C$^0$ and C$^+$ are accumulated if the gas is not removed. 
Viscous disk evolution \citep{Kral2017}, 
EUV photo-evaporation \citep{Nakatani2021}, and disk winds may be important to remove the gas.


%We should note that the comparisons with the observational results are made for the solar metallicity
%because at a given $\nC$, the models with higher metallicities produce less amount of CO because 
%hydrogen is associated with the CO formation.
%If the gas is secondary-origin and the metallicity is as high as $10^3$, 
%CO formation through chemical reactions may be not important.





%% IMPORTANT! The old "\acknowledgment" command has be depreciated. It was
%% not robust enough to handle our new dual anonymous review requirements and
%% thus been replaced with the acknowledgment environment. If you try to 
%% compile with \acknowledgment you will get an error print to the screen
%% and in the compiled pdf.
%% 
%% Also note that the akcnowlodgment environment does not support long amounts of text. If you have a lot of people and institutions to acknowledge, do not use this command. Instead, create a new \section{Acknowledgments}.

\begin{acknowledgments}
We thank the reviewers for providing us many constructive comments that improve this paper signficantly.
We thank Gianni Cataldi, Satoshi Yamamoto, Nami Sakai, Munetake Momose, Kenji Furuya, and Masanobu Kunitomo for valuable discussions.
This work was supported in part by JSPS KAKENHI Grant Numbers 21H00056 (K.I.),
%JP17K05632, JP17H01103, JP17H01105, JP18H05436, JP18H05438 (H.K.), 
22H00179, 22H01278, 21K03642, 18H05436, 18H05438 (H.K.),
18K03713, 19H05090 (A.E.H.), 18H05222, 20H05844, 20H05847 (Y.A.).
Y.A. also acknowledges support by NAOJ ALMA Scientific Research Grant code 2019-13B.

\end{acknowledgments}


{\software{ Meudon PDR code \citep{LeP2006}, numpy \citep{Numpy},
Matplotlib \citep{Matplotlib}}}


%% To help institutions obtain information on the effectiveness of their 
%% telescopes the AAS Journals has created a group of keywords for telescope 
%% facilities.
%
%% Following the acknowledgments section, use the following syntax and the
%% \facility{} or \facilities{} macros to list the keywords of facilities used 
%% in the research for the paper.  Each keyword is check against the master 
%% list during copy editing.  Individual instruments can be provided in 
%% parentheses, after the keyword, but they are not verified.

%\vspace{5mm}
%\facilities{HST(STIS), Swift(XRT and UVOT), AAVSO, CTIO:1.3m,
%CTIO:1.5m,CXO}

%% Similar to \facility{}, there is the optional \software command to allow 
%% authors a place to specify which programs were used during the creation of 
%% the manuscript. Authors should list each code and include either a
%% citation or url to the code inside ()s when available.


%% Appendix material should be preceded with a single \appendix command.
%% There should be a \section command for each appendix. Mark appendix
%% subsections with the same markup you use in the main body of the paper.

%% Each Appendix (indicated with \section) will be lettered A, B, C, etc.
%% The equation counter will reset when it encounters the \appendix
%% command and will number appendix equations (A1), (A2), etc. The
%% Figure and Table counter will not reset.

\appendix

%------------------------------------------------------------------
\section{ An Analytic Formula for the CO Fraction in 
the Simplified Chemical Network without Hydrogen Molecule}\label{app:COana}
%------------------------------------------------------------------

In this appendix, we construct an analytical model for CO formation by the simplified 
CO formation network without H$_2$ shown in Figure \ref{fig:COnetwork}.
With the rate coefficients listed in Table \ref{tab:reaction},
the four chemical balance equations for CO, CO$^+$, OH, and CH$^+$ are solved, where 
$n(e^-)$ is equal to $\nCp$, 
$n(\mathrm{H})$ is set to be $\nH$ in the H$_2$-free environment, most oxygen is in 
the atomic form ($n(\mathrm{O}) \sim n_\mathrm{O}$), and the C$^0$ and C$^+$ fractions are obtained 
from Equation (\ref{nCIana}) with $\nCp = \nC - n(\mathrm{C}^0)$, where 
$\nCO\ll \nCI , \nCp$ is assumed.
To investigate the contribution of each path to the CO fraction, the analytical form of the CO fraction 
is divided into three parts,
\begin{equation}
        \nCOana \equiv \nCO_\mathrm{oh} +\nCO_\mathrm{chp} +  \nCO_\mathrm{woH},
  \label{COana}
\end{equation}
where $\nCO_\mathrm{oh}$, $\nCO_\mathrm{chp}$, and $\nCO_\mathrm{woH}$ are the CO fractions formed 
through the three pathways.


%\begin{equation}
%  \kcoph \nCOp n(\mathrm{H}) + \kohc \nOH n(\mathrm{C^0}) 
%  + \kcto n(\mathrm{C_2}) n(\mathrm{O}) + \kco n(\mathrm{C})n(\mathrm{O}) 
%  = \aCO \nCO.
%  \label{COeq}
%\end{equation}
%for CO, 
%%%%%
%\begin{eqnarray}
%    \kchpo \nCHp n(\mathrm{O})+
%    \kohcp \nOH\nCp + \kcpo \nCp n(\mathrm{O})
%    &=& \nonumber \\
%    && \hspace{-5cm} \aCOp \nCOp + \kcoph \nCOp n(\mathrm{H})  +  \kcope \nCOp n(e^-)
%  \label{COpeq}
%\end{eqnarray}
%for CO$^+$, 
%\begin{equation}
%%  \koh n(\mathrm{O})n(\mathrm{H}) = \aOH \nOH + \koho \nOH n(\mathrm{O}) + \kohcp \nOH \nCp
%%  + \kohc \nOH n(\mathrm{C})
%%  \koh n(\mathrm{O})n(\mathrm{H}) = \aOH \nOH + \kohcp \nOH \nCp
%%  + \kohc \nOH n(\mathrm{C^0})
%  \koh n(\mathrm{O})n(\mathrm{H}) = \aOH \nOH 
%  \label{OHeq}
%\end{equation}
%for OH, 
%\begin{equation}
%    \kcph \nCp n(\mathrm{H}) = 
%    \aCHp \nCHp + \kchph \nCHp n(\mathrm{H}) 
%  \label{CHpeq}
%\end{equation}
%for CH$^+$,
%and 
%\begin{equation}
%        \kcc n(\mathrm{C^0})^2 = \kcto n(\mathrm{C_2})n(\mathrm{O})
%  \label{C2eq}
%\end{equation}
%for C$_2$.
%Eliminating $\nCOp$, $\nOH$, and $n(\mathrm{C}_2)$ by using Equations (\ref{COeq})-(\ref{C2eq}), 
%one obtains the CO fraction.
%As mentioned in Figure \ref{fig:COnetwork}, 
%there are three pathways to form CO, \Pathchp, \Pathoh, and \Pathwoh.
%To investigate the contribution of each path to the CO fraction, the analytical form of the CO fraction 
%is divided into three parts,
%\begin{equation}
%        \nCOana \equiv \nCO_\mathrm{oh} +\nCO_\mathrm{chp} +  \nCO_\mathrm{woH},
%  \label{COana}
%\end{equation}
%where $\nCO_\mathrm{oh}$, $\nCO_\mathrm{chp}$, and $\nCO_\mathrm{woH}$ are the CO fractions formed 
%through the three pathways.
%Their expressions are 
%%%%
%\begin{equation}
%  \nCO_\mathrm{oh} \equiv %\frac{\kcoph\nCOp n(\mathrm{H})}{\aCO} = 
%\frac{\koh}{\aCO \aOH} n_\mathrm{O}\nC n_\mathrm{H}\left(
%%  \left(1 + \frac{\aCOp+ \kcope f_\mathrm{cp}\nC}{\kcoph \nH} \right)^{-1}
%    \Phi
%  \kohcp f_\mathrm{cp} + 
%  \kohc f_\mathrm{c}\right),
%\label{COoh}
%\end{equation}
%%%%
%\begin{equation}
%        \nCO_\mathrm{chp} \equiv 
%  \frac{\kchpo\kcph}{\aCO\kchph}\nO f_\mathrm{cp} \nC
%%  \left(1 + \frac{\aCOp + \kcope f_\mathrm{cp}\nC}{\kcoph \nH}
%%\right)^{-1}
%  \Phi
%\left( 1 + \frac{\aCHp}{\kchph \nH} \right)^{-1},
%\label{COchp}
%\end{equation}
%%%%
%and
%%%%
%\begin{equation}
%  \nCO_\mathrm{woH} \equiv 
%  \frac{\kcc f_\mathrm{c}^2 \nC^2}{\aCO} + \frac{\kco f_\mathrm{c} \nC \nO}{\aCO},
%\label{phioh1}
%\end{equation}
%where $\Phi=(1 + (\aCOp+ \kcope f_\mathrm{cp}\nC)/(\kcoph \nH))^{-1}$, 
%$n(e^-)\sim \nCp$ is used,
%%(Figure \ref{fig:CI_CII}b), 
%and 
%$n(\mathrm{H})$ is set to be $\nH$ in the H$_2$-free environment, most oxygen is in 
%the atomic form, $n(\mathrm{O}) \sim n_\mathrm{O}$, 
%and $f_\mathrm{cp}(\xi)=1-f_\mathrm{c}(\xi)$ is the C$^+$ fraction (Equation (\ref{ionCana})).

\begin{table}[htpb]
	\centering
	\begin{tabular}{|l|l|}
    \hline
		chemical reaction & rate coefficient \\
		\hline
		\hline
                $\mathrm{C^+ + H\rightarrow CH^+ + h\nu}$ & 
                $k_\mathrm{cp,h}=2.29\times 10^{-17}(T/300)^{-0.42}$ \\
                $\mathrm{CH^+ + H\rightarrow H_2 + C^+}$ & 
                $k_\mathrm{chp,h}=7.5\times 10^{-10}$ \\
                $\mathrm{CH^+ + O\rightarrow CO^+ + H}$ & 
                $k_\mathrm{chp,o}=3.5\times 10^{-10}$ \\
    $\mathrm{O + H\rightarrow OH + h\nu}$ & $k_\mathrm{o,h}=9.9\times 10^{-19} (T/300)^{-0.38}$ \\
    $\mathrm{OH + C^+ \rightarrow H + CO^+}$ & $k_\mathrm{oh,cp}=7.7\times 10^{-10} (T/300)^{-0.5}$ \\
%    $\mathrm{OH + O \rightarrow H + O_2}$ & $k_\mathrm{oh,o}=3.5\times 10^{-11}$ \\
    $\mathrm{OH + C \rightarrow H + CO}$ & $k_\mathrm{oh,c}=1.0\times 10^{-10}$ \\
    $\mathrm{CO^+ + e^- \rightarrow C + O}$ & $k_\mathrm{cop,e}=4.58\times 10^{-7}$ \\
    $\mathrm{CO^+ + H \rightarrow CO + H^+}$ & $k_\mathrm{cop,h}=7.5\times 10^{-10}$ \\
    $\mathrm{OH + h\nu \rightarrow O + H}$ & $\alpha_\mathrm{oh}=4.3\times 10^{-6}\chioh$ \\
    $\mathrm{CO^+ + h\nu \rightarrow C^++O}$ & $\alpha_\mathrm{cop}=1.35\times 10^{-8}\chico^{1.3}$ \\
    $\mathrm{CO + h\nu \rightarrow C+O}$ & $\alpha_\mathrm{co}= 10^{-10}\chico$ \\
    \hline
	\end{tabular}
	\caption{
    Chemical reaction rates that are important in the CO formation shown in Figure \ref{fig:COnetwork}.
    The rate coefficients in the first eight reactions are shown 
    in an unit of cm$^3$~s$^{-1}$.
    The last three rate coefficients correspond to the unattenuated photo-dissociation rates in s$^{-1}$.
	}
	\label{tab:reaction}
\end{table}

    The contributions of the three CO formation paths to the CO fraction for $Z=1$ and $Z=10^3$ 
    are illustrated in Figures \ref{fig:ana_z}a and \ref{fig:ana_z}b, respectively. 
    The spectral type is fixed to A5V.
    Which CO formation path is important depends both on the gas metallicity and the gas density.
    For $Z=1$, CO forms mainly from CH$^+$ for low densities (Figure \ref{fig:ana_z}a).  
    Path$_\mathrm{OH}$ becomes the dominant pathway to form CO 
    when $\nC$ exceeds $\sim 10~\pcc$. %, which is comarable to $n_\mathrm{C,cri}$ 
%    above which C$^+$ is depleted.  

    
    As $Z$ increases, 
    both $\nCO_\mathrm{chp}$ and $\nCO_\mathrm{oh}$ decrease at a fixed $\nC$ 
    since $\nH$ decreases with $Z$.
    By contrast, $\nCO_\mathrm{woH}$ is independent of $Z$ at a fixed $\nC$ because hydrogen 
    is not involved in \Pathwoh.
    As shown in Figure \ref{fig:ana_z}b, 
    for $Z=10^3$, $\nCO_\mathrm{woH}$ dominates over $\nCO_\mathrm{chp}$ by an order of magnitude, and 
    {\Pathwoh} becomes the dominant pathway to form CO for lower densities.
    When $\nC>10^3~\pcc$, the CO formation through {\Pathoh} determines the CO fraction.



%The factor $\Phi$ indicates a reduction effect of CO owing to the removal of 
%the intermediate species (CO$^+$ and OH) through the chemical reactions not being 
%involved in CO formation.
%Equation (\ref{nCOana}) shows that $\Phi$ consists of two terms $\Phi_\mathrm{cop}
%\Phi_\mathrm{oh1}$ and $\Phi_\mathrm{oh2}$, which correspond to 
%the path 1 and path 2, respectively, shown in Figure \ref{fig:COnetwork}.

%%%%%%
\begin{figure}[htpb]
  \centering
 % \includegraphics[width=15cm]{CO_ana_z.eps}
  \includegraphics[width=15cm]{Fig12.pdf}
  \caption{
      The contributions of the three different paths to the analytic CO fraction as a function 
      of $\nC$ for (a)$Z=1$ and (b)$Z=10^3$.
  }
  \label{fig:ana_z}
\end{figure}

%Let us examine the asymptotic behaviors of $\Phi_\mathrm{cop}\Phi_\mathrm{oh1}$ and
%$\Phi_\mathrm{oh2}$. 
%
%\subsection{ An approximate Expression of the CO fraction }


We focus on the high density limit where {\Pathoh} is dominant.
The path $\mathrm{O}\rightarrow \mathrm{OH}\rightarrow \mathrm{CO}$ 
contributes to the CO formation more than 
$\mathrm{O}\rightarrow \mathrm{OH}\rightarrow\mathrm{CO^+}\rightarrow \mathrm{CO}$.
Using the fact that the main destruction processes of OH and CO are photo-dissociation, 
we obtain the CO fraction as follows:
%In the high density limit, the path 
%$\mathrm{O}\rightarrow \mathrm{OH}\rightarrow \mathrm{CO}$ dominates over 
%$\mathrm{O}\rightarrow \mathrm{OH}\rightarrow \mathrm{CO^+} \rightarrow \mathrm{CO}$, and 
%the CO fraction can be easily derived analytically as follows.
The OH abundance is determined by the balance between the radiative association 
$\mathrm{O+H\rightarrow OH+h\nu}$ 
(the rate coefficient $\koh$) and the OH photo-dissociation (the rate coefficient $\aOH$).
Eventually, OH is combined with C to form CO with a rate coefficient of $\kohc$.
Considering the CO photo-dissociation (the rate coefficient $\aCO$), 
the analytical form of the CO fraction is given by 
\begin{eqnarray}
    \frac{n(\mathrm{CO})_\mathrm{oh}}{\nC} &=& \frac{\koh\kohc}{\aCO\aOH}\nO\nH \nonumber \\
   &=& 7.4\times 10^{-17}~T_{300}^{-0.38}~\frac{\Ao}{ {\cal A}_\mathrm{O,ism}} \zeta^2,
\label{COana0}
\end{eqnarray}
where
\begin{equation}
    \zeta = \frac{\sqrt{Z}\nH}{\chi},\;\;\;\mathrm{where}\;\;\chi = \sqrt{\chico\chioh},
\end{equation}
$T_{300}=T/300~\mathrm{K}$, 
${\cal A}_\mathrm{O}$ is the oxygen abundance of the gas phase
and ${\cal A}_\mathrm{O,ism} = 3.2\times 10^{-4}Z$ 
(Table \ref{tab:abundance}),  
and we use the fact that $\aCO\propto \chico$ and $\aOH \propto \chioh$.
Equation (\ref{COana0}) shows that 
the CO fraction depends on $\nH Z^{0.5}\chi^{-1}$, which is quite similar to $\eta$.
The difference between $\eta$ and $Z^{0.5}\nH \chi^{-1}$ comes from the 
temperature dependence shown in Equation (\ref{COana0}).



%{\Pathoh} consist of two paths $\mathrm{O}\rightarrow \mathrm{OH}\rightarrow \mathrm{CO}$ 
%and $\mathrm{O}\rightarrow \mathrm{OH}\rightarrow\mathrm{CO^+}\rightarrow \mathrm{CO}$ 
%(Figure \ref{fig:COnetwork}).
%In the high density limit, the former, which corresponds to the second term of the right-hand side 
%of Equation (\ref{COoh}), contributes to the CO formation more than the latter.
%In the density range where {\Pathoh} is the dominant pathway, 
%the CO fraction is expressed analytically because $\nCO_\mathrm{ana}\sim \nCO_\mathrm{OH}$.
%Since $\nC>\nCcri$ as shown in Figure \ref{fig:ana_z}, 
%substituting $\nCp \sim \xi^{-1/2}\nC$, $n(\mathrm{C}^0)\sim \nC$, $\Phi \sim 1$, 
%Substituting $n(\mathrm{C}^0)\sim \nC$ and the rate coefficients shown in 
%Table \ref{tab:reaction} into Equation (\ref{COoh}), one obtains 
%\begin{equation}
%    \frac{\nCO_\mathrm{oh,s}}{\nC} = \frac{\Ao}{3.2\times 10^{-4}}\left[
%    2.3\times 10^{-13}T_{300}^{-0.38} \zeta^2
%        + 9.2\times 10^{-12} \Ac^{-1/2} 
%        T_{300}^{-0.47} \sqrt{\frac{\nH}{\chioh}}\times \zeta 
%\right],
%    \label{nCOohs}
%\end{equation}
%\begin{equation}
%    \frac{\nCO_\mathrm{oh}}{\nC} = 7.4\times 10^{-17}~T_{300}^{-0.38}~\frac{\Ao}{ {\cal A}_\mathrm{O,ism}} 
%\zeta^2,
%    \label{nCOohs}
%\end{equation}
%where 
%\begin{equation}
%    \zeta = \frac{\sqrt{Z}\nH}{\chi},\;\;\;\mathrm{where}\;\;\chi = \sqrt{\chico\chioh}.
%\end{equation}
%where $\Ao$ is the oxygen elemental abundance at $Z=1$.
%The first and second terms in the right-hand side of Equation (\ref{nCOohs}) correspond to 
%the reaction paths $\mathrm{O\rightarrow OH \rightarrow CO}$
%$\mathrm{O\rightarrow OH \rightarrow CO^+ \rightarrow CO}$, respectively (Figure \ref{fig:COnetwork}).
%In the high density limit, the second term dominates over the first term.

Equation (\ref{COana0}) shows that $\nCOana$ is expected to be characterized by $\zeta$ at least for high densities
where $\nCO_\mathrm{ana}\sim \nCO_\mathrm{oh}$. 
In order to check this, we fit $\nCOana$ with a function of $\eta = \nH Z^a \chi^{b}$, 
where $a$ and $b$ are the fitting parameters.
As shown in Figure \ref{fig:COana}, we found that 
$\nCOana$ is insensitive to both $Z$ and $\chi$  for $\eta >10^{5}~\pcc$ 
if $\eta$ with $a=0.4$ and $b=-1.1$ is taken as the horizontal axis.
The difference between $\zeta$ and $\eta$ with $a=0.4$ and $b=-1.1$ comes from 
the negative temperature dependence of $\nCO_\mathrm{oh}$ (Equation (\ref{COana0})).
The gas temperatures increase when either $Z$ or $\chi$ increaes.
Even for low densities where {\Pathoh} is no longer dominant CO formation path,
$\nCOana$ is characterized by $\eta=\nH Z^{0.4}\chi^{-1.1}$ reasonably well. 
A fitting function of the CO fraction is given by 
\begin{equation}
    \frac{\nCO_\mathrm{ana}}{\nC} = \frac{\Ao}{ {\cal A}_\mathrm{O,ism}} 
    \left(
        10^{-14} \eta^{1.8}
    + 6.0\times 10^{-11} \eta \right).
    \label{fitting0}
\end{equation}


%%%%%%
\begin{figure}[htpb]
  \centering
%  \includegraphics[width=7cm]{CO_ana.eps}
  \includegraphics[width=7cm]{Fig13.pdf}
  \caption{
      The CO fraction estimated by Equation (\ref{COana}) for various metallicities and spectral types. 
      The horizontal axis is $\eta = \nH Z^{0.4}\chi^{-1.1} = \zeta (Z\chi)^{-0.1}$.
      The gray dotted line corresponds to the fitting function shown by Equation (\ref{fitting0}).
  }
  \label{fig:COana}
\end{figure}
%%%%%%

%------------------------------------------------------------------
%\section{An Analytic Formula for the CO Fraction in Deep Inside}\label{app:COana}
%------------------------------------------------------------------

%In the regions where $\chico$ becomes sufficiently small owing to the shielding effects, 
%CO destroy through dissociative charge exchange reactions with 
%He$^+$, H$_3^+$, N$^+$, and so on.


%% For this sample we use BibTeX plus aasjournals.bst to generate the
%% the bibliography. The sample631.bib file was populated from ADS. To
%% get the citations to show in the compiled file do the following:
%%
%% pdflatex sample631.tex
%% bibtext sample631
%% pdflatex sample631.tex
%% pdflatex sample631.tex

%\bibliography{sample631}{}
%\bibliography{ms}{}
%\bibliographystyle{aasjournal}
\begin{thebibliography}{}
\expandafter\ifx\csname natexlab\endcsname\relax\def\natexlab#1{#1}\fi
\providecommand{\url}[1]{\href{#1}{#1}}
\providecommand{\dodoi}[1]{doi:~\href{http://doi.org/#1}{\nolinkurl{#1}}}
\providecommand{\doeprint}[1]{\href{http://ascl.net/#1}{\nolinkurl{http://ascl.net/#1}}}
\providecommand{\doarXiv}[1]{\href{https://arxiv.org/abs/#1}{\nolinkurl{https://arxiv.org/abs/#1}}}

\bibitem[{{{\'A}d{\'a}mkovics} {et~al.}(2011){{\'A}d{\'a}mkovics}, {Glassgold},
  \& {Meijerink}}]{Adamkovics2011}
{{\'A}d{\'a}mkovics}, M., {Glassgold}, A.~E., \& {Meijerink}, R. 2011, \apj,
  736, 143

\bibitem[{{Ag{\'u}ndez} {et~al.}(2010){Ag{\'u}ndez}, {Goicoechea},
  {Cernicharo}, {Faure}, \& {Roueff}}]{Agndez2010}
{Ag{\'u}ndez}, M., {Goicoechea}, J.~R., {Cernicharo}, J., {Faure}, A., \&
  {Roueff}, E. 2010, \apj, 713, 662

\bibitem[{{Ahrens} \& {O'Keefe}(1972)}]{Ahrens1972}
{Ahrens}, T.~J., \& {O'Keefe}, J.~D. 1972, Moon, 4, 214

\bibitem[{{Bakes} \& {Tielens}(1994)}]{BakesTielens1994}
{Bakes}, E.~L.~O., \& {Tielens}, A.~G.~G.~M. 1994, \apj, 427, 822

\bibitem[{{Bergin} {et~al.}(2016){Bergin}, {Du}, {Cleeves}, {Blake}, {Schwarz},
  {Visser}, \& {Zhang}}]{Bergin2016}
{Bergin}, E.~A., {Du}, F., {Cleeves}, L.~I., {et~al.} 2016, \apj, 831, 101

\bibitem[{{Bertoldi} \& {Draine}(1996)}]{Bertoldi1996}
{Bertoldi}, F., \& {Draine}, B.~T. 1996, \apj, 458, 222

\bibitem[{{Bron} {et~al.}(2014){Bron}, {Le Bourlot}, \& {Le Petit}}]{Bron2014}
{Bron}, E., {Le Bourlot}, J., \& {Le Petit}, F. 2014, \aap, 569, A100

\bibitem[{{Burns} {et~al.}(1979){Burns}, {Lamy}, \& {Soter}}]{Burns1979}
{Burns}, J.~A., {Lamy}, P.~L., \& {Soter}, S. 1979, \icarus, 40, 1

\bibitem[{{Cataldi} {et~al.}(2014){Cataldi}, {Brandeker}, {Olofsson},
  {Larsson}, {Liseau}, {Blommaert}, {Fridlund}, {Ivison}, {Pantin},
  {Sibthorpe}, {Vandenbussche}, \& {Wu}}]{Cataldi2014}
{Cataldi}, G., {Brandeker}, A., {Olofsson}, G., {et~al.} 2014, \aap, 563, A66

\bibitem[{{Cataldi} {et~al.}(2018){Cataldi}, {Brandeker}, {Wu}, {Chen}, {Dent},
  {de Vries}, {Kamp}, {Liseau}, {Olofsson}, {Pantin}, \&
  {Roberge}}]{Cataldi2018}
{Cataldi}, G., {Brandeker}, A., {Wu}, Y., {et~al.} 2018, \apj, 861, 72

\bibitem[{{Chen} {et~al.}(2006){Chen}, {Sargent}, {Bohac}, {Kim},
  {Leibensperger}, {Jura}, {Najita}, {Forrest}, {Watson}, {Sloan}, \&
  {Keller}}]{Chen2006}
{Chen}, C.~H., {Sargent}, B.~A., {Bohac}, C., {et~al.} 2006, \apjs, 166, 351

\bibitem[{{Compi{\`e}gne} {et~al.}(2011){Compi{\`e}gne}, {Verstraete}, {Jones},
  {Bernard}, {Boulanger}, {Flagey}, {Le Bourlot}, {Paradis}, \&
  {Ysard}}]{Compigne2011}
{Compi{\`e}gne}, M., {Verstraete}, L., {Jones}, A., {et~al.} 2011, \aap, 525,
  A103

\bibitem[{{Cuppen} {et~al.}(2006){Cuppen}, {Morata}, \& {Herbst}}]{Cuppen2006}
{Cuppen}, H.~M., {Morata}, O., \& {Herbst}, E. 2006, \mnras, 367, 1757

\bibitem[{{Dent} {et~al.}(2014){Dent}, {Wyatt}, {Roberge}, {Augereau},
  {Casassus}, {Corder}, {Greaves}, {de Gregorio-Monsalvo}, {Hales}, {Jackson},
  {Hughes}, {Lagrange}, {Matthews}, \& {Wilner}}]{Dent2014}
{Dent}, W.~R.~F., {Wyatt}, M.~C., {Roberge}, A., {et~al.} 2014, Science, 343,
  1490

\bibitem[{{Dohnanyi}(1969)}]{Dohnanyi1969}
{Dohnanyi}, J.~S. 1969, \jgr, 74, 2531

\bibitem[{{Draine}(1978)}]{Draine1978}
{Draine}, B.~T. 1978, \apjs, 36, 595

\bibitem[{{Draine} \& {Bertoldi}(1996)}]{DraineBertoldi1996}
{Draine}, B.~T., \& {Bertoldi}, F. 1996, \apj, 468, 269

\bibitem[{{Draine} \& {Li}(2001)}]{DraineLi2001}
{Draine}, B.~T., \& {Li}, A. 2001, \apj, 551, 807

\bibitem[{{Federman} {et~al.}(1979){Federman}, {Glassgold}, \&
  {Kwan}}]{Fed1979}
{Federman}, S.~R., {Glassgold}, A.~E., \& {Kwan}, J. 1979, \apj, 227, 466

\bibitem[{{Fern{\'a}ndez} {et~al.}(2006){Fern{\'a}ndez}, {Brandeker}, \&
  {Wu}}]{Fernandez2006}
{Fern{\'a}ndez}, R., {Brandeker}, A., \& {Wu}, Y. 2006, \apj, 643, 509

\bibitem[{{Freudling} {et~al.}(1995){Freudling}, {Lagrange}, {Vidal-Madjar},
  {Ferlet}, \& {Forveille}}]{Freudling1995}
{Freudling}, W., {Lagrange}, A.-M., {Vidal-Madjar}, A., {Ferlet}, R., \&
  {Forveille}, T. 1995, \aap, 301, 231

\bibitem[{{Genda} {et~al.}(2015){Genda}, {Kobayashi}, \& {Kokubo}}]{Genda2015}
{Genda}, H., {Kobayashi}, H., \& {Kokubo}, E. 2015, \apj, 810, 136

\bibitem[{{Goicoechea} \& {Le Bourlot}(2007)}]{Goi2007}
{Goicoechea}, J.~R., \& {Le Bourlot}, J. 2007, \aap, 467, 1

\bibitem[{{Goicoechea} \& {Roncero}(2022)}]{Goicoechea2022}
{Goicoechea}, J.~R., \& {Roncero}, O. 2022, \aap, 664, A190

\bibitem[{{Goicoechea} {et~al.}(2016){Goicoechea}, {Pety}, {Cuadrado},
  {Cernicharo}, {Chapillon}, {Fuente}, {Gerin}, {Joblin}, {Marcelino}, \&
  {Pilleri}}]{Goicoechea2016}
{Goicoechea}, J.~R., {Pety}, J., {Cuadrado}, S., {et~al.} 2016, \nat, 537, 207

\bibitem[{{Gorti} {et~al.}(2009){Gorti}, {Dullemond}, \&
  {Hollenbach}}]{Gorti2009}
{Gorti}, U., {Dullemond}, C.~P., \& {Hollenbach}, D. 2009, \apj, 705, 1237

\bibitem[{{Gorti} \& {Hollenbach}(2004)}]{Gorti2004}
{Gorti}, U., \& {Hollenbach}, D. 2004, \apj, 613, 424

\bibitem[{{Gorti} \& {Hollenbach}(2009)}]{GortiHollenbach2009}
---. 2009, \apj, 690, 1539

\bibitem[{{Grigorieva} {et~al.}(2007){Grigorieva}, {Th{\'e}bault},
  {Artymowicz}, \& {Brandeker}}]{Grigorieva2007}
{Grigorieva}, A., {Th{\'e}bault}, P., {Artymowicz}, P., \& {Brandeker}, A.
  2007, \aap, 475, 755

\bibitem[{{Habart} {et~al.}(2011){Habart}, {Abergel}, {Boulanger}, {Joblin},
  {Verstraete}, {Compi{\`e}gne}, {Pineau Des For{\^e}ts}, \& {Le
  Bourlot}}]{Habart2011}
{Habart}, E., {Abergel}, A., {Boulanger}, F., {et~al.} 2011, \aap, 527, A122

\bibitem[{{Habing}(1968)}]{Habing1968}
{Habing}, H.~J. 1968, {Bull. Astron. Inst. Netherlands}, 19, 421

\bibitem[{{Haisch} {et~al.}(2001){Haisch}, {Lada}, \& {Lada}}]{Heisch2001}
{Haisch}, Jr., K.~E., {Lada}, E.~A., \& {Lada}, C.~J. 2001, \apjl, 553, L153

\bibitem[{{Heap} {et~al.}(2000){Heap}, {Lindler}, {Lanz}, {Cornett}, {Hubeny},
  {Maran}, \& {Woodgate}}]{Heap2000}
{Heap}, S.~R., {Lindler}, D.~J., {Lanz}, T.~M., {et~al.} 2000, \apj, 539, 435

\bibitem[{{Heays} {et~al.}(2017){Heays}, {Bosman}, \& {van
  Dishoeck}}]{Heays2017}
{Heays}, A.~N., {Bosman}, A.~D., \& {van Dishoeck}, E.~F. 2017, \aap, 602, A105

\bibitem[{{Herr{\'a}ez-Aguilar} {et~al.}(2014){Herr{\'a}ez-Aguilar},
  {Jambrina}, {Men{\'e}ndez}, {Aldegunde}, {Warmbier}, \&
  {Aoiz}}]{Herrez-Aguilar2014}
{Herr{\'a}ez-Aguilar}, D., {Jambrina}, P.~G., {Men{\'e}ndez}, M., {et~al.}
  2014, Physical Chemistry Chemical Physics (Incorporating Faraday
  Transactions), 16, 24800

\bibitem[{{Hierl} {et~al.}(1997){Hierl}, {Morris}, \& {Viggiano}}]{Hierl1997}
{Hierl}, P.~M., {Morris}, R.~A., \& {Viggiano}, A.~A. 1997, \jcp, 106, 10145

\bibitem[{{Higuchi} {et~al.}(2020){Higuchi}, {K{\'o}sp{\'a}l}, {Mo{\'o}r},
  {Nomura}, \& {Yamamoto}}]{Higuchi2020}
{Higuchi}, A.~E., {K{\'o}sp{\'a}l}, {\'A}., {Mo{\'o}r}, A., {Nomura}, H., \&
  {Yamamoto}, S. 2020, \apj, 905, 122

\bibitem[{{Higuchi} {et~al.}(2017){Higuchi}, {Sato}, {Tsukagoshi}, {Sakai},
  {Iwasaki}, {Momose}, {Kobayashi}, {Ishihara}, {Watanabe}, {Kaneda}, \&
  {Yamamoto}}]{Higuchi2017}
{Higuchi}, A.~E., {Sato}, A., {Tsukagoshi}, T., {et~al.} 2017, \apjl, 839, L14

\bibitem[{{Higuchi} {et~al.}(2019){Higuchi}, {Saigo}, {Kobayashi}, {Iwasaki},
  {Momose}, {Sakai}, {Soon}, {Kunitomo}, {Ishihara}, \&
  {Yamamoto}}]{Higuchi2019}
{Higuchi}, A.~E., {Saigo}, K., {Kobayashi}, H., {et~al.} 2019, submitted in ApJ

\bibitem[{{Hollenbach} {et~al.}(1994){Hollenbach}, {Johnstone}, {Lizano}, \&
  {Shu}}]{Hollenbach1994}
{Hollenbach}, D., {Johnstone}, D., {Lizano}, S., \& {Shu}, F. 1994, \apj, 428,
  654

\bibitem[{{Howarth}(2011)}]{Howarth2011}
{Howarth}, I.~D. 2011, \mnras, 413, 1515

\bibitem[{{Hughes} {et~al.}(2018){Hughes}, {Duch{\^e}ne}, \&
  {Matthews}}]{Hughes2018}
{Hughes}, A.~M., {Duch{\^e}ne}, G., \& {Matthews}, B.~C. 2018, \araa, 56, 541

\bibitem[{{Hughes} {et~al.}(2008){Hughes}, {Wilner}, {Kamp}, \&
  {Hogerheijde}}]{Hughes2008}
{Hughes}, A.~M., {Wilner}, D.~J., {Kamp}, I., \& {Hogerheijde}, M.~R. 2008,
  \apj, 681, 626

\bibitem[{{Hughes} {et~al.}(2017){Hughes}, {Lieman-Sifry}, {Flaherty}, {Daley},
  {Roberge}, {K{\'o}sp{\'a}l}, {Mo{\'o}r}, {Kamp}, {Wilner}, {Andrews},
  {Kastner}, \& {{\'A}brah{\'a}m}}]{Hughes2017}
{Hughes}, A.~M., {Lieman-Sifry}, J., {Flaherty}, K.~M., {et~al.} 2017, \apj,
  839, 86

\bibitem[{{Hunter}(2007)}]{Matplotlib}
{Hunter}, J.~D. 2007, Computing in Science and Engineering, 9, 90

\bibitem[{{Ivanovskaya} {et~al.}(2010){Ivanovskaya}, {Zobelli},
  {Teillet-Billy}, {Rougeau}, {Sidis}, \& {Briddon}}]{Ivanovskaya2010}
{Ivanovskaya}, V.~V., {Zobelli}, A., {Teillet-Billy}, D., {et~al.} 2010, \prb,
  82, 245407

\bibitem[{{Iwasaki} {et~al.}(2001){Iwasaki}, {Tanaka}, {Nakazawa}, \&
  {Hiroyuki}}]{Iwasaki2001}
{Iwasaki}, K., {Tanaka}, H., {Nakazawa}, K., \& {Hiroyuki}, E. 2001, \pasj, 53,
  321

\bibitem[{{Joblin} {et~al.}(2018){Joblin}, {Bron}, {Pinto}, {Pilleri}, {Le
  Petit}, {Gerin}, {Le Bourlot}, {Fuente}, {Berne}, {Goicoechea}, {Habart},
  {K{\"o}hler}, {Teyssier}, {Nagy}, {Montillaud}, {Vastel}, {Cernicharo},
  {R{\"o}llig}, {Ossenkopf-Okada}, \& {Bergin}}]{Joblin2018}
{Joblin}, C., {Bron}, E., {Pinto}, C., {et~al.} 2018, \aap, 615, A129

\bibitem[{{Jura}(1974)}]{Jura1974}
{Jura}, M. 1974, \apj, 191, 375

\bibitem[{{Jura} {et~al.}(1998){Jura}, {Malkan}, {White}, {Telesco}, {Pina}, \&
  {Fisher}}]{Jura1998}
{Jura}, M., {Malkan}, M., {White}, R., {et~al.} 1998, \apj, 505, 897

\bibitem[{{Jura} {et~al.}(1993){Jura}, {Zuckerman}, {Becklin}, \&
  {Smith}}]{Jura1993}
{Jura}, M., {Zuckerman}, B., {Becklin}, E.~E., \& {Smith}, R.~C. 1993, \apjl,
  418, L37

\bibitem[{{Kama} {et~al.}(2016){Kama}, {Bruderer}, {van Dishoeck},
  {Hogerheijde}, {Folsom}, {Miotello}, {Fedele}, {Belloche}, {G{\"u}sten}, \&
  {Wyrowski}}]{Kama2016}
{Kama}, M., {Bruderer}, S., {van Dishoeck}, E.~F., {et~al.} 2016, \aap, 592,
  A83

\bibitem[{{Kamp} \& {Bertoldi}(2000)}]{Kamp2000}
{Kamp}, I., \& {Bertoldi}, F. 2000, \aap, 353, 276

\bibitem[{{Kamp} {et~al.}(2003){Kamp}, {van Zadelhoff}, {van Dishoeck}, \&
  {Stark}}]{Kamp2003}
{Kamp}, I., {van Zadelhoff}, G.-J., {van Dishoeck}, E.~F., \& {Stark}, R. 2003,
  \aap, 397, 1129

\bibitem[{{Kobayashi} \& {L{\"o}hne}(2014)}]{Kobayashi2014}
{Kobayashi}, H., \& {L{\"o}hne}, T. 2014, \mnras, 442, 3266

\bibitem[{{Kobayashi} \& {Tanaka}(2010)}]{Kobayashi2010}
{Kobayashi}, H., \& {Tanaka}, H. 2010, \icarus, 206, 735

\bibitem[{{Kobayashi} {et~al.}(2008){Kobayashi}, {Watanabe}, {Kimura}, \&
  {Yamamoto}}]{Kobayashi2008}
{Kobayashi}, H., {Watanabe}, S., {Kimura}, H., \& {Yamamoto}, T. 2008, \icarus,
  195, 871

\bibitem[{{Kobayashi} {et~al.}(2009){Kobayashi}, {Watanabe}, {Kimura}, \&
  {Yamamoto}}]{Kobayashi2009}
---. 2009, \icarus, 201, 395

\bibitem[{{K{\'o}sp{\'a}l} {et~al.}(2013){K{\'o}sp{\'a}l}, {Mo{\'o}r},
  {Juh{\'a}sz}, {{\'A}brah{\'a}m}, {Apai}, {Csengeri}, {Grady}, {Henning},
  {Hughes}, {Kiss}, {Pascucci}, \& {Schmalzl}}]{Kospal2013}
{K{\'o}sp{\'a}l}, {\'A}., {Mo{\'o}r}, A., {Juh{\'a}sz}, A., {et~al.} 2013,
  \apj, 776, 77

\bibitem[{{Kral} {et~al.}(2019){Kral}, {Marino}, {Wyatt}, {Kama}, \&
  {Matr{\`a}}}]{Kral2019}
{Kral}, Q., {Marino}, S., {Wyatt}, M.~C., {Kama}, M., \& {Matr{\`a}}, L. 2019,
  \mnras, 489, 3670

\bibitem[{{Kral} {et~al.}(2017){Kral}, {Matr{\`a}}, {Wyatt}, \&
  {Kennedy}}]{Kral2017}
{Kral}, Q., {Matr{\`a}}, L., {Wyatt}, M.~C., \& {Kennedy}, G.~M. 2017, \mnras,
  469, 521

\bibitem[{{Kral} {et~al.}(2016){Kral}, {Wyatt}, {Carswell}, {Pringle},
  {Matr{\`a}}, \& {Juh{\'a}sz}}]{Kral2016}
{Kral}, Q., {Wyatt}, M., {Carswell}, R.~F., {et~al.} 2016, \mnras, 461, 845

\bibitem[{{Kraus} {et~al.}(2011){Kraus}, {Senft}, \& {Stewart}}]{Kraus2011}
{Kraus}, R.~G., {Senft}, L.~E., \& {Stewart}, S.~T. 2011, \icarus, 214, 724

\bibitem[{{Kurosawa} {et~al.}(2010){Kurosawa}, {Sugita}, {Kadono}, {Shigemori},
  {Hironaka}, {Otani}, {Sano}, {Shiroshita}, {Ozaki}, {Miyanishi}, {Sakaiya},
  {Sekine}, {Tachibana}, {Nakamura}, {Fukuzaki}, {Ohno}, {Kodama}, \&
  {Matsui}}]{Kurosawa2010}
{Kurosawa}, K., {Sugita}, S., {Kadono}, T., {et~al.} 2010, \grl, 37, L23203

\bibitem[{{Kurucz}(1992)}]{Kurucz1992}
{Kurucz}, R.~L. 1992, \rmxaa, 23, 181

\bibitem[{{Laor} \& {Draine}(1993)}]{LaorDraine1993}
{Laor}, A., \& {Draine}, B.~T. 1993, \apj, 402, 441

\bibitem[{{Lavendy} {et~al.}(1993){Lavendy}, {Robbe}, \&
  {Flament}}]{Lavendy1993}
{Lavendy}, H., {Robbe}, J.~M., \& {Flament}, J.~P. 1993, Chemical Physics
  Letters, 205, 456

\bibitem[{{Le Bourlot} {et~al.}(2012){Le Bourlot}, {Le Petit}, {Pinto},
  {Roueff}, \& {Roy}}]{LeBourlot2012}
{Le Bourlot}, J., {Le Petit}, F., {Pinto}, C., {Roueff}, E., \& {Roy}, F. 2012,
  \aap, 541, A76

\bibitem[{{Le Petit} {et~al.}(2009){Le Petit}, {Barzel}, {Biham}, {Roueff}, \&
  {Le Bourlot}}]{LePetit2009}
{Le Petit}, F., {Barzel}, B., {Biham}, O., {Roueff}, E., \& {Le Bourlot}, J.
  2009, \aap, 505, 1153

\bibitem[{{Le Petit} {et~al.}(2006){Le Petit}, {Nehm{\'e}}, {Le Bourlot}, \&
  {Roueff}}]{LeP2006}
{Le Petit}, F., {Nehm{\'e}}, C., {Le Bourlot}, J., \& {Roueff}, E. 2006, \apjs,
  164, 506

\bibitem[{{Lecavelier des Etangs} {et~al.}(2001){Lecavelier des Etangs},
  {Vidal-Madjar}, {Roberge}, {Feldman}, {Deleuil}, {Andr{\'e}}, {Blair},
  {Bouret}, {D{\'e}sert}, {Ferlet}, {Friedman}, {H{\'e}brard}, {Lemoine}, \&
  {Moos}}]{Lecavelier2001}
{Lecavelier des Etangs}, A., {Vidal-Madjar}, A., {Roberge}, A., {et~al.} 2001,
  \nat, 412, 706

\bibitem[{{Lodders}(2008)}]{Lodders2008}
{Lodders}, K. 2008, \apj, 674, 607

\bibitem[{{Mathis} {et~al.}(1983){Mathis}, {Mezger}, \& {Panagia}}]{Mat1983}
{Mathis}, J.~S., {Mezger}, P.~G., \& {Panagia}, N. 1983, \aap, 128, 212

\bibitem[{{Mathis} {et~al.}(1977){Mathis}, {Rumpl}, \&
  {Nordsieck}}]{Mathis1977}
{Mathis}, J.~S., {Rumpl}, W., \& {Nordsieck}, K.~H. 1977, \apj, 217, 425

\bibitem[{{Matr{\`a}} {et~al.}(2015){Matr{\`a}}, {Pani{\'c}}, {Wyatt}, \&
  {Dent}}]{Matra2015}
{Matr{\`a}}, L., {Pani{\'c}}, O., {Wyatt}, M.~C., \& {Dent}, W.~R.~F. 2015,
  \mnras, 447, 3936

\bibitem[{{Matr{\`a}} {et~al.}(2018){Matr{\`a}}, {Wilner}, {{\"O}berg},
  {Andrews}, {Loomis}, {Wyatt}, \& {Dent}}]{Matra2018}
{Matr{\`a}}, L., {Wilner}, D.~J., {{\"O}berg}, K.~I., {et~al.} 2018, \apj, 853,
  147

\bibitem[{{Matr{\`a}} {et~al.}(2017){Matr{\`a}}, {Dent}, {Wyatt}, {Kral},
  {Wilner}, {Pani{\'c}}, {Hughes}, {de Gregorio-Monsalvo}, {Hales}, {Augereau},
  {Greaves}, \& {Roberge}}]{Matra2017}
{Matr{\`a}}, L., {Dent}, W.~R.~F., {Wyatt}, M.~C., {et~al.} 2017, \mnras, 464,
  1415

\bibitem[{{Mennella}(2006)}]{Mennella2006}
{Mennella}, V. 2006, \apjl, 647, L49

\bibitem[{{Meyer} {et~al.}(1997){Meyer}, {Cardelli}, \& {Sofia}}]{Meyer1997}
{Meyer}, D.~M., {Cardelli}, J.~A., \& {Sofia}, U.~J. 1997, \apjl, 490, L103

\bibitem[{{Meyer} {et~al.}(1998){Meyer}, {Jura}, \& {Cardelli}}]{Meyer1998}
{Meyer}, D.~M., {Jura}, M., \& {Cardelli}, J.~A. 1998, \apj, 493, 222

\bibitem[{{Mizuno} {et~al.}(1978){Mizuno}, {Nakazawa}, \&
  {Hayashi}}]{Mizuno1978}
{Mizuno}, H., {Nakazawa}, K., \& {Hayashi}, C. 1978, Progress of Theoretical
  Physics, 60, 699

\bibitem[{{Montesinos} {et~al.}(2009){Montesinos}, {Eiroa}, {Mora}, \&
  {Mer{\'\i}n}}]{Montesinos2009}
{Montesinos}, B., {Eiroa}, C., {Mora}, A., \& {Mer{\'\i}n}, B. 2009, \aap, 495,
  901

\bibitem[{{Mo{\'o}r} {et~al.}(2019){Mo{\'o}r}, {Kral}, {{\'A}brah{\'a}m},
  {K{\'o}sp{\'a}l}, {Dutrey}, {Di Folco}, {Hughes}, {Juh{\'a}sz}, {Pascucci},
  \& {Pawellek}}]{Moor2019}
{Mo{\'o}r}, A., {Kral}, Q., {{\'A}brah{\'a}m}, P., {et~al.} 2019, \apj, 884,
  108

\bibitem[{{Morton}(1975)}]{Morton1975}
{Morton}, D.~C. 1975, \apj, 197, 85

\bibitem[{{Nakatani} {et~al.}(2018){Nakatani}, {Hosokawa}, {Yoshida}, {Nomura},
  \& {Kuiper}}]{Nakatani2018}
{Nakatani}, R., {Hosokawa}, T., {Yoshida}, N., {Nomura}, H., \& {Kuiper}, R.
  2018, \apj, 865, 75

\bibitem[{{Nakatani} {et~al.}(2021){Nakatani}, {Kobayashi}, {Kuiper}, {Nomura},
  \& {Aikawa}}]{Nakatani2021}
{Nakatani}, R., {Kobayashi}, H., {Kuiper}, R., {Nomura}, H., \& {Aikawa}, Y.
  2021, \apj, 915, 90

\bibitem[{{Navarro-Ruiz} {et~al.}(2015){Navarro-Ruiz},
  {Mart{\'\i}nez-Gonz{\'a}lez}, {Sodupe}, {Ugliengo}, \&
  {Rimola}}]{Navaroo-Ruiz2015}
{Navarro-Ruiz}, J., {Mart{\'\i}nez-Gonz{\'a}lez}, J.~{\'A}., {Sodupe}, M.,
  {Ugliengo}, P., \& {Rimola}, A. 2015, \mnras, 453, 914

\bibitem[{{Navarro-Ruiz} {et~al.}(2014){Navarro-Ruiz}, {Sodupe}, {Ugliengo}, \&
  {Rimola}}]{Navarro-Ruiz2014}
{Navarro-Ruiz}, J., {Sodupe}, M., {Ugliengo}, P., \& {Rimola}, A. 2014,
  Physical Chemistry Chemical Physics (Incorporating Faraday Transactions), 16,
  17447

\bibitem[{{Okeefe} \& {Ahrens}(1982)}]{Okeefe1982}
{Okeefe}, J.~D., \& {Ahrens}, T.~J. 1982, \jgr, 87, 6668

\bibitem[{{Ootsubo} {et~al.}(2012){Ootsubo}, {Kawakita}, {Hamada}, {Kobayashi},
  {Yamaguchi}, {Usui}, {Nakagawa}, {Ueno}, {Ishiguro}, {Sekiguchi}, {Watanabe},
  {Sakon}, {Shimonishi}, \& {Onaka}}]{Ootsubo2012}
{Ootsubo}, T., {Kawakita}, H., {Hamada}, S., {et~al.} 2012, \apj, 752, 15

\bibitem[{{Owen} {et~al.}(2012){Owen}, {Clarke}, \& {Ercolano}}]{Owen2012}
{Owen}, J.~E., {Clarke}, C.~J., \& {Ercolano}, B. 2012, \mnras, 422, 1880

\bibitem[{{Roberge} {et~al.}(2000){Roberge}, {Feldman}, {Lagrange},
  {Vidal-Madjar}, {Ferlet}, {Jolly}, {Lemaire}, \& {Rostas}}]{Roberge2000}
{Roberge}, A., {Feldman}, P.~D., {Lagrange}, A.~M., {et~al.} 2000, \apj, 538,
  904

\bibitem[{{Roberge} {et~al.}(2006){Roberge}, {Feldman}, {Weinberger},
  {Deleuil}, \& {Bouret}}]{Roberge2006}
{Roberge}, A., {Feldman}, P.~D., {Weinberger}, A.~J., {Deleuil}, M., \&
  {Bouret}, J.-C. 2006, \nat, 441, 724

\bibitem[{{Roberge} {et~al.}(2013){Roberge}, {Kamp}, {Montesinos}, {Dent},
  {Meeus}, {Donaldson}, {Olofsson}, {Mo{\'o}r}, {Augereau}, {Howard}, {Eiroa},
  {Thi}, {Ardila}, {Sandell}, \& {Woitke}}]{Roberge2013}
{Roberge}, A., {Kamp}, I., {Montesinos}, B., {et~al.} 2013, \apj, 771, 69

\bibitem[{{Savage} \& {Sembach}(1996)}]{Savage1996}
{Savage}, B.~D., \& {Sembach}, K.~R. 1996, \araa, 34, 279

\bibitem[{{Sha} {et~al.}(2002){Sha}, {Jackson}, \& {Lemoine}}]{Sha2002}
{Sha}, X., {Jackson}, B., \& {Lemoine}, D. 2002, \jcp, 116, 7158

\bibitem[{{Sternberg} {et~al.}(2021){Sternberg}, {Gurman}, \&
  {Bialy}}]{Sternberg2021}
{Sternberg}, A., {Gurman}, A., \& {Bialy}, S. 2021, \apj, 920, 83

\bibitem[{{Sultanov} \& {Balakrishnan}(2005)}]{Sultanov2005}
{Sultanov}, R.~A., \& {Balakrishnan}, N. 2005, \apj, 629, 305

\bibitem[{{Suzuki} \& {Inutsuka}(2009)}]{Suzuki2009}
{Suzuki}, T.~K., \& {Inutsuka}, S.-i. 2009, \apjl, 691, L49

\bibitem[{{Suzuki} \& {Inutsuka}(2014)}]{Suzuki2014}
---. 2014, \apj, 784, 121

\bibitem[{{Tanaka} {et~al.}(1996){Tanaka}, {Inaba}, \& {Nakazawa}}]{Tanaka1996}
{Tanaka}, H., {Inaba}, S., \& {Nakazawa}, K. 1996, \icarus, 123, 450

\bibitem[{{Tanigawa} \& {Ikoma}(2007)}]{Tanigawa2007}
{Tanigawa}, T., \& {Ikoma}, M. 2007, \apj, 667, 557

\bibitem[{{Tieftrunk} {et~al.}(1994){Tieftrunk}, {Pineau des Forets},
  {Schilke}, \& {Walmsley}}]{Tieftrunk1994}
{Tieftrunk}, A., {Pineau des Forets}, G., {Schilke}, P., \& {Walmsley}, C.~M.
  1994, \aap, 289, 579

\bibitem[{{van der Walt} {et~al.}(2011){van der Walt}, {Colbert}, \&
  {Varoquaux}}]{Numpy}
{van der Walt}, S., {Colbert}, S.~C., \& {Varoquaux}, G. 2011, Computing in
  Science and Engineering, 13, 22

\bibitem[{{van Dishoeck} \& {Black}(1988)}]{vanDishoeck1988}
{van Dishoeck}, E.~F., \& {Black}, J.~H. 1988, \apj, 334, 771

\bibitem[{{van Dishoeck} \& {Dalgarno}(1983)}]{vanDishoeck1983}
{van Dishoeck}, E.~F., \& {Dalgarno}, A. 1983, \jcp, 79, 873

\bibitem[{{van Dishoeck} \& {Dalgarno}(1984)}]{vanDishoeck1984}
---. 1984, \apj, 277, 576

\bibitem[{{van Dishoeck} {et~al.}(2006){van Dishoeck}, {Jonkheid}, \& {van
  Hemert}}]{vanDishoeck2006}
{van Dishoeck}, E.~F., {Jonkheid}, B., \& {van Hemert}, M.~C. 2006, Faraday
  Discussions, 133, 231

\bibitem[{{Veselinova} {et~al.}(2021){Veselinova}, {Ag{\'u}ndez}, {Goicoechea},
  {Men{\'e}ndez}, {Zanchet}, {Verdasco}, {Jambrina}, \&
  {Aoiz}}]{Veselinova2021}
{Veselinova}, A., {Ag{\'u}ndez}, M., {Goicoechea}, J.~R., {et~al.} 2021, \aap,
  648, A76

\bibitem[{{Visser} {et~al.}(2009){Visser}, {van Dishoeck}, \&
  {Black}}]{Visser2009}
{Visser}, R., {van Dishoeck}, E.~F., \& {Black}, J.~H. 2009, \aap, 503, 323

\bibitem[{{Weaver} {et~al.}(2011){Weaver}, {Feldman}, {A'Hearn}, {Dello Russo},
  \& {Stern}}]{Weaver2011}
{Weaver}, H.~A., {Feldman}, P.~D., {A'Hearn}, M.~F., {Dello Russo}, N., \&
  {Stern}, S.~A. 2011, \apjl, 734, L5

\bibitem[{{Wishart}(1979)}]{Wishart1979}
{Wishart}, A.~W. 1979, \mnras, 187, 59P

\bibitem[{{Zanchet} {et~al.}(2013{\natexlab{a}}){Zanchet}, {Ag{\'u}ndez},
  {Herrero}, {Aguado}, \& {Roncero}}]{Zanchet2013_Sp}
{Zanchet}, A., {Ag{\'u}ndez}, M., {Herrero}, V.~J., {Aguado}, A., \& {Roncero},
  O. 2013{\natexlab{a}}, \aj, 146, 125

\bibitem[{{Zanchet} {et~al.}(2013{\natexlab{b}}){Zanchet}, {Godard}, {Bulut},
  {Roncero}, {Halvick}, \& {Cernicharo}}]{Zanchet2013}
{Zanchet}, A., {Godard}, B., {Bulut}, N., {et~al.} 2013{\natexlab{b}}, \apj,
  766, 80

\bibitem[{{Zecho} {et~al.}(2002){Zecho}, {Guttler}, {Sha}, {Jackson}, \&
  {Kuppers}}]{Zecho2002}
{Zecho}, T., {Guttler}, A., {Sha}, X., {Jackson}, B., \& {Kuppers}, J. 2002,
  \jcp, 117, 8486

\end{thebibliography}

%% This command is needed to show the entire author+affiliation list when
%% the collaboration and author truncation commands are used.  It has to
%% go at the end of the manuscript.
%\allauthors

%% Include this line if you are using the \added, \replaced, \deleted
%% commands to see a summary list of all changes at the end of the article.
%\listofchanges

\end{document}

% End of file `sample631.tex'.
