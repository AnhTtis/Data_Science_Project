Despite thousands of researchers, engineers, and artists actively working on improving text-to-image generation models, systems often fail to produce images that accurately align with the text inputs. 
%Yet we lack a reliable automatic metric that evaluates the faithfulness between the input text prompt and generated image. 
%\nascomment{maybe cut this sentence, abstract is very long} 
%For example, evaluations based on CLIP are unreliable because CLIP is insensitive to counting, compositionality, etc.
We introduce \NAME (\textbf{T}ext-to-\textbf{I}mage \textbf{F}aithfulness evaluation with question \textbf{A}nswering), an automatic evaluation metric that measures the faithfulness of a generated image to its text input via visual question answering (VQA).
Specifically, given a text input, we automatically generate several question-answer pairs using a language model.
We calculate image faithfulness by checking whether existing VQA models can answer these questions using the generated image.
\NAME is a \emph{reference-free} metric that allows for fine-grained and interpretable evaluations of generated images.
%provides an easy way to detect image deficiencies automatically. 
\NAME also has better correlations with human judgments than existing metrics.
Based on this approach, we introduce \NAME v1.0, a benchmark consisting of 4K diverse text inputs and 25K questions across 12 categories (object,  counting, etc.).
We present a comprehensive evaluation of existing text-to-image models using \NAME v1.0 and highlight the limitations and challenges of current models.
For instance, we find that current text-to-image models, despite doing well on color and material, 
still struggle in counting, spatial relations, and composing multiple objects. 
We hope our benchmark will help carefully measure the research progress in text-to-image synthesis and provide valuable insights for further research.\footnote{Correspondance to <Yushi Hu: \url{yushihu@uw.edu}>.
All data and a pip-installable evaluation package are available on the project page.}