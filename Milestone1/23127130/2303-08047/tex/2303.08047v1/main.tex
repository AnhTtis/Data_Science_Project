% ****** Start of file apssamp.tex ******
%
%   This file is part of the APS files in the REVTeX 4.2 distribution.
%   Version 4.2a of REVTeX, December 2014
%
%   Copyright (c) 2014 The American Physical Society.
%
%   See the REVTeX 4 README file for restrictions and more information.
%

% \pdfoutput=1
% \documentclass[twocolumn,
%  reprint,
% superscriptaddress,
% nofootinbib,
% aps,
% pra]{revtex4-2}

\documentclass[aps,pra,reprint,superscriptaddress,amsmath,amsfonts,amssymb,longbibliography]{revtex4-1}

\usepackage[utf8]{inputenc}
% \usepackage[document]{ragged2e}
%% FIGURES
%\usepackage{subcaption,ragged2e}
\usepackage{graphicx} 
\usepackage{float} 
% \usepackage{caption}

%% TEXT COLOR
\usepackage{xcolor}

%% MATH
%\usepackage{amsmath}
\usepackage{amssymb}
%\usepackage{mathabx}
\usepackage{dsfont}

%% PHYSICS
\usepackage{physics}

%% PYTHON CODE
\usepackage{listings} 

%% TEXT COLOR
\usepackage{color}

%% hypertext capabilities
\usepackage{hyperref}

%% SPACING
\usepackage[nodisplayskipstretch]{setspace}
% \setstretch{1}

\usepackage{hyperref}% add hypertext capabilities

\newcommand{\cy}[1]{{\color{cyan} #1} }
\newcommand{\red}[1]{{\color{red} #1} }
\newcommand{\blue}[1]{{\color{blue} #1} }

\newcommand{\gry}[1]{{\color{gray} #1} }

\newcommand{\aug}[1]{\red{[Augusto: #1]}}
\newcommand{\nac}[1]{{\color{cyan} [Nacho: #1]} }
\begin{document}

\title{Classical approach to equilibrium of out-of-time ordered correlators in mixed systems}

\author{Tomás Notenson}
\affiliation{Departamento de F\'isica ``J. J. Giambiagi'' and IFIBA, FCEyN,
Universidad de Buenos Aires, 1428 Buenos Aires, Argentina}
\author{Ignacio Garc\'ia-Mata}
\affiliation{Instituto de Investigaciones Físicas de Mar del Plata (IFIMAR), Facultad de Ciencias Exactas y Naturales,
Universidad Nacional de Mar del Plata, CONICET, 7600 Mar del Plata, Argentina}
\author{Augusto J. Roncaglia}
\affiliation{Departamento de F\'isica ``J. J. Giambiagi'' and IFIBA, FCEyN,
Universidad de Buenos Aires, 1428 Buenos Aires, Argentina}
\author{Diego A. Wisniacki}
\affiliation{Departamento de F\'isica ``J. J. Giambiagi'' and IFIBA, FCEyN,
Universidad de Buenos Aires, 1428 Buenos Aires, Argentina}

\date{\today}

\begin{abstract}
The out-of-time ordered correlator (OTOC) is a measure of scrambling of quantum information. Scrambling is intuitively considered to be a significant feature of chaotic systems and thus the OTOC is widely used as a measure of chaos. For short times exponential growth is related to the classical Lyapunov exponent, sometimes known as butterfly effect. At long times the OTOC attains an average equilibrium value with possible oscillations. For fully chaotic systems the approach to the asymptotic regime is exponential with a rate given by the classical Ruelle-Pollicott resonances. In this work, we extend this notion by showing that classical generalized resonances govern the relaxation to equilibrium of the OTOC in the ubiquitous case of a system with mixed dynamics, in particular, the  standard map.
% 
\end{abstract}

\maketitle
\section{\label{sec:introduction} Introduction}
The out-of-time ordered correlator (OTOC) has been proposed as a measure of spreading of information and scrambling \cite{larkin1969quasiclassical,maldacena2016bound}. Recently, it has been noticed that for chaotic systems the short time behavior is characterized by an exponential growth which can be directly related (in some cases) to the classical Lyapunov exponent. A straightforward generalization would lead to characterizing systems with exponential growth as chaotic and naming the growth rate as quantum Lyapunov exponent. However this is not a universal behavior. There are systems that are not chaotic where the OTOC growths exponentially  \cite{ali2020chaos,hashimoto2020exponential,morita2021extracting,HummelPRL2019,XuPRL2020,RozenbaumPRL2020,kidd2021saddle} and the converse is also true, there are systems universally accepted as quantum chaotic where the growth is much slower \cite{kukuljan2017,Motrunich2018,PhysRevB.99.054205,PhysRevE.100.042201,Shukla2022,moessner2017}. 

Interestingly, a lot of information about the dynamics, and some underlying classical features, can be obtained  at later times. After the initial growth of the OTOC, the system approaches an equilibrium state with oscillations. The study of both the equilibrium value \cite{hashimoto2017,JGMW2018,rammensee2018,huang2019finite,sinha2021fingerprint,Markovic2022} as well as the oscillations \cite{PhysRevE.100.042201,fortes2020signatures}  has allowed to distinguish chaotic from regular behavior.
However, evaluation of equilibrium based measures requires evolving up to long times and computing  averages over large time windows. This can be experimentally unattainable. 
Nevertheless, at a much shorter time scales there is another classical quantity that can be used to characterize the chaotic nature of a system. In a simplified way, fully chaotic systems present two main features: exponential separation of initial conditions (Lyapunov regime), and mixing. The mixing property implies exponential decay of correlation functions which is characterized by the largest Ruelle-Pollicott resonance. The Ruelle-Pollicott resonances are the point spectrum of the Perron-Frobenius operator expressed in a suitably defined functional space~\cite{Blank2002}. For fully chaotic quantum maps on the torus, it has been shown~\cite{PhysRevLett.121.210601} that the OTOC approaches equilibrium exponentially, and the rate is given by the classical  Ruelle-Pollicott resonance with largest modulus.

The case of systems with mixed dynamics (i.e. neither fully integrable nor fully chaotic), although much more generic, is less studied. In such systems the  Ruelle-Pollicott resonances are not strictly well defined. However, through approximation schemes, the spectrum of the Perron-Frobenius can be obtained. In this case, it is observed that there are eigenvalues that tend to the unit circle, associated to regular behavior, and there are other eigenvalues that persist inside the unit circle, and are considered generalized resonances. The eigenfunctions of these resonances are located in chaotic regions in phase space~\cite{Manderfeld2001,Weber2001,KhodasFishmanAgam,fishman2002relaxation,nonnenmacher2003spectral}.  

In this work, we use numerical methods to obtain the generalized resonances and establish a correspondence with the rate of approach to equilibrium of the OTOC, given by the decay of a four point correlator. The numerical calculations are done on the standard map \cite{Chirikov1979} which can be parametrically tuned from integrable to chaotic. The values obtained for the resonances reproduce very well the structure in the decay rate as a function of the chaos parameter. Interestingly, we also find that in this case, the quantum evolution is able to reveal classical related effects such as the appearance and disappearance of stable islands in the classical phase space. This is clearly reflected in the rich structure of the decay rates as a function of the chaos parameter.

This paper is organized as follows. In Sec.~\ref{sec:O1 in maps} we define the relevant quantities and observables to be used. In Sec.~\ref{sec:standard} we compute numerical the decay rates as a function of the chaos parameter for the standard map.
Then, in Sec.~\ref{sec:sepctrumPF}, we describe the numerical method to compute the spectrum of the Perron-Frobenius operator. Once established the method to compute the generalized resonances, we show their relationship with the decay rates in Sec.~\ref{sec:relationship}, leaving Sec.~\ref{sec:conclusions} for the concluding remarks.

\section{\label{sec:O1 in maps} Out-of-time-order correlators in quantum maps}
%%
Generically, the OTOCs are defined as
\begin{equation}
C(t)=\langle |[{W}(t),{V}]|^2\rangle
\end{equation}
where $V$ and $W$ are two operators, $W(t)$ being the Heisenberg operator at time $t$, and the average is taken with respect to a thermal state (here we consider the infinite temperature case). This quantity was introduced by Larkin and Ovchinnikov to study semiclassical approaches to superconductivity \cite{larkin1969quasiclassical}, and received considerable attention later due to its connection with the scrambling of quantum information 
chaotic systems and the dynamics of black holes \cite{shenker2014black,shenker2014multiple,shenker2015stringy}. 
In fact, when the operators have a local support on different regions of a many body system, the spreading of information can be quantified by the OTOC.
They are called OTOCs, mainly due to the fact that their expansion as $C(t) = -2(O_1 - O_2)$ contains the term
\begin{equation}
 O_1(t)=\langle {W}(t){V}{W}(t){V} \rangle
\end{equation}
which is  an out-of-time-order product, and another correlator $O_2 = \langle {W}(t)^2{V}^2 \rangle$. 

The early time dynamics of the OTOCs in quantum systems with a classical chaotic counterpart is rather universal, and can be characterized by an exponential growth with time and a saturation, as it is depicted in the schematic Fig.~\ref{fig:analysis}(a). This growth shows a clear relationship with the classical Lyapunov exponent, and it lasts up to the so-called scrambling time where this quantity saturates \cite{PhysRevLett.118.086801,PhysRevLett.121.210601,chavez2019quantum,sinha2021fingerprint,hashimoto2020cho,PhysRevE.99.012201}.  Thus, it is remarkable that the short time dynamics of the OTOCs contains a signature of the classical behavior. For one-body systems the saturation time is proportional to the Ehrenfest time \cite{PhysRevLett.118.086801,PhysRevLett.121.210601} ($\tau_{\rm E}\sim \ln N/\lambda$, where $N$ is the Hilbert space size). 

After the Lyapunov regime, the approach to equilibrium of $C(t)$ is governed by the correlator $O_1(t)$. For fully chaotic systems this decay is exponential and it was hinted \cite{polchinski2015}, and demonstrated for chaotic quantum maps~\cite{PhysRevLett.121.210601}, to be related to classical quantities  known as Ruelle-Pollicott resonances \cite{pollicott1985rate,ruelle1986,ruelle1987resonances}. The Ruelle-Pollicott resonances can be loosely defined as the point spectrum of the evolution operator of classical distribution functions in the case of fully chaotic systems.
A classical dynamical system evolves through the one step evolution operator defined by\begin{equation}\label{eq:perron_frobenius}
    \rho(x,t) = \mathcal{L}^t \rho(x,0) = \int_{\mathcal{M}}\!dx_0 \,\delta(x-f^t(x_0)) \rho(x_0,0)
\end{equation}
where $\rho(x,t)$ is the density of representative points at $x$ in the phase space at time $t$ in the zero coarse grained limit, $\mathcal{L}$ is the Perron-Frobenius (PF) evolution operator, $f(\cdot)$ is the map and the integral runs throughout phase space $\mathcal{M}$ ~\cite{cvitanovic2005chaos}. When defined in $\mathbb{L}^2$ the PF is unitary. However this is rather restrictive. A way to unveil the spectrum of PF operator is to allow for distributions. Then one can define a functional space adapted to the dynamics, smooth along the unstable direction and (possibly) singular along the stable dimension. For uniformly hyperbolic, chaotic systems, the spectrum of the thus expressed PF operator consists a point spectrum and an  essential spectrum~\cite{reed1980methods}. The essential spectrum is  bounded by a small radius $r<1$. The eigenvalues composing the point spectrum are the Ruelle-Pollicott resonances $\{\lambda_i\}$, where $r\leq |\lambda_i|\leq 1$~\cite{Blank2002}. The case $\lambda_0 = 1$ corresponds to the invariant density. The correlation function decay is then determined by resonances such that $1>|\lambda_1|>|\lambda_2|>\ldots$ and typically, if there are no degeneracies, it is dominated by $\lambda_1$. For quantum maps on the tours in the fully chaotic regime it has been numerically proven~\cite{PhysRevLett.121.210601} that the decay of $O_1$ is given by 
\begin{equation}
    O_1(t)\sim |\lambda_1|^{2t}.
\end{equation}

The spectrum in the case of mixed phase space dynamics is neither as simple to describe nor to compute, and studies about it do not abound. In \cite{khodas2000relaxation,KhodasFishmanAgam,Manderfeld2001,Weber2001} using coarse graining and truncation schemes, a description of the spectrum is made where eigenvalues that are close to the unit circle are identified with regular island regions and some persistent [``frozen''] eigenvalues, with modulus strictly smaller than one can be identified as corresponding to densities located in the chaotic regions of phase space. The latter are  ``generalized'' Ruelle resonances.

We will show below that for a system with mixed dynamics,
the approach to equilibrium displays a clear exponential regime determined by these so-called generalized resonance. This is the regime that we study in the paper. We have also observed a second power law regime that is far less studied, that lie outside the scope of this work and we leave it for future endeavors.

\begin{figure}
\includegraphics[width=.43\textwidth]{fig1} % {Figures/fig1tomy.pdf} 
\caption{\label{fig:analysis} 
Sketch of the typical time dependence of $O_1 = \langle {W}(t){V}{W}(t){V} \rangle$ for mixed quantum maps exhibiting different
behavior in the short-, intermediate- and long-time regimes. At short times, namely for times below Ehrenfest time $t<t_E$, it has an approximate constant value (blue shading). We name this short-time behavior Lyapunov regime because of the exponential growth in $C(t) = \langle |[{W}(t),{V}] |^2\rangle = -2(O_1 - O_2)$ with $O_2 = \langle {W}(t)^2{V}^2 \rangle$ (see upper panel and \cite{PhysRevLett.121.210601} for more detail). Next, the correlator has an exponential decay related to Ruelle resonances of the classical map (red shading). The succeeding regime is the power law regime which we associate with regular vestiges of the map (green shading). For long times, the $O_1$ oscillates around his saturation value (cyan shading).}
\end{figure}

Let us now define the relevant quantities that we will consider in this work. The numerical calculations will be done on quantum maps on the 2-torus. More specifically the Chirikov standard map~\cite{Chirikov1979,DimaScholar}. After quantization of the two dimensional phase space, the periodic properties impose discreteness on position and momentum, and the effective Planck constant is related to the dimension of Hilbert space $D$ as $\hbar_{\rm eff}=1/(2\pi D)$. The operators we use to evaluate the OTOCs in quantum maps are the so-called position and momentum operators
\begin{equation}
    {X} \equiv \frac{{U}_S - {U}^\dagger_S}{2i}\text{, } \, \, {P} \equiv \frac{{V}_S - 
    {V}^\dagger_S}{2i}
\end{equation}
Which are Hermitian and defined in terms of the unitary Schwinger shift operators $U_S$ and  $V_S$ \cite{schwinger1960unitary}: 
\begin{equation}
    {V}_S =\sum\limits_q |q + 1 \rangle \langle q| \text{, } \, \, {U}_S = \sum\limits_q e^{2 \pi i q/D} |q\rangle \langle q |
\end{equation}
where $|q\rangle$, $|p\rangle$ are, respectively, position and momentum states for a $D$ dimensional system with periodic boundary conditions (with $q, p = 0,..., D - 1$). They are related by the discrete Fourier transform satisfying  $\langle p | q \rangle = e^{-2\pi i qp/D}$  and ${V}_S|D-1\rangle = |0\rangle$. In the semiclassical limit ${X}$ and ${P}$ approximate the position and momentum operators, respectively.
Thus,  for the OTOC we take the average over the maximally mixed state $\rho = \mathds{1}/D$:
\begin{equation}
    O_1(t) = \langle {P}(t){X}{P}(t){X} \rangle = \text{Tr}({P}(t){X}{P}(t){X})/D
\end{equation}
where for a quantum stroboscopic map that evolves with the unitary operator $U$, ${X}(t) = ({U}^\dagger)^t {X} {U}^t$.
%%

\section{Decay of $O_1$ for the standard map}
\label{sec:standard}
\par Mixed systems are ubiquitous within Hamiltonian dynamics, however, they were much less studied than completely chaotic and integrable systems. This is because the coexistence of different dynamics produces very complex structures. In this work, we want to understand the effect that chaotic regions and islands of stability have on the decay of the $O_1$. For this reason, we have focused on the standard map, a paradigmatic model within both classical and quantum studies. This is generated by the time-dependent Hamiltonian 
\begin{equation} 
H(q,p,t)=p^{2}/2+K/(2 \pi)^{2} \cos(2 \pi q)\sum \delta(t-n)
\end{equation}
where $K$ is the strength of delta kicks. Due to the periodicity of $\cos(\cdot)$ the dynamics can be considered on a cylinder (by taking $p \text{ mod } 1$) or on a torus (by taking both $p,q \text{ mod } 1$) \cite{Chirikov:2008}. We are interested in the second description. So, the corresponding classical map is,
\begin{equation} 
\label{eq:clstdmap}
    \left. \begin{array}{ll}
             p_{n+1} &= p_n + \frac{K}{2\pi}\sin(2\pi q_n) \\
             q_{n+1} &= q_n + p_{n+1}
             \end{array}
   \right\}\text{mod }1
\end{equation}

\begin{figure}[h]
\includegraphics[width=0.5\textwidth]{fig2}%{Figures/fig2.pdf}
\caption{\label{fig:edf} Sample phase spaces for the standard map Eq.~(\ref{eq:clstdmap}) and $K=0.4,\, 1,\, 1.8,\,6.6,\,17,\, 18.86$ (from left to right, top to bottom). The labels are the same for every panel. The colors of the last two bottom-left panels are meant to correspond to the $K$ values used in Fig.~\ref{fig:numeric_decay}. Notice that for very large $K$, bottom right panel, small regular islands can be seen.}
\end{figure}

The classical version of this model manifests a regular to chaotic transition (as a function of driving strength $K$), enabling us to benchmark the behavior of the OTOC against the presence or absence of classical chaos and islands of stability.
For small values of $K$ the dynamics is regular. Below a certain critical value $K_c$, the motion in momentum is restricted by the KAM curves. These are invariant curves with an irrational winding number that represent quasi-periodic motion, and they are the most robust orbits under nonlinear perturbations \cite{liebermann1983irregular}. At $K_c = 0.971635406 ...$, the last KAM curve, with most irrational winding number, breaks down \cite{greene1979method}. Above $K_c$, there is unbounded diffusion in $p$. This critical point is important for the map on the cylinder, non-periodic in $p$, not considered in this work. For very large values of $K$ the motion is essentially chaotic with no visible islands. However, as $K$ is increased, at approximately periodic $K$ values, small islands do appear, and then disappear. An illustration of  this behavior is presented is presented in Fig.~\ref{fig:edf}, where for large $K=17$ we see that there are no islands and then at $K=18.86$ two visible islands reappear. 

\par The quantum standard map is defined by the Floquet evolution operator,
\begin{equation}
    {U}(K) = e^{- \frac{i}{\hbar_{\rm eff}} {P}^2/2}e^{- \frac{i}{\hbar_{\rm eff}} \frac{K}{4\pi^2} \cos(2\pi {X})}
\end{equation}
and its dynamics is given by a sequence of free propagations interleaved with periodic kicks. The advantage of the previous formulation is that the numerical implementation of the time evolution and the corresponding diagonalization becomes very efficient using fast Fourier transformations \cite{ketzmerick1999efficient}. For the map we consider, the kick strength is the chaos parameter.

Let us now consider the behavior of the correlator $O_1$ for this map and different values of the kick strength $K$.  In Fig. \ref{fig:numeric_decay} we show a numerical evolution of the $O_1$ up to time $t=36$ (the time unit is a period) for  $K=6.6$ (circles) and for $K=17$ (squares). A Hilbert dimension of $D=5000$ was used in both plots. The four different regimes described in Fig. \ref{fig:analysis} can be clearly seen for $K=6.6$. The dashed black lines in the main panel correspond to an exponential fit $|O_1(t)|\sim e^{-\gamma t}$, while in the inset we show the cases  $K=6.6$ in log-log scale to expose the power-law regime, and the (shifted) black dashed line marks the corresponding fit. After these regimes, $|O_1|$ slowly arrives at the saturation value. In contrast, for $K=17$ (bottom panel Fig. \ref{fig:numeric_decay})   only the exponential regime between $t=2$ and $t=7$ can be observed before saturation.  We also can appreciate differences in the classical phase space structure in these cases. For instance, notice that for $K=6.6$, there are regular islands in this classical phase space, that are appreciable with the naked eye (see the forth panel of Fig.~\ref{fig:numeric_decay}). In contrast, for the $K=17$ case, in the fifth panel of Fig.~\ref{fig:numeric_decay}, no regular structures are visible. In Sec.~\ref{sec:relationship} we will analyse this relation in more detail. 
\begin{figure}[ht]
    \includegraphics[width=0.95\linewidth]{fig3}  %%{Figures/fig3_nacho.pdf}
\caption{\label{fig:numeric_decay} 
$|O_1|$ versus time for $K=6.6$ (circles) and $K=17$ (squares). These two $K$ values correspond to the last two panels of Fig.~\ref{fig:edf}. For $K=6.6$, we observe an exponential decay for  $5 \lesssim t\lesssim  8$ and after that we find  a power-law regime for $9 \lesssim t\lesssim  15$. On the other hand, for $K=17$ only the exponential regime is observed the interval for  $4 \lesssim t\lesssim  7$  before the correlator function saturates for long times. The dashed black lines  in the main panel where obtained by fitting $|O_1(t)|\sim e^{-\gamma t}$. In the  inset we show the case $K=6.6$ in log-log scale and the black dashed line corresponds to a power-law fit $|O_1(t)|\sim t^\alpha$.}
\end{figure}
In Fig.~\ref{fig:gammadeK} we show the decay rate $\gamma$ as a function of $K$ obtained by suitably fitting an exponential in the decay of $|O_1(t)|$. It can be clearly observed that for small $K$ the decay rate is negligible and a transition to strong exponential decay is observed for large $K$, consistent with chaotic behavior. The oscillating structure, marked by (almost) periodic dips, corresponds to the appearance/disappearance of small islands in the classical phase space.
%%%%%%%%%%%%%%%%%%%%%%%%%%
\begin{figure}%[h!]
\includegraphics[width=.45\textwidth]{fig4} %%{Figures/fig_gammadeK.pdf}
\caption{\label{fig:gammadeK} 
Decay rate $\gamma$ as a function of kicking strength $K$ obtained from fitting an exponential  $|O_1(t)|\sim e^{-\gamma t}$ in suitably chosen time intervals. The vertical bars correspond to fitting errors.}
\end{figure}
%%%%%%%%%%%%%%%%%%%%%%%%%%%%%%

\section{\label{sec:sepctrumPF} Numerical computation of the classical spectrum}
As we mentioned in Sec.~\ref{sec:O1 in maps} the Ruelle-Pollicott resonances are the point spectrum of the PF propagator. In the case of hyperbolic systems they establish the exponential rate of relaxation to the classical equilibrium density \cite{pollicott1985rate,ruelle1986,ruelle1987resonances}. 
For mixed systems, the appearance of regular islands in phase space generate sticking to these regions affecting the relaxation rate that in this case depend on the size of the regular regions. In these systems, the Ruelle-Pollicott resonances are not strictly well defined but generalized resonances where obtained by different numerical methods identifying, stable eigenvalues inside the unit circle with functions living inside the chaotic regions \cite{Manderfeld2001,Weber2001,khodas2000relaxation,KhodasFishmanAgam}.

The task of computing the spectrum of the PF operator is not simple in any case (chaotic or mixed), and numerical approximation methods need to be considered. In this section we present the three methods we considered.The first method \cite{blum2000leading, khodas2000relaxation}, that we called \textit{momentum-position method}, consists of expressing the Perron-Frobenius operator directly in a  phase space basis $|q,p\rangle$ 
\begin{equation*}
    ( q',p' | \mathcal{L} |q,p )
    = \delta (p'-[p+\frac{K}{2\pi}\sin(2\pi q)] ) \, \delta\left(q' - [q+p']\right)
\end{equation*}
a periodic $\delta$ function on the torus. In this way, the operator is unitary, thus their spectrum is located in the unit circle. In order to study relaxation in these systems, one can coarse grained the dynamics by adding noise leading to a non-unitary operator. This can be done by replacing the delta function in the above expression by:
\begin{equation}
    \delta(x) \to \sum_n \frac{1}{\pi s} \exp(-(x-n)^2/s),    
\end{equation}
where $s$ is a real parameter.
Then, the approximate spectrum is obtained by diagonalizing the operator in a reduced basis of $D$ vectors. Ideally, the resonances are obtained by diagonalizing the operator for a finite value of $s$ and then taking the limit of noise to zero.

The second method \cite{khodas2000relaxation,KhodasFishmanAgam,fishman2002relaxation} consists in expressing the PF operator in the Fourier-transformed phase space,  $( q,p | k,m ) = \exp(imq)\exp(ikp)/(2 \pi)$. In this case, noise is required for the convergence of the eigenvalues \cite{PhysRevLett.121.210601,garcia2003classical,khodas2000relaxation,fishman2002relaxation,nonnenmacher2003spectral}, especially for values of $K$ where the phase space is mixed.
Thus, after adding noise to $q$ with variance $\sigma^2$, the evolution operator can be written as
\begin{equation}\label{eq:fishman}
\begin{split}
    (k,m | \mathcal{L^{(\sigma)}} |& k',m' ) \\
    &= J_{m-m'}(k'\,K)\exp(-\frac{\sigma^2}{2}m^2)\delta_{k-k',m}
\end{split}
\end{equation}
where $J_\nu (x)$ is the Bessel function of the first kind, $\sigma$ is the noise parameter. 
 Finally, we diagonalize a $D\times D$ matrix which represents a coarse-grained approximation of the PF operator. We will refer to this procedure as the \textit{Fourier method}. The advantage of this approach is that, the thicker the coarse-grained, the greater the importance of low modes associated with big structures of phase space. Namely, a low dimensional matrix approximation can recover important features of the phase space discounting the influence of small structures which usually are regular. The latter carries an advantage in terms of numerical efficiency. 

\par In Fig. \ref{fig:unit circle} we exhibit an example of the spectrum for the truncated Perron-Frobenius operator for $K=6.6$ (left panel) and $K=17$ (right panel). The greater eigenvalue has modulus one, and the next one is the leading eigenvalue $\lambda_1$ that we will associate to the decay of $|O_1|$. In this case, we observe that the modulus of the leading eigenvalue for $K=6.6$ is greater than the one for $K=17$. As we will show bellow, the smallest eigenvalues are associated to higher decay rates.

\begin{figure}[ht]
\includegraphics[width=.49\textwidth]{fig5} %%{Figures/unit_circle_alt.pdf}
\caption{\label{fig:unit circle} Eigenvalues of the truncated Perron-Frobenius operator computed with the Fourier method. The left panel corresponds to $K=6.6$ and the right panel to $K=17$. In these cases we can see that there is a gap between the two leading eigenvalues such that $|\lambda|<1$ which lie in the real axis (big stars).  We will see that in general the larger the chaos parameter $K$, the smaller the leading eigenvalues. The matrix dimension is $D=30$ in both plots.}
\end{figure}

\par Finally, we have also  considered the Ulam method~\cite{ulam1960collection} to calculate the leading eigenvalues of the PF operator. This procedure consists in dividing the phase space into $N_d = M \times M$ cells and propagate $N_c$ trajectories on one map iteration from each cell $j$. Then the matrix $S_{ij}$ is defined by the relation $S_{ij} = N_{ij}/N_c$ where $N_{ij}$ is the number of trajectories arriving from a cell $j$ to a cell $i$. By construction $\sum_i S_{ij} = 1$ and therefore the matrix $S_{ij} \in \mathbb{R}^{N_d \times N_d}$ pertain to the class of PF operators (see \cite{meyer2000perron,pillai2005perron}) and can be considered as a discrete approximation of the PF operator of the continuous classical map \cite{frahm2010ulam}. However, we have observed that when there are regular regions present, this method is less accurate, so we favor the other two methods. 


\section{\label{sec:relationship} Classical decay of $O_1$}
%%
In Sec.~\ref{sec:O1 in maps} we have shown examples of  
 the decay  $|O_1(t)|$,  for different values of the parameter $K$, as well as the dependence of the decay rate with the chaos parameter.   
Here we will show evidence that the  decay rate of $|O_1|$ shown in Fig.~\ref{fig:gammadeK} is closely related to $|\lambda_1|$  by  
\begin{equation}
    \gamma \approx -2\ln(|\lambda_1|). 
    \label{eq:gammalambda}
\end{equation}
We can stress that the left hand side of the equation is obtained from the quantum decay rate of $|O_1|$, while the term at the right hand side is a classical quantity. To this end we  used the methods described in the preceding section to  calculate the leading eigenvalue $|\lambda_1|$ (with modulus smaller than 1) of the Perron-Frobenius operator.
 For the momentum-position method we choose a noise value $s=0.001$ that is based on previous work \cite{blum2000leading}, while for the Fourier method the noise is $\sigma = 0.2$. 
 For the Ulam method, we have observed that the results, for the mixed dynamics case, do not fit  the ones obtained from the decay rates of $|O_1|$ (data not shown).

In Fig.~\ref{fig:resonances_and_decays} we show 
 the leading eigenvalues of the Perron-Frobenius operator $|\lambda_1|$ that, according to Eq.~\eqref{eq:gammalambda}, is equivalent to  the quantity $e^{-\gamma/2}$ obtained from the decay of $|O_1|$.   
In the figure we can see a good agreement between the resonances obtained from both methods, and the data coming from the fitted decay rate. From this  it is evident that $|\lambda_1|$ in general decreases with the chaos parameter $K$. However, we also observe that for large values of $K$, even the system seems to be in a fully chaotic region, there are some bumps in the curve. {Below we show evidence that the four} ``bumps'', in $K\in[8,11]$, $K\in[12,14]$, $K\in[15,17]$ and in $K\in[18,20]$, are associated to the recovery of regularity in the phase space of the classical map. 

A systematic way to show this is to estimate the area in phase space $A_{\rm reg}$ occupied by the regular islands as a function of the chaos parameter $K$.  To do so we choose randomly a large number $N_{\rm tot}$ of initial conditions. We evolve them with the map, but we introduce a hole in the chaotic region. By evolving for a large enough time, the only remaining points  $N_{\rm r}$ will be those that were initially inside the regular regions, which are disconnected from the chaotic ones. After a large number of time steps, and approximation of the regular area is 
\begin{equation}
 A_{\rm reg}\approx N_{\rm r}/N_{\rm tot}.   
\end{equation} 
If the whole phase space is chaotic, then $A_{\rm reg}\to 0$.
In Fig.~\ref{fig:hole} we show how the appearance of islands is manifested in the bumps of the black lines, and these are directly related with the bumps that appear in both the calculation of the resonance $|\lambda_1|$ and in the decay rate of $|O_1|$.
%%%%%%%%%%%%%%%%%%%%%%%%%%%%%%%%%%%%%%%%%%%%
\begin{figure}%[h!]
\includegraphics[width=.45\textwidth]{fig6} %%{Figures/fig6.pdf}
\caption{\label{fig:resonances_and_decays} 
Absolute value $|\lambda_1|$ of the leading eigenvalue of the Perron-Frobenius operator as a function of $K$.
The black line corresponds to the momentum-position method
and the red line to the Fourier method. These are compared to the values obtained from the decay rate of $|O_1|$ as $e^{-\gamma/2}$ (blue circles), the error bars are least squares errors due to the fit. 
For the Fourier method the dimension of the PF approximation was $D=30$ and the noise parameter $\sigma=0.2$, while  for momentum-position method we used $D=90$ and $s=0.001$.}
\end{figure}
%%%%%%%%%%%%%%%%%%%%%%%%%%%%%%%%%%%%%%%%%%%%



 Thus, our results suggest that the intermediate times of the quantum out-of-time-ordered correlator $O_1$ are governed  by classical magnitudes and structures. In this case, the eigenvalues of the PF operator that play a  role similar to the Ruelle-Pollicott resonances in fully chaotic systems. At the same time, these eigenvalues inherit some characteristics of the phase space structure, as we showed by analyzing the revivals of regular islands in the chaotic region. 

\begin{figure}[ht]
\includegraphics[width=.45\textwidth]{fig7} %{Figures/fig7.pdf}
\caption{\label{fig:hole} 
The black thick line (left $y$ axis) represents an estimation of the area of phase space occupied by regular regions $A_{\rm reg}$, for the  standard map as a function of $K$. The light blue line (right $y$ axis) corresponds to the calculation  of $|\lambda_1|$ using the momentum-position method. 
}
\end{figure}


\section{\label{sec:conclusions} Final remarks}
Although the OTOC has been around for quite some time, it was only recently that attracted a lot of attention. The main motivation is related to understand scrambling of quantum information. The other, no less important, is associated to the characterization of the deep relation between chaos and scrambling. As such there are classical quantities that can emerge from the time behavior of the OTOC. The Lyapunov exponent is the most widely known example because of the butterfly effect. However, the chaotic behavior is also characterized by relaxation and this is governed by the Ruelle-Pollicott resonances. In this work, we have extended the results obtained in \cite{PhysRevLett.121.210601} by showing that for a mixed system, generalized resonances dominate the approach to equilibrium (or to the asymptotic behavior) of the OTOC in a wide range of chaos parameter. The results were obtained for the quantum standard map, which is a kicked system, that depends on one parameter which allow us to cover regions ranging from fully integrable to fully chaotic.  In fact, we have also been able to show that the variations in the decay rates mimic very well the intricacies and variations of the resonances that reflect the appearance (and subsequent disappearance) of small regular islands for certain, quasi-periodic, values of the kicking strength. 

Unveiling the classical skeleton of quantum complex systems has, at least, two visible advantages. One is that the classical quantities determining some aspects of quantum time behavior may provide conceptually rich insight and, in some cases, may be easier to compute. Second, in the cases where there is no clear classical analogue, they can provide powerful tools for semiclassical approaches \cite{prosen2004ruelle}.
We expect to make advances in that direction in future efforts. 

\begin{acknowledgments}
 The authors thank M. Saraceno for interesting discussions. 
 This study has been (partially) supported by CONICET (Grant No.~PIP 11220150100493CO) ANCyPT (Grants No.~PICT-2020-SERIEA-00740 and PICT-2020-SERIEA-01082). I.G-M received support from CNRS (France) through the International Research Project (IRP) LICOQ.
\end{acknowledgments}

\appendix

\section*{\label{app:convergence} Appendix}

\subsection*{Eigenvalues of Perron-Frobenius operator}
\begin{figure}
\includegraphics[width=0.45\textwidth]{fig8} %%{Figures/fig8app.pdf}
\caption{\label{fig:fishman} 
\emph{Top Panel:} Eigenvalues $|\lambda_1|$ of the PF operator computed with Fourier method versus chaos parameter for increasing values of $D$ and noise $\sigma = 0.2$.
\emph{Bottom Panel:} Eigenvalues $|\lambda_1|$ of the PF operator calculated with the Fourier method without noise (blue dashed lines) and $e^{-\gamma/2}$ obtained from decay rates $\gamma$ (black dots and solid line) versus chaos parameter. The eigenvalues calculated with noise $\sigma = 0.2$ are shown in gray lines. The PF discrete matrix has dimension $D=30$.}
\end{figure}

 As we discussed in Sec.~\ref{sec:introduction}, the different decay rates  $\gamma$ of the correlator are related to the eigenvalue $\lambda_1$ of the Perron-Frobenius operator which has the greater modulus such that $|\lambda_1|<1$ i.e., the leading eigenvalue inside the unit circle in the complex plane ($|\lambda_0|=1>|\lambda_1|>|\lambda_2|>...>0$).
 Therefore, in order to calculate these eigenvalues, the PF operator is approximated by a finite dimensional one of dimension $D$. Here we show that, for fixed noise in position $\sigma = 0.2$, the eigenvalues of the Perron-Frobenius operator converge rapidly using the Fourier method. In Fig.~\ref{fig:fishman} we show $|\lambda_1|$ in terms of the chaos parameter for increasing matrix dimensions. It is clear that for $D \geq 22$ the eigenvalues associated to the different $K$'s converge their asymptotic value. As we mention in the main text, we need to add a given amount of noise to the calculation in order to converge to the value of $\lambda$ obtained from the decay of $O_1$. This can be observed in the plot of Fig.~\ref{fig:fishman}, where we show  $|\lambda_1|$ calculated with and without noise along with the corresponding $e^{-\gamma/2}$ obtained from the decay rate of $|O_1|$ (see Eq.~\eqref{eq:gammalambda}). It can be seen that in the regions of $K$ where there are ``bumps'' in the decay rates (islands in the classical phase space), the eigenvalues $|\lambda_1|$ do not converge to the fitted values. For these regions, the addition of noise is essential.

%%%%%%%%%%%%%%%%%%%%%
%\bibliography{bibliography}% Produces the bibliography via BibTeX.

%%%%%%%%%%%%%%%%%%%%%%%%%%%%%%%%%%%
%merlin.mbs apsrev4-1.bst 2010-07-25 4.21a (PWD, AO, DPC) hacked
%Control: key (0)
%Control: author (0) dotless jnrlst
%Control: editor formatted (1) identically to author
%Control: production of article title (0) allowed
%Control: page (1) range
%Control: year (0) verbatim
%Control: production of eprint (0) enabled
\providecommand{\noopsort}[1]{}\providecommand{\singleletter}[1]{#1}%
\begin{thebibliography}{57}%
\makeatletter
\providecommand \@ifxundefined [1]{%
 \@ifx{#1\undefined}
}%
\providecommand \@ifnum [1]{%
 \ifnum #1\expandafter \@firstoftwo
 \else \expandafter \@secondoftwo
 \fi
}%
\providecommand \@ifx [1]{%
 \ifx #1\expandafter \@firstoftwo
 \else \expandafter \@secondoftwo
 \fi
}%
\providecommand \natexlab [1]{#1}%
\providecommand \enquote  [1]{``#1''}%
\providecommand \bibnamefont  [1]{#1}%
\providecommand \bibfnamefont [1]{#1}%
\providecommand \citenamefont [1]{#1}%
\providecommand \href@noop [0]{\@secondoftwo}%
\providecommand \href [0]{\begingroup \@sanitize@url \@href}%
\providecommand \@href[1]{\@@startlink{#1}\@@href}%
\providecommand \@@href[1]{\endgroup#1\@@endlink}%
\providecommand \@sanitize@url [0]{\catcode `\\12\catcode `\$12\catcode
  `\&12\catcode `\#12\catcode `\^12\catcode `\_12\catcode `\%12\relax}%
\providecommand \@@startlink[1]{}%
\providecommand \@@endlink[0]{}%
\providecommand \url  [0]{\begingroup\@sanitize@url \@url }%
\providecommand \@url [1]{\endgroup\@href {#1}{\urlprefix }}%
\providecommand \urlprefix  [0]{URL }%
\providecommand \Eprint [0]{\href }%
\providecommand \doibase [0]{http://dx.doi.org/}%
\providecommand \selectlanguage [0]{\@gobble}%
\providecommand \bibinfo  [0]{\@secondoftwo}%
\providecommand \bibfield  [0]{\@secondoftwo}%
\providecommand \translation [1]{[#1]}%
\providecommand \BibitemOpen [0]{}%
\providecommand \bibitemStop [0]{}%
\providecommand \bibitemNoStop [0]{.\EOS\space}%
\providecommand \EOS [0]{\spacefactor3000\relax}%
\providecommand \BibitemShut  [1]{\csname bibitem#1\endcsname}%
\let\auto@bib@innerbib\@empty
%</preamble>
\bibitem [{\citenamefont {Larkin}\ and\ \citenamefont
  {Ovchinnikov}(1969)}]{larkin1969quasiclassical}%
  \BibitemOpen
  \bibfield  {author} {\bibinfo {author} {\bibfnamefont {AI}~\bibnamefont
  {Larkin}}\ and\ \bibinfo {author} {\bibfnamefont {Yu~N}\ \bibnamefont
  {Ovchinnikov}},\ }\bibfield  {title} {\enquote {\bibinfo {title}
  {Quasiclassical method in the theory of superconductivity},}\ }\href@noop {}
  {\bibfield  {journal} {\bibinfo  {journal} {Sov Phys JETP}\ }\textbf
  {\bibinfo {volume} {28}},\ \bibinfo {pages} {1200--1205} (\bibinfo {year}
  {1969})}\BibitemShut {NoStop}%
\bibitem [{\citenamefont {Maldacena}\ \emph {et~al.}(2016)\citenamefont
  {Maldacena}, \citenamefont {Shenker},\ and\ \citenamefont
  {Stanford}}]{maldacena2016bound}%
  \BibitemOpen
  \bibfield  {author} {\bibinfo {author} {\bibfnamefont {Juan}\ \bibnamefont
  {Maldacena}}, \bibinfo {author} {\bibfnamefont {Stephen~H}\ \bibnamefont
  {Shenker}}, \ and\ \bibinfo {author} {\bibfnamefont {Douglas}\ \bibnamefont
  {Stanford}},\ }\bibfield  {title} {\enquote {\bibinfo {title} {A bound on
  chaos},}\ }\href@noop {} {\bibfield  {journal} {\bibinfo  {journal} {JHEP}\
  }\textbf {\bibinfo {volume} {2016}},\ \bibinfo {pages} {1--17} (\bibinfo
  {year} {2016})}\BibitemShut {NoStop}%
\bibitem [{\citenamefont {Ali}\ \emph {et~al.}(2020)\citenamefont {Ali},
  \citenamefont {Bhattacharyya}, \citenamefont {Haque}, \citenamefont {Kim},
  \citenamefont {Moynihan},\ and\ \citenamefont {Murugan}}]{ali2020chaos}%
  \BibitemOpen
  \bibfield  {author} {\bibinfo {author} {\bibfnamefont {Tibra}\ \bibnamefont
  {Ali}}, \bibinfo {author} {\bibfnamefont {Arpan}\ \bibnamefont
  {Bhattacharyya}}, \bibinfo {author} {\bibfnamefont {S.~Shajidul}\
  \bibnamefont {Haque}}, \bibinfo {author} {\bibfnamefont {Eugene~H.}\
  \bibnamefont {Kim}}, \bibinfo {author} {\bibfnamefont {Nathan}\ \bibnamefont
  {Moynihan}}, \ and\ \bibinfo {author} {\bibfnamefont {Jeff}\ \bibnamefont
  {Murugan}},\ }\bibfield  {title} {\enquote {\bibinfo {title} {Chaos and
  complexity in quantum mechanics},}\ }\href {\doibase
  10.1103/PhysRevD.101.026021} {\bibfield  {journal} {\bibinfo  {journal}
  {Phys. Rev. D}\ }\textbf {\bibinfo {volume} {101}},\ \bibinfo {pages}
  {026021} (\bibinfo {year} {2020})}\BibitemShut {NoStop}%
\bibitem [{\citenamefont {Hashimoto}\ \emph {et~al.}(2020)\citenamefont
  {Hashimoto}, \citenamefont {Huh}, \citenamefont {Kim},\ and\ \citenamefont
  {Watanabe}}]{hashimoto2020exponential}%
  \BibitemOpen
  \bibfield  {author} {\bibinfo {author} {\bibfnamefont {Koji}\ \bibnamefont
  {Hashimoto}}, \bibinfo {author} {\bibfnamefont {Kyoung-Bum}\ \bibnamefont
  {Huh}}, \bibinfo {author} {\bibfnamefont {Keun-Young}\ \bibnamefont {Kim}}, \
  and\ \bibinfo {author} {\bibfnamefont {Ryota}\ \bibnamefont {Watanabe}},\
  }\bibfield  {title} {\enquote {\bibinfo {title} {Exponential growth of
  out-of-time-order correlator without chaos: inverted harmonic oscillator},}\
  }\href@noop {} {\bibfield  {journal} {\bibinfo  {journal} {JHEP}\ }\textbf
  {\bibinfo {volume} {2020}},\ \bibinfo {pages} {1--25} (\bibinfo {year}
  {2020})}\BibitemShut {NoStop}%
\bibitem [{\citenamefont {Morita}(2021)}]{morita2021extracting}%
  \BibitemOpen
  \bibfield  {author} {\bibinfo {author} {\bibfnamefont {Takeshi}\ \bibnamefont
  {Morita}},\ }\bibfield  {title} {\enquote {\bibinfo {title} {Extracting
  classical {L}yapunov exponent from one-dimensional quantum mechanics},}\
  }\href {https://arxiv.org/abs/2105.09603} {\bibfield  {journal} {\bibinfo
  {journal} {2105.09603v2}\ } (\bibinfo {year} {2021})}\BibitemShut {NoStop}%
\bibitem [{\citenamefont {Hummel}\ \emph {et~al.}(2019)\citenamefont {Hummel},
  \citenamefont {Geiger}, \citenamefont {Urbina},\ and\ \citenamefont
  {Richter}}]{HummelPRL2019}%
  \BibitemOpen
  \bibfield  {author} {\bibinfo {author} {\bibfnamefont {Quirin}\ \bibnamefont
  {Hummel}}, \bibinfo {author} {\bibfnamefont {Benjamin}\ \bibnamefont
  {Geiger}}, \bibinfo {author} {\bibfnamefont {Juan~Diego}\ \bibnamefont
  {Urbina}}, \ and\ \bibinfo {author} {\bibfnamefont {Klaus}\ \bibnamefont
  {Richter}},\ }\bibfield  {title} {\enquote {\bibinfo {title} {Reversible
  quantum information spreading in many-body systems near criticality},}\
  }\href {\doibase 10.1103/PhysRevLett.123.160401} {\bibfield  {journal}
  {\bibinfo  {journal} {Phys. Rev. Lett.}\ }\textbf {\bibinfo {volume} {123}},\
  \bibinfo {pages} {160401} (\bibinfo {year} {2019})}\BibitemShut {NoStop}%
\bibitem [{\citenamefont {Xu}\ \emph {et~al.}(2020)\citenamefont {Xu},
  \citenamefont {Scaffidi},\ and\ \citenamefont {Cao}}]{XuPRL2020}%
  \BibitemOpen
  \bibfield  {author} {\bibinfo {author} {\bibfnamefont {Tianrui}\ \bibnamefont
  {Xu}}, \bibinfo {author} {\bibfnamefont {Thomas}\ \bibnamefont {Scaffidi}}, \
  and\ \bibinfo {author} {\bibfnamefont {Xiangyu}\ \bibnamefont {Cao}},\
  }\bibfield  {title} {\enquote {\bibinfo {title} {Does scrambling equal
  chaos?}}\ }\href {\doibase 10.1103/PhysRevLett.124.140602} {\bibfield
  {journal} {\bibinfo  {journal} {Phys. Rev. Lett.}\ }\textbf {\bibinfo
  {volume} {124}},\ \bibinfo {pages} {140602} (\bibinfo {year}
  {2020})}\BibitemShut {NoStop}%
\bibitem [{\citenamefont {Rozenbaum}\ \emph {et~al.}(2020)\citenamefont
  {Rozenbaum}, \citenamefont {Bunimovich},\ and\ \citenamefont
  {Galitski}}]{RozenbaumPRL2020}%
  \BibitemOpen
  \bibfield  {author} {\bibinfo {author} {\bibfnamefont {Efim~B.}\ \bibnamefont
  {Rozenbaum}}, \bibinfo {author} {\bibfnamefont {Leonid~A.}\ \bibnamefont
  {Bunimovich}}, \ and\ \bibinfo {author} {\bibfnamefont {Victor}\ \bibnamefont
  {Galitski}},\ }\bibfield  {title} {\enquote {\bibinfo {title} {Early-time
  exponential instabilities in nonchaotic quantum systems},}\ }\href {\doibase
  10.1103/PhysRevLett.125.014101} {\bibfield  {journal} {\bibinfo  {journal}
  {Phys. Rev. Lett.}\ }\textbf {\bibinfo {volume} {125}},\ \bibinfo {pages}
  {014101} (\bibinfo {year} {2020})}\BibitemShut {NoStop}%
\bibitem [{\citenamefont {Kidd}\ \emph {et~al.}(2021)\citenamefont {Kidd},
  \citenamefont {Safavi-Naini},\ and\ \citenamefont {Corney}}]{kidd2021saddle}%
  \BibitemOpen
  \bibfield  {author} {\bibinfo {author} {\bibfnamefont {R.~A.}\ \bibnamefont
  {Kidd}}, \bibinfo {author} {\bibfnamefont {A.}~\bibnamefont {Safavi-Naini}},
  \ and\ \bibinfo {author} {\bibfnamefont {J.~F.}\ \bibnamefont {Corney}},\
  }\bibfield  {title} {\enquote {\bibinfo {title} {Saddle-point scrambling
  without thermalization},}\ }\href {\doibase 10.1103/PhysRevA.103.033304}
  {\bibfield  {journal} {\bibinfo  {journal} {Phys. Rev. A}\ }\textbf {\bibinfo
  {volume} {103}},\ \bibinfo {pages} {033304} (\bibinfo {year}
  {2021})}\BibitemShut {NoStop}%
\bibitem [{\citenamefont {Kukuljan}\ \emph {et~al.}(2017)\citenamefont
  {Kukuljan}, \citenamefont {Grozdanov},\ and\ \citenamefont
  {Prosen}}]{kukuljan2017}%
  \BibitemOpen
  \bibfield  {author} {\bibinfo {author} {\bibfnamefont {Ivan}\ \bibnamefont
  {Kukuljan}}, \bibinfo {author} {\bibfnamefont {Sa\v{s}o}\ \bibnamefont
  {Grozdanov}}, \ and\ \bibinfo {author} {\bibfnamefont {Toma\v{z}}\
  \bibnamefont {Prosen}},\ }\bibfield  {title} {\enquote {\bibinfo {title}
  {Weak quantum chaos},}\ }\href {\doibase 10.1103/PhysRevB.96.060301}
  {\bibfield  {journal} {\bibinfo  {journal} {Phys. Rev. B}\ }\textbf {\bibinfo
  {volume} {96}},\ \bibinfo {pages} {060301} (\bibinfo {year}
  {2017})}\BibitemShut {NoStop}%
\bibitem [{\citenamefont {Lin}\ and\ \citenamefont
  {Motrunich}(2018)}]{Motrunich2018}%
  \BibitemOpen
  \bibfield  {author} {\bibinfo {author} {\bibfnamefont {Cheng-Ju}\
  \bibnamefont {Lin}}\ and\ \bibinfo {author} {\bibfnamefont {Olexei~I.}\
  \bibnamefont {Motrunich}},\ }\bibfield  {title} {\enquote {\bibinfo {title}
  {Out-of-time-ordered correlators in a quantum {I}sing chain},}\ }\href
  {\doibase 10.1103/PhysRevB.97.144304} {\bibfield  {journal} {\bibinfo
  {journal} {Phys. Rev. B}\ }\textbf {\bibinfo {volume} {97}},\ \bibinfo
  {pages} {144304} (\bibinfo {year} {2018})}\BibitemShut {NoStop}%
\bibitem [{\citenamefont {Riddell}\ and\ \citenamefont
  {S\o{}rensen}(2019)}]{PhysRevB.99.054205}%
  \BibitemOpen
  \bibfield  {author} {\bibinfo {author} {\bibfnamefont {Jonathon}\
  \bibnamefont {Riddell}}\ and\ \bibinfo {author} {\bibfnamefont {Erik~S.}\
  \bibnamefont {S\o{}rensen}},\ }\bibfield  {title} {\enquote {\bibinfo {title}
  {Out-of-time ordered correlators and entanglement growth in the random-field
  xx spin chain},}\ }\href {\doibase 10.1103/PhysRevB.99.054205} {\bibfield
  {journal} {\bibinfo  {journal} {Phys. Rev. B}\ }\textbf {\bibinfo {volume}
  {99}},\ \bibinfo {pages} {054205} (\bibinfo {year} {2019})}\BibitemShut
  {NoStop}%
\bibitem [{\citenamefont {Fortes}\ \emph {et~al.}(2019)\citenamefont {Fortes},
  \citenamefont {Garc\'{\i}a-Mata}, \citenamefont {Jalabert},\ and\
  \citenamefont {Wisniacki}}]{PhysRevE.100.042201}%
  \BibitemOpen
  \bibfield  {author} {\bibinfo {author} {\bibfnamefont {Emiliano~M.}\
  \bibnamefont {Fortes}}, \bibinfo {author} {\bibfnamefont {Ignacio}\
  \bibnamefont {Garc\'{\i}a-Mata}}, \bibinfo {author} {\bibfnamefont
  {Rodolfo~A.}\ \bibnamefont {Jalabert}}, \ and\ \bibinfo {author}
  {\bibfnamefont {Diego~A.}\ \bibnamefont {Wisniacki}},\ }\bibfield  {title}
  {\enquote {\bibinfo {title} {Gauging classical and quantum integrability
  through out-of-time-ordered correlators},}\ }\href {\doibase
  10.1103/PhysRevE.100.042201} {\bibfield  {journal} {\bibinfo  {journal}
  {Phys. Rev. E}\ }\textbf {\bibinfo {volume} {100}},\ \bibinfo {pages}
  {042201} (\bibinfo {year} {2019})}\BibitemShut {NoStop}%
\bibitem [{\citenamefont {Shukla}\ \emph {et~al.}(2022)\citenamefont {Shukla},
  \citenamefont {Lakshminarayan},\ and\ \citenamefont {Mishra}}]{Shukla2022}%
  \BibitemOpen
  \bibfield  {author} {\bibinfo {author} {\bibfnamefont {Rohit~Kumar}\
  \bibnamefont {Shukla}}, \bibinfo {author} {\bibfnamefont {Arul}\ \bibnamefont
  {Lakshminarayan}}, \ and\ \bibinfo {author} {\bibfnamefont {Sunil~Kumar}\
  \bibnamefont {Mishra}},\ }\bibfield  {title} {\enquote {\bibinfo {title}
  {Out-of-time-order correlators of nonlocal block-spin and random observables
  in integrable and nonintegrable spin chains},}\ }\href {\doibase
  10.1103/PhysRevB.105.224307} {\bibfield  {journal} {\bibinfo  {journal}
  {Phys. Rev. B}\ }\textbf {\bibinfo {volume} {105}},\ \bibinfo {pages}
  {224307} (\bibinfo {year} {2022})}\BibitemShut {NoStop}%
\bibitem [{\citenamefont {D\'ora}\ and\ \citenamefont
  {Moessner}(2017)}]{moessner2017}%
  \BibitemOpen
  \bibfield  {author} {\bibinfo {author} {\bibfnamefont {Bal\'azs}\
  \bibnamefont {D\'ora}}\ and\ \bibinfo {author} {\bibfnamefont {Roderich}\
  \bibnamefont {Moessner}},\ }\bibfield  {title} {\enquote {\bibinfo {title}
  {Out-of-time-ordered density correlators in luttinger liquids},}\ }\href
  {\doibase 10.1103/PhysRevLett.119.026802} {\bibfield  {journal} {\bibinfo
  {journal} {Phys. Rev. Lett.}\ }\textbf {\bibinfo {volume} {119}},\ \bibinfo
  {pages} {026802} (\bibinfo {year} {2017})}\BibitemShut {NoStop}%
\bibitem [{\citenamefont {Hashimoto}\ \emph {et~al.}(2017)\citenamefont
  {Hashimoto}, \citenamefont {Murata},\ and\ \citenamefont
  {Yoshii}}]{hashimoto2017}%
  \BibitemOpen
  \bibfield  {author} {\bibinfo {author} {\bibfnamefont {Koji}\ \bibnamefont
  {Hashimoto}}, \bibinfo {author} {\bibfnamefont {Keiju}\ \bibnamefont
  {Murata}}, \ and\ \bibinfo {author} {\bibfnamefont {Ryosuke}\ \bibnamefont
  {Yoshii}},\ }\bibfield  {title} {\enquote {\bibinfo {title}
  {Out-of-time-order correlators in quantum mechanics},}\ }\href
  {https://link.springer.com/article/10.1007/JHEP10(2017)138} {\bibfield
  {journal} {\bibinfo  {journal} {J. High Energy Phys.}\ }\textbf {\bibinfo
  {volume} {10}},\ \bibinfo {pages} {138} (\bibinfo {year} {2017})}\BibitemShut
  {NoStop}%
\bibitem [{\citenamefont {Jalabert}\ \emph {et~al.}(2018)\citenamefont
  {Jalabert}, \citenamefont {Garc\'{\i}a-Mata},\ and\ \citenamefont
  {Wisniacki}}]{JGMW2018}%
  \BibitemOpen
  \bibfield  {author} {\bibinfo {author} {\bibfnamefont {Rodolfo~A.}\
  \bibnamefont {Jalabert}}, \bibinfo {author} {\bibfnamefont {Ignacio}\
  \bibnamefont {Garc\'{\i}a-Mata}}, \ and\ \bibinfo {author} {\bibfnamefont
  {Diego~A.}\ \bibnamefont {Wisniacki}},\ }\bibfield  {title} {\enquote
  {\bibinfo {title} {Semiclassical theory of out-of-time-order correlators for
  low-dimensional classically chaotic systems},}\ }\href {\doibase
  10.1103/PhysRevE.98.062218} {\bibfield  {journal} {\bibinfo  {journal} {Phys.
  Rev. E}\ }\textbf {\bibinfo {volume} {98}},\ \bibinfo {pages} {062218}
  (\bibinfo {year} {2018})}\BibitemShut {NoStop}%
\bibitem [{\citenamefont {Rammensee}\ \emph {et~al.}(2018)\citenamefont
  {Rammensee}, \citenamefont {Urbina},\ and\ \citenamefont
  {Richter}}]{rammensee2018}%
  \BibitemOpen
  \bibfield  {author} {\bibinfo {author} {\bibfnamefont {Josef}\ \bibnamefont
  {Rammensee}}, \bibinfo {author} {\bibfnamefont {Juan~Diego}\ \bibnamefont
  {Urbina}}, \ and\ \bibinfo {author} {\bibfnamefont {Klaus}\ \bibnamefont
  {Richter}},\ }\bibfield  {title} {\enquote {\bibinfo {title} {Many-body
  quantum interference and the saturation of out-of-time-order correlators},}\
  }\href {\doibase 10.1103/PhysRevLett.121.124101} {\bibfield  {journal}
  {\bibinfo  {journal} {Phys. Rev. Lett.}\ }\textbf {\bibinfo {volume} {121}},\
  \bibinfo {pages} {124101} (\bibinfo {year} {2018})}\BibitemShut {NoStop}%
\bibitem [{\citenamefont {Huang}\ \emph {et~al.}(2019)\citenamefont {Huang},
  \citenamefont {Brand\~ao},\ and\ \citenamefont {Zhang}}]{huang2019finite}%
  \BibitemOpen
  \bibfield  {author} {\bibinfo {author} {\bibfnamefont {Yichen}\ \bibnamefont
  {Huang}}, \bibinfo {author} {\bibfnamefont {Fernando G. S.~L.}\ \bibnamefont
  {Brand\~ao}}, \ and\ \bibinfo {author} {\bibfnamefont {Yong-Liang}\
  \bibnamefont {Zhang}},\ }\bibfield  {title} {\enquote {\bibinfo {title}
  {Finite-size scaling of out-of-time-ordered correlators at late times},}\
  }\href {\doibase 10.1103/PhysRevLett.123.010601} {\bibfield  {journal}
  {\bibinfo  {journal} {Phys. Rev. Lett.}\ }\textbf {\bibinfo {volume} {123}},\
  \bibinfo {pages} {010601} (\bibinfo {year} {2019})}\BibitemShut {NoStop}%
\bibitem [{\citenamefont {Sinha}\ \emph {et~al.}(2021)\citenamefont {Sinha},
  \citenamefont {Ray},\ and\ \citenamefont {Sinha}}]{sinha2021fingerprint}%
  \BibitemOpen
  \bibfield  {author} {\bibinfo {author} {\bibfnamefont {Sudip}\ \bibnamefont
  {Sinha}}, \bibinfo {author} {\bibfnamefont {Sayak}\ \bibnamefont {Ray}}, \
  and\ \bibinfo {author} {\bibfnamefont {Subhasis}\ \bibnamefont {Sinha}},\
  }\bibfield  {title} {\enquote {\bibinfo {title} {Fingerprint of chaos and
  quantum scars in kicked {D}icke model: an out-of-time-order correlator
  study},}\ }\href {https://doi.org/10.1088/1361-648X/abe26b} {\bibfield
  {journal} {\bibinfo  {journal} {J. Phys. : Condensed Matter}\ }\textbf
  {\bibinfo {volume} {33}},\ \bibinfo {pages} {174005} (\bibinfo {year}
  {2021})}\BibitemShut {NoStop}%
\bibitem [{\citenamefont {Marković}\ and\ \citenamefont
  {Čubrović}(2022)}]{Markovic2022}%
  \BibitemOpen
  \bibfield  {author} {\bibinfo {author} {\bibfnamefont {Dragan}\ \bibnamefont
  {Marković}}\ and\ \bibinfo {author} {\bibfnamefont {Mihailo}\ \bibnamefont
  {Čubrović}},\ }\bibfield  {title} {\enquote {\bibinfo {title} {Detecting
  few-body quantum chaos: out-of-time ordered correlators at saturation},}\
  }\href {\doibase 10.1007/JHEP05(2022)023} {\bibfield  {journal} {\bibinfo
  {journal} {Journal of High Energy Physics}\ ,\ \bibinfo {pages} {23}}
  (\bibinfo {year} {2022})}\BibitemShut {NoStop}%
\bibitem [{\citenamefont {Fortes}\ \emph {et~al.}(2020)\citenamefont {Fortes},
  \citenamefont {Garc{\'\i}a-Mata}, \citenamefont {Jalabert},\ and\
  \citenamefont {Wisniacki}}]{fortes2020signatures}%
  \BibitemOpen
  \bibfield  {author} {\bibinfo {author} {\bibfnamefont {Emiliano~M}\
  \bibnamefont {Fortes}}, \bibinfo {author} {\bibfnamefont {Ignacio}\
  \bibnamefont {Garc{\'\i}a-Mata}}, \bibinfo {author} {\bibfnamefont
  {Rodolfo~A}\ \bibnamefont {Jalabert}}, \ and\ \bibinfo {author}
  {\bibfnamefont {Diego~A}\ \bibnamefont {Wisniacki}},\ }\bibfield  {title}
  {\enquote {\bibinfo {title} {Signatures of quantum chaos transition in short
  spin chains},}\ }\href
  {https://iopscience.iop.org/article/10.1209/0295-5075/130/60001/meta?casa_token=mnx7m-x9KLYAAAAA:5J5sU9fYWkDZQrBpXCTU8LRmJ3AH3YKD1D4t0nGPuKTKhzMv7sl395IV_riKCo9-ROGFKBilWsmmoio}
  {\bibfield  {journal} {\bibinfo  {journal} {EPL (Europhys. Lett.)}\ }\textbf
  {\bibinfo {volume} {130}},\ \bibinfo {pages} {60001} (\bibinfo {year}
  {2020})}\BibitemShut {NoStop}%
\bibitem [{\citenamefont {Blank}\ \emph {et~al.}(2002)\citenamefont {Blank},
  \citenamefont {Keller},\ and\ \citenamefont {Liverani}}]{Blank2002}%
  \BibitemOpen
  \bibfield  {author} {\bibinfo {author} {\bibfnamefont {Michael}\ \bibnamefont
  {Blank}}, \bibinfo {author} {\bibfnamefont {Gerhard}\ \bibnamefont {Keller}},
  \ and\ \bibinfo {author} {\bibfnamefont {Carlangelo}\ \bibnamefont
  {Liverani}},\ }\bibfield  {title} {\enquote {\bibinfo {title}
  {Ruelle~perron~frobenius spectrum for anosov maps},}\ }\href {\doibase
  10.1088/0951-7715/15/6/309} {\bibfield  {journal} {\bibinfo  {journal}
  {Nonlinearity}\ }\textbf {\bibinfo {volume} {15}},\ \bibinfo {pages}
  {1905--1973} (\bibinfo {year} {2002})}\BibitemShut {NoStop}%
\bibitem [{\citenamefont {Garc\'{\i}a-Mata}\ \emph {et~al.}(2018)\citenamefont
  {Garc\'{\i}a-Mata}, \citenamefont {Saraceno}, \citenamefont {Jalabert},
  \citenamefont {Roncaglia},\ and\ \citenamefont
  {Wisniacki}}]{PhysRevLett.121.210601}%
  \BibitemOpen
  \bibfield  {author} {\bibinfo {author} {\bibfnamefont {Ignacio}\ \bibnamefont
  {Garc\'{\i}a-Mata}}, \bibinfo {author} {\bibfnamefont {Marcos}\ \bibnamefont
  {Saraceno}}, \bibinfo {author} {\bibfnamefont {Rodolfo~A.}\ \bibnamefont
  {Jalabert}}, \bibinfo {author} {\bibfnamefont {Augusto~J.}\ \bibnamefont
  {Roncaglia}}, \ and\ \bibinfo {author} {\bibfnamefont {Diego~A.}\
  \bibnamefont {Wisniacki}},\ }\bibfield  {title} {\enquote {\bibinfo {title}
  {Chaos signatures in the short and long time behavior of the out-of-time
  ordered correlator},}\ }\href {\doibase 10.1103/PhysRevLett.121.210601}
  {\bibfield  {journal} {\bibinfo  {journal} {Phys. Rev. Lett.}\ }\textbf
  {\bibinfo {volume} {121}},\ \bibinfo {pages} {210601} (\bibinfo {year}
  {2018})}\BibitemShut {NoStop}%
\bibitem [{\citenamefont {Manderfeld}\ \emph {et~al.}(2001)\citenamefont
  {Manderfeld}, \citenamefont {Weber},\ and\ \citenamefont
  {Haake}}]{Manderfeld2001}%
  \BibitemOpen
  \bibfield  {author} {\bibinfo {author} {\bibfnamefont {Christopher}\
  \bibnamefont {Manderfeld}}, \bibinfo {author} {\bibfnamefont {Joachim}\
  \bibnamefont {Weber}}, \ and\ \bibinfo {author} {\bibfnamefont {Fritz}\
  \bibnamefont {Haake}},\ }\bibfield  {title} {\enquote {\bibinfo {title}
  {Classical versus quantum time evolution of (quasi-) probability densities at
  limited phase-space resolution},}\ }\href {\doibase
  10.1088/0305-4470/34/46/312} {\bibfield  {journal} {\bibinfo  {journal}
  {Journal of Physics A: Mathematical and General}\ }\textbf {\bibinfo {volume}
  {34}},\ \bibinfo {pages} {9893--9905} (\bibinfo {year} {2001})}\BibitemShut
  {NoStop}%
\bibitem [{\citenamefont {Weber}\ \emph {et~al.}(2001)\citenamefont {Weber},
  \citenamefont {Haake}, \citenamefont {Braun}, \citenamefont {Manderfeld},\
  and\ \citenamefont {Seba}}]{Weber2001}%
  \BibitemOpen
  \bibfield  {author} {\bibinfo {author} {\bibfnamefont {Joachim}\ \bibnamefont
  {Weber}}, \bibinfo {author} {\bibfnamefont {Fritz}\ \bibnamefont {Haake}},
  \bibinfo {author} {\bibfnamefont {Petr~A}\ \bibnamefont {Braun}}, \bibinfo
  {author} {\bibfnamefont {Christopher}\ \bibnamefont {Manderfeld}}, \ and\
  \bibinfo {author} {\bibfnamefont {Petr}\ \bibnamefont {Seba}},\ }\bibfield
  {title} {\enquote {\bibinfo {title} {Resonances of the frobenius-perron
  operator for a hamiltonian map with a mixed phase space},}\ }\href {\doibase
  10.1088/0305-4470/34/36/306} {\bibfield  {journal} {\bibinfo  {journal}
  {Journal of Physics A: Mathematical and General}\ }\textbf {\bibinfo {volume}
  {34}},\ \bibinfo {pages} {7195--7211} (\bibinfo {year} {2001})}\BibitemShut
  {NoStop}%
\bibitem [{\citenamefont {Khodas}\ \emph {et~al.}(2000)\citenamefont {Khodas},
  \citenamefont {Fishman},\ and\ \citenamefont {Agam}}]{KhodasFishmanAgam}%
  \BibitemOpen
  \bibfield  {author} {\bibinfo {author} {\bibfnamefont {Maxim}\ \bibnamefont
  {Khodas}}, \bibinfo {author} {\bibfnamefont {Shmuel}\ \bibnamefont
  {Fishman}}, \ and\ \bibinfo {author} {\bibfnamefont {Oded}\ \bibnamefont
  {Agam}},\ }\bibfield  {title} {\enquote {\bibinfo {title} {Relaxation to the
  invariant density for the kicked rotor},}\ }\href {\doibase
  10.1103/PhysRevE.62.4769} {\bibfield  {journal} {\bibinfo  {journal} {Phys.
  Rev. E}\ }\textbf {\bibinfo {volume} {62}},\ \bibinfo {pages} {4769--4783}
  (\bibinfo {year} {2000})}\BibitemShut {NoStop}%
\bibitem [{\citenamefont {Fishman}\ and\ \citenamefont
  {Rahav}(2002)}]{fishman2002relaxation}%
  \BibitemOpen
  \bibfield  {author} {\bibinfo {author} {\bibfnamefont {Shmuel}\ \bibnamefont
  {Fishman}}\ and\ \bibinfo {author} {\bibfnamefont {Saar}\ \bibnamefont
  {Rahav}},\ }\bibfield  {title} {\enquote {\bibinfo {title} {Relaxation and
  noise in chaotic systems},}\ }\href@noop {} {\bibfield  {journal} {\bibinfo
  {journal} {Dynamics of Dissipation}\ ,\ \bibinfo {pages} {165--192}}
  (\bibinfo {year} {2002})}\BibitemShut {NoStop}%
\bibitem [{\citenamefont {Nonnenmacher}(2003)}]{nonnenmacher2003spectral}%
  \BibitemOpen
  \bibfield  {author} {\bibinfo {author} {\bibfnamefont {St{\'e}phane}\
  \bibnamefont {Nonnenmacher}},\ }\bibfield  {title} {\enquote {\bibinfo
  {title} {Spectral properties of noisy classical and quantum propagators},}\
  }\href@noop {} {\bibfield  {journal} {\bibinfo  {journal} {Nonlinearity}\
  }\textbf {\bibinfo {volume} {16}},\ \bibinfo {pages} {1685} (\bibinfo {year}
  {2003})}\BibitemShut {NoStop}%
\bibitem [{\citenamefont {Chirikov}(1979)}]{Chirikov1979}%
  \BibitemOpen
  \bibfield  {author} {\bibinfo {author} {\bibfnamefont {Boris~V}\ \bibnamefont
  {Chirikov}},\ }\bibfield  {title} {\enquote {\bibinfo {title} {A universal
  instability of many-dimensional oscillator systems},}\ }\href {\doibase
  10.1016/0370-1573(79)90023-1} {\bibfield  {journal} {\bibinfo  {journal}
  {Physics Reports}\ }\textbf {\bibinfo {volume} {52}},\ \bibinfo {pages}
  {263--379} (\bibinfo {year} {1979})}\BibitemShut {NoStop}%
\bibitem [{\citenamefont {Shenker}\ and\ \citenamefont
  {Stanford}(2014{\natexlab{a}})}]{shenker2014black}%
  \BibitemOpen
  \bibfield  {author} {\bibinfo {author} {\bibfnamefont {Stephen~H}\
  \bibnamefont {Shenker}}\ and\ \bibinfo {author} {\bibfnamefont {Douglas}\
  \bibnamefont {Stanford}},\ }\bibfield  {title} {\enquote {\bibinfo {title}
  {Black holes and the butterfly effect},}\ }\href@noop {} {\bibfield
  {journal} {\bibinfo  {journal} {JHEP}\ }\textbf {\bibinfo {volume} {2014}},\
  \bibinfo {pages} {1--25} (\bibinfo {year} {2014}{\natexlab{a}})}\BibitemShut
  {NoStop}%
\bibitem [{\citenamefont {Shenker}\ and\ \citenamefont
  {Stanford}(2014{\natexlab{b}})}]{shenker2014multiple}%
  \BibitemOpen
  \bibfield  {author} {\bibinfo {author} {\bibfnamefont {Stephen~H}\
  \bibnamefont {Shenker}}\ and\ \bibinfo {author} {\bibfnamefont {Douglas}\
  \bibnamefont {Stanford}},\ }\bibfield  {title} {\enquote {\bibinfo {title}
  {Multiple shocks},}\ }\href@noop {} {\bibfield  {journal} {\bibinfo
  {journal} {JHEP}\ }\textbf {\bibinfo {volume} {2014}},\ \bibinfo {pages}
  {1--20} (\bibinfo {year} {2014}{\natexlab{b}})}\BibitemShut {NoStop}%
\bibitem [{\citenamefont {Shenker}\ and\ \citenamefont
  {Stanford}(2015)}]{shenker2015stringy}%
  \BibitemOpen
  \bibfield  {author} {\bibinfo {author} {\bibfnamefont {Stephen~H}\
  \bibnamefont {Shenker}}\ and\ \bibinfo {author} {\bibfnamefont {Douglas}\
  \bibnamefont {Stanford}},\ }\bibfield  {title} {\enquote {\bibinfo {title}
  {Stringy effects in scrambling},}\ }\href@noop {} {\bibfield  {journal}
  {\bibinfo  {journal} {JHEP}\ }\textbf {\bibinfo {volume} {2015}},\ \bibinfo
  {pages} {1--34} (\bibinfo {year} {2015})}\BibitemShut {NoStop}%
\bibitem [{\citenamefont {Rozenbaum}\ \emph {et~al.}(2017)\citenamefont
  {Rozenbaum}, \citenamefont {Ganeshan},\ and\ \citenamefont
  {Galitski}}]{PhysRevLett.118.086801}%
  \BibitemOpen
  \bibfield  {author} {\bibinfo {author} {\bibfnamefont {Efim~B.}\ \bibnamefont
  {Rozenbaum}}, \bibinfo {author} {\bibfnamefont {Sriram}\ \bibnamefont
  {Ganeshan}}, \ and\ \bibinfo {author} {\bibfnamefont {Victor}\ \bibnamefont
  {Galitski}},\ }\bibfield  {title} {\enquote {\bibinfo {title} {Lyapunov
  exponent and out-of-time-ordered correlator's growth rate in a chaotic
  system},}\ }\href {\doibase 10.1103/PhysRevLett.118.086801} {\bibfield
  {journal} {\bibinfo  {journal} {Phys. Rev. Lett.}\ }\textbf {\bibinfo
  {volume} {118}},\ \bibinfo {pages} {086801} (\bibinfo {year}
  {2017})}\BibitemShut {NoStop}%
\bibitem [{\citenamefont {Ch\'avez-Carlos}\ \emph {et~al.}(2019)\citenamefont
  {Ch\'avez-Carlos}, \citenamefont {L\'opez-del Carpio}, \citenamefont
  {Bastarrachea-Magnani}, \citenamefont {Str\'ansk\'y}, \citenamefont
  {Lerma-Hern\'andez}, \citenamefont {Santos},\ and\ \citenamefont
  {Hirsch}}]{chavez2019quantum}%
  \BibitemOpen
  \bibfield  {author} {\bibinfo {author} {\bibfnamefont {Jorge}\ \bibnamefont
  {Ch\'avez-Carlos}}, \bibinfo {author} {\bibfnamefont {B.}~\bibnamefont
  {L\'opez-del Carpio}}, \bibinfo {author} {\bibfnamefont {Miguel~A.}\
  \bibnamefont {Bastarrachea-Magnani}}, \bibinfo {author} {\bibfnamefont
  {Pavel}\ \bibnamefont {Str\'ansk\'y}}, \bibinfo {author} {\bibfnamefont
  {Sergio}\ \bibnamefont {Lerma-Hern\'andez}}, \bibinfo {author} {\bibfnamefont
  {Lea~F.}\ \bibnamefont {Santos}}, \ and\ \bibinfo {author} {\bibfnamefont
  {Jorge~G.}\ \bibnamefont {Hirsch}},\ }\bibfield  {title} {\enquote {\bibinfo
  {title} {Quantum and classical {L}yapunov exponents in atom-field interaction
  systems},}\ }\href {\doibase 10.1103/PhysRevLett.122.024101} {\bibfield
  {journal} {\bibinfo  {journal} {Phys. Rev. Lett.}\ }\textbf {\bibinfo
  {volume} {122}},\ \bibinfo {pages} {024101} (\bibinfo {year}
  {2019})}\BibitemShut {NoStop}%
\bibitem [{\citenamefont {Akutagawa}\ \emph {et~al.}(2020)\citenamefont
  {Akutagawa}, \citenamefont {Hashimoto}, \citenamefont {Sasaki},\ and\
  \citenamefont {Watanabe}}]{hashimoto2020cho}%
  \BibitemOpen
  \bibfield  {author} {\bibinfo {author} {\bibfnamefont {Tetsuya}\ \bibnamefont
  {Akutagawa}}, \bibinfo {author} {\bibfnamefont {Koji}\ \bibnamefont
  {Hashimoto}}, \bibinfo {author} {\bibfnamefont {Toshiaki}\ \bibnamefont
  {Sasaki}}, \ and\ \bibinfo {author} {\bibfnamefont {Ryota}\ \bibnamefont
  {Watanabe}},\ }\bibfield  {title} {\enquote {\bibinfo {title}
  {Out-of-time-order correlator in coupled harmonic oscillators},}\ }\href
  {https://doi.org/10.1007/JHEP08(2020)013} {\bibfield  {journal} {\bibinfo
  {journal} {JHEP}\ }\textbf {\bibinfo {volume} {08}},\ \bibinfo {pages} {013}
  (\bibinfo {year} {2020})}\BibitemShut {NoStop}%
\bibitem [{\citenamefont {Lakshminarayan}(2019)}]{PhysRevE.99.012201}%
  \BibitemOpen
  \bibfield  {author} {\bibinfo {author} {\bibfnamefont {Arul}\ \bibnamefont
  {Lakshminarayan}},\ }\bibfield  {title} {\enquote {\bibinfo {title}
  {Out-of-time-ordered correlator in the quantum bakers map and truncated
  unitary matrices},}\ }\href {\doibase 10.1103/PhysRevE.99.012201} {\bibfield
  {journal} {\bibinfo  {journal} {Phys. Rev. E}\ }\textbf {\bibinfo {volume}
  {99}},\ \bibinfo {pages} {012201} (\bibinfo {year} {2019})}\BibitemShut
  {NoStop}%
\bibitem [{\citenamefont {Polchinski}(2015)}]{polchinski2015}%
  \BibitemOpen
  \bibfield  {author} {\bibinfo {author} {\bibfnamefont {Joseph}\ \bibnamefont
  {Polchinski}},\ }\bibfield  {title} {\enquote {\bibinfo {title} {Chaos in the
  black hole {S}-matrix},}\ }\href {\doibase
  https://doi.org/10.48550/arXiv.1505.08108} {\bibfield  {journal} {\bibinfo
  {journal} {arXiv:1505.08108}\ } (\bibinfo {year} {2015}),\
  https://doi.org/10.48550/arXiv.1505.08108}\BibitemShut {NoStop}%
\bibitem [{\citenamefont {Pollicott}(1985)}]{pollicott1985rate}%
  \BibitemOpen
  \bibfield  {author} {\bibinfo {author} {\bibfnamefont {Mark}\ \bibnamefont
  {Pollicott}},\ }\bibfield  {title} {\enquote {\bibinfo {title} {On the rate
  of mixing of {A}xiom {A} flows},}\ }\href
  {https://link.springer.com/article/10.1007/BF01388579#citeas} {\bibfield
  {journal} {\bibinfo  {journal} {Inventiones mathematicae}\ }\textbf {\bibinfo
  {volume} {81}},\ \bibinfo {pages} {413--426} (\bibinfo {year}
  {1985})}\BibitemShut {NoStop}%
\bibitem [{\citenamefont {Ruelle}(1986)}]{ruelle1986}%
  \BibitemOpen
  \bibfield  {author} {\bibinfo {author} {\bibfnamefont {David}\ \bibnamefont
  {Ruelle}},\ }\bibfield  {title} {\enquote {\bibinfo {title} {Resonances of
  chaotic dynamical systems},}\ }\href@noop {} {\bibfield  {journal} {\bibinfo
  {journal} {Phys. Rev. Lett.}\ }\textbf {\bibinfo {volume} {56}},\ \bibinfo
  {pages} {405} (\bibinfo {year} {1986})}\BibitemShut {NoStop}%
\bibitem [{\citenamefont {Ruelle}(1987)}]{ruelle1987resonances}%
  \BibitemOpen
  \bibfield  {author} {\bibinfo {author} {\bibfnamefont {David}\ \bibnamefont
  {Ruelle}},\ }\bibfield  {title} {\enquote {\bibinfo {title} {Resonances for
  {A}xiom {A} flows},}\ }\href
  {https://projecteuclid.org/journals/journal-of-differential-geometry/volume-25/issue-1/Resonances-for-Axiom-bf-A-flows/10.4310/jdg/1214440726.full}
  {\bibfield  {journal} {\bibinfo  {journal} {J. Differential Geom}\ }\textbf
  {\bibinfo {volume} {25}},\ \bibinfo {pages} {99--116} (\bibinfo {year}
  {1987})}\BibitemShut {NoStop}%
\bibitem [{\citenamefont {Cvitanovic}\ \emph {et~al.}(2005)\citenamefont
  {Cvitanovic}, \citenamefont {Artuso}, \citenamefont {Mainieri}, \citenamefont
  {Tanner}, \citenamefont {Vattay}, \citenamefont {Whelan},\ and\ \citenamefont
  {Wirzba}}]{cvitanovic2005chaos}%
  \BibitemOpen
  \bibfield  {author} {\bibinfo {author} {\bibfnamefont {Predrag}\ \bibnamefont
  {Cvitanovic}}, \bibinfo {author} {\bibfnamefont {Roberto}\ \bibnamefont
  {Artuso}}, \bibinfo {author} {\bibfnamefont {Ronnie}\ \bibnamefont
  {Mainieri}}, \bibinfo {author} {\bibfnamefont {Gregor}\ \bibnamefont
  {Tanner}}, \bibinfo {author} {\bibfnamefont {G{\'a}bor}\ \bibnamefont
  {Vattay}}, \bibinfo {author} {\bibfnamefont {Niall}\ \bibnamefont {Whelan}},
  \ and\ \bibinfo {author} {\bibfnamefont {Andreas}\ \bibnamefont {Wirzba}},\
  }\bibfield  {title} {\enquote {\bibinfo {title} {Chaos: classical and
  quantum},}\ }\href@noop {} {\bibfield  {journal} {\bibinfo  {journal}
  {ChaosBook. org (Niels Bohr Institute, Copenhagen 2005)}\ }\textbf {\bibinfo
  {volume} {69}},\ \bibinfo {pages} {25} (\bibinfo {year} {2005})}\BibitemShut
  {NoStop}%
\bibitem [{\citenamefont {Reed}\ and\ \citenamefont
  {Simon}(1980)}]{reed1980methods}%
  \BibitemOpen
  \bibfield  {author} {\bibinfo {author} {\bibfnamefont {Michael}\ \bibnamefont
  {Reed}}\ and\ \bibinfo {author} {\bibfnamefont {Barry}\ \bibnamefont
  {Simon}},\ }\href@noop {} {\emph {\bibinfo {title} {Methods of Modern
  Mathematical Physics: Functional Analysis; Rev. ed}}}\ (\bibinfo  {publisher}
  {Academic press},\ \bibinfo {year} {1980})\BibitemShut {NoStop}%
\bibitem [{\citenamefont {Khodas}\ and\ \citenamefont
  {Fishman}(2000)}]{khodas2000relaxation}%
  \BibitemOpen
  \bibfield  {author} {\bibinfo {author} {\bibfnamefont {Maxim}\ \bibnamefont
  {Khodas}}\ and\ \bibinfo {author} {\bibfnamefont {Shmuel}\ \bibnamefont
  {Fishman}},\ }\bibfield  {title} {\enquote {\bibinfo {title} {Relaxation and
  diffusion for the kicked rotor},}\ }\href@noop {} {\bibfield  {journal}
  {\bibinfo  {journal} {Phys. Rev. Lett.}\ }\textbf {\bibinfo {volume} {84}},\
  \bibinfo {pages} {2837} (\bibinfo {year} {2000})}\BibitemShut {NoStop}%
\bibitem [{\citenamefont {Chirikov}\ and\ \citenamefont
  {Shepelyansky}(2008{\natexlab{a}})}]{DimaScholar}%
  \BibitemOpen
  \bibfield  {author} {\bibinfo {author} {\bibfnamefont {B.}~\bibnamefont
  {Chirikov}}\ and\ \bibinfo {author} {\bibfnamefont {D.~L.}\ \bibnamefont
  {Shepelyansky}},\ }\bibfield  {title} {\enquote {\bibinfo {title} {Chirikov
  standard map},}\ }\href
  {http://www.scholarpedia.org/article/Chirikov_standard_map} {\bibfield
  {journal} {\bibinfo  {journal} {Scholarpedia}\ }\textbf {\bibinfo {volume}
  {3}},\ \bibinfo {pages} {3350} (\bibinfo {year}
  {2008}{\natexlab{a}})}\BibitemShut {NoStop}%
\bibitem [{\citenamefont {Schwinger}(1960)}]{schwinger1960unitary}%
  \BibitemOpen
  \bibfield  {author} {\bibinfo {author} {\bibfnamefont {Julian}\ \bibnamefont
  {Schwinger}},\ }\bibfield  {title} {\enquote {\bibinfo {title} {Unitary
  operator bases},}\ }\href@noop {} {\bibfield  {journal} {\bibinfo  {journal}
  {Proceedings of the National Academy of Sciences}\ }\textbf {\bibinfo
  {volume} {46}},\ \bibinfo {pages} {570--579} (\bibinfo {year}
  {1960})}\BibitemShut {NoStop}%
\bibitem [{\citenamefont {Chirikov}\ and\ \citenamefont
  {Shepelyansky}(2008{\natexlab{b}})}]{Chirikov:2008}%
  \BibitemOpen
  \bibfield  {author} {\bibinfo {author} {\bibfnamefont {B.}~\bibnamefont
  {Chirikov}}\ and\ \bibinfo {author} {\bibfnamefont {D.}~\bibnamefont
  {Shepelyansky}},\ }\bibfield  {title} {\enquote {\bibinfo {title} {{C}hirikov
  standard map},}\ }\href {\doibase 10.4249/scholarpedia.3550} {\bibfield
  {journal} {\bibinfo  {journal} {Scholarpedia}\ }\textbf {\bibinfo {volume}
  {3}},\ \bibinfo {pages} {3550} (\bibinfo {year} {2008}{\natexlab{b}})},\
  \bibinfo {note} {revision \#197507}\BibitemShut {NoStop}%
\bibitem [{\citenamefont {Lieberman}\ and\ \citenamefont
  {Lichtenberg}(1983)}]{liebermann1983irregular}%
  \BibitemOpen
  \bibfield  {author} {\bibinfo {author} {\bibfnamefont {AJ}~\bibnamefont
  {Lieberman}}\ and\ \bibinfo {author} {\bibfnamefont {MA}~\bibnamefont
  {Lichtenberg}},\ }\href@noop {} {\enquote {\bibinfo {title} {Irregular and
  stochastic motion, applied mathematical series, vol. 38},}\ } (\bibinfo
  {year} {1983})\BibitemShut {NoStop}%
\bibitem [{\citenamefont {Greene}(1979)}]{greene1979method}%
  \BibitemOpen
  \bibfield  {author} {\bibinfo {author} {\bibfnamefont {John~M}\ \bibnamefont
  {Greene}},\ }\bibfield  {title} {\enquote {\bibinfo {title} {A method for
  determining a stochastic transition},}\ }\href@noop {} {\bibfield  {journal}
  {\bibinfo  {journal} {J. Math. Phys.}\ }\textbf {\bibinfo {volume} {20}},\
  \bibinfo {pages} {1183--1201} (\bibinfo {year} {1979})}\BibitemShut {NoStop}%
\bibitem [{\citenamefont {Ketzmerick}\ \emph {et~al.}(1999)\citenamefont
  {Ketzmerick}, \citenamefont {Kruse},\ and\ \citenamefont
  {Geisel}}]{ketzmerick1999efficient}%
  \BibitemOpen
  \bibfield  {author} {\bibinfo {author} {\bibfnamefont {R}~\bibnamefont
  {Ketzmerick}}, \bibinfo {author} {\bibfnamefont {K}~\bibnamefont {Kruse}}, \
  and\ \bibinfo {author} {\bibfnamefont {T}~\bibnamefont {Geisel}},\ }\bibfield
   {title} {\enquote {\bibinfo {title} {Efficient diagonalization of kicked
  quantum systems},}\ }\href@noop {} {\bibfield  {journal} {\bibinfo  {journal}
  {Physica D: Nonlinear Phenomena}\ }\textbf {\bibinfo {volume} {131}},\
  \bibinfo {pages} {247--253} (\bibinfo {year} {1999})}\BibitemShut {NoStop}%
\bibitem [{\citenamefont {Blum}\ and\ \citenamefont
  {Agam}(2000)}]{blum2000leading}%
  \BibitemOpen
  \bibfield  {author} {\bibinfo {author} {\bibfnamefont {Galya}\ \bibnamefont
  {Blum}}\ and\ \bibinfo {author} {\bibfnamefont {Oded}\ \bibnamefont {Agam}},\
  }\bibfield  {title} {\enquote {\bibinfo {title} {Leading ruelle resonances of
  chaotic maps},}\ }\href
  {https://journals.aps.org/pre/abstract/10.1103/PhysRevE.62.1977} {\bibfield
  {journal} {\bibinfo  {journal} {Phys. Rev. E}\ }\textbf {\bibinfo {volume}
  {62}},\ \bibinfo {pages} {1977} (\bibinfo {year} {2000})}\BibitemShut
  {NoStop}%
\bibitem [{\citenamefont {Garc{\'\i}a-Mata}\ \emph {et~al.}(2003)\citenamefont
  {Garc{\'\i}a-Mata}, \citenamefont {Saraceno},\ and\ \citenamefont
  {Spina}}]{garcia2003classical}%
  \BibitemOpen
  \bibfield  {author} {\bibinfo {author} {\bibfnamefont {Ignacio}\ \bibnamefont
  {Garc{\'\i}a-Mata}}, \bibinfo {author} {\bibfnamefont {Marcos}\ \bibnamefont
  {Saraceno}}, \ and\ \bibinfo {author} {\bibfnamefont {Mar{\'\i}a~Elena}\
  \bibnamefont {Spina}},\ }\bibfield  {title} {\enquote {\bibinfo {title}
  {Classical decays in decoherent quantum maps},}\ }\href@noop {} {\bibfield
  {journal} {\bibinfo  {journal} {Phys. Rev. Lett.}\ }\textbf {\bibinfo
  {volume} {91}},\ \bibinfo {pages} {064101} (\bibinfo {year}
  {2003})}\BibitemShut {NoStop}%
\bibitem [{\citenamefont {Ulam}(1960)}]{ulam1960collection}%
  \BibitemOpen
  \bibfield  {author} {\bibinfo {author} {\bibfnamefont {Stanislaw~M}\
  \bibnamefont {Ulam}},\ }\href@noop {} {\emph {\bibinfo {title} {A collection
  of mathematical problems}}},\ \bibinfo {number} {8}\ (\bibinfo  {publisher}
  {Interscience Publishers},\ \bibinfo {year} {1960})\BibitemShut {NoStop}%
\bibitem [{\citenamefont {Meyer}(2000)}]{meyer2000perron}%
  \BibitemOpen
  \bibfield  {author} {\bibinfo {author} {\bibfnamefont {Carl}\ \bibnamefont
  {Meyer}},\ }\bibfield  {title} {\enquote {\bibinfo {title} {Perron--frobenius
  theory of nonnegative matrices},}\ }\href@noop {} {\bibfield  {journal}
  {\bibinfo  {journal} {Matrix analysis and applied linear algebra, SIAM}\ }
  (\bibinfo {year} {2000})}\BibitemShut {NoStop}%
\bibitem [{\citenamefont {Pillai}\ \emph {et~al.}(2005)\citenamefont {Pillai},
  \citenamefont {Suel},\ and\ \citenamefont {Cha}}]{pillai2005perron}%
  \BibitemOpen
  \bibfield  {author} {\bibinfo {author} {\bibfnamefont {S~Unnikrishna}\
  \bibnamefont {Pillai}}, \bibinfo {author} {\bibfnamefont {Torsten}\
  \bibnamefont {Suel}}, \ and\ \bibinfo {author} {\bibfnamefont {Seunghun}\
  \bibnamefont {Cha}},\ }\bibfield  {title} {\enquote {\bibinfo {title} {The
  perron-frobenius theorem: some of its applications},}\ }\href@noop {}
  {\bibfield  {journal} {\bibinfo  {journal} {IEEE Signal Processing Magazine}\
  }\textbf {\bibinfo {volume} {22}},\ \bibinfo {pages} {62--75} (\bibinfo
  {year} {2005})}\BibitemShut {NoStop}%
\bibitem [{\citenamefont {Frahm}\ and\ \citenamefont
  {Shepelyansky}(2010)}]{frahm2010ulam}%
  \BibitemOpen
  \bibfield  {author} {\bibinfo {author} {\bibfnamefont {Klaus~M}\ \bibnamefont
  {Frahm}}\ and\ \bibinfo {author} {\bibfnamefont {Dima~L}\ \bibnamefont
  {Shepelyansky}},\ }\bibfield  {title} {\enquote {\bibinfo {title} {Ulam
  method for the chirikov standard map},}\ }\href@noop {} {\bibfield  {journal}
  {\bibinfo  {journal} {Eur. Phys. J. B}\ }\textbf {\bibinfo {volume} {76}},\
  \bibinfo {pages} {57--68} (\bibinfo {year} {2010})}\BibitemShut {NoStop}%
\bibitem [{\citenamefont {Prosen}(2004)}]{prosen2004ruelle}%
  \BibitemOpen
  \bibfield  {author} {\bibinfo {author} {\bibfnamefont {Toma{\v{z}}}\
  \bibnamefont {Prosen}},\ }\bibfield  {title} {\enquote {\bibinfo {title}
  {Ruelle resonances in kicked quantum spin chain},}\ }\href@noop {} {\bibfield
   {journal} {\bibinfo  {journal} {Physica D: Nonlinear Phenomena}\ }\textbf
  {\bibinfo {volume} {187}},\ \bibinfo {pages} {244--252} (\bibinfo {year}
  {2004})}\BibitemShut {NoStop}%
\end{thebibliography}%
%%%
\end{document}
