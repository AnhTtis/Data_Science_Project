\section{Dynamic-Threshold algorithms for $K$ groups}





\subsection{Proof of Lemma \ref{lem:single-thres-recursion}}

\begin{proof}
We consider that $B \geq 1$. The case $B=0$ is treated separately at the end of the proof.
Let $\C_N$ the event $(\forall k \in [K], \forall t \geq 1: ||G^k_t| - \lambda_k t| \leq 4 \sqrt{t \log N})$.
It holds that
\begin{align}
\Pr(\A^B_T \text{ succeeds} \mid \C_N)
&= \sum_{t=T}^N \sum_{k=1}^K \sum_{\ell=1}^K \Pr(\A_T^B \text{ succeeds}, \rho_1=t, g_t = \ell, g^*_{T-1} = k \mid \C_N) \nonumber\\
&= \sum_{t=T}^N \sum_{k=1}^K \sum_{\ell=1}^K \Pr(\A_T^B \text{ succeeds}, R_t = 1, \rho_1=t, g_t = \ell, g^*_{T-1} = k \mid \C_N)\nonumber\\
&\quad + \sum_{t=T}^N \sum_{k=1}^K \sum_{\ell=1}^K \Pr(\A_T^B \text{ succeeds}, R_t \neq 1, \rho_1=t, g_t = \ell, g^*_{T-1} = k \mid \C_N)\;. \label{aligneq:Rt=neq1}
\end{align}
In the following, we will estimate the terms in both sums. We recall that we consider $K$ and $(\lambda_k)_{k\in [K]}$ to be constants, and $T = \w N + o(1)$, with $\alpha$ also a constant. All the $O$ terms appearing in the proof are independent of $t$, as $T \leq t \leq N$, they only depend on $N$, $T/N = \w + o(1)$, and the constant parameters.

For $t \in \{T,\ldots,N\}$, if $\rho_1 = t$ and $R_t=1$, then the Algorithms stops on $x_t$, hence its succeeds if and only if $x_t = \xmax$. The event $x_t = \xmax$ is independent of the group membership of the candidates, thus independent of $\C_N$, and its probability is $1/N$. The event $g_t = \ell$, however, is not independent of $\C_N$, but Lemma \ref{lem:pr-cond-C_N} gives that $\Pr(g_t = \ell \mid \C_N) = \Pr(g_t = \ell) + O(1/N^2) = \lambda_\ell + O(1/N^2)$. Therefore, it holds for all $k,\ell \in [K]$ and $t \in \{T,\ldots,N\}$ that
\begin{align*}
\Pr(\A_T^B \text{ succeeds}, R_t = 1, &\rho_1=t, g_t = \ell, g^*_{T-1} = k \mid \C_N)\\
&= \Pr(x_t = \xmax, \rho_1\geq t,g_t = \ell, g^*_{T-1} = k \mid \C_N)\\
&= \Pr(x_t = \xmax)\Pr(g_t = \ell \mid \C_N) \Pr(\rho_1\geq t, g^*_{T-1} = k \mid \C_N)\\
&= \left(\frac{\lambda_\ell}{N} + O(1/N^3)\right) \Pr(\rho_1\geq t, g^*_{T-1} = k \mid \C_N)\;.
\end{align*}
Note that, for the single-threshold algorithm, we have the equivalence $\rho_1 = t \iff \rho_1 \geq t \text{ and } r_t = 1$.
The event $\rho_1 \geq t$ happens if and only if no candidate $x_s$ for $s \in \{T,\ldots,t-1\}$ in any group $m \in [K]$ exceeds the best candidate seen up to time $T-1$ in the same group:
\[
\forall t \geq T: (\rho_1 \geq t)
\iff (\forall m \in [K]: \max G^m_{T:t-1} < \max G^m_{T-1})\;,
\]
with the convention $\max \emptyset = - \infty$. Consequently, if $\rho_1 \geq t$, then $g^*_{T-1} = k$ means that the best candidate in all groups until time $t-1$ belongs to group $G^k_{T-1}$. Using that $T = \Theta(N)$, this yields
\begin{align}
\Pr(\A_T^B& \text{ succeeds}, R_t = 1, \rho_1=t, g_t = \ell, g^*_{T-1} = k \mid \C_N) \nonumber\\
&= \left(\frac{\lambda_\ell}{N} + O(1/N^3)\right) \Pr(\rho_1 \geq t \text{ and } \max x_{1:t-1} \in G^k_{T-1} \mid C_N)\nonumber\\
&= \left(\frac{\lambda_\ell}{N} + O(1/N^3)\right) \Pr(\max x_{1:t-1} \in G^k_{T-1} \mid C_N) \Pr(\rho_1 \geq t \mid \max x_{1:t-1} \in G^k_{T-1}, C_N)\nonumber\\
&= \left(\frac{\lambda_\ell}{N} + O(1/N^3)\right) \left( \frac{\lambda_k(T-1)}{t-1} + O(1/N^2) \right) \Pr(\rho_1 \geq t \mid \max x_{1:t-1} \in G^k_{T-1}, C_N)\nonumber\\
&= \left(\frac{\lambda_\ell \lambda_k T}{N t} + O(1/N^3)\right) \Pr(\forall m \in [K]\setminus\{k\}: \max G^m_{T:t-1} < \max G^m_{T-1} \mid C_N)\nonumber\\
&= \left(\frac{\lambda_\ell \lambda_k T}{N t} + O(1/N^3)\right) \prod_{m \neq k} \E\left[\tfrac{|G^k_{T-1}|}{|G^k_{t-1}|} \; \Big| \; \C_N \right]\nonumber\\
&= \left(\frac{\lambda_\ell \lambda_k T}{N t} + O(1/N^3)\right) \prod_{m \neq k} \E\left[ \frac{\lambda_m T + O(\sqrt{N\log N})}{\lambda_m t + O(\sqrt{N\log N})} \; \Big| \; \C_N \right]\nonumber\\
&= \left(\frac{\lambda_\ell \lambda_k T}{N t} + O(1/N^3)\right) \prod_{m \neq k} \left(\frac{T}{t} + O\Big( \sqrt{\tfrac{\log N}{N}}\Big) \right) \nonumber\\
&= \left(\frac{\lambda_\ell \lambda_k T}{N t} + O(1/N^3)\right) \left(\frac{T^{K-1}}{t^{K-1}} + O\Big( \sqrt{\tfrac{\log N}{N}}\Big) \right)\nonumber \\
&= \frac{\lambda_\ell \lambda_k}{N} (T/t)^K +  O\Big( \sqrt{\tfrac{\log N}{N^3}}\Big) \;. \label{aligneq:single-thresh-Rt=1}
\end{align}
On the other hand, regarding the terms of the second sum in \eqref{aligneq:Rt=neq1}, if $\rho_1 = t$ but $R_t \neq 1$, the Algorithm uses a comparison to observe $R_t$ but then skips to the next step $t+1$. The budget at step $t+1$ is thus $B-1$ and the group of the best candidate seen so far remains unchanged. Given that the single threshold algorithm is memory-less, its state at time $t+1$ is fully determined by $B-1, g^*_t$ and the number of candidates seen in each group so far, which is controlled by $\C_N$. We deduce that
\begin{align}
\Pr(\A_T^B& \text{ succeeds}, R_t \neq 1, \rho_1=t, g_t = \ell, g^*_{T-1} = k \mid \C_N) \nonumber\\  
&=\Pr(\A_T^B \text{ succeeds} \mid R_t \neq 1, \rho_1=t, g_t = \ell, g^*_{T-1} = k, \C_N)  \Pr(R_t \neq 1, \rho_1=t, g_t = \ell, g^*_{T-1} = k \mid \C_N) \nonumber\\  
&=\Pr(\A^{B-1}_{t+1} \text{ succeeds} \mid g^*_{t} = k, \C_N) \Pr(R_t \neq 1, \rho_1=t, g_t = \ell, g^*_{T-1} = k \mid \C_N)\;, \label{aligneq:single-thresh-Rneq-first}
\end{align}
where $\Pr(\A^{B-1}_{N+1} \text{ succeeds}) = 0$.
For $\ell = k$, the probability $\Pr(R_t \neq 1, \rho_1=t, g_t = \ell, g^*_{T-1} = k \mid \C_N)$ is zero, because if $g^*_{T-1} = k$, $\rho_1 \geq t$ and $g_t = \ell$, then the best candidate up to step $T-1$ belongs to group $G^k$, and no candidate $x_s$ for $s \in \{T,\ldots,t-1\}$ is better than the maximum in its group seen before step $T-1$, thus if $x_t$ belongs to $G^k$ and $r_t = 1$ then necessarily $R_t = 1$.

For $\ell \neq k$, it holds that
\begin{align*}
\Pr(R_t \neq 1,& \rho_1=t, g_t = \ell, g^*_{T-1} = k \mid \C_N)\\
&= \Pr( \rho_1=t, g_t = \ell, \max x_{1:t} \in G^k_{T-1} \mid \C_N)\\
&= \Pr(\max x_{1:t} \in G^k_{T-1} \mid \C_N) \Pr( \rho_1=t, g_t = \ell \mid \max x_{1:t} \in G^k_{T-1}, \C_N)\\
&= \left(\frac{\lambda_k(T-1)}{t-1} + O(1/N^2) \right) \Pr( \rho_1=t, g_t = \ell \mid \max x_{1:t} \in G^k_{T-1}, \C_N)\\
&= \left(\frac{\lambda_k T}{t} + O(1/N^2) \right) \Pr(g_t = \ell \mid \C_N) 
\Pr( \max G^\ell_{T:t-1} < \max G^\ell_{T-1} < x_t \\
&\hspace{3cm} \text{ and } \forall m \in [K]\setminus\{k,\ell\} : \max G^m_{T:t-1} < \max G^m_{T-1}  \mid \C_N)\\
&= \left(\frac{\lambda_k T}{t} + O(1/N^2) \right)(\lambda_\ell + O(1/N^2)) \E\left[ \tfrac{1}{|G^\ell_{t-1}|+1}\cdot\tfrac{|G^\ell_{T-1}|}{|G^\ell_{t-1}|} \; \Big| \; \C_N \right] \prod_{m \notin \{k,\ell\}} \E\left[ \tfrac{|G^m_{t-1}|}{|G^m_{T-1}|} \; \Big| \; \C_N \right]\\
&= \left(\frac{\lambda_\ell \lambda_k T}{t} + O(1/N^2) \right) \E\left[ \tfrac{1}{|G^\ell_{t-1}|+1}\cdot\tfrac{|G^\ell_{T-1}|}{|G^\ell_{t-1}|} \; \Big| \; \C_N \right] \prod_{m \notin \{k,\ell\}} \E\left[ \tfrac{|G^m_{t-1}|}{|G^m_{T-1}|} \; \Big| \; \C_N \right]\\
&= \left(\frac{\lambda_\ell \lambda_k T}{t} + O(1/N^2) \right) \left( \frac{T}{\lambda_\ell t^2} + O\Big( \sqrt{\tfrac{\log N}{N^3}}\Big) \right) \prod_{m \notin \{k,\ell\}} \left(\frac{T}{t} + O\Big( \sqrt{\tfrac{\log N}{N}}\Big)\right)\\
&= \left(\frac{\lambda_\ell \lambda_k T}{t} + O(1/N^2) \right) \left( \frac{T}{\lambda_\ell t^2} + O\Big( \sqrt{\tfrac{\log N}{N^3}}\Big) \right) \left(\frac{T^{K-2}}{t^{K-2}} + O\Big( \sqrt{\tfrac{\log N}{N}}\Big)\right)\\
&= \frac{\lambda_k T^K}{t^{K+1}} + O\Big( \sqrt{\tfrac{\log N}{N^3}}\Big)\;,
\end{align*}
where we used in the last equalities the event $\C_N$ and the assumption $T = \Theta(N)$.
Therefore, substituting into \eqref{aligneq:single-thresh-Rneq-first} gives for all $\ell, k \in [K]$ and $t \in \{T,\ldots,N\}$ that
\begin{align}
\Pr(\A_T^B& \text{ succeeds}, R_t \neq 1, \rho_1=t, g_t = \ell, g^*_{T-1} = k \mid \C_N) \nonumber \\  
&= \left( \frac{\lambda_k T^K}{t^{K+1}} + O\Big( \sqrt{\tfrac{\log N}{N^3}}\Big) \right) \indic{k \neq \ell} \Pr(\A^{B-1}_{t+1} \text{ succeeds} \mid g^*_{t} = k, \C_N)\nonumber \\
&= \left( \frac{\lambda_k T^K}{t^{K+1}} + O\Big( \sqrt{\tfrac{\log N}{N^3}}\Big) \right) \indic{k \neq \ell} \frac{\Pr(\A^{B-1}_{t+1} \text{ succeeds}, g^*_{t} = k\mid \C_N)}{\Pr(g^*_{t} = k \mid \C_N)}\nonumber\\
&= \left( \frac{\lambda_k T^K}{t^{K+1}} + O\Big( \sqrt{\tfrac{\log N}{N^3}}\Big) \right) \indic{k \neq \ell} \frac{\Pr(\A^{B-1}_{t+1} \text{ succeeds}, g^*_{t} = k\mid \C_N)}{\lambda_k + O(1/N^2)}\nonumber\\
&= \left( \frac{T^K}{t^{K+1}} + O\Big( \sqrt{\tfrac{\log N}{N^3}}\Big) \right) \indic{k \neq \ell} \Pr(\A^{B-1}_{t+1} \text{ succeeds}, g^*_{t} = k\mid \C_N) \;.\label{aligneq:single-thresh-Rtneq1}
\end{align}
Finally, substituting \eqref{aligneq:single-thresh-Rt=1} and \eqref{aligneq:single-thresh-Rtneq1} into \eqref{aligneq:Rt=neq1}, and recalling that all the previous $O$ terms are independent of $t$, gives that
\begin{align*}
\Pr(\A^B_T \text{ succeeds} \mid \C_N)
&= \sum_{t=T}^N \sum_{k=1}^K \sum_{\ell=1}^K \left(\frac{\lambda_\ell \lambda_k}{N} (T/t)^K +  O\Big( \sqrt{\tfrac{\log N}{N^3}}\Big) \right)\\
&\quad + \sum_{t=T}^N \sum_{k=1}^K \sum_{\ell \neq k}  \left( \frac{T^K}{t^{K+1}} + O\Big( \sqrt{\tfrac{\log N}{N^3}}\Big) \right) \Pr(\A^{B-1}_{t+1} \text{ succeeds}, g^*_{t} = k\mid \C_N)\\
&= \left(1 + O\Big( \sqrt{\tfrac{\log N}{N}}\Big)\right) \Big( \sum_{t=T}^N \sum_{k=1}^K \sum_{\ell=1}^K \frac{\lambda_\ell \lambda_k}{N} (T/t)^K\\
&\hspace{2.7cm} + \sum_{t=T}^N \sum_{k=1}^K \sum_{\ell \neq k}  \frac{T^K}{t^{K+1}} \Pr(\A^{B-1}_{t+1} \text{ succeeds}, g^*_{t} = k\mid \C_N) \Big)\\
&= \left(1 + O\Big( \sqrt{\tfrac{\log N}{N}}\Big)\right) \Big( \frac{T^K}{N}\sum_{t=T}^N \frac{1}{t^K}
+ (K-1) \sum_{t=T}^N \frac{T^K}{t^{K+1}}\Pr(\A^{B-1}_{t+1} \text{ succeeds}\mid \C_N) \Big)\;.
\end{align*}
Using Riemann sum properties, we obtain 
\[
\frac{T^K}{N}\sum_{t=T}^N \frac{1}{t^K}
= \frac{(T/N)^K}{N}\sum_{t=T}^N \frac{1}{(t/N)^K}
= \w^K \int_{\w}^1 \frac{du}{u^K} + O(1/N)
= \frac{\w - \w^{K}}{K-1} + O(1/N)\;,
\]
and by Lemma \ref{lem:pr-cond-C_N} we have for all $t \in \{T,\ldots,N\}$ and $b \geq 0$ that 
\[
\Pr(\A^{b}_{t} \text{ succeeds}\mid \C_N) = \Pr(\A^{b}_{t} \text{ succeeds}) + O(1/N^2)\;,
\]
with the $O(1/N^2)$ independent of $t$. Observing that $\sum_{t=T}^N \frac{T^K}{t^{K+1}} = \frac{1-\w^K}{K} + o(1) = O(1)$, it follows that
\begin{align*}
\Pr(\A^{B}_{T} \text{ succeeds})
&=  \left(1 + O\Big( \sqrt{\tfrac{\log N}{N}}\Big)\right) \Big(
\frac{\w - \w^{K}}{K-1} + (K-1) \sum_{t=T}^N \frac{T^K}{t^{K+1}}\Pr(\A^{B-1}_{t+1} \text{ succeeds}) + O(\tfrac{1}{N}) \Big)\\
&= \frac{\w - \w^{K}}{K-1} + (K-1) \sum_{t=T}^N \frac{T^K}{t^{K+1}}\Pr(\A^{B-1}_{t+1} \text{ succeeds}) + O\Big( \sqrt{\tfrac{\log N}{N}}\Big)\;.
\end{align*}
This concludes the proof for $B \geq 1$.

For $B = 0$, $\Pr(\A^{0}_{t+1} \text{ succeeds} \mid \C_N)$ can be decomposed as in \eqref{aligneq:Rt=neq1}. However, the terms of the second sum are all zero, because if $\rho_1 = t$ then the algorithm stops at $t$, but since $R_t \neq 1$, the selected candidate is not the best one, and thus the succeeding probability is $0$. All the computations regarding the first sum stay the same, and we obtain
\[
\Pr(\A^{0}_{T} \text{ succeeds}) = \frac{\w - \w^{K}}{K-1} + O\Big( \sqrt{\tfrac{\log N}{N}}\Big)\;.
\]
\end{proof}











\subsection{Proof of Theorem \ref{thm:single-thresh}}


\begin{proof}
Let $\w \in (0,1]$ a constant. For all $w\in [\w,1]$ and $B \geq 0$, we denote by $\phi^B(w)$ the limit $\lim_{N \to \infty} \Pr(\A^B_t \text{ succeeds})$ for $t = \lfloor wN \rfloor$. We will prove by induction over $B$ that this limit exists for all $w\in [\w,1]$, is equal to the expression stated in the theorem, with $u$ instead of $\w$, and satisfies $\Pr(\A^B_t \text{ succeeds}) = \phi^B(w) + O\big( \sqrt{\tfrac{\log N}{N}}\big)$, with the $O$ term only depending on $\w$ and the other constants of the problem. In particular, the $O$ term is independent of $t$.
For $B = 0$, Lemma \ref{lem:single-thres-recursion} gives immediately for any $w\in [\w,1]$ and $t = \lfloor wN \rfloor$ that
\[
\Pr(\A^0_t \text{ succeeds})
= \frac{w- w^{K}}{K-1} + O\Big( \sqrt{\tfrac{\log N}{N}}\Big) 
= \frac{w^K}{K-1}\left( \frac{1}{w^{K-1}} - 1\right) + O\Big( \sqrt{\tfrac{\log N}{N}}\Big)\;.   
\]


The $O$ term depends on $t$, but using the inequalities $\w + o(1) \leq t/N \leq 1$, it can be made only dependent on $\w$.
Let $B \geq 1$ and assume the result is true for $B-1$. Lemma \ref{lem:single-thres-recursion} and the induction hypothesis give for all $w\in [\w,1]$ and $t = \lfloor wN \rfloor$, that
\begin{align*}
\Pr(\A^{B}_{t} \text{ succeeds})
&= \frac{w- w^{K}}{K-1} + (K-1) \sum_{s=t}^N \frac{t^K}{s^{K+1}}\Pr(\A^{B-1}_{s+1} \text{ succeeds}) + O\Big( \sqrt{\tfrac{\log N}{N}}\Big)\\
&= \frac{w- w^{K}}{K-1} + (K-1) \sum_{s=t}^N \frac{t^K}{s^{K+1}}\left(\phi^{B-1}\big( \tfrac{s+1}{N} \big) + O\Big( \sqrt{\tfrac{\log N}{N}}\Big) \right) + O\Big( \sqrt{\tfrac{\log N}{N}}\Big) \\
&= \frac{w- w^{K}}{K-1} + (K-1) \frac{(t/N)^K}{N} \sum_{s=t}^N \frac{\phi^{B-1}\big( \tfrac{s+1}{N} \big)}{(s/N)^{K+1}} + O\Big( \sqrt{\tfrac{\log N}{N}}\Big)\;,
\end{align*}
where we used that the $O$ term in the induction hypothesis is independent of $s$ and that 
\[
\sum_{s=t}^N \frac{t^K}{s^{K+1}}\phi^{B-1} \big(\tfrac{s+1}{N} \big) 
\leq \sum_{s=T}^N \frac{N^K}{s^{K+1}}
\leq \frac{1}{N} \sum_{s = T}^N \frac{1}{(s/N)^{K+1}} = O(1)\;.
\]
Finally, $t/N = w + O(1/N)$, and $\phi^{B-1}$ is, by the induction hypothesis, a continuously differentiable function on $[\w,1]$, therefore, it holds by convergence properties of Riemann sums that
\begin{align*}
\Pr(\A^{B}_{t} \text{ succeeds})
&= \frac{w- w^{K}}{K-1} + (K-1) (w^K + O(\tfrac{1}{N})) \left(\int_w^1 \frac{\phi^{B-1}(u)}{u^{K+1}}du + O(\tfrac{1}{N})\right) + O\Big( \sqrt{\tfrac{\log N}{N}}\Big)\\
&= \frac{w- w^{K}}{K-1} + (K-1) w^K \int_w^1 \frac{\phi^{B-1}(u)}{u^{K+1}}du  + O\Big( \sqrt{\tfrac{\log N}{N}}\Big)\;,
\end{align*}
where the $O$ term depends on $t$ and the constant parameters. Using that $T = \lfloor \w N \rfloor \leq t \leq N$, the $O$ can be made dependent only on $\w$ and the other constant parameters. The limit $\phi^B(w) = \lim_{N \to \infty} \Pr(\A^B_{\lfloor wN \rfloor} \text{ succeeds})$ therefore exists, and is equal to
\[
\phi^B(w)
= \frac{w- w^{K}}{K-1} + (K-1) w^K \int_w^1 \frac{\phi^{B-1}(u)}{u^{K+1}}du\;.
\]
The induction hypothesis gives for all $u \in [\w, 1]$ that 
\[
\phi^{B-1}(u)
= \frac{u^K}{K-1} \sum_{b = 0}^{B-1} \left( \frac{1}{u^{K-1}} - \sum_{\ell = 0}^b \frac{\log(1/u^{K-1})^\ell}{\ell !} \right)\;,
\]
hence
\begin{align*}
(K-1) w^K \int_w^1 \frac{\phi^{B-1}(u)}{u^{K+1}}du
&= \int_w^1 \sum_{b = 0}^{B-1} \left( \frac{1}{u^{K}} - \sum_{\ell = 0}^b \frac{\log(1/u^{K-1})^\ell}{\ell ! u} \right) du\\
&= B w^K \int_w^1 \frac{du}{u^K} - w^K \sum_{b = 0}^{B-1} \sum_{\ell = 0}^b \frac{1}{\ell !} \int_w^1 \frac{\log(1/u^{K-1})^\ell}{u} du\\
&= B w^K \left[ \frac{-1}{(K-1) u^{K-1}} \right]_w^1 - w^K \sum_{b = 0}^{B-1} \sum_{\ell = 0}^b \frac{1}{\ell !} \left[-\frac{\log(1/u^{K-1})^{\ell+1}}{(K-1)(\ell+1)} \right]_w^1\\
&= \frac{B w^K}{K-1}\left( \frac{1}{w^{K-1}} - 1 \right) - w^K\sum_{b = 0}^{B-1} \sum_{\ell = 0}^b \frac{\log(1/w^{K-1})^{\ell+1}}{(K-1)(\ell+1)!}\\
&= \frac{B w^K}{K-1}\left( \frac{1}{w^{K-1}} - 1 \right) - w^K \sum_{b = 1}^{B} \sum_{\ell = 1}^b \frac{\log(1/w^{K-1})^{\ell}}{(K-1)\ell!}\\
&= \frac{w^K}{K-1} \sum_{b = 1}^{B} \left( \frac{1}{w^{K-1}} - 1 - \sum_{\ell = 1}^b \frac{\log(1/w^{K-1})^{\ell}}{\ell!}\right)\\
&= \frac{w^K}{K-1} \sum_{b = 1}^{B} \left( \frac{1}{w^{K-1}} - \sum_{\ell = 0}^b \frac{\log(1/w^{K-1})^{\ell}}{\ell!}\right)\;,
\end{align*}
and it follows that
\begin{align*}
\phi^B(w) 
&= \frac{w^K}{K-1} \left( \frac{1}{w^{K-1}} - 1 +  \sum_{b = 1}^{B} \left( \frac{1}{w^{K-1}} - \sum_{\ell = 0}^b \frac{\log(1/w^{K-1})^{\ell}}{\ell!}\right) \right)\\
&= \frac{w^K}{K-1} \sum_{b = 0}^{B} \left( \frac{1}{w^{K-1}} - \sum_{\ell = 0}^b \frac{\log(1/w^{K-1})^{\ell}}{\ell!}\right)\;.
\end{align*}
In particular, this identity is true for $w = \w$, which gives the wanted result.
\end{proof}









\subsection{Proof of Corollary \ref{cor:single-thresh-factorial-conv}}

\begin{proof}
Let $\w \in (0,1)$ and $T = \lfloor \w N \rfloor$. Lemma \ref{lem:sum-exp-remainder} gives for all $x>0$ that 
\[
\sum_{b=0}^B\left( e^x - \sum_{\ell=0}^b \frac{x^\ell}{\ell !} \right) \geq xe^x\left(1 -  \frac{x^{B+1}}{(B+1)!}\right)\;,
\]
in particular, we obtain for $x = \log(1/\w^{K-1})$ that
\begin{align*}
\lim_{N \to \infty} \Pr(\A_T^B \text{ succeeds})
&= \frac{\w^K}{K-1} \sum_{b = 0}^{\infty} \left( \frac{1}{\w^{K-1}} - \sum_{\ell = 0}^b \frac{\log(1/\w^{K-1})^{\ell}}{\ell!}\right)\\
&\geq \frac{\w^K}{K-1} \cdot \frac{\log(1/\w^{K-1})}{\w^{K-1}}\left(1 - \frac{\log(1/\w^{K-1})^{B+1}}{(B+1)!} \right)\\
&= \w\log(1/\w)\left(1 - \frac{(K-1)^{B+1}\log(1/\w)^{B+1}}{(B+1)!} \right)\;.
\end{align*}
Taking a threshold $T = \lfloor N/e \rfloor$ gives
\[
\lim_{N \to \infty} \Pr(\A_{\lfloor N/e \rfloor}^B \text{ succeeds})
\geq \frac{1}{e}\left(1 - \frac{(K-1)^{B+1}}{(B+1)!} \right)\;.
\]
\end{proof}





