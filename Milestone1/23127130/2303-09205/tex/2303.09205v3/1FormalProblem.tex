\section{Formal Problem}
\label{sec:setup}

We consider a strictly totally ordered set of cardinal $N$, whose elements will be called \textit{candidates}. We assume that these candidates are observed in a uniformly random arrival order $(x_1, \ldots x_N)$, and that they are partitioned into $K$ groups $G^1, \ldots, G^k$. For all $t\in [N]$, we denote by $g_t \in [K]$ the group of $x_t$, i.e. $x_t \in G^{g_t}$, and we assume that $\{g_t\}_{t\in [N]}$ are mutually independent random variables
\[
\quad \Prob(g_t = k) = \lambda_k, \quad \forall t \in [N], \forall k \in [K]\;,
\]
where $\lambda_k > 0$ for all $k \in [K]$ and $\sum_{k=1}^K \lambda_k = 1$.

We assume that comparing candidates of the same group is free, while comparing two candidates of different groups is costly. To address the latter case, we consider that a budget $B \geq 0$ is given for comparisons and we propose two models: the algorithm can pay a cost of $1$ in order to:
\begin{enumerate}
    \item compare a two already observed candidates $x_t$ and $x_s$ belonging to different group, 
    \item determine if the current candidate is the best candidate seen so far among all the groups.
\end{enumerate}
For simplicity, we focus on the second model. However, we explain during the paper how our algorithms adapt to the first model and the cost they incur.

In the online setting, when a candidate arrives, the algorithm can choose to select them, halting the process, or it can choose to skip them, moving on to the next one---hoping to find a better candidate in the future. Once a candidate has been rejected, they cannot be recalled---the decisions are irreversible. Given the total number of candidates $N$, the probabilities $(\lambda_k)_{k \in [K]}$ characterizing the group membership, and a budget $B$, the goal is to derive an algorithm that maximizes the probability of selecting the best overall candidate. In all the following, we refer to the problem as the $(K,B)$-secretary problem




\subsection{Additional notation}
For all $t<s \in [N]$, we denote by $x_{t:s} := \{x_t,\ldots,x_s\}$, and for all $k \in [K]$ we denote by $G^k_t$ the set of candidates of group $G^k$ observed up to step $t$,
\[
G^k_{t:s} := \{ x_r \; : \; t \leq r \leq s \text{ and } g_r = t\}
= x_{s:t} \cap G^k\;.
\]
If $t = 0$, then we lighten the notation $G^k_s := G_{0:s}^k$.
Let $\A$ be any algorithm for the $(K,B)$-secretary problem, we define its stopping time $\tau(\A)$ as the step $t$ when it decides to return the observed candidate. We will often drop the explicit dependency on $\A$ and write $\tau$ when no ambiguity is involved. We will say that $\A$ succeeded if the selected candidate $x_\tau$ is the best among all the candidates $\{x_1, \ldots, x_N\}$.
Let us also define, for any step $t \geq 1$, the random variables

\begin{equation}\label{eq:def-zt}
r_t = \sum_{t'=1}^t \indic{x_t \leq x_i, g_t=g_{t'}}
\quad \text{ and } \quad
R_t = \sum_{t'=1}^t \indic{x_t \leq x_{t'}}\enspace.
\end{equation}
%\begin{equation}\label{eq:def-zt}
%z_t = \textbf{1}(x_t = \best G^{g_t}_t)
%\quad \text{ and } \quad
%Z_t = \textbf{1}(x_t = \best \{ x_1, \ldots, x_{t-1}, %x_t\})\enspace.
%\end{equation}
Both random variables have natural interpretations: given a candidate at time $t$, $r_t$ is its \emph{in-group rank} up to time $t$, while $R_t$ is its \emph{overall rank} up to time $t$.
%$r_t = 1$ indicates that $x_t$ is the best in its own group so far and $R_t = 1$ indicates that $x_t$ is the best candidate over all the past ones. 
Note that the actual values of $x_t$ do not play a role in the secretary problem and we can restrict ourselves to the observations $r_t, g_t, R_t$. While the first two random variables are always available at the beginning of round $t$, information regarding the third can be only acquired utilizing the available budget.
At each step $t \in [N]$, the decision-maker observes $r_t, g_t$ and can perform one of the following three actions:
\begin{enumerate}
    \item $\askip$: reject $x_t$ and move to the next one;
    \item $\astop$: select $x_t$;
    \item $\acomp$: if the comparison budget is not exhausted, use a comparison to determine if $(R_t = 1)$---compare the candidate $x_t$ to the best already seen candidates in the other groups;
\end{enumerate}
Furthermore, if a comparison has been used at time $t$, the algorithm has to perform $\astop$ or $\askip$ afterward. We denote respectively by $\act_{t,1}$ and $\act_{t,2}$ the first and second action made by the algorithm at step $t$. %Protocol~\ref{algo:general} describes possible interactions and defines permitted operations for our problem. 
%The history available to the algorithm after the round $t \in [N]$ is formally given by 
%\[\F_{t} = \big(r_s, g_s, \act_{s, 1}, \indic{R_s=1} \indic{\act_{s, 1} = \acomp}, \act_{s, 2}\indic{\act_{s, 1} = \acomp} \big)_{s \leq t}\;.\]
Let us also define $g^*_t$ the group to which the best candidate observed until step $t$ belongs,
\[
g^*_t = \argmax_{k \in [K]} \{ \max G^k_t \} \quad \forall t \in [N]\;,
\]
and $B_t$ as the budget available for $\A$ at step $t$  
\begin{align*}
B_1 = B \quad \text{and} \quad B_t = B_{t-1} - \indic{\act_{t, 1} = \acomp} \quad \forall t \in [N]\;.
\end{align*}
In the presence of a non-zero budget, the first time when $\A$ decides to make a comparison will be a key parameter in our analysis of the success probability. We denote it by $\rho_1(\A)$,
\begin{equation*}\label{def:rho_b}
\rho_1(\A) = \min \{t \in [N]: \act_{t,1} = \acomp \}\;.
\end{equation*}
As with the stopping time, when there is no ambiguity about $\A$, we simply write $\rho_1$.






\begin{remark}
Although we formalized the problem using the variables $r_t$ and $R_t$, the only information needed at any step $t$ is $\indic{r_t = 1}$ and $\indic{R_t = 1}$, i.e we only need to know if the candidate is the best seen so far, in its own group and overall. 
In practice, $\indic{r_t = 1}$ can be observed by comparing $x_t$ to the best candidate up to $t-1$ belonging to $G^{g_t}$, and if this is the case then $\indic{R_t = 1}$ can be observed by comparing $x_t$ to the best candidate in the other group.
\end{remark}