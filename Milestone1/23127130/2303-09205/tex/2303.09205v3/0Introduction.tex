\section{Introduction}
Online selection is among the most fundamental problems in decision-making under uncertainty. These problems are typically modeled as variations of the secretary problem \cite{freeman1983secretary, ferguson1989solved} or the prophet inequality \cite{krengel1977semiamarts, samuel1984comparison, correa2019recent}.
In the classical secretary problem~\cite{chow1971great, dynkin1963optimum, gardner1970mathematical}, the decision-maker has to identify the best candidate among a pool of totally ordered candidates that it observes sequentially in a uniformly random order. When a new candidate is observed, the decision maker can either select them and halt the process or reject them irrevocably. 
Unlike in prophet inequalities, only the relative ranks of the candidates matter and not their values. 
The optimal strategy is well known and consists of skipping the first $1/\e$ fraction of the candidates and then selecting the first candidate that is better than all previously observed ones. This strategy yields a probability $1/e$ of selecting the best candidate.
A large body of literature is dedicated to the secretary problem and its variants, we refer the interested reader to ~\cite{chow1971great, ferguson1989solved, lindley-1961} for a historical overview of this theoretical problem.

In practice, as pointed out by several social studies, the selection processes often do not reflect the actual relative ranks of the candidates and might be biased with respect to some socioeconomic attributes \cite{salem2022don, Raghavan_Barocas_Kleinberg_Levy20}. To tackle this issue, several works have explored variants of the secretary problem with noisy or biased observations of the ranks \cite{salem2019closing, freij2010partially}.
A particularly interesting setting is that of the multi-color secretary problem \cite{correa2021fairness}, where each candidate belongs to one of $K$ distinct groups, and only candidates of the same group can be compared. This corresponds for example to the case of graduate candidates from different universities, where the within-group orders are freely observable and can be trusted using a metric such as GPA, but inter-group order cannot be obtained by the same metric. This model, however, is too pessimistic, as it overlooks the possibility of obtaining inter-group orders at some cost, through testing and examination. Taking this into account, we study the multicolor secretary problem with a budget for comparisons, where comparing candidates from the same group is free, and comparing candidates from different groups has a fixed cost of $1$. We assume that the decision-maker is allowed at most $B$ comparisons. This budget $B$ represents the amount of time/money that the hiring organization is willing to invest to understand the candidate's ``true'' performance. 
As in the classical secretary problem, an algorithm is said to have \textit{succeeded} if the selected candidate is the best overall, otherwise, it has \textit{failed}. The objective is to design algorithms that maximize the probability of success.





\subsection{Contributions}


After formally defining the problem in Section \ref{sec:setup}, we introduce a class of Dynamic-Threshold (DT) algorithms. These algorithms are defined by distinct acceptance thresholds for each group, that change during the execution depending on the available budget at each step. Analyzing these algorithms in their entirety proves to be highly challenging. Consequently, we delve into their study with specific threshold choices, computing success probabilities in those cases.
Observe that with an infinite budget, the problem becomes equivalent to the classical secretary problem, where the success probability of any algorithm is at most $1/e$. Hence, we evaluate the quality of our algorithms by examining how their success probability converges to $1/e$ as the budget increases.

Our analysis begins by examining the single-threshold algorithm in the case of $K$ distinct groups. The algorithm rejects the first fraction $\w \in (0,1)$ of candidates. Then, whenever a new candidate is best in its group, and there is still a remaining comparison budget, the algorithm utilizes comparisons to determine if it is the best among all previously seen candidates. This algorithm is a particular instance of DT algorithms where all thresholds are equal. It is also a trivial extension of the $1/e$-strategy to the multi-color secretary problem with comparisons. However, in contrast with the case of only one group of candidates, the analysis is intricate due to additional factors such as group memberships, comparison history, and the available budget. By effectively controlling these parameters, we compute the asymptotic success probability of the single-threshold algorithm, demonstrating its rapid convergence to the upper bound of $1/e$.

Subsequently, our focus shifts to the case of two groups, where we explore another variant of DT algorithms: double threshold algorithms. These involve fixed acceptance thresholds for each group, that only depend on the groups' proportions and the initial budget, but do not vary during the execution of the algorithm. While we provide a precise recursive formula for computing the resulting success probability with any threshold choice, exploiting it to derive optimal thresholds proves challenging both analytically and numerically. Nevertheless, we use the formula to establish a simpler lower bound, from which we deduce a concrete threshold choice and provide corresponding guarantees on the success probability.

In the two-group scenario, we also derive the optimal algorithm among those that do not utilize the history of comparisons, we refer to them as memory-less algorithms. We present an efficient implementation of this algorithm and demonstrate, via numerical simulations, that it belongs to the class of DT algorithms when the number of candidates is large. Leveraging this insight, we numerically compute optimal thresholds for the two-group case.



\subsection{Main techniques}
The algorithms we present and analyze belong to the Dynamic-Threshold ($\DT$) algorithm family, described in Section \ref{sec:DT}. These algorithms operate based solely on the current observation at each step, independently of the past candidates' relative orders or group memberships, or the outcomes of previous comparisons. This property ensures that the success probability of these algorithms after any step is entirely determined by the available budget, the number of previously observed candidates in each group, and the group containing the best candidate observed so far. These parameters define the \textit{state} of a $\DT$ algorithm.

To estimate the asymptotic success probability of $\DT$ algorithms, it is crucial to track all the aforementioned parameters during algorithm execution. By using adequate concentration inequalities, we gauge the number of candidates in each group at any given step. Furthermore, to track the available budget and the group containing the best candidate, we derive a recursive formula for the success probability of the algorithm starting from any state. A key variable in this analysis is the time of the first comparison made by the algorithm, denoted as $\rho_1$. We rigorously explore the possible values of $\rho_1$ and their associated probabilities. Subsequently, we examine the algorithm's potential state transitions following this comparison, resulting in the wanted recursion. The following step is to solve this recursion, which we successfully do for the single-threshold algorithm in the case of $K$-groups. However, the task becomes more challenging if multiple thresholds are considered, as we demonstrate through the double threshold algorithm for two groups.





\subsection{Related work}
The secretary problem was introduced by \citet{dynkin1963optimum}, who proposed the $1/e$-threshold algorithm, having a success probability of $1/e$.
%, which is the best possible
Since then, the problem has undergone extensive study and found numerous applications, including in finance \cite{hlynka1988secretary}, mechanism design \cite{kleinberg2005multiple}, and active learning \cite{sabato2018interactive}. A closely related problem is the prophet inequality \cite{krengel1977semiamarts, samuel1984comparison}, where the decision-maker sequentially observes values sampled from known distributions, and its reward is the value of the selected item, in opposite the secretary problem where the reward is binary: $1$ if the selected value is the maximum and $0$ otherwise. Prophet inequalities also have many applications \cite{kleinberg2012matroid, chawla2010multi, feldman2014combinatorial} and have been explored in multiple variants \cite{kennedy1987prophet, azar2018prophet, bubna2023prophet}.


While the classical secretary problem solely assumes a finite sequence of totally ordered items observed in a uniformly random order, several works have investigated variations with additional hypotheses, aiming to devise improved algorithms. For instance, \citet{Gilbert2006} explored the scenario where candidates' values are independently drawn from the same known probability distribution, achieving a success probability of $\approx 0.58$. The case where item values are sampled from different known distributions has also been examined \cite{esfandiari2020prophets}, with authors demonstrating that no algorithm can achieve a success probability better than $1/e$ in certain adversarial settings, thus, in the worst case, gaining nothing compared to the standard secretary problem. Other works have explored potential improvements with limited information beyond distributions, such as samples \cite{correa2021secretary} or advice regarding the value of the current item \cite{antoniadis2020secretary, advice-2021, benomar2023advice}.

In contrast, works more aligned with ours have explored the secretary problem under additional constraints, making it more challenging to solve. Notably, \citet{correa2021fairness} introduced the multi-color secretary problem, where totally ordered candidates belonging to different groups, and only the partial order within each group, consistent with the total order, can be accessed. They designed an asymptotically optimal strategy for selecting the best candidate, in the sense of selecting the best candidate, taking into account fairness considerations. A closely related prophet problem of online multi-group selection has been studied in \cite{arsenis2022individual} and \cite{correa2021fairness}. Similarly, \citet{salem2019closing} analyze a poset-secretary problem, in which a partially ordered set is revealed sequentially and the goal is to select $k$-candidates with the highest score. Partially ordered secretaries have also been previously considered by~\citet{freij2010partially}, with the goal of selecting \emph{any} maximal element.
Another related problem has been considered in~\cite{monahan1980optimal, monahan1982state}, where the goal is to optimally stop a target process assuming that only a related process is observed. In this case, the author introduces a mechanism for acquiring information from the target process. Yet, the budget is not assumed to be fixed, and only a penalized version is considered.

This paper also intersects with other works on online algorithms, where the decision-maker is permitted to query a limited number of hints during execution. Such settings have been studied, for example, in online linear optimization \cite{BhaskaraCKP21} and the caching problem \cite{im2022parsimonious}. Another related field is \textit{advice complexity} \cite{dobrev2009measuring, arora2009computational, bockenhauer2009advice, komm2016introduction, boyar2017online}, where the goal is to determine the minimal size of advice necessary to achieve certain performance guarantees or computational bounds in decision problems.








\subsection{Outline of the paper}

The formal problem and notation are presented in Section \ref{sec:setup}. Section \ref{sec:DT} introduces Dynamic-threshold algorithms, and studies the particular case of the single-threshold algorithm. The rest of the paper focuses on the case of two groups. Double-threshold algorithms are explored in Section \ref{sec:2grps}, while Section \ref{sec:opt-dynprog} investigates the optimal memory-less algorithm. Finally, the findings from numerical experiments are discussed in Section \ref{sec:experiments}.






% We assumed that the hints are error-free, which is a realistic hypothesis in our setting. Indeed, we only require the comparison of already seen candidates and do not ask for any information in the future.
% % since the information we seek is comparing two previously seen candidates, by testing for example, and not estimating some unknown parameters that will be revealed in the future.
% Nevertheless, similarly to the approach employed for the online knapsack problem in \cite{knapsack}, our algorithms can be adapted to handle eventually unreliable comparisons. In particular, running the optimal algorithm presented in \cite{correa2021fairness} for two groups, which is also identical to our DDT algorithm in the case of null budget, with a probability $1 - \gamma$, and an algorithm taking advantage of the hints with a probability $\gamma$. By doing so, we achieve a worst-case success probability of at least $(1-\gamma)\lambda^2\exp(1/\lambda - 2)$ -where $\lambda$ is the proportion of the majority group-,  and a higher success probability when the hints are accurate.



%{\color{red}A broad range of problems within the learning augmented algorithm framework can be studied in similar settings, where knowing some time-variant quantity can enhance the algorithm's performance, and the decision-maker may request predictions about this quantity at any time within a limited budget constraint.\evg{Useless phrase and too long, I would erase it}}




