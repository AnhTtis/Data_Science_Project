%\documentclass[format=acmsmall, review=false]{acmart}
%\usepackage{acm-ec-24}
%\usepackage{booktabs} % For formal tables
%\usepackage[noend, ruled, linesnumbered]{algorithm2e} % For algorithms
%\renewcommand{\algorithmcfname}{ALGORITHM}
%\SetAlFnt{\small}
%\SetAlCapFnt{\small}
%\SetAlCapNameFnt{\small}
%\SetAlCapHSkip{0pt}
%\IncMargin{-\parindent}



\documentclass[11pt]{article}
\usepackage{graphicx} % Required for inserting images
\usepackage[noend, ruled, linesnumbered]{algorithm2e} % For algorithms
\usepackage{amsfonts}
\usepackage{amsmath}
\usepackage{amsthm}
\usepackage[margin=1in]{geometry}
\usepackage{bbm}
\usepackage{xcolor}
\usepackage{comment}
\usepackage{hyperref}
\usepackage[authoryear,round,longnamesfirst, numbers]{natbib}



% Choose a citation style by commenting/uncommenting the appropriate line:
%\setcitestyle{acmnumeric}
%\setcitestyle{authoryear}


\usepackage{dsfont}
\usepackage{tikz}





\newcommand{\interior}{\overset{\circ}}
\newcommand{\norminf}[1]{\left\|#1 \right\|_\infty}
\newcommand{\normun}[1]{\left\|#1 \right\|_1}
\newcommand{\Sig}{\sum_{i=1}^n}
\newcommand{\Supp}{\text{Supp}}
\newcommand{\norm}[1]{\left\Vert#1\right\Vert}
\newcommand{\abs}[1]{\left\vert#1\right\vert}
\newcommand{\Q}{\mathbb Q}
\newcommand{\R}{\mathbb R}
\newcommand{\N}{\mathbb N}
\newcommand{\Z}{\mathbb Z}
\newcommand{\indic}[1]{\mathds{1}\hspace{-2pt}\left(#1\right)}
\newcommand{\cov}{\text{cov}}
\newcommand{\Prob}{\mathbf P}
\renewcommand{\Pr}{\mathbf P}
\newcommand{\eps}{\varepsilon}
\newcommand{\deriv}[2]{\partial #1/\partial #2}
\newcommand{\Deriv}[2]{\frac{\partial #1}{\partial #2}}
\newcommand{\indep}{\perp \!\!\! \perp}
\newcommand{\convP}{\stackrel{\mathbb{P}}{\longrightarrow}}
\newcommand{\convL}{\stackrel{d}{\longrightarrow}}
\DeclareMathOperator{\best}{max}
\newcommand{\E}{\mathbf{E}}
\newcommand{\V}{\mathcal{V}}
\newcommand{\U}{\mathcal{U}}
\newcommand{\A}{\mathcal{A}}
\newcommand{\F}{\mathcal{F}}
\newcommand{\C}{\mathcal{C}}
\newcommand{\askip}{\texttt{skip}}
\newcommand{\astop}{\texttt{stop}}
\newcommand{\acomp}{\texttt{compare}}
\newcommand{\action}{\texttt{action}}
\newcommand{\accept}{\texttt{accept}}
\newcommand{\anone}{\texttt{none}}
\newcommand{\act}{a}
\newcommand{\w}{\alpha}
\newcommand{\diff}{\mathrm{d}}
\newcommand{\comp}{\rho}
\renewcommand{\phi}{\varphi}
\newcommand{\e}{\mathrm{e}}
\newcommand{\eqdef}{\coloneqq}
\newcommand{\sgn}{\text{sgn}}
\newcommand{\st}[1]{\texttt{#1}}
\newcommand{\pl}{\text{pl}}
\newcommand{\xmax}{x_{\max}}
\renewcommand{\S}{S}
\newcommand{\rwd}{\mathcal{R}}
\newcommand{\DT}{\text{DT}}

\DeclareMathOperator*{\argmin}{argmin}
\DeclareMathOperator*{\argmax}{argmax}







\newtheorem{theorem}{Theorem}[section]
\newtheorem{corollary}{Corollary}[theorem]
\newtheorem{lemma}[theorem]{Lemma}
\providecommand*\lemmaautorefname{Lemma}
\newtheorem{remark}[theorem]{Remark}
\newtheorem{definition}{Definition}[section]
\newtheorem{example}{Example}[section]
\newtheorem{proposition}[theorem]{Proposition}





% Title. Note the optional short title for running heads. In the interest of anonymization, please do not include any acknowledgements.
\title{Addressing bias in online selection with limited budget of comparisons}


% Anonymized submission.
\author{Ziyad Benomar}

% Abstract. Note that this must come before \maketitle.


\begin{document}

\date{}

\author{% 
\begin{tabular}{c} Ziyad Benomar \\  CREST, ENSAE, Ecole polytechique \\ ziyad.benomar@ensae.fr \\ \\
Nicolas Schreuder \\ CNRS, Laboratoire d’informatique \\Gaspard Monge (LIGM/UMR 8049)\\  nicolas.schreuder@cnrs.fr \end{tabular} \and
\begin{tabular}{c} Evgenii Chzhen \\  CNRS, LMO, Université Paris-Saclay \\ evgenii.chzhen@universite-paris-saclay.fr \\ \\
Vianney Perchet \\ CREST, ENSAE, Criteo AI LAB \\ vianney.perchet@normalesup.org\\
\end{tabular} }

    
% Title page for title and abstract only.
\begin{titlepage}

\maketitle



\begin{abstract}
Consider a hiring process with candidates coming from different universities. It is easy to order candidates with the same background, yet it can be challenging to compare them otherwise.
The latter case requires additional costly assessments, leading to a potentially high total cost for the hiring organization.
Given an assigned budget, what would be an optimal strategy to select the most qualified candidate?
We model the above problem as a multicolor secretary problem, allowing comparisons between candidates from distinct groups at a fixed cost. Our study explores how the allocated budget enhances the success probability of online selection algorithms.
\end{abstract}




\end{titlepage}

% Paper body
\section{Introduction}
Online selection is among the most fundamental problems in decision-making under uncertainty. These problems are typically modeled as variations of the secretary problem \cite{freeman1983secretary, ferguson1989solved} or the prophet inequality \cite{krengel1977semiamarts, samuel1984comparison, correa2019recent}.
In the classical secretary problem~\cite{chow1971great, dynkin1963optimum, gardner1970mathematical}, the decision-maker has to identify the best candidate among a pool of totally ordered candidates that it observes sequentially in a uniformly random order. When a new candidate is observed, the decision maker can either select them and halt the process or reject them irrevocably. 
Unlike in prophet inequalities, only the relative ranks of the candidates matter and not their values. 
The optimal strategy is well known and consists of skipping the first $1/\e$ fraction of the candidates and then selecting the first candidate that is better than all previously observed ones. This strategy yields a probability $1/e$ of selecting the best candidate.
A large body of literature is dedicated to the secretary problem and its variants, we refer the interested reader to ~\cite{chow1971great, ferguson1989solved, lindley-1961} for a historical overview of this theoretical problem.

In practice, as pointed out by several social studies, the selection processes often do not reflect the actual relative ranks of the candidates and might be biased with respect to some socioeconomic attributes \cite{salem2022don, Raghavan_Barocas_Kleinberg_Levy20}. To tackle this issue, several works have explored variants of the secretary problem with noisy or biased observations of the ranks \cite{salem2019closing, freij2010partially}.
A particularly interesting setting is that of the multi-color secretary problem \cite{correa2021fairness}, where each candidate belongs to one of $K$ distinct groups, and only candidates of the same group can be compared. This corresponds for example to the case of graduate candidates from different universities, where the within-group orders are freely observable and can be trusted using a metric such as GPA, but inter-group order cannot be obtained by the same metric. This model, however, is too pessimistic, as it overlooks the possibility of obtaining inter-group orders at some cost, through testing and examination. Taking this into account, we study the multicolor secretary problem with a budget for comparisons, where comparing candidates from the same group is free, and comparing candidates from different groups has a fixed cost of $1$. We assume that the decision-maker is allowed at most $B$ comparisons. This budget $B$ represents the amount of time/money that the hiring organization is willing to invest to understand the candidate's ``true'' performance. 
As in the classical secretary problem, an algorithm is said to have \textit{succeeded} if the selected candidate is the best overall, otherwise, it has \textit{failed}. The objective is to design algorithms that maximize the probability of success.





\subsection{Contributions}


After formally defining the problem in Section \ref{sec:setup}, we introduce a class of Dynamic-Threshold (DT) algorithms. These algorithms are defined by distinct acceptance thresholds for each group, that change during the execution depending on the available budget at each step. Analyzing these algorithms in their entirety proves to be highly challenging. Consequently, we delve into their study with specific threshold choices, computing success probabilities in those cases.
Observe that with an infinite budget, the problem becomes equivalent to the classical secretary problem, where the success probability of any algorithm is at most $1/e$. Hence, we evaluate the quality of our algorithms by examining how their success probability converges to $1/e$ as the budget increases.

Our analysis begins by examining the single-threshold algorithm in the case of $K$ distinct groups. The algorithm rejects the first fraction $\w \in (0,1)$ of candidates. Then, whenever a new candidate is best in its group, and there is still a remaining comparison budget, the algorithm utilizes comparisons to determine if it is the best among all previously seen candidates. This algorithm is a particular instance of DT algorithms where all thresholds are equal. It is also a trivial extension of the $1/e$-strategy to the multi-color secretary problem with comparisons. However, in contrast with the case of only one group of candidates, the analysis is intricate due to additional factors such as group memberships, comparison history, and the available budget. By effectively controlling these parameters, we compute the asymptotic success probability of the single-threshold algorithm, demonstrating its rapid convergence to the upper bound of $1/e$.

Subsequently, our focus shifts to the case of two groups, where we explore another variant of DT algorithms: double threshold algorithms. These involve fixed acceptance thresholds for each group, that only depend on the groups' proportions and the initial budget, but do not vary during the execution of the algorithm. While we provide a precise recursive formula for computing the resulting success probability with any threshold choice, exploiting it to derive optimal thresholds proves challenging both analytically and numerically. Nevertheless, we use the formula to establish a simpler lower bound, from which we deduce a concrete threshold choice and provide corresponding guarantees on the success probability.

In the two-group scenario, we also derive the optimal algorithm among those that do not utilize the history of comparisons, we refer to them as memory-less algorithms. We present an efficient implementation of this algorithm and demonstrate, via numerical simulations, that it belongs to the class of DT algorithms when the number of candidates is large. Leveraging this insight, we numerically compute optimal thresholds for the two-group case.



\subsection{Main techniques}
The algorithms we present and analyze belong to the Dynamic-Threshold ($\DT$) algorithm family, described in Section \ref{sec:DT}. These algorithms operate based solely on the current observation at each step, independently of the past candidates' relative orders or group memberships, or the outcomes of previous comparisons. This property ensures that the success probability of these algorithms after any step is entirely determined by the available budget, the number of previously observed candidates in each group, and the group containing the best candidate observed so far. These parameters define the \textit{state} of a $\DT$ algorithm.

To estimate the asymptotic success probability of $\DT$ algorithms, it is crucial to track all the aforementioned parameters during algorithm execution. By using adequate concentration inequalities, we gauge the number of candidates in each group at any given step. Furthermore, to track the available budget and the group containing the best candidate, we derive a recursive formula for the success probability of the algorithm starting from any state. A key variable in this analysis is the time of the first comparison made by the algorithm, denoted as $\rho_1$. We rigorously explore the possible values of $\rho_1$ and their associated probabilities. Subsequently, we examine the algorithm's potential state transitions following this comparison, resulting in the wanted recursion. The following step is to solve this recursion, which we successfully do for the single-threshold algorithm in the case of $K$-groups. However, the task becomes more challenging if multiple thresholds are considered, as we demonstrate through the double threshold algorithm for two groups.





\subsection{Related work}
The secretary problem was introduced by \citet{dynkin1963optimum}, who proposed the $1/e$-threshold algorithm, having a success probability of $1/e$.
%, which is the best possible
Since then, the problem has undergone extensive study and found numerous applications, including in finance \cite{hlynka1988secretary}, mechanism design \cite{kleinberg2005multiple}, and active learning \cite{sabato2018interactive}. A closely related problem is the prophet inequality \cite{krengel1977semiamarts, samuel1984comparison}, where the decision-maker sequentially observes values sampled from known distributions, and its reward is the value of the selected item, in opposite the secretary problem where the reward is binary: $1$ if the selected value is the maximum and $0$ otherwise. Prophet inequalities also have many applications \cite{kleinberg2012matroid, chawla2010multi, feldman2014combinatorial} and have been explored in multiple variants \cite{kennedy1987prophet, azar2018prophet, bubna2023prophet}.


While the classical secretary problem solely assumes a finite sequence of totally ordered items observed in a uniformly random order, several works have investigated variations with additional hypotheses, aiming to devise improved algorithms. For instance, \citet{Gilbert2006} explored the scenario where candidates' values are independently drawn from the same known probability distribution, achieving a success probability of $\approx 0.58$. The case where item values are sampled from different known distributions has also been examined \cite{esfandiari2020prophets}, with authors demonstrating that no algorithm can achieve a success probability better than $1/e$ in certain adversarial settings, thus, in the worst case, gaining nothing compared to the standard secretary problem. Other works have explored potential improvements with limited information beyond distributions, such as samples \cite{correa2021secretary} or advice regarding the value of the current item \cite{antoniadis2020secretary, advice-2021, benomar2023advice}.

In contrast, works more aligned with ours have explored the secretary problem under additional constraints, making it more challenging to solve. Notably, \citet{correa2021fairness} introduced the multi-color secretary problem, where totally ordered candidates belonging to different groups, and only the partial order within each group, consistent with the total order, can be accessed. They designed an asymptotically optimal strategy for selecting the best candidate, in the sense of selecting the best candidate, taking into account fairness considerations. A closely related prophet problem of online multi-group selection has been studied in \cite{arsenis2022individual} and \cite{correa2021fairness}. Similarly, \citet{salem2019closing} analyze a poset-secretary problem, in which a partially ordered set is revealed sequentially and the goal is to select $k$-candidates with the highest score. Partially ordered secretaries have also been previously considered by~\citet{freij2010partially}, with the goal of selecting \emph{any} maximal element.
Another related problem has been considered in~\cite{monahan1980optimal, monahan1982state}, where the goal is to optimally stop a target process assuming that only a related process is observed. In this case, the author introduces a mechanism for acquiring information from the target process. Yet, the budget is not assumed to be fixed, and only a penalized version is considered.

This paper also intersects with other works on online algorithms, where the decision-maker is permitted to query a limited number of hints during execution. Such settings have been studied, for example, in online linear optimization \cite{BhaskaraCKP21} and the caching problem \cite{im2022parsimonious}. Another related field is \textit{advice complexity} \cite{dobrev2009measuring, arora2009computational, bockenhauer2009advice, komm2016introduction, boyar2017online}, where the goal is to determine the minimal size of advice necessary to achieve certain performance guarantees or computational bounds in decision problems.








\subsection{Outline of the paper}

The formal problem and notation are presented in Section \ref{sec:setup}. Section \ref{sec:DT} introduces Dynamic-threshold algorithms, and studies the particular case of the single-threshold algorithm. The rest of the paper focuses on the case of two groups. Double-threshold algorithms are explored in Section \ref{sec:2grps}, while Section \ref{sec:opt-dynprog} investigates the optimal memory-less algorithm. Finally, the findings from numerical experiments are discussed in Section \ref{sec:experiments}.






% We assumed that the hints are error-free, which is a realistic hypothesis in our setting. Indeed, we only require the comparison of already seen candidates and do not ask for any information in the future.
% % since the information we seek is comparing two previously seen candidates, by testing for example, and not estimating some unknown parameters that will be revealed in the future.
% Nevertheless, similarly to the approach employed for the online knapsack problem in \cite{knapsack}, our algorithms can be adapted to handle eventually unreliable comparisons. In particular, running the optimal algorithm presented in \cite{correa2021fairness} for two groups, which is also identical to our DDT algorithm in the case of null budget, with a probability $1 - \gamma$, and an algorithm taking advantage of the hints with a probability $\gamma$. By doing so, we achieve a worst-case success probability of at least $(1-\gamma)\lambda^2\exp(1/\lambda - 2)$ -where $\lambda$ is the proportion of the majority group-,  and a higher success probability when the hints are accurate.



%{\color{red}A broad range of problems within the learning augmented algorithm framework can be studied in similar settings, where knowing some time-variant quantity can enhance the algorithm's performance, and the decision-maker may request predictions about this quantity at any time within a limited budget constraint.\evg{Useless phrase and too long, I would erase it}}





\section{Formal Problem}
\label{sec:setup}

We consider a strictly totally ordered set of cardinal $N$, whose elements will be called \textit{candidates}. We assume that these candidates are observed in a uniformly random arrival order $(x_1, \ldots x_N)$, and that they are partitioned into $K$ groups $G^1, \ldots, G^k$. For all $t\in [N]$, we denote by $g_t \in [K]$ the group of $x_t$, i.e. $x_t \in G^{g_t}$, and we assume that $\{g_t\}_{t\in [N]}$ are mutually independent random variables
\[
\quad \Prob(g_t = k) = \lambda_k, \quad \forall t \in [N], \forall k \in [K]\;,
\]
where $\lambda_k > 0$ for all $k \in [K]$ and $\sum_{k=1}^K \lambda_k = 1$.

We assume that comparing candidates of the same group is free, while comparing two candidates of different groups is costly. To address the latter case, we consider that a budget $B \geq 0$ is given for comparisons and we propose two models: the algorithm can pay a cost of $1$ in order to:
\begin{enumerate}
    \item compare a two already observed candidates $x_t$ and $x_s$ belonging to different group, 
    \item determine if the current candidate is the best candidate seen so far among all the groups.
\end{enumerate}
For simplicity, we focus on the second model. However, we explain during the paper how our algorithms adapt to the first model and the cost they incur.

In the online setting, when a candidate arrives, the algorithm can choose to select them, halting the process, or it can choose to skip them, moving on to the next one---hoping to find a better candidate in the future. Once a candidate has been rejected, they cannot be recalled---the decisions are irreversible. Given the total number of candidates $N$, the probabilities $(\lambda_k)_{k \in [K]}$ characterizing the group membership, and a budget $B$, the goal is to derive an algorithm that maximizes the probability of selecting the best overall candidate. In all the following, we refer to the problem as the $(K,B)$-secretary problem




\subsection{Additional notation}
For all $t<s \in [N]$, we denote by $x_{t:s} := \{x_t,\ldots,x_s\}$, and for all $k \in [K]$ we denote by $G^k_t$ the set of candidates of group $G^k$ observed up to step $t$,
\[
G^k_{t:s} := \{ x_r \; : \; t \leq r \leq s \text{ and } g_r = t\}
= x_{s:t} \cap G^k\;.
\]
If $t = 0$, then we lighten the notation $G^k_s := G_{0:s}^k$.
Let $\A$ be any algorithm for the $(K,B)$-secretary problem, we define its stopping time $\tau(\A)$ as the step $t$ when it decides to return the observed candidate. We will often drop the explicit dependency on $\A$ and write $\tau$ when no ambiguity is involved. We will say that $\A$ succeeded if the selected candidate $x_\tau$ is the best among all the candidates $\{x_1, \ldots, x_N\}$.
Let us also define, for any step $t \geq 1$, the random variables

\begin{equation}\label{eq:def-zt}
r_t = \sum_{t'=1}^t \indic{x_t \leq x_i, g_t=g_{t'}}
\quad \text{ and } \quad
R_t = \sum_{t'=1}^t \indic{x_t \leq x_{t'}}\enspace.
\end{equation}
%\begin{equation}\label{eq:def-zt}
%z_t = \textbf{1}(x_t = \best G^{g_t}_t)
%\quad \text{ and } \quad
%Z_t = \textbf{1}(x_t = \best \{ x_1, \ldots, x_{t-1}, %x_t\})\enspace.
%\end{equation}
Both random variables have natural interpretations: given a candidate at time $t$, $r_t$ is its \emph{in-group rank} up to time $t$, while $R_t$ is its \emph{overall rank} up to time $t$.
%$r_t = 1$ indicates that $x_t$ is the best in its own group so far and $R_t = 1$ indicates that $x_t$ is the best candidate over all the past ones. 
Note that the actual values of $x_t$ do not play a role in the secretary problem and we can restrict ourselves to the observations $r_t, g_t, R_t$. While the first two random variables are always available at the beginning of round $t$, information regarding the third can be only acquired utilizing the available budget.
At each step $t \in [N]$, the decision-maker observes $r_t, g_t$ and can perform one of the following three actions:
\begin{enumerate}
    \item $\askip$: reject $x_t$ and move to the next one;
    \item $\astop$: select $x_t$;
    \item $\acomp$: if the comparison budget is not exhausted, use a comparison to determine if $(R_t = 1)$---compare the candidate $x_t$ to the best already seen candidates in the other groups;
\end{enumerate}
Furthermore, if a comparison has been used at time $t$, the algorithm has to perform $\astop$ or $\askip$ afterward. We denote respectively by $\act_{t,1}$ and $\act_{t,2}$ the first and second action made by the algorithm at step $t$. %Protocol~\ref{algo:general} describes possible interactions and defines permitted operations for our problem. 
%The history available to the algorithm after the round $t \in [N]$ is formally given by 
%\[\F_{t} = \big(r_s, g_s, \act_{s, 1}, \indic{R_s=1} \indic{\act_{s, 1} = \acomp}, \act_{s, 2}\indic{\act_{s, 1} = \acomp} \big)_{s \leq t}\;.\]
Let us also define $g^*_t$ the group to which the best candidate observed until step $t$ belongs,
\[
g^*_t = \argmax_{k \in [K]} \{ \max G^k_t \} \quad \forall t \in [N]\;,
\]
and $B_t$ as the budget available for $\A$ at step $t$  
\begin{align*}
B_1 = B \quad \text{and} \quad B_t = B_{t-1} - \indic{\act_{t, 1} = \acomp} \quad \forall t \in [N]\;.
\end{align*}
In the presence of a non-zero budget, the first time when $\A$ decides to make a comparison will be a key parameter in our analysis of the success probability. We denote it by $\rho_1(\A)$,
\begin{equation*}\label{def:rho_b}
\rho_1(\A) = \min \{t \in [N]: \act_{t,1} = \acomp \}\;.
\end{equation*}
As with the stopping time, when there is no ambiguity about $\A$, we simply write $\rho_1$.






\begin{remark}
Although we formalized the problem using the variables $r_t$ and $R_t$, the only information needed at any step $t$ is $\indic{r_t = 1}$ and $\indic{R_t = 1}$, i.e we only need to know if the candidate is the best seen so far, in its own group and overall. 
In practice, $\indic{r_t = 1}$ can be observed by comparing $x_t$ to the best candidate up to $t-1$ belonging to $G^{g_t}$, and if this is the case then $\indic{R_t = 1}$ can be observed by comparing $x_t$ to the best candidate in the other group.
\end{remark}
\section{Dynamic threshold algorithm for $K$ groups}\label{sec:DT}
In this section, we introduce a general family of Dynamic-threshold ($\DT$) algorithms. 
A $\DT$ algorithm for the $(K,B)$-secretary problem is defined by a finite doubly-indexed sequence $(\w_{k,b})_{k \in [K],b \leq B}$ of real numbers in $[0,1]$, which determines the \textit{acceptance} thresholds based on the group of the observed candidate and the available budget. During a run of the algorithm, the thresholds used for each group dynamically change depending on the evolution of the available budget.
We denote this algorithm by $\A^B\big((\w_{k,b})_{k \in [K],b \leq B}\big)$.

Upon the arrival of a new candidate $x_t$, the algorithm observes its group $g_t = k \in [K]$ and its in-group rank $r_t$, and has an available budget of $B_t = b \geq 0$. If $t/N < \w_{k,b}$ or $r_t = 0$, the candidate is immediately rejected. Otherwise, if $t/N \geq \w_{k,b}$ and $r_t = 1$, then the candidate is selected if $b=0$; and if the budget is not yet exhausted ($b>0$), then the algorithm pays a unit cost to observe the variable $\indic{R_t = 1}$. If this variable is $1$, indicating a favorable comparison, the candidate is selected; otherwise, it is rejected.
A formal description is given in Algorithm \ref{algo:DT}, and a visual representation for the case of three groups is provided in Figure \ref{fig:DTviz}.






\begin{figure}
    \centering
    \includegraphics[width=0.8\textwidth]{figs/DTviz.pdf}
    \caption{Schematic description of a DT algorithm in the case of 3 groups}
    \label{fig:DTviz}
\end{figure}


\begin{algorithm}[h!]
\DontPrintSemicolon 
\caption{Dynamic-Threshold algorithm $\A^B\big((\w_{k,b})_{k \in [K],b \leq B}\big)$}\label{algo:DT}
\SetKwInput{Input}{Input}
   \SetKwInOut{Output}{Output}
   \SetKwInput{Initialization}{Initialization}
   \Input{Available budget $B$, thresholds $(\w_{k,b})_{k \in [K],b \leq B}$}
   \Initialization{$b = B$}
\For{$t=1,\ldots,N$}{
    Receive new observation: $(r_t, g_t)$\;
    \If(\tcp*[f]{\small compare in-group}){$t \geq \lfloor \w_{g_t,b} N \rfloor \emph{ and } r_t = 1$}{%\label{algoline:better-than-best}
        \If(\tcp*[f]{\small check budget}){$b > 0$}{
            % \vspace{-0.4cm}
            Update budget: $b \gets b - 1$\;
            \lIf(\tcp*[f]{\small compare inter-group}){$R_t = 1$}{
            Return: $t$
            }
        }
        \lElse{Return: $t$} 
    }
}
\end{algorithm}












\subsection{Single-threshold algorithm for $K$ groups}\label{sec:single-thresh}
In this section, we focus on the single-threshold algorithm, a specific case within the family of $\DT$ algorithms, where all thresholds are identical across groups and budgets. Initially, the algorithm rejects all candidates until step $T-1$, where $T \in [N]$ is a fixed threshold. Upon encountering a new candidate that is the best within its group, if no budget remains, the candidate is selected. Alternatively, if there is still a budget available, the algorithm utilizes it to determine if the current candidate is the best among all groups. If that is the case, the candidate is then selected. We denote by $\A^B_T$ the single-threshold algorithm with threshold $T$ and budget $B$.
%Although seemingly simplistic, 
We demonstrate that this algorithm has an asymptotic success probability converging very rapidly to the upper bound of $1/e$.






In this first lemma, we prove a recursion formula on the success probability of the single-threshold algorithm, with a threshold $T = \lfloor \w N \rfloor$ for some $\w \in [0,1]$.

\begin{lemma}\label{lem:single-thres-recursion}
The success probability of the single threshold algorithm $\A_T^B$ with threshold $T = \lfloor \w N \rfloor$ and budget $B \geq 0$ satisfies the recursion formula
\[
\Pr(\A^{B}_{T} \text{ succeeds})
= \frac{\w - \w^{K}}{K-1} + \indic{B\geq 0} (K-1) \sum_{t=T}^N \frac{T^K}{t^{K+1}}\Pr(\A^{B-1}_{t+1} \text{ succeeds}) + O\Big( \sqrt{\tfrac{\log N}{N}}\Big)\;.
\]
\end{lemma}


The proof hinges on analyzing the behavior of the algorithm following the first comparison. After that comparison, the algorithm halts if $R_t = 1$, and the success probability can be computed in that case. Otherwise, if $R_t \neq 1$, the candidate is rejected, and the algorithm transitions to a new state at step $t+1$, where the available budget reduces to $B-1$.
%Leveraging the memory-less property of the single-threshold algorithm, 
Its success probability becomes precisely that of algorithm $\A^{B-1}_{t+1}$, with budget $B-1$ and threshold $t+1$.






The recursion outlined in Lemma \ref{lem:single-thres-recursion} can be used to calculate the asymptotic success probability of the single-threshold algorithm $\A_{\lfloor \w N \rfloor}^B$ as the number of candidates $N$ approaches infinity. 
%The resultant expression is presented in the following theorem.


\begin{theorem}\label{thm:single-thresh}
The asymptotic success probability of the single threshold algorithm $\A^B_T$ with threshold $T = \lfloor \w N \rfloor$ and budget $B \geq 0$ is 
\[
\lim_{N \to \infty} \Pr(\A^B_{\lfloor \w N \rfloor} \text{ succeeds}) = \frac{\w^K}{K-1} \sum_{b = 0}^B \left( \frac{1}{\w^{K-1}} - \sum_{\ell = 0}^b \frac{\log(1/\w^{K-1})^\ell}{\ell !} \right)\;.
\]
In particular, 
\[
\lim_{B \to \infty} \lim_{N \to \infty} \Pr(\A^B_{\lfloor \w N \rfloor} \text{ succeeds}) = \w \log(1/\w)\;.
\]
\end{theorem}


Note that, for $B = \infty$, the asymptotic success probability in the previous theorem corresponds to the success probability of the algorithm with a threshold $\lfloor \w N \rfloor$ in the secretary problem. 
Indeed, with an unlimited budget, the decision-maker can assess at each step whether the current candidate is the best so far, and the problem becomes equivalent to the classical secretary problem.

\paragraph{Alternative comparison model.}
In the alternative comparison model presented in Section \ref{sec:setup}, the single threshold algorithm $\A^B_T$ can be adapted to guarantee the same success probability at the cost of $K-1$ additional comparisons. After the first $T$  candidates are rejected, $K-1$ comparisons are made between the maximums from each group to identify the best candidate so far. The algorithm then keeps track of the best candidate: whenever a new candidate is the best in their group, they are compared to the current best candidate using a single comparison, and the latter is updated accordingly. This approach enjoys the same guarantees as in Theorem \ref{thm:single-thresh}, but with a budget of $K + B - 1$ instead of $B$.



The next corollary measures how the success probability of the single threshold algorithm, in the setting with $K$ groups, converges to $1/e$ as the budget increases.




\begin{corollary}\label{cor:single-thresh-factorial-conv}
The success probability of the single-threshold algorithm with threshold $T = \lfloor N/e \rfloor$ and budget $B \geq 0$ satisfies
\[
\lim_{N \to \infty} \Pr(\A_{\lfloor N/e \rfloor}^B \text{ succeeds})
\geq \frac{1}{e}\left(1 - \frac{(K-1)^{B+1}}{(B+1)!} \right)\;.
\]
In particular, for all $\eps > 0$, if $K \leq 1 + \frac{B+1}{e}(e\eps)^{\frac{1}{B+1}}$, then $\lim_{N} \Pr(\A_{\lfloor N/e \rfloor}^B \text{ succeeds}) \geq (1-\eps)/e$.
\end{corollary}


This corollary proves that the success probability of $\A_{\lfloor N/e \rfloor}^B$ converges very rapidly to the upper bound $1/e$ as $B$ increases. However, the convergence becomes slower when $K$ is larger.





Surprisingly, the asymptotic success probability of $\A_{\lfloor N/e \rfloor}^B$ is not influenced by the proportions $(\lambda_k)_{k \in [K]}$, but only by the number of groups $K$. 
%For example, let us consider the scenario with only one group, reducing the problem to the classical secretary problem where the budget becomes insignificant. In this case, the asymptotic success probability of the single-threshold algorithm with a threshold $\lfloor \w N\rfloor$ is $\w \log(1/\w)$. However, when there are two groups with respective proportions $\lambda_1 = 1-\delta$ and $\lambda_2 = \delta$, with $\delta > 0$ arbitrarily small, the asymptotic success probability of the single-threshold algorithm becomes $\frac{\w(1-\w)}{2}$, identical to the scenario where $\lambda_1 = \lambda_2 = 1/2$.
This means that the algorithm does not benefit from the cases where there is a majority group $G^k$ with $\lambda_k$ very close to 1, which would make the problem easier. Indeed, it is always possible to achieve a success probability of $\max_{k \in [K]} \lambda_k /e$ by rejecting all the candidates not belonging to the majority group $G^{k^*}$, and using the classical $1/e$-rule counting only elements of $G^{k^*}$. This algorithm can be combined with ours by always running the one with the highest success probability, depending on the available budget, the number of groups, and the group proportions. The resulting algorithm has a success probability that converges to the upper bound $1/e$ both when $B$ increases and when $\max_k \lambda_k$ converges to $1$. Nonetheless, due to the very fast convergence of the success probability of the single threshold algorithm to $1/e$, the improvement brought by having a majority group is only marginal when the budget is sufficient. 

As a consequence, the single threshold algorithm surprisingly constitutes a very efficient solution to the problem even with moderate values of the budget. Computing the optimal thresholds remains however an intriguing question, which we explore in the following sections in the case of two groups. 





\section{The case of two groups}\label{sec:2grps}

In this section, we delve into the particular case of two groups, and we demonstrate how leveraging different thresholds for each group can enhance the success probability. Let $\lambda \in (0,1)$ represent the probability of belonging to group $G^1$, and $1-\lambda$ the probability of belonging to group $G^2$. We examine the success probability of Algorithm $\A^B(\alpha,\beta)$, with threshold $\lfloor \alpha N \rfloor$ for group $G^1$ and $\lfloor \beta N \rfloor$ for group $G^2$, having a budget of $B$ comparisons. This algorithm is a specific instance of the $\DT$ family, wherein the thresholds depend only on the group, and not on the available budget. We call it a \textit{static double-threshold} algorithm.



We assume without loss of generality that $\alpha \leq \beta$, 
and we denote by $\C_N$ the event 
\[
(\C_N): \quad \forall t \geq 1: \max(||G^1_t| - \lambda t|\;,\;||G^2_t| - (1-\lambda) t|) \leq 4 \sqrt{t \log N}\;.
\]
This event provides control over the group sizes at each step. Lemma \ref{lem:conc-card} guarantees that $\C_N$ holds true with a probability at least $1-\frac{1}{N^2}$ for $N \geq 4$. 
Furthermore, for all $t \in [N]$, we denote by $\A^B_t(\alpha,\beta)$ the algorithm with acceptance thresholds $\max\{\lfloor \alpha N \rfloor, t\}$ and $\max\{\lfloor \beta N \rfloor, t\}$ respectively for groups $G^1$ and $G^2$, and we denote by $\U^B_{N,t,k}$ the probability 
\begin{equation}\label{eq:def-U}
\U^B_{N,t,k} = \Pr(\A^B_t(\alpha, \beta) \text{ succeeds},  g^*_{t-1} = k \mid \C_N)\;.    
\end{equation}
%By Lemma \ref{lem:memless}, the state of $\A^B(\alpha, \beta)$ at any step $t$  is completely determined by $(B_t, g^*_{t-1}, |G^1_{t-1}|)$, with the size of $G^2_{t-1}$ implicitly deduced as $|G^2_{t-1}| = t - 1 - |G^1_{t-1}|$. 
Similar to the analysis of the single-threshold algorithm, we establish in Lemma \ref{lem:recursion-U-alpha-beta} a recursion formula satisfied by $(\U^B_{N,t,k})_{B,t,k}$, which we later utilize to derive lower bounds on the asymptotic success probability of $\A^B(\alpha,\beta)$.
To prove this lemma, we study the probability distribution of the occurrence time $\rho_1$ of the first comparison made by $\A^B_t(\alpha, \beta)$, and we examine the algorithm's success probability following it. Essentially, if $\rho_1 = s$, we can compute the probability of stopping and the corresponding success probability, and the distribution of the state of the algorithm at step $s+1$, which yields the recursion. 
Using adequate concentration arguments and Lemma \ref{lem:recursion-U-alpha-beta}, we show the two following results, giving explicit recursive formulas satisfied by the limit of $\U^{B}_{N,t,k}$ when $N \to \infty$, respectively for $k=2$ and $k=1$.

















\begin{lemma}\label{lem:lim-2grp-k=2}
For all $B \geq 0$ and $w \in [\alpha, \beta]$, the limit $\phi^B_2(\alpha,\beta; w) = \lim_{N \to \infty}\U^B_{N,\lfloor w N \rfloor,2}$ exists, 
and it satisfies the following recursion
\begin{align*}
\phi^B_2(\alpha,\beta;w)
&= - \lambda w \log\big((1-\lambda) \tfrac{w}{\beta} + \lambda \big) + \frac{(1-\lambda)\beta w^2 }{(1-\lambda)w + \lambda \beta}  \sum_{b = 0}^B \left( \frac{1}{\beta} - \sum_{\ell = 0}^b \frac{\log(1/\beta)^\ell}{\ell !} \right)\\
& \quad + \indic{B > 0} w^2 \int_w^\beta \frac{ (1-\lambda)^2 w + \lambda(2-\lambda)u}{((1-\lambda)w+\lambda u)^2 u^2}  \phi^{B-1}_2(\alpha, \beta; u) du\;.
\end{align*}
Moreover, $\U^B_{N,\lfloor w N \rfloor,2} = \phi^B_2(\alpha,\beta; w) + O\Big( \sqrt{\tfrac{\log N}{N}} \Big)$.
\end{lemma}




























\begin{lemma}\label{lem:lim-2grp-k=1}
For all $B \geq 0$ and $w \in [\alpha, \beta]$, the limit $\phi^B_1(\alpha,\beta; w) = \lim_{N \to \infty}\U^B_{N,\lfloor w N \rfloor,1}$ exists, 
and it satisfies the following recursion
\begin{align*}
\phi^B_1(\alpha,\beta;w)
&= \lambda w \log\big(1-\lambda + \lambda \tfrac{\beta}{w}\big) 
+ \frac{\lambda w \beta^2}{(1-\lambda) w + \lambda \beta} \sum_{b = 0}^B \left( \frac{1}{\beta} - \sum_{\ell = 0}^b \frac{\log(1/\beta)^\ell}{\ell !} \right) \\
& \quad + \indic{B > 0} \lambda^2 w \int_w^\beta \frac{(u-w) \phi^{B-1}_2(\alpha, \beta; u)}{((1-\lambda)w + \lambda u)^2 u} du\;.
\end{align*}
Moreover, $\U^B_{N,\lfloor w N \rfloor,1} = \phi^B_1(\alpha,\beta; w) + O\Big( \sqrt{\tfrac{\log N}{N}} \Big)$.
\end{lemma}




We deduce that the asymptotic success probability of Algorithm $\A^B_{\lfloor w N\rfloor}$, conditioned on the event $\C_N$, exists and equals $\phi_1^B(\alpha,\beta;w) + \phi_2^B(\alpha,\beta;w)$. Additionally, by applying Lemma \ref{lem:pr-cond-C_N}, we eliminate the conditioning on $\C_N$, thus proving the following theorem.






\begin{theorem}\label{thm:success-2grps}
For all $0 < \alpha \leq \beta \leq 1$, The success probability of Algorithm $\A^B(\alpha,\beta)$ satisfies
\begin{align*}
\Pr(\A^B(\alpha,&\beta) \text{ succeeds}) - O\Big(\sqrt{\tfrac{\log N}{N}}\Big)\\
&= \lambda \alpha \log\big(\tfrac{\beta}{\alpha}\big) + \alpha \beta \sum_{b = 0}^B \left( \frac{1}{\beta} - \sum_{\ell = 0}^b \frac{\log(1/\beta)^\ell}{\ell !} \right)
+ \indic{B > 0} \alpha \int_\alpha^\beta \frac{ \phi^{B-1}_2(\alpha, \beta; u) du}{u^2}\;,
\end{align*}
with $\phi_2^B(\alpha,\beta;\cdot)$ defined in Lemma \ref{lem:lim-2grp-k=2}.
\end{theorem}


It is possible to use Theorem \ref{thm:success-2grps} and \ref{lem:lim-2grp-k=2} to numerically compute the success probability of $\A^B(\alpha, \beta)$. However, this computation is heavy due the recursion defining $\phi^B_2(\alpha, \beta;w)$. Moreover, it is difficult to prove a closed expression, and even more to compute the optimal thresholds.

By disregarding the term containing $\phi^2(\alpha, \beta; \cdot)$ in the theorem, we derive an analytical lower bound expressed as a function of the parameters $\lambda$, $B$, $\alpha$, and $\beta$, allowing a more effective threshold selection. In the subsequent discussion, for all $w \in (0,1]$ and $B \geq 0$, we denote by $S^B(w)$ the following sum:
\[
S^B(w) = \sum_{b = 0}^B \left( \frac{1}{w} - \sum_{\ell = 0}^b \frac{\log(1/w)^\ell}{\ell !} \right)\;.
\]

\begin{corollary}\label{cor:lb2grps}
Assume that $\lambda \geq 1/2$. Let
$
h^B: \beta \mapsto \min \left\{ \frac{\beta}{e} \exp\left(\frac{\beta S^B(\beta)}{\lambda}  \right)\, , \, \beta  \right\}
$,
and $\tilde{\alpha}_B, \tilde{\beta}_B$ the thresholds defined as $\tilde{\alpha}_B = h^B(\tilde{\beta}_B)$, and $\tilde{\beta}_B$ minimizing the mapping
\[
\beta \in [0,1] \mapsto \lambda h(\beta) \log\big(\tfrac{\beta}{h^B(\beta)}\big) + h^B(\beta) \beta S^B(\beta)\;,
\]
then the success probability of $\A^B(\tilde{\alpha}_B,\tilde{\beta}_B)$ satisfies
\[
\lim_{N \to \infty} \Pr(\A^B(\tilde{\alpha}_B,\tilde{\beta}_B) \text{ succeeds}) 
\geq \frac{1}{e} - \min\left\{ \frac{1}{e(B+1)!}, (\tfrac{4}{e}-1)\lambda(1-\lambda) \right\}\;.
\]
\end{corollary}



%If $\lambda < 1/2$, then by symmetry, choosing adequate thresholds yields the same lower bound on the asymptotic success probability of $\A^B(\alpha, \beta)$.
Therefore, in contrast to the single-threshold algorithm, the asymptotic success probability of $\A^B(\tilde{\alpha}_B,\tilde{\beta}_B)$ approaches $1/e$ both when the budget increases and when $\lambda$ approaches $0$ or $1$.



\subsection{Optimal memory-less algorithm for two groups}\label{sec:opt-dynprog-main}

In the following, an algorithm is called \textit{memory-less} if its actions at any step $t \in [N]$ depend only on the current observations $r_t, g_t, \indic{R_t = 1}$, the available budget $B_t$, and the cardinals $(|G^k_{t-1}|)_{k \in [K]}$. 


  We use in this section a dynamic programming approach to determine the optimal memory-less algorithm, which we denote by $\A_*$.
  
  Unlike previous sections, our analysis here is not asymptotic. By meticulously examining how various variables, including the precise number of candidates observed in each group, influence the success probability of $\A_*$, we rigorously analyze its state transitions and corresponding success probabilities to determine optimal actions at each step.
A full description and analysis of the optimal memory-less algorithm can be found in Section \ref{sec:opt-dynprog}. Here, we illustrate its actions through Figure \ref{fig:DPacceptance}. 

\begin{figure}[h!]
    \centering
    \includegraphics[width=0.7\linewidth]{figs/DPacceptanceRegions.pdf} % 
    \caption{Acceptance region of $\A_*$}
    \label{fig:DPacceptance}
\end{figure}
    

\paragraph{Computing optimal thresholds for the DT algorithm.}
Upon observing $(r_t,g_t)$, $\A_*$ makes a decision to accept or reject, where acceptance means $\astop$ if $B=0$ and $\acomp$ otherwise, depending on $t, |G^1_t|, B, g_t$. Figure \ref{fig:DPacceptance} shows its acceptance region (dark green), with $N = 500$, $\lambda = 0.7$, for $B \in \{0,1,2\}$ and for all possible values of $t\in [N]$, $|G^1_t| \leq t$, and $g_t \in \{1,2\}$.
The x- and y-axes display respectively the step $t$ and possible group cardinal $|G_t^{1}|$, which implies $|G_t^{2}|$, up to time $t$. The latter follows a binomial distribution with parameters $(\lambda, t)$, which tightly concentrates around its mean $|G_t^{1}| \approx \lambda t$ (and $|G_t^{2}| \approx (1-\lambda) t$) even for moderate values of $t$. Consequently, when $N$ is large, $|G^1_t| \approx\lambda t$, and the acceptance region is solely defined by a threshold at the intersection of the acceptance region and the line $|G^1_t| \approx\lambda t$.
This observation implies that $\A_*$ behaves as an instance of $\DT$ algorithms when the number of candidates is large. The corresponding thresholds, which we denote by $(\alpha_b^\star, \beta_b^\star)_{b \leq B}$, are necessarily optimal, and can be estimated as the intersection of the acceptance region for $G^k$ and the line $(t,\lambda t)$ for $k \in \{1,2\}$.




\paragraph{Alternative comparison model.} In the particular case of two groups, both comparison models introduced in Section \ref{sec:setup} are equivalent, as freely comparing a candidate with the best in their group and then making one costly comparison with the best candidate from the other group is sufficient to determine if they are the best so far. Therefore, all the results of the current section regarding the static double-threshold algorithm and optimal memory-less algorithm remain true in the alternative comparison model.







\section{Optimal memory-less algorithm for two groups}\label{sec:opt-dynprog}

In this section, we derive an optimal memoryless algorithm employing a dynamic programming approach. We analyze the state transitions depending on the algorithm's actions and the associated success probabilities for each state. Unlike previous sections, our study here is not asymptotic. Therefore, we do not rely on estimating the number of candidates in each group using concentration inequalities. Instead, we consider the exact number of candidates in each group as a parameter for decision-making at each step.

We recall that memoryless algorithms, at any given step $t$, have access to the available budget $B_t$ and the number of previous candidates belonging to each group. In the case of two groups, this information reduces to $(t, B_t, |G^1_{t-1}|)$, since $|G^2_{t-1}| = t-1 - |G^1_{t-1}|$.
The state of the algorithm, which fully determines its success probability, is given by the tuple $(t, B_t, |G^1_{t-1}|, g^*_{t-1})$. However, $g^*_{t-1}$ is not known to the algorithm, hence it must make decisions relying on the limited information it has, to maximize the expected success probability, where the expectation is taken over $g^*_{t-1}$.



\subsection{State transitions}\label{sec:state-transition}
For any memory-less algorithm $\A$, we denote by $\S_t(\A)$ its state at step $t$, which is a tuple $(t, b, m, \ell)$. Here, $t-1$ represents the count of previously rejected candidates, $b \geq 0$ denotes the available budget, $m < t$ indicates the number of prior candidates from group $G^1$, and $\ell \in \{1,2\}$ is the group containing the best-seen candidate so far.

To examine the state transitions of the algorithm, it is imperative to first understand the distribution of the new observations at any given step $t$, depending on $\S_t(\A)$. While the group membership $g_t$ of candidate $x_t$ is independent of $\S_t(\A)$, both $r_t$ and $R_t$ are contingent on it.



\begin{lemma}\label{lem:dynprog-rt-Rt}
For any memory-less algorithm $\A$ and state $(t,m,b,\ell)$, denoting by $k = 3-\ell$ the group index different from $\ell$, it holds that
\begin{align*}
&\Pr(r_t = 1 \mid \S_t(\A) = (t,m,b,\ell), g_t = \ell) 
= \frac{1}{t}\\
&\Pr(r_t = 1 \mid \S_t(\A) = (t,m,b,\ell), g_t = k) 
= \frac{|G^k_{t-1}|+t}{t(|G^k_{t-1}|+1)}\\
&\Pr(R_t = 1 \mid \S_t(\A) = (t,m,b,\ell), g_t = \ell, r_t = 1)
= 1\\
&\Pr(R_t = 1 \mid \S_t(\A) = (t,m,b,\ell), g_t = k, r_t=1) 
= \frac{|G^k_{t-1}|+1}{|G^k_{t-1}|+t}\;,
\end{align*}
where
\[
|G^k_{t-1}| = \left\{
    \begin{array}{ll}
        m & \mbox{if }\; k = 1 \\
        t-1-m & \mbox{if }\; k = 2 \;.
    \end{array}
\right.
\]
\end{lemma}




Using the previous Lemma, we can fully characterize the possible state transitions of a memory-less algorithm.
First, the values of the parameters $|G^1_{t}|$ and $B_{t+1}$ are trivially determined based on the state $S_t$ at the beginning of step $t$, the observations $r_t$ and $g_t$, and the actions of the algorithm:
\[
|G^1_{t}| = |G^1_{t-1}| + \indic{g_t = 1}\;,
\quad B_{t+1} = B_t - \indic{\act_{t,1} = \acomp}\;,
\]
where $\act_{t,1}$ is the action taken by the algorithm, which only depends on the state $S_t$ since the algorithm is memory-less.

Regarding $g^*_t$, if $g_t = g^*_{t-1}$, then $g^*_t = g^*_{t-1}$ remains unchanged with probability $1$. However, if $g_t \neq g^*_{t-1}$ and $r_t = 1$, and if the algorithm skips the candidate without making a comparison, then $g^*_t$ is not deterministic based on the history alone. The probability that $g^*_t = g_t$ in this case is precisely the probability that $R_t = 1$, computed in Lemma \ref{lem:dynprog-rt-Rt}
\[
\Pr(g^*_t = g_t \mid \S_t(\A) = (t,m,b,\ell), g_t = k, r_t=1) 
= \frac{|G^k_{t-1}|+1}{|G^k_{t-1}|+t}\;.
\]





\subsection{Expected action rewards}\label{sec:expected-rwd}
In the following, we denote by $\A_*$ the optimal memory-less algorithm for two groups, and for all $B \geq 0$, $t \in [N]$, $m < t$ and $\ell \in \{1,2\}$, we denote by
\[
\V^B_{t,m,\ell} = \Pr\big(\A_* \text{ succeeds} \mid \tau \geq t, \S_t(\A_*) = (t,B,m,\ell) \big)\;,
\]
which is its success probability starting from state $(t,B,m,\ell)$.








We analyze the expected rewards and state transitions of algorithm $\A_*$ given its limited information access.
When the algorithm receives a new observation $(r_t, g_t)$:
\begin{itemize}
    \item If $r_t \neq 1$, the optimal action is to skip the candidate ($\askip$).
    \item If $r_t = 1$ and $B_t = 0$, the algorithm either stops or skips the candidate. However, if there is a positive budget $B_t$, stopping is suboptimal: it is always better to make a comparison first.
    \item If the algorithm chooses to make a comparison and observes $R_t$:
    \begin{itemize}
        \item If $R_t \neq 1$, the optimal action is to skip the candidate.
        \item If $R_t = 1$, the algorithm must decide whether to skip or stop. However, skipping after observing $R_t = 1$ is suboptimal compared to skipping immediately after observing $r_t = 1$, as the latter conserves the budget.
    \end{itemize}
\end{itemize}





In summary, any rational algorithm follows these decision rules:
\begin{itemize}
    \item If ($r_t \neq 1$) or ($r_t = 1$ and $R_t \neq 1$), then skip the candidate.
    \item If ($r_t = 1$ and $R_t = 1$), select the candidate.
\end{itemize}

Therefore, the main non-trivial decision to make is whether to reject or accept a candidate after observing $r_t = 1$.
Consider an algorithm $\A$ following these rules. At time $t$ with budget $B_t = b$ and $|G^1_{t-1}| = m$, if $g_t = k$ and $r_t = 1$, choosing an action $\act \in \{\askip, \astop, \acomp\}$ based on these rules leads to a new state $\S_{t+1}(\A) = F(t, b, m, k, \act)$, which is a random variable depending on $g^*_{t-1}$ and $R_t$. If $\act \in \{ \astop, \acomp\}$ and $R_t = 1$, then $\S_{t+1}(\A)$ is a final state: success or failure.



With this notation, we define $\rwd^B_{t,m}(\act)$ as the reward that $\A_*$ expects to gain by playing action $\act$ after observing $r_t = 1$ and $g_t = k$ in a state $S_t(\A_*) = (t,b,m,\cdot)$, where it ignores $g^*_{t-1}$
\[
\rwd^B_{t,m,k}(\act)
= \E[\Pr\big(\A_* \text{ succeeds} \mid \S_{t+1}(\A_*) = F(t,B,m,k,\act) \big) \mid S_t(\A_*) = (t,B,m,\cdot), r_t=1, g_t=k]\;.
\]
where the expectation is taken over $g^*_{t-1}$ and $R_t$.
The optimal memory-less action at any state $(t,B,m,\ell)$, knowing that $r_t=1, g_t=k$, is the one maximizing $\rwd^B_{t,m,k}(\act)$. 


\begin{lemma}\label{lem:action-reward}
Consider a state $S_t = (t,B,m,\cdot)$, and let $\{k, \ell\} = \{1,2\}$, $M_k = m + \indic{k = 1}$, then
\begin{align*}
&\rwd^B_{t,m,k}(\astop) = \frac{|G^k_t|}{N}\;,\\
&\rwd^B_{t,m,k}(\askip) = \frac{|G^1_t|}{t} \V^B_{t+1,M_k,1} + \frac{|G^2_t|}{t} \V^B_{t+1,M,2}\;,\\
&\rwd^B_{t,m,k}(\acomp) = \frac{|G^k_t|}{N} + \frac{|G^\ell_t|}{t}\left( \frac{|G^k_{t-1}|+1}{|G^k_{t-1}|+t}\cdot \frac{t}{N} + \frac{t-1}{|G^k_{t-1}|+t} \V^{B-1}_{t+1,M_k,\ell} \right)\;,
\end{align*}
\end{lemma}

Observe that, conditionally to $g_t$ and $|G^1_{t-1}|$, the cardinals of $G^1_{t-1}, G^2_{t-1}, G^1_{t}, G^2_{t}$ are all known: 
\[
|G^1_{t}| = |G^1_{t-1}| + \indic{g_t=1}\;,
\quad |G^2_{t}| = t - |G^1_{t}|, 
\quad |G^2_{t-1}| = t - 1 - |G^1_{t-1}|\;. 
\]



\subsection{Optimal actions and success probability}
Using Lemma \ref{lem:action-reward} and considering the potential state transitions based on the actions, we establish a recursion satisfied by $(\V^B_{t,m,\ell})_{t,B,m,\ell}$. We present the result without distinction between the cases $\ell = 1$ and $\ell = 2$. For simplicity, let $\lambda_k = \Pr(g_t = k)$ for $k = 1,2$, and define $M_k = m + \indic{k=1}$ for all $m \geq 0$. Additionally, for all $(B,t,m,k)$, define
\[
\delta^B_k = \indic{\rwd^B_{t,m,k}(\accept) \geq \rwd^B_{t,m,k}(\askip)}\;,
\]
where the action $\accept$ corresponds to $\acomp$ for $B>0$ and $\astop$ for $B=0$.

\begin{theorem}\label{thm:opt-memless}
For all $t \in [N]$, $m < t$ and $\{k,\ell\} = \{1,2\}$, the success probability of $\A_*$ with zero budget satisfies the recursion
\begin{align*}
\V^0_{t,m,\ell}
&= \lambda_\ell \left(\tfrac{\delta^0_\ell}{N} + \left(1 - \tfrac{\delta^0_\ell}{t}\right)\V^0_{t+1,M_\ell,\ell} \right) 
+ \lambda_k \left( \tfrac{\delta^0_k}{N} + \tfrac{1-\delta^0_k}{t} \V^0_{t+1,M_k,k} + \left(1 - \tfrac{1}{t}\right)\left(2 - \delta^0_k - \tfrac{1}{|G^k_{t-1}|+1} \right) \V^0_{t+1,M_k,\ell} \right)\;,
\end{align*}
and for $B \geq 1$ it satisfies
\begin{align*}
\V^B_{t,m,\ell}
&= \lambda_\ell \left( \tfrac{\delta^B_\ell}{N} + \big( 1 - \tfrac{\delta^B_\ell}{t} \big) \V^B_{t+1,M_\ell,\ell} \right) \\
&\quad+ \lambda_k \left( \tfrac{\delta^B_k}{N} + \tfrac{\delta^B_k}{|G^k_{t-1}|+1}\big( 1 - \tfrac{1}{t}\big) \V^{B-1}_{t+1,M_k,\ell} + \tfrac{1-\delta^B_k}{t} \V^{B}_{t+1,M_k,k} + \big(1-\tfrac{1}{t}\big)\big( 1 - \tfrac{\delta^B_k}{|G^k_{t-1}|+1}\big)  \V^{B}_{t+1,M_k,\ell}  \right)\;,
\end{align*}
where $\V^B_{N+1,m,k} = 0$ for all $B \geq 0$ $m \leq N$ and $k \in \{1,2\}$.
\end{theorem}







Implementing the optimal memory-less algorithm $\A_*$ with budget $B$ requires knowing the 
$(\rwd^b_{t,m,k}(\act))_{t,b,m,k}$ for $\act \in \{\askip, \astop, \acomp\}$, which depend themselves on the table $\big(\V^b_{t,m,k}\big)_{t,b,m,k}$.
Using Lemma \ref{lem:action-reward} and Theorem \ref{thm:opt-memless}, these tables can be computed in a $O(B N^2)$ time as described in Algorithm \ref{algo:computeTables}.


\begin{algorithm}[h!]
\DontPrintSemicolon 
\caption{Computing $(\V^b_{t,m,k})_{t,b,m,k}$ and $(\rwd^b_{t,m,k}(\act))_{t,b,m,k}$}\label{algo:computeTables}
\SetKwInput{Input}{Input}
   \SetKwInOut{Output}{Output}
   \SetKwInput{Initialization}{Initialization}
   \Input{Number of candidates $N$, available budget $B$, probability distribution of $g_t$: $\lambda_1, \lambda_2$}
   \Initialization{$\V^b_{N+1,m,k} \gets 0$ for all $b \leq B , m \leq N, k \in \{1,2\} $}
\For{$b = 1, \ldots, B$}{
    \For{$t=N, N-1,\ldots,1$}{
        \For{$m = 0, \ldots, t$}{
            Compute $\rwd^b_{t,m,k}(\act)$ for $k \in \{1,2\}$ and $\act \in \{\askip, \astop, \acomp\}$ using Lemma \ref{lem:action-reward}\;
            Compute $\V^b_{t,m,k}$ for $k \in \{1,2\}$ using Theorem \ref{thm:opt-memless}\;
        }
    }
}
Return: $(\V^b_{t,m,k})_{t,b,m,k}$, $(\rwd^b_{t,m,k}(\act))_{t,b,m,k}$\;
\end{algorithm}



After computing these tables, the optimal memory-less algorithm $\A_*$ can be implemented by following the rational decision rules outlined in Section \ref{sec:expected-rwd}, and when encountering $r_t = 1$ and needing to choose between accepting or rejecting the candidate, $\A_*$ selects the action that maximizes its expected reward given the information it has about the current state. A detailed description is provided in Algorithm \ref{algo:opt-memless}.

\begin{algorithm}[h!]
\DontPrintSemicolon 
\caption{Optimal memory-less algorithm $\A_*$}\label{algo:opt-memless}
\SetKwInput{Input}{Input}
   \SetKwInOut{Output}{Output}
   \SetKwInput{Initialization}{Initialization}
   \Input{Number of candidates $N$, available budget $B$, probability distribution of $g_t$: $\lambda_1, \lambda_2$}
   \Initialization{$b \gets B$, $m \gets 0$}
   Compute $(\V^b_{t,m,k})_{t,b,m,k}$ and $(\rwd^b_{t,m,k}(\act))_{t,b,m,k,\act}$ using Algorithm \ref{algo:computeTables}\;
\For{$t = 1,\ldots, N$}{
    Receive new observation $(r_t, g_t)$\;
    \If{$r_t = 1$}{
        \If{$b=0$ and $\rwd^0_{t,m,g_t}(\astop) > \rwd^0_{t,m,g_t}(\askip)$}{
        Return: $t$\;
        }
        \If{$b>0$ and $\rwd^b_{t,m,g_t}(\acomp) > \rwd^b_{t,m,g_t}(\askip)$}{
            $b \gets b - 1$\;
            \If{$R_t = 1$}{Return: t}
        }
    }
    $m \gets m + \indic{g_t = 1}$\;
}
\end{algorithm}



Although the structure of Algorithm $\A_*$ is complicated, the discussion, following Figure \ref{fig:DPacceptance}n in the subsequent section provides additional insights into its behavior, offering a visual and simpler understanding of its decision-making rule.


%$b \gets b - 1$\;
%        \If{$R_t = 1$}{Return: t}

\section{Numerical experiments}\label{sec:experiments}

In this section, we confirm our theoretical findings via numerical experiments, and we give further insight regarding the behavior of the algorithms we presented and how they compare to each other. In all the empirical experiments of this section, each point is computed over $10^6$ independent trials.

\subsection{Single-threshold algorithm}
We begin our experimental analysis by examining the single-threshold algorithm, studied in Section \ref{sec:single-thresh}. Its asymptotic success probability, provided in Theorem \ref{thm:single-thresh}, depends on the number $K$ of distinct groups, the budget $B$, and the threshold $\w \in [0,1]$. The optimal threshold, maximizing the success probability, can be computed numerically for fixed $K$ and $B$.

\begin{figure}[h!]
    \centering
    \includegraphics[width=0.4\textwidth]{figs/STthresh_KvsB.pdf} \quad \includegraphics[width=0.4\textwidth]{figs/STproba_KvsB.pdf}
    \caption{Single threshold algorithm: optimal threshold and corresponding success probability}
    \label{fig:ST_KB}
\end{figure}

Figure \ref{fig:ST_KB} presents the optimal threshold for $B \in \{0,30\}$ and $K \in \{2,10,25,50\}$. For any $K \geq 2$, as the budget grows to infinity, the problem becomes akin to the standard secretary problem, leading the optimal threshold to converge to $1/e$. However, as discussed following Corollary \ref{cor:single-thresh-factorial-conv}, this convergence is slower when the number of groups $K$ is higher.


Moreover, Theorem \ref{thm:single-thresh} reveals that the asymptotic success probability is independent of the probabilities of belonging to each group, and it is equal to a value smaller than $1/e$. This indicates a discontinuity of the asymptotic success probability at the extreme points of the polygon defining the possible values of $(\lambda_k)_{k \in [K]}$. Figure \ref{fig:ST_discontinuity} illustrates this behavior for the case of two groups, with $N=500$ candidates.

\begin{figure}[h!]
    \centering
    \includegraphics[width=0.5\textwidth]{figs/ST_lmb_discontinuity.pdf} 
    \caption{Single threshold: success probability for $2$ groups, with $N = 500$ and $\lambda = \Pr(g_t = 1) \in [0,1]$}
    \label{fig:ST_discontinuity}
\end{figure}


The success probability we proved in Theorem \ref{thm:single-thresh} is only asymptotic, it is reached when the number of candidates is very high. Moreover, from Lemma \ref{lem:single-thres-recursion} and from the proof of the theorem, it can be deduced that the difference between the success probability for a given $N$ and the limit is $O\big(\sqrt{\tfrac{\log N}{N}}\big)$. However, this does not comprehend how the success probability varies with the number of candidates. Figure \ref{fig:ST_Ngrows} shows that the success probability is actually better when the number of candidates is small, and it decreases to match the asymptotic expression when $N \to \infty$, represented with dotted lines for $K \in \{2,3,4\}$. 


\begin{figure}[h!]
    \centering
    \includegraphics[width=0.5\textwidth]{figs/ST_Ngrows.pdf} 
    \caption{Convergence to the asymptotic success probability, with $\lambda_k = 1/K$ for all $k \in [K]$}
    \label{fig:ST_Ngrows}
\end{figure}







\subsection{The case of two groups}
Moving to the case of two groups, Theorem \ref{thm:opt-memless} shows a recursive formula for computing the success probability of the optimal memory-less algorithm $\A_*$ for all $N \geq 1$, $B \geq 0$, and $\lambda \in (0,1)$. Figure \ref{fig:DP-success-lb} displays this success probability for $N=500$ and $B \in \{0,1,2\}$, along with the success probability of the double threshold algorithm $\A^B(\Tilde{\alpha}_B, \Tilde{\beta}_B)$, where $\Tilde{\alpha}_B, \Tilde{\beta}_B$ are defined in Corollary \ref{cor:lb2grps}.


\begin{figure}[h!]
  \centering
    \includegraphics[width=0.6\textwidth]{figs/lowerBoundCor2grps.pdf}
    \caption{Success probability of the optimal memory-less algorithm, and the lower bound of Corollary \ref{cor:lb2grps}}\label{fig:DP-success-lb}
\end{figure}

The figure demonstrates that for $B = 0$, or for $\lambda = 0.5$, Algorithm $\A^B(\Tilde{\alpha}_B, \Tilde{\beta}_B)$ matches the performance of $\A_*$ closely, despite having a much simpler structure. 

We recall that $\A_*$ has access to the available budget $B$, and the number of previous candidates observed in each group, which reduces to knowing $|G^1_t|$ in the case of two groups. Upon observing $(r_t,g_t)$, it makes a decision to accept or reject, where acceptance means $\astop$ if $B=0$ and $\acomp$ otherwise, depending on $t, |G^1_t|, B, g_t$. Figure \ref{fig:DPacceptance} shows the acceptance region (dark green) of the optimal algorithm, with $N = 500$, $\lambda = 0.7$, for $B \in \{0,1,2\}$ and for all possible values of $t\in [N]$, $|G^1_t| \leq t$, and $g_t \in \{1,2\}$.


\begin{figure}[h!]
    \centering
    \includegraphics[width=0.7\textwidth]{figs/DPacceptanceRegions.pdf}
    \caption{Acceptance region of $\A_*$ depending on the budget}
    \label{fig:DPacceptance}
\end{figure}


The x- and y-axes display the step $t$ and possible group cardinal $|G_t^{1 / 2}|$ up to time $t$ respectively. The latter follows a binomial random distribution with parameters $(\lambda, t)$ and tightly clusters around its mean $|G_t^{1}| \approx \lambda t$ (and $|G_t^{2}| \approx (1-\lambda) t$) even for moderate values of $t$. Consequently, when $N$ is large, $|G^1_t| \approx\lambda t$, and the acceptance region is solely defined by a threshold at the intersection of the acceptance region and the line $|G^1_t| \approx\lambda t$.
This observation directly implies that $\A_*$ behaves as an instance of $\DT$ algorithms. The optimal $\DT$ thresholds for any $\lambda$, $B \geq 0$, and $g \in \{1,2\}$ can therefore be estimated as the intersection of the acceptance region for $G^g$ and the line $(t,\lambda t)$.



\begin{figure}[h!]
  \centering
    \includegraphics[width=0.48\textwidth]{figs/threshAlpha.pdf}
    \hfill
    \includegraphics[width=0.48\textwidth]{figs/threshBeta.pdf}
    \caption{Evolution of $(\alpha_b^\star, \beta_b^\star)$ as a function of $\lambda$.}\label{fig:threhsolds_alphas_betas}
\end{figure}



Figure \ref{fig:threhsolds_alphas_betas} shows the thresholds computed by this method, for all $\lambda \in [0.5,1]$ and $B \in \{0,1,2\}$. The thresholds $\alpha_b^\star, \beta_b^\star$ are continuous functions of $\lambda$, both converging to $1/e$ very rapidly as the budget increases. Indeed, for $B \geq 1$, they both become very close to $1/e$, as for $B = 0$, the optimal thresholds are exactly equal to $\alpha^\star_0 = \lambda \exp(\tfrac{1}{\lambda}-2)$ and $\beta^\star_0 = \lambda$, which correspond to the optimal thresholds described in Corollary \ref{cor:lb2grps} for $B = 0$ (See the proof of the corollary). Another remark is that when $\lambda$ approaches $1$, the thresholds $\alpha^\star_B$ for $G^1$ converge to $1/e$, as the problem again becomes equivalent to the standard secretary problem.



Figure \ref{fig:DPvsDDT} compares the empirical success probabilities of both $\A_*$ and the $\DT$ algorithm with thresholds $(\alpha^\star_B,\beta^\star_B)_{B \geq 0}$.

\begin{figure}[h!]
  \centering
    \includegraphics[width=0.32\textwidth]{figs/DPvsDDT_05.pdf}
    \hfill
    \includegraphics[width=0.32\textwidth]{figs/DPvsDDT_07.pdf}
    \hfill
    \includegraphics[width=0.32\textwidth]{figs/DPvsDDT_095.pdf}
    \caption{Empirical comparison of the success probabilities of $\A_*$ and the $\DT$ algorithm with optimal thresholds}\label{fig:DPvsDDT}
\end{figure}


For $\lambda \in \{0.5, 0.7, 0.95\}$, the $\DT$ algorithm achieves a success probability very close to that of $\A_*$. While there may be slight differences for small values of $N$ when $B = 0$, the curves almost perfectly align for $B \geq 1$ and for larger values of $N$. This observation confirms that, despite the intricate structure of the optimal memory-less algorithm, in most scenarios, it does not surpass the performance of the $\DT$ algorithm with optimal thresholds. Nonetheless, the analysis of the optimal memory-less algorithm is what enables the numerical computation of the optimal thresholds, as explained previously.











\section{Conclusion}
\label{sec:conclusion}

In this paper, we investigated the problem of aligning instructional videos with a high-level schematic representation of the task, depicted by abstract instructional diagrams showing the steps in the process.
We proposed a method based on contrastive learning to align video and diagram features using three novel losses designed specifically for this task.
Our focus is on Ikea furniture assembly where alignment is done between in-the-wild videos and the corresponding official assembly manuals.
To this end, we also collected a dataset of 183 hours of in-the-wild assembly videos and nearly 8,300 diagrams.
Two tasks are designed on this dataset to evaluate the performance of our method: (i) a nearest neighbor retrieval task between video clips and instructional diagrams, (ii) alignment of the instruction diagrams to their corresponding assembly video clips.
On both tasks, experimental results show that our proposed sinusoidal progress rate feature and optimal transport modules lead to better temporal alignment and each one of the proposed losses enables the model to learn better representations, compared with compelling alternatives that do not take into account the unique nature of the problem.

Our work suggests several directions for future work.
First, it would be interesting to consider including additional modalities such as video narrations into our framework.
Second, extending the task to unsupervised or weakly supervised settings would overcome our current limitation of requiring ground truth alignments for learning.
Last, an ambitious long-term goal is to develop applications, built on our alignment model, that automatically monitor and guide a user through an assembly process or facilitate robot-human collaboration on instructional tasks.









% In the interest of anonymization, please do not include acknowledgements in your submission.
%
%\begin{acks}
%
%	The authors would like to thank Dr. Maura Turolla of Telecom
%	Italia for providing specifications about the application scenario.
%
%	The work is supported by the \grantsponsor{GS501100001809}{National
%		Natural Science Foundation of
%		China}{http://dx.doi.org/10.13039/501100001809} under Grant
%	No.:~\grantnum{GS501100001809}{61273304\_a}
%	and~\grantnum[http://www.nnsf.cn/youngscientsts]{GS501100001809}{Young
%		Scientsts' Support Program}.
%
%
%\end{acks}

% Bibliography
\bibliographystyle{plainnat}
\bibliography{bibliography}

% Appendix
\appendix

\section{Appendix for Proofs}

\paragraph{Proof of Theorem \ref{thm:main}.}

\begin{proof}
\label{proof:main}
Our proof has two steps. In Step 1, we will show that SimCLR is equivalent to minimizing the cross entropy loss defined in Eqn.~(\ref{eqn:cross-entropy}). 
In Step 2, we will show  that minimizing the cross-entropy loss 
is equivalent to spectral clustering on $\bfpi$. 
Combining the two steps together, we have proved our theorem. 

\textbf{Step 1: } SimCLR is equivalent to minimizing the cross entropy loss.

The cross-entropy loss takes expectation over 
$\bfW_\bfX\sim \mathbb{P}(\cdot ; \bfpi)$, 
which means $\bfW_\bfX$ has exactly one non-zero entry in each row $i$. By Lemma~\ref{lem:multinomial}, we know every row $i$ of $\bfW_\bfX$ is independent of other rows. Moreover, 
$\bfW_{\bfX,i}\sim \mathcal{M}(1, \bfpi_i/\sum_j \bfpi_{i,j})=\mathcal{M}(1, \bfpi_i)$, because $\bfpi_i$ itself is a probability distribution.
Similarly, we know $\bfW_\bfZ$ also has the row-independent property by sampling over $\mathbb{P}(\cdot;\bfK_\bfZ)$.
Therefore, by Lemma~\ref{lem:cross_split}, we know Eqn.~(\ref{eqn:cross-entropy}) is equivalent to:
\[
 -\sum_{i=1}^n \mathbb{E}_{\bfW_{\bfX,i}}[\log \mathbb{P}(\bfW_{\bfZ,i}=\bfW_{\bfX,i};\bfK_\bfZ)],
\]

This expression takes expectation over $\bfW_{\bfX,i}$ for the given row $i$. Notice that 
$\bfW_{\bfX,i}$ has exactly one non-zero entry, which equals $1$ (same for $\bfW_{\bfZ,i}$). 
As a result
we expand the above expression to be:
\begin{equation}
 -\sum_{i=1}^n \sum_{j\neq i} \Pr(\bfW_{\bfX,i,j}=1)\log \Pr(\bfW_{\bfZ,i,j}=1).
\label{eqn:detailed-expansion}    
\end{equation}


By Lemma~\ref{lem:multinomial}, $\Pr(\bfW_{\bfZ,i,j}=1)=\bfK_{\bfZ,i,j}/\|\bfK_{\bfZ,i}\|_1$ for $j\neq i$. Recall that $\bfK_\bfZ=(k(\bfZ_i-\bfZ_j))_{(i,j)\in[n]^2}$, which means 
$\bfK_{\bfZ,i,j}/\|\bfK_{\bfZ,i}\|_1=\frac{\exp(-\|\bfZ_i-\bfZ_j\|^2/{2\tau})}{\sum_{k\neq i}
\exp(-\|\bfZ_i-\bfZ_k\|^2/{2\tau})
}$ for $j\neq i$, when $k$ is the Gaussian kernel with variance $\tau$. 

Notice that $\bfZ_i=f(\bfX_i)$, so we know
\begin{equation}
-\log \Pr(\bfW_{\bfZ,i,j}=1)=
-\log \frac{\exp(-\|f(\bfX_i)-f(\bfX_j)\|^2/{2\tau})}{\sum_{k\neq i}
\exp(-\|f(\bfX_i)-f(\bfX_k)\|^2/{2\tau}),
}
\label{eqn:infonce-equivalence}    
\end{equation}


The right hand side is exactly the InfoNCE loss defined in Eqn.~(\ref{eqn:infonce}).
Inserting Eqn.~(\ref{eqn:infonce-equivalence}) into Eqn.~(\ref{eqn:detailed-expansion}), we get the SimCLR algorithm, which first samples augmentation pairs $(i,j)$ with $\Pr(\bfW_{\bfX,i,j}=1)$ for each row $i$, and then optimize the InfoNCE loss. 

\textbf{Step 2: } minimizing the cross entropy loss 
is equivalent to spectral clustering on $\bfpi$.


By Lemma~\ref{lem:convert_to_spectral}, we may further convert the loss to 
\begin{equation}
\label{eqn:main-theorem-repul-attr}
\min_{\bfZ}
-\sum_{(i,j)\in [n]^2} \mathbf{P}_{i,j}
\log k (\bfZ_i-\bfZ_j)+\log \mathbf{R}(\bfZ).
\end{equation}
Since $k$ is the Gaussian kernel, this reduces to \[
\min_\bfZ \mathrm{tr}(\bfZ^\top \mathbf{L}(\bfpi) \bfZ)
+\log \mathbf{R}(\bfZ),
\]

where we use the fact that $\mathbb{E}_{\bfW_\bfX\sim \mathbb{P}(\cdot; \bfpi)}[\mathbf{L}(\bfW_\bfX)]
=\mathbf{L}(\bfpi)
$, because the Laplacian operator is linear and $
\mathbb{E}_{\bfW_\bfX\sim \mathbb{P}(\cdot; \bfpi)}(\bfW_\bfX)=\bfpi
$.
\end{proof}

\paragraph{Proof of Theorem \ref{thm:clip}.}
\begin{proof}
Since $\bfW_\bfX\sim \mathbb{P}(\cdot;\bfpi_{\mathbf{A}, \mathbf{B}})$, we know 
$\bfW_\bfX$ has exactly one non-zero entry in each row, denoting the pair that got sampled. 
A notable difference compared to the previous proof is we now have $n_\mathcal{A}+n_\mathcal{B}$ objects in our graph. CLIP deals with this by taking a mini-batch of size $2N$, 
such that $n_\mathcal{A}=n_\mathcal{B}=N$, and adding the $2N$ InfoNCE losses together. We label the objects in $\mathcal{A}$ as $[n_\mathcal{A}]$, and the objects in $\mathcal{B}$ as $\{n_\mathcal{A}+1, \cdots, n_\mathcal{A}+n_\mathcal{B}\}$. 

Notice that $\bfpi_{\mathbf{A}, \mathbf{B}}$ is a bipartite graph, so the edges of objects in $\mathcal{A}$ will only connect to object in $\mathcal{B}$ and vice versa. We can define the similarity matrix in $\cZ$ as $\bfK_\bfZ$, 
where $\bfK_\bfZ(i, j+n_\mathcal{A})=\bfK_\bfZ(j+n_\mathcal{A},i)= k(\bfZ_i-\bfZ_j)$ for $i\in [n_\mathcal{A}], j\in [n_\mathcal{B}]$, and otherwise we set $\bfK_\bfZ(i,j)=0$. 
The rest is same as the previous proof. 
\end{proof}

\paragraph{Proof of Theorem \ref{thm:exponential}.}

\begin{proof}
\label{proof:exponential}
Since the objective function consists of a linear term combined with an entropy regularization, which is a strongly concave function, the maximization problem is a convex optimization problem. Owing to the implicit constraints provided by the entropy function, the problem is equivalent to having only the equality constraint. We then introduce the Lagrangian multiplier $\lambda$ and obtain the following relaxed problem:

$$
\widetilde{E}(\boldsymbol{\alpha})=\psi_{1}-\sum_{i=1}^n \alpha_{i} \psi_{i}+\tau \sum_{i=1}^n \alpha_{i}\log \alpha_{i}+\lambda\left(\boldsymbol{\alpha}^{\top} \mathbf{1}_n-1\right).
$$

As the relaxed problem is unconstrained, taking the derivative with respect to $\alpha_{i}$ yields

$$
\frac{\partial \widetilde{E}(\boldsymbol{\alpha})}{\partial \alpha_{i}}=-\psi_{i}+\tau\left(\log \alpha_{i}+\alpha_{i} \frac{1}{\alpha_{i}}\right)+\lambda=0.
$$

Solving the above equation implies that $\alpha_{i}$ takes the form
$
\alpha_{i}=\exp \left(\frac{1}{\tau} \psi_{i}\right) \exp \left(\frac{-\lambda}{\tau}-1\right).
$ Since $\alpha_{i}$ lies on the probability simplex, the optimal $\alpha_{i}$ is explicitly given by
$
\alpha^{*}_{i}=\frac{\exp \left(\frac{1}{\tau} \psi_{i}\right)}{\sum_{i^{\prime}=1}^n \exp \left(\frac{1}{\tau} \psi_{i^{\prime}}\right)} .
$ Substituting the optimal point into the objective function, we obtain
$$
\begin{aligned}
E\left(\boldsymbol{\alpha}^*\right)  &=\psi_1-\sum_{i=1}^n \frac{\exp \left(\frac{1}{\tau} \psi_{i}\right)}{\sum_{i^{\prime}=1}^n \exp \left(\frac{1}{\tau} \psi_{i^{\prime}}\right)} \psi_{i}+\tau \sum_{i=1}^n \frac{\exp \left(\frac{1}{\tau} \psi_{i}\right)}{\sum_{i^{\prime}=1}^n \exp \left(\frac{1}{\tau} \psi_{i^{\prime}}\right)}\log \frac{\exp \left(\frac{1}{\tau} \psi_{i}\right)}{\sum_{i^{\prime}=1}^n \exp \left(\frac{1}{\tau} \psi_{i^{\prime}}\right)} \\
& =\psi_1 - \tau \log \left(\sum_{i=1}^n \exp \left(\frac{1}{\tau} \psi_{i}\right)\right).
\end{aligned}
$$
Thus, the Lagrangian dual function is given by
\begin{equation*}
-E\left(\boldsymbol{\alpha}^*\right)= -\tau \log \frac{\exp \left(\frac{1}{\tau} \psi_{1}\right)}{\sum_{i=1}^n \exp \left(\frac{1}{\tau} \psi_{i}\right)}.\qedhere
\end{equation*}
\end{proof}



\section{More on Experiments} \label{section: experiment_details}

\paragraph{CIFAR-10 and CIFAR-100} CIFAR-10 ~\citep{krizhevsky2009learning} and CIFAR-100 ~\citep{krizhevsky2009learning} are well-known classic image classification datasets. Both CIFAR-10 and CIFAR-100 contain a total of 60k $32 \times 32$ labeled images of different classes, with 50k for training and 10k for testing. CIFAR-10 is similar to CIFAR-100, except there are 10 different classes in CIFAR-10 and 100 classes in CIFAR-100.

\paragraph{TinyImageNet} TinyImageNet ~\citep{le2015tiny} is a subset of ImageNet ~\citep{deng2009imagenet}. There are 200 different object classes in TinyImageNet, with 500 training images, 50 validation images, and 50 test images for each class. All the images in TinyImageNet are colored and labeled with a size of $64 \times 64$.

\textbf{Pseudo-code.} Algorithm \ref{alg:Training Procedure} presents the pseudo-code for our empirical training procedure.

\begin{algorithm}[!htbp]
\caption{Training Procedure}
\label{alg:Training Procedure}
\begin{algorithmic}[1]
\REQUIRE trainable encoder network $f$, batch size $N$, augmentation strategy \textit{aug}, loss function $L$ with hyperparameters \textit{args}
\FOR {sampled minibatch ${x_i}_{i=1}^N$}
\FORALL{$i \in { 1, ..., N }$}
\STATE draw two augmentations $t_i = \textit{aug}\left(x_i\right) $, $t_i' = \textit{aug}\left(x_i\right) $
\STATE $z_i = f\left(t_i\right)$, $z_i' = f\left(t_i'\right)$
\ENDFOR
\STATE compute loss $\mathcal{L} = L(N, z, z', \textit{args})$
\STATE update encoder network $f$ to minimize $\mathcal{L}$
\ENDFOR
\STATE \textbf{Return} encoder network $f$
\end{algorithmic}
\end{algorithm}

We also provide the pseudo-code for our core loss function used in the training procedure in Algorithm \ref{alg:Core loss}. The pseudo-code is almost identical to SimCLR's loss function, with the exception of an extra parameter $\gamma$.

\begin{algorithm}[!htbp]
\caption{Core loss function $\mathcal{C}$}
\label{alg:Core loss}
\begin{algorithmic}[1]
\REQUIRE batch size $N$, two encoded minibatches $z_1, z_2$, $\gamma$, temperature $\tau$
\STATE $z = \textit{concat}\left(z_1, z_2\right)$
\FOR {$i \in {1, ..., 2N }, j \in {1, ..., 2N}$ }
\STATE $s_{i,j} = \Vert z_i - z_j \Vert_2^{\gamma}$
\ENDFOR
\STATE \textbf{define} $l(i, j)$ \textbf{as} $l(i, j) = - \log \frac{exp\left(s_{i,j}/\tau \right)}{\sum_{k=1}^{2N} \mathbf{1}{[k \ne i]} exp\left(s{i, j} / \tau \right)} $
\STATE \textbf{Return} $\frac{1}{2N} \sum_{k=1}^N\left[l(i, i+N) + l(i+N, i)\right]$
\end{algorithmic}
\end{algorithm}

Utilizing the core loss function $\mathcal{C}$, we can define all kernel loss functions used in our experiments in Table \ref{table: loss definition}. For all $z_i \in z$ with even dimensions $n$, we define $z_{L_i} = z_i\left[0:n/2\right]$ and $z_{R_i} = z_i\left[n/2:n\right]$.

\begin{table}[ht]
\centering
\begin{tabular}{{@{}l|l@{}}}
Kernel  &  Loss function \\ \midrule
Laplacian & $\mathcal{C}\left(N, z, z', \gamma=1, \tau\right)$\\ \midrule
Sum       & $\lambda * \mathcal{C}\left(N, z, z', \gamma=1, \tau_1\right) + (1-\lambda) * \mathcal{C}\left(N, z, z', \gamma=2, \tau_2\right)$  \\ \midrule
Concatenation Sum&$\lambda * \mathcal{C}\left(N, z_L, z'_L, \gamma=1, \tau_1\right) + (1-\lambda) * \mathcal{C}\left(N, z_R, z'_R, \gamma=2, \tau_2\right)$\\ \midrule
$\gamma = 0.5$ & $\mathcal{C}\left(N, z, z', \gamma=0.5, \tau\right)$          \\ 

\end{tabular}

\caption{Definition of kernel loss functions in our experiments}
\label {table: loss definition}
\end{table}

\textbf{Baselines.} We reproduce the SimCLR algorithm using PyTorch Lightning~\citep{PytorchLightning}.

\textbf{Encoder details.}
The encoder $f$ consists of a backbone network and a projection network. We employ ResNet50~\citep{ResNet} as the backbone and a 2-layer MLP (connected by a batch normalization~\citep{ioffe2015batch} layer and a ReLU \cite{nair2010rectified} layer) with hidden dimensions 2048 and output dimensions 128 (or 256 in the concatenation kernel case).

\textbf{Encoder hyperparameter tuning.}
For each encoder training case, we randomly sample 500 hyperparameter groups (sample details are shown in Table \ref{table: Hyperparameter sample}) and train these samples simultaneously using Ray Tune ~\citep{RayTune}, with the ASHA scheduler~\citep{li2018massively}. Ultimately, the hyperparameter group that maximizes the online validation accuracy (integrated in PyTorch Lightning) within 5000 validation steps is chosen for the given encoder training case.

\begin{table}[ht]
\centering

\begin{tabular}{@{}l|l|l@{}}
\midrule
Hyperparameter  & Sample Range & Sample Strategy \\ \midrule
start learning rate & $\left[10^{-2}, 10\right]$ & log uniform \\ \midrule
$\lambda$       & $\left[0, 1\right]$ & uniform \\ \midrule
$\tau$, $\tau_1$, $\tau_2$ & $\left[0, 1\right]$ & log uniform \\ \midrule
\end{tabular}

\caption{Hyperparameters sample strategy}
\label {table: Hyperparameter sample}
\end{table}

\textbf{Encoder training.} 
We train each encoder using the LARS optimizer~\citep{LARSOptimizer}, LambdaLR Scheduler in PyTorch, momentum 0.9, weight decay $10^{-6}$, batch size 256, and the aforementioned hyperparameters for 400 epochs on a single A-100 GPU.

\textbf{Image transformation.} The image transformation strategy, including augmentation, is identical to the default transformation strategy provided by PyTorch Lightning.

\textbf{Linear evaluation.}
The linear head is trained using the SGD optimizer with a cosine learning rate scheduler, batch size 64, and weight decay $10^{-6}$ for 100 epochs. The learning rate starts at $0.3$ and ends at $0$.

\textbf{Moco Experiments.} We also tested our method based on MoCo~\citep{he2019moco}. The results are summarized in Table \ref{tab:results-moco}. Here we choose ResNet18~\citep{ResNet} as the backbone and set a temperature of $0.1$ as default. For our simple sum kernel, we set $\lambda=0.8$. The results show that our method outperforms the original MoCo method.

\begin{table}[thb]
\centering
\caption{MoCo Experiment Results on CIFAR-10 and CIFAR-100.}
\label{tab:results-moco}
\resizebox{\textwidth}{!}{%
\begin{tabular}{@{}c|ccc|ccc@{}}
\toprule
\multirow{3}{*}{Method} & \multicolumn{3}{c|}{CIFAR-10} & \multicolumn{3}{c}{CIFAR-100} \\ \cmidrule(lr){2-4} \cmidrule(lr){5-7} 
                        & 200 epochs & 400 epochs    & 1000 epochs   & 200 epochs & 400 epochs & 1000 epochs         \\ \midrule
MoCo (repro.)         & $76.41 \pm 0.12$    & $80.01 \pm 0.15$          & $84.45 \pm 0.08$    & $\mathbf{47.02 \pm 0.11}$ & $52.50 \pm 0.07$ & $57.62 \pm 0.15$            \\
\midrule
Laplacian Kernel        & ${78.09 \pm 0.10}$    & $\mathbf{83.85 \pm 0.09}$          & $\mathbf{88.34 \pm 0.16}$    & $46.12 \pm 0.22$   & $53.44 \pm 0.17$ & $59.10 \pm 0.14$        \\
Simple Sum Kernel & $\mathbf{78.12 \pm 0.15}$   & $83.23 \pm 0.18$ & $87.50 \pm 0.20$ & $46.65 \pm 0.06$ & $\mathbf{53.62 \pm 0.19}$ & $\mathbf{59.83 \pm 0.12}$\\
\bottomrule
\end{tabular}
}
\end{table}



\section{More Experiments on Synthetic Data}


Consider a scenario with $n$ clusters, each containing $k$ vertices. Let the probability of vertices $u$ and $v$ from the same cluster belonging to $\bfpi$ be $p$. Conversely, for vertices $u$ and $v$ from different clusters, let the probability of belonging to $\pi$ be $q$. We generate the graph $\bfpi$ randomly, based on $p$ and $q$. We experiment with values of $k=100$ and $n=6$ for ease of visualization, embedding all points in a two-dimensional space. Each vertex's initial position originates from a normal distribution. In each iteration, we sample a subgraph of $\bfpi$ uniformly, ensuring each vertex has an out-degree of $1$. We then optimize the corresponding vectors using InfoNCE loss with an SGD optimizer and iterate until convergence. Our experimental setup consists of an SGD learning rate of $1$, an InfoNCE loss temperature of $0.5$, and a batch size of $50$. We evaluate two scenarios with different $p$ and $q$ values: $p=1$, $q=0$, and $p=0.75$, $q=0.2$. The results of these experiments are visualized in Figure \ref{fig:vis-spectral-cluster}. The obtained embeddings exhibit the hallmark pattern of spectral clustering of graph $\bfpi$.

\begin{figure}[!tb]
\centering
\subfigure{
\includegraphics[width=1\textwidth]{Figures/cluster_pi.png}
\label{fig:vis-cluster}
}
\subfigure{
\includegraphics[width=1\textwidth]{Figures/noised_cluster_pi.png}
\label{fig:vis-noised-cluster}
}
\caption{Visualizations of the optimization process using InfoNCE Loss on the vectors corresponding to $\bfpi$. Points of identical color belong to the same cluster within $\bfpi$. To showcase the internal structure of $\bfpi$, we randomly select 10 vertices from each cluster to display the edge distribution of $\bfpi$.}
\label{fig:vis-spectral-cluster}
\end{figure}


\section{Dynamic-Threshold algorithms for $K$ groups}





\subsection{Proof of Lemma \ref{lem:single-thres-recursion}}

\begin{proof}
We consider that $B \geq 1$. The case $B=0$ is treated separately at the end of the proof.
Let $\C_N$ the event $(\forall k \in [K], \forall t \geq 1: ||G^k_t| - \lambda_k t| \leq 4 \sqrt{t \log N})$.
It holds that
\begin{align}
\Pr(\A^B_T \text{ succeeds} \mid \C_N)
&= \sum_{t=T}^N \sum_{k=1}^K \sum_{\ell=1}^K \Pr(\A_T^B \text{ succeeds}, \rho_1=t, g_t = \ell, g^*_{T-1} = k \mid \C_N) \nonumber\\
&= \sum_{t=T}^N \sum_{k=1}^K \sum_{\ell=1}^K \Pr(\A_T^B \text{ succeeds}, R_t = 1, \rho_1=t, g_t = \ell, g^*_{T-1} = k \mid \C_N)\nonumber\\
&\quad + \sum_{t=T}^N \sum_{k=1}^K \sum_{\ell=1}^K \Pr(\A_T^B \text{ succeeds}, R_t \neq 1, \rho_1=t, g_t = \ell, g^*_{T-1} = k \mid \C_N)\;. \label{aligneq:Rt=neq1}
\end{align}
In the following, we will estimate the terms in both sums. We recall that we consider $K$ and $(\lambda_k)_{k\in [K]}$ to be constants, and $T = \w N + o(1)$, with $\alpha$ also a constant. All the $O$ terms appearing in the proof are independent of $t$, as $T \leq t \leq N$, they only depend on $N$, $T/N = \w + o(1)$, and the constant parameters.

For $t \in \{T,\ldots,N\}$, if $\rho_1 = t$ and $R_t=1$, then the Algorithms stops on $x_t$, hence its succeeds if and only if $x_t = \xmax$. The event $x_t = \xmax$ is independent of the group membership of the candidates, thus independent of $\C_N$, and its probability is $1/N$. The event $g_t = \ell$, however, is not independent of $\C_N$, but Lemma \ref{lem:pr-cond-C_N} gives that $\Pr(g_t = \ell \mid \C_N) = \Pr(g_t = \ell) + O(1/N^2) = \lambda_\ell + O(1/N^2)$. Therefore, it holds for all $k,\ell \in [K]$ and $t \in \{T,\ldots,N\}$ that
\begin{align*}
\Pr(\A_T^B \text{ succeeds}, R_t = 1, &\rho_1=t, g_t = \ell, g^*_{T-1} = k \mid \C_N)\\
&= \Pr(x_t = \xmax, \rho_1\geq t,g_t = \ell, g^*_{T-1} = k \mid \C_N)\\
&= \Pr(x_t = \xmax)\Pr(g_t = \ell \mid \C_N) \Pr(\rho_1\geq t, g^*_{T-1} = k \mid \C_N)\\
&= \left(\frac{\lambda_\ell}{N} + O(1/N^3)\right) \Pr(\rho_1\geq t, g^*_{T-1} = k \mid \C_N)\;.
\end{align*}
Note that, for the single-threshold algorithm, we have the equivalence $\rho_1 = t \iff \rho_1 \geq t \text{ and } r_t = 1$.
The event $\rho_1 \geq t$ happens if and only if no candidate $x_s$ for $s \in \{T,\ldots,t-1\}$ in any group $m \in [K]$ exceeds the best candidate seen up to time $T-1$ in the same group:
\[
\forall t \geq T: (\rho_1 \geq t)
\iff (\forall m \in [K]: \max G^m_{T:t-1} < \max G^m_{T-1})\;,
\]
with the convention $\max \emptyset = - \infty$. Consequently, if $\rho_1 \geq t$, then $g^*_{T-1} = k$ means that the best candidate in all groups until time $t-1$ belongs to group $G^k_{T-1}$. Using that $T = \Theta(N)$, this yields
\begin{align}
\Pr(\A_T^B& \text{ succeeds}, R_t = 1, \rho_1=t, g_t = \ell, g^*_{T-1} = k \mid \C_N) \nonumber\\
&= \left(\frac{\lambda_\ell}{N} + O(1/N^3)\right) \Pr(\rho_1 \geq t \text{ and } \max x_{1:t-1} \in G^k_{T-1} \mid C_N)\nonumber\\
&= \left(\frac{\lambda_\ell}{N} + O(1/N^3)\right) \Pr(\max x_{1:t-1} \in G^k_{T-1} \mid C_N) \Pr(\rho_1 \geq t \mid \max x_{1:t-1} \in G^k_{T-1}, C_N)\nonumber\\
&= \left(\frac{\lambda_\ell}{N} + O(1/N^3)\right) \left( \frac{\lambda_k(T-1)}{t-1} + O(1/N^2) \right) \Pr(\rho_1 \geq t \mid \max x_{1:t-1} \in G^k_{T-1}, C_N)\nonumber\\
&= \left(\frac{\lambda_\ell \lambda_k T}{N t} + O(1/N^3)\right) \Pr(\forall m \in [K]\setminus\{k\}: \max G^m_{T:t-1} < \max G^m_{T-1} \mid C_N)\nonumber\\
&= \left(\frac{\lambda_\ell \lambda_k T}{N t} + O(1/N^3)\right) \prod_{m \neq k} \E\left[\tfrac{|G^k_{T-1}|}{|G^k_{t-1}|} \; \Big| \; \C_N \right]\nonumber\\
&= \left(\frac{\lambda_\ell \lambda_k T}{N t} + O(1/N^3)\right) \prod_{m \neq k} \E\left[ \frac{\lambda_m T + O(\sqrt{N\log N})}{\lambda_m t + O(\sqrt{N\log N})} \; \Big| \; \C_N \right]\nonumber\\
&= \left(\frac{\lambda_\ell \lambda_k T}{N t} + O(1/N^3)\right) \prod_{m \neq k} \left(\frac{T}{t} + O\Big( \sqrt{\tfrac{\log N}{N}}\Big) \right) \nonumber\\
&= \left(\frac{\lambda_\ell \lambda_k T}{N t} + O(1/N^3)\right) \left(\frac{T^{K-1}}{t^{K-1}} + O\Big( \sqrt{\tfrac{\log N}{N}}\Big) \right)\nonumber \\
&= \frac{\lambda_\ell \lambda_k}{N} (T/t)^K +  O\Big( \sqrt{\tfrac{\log N}{N^3}}\Big) \;. \label{aligneq:single-thresh-Rt=1}
\end{align}
On the other hand, regarding the terms of the second sum in \eqref{aligneq:Rt=neq1}, if $\rho_1 = t$ but $R_t \neq 1$, the Algorithm uses a comparison to observe $R_t$ but then skips to the next step $t+1$. The budget at step $t+1$ is thus $B-1$ and the group of the best candidate seen so far remains unchanged. Given that the single threshold algorithm is memory-less, its state at time $t+1$ is fully determined by $B-1, g^*_t$ and the number of candidates seen in each group so far, which is controlled by $\C_N$. We deduce that
\begin{align}
\Pr(\A_T^B& \text{ succeeds}, R_t \neq 1, \rho_1=t, g_t = \ell, g^*_{T-1} = k \mid \C_N) \nonumber\\  
&=\Pr(\A_T^B \text{ succeeds} \mid R_t \neq 1, \rho_1=t, g_t = \ell, g^*_{T-1} = k, \C_N)  \Pr(R_t \neq 1, \rho_1=t, g_t = \ell, g^*_{T-1} = k \mid \C_N) \nonumber\\  
&=\Pr(\A^{B-1}_{t+1} \text{ succeeds} \mid g^*_{t} = k, \C_N) \Pr(R_t \neq 1, \rho_1=t, g_t = \ell, g^*_{T-1} = k \mid \C_N)\;, \label{aligneq:single-thresh-Rneq-first}
\end{align}
where $\Pr(\A^{B-1}_{N+1} \text{ succeeds}) = 0$.
For $\ell = k$, the probability $\Pr(R_t \neq 1, \rho_1=t, g_t = \ell, g^*_{T-1} = k \mid \C_N)$ is zero, because if $g^*_{T-1} = k$, $\rho_1 \geq t$ and $g_t = \ell$, then the best candidate up to step $T-1$ belongs to group $G^k$, and no candidate $x_s$ for $s \in \{T,\ldots,t-1\}$ is better than the maximum in its group seen before step $T-1$, thus if $x_t$ belongs to $G^k$ and $r_t = 1$ then necessarily $R_t = 1$.

For $\ell \neq k$, it holds that
\begin{align*}
\Pr(R_t \neq 1,& \rho_1=t, g_t = \ell, g^*_{T-1} = k \mid \C_N)\\
&= \Pr( \rho_1=t, g_t = \ell, \max x_{1:t} \in G^k_{T-1} \mid \C_N)\\
&= \Pr(\max x_{1:t} \in G^k_{T-1} \mid \C_N) \Pr( \rho_1=t, g_t = \ell \mid \max x_{1:t} \in G^k_{T-1}, \C_N)\\
&= \left(\frac{\lambda_k(T-1)}{t-1} + O(1/N^2) \right) \Pr( \rho_1=t, g_t = \ell \mid \max x_{1:t} \in G^k_{T-1}, \C_N)\\
&= \left(\frac{\lambda_k T}{t} + O(1/N^2) \right) \Pr(g_t = \ell \mid \C_N) 
\Pr( \max G^\ell_{T:t-1} < \max G^\ell_{T-1} < x_t \\
&\hspace{3cm} \text{ and } \forall m \in [K]\setminus\{k,\ell\} : \max G^m_{T:t-1} < \max G^m_{T-1}  \mid \C_N)\\
&= \left(\frac{\lambda_k T}{t} + O(1/N^2) \right)(\lambda_\ell + O(1/N^2)) \E\left[ \tfrac{1}{|G^\ell_{t-1}|+1}\cdot\tfrac{|G^\ell_{T-1}|}{|G^\ell_{t-1}|} \; \Big| \; \C_N \right] \prod_{m \notin \{k,\ell\}} \E\left[ \tfrac{|G^m_{t-1}|}{|G^m_{T-1}|} \; \Big| \; \C_N \right]\\
&= \left(\frac{\lambda_\ell \lambda_k T}{t} + O(1/N^2) \right) \E\left[ \tfrac{1}{|G^\ell_{t-1}|+1}\cdot\tfrac{|G^\ell_{T-1}|}{|G^\ell_{t-1}|} \; \Big| \; \C_N \right] \prod_{m \notin \{k,\ell\}} \E\left[ \tfrac{|G^m_{t-1}|}{|G^m_{T-1}|} \; \Big| \; \C_N \right]\\
&= \left(\frac{\lambda_\ell \lambda_k T}{t} + O(1/N^2) \right) \left( \frac{T}{\lambda_\ell t^2} + O\Big( \sqrt{\tfrac{\log N}{N^3}}\Big) \right) \prod_{m \notin \{k,\ell\}} \left(\frac{T}{t} + O\Big( \sqrt{\tfrac{\log N}{N}}\Big)\right)\\
&= \left(\frac{\lambda_\ell \lambda_k T}{t} + O(1/N^2) \right) \left( \frac{T}{\lambda_\ell t^2} + O\Big( \sqrt{\tfrac{\log N}{N^3}}\Big) \right) \left(\frac{T^{K-2}}{t^{K-2}} + O\Big( \sqrt{\tfrac{\log N}{N}}\Big)\right)\\
&= \frac{\lambda_k T^K}{t^{K+1}} + O\Big( \sqrt{\tfrac{\log N}{N^3}}\Big)\;,
\end{align*}
where we used in the last equalities the event $\C_N$ and the assumption $T = \Theta(N)$.
Therefore, substituting into \eqref{aligneq:single-thresh-Rneq-first} gives for all $\ell, k \in [K]$ and $t \in \{T,\ldots,N\}$ that
\begin{align}
\Pr(\A_T^B& \text{ succeeds}, R_t \neq 1, \rho_1=t, g_t = \ell, g^*_{T-1} = k \mid \C_N) \nonumber \\  
&= \left( \frac{\lambda_k T^K}{t^{K+1}} + O\Big( \sqrt{\tfrac{\log N}{N^3}}\Big) \right) \indic{k \neq \ell} \Pr(\A^{B-1}_{t+1} \text{ succeeds} \mid g^*_{t} = k, \C_N)\nonumber \\
&= \left( \frac{\lambda_k T^K}{t^{K+1}} + O\Big( \sqrt{\tfrac{\log N}{N^3}}\Big) \right) \indic{k \neq \ell} \frac{\Pr(\A^{B-1}_{t+1} \text{ succeeds}, g^*_{t} = k\mid \C_N)}{\Pr(g^*_{t} = k \mid \C_N)}\nonumber\\
&= \left( \frac{\lambda_k T^K}{t^{K+1}} + O\Big( \sqrt{\tfrac{\log N}{N^3}}\Big) \right) \indic{k \neq \ell} \frac{\Pr(\A^{B-1}_{t+1} \text{ succeeds}, g^*_{t} = k\mid \C_N)}{\lambda_k + O(1/N^2)}\nonumber\\
&= \left( \frac{T^K}{t^{K+1}} + O\Big( \sqrt{\tfrac{\log N}{N^3}}\Big) \right) \indic{k \neq \ell} \Pr(\A^{B-1}_{t+1} \text{ succeeds}, g^*_{t} = k\mid \C_N) \;.\label{aligneq:single-thresh-Rtneq1}
\end{align}
Finally, substituting \eqref{aligneq:single-thresh-Rt=1} and \eqref{aligneq:single-thresh-Rtneq1} into \eqref{aligneq:Rt=neq1}, and recalling that all the previous $O$ terms are independent of $t$, gives that
\begin{align*}
\Pr(\A^B_T \text{ succeeds} \mid \C_N)
&= \sum_{t=T}^N \sum_{k=1}^K \sum_{\ell=1}^K \left(\frac{\lambda_\ell \lambda_k}{N} (T/t)^K +  O\Big( \sqrt{\tfrac{\log N}{N^3}}\Big) \right)\\
&\quad + \sum_{t=T}^N \sum_{k=1}^K \sum_{\ell \neq k}  \left( \frac{T^K}{t^{K+1}} + O\Big( \sqrt{\tfrac{\log N}{N^3}}\Big) \right) \Pr(\A^{B-1}_{t+1} \text{ succeeds}, g^*_{t} = k\mid \C_N)\\
&= \left(1 + O\Big( \sqrt{\tfrac{\log N}{N}}\Big)\right) \Big( \sum_{t=T}^N \sum_{k=1}^K \sum_{\ell=1}^K \frac{\lambda_\ell \lambda_k}{N} (T/t)^K\\
&\hspace{2.7cm} + \sum_{t=T}^N \sum_{k=1}^K \sum_{\ell \neq k}  \frac{T^K}{t^{K+1}} \Pr(\A^{B-1}_{t+1} \text{ succeeds}, g^*_{t} = k\mid \C_N) \Big)\\
&= \left(1 + O\Big( \sqrt{\tfrac{\log N}{N}}\Big)\right) \Big( \frac{T^K}{N}\sum_{t=T}^N \frac{1}{t^K}
+ (K-1) \sum_{t=T}^N \frac{T^K}{t^{K+1}}\Pr(\A^{B-1}_{t+1} \text{ succeeds}\mid \C_N) \Big)\;.
\end{align*}
Using Riemann sum properties, we obtain 
\[
\frac{T^K}{N}\sum_{t=T}^N \frac{1}{t^K}
= \frac{(T/N)^K}{N}\sum_{t=T}^N \frac{1}{(t/N)^K}
= \w^K \int_{\w}^1 \frac{du}{u^K} + O(1/N)
= \frac{\w - \w^{K}}{K-1} + O(1/N)\;,
\]
and by Lemma \ref{lem:pr-cond-C_N} we have for all $t \in \{T,\ldots,N\}$ and $b \geq 0$ that 
\[
\Pr(\A^{b}_{t} \text{ succeeds}\mid \C_N) = \Pr(\A^{b}_{t} \text{ succeeds}) + O(1/N^2)\;,
\]
with the $O(1/N^2)$ independent of $t$. Observing that $\sum_{t=T}^N \frac{T^K}{t^{K+1}} = \frac{1-\w^K}{K} + o(1) = O(1)$, it follows that
\begin{align*}
\Pr(\A^{B}_{T} \text{ succeeds})
&=  \left(1 + O\Big( \sqrt{\tfrac{\log N}{N}}\Big)\right) \Big(
\frac{\w - \w^{K}}{K-1} + (K-1) \sum_{t=T}^N \frac{T^K}{t^{K+1}}\Pr(\A^{B-1}_{t+1} \text{ succeeds}) + O(\tfrac{1}{N}) \Big)\\
&= \frac{\w - \w^{K}}{K-1} + (K-1) \sum_{t=T}^N \frac{T^K}{t^{K+1}}\Pr(\A^{B-1}_{t+1} \text{ succeeds}) + O\Big( \sqrt{\tfrac{\log N}{N}}\Big)\;.
\end{align*}
This concludes the proof for $B \geq 1$.

For $B = 0$, $\Pr(\A^{0}_{t+1} \text{ succeeds} \mid \C_N)$ can be decomposed as in \eqref{aligneq:Rt=neq1}. However, the terms of the second sum are all zero, because if $\rho_1 = t$ then the algorithm stops at $t$, but since $R_t \neq 1$, the selected candidate is not the best one, and thus the succeeding probability is $0$. All the computations regarding the first sum stay the same, and we obtain
\[
\Pr(\A^{0}_{T} \text{ succeeds}) = \frac{\w - \w^{K}}{K-1} + O\Big( \sqrt{\tfrac{\log N}{N}}\Big)\;.
\]
\end{proof}











\subsection{Proof of Theorem \ref{thm:single-thresh}}


\begin{proof}
Let $\w \in (0,1]$ a constant. For all $w\in [\w,1]$ and $B \geq 0$, we denote by $\phi^B(w)$ the limit $\lim_{N \to \infty} \Pr(\A^B_t \text{ succeeds})$ for $t = \lfloor wN \rfloor$. We will prove by induction over $B$ that this limit exists for all $w\in [\w,1]$, is equal to the expression stated in the theorem, with $u$ instead of $\w$, and satisfies $\Pr(\A^B_t \text{ succeeds}) = \phi^B(w) + O\big( \sqrt{\tfrac{\log N}{N}}\big)$, with the $O$ term only depending on $\w$ and the other constants of the problem. In particular, the $O$ term is independent of $t$.
For $B = 0$, Lemma \ref{lem:single-thres-recursion} gives immediately for any $w\in [\w,1]$ and $t = \lfloor wN \rfloor$ that
\[
\Pr(\A^0_t \text{ succeeds})
= \frac{w- w^{K}}{K-1} + O\Big( \sqrt{\tfrac{\log N}{N}}\Big) 
= \frac{w^K}{K-1}\left( \frac{1}{w^{K-1}} - 1\right) + O\Big( \sqrt{\tfrac{\log N}{N}}\Big)\;.   
\]


The $O$ term depends on $t$, but using the inequalities $\w + o(1) \leq t/N \leq 1$, it can be made only dependent on $\w$.
Let $B \geq 1$ and assume the result is true for $B-1$. Lemma \ref{lem:single-thres-recursion} and the induction hypothesis give for all $w\in [\w,1]$ and $t = \lfloor wN \rfloor$, that
\begin{align*}
\Pr(\A^{B}_{t} \text{ succeeds})
&= \frac{w- w^{K}}{K-1} + (K-1) \sum_{s=t}^N \frac{t^K}{s^{K+1}}\Pr(\A^{B-1}_{s+1} \text{ succeeds}) + O\Big( \sqrt{\tfrac{\log N}{N}}\Big)\\
&= \frac{w- w^{K}}{K-1} + (K-1) \sum_{s=t}^N \frac{t^K}{s^{K+1}}\left(\phi^{B-1}\big( \tfrac{s+1}{N} \big) + O\Big( \sqrt{\tfrac{\log N}{N}}\Big) \right) + O\Big( \sqrt{\tfrac{\log N}{N}}\Big) \\
&= \frac{w- w^{K}}{K-1} + (K-1) \frac{(t/N)^K}{N} \sum_{s=t}^N \frac{\phi^{B-1}\big( \tfrac{s+1}{N} \big)}{(s/N)^{K+1}} + O\Big( \sqrt{\tfrac{\log N}{N}}\Big)\;,
\end{align*}
where we used that the $O$ term in the induction hypothesis is independent of $s$ and that 
\[
\sum_{s=t}^N \frac{t^K}{s^{K+1}}\phi^{B-1} \big(\tfrac{s+1}{N} \big) 
\leq \sum_{s=T}^N \frac{N^K}{s^{K+1}}
\leq \frac{1}{N} \sum_{s = T}^N \frac{1}{(s/N)^{K+1}} = O(1)\;.
\]
Finally, $t/N = w + O(1/N)$, and $\phi^{B-1}$ is, by the induction hypothesis, a continuously differentiable function on $[\w,1]$, therefore, it holds by convergence properties of Riemann sums that
\begin{align*}
\Pr(\A^{B}_{t} \text{ succeeds})
&= \frac{w- w^{K}}{K-1} + (K-1) (w^K + O(\tfrac{1}{N})) \left(\int_w^1 \frac{\phi^{B-1}(u)}{u^{K+1}}du + O(\tfrac{1}{N})\right) + O\Big( \sqrt{\tfrac{\log N}{N}}\Big)\\
&= \frac{w- w^{K}}{K-1} + (K-1) w^K \int_w^1 \frac{\phi^{B-1}(u)}{u^{K+1}}du  + O\Big( \sqrt{\tfrac{\log N}{N}}\Big)\;,
\end{align*}
where the $O$ term depends on $t$ and the constant parameters. Using that $T = \lfloor \w N \rfloor \leq t \leq N$, the $O$ can be made dependent only on $\w$ and the other constant parameters. The limit $\phi^B(w) = \lim_{N \to \infty} \Pr(\A^B_{\lfloor wN \rfloor} \text{ succeeds})$ therefore exists, and is equal to
\[
\phi^B(w)
= \frac{w- w^{K}}{K-1} + (K-1) w^K \int_w^1 \frac{\phi^{B-1}(u)}{u^{K+1}}du\;.
\]
The induction hypothesis gives for all $u \in [\w, 1]$ that 
\[
\phi^{B-1}(u)
= \frac{u^K}{K-1} \sum_{b = 0}^{B-1} \left( \frac{1}{u^{K-1}} - \sum_{\ell = 0}^b \frac{\log(1/u^{K-1})^\ell}{\ell !} \right)\;,
\]
hence
\begin{align*}
(K-1) w^K \int_w^1 \frac{\phi^{B-1}(u)}{u^{K+1}}du
&= \int_w^1 \sum_{b = 0}^{B-1} \left( \frac{1}{u^{K}} - \sum_{\ell = 0}^b \frac{\log(1/u^{K-1})^\ell}{\ell ! u} \right) du\\
&= B w^K \int_w^1 \frac{du}{u^K} - w^K \sum_{b = 0}^{B-1} \sum_{\ell = 0}^b \frac{1}{\ell !} \int_w^1 \frac{\log(1/u^{K-1})^\ell}{u} du\\
&= B w^K \left[ \frac{-1}{(K-1) u^{K-1}} \right]_w^1 - w^K \sum_{b = 0}^{B-1} \sum_{\ell = 0}^b \frac{1}{\ell !} \left[-\frac{\log(1/u^{K-1})^{\ell+1}}{(K-1)(\ell+1)} \right]_w^1\\
&= \frac{B w^K}{K-1}\left( \frac{1}{w^{K-1}} - 1 \right) - w^K\sum_{b = 0}^{B-1} \sum_{\ell = 0}^b \frac{\log(1/w^{K-1})^{\ell+1}}{(K-1)(\ell+1)!}\\
&= \frac{B w^K}{K-1}\left( \frac{1}{w^{K-1}} - 1 \right) - w^K \sum_{b = 1}^{B} \sum_{\ell = 1}^b \frac{\log(1/w^{K-1})^{\ell}}{(K-1)\ell!}\\
&= \frac{w^K}{K-1} \sum_{b = 1}^{B} \left( \frac{1}{w^{K-1}} - 1 - \sum_{\ell = 1}^b \frac{\log(1/w^{K-1})^{\ell}}{\ell!}\right)\\
&= \frac{w^K}{K-1} \sum_{b = 1}^{B} \left( \frac{1}{w^{K-1}} - \sum_{\ell = 0}^b \frac{\log(1/w^{K-1})^{\ell}}{\ell!}\right)\;,
\end{align*}
and it follows that
\begin{align*}
\phi^B(w) 
&= \frac{w^K}{K-1} \left( \frac{1}{w^{K-1}} - 1 +  \sum_{b = 1}^{B} \left( \frac{1}{w^{K-1}} - \sum_{\ell = 0}^b \frac{\log(1/w^{K-1})^{\ell}}{\ell!}\right) \right)\\
&= \frac{w^K}{K-1} \sum_{b = 0}^{B} \left( \frac{1}{w^{K-1}} - \sum_{\ell = 0}^b \frac{\log(1/w^{K-1})^{\ell}}{\ell!}\right)\;.
\end{align*}
In particular, this identity is true for $w = \w$, which gives the wanted result.
\end{proof}









\subsection{Proof of Corollary \ref{cor:single-thresh-factorial-conv}}

\begin{proof}
Let $\w \in (0,1)$ and $T = \lfloor \w N \rfloor$. Lemma \ref{lem:sum-exp-remainder} gives for all $x>0$ that 
\[
\sum_{b=0}^B\left( e^x - \sum_{\ell=0}^b \frac{x^\ell}{\ell !} \right) \geq xe^x\left(1 -  \frac{x^{B+1}}{(B+1)!}\right)\;,
\]
in particular, we obtain for $x = \log(1/\w^{K-1})$ that
\begin{align*}
\lim_{N \to \infty} \Pr(\A_T^B \text{ succeeds})
&= \frac{\w^K}{K-1} \sum_{b = 0}^{\infty} \left( \frac{1}{\w^{K-1}} - \sum_{\ell = 0}^b \frac{\log(1/\w^{K-1})^{\ell}}{\ell!}\right)\\
&\geq \frac{\w^K}{K-1} \cdot \frac{\log(1/\w^{K-1})}{\w^{K-1}}\left(1 - \frac{\log(1/\w^{K-1})^{B+1}}{(B+1)!} \right)\\
&= \w\log(1/\w)\left(1 - \frac{(K-1)^{B+1}\log(1/\w)^{B+1}}{(B+1)!} \right)\;.
\end{align*}
Taking a threshold $T = \lfloor N/e \rfloor$ gives
\[
\lim_{N \to \infty} \Pr(\A_{\lfloor N/e \rfloor}^B \text{ succeeds})
\geq \frac{1}{e}\left(1 - \frac{(K-1)^{B+1}}{(B+1)!} \right)\;.
\]
\end{proof}






\section{Static Double-threshold algorithm for two groups}


\subsection{Recursion lemma}
We first prove a recursion satisfied by $\U^B_{N,t,k}$ \eqref{eq:def-U}, which we use to prove the subsequent results in this section.

\begin{lemma}\label{lem:recursion-U-alpha-beta}
For all $B \geq 0$, $t \in \{\lfloor \alpha N \rfloor, \ldots, \lfloor \beta N \rfloor - 1\}$, and $k \in \{1,2\}$,  $\U^B_{N,t,k}$ satisfies
\begin{align*}
\U^B_{N,t,k} 
&=   \frac{\lambda}{N} \sum_{s = t}^{\lfloor \beta N \rfloor-1} \Pr(\rho_1 \geq s, g^*_{t-1} = k \mid \C_N)\\
& \quad +  \frac{\indic{B>0}}{1-\lambda} \sum_{s = t}^{\lfloor \beta N \rfloor-1} \Pr(g^*_{t-1} = k, \rho_1 = s, R_s \neq 1 \mid \C_N) \U^{B-1}_{N,s+1,2} \\
&\quad + \Pr(\A^B_t(\alpha, \beta) \text{ succeeds},  g^*_{t-1} = k, \rho_1 \geq \beta N \mid \C_N) + O\big(\tfrac{1}{N}\big)\;.\\
\end{align*}
\end{lemma}

Assuming that $\rho_1 = s \in \{t, \ldots, \lfloor \beta N\rfloor -1 \}$, the first sum corresponds to the success probability if $R_s = 1$ and the algorithm selects the candidate $x_s$. The terms of the second sum represent the success probability after using a comparison at step $s$ but observing $R_t \neq 1$, resulting in the rejection of the candidate. Therefore, the available budget at step $s+1$ is $B-1$, and necessarily $g^*_s = 2$, because a comparison at step $s$ can only occur if $g_s = 1$ by definition of the algorithm. Hence, only the term $\U^{B-1}_{N,s+1,2}$ appears in the recursion, not $\U^{B-1}_{N,s+1,1}$. Finally, the last term represents the probability of success if no comparison has been made before step $\lfloor \beta N \rfloor$


\begin{proof}
Let $B \geq 0$. For all $t  \in \{\lfloor \alpha N \rfloor, \ldots,\lfloor\beta N \rfloor - 1)$ and $k \in \{1,2\}$, it holds that
\begin{align}
\U^B_{N,t,k} 
&= \Pr(\A^B_t(\alpha, \beta) \text{ succeeds},  g^*_{t-1} = k \mid \C_N)\nonumber\\
&= \sum_{s = t}^{\lfloor \beta N \rfloor-1} \Pr(\A^B_t(\alpha, \beta) \text{ succeeds},  g^*_{t-1} = k, \rho_1 = s \mid \C_N) \nonumber \\
&\qquad + \Pr(\A^B_t(\alpha, \beta) \text{ succeeds},  g^*_{t-1} = k, \rho_1 \geq \beta N \mid \C_N) \nonumber\\
&= \sum_{s = t}^{\lfloor \beta N \rfloor-1} \Pr(\A^B_t(\alpha, \beta) \text{ succeeds},  g^*_{t-1} = k, \rho_1 = s, R_s = 1 \mid \C_N) \label{eqref:decom-U-1}\\
& \quad + \sum_{s = t}^{\lfloor \beta N \rfloor-1} \Pr(\A^B_t(\alpha, \beta) \text{ succeeds},  g^*_{t-1} = k, \rho_1 = s, R_s \neq 1 \mid \C_N) \label{eqref:decom-U-2} \\
&\quad + \Pr(\A^B_t(\alpha, \beta) \text{ succeeds},  g^*_{t-1} = k, \rho_1 \geq \beta N \mid \C_N)\;. \label{eqref:decom-U-3}
\end{align}

For all $s \in \{t,\ldots, \lfloor \beta N \rfloor -1\}$, by definition of Algorithm $\A^B_t(\alpha, \beta)$, if $\rho_1 = s$ and $R_s = 1$ then the candidate $x_s$ is selected, and the algorithm succeeds if only if $x_s = \xmax$. Moreover, the event $\{\rho_1 = s\}$ is equivalent to $\{\rho_1 \geq s, g_s = 1, r_s = 1\}$, hence the terms in \eqref{eqref:decom-U-1} can be written as
\begin{align*}
\Pr(\A^B_t(\alpha, \beta) \text{ succeeds},  g^*_{t-1} = k, &\rho_1 = s, R_s = 1 \mid \C_N)\\
&= \Pr(x_s = \xmax,  g^*_{t-1} = k, \rho_1 = s, R_s = 1 \mid \C_N)\\
&= \Pr(x_s = \xmax,  g^*_{t-1} = k, \rho_1 \geq s, g_s = 1 \mid \C_N)\\
&= \frac{\Pr(g_s = 1 \mid \C_N)}{N} \Pr(\rho_1 \geq s, g^*_{t-1} = k \mid \C_N)\\
&= \left( \frac{\lambda}{N} + O\big(\tfrac{1}{N^3}\big)\right) \Pr(\rho_1 \geq s, g^*_{t-1} = k \mid \C_N)\;,
\end{align*}
where used for that the event $\{x_s = \xmax\}$ is independent of the group memberships, thus independent of $\C_N$, and that it is also independent of $\{\rho_1 \geq s, g^*_{t-1} = k\}$, because a realization of the latter event is determined only by the groups and relative ranks of the candidates $\{x_1,\ldots,x_{s-1}\}$. For the last equality, we used Lemma \ref{lem:pr-cond-C_N}.

Secondly, in the case where $\rho_1 = s$ and $R_s \neq 1$, if $B = 0$ then the algorithm selects candidate $x_s$ which is not the best overall, hence its probability of succeeding is zero. If $B \geq 1$, the algorithm makes a comparison but then rejects the candidate. Moreover, for $s \in [t,\beta N]$, if $\rho_1 = s$ then necessarily $g_s = 1$, and having $R_s \neq 1$ implies that $g^*_s = 2$.  The success probability of $\A^B_t(\alpha, \beta)$ given that $\rho_1 = s, R_s \neq 1$ is the same as the success probability of $\A^{B-1}_{s+1}(\alpha, \beta)$ given that $g^*_{s} = 2$. Therefore, the terms of \eqref{eqref:decom-U-2} satisfy
\begin{align*}
\Pr&(\A^B_t(\alpha, \beta) \text{ succeeds},  g^*_{t-1} = k, \rho_1 = s, R_s \neq 1 \mid \C_N) \\
&= \indic{B>0} \Pr(g^*_{t-1} = k, \rho_1 = s, R_s \neq 1 \mid \C_N) \Pr(\A^B_t(\alpha, \beta) \text{ succeeds} \mid g^*_{t-1} = k, \rho_1 = s, R_s \neq 1, \C_N)\\
&= \indic{B>0}\Pr(g^*_{t-1} = k, \rho_1 = s, R_s \neq 1 \mid \C_N) \Pr(\A^{B-1}_{s+1}(\alpha, \beta) \text{ succeeds} \mid g^*_{s} = 2, \C_N)\\
&= \indic{B>0}\Pr(g^*_{t-1} = k, \rho_1 = s, R_s \neq 1 \mid \C_N) \frac{\Pr(\A^{B-1}_{s+1}(\alpha, \beta) \text{ succeeds}, g^*_{s} = 2 \mid \C_N)}{\Pr(g^*_s = 2 \mid \C_N)}\\
&= \indic{B>0}\Pr(g^*_{t-1} = k, \rho_1 = s, R_s \neq 1 \mid \C_N) \frac{\U^{B-1}_{N,s+1,2}}{1-\lambda + O( \tfrac{1}{N^2})}\;,
\end{align*}
where we used again Lemma \ref{lem:pr-cond-C_N}. Given that the $O$ terms are independent of $s$, We deduce that 
\begin{align*}
\U^B_{N,t,k} 
&=  \left( \frac{\lambda}{N} + O\big(\tfrac{1}{N^3}\big)\right) \sum_{s = t}^{\lfloor \beta N \rfloor-1} \Pr(\rho_1 \geq s, g^*_{t-1} = k \mid \C_N)\\
& \quad + \indic{B>0} \Big(\frac{1}{1-\lambda} + O\big( \tfrac{1}{N^2}\big) \Big) \sum_{s = t}^{\lfloor \beta N \rfloor-1} \Pr(g^*_{t-1} = k, \rho_1 = s, R_s \neq 1 \mid \C_N) \U^{B-1}_{N,s+1,2} \\
&\quad + \Pr(\A^B_t(\alpha, \beta) \text{ succeeds},  g^*_{t-1} = k, \rho_1 \geq \beta N \mid \C_N)\\
&=   \frac{\lambda}{N} \sum_{s = t}^{\lfloor \beta N \rfloor-1} \Pr(\rho_1 \geq s, g^*_{t-1} = k \mid \C_N)\\
& \quad + \frac{\indic{B>0}}{1-\lambda} \sum_{s = t}^{\lfloor \beta N \rfloor-1} \Pr(g^*_{t-1} = k, \rho_1 = s, R_s \neq 1 \mid \C_N) \U^{B-1}_{N,s+1,2} \\
&\quad + \Pr(\A^B_t(\alpha, \beta) \text{ succeeds},  g^*_{t-1} = k, \rho_1 \geq \beta N \mid \C_N) + O\big(\tfrac{1}{N}\big)\;.
\end{align*}

\end{proof}






\subsection{Additional lemmas}

In the following two lemmas, we compute the probabilities appearing in Lemma \ref{lem:recursion-U-alpha-beta} for all $s \in \{t,\ldots, \lfloor \beta N \rfloor -1$ and $k\in \{1,2\}$

\begin{lemma}\label{lem:2grp-rho>s}
Let $\lfloor \alpha N \rfloor \leq t \leq s < \lfloor \beta N \rfloor$, and consider a run of Algorithm $\A_t^B(\alpha, \beta)$, then it holds that
\begin{align*}
\Pr(\rho_1 \geq s , g^*_{t-1} = 1 \mid \C_N)
&= \frac{\lambda t}{(1-\lambda)t + \lambda s} + O\Big( \sqrt{\tfrac{\log N}{N}} \Big)\\
\Pr(\rho_1 \geq s, g^*_{t-1} = 2 \mid \C_N)
&= \frac{(1-\lambda)t^2}{((1-\lambda)t + \lambda s)s} + O\Big( \sqrt{\tfrac{\log N}{N}} \Big)\;.
\end{align*}
\end{lemma}


\begin{proof}
Since there are only $2$ groups, the event $g^*_{t-1} = 1$ is equivalent to $\max G^2_{t-1} < \max G^1_{t-1}$.
For $s \in \{t, \ldots, \lfloor \beta N \rfloor\}$, Algorithm $\A_t^B(\alpha, \beta)$ only makes a comparison (or stops in the case of $B=0$) at step $s$ only if $g_s = 1$ and $r_s = 1$. Therefore, $\rho_1 \geq s$ if and only if no candidate belonging to $G^1_{t:s-1}$ surpasses the maximum value observed in $G^1_{t-1}$
\begin{align*}
\Pr(\rho_1 \geq s, g^*_{t-1} = 1 \mid \C_N)
&=\Pr(\max G^1_{t:s-1} < G^1_{t-1},  \max G^2_{t-1} < \max G^1_{t-1} \mid \C_N)\\
&= \Pr(\max (G^1_{t:s-1} \cup G^2_{t-1}) < G^1_{t-1}\mid \C_N)\\
&= \E\left[ \frac{|G^1_{t-1}|}{|G^1_{t-1}|+|G^1_{t:s-1}|+|G^2_{t-1}|} \; \Big| \; \C_N \right]\\
&= \E\left[ \frac{|G^1_{t-1}|}{t-1 +|G^1_{t:s-1}|} \; \Big| \; \C_N \right]\\
&= \frac{\lambda t + O(\sqrt{N\log N})}{t + \lambda(s-t) + O(\sqrt{N\log N})}\\
&= \frac{\lambda t}{(1-\lambda)t + \lambda s} + O\Big( \sqrt{\tfrac{\log N}{N}} \Big)\;.
\end{align*}
For the case $g_t^* = 2$, we obtain
\begin{align*}
\Pr(\rho_1 \geq s, g^*_{t-1} = 2 \mid \C_N)
&=\Pr(\max G^1_{t:s-1} < G^1_{t-1},  \max G^1_{t-1} < \max G^2_{t-1} \mid \C_N)\\
&= \max G^1_{t:s-1} < G^1_{t-1} < \max G^2_{t-1} \mid \C_N)\\
&= \E\left[ \frac{|G^2_{t-1}|}{|G^1_{t-1}|+|G^1_{t:s-1}|+|G^2_{t-1}|} \cdot \frac{|G^1_{t-1}|}{|G^1_{t:s-1}| + |G^1_{t-1}|} \; \Big| \; \C_N \right]\\
&= \E\left[ \frac{|G^2_{t-1}|}{t-1 +|G^1_{t:s-1}|} \cdot \frac{|G^1_{t-1}|}{|G^1_{s-1}|} \; \Big| \; \C_N \right]\\
&= \frac{(1-\lambda)t + O(\sqrt{N\log N})}{t+\lambda(s-t) + O(\sqrt{N\log N})} \cdot \frac{\lambda t + O(\sqrt{N\log N})}{\lambda s + O(\sqrt{N\log N})}\\
&= \frac{(1-\lambda)t^2}{((1-\lambda)t + \lambda s)s} + O\Big( \sqrt{\tfrac{\log N}{N}} \Big)\;.
\end{align*}
\end{proof}





\begin{lemma}\label{lem:2grp-rho=s}
Let $\lfloor \alpha N \rfloor \leq t \leq s < \lfloor \beta N \rfloor$, and consider a run of Algorithm $\A_t^B(\alpha, \beta)$, then 
\begin{align*}
\Pr(\rho_1 = s, R_s \neq 1, g^*_{t-1} = 1 \mid \C_N)
&= \frac{\lambda^2(1-\lambda)(s-t)t}{((1-\lambda)t + \lambda s)^2s} + O\Big( \sqrt{\tfrac{\log N}{N^3}} \Big)\\
\Pr(\rho_1 = s, R_s \neq 1, g^*_{t-1} = 2 \mid \C_N)
&= \frac{(1-\lambda)\left( (1-\lambda)^2 t + \lambda(2-\lambda)s\right)t^2}{((1-\lambda)t+\lambda s)^2s^2} + O\Big( \sqrt{\tfrac{\log N}{N^3}} \Big)\;.
\end{align*}
\end{lemma}

\begin{proof}
For Algorithm $\A_t^B(\alpha, \beta)$ and $s \in \{t, \ldots, \lfloor \beta N \rfloor-1\}$, $\rho_1 = s$ if and only if $x_s$ is the first element in $G^1$ since step $t$ for which $r_s = 1$, thus
\[
\rho_1 = s \iff g_s = 1 \text{ and } \max G^1_{t:s-1} < \max G^1_{t-1} < x_s\;.
\]
Furthermore, Lemma \ref{lem:pr-cond-C_N} gives that $\Pr(g_s = 1 \mid \C_N) = \Pr(g_s = 1) + O(1/N) = \lambda + O(1/N)$. Therefore, it holds that
\begin{align*}
\Pr(\rho_1 &= s, R_s \neq 1, g^*_{t-1} = 1 \mid \C_N)\\
&= \Pr( g_s = 1 \; , \; \max G^1_{t:s-1} < \max G^1_{t-1} < x_s \; , \; x_s < \max G^2_{s-1} \; , \; \max G^2_{t-1} < \max G^1_{t-1} \mid \C_N)\\
&= \Pr( g_s = 1 \mid C_N) \Pr( \max G^1_{t:s-1} < \max G^1_{t-1} < x_s < \max G^2_{t:s-1}, \max G^2_{t-1} < \max G^1_{t-1} \mid \C_N)\\
&= \Pr( g_s = 1 \mid C_N) \Pr( \max (G^1_{t:s-1} \cup  G^2_{t-1}) < \max G^1_{t-1} < x_s < \max G^2_{t:s-1} \mid \C_N)\\
&= (\lambda + O(\tfrac{1}{N}))\E\Big[ \tfrac{|G^2_{t:s-1}|}{|G^1_{t:s-1}|+|G^2_{t-1}|+|G^1_{t-1}|+1+ |G^2_{t:s-1}|} \cdot \tfrac{1}{|G^1_{t:s-1}|+|G^2_{t-1}|+|G^1_{t-1}|+1} 
\cdot \tfrac{|G^1_{t-1}|}{|G^1_{t:s-1}|+|G^2_{t-1}|+|G^1_{t-1}|} \; \Big| \; \C_N \Big]\\
&= (\lambda + O(\tfrac{1}{N}))\E\left[ \frac{|G^2_{t:s-1}|}{s}\cdot \frac{1}{t + |G^1_{t:s-1}|}\cdot \frac{ |G^1_{t-1}|}{t-1 +  |G^1_{t:s-1}|} \; \Big| \; \C_N \right]\\
&= (\lambda + O(\tfrac{1}{N}))\frac{(1-\lambda)(s-t) + O(\sqrt{N\log N})}{s(t+\lambda(s-t) + O(\sqrt{N\log N}))}\cdot \frac{\lambda t + O(\sqrt{N\log N})}{t + \lambda(s-t) + O(\sqrt{N\log N})}\\
&= \frac{\lambda^2(1-\lambda)(s-t)t}{((1-\lambda)t + \lambda s)^2s} + O\Big( \sqrt{\tfrac{\log N}{N^3}} \Big)\;.
\end{align*}
On the other hand, in the case where $g^*_{t-1} = 2$, we obtain
\begin{align*}
\Pr(\rho_1 &= s, R_s \neq 1, g^*_{t-1} = 2 \mid \C_N)\\
&= \Pr( g_s = 1 \; , \; \max G^1_{t:s-1} < \max G^1_{t-1} < x_s \; , \; x_s < \max G^2_{s-1} \; , \; \max G^1_{t-1} < \max G^2_{t-1} \mid \C_N)\\
&= \Pr( g_s = 1 \mid C_N) \Pr( \max G^1_{t:s-1} < \max G^1_{t-1} < x_s < \max G^2_{s-1}\; , \; \max G^1_{t-1} < \max G^2_{t-1} \mid \C_N)\\
&= \Pr( g_s = 1 \mid C_N) \Pr( a < b < x_s < \max(c,d)\; , \; b < c \mid \C_N)\;,
\end{align*}
where $a =  \max G^1_{t:s-1}$, $b = \max G^1_{t-1}$, $c = \max G^2_{t-1}$ and $d = \max G^2_{t:s-1}$. Let us denote by $\mathcal{E}$ the event $\{a < b < x_s < \max(c,d)\} \cap \{b < c\}$. It holds that
\begin{align*}
\mathcal{E} \cap \{c<d\}
&= \{a < b < x_s < \max(c,d)\} \cap \{b < c\} \cap \{c<d\}\\
&= \{ a<b<c<x_s<d \} \cup \{ a<b<x_s<c<d \}\\
&= \{ a<b<c<x_s<d \} \cup \big(\{ a<b<x_s<c \} \cap \{ c < d\}\big)\;,\\
\mathcal{E} \cap \{d<c\}
&= \{a < b < x_s < \max(c,d)\} \cap \{b < c\} \cap \{d<c\}\\
&= \{a < b < x_s < c\} \cap \{d<c\}\;,
\end{align*}
which yields
\begin{align*}
\mathcal{E} 
&= \big( \mathcal{E} \cap \{c<d\} \big) \big( \mathcal{E} \cap \{d<c\} \big)\\
&= \{ a<b<c<x_s<d \} \cup \big(\{ a<b<x_s<c \} \cap \{ c < d\}\big) \cup \big(\{ a<b<x_s<c \} \cap \{d<c\}\big)\\
&= \{ a<b<c<x_s<d \} \cup \{ a<b<x_s<c \}\;.
\end{align*}
The two events above are disjoint, and we have
\begin{align*}
\Pr&(a<b<c<x_s<d \mid \C_N)\\
&= \Pr(\max G^1_{t:s-1} < \max G^1_{t-1} < \max G^2_{t-1}< x_s < G^2_{t:s-1} \mid \C_N)\\
&= \E\left[\tfrac{|G^2_{t:s-1}|}{|G^1_{t:s-1}|+|G^1_{t-1}|+|G^2_{t-1}|+1+|G^2_{t:s-1}|}
\cdot \tfrac{1}{|G^1_{t:s-1}|+|G^1_{t-1}|+|G^2_{t-1}|+1} 
\cdot \tfrac{|G^2_{t-1}|}{|G^1_{t:s-1}|+|G^1_{t-1}|+|G^2_{t-1}|}
\cdot \tfrac{|G^1_{t-1}|}{|G^1_{t:s-1}|+|G^1_{t-1}|}
\cdot \mid \C_N \right]\\
&= \E\left[\frac{|G^2_{t:s-1}|}{s}
\cdot \frac{1}{t+|G^1_{t:s-1}|} 
\cdot \frac{|G^2_{t-1}|}{t-1+|G^1_{t:s-1}|}
\cdot \frac{|G^1_{t-1}|}{|G^1_{s-1}|}
 \; \Big| \; \C_N \right]\\
&=\frac{(1-\lambda)(s-t) + O(\sqrt{N \log N})}{s(t + \lambda(s-t) + O(\sqrt{N \log N}))} \cdot \frac{(1-\lambda)t + O(\sqrt{N \log N})}{t + \lambda(s-t) + O(\sqrt{N \log N})}\cdot \frac{\lambda t + O(\sqrt{N \log N})}{\lambda s + O(\sqrt{N \log N})}\\
&= \frac{(1-\lambda)^2 (s-t) t^2}{((1-\lambda)t + \lambda s)^2 s^2} + O\Big( \sqrt{\tfrac{\log N}{N^3}} \Big)\;.
\end{align*}
The probability of the second event is
\begin{align*}
\Pr(a < b < x_s < c \mid \C_N)
&= \Pr(\max G^1_{t:s-1} < \max G^1_{t-1} < x_s <  \max G^2_{t-1} \mid \C_N)\\
&= \E\left[ \tfrac{|G^2_{t-1}|}{|G^1_{t:s-1}| + |G^1_{t-1}| + 1 + |G^2_{t-1}|}
\cdot \tfrac{1}{|G^1_{t:s-1}| + |G^1_{t-1}| + 1}
\cdot \tfrac{|G^1_{t-1}|}{|G^1_{t:s-1}| + |G^1_{t-1}| }
 \;\Big|\; \C_N \right]\\
&= \E\left[ \frac{|G^2_{t-1}|}{t + |G^1_{t:s-1}|}
\cdot \frac{1}{|G^1_{s-1}| + 1}
\cdot \frac{|G^1_{t-1}|}{|G^1_{s-1}|}
 \;\Big|\; \C_N \right]\\
&= \frac{(1-\lambda)t + O(\sqrt{N \log N})}{t + \lambda(s-t) + O(\sqrt{N \log N})} \cdot \frac{1}{\lambda s + O(\sqrt{N \log N})} \cdot \frac{\lambda t + O(\sqrt{N \log N})}{\lambda s + O(\sqrt{N \log N})}\\
&= \frac{(1-\lambda)t^2}{\lambda((1-\lambda)t + \lambda s) s^2} + O\Big( \sqrt{\tfrac{\log N}{N^3}} \Big)\;.
\end{align*}
Finally, Lemma \ref{lem:pr-cond-C_N} shows that $\Pr( g_s = 1 \mid C_N) = \lambda + O(1/N^2)$, and we deduce 
\begin{align*}
\Pr(\rho_1 = s, R_s \neq 1, g^*_{t-1} = 2 \mid \C_N)
&= \lambda \Pr(\mathcal{E} \mid \C_N) + O(\tfrac{1}{N^2})\\
&= \lambda \Pr(a<b<c<x_s<d \mid \C_N) + \lambda \Pr(a < b < x_s < c \mid \C_N) + O(\tfrac{1}{N^2})\\
&= \frac{\lambda(1-\lambda)^2 (s-t) t^2}{((1-\lambda)t + \lambda s)^2 s^2} + \frac{\lambda(1-\lambda)t^2}{\lambda((1-\lambda)t + \lambda s) s^2} + O\Big( \sqrt{\tfrac{\log N}{N^3}} \Big)\\
&= \frac{(1-\lambda)t^2}{((1-\lambda)t+\lambda s)^2s^2}\left( \lambda(1-\lambda)(s-t) + (1-\lambda)t+\lambda s\right) + O\Big( \sqrt{\tfrac{\log N}{N^3}} \Big)\\
&= \frac{(1-\lambda)t^2}{((1-\lambda)t+\lambda s)^2s^2}\left((1-\lambda)^2 t + \lambda(2-\lambda)s \right) + O\Big( \sqrt{\tfrac{\log N}{N^3}} \Big)\;.\\
\end{align*}
\end{proof}





\begin{lemma}\label{lem:2grp-rho>betaN}
Let $\lfloor \alpha N \rfloor \leq t < \lfloor \beta N \rfloor$, and consider a run of Algorithm $\A_t^B(\alpha, \beta)$, then
\begin{align*}
\Pr(\A_t^B(\alpha,\beta) \text{ succeeds}&, \rho_1 \geq \beta N, g^*_{t-1} = 1 \mid \C_N)\\
&=  \frac{t}{\beta N} \U^B_{N, \lfloor \beta N \rfloor, 1}
+ \frac{\lambda(\beta N -t)t}{((1-\lambda) t + \lambda \beta N) \beta N} \U^B_{N, \lfloor \beta N \rfloor, 2} +  O\Big(\sqrt{\tfrac{\log N}{N}}\Big)\;, \\
\Pr(\A_t^B(\alpha,\beta) \text{ succeeds}&, \rho_1 \geq \beta N, g^*_{t-1} = 2 \mid \C_N)\\
&= \frac{t^2 }{((1-\lambda)t + \lambda \beta N) \beta N} \U^B_{N,\lfloor \beta N \rfloor,2} +  O\Big(\sqrt{\tfrac{\log N}{N}}\Big)\;.
\end{align*} 
\end{lemma}

\begin{proof}
Since Algorithm $\A_t^B(\alpha, \beta)$ is memoryless, if it does not stop before step $\lfloor \beta N \rfloor$, then its success probability is the same as that of $\A^B_{\lfloor \beta N \rfloor}(\alpha, \beta) = \A^B_0(\beta, \beta)$, which has the same threshold $\beta$ for both groups, if it is in the same state $(g^*_{\lfloor \beta N \rfloor - 1}, |G^1_{\lfloor \beta N \rfloor}|)$. In all the proof, $\rho_1$ is relative to Algorithm $\A_t^B(\alpha, \beta)$, not $\A^B_0(\beta, \beta)$. It holds that
\begin{align*}
\Pr&(\A_t^B(\alpha,\beta) \text{ succeeds}, \rho_1 \geq \beta N, g^*_{t-1} = 1 \mid \C_N)\\
&= \sum_{\ell \in \{1,2\}}\Pr(\A_t^B(\alpha,\beta) \text{ succeeds}, \rho_1 \geq \beta N, g^*_{t-1} = 1, g^*_{\lfloor \beta N \rfloor - 1} = \ell \mid \C_N)\\
&= \sum_{\ell \in \{1,2\}} \Pr(\rho_1 \geq \beta N, g^*_{t-1} = 1, g^*_{\lfloor \beta N \rfloor - 1} = \ell \mid \C_N))\\
&\hspace{1.35cm}\times \Pr(\A_t^B(\alpha,\beta) \text{ succeeds} \mid \rho_1 \geq \beta N, g^*_{t-1} = 1, g^*_{\lfloor \beta N \rfloor - 1} = \ell, \C_N) \\
&= \sum_{\ell \in \{1,2\}} \Pr(\rho_1 \geq \beta N, g^*_{t-1} = 1, g^*_{\lfloor \beta N \rfloor - 1} = \ell \mid \C_N) \Pr(\A^B_0(\beta,\beta) \text{ succeeds} \mid g^*_{\lfloor \beta N \rfloor - 1} = \ell, \C_N)\\
&= \sum_{\ell \in \{1,2\}} \Pr(\rho_1 \geq \beta N, g^*_{t-1} = 1, g^*_{\lfloor \beta N \rfloor - 1} = \ell \mid \C_N) \frac{\Pr(\A^B_0(\beta,\beta) \text{ succeeds}, g^*_{\lfloor \beta N \rfloor - 1} = \ell \mid \C_N)}{\Pr(g^*_{\lfloor \beta N \rfloor - 1} = \ell \mid \C_N)}\\
&= \sum_{\ell \in \{1,2\}} \Pr(\rho_1 \geq \beta N, g^*_{t-1} = 1, g^*_{\lfloor \beta N \rfloor - 1} = \ell \mid \C_N) \frac{\U^B_{N,\lfloor \beta N \rfloor,\ell}}{\Pr(g^*_{\lfloor \beta N \rfloor - 1} = \ell) + O(\tfrac{1}{N^2})}\;,\\
\end{align*} 
where we used Lemma \ref{lem:pr-cond-C_N} and the definition \eqref{eq:def-U} of $\U^B_{N,s,\ell}$. Let us now compute the probability of the event $\{\rho_1 \geq \beta N, g^*_{t-1} = 1, g^*_{\lfloor \beta N \rfloor - 1} = \ell\}$ conditional to $\C_N$.
For $\ell = 1$, we have 
\begin{align*}
\Pr&(\rho_1 \geq \beta N, g^*_{t-1} = 1, g^*_{\lfloor \beta N \rfloor - 1} = 1 \mid \C_N)\\
&= \Pr(\max G^1_{t:\lfloor \beta N \rfloor-1} < \max G^1_{t-1} \;,\; \max G^2_{t-1} < \max G^1_{t-1} \;,\; \max G^2_{\lfloor \beta N \rfloor-1} < \max G^1_{\lfloor \beta N \rfloor-1} \mid \C_N)\\
&= \Pr(\max x_{1:\lfloor \beta N \rfloor - 1} \in G^1_{t-1}\mid \C_N)\\
&= \E\left[ \frac{|G^1_{t-1}|}{\lfloor \beta N \rfloor-1} \Big| \C_N\right]\\
&= \frac{\lambda t + O(\sqrt{N \log N})}{\beta N + O(1)}
= \frac{\lambda t}{\beta N} + O\Big( \sqrt{\tfrac{\log N}{N}} \Big)\;.
\end{align*}
For $\ell = 2$, we first compute the following
\begin{align*}
\Pr(\rho_1 \geq \beta N, g^*_{t-1} = 1\mid \C_N)
&= \Pr( \max G^1_{t:\lfloor \beta N \rfloor-1} < \max G^1_{t-1} \;,\; \max G^2_{t-1} < \max G^1_{t-1} \mid \C_N)\\
&= \Pr( \max( G^1_{t:\lfloor \beta N \rfloor-1} \cup G^2_{t-1} ) < \max G^1_{t-1} \mid \C_N)\\
&= \E\left[ \frac{|G^1_{t-1}|}{|G^1_{t:\lfloor \beta N \rfloor-1}| + |G^2_{t-1}| + |G^1_{t-1}|} \;\Big|\; \C_N \right]\\
&= \frac{\lambda t + O(\sqrt{N \log N})}{t + \lambda(\beta N - t) + O(\sqrt{N \log N})}\\
&= \frac{\lambda t }{(1-\lambda) t + \lambda \beta N} + O\Big( \sqrt{\tfrac{\log N}{N}} \Big)\;,
\end{align*}
and it follows that
\begin{align*}
\Pr&(\rho_1 \geq \beta N, g^*_{t-1} = 1, g^*_{\lfloor \beta N \rfloor - 1} = 2 \mid \C_N)\\
&= \Pr(\rho_1 \geq \beta N, g^*_{t-1} = 1 \mid \C_N) - \Pr(\rho_1 \geq \beta N, g^*_{t-1} = 1, g^*_{\lfloor \beta N \rfloor - 1} = 1 \mid \C_N)\\
&= \frac{\lambda t }{(1-\lambda) t + \lambda \beta N} - \frac{\lambda t}{\beta N} +  O\Big( \sqrt{\tfrac{\log N}{N}} \Big)\\
&= \frac{\lambda(1-\lambda)(\beta N -t)t}{((1-\lambda) t + \lambda \beta N) \beta N}  + O\Big( \sqrt{\tfrac{\log N}{N}} \Big)\;.
\end{align*}
All in all, we deduce that 
\begin{align*}
\Pr(\A_t^B&(\alpha,\beta) \text{ succeeds}, \rho_1 \geq \beta N, g^*_{t-1} = 1 \mid \C_N)\\
&= \Big(\frac{\lambda t}{\beta N} + O\Big( \sqrt{\tfrac{\log N}{N}} \Big) \Big) \frac{\U^B_{N, \lfloor \beta N \rfloor, 1}}{\lambda + O(\tfrac{1}{N^2})}
+ \left(\frac{\lambda(1-\lambda)(\beta N -t)t}{((1-\lambda) t + \lambda \beta N) \beta N} + O\Big( \sqrt{\tfrac{\log N}{N}} \Big) \right) \frac{\U^B_{N, \lfloor \beta N \rfloor, 2}}{1-\lambda + O(\tfrac{1}{N^2})}\\
&= \Big(\frac{t}{\beta N} + O\Big( \sqrt{\tfrac{\log N}{N}} \Big) \Big) \U^B_{N, \lfloor \beta N \rfloor, 1}
+ \left(\frac{\lambda(\beta N -t)t}{((1-\lambda) t + \lambda \beta N) \beta N} + O\Big( \sqrt{\tfrac{\log N}{N}} \Big) \right) \U^B_{N, \lfloor \beta N \rfloor, 2}\\
&=  \frac{t}{\beta N} \U^B_{N, \lfloor \beta N \rfloor, 1}
+ \frac{\lambda(\beta N -t)t}{((1-\lambda) t + \lambda \beta N) \beta N} \U^B_{N, \lfloor \beta N \rfloor, 2} + O\Big(\sqrt{\tfrac{\log N}{N}}\Big) \;.
\end{align*} 

On the other hand, if $g^*_{t-1} = 2$ and $\rho_1 \geq \beta N$, then necessarily $g^*_{\lfloor \beta N \rfloor-1} = 2$, because no candidate in $G^1$ up to step $\lfloor \beta N \rfloor-1$ surpasses $\max G^1_{t-1}$, which is less than $\max G^2_{t-1}$. Therefore
\begin{align*}
\Pr(\A_t^B&(\alpha,\beta) \text{ succeeds}, \rho_1 \geq \beta N, g^*_{t-1} = 2 \mid \C_N)\\
&= \Pr(\rho_1 \geq \beta N, g^*_{t-1} = 2 \mid \C_N) \Pr(\A_t^B(\alpha, \beta) \mid  \rho_1 \geq \beta N, g^*_{t-1} = 2, \C_N)\\
&= \Pr(\max G^1_{t:\lfloor \beta N \rfloor-1} < \max G^1_{t-1} < \max G^2_{t-1} \mid \C_N) \Pr(\A_0^B(\beta, \beta) \mid  g^*_{\lfloor \beta N \rfloor-1} = 2, \C_N)\\
&= \E\left[ \frac{|G^2_{t-1}|}{t-1 + |G^1_{t:\lfloor \beta N \rfloor-1}|} \cdot \frac{|G^1_{t-1}|}{|G^1_{\lfloor \beta N \rfloor-1}|} \;\Big|\; \C_N \right] \frac{\Pr(\A_0^B(\beta, \beta),  g^*_{\lfloor \beta N \rfloor-1} = 2 \mid \C_N)}{\Pr(g^*_{\lfloor \beta N \rfloor-1} = 2 \mid \C_N)}\\
&= \frac{(1-\lambda)t + O(\sqrt{N\log N})}{t+ \lambda(\beta N -t) + O(\sqrt{N\log N})}\cdot \frac{\lambda t + O(\sqrt{N\log N})}{\lambda \beta N + O(\sqrt{N\log N})}\cdot \frac{\U^B_{N,\lfloor \beta N \rfloor,2}}{1 - \lambda + O(\tfrac{1}{N^2})}\\
&= \frac{t^2 }{((1-\lambda)t + \lambda \beta N) \beta N} \U^B_{N,\lfloor \beta N \rfloor,2} +  O\Big(\sqrt{\tfrac{\log N}{N}}\Big)\;.
\end{align*}
\end{proof}




In the following lemma, we compute the exact limit of $\U^B_{N,\lfloor \beta N \rfloor,k}$ when the number of candidates goes to infinity.

\begin{lemma}\label{lem:2grp-Ubb}
For all $B \geq 0$ and $k \in \{1,2\}$, 
\[
\U^B_{N,\lfloor \beta N \rfloor,k}
= \lambda_k \beta^2\sum_{b = 0}^B \left( \frac{1}{\beta} - \sum_{\ell = 0}^b \frac{\log(1/\beta)^\ell}{\ell !} \right) + O\Big( \sqrt{\tfrac{\log N}{N}}\Big)\;.
\]
\end{lemma}


\begin{proof}
By definition \eqref{eq:def-U} of $\U^B_{N,t,k}$, we have
\begin{align*}
\U^B_{N,t,k} 
&= \Pr(\A^B_{\lfloor \beta N \rfloor}(\alpha, \beta) \text{ succeeds},  g^*_{t-1} = k \mid \C_N)\;,
\end{align*}
and $\A^B_{\lfloor \beta N \rfloor}(\alpha, \beta)$ is simply the single-threshold algorithm with threshold $\beta N$ and budget $B$. Let $T = \lfloor \beta N \rfloor$. As in the proof of Lemma \ref{lem:single-thres-recursion}, we decompose the success probability of $\A^B_{\lfloor \beta N \rfloor}$ as follows
\begin{align*}
\U^B_{N,T,k} 
&= \Pr(\A^B_{T}(\alpha, \beta) \text{ succeeds},  g^*_{T-1} = k \mid \C_N)\\
&= \sum_{t=T}^N \sum_{\ell \in \{1,2\}} \Pr(\A^B_{T}(\alpha, \beta) \text{ succeeds}, \rho_1=t, g_t = \ell,  g^*_{T-1} = k \mid \C_N)\\
&= \sum_{t=T}^N \sum_{\ell \in \{1,2\}} \bigg(\Pr(\A^B_{T}(\alpha, \beta) \text{ succeeds}, \rho_1=t, R_t = 1, g_t = \ell,  g^*_{T-1} = k \mid \C_N)\\
& \hspace{1.9cm} + \Pr(\A^B_{T}(\alpha, \beta) \text{ succeeds}, \rho_1=t, R_t \neq 1, g_t = \ell,  g^*_{T-1} = k \mid \C_N) \bigg)\;.
\end{align*}

The terms appearing in the sums above were computed in the proof of Lemma \ref{lem:single-thres-recursion}. It follows respectively from \eqref{aligneq:single-thresh-Rt=1} and \eqref{aligneq:single-thresh-Rtneq1}, with $K = 2$, that
\begin{align*}
\Pr(\A^B_{T}(\alpha, \beta) \text{ succeeds}, \rho_1=t, R_t = 1&, g_t = \ell,  g^*_{T-1} = k \mid \C_N)
= \frac{\lambda_\ell \lambda_k}{N} (T/t)^2 +  O\Big( \sqrt{\tfrac{\log N}{N^3}}\Big)\;,\\
\Pr(\A^B_{T}(\alpha, \beta) \text{ succeeds}, \rho_1=t, R_t \neq 1&, g_t = \ell,  g^*_{T-1} = k \mid \C_N)\\
&= \left( \frac{T^2}{t^{3}} + O\Big( \sqrt{\tfrac{\log N}{N^3}}\Big) \right) \indic{B>0, k \neq \ell} \U^{B-1}_{N,t+1,k}  \;,
\end{align*}
where the $O$ terms are independent of $t$, they only depend on $\beta$. Therefore,
\begin{align*}
\U^B_{N,T,k} 
&= \left(1 + O\Big( \sqrt{\tfrac{\log N}{N}}\Big) \right)\sum_{t=T}^N \sum_{\ell \in \{1,2\}} \bigg(\frac{\lambda_\ell \lambda_k}{N} (T/t)^2 + \frac{T^2}{t^{3}} \indic{B>0, k \neq \ell} \U^{B-1}_{N,t+1,k} \bigg)\\
&= \left(1 + O\Big( \sqrt{\tfrac{\log N}{N}}\Big) \right)\sum_{t=T}^N \bigg(\frac{\lambda_k}{N} (T/t)^2 + \frac{T^2}{t^{3}} \indic{B>0} \U^{B-1}_{N,t+1,k} \bigg)\;.
\end{align*}
The first sum can easily be computed
\begin{align*}
\sum_{t=T}^N \frac{\lambda_k}{N} (T/t)^2
&= \frac{\lambda_k (T/N)^2}{N} \sum_{t=T}^N \frac{1}{(t/N)^2}\\
&= \lambda_k (\beta^2 + O(\tfrac{1}{N})) \int_\beta^1 \frac{du}{u^2} + O(\tfrac{1}{N})\\
&= \lambda_k \beta(1- \beta) + O(\tfrac{1}{N})\;.    
\end{align*}
Therefore,
\begin{align*}
\U^B_{N,T,k}
&= \left(1 + O\Big( \sqrt{\tfrac{\log N}{N}}\Big) \right)\bigg(\lambda_k \beta(1- \beta) +  \indic{B>0}\sum_{t=T}^N \frac{T^2}{t^{3}}\U^{B-1}_{N,t+1,k} + O(\tfrac{1}{N}) \bigg)\\
&= \lambda_k \beta(1- \beta) +  \indic{B>0}\sum_{t=T}^N \frac{T^2}{t^{3}}\U^{B-1}_{N,t+1,k} + O\Big( \sqrt{\tfrac{\log N}{N}}\Big) \;.
\end{align*}
Dividing by $\lambda_k$ yields
\begin{align*}
\big(\lambda_k^{-1} \U^B_{N,T,k} \big)
=  \beta(1- \beta) +  \indic{B>0}\sum_{t=T}^N \frac{T^2}{t^{3}} \big( \lambda_k^{-1} \U^{B-1}_{N,t+1,k} \big) + O\Big( \sqrt{\tfrac{\log N}{N}}\Big) \;.
\end{align*}
thus the double-indexed sequence $(\lambda_k^{-1}\U^b_{N,t,k})_{b,t}$ satisfies the same recursion and initial condition as $(\Pr(\A^b_t(0,0) \text{ succeeds}))_{b,t}$ (see proof of Lemma \ref{lem:single-thres-recursion}) with $\beta$ instead of $w$ and $K=2$. Therefore, we deduce immediately that:
\[
\lambda_k^{-1}\U^B_{N,T,k} = \beta^2 \sum_{b = 0}^B \left( \frac{1}{\beta} - \sum_{\ell = 0}^b \frac{\log(1/\beta)^\ell}{\ell !} \right) + O\Big( \sqrt{\tfrac{\log N}{N}}\Big)\;.
\]
\end{proof}


















\subsection{Proof of Lemma \ref{lem:lim-2grp-k=2}}

\begin{proof}
Using Lemmas \ref{lem:recursion-U-alpha-beta}, \ref{lem:2grp-rho>s}, \ref{lem:2grp-rho=s} and \ref{lem:2grp-rho>betaN} for $k=2$, we obtain for all $t \in \{ \lfloor\alpha N \rfloor, \ldots, \lfloor\beta N \rfloor - 1\}$
\begin{align*}
\U^B_{N,t,2} 
&=  \frac{\lambda}{N} \sum_{s = t}^{\lfloor \beta N \rfloor-1} \left(\frac{(1-\lambda)t^2}{((1-\lambda)t + \lambda s)s} + O\Big( \sqrt{\tfrac{\log N}{N}} \Big) \right)\\
& \quad +\frac{\indic{B>0}}{1-\lambda}  \sum_{s = t}^{\lfloor \beta N \rfloor-1} \left( \frac{(1-\lambda)\left( (1-\lambda)^2 t + \lambda(2-\lambda)s\right)t^2}{((1-\lambda)t+\lambda s)^2s^2} + O\Big( \sqrt{\tfrac{\log N}{N^3}} \Big)\right) \U^{B-1}_{N,s+1,2} \\
&\quad + \frac{t^2 }{((1-\lambda)t + \lambda \beta N) \beta N} \U^B_{N,\lfloor \beta N \rfloor,2} + O\Big(\sqrt{\tfrac{\log N}{N}}\Big)\;.
\end{align*}
The $O$ terms inside the sums depend on the ratios $t/N$ and $s/N$, but using that $\alpha \leq t/N \leq s/N \leq 1$, it can be made only dependent on $\alpha$ and the other constants of the problem. Moreover, Thus we can write
\begin{align*}
\U^B_{N,t,2} 
&= \frac{\lambda(1-\lambda) t^2}{N} \sum_{s = t}^{\lfloor \beta N \rfloor-1} \frac{1}{((1-\lambda)t + \lambda s)s} \nonumber\\
& \quad + \indic{B>0} t^2 \sum_{s = t}^{\lfloor \beta N \rfloor-1}  \frac{\left( (1-\lambda)^2 t + \lambda(2-\lambda)s\right)}{((1-\lambda)t+\lambda s)^2s^2}  \U^{B-1}_{N,s+1,2}\nonumber \\
&\quad +\frac{t^2 }{((1-\lambda)t + \lambda \beta N) \beta N} \U^B_{N,\lfloor \beta N \rfloor,2} + O\Big( \sqrt{\tfrac{\log N}{N}} \Big) \nonumber\\
&=   \lambda(1-\lambda)(\tfrac{t}{N})^2 \frac{1}{N} \sum_{s = t}^{\lfloor \beta N \rfloor-1} \frac{1}{((1-\lambda)\tfrac{t}{N} + \lambda \tfrac{s}{N})\tfrac{s}{N}} \nonumber\\
& \quad + \indic{B>0} (\tfrac{t}{N})^2 \frac{1}{N} \sum_{s = t}^{\lfloor \beta N \rfloor-1}  \frac{ (1-\lambda)^2 \tfrac{t}{N} + \lambda(2-\lambda)\tfrac{s}{N}}{((1-\lambda)\tfrac{t}{N}+\lambda \frac{s}{N})^2 (\tfrac{s}{N})^2}  \U^{B-1}_{N,s+1,2} \nonumber\\
&\quad +\frac{(t/N)^2 }{((1-\lambda)\tfrac{t}{N} + \lambda \beta) \beta} \U^B_{N,\lfloor \beta N \rfloor,2} + O\Big( \sqrt{\tfrac{\log N}{N}} \Big)\;.
\end{align*}
Taking $t = \lfloor wN \rfloor = wN + O(1)$ and using Riemann sum convergence properties yields
\begin{align}
(\tfrac{t}{N})^2 \frac{1}{N} \sum_{s = t}^{\lfloor \beta N \rfloor-1} \frac{1}{((1-\lambda)\tfrac{t}{N} + \lambda \tfrac{s}{N})\tfrac{s}{N}}
&= \frac{w^2}{N} \sum_{s = \lfloor w N \rfloor}^{\lfloor \beta N \rfloor-1} \frac{1}{((1-\lambda)w + \lambda \tfrac{s}{N})\tfrac{s}{N}} + O\big( \tfrac{1}{N}\big) \nonumber \\
&= w^2 \int_w^\beta \frac{du}{((1-\lambda)w + \lambda u)u} + O\big( \tfrac{1}{N}\big) \nonumber \\
&= \frac{w}{1-\lambda} \left( \int_w^\beta \frac{du}{ u} - \int_w^\beta \frac{du}{(1/\lambda-1)w + u}  \right) + O\big( \tfrac{1}{N}\big) \nonumber \\
&= \frac{w}{1 - \lambda} \left( - \log(w/\beta) - \log(1-\lambda + \lambda \beta/w) \right) + O\big( \tfrac{1}{N}\big) \nonumber \\
&= \frac{- w \log\big((1-\lambda) \tfrac{w}{\beta} + \lambda \big)}{1 - \lambda} + O\big( \tfrac{1}{N}\big)\;.
\end{align}
On the other hand,
\[
\frac{(t/N)^2 }{((1-\lambda)\tfrac{t}{N} + \lambda \beta) \beta} = \frac{w^2 }{((1-\lambda)w + \lambda \beta) \beta} + O(\tfrac{1}{N})\;,
\]
and Lemma \ref{lem:2grp-Ubb} gives for $k=2$ that $\phi^B_2(\alpha, \beta; \beta)$ exists, its expression is
\begin{equation}\label{eq:phiB2-bb}
\phi^B_2(\alpha, \beta; \beta)
= (1-\lambda)\beta^2 \sum_{b = 0}^B \left( \frac{1}{\beta} - \sum_{\ell = 0}^b \frac{\log(1/\beta)^\ell}{\ell !} \right)\;,       
\end{equation}
and it satisfies 
$\U^B_{N,\lfloor \beta N \rfloor,2}
= \phi^B_2(\alpha, \beta; \beta) + O\Big( \sqrt{\tfrac{\log N}{N}}\Big)$. Consequently,
\begin{align}
\U^B_{N,\lfloor w N \rfloor,2} 
&= - \lambda w \log\big((1-\lambda) \tfrac{w}{\beta} + \lambda \big) \nonumber\\
& \quad + \indic{B>0} \frac{w^2}{N} \sum_{s = \lfloor w N \rfloor}^{\lfloor \beta N \rfloor-1}  \frac{ (1-\lambda)^2 w + \lambda(2-\lambda)\tfrac{s}{N}}{((1-\lambda)w+\lambda \frac{s}{N})^2 (\tfrac{s}{N})^2}  \U^{B-1}_{N,s+1,2} \nonumber\\
&\quad +\frac{w^2 }{((1-\lambda)w + \lambda \beta) \beta} \phi^B_2(\alpha, \beta; \beta) + O\Big( \sqrt{\tfrac{\log N}{N}} \Big)\;. \label{aligneq:U-rec}
\end{align}



Using this equality, we will prove by induction over $B \geq 0$ that, for all $w \in [\alpha, \beta]$, the limit
$\phi^B(\alpha,\beta; w) :=  \lim_{N \to \infty} \U^B_{N,\lfloor w N \rfloor,2}$
exists, is continuous, and satisfies 
\[
\U^B_{N,\lfloor w N \rfloor,2} 
= \phi^B_2(\alpha,\beta; w) + O\Big( \sqrt{\tfrac{\log N}{N}} \Big)\;.
\]



\noindent
\textbf{Initialization.}
For $B=0$ and $w \in [\alpha, \beta]$, \eqref{aligneq:U-rec} gives immediately
\begin{align}
\U^B_{N,\lfloor wN \rfloor,2} 
&= - \lambda w \log\big((1-\lambda) \tfrac{w}{\beta} + \lambda \big) 
+\frac{w^2 }{((1-\lambda)w + \lambda \beta) \beta} \phi^B_2(\alpha, \beta; \beta) + O\Big( \sqrt{\tfrac{\log N}{N}} \Big)\;.
\end{align}

\noindent
\textbf{Induction.} Let $B \geq 1$, $w \in [\alpha, \beta]$, and assume that 
$\U^{B-1}_{N,\lfloor u N \rfloor,2} 
= \phi^B_2(\alpha,\beta; u) + O\Big( \sqrt{\tfrac{\log N}{N}} \Big)$ for all $u \in [\alpha, \beta]$, where the $O$ does not depend on $u$. Using this hypothesis for $u = \tfrac{s+1}{N}$, with $s \in \{t, \ldots,\lfloor \beta N\rfloor -1\}$, along with te continuity of $\phi^{B-1}_2(\alpha, \beta; \cdot)$ and Riemann sums convergence properties, yields
\begin{align*}
\frac{w^2}{N} \sum_{s = \lfloor w N \rfloor}^{\lfloor \beta N \rfloor-1} &  \frac{ (1-\lambda)^2 w + \lambda(2-\lambda)\tfrac{s}{N}}{((1-\lambda)w+\lambda \frac{s}{N})^2 (\tfrac{s}{N})^2}  \U^{B-1}_{N,s+1,2}\\
&= \frac{w^2}{N} \sum_{s = \lfloor w N \rfloor}^{\lfloor \beta N \rfloor-1}  \frac{ (1-\lambda)^2 w + \lambda(2-\lambda)\tfrac{s}{N}}{((1-\lambda)w+\lambda \frac{s}{N})^2 (\tfrac{s}{N})^2}  \phi^{B-1}_2(\alpha, \beta; \tfrac{s+1}{N}) + O\Big( \sqrt{\tfrac{\log N}{N}} \Big)\\
&= w^2 \int_w^\beta \frac{ (1-\lambda)^2 w + \lambda(2-\lambda)u}{((1-\lambda)w+\lambda u)^2 u^2}  \phi^{B-1}_2(\alpha, \beta; u) du + O\Big( \sqrt{\tfrac{\log N}{N}} \Big)\;.
\end{align*}
Therefore, we deduce by \eqref{aligneq:U-rec} that
\begin{align*}
\U^B_{N,\lfloor w N \rfloor,2} 
&= - \lambda w \log\big((1-\lambda) \tfrac{w}{\beta} + \lambda \big)\\
& \quad + w^2 \int_w^\beta \frac{ (1-\lambda)^2 w + \lambda(2-\lambda)u}{((1-\lambda)w+\lambda u)^2 u^2}  \phi^{B-1}_2(\alpha, \beta; u) du\\
&\quad +\frac{w^2 }{((1-\lambda)w + \lambda \beta) \beta} \phi^B_2(\alpha, \beta; \beta) + O\Big( \sqrt{\tfrac{\log N}{N}} \Big)\;.
\end{align*}
This proves that $\U^B_{N,\lfloor w N \rfloor,2} = \phi^B_2(\alpha,\beta;w) + O\Big( \sqrt{\tfrac{\log N}{N}} \Big)$, where 
\begin{align*}
\phi^B_2(\alpha,\beta;w)
&= - \lambda w \log\big((1-\lambda) \tfrac{w}{\beta} + \lambda \big)\\
& \quad + w^2 \int_w^\beta \frac{ (1-\lambda)^2 w + \lambda(2-\lambda)u}{((1-\lambda)w+\lambda u)^2 u^2}  \phi^{B-1}_2(\alpha, \beta; u) du\\
&\quad +\frac{w^2 }{((1-\lambda)w + \lambda \beta) \beta} \phi^B_2(\alpha, \beta; \beta)\;,
\end{align*}
where the expression of $\phi^B_2(\alpha, \beta; \beta)$ is given in \eqref{eq:phiB2-bb}.
\end{proof}










\subsection{Proof of Lemma \ref{lem:lim-2grp-k=1}}


\begin{proof}
Using Lemmas \ref{lem:recursion-U-alpha-beta}, \ref{lem:2grp-rho=s}, \ref{lem:2grp-rho>s} and \ref{lem:2grp-rho>betaN} for $k=2$, we obtain for all $t \in \{ \lfloor\alpha N \rfloor, \ldots, \lfloor\beta N \rfloor - 1\}$
\begin{align*}
\U^B_{N,t,1} 
&=   \frac{\lambda}{N} \sum_{s = t}^{\lfloor \beta N \rfloor-1} \left(\frac{\lambda t}{(1-\lambda)t + \lambda s} + O\Big( \sqrt{\tfrac{\log N}{N}} \Big) \right)\\
& \quad +  \frac{\indic{B>0}}{1-\lambda} \sum_{s = t}^{\lfloor \beta N \rfloor-1} \left( \frac{\lambda^2(1-\lambda)(s-t)t}{((1-\lambda)t + \lambda s)^2s} + O\Big( \sqrt{\tfrac{\log N}{N^3}} \Big) \right) \U^{B-1}_{N,s+1,2} \\
&\quad + \frac{t}{\beta N} \U^B_{N, \lfloor \beta N \rfloor, 1} + \frac{\lambda(\beta N -t)t}{((1-\lambda) t + \lambda \beta N) \beta N} \U^B_{N, \lfloor \beta N \rfloor, 2} +  O\Big(\sqrt{\tfrac{\log N}{N}}\Big)\\
&= \frac{\lambda^2 (t/N)}{N} \sum_{s = t}^{\lfloor \beta N \rfloor-1} \frac{1}{(1-\lambda)\tfrac{t}{N} + \lambda \tfrac{s}{N}}\\
& \quad + \indic{B>0}  \frac{\lambda^2(t/N)}{N} \sum_{s = t}^{\lfloor \beta N \rfloor-1} \frac{\tfrac{s}{N}-\tfrac{t}{N}}{((1-\lambda)\tfrac{t}{N} + \lambda \tfrac{s}{N})^2 \tfrac{s}{N}} \U^{B-1}_{N,s+1,2} \\
&\quad + \frac{t/N}{\beta} \U^B_{N, \lfloor \beta N \rfloor, 1} + \frac{\lambda(\beta - \tfrac{t}{N})\tfrac{t}{N}}{((1-\lambda) \tfrac{t}{N} + \lambda \beta) \beta} \U^B_{N, \lfloor \beta N \rfloor, 2} +  O\Big(\sqrt{\tfrac{\log N}{N}}\Big)\;.
\end{align*}
Consider in the following $t = \lfloor wN \rfloor = wN + O(1)$. Using Riemann sums convergence properties, we have
\begin{align*}
\frac{\lambda^2 (t/N)}{N} \sum_{s = t}^{\lfloor \beta N \rfloor-1} \frac{1}{(1-\lambda)\tfrac{t}{N} + \lambda \tfrac{s}{N}}
&= \lambda^2 w \int_w^\beta \frac{du}{(1-\lambda)w + \lambda u} + O(\tfrac{1}{N})\\
&= \lambda w \left[ \log((1-\lambda)w + \lambda u) \right]_w^\beta + O(\tfrac{1}{N})\\
&= \lambda w \log\big(1-\lambda + \lambda \tfrac{\beta}{w}\big)+ O(\tfrac{1}{N})\;.
\end{align*}
Since $\U^b_{N,s,k} \leq 1$ for all $b,s,k$, and $t = wN + O(1)$, as in the proof of Lemma \ref{lem:lim-2grp-k=2}, we obtain
\begin{align*}
\U^B_{N,\lfloor wN \rfloor,1} 
&= \lambda w \log\big(1-\lambda + \lambda \tfrac{\beta}{w}\big) + O(\tfrac{1}{N}) \nonumber\\
& \quad + \indic{B>0}  \frac{\lambda^2 w}{N} \sum_{s = \lfloor w N\rfloor}^{\lfloor \beta N \rfloor-1} \frac{\tfrac{s}{N}-w}{((1-\lambda)w + \lambda \tfrac{s}{N})^2 \tfrac{s}{N}} \U^{B-1}_{N,s+1,2} + O(\tfrac{1}{N}) \nonumber\\
&\quad + \frac{w}{\beta} \U^B_{N, \lfloor \beta N \rfloor, 1} + \frac{\lambda(\beta - w)w}{((1-\lambda) w + \lambda \beta) \beta} \U^B_{N, \lfloor \beta N \rfloor, 2} +  O\Big(\sqrt{\tfrac{\log N}{N}}\Big) \nonumber\\
&= \lambda w \log\big(1-\lambda + \lambda \tfrac{\beta}{w}\big) \nonumber\\
& \quad + \indic{B>0}  \frac{\lambda^2 w}{N} \sum_{s = \lfloor w N\rfloor}^{\lfloor \beta N \rfloor-1} \frac{\tfrac{s}{N}-w}{((1-\lambda)w + \lambda \tfrac{s}{N})^2 \tfrac{s}{N}} \U^{B-1}_{N,s+1,2} \nonumber \\
&\quad + \frac{w}{\beta} \phi^B_1(\alpha, \beta; \beta) + \frac{\lambda(\beta - w)w}{((1-\lambda) w + \lambda \beta) \beta} \phi^B_2(\alpha, \beta; \beta) +  O\Big(\sqrt{\tfrac{\log N}{N}}\Big)\;,
\end{align*}
where we used Lemma \ref{lem:2grp-Ubb} in the last inequality, which guarantees that $\U^B_{N, \lfloor \beta N \rfloor, k} = \phi^B_k(\alpha, \beta; \beta) + O\Big(\sqrt{\tfrac{\log N}{N}}\Big)$ for $k \in \{1,2\}$, with
\[
\phi^B_k(\alpha, \beta; \beta)
= \lambda_k\beta^2 \sum_{b = 0}^B \left( \frac{1}{\beta} - \sum_{\ell = 0}^b \frac{\log(1/\beta)^\ell}{\ell !} \right)\;.
\]
Denoting by $\phi^B(\alpha,\beta;\beta) = \phi^B_1(\alpha,\beta;\beta) + \phi^B_2(\alpha,\beta;\beta)$, i.e. $\phi^B(\alpha,\beta;\beta) = \tfrac{1}{\lambda_k}\phi^B_k(\alpha,\beta;\beta)$, we have
\begin{align*}
\frac{w}{\beta} \phi^B_1(\alpha, \beta; \beta) + \frac{\lambda(\beta - w)w}{((1-\lambda) w + \lambda \beta) \beta} \phi^B_2(\alpha, \beta; \beta)
&= \left( \frac{\lambda w}{\beta} + \frac{\lambda(1-\lambda)(\beta - w)w}{((1-\lambda) w + \lambda \beta) \beta}\right) \phi^B(\alpha, \beta; \beta)\\
&= \frac{\lambda w}{\beta}\left( 1 + \frac{(1-\lambda)(\beta - w)}{(1-\lambda) w + \lambda \beta}\right) \phi^B(\alpha, \beta; \beta)\\
&= \frac{\lambda w}{\beta}\left( \frac{\beta}{(1-\lambda) w + \lambda \beta}\right) \phi^B(\alpha, \beta; \beta)\\
&=  \frac{\lambda w}{(1-\lambda) w + \lambda \beta} \phi^B(\alpha, \beta; \beta)\;.
\end{align*}
Thus
\begin{align}
\U^B_{N,\lfloor wN \rfloor,1} 
&= \lambda w \log\big(1-\lambda + \lambda \tfrac{\beta}{w}\big) 
+ \frac{\lambda w}{(1-\lambda) w + \lambda \beta} \phi^B(\alpha, \beta; \beta) \nonumber\\
& \quad + \indic{B>0}  \frac{\lambda^2 w}{N} \sum_{s = \lfloor w N\rfloor}^{\lfloor \beta N \rfloor-1} \frac{\tfrac{s}{N}-w}{((1-\lambda)w + \lambda \tfrac{s}{N})^2 \tfrac{s}{N}} \U^{B-1}_{N,s+1,2} +  O\Big(\sqrt{\tfrac{\log N}{N}}\Big)\;. \label{aligneq:UB1-recursion}
\end{align}



Now, we will prove by induction over $B$ that $\U^B_{N, \lfloor w N \rfloor, 1} = \phi^B_1(\alpha, \beta; w) + O\Big(\sqrt{\tfrac{\log N}{N}}\Big)$ for all $w \in [\alpha, \beta]$, with $\phi^B_1(\alpha, \beta; \cdot)$ a continuous function satisfying the recursion stated in the Lemma.


\noindent
\textbf{Initialization} 
For $B = 0$, \eqref{aligneq:UB1-recursion} yields immediately for all $w \in [\alpha, \beta]$
\begin{align*}
\U^B_{N,\lfloor wN \rfloor,1} 
&= \lambda w \log\big(1-\lambda + \lambda \tfrac{\beta}{w}\big) 
+ \frac{\lambda w}{(1-\lambda) w + \lambda \beta} \phi^B(\alpha, \beta; \beta) +  O\Big(\sqrt{\tfrac{\log N}{N}}\Big)\;.
\end{align*}
\noindent
\textbf{Induction} Let $B \geq 1$, and assume that $\U^{B-1}_{N, \lfloor u N \rfloor, 1} = \phi^{B-1}_1(\alpha, \beta; u) + O\Big(\sqrt{\tfrac{\log N}{N}}\Big)$ for all $u \in [\alpha, \beta]$, and that $\phi^B_1(\alpha, \beta; \cdot)$ is continuous. Consequently, using Riemann sums convergence properties, it holds for all $w \in [\alpha, \beta]$ that
\begin{align*}
\frac{\lambda^2 w}{N} \sum_{s = \lfloor w N\rfloor}^{\lfloor \beta N \rfloor-1}& \frac{\tfrac{s}{N}-w}{((1-\lambda)w + \lambda \tfrac{s}{N})^2 \tfrac{s}{N}} \U^{B-1}_{N,s+1,2}\\
&= \frac{\lambda^2 w}{N} \sum_{s = \lfloor w N\rfloor}^{\lfloor \beta N \rfloor-1} \frac{\tfrac{s}{N}-w}{((1-\lambda)w + \lambda \tfrac{s}{N})^2 \tfrac{s}{N}} \left( \phi^{B-1}_2(\alpha, \beta; \tfrac{s+1}{N}) + O\Big(\sqrt{\tfrac{\log N}{N}}\Big) \right)\\
&= \frac{\lambda^2 w}{N} \sum_{s = \lfloor w N\rfloor}^{\lfloor \beta N \rfloor-1} \frac{\tfrac{s}{N}-w}{((1-\lambda)w + \lambda \tfrac{s}{N})^2 \tfrac{s}{N}} \phi^{B-1}_2(\alpha, \beta; \tfrac{s+1}{N}) + O\Big(\sqrt{\tfrac{\log N}{N}}\Big) \\
&= \lambda^2 w \int_w^\beta \frac{(u-w) \phi^{B-1}_2(\alpha, \beta; u)}{((1-\lambda)w + \lambda u)^2 u} du  + O\Big(\sqrt{\tfrac{\log N}{N}}\Big)\;,
\end{align*}
thus, we have by substituting into \eqref{aligneq:UB1-recursion}
\begin{align*}
\U^B_{N,\lfloor wN \rfloor,1} 
&= \lambda w \log\big(1-\lambda + \lambda \tfrac{\beta}{w}\big) 
+ \frac{\lambda w}{(1-\lambda) w + \lambda \beta} \phi^B(\alpha, \beta; \beta) \\
& \quad +  \lambda^2 w \int_w^\beta \frac{(u-w) \phi^{B-1}_2(\alpha, \beta; u)}{((1-\lambda)w + \lambda u)^2 u} du  +  O\Big(\sqrt{\tfrac{\log N}{N}}\Big)\\
&= \phi_1^B( \alpha, \beta; w) + O\Big(\sqrt{\tfrac{\log N}{N}}\Big)\;.
\end{align*}
\end{proof}



\subsection{Proof of Corollary \ref{cor:lb2grps}}
\begin{proof}
Assume that $\lambda \geq 1/2$. For any thresholds $0 < \alpha \leq \beta \leq 1$ Theorem \ref{thm:success-2grps} yields 
\[
\lim_{N \to \infty} \Pr(\A^B(\alpha,\beta) \text{ succeeds})
\geq \lambda \alpha \log\big(\tfrac{\beta}{\alpha}\big) + \alpha \beta S^B(\beta)\;.
\]
We will now determine thresholds maximizing this lower bound. For a fixed $\beta$, we have
\begin{align*}
\frac{\partial }{\partial \alpha} \big( \lambda \alpha \log\big(\tfrac{\beta}{\alpha}\big) + \alpha \beta S^B(\beta) \big)
\geq 0
&\iff \lambda \log\big(\tfrac{\beta}{\alpha}\big) - \lambda + \beta S^B(\beta) \geq 0\\
&\iff \lambda \log\big(\tfrac{\alpha}{\beta}\big) \leq  \beta S^B(\beta) - \lambda\\
&\iff \alpha \leq \frac{\beta}{e} \exp\left( \frac{\beta}{\lambda} S^B(\beta) \right)\;.
\end{align*}
This proves that, for fixed $\beta$ the lower bound $\lambda \alpha \log\big(\tfrac{\beta}{\alpha}\big) + \alpha \beta S^B(\beta)$ is maximized on $[0,\beta]$ for $\alpha = \min(\beta, \frac{\beta}{e} \exp( \tfrac{\beta}{\lambda} S^B(\beta) )) = h^B(\beta)$. With this choice of $\alpha$, the optimal choice of $\beta$ is the one maximizing the mapping $\beta \mapsto \lambda h^B(\beta) \log\big(\tfrac{\beta}{h^B(\beta)}\big) + h^B(\beta) \beta S^B(\beta)$. 

In particular, for $B = 0$, we obtain $\tilde{\alpha}_0 = \lambda \exp(\tfrac{1}{\lambda}-2)$ and $\tilde{\beta}_0 = \lambda$. They guarantee an asymptotic success probability of at least $\lambda^2 \exp(\tfrac{1}{\lambda}-2)$. Given that the sequence $(S^B(w))_B$ is non-decreasing for all $w \in (0,1]$, it holds for all $B \geq 0$ that
\begin{align*}
\lim_{N \to \infty} \Pr(\A^B(\tilde{\alpha}_B,\tilde{\beta}_B) \text{ succeeds})
&\geq \lambda \tilde{\alpha}_B \log\big(\tfrac{\tilde{\beta}_B}{\tilde{\alpha}_B}\big) + \tilde{\alpha}_B \tilde{\beta}_B S^B(\tilde{\beta}_B)\\
&= \max_{\alpha \leq \beta} \left\{ \lambda \alpha \log\big(\tfrac{\beta}{\alpha}\big) + \alpha \beta S^B(\beta) \right\}\\
&\geq \max_{\alpha \leq \beta} \left\{ \lambda \alpha \log\big(\tfrac{\beta}{\alpha}\big) + \alpha \beta S^0(\beta) \right\}\\
&= \lambda \tilde{\alpha}_0 \log\big(\tfrac{\tilde{\beta}_0}{\tilde{\alpha}_0}\big) + \tilde{\alpha}_0 \tilde{\beta}_0 S^0(\tilde{\beta}_0)\\
&= \lambda^2 \exp ( \tfrac{1}{\lambda} - 2)\\
&\geq \tfrac{1}{e} - (\tfrac{4}{e} - 1)\lambda(1-\lambda)\;.
\end{align*}
On the other hand, taking equal thresholds $\alpha = \beta = \tfrac{1}{e}$ then using Corollary \ref{cor:single-thresh-factorial-conv} with $K=2$ gives 
\begin{align*}
\lim_{N \to \infty} \Pr(\A^B(\tilde{\alpha}_B,\tilde{\beta}_B) \text{ succeeds})
&\geq \max_{\alpha \leq \beta} \left\{ \lambda \alpha \log\big(\tfrac{\beta}{\alpha}\big) + \alpha \beta S^B(\beta) \right\}\\
&\geq \frac{S^B(1/e)}{e^2}  \\
&\geq \frac{1}{e} - \frac{1}{e (B+1)!}\;.
\end{align*}
Thus, we deduce that
\[
\lim_{N \to \infty} \Pr(\A^B(\tilde{\alpha}_B,\tilde{\beta}_B) \text{ succeeds}) 
\geq \frac{1}{e} - \min\left\{ \frac{1}{e(B+1)!}, (\tfrac{1}{e}-1)\lambda(1-\lambda) \right\}\;.
\]
\end{proof}








\subsection{Proof of Theorem \ref{thm:success-2grps}}

\begin{proof}
By Lemmas \ref{lem:pr-cond-C_N}, \ref{lem:lim-2grp-k=2} and \ref{lem:lim-2grp-k=1}, The success probability of Algorithm $\A(\alpha,\beta)^B$ can be written as
\begin{align*}
\Pr(\A^B(\alpha,\beta) \text{ succeeds}) 
&= \Pr(\A^B_{\lfloor \alpha N \rfloor}(\alpha,\beta) \text{ succeeds} \mid \C_N) + O(\tfrac{1}{N^2})\\
&= \Pr(\A^B_{\lfloor \alpha N \rfloor}(\alpha,\beta) \text{ succeeds}, g^*_{\lfloor \alpha N\rfloor - 1} = 1 \mid \C_N) \\
&\quad + \Pr(\A^B_{\lfloor \alpha N \rfloor}(\alpha,\beta) \text{ succeeds}, g^*_{\lfloor \alpha N\rfloor - 1} = 2 \mid \C_N) + O(\tfrac{1}{N^2})\\
&= \U^B_{N,\lfloor \alpha N\rfloor,1} + \U^B_{N,\lfloor \alpha N\rfloor,2} + O(\tfrac{1}{N^2})\\
&= \phi_1^B( \alpha, \beta; \alpha) + \phi_2^B( \alpha, \beta; \alpha) + O\Big(\sqrt{\tfrac{\log N}{N}}\Big)\\
&= \lambda \alpha \log\big(1-\lambda + \lambda \tfrac{\beta}{\alpha}\big) 
+ \frac{\lambda \alpha \beta^2}{(1-\lambda) \alpha + \lambda \beta} \sum_{b = 0}^B \left( \frac{1}{\beta} - \sum_{\ell = 0}^b \frac{\log(1/\beta)^\ell}{\ell !} \right) \\
& \quad + \indic{B > 0} \lambda^2 \alpha \int_\alpha^\beta \frac{(u-\alpha) \phi^{B-1}_2(\alpha, \beta; u)}{((1-\lambda)\alpha + \lambda u)^2 u} du\\
&\quad - \lambda \alpha \log\big((1-\lambda) \tfrac{\alpha}{\beta} + \lambda \big) + \frac{(1-\lambda)\beta \alpha^2 }{(1-\lambda)\alpha + \lambda \beta}  \sum_{b = 0}^B \left( \frac{1}{\beta} - \sum_{\ell = 0}^b \frac{\log(1/\beta)^\ell}{\ell !} \right)\\
& \quad + \indic{B > 0} \alpha^2 \int_\alpha^\beta \frac{ (1-\lambda)^2 \alpha + \lambda(2-\lambda)u}{((1-\lambda)\alpha+\lambda u)^2 u^2}  \phi^{B-1}_2(\alpha, \beta; u) du + O\Big(\sqrt{\tfrac{\log N}{N}}\Big)\;,
\end{align*}
then, regrouping the terms yields
\begin{align*}
\Pr(&\A^B(\alpha,\beta) \text{ succeeds}) \\
&= \lambda \alpha \log\big(1-\lambda + \lambda \tfrac{\beta}{\alpha}\big) -  \lambda \alpha \log\big((1-\lambda) \tfrac{\alpha}{\beta} + \lambda \big)\\
& \quad + \frac{\alpha \beta}{(1-\lambda)\alpha + \lambda \beta} \left( \lambda \beta + (1-\lambda) \alpha  \right)  \sum_{b = 0}^B \left( \frac{1}{\beta} - \sum_{\ell = 0}^b \frac{\log(1/\beta)^\ell}{\ell !} \right)\\
&\quad + \indic{B > 0} \alpha \int_\alpha^\beta \left( \lambda^2 (1 - \tfrac{\alpha}{u}) + \tfrac{\alpha}{u} \big( (1-\lambda)^2 \tfrac{\alpha}{u} + \lambda(2-\lambda) \big) \right) \frac{ \phi^{B-1}_2(\alpha, \beta; u) du}{((1-\lambda)\alpha + \lambda u)^2} + O\Big(\sqrt{\tfrac{\log N}{N}}\Big)\;.
\end{align*}
Finally, observing that 
\begin{align*}
(1-\lambda)^2 \tfrac{\alpha}{u} + \lambda(2-\lambda) \big)
&= (1- \lambda)^2 \tfrac{\alpha^2}{u^2} + 2\lambda (1-\lambda) \tfrac{\alpha}{u} + \lambda^2\\
&= \tfrac{1}{u^2} \big( (1-\lambda) \alpha + \lambda u \big)^2\;,
\end{align*}
we deduce the result
\begin{align*}
\Pr&(\A^B(\alpha,\beta) \text{ succeeds})\\
&= \lambda \alpha \log\big(\tfrac{\beta}{\alpha}\big) + \alpha \beta \sum_{b = 0}^B \left( \frac{1}{\beta} - \sum_{\ell = 0}^b \frac{\log(1/\beta)^\ell}{\ell !} \right)
+ \indic{B > 0} \alpha \int_\alpha^\beta \frac{ \phi^{B-1}_2(\alpha, \beta; u) du}{u^2} + O\Big(\sqrt{\tfrac{\log N}{N}}\Big)\;,
\end{align*}
%where the $O$ term depends on $\alpha$.
\end{proof}








\section{Optimal memory-less algorithm for two groups}

\subsection{Proof of Lemma \ref{lem:memless}}

Before studying the optimal memory-less algorithm, we show that the state of a memory-less algorithm at any step $t$, fully characterizing its success probability, can be reduced to a few parameters, instead of the the history of the algorithm from the beginning of its execution.

\begin{proof}
Let $\A$ be a memory-less algorithm, and let us denote by $\tau$ its stopping time. Conditionally to the history of the algorithm until step $t-1$ and to the event $\{\tau \geq t\}$, the success probability of $\A$ depends on the future observations and the future actions of the algorithm.

Given that the algorithm is memory-less, at any step $s \geq t$, its actions $\act_{s,1}, \act_{s,2}$ depend on the observations $r_t, g_t, R_t$, the budget $B_t$ and $(|G^k_{s-1}|)_{k \in [K]}$. 

Conditionally to the cardinals of the groups at step $t-1$, the cardinals $(|G^k_{s-1}|)_{k \in [K]}$ are independent of the history $F_{t-1}$ because
\[
|G^k_{s-1}| = |G^k_{t-1}| + \sum_{u = t}^{s-1} \indic{g_u = k} \quad \forall k \in [K]\;,
\]
Moreover, since the candidates are observed in a uniformly random order, and the group memberships are also i.i.d random variables, then for all $s \geq 2$ distributions of $r_s, R_s$ depend only on the cardinals of each group at step $s-1$, on $g_s$ and $g^*_{s-1}$. Also $g^*_s$ is a function of $g^*_{s-1}, g_s$ and $\indic{R_s = 1}$:
\[
g^*_s = \indic{R_s \neq 1} g^*_{s-1} + \indic{R_s \neq 1} g_s\;,
\]
and the budget $B_s$ satisfies
\[
B_s = B_{s-1} - \indic{a_{s-1,2} = \acomp}\;.
\]

Therefore, Conditionally to the $B_t, (|G^k_{t-1}|)_{k \in [K]}, g^*_{t-1}$, 
the distributions of the observations and of the algorithm's actions at any step $s \geq t$ are independent of the history before step $t$.
\end{proof}







\subsection{Proof of Lemma \ref{lem:dynprog-rt-Rt}}
\begin{proof}
If $\S_t(\A) = (t,m,b,\ell)$, then in particular $g^*_{t-1} = \ell$, i.e. $\max G^\ell_{t-1} > \max G^k_{t-1}$, thus
\begin{align*}
\Pr(r_t = 1 \mid \S_t(\A) = (t,m,b,\ell), g_t = \ell) 
&= \Pr( x_t > \max G^\ell_{t-1} \mid \S_t(\A) = (t,m,b,\ell), g_t = \ell) \\
&= \Pr( x_t > \max x_{1:t-1} \mid \S_t(\A) = (t,m,b,\ell), g_t = \ell) \\
&= \frac{1}{t}\;,
\end{align*}
because the rank of $x_t$ among previous candidates is independent of their relative ranks and groups, thus independent of the state of the algorithm. Moreover, if $r_t = 1$, $g_t = 1$ and $g^*_{t-1} = \ell$, then $x_t$ is the better than the maximum of $G^\ell_{t-1}$, which is the maximum of $x_{1:t-1}$, thus necessarily $R_t = 1$,
\[
\Pr(R_t = 1 \mid \S_t(\A) = (t,m,b,\ell), g_t = \ell, r_t = 1) = 1\;.
\]

On the other hand, if $g_t = k \neq \ell = g^*_{t-1}$, assume that $|G^\ell_{t-1}| > 0$. It holds that
\begin{align*}
\Pr(r_t = 1 \mid \S_t(\A) = (t,m,b,\ell), g_t = k) 
&= \Pr( r_t = 1 \mid g^*_{t-1} = \ell, g_t = k, |G^1_{t-1}| = m) \\
&= \frac{\Pr( r_t = 1, g^*_{t-1} = \ell \mid g_t = k, |G^1_{t-1}| = m)}{\Pr(g^*_{t-1} = \ell \mid |G^1_{t-1}| = m)}\;.
\end{align*}
We have immediately that 
\[
\Pr(g^*_{t-1} = \ell \mid |G^1_{t-1}| = m)
= \Pr(\max G^\ell_{t-1} > \max G^k_{t-1} \mid |G^1_{t-1}| = m)
= \frac{|G^\ell_{t-1}|}{t-1}\;,
\]
and the numerator can be computed as
\begin{align}
\Pr( r_t = 1, g^*_{t-1} = \ell \mid g_t = k, |G^1_{t-1}| = m)
=& \Pr( x_t > \max G^k_{t-1}, g^*_{t-1} = \ell\mid  |G^1_{t-1}| = m) \nonumber\\ 
&= \Pr(x_t > \max G^k_{t-1}, \max G^\ell_{t-1} > \max G^k_{t-1} \mid |G^1_{t-1}| = m)\nonumber \\
&= \Pr(x_t > \max G^\ell_{t-1} > \max G^k_{t-1} \mid |G^1_{t-1}| = m) \nonumber \\
&\quad + \Pr(\max G^\ell_{t-1} > x_t > \max G^k_{t-1} \mid |G^1_{t-1}| = m) \nonumber \\
&= \frac{1}{t}\cdot \frac{|G^\ell_{t-1}|}{t-1} + \frac{|G^\ell_{t-1}|}{t} \cdot \frac{1}{|G^k_{t-1}|+1} \nonumber \\
&= \frac{|G^\ell_{t-1}|}{t} \left( \frac{1}{t-1} + \frac{1}{|G^k_{t-1}| + 1} \right)\;, \label{eq:term(q)}
\end{align}
which yields
\begin{align*}
\Pr(r_t = 1 \mid \S_t(\A) = (t,m,b,\ell), g_t = k) 
&= \frac{t-1}{|G^\ell_{t-1}|} \cdot \frac{|G^\ell_{t-1}|}{t} \left( \frac{1}{t-1} + \frac{1}{|G^k_{t-1}| + 1} \right)\\
&= \frac{1}{t} \left( 1 + \frac{t-1}{|G^k_{t-1}| + 1} \right)\\
&= \frac{|G^k_{t-1}|+t}{t(|G^k_{t-1}|+1)}\;.
\end{align*}

Finally, 

\begin{align*}
\Pr(R_t = 1 \mid \S_t(\A) = (t,m,b,\ell), g_t = k, r_t = 1) 
&= \frac{\Pr(R_t = 1, r_t = 1, g^*_{t-1} = \ell \mid g_t = k, |G^1_{t-1}|) }{\Pr(r_t = 1, g^*_{t-1} = \ell \mid g_t = k, |G^1_{t-1}|)}\;.
\end{align*}
We computed the denominator term in \eqref{eq:term(q)}, and the numerator satisfies
\begin{align*}
\Pr(R_t = 1, r_t = 1, g^*_{t-1} = \ell \mid g_t = k, |G^1_{t-1}|)
&= \Pr(R_t = 1, g^*_{t-1} = \ell \mid g_t = k, |G^1_{t-1}|)\\
&= \Pr(x_t > \max G^\ell G^\ell_{t-1} > \max G^k_{t-1} \mid |G^1_{t-1}|)\\
&= \frac{1}{t} \cdot \frac{|G^\ell_{t-1}|}{t-1}\;,
\end{align*}
hence
\begin{align*}
\Pr(R_t = 1 \mid \S_t(\A) = (t,m,b,\ell), g_t = k, r_t = 1) 
&= \frac{ \frac{|G^\ell_{t-1}|}{t(t-1)} }{\frac{|G^\ell_{t-1}|}{t} \left( \frac{1}{t-1} + \frac{1}{|G^k_{t-1}| + 1} \right)}\\
&= \frac{1}{ 1 + \frac{t-1}{|G^k_{t-1}| + 1} }\\
&= \frac{|G^k_{t-1}| + 1}{|G^k_{t-1}| + t}\;.
\end{align*}
This concludes the proof when $|G^\ell_{t-1}| > 0$. If $|G^\ell_{t-1}| = 0$, then the same identities remain trivially true.
\end{proof}






\subsection{Proof of Theorem \ref{thm:opt-memless}}
\begin{proof}
Using the results from Section \ref{sec:state-transition} and \ref{sec:expected-rwd}, the actions of $\A_*$ and the resulting state transitions are as follows.
If the state of $\A_*$ at step $t$ is $\S_t(\A_*) = (t,B,m,\ell)$ for some $B \geq 1$, $m<t$ and $\ell \in \{1,2\}$:
If $g_t = \ell$, denoting by $M_\ell = m + \indic{\ell = 1}$, we have
\begin{itemize}
    \item with probability $1-1/t$: $r_t = 0$, and the algorithm rejects the candidate, transitioning to the state $(t+1,B,M_\ell,\ell)$.
    \item with probability $1/t$: $r_t = 1$, and necessarily $R_t = 1$, because $g_t = g^*_{t-1} = \ell$. 
    \begin{itemize}
        \item If $\rwd^B_{t,m,k}(\acomp) > \rwd^B_{t,m,k}(\askip)$, then the algorithm uses a comparison and observes $R_t = 1$, hence accepts the candidate. The success probability in that case is $t/N$.
        \item Otherwise, the candidate is rejected and the algorithm goes to state $(t+1,B,M_\ell,\ell)$
    \end{itemize}
\end{itemize}
On the other hand, if $g_t = k \neq g^*_{t-1}$, then denoting by $M_k = m + \indic{k = 1}$, we have
\begin{itemize}
    \item with probability $\frac{|G^k_{t-1}|(t+1)}{t(|G^k_{t-1}|+1)}$: $r_t = 0$, and the algorithm rejects the candidate, transitioning to the state $(t+1,B,M_k,\ell)$.
    \item with probability $\frac{|G^k_{t-1}| + t}{t(|G^k_{t-1}|+1)}$: $r_t = 1$
    \begin{itemize}
        \item If $\rwd^B_{t,m,k}(\acomp) > \rwd^B_{t,m,k}(\askip)$, then the algorithm uses a comparison
        \begin{itemize}
            \item with probability $\frac{|G^k_{t-1}| + 1}{|G^k_{t-1}|+t}$: $R_t = 1$ and the algorithm stops, its success probability is $t/N$
            \item with probability $\frac{t- 1}{|G^k_{t-1}|+t}$: $R_t = 0$, the candidate is rejected, and the algorithm goes to state $(t+1,B-1,M_k,\ell)$
        \end{itemize}
        \item Otherwise, the candidate is rejected and 
        \begin{itemize}
            \item with probability $\frac{|G^k_{t-1}| + 1}{|G^k_{t-1}|+t}$: the algorithm goes to state $(t+1,B, M_k,k)$
            \item with probability $\frac{t- 1}{|G^k_{t-1}|+t}$: the algorithm goes to state $(t+1,B, M_k,\ell)$
        \end{itemize}
    \end{itemize}
\end{itemize}

In the case of a zero budget, the algorithm compares $\rwd^B_{t,m,k}(\askip)$ to $\rwd^B_{t,m,k}(\acomp)$ instead of $\rwd^B_{t,m,k}(\acomp)$. If the algorithm decides to reject the candidate then the same state transition occurs. However, if the candidate is selected and if $g_t = g^*_{t-1} = \ell$ then the success probability is $t/N$. If On the other hand, if it is selected and $g_t = k \neq g^*_{t-1} = \ell$ then the probability that the current candidate is the best overall is $\frac{|G^k_{t-1}| + 1}{|G^k_{t-1}|+t} \times \frac{t}{N}$.

All in all, for $B = 0$, then
\begin{align*}
(\V^0_{t,m,\ell} \mid g_t = \ell)
&= \frac{1}{t}\left( \delta^0_\ell \frac{t}{N} + (1 - \delta^0_\ell) \V^0_{t+1,M_\ell,\ell}  \right) + \left(1 - \frac{1}{t} \right) \V^0_{t+1,M_\ell,\ell}\\
&= \left(1 - \tfrac{\delta^0_\ell}{t}\right)\V^0_{t+1,M_\ell,\ell} + \tfrac{\delta^0_\ell}{N}\\
(\V^0_{t,m,\ell} \mid g_t = k)
&= \tfrac{|G^k_{t-1}| + t}{t(|G^k_{t-1}|+1)}\left( \delta^0_k \tfrac{t(|G^k_{t-1}| + 1)}{N(|G^k_{t-1}|+t)} + (1-\delta^0_k) \left( \tfrac{|G^k_{t-1}| + 1}{|G^k_{t-1}|+t} \V^0_{t+1,M_\ell,l} + \tfrac{t- 1}{|G^k_{t-1}|+t} \V^0_{t+1,M_\ell,\ell} \right) \right)\\
&\quad + \tfrac{|G^k_{t-1}| + t}{t(|G^k_{t-1}|+1)}\V^0_{t+1,M_k,\ell}\\
&=  \tfrac{\delta^0_k}{N} + \tfrac{1-\delta^0_k}{t} \V^0_{t+1,M_k,k} + \left(1 - \tfrac{1}{t}\right)\left(2 - \delta^0_k - \tfrac{1}{|G^k_{t-1}|+1} \right) \V^0_{t+1,M_k,\ell}\;,
\end{align*}
and we deduce that 
\begin{align*}
\V^0_{t,m,\ell}
&= \lambda_\ell \left( \left(1 - \tfrac{\delta^0_\ell}{t}\right)\V^0_{t+1,M_\ell,\ell} + \tfrac{\delta^0_\ell}{N} \right) 
+ \lambda_k \left( \tfrac{\delta^0_k}{N} + \tfrac{1-\delta^0_k}{t} \V^0_{t+1,M_k,k} + \left(1 - \tfrac{1}{t}\right)\left(2 - \delta^0_k - \tfrac{1}{|G^k_{t-1}|+1} \right) \V^0_{t+1,M_k,\ell} \right)\;.
\end{align*}

For $B \geq 1$, we obtain
\begin{align*}
(\V^B_{t,m,\ell} \mid g_t = \ell)
&= \frac{1}{t}\left( \delta^B_\ell \frac{t}{N} + (1 - \delta^B_\ell) \V^B_{t+1,M_\ell,\ell}  \right) + \left(1 - \frac{1}{t} \right) \V^B_{t+1,M_\ell,\ell}\\
&= \left(1 - \tfrac{\delta^B_\ell}{t}\right)\V^B_{t+1,M_\ell,\ell} + \tfrac{\delta^B_\ell}{N}\\
(\V^B_{t,m,\ell} \mid g_t = k)
&= \tfrac{|G^k_{t-1}| + t}{t(|G^k_{t-1}|+1)}\bigg[ \delta^B_k \left( \tfrac{|G^k_{t-1}| + 1}{|G^k_{t-1}|+t} \V^B_{t+1,M_k,k} + \tfrac{t - 1}{|G^k_{t-1}|+t} \V^B_{t+1,M_k,\ell} \right) \\
&\quad + (1-\delta^B_k) \left( \tfrac{|G^k_{t-1}| + 1}{|G^k_{t-1}|+t} \V^0_{t+1,M_k,l} + \tfrac{t- 1}{|G^k_{t-1}|+t} \V^0_{t+1,M_k,\ell} \right) \bigg] + \tfrac{|G^k_{t-1}| + t}{t(|G^k_{t-1}|+1)}\V^B_{t+1,M_k,\ell}\\
&=  \tfrac{\delta^B_k}{N} + \tfrac{\delta^B_k}{|G^k_{t-1}|+1}\big( 1 - \tfrac{1}{t}\big) \V^{B-1}_{t+1,M_k,\ell} + \tfrac{1-\delta^B_k}{t} \V^{B}_{t+1,M_k,k} + \big(1-\tfrac{1}{t}\big)\big( 1 - \tfrac{\delta^B_k}{|G^k_{t-1}|+1}\big)  \V^{B}_{t+1,M_k,\ell}\;,
\end{align*}
hence
\begin{align*}
\V^B_{t,m,\ell}
&= \lambda_\ell \left( \tfrac{\delta^B_\ell}{N} + \big( 1 - \tfrac{\delta^B_\ell}{t} \big) \V^B_{t+1,M_\ell,\ell} \right) \\
&\quad+ \lambda_k \left( \tfrac{\delta^B_k}{N} + \tfrac{\delta^B_k}{|G^k_{t-1}|+1}\big( 1 - \tfrac{1}{t}\big) \V^{B-1}_{t+1,M_k,\ell} + \tfrac{1-\delta^B_k}{t} \V^{B}_{t+1,M_k,k} + \big(1-\tfrac{1}{t}\big)\big( 1 - \tfrac{\delta^B_k}{|G^k_{t-1}|+1}\big)  \V^{B}_{t+1,M_k,\ell}  \right)\;,
\end{align*}
which concludes the proof.
\end{proof}




\end{document}
