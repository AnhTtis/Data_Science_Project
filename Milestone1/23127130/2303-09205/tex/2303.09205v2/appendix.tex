% Appendix
\newpage

\appendix

\setcounter{page}{1}
\begin{center}
    {\Large\bf Supplementary material for ``Addressing bias in online selection with limited budget of comparisons’’}\\
    \vspace{1cm}
\end{center}
The appendices are organized as follows. 
In Appendices \ref{appendix:analysisDDT} and \ref{appendix:optMemless}, we give the omitted proofs from the paper. Then in Appendix \ref{appx:choose-opt-thresh} we show how to numerically compute the optimal thresholds $(\alpha_b, \beta_b)_{b=0}^B$ exploiting our previous results. Finally, in appendix \ref{appx:bench-algo}, we study non-asymptotically the unique-threshold algorithm, which is a particular case of DDT algorithms with $\alpha_0 = \ldots = \alpha_B = \beta_0 = \ldots = \beta_B$. We give a lower bound showing the convergence rate of its success probability to the limit we exhibited in the paper. This analysis uses different technique proofs and explicitly shows the impact of $\lambda$ on this convergence.



\begin{center}
    {\large CONTENTS}
\end{center}
\startcontents[appendices]
\printcontents[appendices]{}{1}{\setcounter{tocdepth}{2}}



\subsection{DDT with $B=0$: Proofs of Theorem \ref{thm:prob-win0} and Proposition \ref{prop:asymptoticDDT0}}\label{appx:DDT-0}

\subsubsection{Proof of Theorem \ref{thm:prob-win0}}
\begin{proof}
Let $\alpha_0,\beta_0 \in (0,1)$. We will consider for now that $\alpha_0 \leq \beta_0 \in (0,1)$, and we will compute the asymptotic success probability of $\A(\alpha_0, \beta_0)$.
We define the event
\begin{equation}\label{eq:conc_0comp}\tag{$\mathcal{C}_{\alpha_0}$}
    \forall \alpha_0 N \leq t \leq N \; : \;
            \left| |G^1_t| - \lambda t \right|
            < A_{\alpha_0} \sqrt{t \log N}
\end{equation}
we have from Lemma \ref{lem:conc-card} that $\mathcal{C}_{\alpha_0}$ is true with probability at least $1 - \frac{2}{\alpha_0 N}$, where  $A_{\alpha_0} \eqdef \dfrac{1+\sqrt{{\alpha_0}}}{2\sqrt{2}{\alpha_0}^{1/4}}$, hence we can write
\begin{align}
\Prob(\A(\alpha_0, \beta_0) \text{ succeeds}) \nonumber \label{eq:prob0-A20}
&= \Prob(\A(\alpha_0, \beta_0) \text{ succeeds} \mid \mathcal{C}_{\alpha_0})\left( 1 - O(1/N) \right)\\
&= \Prob(\A(\alpha_0, \beta_0) \text{ succeeds} \mid \mathcal{C}_{\alpha_0}) - O(1/N)
\end{align}
where the $O$ constant depends only on $\alpha_0$. Since $\mathcal{C}_{\alpha_0}$ only depends on the groups of the candidates and not their values, it is independent of the position of the best candidate, therefore by denoting $\tau$ the stopping time of $\A(\alpha_0, \beta_0)$ and $t^\star$ the position of best candidate we have

\begin{align}
\Prob(\A(\alpha_0, \beta_0) \text{ succeeds} \mid \mathcal{C}_{\alpha_0}) \nonumber \label{eq:prob1-A20}
&= \sum_{t=\alpha_0 N}^N \Prob(\tau = t \mid t = t^\star,  \mathcal{C}_{\alpha_0}) \Prob(t = t^\star)\\
&= \frac{1}{N} \sum_{t=\alpha_0 N}^N \Prob(\tau = t \mid t = t^\star,  \mathcal{C}_{\alpha_0}).
\end{align}
For $t \in \{ \alpha_0 N, \ldots , \beta_0 N - 1\}$, since group $G^2$ is still being observed, $\A(\alpha_0, \beta_0)$ will stop on $t$ only if $g_t = 1$ and $x_t$ is the first candidate of $G^1$ such that $x_t > \best G^1_{\alpha_0 N}$. Conditioning on $t = t^\star$, this is equivalent to simply having $g_t = 1$ and $\best G^1_{t-1} \in G^1_{\alpha_0 N-1}$ hence
\begin{align*}
\Prob(\tau = t\mid t = t^\star,  \mathcal{C}_{\alpha_0})
&= \Prob(g_t = 1 \text{ and } \best G^1_{t-1} \in G^1_{\alpha_0 N - 1} \mid t = t^\star,  \mathcal{C}_{\alpha_0})\\
&= \lambda \Prob(\best G^1_{t-1} \in G^1_{\alpha_0 N - 1} \mid \mathcal{C}_{\alpha_0})\\
&= \lambda \E \left[ \left. \frac{ |G^1_{\alpha_0 N - 1}|}{|G^1_{t-1}|} \right| \mathcal{C}_{\alpha_0} \right]
\end{align*}
Conditioning on $\mathcal{C}_{\alpha_0}$ we have that
\begin{align*}
    &|G^1_{\alpha_0 N - 1}| 
    = |G^1_{\alpha_0 N}| + O(1)
    = \lambda \alpha_0 N + O(\sqrt{N \log N})
    = \lambda \alpha_0 N\left( 1 + O\left(\sqrt{\log N / N}\right)\right)\\
    &|G^1_{t-1}|
    = |G^1_{t}| + O(1)
    = \lambda t + O(\sqrt{N \log N})
    = \lambda t \left( 1 + O\left(\sqrt{\log N / N}\right)\right)\\
\end{align*}
where the $O$ constants only depend on $\lambda$ and $\alpha_0$ because $\alpha_0 N \leq t \leq N$, this gives 
\begin{align}
\Prob(\tau = t \mid t = t^\star,  \mathcal{C}_{\alpha_0}) \nonumber \label{eq:prob2-A20}
&= \frac{\lambda^2 \alpha_0 N\left( 1 + O\left(\sqrt{\log N / N}\right)\right)}{\lambda t \left( 1 + O\left(\sqrt{\log N / N}\right)\right)}\\ 
&= \frac{\lambda \alpha_0 N}{t} + O\left( \sqrt{\frac{\log N}{N}} \right)\enspace.
\end{align}

For $t \in \{\beta_0 N, \ldots, N\}$, $\A(\alpha_0, \beta_0)$ stops on $t$ if $x_t$ is the best candidate of its group so far, i.e $r_t = 1$. In order for $\A(\alpha_0, \beta_0)$ to reach the step $t$ and not stop before it, we must have $\best G^1_{t-1} \in G^1_{\alpha_0 N - 1}$ and $\best G^2_{t-1} \in G^2_{\beta_0 N - 1}$, which means that no candidate before $x_t$ surpasses the maximum of its group from the exploration phase, and conditioning on $t = t^\star$, the previous condition is guarantees that the algorithm will stop on $t$, therefore
\begin{align*}
\Prob(\tau = t & \mid t = t^\star,  \mathcal{C}_{\alpha_0})\\
&= \Prob( \best G^1_{t-1} \in G^1_{\alpha_0 N - 1} \text{ and } \best G^2_{t-1} \in G^2_{\beta_0 N - 1} \mid \mathcal{C}_{\alpha_0})\\
&= \E \left[ \left. \frac{ |G^1_{\alpha_0 N - 1}|}{|G^1_{t-1}|} \times \frac{ |G^2_{\beta_0 N - 1}|}{|G^2_{t-1}|}\right| \mathcal{C}_{\alpha_0} \right],
\end{align*}
proceeding as we did for earlier in the proof, we obtain that
\begin{equation}\label{eq:prob3-A20}
\Prob(\tau = t \mid t = t^\star,  \mathcal{C}_{\alpha_0})
= \frac{\alpha_0 \beta_0 N^2}{t^2} + O\left( \sqrt{\frac{\log N}{N}} \right)
\end{equation}
where the $O$ constant only depend on $\lambda$ and $\alpha_0$. ($\alpha_0 \leq \beta_0 \leq 1)$\\
Combining Equations \ref{eq:prob0-A20}, \ref{eq:prob1-A20}, \ref{eq:prob2-A20} and \ref{eq:prob3-A20} we obtain that
\begin{align*}
\Prob(\A(\alpha_0, \beta_0)& \text{ succeeds})\\
&= \lambda \alpha_0 \sum_{t=\alpha_0 N}^{\beta_0 N - 1} \frac{1}{t}
+ \alpha_0 \beta_0 N \sum_{t=\beta_0 N}^{N} \frac{1}{t^2} + O\left(\sqrt{\frac{\log N}{N}}\right)\\
&= \lambda \alpha_0 \left( \log\frac{\beta_0}{\alpha_0} + O(1/N) \right)
  + \alpha_0 \beta_0 N \left( \frac{1}{\beta_0 N} - \frac{1}{N} + O(1/N^2) \right) + O\left(\sqrt{\frac{\log N}{N}}\right)\\
 &= \lambda \alpha_0 \log\frac{\beta_0}{\alpha_0} + \alpha_0 - \alpha_0 \beta_0 + O\left(\sqrt{\frac{\log N}{N}}\right)\enspace.
\end{align*}
The limit success probability is therefore 
\[
\Tilde{\Prob}(\A(\alpha_0, \beta_0) \text{ succeeds}) 
= \lambda \alpha_0 \log\frac{\beta_0}{\alpha_0} + \alpha_0 - \alpha_0 \beta_0,
\]
differentiating with respect to $\alpha_0$ then $\beta_0$ gives that this success probability is maximal over $\{\alpha \leq \beta \in (0,1) \}$ for the thresholds
\[
\alpha_0^\star =  \lambda \exp\left( \frac{1}{\lambda} - 2 \right) 
\quad \text{ and } \quad
\beta_0^\star = \lambda.
\]
Observe that we have indeed $\alpha_0 \leq \beta_0$ with equality only for $\lambda = 1/2$, and
\[
\Tilde{\Prob}(\A(\alpha_0^\star, \beta_0^\star))
=  \lambda^2 \exp\left( \frac{1}{\lambda} - 2 \right)\enspace.
\]

Now if we consider the case $\beta_0 \leq \alpha_0$, since $G^1, G^2$ play symmetric roles, we obtain that the optimal thresholds are $\alpha_0' = 1 - \lambda$ and $\beta_0' = (1-\lambda)^2\exp(1/(1-\lambda) - 2)$, and the limit success probability is 
$
\Tilde{\Prob}(\A(\alpha_0', \beta_0'))
=  (1-\lambda)^2 \exp\left( \frac{1}{1-\lambda} - 2 \right)\enspace.
$\\
Assuming that $\lambda \geq 1/2$, and given that the function $x \in (0,1) \mapsto x^2 \exp\left( \frac{1}{x} - 2 \right)$ is increasing, we deduce that 
\[
\Tilde{\Prob}(\A(\alpha_0', \beta_0')) \leq 
\Tilde{\Prob}(\A(\alpha_0^\star, \beta_0^\star)).
\]
and thus the result.
\end{proof}










\subsubsection{Proof of Proposition \ref{prop:asymptoticDDT0}}
\begin{proof}
Since $\tau \geq \alpha_0 N$ with probability 1, we have for any $w \leq \alpha_0$
\[
\Prob(\A_2(\alpha_0, \beta_0) \text{ succeeds} \mid \tau \geq w N)
= \Prob(\A_2(\alpha_0, \beta_0) \text{ succeeds})
\]
thus it converges to $\varphi_0((\alpha_0, \beta_0), \alpha_0)$ as proved in Theorem \ref{thm:prob-win0}. If $\alpha_0 \leq w < \beta_0$ then the success probability of $\A_2(\alpha_0, \beta_0)$ knowing that $\tau \geq w$ is the same as the success probability of $\A_2(w, \beta_0)$, and we have from Theorem \ref{thm:prob-win0} that is converges to $\varphi_0((\alpha_0, \beta_0);w)$. Moreover, we have from the proof of Theorem \ref{thm:prob-win0} that
\[
\Prob(\A_2(\alpha_0, \beta_0) \text{ succeeds} \mid \tau \geq w N)
= \lambda w \log \frac{\beta_0}{w} + w(1 - \beta_0) + O\left( \sqrt{\frac{\log N}{N}}\right),
\]
the $O$ constant depends in that case on $w$, but we can make it only dependent of $\alpha_0$ using the bounds $\alpha_0 \leq w \leq 1$.\\
Now for $\beta_0 \leq w \leq 1$, denoting $t^\star$ the position of the best candidate overall and $\tau$ the stopping time of the algorithm, we have
\begin{align*}
\Prob(\A_2(\alpha_0, \beta_0) \text{ succeeds} \mid \tau \geq wN)
&= \Prob(\tau = t^\star \mid \tau \geq wN)\\
&= \sum_{t=wN}^N \Prob(t=t^\star \mid \tau = t) \frac{\Prob(\tau = t)}{\Prob(\tau \geq wN)}\\
&= \frac{1}{\Prob(\tau \geq wN)} \sum_{t=wN}^N \Prob(\tau = t \mid t=t^\star) \Prob(t = t^\star)\\
&= \frac{1}{N\Prob(\tau \geq wN)} \sum_{t=wN}^N \Prob(\tau = t \mid t=t^\star).
\end{align*}
where the last equation is obtained using the Bayes rule.
By Lemma \ref{lem:conc-card}, the event
\begin{equation}
%\label{eq:conc_0comp}
\tag{$\mathcal{C}_{\alpha_0}$}
    \forall \alpha_0 N \leq t \leq N \; : \;
            \left| |G^1_t| - \lambda t \right|
            < A_{\alpha_0} \sqrt{t \log N}
\end{equation}
is true with probability at least $1 - \frac{2}{\alpha_0 N}$, where  $A_{\alpha_0}$ a constant depending on $\alpha_0$. We showed in the proof of Theorem \ref{thm:prob-win0} that
\[
\Prob(\tau = t \mid t = t^\star) = \Prob(\tau = t \mid t = t^\star, \mathcal{C}_{\alpha_0})(1 - O(1/N))
= \frac{\alpha_0 \beta_0 N^2}{t^2} + O\left( \sqrt{\frac{\log N}{N}}\right),
\]
with the same proof technique we derive
\begin{align*}
\Prob(\tau \geq wN \mid \mathcal{C}_{\alpha_0})
&= \Prob(\best G^1_{wN-1} \in G^1_{\alpha_0 N-1}, \best G^2_{wN-1} \in G^2_{\beta_0 N-1} \mid \mathcal{C}_{\alpha_0})\\
&= \E\left[ \left. \frac{|G^1_{\alpha_0 N - 1}|}{|G^1_{wN-1}|}
   \frac{|G^2_{\beta_0 N - 1}|}{|G^2_{wN-1}|} \right| \mathcal{C}_{\alpha_0} \right] \\
&= \left( \frac{\alpha_0}{w} +  O\left( \sqrt{\frac{\log N}{N}}\right) \right) \left( \frac{\beta_B}{w} +  O\left( \sqrt{\frac{\log N}{N}}\right) \right)\\
&= \frac{\alpha_0 \beta_0}{w^2} + O\left( \sqrt{\frac{\log N}{N}}\right),
\end{align*}
thus 
\[
\Prob(\tau \geq wN)
= \Prob(\tau \geq wN \mid \mathcal{C}_{\alpha_0})(1-O(1/N))
= \frac{\alpha_0 \beta_0}{w^2} + O\left( \sqrt{\frac{\log N}{N}}\right)\enspace.
\]
the $O$ can be made independent of $w$ using that $\alpha_0 \leq w \leq 1$. Finally we obtain
\begin{align*}
\Prob(\A_2(\alpha_0, \beta_0) &\text{ succeeds} \mid \tau \geq wN, \mathcal{C}_{\alpha_0})\\
&= \left( \frac{w^2}{\alpha_0 \beta_0} + O\left( \sqrt{\frac{\log N}{N}}\right) \right)\frac{1}{N} \sum_{t=wN}^N
\left( \frac{\alpha_0 \beta_0 N^2}{t^2} + O\left( \sqrt{\frac{\log N}{N}}\right) \right)\\
&= \left( \frac{w^2}{\alpha_0 \beta_0} + O\left( \sqrt{\frac{\log N}{N}}\right) \right) 
\left( \alpha_0 \beta_0\left(\frac{1}{w} - 1\right) + 
O\left( \sqrt{\frac{\log N}{N}}\right) 
\right)\\
&= w - w^2 + O\left( \sqrt{\frac{\log N}{N}}\right)\enspace.
\end{align*}
\end{proof}




\subsubsection{Strong optimality of the thresholds $\alpha^\star_0, \beta^\star_0$}
The following Proposition states that for any DDT algorithm with budget $B \geq 0$, if at some step $w N$ the algorithm still did not stop and does not have any comparisons left, then its success probability is upper bounded by $\varphi_0(w; (\alpha_0^\star, \beta_0^\star))$. In particular, this implies that whatever is the initial budget $B$, choosing $\alpha_0 = \alpha^\star_0, \beta_0 = \beta^\star_0$ is optimal.

\begin{proposition}\label{prop:DDT0-optanyw}
For all $w \in [0, 1]$ and $\lambda \geq 1/2$, the mapping $(\alpha_0, \beta_0) \mapsto\varphi_0(w; (\alpha_0, \beta_0))$ is maximized over $\alpha_0 \leq \beta_0$ at
\[
    \alpha_0^\star = \lambda \exp\left( \frac{1}{\lambda} - 2\right)
    \quad \text{ and } \quad
    \beta_0^\star  = \lambda\enspace.
\]

\end{proposition}


\begin{proof}
   Fix some $\lambda \geq 1/2$. For convenience, let us drop the subscript $0$ from $g_0$.
%    and let $g(a, b) = \lambda a\log(b/a) + a(1 - b)$. Then, $\phi_0$ can be written as
%    \[
% \varphi_0((\alpha, \beta);w)
% = \left\{
%     \begin{array}{ll}
%         g(\alpha, \beta) & \mbox{if } w < \alpha,\\
%         g(w, \beta) & \mbox{if } \alpha \leq w < \beta,  \\
%         g(w, w) & w\geq \beta.
%     \end{array}
% \right.
% \]
Our goal is to show that for all $w \in [0, 1]$ and all $\alpha \leq 
\beta$ we have
\begin{align*}
    \begin{cases}
        g(\alpha, \beta) & \mbox{if } w < \alpha,\\
        g(w, \beta) & \mbox{if } \alpha \leq w < \beta,  \\
        g(w, w) & w\geq \beta.
    \end{cases} \quad\leq\quad
    \begin{cases}
        g(\alpha^\star, \lambda) & \mbox{if } w < \alpha^\star,\\
        g(w, \lambda) & \mbox{if } \alpha^\star \leq w < \lambda,  \\
        g(w, w) & w\geq \lambda
    \end{cases}\enspace,
\end{align*}
where $\alpha^\star = \lambda \exp(\lambda^{-1}-2)$.
Let $\alpha^*(\beta) = \beta\exp((1-\beta)\lambda^{-1}-1)$, the value that maximizes $g(\alpha, \beta)$ over the first coordinate.
Note that $\alpha^*(\beta)$ is maximized at $\beta = \lambda$ and $\alpha^*(\lambda) \leq \lambda $ as long as $\lambda \geq 1/2$, implying that $\alpha^*(\beta) \leq \lambda$ for all $\beta \in [0,1]$.
The following properties of $g$ hold if $\alpha \leq \beta$, which are direct consequence of the fact that $g(\alpha, \beta)$ is concave on $\alpha$, concave on $\beta$, maximized in $(\alpha^\star, \lambda)$ and the maximum of $g(\alpha, \beta)$ over $\alpha$ is always upper bounded by $\lambda$:
\begin{enumerate}
    \item\label{item_prop:1} $g(\alpha, \beta) \leq g(\alpha^\star, \lambda)$ for all $\alpha, \beta$ \,\,\,(i.e., $g(\cdot, \cdot)$ is maximized at $(\alpha^\star, \lambda)$);
    \item\label{item_prop:2} $g(\alpha, \beta) \leq g(\alpha, \lambda)$ for all $\alpha, \beta$ \,\,\,(i.e., $g(\alpha, \cdot)$ is maximized at $\lambda$);
    \item\label{item_prop:3} $g(\alpha, \lambda) \leq g(w, \lambda)$ for all $\alpha^\star \leq w \leq \alpha$ \,\,\,(i.e., $g(\cdot, \lambda)$ decreases on $[\alpha^\star, 1]$);
    \item\label{item_prop:5} $g(\alpha, \beta) \leq g(\alpha, w)$ for all $\lambda \leq w \leq \beta$ and all $\alpha$ \,\,\,(i.e., $g(\alpha, \cdot)$ decreases on $[\lambda, 1]$);
    \item\label{item_prop:7} $g(\alpha, \beta) \leq g(w, \beta)$ for all $\lambda \leq w \leq \alpha$ and all $\beta$ (i.e., $g(\cdot, \beta)$ decreases on $[\alpha^*(\beta), 1]$ and hence on $[\lambda, 1]$);
\end{enumerate}
Fix some $\alpha \leq \beta$.\\
\textbf{Case 1 ($w \leq \alpha^{\star}$)}: by item \eqref{item_prop:1} the desired statement holds. \\
\textbf{Case 2 ($w \in [\alpha^\star, \lambda)$)}: thanks to items~\eqref{item_prop:2},\eqref{item_prop:3} if $w \in [\alpha^\star, \alpha)$,
item~\eqref{item_prop:2} if $\alpha^* \leq \alpha \leq w \leq \beta$,
item~\eqref{item_prop:2} if $w \in [\beta, \lambda)$
\begin{align*}
    \begin{cases}
        g(\alpha, \beta) & \mbox{if } w < \alpha,\\
        g(w, \beta) & \mbox{if } \alpha \leq w < \beta,  \\
        g(w, w) & w\geq \beta.
    \end{cases} \quad\leq\quad g(w, \lambda)\enspace.
\end{align*}
and the desired statement holds.\\
\textbf{Case 3 ($w \geq \lambda$)}: thanks to items~\eqref{item_prop:5},\eqref{item_prop:7} if $w \leq \alpha$ and $\beta \geq \alpha \geq \lambda$, item~\eqref{item_prop:5} if $w \in [\alpha, \beta)$ and $\alpha \geq \lambda$
\begin{align*}
    \begin{cases}
        g(\alpha, \beta) & \mbox{if } w < \alpha,\\
        g(w, \beta) & \mbox{if } \alpha \leq w < \beta,  \\
        g(w, w) & w\geq \beta.
    \end{cases} \quad\leq\quad g(w, w)
\end{align*}
and the desired statement holds.
\end{proof}















\subsection{DDT with $B \geq 1$: Proof of Theorem \ref{thm:asymptoticDDT} and associated Lemmas}\label{appx:DDT-B}
In this first Lemma we estimate the probability of the events $\{\rho_1 = t\}$ and $\{\rho_1 \geq t\}$ for any $t$.
\begin{lemma}\label{lem:rho1=t}
Let $\rho_1 = \min\{ t \geq \alpha_B N : r_t = 1\}$, then we have for any $t \in \{ \alpha_B N, \ldots, \beta_B N-1\}$
\[
\Prob(\rho_1 \geq t) = \frac{\alpha_B N}{t} +  O\left( \sqrt{\frac{\log N}{N}}\right),
\qquad
\Prob(\rho_1 = t) = \frac{\alpha_B N}{t^2} +  O\left( \sqrt{\frac{\log N}{N^3}}\right),
\]
and for any $t \in \{ \beta_B N, \ldots, N \}$
\[
\Prob(\rho_1 \geq t) = \frac{\alpha_B \beta_B N^2}{t^2} +  O\left( \sqrt{\frac{\log N}{N}}\right),
\qquad
\Prob(\rho_1 = t) = \frac{2 \alpha_B \beta_B N^2}{t^3} +  O\left( \sqrt{\frac{\log N}{N^3}}\right),
\]
where all the $O$ constants depend only on $\lambda$ and $\alpha_B$.
\end{lemma}
\begin{proof}
By Lemma \ref{lem:conc-card} we have that the event
\begin{equation}
%\label{eq:conc_0comp}
\tag{$\mathcal{C}_{\alpha_B}$}
    \forall \alpha_B N \leq t \leq N \; : \;
            \left| |G^1_t| - \lambda t \right|
            < A_{\alpha_B} \sqrt{t \log N}
\end{equation}
is true with probability at least $1 - \frac{2}{\alpha_B N}$, where  $A_{\alpha_B} \eqdef \dfrac{1+\sqrt{{\alpha_B}}}{2\sqrt{2}{\alpha_B}^{1/4}}$,
thus for any $t \geq \alpha_B N$ we have 
\[
\Prob(\rho_1 \geq t)
= \Prob(\rho_1 \geq t \mid \mathcal{C}_{\alpha_B})(1 - O(1/N))
\]
and the same goes for $\Prob(\rho_1 = t)$. Conditionally to $\mathcal{C}_{\alpha_B}$ we have for any $t \geq \alpha_B N - 1$
\begin{align*}
|G^1_t| &\geq |G^1_{t+1}|-1 \geq \lambda t -1 - A_{\alpha_B} \sqrt{t \log N},\\
|G^2_t| &\geq |G^2_{t+1}|-1 \geq (1-\lambda) t -1 - A_{\alpha_B} \sqrt{t \log N},
\end{align*}
and therefore if $N$ is large enough then $|G^1_t|, |G^2_t|$ are both positive.\\
If $\alpha_B N \leq t < \beta_B N$ then
\begin{align*}
\Prob(\rho_1 \geq t \mid \mathcal{C}_{\alpha_B})
&= \Prob(\best G^1_{ t-1} \in G^1_{\alpha_B N-1} \mid \mathcal{C}_{\alpha_B})
= \E\left[ \left. \frac{|G^1_{\alpha_B N - 1}|}{|G^1_{t-1}|} \right| \mathcal{C}_{\alpha_B} \right] \\
&= \frac{\lambda \alpha_B N + O(\sqrt{N \log N})}{\lambda t + O(\sqrt{t\log t})} 
= \frac{\alpha_B N}{t}\left( 1 + O\left( \sqrt{\frac{\log N}{N}}\right) \right)\left( 1 - O\left( \sqrt{\frac{\log t}{t}}\right) \right)\\
&= \frac{\alpha_B N}{t} +  O\left( \sqrt{\frac{\log N}{N}}\right),
\end{align*}
since $\alpha_B N \leq t \leq N$ we have that $\alpha_B N/t = O(1)$ and thus
\[
\Prob(\rho_1 \geq t)
= \left(\frac{\alpha_B N}{t} +  O\left( \sqrt{\frac{\log N}{N}}\right) \right) \left(1 - O(1/N) \right)
= \frac{\alpha_B N}{t} + O\left( \sqrt{\frac{\log N}{N}}\right)
\]
where the $O$ constant only depends on $\lambda$ and $\alpha_B$ because $\alpha_B N \leq t \leq N$. On the other hand, since the event $\{r_t = 1\}$ is independent of the past observations conditionally to $(g_t, |G^{g_t}_t|)$, we can write
\begin{align*}
\Prob(\rho_1 = t \mid \mathcal{C}_{\alpha_B})
&= \Prob(\rho_1 = t \mid \rho_1 \geq t, \mathcal{C}_{\alpha_B})\Prob(\rho_1 \geq t \mid \mathcal{C}_{\alpha_B})\\
&= \Prob(r_t = 1, g_t = 1 \mid \mathcal{C}_{\alpha_B})\Prob(\rho_1 \geq t \mid \mathcal{C}_{\alpha_B})\\
&= \lambda \E\left[ \left. \frac{1}{|G^1_t|} \right| \mathcal{C}_{\alpha_B} \right]
\Prob(\rho_1 \geq t \mid \mathcal{C}_{\alpha_B})\\
&= \left( \frac{1}{t} + O\left( \sqrt{\frac{\log N}{N^3}} \right)\right)\left( \frac{\alpha_B N}{t} +  O\left( \sqrt{\frac{\log N}{N}}\right) \right)
= \frac{\alpha_B N}{t^2} + O\left( \sqrt{\frac{\log N}{N^3}} \right),
\end{align*}
we deduce that
\[
\Prob(\rho_1 = t)
= \left( \frac{\alpha_B N}{t^2} + O\left( \sqrt{\frac{\log N}{N^3}} \right) \right) (1 - O(1/N))
= \frac{\alpha_B N}{t^2} + O\left( \sqrt{\frac{\log N}{N^3}} \right)\enspace.
\]

Similarly for $t \geq \beta_0 N$ we have
\begin{align*}
\Prob(\rho_1 \geq t \mid \mathcal{C}_{\alpha_B})
&= \Prob(\best G^1_{t-1} \in G^1_{\alpha_B N-1}, \best G^2_{t-1} \in G^2_{\beta_B N-1} \mid \mathcal{C}_{\alpha_B})\\
&= \E\left[ \left. \frac{|G^1_{\alpha_B N - 1}|}{|G^1_{t-1}|}
   \frac{|G^2_{\beta_B N - 1}|}{|G^2_{t-1}|} \right| \mathcal{C}_{\alpha_B} \right] \\
&= \left( \frac{\alpha_B N}{t} +  O\left( \sqrt{\frac{\log N}{N}}\right) \right) \left( \frac{\beta_B N}{t} +  O\left( \sqrt{\frac{\log N}{N}}\right) \right)\\
&= \frac{\alpha_B \beta_B N^2}{t^2} + O\left( \sqrt{\frac{\log N}{N}}\right),
\end{align*}
and also
\begin{align*}
\Prob(\rho_1 = t \mid \mathcal{C}_{\alpha_B})
&= \Prob(\rho_1 = t \mid \rho_1 \geq t, \mathcal{C}_{\alpha_B})\Prob(\rho_1 \geq t \mid \mathcal{C}_{\alpha_B})\\
&= \Prob(r_t = 1 \mid \mathcal{C}_{\alpha_B})\Prob(\rho_1 \geq t \mid \mathcal{C}_{\alpha_B})\\
&= \lambda \left( \lambda \E\left[ \left. \frac{1}{|G^1_t|} \right| \mathcal{C}_{\alpha_B} \right] 
+(1-\lambda) \E\left[ \left. \frac{1}{|G^2_t|} \right| \mathcal{C}_{\alpha_B} \right] \right)
\Prob(\rho_1 \geq t \mid \mathcal{C}_{\alpha_B})\\
&= \left( \frac{2}{t} + O\left( \sqrt{\frac{\log N}{N^3}} \right)\right)\left( \frac{\alpha_B \beta_B N^2}{t^2} +  O\left( \sqrt{\frac{\log N}{N}}\right) \right)\\
&= \frac{2\alpha_B \beta_B N^2}{t^3} + O\left( \sqrt{\frac{\log N}{N^3}} \right),
\end{align*}
we deduce $\Prob(\rho_1 \geq t), \Prob(\rho_1 = t)$ as in the case $\alpha_B \leq t \leq \beta_B$.
\end{proof}


\subsubsection{Success probability knowing $\rho_1$}
Now using Lemma \ref{lem:rho1=t}, we estimate the success probability of the algorithm conditionally to $\rho_1$.

\begin{lemma}\label{lem:prob-win-rho1}
For any $t \in \{\alpha_B N, \ldots, N\}$ we have
\[
\Prob(\A^B \text{ succeeds} \mid \rho_1 = t)
= \left\{
    \begin{array}{ll}
        \dfrac{\lambda t}{N} + (1-\lambda)\rd^{B-1}_{N,t+1} + O\left( \sqrt{\frac{\log N}{N}} \right)  & \mbox{if } \alpha_B \leq \frac{t}{N} < \beta_B \\
        \dfrac{t}{2N} + \dfrac{1}{2}\rd^{B-1}_{N,t+1} + O\left( \sqrt{\frac{\log N}{N}} \right) & \mbox{if } \frac{t}{N} \geq \beta_B
    \end{array}\enspace.
\right.
\]
where the $O$ constants depend on $\lambda$ and $\alpha_B$.
\end{lemma}


\begin{proof}
We denote in all the proof $t^\star$ the position of the best candidate overall.
For any $\alpha_B N \leq t \leq \beta_B N$ we have
\begin{align*}
\Prob(\A^B \text{ succeeds} \mid \rho_1 = t)
=& \Prob(\A^B \text{ succeeds}, R_t = 1 \mid \rho_1 = t)\\
& + \Prob(\A^B \text{ succeeds} \mid R_t \neq 1, \rho_1 = t)\Prob(R_t \neq 1 \mid \rho_1 = t),
\end{align*}
conditionally to $\rho_1 = t$, we have $\{ \A^B \text{ succeeds}, R_t = 1\} = \{t = t^\star \}$, because if $R_t = 1$ then $\A^B$ necessarily stops on $t$. Using this observation, the Bayes rule and Lemma \ref{lem:rho1=t} we obtain
\begin{align*}
\Prob(\A^B \text{ succeeds}, R_t = 1 \mid \rho_1 = t)
= \Prob(t = t^\star \mid \rho_1 = t)
= \Prob(\rho_1 = t \mid t = t^\star)\frac{\Prob(t = t^\star)}{\Prob(\rho_1=t)}\enspace,
\end{align*}
secondly, we have 
\[
\Prob(\A^B \text{ succeeds}, R_t \neq 1 \mid \rho^B_1 = t)
= \Prob(\A_{B-1} \text{ succeeds} \mid \rho^{B-1}_1 \geq t+1)
= \rd^{B-1}_{N,t+1}\enspace,
\]
in fact, if $\rho^B_1 = t$ and $R_t \neq 1$ then $\A^B$ uses a comparison but does not stop and moves to the next candidate, $\A^B$ is therefore exactly in the same state as $\A_{B-1}$ if the latter still has not used any comparison. Finally, using the Bayes rule again gives
\[
\Prob(R_t \neq 1 \mid \rho_1 = t)
= 1 - \Prob(\rho_1 = t \mid R_t = 1) \frac{\Prob(R_t = 1)}{\Prob(\rho_1 = t)}\enspace,
\]
it yields
\begin{align*}
\Prob(\A^B \text{ succeeds} \mid \rho_1 = t)
=& \Prob(\rho_1 = t \mid t = t^\star)\frac{\Prob(t = t^\star)}{\Prob(\rho_1=t)}\\
&\quad + \rd^{B-1}_{N,t+1} \left( 1 - \Prob(\rho_1 = t \mid R_t = 1) \frac{\Prob(R_t = 1)}{\Prob(\rho_1 = t)} \right)\\
=& \frac{\Prob(\rho_1 = t \mid t = t^\star)}{N \Prob(\rho_1 = t)}
+ \rd^{B-1}_{N,t+1} \left( 1 - \frac{\Prob(\rho_1 = t \mid R_t = 1)}{t\Prob(\rho_1 = t)} \right)\enspace.
\end{align*}
The expression above must be evaluated differently depending on the position of $t$ relatively to $\alpha_B N$ and $\beta_B N$. If $\alpha_B N \leq t < \beta_B N$ then conditionally to $t = t^\star$ we have $\{ \rho_1 = t \} = \{ \rho_1 \geq t, g_t = 1\}$, because if the latter event happens then the algorithm will by design make a comparison at $t$, and if $\rho_1 = t$ then necessary $g_t = 1$ since the algorithm can only select elements of $G^1$ at steps $\{\alpha_B N, \ldots, \beta_B N - 1\}$. Using this argument and Lemma \ref{lem:rho1=t} gives 
\begin{align*}
\frac{\Prob(\rho_1 = t \mid t = t^\star)}{\Prob(\rho_1 = t)}
&= \frac{\Prob(\rho_1 \geq t, g_t = 1 \mid t = t^\star)}{\Prob(\rho_1 = t)}
= \frac{\lambda \Prob(\rho_1 \geq t)}{\Prob(\rho_1 = t)}\\
&= \lambda \frac{\frac{\alpha_B N}{t}\left(1 + O\left( \sqrt{\frac{\log N}{N}} \right) \right)}{\frac{\alpha_B N}{t^2}\left(1 + O\left( \sqrt{\frac{\log N}{N}} \right) \right)}
= \lambda t\left(1 + O\left( \sqrt{\frac{\log N}{N}} \right)\right),
\end{align*}
the event $\{ \rho_1 \geq t \}$ is independent of $\{t = t^\star\}$ because it is measurable with respect to the ranks and groups of the past candidates, while $\{t = t^\star\}$ is independent of them.\\
Conditionally to $R_t = 1$ we can use the same argument to compute $\Prob(\rho_1 = t \mid R_t = 1)$, we obtain 
\[
\frac{\Prob(\rho_1 = t \mid R_t = 1)}{\Prob(\rho_1 = t)}
= \lambda t\left(1 + O\left( \sqrt{\frac{\log N}{N}} \right)\right),
\]
consequently,
\begin{align*}
\Prob(\A^B \text{ succeeds} \mid \rho_1 = t)
&= \frac{\lambda t}{N}\left(1 + O\left( \sqrt{\frac{\log N}{N}} \right)\right)
+ \rd^{B-1}_{t+1} \left( 1 - \lambda \left(1 + O\left( \sqrt{\frac{\log N}{N}} \right)\right) \right)\\
&= \frac{\lambda t}{N} + (1-\lambda)\rd^{B-1}_{t+1} + O\left( \sqrt{\frac{\log N}{N}} \right)\enspace.    
\end{align*}
with the $O$ constant only depending on $\lambda$ and $\alpha_B$.

Let us now consider $\beta_B N \leq t \leq N$. Since $\rho_1$ can belong to any of the two groups, and conditionally to $\{ t = t^\star \}$ we have that $\{ \rho_1 = t \} = \{ \rho_1 \geq t \}$, and we using Lemma \ref{lem:rho1=t}
\[
\frac{\Prob(\rho_1 = t \mid t = t^\star)}{\Prob(\rho_1 = t)}
= \frac{\Prob(\rho_1 \geq t)}{\Prob(\rho_1 = t)}
= \frac{\lambda \Prob(\rho_1 \geq t)}{\Prob(\rho_1 = t)}
= \frac{t}{2}\left(1 + O\left( \sqrt{\frac{\log N}{N}} \right)\right),
\]
and in the same way
\[
\frac{\Prob(\rho_1 = t \mid R_t = 1)}{\Prob(\rho_1 = t)}
= \frac{t}{2}\left(1 + O\left( \sqrt{\frac{\log N}{N}} \right)\right),
\]
we deduce that
\[
\Prob(\A^B \text{ succeeds} \mid \rho_1 = t)
= \frac{t}{2N} + \frac{1}{2}\rd^{B-1}_{t+1} + O\left( \sqrt{\frac{\log N}{N}} \right)\enspace.    
\]

\end{proof}



\subsubsection{Recursive formula for $\rd^B_{N,wN}$}

\begin{lemma}\label{lem:induction-Ubn}
For any $w \in [\alpha_B,\beta_B)$ we have 
\begin{align*}
\rd^B_{N,wN}
=& \lambda w \log\frac{\beta_B}{w} + w(1 - \beta_B)
+ (1-\lambda)w \left(\frac{1}{N} \sum_{t=wN}^{\beta_B N}
        \frac{\rd^{B-1}_{N,t+1}}{(t/N)^2}\right)\\
&\quad + w\beta_B \left(\frac{1}{N} \sum_{t=\beta_B N + 1}^{N} \frac{\rd^{B-1}_{N,t+1}}{(t/N)^3}\right)
+ O\left( \sqrt{\frac{\log N}{N}}\right),
\end{align*}
and for any $w \in [\beta_B, 1]$
\[
\rd^B_{N,wN}
= w - w^2 + w^2 \left( \frac{1}{N}\sum_{t=wN}^{N} \frac{\rd^{B-1}_{N,t+1}}{(t/N)^3} \right) + O\left( \sqrt{\frac{\log N}{N}}\right)\enspace.
\]
where the $O$ constants only depend on $\lambda$ and $\alpha_B$.
\end{lemma}


\begin{proof}
For any $w \in [\alpha_B,1]$, we can write
\begin{align}\label{eq:lemU-0}
\rd^B_{N,wN}
&= \sum_{t=wN}^{N} \Prob(\A^B \text{ succeeds} \mid \rho_1 = t)\Prob(\rho_1 = t \mid \rho_1 \geq wN)\\ \nonumber
&= \frac{1}{\Prob(\rho_1 \geq wN)}
    \sum_{t=wN}^{N} \Prob(\A^B \text{ succeeds} \mid \rho_1 = t)\Prob(\rho_1 = t).
\end{align}

If $w \geq \beta_B$ then by Lemmas \ref{lem:rho1=t} and \ref{lem:prob-win-rho1}
\begin{align}\nonumber
\sum_{t=wN}^{N} \Prob(\A^B& \text{ succeeds} \mid \rho_1 = t)\Prob(\rho_1 = t)\\ \nonumber
&= \sum_{t=wN}^{N} \left( \frac{t}{2N} + \frac{1}{2}\rd^{B-1}_{N,t+1} + O\left( \sqrt{\frac{\log N}{N}} \right) \right)
\left( \frac{2 \alpha_B \beta_B N^2}{t^3} +  O\left( \sqrt{\frac{\log N}{N^3}}\right) \right)\\ \nonumber
&= \sum_{t=wN}^{N} \left( 
\frac{\alpha_B \beta_B N}{t^2} + \frac{\alpha_B \beta_B N^2 \rd^{B-1}_{N,t+1}}{t^3} + O\left( \sqrt{\frac{\log N}{N^3}}\right)
\right)\\ \label{eq:lemU-1}
&= \alpha_B\beta_B\left( \frac{1}{w} - 1 \right)
+ \frac{\alpha_B\beta_B }{N}\sum_{t=wN}^{N} \frac{\rd^{B-1}_{N,t+1}}{(t/N)^3}
+ O\left( \sqrt{\frac{\log N}{N}}\right),
\end{align}
consequently, using Lemma \ref{lem:rho1=t} to estimate $\Prob(\rho_1 \geq wN)$ and substituting into Equation \ref{eq:lemU-0}
\begin{align*}
\rd^B_{N,wN}
&= \frac{1}{\Prob(\rho_1 \geq wN)}
    \sum_{t=wN}^{N} \Prob(\A^B \text{ succeeds} \mid \rho_1 = t)\Prob(\rho_1 = t)\\
&= \left( \frac{w^2}{\alpha_B\beta_B} + O\left( \sqrt{\frac{\log N}{N}}\right) \right) 
\left(
\alpha_B\beta_B\left( \frac{1}{w} - 1 \right)
+ \frac{\alpha_B\beta_B }{N}\sum_{t=wN}^{N} \frac{\rd^{B-1}_{N,t+1}}{(t/N)^3}
+ O\left( \sqrt{\frac{\log N}{N}}\right)
\right)\\
&= w - w^2 + w^2 \left( \frac{1}{N}\sum_{t=wN}^{N} \frac{\rd^{B-1}_{N,t+1}}{(t/N)^3} \right) + O\left( \sqrt{\frac{\log N}{N}}\right)\enspace.
\end{align*}

If we consider $w \in [\alpha_B, \beta_B)$ then again by Lemmas \ref{lem:rho1=t} and \ref{lem:prob-win-rho1}
\begin{align*}
\sum_{t=wN}^{\beta_B N-1} \Prob(\A^B& \text{ succeeds} \mid \rho_1 = t)\Prob(\rho_1 = t)\\
&= \sum_{t=wN}^{\beta_B N -1} \left( \frac{\lambda t}{N} + (1-\lambda)\rd^{B-1}_{N,t+1} + O\left( \sqrt{\frac{\log N}{N}} \right) \right)
\left( \frac{\alpha_B N}{t^2} +  O\left( \sqrt{\frac{\log N}{N^3}}\right) \right)\\
&= \sum_{t=wN}^{\beta_B N -1} \left( 
\frac{\lambda \alpha_B}{t} + (1-\lambda) \frac{\alpha_B N \rd^{B-1}_{N,t+1}}{t^2} + O\left( \sqrt{\frac{\log N}{N^3}}\right)
\right)\\
&= \lambda \alpha_B \log \frac{\beta_B}{w}
+ (1-\lambda) \frac{\alpha_B}{N} \sum_{t=wN}^{\beta_B N -1} \frac{\rd^{B-1}_{N,t+1}}{(t/N)^2}
+ O\left( \sqrt{\frac{\log N}{N}}\right),
\end{align*}
and using Equation \ref{eq:lemU-1} for $\beta_B$ instead of $w$ we get
\begin{align*}
\sum_{t=\beta_B N}^{N} \Prob(\A^B& \text{ succeeds} \mid \rho_1 = t)\Prob(\rho_1 = t)\\
&= \alpha_B( 1 - \beta_B )
+ \frac{\alpha_B\beta_B }{N}\sum_{t=\beta_B N}^{N} \frac{\rd^{B-1}_{N,t+1}}{(t/N)^3}
+ O\left( \sqrt{\frac{\log N}{N}}\right),
\end{align*}
finally, using Lemma \ref{lem:rho1=t} to estimate $\Prob(\rho_1 \geq wN)$ and by Equation \ref{eq:lemU-0}

\begin{align*}
\rd^B_{N,wN}
=& \left( \frac{w}{\alpha_B} + O\left(\sqrt{\frac{\log N}{N}}\right)\right)
    \sum_{t=w N}^{N} \Prob(\A^B \text{ succeeds} \mid \rho_1 = t)\Prob(\rho_1 = t)\\
=& \frac{w}{\alpha_B}\sum_{t=w N}^{N} \Prob(\A^B \text{ succeeds} \mid \rho_1 = t)\Prob(\rho_1 = t) + O\left(\sqrt{\frac{\log N}{N}}\right)\\
=& \lambda w \log\frac{\beta_B}{w}
+ (1-\lambda) \frac{w}{N} \sum_{t=wN}^{\beta_B N -1} \frac{\rd^{B-1}_{N,t+1}}{(t/N)^2}\\
& \quad + w( 1 - \beta_B )
+ \frac{w\beta_B }{N}\sum_{t=\beta_B N}^{N} \frac{\rd^{B-1}_{N,t+1}}{(t/N)^3}
+ O\left( \sqrt{\frac{\log N}{N}}\right),
\end{align*}
and this concludes the proof.
\end{proof}





\subsubsection{Proof of the Theorem}

Finally, we remind this classical result on Riemann sums that we will use to prove Theorem \ref{thm:asymptoticDDT}.
\begin{lemma}\label{lem:riemann-sum}
If $f$ is a continuous and piecewise $C^1$ function on $[a,b]$ then for any positive integer $N$ we have
\[
\left|
\frac{b-a}{N}\sum_{t=1}^N f\left( a + \frac{b-a}{N} \right)
- \int_a^b f(u)du 
\right|
\leq \left( \max_{[a,b]}|f'| \right) \frac{(b-a)^2}{2N}\enspace.
\]
\end{lemma}

Using the previous lemmas, we will now prove the theorem by induction over $B$.

\begin{proof}[Proof of Theorem \ref{thm:asymptoticDDT}]
For $B = 0$. we have by Proposition \ref{prop:asymptoticDDT0} that $\rd^0_{N,wN}$ converges to $\varphi_0(w)$, which is a continuous and piecewise $C^1$ function on $(0,1]$. Moreover, for any $w \in (0,1]$ we have from the proof of Proposition \ref{prop:asymptoticDDT0} that
\[
\rd^0_{N,wN} 
=\varphi_0(w) + O\left(\sqrt{\frac{\log N}{N}}\right)\enspace,
\]
with the $O$ constant depending only on $\lambda, \alpha_0$. We will show during the induction that this property is conserved and that the $O$ constants depend only on $\lambda, B$ and on the thresholds.\\
Now let $B \geq 1$ and assume that the result is true for $B-1$. For $w < \alpha_B$ it is immediately true because $w \mapsto \rd^B_{N,wN}$ is constant on $[0,\alpha_B]$ and equal to $\rd^B_{N,\alpha_B N}$. If $w \in [\alpha_B, \beta_B)$ then the induction hypothesis gives
\begin{align*}
\frac{1}{N} \sum_{t=wN}^{\beta_B N} \frac{\rd^{B-1}_{N,t+1}}{(t/N)^2}
&= \frac{1}{N} \sum_{t=wN}^{\beta_B N} \frac{1}{(t/N)^2}\left( \varphi_{B-1}\left(\frac{t+1}{N}\right) + O\left( \sqrt{\frac{\log N}{N}}\right) \right)\\
&= \frac{1}{N} \sum_{t=wN}^{\beta_B N} \frac{\varphi_{B-1}\left(\frac{t+1}{N}\right)}{(t/N)^2} + O\left( \sqrt{\frac{\log N}{N}}\right)\\
&= \int_w^{\beta_B} \frac{\varphi_{B-1}(u)}{u^2}du + O\left( \sqrt{\frac{\log N}{N}}\right)\enspace,
\end{align*}
where we used for the second equality that 
$\sum_{t=wN}^{\beta_B N} \frac{1}{t^2}
\leq \sum_{wN+1}^{\beta_B N}(\frac{1}{t-1} - \frac{1}{t})
= \frac{1}{wN}- \frac{1}{\beta_B N} \leq \frac{1}{\alpha_B N}
$,
and for the last one we used Lemma \ref{lem:riemann-sum} and the continuity of $\varphi_{B-1}$. The initial $O$ constant depends on $\lambda, \alpha_0, \ldots, \alpha_{B-1}$, to which we add $1/\alpha_B$ and $\sup_{[\alpha_B, \beta_B]}(\frac{d}{du}\frac{\varphi_{B-1}(u)}{u^2})$, which is well defined because $\varphi_{B-1}(u)/u^2$ is piecewise $C^1$ on $[\alpha_B,1]$ and which is a constant depending on $\lambda, B, (\alpha_0,\beta_0), \ldots, (\alpha_{B}, \beta_{B})$.\\
In the same fashion we prove that 
\[
\frac{1}{N} \sum_{t=\beta_B N + 1}^{N} \frac{\rd^{B-1}_{N,t+1}}{(t/N)^3}
= \int_{\beta_B}^1 \frac{\varphi_{B-1}(u)}{u^3}du + O\left( \sqrt{\frac{\log N}{N}}\right)\enspace,
\]
and that for $w \in [\beta_B, 1)$
\[
\frac{1}{N} \sum_{t=w N}^{N} \frac{\rd^{B-1}_{N,t+1}}{(t/N)^2}
= \int_{w}^1 \frac{\varphi_{B-1}(u)}{u^2}du + O\left( \sqrt{\frac{\log N}{N}}\right)\enspace,
\]
this equation with Lemma \ref{lem:induction-Ubn} prove that on both intervals $[\alpha_B, \beta_B)$ and $[\beta_B, 1]$ we can write
\[
\rd^B_{N,wN}
= \varphi_B(w) + O\left( \sqrt{\frac{\log N}{N}}\right)\enspace,
\]
with $\varphi_B$ is defined on $[\alpha_B, \beta_B)$ by
\begin{align*}
\varphi_B(w)
= \lambda w \log\frac{\beta_B}{w} + w(1 - \beta_B)
+ (1-\lambda)w \int_w^{\beta_B} \frac{\varphi_{B-1}(u)}{u^2}\diff u
+ w \beta_B \int_{\beta_B}^1 \frac{\varphi_{B-1}(u)}{u^3}\diff u\enspace,
\end{align*}
and on $[\beta_B,1)$ by
\[
\varphi_B(w)
= w - w^2 + w^2\int_w^1 \frac{\varphi_{B-1}(u)}{u^3}du.
\]
which can be expressed as stated in the theorem. We verify that it is continuous in $\beta_B$, and the induction hypothesis guarantees that $\varphi_B$ is continuous and piecewise $C^1$ on $[\alpha_B, 1]$.
\end{proof}












\subsection{Success probability of optimal DDT algorithm: Proof of Proposition \ref{prop:lb-optDDT} }\label{appx:lb-optDDT}

\begin{proof}[Proof of Proposition \ref{prop:lb-optDDT}]
Since $(\alpha^\star_b, \beta^\star_b)_{b=0}^B$ are optimal thresholds we have that the limit success probability of the algorithm is higher than the success probability of any other DDT algorithm, in particular if we choose all the thresholds equal to the same value 
\begin{align*}
\lim_{N \to \infty} \Prob( \A_2((\alpha^\star_b, \beta^\star_b)_{b=0}^B) \text{ succeeds})
&= \varphi_B(0; (\alpha^\star_b, \beta^\star_b)_{b=0}^B)\\
&\geq \sup_{\w \in (0,1)} \varphi_B(0; (\w, \w)_{b=0}^B)\\
&\geq \frac{1}{e} - \frac{1}{e(B+1)!}\enspace,
\end{align*}
where the last lower bound is deduced from Corollary \ref{cor:success_proba_bound_budget}.  We also have by Proposition \ref{prop:asymptoticDDT0} that $\lambda \geq 1/2$ then
\[
\varphi_B(0; (\alpha^\star_b, \beta^\star_b)_{b=0}^B)
\geq \sup_{(\alpha_0, \beta_0) \in (0,1)} \varphi_0(0; (\alpha_0, \beta_0))
\geq \lambda^2 \exp\left( \frac{1}{\lambda} - 2 \right)\enspace,
\]
with simple computations we obtain that the latter expression is lower bounded by $\frac{1}{e} - \frac{\lambda(1-\lambda)}{2}$, and this bound remains true also if $\lambda < 1/2$ (replace $\lambda$ by $1-\lambda$), hence
\[
\lim_{N \to \infty} \Prob( \A_2((\alpha^\star_b, \beta^\star_b)_{b=0}^B) \text{ succeeds})
\geq 
\frac{1}{e} - \frac{\lambda(1-\lambda)}{2}\enspace,
\]
which concludes the proof.
\end{proof}

\subsection{Proof of Proposition \ref{prop:success-memless-skip} and additional properties}\label{appx:memless}


\subsubsection{Proof of Proposition \ref{prop:success-memless-skip}}
\begin{proof}
We show that the following claims hold
\begin{enumerate}[label=(\roman*)]
\item $\Prob(R_t = 1 \mid \F_{t-1}) = \frac{1}{t}$,
\item knowing $(g_t, N^1_{t-1})$, the distribution of $r_t$ is independent of $\F_{t-1}$,
\item if $s \geq t$ then $A_{s,1}$ is measurable with respect to $(N^1_{t-1},B_t,\{ g_u, r_u, \xi_{u,1} \}_{u=t}^{s})$.
\end{enumerate} 
\textbf{First claim} 
We remind that 
$\F_{t-1} = \{r_s,g_s, \act_{s, 1}, R_s\indic{\act_{s, 1} = \acomp}, \act_{s, 2}\indic{\act_{s, 1} = \acomp}\}_{s=1}^{t-1}$. Since the distributions of $\act_{s, 1}, \act_{s, 2}$ are measurable with respect to $\{r_s,R_s,g_s\}_{s=1}^{t-1}$, we only need to show the independence between $R_t$ and the past observations.\\  
We have that $\{R_t = 1\} = \{x_t > \max\{x_s\}_{s=1}^{t-1}\}$, this event is independent of the groups of the past candidates, and by Proposition \ref{prop:max2sets} it is also independent of their relative ranks, therefore $R_t$ is independent of $\F_{t-1}$ and 
$\Prob(R_t = 1 \mid \F_{t-1}) = \Prob(R_t = 1)
=\frac{1}{t}$.\\
\textbf{Second claim}
We only need to show the independence between $r_t$ and the past observations conditionally to $(g_t, N^1_{t-1})$. 
Knowing $G^1_t$, we have that $r_t$ is independent of the relative ranks of $\{x_1, \ldots, x_N\}$ by Proposition \ref{prop:max2sets}, thus it is independent of $\{r_s\}_{s<t}, \{R_s\}_{s<t}$. We also have by  Proposition \ref{prop:max2sets} that knowing $(g_t, N^1_{t-1})$, the probability of $\{r_t = 1\}$ is $1/N^{g_t}_t$ independently of $G^1_t$, where 
\[
N^{g_t}_t = \indic{g_t = 1}(N^1_{t-1} +1) + (1- \indic{g_t = 1})(t - N^1_{t-1}),
\]
hence knowing $(g_t, N^1_{t-1})$,$r_t$ is independent of $\{r_s,R_s,g_s\}_{s=1}^t$, thus independent of $\F_{t-1}$. \\
\textbf{Third claim} Let us assume for simplicity that the randomness in $\act_{s, 1}, \act_{s, 2}$ knowing $\F_t$ is due to some random noise $\xi_{s,1}, \xi_{s,2}$ independent of all the rest. We will prove the claim using a simple induction. For $s=t$ it is true because $\A$ is memory-less. Assume that it is true up to some $t\leq s < N$, $A_{s+1,1}$ is $(N^1_{s-1}, B_{s+1}, g_{s+1}, r_{s+1}, \xi_{s+1,1})-$measurable, and we have
\[
N^1_{s} = N^1_{t-1} + \sum_{u=t}^s \indic{g_t = 1}, \qquad
B_{s+1} = B_t - \sum_{u=t}^{s} \indic{\act_{u, 1} = \acomp},
\]
$N^1_{s}$ is therefore $(N^1_{s}, \{g_u\}_{u=t}^s)$-measurable and $B_{s+1}$ is $(B_t,\{ \act_{u, 1} \}_{t \leq u\leq s})$-measurable, using the induction hypothesis we deduce that it is measurable with respect to $(N^1_{t-1}, B_t,\{ g_u, r_u, \xi_{u,1} \}_{u=t}^{s})$, and thus the result.\\
\textbf{Proof of the proposition} 
Let $t^\star$ be the the position of the best candidates overall,
\[
\Prob(\A \text{ succeeds} \mid \tau \geq t, \F_{t-1}) 
= \Prob(\tau = {t^\star} \mid \tau \geq t, \F_{t-1}) 
= \E_T \left[ 
        \Prob(\tau = {t^\star} \mid \tau \geq t, \F_{t-1}, {t^\star}) 
    \right],
\]
if ${t^\star} < t$ then $\Prob(\A \text{ succeeds} \mid \tau \geq t, \F_{t-1}, T) = 0$, otherwise this probability only depends on the future actions of $\A$, because conditionally to $\{ \tau \geq t\}$, we have that 
\[
\tau = \min \{s \geq t \mid \act_{s, 1} = \astop \text{ or } \act_{s, 2} = \astop \}.
\]
Conditionally to $(N^1_{t-1}, B_t)$, we have that the distributions of $\{\act_{s, 1}, \act_{s, 2}\}_{s \geq t}$ are measurable with respect to $\{g_u, r_u, R_u\}_{u=t}^s$. The variables $\{g_u\}_{u \geq t}$ are independent of $\F_{t-1}$, and $\{r_u, R_u\}_{u \geq t}$ are also independent of $\F_{t-1}$ knowing $(N^1_{t-1}, B_t)$, hence $\{\act_{s, 1}, \act_{s, 2}\}_{s \geq t}$ are independent of $\F_{t-1}$ and thus the result.
\end{proof}







\subsubsection{Another useful property of memory-less algorithms}
The following proposition states that when a memory-less stops at some $t$, then knowing $(r_t,R_t,N^1_t)$ we can easily express its success probability.

\begin{proposition}\label{prop:success-memless-stop}
Under the same assumptions of the last proposition, we have
\begin{align*}
\Prob(\A \text{ succeeds} \mid \act_{t, 1} = \astop, r_t, g_t, B_t, N^1_t) 
&= \indic{r_t = 1}\frac{ N^{g_t}_t}{N},
\\
\Prob(\A \text{ succeeds} \mid \act_{t, 2} = \astop, r_t, g_t, B_t, N^1_t, R_t) 
&= \indic{R_t = 1}\frac{t}{N}.
\end{align*}
\end{proposition}

\begin{proof}
Let us denote $t^\star$ the position of the best candidate overall. Conditionally to $\act_{t,1} = \astop$, the $\{ \A \text{ succeeds} \};= \{ t^\star = t \}$. If $r_t \neq 1$ then with probability 1 we have $t^\star \neq t$, and if $r_t = 1$, then the only information about the rank of $x_t$ contained in $r_t, g_t, B_t, N^1_t$ is that $x_t$ is better than all the past candidates of $G^{g_t}$, i.e that it is the best among a collection of $N^{g_t}_t$ candidates, which happens with probability $1/N^{g_t}_t$. The success probability is therefore
\begin{align*}
\Prob(\A \text{ succeeds} &\mid \act_{t, 1} = \astop, r_t=1, g_t, B_t, N^1_t)\\
&= \Prob(t^\star = t \mid \act_{t, 1} = \astop, r_t=1, g_t, B_t, N^1_t)
= \Prob(t^\star = t \mid  r_t=1, g_t, N^1_t)\\
&= \frac{\Prob(t^\star = t \mid  g_t, N^1_t)}{\Prob(r_t=1 \mid g_t, N^1_t)}
= \frac{N^{g_t}_t}{N}.
\end{align*}
which means that it is null when $r_t \neq 1$ and equal to $N^{g_t}_t/N$ otherwise.\\

If $\act_{t, 2} = \astop$, then necessarily $(\act_{t, 1} = \acomp)$ and $R_t$ is known. If $R_t \neq 1$ then $\Prob(t^\star = t \mid \act_{t, 2} = \astop, R_t \neq 1, \F_t) = 0$, otherwise $R_t = 1$, implying that $r_t = 1$, thus $R_t = 1$ is the maximal information that we get from all the past on the absolute rank of $x_t$, thus 
\[
\Prob(\A \text{ succeeds} \mid \act_{t, 1} = \astop, R_t = 1)
= \Prob(t^\star = t \mid R_t = 1)
= \frac{t}{N}.
\]
\end{proof}








\subsection{Optimal algorithm via dynamic programming: Proofs of Section \ref{sec:opt-dynprog}}\label{appx:opt-dynprog}

\subsubsection{Proof of Proposition \ref{prop:dynprog0}}

\begin{proof}
Let $\A$ be any memory-less algorithm for the 2-groups secretary problem with no comparisons between the two groups, with a stopping time $\tau$, and let us denote for $t \geq 1$
\[
\rwd^0_{t,m} \eqdef \Prob (\A \text{ succeeds} \mid \tau \geq t, N^1_t).
\]

$\{\tau \geq t\}$ is $\F_{t-1}$-measurable, and the distribution of $r_t$ is independent of $\F_{t-1}$ knowing $(N^1_{t-1}, g_t)$, therefore $r_t$ is independent of $\{ \tau \geq t\}$, thus we have
\begin{align}
\rwd^0_{t,N^1_{t-1}t} \nonumber \label{eq:req-rwd0-0}
=& \E_{g_t}[ \Prob(r_t=1 \mid \tau \geq t, N^1_{t-1}, g_t)
\Prob (\A \text{ succeeds} \mid \tau \geq t, N^1_{t-1}, r_t=1, g_t) \\\nonumber
&\qquad + \Prob(r_t \neq 1 \mid \tau \geq t, N^1_{t-1}, g_t)
\Prob (\A \text{ succeeds} \mid \tau \geq t, N^1_{t-1}, r_t \neq 1, g_t) ]\\ \nonumber
=& \E_{g_t} \left[  \frac{1}{1 + N^{g_t}_{t-1}}
    \Prob (\A \text{ succeeds} \mid \tau \geq t, N^1_{t-1}, r_t=1, g_t)  \right.\\
&\qquad \left.+ \left( 1 - \frac{1}{1 + N^{g_t}_{t-1}}\right) 
    \Prob (\A \text{ succeeds} \mid \tau \geq t, N^1_{t-1}, r_t \neq 1, g_t) \right]\enspace.
\end{align}

If we denote $q$ the probability that $\act_{t, 1} = \askip$ knowing $(N^1_{t-1}, r_t, g_t)$, we have that

\begin{align} \label{eq:opt0-ineq}
\Prob (\A \text{ succeeds} \mid \tau \geq t, &N^1_{t-1}, r_t, g_t)\\ \nonumber
&= q\Prob (\A \text{ succeeds} \mid \tau \geq t, N^1_{t-1}, r_t, g_t, \act_{t, 1} = \askip) \\ \nonumber
& \qquad + (1-q) \Prob (\A \text{ succeeds} \mid \tau \geq t, N^1_{t-1}, r_t, g_t, \act_{t, 1} = \astop)\\ \nonumber
&\leq \max\{
    \Prob (\A \text{ succeeds} \mid \tau \geq t,  N^1_{t-1}, r_t, g_t, \act_{t, 1} = \askip)\\ \nonumber
&\hspace{35pt} + \Prob (\A \text{ succeeds} \mid \tau \geq t, N^1_{t-1}, r_t, g_t, \act_{t, 1} = \astop) 
\}
\end{align}
The first term can be simplified using the implication $(\tau \geq t, A_{t,1} = \askip) \implies (\tau \geq t+1)$ and the memory-less property of $\A$ gives for any value of $r_t$ that
\begin{align*}
\Prob (\A \text{ succeeds} &\mid \tau \geq t, N^1_{t-1}, r_t, g_t, \act_{t, 1} = \askip)\\
&= \Prob (\A \text{ succeeds} \mid \tau \geq t+1, N^1_t = N^1_{t-1} + \indic{g_t=1})\\
&= \rwd^0_{t+1, N^1_{t-1} +\indic{g_t=1}}
\end{align*}
The second term is null for $r_t \neq 1$. For $r_t=1$, we have that $(\act_{t, 1} = \astop) \iff ( \tau = t)$, the event $\{ \A \text{ succeeds}\}$ becomes equivalent to $\{t = t^\star\}$, and knowing $r_t, g_t, N^1_{t-1}$, it is independent of the distribution $Q_{t,1}$ of $\act_{t, 1}$ because $\A$ is memory-less, it yields
\begin{align*}
\Prob (\A \text{ succeeds} \mid \tau \geq t,  N^1_{t-1}, r_t=1, g_t, \act_{t, 1} = \astop)
= \Prob (t = t^\star \mid N^1_{t-1}, r_t=1, g_t)
= \frac{1+N^{g_t}_{t-1}}{N}.
\end{align*}
Substituting into Equation \ref{eq:req-rwd0-0} gives
\begin{align*}
\rwd^0_{t,m}
&\leq \E_{g_t} \left[ \frac{1}{1+N^{g_t}_{t-1}}\max\left\{  
    \rwd^0_{t+1,m+\indic{g_t=1}}, \frac{1+N^{g_t}_{t-1}}{N} \right\}
    + \left(1 - \frac{1}{1+N^{g_t}_{t-1}} \right) \rwd^0_{t+1,m+\indic{g_t=1}} \right]\\
&=  \E_{g_t} \left[   \max\left\{  
    \rwd^0_{t+1,m+\indic{g_t=1}}, \frac{1}{N} + \left(1 - \frac{1}{1+N^{g_t}_{t-1}} \right) \rwd^0_{t+1,m+\indic{g_t=1}}\right\} \right]\\
&= \lambda \max\left\{  
    \rwd^0_{t+1,m+1}, \frac{1}{N} + \left(1 - \frac{1}{1+m} \right) \rwd^0_{t+1,m+1}\right\}\\
&\qquad + (1 - \lambda) \max\left\{  
    \rwd^0_{t+1,m}, \frac{1}{N} + \left(1 - \frac{1}{t-m} \right) \rwd^0_{t+1,m}\right\}.
\end{align*}
\end{proof}



\subsubsection{Proof of Corollary \ref{cor:optAction0}}

\begin{proof}
If an algorithm is such that the previous dynamic programming inequality is an equality then it is necessarily optimal. In order to have that, we only need Inequality \ref{eq:opt0-ineq} from the previous proof to be an equality, and this can be achieved by stopping with probability 1 if 
\[
r_s = 1 \text{ and }  N \rwd^0_{s+1, N^1_s} \leq N^{g_s}_s\enspace.
\]
\end{proof}



\subsubsection{Proof of Proposition \ref{prop:dynprogb}}

\begin{proof}
Let $1 \leq t \leq N$ and assume that $B_t = b > 1$. The tree of all possible observations $(r_t, R_t)$ and actions $\act_{t, 1}, \act_{t, 2}$ that can happen at step $t$ are presented in Figure \ref{fig:tree-b>1}. Before trying to determine the optimal action at this step, let us start by establishing that some actions are never optimal when we still have a positive budget $b$: 
\begin{itemize}
    \item if $r_t \neq 1$, then stopping is not optimal because the success probability would be $0$,
    \item same thing if $R_t \neq 1$,
    \item if $r_t = 1$, stopping is not optimal because it is equivalent to comparing then stopping with probability 1 since $b>0$, therefore comparing then choosing an optimal action given $R_t$ is better than immediately stopping,
    \item a reasonable algorithm never skips after observing $R_t = 1$. In fact, since we consider memory-less algorithms, by Proposition \ref{prop:success-memless-stop} we deduce that skipping $x_t$ after observing $R_t$ has no impact on the success probability compared to skipping after just observing $r_t$, except having $B_{t+1} = b-1$ instead of $B_{t+1} = b$. A reasonable algorithm must therefore use a comparison only if it is ready to stop if $R_t = 1$. 
\end{itemize}

If we want to determine the optimal algorithm, the only sequences of actions and observations possible at step $t$ are $(r_t=1, \acomp, R_t=1, \astop)$, $(r_t=1, \acomp, R_t \neq 1, \askip)$, $(r_t=1, \askip)$, and $(r_t \neq 1, \askip)$.

\begin{figure}[h!]
\centering
\begin{tikzpicture}
\node {$(r_t,g_t)$}[grow=right][sibling distance = 1cm][level distance=2.5cm]
    child [yshift = -0.8cm]{node {$r_t \neq 1$}
        child {node {$\askip$}}
        child {node {\color{red} $\astop$} edge from parent [red]}
    }
    child [yshift = 0.8cm] {node {$r_t=1$}
        child {node {$\askip$}}
        child {node {\color{red} $\astop$} edge from parent [red]}
        child {node {$\acomp$}
            child [yshift = -.5cm]  {node {$R_t \neq 1$} 
                child {node {$\askip$}}
                child {node {\color{red} $\astop$} edge from parent [red]}
            }
            child [yshift = .5cm]  {node {$R_t=1$}
                child {node {\color{red} $\askip$} edge from parent [red]}
                child {node {$\astop$}}
            }
        }
    };
\end{tikzpicture}
\caption{Tree of possible observations and actions when a new observation $(r_t, g_t)$ arrives and $B_t > 0$.} 
\label{fig:tree-b>1}
\end{figure}

We showed in the proof of Proposition \ref{prop:success-memless-stop} that conditionally to $(g_t, N^1_{t-1})$, the distribution of $r_t$ is independent of $\F_{t-1}$, thus we have 
\[
\Prob(r_t = 1 \mid g_t, N^1_{t-1}) = \frac{1}{N^{g_t}_t},
\]
with $N^{g_t}_t = \indic{g_t = 1}(N^1_{t-1} + 1) + (1 - \indic{g_t = 1} )(t - N^1_{t-1})$. In particular, the event $\{ \tau \geq t\}$ is $\F_{t-1}$-measurable, hence
\begin{align*}
\Prob(\A \text{ succeeds} \mid \tau \geq t, B_t, N^1_{t-1}, g_t)
=& \frac{1}{N^{g_t}_t} \Prob(\A \text{ succeeds} \mid \tau \geq t, B_t, N^1_{t-1}, g_t, r_t=1)\\
&+ \left(1 - \frac{1}{N^{g_t}_t} \right) 
\Prob(\A \text{ succeeds} \mid \tau \geq t, B_t, N^1_{t-1}, g_t, r_t \neq 1)    
\end{align*}
If $r_t \neq 1$, the only possible action is $\act_{t, 1} = \askip$, consequently
\begin{align*}
\Prob(\A \text{ succeeds}& \mid \tau \geq t, B_t, N^1_{t-1}, g_t, r_t \neq 1)\\
&= \Prob(\A \text{ succeeds} \mid \tau \geq t+1, B_t, N^1_{t-1}, g_t, r_t \neq 1, \act_{t, 1}=\askip)\\
&= \Prob(\A \text{ succeeds} \mid \tau \geq t+1, B_{t+1} = B_t, N^1_{t} = N^1_{t-1} + \indic{g_t=1})\\
&=\rwd^{B_t}_{t+1, N^1_{t-1} + \indic{g_t=1}}.
\end{align*}
If $r_t = 1$, then
\begin{align}\label{eq:rwdb-max}
\Prob(\A &\text{ succeeds} \mid \tau \geq t, B_t, N^1_{t-1}, g_t, r_t=1)\\ \nonumber
&\leq \max \{
\Prob(\A \text{ succeeds} \mid \tau \geq t, B_t, N^1_{t-1}, g_t, r_t=1, \act_{t, 1} = \askip),\\ \nonumber
&\qquad \Prob(\A \text{ succeeds} \mid \tau \geq t, B_t, N^1_{t-1}, g_t, r_t=1, \act_{t, 1} = \acomp)\}\\ \nonumber
&= \max\{ 
\rwd^{B_t}_{t+1, N^1_{t-1} + \indic{g_t=1}},
\Prob(\A \text{ succeeds} \mid \tau \geq t, B_t, N^1_{t-1}, g_t, r_t=1, \act_{t, 1} = \acomp)\}
\end{align}
With the same arguments as in the proof of Proposition \ref{prop:success-memless-stop}, we have that 
\[
\Prob(R_t=1 \mid N^1_{t-1}, g_t, r_t=1, \act_{t, 1}, B_t)
= \Prob(R_t=1 \mid N^1_{t-1}, g_t, r_t=1)
= \frac{\Prob(R_t=1 \mid N^1_{t-1}, g_t)}{\Prob(r_t=1 \mid N^1_{t-1}, g_t)}
= \frac{N^{g_t}_t}{t},
\]
therefore
\begin{align*}
\Prob(\A \text{ succeeds} &\mid \tau \geq t, B_t, N^1_{t-1}, g_t, r_t=1, \act_{t, 1} = \acomp)\\
=& \frac{N^{g_t}_t}{t}
  \Prob(\A \text{ succeeds} \mid \tau \geq t, B_t, N^1_{t-1}, g_t, r_t=1, \act_{t, 1} = \acomp, R_t=1)\\
&+ \left(1 - \frac{N^{g_t}_t}{t} \right) \Prob(\A \text{ succeeds} \mid \tau \geq t, B_t, N^1_{t-1}, g_t, r_t=1, \act_{t, 1} = \acomp, R_t \neq 1)\\
=& \frac{N^{g_t}_t}{t}
  \Prob(\A \text{ succeeds} \mid R_t=1, \act_{t, 2} = \astop)\\
&+ \left(1 - \frac{N^{g_t}_t}{t} \right) \Prob(\A \text{ succeeds} \mid \tau \geq t+1, B_{t+1}=B_t - 1, N^1_t = N^1_{t-1} + \indic{g_t = 1})\\
=& \frac{N^{g_t}_t}{N} + \left(1 - \frac{N^{g_t}_t}{t} \right)
    \rwd^{B_t-1}_{t+1, N^1_{t-1} + \indic{g_t=1}}\\
\end{align*}
where we used in the second equality that the only possible action for the optimal algorithm after observing $R_t \neq 1$ is $\act_{t, 2} = \askip$ and the only action after $R_t = 1$ is $\act_{t, 2} = \astop$, and for the last equality we used Propositions \ref{prop:success-memless-stop} and \ref{prop:success-memless-skip}.\\
Substituting into Inequality \ref{eq:rwdb-max} yields
\begin{align*}
\Prob(\A \text{ succeeds}& \mid \tau \geq t, B_t, N^1_{t-1}, g_t, r_t=1)\\
&\leq \max\left\{
\rwd^{B_t-1}_{t+1, N^1_{t-1} + \indic{g_t=1}},
\frac{N^{g_t}_t}{N} + \left(1 - \frac{N^{g_t}_t}{t} \right)
\rwd^{B_t}_{t+1, N^1_{t-1} + \indic{g_t=1}}
\right\},
\end{align*}
and finally
\begin{align*}
\Prob(\A &\text{ succeeds} \mid \tau \geq t, B_t, N^1_{t-1}, g_t)\\
&\leq \frac{1}{N^{g_t}_t} \max\left\{
\rwd^{B_t}_{t+1, N^1_{t-1} + \indic{g_t=1}},
\frac{N^{g_t}_t}{N} + \left(1 - \frac{N^{g_t}_t}{t} \right)
\rwd^{B_t-1}_{t+1, N^1_{t-1} + \indic{g_t=1}}
\right\}
+ \left(1 - \frac{1}{N^{g_t}_t} \right) 
\rwd^{B_t}_{t+1, N^1_{t-1} + \indic{g_t=1}}\\
&=\max\left\{
\rwd^{B_t}_{t+1, N^1_{t-1} + \indic{g_t=1}},
\frac{1}{N} + \left(\frac{1}{N^{g_t}_t} - \frac{1}{t} \right)
\rwd^{B_t-1}_{t+1, N^1_{t-1} + \indic{g_t=1}}
+ \left(1 - \frac{1}{N^{g_t}_t} \right) 
\rwd^{B_t}_{t+1, N^1_{t-1} + \indic{g_t=1}}
\right\},
\end{align*}
If $B_t = b \geq 1$ and $N_{t-1}^1 = m$, the previous inequality translates, in expectation over $g_t$, as
\begin{align*}
\rwd^b_{t,m}
\leq& \lambda \max\left\{
\rwd^{b}_{t+1, m+1},
\frac{1}{N} + \left(\frac{1}{m+1} - \frac{1}{t} \right)
\rwd^{b-1}_{t+1, m+1}
+ \left(1 - \frac{1}{m+1} \right) 
\rwd^{b}_{t+1, m+1}
\right\}\\
&+ (1 - \lambda) \max\left\{
\rwd^{b}_{t+1, m},
\frac{1}{N} + \left(\frac{1}{t-m} - \frac{1}{t} \right)
\rwd^{b-1}_{t+1, m}
+ \left(1 - \frac{1}{t-m} \right) 
\rwd^{b}_{t+1, m}
\right\}.
\end{align*}

\end{proof}







\subsubsection{Proof of Corollary \ref{cor:optActionB}}
\begin{proof}
Similarly to the proof of Corollary \ref{cor:optAction0}, an algorithm is optimal if the inequality in Proposition \ref{prop:dynprogb} is an equality, and this true only if Inequality \ref{eq:rwdb-max} in the proof of Proposition \ref{prop:dynprogb} is an equality, which can be achieved by using a comparison with probability 1 if
\[
r_s = 1 \text{ and }
\rwd^{b}_{s+1, N^1_s}
- \left(1 - \frac{N^{g_s}_s}{s} \right) \rwd^{b-1}_{s+1, N^1_s}
<
\frac{ N^{g_s}_s}{N}\enspace.
\]
\end{proof}




\section{How to choose optimal thresholds ?}
\label{appx:choose-opt-thresh}

\subsection{Strong optimality of the thresholds $\alpha^\star_0, \beta^\star_0$}
The following Proposition states that for any DDT algorithm with budget $B \geq 0$, if at some step $w N$ the algorithm still did not stop and does not have any comparisons left, then its success probability is upper bounded by $\varphi_0(w; (\alpha_0^\star, \beta_0^\star))$, where $(\alpha_0^\star, \beta_0^\star) = (\lambda \exp(1/\lambda - 2), \lambda)$. In particular, this implies that whatever is the initial budget $B$, choosing $\alpha_0 = \alpha^\star_0, \beta_0 = \beta^\star_0$ is optimal.

\begin{proposition}\label{prop:DDT0-optanyw}
For all $w \in [0, 1]$ and $\lambda \geq 1/2$, the mapping $(\alpha_0, \beta_0) \mapsto\varphi_0(w; (\alpha_0, \beta_0))$ is maximized over $\{\alpha_0, \beta_0 \in (0,1)\}$ at
\[
    \alpha_0^\star = \lambda \exp\left( \frac{1}{\lambda} - 2\right)
    \quad \text{ and } \quad
    \beta_0^\star  = \lambda\enspace.
\]

\end{proposition}


\begin{proof}
   Fix some $\lambda \geq 1/2$. For convenience, let us drop the subscript $0$ from $g_0$.
Our goal is to show that for all $w \in [0, 1]$ and all $\alpha \leq 
\beta$ we have
\begin{align*}
    \begin{cases}
        g(\alpha, \beta) & \mbox{if } w < \alpha,\\
        g(w, \beta) & \mbox{if } \alpha \leq w < \beta,  \\
        g(w, w) & w\geq \beta.
    \end{cases} \quad\leq\quad
    \begin{cases}
        g(\alpha^\star, \lambda) & \mbox{if } w < \alpha^\star,\\
        g(w, \lambda) & \mbox{if } \alpha^\star \leq w < \lambda,  \\
        g(w, w) & w\geq \lambda
    \end{cases}\enspace,
\end{align*}
where $\alpha^\star = \lambda \exp(\lambda^{-1}-2)$.
Let $\alpha^*(\beta) = \beta\exp((1-\beta)\lambda^{-1}-1)$, the value that maximizes $g(\alpha, \beta)$ over the first coordinate.
Note that $\alpha^*(\beta)$ is maximized at $\beta = \lambda$ and $\alpha^*(\lambda) \leq \lambda $ as long as $\lambda \geq 1/2$, implying that $\alpha^*(\beta) \leq \lambda$ for all $\beta \in [0,1]$.
The following properties of $g$ hold if $\alpha \leq \beta$, which are direct consequence of the fact that $g(\alpha, \beta)$ is concave on $\alpha$, concave on $\beta$, maximized in $(\alpha^\star, \lambda)$ and the maximum of $g(\alpha, \beta)$ over $\alpha$ is always upper bounded by $\lambda$:
\begin{enumerate}
    \item\label{item_prop:1} $g(\alpha, \beta) \leq g(\alpha^\star, \lambda)$ for all $\alpha, \beta$ \,\,\,(i.e., $g(\cdot, \cdot)$ is maximized at $(\alpha^\star, \lambda)$);
    \item\label{item_prop:2} $g(\alpha, \beta) \leq g(\alpha, \lambda)$ for all $\alpha, \beta$ \,\,\,(i.e., $g(\alpha, \cdot)$ is maximized at $\lambda$);
    \item\label{item_prop:3} $g(\alpha, \lambda) \leq g(w, \lambda)$ for all $\alpha^\star \leq w \leq \alpha$ \,\,\,(i.e., $g(\cdot, \lambda)$ decreases on $[\alpha^\star, 1]$);
    \item\label{item_prop:5} $g(\alpha, \beta) \leq g(\alpha, w)$ for all $\lambda \leq w \leq \beta$ and all $\alpha$ \,\,\,(i.e., $g(\alpha, \cdot)$ decreases on $[\lambda, 1]$);
    \item\label{item_prop:7} $g(\alpha, \beta) \leq g(w, \beta)$ for all $\lambda \leq w \leq \alpha$ and all $\beta$ (i.e., $g(\cdot, \beta)$ decreases on $[\alpha^*(\beta), 1]$ and hence on $[\lambda, 1]$);
\end{enumerate}
Fix some $\alpha \leq \beta$.\\
\textbf{Case 1 ($w \leq \alpha^{\star}$)}: by item \eqref{item_prop:1} the desired statement holds. \\
\textbf{Case 2 ($w \in [\alpha^\star, \lambda)$)}: thanks to items~\eqref{item_prop:2},\eqref{item_prop:3} if $w \in [\alpha^\star, \alpha)$,
item~\eqref{item_prop:2} if $\alpha^* \leq \alpha \leq w \leq \beta$,
item~\eqref{item_prop:2} if $w \in [\beta, \lambda)$
\begin{align*}
    \begin{cases}
        g(\alpha, \beta) & \mbox{if } w < \alpha,\\
        g(w, \beta) & \mbox{if } \alpha \leq w < \beta,  \\
        g(w, w) & w\geq \beta.
    \end{cases} \quad\leq\quad g(w, \lambda)\enspace.
\end{align*}
and the desired statement holds.\\
\textbf{Case 3 ($w \geq \lambda$)}: thanks to items~\eqref{item_prop:5},\eqref{item_prop:7} if $w \leq \alpha$ and $\beta \geq \alpha \geq \lambda$, item~\eqref{item_prop:5} if $w \in [\alpha, \beta)$ and $\alpha \geq \lambda$
\begin{align*}
    \begin{cases}
        g(\alpha, \beta) & \mbox{if } w < \alpha,\\
        g(w, \beta) & \mbox{if } \alpha \leq w < \beta,  \\
        g(w, w) & w\geq \beta.
    \end{cases} \quad\leq\quad g(w, w)
\end{align*}
and the desired statement holds.
\end{proof}



While the recursive expression in Theorem \ref{thm:asymptoticDDT} on the success probability gives a way to numerically compute the optimal thresholds, it is still unclear if they are universal and do not depend on the initial budget.
For any initial budget $B \geq 0$, let $(\alpha^B_b, \beta^B_b)_{b=0}^B$ be some optimal choice of thresholds, i.e
\[
(\alpha^B_b, \beta^B_b)_{b=0}^B
\in \argmax_{(\alpha_b, \beta_b)_{b=0}^B} \varphi_B\left(w;(\alpha_b, \beta_b)_{b=0}^B\right) \enspace.
\]
For $B = 0$, Proposition \ref{prop:DDT0-optanyw} establishes that the optimal thresholds $\alpha^0_0, \beta^0_0$ maximize the mapping $(\alpha, \beta) \mapsto \varphi_0(w; (\alpha, \beta))$. Given that $\varphi\big(0,(\alpha^1_b, \beta^1_b)_{b=0,1}\big) = g_1(\alpha^1_1, \beta^1_1)$ and by definition of $g_1$, the previous statement implies that 
\begin{align*}
\varphi_1(0;(&\alpha^1_b, \beta^1_b)_{b=0,1})\\
&= g_0(\alpha^1_0, \beta^1_0)+ (1 - \lambda) \alpha^1_1 \int_{\alpha^1_1}^{\beta^1_1} \frac{\phi_{0}\left(u ; (\alpha^1_0, \beta^1_0)\right)}{u^2} \diff u {+} \alpha^1_1\beta^1_1 \int_{\beta^1_1}^1 \frac{\phi_{0}\left(u ; (\alpha^1_0, \beta^1_0) \right)}{u^3} \diff u\\
&\leq g_0(\alpha^0_0, \beta^0_0)+ (1 - \lambda) \alpha_1 \int_{\alpha_1}^{\beta_1} \frac{\phi_{0}\left(u ; (\alpha^0_0, \beta^0_0)\right)}{u^2} \diff u {+} \alpha_1\beta_1 \int_{\beta^1_1}^1 \frac{\phi_{0}\left(u ; (\alpha^0_0, \beta^0_0) \right)}{u^3} \diff u\\
&= \varphi_1(0;(\alpha^0_0, \beta^0_0), (\alpha^1_1, \beta^1_1))\enspace.
\end{align*}
Thus, the thresholds $((\alpha^0_0, \beta^0_0), (\alpha^1_1, \beta^1_1))$ are also optimal. We can prove by induction that for any $B \geq 0$, replacing $(\alpha^B_0, \beta^B_0)$ by $(\alpha^0_0, \beta^0_0)$ can only increase the success probability, deducing that optimal values $\alpha_0, \beta_0$ can be chosen independently of the initial budget.


Deriving this property for any $b \geq 1$ is more challenging. The main difficulty lies in the increasing complexity of functions $g_b$ compared to $g_0$. Nevertheless, numerical simulations show that indeed the optimal thresholds are independent of the initial budget.
If supported by theoretical analysis, the above would imply optimal thresholds $(\alpha^\star_b, \beta^\star_b)_{b\geq 0}$ can be obtained by setting $(\alpha^\star_0, \beta^\star_0)$ as in Theorem \ref{thm:prob-win0}, then for any $B \geq 1$ computing
\[
(\alpha^\star_B, \beta^\star_B) \in 
\argmax_{(\alpha_B, \beta_B)} \varphi_B\big{(}0;((\alpha^\star_b, \beta^\star_b)_{b=0}^{B-1}, (\alpha_B, \beta_B))\big{)} \enspace.
\]
Theoretically, and practically, the above characterization of the optimal thresholds is much simpler and more convenient. The rigorous analysis of this phenomenon is left for future work.


\subsection{Using the optimal memory-less algorithm}
In this section, we give a numerical simple way for computing the optimal thresholds, using the intuitions, presented in Section \ref{sec:opt-memless-asymptotic}, on the asymptotic behavior of the optimal memory-less algorithm. First, we prove the differential equation, stated in that section, verified by $\varphi_0(\cdot;(\alpha_0^\star, \beta_0^\star))$.
\begin{proposition}\label{prop:phi0-DE}
the function $w \mapsto \varphi_0(w;(\alpha_0^\star, \beta_0^\star))$ verifies the following differential equation
\[
- \varphi_0'(w;(\alpha_0^\star, \beta_0^\star))
= \left( \lambda - \frac{\varphi_0(w;(\alpha_0^\star, \beta_0^\star))}{w} \right)_{+}
+ \left( 1 - \lambda - \frac{\varphi_0(w;(\alpha_0^\star, \beta_0^\star))}{w} \right)_{+}.
\]
\end{proposition}

\begin{proof}
Let us denote simply $\varphi_0(w)$ instead of $\varphi_0(w;(\alpha_0^\star, \beta_0^\star))$. With Theorem \ref{thm:asymptoticDDT}, we can write $\varphi_0(w) = g_0(\max(\alpha_0^\star, w), \max(\beta_0^\star, w))$, where $(\alpha_0^\star, \beta_0^\star) = (\lambda \exp(1/\lambda - 2), \lambda)$. We will prove the result of the proposition on each of the intervals $[0,\alpha_0^\star), [\alpha_0^\star, \beta_0^\star)$ and $[\beta_0^\star, 1]$. Recall that $\lambda \geq 1/2$.

For $w \leq \alpha_0^\star$ we have $\varphi_0'(w) = \lambda \exp(1/\lambda - 2) = \lambda \alpha_0^\star$. Thus $\frac{\varphi_0(w)}{w} \geq \frac{\varphi_0(w)}{\alpha£_0^\star} = \lambda$. Therefore
\[
- \varphi_0'(w)
= 0
= \left( \lambda - \frac{\varphi_0(w)}{w} \right)_{+}
+ \left( 1 - \lambda - \frac{\varphi_0(w)}{w} \right)_{+}.
\]

For $\alpha_0^\star \leq w < \beta_0^\star$ whe have
\[
\varphi_0(w) 
= g_0(w, \lambda)
= \lambda w \log(\lambda/w) + (1-\lambda)w, 
\]
and $\varphi_0(w) / w = \lambda \log(\lambda/w) + (1-\lambda)$, which is a decreasing function of $w$, and that equals $\lambda$ for $ w = \alpha_0^\star$ and $1-\lambda$ for $w = \lambda = \beta_0^\star$. Therefore $1-\lambda \leq \frac{\varphi_0(w)}{w} \leq \lambda$, and
\begin{align*}
- \varphi_0'(w)
&= \lambda \log(w/\lambda) + \lambda - (1 - \lambda)\\
&= \lambda - (\lambda \log(\lambda/w) + 1 - \lambda)\\
&= \left( \lambda - \frac{\varphi_0(w)}{w} \right)_{+}
+ \left( 1 - \lambda - \frac{\varphi_0(w)}{w} \right)_{+}. 
\end{align*}

Finally, for $w \geq \beta_0^\star$, we have
$\varphi_0(w) = w - w^2$, thus $\frac{\varphi_0(w)}{w} = 1 - w \leq 1 - \beta_0^\star =  1 - \lambda$, and 
\begin{align*}
- \varphi_0'(w)
&= 2w - 1
= (\lambda - (1-w)) + (1 - \lambda - (1-w))\\
&= \left( \lambda - \frac{\varphi_0(w)}{w} \right)_{+}
+ \left( 1 - \lambda - \frac{\varphi_0(w)}{w} \right)_{+}. 
\end{align*}
\end{proof}


\paragraph{Acceptance regions of Algorithm \ref{algo:opt-memless}}

\begin{figure}
    \centering
    \includegraphics[width=0.85\textwidth]{pics/acceptance_region_N50.pdf}
    \includegraphics[width=0.85\textwidth]{pics/acceptance_region_N1000.pdf}
    \caption{Acceptance regions of the optimal memory-less algorithm, with $\lambda = 0.6$, $B= 0$ and $N = 50, 1000$.}
    \label{fig:acceptance_opt}
\end{figure}

Figure \ref{fig:acceptance_opt} shows the acceptance region (dark green) of the optimal algorithm, when $\lambda = 0.6$ and $B=0$, for $N = 50,1000$, i.e., the region where it will return a candidate if $r_t = 1$.
On the x- and y-axes we display the time $t$ and possible group cardinal $|G_t^{1}|$ up to time $t$ respectively. The latter follows binomial random distribution with parameters $(\lambda, t)$ and tightly concentrates around its mean $|G_t^{1}| \approx \lambda t$ (resp. $|G_t^{2}| \approx (1-\lambda) t$) for already moderate values of $t$.
We first remark that the acceptance region depends on the group, since the group membership is stochastic with parameter $\lambda$, the algorithm tries to sufficiently explore either of the groups.
% if at a given step $t$ we have for example $|G^1_t| = 0$ and the candidate who arrives at the next step is in the group $t$, the algorithm will not return it because it did not sufficiently observe the elements of this group yet.
% The same remark is valid for group $G^2$.
Then, let us observe that the shape of the acceptance region is almost identical for $N = 50$ and $N = 1000$, i.e., the curve defining this region converges rapidly to a curve that only depends on $(t/N, |G^1|/N)$.
As recalled above, when $N$ is very large we have that $|G^1_t| \approx\lambda t$, thus, the acceptance region is defined only by a threshold at the intersection of the acceptance curve and the line $|G^1_t| \approx\lambda t$.
Numerically, this intersection for $N=1000$ occurs for $t/N \approx 0.428$ for $G^1$, and $t/N \approx 0.598$ for $G^2$, while the optimal thresholds of the DDT algorithm, given by Proposition \ref{prop:asymptoticDDT0}, are $\alpha_0^\star = 0.430$ and $\beta_0^\star = 0.6$.

The same observations hold for superior values of $B$ and with different values of $\lambda$, where the acceptance region is where the algorithm uses a comparison if $r_t = 1$. 
This shows that the optimal thresholds for any $\lambda$, any $B$, and any $g \in \{1,2\}$ can be estimated as the intersection of the acceptance region for $G^g$ and the line $(t,\lambda t)$, with $N = 500$ for example.

\paragraph{Optimal thresholds via Equation \eqref{eq:Phib}}
To fully exploit the asymptotic behavior of Algorithm \ref{algo:opt-memless}, for a given value of $\lambda$, the optimal thresholds $(\alpha_b^\star(\lambda), \beta_b^\star(\lambda))_{b=1}^B$ can be estimated by numerically solving Equations \eqref{eq:Phib} for $1 \leq b \leq B$, then taking for $b$ as thresholds the respective solutions to 
\[
\lambda \alpha 
= \Phi_b(\alpha) - (1-\lambda)\Phi_b(\alpha)
\quad \text{ and } \quad
(1-\lambda) \beta
= \Phi_b(\beta) - \lambda \Phi_b(\beta).
\]








\section{Non-asymptotic analysis of the unique-threshold algorithm}\label{appx:bench-algo}

While we only proved previously asymptotic guarantees for DDT algorithms, this section focuses on how fast the success probability converges to the asymptotic regime. For that, we consider a particular case of DDT algorithm where all the thresholds equal the same value $\w \in (0,1)$. We show, in this case, that the asymptotic success probability is independent of $\lambda$, but that the convergence rate strongly depends on it.
The approach considered here is presented in Algorithm~\ref{algo:bench-algo}. The proof techniques here differ significantly from the ones used for the asymptotic analysis of DDT algorithms. We explain this with more details in the next section.


\begin{algorithm}
\DontPrintSemicolon 
\caption{Benchmark algorithm with $B$ comparisons}\label{algo:bench-algo}
\SetKwInput{Input}{Input}
   \SetKwInOut{Output}{Output}
   \SetKwInput{Initialization}{Initialization}
   \Input{number of explorations $\w N$}
   \Initialization{budget $B$}
\For{$t=1,\ldots,N$}{
    Receive new observation: $(r_t, g_t) \in  \{1, \ldots, t\} \times \{1, 2\}$\;
    \If(\tcp*[f]{\small compare in-group}){$t \ge \w N \And r_t=1$}{%\label{algoline:better-than-best}
        
        \If(\tcp*[f]{\small check budget}){$B > 0$}{
            Update budget: $B \gets B - 1$\;


            \If(\tcp*[f]{\small compare inter-group}){$R_t=1$}{\label{algoline:compare-inter-group}
            Return: $t$
            }
        }
        \lElse
        {
        Return: $t$} 
        % \EndIf
    }
}

\end{algorithm}

\subsection{Main result}
We estimate in this section the success probability of Algorithm \ref{algo:bench-algo}. The parameters of the algorithm are the total number of candidates $N$, the fraction of skipped candidates in the exploration phase $\w$, and the budget of comparisons $B$. 
The main result of this section is concerned with the behavior of
\begin{align*}
    P_{N}^B(\alpha) \eqdef \Prob(\A(\w) \text{ succeeds})\enspace,
\end{align*}
where $\A(\w)$ is the Algorithm \ref{algo:bench-algo} with budget $B$ and exploration phase of size $\w N$. As in the main body of the paper, we assume here also that $\lambda \geq 1/2$. Our main result is the following theorem.

\begin{theorem}\label{thm:lb-PB(s,N)}
Let $\w \in (0,1)$ and $N \in \mathbb{N}$ be such that $\frac{N}{\log N} \geq \frac{(1 + 1/\sqrt{\w})^2}{2\sqrt{\w}(1 - \lambda)^2}$.
% The success probability $P_B(\w N,N)$ of Algorithm \ref{algo:bench-algo} with budget $B$ satisfies
Then,
\begin{align*}
    P_{N}^B(\alpha) \geq P_{\infty}^B(\alpha) - K_{\lambda,\w,B} \sqrt{\frac{\log N}{N}}\enspace,
\end{align*}
with
\begin{align*}
P_{\infty}^B(\alpha) 
&= \w^2 \sum_{b=0}^B \left( \frac{1}{\w} - \sum_{\ell=0}^{b} \frac{\log(1/ \w)^\ell}{\ell !}   \right) \\
\text{ and }\, \quad
K_{\lambda,\w,B} 
&= \frac{\sqrt{2}(B+1)\log(1/ \w)}{\lambda(1-\lambda)} 
  {+} \frac{\e}{\sqrt{\w}} \sum_{b=0}^{B-1} \log(1/ \w)^b\enspace.
\end{align*}
\end{theorem}
Let us comment on the above derived lower bound on the success probability of Algorithm \ref{algo:bench-algo}. The leading term depends neither on $N$ nor on $\lambda$ and, it corresponds exactly to the asymptotic success probability of Algorithm \ref{algo:bench-algo} and it only depends on $\w$ and $B$---the maximization of the leading term in $\w$ does not require the knowledge of $\lambda$. The second term decreases in $N$ and essentially corresponds to the speed of concentration of $|G^1_t|$ around its mean. However, the multiplicative constant depends on $\lambda$ and becomes arbitrarily large when $\lambda$ approaches $1$. We recall that both $\lambda$ and $B$ are assumed to be fixed constants that do not depend on $N$. 
The full proof of this Theorem is given in Section \ref{sec:proof-benchmark} below. However, we give here a sketch of the main steps.
\begin{proof}[Sketch of the proof]

We observe first that the times $\{ \rho_b \}_{b=1}^{B+1}$ when Algorithm \ref{algo:bench-algo} uses comparisons (as defined in Section \ref{sec:setup}) are such that for any $1 \leq b \leq B+1$
\begin{align*}
\rho_b = \min\left\{ t > \rho_{b-1} \mid  r_t = 1\right\}\enspace.
\end{align*}
with $\rho_0 \eqdef \w N$. These times are central in the analysis of Algorithm~\ref{algo:bench-algo} as they give a convenient decomposition of the success probability of the algorithm. In particular, we show that if $\tau$ is stopping time of Algorithm~\ref{algo:bench-algo}, then
\begin{align}
    \label{eq:benchmark_prob_decomposition}
        P_{N}^B(\w) = \sum_{b=1}^{B+1} p_{N}^b(\w)\enspace,
\end{align}
where 
$p_{N}^b(\w) \eqdef \Prob (\tau = t^\star \text{ and } \tau = \rho_b)$ for any $1 \leq b \leq B+1$.

We show that $p_{N}^1(\w) = \w - \w^2 - O(\sqrt{\log N / N})$ and that for $2 \leq b \leq B+1$
\[
p_{N}^b(\w) \geq \left(\w^2 N \sum_{i=r}^N \frac{1}{i^2} \sum_{s \leq t_1 < \ldots < t_{b-1} < i} \prod_{\ell=1}^{b-1} \frac{1}{t_\ell}\right)
- O\left( \sqrt{\frac{\log N}{N}} \right)
\]
with explicit $O$ constants.
Finally, a good estimation of the sum above gives the claimed result.
\end{proof}


\begin{remark}
While the previous theorem only establishes a lower bound of the form $P_{N}^B(\w) \geq P_{\infty}^B( \w) - O(\sqrt{\log N /N})$, one analogously obtains $P_{N}^B(\w) \leq P_{\infty}^B(\w) + O(\sqrt{\log N /N})$, implying that 
\[
\lim_{N \to \infty} P_{N}^B(\w) = P_{\infty}^B( \w)\enspace.
\]
Moreover, using Theorem \ref{thm:asymptoticDDT} giving the asymptotic success probability of DDT algorithms, we can verify by induction that setting all the thresholds equal to the same value $\w$ gives $P_{\infty}(\w)$.
\end{remark}

Although the success probability of Algorithm \ref{algo:bench-algo} with a good threshold choice converges to $1/\e$ as $B$ increases, a weakness of the algorithm is that its asymptotic success probability does not take into account the distribution of the groups $\lambda$. Indeed, when $\lambda$ approaches $1$, one should be able to reach nearly the $1/\e$ asymptotic success probability, since a naive algorithm that ignores the elements of the group $G^2$ and uses the classical reject-accept strategy only on $G^1$ results in a success probability of $\lambda/\e$. Yet, even in the case $\lambda$ goes to $1$, the asymptotic success probability of Algorithm \ref{algo:bench-algo} stays away from $1/\e$. 






\subsection{Proof of Theorem \ref{thm:lb-PB(s,N)}}\label{sec:proof-benchmark}


\begin{lemma}\label{lem:tau=rho}
With $B$ comparisons allowed, the stopping time $\tau$ of Algorithm \ref{algo:bench-algo} is such that 
\begin{enumerate}
    \item $\tau \in \{\rho_1, \ldots, \rho_{B+1} \}\cup\{\infty\}$;
    \item if $\tau = \rho_b$ for some $b \in \{2,\ldots, B+1\}$, then $g_{\rho_1}=g_{\rho_2}=\ldots=g_{\rho_{b-1}}$;
    \item denoting by $t^\star$ the index of the best candidate, the success probability of Algorithm \ref{algo:bench-algo} can be expressed as
\begin{align}
    \label{eq:benchmark_prob_decomposition_app}
        P^B_N(\w) = \sum_{b=1}^{B+1} p^b_N(\w)\enspace,
\end{align}
where 
\begin{align*}
   p^b_N(\w) \eqdef \Prob ( \tau = t^\star \text{ and } \tau = \rho_b)
   \quad 1 \leq b \leq B+1\enspace.
\end{align*}
%    \item if $x_{\max}$ is the best candidate, then the success probability of Algorithm \ref{algo:bench-algo} can be expressed as
%\begin{align}
%    \label{eq:benchmark_prob_decomposition}
%        P_B(s,N) = \sum_{b=1}^{B+1} p_b(s,N)\enspace,
%\end{align}
%where 
%\begin{align*}
%   p_b(s,N) \eqdef \Prob [ x_\tau = x_{\max} \text{ and } \tau = \rho_b]
%   \quad 1 \leq b \leq B+1\enspace.
%\end{align*}
\end{enumerate}
\end{lemma}
%\nic{The inequality $x_\tau > \best G^{g_\tau}_{\tau - 1}$ is strict in the proof while it is not in the algorithm. Which is better?}
\begin{proof}
% \paragraph{11}
    Let us prove the three claimed statements.\\
    \textbf{First claim.} 
    If the algorithm does not return any candidate, $\tau = \infty$ by convention. Otherwise, it must stop at some time $t \geq s$ such that $r_t = 1$, \textit{i.e.}, $\tau \in \{ \rho_b \mid b \geq 1\}$. Moreover, in that case and given the available budget $B$, the stopping time $\tau$ is upper bounded by $\rho_{B+1}$.\\
    %If the algorithm does not return any candidate, $\tau = \infty$ by convention. Otherwise, it must stop at some time $t \geq s$ such that $x_t \geq \best G^{g_t}_{t - 1}$, \textit{i.e.}, $\tau \in \{ \rho_b \mid b \geq 1\}$. Moreover, in that case and given the available budget $B$, the stopping time $\tau$ is upper bounded by $\rho_{B+1}$.\\
    %It is clear from the algorithm that we can only stop on $\tau$ satisfying $x_\tau > \best G^{g_\tau}_{\tau - 1}$ which means that $\tau \in \{ \rho_b \mid b \geq 1\}$ by Definition \ref{def:bench-algo-rho}, and since we have a budget of $B$ comparisons, this condition can be only satisfied $B+1$ times during a run of the algorithm. In the case where the algorithm does not return any candidate, we have by convention that $\tau = \infty$.\\
    %\textbf{Second claim.} Let $b \geq 2$. Assume that $\tau = \rho_b$ and, w.l.o.g., that $g_{\rho_1}  = 1$. Since $\tau \neq \rho_1$, then $\best G^1_{r-1} < \best G^2_{r-1}$. Indeed, otherwise the algorithm would have terminated at $\rho_1$. \nic{I am not convinced by this: $\tau \neq \rho_1$ implies $x_{\rho_1} \geq G_{\rho_1}^1$ and $x_{\rho_1} < G_{\rho_1}^2$ but it does not tell us how $\best G^1_{r-1}$ and $\best G^2_{r-1}$ compare.}
    %Assume that $\rho_2, \ldots, \rho_{b-1}$ are not all in  $G^1$ and let $\ell$ be the smallest index in $\{2, \ldots, b-1\}$ such that $g_{\rho_\ell} = 2$. On the one hand, since algorithm did not terminate at $\rho_\ell$, then no candidate in $\{x_r, \ldots, x_{\rho_\ell-1}\}$ is better than $\best G^2_{r-1}$. On the other hand, by definition of $\rho_\ell$, we have $x_{\rho_\ell} > \best G^2_{\rho_\ell-1} = \best G^2_{r-1} > \best G^1_{\rho_\ell-1}$. Thus, the algorithm must terminate on $\rho_\ell$ contradicting our assumption that $\rho_\ell < \rho_b = \tau$.\\
    \textbf{Second claim.}  Assume that $\tau = \rho_b$ for some $b \geq 2$. Moreover, let us assume without loss of generality that $g_{\rho_1}  = 1$. 
    %Since $\tau \neq \rho_1$, then $\best G^1_{\rho_1} < \best G^2_{\rho_1}$. Indeed, otherwise the algorithm would have terminated at $\rho_1$. 
    By contradiction, assume that the potential stopping times $\rho_1, \ldots, \rho_{b-1}$ are not all in the first group.
    Let $\ell \coloneqq \min\{ i \leq b-1 : g_{\rho_\ell} = 2 \}$.
    Let us show that no candidate in $\{x_r, \ldots, x_{\rho_\ell-1}\}$ is better than $\best G^2_{r-1}$.
    By definition of the index $\ell$, $ \best G^2_{\rho_\ell -1} = \best G^2_{\rho_0}$.
    Moreover, since the algorithm did not stop at $\rho_1,\ldots,\rho_{\ell-1}$, no candidate from group 1 in $\{x_r, \ldots, x_{\rho_\ell-1}\}$ is better than $\best G^2_{r-1}$.
    By definition of $\rho_\ell$,  
    \[
    x_{\rho_\ell} \geq \best G^2_{\rho_\ell-1} = \best G^2_{\rho_0} \geq \best G^1_{\rho_0}.
    \]
    Thus, the algorithm must terminate on $\rho_\ell$ contradicting our assumption that $\rho_\ell < \rho_b = \tau$. \\
    \textbf{Third claim.} The claim follows immediately from the first claim.
    % \[
    %     P_B(s,N) = \Prob [ x_\tau = x_{\max}]
    %              = \sum_{b=1}^{B+1} p_b(s,N)\enspace.
    % \]
\end{proof}


Lemma \ref{lem:tau=rho} shows that estimating $P^B_N(\w)$ comes down to estimating the probabilities $p^b_N(\w), b \in \{1, \ldots, B+1\}$. In order to do that
we will need a concentration inequality for controlling the cardinals of the sets $G^1_t$ and $G^2_t$ for $t \in \{s, \ldots, N\}$. We use for that we use Lemma \ref{lem:conc-card}, which was also used in the asymptotic analysis of DDT algorithms.


\begin{remark}
For any $t \geq 1$ we have
\begin{align*}
\left| |G^1_t| - \lambda t \right|
&= \left| t - |G^2_t| - \lambda t \right|\\
&= \left| |G^2_t| - (1 - \lambda) t \right|\enspace.
\end{align*}
Consequently, the above Lemma provides a concentration inequality for the cardinality of both $G^1_t$ and $G^2_t$ for any time $t$.
\end{remark}

The next sequence of results provide the control of $p_b(r, N)$ appearing in the decomposition of Eq.~\eqref{eq:benchmark_prob_decomposition_app}. We treat separately the case $b=1$ from the other cases.
This first Lemma is purely technical and provides some bounds that we use in the proofs. We remind that we assume $\lambda \geq 1/2$.
\begin{lemma}\label{lem:N-suff-large}
Let $N,s$ be two positive integers such that $s = \w N$ fro some $\w \in (0,1)$. If and if $\frac{N}{\log N} \geq \frac{(1 + 1/\sqrt{\w})^2}{2\sqrt{\w}(1 - \lambda)^2}$ then
\[
    1 \leq A_\w \sqrt{s \log s} \leq \frac{(1-\lambda)s}{2}
\]
\end{lemma}
\begin{proof}
The proof is immediate
\[
\frac{s}{\log s}
\geq \frac{\w N}{\log N}
\geq \frac{(1 +\sqrt{\w})^2}{2\sqrt{\w}(1 - \lambda)^2}
= \frac{4 A_\w^2}{(1-\lambda)^2}
\]
therefore $\sqrt{s/\log s} \geq 2 A_\w / (1-\lambda)$, multiplying by $(1-\lambda)\sqrt{s \log s}$ gives
\[
2 A_\w \sqrt{s \log s} \leq (1-\lambda)s.
\]
For the other inequality, using $1-\lambda \leq 1/2$ and observing that $A_\w \geq 1/\sqrt{2}$ we obtain
\[
s \geq \frac{s}{\log s} \geq \frac{4A_\w^2}{(1-\lambda)^2} 
\geq 8,
\]
thus $A_\w \sqrt{s \log s} \geq A_\w \sqrt{s} \geq \frac{1}{\sqrt{2}} 2 \sqrt{2}$.
\end{proof}


\begin{lemma}\label{lem:p1(s,N)}
% Assume that $s = \w N$.
Set $s = \w N$, if $\frac{N}{\log N} \geq \frac{(1 + 1/\sqrt{\w})^2}{2\sqrt{\w}(1 - \lambda)^2}$, then
\begin{align*}
p_N^1(\w)
\geq (\w - \w^2) \left( 1 - \frac{4A_\w}{\lambda(1 - \lambda)}\sqrt{\frac{\log s}{s}} \right)\enspace.
\end{align*}
\end{lemma}

\begin{proof} Let $N\geq2$ be an integer and let $s = \w N$ such that $s \in \{1, \ldots, N\}$. By definition of the probability $p^1_N(\w)$, the law of total probability gives
\begin{align*}
    p^1_N(\w)  
    %=\Prob [ \tau = t^\star \text{ and } \tau = \rho_1]
    = \frac{1}{N} \sum_{i=s}^N \Prob( \tau = \rho_1 = i \mid t^\star = i)\enspace.
 \end{align*}
Let $i \in \{s, \ldots, N\}$. Conditionally on the event that $t^\star=i$, one can check that the equalities $\tau = \rho_1 = i$ hold if and only if none of the candidates between times $s$ and $i{-}1$ are ranked first among the observed candidates inside their groups. Moreover, the latter event depends solely on the relative ranks of the first $i{-}1$ candidates, it is independent of the event that the $i$-th candidate is the best overall. Consequently, we can write
\begin{align*}
 \Prob( \tau = \rho_1 = i \mid t^\star = i) = \Prob\left(\cap_{j=s,\ldots,i-1} \{r_j \neq 1\} \right)\enspace.
\end{align*}
Since the ranks are mutually independent and uniformly distributed conditionally on $(G_{i-1}^1, G_{i-1}^2)$ (see Lemma~\ref{lem:rank_indep_unif}), we obtain
%Enumerating the cases constituting the event $\cap_{j=s,\ldots,i-1} \{r_j \neq 1\}$, we obtain 
\begin{align*}
    \Prob\left(\cap_{j=s,\ldots,i-1} \{r_j \neq 1\} \right)
    = \E \left[ \prod_{j = s}^{i-1} \left(1 - \frac{1}{|G^{g_j}_j|}\right) \right]
    =  \E \left[\frac{|G^1_{s-1}|}{|G^1_{i-1}|} \times \frac{|G^2_{s-1}|}{|G^2_{i-1}|} \right]\enspace.
\end{align*}
%with the convention that $-1/0=1$.

% %We recall that Lemma~\ref{lem:conc-card} ensures that the event
% \begin{align*}
%     \mathcal{C}_{s,N} = \left\{\forall t \in \{s, \ldots, N\} \qquad
%     \left| |G^1_t| - \lambda t \right| < A_\w \sqrt{t \log s}
%     \right\}
% \end{align*}
% holds with probability at least $1 - 2/s$.  Applying the Bayes rule, we then get
% \begin{align*}
%     \Pr\left(\min_{j=s,\ldots,i-1} r_j \neq 1 \right) \geq \Pr\left(\min_{j=s,\ldots,i-1} r_j \neq 1 \mid \mathcal{C}_{s,N}\right)\left(1-\frac{2}{s}\right)\enspace.
% \end{align*}
% Finally, enumerating the cases in which $\min_{j=s,\ldots,i-1} r_j \neq 1$ holds, we obtain
% \begin{align*}
%     \Pr\left(\min_{j=s,\ldots,i-1} r_j \neq 1 \mid \mathcal{C}_{s,N}\right)
%     = \E \left( \prod_{j = s}^{i-1} \left(1 - \frac{1}{|G^{g_j}_j|}\right) \,\big|\, \mathcal{C}_{s,N}\right)
%     =  \E \left( \left.  \frac{|G^1_{s-1}|}{|G^1_{i-1}|} \times \frac{|G^2_{s-1}|}{|G^2_{i-1}|} \right| \mathcal{C}_{s,N} \right)\enspace.
% \end{align*}
% \begin{align*}
%     p_1(s,N)  &=  \Prob [ x_\tau = x_{\max} \text{ and } \tau = \rho_1]\\
%     &= \frac{1}{N} \sum_{i=r}^N \Prob[ \rho_1 = \tau = i \mid x_i = x_{\max}]\\
%     &= \frac{1}{N} \sum_{i=r}^N \Prob[ \best G^1_{i-1} \in G^1_{r-1} \text{ and } \best G^2_{i-1} \in G^2_{r-1}]\\
%     &\geq \frac{1}{N} \sum_{i=r}^N \Prob[ \best G^1_{i-1} \in G^1_{r-1} \text{ and } \best G^2_{i-1} \in G^2_{r-1} \mid \mathcal{C}_{s,N}](1-2/s)\\
%     &= \frac{1}{N} \sum_{i=r}^N \E \left( \left.  \frac{|G^1_{r-1}|}{|G^1_{i-1}|} \times \frac{|G^2_{r-1}|}{|G^2_{i-1}|} \right| \mathcal{C}_{s,N} \right) \left(1 - \frac{2}{s} \right) \\
% \end{align*}
%\nic[inline]{Are you sure that $\Prob[ \rho_1 = \tau = i \mid x_i = x_{\max}] = \Prob[ \best G^1_{i-1} \in G^1_{r-1} \text{ and } \best G^2_{i-1} \in G^2_{r-1}]$? We have $\Prob[ \rho_1 = \tau = i \mid x_i = x_{\max}] = \Prob[ \best G^1_{i-1} \in G^1_{r-1} \text{ and } \best G^2_{i-1} \in G^2_{r-1} \mid x_i = x_{\max}]  \geq \Prob[ \best G^1_{i-1} \in G^1_{r-1} \text{ and } \best G^2_{i-1} \in G^2_{r-1}]$ but I am not 100\% confident about the equality.}
%\ziy[inline]{It's an equality because the event inside the proba depends only on the relative ranks of the elements of $G^1_{i-1}, G^2_{i-1}$, it is thus independent of their rank compared to $x_i$. This argument is very similar to the one used in the classical proof of the success probability in the standard secretary problem.}
%\nic[inline]{Thanks for the clarification, I agree! Let's add a sentence about this in the proof.}
%\nic[inline]{For full rigor, should we state somewhere that we use the convention $\max \varnothing \in \varnothing$ is true? It might happen if we only observe candidates from one group.}
%\ziy[inline]{yes i see. If we condition on $\mathcal{C}_{s,N}$ before it will be good ?}
%\nic[inline]{Yes indeed}
% \evg[inline]{Why don't you condition on $\mathcal{C}_{r-1,N}$ instead of $\mathcal{C}_{s,N}$? maybe you can win some constants, with the former, no?}
%
Let us now provide a lower bound on the expression inside the expectation, conditionally on the event $\mathcal{C}_{s,N}$. 
Note that, on this event, the denominators of this expression are guaranteed to be non-zero for thanks under the hypothesis of the Lemma. Indeed, $|G^1_{i-1}| \geq |G^1_{s-1}| \geq |G^1_{s}|-1$ by monotonicity of $(|G^1_{i}|)_i$ and, on $\mathcal{C}_{s,N}$, $|G^1_{s}| \geq \lambda \w N - A_\w \sqrt{s\log s}$. Therefore, Lemma \ref{lem:N-suff-large} gives that $|G^1_{i-1}| > 0$ holds almost surely conditionally on $\mathcal{C}_{s,N}$. A similar argument shows that the same holds for $|G^2_{i-1}|$.

Assume the event $\mathcal{C}_{s,N}$. Since the sequence $(|G^1_j|)_j$ is non-decreasing,
%\begin{align*}
%\frac{|G^1_{s}|-1}{|G^1_{i-1}|} \geq \frac{|G^1_s|}{|G^1_i|}\enspace.
%\end{align*}
we have
\begin{align*}
    \frac{|G^1_{s-1}|}{|G^1_{i-1}|} \geq \frac{|G^1_s| - 1}{|G^1_i|} \geq \frac{\lambda s - 2 A_\w\sqrt{s\log s}}{\lambda i + A_\w \sqrt{i \log s}} \geq \frac{s}{i}
      \left( 1 - \frac{2 A_\w}{\lambda}\sqrt{\frac{\log s}{s}} \right)
      \left( 1 - \frac{A_\w}{\lambda}\sqrt{\frac{\log s}{i}} \right)\enspace,
\end{align*}
using the fact that $1 \leq A_\w \sqrt{s \log s}$ for the second inequality and $(1+x)(1-x) \leq 1$ for $x \geq 0$ for the third inequality. Finally, using the inequality $(1-a)(1-b) \geq 1-a-b$ for $a, b \geq 0$ gives
\begin{align*}
\frac{|G^1_{s-1}|}{|G^1_{i-1}|} \geq \frac{s}{i} \left( 1 - \frac{3A_\w}{\lambda}\sqrt{\frac{\log s}{s}} \right)\enspace.
\end{align*}
%\nic{Is $|G^1_{i-1}|$ always non-zero on $\mathcal{C}_{s,N}$?}
%\ziy{\we have $|G^1_{i-1}| \geq |G^1_{r-1}|$. If $N$ is sufficiently large then $\mathcal{C}_{s,N}$ guarantees having $|G^1_{r-1}| > 0$ }
%\nic{Thanks I agree! Let's add a sentence about this at the beginning of the proof.}
%\nic{Actually, we might need to condition on $\mathcal{C}_{r-1, N}$ as suggested by Evgenii right? Because $\mathcal{C}_{s, N}$ does not give a direct control on $|G^1_{r-1}|$.}
%the first inequality is immediate because $t \mapsto |G^1_t|$ is non-decreasing and $|G^1_t| - |G^1_{t-1}| \in \{0,1\}$, the second one follows from $1 \leq A_\w \sqrt{r \log s}$ and the fact that $\mathcal{C}_{s,N}$ holds (we can use it since $i, r \in \{s,\ldots,N\}$). Then, we use the fact that $(1+x)(1-x) \leq 1$ for $x \geq 0$. Finally, Bernoulli's inequality gives
% \[
% \frac{|G^1_{r-1}|}{|G^1_{i-1}|}
% \geq \frac{s}{i}
%       \left( 1 - \frac{2A_\w}{\lambda}\sqrt{\frac{\log s}{s}} - \frac{A_\w}{\lambda}\sqrt{\frac{\log s}{i}} \right)\\
% \geq \frac{s}{i} \left( 1 - \frac{3A_\w}{\lambda}\sqrt{\frac{\log s}{s}} \right).
% \]
The same argument gives a similar lower bound on $\frac{|G^2_{s-1}|}{|G^2_{i-1}|}$ in which $\lambda$ is replaced by $1-\lambda$. 

Overall, we obtain the lower bound
\begin{align*}
p^1_N(\w) 
&\geq \frac{1}{N} \sum_{i=r}^N 
      \frac{s^2}{i^2} \left( 1 - \frac{3A_\w}{\lambda}\sqrt{\frac{\log s}{s}} \right) 
                      \left( 1 - \frac{3A_\w}{1-\lambda}\sqrt{\frac{\log s}{s}} \right)
                      \left( 1 -\frac{2}{s} \right)\enspace,
\end{align*}
that can be loosened to
\begin{align}
\label{eq:LB_p1}
p^1_N(\w) 
&\geq \left( 1 - \frac{4A_\w}{\lambda(1 - \lambda)}\sqrt{\frac{\log s}{s}} \right)
      \frac{s^2}{N} \sum_{i=s}^N \frac{1}{i^2}\enspace,
\end{align}
 using $\frac{2}{s} \leq 4 A_\w \sqrt{\frac{\log s}{s}} \leq \frac{A_\w}{\lambda(1 - \lambda)}\sqrt{\frac{\log s}{s}}$ and the inequality $(1-a)(1-b)\geq 1-a-b$ for $a,b\geq0$. Observing that
\begin{align*}
\frac{s^2}{N} \sum_{i=s}^N \frac{1}{i^2} 
\geq \frac{s^2}{N} \sum_{i=s}^{N-1} \frac{1}{i^2}
\geq \frac{s^2}{N} \int_{s}^N \frac{dx}{x^2} 
= \frac{s^2}{N} \left( \frac{1}{s} - \frac{1}{N} \right)
= \w - \w^2\enspace,
\end{align*}
we conclude by substituting the above inequality into \eqref{eq:LB_p1}.
% we obtain the claimed lower bound
% \begin{align*}
% p_1(s,N)
% \geq \left( 1 - \frac{4A_\w}{\lambda(1 - \lambda)}\sqrt{\frac{\log s}{s}} \right)(w - \w^2)\enspace.
% \end{align*}
\end{proof}

\begin{lemma}\label{lem:pb-sum}
Set $s = \w N$, if $\frac{N}{\log N} \geq \frac{(1 + 1/\sqrt{\w})^2}{2\sqrt{\w}(1 - \lambda)^2}$, then for any $b\geq 2$,
\[
p^b_N(\w) 
\geq
\left(
\frac{s^2}{N} \sum_{i=s}^N \frac{1}{i^2} \sum_{s \leq t_1 < \ldots < t_{b-1} < i} \prod_{\ell=1}^{b-1} \frac{1}{t_\ell}\right)
\left( 1 - \frac{2(b + 1)A_\w}{\lambda(1 - \lambda)} \sqrt{\frac{\log s}{s}} \right)\enspace,
\]
with $A_\w$ defined in Lemma \ref{lem:conc-card}.
\end{lemma}

\begin{proof}

Let $b \geq 2$. Using the law of total probability and the Bayes rule, we have
\begin{align*}
    p^b_N(\w) 
    %&= \Prob [ \tau = t^\star \text{ and } \tau = \rho_b]\\
    &= \frac{1}{N} \sum_{i=s}^N \Prob( \rho_b = \tau = i \mid t^\star = i)\\
    &= \frac{1}{N} \sum_{i=s}^N \sum_{r \leq t_1 < \ldots < t_{b-1} < i} \Prob( \rho_b = \tau = i, \rho_1 = t_1, \ldots, \rho_{b-1} = t_{b-1} \mid t^\star = i)\enspace.
\end{align*}
Fix  $s \leq i \leq N$ and $s{-}1 < t_1 < t_2 < \ldots < t_{b-1}<i$. Define the set $T \coloneqq \{t_1, t_2, \ldots, t_{b-1}\}$ and the events
\begin{align*}
\mathcal{W}_{T}^{(s,i)} &\coloneqq \bigcap_{j \in \{s, \ldots, i-1\}\setminus T} \{ r_j \neq 1\} \text{ and } \mathcal{B}_T \coloneqq \bigcap_{j \in T} \{r_j = 1, R_j \neq 1\}\enspace.
\end{align*}
Introducing $\mathcal{I}_{s, i, T} \coloneqq \mathcal{W}_{T}^{(s,i)} \cap \mathcal{B}_T \cap \{r_i =1, R_i=1\}$, one can check that
\begin{align*}
\Prob( \rho_b = \tau = i, \rho_1 = t_1, \ldots, \rho_{b-1} = t_{b-1} \mid t^\star = i) = \Prob(\mathcal{I}_{s, i, T} \mid t^\star=i)\enspace.
\end{align*}
Let us denote $g^\star_{< s}$ the group to which belongs the best candidate among the $s$ first ones, then  using the law of total probability, we have 
\[
\Pr(\mathcal{I}_{s, i, T} \mid t^\star=i)
= (1-\lambda) \Pr(\mathcal{I}_{s, i, T} \mid t^\star=i, g^\star_{< s} = 2)
+ \lambda \Pr(\mathcal{I}_{s, i, T} \mid t^\star=i, g^\star_{< s} = 1)\enspace,
\]
and we proved in Lemma~\ref{lem:tau=rho} that if $t_b = \tau$ then $g_{t_1} = g_{t_2} =  \ldots = g_{t_{b-1}}$, and if their value is equal to $g^\star_{< s}$, then the algorithm would have stopped at $t_1$ and the the probability of $\mathcal{I}_{s, i, T}$ would be null. Consequently,
\begin{align} \nonumber
\Prob(\mathcal{I}_{s, i, T} \mid t^\star=i)
=& (1-\lambda) \Pr((g_{t}=1)_{t \in T} \mid t^\star=i, g^\star_{< s} = 2 ) \Prob(\mathcal{I}_{s, i, T} \mid t^\star=i, g^\star_{< s} = 2, (g_{t}=1)_{t \in T}) 
\\ \nonumber
&+ \lambda \Prob((g_{t}=2)_{t \in T} \mid t^\star=i, g^\star_{< s} = 1)
\Prob(\mathcal{I}_{s, i, T} \mid t^\star=i, g^\star_{< s} = 1, (g_{t}=2)_{t \in T})
\\ \nonumber
=& (1 - \lambda)\lambda^{b-1} \Prob(\mathcal{I}_{s, i, T} \mid t^\star=i, g^\star_{< s} = 2, (g_{t}=1)_{t \in T})\\ \label{proofeq:pr(I)}
&+ \lambda (1-\lambda)^{b-1} \Prob(\mathcal{I}_{s, i, T} \mid t^\star=i, g^\star_{< s} = 1, (g_{t}=2)_{t \in T})
\enspace.
\end{align}
where the last equation holds because $T \subset \{ s, \ldots, i-1 \}$ and thus $(g_t)_{t \in T}$ are independent of $g_{< s}^\star$ and of the event $t^\star = i$.
%We want to control the probability of the intersection of events
%\begin{align*}
%\mathcal{W}_{T}^{(s,i)} \cap \mathcal{B}_T \cap \{r_i =1, R_i=1\}\enspace,
%\end{align*}
%conditionally on the event $t^\star=i$.
%\begin{align*}
%\Prob[ \rho_b = \tau = i, \rho_1 = t_1, \ldots, \rho_{b-1} = &t_{b-1} \mid t^\star = i, g_{t_1}=1] \\=& \lambda^{b-2} \Prob[\mathcal{W}_{T}^{(s,i)} \cap \mathcal{B}_T \cap \{r_i =1, R_i=1\} \mid t^\star=i, g_{t_1}=1] \enspace.
%\end{align*}
% The events $\mathcal{W}_{T}^{(s,i)}, \mathcal{B}_T$ and $\{r_i =1, R_i=1\}$ are mutually independent conditionally on $G_i \coloneqq (G_i^1, G_i^2)$. Moreover, the first two events are independent of the event $t^\star = i$. Hence, it suffices to compute the probability of each event separately.
% Since the ranks are mutually independent and uniformly distributed conditionally on $G_i$, we get
% \begin{align*}
% \Prob[\mathcal{W}_{T}^{(s,i)}\mid G_i, g_{t_1}=1] = \prod_{j \in \{s, \ldots, i-1\}\setminus T} \Prob[r_j \neq 1 \mid G_j^{g_j}] = \prod_{j \in \{1, \ldots, i-1\}\setminus T} \frac{\lvert G_j^{g_j} \rvert - 1}{\lvert G_j^{g_j} \rvert}
% \end{align*}
% and
% \begin{align*}
% \Prob[\mathcal{B}_T \mid G_i, g_{t_1}=1] = \prod_{j \in T} \Prob[r_j = 1, R_j \neq 1, g_j=1 \mid G_j, g_{t_1}=1] = \lambda^{b-2} \prod_{j \in T} \frac{1}{\lvert G_j^{1} \rvert} \frac{\lvert G_j^{2} \rvert}{\lvert G_j^{2} \rvert +1}\enspace.
% \end{align*}
% Moreover,
% \begin{align*}
% \Prob[r_i = 1, R_i = 1 \mid t^\star=i] = 1\enspace.
% \end{align*}
% %\begin{align*}
% %\Prob[r_i = 1, R_i = 1] = \mathbb{E}\left(\frac{1}{\lvert G_i^{g_i} \rvert} \frac{1}{\lvert G_i^{\bar{g_i}} \rvert + 1}\right)
% %\end{align*}
% Consequently, by conditional independence of $\mathcal{W}_{T}^{(s,i)}$, $\mathcal{B}_T$ and $\{r_i=1, R_i=1\}$,
% \begin{align*}
% \Prob[\mathcal{I}_{s, i, T} \mid t^\star=i, g_{t_1}=1] &= \mathbb{E}\left(\Prob[\mathcal{W}_{T}^{(s,i)}\mid G_i] \times \Prob[\mathcal{B}_T \mid G_i, g_{t_1}=1] \times \Prob[r_i=1, R_i=1 \mid t^\star = i] \right)\\
% &= \mathbb{E}\left(\left(\prod_{j=s}^{i-1} \frac{1}{\lvert G_j^{g_j}\rvert} \right) \times \left( \prod_{j \in \{1, \ldots, i-1\} \setminus T} (\lvert G_j^{g_j}\rvert - 1) \right)\times\left( \prod_{j \in T} \frac{\lvert G_j^{2} \rvert}{\lvert G_j^{2} \rvert +1}\right)\right)\\
% &= \mathbb{E}\left(\left(\prod_{j=s}^{i-1} \frac{\lvert G_j^{g_j}\rvert - 1}{\lvert G_j^{g_j}\rvert} \right) \times \left( \prod_{j \in T} \frac{\lvert G_j^{2} \rvert}{(\lvert G_j^{2} \rvert +1)( \lvert G_j^{1}\rvert - 1)}\right)\right)\\
% &= \mathbb{E}\left(\left(\frac{\lvert G_{s-1}^1 \rvert \cdot \lvert G_{s-1}^2 \rvert}{\lvert G_{i-1}^1 \rvert \cdot \lvert G_{i-1}^2 \rvert}\right) \times \left( \prod_{j \in T} \frac{\lvert G_j^{2} \rvert}{(\lvert G_j^{2} \rvert +1)( \lvert G_j^{1}\rvert - 1)}\right)\right)
% \end{align*}
% Observing that
% \begin{align*}
% \prod_{j \in T} \frac{\lvert G_j^{2}\rvert}{\lvert G_j^{2}\rvert + 1} \geq \prod_{k=1}^{b-1} \frac{k}{k+1} = \frac{1}{b}\enspace,
% \end{align*}
% we obtain the lower bound
% \begin{align*}
% \Prob[\mathcal{I}_{s, i, T} \mid t^\star=i, g_{t_1}=1] &\geq \frac{\lambda^{b-1}}{b} \mathbb{E}\left(\left(\frac{\lvert G_{s-1}^1 \rvert \cdot \lvert G_{s-1}^2 \rvert)}{\lvert G_{i-1}^1 \rvert \cdot \lvert G_{i-1}^2 \rvert}\right) \times \left( \prod_{j \in T} \frac{1}{ \lvert G_j^{1}\rvert - 1}\right)\right)\enspace.
% \end{align*}
Let us denote by $\Prob_G$ the probability measure $\Pr$ conditionally on the groups $G_i^1$, $G_i^2$, the events $\cap_{k=1, \ldots, b-1} \{g_{t_k}=1\}$, $\{g^\star_{<s} = 2\}$ and $\{t^\star=i\}$. In what follows, we derive an explicit formula for the probability
\begin{align*}
\Prob(\mathcal{I}_{s, i, T} \mid t^\star=i,g^\star_{<s} = 2 ,(g_{t}=1)_{t \in T}) = \mathbb{E}[\Prob_G(\mathcal{I}_{s, i, T})]\enspace.
\end{align*}
Note that a similar formula can be derived for $\Prob(\mathcal{I}_{s, i, T} \mid t^\star=i, (g_{t}=2)_{t \in T})$ following exactly the same steps.
Applying the chain rule, we can decompose the probability of interest as 
\begin{align*}
\Prob_G(\mathcal{I}_{s, i, T}) 
=& \Prob_G\left( \bigcap_{j \in \llbracket s, t_1-1\rrbracket} \{r_j \neq 1\} \right)\\
& \times\prod_{k=1}^{b-1}
\Prob_G \left( r_{t_k}=1, R_{t_k} \neq 1 \,\middle\vert\,  \bigcap_{j \in \llbracket s, t_{k}\rrbracket \setminus T} \{\ r_j \neq 1\}, \bigcap_{j \in \llbracket s, t_{k-1}\rrbracket \cap T} \{r_j=1, R_j \neq 1\}  \right)\\
&\times \Prob_G \left(\bigcap_{j \in \llbracket t_{k}, t_{k+1}-1\rrbracket} \{r_j \neq 1\} \,\middle\vert\,  \bigcap_{j \in \llbracket s, t_{k}\rrbracket \setminus T} \{\ r_j \neq 1\}, \bigcap_{j \in \llbracket s, t_{k}\rrbracket \cap T} \{r_j=1, R_j \neq 1\}\right)\\
&\times \Prob_G\left(r_i=R_i=1 \mid  \,\middle\vert\,  \bigcap_{j \in \llbracket s, i-1\rrbracket \setminus T} \{\ r_j \neq 1\}, \bigcap_{j \in \llbracket s, i-1\rrbracket \cap T} \{r_j=1, R_j \neq 1\} \right)\enspace.
\end{align*}
By mutual independence of the within-group ranks conditionally on the groups,
\begin{align*}
\Prob_G\left( \cap_{j \in \llbracket s, t_1-1\rrbracket} \{r_j \neq 1\} \right) = \prod_{j \in \llbracket s, t_1-1\rrbracket} \Prob(r_j \neq 1 \mid G_j^{g_j}) = \prod_{j \in \llbracket s, t_1-1\rrbracket} \frac{\lvert G_j^{g_j}\rvert-1}{\lvert G_j^{g_j} \rvert}\enspace,
\end{align*}
and, for any $1 \leq k \leq b-1$,
\begin{align*}
\Prob_G &\left(\bigcap_{j \in \llbracket t_{k}+1, t_{k+1}-1\rrbracket} \{r_j \neq 1\} \,\middle\vert\,  \bigcap_{j \in \llbracket s, t_{k}\rrbracket \setminus T} \{\ r_j \neq 1\}, \bigcap_{j \in \llbracket s, t_{k}\rrbracket \cap T} \{r_j=1, R_j \neq 1, g_j=1\}\right)\\
&= \Prob_G \left(\bigcap_{j \in \llbracket t_{k}, t_{k+1}-1\rrbracket} \{r_j \neq 1\} \right)\\
&= \prod_{j \in \llbracket t_{k}+1, t_{k+1}-1\rrbracket} \Prob(r_j \neq 1 \mid G_j^{g_j})\\
&= \prod_{j \in \llbracket t_{k}+1, t_{k+1}-1\rrbracket} \frac{\lvert G_j^{g_j}\rvert-1}{\lvert G_j^{g_j} \rvert}\enspace.
\end{align*}

Applying once again the chain rule, we get for any $1 \leq k \leq b-1$,
\begin{align*}
\Prob_G &\left( r_{t_k}=1, R_{t_k} \neq 1 \,\middle\vert\,  \bigcap_{j \in \llbracket s, t_{k}\rrbracket \setminus T} \{\ r_j \neq 1\}, \bigcap_{j \in \llbracket s, t_{k-1}\rrbracket \cap T} \{r_j=1, R_j \neq 1, g_j=1\}  \right) 
\\=& \Prob(r_{t_k}=1 \mid g_{t_k}=1, G_{t_k}^1)\\ &\times\quad  \Prob\left(R_{t_k}\neq 1 \,\middle\vert\,  \bigcap_{j \in \llbracket s, t_{k}\rrbracket \setminus T} \{\ r_j \neq 1\}, \bigcap_{j \in \llbracket s, t_{k-1}\rrbracket \cap T} \{r_j=1, R_j \neq 1, g_j=1\}, r_{t_k}=1\right)\enspace.
\end{align*}

First, notice that
$\Prob(r_{t_k}=1\mid g_{t_k}=1, G_{t_k}^1) = \frac{1}{\lvert G_{t_k}^1 \rvert}$. Then, 
\begin{align*}
\Prob_G & \left(R_{t_k} =  1 \,\middle\vert\,  \bigcap_{j \in \llbracket s, t_{k}\rrbracket \setminus T} \{\ r_j \neq 1\}, \bigcap_{j \in \llbracket s, t_{k-1}\rrbracket \cap T} \{r_j=1, R_j \neq 1, g_j=1\}, r_{t_k}=1, \right)\\
= &\Prob_G\left(x_{t_k} > \max_{1 \leq j \leq t_k-1} x_j \,\middle\vert\,  \bigcap_{j \in \llbracket s, t_{k}\rrbracket \setminus T} \{\ r_j \neq 1\}, \bigcap_{j \in \llbracket s, t_{k-1}\rrbracket \cap T} \{r_j=1, R_j \neq 1, g_j=1\}, r_{t_k}=1\right)\\
= &\Prob_G\left(x_{t_k} > \max_{\substack{1 \leq j \leq s-1 \\ g_j=2}} x_j \,\middle\vert\,  \bigcap_{j \in \llbracket s, t_{k}\rrbracket \setminus T} \{\ r_j \neq 1\}, \bigcap_{j \in \llbracket s, t_{k-1}\rrbracket \cap T} \{r_j=1, R_j \neq 1, g_j=1\}, r_{t_k}=1\right)\\
&= \Prob_G\left(x_{t_k} > \max_{\substack{1 \leq j \leq s-1 \\ g_j=2}} x_j \right)\\
&= \frac{1}{1 + \lvert G_{s-1}^2\rvert}\enspace.
\end{align*}
We used the crucial fact that, on the conditioning event, the best candidate from group $2$ up until time $i-1$ showed up among the first $s-1$ candidates. All in all, we obtain
\begin{align*}
\Prob_G(\mathcal{I}_{s, i, T}) &= \left(\prod_{j \in \llbracket s, t_1-1\rrbracket} \frac{\lvert G_j^{g_j}\rvert-1}{\lvert G_j^{g_j} \rvert}\right) \times \prod_{k=1}^{b-1} \left(\frac{1}{\lvert G_{t_k}^1\rvert} \frac{\lvert G_{s-1}^2\rvert}{1 + \lvert G_{s-1}^2\rvert}  \prod_{j \in \llbracket t_{k}+1, t_{k+1}-1\rrbracket} \frac{\lvert G_j^{g_j}\rvert-1}{\lvert G_j^{g_j} \rvert}\right)\\
&= \prod_{j \in \llbracket s, i-1\rrbracket} \frac{\lvert G_j^{g_j}\rvert-1}{\lvert G_j^{g_j} \rvert} \times \prod_{k=1}^{b-1} \frac{1}{\lvert G_{t_k}^{1}\rvert-1} \times \left(\frac{\lvert G_{s-1}^2\rvert}{1 + \lvert G_{s-1}^2\rvert} \right)^{b-1}\\
&= \frac{\lvert G_{s-1}^1 \rvert \lvert G_{s-1}^2 \rvert}{\lvert G_{i-1}^1 \rvert \lvert G_{i-1}^2 \rvert} \left(\frac{\lvert G_{s-1}^2\rvert}{1 + \lvert G_{s-1}^2\rvert} \right)^{b-1} \prod_{k=1}^{b-1} \frac{1}{\lvert G_{t_k}^{1}\rvert-1}\\
&\geq \frac{\lvert G_{s-1}^1 \rvert \lvert G_{s-1}^2 \rvert}{\lvert G_{i-1}^1 \rvert \lvert G_{i-1}^2 \rvert}
\left(1 - \frac{b-1}{1+\lvert G^2_{s-1} \rvert} \right)
\prod_{k=1}^{b-1} \frac{1}{\lvert G_{t_k}^{1}\rvert}
\enspace.
\end{align*}
where we used Bernoulli's inequality in the last line.
Conditioning on $\mathcal{C}_{s,N}$, similarly to the estimations we did in previous proofs, we obtain
\[
\frac{\lvert G^1_{s-1}\rvert}{\lvert G^1_{i-1}\rvert}
\geq \frac{s}{i}\left( 1 - \frac{3A_\w}{\lambda}\sqrt{\frac{\log s}{s}} \right)\enspace,
\]
and we have the same for $\lvert G^2_{s-1}\rvert / \lvert G^2_{i-1}\rvert$ replacing $\lambda$ by $1-\lambda$. We also have that
\begin{align*}
\frac{1}{{1+\lvert G^2_{s-1} \rvert}}
&\leq \frac{1}{\lambda s - A_\w \sqrt{s \log s}}\\
&\leq \frac{2}{\lambda s}
\leq \frac{\sqrt{2}A_\w}{\lambda} \sqrt{\frac{\log s}{s}}    
\end{align*}


because $A_\w \geq 1/\sqrt{2}$, and
\begin{align*}
\prod_{k=1}^{b-1} \frac{1}{\lvert G_{t_k}^{1}\rvert}
&\geq \prod_{k=1}^{b-1} \left(\frac{1}{\lambda t_k} \left( 1 - \frac{A_\w}{\lambda}\sqrt{\frac{\log s}{s}} \right) \right)\\
&\geq \left( \lambda^{-b+1} \prod_{k=1}^{b-1} \frac{1}{t_k} \right)
\left( 1 - \frac{(b-1)A_\w}{\lambda}\sqrt{\frac{\log s}{s}}
\right)\enspace.
\end{align*}
It yields using Bernoulli's inequality
\begin{align*}
\Prob_G(\mathcal{I}_{s, i, T} \mid \mathcal{C}_{s,N})
&\geq
\lambda^{-b+1} \frac{s^2}{i^2} \prod_{k=1}^{b-1} \frac{1}{t_k} \left( 1 -\left( \frac{3}{\lambda (1 - \lambda)} - \frac{(1+\sqrt{2})(b-1)}{\lambda} \right) A_\w \sqrt{\frac{\log s}{s}} \right)\enspace.
\end{align*}
Replacing $G^1$ by $G^2$ we obtain the same lower bound with $1-\lambda$ instead of $\lambda$, hence substituting into \ref{proofeq:pr(I)} gives
\begin{align*}
\Prob(\mathcal{I}_{s, i, T} \mid t^\star=i, \mathcal{C}_{s,N}) 
&\geq 
\frac{s^2}{i^2} \prod_{k=1}^{b-1} \frac{1}{t_k} \left( 
1 - \left( \frac{3}{\lambda} + (1-\lambda)\frac{(1+\sqrt{2})(b-1)}{\lambda} \right. \right.\\
& \hspace{2cm} \left. \left.+ \frac{3}{1 - \lambda} + \lambda\frac{(1+\sqrt{2})(b-1)}{1-\lambda}
\right)A_\w\sqrt{\frac{\log s}{s}}
\right)\\
&\geq \frac{s^2}{i^2} \prod_{k=1}^{b-1} \frac{1}{t_k} \left( 1 - \frac{(2b + 1)A_\w}{\lambda(1 - \lambda)} \sqrt{\frac{\log s}{s}} \right)\enspace,
\end{align*}
finally, since $\mathcal{C}_{s,N}$ is true with probability at least $2/s$ and given that $A_\w \geq 1/\sqrt{2}$ and $\lambda(1-\lambda) \leq 1/4$ we deduce that
\[
\Prob(\mathcal{C}_{s,N})
\geq 1 - \frac{2}{s} \geq 1 - \frac{A_\w}{\lambda(1-\lambda)}\sqrt{\frac{\log s}{s}}\enspace,
\]
and thus with Bernoulli's inequality
\[
\Prob(\mathcal{I}_{s, i, T} \mid t^\star=i)
\geq \frac{s^2}{i^2} \prod_{k=1}^{b-1} \frac{1}{t_k} \left( 1 - \frac{2(b + 1)A_\w}{\lambda(1 - \lambda)} \sqrt{\frac{\log s}{s}} \right)\enspace.
\]
the claim of the Lemma follows immediately from.
\[
p^b_N(\w)  = \frac{1}{N}\sum_{i=s}^N \sum_{s \leq t_1 < \ldots, t_{b-1} < i} \Prob(\mathcal{I}_{s, i, T} \mid t^\star=i)\enspace.
\]


\end{proof}

The following step is to estimate the sum appearing in the previous Lemma. We do that using computational results from Section \ref{sec:comp}.

\begin{lemma}\label{lem:pb(s,N)}
Assume that $s = \w N$. We have  for any $b\geq 2$,
\[
\frac{s^2}{N} \sum_{i=s}^N \frac{1}{i^2} \sum_{s \leq t_1 < \ldots < t_{b-1} < i} \prod_{\ell=1}^{b-1} \frac{1}{t_\ell}
\geq {p}^b_{\infty}( \w)
- \frac{\e}{s} \log(1/ \w)^{b-2},
\]
with $A_\w$ defined in Lemma \ref{lem:conc-card}, and  
\[
p^b_{\infty}( \w) = \w^2\left( \frac{1}{\w} - \sum_{\ell = 0}^{b-1} \frac{\log(1/ \w)^\ell}{\ell !} \right).
\]
\end{lemma}

\begin{proof}
Recall that $s = \w N$.
Lemma \ref{lem:pb-sum} gives that
\begin{equation}\label{eq:pb-Sb}
p^b_N(\w) 
\geq
\left( 1 - \frac{2(b+1)A_\w}{\lambda(1-\lambda)} \sqrt{\frac{\log s}{s}}\right)
\frac{s^2}{N} \sum_{i=r}^N \frac{1}{i^2} S_{b-1}(r,i),
\end{equation}

with 
\[
S_{b-1}(r,i) := \sum_{r \leq t_1 < \ldots < t_{b-1} < i} \prod_{\ell=1}^{b-1} \frac{1}{t_\ell},
\]
Lemma \ref{lem:Sm(s,N)} shows that 
\[
    S_{b-1}(r,i) \geq \frac{\log(i/s)^{b-1}}{(b-1)!} - \frac{e\log(1/ \w)^{b-2}}{s},
\]
therefore we have
\begin{align*}
\frac{s^2}{N} \sum_{i=r}^N \frac{1}{i^2} S_{b-1}(r,i)
&\geq \frac{s^2}{N} \sum_{i=r}^N \frac{1}{i^2} \left( \frac{\log(i/s)^{b-1}}{(b-1)!} 
     - \frac{e\log(1/ \w)^{b-2}}{s} \right)\\
&= \frac{s^2}{N(b-1)!} \sum_{i=r}^N \frac{\log(i/s)^{b-1}}{i^2}
   - e\log(1/ \w)^{b-2} \frac{s}{N} \sum_{i=r}^N \frac{1}{i^2}
\end{align*}
we immediately have
\begin{align*}
\frac{s}{N} \sum_{i=r}^N \frac{1}{i^2}
\leq \frac{s}{N} \int_{s}^N \frac{d}{x^2}
= \frac{s}{N} \left( \frac{1}{s} - \frac{1}{N} \right)
\leq \frac{1}{N},
\end{align*}
and using Lemmas \ref{lem:sum-int} then \ref{lem:Im} we have
\begin{align*}
\frac{s^2}{N(b-1)!} \sum_{i=r}^N \frac{\log(i/s)^{b-1}}{i^2}
&\geq \frac{s^2}{N(b-1)!}\left( \int_s^N  \frac{\log(x/s)^{b-1}}{x^2}dx
      - \frac{\log(1/ \w)^{b-1}}{s^2}\right)\\
&= \frac{s^2}{N} \left( \frac{1}{s} - \frac{1}{N} \sum_{\ell = 0}^{b-1}
    \frac{\log(1/ \w)^\ell}{\ell !} \right)
    - \frac{\log(1/ \w)^{b-1}}{N(b-1)!}\\
&= \left( \w - \w^2 \sum_{\ell = 0}^{b-1} \frac{\log(1/ \w)^\ell}{\ell !} \right)
    - \frac{\log(1/ \w)^{b-1}}{N(b-1)!}\\
&= \tilde{p}_b( \w) - \frac{\log(1/ \w)^{b-1}}{N(b-1)!}.
\end{align*}
We deduce that
\begin{align*}
\frac{s^2}{N} \sum_{i=r}^N \frac{1}{i^2} S_{b-1}(r,i)
&\geq \tilde{p}_b( \w)
    - \frac{\log(1/ \w)^{b-1}}{N(b-1)!}
    - \frac{e\log(1/ \w)^{b-2}}{N}\\
&\geq \tilde{p}_b( \w)
    - \frac{\e}{s} \log(1/ \w)^{b-2},
\end{align*}
where we used for the last inequality that $\frac{1}{(b-1)!} \leq e$, $\log(1/ \w) \leq 1/\w - 1$ and $\w N = s$.
\end{proof}


With the previous lemmas, we can finally prove Theorem \ref{thm:lb-PB(s,N)}. 

\begin{proof}[Proof of Theorem \ref{thm:lb-PB(s,N)}]
From Lemmas \ref{lem:p1(s,N)} and \ref{lem:pb(s,N)}, we have that for any $b \geq 1$
\begin{align*}
p^b_N(\w) 
\geq 
\left( p^b_{\infty}( \w) - \frac{\e \indic{b\geq2}}{s} \log(1/ \w)^{b-2} \right)
\left( 1 - \frac{2(b+1)A_\w}{\lambda(1-\lambda)} \sqrt{\frac{\log s}{s}}\right)\\
\geq 
\left( p^b_{\infty}( \w) - \frac{\e \indic{b\geq2}}{s} \log(1/ \w)^{b-2} \right)
\left( 1 - \frac{2(B+1)A_\w}{\lambda(1-\lambda)} \sqrt{\frac{\log s}{s}}\right)
\end{align*}
where 
\[
p^b_{\infty}( \w) = \w^2\left( \frac{1}{\w} - \sum_{\ell = 0}^{b-1} \frac{\log(1/ \w)^\ell}{\ell !} \right).
\]
Using Lemma \ref{lem:tau=rho}, and denoting $P^B_{\infty}( \w):= \sum_{b=1}^{B+1}p^b_{\infty}( \w)$, we deduce that
\begin{align*}
P^B_N(\w) 
&= \sum_{b=1}^{B+1} p_b(s,N)\\
&\geq 
\left(P^B_{\infty}( \w) - \frac{\e}{s}\sum_{b=0}^{B-1}\log(1/ \w)^b \right)
\left( 1 - \frac{2(B+1)A_\w}{\lambda(1-\lambda)} \sqrt{\frac{\log s}{s}}\right) \\
&\geq 
P^B_{\infty}( \w) - \frac{2(B+1)\tilde{P}_B( \w)A_\w}{\lambda(1-\lambda)}\sqrt{\frac{\log s}{s}} - \frac{\e}{s}\sum_{b=0}^{B-1}\log(1/ \w)^b.
\end{align*}
Lemma \ref{lem:sum-exp-remainder} with $x= \log(1/ \w)$ gives 
\begin{align*}
P^B_{\infty}( \w):=
&\w^2 \sum_{b=1}^{B+1} \left( \frac{1}{\w} - \sum_{\ell=1}^{b-1} \frac{\log(1/ \w)^\ell}{\ell !}   \right)\\
&= \w^2 \sum_{b=0}^B \left( \frac{1}{\w} - \sum_{\ell=0}^{b} \frac{\log(1/ \w)^\ell}{\ell !}   \right)\\
&\leq \w \log(1/ \w),    
\end{align*}

and thus we have
\begin{align*}
P^B_{\infty}( \w)A_\w 
&= P^B_{\infty}( \w)\frac{1 + \sqrt{\w}}{2\sqrt{2}\w^{\frac{1}{4}}}\\
&\leq \log(1/ \w) \w^{\frac{3}{4}} \frac{1 + \sqrt{\w}}{2\sqrt{2}}\\
&\leq \log(1/ \w) \sqrt{\frac{\w}{2}},
\end{align*}
It yields
\begin{align*}
P_N^B(\w) 
&\geq 
P^B_{\infty}( \w)
- \left( \frac{(B+1)\sqrt{2\w}\log(1/ \w)}{\lambda(1-\lambda)} + \e \sum_{b=0}^{B-1} \log(1/ \w)^b \right) \sqrt{\frac{\log s}{s}}\\
&\geq P^B_{\infty}( \w)
- \left( \frac{\sqrt{2}(B+1)\log(1/ \w)}{\lambda(1-\lambda)} + \frac{\e}{\sqrt{\w}} \sum_{b=0}^{B-1} \log(1/ \w)^b \right) \sqrt{\frac{\log N}{N}}\enspace,
\end{align*}
we used in the last line the inequality $\sqrt{\log s /s} \leq \sqrt{\log N / (\w N)}.$
\end{proof}







\section{Auxiliary Results}\label{sec:comp}
This section contains several general results that are used in other proofs. We begin by introducing some properties of the variables $(r_t)_t$ that were defined in Section \ref{sec:setup}.
\begin{definition}\label{def:inrank}
If $X$ is a totally ordered finite set, we define the inside-ranking function $\pi$ as
\[
\pi(X) := (r_1, \ldots, r_{|X|}),
\]
where $\{r_1, \ldots, r_{|X|}\} = \{1, \ldots, |X|\}$ and for any $1 \leq i \leq |X|$, $r_i$ is the rank of $x_i$ inside $X$.
\end{definition}


\begin{proposition}\label{prop:max2sets}
Let $I,J$ be two disjoint subsets of $\{1,\ldots,N\}$, and $\{x_1, \ldots, x_N\}$ a uniform random permutation of $\{1,\ldots,N\}$. If $X=\{x_i\}_{i\in I}$ and $Y = \{x_j\}_{j\in J}$ then the event $\{\max X < \max Y\}$ is independent of the inside rankings of $X$ and $Y$: $\pi(X)$ and $\pi(Y)$ (See Definition \ref{def:inrank}). Moreover 
\[
\Prob(\max X < \max Y)
= \frac{|Y|}{|X| + |Y|}.
\]
\end{proposition}

\begin{proof}
Since $(x_1,\ldots,x_N)$ is randomly permuted, every element in $X \cup Y$ has the same probability of being the maximum in $X \cup Y$, therefore
\begin{align*}
\Prob(\max X < \max Y)
&= \Prob(\max\{X\cup Y\} \in Y) 
= \sum_{y \in Y} \Prob(\max\{X\cup Y\} = y)
= \frac{|Y|}{|X| + |Y|}.
\end{align*}
\end{proof}


\begin{lemma}\label{lem:rank_indep_unif}
    For any positive integer $t$, conditionally to $G^1_t, G^2_t$, the relative ranks $(r_j)_{j=1}^t$, defined in Section \ref{sec:setup}, are mutually independent. Moreover,  the conditional distribution of the rank $r_t$ knowing $\lvert G_t^{g_t}\rvert$ is uniform over the set $\{1, \dots, \lvert G_t^{g_t}\rvert\}$.
\end{lemma}
\begin{proof}
Since the arrival order of the candidates is chosen uniformly at random, any subset of $(x_1, \ldots, x_N)$ is also uniformly shuffled. In particular, if $g_t = g$, this is the case for the subset $G^g_t$. The random variable $r_t$, indicating the rank of $x_t$ in $G^g_t$, is therefore a uniform random variable in $\{1, \ldots, |G^g_t|\}$. It is independent of the relative ranks of the past candidates in $G^g_{t-1}$ because all the possible relative ranks of $G^g_{t-1}$ are equally likely.
\end{proof}

Next, We remind Hoeffding's maximal inequality, from which we deduce a concentration bound on the random variables $|G^g_t|$ for $s \leq t \leq N$.
\begin{lemma}\label{lem:hoeff-max}
    Let $a$ and $b$ be two real numbers such that $a < b$. Let $(X_t)_{t \in \mathbb{N}}$ be a martingale difference sequence such that $a \leq X_t \leq b$ for any $t \in \mathbb{N}$. Let $r$ and $N$ be two integers such that $r \leq N$.
    Then, for any $\delta > 0$, 
    \[
      \Prob \left(
        %\exists r \leq t \leq N \; : \; 
        \max_{r \leq t \leq N}
        \frac{1}{t} \sum_{i=1}^t X_i \geq \sqrt{\frac{(b-a)^2}{2t}\log(1/\delta)\Phi(N/r)}  \;
        \right)
        \leq \delta\enspace,
    \] 
    where $\Phi : x>0 \mapsto \frac{(1 + \sqrt{x})^2}{4\sqrt{x}}$.

\end{lemma}

The following lemma is a corollary of the previous one.
\begin{lemma}\label{lem:conc-card}
    If $s = \w N$ with $\w \in (0,1)$, then the event
    \begin{equation}\label{eq:conc}\tag{$\mathcal{C}_{s,N}$}
    \mathcal{C}_{s,N} \coloneqq \left\{\forall s \leq t \leq N \; : \;
            \left| |G^1_t| - \lambda t \right|
            < A_\w \sqrt{t \log s}\right\}
    \end{equation}
    holds with probability at least $1 - \dfrac{2}{s}$, where $A_\w := \dfrac{1+\sqrt{\w}}{2\sqrt{2}\w^{1/4}}$.
\end{lemma}


\begin{proof}
We apply Hoeffding's maximal inequality, reminded in Lemma~\ref{lem:hoeff-max}, with $\delta = 1/s$, first for the random variables $(\indic{x_t \in G^1} - \lambda)_{r\leq t \leq N}$, then for the random variables $(\lambda - \indic{x_t \in G^1} )_{s \leq t \leq N}$. A union bound then gives
    \begin{align*}
        \Prob\left(
            \exists s \leq t \leq N \; : \;
            \left| |G^1_t| - \lambda t \right| \geq \left( \frac{\Phi(1/ \w)}{2}\right)^{\frac{1}{2}} \sqrt{t \log s}
        \right)
        \leq \frac{2}{s}\enspace,
    \end{align*} 
    where $\Phi : x>0 \mapsto \frac{(1 + \sqrt{x})^2}{4\sqrt{x}}$. An immediate computation yields $\sqrt{\dfrac{\Phi(1/ \w)}{2}} =  \dfrac{1+\sqrt{\w}}{2\sqrt{2}\w^{1/4}}$.
\end{proof}


\begin{comment}
\begin{lemma}\label{lem:sum-int}
    If $f:(0,\infty) \to \R$ and there exists $c > 0$ such that f is non-decreasing on $(0,c]$ and non-increasing on $[c,\infty)$ then for any integers $r,s$ such that $r + 2 \leq s$, if $f$ is non-negative on $[r,s]$ then we have
    \[
    \sum_{t=r+1}^{s-1} f(t) \geq \int_r^s f(x)dx - \max_{x \in [r,s]}f(x).
    \]
\end{lemma}
\begin{proof}
Let $f$ be such a function and $r < s$ two integers. If $s - 1\leq c$ then $f$ is non-decreasing on $[1,s-1]$ thus
\begin{align*}
\sum_{t=r+1}^{s-1} f(t)
&\geq \sum_{t=r+1}^{s-1} \int_{t-1}^t f(x)dx
= \int_r^{s-1} f(x)dx\\
&= \int_r^s f(x)dx - \int_{s-1}^s f(x)dx
\geq \int_r^s f(x)dx - \max_{x \in [r,s]}f(x), 
\end{align*}

In the other case, if $s - 1 > c$ then $s -1  \geq \lfloor c \rfloor + 1$, and we have
\begin{align*}
\sum_{t=r+1}^{s-1} f(t) 
&= \sum_{t=r+1}^{\lfloor c \rfloor} f(t) + \sum_{t=\lfloor c \rfloor + 1}^{s-1} f(t)\\
&\geq \sum_{t=r+1}^{\lfloor c \rfloor} \int_{t-1}^t f(x)dx
      + \sum_{t=\lfloor c \rfloor + 1}^{s-1} \int_t^{t+1} f(x)dx\\
&= \int_r^{s} f(x) dx - \int_{\lfloor c \rfloor}^{\lfloor c \rfloor+1} f(x)dx\\
&\geq \int_r^{s} f(x) dx - \max_{x \in [r,s]}f(x).
\end{align*}
\end{proof}


\begin{lemma}\label{lem:Sm(s,N)}
    For any integers $m \geq 1$ and $r,s$ such that $r + m \leq s$, we define
    \[
    S_m(r,s) := \sum_{r \leq t_1 < \ldots < t_{m} \leq s-1} \prod_{\ell=1}^{m} \frac{1}{t_\ell}.
    \]
    if $r,s$ are such that $r + m \leq s \leq \frac{r}{w}$ for some $w \in (0,1)$ then
    \[
    S_m(r,s) \geq \frac{\log(s/r)^m}{m!} - \frac{e\log(1/w)^{m-1}}{r}.
    \]
\end{lemma}

\begin{proof}
Observe that for $m \geq 2$ and $r,s$ such that $r+m \leq s$ we have
\begin{align*}
    S_m(r,s)
    &= \sum_{r \leq t_1 < \ldots < t_{m} \leq s-1} \prod_{\ell=1}^{m} \frac{1}{t_\ell}\\
    &= \sum_{t_m=r+m-1}^{s-1} \frac{1}{t_m} \left(
      \sum_{r \leq t_1 < \ldots < t_{m-1} \leq t_m-1} \prod_{\ell=1}^{m-1} \frac{1}{t_\ell} \right)\\
    &= \sum_{t=r+m-1}^{s-1} \frac{S_{m-1}(r,t)}{t}.
\end{align*}
Using this equation, we will prove by induction on $m$ that for $r,s$ such that $r+m \leq s \leq r/w$
\begin{equation}\label{eq:sm-induc}
    S_m(r,s) \geq \frac{\log(s/r)^m}{m!} - \frac{\alpha_m\log(1/w)^{m-1}}{r},
\qquad \text{where } \alpha_m := \sum_{\ell = 0}^{m-2} \frac{1}{\ell !},
\end{equation}


with the convention $\alpha_1 = 0$. For $m = 1$ we have for any $r,s$ such that $r +1 \leq s$
\[
S_1(r,s)
= \sum_{t=r}^{s-1} \frac{1}{t}
\geq \sum_{t=r}^{s-1} \int_t^{t+1} \frac{dx}{x}
= \int_r^s \frac{dx}{x}
= \log(s/r).
\]
Let $m \geq 2$ and assume that the lower bound (\ref{eq:sm-induc}) holds for $m-1$. We have for any $r,s$ such that $r+m \leq s \leq \frac{r}{w}$
\begin{align*}
S_m(r,s) 
&= \sum_{t=r+m-1}^{s-1} \frac{S_{m-1}(r,t)}{t}\\
&\geq \sum_{t=r+m-1}^{s-1} \left( \frac{\log(t/r)^{m-1}}{(m-1)!t} - \frac{\alpha_{m-1}\log(1/w)^{m-2}}{rt} \right)\\
& = \frac{1}{(m-1)!} \sum_{t=r+m-1}^{s-1}  \frac{1}{t}\log(t/r)^{m-1}
    - \frac{\alpha_{m-1}}{r}\log(1/w)^{m-2} \sum_{t=r+m-1}^{s-1} \frac{1}{t},
\end{align*}
the first sum can be lower bounded using Lemma \ref{lem:sum-int} with the function $t \to \frac{1}{t}\log(t/r)^{m-1}$:
\begin{align*}
    \sum_{t=r+m-1}^{s-1}  \frac{1}{t}\log(t/r)^{m-1}
    &= \sum_{t=r+1}^{s-1}  \frac{1}{t}\log(t/r)^{m-1} 
        - \sum_{t=r+1}^{r+m-2} \frac{1}{t}\log(t/r)^{m-1}\\
    &\geq \sum_{t=r+1}^{s-1}  \frac{1}{t}\log(t/r)^{m-1} 
        - (m-2)\max_{x\in [r,s]}\frac{1}{x}\log(x/r)^{m-1}\\
    &\geq \int_r^s \frac{1}{x}\log(x/r)^{m-1} dx 
        - (m-1) \max_{x\in [r,s]}\frac{1}{x}\log(x/r)^{m-1}
\end{align*}
we have directly that 
\[
\int_r^s \frac{1}{x}\log(x/r)^{m-1} dx
= \left[ \frac{\log(x/r)^m}{m}\right]_r^s 
=  \frac{\log(s/r)^m}{m},
\]
and since $s \leq r/w$ we have for any $x \in [r,s]$
\[
\frac{1}{x}\log(x/r)^{m-1} 
\leq \frac{1}{r}\log(s/r)^{m-1}
\leq \frac{1}{r}\log(1/w)^{m-1}
\]
therefore 
\[
\sum_{t=r+m-1}^{s-1}  \frac{1}{t}\log(t/r)^{m-1}
\geq \frac{\log(s/r)^m}{m} - \frac{m-1}{r}\log(1/w)^{m-1}.
\]

On the other hand, we have
\[
\sum_{t=r+m-1}^{s-1} \frac{1}{t}
\leq \int_r^s \frac{1}{t}
= \log(r/r)
\leq \log(1/w),
\]
we deduce that
\begin{align*}
S_m(r,s)
&\geq \frac{\log(s/r)^m}{m!} - \frac{1}{(m-2)!r}\log(1/w)^{m-1} 
   - \frac{\alpha_{m-1}}{r}\log(1/w)^{m-1}\\
&= \frac{\log(s/r)^m}{m!} - \frac{\alpha_m}{r}\log(1/w)^{m-1},
\end{align*}
which concludes the induction. Finally, given that $\alpha_m \leq e$, we deduce the lower bound stated in the lemma.
\end{proof}


\begin{lemma}\label{lem:Im}
If $r,N$ are two real numbers such that $r = wN$ for some $w \in (0,1)$ then for any integer $m \geq 0$ we have
\[
\frac{1}{m!}\int_r^N \frac{\log(x/r)^m}{x^2}dx
= \frac{1}{r} - \frac{1}{N} \sum_{\ell = 0}^m \frac{\log(1/w)^\ell}{\ell !}.
\]
\end{lemma}

\begin{proof}
Let us denote $I_m := \frac{1}{m!}\int_r^N \frac{\log(x/r)^m}{x^2}dx$. 
For $m = 0$ we have
\[
I_0 = \int_r^N \frac{dx}{x^2} = \frac{1}{r} - \frac{1}{N},
\]
and for any $m \geq 1$, by integration by parts we obtain
\begin{align*}
I_m 
&= \frac{1}{m!}\left(  \int_r^N m  \frac{\log(x/r)^{m-1}}{x^2}dx
    +< \left[ -\frac{\log(x/r)^{m}}{x} \right]_r^N \right)\\
&=  I_{m-1} - \frac{\log(1/w)^m}{m!N},
\end{align*}
thus 
\[
I_m = I_0 - \frac{1}{N} \sum_{\ell = 1}^m \frac{\log(1/w)^\ell}{\ell !}
= \frac{1}{r} - \frac{1}{N} \sum_{\ell = 0}^m \frac{\log(1/w)^\ell}{\ell !}.
\]
\end{proof}

\end{comment}


Finally, the next lemma gives an upper bound that is used in the proof of Proposition \ref{prop:lb-optDDT}.
\begin{lemma}\label{lem:sum-exp-remainder}
For any $x > 0$ and for any $B \geq 1$ we have
\[
0 \leq
xe^x - \sum_{b=0}^B\left( e^x - \sum_{\ell=0}^b \frac{x^\ell}{\ell !} \right)
\leq e^x\frac{x^{B+2}}{(B+1)!}.
\]
\end{lemma}

\begin{proof}
Let $x>0$ and $B \geq 1$. First, observe that
\[
\sum_{b=0}^\infty\left( e^x - \sum_{\ell=0}^b \frac{x^\ell}{\ell !} \right)
= \sum_{b=0}^\infty \sum_{\ell=b+1}^\infty \frac{x^\ell}{\ell !}
= \sum_{\ell = 1}^\infty \sum_{b=0}^{\ell-1}  \frac{x^\ell}{\ell !} 
= \sum_{\ell = 1}^\infty \frac{x^\ell}{(\ell - 1) !}
= x \sum_{\ell = 0}^\infty \frac{x^\ell}{\ell !}
= x e^x,
\]
therefore 
\[
\sum_{b=0}^B\left( e^x - \sum_{\ell=0}^b \frac{x^\ell}{\ell !} \right)
\leq x e^x,
\]
and we have
\begin{align*}
xe^x - \sum_{b=0}^B\left( e^x - \sum_{\ell=0}^b \frac{x^\ell}{\ell !} \right)
&= \sum_{b=B+1}^\infty \sum_{\ell=b+1}^\infty \frac{x^\ell}{\ell !}
\quad= \sum_{\ell=B+2}^\infty \sum_{b=B+1}^{\ell - 1} \frac{x^\ell}{\ell !}
\quad= \sum_{\ell=B+2}^\infty (\ell - B - 1) \frac{x^\ell}{\ell !}\\
&\leq \sum_{\ell=B+2}^\infty  \frac{x^\ell}{(\ell-1) !}
\quad\leq x \sum_{\ell=B+1}^\infty \frac{x^\ell}{\ell !}
\quad\leq e^x\frac{x^{B+2}}{(B+1)!},
\end{align*}
where we used for the last step the classical inequality 
\[
\sum_{\ell=B+1}^\infty \frac{x^\ell}{\ell !}
= e^x - \sum_{\ell=0}^B \frac{x^\ell}{\ell !}
\leq e^x\frac{x^{B+1}}{(B+1)!}.
\]
\end{proof}








\begin{lemma}\label{lem:sum-int}
    If $f:(0,\infty) \to \R$ and there exists $c > 0$ such that f is non-decreasing on $(0,c]$ and non-increasing on $[c,\infty)$ then for any integers $r,s$ such that $r + 2 \leq s$, if $f$ is non-negative on $[r,s]$ then we have
    \[
    \sum_{t=r+1}^{s-1} f(t) \geq \int_r^s f(x)dx - \max_{x \in [r,s]}f(x).
    \]
\end{lemma}
\begin{proof}
Let $f$ be such a function and $r < s$ two integers. If $s - 1\leq c$ then $f$ is non-decreasing on $[1,s-1]$ thus
\begin{align*}
\sum_{t=r+1}^{s-1} f(t)
&\geq \sum_{t=r+1}^{s-1} \int_{t-1}^t f(x)dx
= \int_r^{s-1} f(x)dx\\
&= \int_r^s f(x)dx - \int_{s-1}^s f(x)dx
\geq \int_r^s f(x)dx - \max_{x \in [r,s]}f(x), 
\end{align*}

In the other case, if $s - 1 > c$ then $s -1  \geq \lfloor c \rfloor + 1$, and we have
\begin{align*}
\sum_{t=r+1}^{s-1} f(t) 
&= \sum_{t=r+1}^{\lfloor c \rfloor} f(t) + \sum_{t=\lfloor c \rfloor + 1}^{s-1} f(t)\\
&\geq \sum_{t=r+1}^{\lfloor c \rfloor} \int_{t-1}^t f(x)dx
      + \sum_{t=\lfloor c \rfloor + 1}^{s-1} \int_t^{t+1} f(x)dx\\
&= \int_r^{s} f(x) dx - \int_{\lfloor c \rfloor}^{\lfloor c \rfloor+1} f(x)dx\\
&\geq \int_r^{s} f(x) dx - \max_{x \in [r,s]}f(x).
\end{align*}
\end{proof}


\begin{lemma}\label{lem:Sm(s,N)}
    For any integers $m \geq 1$ and $r,s$ such that $r + m \leq s$, we define
    \[
    S_m(r,s) := \sum_{r \leq t_1 < \ldots < t_{m} \leq s-1} \prod_{\ell=1}^{m} \frac{1}{t_\ell}.
    \]
    if $r,s$ are such that $r + m \leq s \leq \frac{r}{w}$ for some $w \in (0,1)$ then
    \[
    S_m(r,s) \geq \frac{\log(s/r)^m}{m!} - \frac{e\log(1/w)^{m-1}}{r}.
    \]
\end{lemma}

\begin{proof}
Observe that for $m \geq 2$ and $r,s$ such that $r+m \leq s$ we have
\begin{align*}
    S_m(r,s)
    &= \sum_{r \leq t_1 < \ldots < t_{m} \leq s-1} \prod_{\ell=1}^{m} \frac{1}{t_\ell}\\
    &= \sum_{t_m=r+m-1}^{s-1} \frac{1}{t_m} \left(
      \sum_{r \leq t_1 < \ldots < t_{m-1} \leq t_m-1} \prod_{\ell=1}^{m-1} \frac{1}{t_\ell} \right)\\
    &= \sum_{t=r+m-1}^{s-1} \frac{S_{m-1}(r,t)}{t}.
\end{align*}
Using this equation, we will prove by induction on $m$ that for $r,s$ such that $r+m \leq s \leq r/w$
\begin{equation}\label{eq:sm-induc}
    S_m(r,s) \geq \frac{\log(s/r)^m}{m!} - \frac{\alpha_m\log(1/w)^{m-1}}{r},
\qquad \text{where } \alpha_m := \sum_{\ell = 0}^{m-2} \frac{1}{\ell !},
\end{equation}


with the convention $\alpha_1 = 0$. For $m = 1$ we have for any $r,s$ such that $r +1 \leq s$
\[
S_1(r,s)
= \sum_{t=r}^{s-1} \frac{1}{t}
\geq \sum_{t=r}^{s-1} \int_t^{t+1} \frac{dx}{x}
= \int_r^s \frac{dx}{x}
= \log(s/r).
\]
Let $m \geq 2$ and assume that the lower bound (\ref{eq:sm-induc}) holds for $m-1$. We have for any $r,s$ such that $r+m \leq s \leq \frac{r}{w}$
\begin{align*}
S_m(r,s) 
&= \sum_{t=r+m-1}^{s-1} \frac{S_{m-1}(r,t)}{t}\\
&\geq \sum_{t=r+m-1}^{s-1} \left( \frac{\log(t/r)^{m-1}}{(m-1)!t} - \frac{\alpha_{m-1}\log(1/w)^{m-2}}{rt} \right)\\
& = \frac{1}{(m-1)!} \sum_{t=r+m-1}^{s-1}  \frac{1}{t}\log(t/r)^{m-1}
    - \frac{\alpha_{m-1}}{r}\log(1/w)^{m-2} \sum_{t=r+m-1}^{s-1} \frac{1}{t},
\end{align*}
the first sum can be lower bounded using Lemma \ref{lem:sum-int} with the function $t \to \frac{1}{t}\log(t/r)^{m-1}$:
\begin{align*}
    \sum_{t=r+m-1}^{s-1}  \frac{1}{t}\log(t/r)^{m-1}
    &= \sum_{t=r+1}^{s-1}  \frac{1}{t}\log(t/r)^{m-1} 
        - \sum_{t=r+1}^{r+m-2} \frac{1}{t}\log(t/r)^{m-1}\\
    &\geq \sum_{t=r+1}^{s-1}  \frac{1}{t}\log(t/r)^{m-1} 
        - (m-2)\max_{x\in [r,s]}\frac{1}{x}\log(x/r)^{m-1}\\
    &\geq \int_r^s \frac{1}{x}\log(x/r)^{m-1} dx 
        - (m-1) \max_{x\in [r,s]}\frac{1}{x}\log(x/r)^{m-1}
\end{align*}
we have directly that 
\[
\int_r^s \frac{1}{x}\log(x/r)^{m-1} dx
= \left[ \frac{\log(x/r)^m}{m}\right]_r^s 
=  \frac{\log(s/r)^m}{m},
\]
and since $s \leq r/w$ we have for any $x \in [r,s]$
\[
\frac{1}{x}\log(x/r)^{m-1} 
\leq \frac{1}{r}\log(s/r)^{m-1}
\leq \frac{1}{r}\log(1/w)^{m-1}
\]
therefore 
\[
\sum_{t=r+m-1}^{s-1}  \frac{1}{t}\log(t/r)^{m-1}
\geq \frac{\log(s/r)^m}{m} - \frac{m-1}{r}\log(1/w)^{m-1}.
\]

On the other hand, we have
\[
\sum_{t=r+m-1}^{s-1} \frac{1}{t}
\leq \int_r^s \frac{1}{t}
= \log(r/r)
\leq \log(1/w),
\]
we deduce that
\begin{align*}
S_m(r,s)
&\geq \frac{\log(s/r)^m}{m!} - \frac{1}{(m-2)!r}\log(1/w)^{m-1} 
   - \frac{\alpha_{m-1}}{r}\log(1/w)^{m-1}\\
&= \frac{\log(s/r)^m}{m!} - \frac{\alpha_m}{r}\log(1/w)^{m-1},
\end{align*}
which concludes the induction. Finally, given that $\alpha_m \leq e$, we deduce the lower bound stated in the lemma.
\end{proof}


\begin{lemma}\label{lem:Im}
If $r,N$ are two real numbers such that $r = wN$ for some $w \in (0,1)$ then for any integer $m \geq 0$ we have
\[
\frac{1}{m!}\int_r^N \frac{\log(x/r)^m}{x^2}dx
= \frac{1}{r} - \frac{1}{N} \sum_{\ell = 0}^m \frac{\log(1/w)^\ell}{\ell !}.
\]
\end{lemma}

\begin{proof}
Let us denote $I_m := \frac{1}{m!}\int_r^N \frac{\log(x/r)^m}{x^2}dx$. 
For $m = 0$ we have
\[
I_0 = \int_r^N \frac{dx}{x^2} = \frac{1}{r} - \frac{1}{N},
\]
and for any $m \geq 1$, by integration by parts we obtain
\begin{align*}
I_m 
&= \frac{1}{m!}\left(  \int_r^N m  \frac{\log(x/r)^{m-1}}{x^2}dx
    +< \left[ -\frac{\log(x/r)^{m}}{x} \right]_r^N \right)\\
&=  I_{m-1} - \frac{\log(1/w)^m}{m!N},
\end{align*}
thus 
\[
I_m = I_0 - \frac{1}{N} \sum_{\ell = 1}^m \frac{\log(1/w)^\ell}{\ell !}
= \frac{1}{r} - \frac{1}{N} \sum_{\ell = 0}^m \frac{\log(1/w)^\ell}{\ell !}.
\]
\end{proof}