\section{Dynamic threshold algorithm for $K$ groups}\label{sec:DT}
In this section, we introduce a general family of Dynamic-threshold ($\DT$) algorithms. 
A $\DT$ algorithm for the $(K,B)$-secretary problem is defined by a finite doubly-indexed sequence $(\w_{k,b})_{k \in [K],b \leq B}$ of real numbers in $[0,1]$, which determines the \textit{acceptance} thresholds based on the group of the observed candidate and the available budget. During a run of the algorithm, the thresholds used for each group dynamically change depending on the evolution of the available budget.
We denote this algorithm by $\A^B\big((\w_{k,b})_{k \in [K],b \leq B}\big)$.

Upon the arrival of a new candidate $x_t$, the algorithm observes its group $g_t = k \in [K]$ and its in-group rank $r_t$, and has an available budget of $B_t = b \geq 0$. If $t/N < \w_{k,b}$ or $r_t = 0$, the candidate is immediately rejected. Otherwise, if $t/N \geq \w_{k,b}$ and $r_t = 1$, then the candidate is selected if $b=0$; and if the budget is not yet exhausted ($b>0$), then the algorithm pays a unit cost to observe the variable $\indic{R_t = 1}$. If this variable is $1$, indicating a favorable comparison, the candidate is selected; otherwise, it is rejected.
A formal description is given in Algorithm \ref{algo:DT}, and a visual representation for the case of three groups is provided in Figure \ref{fig:DTviz}.






\begin{figure}
    \centering
    \includegraphics[width=0.8\textwidth]{figs/DTviz.pdf}
    \caption{Schematic description of a DT algorithm in the case of 3 groups}
    \label{fig:DTviz}
\end{figure}


\begin{algorithm}[h!]
\DontPrintSemicolon 
\caption{Dynamic-Threshold algorithm $\A^B\big((\w_{k,b})_{k \in [K],b \leq B}\big)$}\label{algo:DT}
\SetKwInput{Input}{Input}
   \SetKwInOut{Output}{Output}
   \SetKwInput{Initialization}{Initialization}
   \Input{Available budget $B$, thresholds $(\w_{k,b})_{k \in [K],b \leq B}$}
   \Initialization{$b = B$}
\For{$t=1,\ldots,N$}{
    Receive new observation: $(r_t, g_t)$\;
    \If(\tcp*[f]{\small compare in-group}){$t \geq \lfloor \w_{g_t,b} N \rfloor \emph{ and } r_t = 1$}{%\label{algoline:better-than-best}
        \If(\tcp*[f]{\small check budget}){$b > 0$}{
            % \vspace{-0.4cm}
            Update budget: $b \gets b - 1$\;
            \lIf(\tcp*[f]{\small compare inter-group}){$R_t = 1$}{
            Return: $t$
            }
        }
        \lElse{Return: $t$} 
    }
}
\end{algorithm}












\subsection{Single-threshold algorithm for $K$ groups}\label{sec:single-thresh}
In this section, we focus on the single-threshold algorithm, a specific case within the family of $\DT$ algorithms, where all thresholds are identical across groups and budgets. Initially, the algorithm rejects all candidates until step $T-1$, where $T \in [N]$ is a fixed threshold. Upon encountering a new candidate that is the best within its group, if no budget remains, the candidate is selected. Alternatively, if there is still a budget available, the algorithm utilizes it to determine if the current candidate is the best among all groups. If that is the case, the candidate is then selected. We denote by $\A^B_T$ the single-threshold algorithm with threshold $T$ and budget $B$.
%Although seemingly simplistic, 
We demonstrate that this algorithm has an asymptotic success probability converging very rapidly to the upper bound of $1/e$.






In this first lemma, we prove a recursion formula on the success probability of the single-threshold algorithm, with a threshold $T = \lfloor \w N \rfloor$ for some $\w \in [0,1]$.

\begin{lemma}\label{lem:single-thres-recursion}
The success probability of the single threshold algorithm $\A_T^B$ with threshold $T = \lfloor \w N \rfloor$ and budget $B \geq 0$ satisfies the recursion formula
\[
\Pr(\A^{B}_{T} \text{ succeeds})
= \frac{\w - \w^{K}}{K-1} + \indic{B\geq 0} (K-1) \sum_{t=T}^N \frac{T^K}{t^{K+1}}\Pr(\A^{B-1}_{t+1} \text{ succeeds}) + O\Big( \sqrt{\tfrac{\log N}{N}}\Big)\;.
\]
\end{lemma}


The proof hinges on analyzing the behavior of the algorithm following the first comparison. After that comparison, the algorithm halts if $R_t = 1$, and the success probability can be computed in that case. Otherwise, if $R_t \neq 1$, the candidate is rejected, and the algorithm transitions to a new state at step $t+1$, where the available budget reduces to $B-1$.
%Leveraging the memory-less property of the single-threshold algorithm, 
Its success probability becomes precisely that of algorithm $\A^{B-1}_{t+1}$, with budget $B-1$ and threshold $t+1$.






The recursion outlined in Lemma \ref{lem:single-thres-recursion} can be used to calculate the asymptotic success probability of the single-threshold algorithm $\A_{\lfloor \w N \rfloor}^B$ as the number of candidates $N$ approaches infinity. 
%The resultant expression is presented in the following theorem.


\begin{theorem}\label{thm:single-thresh}
The asymptotic success probability of the single threshold algorithm $\A^B_T$ with threshold $T = \lfloor \w N \rfloor$ and budget $B \geq 0$ is 
\[
\lim_{N \to \infty} \Pr(\A^B_{\lfloor \w N \rfloor} \text{ succeeds}) = \frac{\w^K}{K-1} \sum_{b = 0}^B \left( \frac{1}{\w^{K-1}} - \sum_{\ell = 0}^b \frac{\log(1/\w^{K-1})^\ell}{\ell !} \right)\;.
\]
In particular, 
\[
\lim_{B \to \infty} \lim_{N \to \infty} \Pr(\A^B_{\lfloor \w N \rfloor} \text{ succeeds}) = \w \log(1/\w)\;.
\]
\end{theorem}


Note that, for $B = \infty$, the asymptotic success probability in the previous theorem corresponds to the success probability of the algorithm with a threshold $\lfloor \w N \rfloor$ in the secretary problem. 
Indeed, with an unlimited budget, the decision-maker can assess at each step whether the current candidate is the best so far, and the problem becomes equivalent to the classical secretary problem.

\paragraph{Alternative comparison model.}
In the alternative comparison model presented in Section \ref{sec:setup}, the single threshold algorithm $\A^B_T$ can be adapted to guarantee the same success probability at the cost of $K-1$ additional comparisons. After the first $T$  candidates are rejected, $K-1$ comparisons are made between the maximums from each group to identify the best candidate so far. The algorithm then keeps track of the best candidate: whenever a new candidate is the best in their group, they are compared to the current best candidate using a single comparison, and the latter is updated accordingly. This approach enjoys the same guarantees as in Theorem \ref{thm:single-thresh}, but with a budget of $K + B - 1$ instead of $B$.



The next corollary measures how the success probability of the single threshold algorithm, in the setting with $K$ groups, converges to $1/e$ as the budget increases.




\begin{corollary}\label{cor:single-thresh-factorial-conv}
The success probability of the single-threshold algorithm with threshold $T = \lfloor N/e \rfloor$ and budget $B \geq 0$ satisfies
\[
\lim_{N \to \infty} \Pr(\A_{\lfloor N/e \rfloor}^B \text{ succeeds})
\geq \frac{1}{e}\left(1 - \frac{(K-1)^{B+1}}{(B+1)!} \right)\;.
\]
In particular, for all $\eps > 0$, if $K \leq 1 + \frac{B+1}{e}(e\eps)^{\frac{1}{B+1}}$, then $\lim_{N} \Pr(\A_{\lfloor N/e \rfloor}^B \text{ succeeds}) \geq (1-\eps)/e$.
\end{corollary}


This corollary proves that the success probability of $\A_{\lfloor N/e \rfloor}^B$ converges very rapidly to the upper bound $1/e$ as $B$ increases. However, the convergence becomes slower when $K$ is larger.





Surprisingly, the asymptotic success probability of $\A_{\lfloor N/e \rfloor}^B$ is not influenced by the proportions $(\lambda_k)_{k \in [K]}$, but only by the number of groups $K$. 
%For example, let us consider the scenario with only one group, reducing the problem to the classical secretary problem where the budget becomes insignificant. In this case, the asymptotic success probability of the single-threshold algorithm with a threshold $\lfloor \w N\rfloor$ is $\w \log(1/\w)$. However, when there are two groups with respective proportions $\lambda_1 = 1-\delta$ and $\lambda_2 = \delta$, with $\delta > 0$ arbitrarily small, the asymptotic success probability of the single-threshold algorithm becomes $\frac{\w(1-\w)}{2}$, identical to the scenario where $\lambda_1 = \lambda_2 = 1/2$.
This means that the algorithm does not benefit from the cases where there is a majority group $G^k$ with $\lambda_k$ very close to 1, which would make the problem easier. Indeed, it is always possible to achieve a success probability of $\max_{k \in [K]} \lambda_k /e$ by rejecting all the candidates not belonging to the majority group $G^{k^*}$, and using the classical $1/e$-rule counting only elements of $G^{k^*}$. This algorithm can be combined with ours by always running the one with the highest success probability, depending on the available budget, the number of groups, and the group proportions. The resulting algorithm has a success probability that converges to the upper bound $1/e$ both when $B$ increases and when $\max_k \lambda_k$ converges to $1$. Nonetheless, due to the very fast convergence of the success probability of the single threshold algorithm to $1/e$, the improvement brought by having a majority group is only marginal when the budget is sufficient. 

As a consequence, the single threshold algorithm surprisingly constitutes a very efficient solution to the problem even with moderate values of the budget. Computing the optimal thresholds remains however an intriguing question, which we explore in the following sections in the case of two groups. 




