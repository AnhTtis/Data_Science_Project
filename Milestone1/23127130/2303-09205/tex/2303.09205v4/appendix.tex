\newpage


\section{Preliminaries}
With the assumption that the group membership of each candidate is a random variable, to compute the asymptotic success probabilities of the algorithms presented in the paper, it is necessary to use concentration inequalities to estimate the number of candidates in each group. We use Lemma 3 from \cite{jamieson2014lil} to prove the following.

\begin{lemma}[\cite{jamieson2014lil}]
Let $(X_t)_{t \geq 1}$ be i.i.d. Bernoulli random variables, $N$ a positive integer, and $m > 0$ satisfying $N^m > 8$, then it holds with a probability of at least $1 - \frac{25}{N^{2m}}$ that 
\[
\forall t \geq 1: \sum_{s=1}^t (X_s-\E[X_s]) \leq 2 \sqrt{(m+1)t \log N}
\]
\end{lemma}

\begin{proof}
Bernoulli random variables are sub-Gaussian with scale parameter $1/2$, hence Lemma 3 from \cite{jamieson2014lil} with $\eps = 1$ guarantees that,  with a probability of at least $1 - 3(\frac{\delta}{\log 2})^2$, the following holds
\[
\forall t \geq 1: \sum_{s=1}^t (X_s-\E[X_s]) \leq 2 \sqrt{t \log\left( \tfrac{\log(2t)}{\delta} \right)}\;.
\]
Consider positive integers $T \leq N$, $m \geq 1$ such that $N^m > 8$, and $\delta = 2/N^m$. Then for all $t \in \{T,\ldots,N\}$
\[
\frac{\log(2t)}{\delta} \leq \frac{2t}{\delta} = N^m t \leq N^{m+1}
\]
and we deduce that
\begin{align*}
\Pr\Big( \forall t \geq 1: &\sum_{s=1}^t (X_s-\E[X_s]) \leq 2 \sqrt{(m+1)t \log N} \Big)\\
&\geq \Pr\left( \forall t \geq 1: \sum_{s=1}^t (X_s-\E[X_s]) \leq 2 \sqrt{t \log\left(\tfrac{\log(2t)}{2/N^m}\right)} \right)\\
&\geq 1 - \frac{3(2/\log 2)^2}{N^{2m}}\\
&\geq 1 - \frac{25}{N^{2m}}\;.
\end{align*}
\end{proof}




In the context of the $K$-group secretary problem, adequately using the previous Lemma and using union bounds yields a concentration inequality on the number of candidates belonging to each group.

\begin{lemma}\label{lem:conc-card}
Let $N \geq \max(4,K)$, and consider the following event 
\[
\forall k \in [K], \forall t \geq 1: ||G^k_t| - \lambda_k t| \leq 4 \sqrt{t \log N}\;,
\]
which we denote by $\C_{N}$. Then it holds that $\Pr( \C_{N}) \geq 1 - \frac{1}{N^2}$.
\end{lemma}


\begin{proof}
The previous Lemma with $N \geq 4$, $m = 3$ and respectively with the Bernoulli random variables $X_{t,k,0} = \indic{g_t = k}$ and $X_{t,k,1} = 1 - \indic{g_t = k}$, gives for all $i \in \{0,1\}$ and $k \in [K]$ that
\[
\Pr\left( \forall t \geq 1: (-1)^i(|G^k_t| - \lambda_k t) \leq 4 \sqrt{t \log N} \right)
\geq 1 - \frac{25}{N^{6}}\;,
\]
and we obtain by a union bound that $\Pr(\C_{N}) \geq 1 - \frac{50K}{N^6}$.
Given that $N \geq \max(4,K)$, it follows that 
\[
\Pr(\C_{N}) \geq 1 - \frac{K}{N}\cdot \frac{50}{N^3} \cdot \frac{1}{N^2} \geq  1 - \frac{1}{N^2}\;.
\]
\end{proof}



\begin{lemma}\label{lem:pr-cond-C_N}
Let $\mathcal{E}$ be any event, not necessarily independent of $\C_N$ (defined in Lemma \ref{lem:conc-card}), then 
\[
\Pr(\mathcal{E} \mid \C_N) = \Pr(\mathcal{E}) + O(1/N^2)\;.
\]
\end{lemma}

\begin{proof}
We have by Lemma \ref{lem:conc-card} that
\[
\Pr(\mathcal{E} \mid \C_N) 
= \frac{\Pr(\mathcal{E} \cap \C_N)}{\Pr(\C_N)}
\leq \frac{\Pr(\mathcal{E})}{1 - \frac{1}{N^2}}
\leq \left( 1 + \frac{2}{N^2} \right) \Pr(\mathcal{E})
\leq \Pr(\mathcal{E}) + \frac{2}{N^2}\;.
\]
Using the same inequality with the complementary event $\mathcal{E}^c$ of $\mathcal{E}$ gives
\[
\Pr(\mathcal{E} \mid \C_N) 
= 1 - \Pr(\mathcal{E}^c \mid \C_N)
\geq 1 - \left(\Pr(\mathcal{E}^c) + \frac{2}{N^2}\right)
= \Pr(\mathcal{E}) - \frac{2}{N^2}\;,
\]
which concludes the proof.
\end{proof}










\begin{lemma}\label{lem:sum-exp-remainder}
For any $x > 0$ and for any $B \geq 1$ we have
\[
0 \leq
xe^x - \sum_{b=0}^B\left( e^x - \sum_{\ell=0}^b \frac{x^\ell}{\ell !} \right)
\leq e^x\frac{x^{B+2}}{(B+1)!}.
\]
\end{lemma}

\begin{proof}
Let $x>0$ and $B \geq 1$. First, observe that
\[
\sum_{b=0}^\infty\left( e^x - \sum_{\ell=0}^b \frac{x^\ell}{\ell !} \right)
= \sum_{b=0}^\infty \sum_{\ell=b+1}^\infty \frac{x^\ell}{\ell !}
= \sum_{\ell = 1}^\infty \sum_{b=0}^{\ell-1}  \frac{x^\ell}{\ell !} 
= \sum_{\ell = 1}^\infty \frac{x^\ell}{(\ell - 1) !}
= x \sum_{\ell = 0}^\infty \frac{x^\ell}{\ell !}
= x e^x,
\]
therefore 
\[
\sum_{b=0}^B\left( e^x - \sum_{\ell=0}^b \frac{x^\ell}{\ell !} \right)
\leq x e^x,
\]
and we have
\begin{align*}
xe^x - \sum_{b=0}^B\left( e^x - \sum_{\ell=0}^b \frac{x^\ell}{\ell !} \right)
&= \sum_{b=B+1}^\infty \sum_{\ell=b+1}^\infty \frac{x^\ell}{\ell !}
\quad= \sum_{\ell=B+2}^\infty \sum_{b=B+1}^{\ell - 1} \frac{x^\ell}{\ell !}
\quad= \sum_{\ell=B+2}^\infty (\ell - B - 1) \frac{x^\ell}{\ell !}\\
&\leq \sum_{\ell=B+2}^\infty  \frac{x^\ell}{(\ell-1) !}
\quad\leq x \sum_{\ell=B+1}^\infty \frac{x^\ell}{\ell !}
\quad\leq e^x\frac{x^{B+2}}{(B+1)!},
\end{align*}
where we used for the last step the classical inequality 
\[
\sum_{\ell=B+1}^\infty \frac{x^\ell}{\ell !}
= e^x - \sum_{\ell=0}^B \frac{x^\ell}{\ell !}
\leq e^x\frac{x^{B+1}}{(B+1)!}.
\]
\end{proof}































