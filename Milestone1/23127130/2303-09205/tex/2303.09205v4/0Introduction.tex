\section{Introduction}
Online selection is among the most fundamental problems in decision-making under uncertainty, 
Multiple problems within this framework can be modeled as variants of the secretary problem \citep{dynkin1963optimum, chow1971great},
where the decision-maker has to identify the best candidate among a pool of totally ordered candidates, observed sequentially in a uniformly random order. When a new candidate is observed, the decision maker can either select them and halt the process or reject them irrevocably. 
%Unlike in prophet inequalities \citep{krengel1977semiamarts}, only the relative ranks of the candidates matter and not their values. 
The optimal strategy is well known and consists of skipping the first $1/e$ fraction of the candidates and then selecting the first candidate that is better than all previously observed ones. This strategy yields a probability $1/e$ of selecting the best candidate.
A large body of literature is dedicated to the secretary problem and its variants, we refer the interested reader to ~\citep{chow1971great, lindley-1961} for a historical overview of this theoretical problem.

In practice, as pointed out by several social studies, the selection processes often do not reflect the actual relative ranks of the candidates and might be biased with respect to some socioeconomic attributes \citep{salem2022don, Raghavan_Barocas_Kleinberg_Levy20}. To tackle this issue, several works have explored variants of the secretary problem with noisy or biased observations 
%of the ranks 
\citep{salem2019closing, freij2010partially}.
In particular, \citet{correa2021fairness}
studied the \textit{multi-color secretary problem}, where each candidate belongs to one of $K$ distinct groups, and only candidates of the same group can be compared. This corresponds for example to the case of graduate candidates from different universities, where the within-group orders are freely observable and can be trusted using a metric such as GPA, but inter-group order cannot be obtained by the same metric. This model, however, is too pessimistic, as it overlooks the possibility of obtaining inter-group orders at some cost, through testing and examination. Taking this into account, we study the multicolor secretary problem with a budget for comparisons, where comparing candidates from the same group is free, and comparing candidates from different groups has a fixed cost of $1$. We assume that the decision-maker is allowed at most $B$ comparisons. This budget $B$ represents the amount of time/money that the hiring organization is willing to invest to understand the candidate's ``true'' performance. 
As in the classical secretary problem, an algorithm is said to have \textit{succeeded} if the selected candidate is the best overall, otherwise, it has \textit{failed}. The objective is to design algorithms that maximize the probability of success.





\subsection{Contributions}
The paper studies an extension of the \textit{multi-color} secretary problem \citep{correa2021fairness}, where comparing candidates from different groups is possible at a cost. This makes the setting more realistic and paves the way for more practical applications, but also introduces new analytical challenges.

In Section \ref{sec:setup}, we describe a general class of Dynamic-Threshold (DT) algorithms, defined by distinct acceptance thresholds for each group, that can change over time depending on the available budget.  
First, we examine a particular case where all thresholds are equal, which can be viewed as an extension of the classical $1/e$-strategy. However, the analysis is intricate due to additional factors such as group memberships, comparison history, and the available budget. By carefully controlling these parameters, we compute the asymptotic success probability of the algorithm, demonstrating its extremely rapid convergence to the upper bound of $1/e$ when the budget increases, hence constituting a first efficient solution to the problem.

Subsequently, our focus shifts to the case of two groups, where we explore another particular case of DT algorithms: static double threshold algorithms. These involve different acceptance thresholds for each group, that depend on the groups' proportions and the initial budget, but do not vary during the execution of the algorithm. We prove a recursive formula for computing the resulting success probability, and we exploit it to establish a closed-form lower bound and compute explicit thresholds.

In the two-group scenario, we also derive the optimal algorithm among those that do not utilize the history of comparisons, which we refer to as \textit{memory-less}. We present an efficient implementation of this algorithm and demonstrate, via numerical simulations, that it belongs to the class of DT algorithms when the number of candidates is large. Leveraging this insight, we numerically compute optimal thresholds for the two-group case. 
%Finally, we propose a threshold selection rule in the $K$-group scenario, which we numerically prove to be efficient.

%\subsection{Main techniques}
%The algorithms we present and analyze belong to the Dynamic-Threshold ($\DT$) algorithm family, described in Section \ref{sec:DT}. These algorithms operate based solely on the current observation at each step, independently of the past candidates' relative orders or group memberships, or the outcomes of previous comparisons. This property ensures that the success probability of these algorithms after any step is entirely determined by the available budget, the number of previously observed candidates in each group, and the group containing the best candidate observed so far. These parameters define the \textit{state} of a $\DT$ algorithm.

%To estimate the asymptotic success probability of $\DT$ algorithms, it is crucial to track all the aforementioned parameters during algorithm execution. By using adequate concentration inequalities, we gauge the number of candidates in each group at any given step. Furthermore, to track the available budget and the group containing the best candidate, we derive a recursive formula for the success probability of the algorithm starting from any state. A key variable in this analysis is the time of the first comparison made by the algorithm, denoted as $\rho_1$. We rigorously explore the possible values of $\rho_1$ and their associated probabilities. Subsequently, we examine the algorithm's potential state transitions following this comparison, resulting in the wanted recursion. The following step is to solve this recursion, which we successfully do for the single-threshold algorithm in the case of $K$-groups. However, the task becomes more challenging if multiple thresholds are considered, as we demonstrate through the double threshold algorithm for two groups.





\subsection{Related work}
%The secretary problem was originally introduced by 
\paragraph{The secretary problem} The secretary problem was introduced by \citet{dynkin1963optimum}, who proposed the $1/e$-threshold algorithm, having a success probability of $1/e$, which is the best possible. Since then, the problem has undergone extensive study and found numerous applications, including in finance \citep{hlynka1988secretary}, mechanism design \citep{kleinberg2005multiple}, Nested Rollout Policy Adaptation (NRPA) \citep{dang2023warm},  active learning \citep{fujii2016budgeted}, and the design of interactive algorithms \citep{sabato2016interactive, sabato2018interactive}. Moreover, the secretary problem has multiple variants \citep{karlin2015competitive, bei2022secretary,assadi2019secretary, keller2015better}, and has inspired other works, for instance, related to matching \citep{goyal2022secretary, dickerson2019balancing} or ranking \citep{jiang2021online, elactive}. A closely related problem is the prophet inequality \citep{krengel1977semiamarts, samuel1984comparison}, where the decision-maker sequentially observes values sampled from known distributions, and its reward is the value of the selected item, in opposite the secretary problem where the reward is binary: $1$ if the selected value is the maximum and $0$ otherwise. 
Prophet inequalities also have many applications \citep{kleinberg2012matroid, chawla2010multi, feldman2014combinatorial} and have been explored in multiple variants \citep{kennedy1987prophet, azar2018prophet, bubna2023prophet, benomar2024lookback}.

\paragraph{Different information settings}
In some practical scenarios, the secretary problem may present a pessimistic model. Therefore, variants with additional information have been studied. For example, \citet{Gilbert2006} explored a scenario where candidates' values are independently drawn from a known distribution. Other studies have examined potential improvements with other types of information, such as samples \citep{correa2021secretary} or machine-learned advice \citep{antoniadis2020secretary, advice-2021, benomar2023advice}.
Conversely, some works more closely aligned with ours have investigated more constrained settings. Notably, \citet{correa2021fairness} introduced the multi-color secretary problem, where totally ordered candidates belong to different groups, and only the partial order within each group, consistent with the total order, can be accessed. Under fairness constraints, they designed an asymptotically optimal strategy for selecting the best candidate. Other settings with only partial information have been studied as well. For example, ~\cite{monahan1980optimal, monahan1982state} addressed the optimal stopping of a target process when only a related process is observed, and they designed mechanisms for acquiring information from the target process. However, these works do not assume a fixed budget and instead consider a penalized version of the problem.


\paragraph{Online algorithms with limited advice}
This paper also relates to other works on online algorithms, where the decision-maker is allowed to query a limited number of hints during execution. The objective of these analyses is to measure how the performance improves with the number of permitted hints. Such settings have been studied, for example, in online linear optimization \citep{BhaskaraCKP21}, caching, \citep{im2022parsimonious}, paging \citep{antoniadis2023paging}, scheduling \citep{benomarnon}, metrical task systems \citep{sadek2024algorithms}, clustering \citep{silwal2023kwikbucks}, and sorting \citep{bai2024sorting, benomar2024learning}. Another related paper by \cite{drygala2023online} studies a penalized version of the Bahncard problem with costly hints.






