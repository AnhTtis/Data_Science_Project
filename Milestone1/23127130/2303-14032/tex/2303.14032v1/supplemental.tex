% ****** Start of file supplemental.tex ******
\pdfoutput=1

\documentclass[%
 reprint,
 superscriptaddress,
%groupedaddress,
%unsortedaddress,
%runinaddress,
%frontmatterverbose, 
%preprint,
%preprintnumbers,
%nofootinbib,
%nobibnotes,
%bibnotes,
 amsmath,amssymb,
%aps,
 prl,
%pra,
%prb,
%rmp,
%prstab,
%prstper,
%floatfix,
]{revtex4-2}

\usepackage{float}
\usepackage{graphicx}% Include figure files
\usepackage{caption}
\usepackage{subcaption}
\usepackage{dcolumn}% Align table columns on decimal point
\usepackage{bm}% bold math
\usepackage{hyperref}% add hypertext capabilities
\usepackage{multirow}

%Author commands
\usepackage{siunitx}
\renewcommand{\theequation}{SM--\arabic{equation}}



\begin{document}

\title{Supplemental Material for  ``Stability of the Modulator in a Plasma-Modulated Plasma Accelerator''}% Force line breaks with \\

\author{J. J. van de Wetering}
\email{johannes.vandewetering@physics.ox.ac.uk}
\affiliation{John Adams Institute for Accelerator Science and Department of Physics, University of Oxford, Denys Wilkinson Building, Keble Road, Oxford OX1 3RH, United Kingdom}%
\author{S. M. Hooker}%
\affiliation{John Adams Institute for Accelerator Science and Department of Physics, University of Oxford, Denys Wilkinson Building, Keble Road, Oxford OX1 3RH, United Kingdom}%
\author{R. Walczak}%
\affiliation{John Adams Institute for Accelerator Science and Department of Physics, University of Oxford, Denys Wilkinson Building, Keble Road, Oxford OX1 3RH, United Kingdom}%
\affiliation{Somerville College, Woodstock Road, Oxford OX2 6HD, United  Kingdom}%

\date{\today}% It is always \today, today,
             %  but any date may be explicitly specified

\maketitle

\section{Paraxial Description of Pulses in Plasma Channels}

\subsection{The Envelope Model}

Considering only the high frequency fields associated with the laser pulse, from Amp\`{e}re's law we find that the normalized vector potential $\bm{a}= e\bm{A}/m_ec$ evolves according to the wave equation
\begin{align}\label{eq:ampere}
    \left[\frac{\partial^2}{\partial t^2}-c^2\Delta+\frac{\omega_p^2}{n_0\gamma}\left(n_0+\delta n\right)\right]\bm{a} &= 0
\end{align}
where $\gamma = (1+\bm{p}^2+\bm{a}^2)^{1/2}$, $\bm{p}=\gamma m_e\bm{v}_\text{hf}$ and $\bm{v}_\text{hf}$ represents the high frequency quiver velocity of electrons which ignores the low frequency bulk fluid velocity from a wakefield response, $n_0$ is the unperturbed pre-formed plasma channel density and $\delta n$ represents the change in density due to the wake driven by the laser pulse. We have also assumed that the electrostatic response is small compared to the transverse current. To construct an envelope model for the laser evolution, consider a pulse of the form
\begin{align}
    \bm{a} = \hat{e}_La(r,\theta,z,t)\exp[i(k_Lz-\omega_Lt)]
\end{align}
where the laser frequency and wavenumber in free space $\omega_L, k_L$ are constants and $\hat{e}_L$ is the polarization of the laser, which we will treat as linearly polarized throughout this paper. It is convenient to shift from the lab frame axial coordinate $z$ and time $t$ to co-moving coordinates $\xi=z-v_gt$, and $\tau=t$, where $v_g/c = (1-\omega_{p0}^2/\omega_L^2)^{1/2}$ is defined as the group velocity of electromagnetic plane waves in uniform plasma at the on-axis plasma channel density $n_{00}=n_0(r=0)$, $\omega_{p0}=\omega_p(r=0)$, which may differ from the group velocity of a tightly focused laser pulse \cite{SpotVG}. Substituting a pulse of this form into Eq. (\ref{eq:ampere}) in the weakly relativistic and linear wake limit yields the following expression for the evolution of the envelope $a(r,\theta,\xi,\tau)$
\begin{align}\label{eq:full_envelope}
    &\Big[-2i\omega_L\frac{\partial }{\partial\tau}-2v_g\frac{\partial^2}{\partial\xi\partial\tau}-c^2\Delta_\perp + \frac{\partial^2}{\partial\tau^2} \nonumber \\ 
    &-(c^2-v_g^2)\frac{\partial^2}{\partial\xi^2} + \frac{\omega_p^2}{n_0}\left(\delta n_0+\delta n-n_0|a|^2/4\right)\Big]a = 0
\end{align}
where $\Delta_\perp\equiv(1/r)(\partial/\partial r)(r\partial/\partial r)+(1/r^2)\partial^2/\partial\theta^2$ and $\delta n_0(r)=n_0(r)-n_{00}$. This PDE, coupled to a self-consistent wake solution for $\delta n$, fully describes the evolution of the laser envelope in the axisymmetric, weakly relativistic, quasi-static linear wakefield regime. 

\subsection{The Paraxial Approximation}

To further simplify Eq. (\ref{eq:full_envelope}) to the paraxial approximation, we assume that the discrepancy between the group velocity and speed of light in vacuum is negligible ($v_g\approx c$) and that the pulse envelope has a large enough longitudinal extent that the term $\partial^2/\partial\xi^2$ can be neglected. We also assume that the envelope evolution is slow relative to the carrier frequency so that the terms $v_g\partial^2/\partial\xi\partial\tau$ and $\partial^2/\partial\tau^2$ time derivative terms can be dropped. This yields a generalized nonlinear paraxial wave equation in an axisymmetric plasma channel $n_0(r)=n_{00}+\delta n_0(r)$
\begin{align}\label{eq:GNLS}
    &\left[\frac{i}{\omega_L}\frac{\partial}{\partial\tau} + \frac{c^2}{2\omega_L^2}\Delta_\perp\right] a = \nonumber \\ &\frac{\omega_p^2}{2\omega_L^2n_0}\left[\delta n_0(r) + \delta n(r,\xi;|a|^2) - n_0(r)|a|^2/4\right]a
\end{align}
where the nonlinearities come from relativistic effects and the interaction between the pulse and its own excited wake.

\subsection{Matched Plasma Channels}

Using Eq. (\ref{eq:GNLS}) we can derive the form of the matched plasma channel, which confines a Gaussian beam with a constant spot size as it propagates. Assuming no contributions from relativistic self-focusing nor from plasma wakes, this yields the linear paraxial wave equation in a channel 
\begin{align}
    \left[\frac{i}{\omega_L}\frac{\partial}{\partial\tau} + \frac{c^2}{2\omega_L^2}\Delta_\perp\right] a &= \frac{\omega_p^2\delta n_0(r)}{2\omega_L^2n_0}a\,.
\end{align}
Substituting a Gaussian pulse with a fixed spot size $w_0$ yields the following stable solution
\begin{align}\label{eq:gausschannel}
    &a(r,\xi,\tau) = a_0f(\xi)\exp\left(-\frac{r^2}{w_0^2}-i\omega_L\tau\frac{2c^2}{\omega_L^2w_0^2}\right)\,,\nonumber \\
    &n_0(r) = n_{00} + \Delta n(r/w_0)^2\,,\,\,\,\, \Delta n = (\pi r_ew_0^2)^{-1}
\end{align}
where the longitudinal profile of the pulse, $0\leq f(\xi)\leq1$, is assumed to be sufficiently slowly varying for the paraxial approximation to hold, $n_{00}$ is an arbitrary on-axis density and $r_e$ is the classical electron radius. This result for the matched channel can also be shown to confine all Laguerre-Gaussian modes with envelope solutions of the form
\begin{align}\label{eq:LGchannel}
    &a_{pm}(r,\theta,\xi,\tau) = \nonumber \\ &\alpha_{pm}(\xi)\exp\left(-i\omega_L\tau(2p+|m|+1)\frac{2c^2}{\omega_L^2w_0^2}\right)\text{LG}_{pm}\,, \nonumber \\
    &\text{LG}_{pm}(r,\theta) = \nonumber \\ &\sqrt{\frac{p!}{(p+|m|)!}}\left(\frac{\sqrt{2}\,r}{w_0}\right)^{|m|}\exp\left(-\frac{r^2}{w_0^2}+im\theta\right)L^{|m|}_p\left(\frac{2r^2}{w_0^2}\right)\,, \nonumber \\
    &\langle\text{LG}_{p'm'}|(\ldots)|\text{LG}_{pm}\rangle \equiv \nonumber \\
    &\frac{2}{\pi w_0^2}\int_0^{2\pi}d\theta\int_0^\infty rdr \text{LG}_{p'm'}^*(\ldots)\text{LG}_{pm}
\end{align}
where the integers $p\geq0$ and $m$ are the radial and azimuthal indexes respectively and the $L^{|m|}_p$ functions are the generalized Laguerre polynomials. Note that it is the interference between the fundamental and first azimuthal and radial modes that set the laser centroid and spot size oscillation frequencies $\omega_c=\omega_w/2=2c^2/\omega_Lw_0^2$ respectively.

\section{2D Slab vs 3D Cylindrical Geometry}

To justify the use of 2D PIC simulations to study the stability of the plasma modulator, we outline the pulse propagation theory in 2D slab geometry. The physical description of seeded spectral modulation in 2D slab and 3D cylindrical are similar, but they have some key differences which come from the transverse Laplacian operator $\Delta_\perp$ taking a different form. The matched parabolic channel for a Gaussian beam with spot size $w_0$ still takes the same form $n_0(x)=n_{00}+\Delta n(x/w_0)^2$, but the confined modes are instead described by Hermite-Gaussian functions
\begin{align}\label{eq:HGchannel}
    &a_l(x,\xi,\tau) = \alpha_l(\xi)\exp\left(-i\omega_L\tau(l+1)\frac{2c^2}{\omega_L^2w_0^2}\right)\text{HG}_l\,, \nonumber \\
    &\text{HG}_l(x) = \sqrt{\frac{1}{2^l l!}}\exp\left(-\frac{x^2}{w_0^2}\right)H_l\left(\frac{\sqrt{2}\,x}{w_0}\right)\,, \nonumber \\
    &\langle\text{HG}_{l'}|(...)|\text{HG}_l\rangle \equiv \sqrt{\frac{2}{\pi w_0^2}}\int_{-\infty}^\infty dx \text{HG}_{l'}^*(...)\text{HG}_l
\end{align}
where integer $l\geq 0$ is the transverse index and the $H_l$ functions are the (physicist's) Hermite polynomials. Note that the laser centroid and spot size oscillation frequencies are the same in both 2D slab and 3D cylindrical geometry, so we would expect processes tied to spot/centroid oscillations to behave similarly in 2D and 3D. However, the rate of spectral modulation does depend on the dimensionality. For example, consider the shallow channel limit $\Delta n\ll n_{00}$ where the seed wake can be written in the form
\begin{align}
    \delta n = \delta n_s\cos(k_{p0}\xi)
    \begin{cases}
        1 & \text{1D}\\
        e^{-2x^2/w_0^2} & \text{2D}\\
        e^{-2r^2/w_0^2} & \text{3D}
    \end{cases} 
\end{align}
where $\delta n_s$ denotes the on-axis seed wake amplitude. Assuming that the modulating drive pulse remains in the fundamental channel mode, the spectral modulation rate parameter $\Omega_s$ is given by
\begin{align}
    \Omega_s = \frac{1}{4}\frac{\omega_{p0}^2}{\omega_L}\frac{\delta n_s}{n_{00}}
    \begin{cases}
         1 & \text{1D}\\
        \langle\text{HG}_0|e^{-2x^2/w_0^2}|\text{HG}_0\rangle = 1/\sqrt{2} & \text{2D}\\
        \langle\text{LG}_{00}|e^{-2r^2/w_0^2}|\text{LG}_{00}\rangle = 1/2 & \text{3D}
    \end{cases} 
\end{align}
hence 2D PIC simulations are expected to spectrally modulate $\sim\sqrt{2}$ times faster than predicted by 3D cylindrical theory (and $\sim\sqrt{2}$ times slower than 1D theory).

The suppression of spectral modulation by wake phase-front curvature caused by the plasma channel, as described in the paper, also changes depending on the dimensionality. This is again primarily caused by the difference between $\langle\text{HG}_0|\delta n(x,\xi)|\text{HG}_0\rangle$ and $\langle\text{LG}_{00}|\delta n(r,\xi)|\text{LG}_{00}\rangle$, resulting in the suppression towards the pulse tail being more pronounced in the 3D cylindrical case. There is also a smaller effect due to differences in the wake structure itself between 2D and 3D.

\section{Spectral Phase for Pulse Train Formation}

As derived previously in 1D by Jakobsson et al \cite{PhysRevLett.127.184801}, to first order the spectral modulation takes the form of a set of sidebands of angular frequencies $\omega_m = \omega_L + m \omega_{p0}$, where $m = \pm1, \pm2, \pm3, \ldots$ is the sideband order. These sidebands will each have a relative spectral phase $\psi_m = -|m|\pi/2$. With a more detailed analysis, which can be derived from Eq. \eqref{eq:TDPT_nonparaxial_2level} assuming a narrow bandwidth $\tau_\text{drive}^{-1}\ll \omega_{p0}$ and that light only remains in the fundamental channel mode, the full relative spectral phase of the modulated drive pulse before compression into a pulse train can be approximately described by the following nearest integer staircase function
\begin{align}
    \psi(\omega) = -\left|\text{nint}\left(\frac{\omega-\omega_L}{\omega_{p0}}\right)\right|\frac{\pi}{2}\,.
\end{align}
To form a pulse train, we wish to remove this spectral phase from the pulse. However, it is not practical to remove a spectral phase of this form with a dispersive optic. Instead, we can take advantage of the narrow bandwidth of each of the sidebands to approximately remove this spectral phase by applying a continuous dispersion function of the form
\begin{align}
    \psi^\text{opt}(\omega) = +\left|\frac{\omega-\omega_L}{\omega_{p0}}\right|\frac{\pi}{2}.
\end{align}
This form of $\psi^\text{opt}(\omega)$ was used in the main paper to evaluate the pulse trains that can be formed from the spectrally modulated drive pulses.

\section{Non-Paraxial Description of Seeded Spectral Modulation in Plasma Channels}

Unlike the paraxial equation, this description will include group velocity dispersion as well as asymmetries between the dynamics of the generated Stokes and anti-Stokes sidebands in the plasma modulator. We will start by following the procedure outlined by \cite{SpotVG} for deriving the non-paraxial description of pulses in fully ionized plasma, but here we will include contributions from a parabolic plasma channel and a seed wake. We begin with the full 3D wave equation for a laser pulse propagating in a fully ionized plasma
\begin{align}
    \left(\Delta-\frac{1}{c^2}\frac{\partial^2}{\partial t^2}-k_p^2\right)\bm{a} = 0
\end{align}
where $\bm{a}=e\bm{A}/m_ec$ is the normalized vector potential and we have neglected higher order plasma source terms from having a finite electrostatic potential \cite{SpotVGplasma}. We now switch to new coordinates $\xi=z-ct$, $\eta=(z+ct)/2$, which yields
\begin{align}
    \left(2\frac{\partial^2}{\partial\xi\partial\eta}+\Delta_\perp-k_p^2\right)\bm{a} = 0\,.
\end{align}
We seek envelope solutions in the form $\bm{a}=\left[ a\exp(ik_L\xi)+\text{c.c.}\right]\hat{e}_L/2$ where $k_L$ is a constant. This yields the envelope PDE
\begin{align}
    \left[2\left(ik_L+\frac{\partial}{\partial\xi}\right)\frac{\partial}{\partial\eta}+\Delta_\perp-k_p^2\right] a(r,\xi,\eta) = 0\,.
\end{align}
We then take the Fourier transform in the variable $\xi$ and apply the convolution theorem
\begin{align}
    &\left[2i\left(k_L+k\right)\frac{\partial}{\partial\eta}+\Delta_\perp\right] a_k = \left(k_p^2\right)_k\ast a_k\,, \nonumber \\
    & a_k(r,k,\eta) = \frac{1}{\sqrt{2\pi}}\int_{-\infty}^\infty d\xi e^{-ik\xi} a(r,\xi,\eta)\,,\nonumber \\
    &(f\ast g)(k) := \frac{1}{\sqrt{2\pi}}\int_{-\infty}^\infty f(k')g(k-k')dk'
\end{align}
where we have included a $1/\sqrt{2\pi}$ normalization in the definition of the convolution for notational convenience. Ignoring relativistic effects, we can split $k_p^2(r,\xi,\eta)$ into contributions from a pre-formed axisymmetric plasma channel $n_0(r)$ and a plasma wake $\delta n(r,\xi,\eta)$
\begin{align}
    &\left[i(1+k/k_L)\frac{1}{k_L}\frac{\partial}{\partial\eta}+\frac{1}{2k_L^2}\Delta_\perp-\frac{2}{k_L^2w_0^2}\frac{n_0(r)}{\Delta n}\right] a_k = \nonumber \\
    &\frac{2}{k_L^2w_0^2}\left(\frac{\delta n(r,\xi,\eta)}{\Delta n}\right)_k\ast a_k
\end{align}
where $\Delta n \equiv (\pi r_e w_0^2)^{-1}$ is the channel depth parameter. Ignoring the wake for now, a matched parabolic channel $n_0(r)=n_{00}+\Delta n(r/w_0)^2$ can guide any linear combination of Laguerre-Gaussian modes of the form
\begin{align}
    & a_k^{pm}(r,\theta,k,\eta) =   \alpha_k^{pm}(k)\exp[-i\tilde{k}_k^{pm}(k)\eta]\text{LG}_{pm}(r,\theta)\,, \nonumber \\
    &\tilde{k}_k^{pm}(k) = \frac{2}{k_Lw_0^2}\frac{2p+|m|+1+n_{00}/\Delta n}{1+k/k_L}\,, \nonumber \\
    &\text{LG}_{pm}(r,\theta) = \nonumber \\
    &\sqrt{\frac{p!}{(p+|m|)!}}\left(\frac{\sqrt{2}\,r}{w_0}\right)^{|m|}\exp\left(-\frac{r^2}{w_0^2}+im\theta\right)L^{|m|}_p\left(\frac{2r^2}{w_0^2}\right)\,.    
\end{align}
Assuming that $k/k_L\ll 1$ remains valid for the majority of the pulse $a_k$, (i.e. for pulse durations that are not too short relative to the laser cycle period), we can also write the Laguerre-Gaussian mode solutions in real space to first order in the form
\begin{align}
     a^{pm}(r,\theta,\xi,\eta) &=   \alpha^{pm}(\xi)\ast\left(\exp[-i\tilde{k}_k^{pm}(k)\eta]\right)_\xi\text{LG}_{pm}(r,\theta) \nonumber \\
    &\approx   \alpha^{pm}(\xi+\tilde{k}_0^{pm}\eta/k_L)\exp(-i\tilde{k}_0^{pm}\eta)\text{LG}_{pm}(r,\theta)\,, \nonumber \\
    \tilde{k}_0^{pm}/k_L &= \frac{\omega_{p0}^2}{2k_L^2c^2}+(2p+|m|+1)\frac{2}{k_L^2w_0^2}
\end{align}
where $\alpha^{pm}(\xi)$ is the inverse Fourier transform of $\alpha^{pm}_k(k)$. This describes the first order group velocity dispersion and wavenumbers of Laguerre-Gaussian modes due to the on-axis plasma density and finite spot size effects. We can see that for a pulse primarily in the fundamental mode, after every spot size oscillation the first radial mode will fall behind the fundamental mode by a laser wavelength (and similarly for centroid oscillations). Hence we eventually need to take this group velocity dispersion into account if the pulse propagates over many spot size oscillations in a long plasma channel.

If we can treat the wake contribution as a small perturbation to the matched parabolic plasma channel, we can use time-dependent perturbation theory to calculate the transitions between the channel modes with the following expression
\begin{widetext}
\begin{align}\label{eq:TDPT}
    &i(1+k/k_L)\frac{\partial  \alpha_k^{pm}(k,\eta)}{\partial\eta}\exp(-i\tilde{k}_k^{pm}\eta) = k_c\sum_{p'm'}\Big\langle\text{LG}_{pm}\Big|\left(\frac{\delta n(r,\xi,\eta)}{\Delta n}\right)_k\ast\left(  \alpha_k^{p'm'}(k,\eta)\exp(-i\tilde{k}_k^{p'm'}\eta)\right)\Big|\text{LG}_{p'm'}\Big\rangle
\end{align}
\end{widetext}
where $k_c=k_w/2=2/k_Lw_0^2$ are the centroid and spot size oscillation wavenumbers respectively.

Assume that the short seed pulse has the same wavelength as the drive pulse, has many laser cycles in its duration, is in the fundamental mode and does not appreciably deplete. The seed pulse will then have a group velocity of
\begin{align}
    v_{g,s}/c = 1 - \tilde{k}_0^{00}/k_L = 1 - \frac{\omega_{p0}^2}{2k_L^2c^2}-\frac{2}{k_L^2w_0^2}\,.
\end{align}
Note that the group velocity is slowed by both plasma and finite spot size effects. This means that in general the seed wake will be in the form
\begin{align}
    \delta n(r,\xi,\eta) = \delta n(r,\xi+\tilde{k}_0^{00}\eta/k_L)\,.
\end{align}
We now choose to work in the shallow channel limit $\Delta n \ll n_{00}$ to ignore the non-separable transverse wake structure introduced by the channel \cite{Andreev96}. Note that using a square-like channel would achieve a similar effect, but would have Bessel modes rather than Laguerre-Gaussian modes. The wake excited by the seed pulse considering both plasma and finite spot size effects on its group velocity in this limit takes the form
\begin{align}
    \delta n(r,\xi,\eta) = \delta n_s\text{LG}_{00}^2(r)\cos\left[k_{p0}\left(\xi+\tilde{k}_0^{00}\eta/k_L\right)\right]\,.
\end{align}
Substituting this seed wake into Eq. \eqref{eq:TDPT} yields the non-paraxial plasma modulator equation
\begin{widetext}
\begin{align}\label{eq:TDPT_nonparaxial}
    &i\frac{1+k/k_L}{k_L}\frac{\partial  \alpha_k^{pm}}{\partial\eta} = \frac{1}{4}\frac{\omega_{p0}^2}{k_L^2c^2}\frac{\delta n_s}{n_{00}}\sum_{p'm'}  \alpha^{p'm'}_{k\pm k_{p0}}\exp\left[-i\left(\tilde{k}_{k\pm k_{p0}}^{p'm'}-\tilde{k}_k^{pm}\pm(k_{p0}/k_L)\tilde{k}_0^{00}\right)\eta\right]\langle\text{LG}_{pm}|\text{LG}_{00}^2|\text{LG}_{p'm'}\rangle
\end{align}
where we can now clearly see that the seed wake modulating a pulse with an initially short bandwidth will generate Stokes and anti-Stokes sidebands in $k$-space separated by integer multiples of $k_{p0}$. Assuming that $k,k_{p0}\ll k_L$, the wavenumber shifts approximate to
\begin{align}
    &\left(\tilde{k}_{k\pm k_{p0}}^{p'm'}-\tilde{k}_k^{pm}\pm(k_{p0}/k_L)\tilde{k}_0^{00}\right)/k_c \approx 
    \left[2(p'-p)+(|m'|-|m|)\right](1-k/k_L) \mp \left(2p'+|m'|\right)(k_{p0}/k_L)\,.
\end{align}
Using this expression and assuming that most of the light remains in the fundamental mode and that no azimuthal modes are present, we can approximate the plasma modulator equation as a two-level system of the fundamental and first radial modes
\begin{align}\label{eq:TDPT_nonparaxial_2level}
    &i\frac{1+k/k_L}{k_L}\frac{\partial  \alpha_k^{00}}{\partial\eta} = \frac{1}{8}\frac{\omega_{p0}^2}{k_L^2c^2}\frac{\delta n_s}{n_{00}}\left(  \alpha^{00}_{k\pm k_{p0}}
    + \tfrac{1}{2}  \alpha^{10}_{k\pm k_{p0}}\exp\left[-ik_w\left(1-k/k_L\mp k_{p0}/k_L\right)\eta\right]\right)\,, \nonumber \\
    &i\frac{1+k/k_L}{k_L}\frac{\partial  \alpha_k^{10}}{\partial\eta} = \frac{1}{8}\frac{\omega_{p0}^2}{k_L^2c^2}\frac{\delta n_s}{n_{00}}\left(\tfrac{1}{2}  \alpha^{00}_{k\pm k_{p0}}\exp\left[ik_w(1-k/k_L)\eta\right]
    + \tfrac{1}{2}  \alpha^{10}_{k\pm k_{p0}}\exp\left[\pm ik_w\left(k_{p0}/k_L\right)\eta\right]\right)
\end{align}
\end{widetext}
which to zeroth order in $k/k_L$ and $k_{p0}/k_L$ gives the same result given by the paraxial equation used in the paper (apart from the slightly different propagation variable $\eta$). However, to first order we see symmetry-breaking between the Stokes and anti-Stokes sidebands which was not captured by the paraxial equation. We see here that the Stokes sidebands are generated faster by the seed wake and also undergo faster spot size oscillations and transverse mode transitions than the anti-Stokes. This asymmetry in the dynamics between the Stokes and anti-Stokes sidebands is necessary to explain the transverse separation of Stokes and anti-Stokes light observed in PIC simulations in the regime where the self-wake of the modulating drive pulse is no longer negligible.

\section{Particle-in-Cell Simulations}

Two-dimensional simulations were performed with the PIC code WarpX (version 22.07) \cite{WarpX}. Results from eight simulations are included in the paper with laser-plasma parameters and respective figures outlined in Table \ref{simParams}. All of these eight simulations were performed at an on-axis density of $n_{00}=\SI{2.5e17}{cm^{-3}}$ in a modulator of length $L_\text{mod}=\SI{110}{mm}$ with seed and drive pulses with the same wavelength $\lambda_L=\SI{1030}{nm}$ and spot size $w_0=$ 30 or 50 $\SI{}{\micro m}$. The seed and drive pulses were polarized out of and in the plane of simulation respectively. All simulations had a longitudinal resolution of $\Delta z = \lambda_L/50$ and transverse resolution of $\Delta x = \lambda_L/2.5$ using a second order Yee field solver and perfectly matched layer (PML) transverse boundary conditions. The transverse window size for all simulations was at least $2.67w_0$ away from the axis. The ``square'' and ``parabolic'' plasma channels were parameterized in the following form
\begin{align}
    &\frac{n_0^\text{square}(r)-n_{00}}{\Delta n} =  \nonumber \\
    &\begin{cases} 
        (r/w_0)^{10} & 0 \leq r < 1.2w_0 \\ 
        1.2^{10} & 1.2w_0 \leq r < 1.2w_0 + d \\ 
        1.2^{10}\left(1-\frac{r-1.2w_0-d}{d}\right) & 1.2w_0 + d \leq r < 1.2w_0 + 2d \\
        0 & r \geq 1.2w_0 + 2d
    \end{cases} \nonumber \\
    &\frac{n_0^\text{parabolic}(r)-n_{00}}{\Delta n} =  \nonumber \\
    &\begin{cases} 
        (r/w_0)^{2} & 0 \leq r < 2w_0 \\ 
        4 & 2w_0 \leq r < 2w_0 + d \\ 
        4\left(1-\frac{r-2w_0-d}{d}\right) & 2w_0 + d \leq r < 2w_0 + 2 d \\
        0 & r \geq 2 w_0 + 2 d
    \end{cases} \nonumber \\
\end{align}
where $d=\SI{10}{\micro m}$. Note that all simulations were initialized with a laser pulse with a gaussian transverse profile of spot size $w_0$, which differs slightly from the fundamental guiding mode of the square channel.


\begin{widetext}
\begin{center}
\begin{table}[!h]
\begin{tabular}{ |p{1.8cm}||p{1.6cm}|p{1.4cm}|p{1.8cm}|p{1.8cm}|p{1.6cm}|p{1.6cm}|p{1.8cm}|  }
 \hline
 \multicolumn{8}{|c|}{PIC Simulation Parameters} \\
 \hline
 Simulation & Figures & $w_0$ ($\SI{}{\micro m}$) & $W_\text{seed}$ (mJ) & $W_\text{drive}$ (J) & $\tau_\text{seed}$ (fs) & $\tau_\text{drive}$ (ps) & Channel\\
 \hline
 $(i)$ & 2, 3 & 30 & 50 & 0.6 & 40 & 1.0 & parabolic \\
 $(ii)$ & 3, 4a & 30 & 50 & 0.6 & 40 & 1.0 & square \\
 $(iii)$ & 3 & 50 & 139 & 1.67 & 40 & 1.0 & parabolic \\
 $(iv)$ & 4b, 5b & 30 & 50 & 1.2 & 40 & 1.0 & square \\
 $(v)$ & 4c & 30 & 50 & 2.4 & 40 & 4.0 & square \\
 $(vi)$ & 5a & 30 & 50 & 1.2 & 40 & 0.25 & square \\
 $(vii)$ & 5c & 30 & 50 & 1.2 & 40 & 2.0 & square \\
 $(viii)$ & 5d & 30 & 50 & 1.2 & 40 & 4.0 & square \\
 \hline
\end{tabular}
\caption{\label{simParams}}
\end{table}
\end{center}
\end{widetext}

\acknowledgements

This research was funded in whole, or in part, by EPSRC and STFC, which are Plan S funders. For the purpose of Open Access, the author has applied a CC BY public copyright licence to any Author Accepted Manuscript version arising from this submission.

\bibliography{references_supp}% Produces the bibliography via BibTeX.

\end{document}
%
% ****** End of file supplemental.tex ******
