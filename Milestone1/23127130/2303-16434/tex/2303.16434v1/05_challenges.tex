\section{Challenges} \label{sec:exp}
Although we have conducted preliminary validation of tasks in several scenarios, there are still many challenges to fully realizing TaskMatrix.AI's vision.

\textbf{Multimodal Conversational Foundation Model} To handle various tasks, TaskMatrix.AI needs a powerful foundation model that can work with different kinds of inputs (such as text, image, video, audio, code, etc.), learn from context, use common sense to reason and plan, and generate high-quality codes based on APIs to complete tasks. ChatGPT can only deal with text and code and is not able to handle tasks that involve physical and multimodal inputs. GPT-4~\citep{openai2023gpt4} is a better choice as it can process multimodal inputs (i.e., text and image). However, TaskMatrix.AI also needs to handle other modalities that may be returned by different APIs besides text, code and image. It is still challenging to figure out the minimum set of modalities that TaskMatrix.AI requires and train an MCFM for it.

\textbf{API Platform} Creating and maintaining a platform that hosts millions of APIs requires solving several challenges, such as: 1) API documentation generation: In the early stage, most APIs lack proper documentations that are friendly for MCFM to understand and invoke. 2) API quality assurance: The quality and dependability of APIs can differ greatly. It is crucial to ensure that the APIs in the platform meet the necessary quality criteria and are trustworthy for the success of TaskMatrix.AI. 3) API creation suggestion: Based on the user feedback, TaskMatrix.AI can recognize the shortcomings of existing APIs and know which kinds of tasks they cannot handle. Based on these, the API platform should provide further guidance for API developers to create new APIs to address such tasks.

\textbf{API Calling} Leveraging millions of APIs to accomplish user instructions raise new challenges that go beyond free text generation:  1) API selection: When there are many APIs available, recommending related APIs to MCFM for solving a specific task is vital. It requires TaskMatrix.AI to have a strong ability to plan reasonable solutions that can link user intentions with suitable APIs based on their documentation and previous usage history. 2) Online planning: For complex tasks, TaskMatrix.AI may not be able to come up with a solution right away. Instead, MCFM should interact with users and try different possible solutions to figure out the most suitable one. 

\textbf{Security and Privacy} When the API can access the physical world, and digital world to make real changes, we need to make sure the model: 1) Faithful to user instructions. We need to verify that the model accomplishes user instructions and doesn't do anything more than the user's intent. 2) Keep data private. The data transmission should be secure and data access should be authorized when integrating with various APIs from different domains that require access to sensitive data.

\textbf{Personalization} TaskMatrix.AI requires a personalization strategy to assist individual developers in building their own personalized AI interfaces for their products, as well as to help users have their own personal assistants. Personalization faces two challenges: 1) Reducing scaling cost. Since the learning needs to apply to numerous developers and users, it is not feasible to fine-tune one model for each scenario. 2) alignment with the user with few-shot examples. As users may only provide a few demonstrations or feedback and the model needs to efficiently learn their preferences. One promising direction is to generate a preference-aware solution outline.
