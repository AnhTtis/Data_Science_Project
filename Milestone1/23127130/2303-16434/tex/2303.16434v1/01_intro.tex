\section{Introduction} 

\begin{center} ``\textit{The amount of intelligence in the universe doubles every 18 months.'' ~-- ~Sam Altman, OpenAI CEO} \end{center}

Foundation models have made remarkable progress in this decade, from understanding models (e.g., BERT~\citep{devlin2018bert}, ViT~\citep{DBLP:conf/iclr/DosovitskiyB0WZ21vit}, Whisper~\citep{radford2022whisper}) that can process and comprehend data of different modalities, to generative models (e.g., GPT-4~\citep{openai2023gpt4}, GPT-3~\citep{brown2020gpt3}, Codex~\citep{chen2021codex}, DALL·E~\citep{ramesh2021dalle}) that can produce various kinds of outputs to interact with the world. ChatGPT is so impressive that many people think it is a sign of Artificial General Intelligence (AGI) coming soon. However, foundation models still face limitations and challenges in doing some specialized tasks, such as performing accurate mathematical calculations or completing a multi-step task in the real world that requires both textual and visual processing skills. Meanwhile, there are existing models and systems (based on mathematical theories, symbolic rules, or neural networks) that can perform very well on some domain-specific tasks. But due to the different implementation or working mechanisms, they are not readily available or compatible with foundation models. Therefore, there is an urgent and obvious need for a mechanism that can link the foundation models with the off-the-shelf models and systems with special functionalities to finish diversified tasks in both the digital and physical worlds.

Motivated by this, we present our vision of building a new AI ecosystem, named as \textbf{TaskMatrix.AI}, for linking foundation models with millions of existing models and system APIs to finish diversified tasks. Different from any single AI model, TaskMatrix.AI can be seen as a super-AI with abilities to execute both digital and physical tasks, which has the following key advantages:

\begin{itemize}
\item{\textbf{TaskMatrix.AI can perform both digital and physical tasks} by using the foundation model as a core system to understand different types of inputs (such as text, image, video, audio, and code) first and then generate codes that can call APIs for task completion.}
\item{\textbf{TaskMatrix.AI has an API platform as a repository of various task experts}. All the APIs on this platform have a consistent documentation format that makes them easy for the foundation model to use and for developers to add new ones.} 
\item{\textbf{TaskMatrix.AI has a powerful lifelong learning ability}, as it can expand its skills to deal with new tasks by adding new APIs with specific functions to the API platform.} 
\item{\textbf{TaskMatrix.AI has better interpretability for its responses}, as both the task-solving logic (i.e., action codes) and the outcomes of the APIs are understandable.}

\end{itemize}

Our vision is to build an ecosystem that can leverage both foundation models and other models and systems that are good at specific tasks and can be accessed as APIs. By connecting the foundation model with APIs, TaskMatrix.AI can smoothly integrate neural and symbolic systems, accomplish digital and physical tasks, and provide strong lifelong learning and reliable capabilities.