\newsavebox\lstboxnoorder
\begin{lrbox}{\lstboxnoorder}\begin{minipage}{0.98\linewidth}
MCFM: Sure, here's the code to generate slides for each company:
\begin{pythongray}
create_slide();
insert_text("Microsoft");
select_title();
insert_text("Founder: Bill Gates, Location: Redmond, Washington, Mission: To empower every person and every organization on the planet to achieve more, Products: Windows, Office, Xbox, Subsidiaries: LinkedIn, Skype");
select_content();
create_slide();
insert_text("Apple");
select_title();
insert_text("Founder: Steve Jobs and Steve Wozniak, Location: Cupertino, California, Mission: To bring the best user experience to its customers through its innovative hardware, software, and services, Products: iPhone, iPad, Mac, Apple Watch, Subsidiaries: Beats Electronics, Shazam");
select_content();
...
\end{pythongray}
\end{minipage}\end{lrbox} 

% \noindent\fbox{%
%     \parbox{\textwidth}{%


\begin{tcolorbox}
\begin{tcolorbox}[enhanced,colframe=black!30,title=API Documentations,attach boxed title to top left={xshift=0.5cm,yshift=-\tcboxedtitleheight/2},grow to right by=1mm] 
As an assistant for generating and editing slides, you have access to a list of APIs to control PowerPoint with the following functions:\\
\pyth{create_slide()}: This API is used to create a new slide.\\
\pyth{insert_text(text:str)}: This API is used to insert text into a text box. \colorbox{mylightgreen}{\st{The content of each}} \colorbox{mylightgreen}{\st{slide should contain several sentences, you can call insert\_text multiple times, or split the}} \colorbox{mylightgreen}{\st{multiple sentences by '\textbackslash n'.}}\\
\pyth{select_title()}: This API is used to select the text box of the title. \colorbox{mylightgreen}{\st{You should first select}} \colorbox{mylightgreen}{\st{the text box of the title and then insert or delete the text in the text box of the title.}}\\
\pyth{select_content()}: This API is used to select the text box of the content. \colorbox{mylightgreen}{\st{You should first}} \colorbox{mylightgreen}{\st{select the text box of the content and then insert or delete the text in the text box of the content.}}\\
\pyth{move_to_slide(slide_id:int)}: This API is used to move to a specific slide in the presentation. It can take one parameter, \pyth{slide_id}: the ID of the slide to move to as an integer. \\
...
\end{tcolorbox}
...
\vspace{0.2cm}

\fcolorbox{white}{gray!15}{\usebox\lstboxnoorder}
\end{tcolorbox}

\begin{center}
\captionof{figure}{When composition instructions are removed, MCFM may generate APIs in an incorrect order, such as inserting text before selecting the target text box. Additionally, MCFM may generate a long paragraph for each slide without proper line-breaking.}
% \vspace{0.5cm}
\label{fig:ppt_case3}
\end{center}