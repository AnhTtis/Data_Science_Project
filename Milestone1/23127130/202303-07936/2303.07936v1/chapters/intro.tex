\vspace{0.2cm}
\section{Introduction}
At intersections (e.g. Y, T and X junctions as well as roundabouts) and highway mergings (entering plus leaving ramps and overtaking), the driving task is simplified with traffic codes for prioritization \cite{trafficcode2000}. For cars, there are right-of-ways, stop lines or traffic lights. Pedestrians and bicycles utilize crosswalks and have general priority over cars. Even in simpler longitudinal scenarios, traffic participants should follow the direction of travel, keep on one side for multi-track streets and obey speed limits. These regulatory risks not only define who goes first, but also constrain the agents in their choice of actions and thereby make driving safer. Intentions become transparent and accessable for other agents so that they can be considered for motion planning.

Previous work models rules with state machines \cite{buehler2009}, ordering suitable maneuvers (keep distance, drive inside, etc.) based on the current kinematics of vehicle pairs. When entities do not comply to the norms, fallback plans avoid possible deadlock situations. Particularly for crossroads, the system should leave way or come to a stop when an obstacle takes precedence. 
This works reliably for normal driving, in critical conditions however it may generate reactive solutions that solely center on safety, effectively neglecting efficiency for the traffic flow. To robustly deal with a variable interplay of cars, it is therefore better to control behaviors dynamically using prediction-evaluation cycles that directly incorporate priority in the behavior finding procedure. %we present a priority velocity, efficiency and comfort 
%optimization

Our Risk Optimization Method (ROPT) presented in \cite{puphal2018} is an uncertainty-aware velocity planner which balances future integral risk from collisions and road structure with comfort and utility of the travel. In this paper, we expand its functionality to handle regulatory risk in vehicle-to-vehicle interactions (i.e., front, back, right and left geometric relations). Depending on the arising priority, ROPT first alters assumed trajectories as well as discounts awareness horizons for the respective other cars. %As an example, obstacles for which the ego car has precedence are considered only short times into the future and extrapolated to decelerate. 
%Segment heights and lags of multiple velocity snakes are then fine-tuned.
Next, fine-tuning slope and lag from segment-wise linear velocity profiles results into smooth ego car responses. We show in large-scale simulations that ROPT hereby successfully approaches, crosses and leaves uncontrolled intersections with varying behaviors (including cases where priority is violated) and taken paths of encountered traffic participants. %\ref{sec:outl} (including cases where priority compliance is violated), several

Section \ref{sec:rel} summarizes longitudinal, lateral and cooperative planning techniques in state-of-the-art. The description of a general multi-agent optimization framework is given in Section \ref{sec:fram} with emphasis on risk and comfort modeling. %into trajectory generation with Section \ref{sec:trajgen} and cost evaluation with Section \ref{sec:trajeval}. 
Sections \ref{sec:orderas} and \ref{sec:predprio} continue with priority assignment and prediction under regulatory risks. Finally, the analytical plus statistical experiments and evaluations are outlined in Section \ref{sec:exp} and Section \ref{sec:outl} presents our conclusion and prospects for future research. %for dynamic following as well as crossroads
%We show in large-scale simulations that ROPT thus finds 

\subsection{Related Work}
\label{sec:rel}
Vehicle control along same or parallel lanes is well established in the automotive industry. Here, the focus especially shifts from collision-free to likewise beneficial plans. Exemplarily in platooning \cite{geiger2012}, the minimized cost functional comprises the distance to all front vehicles for stable following. Traffic light assists \cite{treiber2014} create fuel-saving traverses during experienced phase switches (green, orange and red). When taking curves, \cite{liebner2013} apply a proactive deceleration with driver models and the course of lane changes are interpolated using Bezier curves in \cite{qian2016}. 

For crossing lanes with varying angles, possible driver intentions and ways of interaction become extensive. %\cite{galceran2015}.  as interactions  and their combinations
As a result on an intersection, right-of-way matrices \cite{erdmann2011} typically fill each lane relation with passing orders  %Finding the path conflicts and traversing distance to collision points, allows to define right-of-way matrices \cite{erdmann2011}. 
and safe maneuvers are then coordinated between the actors via if-then transition of defined driving states \cite{kammel2008}. While doing so, a fuzzy system \cite{lee1999} could dynamically alter single entries of the priorities, e.g. on behalf of emergency vehicles. In contrast to conventional heuristics, \cite{gregoire2014} also employ priority graphs to construct continous trajectories with safe gaps and \cite{plessen2016} iterate through priority schemes for realising orders even when each vehicle has to yield to another vehicle. 

\begin{figure}[t!]
      \centering      
      \resizebox{\linewidth}{!}{
      \import{img/}{overview.pdf_tex}}
      \caption{Concept of predictive velocity optimization. Assessing situation costs follows adapting multiple ego velocity profiles in repetitive cycles.} %block diagram
      \label{fig:simulator}
\end{figure}

Alongside lateral planning, recent research involves cooperative planning which considers optimized plans of other cars to locate global solutions. In \cite{frese2011}, priority-based approaches were evaluated as most efficient, but they cannot handle all scenarios. As a comparison, Monte Carlo tree search \cite{wolf2018} is used for lane merging with semantic vehicle-lane relationships and \cite{sezer2015} tested Markov decision processes under mixed observability for unsignalized intersections. Both methods lead to sensible ordering behaviors for specific complex scenarios which are implicitly influenced from the learned policy. %Merge or crossing orders are implictly given, because the global costs are otherwise elevated.
%solely

