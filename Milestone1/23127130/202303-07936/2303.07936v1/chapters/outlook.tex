\vspace{0.4cm}
\section{Conclusion and Outlook}
\label{sec:outl}
In this work, we presented an optimization framework to predictively plan dynamic velocity curves under right-of-ways. ROPT considers risk (expected damage caused by collision and curvature), utility (distance travelled plus deviation to desired speed) and comfort (strength or frequency of behavior change) in one scalar cost function. The chosen parametric snake profile is composed of multiple ramps with an initial variable lag and smoothed transition points. Minimal and maximal values from engine and brake characteristics thereby maintain in ROPT realistic driving constraints.  %driving

%present
After geometrically determining the trajectory relationship between vehicle pairs, ROPT discounts the corresponding collision risk if other cars are longitudinally inferior (i.e., to the back). In lateral traffic situations, other participants are simultaneously assumed to prototypically decelerate when inferior (e.g. on the left) or accelerate when superior (e.g. on the right). %toafter a short-term constant velocity phase  
In this context, left-before-right settings can be easily applied in the same way. Furthermore, the fixed predictions are enhanced for ROPT by adhering to possible velocities in curves and to permitted road limits.
%, ROPT is hence superior to obstacles to the right and inferior inferiority applies to left and superiority to right relations.

Analytical experiments demonstrated that our method is able to effectively follow priority rules when necessary. While ROPT admits lower distances to following than leading cars for push reduction from tailgaiting, it drives in more instances before other vehicles when having precedence at intersections. Subsequently with path randomizations, we also proved that if the encountered obstacle is inattentive, ROPT avoids an accident while having good risk-comfort tradeoffs. Otherwise, the order of who goes first is safely kept and ROPT proactively manages intersection passings.

At this point, the path relations (e.g. front or back) are matched online with given regulatories (e.g. front-before-back). In large road junctions however, map data usually predefines the lane ranks %, e.g. turning main road or roundabouts. 
and other priority elements, such as traffic signs, should be incorporated as well. On this matter, the application of all-way stops shows to be promising, because they require to flexibly assign priority based on arrival time and might have superior pedestrian crossings. 
%Another direction would be four-way stops that dynamically assign priority based on who arrives first. 

Overall, ROPT is interaction-aware and could create cooperative behaviors in highway situations when permitting drivers on ramps to overtake or by prioritizing participants from faster lanes. In a next step, not only the interaction between pairs, but among all involved agents needs to be covered in the planning scheme. Rule deadlocks when cars on each incoming lane come together at an intersection are then solvable.

At last, real-time capability on a modern processor is not yet guaranteed with ROPT. %ROPT ran in the experiments with a time step of $\unit[200]{\text{msec}}$ on an Intel Core i5-6500 CPU and 8 \hspace{-0.1cm}GB RAM. 
For the application on a test car, it is possible to improve the computation time in two ways. On the one hand, the optimizer can leverage a numerical gradient from the explicit risk modeling. Parallely applied constructive heuristics (i.e., ramps are fine-tuned in succession to obtain one complete snake) enable here to get quicker out of local minima. On the other hand, a controller which executes the found ego trajectory for some timesteps supports ROPT to plan with lower update frequencies.
