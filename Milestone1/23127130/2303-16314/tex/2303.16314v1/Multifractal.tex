\batchmode
\makeatletter
\def\input@path{{/Users/axelaraneda/Desktop/Research/Hurst/}}
\makeatother
\documentclass[english]{article}
\usepackage{lmodern}
\usepackage[T1]{fontenc}
\usepackage[latin9]{inputenc}
\usepackage[a4paper]{geometry}
\geometry{verbose,tmargin=2.5cm,bmargin=2.5cm,lmargin=2.5cm,rmargin=2.5cm}
\usepackage{units}
\usepackage{amsmath}
\usepackage{amsthm}
\usepackage{amssymb}
\usepackage{graphicx}
\usepackage[numbers,numbers,sort&compress]{natbib}

\makeatletter
%%%%%%%%%%%%%%%%%%%%%%%%%%%%%% Textclass specific LaTeX commands.
\newcommand{\lyxaddress}[1]{
	\par {\raggedright #1
	\vspace{1.4em}
	\noindent\par}
}

%%%%%%%%%%%%%%%%%%%%%%%%%%%%%% User specified LaTeX commands.
%\usepackage{hyperref}
%\biboptions{sort&compress}
\date{}
%\usepackage[firstpage]{draftwatermark}
%\SetWatermarkColor[rgb]{1,0,0}
%\SetWatermarkText{Draft}
%\SetWatermarkScale{0.5}
%\usepackage{bibentry}
%\nobibliography*

\usepackage{todonotes}
\usepackage{bbm}
\usepackage{fontawesome5}
%\usepackage{orcidlink}
%\usepackage{doi}

\makeatother

\usepackage{babel}
\begin{document}
\title{\textbf{A multifractional option pricing formula}}
\author{Axel A.~Araneda\thanks{Email: \protect\href{mailto:axelaraneda@mail.muni.cz}{\texttt{axelaraneda@mail.muni.cz}}}}
\maketitle

\lyxaddress{\begin{center}
\vspace{-2em} Institute of Financial Complex Systems \\ Department
of Finance\\ Masaryk University\\ 602 00 Brno, Czech Republic.
\par\end{center}}

\begin{center}
\vspace{-1em} This version: March 27, 2023 \vspace{1.5em}
\par\end{center}
\begin{abstract}
Fractional Brownian motion has become a standard tool to address long-range
dependence in financial time series. However, a constant memory parameter
is too restrictive to address different market conditions. Here we
model the price fluctuations using a multifractional Brownian motion
assuming that the Hurst exponent is a time-deterministic function.
Through the multifractional Ito calculus, both the related transition
density function and the analytical European Call option pricing formula
are obtained. The empirical performance of the multifractional Black-Scholes
models is tested and appears superior to its fractional and standard
counterparts.

\textit{Keywords}: Multifractional Brownian motion, Hurst exponent,
Long-range dependence, European option pricing.

\vspace{2em}
\end{abstract}

\section{Introduction}

Since the Black and Scholes seminal paper \citep{black1973pricing},
diffusion processes driven by standard Brownian motions have been
the cornerstone of financial engineering. However, the long-range
dependence or long memory has been established as `stylized fact'
in the analysis of financial time series \citep{willinger1999stock,caporale2019long}.
In order to address this issue, many approaches based on fractional
Brownian motion (fBm) \citep{necula2002option,araneda2020fractional,he2021fractional,costabile2023lattice}
but also alternative process as the sub-fractional Brownian motion
\citep{xu2019pricing,araneda2021sub,bian2021fuzzy,wang2022pricing}
have been proposed.

However, the assumption of a constant Holder regularity (Hurst exponent)
in financial time-series seems to be too rigid to address some particularities
of a market beyond tranquil periods, namely bull or bear markets,
and both memory and memory-less can be present in the same financial
data \citep{bianchi2005pathwise,bianchi2013modeling}. Indeed, some
scholars empirically state a time-varying behavior for the memory
parameter \citep{guangxi2014time,guedes2022efficiency}. In terms
of modelling, the mathematical compatible with this behavior is called
multifractional Brownian motion (mBm) \citep{peltier1995multifractional,benassi1997elliptic}.
This centered Gaussian process acts as a generalization of fBm in
the sense that it allows to the Hurst exponent becomes a time-deterministic\footnote{Some developments \citep{ayache2005multifractional,ayache2022moving}
extend this local behavior to non-deterministic cases or stochastic
processes.} local quantity. The implications of using mBm as the driven process
in price fluctuations are listed in ref. \citep{Bianchi2015}, and
among them is the compatibility with Lo's adaptive market hypothesis
\citep{lo2005reconciling} which dismiss the efficiency/inefficiency
dichotomy arguing that the level of efficiency changes on time around
the efficient state.

The literature offers some examples of the uses of mBm in option pricing.
For instance, Wang \citep{wang2010scaling} addresses the problem
under transaction cost, where a discrete-time setting obtains the
minimal value for a European Call by delta hedging arguments. In addition,
Mattera and Sciorio \citep{mattera2021option} elaborated a numerical
procedure to value a European Call option in a multifractional environment,
considering an autoregressive behavior for the Hurst exponent. On
the other hand, Corlay et al. \citep{corlay2014multifractional} arise
a multifractional version for both Hull \& White and log-normal SABR
stochastic volatility models with the aim to fit the shape of the
smile at different maturities. Similarly, Ayache and Peng \citep{ayache2012stochastic}
discuss parameter estimation for the integrated volatility driven
by mBm.

Our insight here is slightly different and focused on the analytical
results for option pricing in a continuous-time setting. First, we
assume that the noise behavior of the asset dynamics can be modeled
using an mBm with Hurst exponent described by a time-deterministic
function, and second, taking borrow some results based on stochastic
calculus related to mBm, the respective effective Fokker-Planck equation
is derived and solved, and consequently, the option pricing formula
is addressed.

The paper is organized as the following. First, we listed some general
properties and auxiliary results for the mBm, particularly the multifractional
It\^o's lemma and the obtention of the related Fokker-Planck equation.
Later, we deal with the pricing procedure, focusing on an analytic
solution for the transition density and the proper pricing formula
using the actuarial approach. Finally, the main conclusions are listed.

\section{On the Multifractional Brownian motion}

\paragraph{Covariance: }

Let $h:\left[0,\infty\right)\rightarrow\left[l,m\right]\subset\left(0,1\right)$
 and $W_{h\left(t\right)}$ a standard mBm; i.e., $\text{var}\left[W_{h\left(1\right)}\right]=1$.
Then:

\[
\text{\ensuremath{\mathbb{E}\left[W_{h\left(t\right)}\cdot W_{h\left(s\right)}\right]=D\left(t,s\right)\left[t^{h(t)+h(s)}+s^{h(t)+h(s)}+\left|t-s\right|^{h(t)+h(s}\right]}}
\]
\noindent where

\[
D\left(t,s\right)=\frac{\sqrt{\Gamma\left(2t+1\right)\Gamma\left(2s+1\right)\sin\left(\pi t\right)\sin\left(\pi s\right)}}{2\Gamma\left(t+s+1\right)\sin\left[\frac{\pi}{2}\left(t+s\right)\right]}
\]

It should be noted that this covariance structure exhibits long-range
dependence \citep{Ayache2000}. Moreover:

\[
\mathbb{E}\left[\left(W_{h\left(t\right)}\right)^{2}\right]=t^{2h\left(t\right)}
\]


\paragraph{Multifractional Ito lemma \citep{lebovits2014white}:}

Let $F\in C^{2}\left(\mathbb{R}\right)$ and $y_{t}$ a generic stochastic
process driven by a multifractional Brownian motion:

\begin{equation}
\text{d}y_{t}=a\left(y_{t},t\right)\text{dt}+b\left(t,y_{t}\right)\text{d}W_{h\left(t\right)}\label{eq:gen}
\end{equation}

Then, the following equality holds:

\begin{eqnarray}
\text{d}F\left(t,y_{t}\right) & = & \frac{\partial F}{\partial t}\text{d}t+\frac{\partial F}{\partial y_{t}}\text{d}y_{t}+\frac{1}{2}\left\{ \frac{\text{d}}{\text{d}t}\left[t^{2h\left(t\right)}\right]\right\} b^{2}\frac{\partial^{2}F}{\partial t^{2}}\text{d}t\nonumber \\
 & = & \left\{ \frac{\partial F}{\partial t}+a\frac{\partial F}{\partial y_{t}}+b^{2}t^{2h(t)-1}\left[h'\left(t\right)t\ln t+h\left(t\right)\right]\frac{\partial^{2}F}{\partial t^{2}}\right\} \text{d}t+b\frac{\partial F}{\partial y_{t}}\text{d}W_{h\left(t\right)}\label{eq:mfI}
\end{eqnarray}

For a constant function $h(t)=H,$ the above theorem is reduced to
the Fractional Ito formula addressed by Bender \citep{bender2003ito},
while for the fixed value $h(t)=H=1/2$, the standard It\^o's lemma
is recovered.

\paragraph{Effective Fokker-Planck equation:}

Let $g\left(y_{t}\right)$a twice-differentiable scalar function and
$y_{t}$ the generic process (\ref{eq:gen}). Using the multifractional
It\^o calculus and taking expectations, we get:

\[
\frac{\text{d}\mathbb{E}\left(g\right)}{\text{d}t}=\mathbb{E}\left(a\frac{\partial g}{\partial y}\right)+\mathbb{E}\left\{ b^{2}t^{2h(t)-1}\left[h'\left(t\right)t\ln t+h\left(t\right)\right]\frac{\partial^{2}g}{\partial y^{2}}\right\} 
\]

Recalling the definition of expectations by means of the transition
density function $P$, and after some calculus, the effective Fokker-Planck
related to the process (\ref{eq:gen}) emerges:

\begin{equation}
\frac{\partial P}{\partial t}=t^{2h(t)-1}\left[h'\left(t\right)t\ln t+h\left(t\right)\right]\frac{\partial^{2}\left(b^{2}P\right)}{\partial y^{2}}-a\frac{\partial\left(aP\right)}{\partial y}\label{eq:TDF}
\end{equation}


\section{The multifractional Black-Scholes model and its transition density
function}

The geometric Brownian motion, real-world physical measure $\mathbb{P}$

\[
\text{d}S_{t}=\text{\ensuremath{\mu S_{t}\text{d}t}+\ensuremath{\sigma S_{t}W_{t}}}
\]

\noindent where the constant values $\mu$ and $\sigma$ represent
the yearly drift and volatility for the instantaneous return, and
$W_{t}$ an standard Gauss-Wiener process. In order to address ...,
we replace $W_{t}$ by a multifractional Brownian motion $W_{t\left(h\right)}$:

\[
\text{d}S_{t}=\text{\ensuremath{\mu S_{t}\text{d}t}+\ensuremath{\sigma S_{t}W_{h\left(t\right)}}}
\]

The Holderian function of the mBm $W_{t(h)}$; i.e., $H(t)$, is assumed
known and time-deterministic (see for example ref. \citep{corlay2014multifractional}
for the case of a time-dependent sinusoidal function).

By the substitution $x_{t}=\ln S_{t}-\mu t$ , he multifractional
It\^o's lemma (Eq. \ref{eq:mfI}) leads to:

\begin{equation}
\text{d}x_{t}=-\sigma^{2}t^{2h(t)-1}\left[h'\left(t\right)t\ln t+h\left(t\right)\right]\text{d}t+\sigma W_{h\left(t\right)}\label{eq:dx}
\end{equation}

According to the multifractional Fokker-Planck equation (Eq. \ref{eq:TDF}),
the transition density $P=P\left(x_{t},t\right)$ related to the process
(\ref{eq:dx}) obeys:

\begin{equation}
\frac{\partial P}{\partial t}=\sigma^{2}t^{2h(t)-1}\left[h'\left(t\right)t\ln t+h\left(t\right)\right]\left[\frac{\partial P}{\partial x}+\frac{\partial^{2}P}{\partial x^{2}}\right]\label{eq:FP_x}
\end{equation}

Using the time substitution:

\[
\bar{t}=\sigma^{2}t^{2h\left(t\right)}
\]

\noindent and defining the moving frame of reference:

\[
\bar{x}=x+\frac{\bar{t}}{2}
\]

Eq. (\ref{eq:FP_x}) goes to:

\[
\frac{\partial P}{\partial\bar{t}}=\frac{1}{2}\frac{\partial^{2}P}{\partial\bar{x}^{2}}
\]

The fundamental solution for the above equation (heat kernel with
constant thermal difussivity equal to $1/2$) is given by:

\[
P\left(\bar{x},\bar{t}\right)=\frac{1}{\sqrt{2\pi\bar{t}}}\exp\left[-\frac{\left(\bar{x}-\bar{x}_{0}\right)}{2\bar{t}}\right]
\]

\noindent where $P\left(\bar{x},0\right)=P\left(\bar{x}_{0}\right)=\delta\left(\bar{x}_{0}\right)$.
The initial condition is given by knowing the state of the asset at
the inception time; i.e., $S\left(t=0\right)=S_{0}=\text{e}^{x_{0}}=\text{e}^{\bar{x}_{0}}$.

Coming back to the variable $x$and the original time $t$, the transition
density is expressed as:

\[
P\left(x,t\right)=\frac{1}{\sqrt{2\pi\sigma^{2}t^{2h\left(t\right)}}}\exp\left[-\frac{\left(x-x_{0}+\frac{1}{2}\sigma^{2}t^{2h\left(t\right)}\right)^{2}}{2\sigma^{2}t^{2h\left(t\right)}}\right]
\]

From the previous result, we can compute the first moment for asset
price in a future time $t=T$ subject to its value at the inception
$t=0$:

\begin{eqnarray}
\mathbb{E^{P}}\left(S_{T}\right) & = & \int_{0}^{\infty}S_{T}P\left(S_{T},T\right)\text{d}S_{T}\nonumber \\
 & = & \int_{0}^{\infty}\text{e}^{x_{T}+\mu T}P\left(x_{T},T\right)\text{d}x_{T}\nonumber \\
 %& = & \frac{\text{e}^{x_{0}+\mu T}}{\sqrt{2\pi\sigma^{2}T^{2h\left(T\right)}}}\int_{0}^{\infty}\exp\left[-\frac{\left(x-x_{0}-\frac{1}{2}\sigma^{2}T^{2h\left(T\right)}\right)^{2}}{2\sigma^{2}T^{2h\left(T\right)}}\right]\text{d}x_{T	}\nonumber \\
 %& = & \frac{S_{0}\text{e}^{\mu T}}{\sqrt{2\pi}}\int_{0}^{\infty}\exp\left[-\frac{u^{2}}{2}\right]\text{d}u\nonumber \\
 & = & S_{0}\text{e}^{\mu T}\label{eq:S0}
\end{eqnarray}

\noindent where no differences appear concerning the expectation
in the classical Black Scholes world, while in the second central
moment, they differ\footnote{In the standard Geometric Brownian motion, the variance for the price
at time $T$ is equal to $S_{0}\text{e}^{2\mu T}\left(\text{e}^{\sigma^{2}T}-1\right)$.}:

\begin{eqnarray}
\mathbb{E^{P}}\left[\left(S_{T}-\mathbb{E^{P}}\left(S_{T}\right)\right)^{2}\right] & = & \int_{0}^{\infty}S_{T}^{2}P\left(S_{T},T\right)\text{d}S_{T}-S_{0}\text{e}^{2\mu T}\nonumber \\
 & = & \int_{0}^{\infty}\text{e}^{2x_{T}+2\mu T}P\left(x_{T},T\right)\text{d}x_{T}-S_{0}\text{e}^{2\mu T}\nonumber \\
 %& = & \frac{\text{e}^{x_{0}+\mu T}}{\sqrt{2\pi\sigma^{2}T^{2h\left(T\right)}}}\int_{0}^{\infty}\exp\left[-\frac{\left(x-x_{0}-\frac{1}{2}\sigma^{2}T^{2h\left(T\right)}\right)^{2}}{2\sigma^{2}T^{2h\left(T\right)}}\right]\text{d}x_{T}-S_{0}\text{e}^{2\mu T}\nonumber \\
 %& = & \frac{S_{0}\text{e}^{\mu T}}{\sqrt{2\pi}}\int_{0}^{\infty}\exp\left[-\frac{u^{2}}{2}\right]\text{d}u-S_{0}\text{e}^{2\mu T}\nonumber \\
 & = & S_{0}\text{e}^{2\mu T}\left(\text{e}^{\sigma^{2}T^{2h\left(T\right)}}-1\right)\label{eq:S0-1}
\end{eqnarray}


\section{The option pricing formula}

Under mBm diffusion there is no equivalent martingale measure, so
the risk risk-neutral pricing is not available \citep{mattera2021option}.
However, we can apply the actuarial approach \citep{bladt1998actuarial}
in order to get a fair option pricing formula without the semi-martingale
assumption.

Let $\text{e}^{\mu T}=\frac{\mathbb{E^{P}}\left(S_{T}\right)}{S_{0}}$
the expected rate of return for the asset $S$ at time $T$ (see Eq.
\ref{eq:S0}). By the actuarial approach, the fair premium for a vanilla
European Call option with maturity $T$ and exercise price $K$ is
given by \citep{bladt1998actuarial}:

\[
C\left(K,T\right)=\mathbb{E^{P}}\left[\left(\text{e}^{-\mu T}S_{T}-\text{e}^{-rT}K\right)^{+}\right]
\]

Since:

\begin{eqnarray*}
\text{e}^{-\mu T}S_{T}>\text{e}^{-rT}K & \iff & \text{e}^{x_{T}}>\text{e}^{-rT}K\\
 & \iff & x_{T}>\ln K-rT
\end{eqnarray*}

\noindent we have:

\begin{eqnarray}
C\left(K,T\right) & = & \int_{\ln K-rT}^{\infty}\left(\text{e}^{x_{T}}-\text{e}^{-rT}K\right)P\left(x_{T},T\right)\text{d}x_{T}\nonumber \\
 & = & \int_{\ln K-rT}^{\infty}\text{e}^{x_{T}}P\left(x_{T},T\right)\text{d}x_{T}-K\text{e}^{-rT}\int_{\ln K-rT}^{\infty}P\left(x_{T},T\right)\text{d}x_{T}\label{eq:C}
\end{eqnarray}

Given that,

\begin{eqnarray*}
\int_{\ln K-rT}^{\infty}\text{e}^{x_{T}}P\left(x_{T},T\right)\text{d}x_{T} & = & \frac{\text{e}^{x_{T}}}{\sqrt{2\pi\sigma^{2}T^{2h\left(t\right)}}}\exp\left[-\frac{\left(x-x_{0}+\frac{1}{2}\sigma^{2}T^{2h\left(T\right)}\right)^{2}}{2\sigma^{2}T^{2h\left(T\right)}}\right]\\
 & = & \frac{\text{e}^{x_{0}}}{\sqrt{2\pi\sigma^{2}T^{2h\left(T\right)}}}\int_{0}^{\infty}\exp\left[-\frac{\left(x-x_{0}-\frac{1}{2}\sigma^{2}T^{2h\left(T\right)}\right)^{2}}{2\sigma^{2}T^{2h\left(T\right)}}\right]\text{d}x_{T}\\
 & = & -\frac{\text{e}^{x_{0}}}{\sqrt{2\pi}}\int_{\frac{x_{0}-\ln K+rT+\frac{1}{2}\sigma^{2}T^{2h\left(T\right)}}{\sqrt{\sigma^{2}T^{2h\left(T\right)}}}}^{\infty}\text{e}^{-\frac{v^{2}}{2}}\text{d}v\\
 & = & \text{e}^{x_{0}}N\left(d_{1}\right)
\end{eqnarray*}

\begin{eqnarray*}
\int_{\ln K-rT}^{\infty}P\left(x_{T},T\right)\text{d}x_{T} & = & \frac{1}{\sqrt{2\pi\sigma^{2}T^{2h\left(t\right)}}}\int_{\ln K-\mu T}^{\infty}\exp\left[-\frac{\left(x-x_{0}+\frac{1}{2}\sigma^{2}T^{2h\left(t\right)}\right)^{2}}{2\sigma^{2}T^{2h\left(t\right)}}\right]\text{d}x_{T}\\
 & = & -\frac{1}{\sqrt{2\pi}}\int_{\frac{x_{0}-\ln K+rT-\frac{1}{2}\sigma^{2}T^{2h\left(T\right)}}{\sqrt{\sigma^{2}T^{2h\left(T\right)}}}}^{\infty}\text{e}^{-\frac{w^{2}}{2}}\text{d}w\\
 & = & N\left(d_{2}\right)
\end{eqnarray*}

\noindent where $N\left(\cdot\right)$ stands for the standard normal
cumulative density and:

\[
d_{1}=\frac{x_{0}-\ln K+rT+\frac{1}{2}\sigma^{2}T^{2h\left(T\right)}}{\sqrt{\sigma^{2}T^{2h\left(T\right)}}}=\frac{\ln\left(\nicefrac{S_{0}}{K}\right)+rT+\frac{1}{2}\sigma^{2}T^{2h\left(T\right)}}{\sqrt{\sigma^{2}T^{2h\left(T\right)}}}
\]

\[
d_{2}=d_{1}-\sqrt{\sigma^{2}T^{2h\left(T\right)}}
\]

Consequently, after replace the above computations into Eq. (\ref{eq:C}),
we can arrive at the pricing for a European Call under multifractional
diffusion:

\begin{equation}
C\left(K,T\right)=S_{0}N\left(d_{1}\right)-K\text{e}^{-rT}N\left(d_{2}\right)\label{eq:BS}
\end{equation}

The formula (\ref{eq:BS}) is also a generalization of the previous
approaches. If $h(t)=H$ is fixed to some value in its domain, the
above result is equivalent to the fractional Black-Scholes formula
\citep{necula2002option}, while for $h\left(t\right)=1/2$, the standard
Black-Scholes premium is recovered. 

With respect to market quotes, the performance of the multifractional
BS model is clearly superior to its fractional and classical counterparts.
This can be observed by calibrating the model prices to the at-the-money
market European Call values, by means of the minimization of the squared
residuals. Fig. \ref{fig:Performance-of-the} shows a set of market
quotes for SPX at-the-money call option (stock price=\$3970.99, strike
price=3970) written on March 24, 2023, considering maturities from
the range of 1 day to 5 months (data retrieved from Yahoo! Finance).
We use the 252 yearly days convention and set $r$ equal to the 13-week
T-Bill rate quoted on the inception time (4.5013\%). For the multifractional
Black-Scholes, as in ref. \citep{corlay2014multifractional}, we select
a 6-week ($\sim30$ trading days) periodic sinusoidal function for
the point-wise regularity, particularly, $\hat{h}(t)=A\cos\left[2\pi\left(\frac{252}{30}\right)t+B\right]+C$;
where $A$, $B$, and $C$, in addition to $\sigma$, are parameters
that should be estimated. The mean square errors of the market quotes
compared with model prices are lower for the multifractional approach
(456.8), followed by the fractional (493.7), and the standard BS (555.5).

\begin{figure}
\includegraphics[width=1\textwidth]{2_Users_axelaraneda_Desktop_Research_Hurst_optionprices.pdf}

\caption{Performance of the pricing models\label{fig:Performance-of-the}}

\end{figure}


\section{Summary}

We have modeled the price fluctuation by means of a Geometric Brownian
motion driven by a multifractional Brownian motion where the Hurst
exponent is an exclusive function of time. Our main result here is
the obtention of the analytical multifractional Black-Scholes formula
by means of the multifractional Ito calculus, the related Fokker-Planck
equation, and the actuarial approach to price option under the physical
measure $\mathbb{P}$. Since mBm is a generalization for both Bm and
fBm, the classical and fractional Black Scholes option pricing are
recovered. Empirical fits over SPX ATM European Call options show
better performance using a time-varying Hurst exponent. The option
pricing under different extensions of the multifractional Brownian
motion is a field of further research.

\bibliographystyle{ws-fnl}
\bibliography{multifractional}

\end{document}
