\documentclass[3p,review]{elsarticle}

\usepackage{lineno,hyperref,graphicx,epstopdf}
\usepackage{times}
\usepackage{type1cm}
\fontsize{12pt}{18pt}

\usepackage{braket}
\usepackage{amsmath}
\usepackage{amsfonts}
\usepackage{amssymb}
\usepackage{amsthm}
\usepackage{algorithm}
% \usepackage{algorithm2e}
\usepackage{algpseudocode}
\usepackage {braket}
\usepackage{diagbox}
\usepackage{makecell}

\theoremstyle{plain}
\newtheorem{theorem}{Theorem}
\newtheorem{restate}{Theorem}
\newtheorem{lemma}{Lemma}
\newtheorem{proposition}{Proposition}
\newtheorem{corollary}{Corollary}

\theoremstyle{definition}
\newtheorem{defi}{Definition}
\newtheorem{conjecture}{Conjecture}
\newtheorem{remark}{Remark}
\newtheorem{fact}{Fact}

\newtheorem{example}{Example}

\bibliographystyle{elsarticle-num}
\journal{}

\newcommand{\totalref}[2]{{#1 \ref{#2}}}
\date{}

\begin{document}
\begin{frontmatter}
\title{Distributed exact quantum algorithms for Deutsch-Jozsa problem}

\author{Hao Li$^{1,2}$}
\author{Daowen Qiu$^{1,2,4}$\corref{one}}
\author{Le Luo$^{3,4}$}
\cortext[one]{Corresponding author (D. Qiu). {\it E-mail addresses:} issqdw@mail.sysu.edu.cn (D. Qiu)}

\address{
   %$^1$
   $^{1}$Institute of Quantum Computing and Computer Theory, School of Computer Science and Engineering, Sun Yat-sen University, Guangzhou 510006, China\\
   $^{2}$The Guangdong Key Laboratory of Information Security Technology, Sun Yat-sen University, 510006, China\\
     $^{3}$School of Physics and Astronomy, Sun Yat-sen University, Zhuhai 519082, China\\
 $^{4}$QUDOOR Technologies Inc., Guangzhou,  China }


\begin{abstract}
%Limited by current physical technology, quantum circuits with long depth are usually noisy and difficult to be realized in practice. The innovative computing architecture of distributed quantum computing is expected to reduce the noise and depth of quantum circuits. In this paper, we study the Deutsch-Jozsa (DJ) problem in distributed scenarios and design three classes of distributed quantum algorithms to solve it.  The previous distributed quantum algorithm of the  DJ problem only considers the case of {\it two computing nodes} (i.e.,two subproblems), our distributed quantum algorithm further considers the case of extending to {\it multiple computing nodes} (i.e.,multiple subproblems). In particular, the algorithm proposed by us has the advantage of  exactness compared with the best distributed quantum algorithm proposed previously.\\

Deutsch-Jozsa (DJ) problem is one of the most important problems demonstrating the power of quantum algorithm. DJ problem can be described as a Boolean function $f$: $\{0,1\}^n\rightarrow \{0,1\}$ with promising it is either constant or balanced, and the purpose is to determine which type it is. DJ algorithm can solve it exactly with one query.   In this paper, we first discover the inherent structure of  DJ problem in distributed scenario by giving a number of equivalence characterizations between $f$ being constant (balanced) and some properties of $f$'s subfunctions, and then we  propose three distributed exact quantum algorithms for solving DJ problem. %by combines the design idea  of distributed Simon's quantum algorithm \cite{simon_power_1997} and HHL algorithm \cite{HHL_2009}.
Our algorithms have essential acceleration over distributed classical deterministic algorithm, and
  can be extended to the case  of multiple computing nodes.  Compared with  DJ algorithm, our algorithms can reduce the number of qubits and the depth of circuit implementing a single query operator. Therefore, we find that the structure of problem should be clarified for designing distributed quantum algorithm to solve it.
%In particular, the procedure of designing algorithms shows that it is important to discover the properties of problems in distributed  situation for  designing distributed quantum algorithms.


%Deutsch-Jozsa problem is an important problem in the histroy of quantum computing.
\end{abstract}

\begin{keyword}
%% keywords here, in the form: keyword \sep keyword
Deutsch-Jozsa problem \sep Distributed quantum algorithm \sep Circuit depth  
%\sep Quantum teleportation
\end{keyword}

\end{frontmatter}

%取消注释这行可显示行号
%\linenumbers

\section{Introduction}\label{sec:introduction}
	%Quantum computing \cite{nielsen_quantum_2010}\cite{kaye_introduction_2007}  has been proved to have great potential in factorizing  large numbers  \cite{shor_polynomial-time_1997}, unordered database search \cite{grover_fast_1996} and solving linear systems of equations \cite{HHL_2009}. %and chemical molecular simulation \cite{aspuru-guzik_simulated_2005}\cite{cao_quantum_2019}. 
%Nevertheless, due to the limitations of current physical technology, large-scale universal quantum computers have not been realized. At present, quantum technology has been entered to the NISQ (Noisy Intermediate-scale Quantum) era \cite{preskill_quantum_2018}, so it is possible for us to implement quantum algorithms on middle-scale quantum circuits.

Quantum computing \cite{nielsen_quantum_2010%,kaye_introduction_2007
} has been proved to have great potential in factorizing  large numbers \cite{shor_polynomial-time_1997},  searching unordered database \cite{grover_fast_1996} and solving linear systems of equations \cite{HHL_2009}. %and chemical molecular simulation \cite{aspuru-guzik_simulated_2005}\cite{cao_quantum_2019}. 
Due to the limitations of current physical technology, large-scale universal quantum computers have not been realized. At present, quantum technology has been entered to the Noisy Intermediate-scale Quantum (NISQ) era \cite{preskill_quantum_2018}, it is possible for us to implement quantum algorithms on middle-scale quantum circuits.

Distributed quantum computing is an innovative computing architecture, which combines quantum computing with distributed computing \cite{goos_distributed_2003,beals_efficient_2013,Qiu2017DQC,Tan2022DQCSimon,Qiu22,Xiao2023DQAShor}. In distributed quantum computing, multiple  computing nodes communicate with each other and cooperate to complete computing tasks, the circuit size and depth can be reduced% by using distributed quantum computing
, which  is beneficial to reduce the noise of circuits. 

%Distributed quantum computing is an innovative and fascinating computing architecture, which organically combines quantum computing with distributed computing \cite{goos_distributed_2003}\cite{beals_efficient_2013}\cite{Qiu2017DQC}\cite{Tan2022DQCSimon}\cite{Qiu22}\cite{Xiao2023DQAShor}. In distributed quantum computing architecture, multiple quantum computing nodes communicate with each other through channels and cooperate to complete computing tasks. Compared with centralized quantum computing, the circuit size and depth can be reduced by using distributed quantum computing, which  is beneficial to reduce the noise of circuits. In the current NISQ era, the adoption of distributed quantum computing technology is likely to facilitate the practical application of quantum algorithms.


As mentioning above, DJ problem can be described as a Boolean function $f:\{0,1\}^n\rightarrow \{0,1\}$ with promising it is either constant or balanced, and the purpose is to decide what function  it is. DJ algorithm can solve it exactly with one query, but it is easy to know that the query complexity of classical deterministic algorithm is $2^{n-1}+1$ in the worst case.
DJ problem is the first  to show that quantum algorithm has essential acceleration compared with classical deterministic algorithm \cite{deutsch_rapid_1992}, which is an extension of  Deutsch's problem \cite{deutsch_quantum_1985}.  Furthermore, Qiu and Zheng \cite{Generalized Deutsch-Jozsa proble_2018} extended DJ problem to  generalized DJ problem, and gave an optimal algorithm for  generalized DJ problem. DJ algorithm  \cite{deutsch_rapid_1992} presents a basic framework of quantum algorithms, and in a way provides inspiration for Simon’s algorithm \cite{simon_power_1997}, Shor’s algorithm \cite{shor_polynomial-time_1997} and Grover’s algorithm \cite{grover_fast_1996}. In particular, Qiu and Zheng  \cite{Revisiting Deutsch-Jozsa algorithm_2018} proved that  DJ algorithm can compute any symmetric partial Boolean function with exact quantum 1-query complexity.
 In fact, DJ algorithm can be used to test some properties of Boolean functions \cite{Testing Boolean Functions Properties_2021}.
In addition, it is worth mentioning that  there is also relevant  physical realization on  DJ algorithm \cite{Implementation of the Deutsch-Jozsa Algorithm_2013}.







%\cite{cai_optimal_2018}. %Simon's algorithm has a great enlightening effect on the subsequent proposal of Shor's algorithm which can decompose large numbers and compute discrete logarithms in polynomial time \cite{shor_polynomial-time_1997}. 
%Avron et al. proposed a distributed quantum algorithm to solve DJ problem \cite{avron_quantum_2021}. Their algorithm has the advantage of polynomial acceleration compared with the random algorithm. Since their algorithm directly runs DJ algorithm in multiple distributed nodes, there is no communication between the nodes, their algorithm is imprecise.

%In this paper, we study the DJ problem in distributed scenarios and design the  distributed quantum algorithm  to solve DJ problem exactly. We consider designing distributed quantum algorithms for the specific structure of the DJ problem, and utilize quantum teleportation technology to realize quantum gates spanning multiple computing nodes. After the oracle query for multiple quantum computing nodes, the arithmetic  operator and controlled rotation  operator are used to process the query results of multiple quantum computing nodes. After performing the above operations, we can obtain a quantum state containing the structure similar to the original DJ problem. Taking advantage of this structure, we can solve the DJ problem by two depth of query complexity exactly. The distributed quantum algorithm we design is closely related to the structure of DJ problem. %We have not found evidence that our algorithm can be directly extended to some quantum query algorithms, such as the Deutsch-Josza problem and the Grover's search problem. However, we believe that the design of distributed quantum algorithms with advantages for Deutsch-Josza problem and the Grover's search problem also needs to explore the structure of the problem. 
%it may shed some light on the design of distributed quantum algorithms for generalized DJ problem.


%In 2021, Avron et al \cite{avron_quantum_2021} proposed a distributed quantum algorithm for DJ problem. However, in their algorithm, they only consider the case of  two computing nodes, and  it directly runs DJ algorithm for each node without using quantum communication between nodes. In this way, the result of measuring in their algorithm is not exact and has error. In effect, their algorithm do not consider exploiting the essential structure of  DJ problem.
	
	 %In this distributed algorithm, it directly divides the qubits into several parts reasonably, so each part has fewer qubits than the original one. Since all unitary operators can be decomposed into single qubit quantum gates and CNOT gates \cite{nielsen2000quantum}, they only need to consider how to implement CNOT gates acting on different parts, while a CNOT gate acting on different parts can be implemented by means of pre-sharing EPR pairs, local operations and classical communication. Their distributed algorithm needs to communicate $O(L^2)$ classical bits.

In this paper, we first discover the inherent structure of DJ problem in distributed scenario by means of presenting a number of equivalence characterizations concerning the features of DJ problem.  More exactly,
let DJ problem be represented by a Boolean function $f:\{0,1\}^n\rightarrow\{0,1\}$, $f$ is decomposed  into $2^t$ subfunctions $f_w:\{0,1\}^{n-t}\rightarrow\{0,1\}$, where $f_w(u)=f(uw)$ $(u\in\{0,1\}^{n-t}, w\in\{0,1\}^t)$.
Then we prove a number of equivalence relationships between $f$ being constant (balanced) and some properties of its subfunctions.
In particular, we
 design three distributed exact quantum algorithms  for solving it, while we utilize quantum teleportation technology \cite{bennett_teleporting_1993} to realize quantum gates spanning multiple computing nodes \cite{caleffi_quantum_2018}. %exactly, and the case of extending to {\it multiple computing nodes} (i.e.,multiple subproblems) is considered.
%We consider designing distributed quantum algorithm for the specific structure  of  DJ problem, and utilize quantum teleportation technology \cite{bennett_teleporting_1993} to realize quantum gates spanning multiple computing nodes \cite{caleffi_quantum_2018}. 
In our  algorithms, after the oracle queries for multiple computing nodes, the specific unitary operator we design is used to process the query results. Eventually,  the quantum state containing the structure consistent with DJ problem is obtained, and then the solution of DJ problem is concluded exactly by measuring this state.
% in the case of  multiple computing nodes.% by using it. 
%After the original Boolean function of  DJ problem is decomposed into multiple subfunctions, the circuit depth of the oracle that implements the subfunction is less than that of the oracle that implements the original Boolean function of  DJ problem because the subfunction is simpler than the original function. Therefore, the circuit depth of the oracle implementing the Boolean function of  DJ problem is reduced, which helps to improve the anti-noise ability of the circuit.

%Compared with the classical distributed algorithm to solve DJ problem, the quantum algorithm proposed by us has the advantage of exponential acceleration. The distributed quantum algorithm of the previous DJ problem only considers the case of {\it two computing nodes} (i.e.,two subproblems), but our distributed quantum algorithm also considers the case of extending to {\it multiple computing nodes} (i.e.,multiple subproblems),  which is important in distributed quantum computing. In particular, our algorithm has the advantage of  exactness compared with the best distributed quantum algorithm proposed previously \cite{avron_quantum_2021}, which is profound and wonderful.



%and has the advantage of exactness compared with the best distributed quantum algorithm proposed before \cite{avron_quantum_2021}. In particular, the previous distributed quantum algorithm can not be extended to the case of more than two computing nodes (i.e. two subproblems), but our distributed quantum algorithm can also be extended to the case of multiple computing nodes (i.e. multiple subproblems), which is important in distributed quantum computing.


%Actually, we design three distributed exact quantum algorithms to solve DJ problem, and our algorithms can be summarized into two different  methods, which are denoted as method (I) and method (II) respectively. The algorithm belonging to method (I) is Algorithm 1, and the algorithms belonging to method (II)  are Algorithm 2 and Algorithm 3. The design with method (I) makes full use of the structural feature of  DJ problem in  distributed scenario, which reveals the relationship between the multiple subfunctions of DJ problem in  distributed scenario and the properties of the original Boolean function of DJ problem. The design with method (II) makes full use of the new structural feature of  DJ problem in  distributed scenario, which reveals the relationship between the properties of the pairwise composition subfunctions of DJ problem in  distributed scenario and the original Boolean function of DJ problem.



Actually, we design three distributed exact quantum algorithms to solve DJ problem, namely Algorithm \ref{algorithm3}, Algorithm \ref{algorithm2} and Algorithm \ref{algorithm5}. The design of Algorithm \ref{algorithm3} 
is to associate the two subfunctions $f_0$ and $f_1$ of  $f$  with xor operation, extracting the value of $f_0$ by using the Pauli operator $Z$. However, this algorithm can only solve DJ problem in distributed scenario with two subfunctions.


The design of Algorithm \ref{algorithm2} draws on the ideas and methods of distributed Simon's algorithm by Tan, Xiao, and Qiu et al \cite{Tan2022DQCSimon} and HHL algorithm \cite{HHL_2009}. In fact, Algorithm \ref{algorithm2} uses the specific unitary operator to perform addition and subtraction operations on the values of multiple subfunctions $f_w$ $(w\in\{0,1\}^t)$ of $f$,  and then  integrates the specific controlled rotation operator from HHL algorithm \cite{HHL_2009} to extract the values of the joint operation of the multiple subfunctions.


  Algorithm \ref{algorithm5} combines the  ideas and methods of Algorithm \ref{algorithm3} and Algorithm \ref{algorithm2}, which can solve DJ problem in distributed scenario with multiple computing nodes, and some of its single quantum gates require less qubits than Algorithm \ref{algorithm2}. Algorithm \ref{algorithm5} also draws on the ideas of distributed Simon's algorithm \cite{Tan2022DQCSimon} and HHL algorithm \cite{HHL_2009}.


The algorithms we  design have the following advantages. First, compared with the distributed quantum algorithm for solving  DJ problem in \cite{avron_quantum_2021}, our algorithms are exact and have better scalability. Second, compared with distributed classical deterministic  algorithm, our algorithms have  exponential advantage in query complexity. Finally, compared with  DJ algorithm, the single query operator in our algorithms requires fewer qubits, and the depth of circuit is reduced \cite{Elementary_gates_1995}, which has better anti-noise performance.





The remainder of the paper is organized as follows.  
In Section \ref{sec:preliminaries}, after briefly recalling  DJ problem and DJ algorithm, we  describe DJ problem in distributed scenario and  review the distributed  DJ  algorithm with errors  \cite{avron_quantum_2021} that motivates the study of designing distributed exact DJ algorithm. 
In Section \ref{sec:Characterization of the structure of DJ problem in distributed scenario}, we study a number of properties of DJ problem in  distributed scenario in detail, which are key to designing distributed exact DJ algorithms.
In section \ref{sec:Distributed quantum algorithm for  DJ problem (Algorithm 1)}, we first describe the ideas and methods for designing distributed exact DJ algorithms, and then  give  Algorithm \ref{algorithm3}. % The  advantage of Algorithm \ref{algorithm3} is that it requires fewer qubits and quantum gates than DJ algorithm. 
 Algorithm \ref{algorithm3} can solve  DJ problem in  distributed scenario with only  two computing nodes.
In section \ref{sec:Distributed quantum algorithm for  DJ problem (Algorithm 2)},  we describe  Algorithm \ref{algorithm2}. Algorithm \ref{algorithm2} can solve  DJ problem in  distributed scenario with multiple computing nodes, and its  key technique is to use the controlled rotation operator in  HHL algorithm \cite{HHL_2009} to extract the information related to  the structure of  DJ problem.
In section \ref{sec:Distributed quantum algorithm for  DJ problem (Algorithm 3)}, we describe  Algorithm \ref{algorithm5} and also compare the three algorithms we propose. Also, Algorithm \ref{algorithm5}  can solve  DJ problem in  distributed scenario with multiple computing nodes,  and   fewer qubits in this algorithm are required to implement certain unitary operators  compared to Algorithm \ref{algorithm2}. Finally,
in Section \ref{sec:conclusions}, we conclude with a summary. 

%In section \ref{sec:Distributed quantum algorithm for  DJ problem (Algorithm 2)} , we first give a number of characterizations of DJ problem in distributed scenario, and then present  distributed quantum algorithms for solving  DJ problem, analyse the correctness of our algorithms, and compare with other algorithms.In section \ref{sec:Distributed quantum algorithm for  DJ problem (Algorithm 3)} , we first give a number of characterizations of DJ problem in distributed scenario, and then present  distributed quantum algorithms for solving  DJ problem, analyse the correctness of our algorithms, and compare with other algorithms.
 %The correctness of our  algorithm is thoroughly proved  in Ref. \cite{SM}. 
%Next, we compare  our algorithm with distributed classical deterministic algorithm and another distributed quantum algorithm proposed in \cite{avron_quantum_2021}. 
% In Section \ref{sec:complexity analysis}, we make comparisons between our three algorithms designed  and analyse their advantages.
%Finally in Section \ref{sec:conclusions}, we conclude with a summary. 


\section{Preliminaries}\label{sec:preliminaries}
In this section, we first recall DJ problem and DJ algorithm, and then describe DJ problem in distributed scenario as well as review the distributed  DJ  algorithm with errors \cite{avron_quantum_2021}.  We assume that the readers are familiar with  liner algebra, probability theory and
basic notations in quantum computing \cite{nielsen_quantum_2010}.

\subsection{DJ problem and DJ algorithm}
  DJ problem can be described as follows: Consider a Boolean function $f:\{0,1\}^n \rightarrow \{0,1\}$, where we have the promise that $f$ is either constant or balanced. 

Call $f$ is constant if and only if 
\begin{equation}
f(x) \equiv 0
\end{equation}
 or 
 \begin{equation}
 f(x) \equiv 1. 
\end{equation}

Call $f$ is balanced if and only if  
\begin{equation}
|\{x|f(x)=0\}|=|\{x|f(x)=1\}|
=2^{n-1}. 
\end{equation}

Suppose we have an oracle $O_f$ that can query the value of Boolean function $f$, where the  oracle $O_f$ is defined as:
\begin{align}
O_{f}\ket{x}\ket{b}=&\ket{x}\ket{b\oplus f(x)},
\end{align}
 where $x\in\{0,1\}^{n}$ and $b\in\{0,1\}$.
%In classical computing, for any $x \in \{0,1\}^n$ and $y \in \{0,1\}$, if we input $(x, y)$ into the oracle, we will get $(x, y \oplus f(x))$. In quantum computing, for any $x \in \{0,1\}^n$ and $y \in \{0,1\}$, if we input $|x\rangle|y\rangle$ into the oracle $O_f$, we will get $|x\rangle|y \oplus f(x)\rangle$. 

The goal of   DJ problem is to determine whether $f$ is constant or balanced.% by performing the minimum number of queries to $f$.



%In the following, we recall DJ algorithm.

%Alice has an $n$ qubit register to store her query in, and a single qubit register which she will give to Bob, to store the answer in. She
%begins by preparing both her query and answer registers in a superposition state. Bob will evaluate $f$ using quantum parallelism and leave the result in the answer register.
%Alice then interferes states in the superposition using a Hadamard transform on the query register, and finishes by performing a suitable measurement to determine whether $f$ was constant or balanced.


 DJ algorithm is  the first quantum algorithm that is essentially faster than
any possible deterministic classical algorithm for solving DJ problem. %It was proposed by Deutsch and Jozsa in 1992 with improvements
%by Cleve, Ekert, Macchiavello, and Mosca in 1998. In this article, we generalize the Deutsch-Jozsa problem. 
In DJ algorithm, it first generates a uniform superposition state by acting on an initial state with  Hadamard transform, and then the query operator $O_f$ and   Hadamard transform are applied sequently, finally it performs measurement to determine whether $f$ is constant or balanced. The details of DJ algorithm are described in \ref{DJ Algorithm}.













%The best centralized classical deterministic algorithm for solving DJ problem requires $2^{n-1}+1$ queries. The best centralized quantum algorithm to solve DJ problem only requires one query. %In the history of quantum computing, 
%DJ problem is the first  to show that quantum algorithm has exponential acceleration compared with classical deterministic algorithm.

% \subsection{Operations of binary strings}

% We define operations of binary strings used in this paper as follows:

% \begin{defi}
%   For any binary string $u \in \{0,1\}^{n}$ and character $c \in \{0,1\}$, $uc$ is a binary string of length $n + 1$ which is the concatenation of the string $u$ and the character $c$.
% \end{defi}

% \begin{defi}
%   For any binary string $u \in \{0,1\}^{n}$ and character $v \in \{0,1\}^{m}$, $uv$ is a binary string of length $n + m$ which is the concatenation of the string $u$ and the string $v$.
% \end{defi}

% \begin{defi}
%   For any character $c \in \{0,1\}$ and positive integer $n$, $c^n$ is a binary string of length $n$ which represents $\underbrace{cc\ldots c}_{n}$.
% \end{defi}

% \begin{defi}
%   For any binary string $u,v \in \{0,1\}^{n}$, $u \oplus v$ denotes bitwise xor of $u$ and $v$.
% \end{defi}

% \begin{defi}
%   For any binary string $u,v \in \{0,1\}^{n}$, $u \cdot v$ denotes $(\sum_{i = 1}^{n}u_i \cdot v_i) \mod 2$.
% \end{defi}

% \begin{defi}
%   For any binary string $u \in \{0,1\}^{n}$, $u^{\perp}$ denotes $\{v \in \{0,1\}^{n}| u \cdot v = 0\}$.
% \end{defi}

%\subsection{Quantum teleportation}\label{quantum_teleportation}

\iffalse
\subsection{Quantum teleportation}
Quantum teleportation is a magical and delicate protocol \cite{bennett_teleporting_1993}. By sharing a pair of entangled states, one can transmit an unknown quantum state over a classical channel to another distant location without actually transmitting physical qubits. The protocol protects the qubits from being destroyed during transmission.

In distributed quantum computing, quantum gates spanning multiple computing nodes may be required. We can teleport a quantum state from one computing node to another \cite{caleffi_quantum_2018}, and  then apply a multiqubit gate on the combined state. %of the two nodes at the computing node being teleported.
In this way, we can implement quantum gates that span multiple computing nodes.
\fi

%\subsection{DJ problem in distributed scenario} \label{sec:distributed order finding algorithm}
	%For the sake of comparing our algorithm with the classical distributed algorithm, the DJ problem in distributed scenarios is described below.

\subsection{DJ problem in distributed scenario}
In the following, we first describe DJ problem in distributed scenario with two distributed computing nodes,  then describe DJ problem in distributed scenario with multiple distributed computing nodes.


In the case of two distributed computing nodes,  Boolean function $f$ corresponding to DJ problem is divided into two subfunctions $f_0:\{0,1\}^{n-1}\rightarrow\{0,1\}$ and $f_1:\{0,1\}^{n-1}\rightarrow\{0,1\}$ as follows.
For all $ u \in \{0,1\}^{n-1}$, let
\begin{equation}
f_0(u)=f(u0)
\end{equation}
and
\begin{equation}
f_1(u)=f(u1).
\end{equation}

 %where $f_0(u)=f(u0)$ and $f_1(u)=f(u1)$ for all $ u \in \{0,1\}^{n-1}$.
  Suppose Alice has an oracle $O_{f_0}$ that can query $f_0(u)$ for all $u \in \{0,1\}^{n-1}$,  Bob has an oracle $O_{f_1}$ that can query  $f_1(u)$ for all $u \in \{0,1\}^{n-1}$, %Here $u0$ (or $u1$)  represents the concatenation between string $u$ and character $0$ (or $1$). %Alice and Bob's oracles split the domain of function $f$ into two parts at the last bit of the domain. Each of them know half of the information about $f$, but neither knows all of the information about $f$. 
where the oracle $O_{f_0}$ and $O_{f_1}$  are defined as:
\begin{align}
O_{f_0}\ket{u}\ket{b}=&\ket{u}\ket{b\oplus f_0(u)},\label{Of0}\\
O_{f_1}\ket{u}\ket{b}=&\ket{u}\ket{b\oplus f_1(u)},\label{Of1}
\end{align}
 where $u\in\{0,1\}^{n-1}$ and $b\in\{0,1\}$.
   
  
They need to determine whether $f$ is constant or balanced by querying their own oracle and communicating with each other as few times as possible. 

%The algorithm in \cite{avron_quantum_2021} can not solve this problem exactly, so here
%we propose new distributed quantum algorithms to solve it. 

In the case of multiple distributed computing nodes,  Boolean function $f$ corresponding to DJ problem is divided into $2^t$ subfunctions $f_w:\{0,1\}^{n-t}\rightarrow\{0,1\}$ as follows.
For all $ u \in \{0,1\}^{n-t}$, let
\begin{equation}\label{General method of function decomposition}
f_w(u)=f(uw),
\end{equation}
where $w\in\{0,1\}^t$. 

%$f_w(u)=f(uw)$ for all $ u \in \{0,1\}^{n-t}$, $w\in\{0,1\}^t$. 

Suppose there are $2^t$ people, each of whom has an oracle $O_{f_w}$ that can query all $f_w(u)=f(uw)$ for all $u \in \{0,1\}^{n-t}$, $w \in \{0,1\}^t$, %($uw$ here represents the concatenation between string $u$ and string $w$).
where the oracle $O_{f_w}$ is defined as:
\begin{equation}\label{Ofw}
O_{f_w}\ket{u}\ket{b}=\ket{u}\ket{b\oplus f_w(u)},
\end{equation}
where $ w\in \{0,1\}^t$, $u\in\{0,1\}^{n-t}$, $b\in\{0,1\}$.
 

Each person can access $2^{n-t}$ values of $f$. %Note that there is no common element between the values of the function $f$ that everyone can access.
They need to determine whether $f$ is constant or balanced by querying their own oracle and communicating with each other as few times as possible. 


%This problem is not considered in \cite{avron_quantum_2021}, so we consider this problem  and design distributed quantum algorithms to solve it  exactly .  

\subsection{Distributed  DJ  algorithm with errors}


In 2021, Avron et al \cite{avron_quantum_2021} proposed a distributed DJ  algorithm. However, their algorithm only considers the case of  two computing nodes, and  it directly runs DJ algorithm for each node (i.e., subfunction) without using quantum communication between nodes. In this way, the result of measurement in their algorithm is not exact and has error. 

In the following we review the algorithm in \cite{avron_quantum_2021}  and give its error analysis.

Given a Boolean function $f:\{0,1\}^n\rightarrow\{0,1\}$ of  DJ problem,  and it is decomposed  into two subfunctions $f_0:\{0,1\}^{n-1}\rightarrow\{0,1\}$ and $f_1:\{0,1\}^{n-1}\rightarrow\{0,1\}$. The algorithm in \cite{avron_quantum_2021} runs the DJ algorithm directly for $f_0$ and $f_1$ respectively, obtaining the measuring  results denoted as $M_0$ and $M_1$ (if $M_0=M_1=0^{n-1}$, then $f$ is concluded to be constant), respectively. Let $k_0=|\{u\in\{0,1\}^{n-1}|f_0(u)=1\}|$, $k_1=|\{u\in\{0,1\}^{n-1}|f_1(u)=1\}|$, $N=2^n$.

The  probability that the algorithm in \cite{avron_quantum_2021} misidentifies a balanced function as constant is
\begin{equation}\label{DDJA_error_2node_probability}
\begin{split}
&\Pr(f\ is\ \ balanced,M_0=M_1=0^{n-1})\\
=&\sum_{\substack{k_0+k_1=\frac{N}{2}\\ 0\leq k_0\leq\frac{N}{2}\\0\leq k_1\leq\frac{N}{2}  }}\dfrac{\dbinom{N/2}{k_0}\dbinom{N/2}{k_1}}{\dbinom{N}{N/2}}\left(\dfrac{2^{2}}{N}\right)^2\left(k_0-\dfrac{N}{2^{2}}\right)^2\left(\dfrac{2^{2}}{N}\right)^2\left(k_1-\dfrac{N}{2^{2}}\right)^2\\
%=&\dfrac{3N-8}{N(N-1)(N-3)}\\
>&0.
\end{split}
\end{equation}

From equation (\ref{DDJA_error_2node_probability}), it is clear that there is a situation where the algorithm in \cite{avron_quantum_2021} has errors. In effect, the algorithm in \cite{avron_quantum_2021} does not consider exploiting the essential structure of  DJ problem.
%Although the algorithm in \cite{avron_quantum_2021} can be extended to  the general case of multiple distributed computing nodes, the error of applying the algorithm in \cite{avron_quantum_2021} to solve DJ problem will be higher as the number of distributed computing nodes increases. This is because it breaks the structure of the original Boolean function of DJ problem.
In \ref{Distributed DJ algorithm for multiple computing nodes with errors}, we generalize the algorithm in \cite{avron_quantum_2021} to solve the case of multiple subfunctions and analyze its errors. %\cite{avron_quantum_2021} and give its error analysis.  













\iffalse

Next, we  introduce some notations and theorems that will be used in this paper.




%\begin{defi}
%	For all $u \in \{0,1\}^{n-t}$, let multiset $G(u) = \{f(uw)|w \in \{0,1\}^t\}$.
%\end{defi}

\begin{defi}\label{deltadef}
Suppose function $f:\{0,1\}^n \rightarrow \{0,1\}$, for all $u \in \{0,1\}^{n-t}$, let  $\delta(u) = |\{w\in\{0,1\}^t|f(uw)=0\}|-|\{w\in\{0,1\}^t|f(uw)=1\}|$.%=2^t-|\{w\in\{0,1\}^t|f(uw)=1\}|-|\{w\in\{0,1\}^t|f(uw)=1\}|=2^t-2|\{w\in\{0,1\}^t|f(uw)=1\}|$.
\end{defi}

\iffalse

According to the definition of $\delta(u)$, we have
\begin{equation}\label{delta1}
\begin{split}
&\delta(u)\\
 =&|\{w\in\{0,1\}^t|f(uw)=0\}|\\
              &\quad-|\{w\in\{0,1\}^t|f(uw)=1\}|\\
  =&2^t-|\{w\in\{0,1\}^t|f(uw)=1\}|\\
  &\quad -|\{w\in\{0,1\}^t|f(uw)=1\}|\\
  =&2^t-2|\{w\in\{0,1\}^t|f(uw)=1\}|\\ 
  =&2^t-2\sum_{w\in\{0,1\}^t}f(uw).
\end{split}
\end{equation}

\fi

%Notice that there could be multiple identical elements in $G(u)$. 
%An example of $\delta(u)$ is shown in Appendix \ref{example}.



The following theorems concerning $\delta(u)$ are useful and important.





\begin{theorem}\label{The1} Suppose function $f:\{0,1\}^n \rightarrow \{0,1\}$, satisfies that it is either constant or balanced, it is divided into $2^t$ subfunctions $f_w$ $(\forall u \in \{0,1\}^{n-t}, w \in \{0,1\}^{t}, f_w(u)=f(uw))$. If $\exists$ $u\in\{0,1\}^{n-t}$ such that $|\delta(u)|\neq 2^t$, then $f$ is balanced. %a string $s \in \{0,1\}^n$  with $s\neq 0^n$, such that $f(x) = f(y)$ if and only if $x = y$ or $x \oplus y = s$. 
%Then
%  $\forall u,v \in \{0,1\}^{n-t},S(u)=S(v)$ if and only if $u \oplus v = 0^{n-t}$ or $u \oplus v = s_1$, where $s=s_1s_2$.
\end{theorem}
\iffalse
\begin{proof}
Suppose $\exists$ $u\in\{0,1\}^{n-t}$ such that $|\delta(u)|\neq 2^t$, then $f$ is constant. 
If $f$ is  constant, then $\forall u\in\{0,1\}^{n-t}$, $\sum_{w\in\{0,1\}^t}f(uw)=0$ or $\sum_{w\in\{0,1\}^t}f(uw)=2^t$. According to  equation $(\ref{delta1})$, then we have $|\delta(u)|= 2^t$, which is contrary to the assumption $|\delta(u)|\neq 2^t$. Therefore, if $\exists$ $u\in\{0,1\}^{n-t}$ such that $|\delta(u)|\neq 2^t$, then $f$ is balanced.
\end{proof}
\fi

\begin{theorem}\label{The2} Suppose function $f:\{0,1\}^n \rightarrow \{0,1\}$, satisfies that it is either constant or balanced, it is divided into $2^t$ subfunctions $f_w$ $(\forall u \in \{0,1\}^{n-t}, w \in \{0,1\}^{t}, f_w(u)=f(uw))$. Then $f$ is balanced if and only if $\sum\nolimits_{u\in\{0,1\}^{n-t}} \delta(u) = 0$. %Then $f$ is constant  if and only if $\sum\nolimits_{u\in\{0,1\}^{n-t}} \delta(u) =\pm 2^n$.
Then $f$ is constant  if and only if for all $u \in \{0,1\}^{n-t}$, $\delta(u) =2^t$ or for all $u \in \{0,1\}^{n-t}$, $\delta(u) =-2^t$. 
\end{theorem}
\iffalse
\begin{proof}
Firstly, we prove that $f$ is balanced if and only if $\sum\nolimits_{u\in\{0,1\}^{n-t}} \delta(u) = 0$.

(\romannumeral1) $\Longleftarrow$. If $\sum\nolimits_{u\in\{0,1\}^{n-t}} \delta(u)=0$, according to equation $(\ref{delta1})$, then we have 
\begin{equation}
\begin{split}
&\sum\nolimits_{u\in\{0,1\}^{n-t}} \delta(u)\\
=&\sum\nolimits_{u\in\{0,1\}^{n-t}} (2^t-2\sum_{w\in\{0,1\}^t}f(uw))\\
=&\sum\nolimits_{u\in\{0,1\}^{n-t}}2^t-2\sum\nolimits_{u,w\in\{0,1\}^{n-t}}f(uw)\\
=&2^n-2\sum\nolimits_{x\in\{0,1\}^{n}}f(x)\\
=&0, 
\end{split}
\end{equation}
that is $\sum\nolimits_{x\in\{0,1\}^{n}}f(x)=2^{n-1}$.

So we have 
\begin{equation}
\begin{split}
&|\{x\in\{0,1\}^n|f(x)=1\}|\\
=&\sum\nolimits_{x\in\{0,1\}^{n}}f(x)=2^{n-1}. 
\end{split}
\end{equation}
\begin{equation}
\begin{split}
&|\{x\in\{0,1\}^n|f(x)=0\}|\\
=&2^n-|\{x\in\{0,1\}^n|f(x)=1\}|\\
=&2^{n-1}. 
\end{split}
\end{equation}
Therefore, $f$ is balanced.

(\romannumeral2) $\Longrightarrow$. If $f$ is balanced, then  we have $|\{x\in\{0,1\}^n|f(x)=0\}|=|\{x\in\{0,1\}^n|f(x)=1\}|=2^{n-1}$. According to equation $(\ref{delta1})$, then we have 
\begin{equation}
\begin{split}
&\sum\nolimits_{u\in\{0,1\}^{n-t}} \delta(u)\\
=&\sum\nolimits_{u\in\{0,1\}^{n-t}} (2^t-2\sum_{w\in\{0,1\}^t}f(uw))\\
=&\sum\nolimits_{u\in\{0,1\}^{n-t}}2^t-2\sum\nolimits_{u,w\in\{0,1\}^{n-t}}f(uw)\\
=&2^n-2\sum\nolimits_{x\in\{0,1\}^{n}}f(x)\\
=&2^n-2|\{x\in\{0,1\}^n|f(x)=1\}|\\
=&0.
\end{split}
\end{equation}

\iffalse

Secondly, we prove that $f$ is constant if and only if $\sum\nolimits_{u\in\{0,1\}^{n-t}} \delta(u) = 2^n$. 



(\romannumeral3) $\Longleftarrow$. If $\sum\nolimits_{u\in\{0,1\}^{n-t}} \delta(u)=\pm 2^n$, according to equation $(\ref{delta1})$, then we have 
\begin{equation}
\begin{split}
&\sum\nolimits_{u\in\{0,1\}^{n-t}} \delta(u)\\
=&\sum\nolimits_{u\in\{0,1\}^{n-t}} (2^t-2\sum_{w\in\{0,1\}^t}f(uw))\\
=&\sum\nolimits_{u\in\{0,1\}^{n-t}}2^t-2\sum\nolimits_{u,w\in\{0,1\}^{n-t}}f(uw)\\
=&2^n-2\sum\nolimits_{x\in\{0,1\}^{n}}f(x)=\pm 2^n,
\end{split}
\end{equation}
that is $\sum\nolimits_{x\in\{0,1\}^{n}}f(x)=0$ or $2^n$. 

So $|\{x\in\{0,1\}^n|f(x)=1\}|=0$ or $2^n$, that is $f(x) \equiv 0$ or $f(x) \equiv 1$. Therefore, $f$ is constant.

(\romannumeral4) $\Longrightarrow$. If $f$ is constant, according to the definition of DJ problem, then we have $f(x) \equiv 0$ or $f(x) \equiv 1$. So $|\{x\in\{0,1\}^n|f(x)=1\}|=0$ or $2^n$, that is $\sum\nolimits_{x\in\{0,1\}^{n}}f(x)=0$ or $2^n$. According to equation $(\ref{delta1})$, then we have 
\begin{equation}
\begin{split}
&\sum\nolimits_{u\in\{0,1\}^{n-t}} \delta(u)\\
=&\sum\nolimits_{u\in\{0,1\}^{n-t}} (2^t-2\sum_{w\in\{0,1\}^t}f(uw))\\
=&\sum\nolimits_{u\in\{0,1\}^{n-t}}2^t-2\sum\nolimits_{u,w\in\{0,1\}^{n-t}}f(uw)\\
=&2^n-2\sum\nolimits_{x\in\{0,1\}^{n}}f(x)\\
=&\pm 2^n.
\end{split}
\end{equation}


\fi



Secondly, we prove that $f$ is constant if and only if for all $u \in \{0,1\}^{n-t}$, $\delta(u) =2^t$ or for all $u \in \{0,1\}^{n-t}$, $\delta(u) =-2^t$. 



(\romannumeral3) $\Longleftarrow$. If  for all $u \in \{0,1\}^{n-t}$, $\delta(u) =2^t$, then according to equation (\ref{delta1}), for all $u \in \{0,1\}^{n-t}$, we have $\sum_{w\in\{0,1\}^t}f(uw)=0$. So for all $x \in \{0,1\}^{n}$, $f(x)=0$, that is $f(x) \equiv 0$.





 If  for all $u \in \{0,1\}^{n-t}$, $\delta(u) =-2^t$, then according to equation (\ref{delta1}), for all $u \in \{0,1\}^{n-t}$, we have $\sum_{w\in\{0,1\}^t}f(uw)=2^t$. So for all $x \in \{0,1\}^{n}$, $f(x)=1$, that is $f(x) \equiv 1$.
 
 Therefore, if for all $u \in \{0,1\}^{n-t}$, $\delta(u) =2^t$ or for all $u \in \{0,1\}^{n-t}$, $\delta(u) =-2^t$, then $f$ is constant.

(\romannumeral4) $\Longrightarrow$. If $f$ is constant, according to the definition of DJ problem, then we have $f(x) \equiv 0$ or $f(x) \equiv 1$. 


If $f(x) \equiv 0$, then for all $x \in \{0,1\}^{n}$, $f(x)=0$. So for all $u \in \{0,1\}^{n-t}$, we have $\sum_{w\in\{0,1\}^t}f(uw)=0$.  According to equation (\ref{delta1}), for all $u \in \{0,1\}^{n-t}$, we have $\delta(u) =2^t$.


If $f(x) \equiv 1$, then for all $x \in \{0,1\}^{n}$, $f(x)=1$. So for all $u \in \{0,1\}^{n-t}$, we have $\sum_{w\in\{0,1\}^t}f(uw)=2^t$.  According to equation (\ref{delta1}), for all $u \in \{0,1\}^{n-t}$, we have $\delta(u) =-2^t$.

 Therefore, if $f$ is constant, then for all $u \in \{0,1\}^{n-t}$, $\delta(u) =2^t$ or for all $u \in \{0,1\}^{n-t}$, $\delta(u) =-2^t$.

\end{proof}

\fi




The following theorems %concerning DJ problem
 are profound and interesting.




\begin{theorem}\label{The3} Suppose function $f:\{0,1\}^n \rightarrow \{0,1\}$, satisfies that it is either constant or balanced, it is divided into subfunctions $f_0$ and $f_1$. If $\exists$ $u\in\{0,1\}^{n-1}$ such that $f(u0)\oplus f(u1)=1$, then $f$ is balanced.
\end{theorem}
\iffalse
\begin{proof}
Suppose $\exists$ $u\in\{0,1\}^{n-1}$ such that $f(u0)\oplus f(u1)=1$, then $f$ is constant. 
If $f$ is  constant, then $\forall u\in\{0,1\}^{n-1}$, $f(u0)=f(u1)=0$ or $f(u0)=f(u1)=1$. So $f(u0)\oplus f(u1)=0$, which is contrary to the assumption $f(u0)\oplus f(u1)=1$. Therefore, if $\exists$ $u\in\{0,1\}^{n-1}$ such that $f(u0)\oplus f(u1)=1$, then $f$ is balanced.
\end{proof}
\fi

\begin{defi}
Suppose function $f:\{0,1\}^n \rightarrow \{0,1\}$, let $B_0=|\{u\in\{0,1\}^{n-1}|f(u0)\oplus f(u1)=0, f(u0)=0\}|$, $B_1=|\{u\in\{0,1\}^{n-1}|f(u0)\oplus f(u1)=0, f(u0)=1\}|$.
\end{defi}


\begin{defi}
 Suppose function $f:\{0,1\}^n \rightarrow \{0,1\}$, let $M=|\{u\in\{0,1\}^{n-1}|f(u0)\oplus f(u1)=0\}|$ $(0\leq M\leq 2^{n-1})$. 
\end{defi}

\begin{theorem}\label{The4} Suppose function $f:\{0,1\}^n \rightarrow \{0,1\}$, satisfies that it is either constant or balanced, it is divided into subfunctions $f_0$ and $f_1$. Then $f$ is balanced if and only if $B_0=B_1=M/2$. %Then $f$ is constant  if and only if $|\{u\in\{0,1\}^{n-1}|f(u0)=0\}|=|\{u\in\{0,1\}^{n-1}|f(u1)=0\}|=2^{n-1}$ or $|\{u\in\{0,1\}^{n-1}|f(u0)=1\}|=|\{u\in\{0,1\}^{n-1}|f(u1)=1\}|=2^{n-1}$. 
\end{theorem}
\iffalse
\begin{proof}
%First, we prove that $f$ is balanced if and only if $B_0=B_1=M/2$.

(\romannumeral1) $\Longleftarrow$. Suppose $B_0=B_1=M/2$ $(0\leq M\leq 2^{n-1})$, then $f$ is constant. If $f$ is constant, then $f\equiv 0$ or $f\equiv 1$, that is $|\{u\in\{0,1\}^{n-1}|f(u0)=f(u1)=0\}|=2^{n-1}$ or $|\{u\in\{0,1\}^{n-1}|f(u0)=f(u1)=1\}|=2^{n-1}$. So $B_0=2^{n-1}$ or $B_1=2^{n-1}$, which is contrary to the assumption $B_0=B_1=M/2$ $(0\leq M\leq 2^{n-1})$. Therefore, if $B_0=B_1=M/2$, then $f$ is balanced, where $M=|\{u\in\{0,1\}^{n-1}|f(u0)\oplus f(u1)=0\}|$.


(\romannumeral2) $\Longrightarrow$. %If $f$ is balanced, according to the definition of DJ problem, then   $|\{x\in\{0,1\}^n|f(x)=0\}|=|\{x\in\{0,1\}^n|f(x)=1\}|=2^{n-1}$. 
Since 
\begin{equation}
\begin{split}
M=&|\{u\in\{0,1\}^{n-1}|f(u0)\oplus f(u1)=0\}|\\
=&|\{u\in\{0,1\}^{n-1}|f(u0)= f(u1)=0\}|\\
&+|\{u\in\{0,1\}^{n-1}|f(u0)= f(u1)=1\}|,
\end{split}
\end{equation}
 therefore $|\{u\in\{0,1\}^{n-1}|f(u0)\oplus f(u1)=1\}|=2^{n-1}-M$, that is 
\begin{equation}
\begin{split}
& |\{u\in\{0,1\}^{n-1}|f(u0)=0,f(u1)=1\}|\\
&+|\{u\in\{0,1\}^{n-1}|f(u0)=1,f(u1)=0\}|\\
 =&2^{n-1}-M. 
 \end{split}
\end{equation}

 We have 
 \begin{equation}
\begin{split}
 &|\{u\in\{0,1\}^{n-1}|f(u0)= f(u1)=0\}|\\
 =&|\{u\in\{0,1\}^{n-1}|f(u0)=0\}|\\
 &-|\{u\in\{0,1\}^{n-1}|f(u0)=0,f(u1)=1\}|\\
 =&|\{u\in\{0,1\}^{n-1}|f(u1)=0\}|\\
 &-|\{u\in\{0,1\}^{n-1}|f(u0)=1,f(u1)=0\}|. 
\end{split}
\end{equation}

Let 
 \begin{equation}
\begin{split}
S_0=&|\{u\in\{0,1\}^{n-1}|f(u0)=0\}|\\
&-|\{u\in\{0,1\}^{n-1}|f(u0)=0,f(u1)=1\}|.
\end{split}
\end{equation}
\begin{equation}
\begin{split}
S_1=&|\{u\in\{0,1\}^{n-1}|f(u1)=0\}|\\
&-|\{u\in\{0,1\}^{n-1}|f(u0)=1,f(u1)=0\}|. 
\end{split}
\end{equation}

Then 
\begin{equation}
\begin{split}
S_0=S_1=|\{u\in\{0,1\}^{n-1}|f(u0)= f(u1)=0\}|.
\end{split}
\end{equation}

Then 
\begin{equation}
\begin{split}
&S_0+S_1\\
=&(|\{u\in\{0,1\}^{n-1}|f(u0)=0\}|\\
&-|\{u\in\{0,1\}^{n-1}|f(u0)=0,f(u1)=1\}|)\\
&+(|\{u\in\{0,1\}^{n-1}|f(u1)=0\}|\\
&-|\{u\in\{0,1\}^{n-1}|f(u0)=1,f(u1)=0\}|)\\
=&(|\{u\in\{0,1\}^{n-1}|f(u0)=0\}|\\
&+|\{u\in\{0,1\}^{n-1}|f(u1)=0\}|)\\
&-(|\{u\in\{0,1\}^{n-1}|f(u0)=0,f(u1)=1\}|\\
&+|\{u\in\{0,1\}^{n-1}|f(u0)=1,f(u1)=0\}|)\\
=&(|\{u\in\{0,1\}^{n-1}|f(u0)=0\}|\\
&+|\{u\in\{0,1\}^{n-1}|f(u1)=0\}|)-(2^{n-1}-M). 
\end{split}
\end{equation}

If $f$ is balanced, according to the definition of DJ problem, then we have $|\{x\in\{0,1\}^n|f(x)=0\}|=|\{x\in\{0,1\}^n|f(x)=1\}|=2^{n-1}$. So $|\{u\in\{0,1\}^{n-1}|f(u0)=0\}|+|\{u\in\{0,1\}^{n-1}|f(u1)=0\}|=|\{x\in\{0,1\}^n|f(x)=0\}|=2^{n-1}$. Therefore, we have
\begin{equation}
\begin{split}
&S_0+S_1\\
=&(|\{u\in\{0,1\}^{n-1}|f(u0)=0\}|\\
&+|\{u\in\{0,1\}^{n-1}|f(u1)=0\}|)-(2^{n-1}-M)\\
=&2^{n-1}-(2^{n-1}-M)\\
=&M, 
\end{split}
\end{equation}
then from $S_0=S_1=|\{u\in\{0,1\}^{n-1}|f(u0)= f(u1)=0\}|$, we have $S_0=S_1=|\{u\in\{0,1\}^{n-1}|f(u0)= f(u1)=0\}|=M/2$. 


Further, from $M=|\{u\in\{0,1\}^{n-1}|f(u0)= f(u1)=0\}|+|\{u\in\{0,1\}^{n-1}|f(u0)= f(u1)=1\}|$, we have 
\begin{equation}
\begin{split}
&|\{u\in\{0,1\}^{n-1}|f(u0)= f(u1)=1\}|\\
=&M-|\{u\in\{0,1\}^{n-1}|f(u0)= f(u1)=0\}|\\
=&M-M/2\\
=&M/2.
\end{split}
\end{equation}

Since 
\begin{equation}
\begin{split}
B_0=&|\{u\in\{0,1\}^{n-1}|f(u0)\oplus f(u1)=0, f(u0)=0\}|\\
=&|\{u\in\{0,1\}^{n-1}|f(u0)= f(u1)=0\}|.
\end{split}
\end{equation}
\begin{equation}
\begin{split}
B_1=&|\{u\in\{0,1\}^{n-1}|f(u0)\oplus f(u1)=0, f(u0)=1\}|\\
=&|\{u\in\{0,1\}^{n-1}|f(u0)= f(u1)=1\}|.
\end{split}
\end{equation}
Therefore, we have $B_0=B_1=M/2$.

%$S_0=|\{u\in\{0,1\}^{n-1}|f(u0)=0\}|-|\{u\in\{0,1\}^{n-1}|f(u0)=0,f(u1)=1\}|=B_0$, $S_1=|\{u\in\{0,1\}^{n-1}|f(u1)=0\}|-|\{u\in\{0,1\}^{n-1}|f(u0)=1,f(u1)=0\}|=B_1$, we have $B_0=B_1=M/2$.



% According to equation $(\ref{delta1})$, then we have $\sum\nolimits_{u\in\{0,1\}^{n-t}} \delta(u)=\sum\nolimits_{u\in\{0,1\}^{n-t}} (2^t-2\sum_{w\in\{0,1\}^t}f(uw))=\sum\nolimits_{u\in\{0,1\}^{n-t}}2^t-2\sum\nolimits_{u,w\in\{0,1\}^{n-t}}f(uw)=2^n-2\sum\nolimits_{x\in\{0,1\}^{n}}f(x)=2^n-2|\{x\in\{0,1\}^n|f(x)=1\}|=0$.

%Second, we prove that $f$ is constant if and only if $|\{u\in\{0,1\}^{n-1}|f(u0)=0\}|=|\{u\in\{0,1\}^{n-1}|f(u1)=0\}|=2^{n-1}$ or $|\{u\in\{0,1\}^{n-1}|f(u0)=1\}|=|\{u\in\{0,1\}^{n-1}|f(u1)=1\}|=2^{n-1}$. 

%(3) $\Longleftarrow$. If $|\{u\in\{0,1\}^{n-1}|f(u0)=0\}|=|\{u\in\{0,1\}^{n-1}|f(u1)=0\}|=2^{n-1}$ or $|\{u\in\{0,1\}^{n-1}|f(u0)=1\}|=|\{u\in\{0,1\}^{n-1}|f(u1)=1\}|=2^{n-1}$, then $|\{x\in\{0,1\}^n|f(x)=1\}|=0$ or $2^n$, that is $f(x) \equiv 0$ or $f(x) \equiv 1$. Therefore, $f$ is constant.

%(4) $\Longrightarrow$. If $f$ is constant, according to the definition of DJ problem, then  $f(x) \equiv 0$ or $f(x) \equiv 1$. So $|\{x\in\{0,1\}^n|f(x)=1\}|=0$ or $2^n$. Therefore $|\{u\in\{0,1\}^{n-1}|f(u0)=0\}|=|\{u\in\{0,1\}^{n-1}|f(u1)=0\}|=2^{n-1}$ or $|\{u\in\{0,1\}^{n-1}|f(u0)=1\}|=|\{u\in\{0,1\}^{n-1}|f(u1)=1\}|=2^{n-1}$.

\end{proof}
\fi

\begin{theorem}\label{The5} Suppose function $f:\{0,1\}^n \rightarrow \{0,1\}$, satisfies that it is either constant or balanced, it is divided into subfunctions $f_0$ and $f_1$. Then $f$ is constant  if and only if $|\{u\in\{0,1\}^{n-1}|f(u0)=0\}|=|\{u\in\{0,1\}^{n-1}|f(u1)=0\}|=2^{n-1}$ or $|\{u\in\{0,1\}^{n-1}|f(u0)=1\}|=|\{u\in\{0,1\}^{n-1}|f(u1)=1\}|=2^{n-1}$. 
\end{theorem}
\iffalse
\begin{proof}
%Second, we prove that $f$ is constant if and only if $|\{u\in\{0,1\}^{n-1}|f(u0)=0\}|=|\{u\in\{0,1\}^{n-1}|f(u1)=0\}|=2^{n-1}$ or $|\{u\in\{0,1\}^{n-1}|f(u0)=1\}|=|\{u\in\{0,1\}^{n-1}|f(u1)=1\}|=2^{n-1}$. 

(\romannumeral1) $\Longleftarrow$. If $|\{u\in\{0,1\}^{n-1}|f(u0)=0\}|=|\{u\in\{0,1\}^{n-1}|f(u1)=0\}|=2^{n-1}$ or $|\{u\in\{0,1\}^{n-1}|f(u0)=1\}|=|\{u\in\{0,1\}^{n-1}|f(u1)=1\}|=2^{n-1}$, then $|\{x\in\{0,1\}^n|f(x)=1\}|=0$ or $2^n$, that is $f(x) \equiv 0$ or $f(x) \equiv 1$. Therefore, $f$ is constant.

(\romannumeral2) $\Longrightarrow$. If $f$ is constant, according to the definition of DJ problem, then  $f(x) \equiv 0$ or $f(x) \equiv 1$. So $|\{x\in\{0,1\}^n|f(x)=1\}|=0$ or $2^n$. Therefore $|\{u\in\{0,1\}^{n-1}|f(u0)=0\}|=|\{u\in\{0,1\}^{n-1}|f(u1)=0\}|=2^{n-1}$ or $|\{u\in\{0,1\}^{n-1}|f(u0)=1\}|=|\{u\in\{0,1\}^{n-1}|f(u1)=1\}|=2^{n-1}$.

\end{proof}
\fi

%Next, we further introduce some notations that will be used in this paper.

%\begin{defi}
%For all $u \in \{0,1\}^{n-t}$, let  $D(u) = |\{w\in\{0,1\}^{t-1}0|f(uw)\oplus f(u(w+1))=1\}|$.
%\end{defi}


\begin{defi}\label{Deltadef}
Suppose function $f:\{0,1\}^n \rightarrow \{0,1\}$, for all $u \in \{0,1\}^{n-t}$, let  $\Delta(u) = |\{w\in\{0,1\}^{t-1}0|f(uw)=f(u(w+1))=0\}|-|\{w\in\{0,1\}^{t-1}0|f(uw)=f(u(w+1))=1\}|$.%=2^t-|\{w\in\{0,1\}^t|f(uw)=1\}|-|\{w\in\{0,1\}^t|f(uw)=1\}|=2^t-2|\{w\in\{0,1\}^t|f(uw)=1\}|$.
\end{defi}

\iffalse

Since 
\begin{equation}
\begin{split}
 &|\{w\in\{0,1\}^{t-1}0|f(uw)\oplus f(u(w+1))=1\}|\\
 &+|\{w\in\{0,1\}^{t-1}0|f(uw)\oplus f(u(w+1))=0\}|\\
 = &|\{w\in\{0,1\}^{t-1}0|f(uw)\oplus f(u(w+1))=1\}|\\
    &+|\{w\in\{0,1\}^{t-1}0|f(uw)=f(u(w+1))=0\}|\\
    &+|\{w\in\{0,1\}^{t-1}0|f(uw)=f(u(w+1))=1\}|\\
= &2^{t-1},
\end{split}
\end{equation}
then according to the definition of $\Delta(u)$, for all $u \in \{0,1\}^{n-t}$, we have
\begin{equation}
\begin{split}
-2^{t-1}\leq\Delta(u)\leq 2^{t-1}.
\end{split}
\end{equation}

\fi

\iffalse
According to the definition of $\Delta(u)$,  for all $u \in \{0,1\}^{n-t}$, we have
\begin{equation}\label{Delta1}
\begin{split}
 &\quad\Delta(u)\\
 &=|\{w\in\{0,1\}^{t-1}0|f(uw)=f(u(w+1))=0\}|\\
  &\quad-|\{w\in\{0,1\}^{t-1}0|f(uw)=f(u(w+1))=1\}|\\
  &=(2^{t-1}-|\{w\in\{0,1\}^{t-1}0|f(uw)=f(u(w+1))=1\}|\\
  &\quad-|\{w\in\{0,1\}^{t-1}0|f(uw)\oplus f(u(w+1))=1\}|)\\
  &\quad- |\{w\in\{0,1\}^{t-1}0|f(uw)=f(u(w+1))=1\}|\\
  &=2^{t-1}-|\{w\in\{0,1\}^{t-1}0|f(uw)\oplus f(u(w+1))=1\}|\\
  &\quad-2|\{w\in\{0,1\}^{t-1}0|f(uw)=f(u(w+1))=1\}|\\
  &=2^{t-1}-\sum_{w\in\{0,1\}^{t-1}0}f(uw)\oplus f(u(w+1))\\
  &\quad-2\sum_{w\in\{0,1\}^{t-1}0}f(uw)\land f(u(w+1)).
\end{split}
\end{equation}

According to the definition of $\Delta(u)$, we also have
\begin{equation}\label{Delta2}
\begin{split}
 &\quad\Delta(u)\\
 &=|\{w\in\{0,1\}^{t-1}0|f(uw)=f(u(w+1))=0\}|\\
  &\quad-|\{w\in\{0,1\}^{t-1}0|f(uw)=f(u(w+1))=1\}|\\
  &=|\{w\in\{0,1\}^{t-1}0|f(uw)=f(u(w+1))=0\}|\\
  &\quad-(2^{t-1}-|\{w\in\{0,1\}^{t-1}0|f(uw)=f(u(w+1))=0\}|\\
  &\quad-|\{w\in\{0,1\}^{t-1}0|f(uw)\oplus f(u(w+1))=1\}|)\\
  &=-2^{t-1}+|\{w\in\{0,1\}^{t-1}0|f(uw)\oplus f(u(w+1))=1\}|\\
  &\quad+2|\{w\in\{0,1\}^{t-1}0|f(uw)=f(u(w+1))=0\}|\\
  &=-2^{t-1}+\sum_{w\in\{0,1\}^{t-1}0}f(uw)\oplus f(u(w+1))\\
  &\quad+2\sum_{w\in\{0,1\}^{t-1}0}(\lnot f(uw))\land(\lnot f(u(w+1))).
\end{split}
\end{equation}
%Notice that there could be multiple identical elements in $G(u)$. 
%An example of $\delta(u)$ is shown in Appendix \ref{example}.
\fi


The following theorems concerning $\Delta(u)$ are useful and important.

\begin{theorem}\label{The6} Suppose function $f:\{0,1\}^n \rightarrow \{0,1\}$, satisfies that it is either constant or balanced. If $\exists$ $u\in\{0,1\}^{n-t}$ such that $|\Delta(u)|\neq 2^{t-1}$, then $f$ is balanced. %a string $s \in \{0,1\}^n$  with $s\neq 0^n$, such that $f(x) = f(y)$ if and only if $x = y$ or $x \oplus y = s$. 
%Then
%  $\forall u,v \in \{0,1\}^{n-t},S(u)=S(v)$ if and only if $u \oplus v = 0^{n-t}$ or $u \oplus v = s_1$, where $s=s_1s_2$.
\end{theorem}
\iffalse
\begin{proof}
Suppose $\exists$ $u\in\{0,1\}^{n-t}$ such that $|\Delta(u)|\neq 2^{t-1}$, then $f$ is constant. 
If $f$ is  constant, then $\forall u\in\{0,1\}^{n-t}$, $\sum_{w\in\{0,1\}^t}f(uw)=0$ or $\sum_{w\in\{0,1\}^t}f(uw)=2^t$. According to  equation $(\ref{Delta1})$ or $(\ref{Delta2})$, then we have $|\Delta(u)|= 2^{t-1}$, which is contrary to the assumption $|\Delta(u)|\neq 2^{t-1}$. Therefore, if $\exists$ $u\in\{0,1\}^{n-t}$ such that $|\Delta(u)|\neq 2^{t-1}$, then $f$ is balanced.
\end{proof}
\fi


\begin{defi}
Suppose function $f:\{0,1\}^n \rightarrow \{0,1\}$, let $G_0=\sum\nolimits_{w\in\{0,1\}^{t-1}0}|\{u\in\{0,1\}^{n-t}|f(uw)\oplus f(u(w+1))=0, f(uw)=0\}|$, $G_1=\sum\nolimits_{w\in\{0,1\}^{t-1}0}|\{u\in\{0,1\}^{n-t}|f(uw)\oplus f(u(w+1))=0, f(uw)=1\}|$.
\end{defi}


\begin{defi}
Suppose function $f:\{0,1\}^n \rightarrow \{0,1\}$, let $D=\sum\nolimits_{w\in\{0,1\}^{t-1}0}|\{u\in\{0,1\}^{n-t}|f(uw)\oplus f(u(w+1))=0\}|$ $(0\leq D\leq 2^{n-1})$. 
\end{defi}


\begin{lemma}\label{Lem1} Suppose function $f:\{0,1\}^n \rightarrow \{0,1\}$, satisfies that it is either constant or balanced, it is divided into $2^t$ subfunctions $f_w$ $(\forall u \in \{0,1\}^{n-t}, w \in \{0,1\}^{t}, f_w(u)=f(uw))$ . Then $f$ is balanced if and only if $G_0=G_1=D/2$. %Then $f$ is constant  if and only if for all $w\in\{0,1\}^t$, $|\{u\in\{0,1\}^{n-t}|f(uw)=0\}|=2^{n-t}$ or $|\{u\in\{0,1\}^{n-t}|f(uw)=1\}|=2^{n-t}$.
\end{lemma}
\iffalse
\begin{proof}
%First, we prove that $f$ is balanced if and only if $G_0=G_1=D/2$.

(\romannumeral1) $\Longleftarrow$. Suppose $G_0=G_1=D/2$ $(0\leq D\leq 2^{n-1})$, then $f$ is constant. If $f$ is constant, then $f\equiv 0$ or $f\equiv 1$, that is $\sum\nolimits_{w\in\{0,1\}^{t-1}0}|\{u\in\{0,1\}^{n-t}|f(uw)\oplus f(u(w+1))=0, f(uw)=0\}|=2^{n-1}$ or $\sum\nolimits_{w\in\{0,1\}^{t-1}0}|\{u\in\{0,1\}^{n-t}|f(uw)\oplus f(u(w+1))=0, f(uw)=1\}|=2^{n-1}$. So $G_0=2^{n-1}$ or $G_1=2^{n-1}$, which is contrary to the assumption $G_0=G_1=D/2$ $(0\leq D\leq 2^{n-1})$. Therefore, if $G_0=G_1=D/2$, then $f$ is balanced, where $D=\sum\nolimits_{w\in\{0,1\}^{t-1}0}|\{u\in\{0,1\}^{n-t}|f(uw)\oplus f(u(w+1))=0\}|$.




(\romannumeral2) $\Longrightarrow$. %If $f$ is balanced, according to the definition of DJ problem, then   $|\{x\in\{0,1\}^n|f(x)=0\}|=|\{x\in\{0,1\}^n|f(x)=1\}|=2^{n-1}$. 
Since 
\begin{equation}
\begin{split}
&D\\
=&\sum\limits_{w\in\{0,1\}^{t-1}0}|\{u\in\{0,1\}^{n-t}|f(uw)\oplus f(u(w+1))=0\}|\\=&\sum\limits_{w\in\{0,1\}^{t-1}0}|\{u\in\{0,1\}^{n-t}|f(uw)= f(u(w+1))=0\}|\\
&+\sum\limits_{w\in\{0,1\}^{t-1}0}|\{u\in\{0,1\}^{n-t}|f(uw)=f(u(w+1))\\
&\qquad\qquad\qquad=1\}|, 
\end{split}
\end{equation}
therefore
\begin{equation}
\begin{split}
&\sum\limits_{w\in\{0,1\}^{t-1}0}|\{u\in\{0,1\}^{n-t}|f(uw)\oplus f(u(w+1))=1\}|\\
=&2^{n-1}-D, 
\end{split}
\end{equation}
that is 
\begin{equation}
\begin{split}
&\sum\limits_{w\in\{0,1\}^{t-1}0}|\{u\in\{0,1\}^{n-t}|f(uw)=0, \\
&\qquad\qquad\quad f(u(w+1))=1\}|\\
&+\sum\limits_{w\in\{0,1\}^{t-1}0}|\{u\in\{0,1\}^{n-t}|f(uw)=1, \\
&\qquad\qquad\qquad f(u(w+1))=0\}|\\
=&2^{n-1}-D. 
\end{split}
\end{equation}

We have 
\begin{equation}
\begin{split}
&\sum\limits_{w\in\{0,1\}^{t-1}0}|\{u\in\{0,1\}^{n-t}|f(uw)=f(u(w+1))=0\}|\\
=&\sum\limits_{w\in\{0,1\}^{t-1}0}|\{u\in\{0,1\}^{n-t}|f(uw)=0\}|\\
&-\sum\limits_{w\in\{0,1\}^{t-1}0}|\{u\in\{0,1\}^{n-t}|f(uw)=0,\\
&\qquad\qquad\qquad f(u(w+1))=1\}|\\
=&\sum\limits_{w\in\{0,1\}^{t-1}0}|\{u\in\{0,1\}^{n-t}|f(u(w+1))=0\}|\\
&-\sum\limits_{w\in\{0,1\}^{t-1}0}|\{u\in\{0,1\}^{n-t}|f(uw)=1,\\
&\qquad\qquad\qquad f(u(w+1))=0\}|.
\end{split}
\end{equation}


Let 
\begin{equation}
\begin{split}
T_0=&\sum\limits_{w\in\{0,1\}^{t-1}0}|\{u\in\{0,1\}^{n-t}|f(uw)=0\}|\\
&-\sum\limits_{w\in\{0,1\}^{t-1}0}|\{u\in\{0,1\}^{n-t}|f(uw)=0,\\
&\qquad\qquad\qquad f(u(w+1))=1\}|.
\end{split}
\end{equation}
\begin{equation}
\begin{split}
T_1=&\sum\limits_{w\in\{0,1\}^{t-1}0}|\{u\in\{0,1\}^{n-t}|f(u(w+1))=0\}|\\
&-\sum\limits_{w\in\{0,1\}^{t-1}0}|\{u\in\{0,1\}^{n-t}|f(uw)=1,\\
&\qquad\qquad\qquad f(u(w+1))=0\}|.
\end{split}
\end{equation}

Then
\begin{equation}
\begin{split}
T_0=&T_1\\
=&\sum\limits_{w\in\{0,1\}^{t-1}0}|\{u\in\{0,1\}^{n-t}|f(uw)=\\
&\qquad\qquad\quad f(u(w+1))=0\}|.
\end{split}
\end{equation}

Then 
\begin{equation}
\begin{split}
&T_0+T_1\\
=&\sum\limits_{w\in\{0,1\}^{t-1}0}|\{u\in\{0,1\}^{n-t}|f(uw)=0\}|\\
&-\sum\limits_{w\in\{0,1\}^{t-1}0}|\{u\in\{0,1\}^{n-t}|f(uw)=0,\\
&\qquad\qquad\qquad f(u(w+1))=1\}|\\
&+\sum\limits_{w\in\{0,1\}^{t-1}0}|\{u\in\{0,1\}^{n-t}|f(u(w+1))=0\}|\\
&-\sum\limits_{w\in\{0,1\}^{t-1}0}|\{u\in\{0,1\}^{n-t}|f(uw)=1,\\
&\qquad\qquad\qquad f(u(w+1))=0\}|.\\
=&(\sum\limits_{w\in\{0,1\}^{t-1}0}|\{u\in\{0,1\}^{n-t}|f(uw)=0\}|\\
&+\sum\limits_{w\in\{0,1\}^{t-1}0}|\{u\in\{0,1\}^{n-t}|f(u(w+1))=0\}|)\\
&-(\sum\limits_{w\in\{0,1\}^{t-1}0}|\{u\in\{0,1\}^{n-t}|f(uw)=0,\\
&\qquad\qquad\qquad f(u(w+1))=1\}|\\
&+\sum\limits_{w\in\{0,1\}^{t-1}0}|\{u\in\{0,1\}^{n-t}|f(uw)=1,\\
&\qquad\qquad\qquad f(u(w+1))=0\}|)\\
=&(\sum\limits_{w\in\{0,1\}^{t-1}0}|\{u\in\{0,1\}^{n-t}|f(uw)=0\}|\\
&+\sum\limits_{w\in\{0,1\}^{t-1}0}|\{u\in\{0,1\}^{n-t}|f(u(w+1))=0\}|)\\
&-(2^{n-1}-D). 
\end{split}
\end{equation}

If $f$ is balanced, according to the definition of DJ problem, then we have $|\{x\in\{0,1\}^n|f(x)=0\}|=|\{x\in\{0,1\}^n|f(x)=1\}|=2^{n-1}$. 

So 
\begin{equation}
\begin{split}
&\sum\limits_{w\in\{0,1\}^{t-1}0}|\{u\in\{0,1\}^{n-t}|f(uw)=0\}|\\
&+\sum\limits_{w\in\{0,1\}^{t-1}0}|\{u\in\{0,1\}^{n-t}|f(u(w+1))=0\}|\\
=&|\{x\in\{0,1\}^n|f(x)=0\}|\\
=&2^{n-1}.
\end{split}
\end{equation}

Therefore, we have
\begin{equation}
\begin{split}
&T_0+T_1\\
=&(\sum\limits_{w\in\{0,1\}^{t-1}0}|\{u\in\{0,1\}^{n-t}|f(uw)=0\}|\\
&+\sum\limits_{w\in\{0,1\}^{t-1}0}|\{u\in\{0,1\}^{n-t}|f(u(w+1))=0\}|)\\
&-(2^{n-1}-D)\\
=&2^{n-1}-(2^{n-1}-D)\\
=&D, 
\end{split}
\end{equation}
then from $T_0=T_1=\sum\limits_{w\in\{0,1\}^{t-1}0}|\{u\in\{0,1\}^{n-t}|f(uw)=f(u(w+1))=0\}|$, we have $T_0=T_1=\sum\limits_{w\in\{0,1\}^{t-1}0}|\{u\in\{0,1\}^{n-t}|f(uw)=f(u(w+1))=0\}|=D/2$. 


Further, from $D=\sum\limits_{w\in\{0,1\}^{t-1}0}|\{u\in\{0,1\}^{n-t}|f(uw)= f(u(w+1))=0\}|+\sum\limits_{w\in\{0,1\}^{t-1}0}|\{u\in\{0,1\}^{n-t}|f(uw)=f(u(w+1))=1\}|$, we have 
\begin{equation}
\begin{split}
&\sum\limits_{w\in\{0,1\}^{t-1}0}|\{u\in\{0,1\}^{n-t}|f(uw)=f(u(w+1))=1\}|\\
=&D-\sum\limits_{w\in\{0,1\}^{t-1}0}|\{u\in\{0,1\}^{n-t}|f(uw)=f(u(w+1))\\
&\qquad\qquad\qquad\quad=0\}|\\
=&D-D/2\\
=&D/2.
\end{split}
\end{equation}



Since 
\begin{equation}
\begin{split}
&G_0\\
=&\sum\limits_{w\in\{0,1\}^{t-1}0}|\{u\in\{0,1\}^{n-t}|f(uw)\oplus f(u(w+1))=0,\\ &\qquad\qquad\quad f(uw)=0\}|\\
=&\sum\limits_{w\in\{0,1\}^{t-1}0}|\{u\in\{0,1\}^{n-t}|f(uw)=f(u(w+1))=0\}|.
\end{split}
\end{equation}
\begin{equation}
\begin{split}
&G_1\\
=&\sum\limits_{w\in\{0,1\}^{t-1}0}|\{u\in\{0,1\}^{n-t}|f(uw)\oplus f(u(w+1))=0,\\ &\qquad\qquad\quad f(uw)=1\}|\\
=&\sum\limits_{w\in\{0,1\}^{t-1}0}|\{u\in\{0,1\}^{n-t}|f(uw)=f(u(w+1))=1\}|.
\end{split}
\end{equation}
Therefore, we have $G_0=G_1=D/2$.

%Let $S_0=|\{u\in\{0,1\}^{n-1}|f(u0)=0\}|-|\{u\in\{0,1\}^{n-1}|f(u0)=0,f(u1)=1\}|$, $S_1=|\{u\in\{0,1\}^{n-1}|f(u1)=0\}|-|\{u\in\{0,1\}^{n-1}|f(u0)=1,f(u1)=0\}|$. 

%Then $S_0+S_1=(|\{u\in\{0,1\}^{n-1}|f(u0)=0\}|-|\{u\in\{0,1\}^{n-1}|f(u0)=0,f(u1)=1\}|)+(|\{u\in\{0,1\}^{n-1}|f(u1)=0\}|-|\{u\in\{0,1\}^{n-1}|f(u0)=1,f(u1)=0\}|)=(|\{u\in\{0,1\}^{n-1}|f(u0)=0\}|+|\{u\in\{0,1\}^{n-1}|f(u1)=0\}|)-(|\{u\in\{0,1\}^{n-1}|f(u0)=0,f(u1)=1\}|+|\{u\in\{0,1\}^{n-1}|f(u0)=1,f(u1)=0\}|)=(|\{u\in\{0,1\}^{n-1}|f(u0)=0\}|+|\{u\in\{0,1\}^{n-1}|f(u1)=0\}|)-(2^{n-1}-M)$. 




%If $f$ is balanced, according to the definition of DJ problem, then we have $|\{x\in\{0,1\}^n|f(x)=0\}|=|\{x\in\{0,1\}^n|f(x)=1\}|=2^{n-1}$. So $|\{u\in\{0,1\}^{n-1}|f(u0)=0\}|+|\{u\in\{0,1\}^{n-1}|f(u1)=0\}|=|\{x\in\{0,1\}^n|f(x)=0\}|=2^{n-1}$. Therefore, $S_0+S_1=(|\{u\in\{0,1\}^{n-1}|f(u0)=0\}|+|\{u\in\{0,1\}^{n-1}|f(u1)=0\}|)-(2^{n-1}-M)=2^{n-1}-(2^{n-1}-M)=M$. Then from $S_0=S_1=|\{u\in\{0,1\}^{n-1}|f(u0)= f(u1)=0\}|$, we have $S_0=S_1=M/2$. 

%Further, from $S_0=|\{u\in\{0,1\}^{n-1}|f(u0)=0\}|-|\{u\in\{0,1\}^{n-1}|f(u0)=0,f(u1)=1\}|=B_0$, $S_1=|\{u\in\{0,1\}^{n-1}|f(u1)=0\}|-|\{u\in\{0,1\}^{n-1}|f(u0)=1,f(u1)=0\}|=B_1$, we have $B_0=B_1=M/2$.



% According to equation $(\ref{delta1})$, then we have $\sum\nolimits_{u\in\{0,1\}^{n-t}} \delta(u)=\sum\nolimits_{u\in\{0,1\}^{n-t}} (2^t-2\sum_{w\in\{0,1\}^t}f(uw))=\sum\nolimits_{u\in\{0,1\}^{n-t}}2^t-2\sum\nolimits_{u,w\in\{0,1\}^{n-t}}f(uw)=2^n-2\sum\nolimits_{x\in\{0,1\}^{n}}f(x)=2^n-2|\{x\in\{0,1\}^n|f(x)=1\}|=0$.

%Second, we prove that $f$ is constant if and only if $|\{u\in\{0,1\}^{n-1}|f(u0)=0\}|=|\{u\in\{0,1\}^{n-1}|f(u1)=0\}|=2^{n-1}$ or $|\{u\in\{0,1\}^{n-1}|f(u0)=1\}|=|\{u\in\{0,1\}^{n-1}|f(u1)=1\}|=2^{n-1}$. 

%(3) $\Longleftarrow$. If $|\{u\in\{0,1\}^{n-1}|f(u0)=0\}|=|\{u\in\{0,1\}^{n-1}|f(u1)=0\}|=2^{n-1}$ or $|\{u\in\{0,1\}^{n-1}|f(u0)=1\}|=|\{u\in\{0,1\}^{n-1}|f(u1)=1\}|=2^{n-1}$, then $|\{x\in\{0,1\}^n|f(x)=1\}|=0$ or $2^n$, that is $f(x) \equiv 0$ or $f(x) \equiv 1$. Therefore, $f$ is constant.

%(4) $\Longrightarrow$. If $f$ is constant, according to the definition of DJ problem, then  $f(x) \equiv 0$ or $f(x) \equiv 1$. So $|\{x\in\{0,1\}^n|f(x)=1\}|=0$ or $2^n$. Therefore $|\{u\in\{0,1\}^{n-1}|f(u0)=0\}|=|\{u\in\{0,1\}^{n-1}|f(u1)=0\}|=2^{n-1}$ or $|\{u\in\{0,1\}^{n-1}|f(u0)=1\}|=|\{u\in\{0,1\}^{n-1}|f(u1)=1\}|=2^{n-1}$.

\end{proof}
\fi

\begin{theorem}\label{The7} Suppose function $f:\{0,1\}^n \rightarrow \{0,1\}$, satisfies that it is either constant or balanced, it is divided into $2^t$ subfunctions $f_w$ $(\forall u \in \{0,1\}^{n-t}, w \in \{0,1\}^{t}, f_w(u)=f(uw))$. %Then $f$ is balanced if and only if $G_0=G_1=D/2$. 
Then $f$ is balanced  if and only if $\sum_{u\in\{0,1\}^{n-t}}\Delta(u)=0$.
\end{theorem}
\iffalse
\begin{proof}
Since
\begin{equation}
\begin{split}
&\sum_{u\in\{0,1\}^{n-t}}\Delta(u)\\
 =&\sum_{u\in\{0,1\}^{n-t}}(|\{w\in\{0,1\}^{t-1}0|f(uw)=f(u(w+1))=0\}|\\
&-|\{w\in\{0,1\}^{t-1}0|f(uw)=f(u(w+1))=1\}|)\\
=&\sum_{u\in\{0,1\}^{n-t}}|\{w\in\{0,1\}^{t-1}0|f(uw)=f(u(w+1))=0\}|\\
&-\sum_{u\in\{0,1\}^{n-t}}|\{w\in\{0,1\}^{t-1}0|f(uw)=f(u(w+1))=\\
&\qquad\qquad\qquad1\}|\\
=&\sum_{w\in\{0,1\}^{t-1}0}|\{u\in\{0,1\}^{n-t}|f(uw)=f(u(w+1))=0\}|\\
&-\sum_{w\in\{0,1\}^{t-1}0}|\{u\in\{0,1\}^{n-t}|f(uw)=f(u(w+1))=\\
&\qquad\qquad\qquad1\}|\\
&=G_0-G_1,
\end{split}
\end{equation}
 therefore, according to Lemma \ref{Lem1}, we have $f$ is balanced  if and only if $\sum_{u\in\{0,1\}^{n-t}}\Delta(u)=0$.

\end{proof}
\fi


\begin{theorem}\label{The8} Suppose function $f:\{0,1\}^n \rightarrow \{0,1\}$, satisfies that it is either constant or balanced, it is divided into $2^t$ subfunctions $f_w$ $(\forall u \in \{0,1\}^{n-t}, w \in \{0,1\}^{t}, f_w(u)=f(uw))$. %Then $f$ is balanced if and only if $G_0=G_1=D/2$. 
Then $f$ is constant  if and only if for all $u \in \{0,1\}^{n-t}$, $\Delta(u) = 2^{t-1}$ or $u \in \{0,1\}^{n-t}$, $\Delta(u) = -2^{t-1}$.
\end{theorem}
\iffalse
\begin{proof}

(\romannumeral1) $\Longleftarrow$. If  for all $u \in \{0,1\}^{n-t}$, $\Delta(u) =2^{t-1}$, then according to equation (\ref{Delta1}), for all $u \in \{0,1\}^{n-t}$, we have $\sum_{w\in\{0,1\}^{t-1}0}f(uw)\oplus f(u(w+1))=0$, $\sum_{w\in\{0,1\}^{t-1}0}f(uw)\land f(u(w+1))=0$. So for all $x \in \{0,1\}^{n}$, $f(x)=0$, that is $f(x) \equiv 0$.



 If  for all $u \in \{0,1\}^{n-t}$, $\Delta(u) =-2^{t-1}$, then according to equation (\ref{Delta2}), for all $u \in \{0,1\}^{n-t}$, we have $\sum_{w\in\{0,1\}^{t-1}0}f(uw)\oplus f(u(w+1))=0$, $\sum_{w\in\{0,1\}^{t-1}0}(\lnot f(uw))\land (\lnot f(u(w+1)))=0$. So for all $x \in \{0,1\}^{n}$, $f(x)=1$, that is $f(x) \equiv 1$.
 
 Therefore, if for all $u \in \{0,1\}^{n-t}$, $\Delta(u) =2^{t-1}$ or for all $u \in \{0,1\}^{n-t}$, $\Delta(u) =-2^{t-1}$, then $f$ is constant.

(\romannumeral2) $\Longrightarrow$. If $f$ is constant, according to the definition of DJ problem, then we have $f(x) \equiv 0$ or $f(x) \equiv 1$. 


If $f(x) \equiv 0$, then for all $x \in \{0,1\}^{n}$, $f(x)=0$. So for all $u \in \{0,1\}^{n-t}$, we have $\sum_{w\in\{0,1\}^{t-1}0}f(uw)\oplus f(u(w+1))=0$, $\sum_{w\in\{0,1\}^{t-1}0}f(uw)\land f(u(w+1))=0$.  According to equation (\ref{Delta1}), for all $u \in \{0,1\}^{n-t}$, we have $\Delta(u) =2^{t-1}$.


If $f(x) \equiv 1$, then for all $x \in \{0,1\}^{n}$, $f(x)=1$. So for all $u \in \{0,1\}^{n-t}$, we have $\sum_{w\in\{0,1\}^{t-1}0}f(uw)\oplus f(u(w+1))=0$, $\sum_{w\in\{0,1\}^{t-1}0}(\lnot f(uw))\land (\lnot f(u(w+1)))=0$.  According to equation (\ref{Delta2}), for all $u \in \{0,1\}^{n-t}$, we have $\Delta(u) =-2^{t-1}$.

 Therefore, if $f$ is constant, then for all $u \in \{0,1\}^{n-t}$, $\Delta(u) =2^{t-1}$ or for all $u \in \{0,1\}^{n-t}$, $\Delta(u) =-2^{t-1}$.
 
\iffalse
(\romannumeral1) $\Longleftarrow$. If $\sum\nolimits_{u\in\{0,1\}^{n-t}} \Delta(u) =2^{n-1}$, according to equation $(\ref{Delta1})$, then we have 
\begin{equation}
\begin{split}
&\sum\nolimits_{u\in\{0,1\}^{n-t}} \Delta(u)\\
=&\sum\nolimits_{u\in\{0,1\}^{n-t}} (2^{t-1}-\sum_{w\in\{0,1\}^{t-1}0}f(uw)\oplus f(u(w+1))\\
  &-2\sum_{w\in\{0,1\}^{t-1}0}f(uw)\land f(u(w+1)))\\
=&2^{n-1},
\end{split}
\end{equation}
and since for all $u \in \{0,1\}^{n-t}$, $-2^{t-1}\leq\Delta(u)\leq 2^{t-1}$, therefore for all $u \in \{0,1\}^{n-t}$, $\Delta(u)=2^{t-1}$.

And since  for all $u \in \{0,1\}^{n-t}$, we have
\begin{equation}
\begin{split}
&\Delta(u)\\
=&2^{t-1}-\sum_{w\in\{0,1\}^{t-1}0}f(uw)\oplus f(u(w+1))\\
  &-2\sum_{w\in\{0,1\}^{t-1}0}f(uw)\land f(u(w+1)),\\
\end{split}
\end{equation}
therefor  for all $u \in \{0,1\}^{n-t}$, we have
\begin{equation}
\begin{split}
\sum_{w\in\{0,1\}^{t-1}0}f(uw)\oplus f(u(w+1))=0.
\end{split}
\end{equation}
\begin{equation}
\begin{split}
\sum_{w\in\{0,1\}^{t-1}0}f(uw)\land f(u(w+1))=0.
\end{split}
\end{equation}

Therefore, for all $u \in \{0,1\}^{n-t}$, we have
\begin{equation}
\begin{split}
&|\{w\in\{0,1\}^{t-1}0|f(uw)=f(u(w+1))=0\}|\\
=&2^{t-1}-(|\{w\in\{0,1\}^{t-1}0|f(uw)\oplus f(u(w+1))=1\}|\\
&+|\{w\in\{0,1\}^{t-1}0|f(uw)=f(u(w+1))=1\}|)\\
=&2^{t-1}-(\sum_{w\in\{0,1\}^{t-1}0}f(uw)\oplus f(u(w+1))\\
&+\sum_{w\in\{0,1\}^{t-1}0}f(uw)\land f(u(w+1)))\\
=&2^{t-1}.
\end{split}
\end{equation}

So $|\{x\in\{0,1\}^n|f(x)=1\}|=0$, that is $f(x) \equiv 0$.

If $\sum\nolimits_{u\in\{0,1\}^{n-t}} \Delta(u) =-2^{n-1}$, according to equation $(\ref{Delta2})$, then we have 
\begin{equation}
\begin{split}
&\sum\nolimits_{u\in\{0,1\}^{n-t}} \Delta(u)\\
  =&\sum\nolimits_{u\in\{0,1\}^{n-t}}(-2^{t-1}+\sum_{w\in\{0,1\}^{t-1}0}f(uw)\oplus f(u(w+1))\\
  &+2\sum_{w\in\{0,1\}^{t-1}0}(\lnot f(uw))\land(\lnot f(u(w+1))))\\
=&-2^{n-1},
\end{split}
\end{equation}
and since for all $u \in \{0,1\}^{n-t}$, $-2^{t-1}\leq\Delta(u)\leq 2^{t-1}$, therefore for all $u \in \{0,1\}^{n-t}$, $\Delta(u)=-2^{t-1}$.

And since  for all $u \in \{0,1\}^{n-t}$, we have
\begin{equation}
\begin{split}
&\Delta(u)\\
=&-2^{t-1}+\sum_{w\in\{0,1\}^{t-1}0}f(uw)\oplus f(u(w+1))\\
  &+2\sum_{w\in\{0,1\}^{t-1}0}(\lnot f(uw))\land(\lnot f(u(w+1))),\\
\end{split}
\end{equation}
therefor  for all $u \in \{0,1\}^{n-t}$, we have
\begin{equation}
\begin{split}
\sum_{w\in\{0,1\}^{t-1}0}f(uw)\oplus f(u(w+1))=0.
\end{split}
\end{equation}
\begin{equation}
\begin{split}
\sum_{w\in\{0,1\}^{t-1}0}(\lnot f(uw))\land(\lnot f(u(w+1))=0.
\end{split}
\end{equation}

Therefore, for all $u \in \{0,1\}^{n-t}$, we have
\begin{equation}
\begin{split}
&|\{w\in\{0,1\}^{t-1}0|f(uw)=f(u(w+1))=1\}|\\
=&2^{t-1}-(|\{w\in\{0,1\}^{t-1}0|f(uw)\oplus f(u(w+1))=1\}|\\
&+|\{w\in\{0,1\}^{t-1}0|f(uw)=f(u(w+1))=0\}|)\\
=&2^{t-1}-(\sum_{w\in\{0,1\}^{t-1}0}f(uw)\oplus f(u(w+1))\\
&+\sum_{w\in\{0,1\}^{t-1}0}(\lnot f(uw))\land(\lnot f(u(w+1)))\\
=&2^{t-1}.
\end{split}
\end{equation}

So $|\{x\in\{0,1\}^n|f(x)=1\}|=2^n$, that is $f(x) \equiv 1$.

Therefore, if $\sum\nolimits_{u\in\{0,1\}^{n-t}} \Delta(u) =\pm 2^{n-1}$, then $f$ is constant.

(\romannumeral2) $\Longrightarrow$. If $f$ is constant, according to the definition of DJ problem, then we have $f(x) \equiv 0$ or $f(x) \equiv 1$. 

If $f(x) \equiv 0$, then $|\{x\in\{0,1\}^n|f(x)=0\}|=2^n$. Therefore, for all $u \in \{0,1\}^{n-t}$, we have $|\{w\in\{0,1\}^{t-1}0|f(uw)=f(u(w+1))=0\}|=2^{t-1}$ and $|\{w\in\{0,1\}^{t-1}0|f(uw)=f(u(w+1))=1\}|=0$.


So according to the definition of $\Delta(u)$,we have
\begin{equation}
\begin{split}
&\sum\limits_{u\in\{0,1\}^{n-t}} \Delta(u)\\
=&\sum\limits_{u\in\{0,1\}^{n-t}} (|\{w\in\{0,1\}^{t-1}0|f(uw)=f(u(w+1))=0\}|\\
&\quad-|\{w\in\{0,1\}^{t-1}0|f(uw)=f(u(w+1))=1\}|)\\
=&\sum\limits_{u\in\{0,1\}^{n-t}}2^{t-1}\\
=&2^{n-1}.
\end{split}
\end{equation}



If $f(x) \equiv 1$, then $|\{x\in\{0,1\}^n|f(x)=1\}|=2^n$. Therefore, for all $u \in \{0,1\}^{n-t}$, we have $|\{w\in\{0,1\}^{t-1}0|f(uw)=f(u(w+1))=0\}|=0$ and $|\{w\in\{0,1\}^{t-1}0|f(uw)=f(u(w+1))=1\}|=2^{t-1}$.


So according to the definition of $\Delta(u)$,we have
\begin{equation}
\begin{split}
&\sum\limits_{u\in\{0,1\}^{n-t}} \Delta(u)\\
=&\sum\limits_{u\in\{0,1\}^{n-t}} (|\{w\in\{0,1\}^{t-1}0|f(uw)=f(u(w+1))=0\}|\\
&\quad-|\{w\in\{0,1\}^{t-1}0|f(uw)=f(u(w+1))=1\}|)\\
=&\sum\limits_{u\in\{0,1\}^{n-t}}-2^{t-1}\\
=&-2^{n-1}.
\end{split}
\end{equation}

Therefore, if $f$ is constant, then $\sum\nolimits_{u\in\{0,1\}^{n-t}} \Delta(u) =\pm 2^{n-1}$.

\fi

\end{proof}
\fi




\fi

%\section{Distributed quantum algorithm for  DJ problem}\label{Sec4}






\section{Characterizations of DJ problem in distributed scenario} \label{sec:Characterization of the structure of DJ problem in distributed scenario}


%\subsection{Structure of  DJ problem in  distributed scenario for  designing  Algorithm \ref{algorithm3}} \label{sec:Structure of  DJ problem in  distributed scenario for  designing  Algorithm 1}


In this section, we characterize the essential structure and features of  DJ problem in  distributed scenario.
 Intuitively, the structure of  DJ problem in distributed scenario can be represented from its corresponding structure table, and we construct   related examples in \ref{Examples of the structure of  DJ problem in  distributed scenario}.


%The following Theorem \ref{The3} provides a sufficient condition for determining whether or not $f$ is balanced. 
%, which can be used to ensure the correctness of Algorithm \ref{algorithm3} after the first measurement.

%In the following, we first describe %Theorem \ref{The3} and 
%Theorem \ref{The4}, which portrays the structure of  DJ problem in  distributed scenario with two subfunctions.


%%%%%%以下为原来定理1简介段落%%%%%%

%The  Theorem \ref{The3} below provides a sufficient condition for determining whether a given function $f$ of  DJ problem is balanced by the joint operation of  functions $f_0$ and $f_1$, where $f_0$ and $f_1$ are subfunctions of $f$.

%%%%%%以上为原来定理1简介段落%%%%%%



%%%%以下为原来定理1%%%%%

\iffalse
\begin{theorem}\label{The3} Suppose function $f:\{0,1\}^n \rightarrow \{0,1\}$, satisfies that it is either constant or balanced, and it is divided into subfunctions $f_0$ and $f_1$. If $\exists$ $u\in\{0,1\}^{n-1}$ such that $f(u0)\oplus f(u1)=1$, then $f$ is balanced.
\end{theorem}
\begin{proof}
%We prove the theorem by contradiction. 
Suppose $\exists$ $u\in\{0,1\}^{n-1}$ such that $f(u0)\oplus f(u1)=1$, but $f$ is constant. 

From $f$ being  constant, it follows that $\forall u\in\{0,1\}^{n-1}$, 
\begin{equation}
f(u0)=f(u1)=0 
\end{equation}
or
\begin{equation}
f(u0)=f(u1)=1. 
\end{equation}

So 
\begin{equation}
f(u0)\oplus f(u1)=0, 
\end{equation}
which is contrary to the assumption $f(u0)\oplus f(u1)=1$. 

Therefore, if $\exists$ $u\in\{0,1\}^{n-1}$ such that $f(u0)\oplus f(u1)=1$, then $f$ is balanced.
\end{proof}


Intuitively, Theorem \ref{The3} states  that after decomposing a given function $f$ of DJ problem into two subfunctions $f_0$ and $f_1$, if there exist inputs such that the values of $f_0$ and $f_1$ are different, then $f$ is necessarily balanced by the definition of  DJ problem.

\fi

%%%%以上为原来定理1%%%%%



%We first describe Theorem \ref{The4}, which explores the hidden structure and features inherent in  DJ problem in distributed scenario with two subfunctions. This is a new class of structural characterizations of  DJ problem. 

In the interest of simplicity, we give  a number of  notations.
Suppose Boolean function $f:\{0,1\}^n \rightarrow \{0,1\}$, $f_0:\{0,1\}^{n-1} \rightarrow \{0,1\}$, $f_1:\{0,1\}^{n-1} \rightarrow \{0,1\}$,  where $f_0(u)=f(u0)$, $f_1(u)=f(u1)$,  $u\in\{0,1\}^{n-1}$, denote by 

\begin{align}
C_{00}=&|\{u|f_0(u)=0\}|.\\
C_{01}=&|\{u|f_0(u)=1\}|.\\
C_{10}=&|\{u|f_1(u)=0\}|.\\
C_{11}=&|\{u|f_1(u)=1\}|.\\
B_{00}=&|\{u|f_0(u)=0, f_1(u)=0\}|.\\
B_{01}=&|\{u|f_0(u)=0, f_1(u)=1\}|.\\
B_{10}=&|\{u|f_0(u)=1, f_1(u)=0\}|.\\
B_{11}=&|\{u|f_0(u)=1, f_1(u)=1\}|.\\
 M=&|\{u|f_0(u)\oplus f_1(u)=0\}|.
\end{align}

It is clear that 
\begin{equation}\label{M=B_0+B_1}
M=B_{00}+B_{11}
\end{equation}
and $0\leq M\leq 2^{n-1}$.

%The following Theorem \ref{The4} provides a sufficient and necessary condition to determine that $f$ is balanced. 
%, which can be used to ensure the correctness of Algorithm \ref{algorithm3} after the second measurement.


Theorem \ref{The4} below provides a sufficient and necessary condition for determining whether a given Boolean function $f$ of  DJ problem is constant or balanced.

\begin{theorem}\label{The4} Suppose Boolean  function $f:\{0,1\}^n \rightarrow \{0,1\}$, satisfies that it is either constant or balanced, and it is divided into subfunctions $f_0$ and $f_1$. Then:
\begin{enumerate}[(1)]
\item $f$ is constant  if and only if $C_{00}=C_{10}=2^{n-1}$
 or $C_{01}=C_{11}=2^{n-1}$;
%Then $f$ is constant  if and only if $|\{u\in\{0,1\}^{n-1}|f(u0)=0\}|=|\{u\in\{0,1\}^{n-1}|f(u1)=0\}|=2^{n-1}$ or $|\{u\in\{0,1\}^{n-1}|f(u0)=1\}|=|\{u\in\{0,1\}^{n-1}|f(u1)=1\}|=2^{n-1}$. 
\item $f$ is balanced if and only if $B_{00}=B_{11}=M/2$. 
 \end{enumerate}
\end{theorem}
\begin{proof}

Firstly, we prove that $f$ is constant  if and only if $C_{00}=C_{10}=2^{n-1}$
 or $C_{01}=C_{11}=2^{n-1}$. 
%\begin{equation}
%|\{u\in\{0,1\}^{n-1}|f(u0)=0\}|=|\{u\in\{0,1\}^{n-1}|f(u1)=0\}|=2^{n-1}
%\end{equation}
% or 
% \begin{equation}
% |\{u\in\{0,1\}^{n-1}|f(u0)=1\}|=|\{u\in\{0,1\}^{n-1}|f(u1)=1\}|=2^{n-1}. 
% \end{equation}

(\romannumeral1) $\Longleftarrow$. If $C_{00}=C_{10}=2^{n-1}$
 or $C_{01}=C_{11}=2^{n-1}$,
%\begin{equation}
%|\{u\in\{0,1\}^{n-1}|f(u0)=0\}|=|\{u\in\{0,1\}^{n-1}|f(u1)=0\}|
%=2^{n-1} 
%\end{equation}
%or 
%\begin{equation}
%|\{u\in\{0,1\}^{n-1}|f(u0)=1\}|=|\{u\in\{0,1\}^{n-1}|f(u1)=1\}|
%=2^{n-1}, 
%\end{equation}
then 
\begin{equation}
|\{x|f(x)=1\}|=0
\end{equation}
or 
\begin{equation}
|\{x|f(x)=1\}|=2^n,
\end{equation}
that is $f(x) \equiv 0$ or $f(x) \equiv 1$. Therefore, $f$ is constant.

(\romannumeral2) $\Longrightarrow$. If $f$ is constant, then  $f(x) \equiv 0$ or $f(x) \equiv 1$. 

So
\begin{equation}
 |\{x|f(x)=1\}|=0
\end{equation}
or
\begin{equation}
 |\{x|f(x)=1\}|=2^n. 
\end{equation}

Therefore 
\begin{equation}
C_{00}=C_{10}=2^{n-1}
\end{equation}
or 
\begin{equation}
C_{01}=C_{11}=2^{n-1}.
\end{equation}



Secondly, we prove that $f$ is balanced if and only if $B_0=B_1=M/2$.

(\romannumeral3) 
$\Longleftarrow$. %We prove the theorem by contradiction. 
Suppose $B_{00}=B_{11}=M/2$ $(0\leq M\leq 2^{n-1})$, then $f$ is constant. 

If $f$ is constant, then $f\equiv 0$ or $f\equiv 1$. % that is 
%\begin{equation}
%|\{u\in\{0,1\}^{n-1}|f(u0)=f(u1)=0\}|=2^{n-1}
%\end{equation}
%or
%\begin{equation}
%|\{u\in\{0,1\}^{n-1}|f(u0)=f(u1)=1\}|=2^{n-1}. 
%\end{equation}
So $B_{00}=2^{n-1}$ or $B_{11}=2^{n-1}$, which is contrary to the assumption $B_{00}=B_{11}=M/2$ $(0\leq M\leq 2^{n-1})$. 

Therefore, if $B_{00}=B_{11}=M/2$, then $f$ is balanced.%, where $M=|\{u\in\{0,1\}^{n-1}|f(u0)\oplus f(u1)=0\}|$.


(\romannumeral4) $\Longrightarrow$. %If $f$ is balanced, according to the definition of DJ problem, then   $|\{x\in\{0,1\}^n|f(x)=0\}|=|\{x\in\{0,1\}^n|f(x)=1\}|=2^{n-1}$. 
Since 
\begin{equation}\label{thm4equ6}
\begin{split}
&B_{00}
+B_{11}
+B_{01}+B_{10}\\
=&M+B_{01}+B_{10}\\
=&2^{n-1},
 \end{split}
\end{equation}
 we have %$|\{u\in\{0,1\}^{n-1}|f(u0)\oplus f(u1)=1\}|=2^{n-1}-M$, that is 
\begin{equation}\label{thm4equ3}
B_{01}+B_{10}=2^{n-1}-M. 
\end{equation}

%We have 
%\begin{equation}
%\begin{split}
% &|\{u\in\{0,1\}^{n-1}|f(u0)= f(u1)=0\}|\\
% =&|\{u\in\{0,1\}^{n-1}|f(u0)=0\}|-|\{u\in\{0,1\}^{n-1}|f(u0)=0,f(u1)=1\}|\\
 %=&|\{u\in\{0,1\}^{n-1}|f(u1)=0\}|-|\{u\in\{0,1\}^{n-1}|f(u0)=1,f(u1)=0\}|. 
%\end{split}
%\end{equation}

Denote
 \begin{equation}\label{S_0_eq}
S_0=C_{00}-B_{01},%|\{u\in\{0,1\}^{n-1}|f(u0)=0\}|-|\{u\in\{0,1\}^{n-1}|f(u0)=0,f(u1)=1\}|,
\end{equation}
\begin{equation}\label{S_1_eq}
S_1=C_{10}-B_{10}.%|\{u\in\{0,1\}^{n-1}|f(u1)=0\}|-|\{u\in\{0,1\}^{n-1}|f(u0)=1,f(u1)=0\}|. 
\end{equation}

Then 
\begin{equation}\label{thm4equ4}
%\begin{split}
S_0=S_1%\\
%=&|\{u\in\{0,1\}^{n-1}|f(u0)= f(u1)=0\}|\\
=B_{00}.
%\end{split}
\end{equation}

According to equation (\ref{thm4equ3}),equation (\ref{S_0_eq}) and equation (\ref{S_1_eq}), we have
\begin{equation}\label{thm2_eq_S0+S1}
\begin{split}
S_0+S_1
%=&\left(|\{u\in\{0,1\}^{n-1}|f(u0)=0\}|-|\{u\in\{0,1\}^{n-1}|f(u0)=0,f(u1)=1\}|\right)\\
%&+\left(|\{u\in\{0,1\}^{n-1}|f(u1)=0\}|-|\{u\in\{0,1\}^{n-1}|f(u0)=1,f(u1)=0\}|\right)\\
=&C_{00}+C_{10}-(B_{01}+B_{10})\\
=&C_{00}+C_{10}-\left(2^{n-1}-M\right).
%=&|\{u\in\{0,1\}^{n-1}|f(u0)=0\}|+|\{u\in\{0,1\}^{n-1}|f(u1)=0\}|\\&-\left(|\{u\in\{0,1\}^{n-1}|f(u0)=0,f(u1)=1\}|+|\{u\in\{0,1\}^{n-1}|f(u0)=1,f(u1)=0\}|\right)\\
%=&|\{u\in\{0,1\}^{n-1}|f(u0)=0\}|+|\{u\in\{0,1\}^{n-1}|f(u1)=0\}|-\left(2^{n-1}-M\right). 
\end{split}
\end{equation}

If $f$ is balanced, %according to the definition of DJ problem, then we have 
then
%\begin{equation}
%|\{x\in\{0,1\}^n|f(x)=0\}|=|\{x\in\{0,1\}^n|f(x)=1\}|
%=2^{n-1}. 
%\end{equation}
%So 
\begin{equation}\label{thm2_eq_S0+S1_f(x)=0}
%\begin{split}
%&|\{u\in\{0,1\}^{n-1}|f(u0)=0\}|+|\{u\in\{0,1\}^{n-1}|f(u1)=0\}|\\
%=&|\{x\in\{0,1\}^n|f(x)=0\}|\\
C_{00}+C_{10}=2^{n-1}. 
%\end{split}
\end{equation}

Therefore, by equation (\ref{thm2_eq_S0+S1}) and equation (\ref{thm2_eq_S0+S1_f(x)=0}), we have
\begin{equation}\label{thm4equ5}
\begin{split}
S_0+S_1
%=&\left(|\{u\in\{0,1\}^{n-1}|f(u0)=0\}|+|\{u\in\{0,1\}^{n-1}|f(u1)=0\}|\right)-\left(2^{n-1}-M\right)\\
=&2^{n-1}-\left(2^{n-1}-M\right)\\
=&M.
\end{split}
\end{equation}

According to equation (\ref{thm4equ4}) and equation (\ref{thm4equ5}), we have
%then from $S_0=S_1=|\{u\in\{0,1\}^{n-1}|f(u0)= f(u1)=0\}|$, we have 
\begin{equation}\label{thm4equ1}
S_0=S_1
=B_{00}
=M/2. 
\end{equation}

%Since
%\begin{equation}\label{M=B_0+B_1}
%M=B_0+B_1
%\end{equation}

With equation  (\ref{M=B_0+B_1}) and equation (\ref{thm4equ1}), we have
%Further, from $M=|\{u\in\{0,1\}^{n-1}|f(u0)= f(u1)=0\}|+|\{u\in\{0,1\}^{n-1}|f(u0)= f(u1)=1\}|$, we have 
\begin{equation}\label{thm4equ2}
\begin{split}
B_{11}=&M-B_{00}\\
=&M-M/2\\
=&M/2.
\end{split}
\end{equation}

%Since 
%\begin{equation}
%B_0=|\{u\in\{0,1\}^{n-1}|f(u0)\oplus f(u1)=0, f(u0)=0\}|
%=|\{u\in\{0,1\}^{n-1}|f(u0)= f(u1)=0\}|,
%\end{equation}
%\begin{equation}
%B_1=|\{u\in\{0,1\}^{n-1}|f(u0)\oplus f(u1)=0, f(u0)=1\}|\\
%=|\{u\in\{0,1\}^{n-1}|f(u0)= f(u1)=1\}|,
%\end{equation}
Combining equation (\ref{thm4equ1}) with equation (\ref{thm4equ2}), we have $B_{00}=B_{11}=M/2$.

%$S_0=|\{u\in\{0,1\}^{n-1}|f(u0)=0\}|-|\{u\in\{0,1\}^{n-1}|f(u0)=0,f(u1)=1\}|=B_0$, $S_1=|\{u\in\{0,1\}^{n-1}|f(u1)=0\}|-|\{u\in\{0,1\}^{n-1}|f(u0)=1,f(u1)=0\}|=B_1$, we have $B_0=B_1=M/2$.



% According to equation $(\ref{delta1})$, then we have $\sum\nolimits_{u\in\{0,1\}^{n-t}} \delta(u)=\sum\nolimits_{u\in\{0,1\}^{n-t}} (2^t-2\sum_{w\in\{0,1\}^t}f(uw))=\sum\nolimits_{u\in\{0,1\}^{n-t}}2^t-2\sum\nolimits_{u,w\in\{0,1\}^{n-t}}f(uw)=2^n-2\sum\nolimits_{x\in\{0,1\}^{n}}f(x)=2^n-2|\{x\in\{0,1\}^n|f(x)=1\}|=0$.

%Second, we prove that $f$ is constant if and only if $|\{u\in\{0,1\}^{n-1}|f(u0)=0\}|=|\{u\in\{0,1\}^{n-1}|f(u1)=0\}|=2^{n-1}$ or $|\{u\in\{0,1\}^{n-1}|f(u0)=1\}|=|\{u\in\{0,1\}^{n-1}|f(u1)=1\}|=2^{n-1}$. 

%(3) $\Longleftarrow$. If $|\{u\in\{0,1\}^{n-1}|f(u0)=0\}|=|\{u\in\{0,1\}^{n-1}|f(u1)=0\}|=2^{n-1}$ or $|\{u\in\{0,1\}^{n-1}|f(u0)=1\}|=|\{u\in\{0,1\}^{n-1}|f(u1)=1\}|=2^{n-1}$, then $|\{x\in\{0,1\}^n|f(x)=1\}|=0$ or $2^n$, that is $f(x) \equiv 0$ or $f(x) \equiv 1$. Therefore, $f$ is constant.

%(4) $\Longrightarrow$. If $f$ is constant, according to the definition of DJ problem, then  $f(x) \equiv 0$ or $f(x) \equiv 1$. So $|\{x\in\{0,1\}^n|f(x)=1\}|=0$ or $2^n$. Therefore $|\{u\in\{0,1\}^{n-1}|f(u0)=0\}|=|\{u\in\{0,1\}^{n-1}|f(u1)=0\}|=2^{n-1}$ or $|\{u\in\{0,1\}^{n-1}|f(u0)=1\}|=|\{u\in\{0,1\}^{n-1}|f(u1)=1\}|=2^{n-1}$.








\end{proof}


\iffalse


The following Theorem \ref{The5} provides a sufficient and necessary condition to determine whether or not $f$ is constant, which can be used to ensure the correctness of Algorithm \ref{algorithm3} after the second measurement.

\begin{theorem}\label{The5} Suppose function $f:\{0,1\}^n \rightarrow \{0,1\}$, satisfies that it is either constant or balanced, and it is divided into subfunctions $f_0$ and $f_1$. Then $f$ is constant  if and only if 
\begin{equation}
|\{u\in\{0,1\}^{n-1}|f(u0)=0\}|=|\{u\in\{0,1\}^{n-1}|f(u1)=0\}|=2^{n-1}
\end{equation}
 or 
 \begin{equation}
 |\{u\in\{0,1\}^{n-1}|f(u0)=1\}|=|\{u\in\{0,1\}^{n-1}|f(u1)=1\}|=2^{n-1}. 
 \end{equation}
\end{theorem}
\begin{proof}
%Second, we prove that $f$ is constant if and only if $|\{u\in\{0,1\}^{n-1}|f(u0)=0\}|=|\{u\in\{0,1\}^{n-1}|f(u1)=0\}|=2^{n-1}$ or $|\{u\in\{0,1\}^{n-1}|f(u0)=1\}|=|\{u\in\{0,1\}^{n-1}|f(u1)=1\}|=2^{n-1}$. 

(\romannumeral1) $\Longleftarrow$. If 
\begin{equation}
|\{u\in\{0,1\}^{n-1}|f(u0)=0\}|=|\{u\in\{0,1\}^{n-1}|f(u1)=0\}|
=2^{n-1} 
\end{equation}
or 
\begin{equation}
|\{u\in\{0,1\}^{n-1}|f(u0)=1\}|=|\{u\in\{0,1\}^{n-1}|f(u1)=1\}|
=2^{n-1}, 
\end{equation}
then 
\begin{equation}
|\{x\in\{0,1\}^n|f(x)=1\}|=0
\end{equation}
or 
\begin{equation}
|\{x\in\{0,1\}^n|f(x)=1\}|=2^n,
\end{equation}
that is $f(x) \equiv 0$ or $f(x) \equiv 1$. Therefore, $f$ is constant.

(\romannumeral2) $\Longrightarrow$. If $f$ is constant, according to the definition of DJ problem, then  $f(x) \equiv 0$ or $f(x) \equiv 1$. 

So
\begin{equation}
 |\{x\in\{0,1\}^n|f(x)=1\}|=0
\end{equation}
or
\begin{equation}
 |\{x\in\{0,1\}^n|f(x)=1\}|=2^n. 
\end{equation}

Therefore 
\begin{equation}
|\{u\in\{0,1\}^{n-1}|f(u0)=0\}|=|\{u\in\{0,1\}^{n-1}|f(u1)=0\}|
=2^{n-1}
\end{equation}
or 
\begin{equation}
|\{u\in\{0,1\}^{n-1}|f(u0)=1\}|=|\{u\in\{0,1\}^{n-1}|f(u1)=1\}|
=2^{n-1}.
\end{equation}
\end{proof}

\fi


%\subsection{Structure of  DJ problem in  distributed scenario for  designing  Algorithm \ref{algorithm2}} \label{sec:Structure of  DJ problem in  distributed scenario for  designing  Algorithm 2}


In light of Theorem \ref{The4},  we have   Corollary \ref{Cor3} below, which provides a  sufficient  condition for determining whether a given Boolean function $f$ of  DJ problem is balanced by  the xor value of the subfunctions $f_0$ and $f_1$.


%From Theorem \ref{The4} we have the following  Corollary \ref{Cor3}.




\begin{corollary}\label{Cor3} Suppose Boolean function $f:\{0,1\}^n \rightarrow \{0,1\}$, satisfies that it is either constant or balanced, and it is divided into subfunctions $f_0$ and $f_1$. If $\exists$ $u\in\{0,1\}^{n-1}$ such that $f_0(u)\oplus f_1(u)=1$, then $f$ is balanced.
\end{corollary}



%Intuitively, Corollary \ref{Cor3} states  that after decomposing a given function $f$ of DJ problem into two subfunctions $f_0$ and $f_1$, if there exist inputs such that the values of $f_0$ and $f_1$ are different, then $f$ is necessarily balanced by the definition of  DJ problem.
















In the following, we  describe %Theorem \ref{The1} and 
Theorem \ref{The2}, which characterizes the basic structure of  DJ problem in  distributed scenario with multiple subfunctions. Also, we give some  notations in order to describe its procedure of proof more clearly.






%In the following, we describe some  theorems  characterizing the essential structure of the DJ problem in distributed scenario and are used in the design and correctness analysis of Algorithm \ref{algorithm2} .






%\begin{defi}\label{deltadef}
Suppose Boolean function $f:\{0,1\}^n \rightarrow \{0,1\}$, $f_w:\{0,1\}^{n-t} \rightarrow \{0,1\}$, where $f_w(u)=f(uw)$, $u\in\{0,1\}^{n-t}$, $w\in\{0,1\}^t$. For all $u \in \{0,1\}^{n-t}$, denote
\begin{equation}\label{deltadef}
 \delta(u) = |\{w|f_w(u)=0\}|-|\{w|f_w(u)=1\}|.
 \end{equation}
 
 %=2^t-|\{w\in\{0,1\}^t|f(uw)=1\}|-|\{w\in\{0,1\}^t|f(uw)=1\}|=2^t-2|\{w\in\{0,1\}^t|f(uw)=1\}|$.
%\end{defi}




According to  equation (\ref{deltadef}) %of $\delta(u)$
, we have
\begin{equation}\label{delta1}
\begin{split}
\delta(u)=&|\{w|f_w(u)=0\}|-|\{w|f_w(u)=1\}|\\
  =&2^t-|\{w|f_w(u)=1\}|-|\{w|f_w(u)=1\}|\\
  =&2^t-2|\{w|f_w(u)=1\}|\\ 
  =&2^t-2\sum_{w\in\{0,1\}^t}f_w(u).
\end{split}
\end{equation}

%Notice that there could be multiple identical elements in $G(u)$. 
%An example of $\delta(u)$ is shown in Appendix \ref{example}.



\iffalse

The  Theorem \ref{The1} below provides a  sufficient  condition for determining whether a given function $f$ of  DJ problem is balanced by the absolute value of $\delta(u)$.

\begin{theorem}\label{The1} Suppose function $f:\{0,1\}^n \rightarrow \{0,1\}$, satisfies that it is either constant or balanced, and it is divided into $2^t$ subfunctions $f_w$ $(\forall u \in \{0,1\}^{n-t}, w \in \{0,1\}^{t}, f_w(u)=f(uw))$. If $\exists$ $u\in\{0,1\}^{n-t}$ such that $|\delta(u)|\neq 2^t$, then $f$ is balanced. %a string $s \in \{0,1\}^n$  with $s\neq 0^n$, such that $f(x) = f(y)$ if and only if $x = y$ or $x \oplus y = s$. 
%Then
%  $\forall u,v \in \{0,1\}^{n-t},S(u)=S(v)$ if and only if $u \oplus v = 0^{n-t}$ or $u \oplus v = s_1$, where $s=s_1s_2$.
\end{theorem}
\begin{proof} %We prove the theorem by contradiction.
Suppose $\exists$ $u\in\{0,1\}^{n-t}$ such that $|\delta(u)|\neq 2^t$, but $f$ is constant. 

 From $f$ being  constant, it follows that $\forall u\in\{0,1\}^{n-t}$, 
 \begin{equation}
 \sum_{w\in\{0,1\}^t}f(uw)=0
 \end{equation}
 or 
 \begin{equation}
 \sum_{w\in\{0,1\}^t}f(uw)=2^t. 
 \end{equation}
 
 According to  equation $(\ref{delta1})$,  we have $|\delta(u)|= 2^t$, which is contrary to the assumption $|\delta(u)|\neq 2^t$. 
 
 Therefore, if $\exists$ $u\in\{0,1\}^{n-t}$ such that $|\delta(u)|\neq 2^t$, then $f$ is balanced.
\end{proof}


Intuitively, Theorem \ref{The1}  states that a function $f$ of  DJ problem is decided to be balanced if there is some input for which the values of multiple subfunctions are not simultaneously $0$ or simultaneously $1$.


\fi

  Theorem \ref{The2} below provides a sufficient and necessary condition for determining whether a given Boolean function $f$ of  DJ problem is constant or balanced by means of $\delta(u)$.



%The following Theorem \ref{The2} provides a sufficient and necessary condition for $\delta(u)$ to determine whether $f$ is constant or balanced. 
%, which can be used to ensure the correctness of Algorithm \ref{algorithm2} after the second measurement.

\begin{theorem}\label{The2} Suppose Boolean function $f:\{0,1\}^n \rightarrow \{0,1\}$, satisfies that it is either constant or balanced, and it is divided into $2^t$ subfunctions $f_w$ $(\forall u \in \{0,1\}^{n-t}, w \in \{0,1\}^{t}, f_w(u)=f(uw))$. Then: 
\begin{enumerate}[(1)]
\item $f$ is constant  if and only if for all $u \in \{0,1\}^{n-t}$, $\delta(u) =2^t$ or for all $u \in \{0,1\}^{n-t}$, $\delta(u) =-2^t$; 
\item $f$ is balanced if and only if $\sum\nolimits_{u\in\{0,1\}^{n-t}} \delta(u) = 0$. %Then $f$ is constant  if and only if $\sum\nolimits_{u\in\{0,1\}^{n-t}} \delta(u) =\pm 2^n$.
%\item $f$ is constant  if and only if for all $u \in \{0,1\}^{n-t}$, $\delta(u) =2^t$ or for all $u \in \{0,1\}^{n-t}$, $\delta(u) =-2^t$. 
\end{enumerate}
\end{theorem}
\begin{proof}
Firstly, we prove that $f$ is constant if and only if for all $u \in \{0,1\}^{n-t}$, $\delta(u) =2^t$ or for all $u \in \{0,1\}^{n-t}$, $\delta(u) =-2^t$. 



(\romannumeral1) $\Longleftarrow$. If  for all $u \in \{0,1\}^{n-t}$, $\delta(u) =2^t$, then according to equation (\ref{delta1}), for all $u \in \{0,1\}^{n-t}$, we have 
\begin{equation}
\sum_{w\in\{0,1\}^t}f_w(u)=0. 
\end{equation}

So for all $x \in \{0,1\}^{n}$, $f(x)=0$, that is $f(x) \equiv 0$.





 If  for all $u \in \{0,1\}^{n-t}$, $\delta(u) =-2^t$, then according to equation (\ref{delta1}), for all $u \in \{0,1\}^{n-t}$, we have 
 \begin{equation}
 \sum_{w\in\{0,1\}^t}f_w(u)=2^t. 
 \end{equation}
 
 So for all $x \in \{0,1\}^{n}$, $f(x)=1$, that is $f(x) \equiv 1$.
 
 Therefore, if for all $u \in \{0,1\}^{n-t}$, $\delta(u) =2^t$ or for all $u \in \{0,1\}^{n-t}$, $\delta(u) =-2^t$, then $f$ is constant.

(\romannumeral2) $\Longrightarrow$. If $f$ is constant, then we have $f(x) \equiv 0$ or $f(x) \equiv 1$. 


If $f(x) \equiv 0$, then for all $x \in \{0,1\}^{n}$, $f(x)=0$. So for all $u \in \{0,1\}^{n-t}$, we have 
\begin{equation}
\sum_{w\in\{0,1\}^t}f_w(u)=0. 
 \end{equation}
 
 According to equation (\ref{delta1}), for all $u \in \{0,1\}^{n-t}$, we have $\delta(u) =2^t$.


If $f(x) \equiv 1$, then for all $x \in \{0,1\}^{n}$, $f(x)=1$. So for all $u \in \{0,1\}^{n-t}$, we have 
\begin{equation}
\sum_{w\in\{0,1\}^t}f_w(u)=2^t.  
 \end{equation}
 
According to equation (\ref{delta1}), for all $u \in \{0,1\}^{n-t}$, we have $\delta(u) =-2^t$.

 Therefore, if $f$ is constant, then for all $u \in \{0,1\}^{n-t}$, $\delta(u) =2^t$ or for all $u \in \{0,1\}^{n-t}$, $\delta(u) =-2^t$.
 
 




Secondly, we prove that $f$ is balanced if and only if $\sum\nolimits_{u\in\{0,1\}^{n-t}} \delta(u) = 0$.

(\romannumeral3) $\Longleftarrow$. If $\sum\nolimits_{u\in\{0,1\}^{n-t}} \delta(u)=0$, according to equation $(\ref{delta1})$, then we have 
\begin{equation}
\begin{split}
\sum\limits_{u\in\{0,1\}^{n-t}} \delta(u)
=&\sum\limits_{u\in\{0,1\}^{n-t}} \left(2^t-2\sum\limits_{w\in\{0,1\}^t}f_w(u)\right)\\
=&\sum\limits_{u\in\{0,1\}^{n-t}}2^t-2\sum\limits_{\substack{u\in\{0,1\}^{n-t}\\w\in\{0,1\}^{t}}}f_w(u)\\
=&2^n-2\sum\limits_{x\in\{0,1\}^{n}}f(x)
=0, 
\end{split}
\end{equation}
that is $\sum\nolimits_{x\in\{0,1\}^{n}}f(x)=2^{n-1}$.
So we have 
\begin{equation}
|\{x|f(x)=1\}|
=\sum\limits_{x\in\{0,1\}^{n}}f(x)
=2^{n-1}. 
\end{equation}

\begin{equation}
|\{x|f(x)=0\}|
=2^n-|\{x|f(x)=1\}|
=2^{n-1}.
\end{equation}

Therefore, $f$ is balanced.

(\romannumeral4) $\Longrightarrow$. If $f$ is balanced, then  we have 
\begin{equation}\label{thm5_balanced_1}
|\{x|f(x)=0\}|=|\{x|f(x)=1\}|=2^{n-1}. 
\end{equation}

According to equation $(\ref{delta1})$ and equation $(\ref{thm5_balanced_1})$,  we have 
\begin{equation}
\begin{split}
\sum\limits_{u\in\{0,1\}^{n-t}} \delta(u)
=&\sum\limits_{u\in\{0,1\}^{n-t}} \left(2^t-2\sum\limits_{w\in\{0,1\}^t}f_w(u)\right)\\
=&\sum\limits_{u\in\{0,1\}^{n-t}}2^t-2\sum\limits_{\substack{u\in\{0,1\}^{n-t}\\w\in\{0,1\}^{t}}}f_w(u)\\
=&2^n-2\sum\limits_{x\in\{0,1\}^{n}}f(x)\\
=&2^n-2|\{x|f(x)=1\}|
=0.
\end{split}
\end{equation}



\iffalse

Secondly, we prove that $f$ is constant if and only if $\sum\nolimits_{u\in\{0,1\}^{n-t}} \delta(u) = 2^n$. 



(\romannumeral3) $\Longleftarrow$. If $\sum\nolimits_{u\in\{0,1\}^{n-t}} \delta(u)=\pm 2^n$, according to equation $(\ref{delta1})$, then we have 
\begin{equation}
\begin{split}
&\sum\nolimits_{u\in\{0,1\}^{n-t}} \delta(u)\\
=&\sum\nolimits_{u\in\{0,1\}^{n-t}} (2^t-2\sum_{w\in\{0,1\}^t}f(uw))\\
=&\sum\nolimits_{u\in\{0,1\}^{n-t}}2^t-2\sum\nolimits_{u,w\in\{0,1\}^{n-t}}f(uw)\\
=&2^n-2\sum\nolimits_{x\in\{0,1\}^{n}}f(x)=\pm 2^n,
\end{split}
\end{equation}
that is $\sum\nolimits_{x\in\{0,1\}^{n}}f(x)=0$ or $2^n$. 

So $|\{x\in\{0,1\}^n|f(x)=1\}|=0$ or $2^n$, that is $f(x) \equiv 0$ or $f(x) \equiv 1$. Therefore, $f$ is constant.

(\romannumeral4) $\Longrightarrow$. If $f$ is constant, according to the definition of DJ problem, then we have $f(x) \equiv 0$ or $f(x) \equiv 1$. So $|\{x\in\{0,1\}^n|f(x)=1\}|=0$ or $2^n$, that is $\sum\nolimits_{x\in\{0,1\}^{n}}f(x)=0$ or $2^n$. According to equation $(\ref{delta1})$, then we have 
\begin{equation}
\begin{split}
&\sum\nolimits_{u\in\{0,1\}^{n-t}} \delta(u)\\
=&\sum\nolimits_{u\in\{0,1\}^{n-t}} (2^t-2\sum_{w\in\{0,1\}^t}f(uw))\\
=&\sum\nolimits_{u\in\{0,1\}^{n-t}}2^t-2\sum\nolimits_{u,w\in\{0,1\}^{n-t}}f(uw)\\
=&2^n-2\sum\nolimits_{x\in\{0,1\}^{n}}f(x)\\
=&\pm 2^n.
\end{split}
\end{equation}


\fi


\end{proof}

In light of Theorem \ref{The2},  we have   Corollary \ref{Cor1} below, which provides a  sufficient  condition for determining whether a given Boolean function $f$ of  DJ problem is balanced by the absolute value of $\delta(u)$.

%From Theorem \ref{The2} we have the following  Corollary \ref{Cor1}.

\begin{corollary}\label{Cor1} Suppose Boolean function $f:\{0,1\}^n \rightarrow \{0,1\}$, satisfies that it is either constant or balanced, and it is divided into $2^t$ subfunctions $f_w$ $(\forall u \in \{0,1\}^{n-t}, w \in \{0,1\}^{t}, f_w(u)=f(uw))$. If $\exists$ $u\in\{0,1\}^{n-t}$ such that $|\delta(u)|\neq 2^t$, then $f$ is balanced. %a string $s \in \{0,1\}^n$  with $s\neq 0^n$, such that $f(x) = f(y)$ if and only if $x = y$ or $x \oplus y = s$. 
%Then
%  $\forall u,v \in \{0,1\}^{n-t},S(u)=S(v)$ if and only if $u \oplus v = 0^{n-t}$ or $u \oplus v = s_1$, where $s=s_1s_2$.
\end{corollary}

\iffalse
\begin{proof} %We prove the theorem by contradiction.
\iffalse
Suppose $\exists$ $u\in\{0,1\}^{n-t}$ such that $|\delta(u)|\neq 2^t$, but $f$ is constant. 

 From $f$ being  constant, it follows that $\forall u\in\{0,1\}^{n-t}$, 
 \begin{equation}
 \sum_{w\in\{0,1\}^t}f(uw)=0
 \end{equation}
 or 
 \begin{equation}
 \sum_{w\in\{0,1\}^t}f(uw)=2^t. 
 \end{equation}
 
 According to  equation $(\ref{delta1})$,  we have $|\delta(u)|= 2^t$, which is contrary to the assumption $|\delta(u)|\neq 2^t$. 
 
 Therefore, if $\exists$ $u\in\{0,1\}^{n-t}$ such that $|\delta(u)|\neq 2^t$, then $f$ is balanced.

\fi

 If $\exists$ $u\in\{0,1\}^{n-t}$ such that $|\delta(u)|\neq 2^t$, then for all $u \in \{0,1\}^{n-t}$, $\delta(u) =2^t$ does not hold and for all $u \in \{0,1\}^{n-t}$, $\delta(u) =-2^t$ does not hold. According to Theorem \ref{The2} it follows that $f$ is not constant. Therefore, $f$ is balanced.
\end{proof}
\fi

%Intuitively, Corollary \ref{Cor1}  states that a function $f$ of  DJ problem is decided to be balanced if there is some input for which the values of multiple subfunctions are not simultaneously $0$ or simultaneously $1$.






%\subsection{Structure of  DJ problem in  distributed scenario for  designing  Algorithm \ref{algorithm5}} \label{sec:Structure of  DJ problem in  distributed scenario for  designing  Algorithm 5}


%In the following, we first describe some notations and theorems as well as lemmas that are closely related to DJ problem in distributed scenario and are used in the correctness analysis of Algorithm \ref{algorithm5} .


%Below, we describe some theorems characterizing the structural feature of  DJ problem in  distributed scenario. 
%, which are used in the design and  correctness analysis of Algorithm \ref{algorithm5}.





By combining Theorem \ref{The4} with Theorem \ref{The2}, we can obtain Theorem 
\ref{The7} in the following. Also, we need some notations for convenience.








%\begin{defi}\label{Deltadef}
Suppose Boolean function $f:\{0,1\}^n \rightarrow \{0,1\}$, $f_{w'0}:\{0,1\}^{n-t} \rightarrow \{0,1\}$, $f_{w'1}:\{0,1\}^{n-t} \rightarrow \{0,1\}$, where $f_{w'0}(u)=f(uw'0)$, $f_{w'1}(u)=f(uw'1)$, $u\in\{0,1\}^{n-t}$, $w'\in\{0,1\}^{t-1}$. For all $u \in \{0,1\}^{n-t}$, denote
\begin{align}
E_{00}(u)=&|\{w'|f_{w'0}(u)=0,f_{w'1}(u)=0\}|.\\
E_{01}(u)=&|\{w'|f_{w'0}(u)=0,f_{w'1}(u)=1\}|.\\
E_{10}(u)=&|\{w'|f_{w'0}(u)=1,f_{w'1}(u)=0\}|.\\
E_{11}(u)=&|\{w'|f_{w'0}(u)=1,f_{w'1}(u)=1\}|.\\
\Delta(u)=&E_{00}(u)-E_{11}(u).\\
K(u)=&|\{w'|f_{w'0}(u)\oplus f_{w'1}(u)=1\}|.
\end{align}

It is clear that 
\begin{equation}\label{K=E_01+E_10}
K(u)=E_{01}(u)+E_{10}(u).
\end{equation}
\iffalse
Suppose function $f:\{0,1\}^n \rightarrow \{0,1\}$, for all $u \in \{0,1\}^{n-t}$, let
\begin{align}
%K(u)=&|\{w\in\{0,1\}^{t-1}0|f(uw)\oplus f(u(w+1))=1\}|.\\
%\Delta(u)=&E_{00}(u)-E_{11}(u).\\
F_0(u)=&|\{w\in\{0,1\}^{t-1}0|f(uw)=0\}|.\\
H_0(u)=&|\{w\in\{0,1\}^{t-1}0|f(u(w+1))=0\}|.
\end{align}

\fi
 
%\begin{equation}
% \Delta(u) = |\{w\in\{0,1\}^{t-1}0|f(uw)=f(u(w+1))=0\}|-|\{w\in\{0,1\}^{t-1}0|f(uw)=f(u(w+1))=1\}|.
% \end{equation}
 
 %=2^t-|\{w\in\{0,1\}^t|f(uw)=1\}|-|\{w\in\{0,1\}^t|f(uw)=1\}|=2^t-2|\{w\in\{0,1\}^t|f(uw)=1\}|$.
%\end{defi}

\iffalse

Since 
\begin{equation}
\begin{split}
 &|\{w\in\{0,1\}^{t-1}0|f(uw)\oplus f(u(w+1))=1\}|\\
 &+|\{w\in\{0,1\}^{t-1}0|f(uw)\oplus f(u(w+1))=0\}|\\
 = &|\{w\in\{0,1\}^{t-1}0|f(uw)\oplus f(u(w+1))=1\}|\\
    &+|\{w\in\{0,1\}^{t-1}0|f(uw)=f(u(w+1))=0\}|\\
    &+|\{w\in\{0,1\}^{t-1}0|f(uw)=f(u(w+1))=1\}|\\
= &2^{t-1},
\end{split}
\end{equation}
then according to the definition of $\Delta(u)$, for all $u \in \{0,1\}^{n-t}$, we have
\begin{equation}
\begin{split}
-2^{t-1}\leq\Delta(u)\leq 2^{t-1}.
\end{split}
\end{equation}

\fi

According to the definition of $\Delta(u)$, for all $u \in \{0,1\}^{n-t}$, we have
\begin{equation}\label{Delta1}
\begin{split}
  \Delta(u)&=E_{00}(u)-E_{11}(u)\\
  &=\left[2^{t-1}-E_{11}(u)-K(u)\right]-E_{11}(u)\\
  %|\{w\in\{0,1\}^{t-1}0|f(uw)=f(u(w+1))=0\}|-|\{w\in\{0,1\}^{t-1}0|f(uw)=f(u(w+1))=1\}|\\
  %&=\left(2^{t-1}-|\{w\in\{0,1\}^{t-1}0|f(uw)=f(u(w+1))=1\}|-|\{w\in\{0,1\}^{t-1}0|f(uw)\oplus f(u(w+1))=1\}|\right)\\
  %&\quad- |\{w\in\{0,1\}^{t-1}0|f(uw)=f(u(w+1))=1\}|\\
  &=2^{t-1}-K(u)-2E_{11}(u)\\
  &=2^{t-1}-\sum_{w'\in\{0,1\}^{t-1}}f_{w'0}(u)\oplus f_{w'1}(u)-2\sum_{w'\in\{0,1\}^{t-1}}f_{w'0}(u)\land f_{w'1}(u).
\end{split}
\end{equation}
and
\begin{equation}\label{Delta2}
\begin{split}
  \Delta(u)&=E_{00}(u)-E_{11}(u)\\
  &=E_{00}(u)-\left[2^{t-1}-E_{00}(u)-K(u)\right]\\
  %|\{w\in\{0,1\}^{t-1}0|f(uw)=f(u(w+1))=0\}|-|\{w\in\{0,1\}^{t-1}0|f(uw)=f(u(w+1))=1\}|\\
  %&=\left(2^{t-1}-|\{w\in\{0,1\}^{t-1}0|f(uw)=f(u(w+1))=1\}|-|\{w\in\{0,1\}^{t-1}0|f(uw)\oplus f(u(w+1))=1\}|\right)\\
  %&\quad- |\{w\in\{0,1\}^{t-1}0|f(uw)=f(u(w+1))=1\}|\\
  &=-2^{t-1}+K(u)+2E_{00}(u)\\
  &=-2^{t-1}+\sum_{w'\in\{0,1\}^{t-1}}f_{w'0}(u)\oplus f_{w'1}(u)+2\sum_{w'\in\{0,1\}^{t-1}}(\lnot f_{w'0}(u))\land (\lnot f_{w'1}(u)).
\end{split}
\end{equation}

%Intuitively, $\Delta(u)$ denotes the number of inputs for which both subfunctions are simultaneously $0$ minus the number of inputs for which both subfunctions are simultaneously $1$, after combining the subfunctions two by two.





%Notice that there could be multiple identical elements in $G(u)$. 
%An example of $\delta(u)$ is shown in Appendix \ref{example}.



%The following theorems concerning $\Delta(u)$ are useful and important, which characterizes the intrinsic structure of  DJ problem in distributed scenario.

%The following Theorem \ref{The6} gives a sufficient condition for $\Delta(u)$ to determine whether or not $f$ is balanced. %, which can be used to ensure the correctness of Algorithm \ref{algorithm5} after the first measurement.



%%%%%以下是原来定理5%%%%


\iffalse


Theorem \ref{The6} below provides a  sufficient  condition for determining whether a given function $f$ of  DJ problem is balanced in light of the absolute value of $\Delta(u)$.


\begin{theorem}\label{The6} Suppose function $f:\{0,1\}^n \rightarrow \{0,1\}$, satisfies that it is either constant or balanced.
 If $\exists$ $u\in\{0,1\}^{n-t}$ such that $|\Delta(u)|\neq 2^{t-1}$, then $f$ is balanced. %a string $s \in \{0,1\}^n$  with $s\neq 0^n$, such that $f(x) = f(y)$ if and only if $x = y$ or $x \oplus y = s$. 
%Then
%  $\forall u,v \in \{0,1\}^{n-t},S(u)=S(v)$ if and only if $u \oplus v = 0^{n-t}$ or $u \oplus v = s_1$, where $s=s_1s_2$.
\end{theorem}
\begin{proof}
%We prove the theorem by contradiction. 
Suppose $\exists$ $u\in\{0,1\}^{n-t}$ such that $|\Delta(u)|\neq 2^{t-1}$, but $f$ is constant. 

From $f$ being  constant, it follows that $\forall u\in\{0,1\}^{n-t}$, 
\begin{equation}
\sum_{w\in\{0,1\}^t}f(uw)=0
\end{equation}
or 
\begin{equation}
\sum_{w\in\{0,1\}^t}f(uw)=2^t. 
\end{equation}

According to  equation $(\ref{Delta1})$ or equation $(\ref{Delta2})$, we have $|\Delta(u)|= 2^{t-1}$, which is contrary to the assumption $|\Delta(u)|\neq 2^{t-1}$. 

Therefore, if $\exists$ $u\in\{0,1\}^{n-t}$ such that $|\Delta(u)|\neq 2^{t-1}$, then $f$ is balanced.
\end{proof}



Intuitively, Theorem \ref{The6}  states that a function $f$ of  DJ problem is decided to be balanced if there is some input for which the values of multiple subfunctions are not simultaneously $0$ or simultaneously $1$.


\fi

%%%%%以上是原来定理5%%%%

%\begin{defi}
%Suppose function $f:\{0,1\}^n \rightarrow \{0,1\}$, let $G_0=\sum\nolimits_{w\in\{0,1\}^{t-1}0}|\{u\in\{0,1\}^{n-t}|f(uw)\oplus f(u(w+1))=0, f(uw)=0\}|$, $G_1=\sum\nolimits_{w\in\{0,1\}^{t-1}0}|\{u\in\{0,1\}^{n-t}|f(uw)\oplus f(u(w+1))=0, f(uw)=1\}|$.
%\end{defi}


%\begin{defi}
%Suppose function $f:\{0,1\}^n \rightarrow \{0,1\}$, let $D=\sum\nolimits_{w\in\{0,1\}^{t-1}0}|\{u\in\{0,1\}^{n-t}|f(uw)\oplus f(u(w+1))=0\}|$ $(0\leq D\leq 2^{n-1})$. 
%\end{defi}

\iffalse


The following Lemma \ref{Lem1} provides a sufficient and necessary condition to determine whether $f$ is balanced.
\begin{lemma}\label{Lem1} Suppose function $f:\{0,1\}^n \rightarrow \{0,1\}$, satisfies that it is either constant or balanced, and it is divided into $2^t$ subfunctions $f_w$ $(\forall u \in \{0,1\}^{n-t}, w \in \{0,1\}^{t}, f_w(u)=f(uw))$ . Then $f$ is balanced if and only if $G_0=G_1=D/2$. %Then $f$ is constant  if and only if for all $w\in\{0,1\}^t$, $|\{u\in\{0,1\}^{n-t}|f(uw)=0\}|=2^{n-t}$ or $|\{u\in\{0,1\}^{n-t}|f(uw)=1\}|=2^{n-t}$.
\end{lemma}
\begin{proof}
%First, we prove that $f$ is balanced if and only if $G_0=G_1=D/2$.

(\romannumeral1) $\Longleftarrow$. Suppose $G_0=G_1=D/2$ $(0\leq D\leq 2^{n-1})$, then $f$ is constant. If $f$ is constant, then $f\equiv 0$ or $f\equiv 1$, that is $\sum\nolimits_{w\in\{0,1\}^{t-1}0}|\{u\in\{0,1\}^{n-t}|f(uw)\oplus f(u(w+1))=0, f(uw)=0\}|=2^{n-1}$ or $\sum\nolimits_{w\in\{0,1\}^{t-1}0}|\{u\in\{0,1\}^{n-t}|f(uw)\oplus f(u(w+1))=0, f(uw)=1\}|=2^{n-1}$. So $G_0=2^{n-1}$ or $G_1=2^{n-1}$, which is contrary to the assumption $G_0=G_1=D/2$ $(0\leq D\leq 2^{n-1})$. Therefore, if $G_0=G_1=D/2$, then $f$ is balanced.%, where $D=\sum\nolimits_{w\in\{0,1\}^{t-1}0}|\{u\in\{0,1\}^{n-t}|f(uw)\oplus f(u(w+1))=0\}|$.




(\romannumeral2) $\Longrightarrow$. %If $f$ is balanced, according to the definition of DJ problem, then   $|\{x\in\{0,1\}^n|f(x)=0\}|=|\{x\in\{0,1\}^n|f(x)=1\}|=2^{n-1}$. 
Since 
\begin{equation}\label{lemma1eq8}
\begin{split}
D%&\sum\limits_{w\in\{0,1\}^{t-1}0}|\{u\in\{0,1\}^{n-t}|f(uw)\oplus f(u(w+1))=0\}|\\
=&\sum\limits_{w\in\{0,1\}^{t-1}0}|\{u\in\{0,1\}^{n-t}|f(uw)= f(u(w+1))=0\}|\\
&+\sum\limits_{w\in\{0,1\}^{t-1}0}|\{u\in\{0,1\}^{n-t}|f(uw)=f(u(w+1))=1\}|, 
\end{split}
\end{equation}
we have
%\begin{equation}
%\begin{split}
%\sum\limits_{w\in\{0,1\}^{t-1}0}|\{u\in\{0,1\}^{n-t}|f(uw)\oplus f(u(w+1))=1\}|
%=2^{n-1}-D, 
%\end{split}
%\end{equation}
%that is 
\begin{equation}\label{lemma1eq3}
\begin{split}
&\sum\limits_{w\in\{0,1\}^{t-1}0}|\{u\in\{0,1\}^{n-t}|f(uw)=0, f(u(w+1))=1\}|\\
&+\sum\limits_{w\in\{0,1\}^{t-1}0}|\{u\in\{0,1\}^{n-t}|f(uw)=1, f(u(w+1))=0\}|\\
=&2^{n-1}-D. 
\end{split}
\end{equation}

%We have 
%\begin{equation}
%\begin{split}
%&\sum\limits_{w\in\{0,1\}^{t-1}0}|\{u\in\{0,1\}^{n-t}|f(uw)=f(u(w+1))=0\}|\\
%=&\sum\limits_{w\in\{0,1\}^{t-1}0}|\{u\in\{0,1\}^{n-t}|f(uw)=0\}|-\sum\limits_{w\in\{0,1\}^{t-1}0}|\{u\in\{0,1\}^{n-t}|f(uw)=0,f(u(w+1))=1\}|\\
%=&\sum\limits_{w\in\{0,1\}^{t-1}0}|\{u\in\{0,1\}^{n-t}|f(u(w+1))=0\}|-\sum\limits_{w\in\{0,1\}^{t-1}0}|\{u\in\{0,1\}^{n-t}|f(uw)=1, f(u(w+1))=0\}|.
%\end{split}
%\end{equation}


Let 
\begin{equation}
\begin{split}
T_0=&\sum\limits_{w\in\{0,1\}^{t-1}0}|\{u\in\{0,1\}^{n-t}|f(uw)=0\}|\\&-\sum\limits_{w\in\{0,1\}^{t-1}0}|\{u\in\{0,1\}^{n-t}|f(uw)=0, f(u(w+1))=1\}|.
\end{split}
\end{equation}
\begin{equation}
\begin{split}
T_1=&\sum\limits_{w\in\{0,1\}^{t-1}0}|\{u\in\{0,1\}^{n-t}|f(u(w+1))=0\}|\\&-\sum\limits_{w\in\{0,1\}^{t-1}0}|\{u\in\{0,1\}^{n-t}|f(uw)=1, f(u(w+1))=0\}|.
\end{split}
\end{equation}

Then
\begin{equation}\label{lemma1eq6}
\begin{split}
T_0=&T_1\\
=&\sum\limits_{w\in\{0,1\}^{t-1}0}|\{u\in\{0,1\}^{n-t}|f(uw)= f(u(w+1))=0\}|.
\end{split}
\end{equation}

According to equation (\ref{lemma1eq3}), we have
\begin{equation}\label{lemma1eq4}
\begin{split}
&T_0+T_1\\
%=&\left(\sum\limits_{w\in\{0,1\}^{t-1}0}|\{u\in\{0,1\}^{n-t}|f(uw)=0\}|-\sum\limits_{w\in\{0,1\}^{t-1}0}|\{u\in\{0,1\}^{n-t}|f(uw)=0, f(u(w+1))=1\}|\right)\\
%&+\left(\sum\limits_{w\in\{0,1\}^{t-1}0}|\{u\in\{0,1\}^{n-t}|f(u(w+1))=0\}|\right.\\
%&\left.-\sum\limits_{w\in\{0,1\}^{t-1}0}|\{u\in\{0,1\}^{n-t}|f(uw)=1, f(u(w+1))=0\}|\right)\\
=&\sum\limits_{w\in\{0,1\}^{t-1}0}|\{u\in\{0,1\}^{n-t}|f(uw)=0\}|+\sum\limits_{w\in\{0,1\}^{t-1}0}|\{u\in\{0,1\}^{n-t}|f(u(w+1))=0\}|\\
&-\left(\sum\limits_{w\in\{0,1\}^{t-1}0}|\{u\in\{0,1\}^{n-t}|f(uw)=0, f(u(w+1))=1\}|\right.\\
&+\left.\sum\limits_{w\in\{0,1\}^{t-1}0}|\{u\in\{0,1\}^{n-t}|f(uw)=1, f(u(w+1))=0\}|\right)\\
=&\sum\limits_{w\in\{0,1\}^{t-1}0}|\{u\in\{0,1\}^{n-t}|f(uw)=0\}|+\sum\limits_{w\in\{0,1\}^{t-1}0}|\{u\in\{0,1\}^{n-t}|f(u(w+1))=0\}|-\left(2^{n-1}-D\right). 
\end{split}
\end{equation}

If $f$ is balanced, %according to the definition of DJ problem, 
then %we have 
$|\{x\in\{0,1\}^n|f(x)=0\}|=|\{x\in\{0,1\}^n|f(x)=1\}|=2^{n-1}$. 
So we have
\begin{equation}\label{lemma1eq5}
\begin{split}
&\sum\limits_{w\in\{0,1\}^{t-1}0}|\{u\in\{0,1\}^{n-t}|f(uw)=0\}|+\sum\limits_{w\in\{0,1\}^{t-1}0}|\{u\in\{0,1\}^{n-t}|f(u(w+1))=0\}|\\
=&|\{x\in\{0,1\}^n|f(x)=0\}|\\\
=&2^{n-1}.
\end{split}
\end{equation}

Therefore, according to equation (\ref{lemma1eq4}) and equation (\ref{lemma1eq5}), we have
\begin{equation}\label{lemma1eq7}
\begin{split}
&T_0+T_1\\
=&\sum\limits_{w\in\{0,1\}^{t-1}0}|\{u\in\{0,1\}^{n-t}|f(uw)=0\}|+\sum\limits_{w\in\{0,1\}^{t-1}0}|\{u\in\{0,1\}^{n-t}|f(u(w+1))=0\}|-(2^{n-1}-D)\\
=&2^{n-1}-(2^{n-1}-D)\\
=&D.
\end{split}
\end{equation}

According to equation (\ref{lemma1eq6}) and equation (\ref{lemma1eq7}), we have
%\begin{equation}
%T_0=T_1=\sum\limits_{w\in\{0,1\}^{t-1}0}|\{u\in\{0,1\}^{n-t}|f(uw)=f(u(w+1))=0\}|,
%\end{equation} 
%we have 
\begin{equation}\label{lemma1eq1}
%\begin{split}
T_0=T_1
=\sum\limits_{w\in\{0,1\}^{t-1}0}|\{u\in\{0,1\}^{n-t}|f(uw)=f(u(w+1))=0\}|
=D/2. 
%\end{split}
\end{equation}

With equation (\ref{lemma1eq8}) and equation (\ref{lemma1eq1}), we have
%\begin{equation}
%\begin{split}
%D=&\sum\limits_{w\in\{0,1\}^{t-1}0}|\{u\in\{0,1\}^{n-t}|f(uw)= f(u(w+1))=0\}|\\
%&+\sum\limits_{w\in\{0,1\}^{t-1}0}|\{u\in\{0,1\}^{n-t}|f(uw)=f(u(w+1))=1\}|, 
%\end{split}
%\end{equation} 
%we have 
\begin{equation}\label{lemma1eq2}
\begin{split}
&\sum\limits_{w\in\{0,1\}^{t-1}0}|\{u\in\{0,1\}^{n-t}|f(uw)=f(u(w+1))=1\}|\\
=&D-\sum\limits_{w\in\{0,1\}^{t-1}0}|\{u\in\{0,1\}^{n-t}|f(uw)=f(u(w+1))=0\}|\\
=&D-D/2
=D/2.
\end{split}
\end{equation}





%Since 
%\begin{equation}
%G_0
%=&\sum\limits_{w\in\{0,1\}^{t-1}0}|\{u\in\{0,1\}^{n-t}|f(uw)\oplus f(u(w+1))=0, f(uw)=0\}|\\
%=\sum\limits_{w\in\{0,1\}^{t-1}0}|\{u\in\{0,1\}^{n-t}|f(uw)=f(u(w+1))=0\}|,
%\end{equation}
%\begin{equation}
%G_1
%=&\sum\limits_{w\in\{0,1\}^{t-1}0}|\{u\in\{0,1\}^{n-t}|f(uw)\oplus f(u(w+1))=0, f(uw)=1\}|\\
%=\sum\limits_{w\in\{0,1\}^{t-1}0}|\{u\in\{0,1\}^{n-t}|f(uw)=f(u(w+1))=1\}|,
%\end{equation}
In addition, by equation (\ref{lemma1eq1}) and equation (\ref{lemma1eq2}), we have $G_0=G_1=D/2$.

%Let $S_0=|\{u\in\{0,1\}^{n-1}|f(u0)=0\}|-|\{u\in\{0,1\}^{n-1}|f(u0)=0,f(u1)=1\}|$, $S_1=|\{u\in\{0,1\}^{n-1}|f(u1)=0\}|-|\{u\in\{0,1\}^{n-1}|f(u0)=1,f(u1)=0\}|$. 

%Then $S_0+S_1=(|\{u\in\{0,1\}^{n-1}|f(u0)=0\}|-|\{u\in\{0,1\}^{n-1}|f(u0)=0,f(u1)=1\}|)+(|\{u\in\{0,1\}^{n-1}|f(u1)=0\}|-|\{u\in\{0,1\}^{n-1}|f(u0)=1,f(u1)=0\}|)=(|\{u\in\{0,1\}^{n-1}|f(u0)=0\}|+|\{u\in\{0,1\}^{n-1}|f(u1)=0\}|)-(|\{u\in\{0,1\}^{n-1}|f(u0)=0,f(u1)=1\}|+|\{u\in\{0,1\}^{n-1}|f(u0)=1,f(u1)=0\}|)=(|\{u\in\{0,1\}^{n-1}|f(u0)=0\}|+|\{u\in\{0,1\}^{n-1}|f(u1)=0\}|)-(2^{n-1}-M)$. 




%If $f$ is balanced, according to the definition of DJ problem, then we have $|\{x\in\{0,1\}^n|f(x)=0\}|=|\{x\in\{0,1\}^n|f(x)=1\}|=2^{n-1}$. So $|\{u\in\{0,1\}^{n-1}|f(u0)=0\}|+|\{u\in\{0,1\}^{n-1}|f(u1)=0\}|=|\{x\in\{0,1\}^n|f(x)=0\}|=2^{n-1}$. Therefore, $S_0+S_1=(|\{u\in\{0,1\}^{n-1}|f(u0)=0\}|+|\{u\in\{0,1\}^{n-1}|f(u1)=0\}|)-(2^{n-1}-M)=2^{n-1}-(2^{n-1}-M)=M$. Then from $S_0=S_1=|\{u\in\{0,1\}^{n-1}|f(u0)= f(u1)=0\}|$, we have $S_0=S_1=M/2$. 

%Further, from $S_0=|\{u\in\{0,1\}^{n-1}|f(u0)=0\}|-|\{u\in\{0,1\}^{n-1}|f(u0)=0,f(u1)=1\}|=B_0$, $S_1=|\{u\in\{0,1\}^{n-1}|f(u1)=0\}|-|\{u\in\{0,1\}^{n-1}|f(u0)=1,f(u1)=0\}|=B_1$, we have $B_0=B_1=M/2$.



% According to equation $(\ref{delta1})$, then we have $\sum\nolimits_{u\in\{0,1\}^{n-t}} \delta(u)=\sum\nolimits_{u\in\{0,1\}^{n-t}} (2^t-2\sum_{w\in\{0,1\}^t}f(uw))=\sum\nolimits_{u\in\{0,1\}^{n-t}}2^t-2\sum\nolimits_{u,w\in\{0,1\}^{n-t}}f(uw)=2^n-2\sum\nolimits_{x\in\{0,1\}^{n}}f(x)=2^n-2|\{x\in\{0,1\}^n|f(x)=1\}|=0$.

%Second, we prove that $f$ is constant if and only if $|\{u\in\{0,1\}^{n-1}|f(u0)=0\}|=|\{u\in\{0,1\}^{n-1}|f(u1)=0\}|=2^{n-1}$ or $|\{u\in\{0,1\}^{n-1}|f(u0)=1\}|=|\{u\in\{0,1\}^{n-1}|f(u1)=1\}|=2^{n-1}$. 

%(3) $\Longleftarrow$. If $|\{u\in\{0,1\}^{n-1}|f(u0)=0\}|=|\{u\in\{0,1\}^{n-1}|f(u1)=0\}|=2^{n-1}$ or $|\{u\in\{0,1\}^{n-1}|f(u0)=1\}|=|\{u\in\{0,1\}^{n-1}|f(u1)=1\}|=2^{n-1}$, then $|\{x\in\{0,1\}^n|f(x)=1\}|=0$ or $2^n$, that is $f(x) \equiv 0$ or $f(x) \equiv 1$. Therefore, $f$ is constant.

%(4) $\Longrightarrow$. If $f$ is constant, according to the definition of DJ problem, then  $f(x) \equiv 0$ or $f(x) \equiv 1$. So $|\{x\in\{0,1\}^n|f(x)=1\}|=0$ or $2^n$. Therefore $|\{u\in\{0,1\}^{n-1}|f(u0)=0\}|=|\{u\in\{0,1\}^{n-1}|f(u1)=0\}|=2^{n-1}$ or $|\{u\in\{0,1\}^{n-1}|f(u0)=1\}|=|\{u\in\{0,1\}^{n-1}|f(u1)=1\}|=2^{n-1}$.

\end{proof}





\fi





Theorem \ref{The7} below provides a sufficient and necessary condition for determining whether a given Boolean function $f$ of  DJ problem is constant or balanced in the light of $\Delta(u)$.

\begin{theorem}\label{The7} Suppose Boolean function $f:\{0,1\}^n \rightarrow \{0,1\}$, satisfies that it is either constant or balanced, and it is divided into $2^t$ subfunctions $f_{w'0}$ and  $f_{w'1}$ $(\forall u \in \{0,1\}^{n-t}, w' \in \{0,1\}^{t-1}, f_{w'0}(u)=f(uw'0), f_{w'1}(u)=f(uw'1))$. %Then $f$ is balanced if and only if $G_0=G_1=D/2$. 
Then:
\begin{enumerate}[(1)]
 \item $f$ is constant  if and only if for all $u \in \{0,1\}^{n-t}$, $\Delta(u) = 2^{t-1}$ or for all $u \in \{0,1\}^{n-t}$, $\Delta(u) = -2^{t-1}$;
 \item $f$ is balanced  if and only if $\sum_{u\in\{0,1\}^{n-t}}\Delta(u)=0$.
 \end{enumerate}
\end{theorem}
\begin{proof}
Firstly, we prove that $f$ is constant if and only if for all $u \in \{0,1\}^{n-t}$, $\Delta(u) = 2^{t-1}$ or for all $u \in \{0,1\}^{n-t}$, $\Delta(u) = -2^{t-1}$.

(\romannumeral1) $\Longleftarrow$. If  for all $u \in \{0,1\}^{n-t}$, $\Delta(u) =2^{t-1}$, then according to equation (\ref{Delta1}), for all $u \in \{0,1\}^{n-t}$, we have 
\begin{equation}
\sum_{w'\in\{0,1\}^{t-1}}f_{w'0}(u)\oplus f_{w'1}(u)=0
\end{equation}
and
\begin{equation}
\sum_{w'\in\{0,1\}^{t-1}}f_{w'0}(u)\land f_{w'1}(u)=0. 
\end{equation}

So for all $x \in \{0,1\}^{n}$, $f(x)=0$, that is $f(x) \equiv 0$.



 If  for all $u \in \{0,1\}^{n-t}$, $\Delta(u) =-2^{t-1}$, then by equation (\ref{Delta2}), for all $u \in \{0,1\}^{n-t}$, we have 
\begin{equation}
\sum_{w'\in\{0,1\}^{t-1}}f_{w'0}(u)\oplus f_{w'1}(u)=0
\end{equation}
 and
 \begin{equation}
 \sum_{w'\in\{0,1\}^{t-1}}(\lnot f_{w'0}(u))\land (\lnot f_{w'1}(u))=0. 
  \end{equation}
  
 So for all $x \in \{0,1\}^{n}$, $f(x)=1$, that is $f(x) \equiv 1$.
 
 Therefore, if for all $u \in \{0,1\}^{n-t}$, $\Delta(u) =2^{t-1}$ or for all $u \in \{0,1\}^{n-t}$, $\Delta(u) =-2^{t-1}$, then $f$ is constant.

(\romannumeral2) $\Longrightarrow$. If $f$ is constant, then we have $f(x) \equiv 0$ or $f(x) \equiv 1$. 


If $f(x) \equiv 0$, then for all $x \in \{0,1\}^{n}$, $f(x)=0$. So for all $u \in \{0,1\}^{n-t}$, we have 
 \begin{equation}
\sum_{w'\in\{0,1\}^{t-1}}f_{w'0}(u)\oplus f_{w'1}(u)=0
\end{equation}
and
\begin{equation}
\sum_{w'\in\{0,1\}^{t-1}}f_{w'0}(u)\land f_{w'1}(u)=0.
\end{equation}

With equation (\ref{Delta1}), for all $u \in \{0,1\}^{n-t}$, we have $\Delta(u) =2^{t-1}$.


If $f(x) \equiv 1$, then for all $x \in \{0,1\}^{n}$, $f(x)=1$. So for all $u \in \{0,1\}^{n-t}$, we have 
\begin{equation}
\sum_{w'\in\{0,1\}^{t-1}}f_{w'0}(u)\oplus f_{w'1}(u)=0
\end{equation}
and
\begin{equation}
\sum_{w'\in\{0,1\}^{t-1}}(\lnot f_{w'0}(u))\land (\lnot f_{w'1}(u))=0. 
\end{equation}

By equation (\ref{Delta2}), for all $u \in \{0,1\}^{n-t}$, we have $\Delta(u) =-2^{t-1}$.

 Therefore, if $f$ is constant, then for all $u \in \{0,1\}^{n-t}$, $\Delta(u) =2^{t-1}$ or for all $u \in \{0,1\}^{n-t}$, $\Delta(u) =-2^{t-1}$.



Secondly, we prove that $f$ is balanced if and only if $\sum_{u\in\{0,1\}^{n-t}}\Delta(u)=0$.


(\romannumeral3) $\Longleftarrow$. Suppose $\sum_{u\in\{0,1\}^{n-t}}\Delta(u)=0$, but $f$ is constant. 
From $f$ being  constant, it follows that $f\equiv 0$ or $f\equiv 1$, that is 
\begin{equation}
\sum\limits_{u\in\{0,1\}^{n-t}}E_{00}(u)=2^{n-1} 
\end{equation}
or 
\begin{equation}
\sum\limits_{u\in\{0,1\}^{n-t}}E_{11}(u)=2^{n-1}. 
\end{equation}


Therefore, we have
\begin{equation}
\sum\limits_{u\in\{0,1\}^{n-t}}\Delta(u)=2^{n-1}
\end{equation}
or
\begin{equation}
\sum\limits_{u\in\{0,1\}^{n-t}}\Delta(u)=-2^{n-1},
\end{equation}
which is contrary to the assumption $\sum_{u\in\{0,1\}^{n-t}}\Delta(u)=0$. 

As a result, if $\sum_{u\in\{0,1\}^{n-t}}\Delta(u)=0$, then $f$ is balanced.

(\romannumeral4) $\Longrightarrow$. 
If $f$ is balanced, %according to the definition of DJ problem, 
then %we have 
$|\{x|f(x)=0\}|=|\{x|f(x)=1\}|=2^{n-1}$. 
For all $u \in \{0,1\}^{n-t}$,  denote
\begin{equation}
F_0(u)=|\{w'|f_{w'0}(u)=0\}|.
\end{equation}

For all $u \in \{0,1\}^{n-t}$, denote
\begin{equation}
H_0(u)=|\{w'|f_{w'1}(u)=0\}|.
\end{equation}

Then we have
\begin{equation}\label{lemma1eq5}
\begin{split}
&\sum\limits_{u\in\{0,1\}^{n-t}}[F_0(u)+H_0(u)]\\
=&|\{x|f(x)=0\}|\\
=&2^{n-1}.
\end{split}
\end{equation}


Let
\begin{equation}\label{D}
D=\sum\limits_{u\in\{0,1\}^{n-t}}[E_{00}(u)+E_{11}(u)].
\end{equation}
\begin{equation}
T_0=\sum\limits_{u\in\{0,1\}^{n-t}}[F_0(u)-E_{01}(u)].
\end{equation}
\begin{equation}
T_1=\sum\limits_{u\in\{0,1\}^{n-t}}[H_0(u)-E_{10}(u)].
\end{equation}

Then
\begin{equation}\label{T_0=T_1}
T_0=T_1=\sum\limits_{u\in\{0,1\}^{n-t}}E_{00}(u).
\end{equation}
\begin{equation}\label{T_0+T_1}
\begin{split}
T_0+T_1=&\sum\limits_{u\in\{0,1\}^{n-t}}[F_0(u)+H_0(u)]-\sum\limits_{u\in\{0,1\}^{n-t}}[E_{01}(u)+E_{10}(u)]\\
&=2^{n-1}-(2^{n-1}-D)\\
&=D.
\end{split}
\end{equation}

Combining equation (\ref{T_0=T_1}) with equation (\ref{T_0+T_1}), we have
\begin{equation}\label{E_00=D/2}
\sum\limits_{u\in\{0,1\}^{n-t}}E_{00}(u)=\frac{D}{2}.
\end{equation}

Due to equation (\ref{D}) and equation (\ref{E_00=D/2}), we have
\begin{equation}\label{E_11=D/2}
\sum\limits_{u\in\{0,1\}^{n-t}}E_{11}(u)=\frac{D}{2}.
\end{equation}

From equation (\ref{E_00=D/2}) and equation (\ref{E_11=D/2}), we can deduce
\begin{equation}\label{Delta=0}
\begin{split}
\sum\limits_{u\in\{0,1\}^{n-t}}\Delta(u)=&\sum\limits_{u\in\{0,1\}^{n-t}}[E_{00}(u)-E_{11}(u)]\\
=&\sum\limits_{u\in\{0,1\}^{n-t}}E_{00}(u)-\sum\limits_{u\in\{0,1\}^{n-t}}E_{11}(u)\\
=&0.
\end{split}
\end{equation}

\iffalse
Since
\begin{equation}
\begin{split}
&\sum_{u\in\{0,1\}^{n-t}}\Delta(u)\\
 =&\sum_{u\in\{0,1\}^{n-t}}\left(|\{w\in\{0,1\}^{t-1}0|f(uw)=f(u(w+1))=0\}|-|\{w\in\{0,1\}^{t-1}0|f(uw)=f(u(w+1))=1\}|\right)\\
=&\sum_{w\in\{0,1\}^{t-1}0}\left(|\{u\in\{0,1\}^{n-t}|f(uw)=f(u(w+1))=0\}|-|\{u\in\{0,1\}^{n-t}|f(uw)=f(u(w+1))=1\}|\right)\\
=&\sum_{w\in\{0,1\}^{t-1}0}|\{u\in\{0,1\}^{n-t}|f(uw)=f(u(w+1))=0\}|\\
&-\sum_{w\in\{0,1\}^{t-1}0}|\{u\in\{0,1\}^{n-t}|f(uw)=f(u(w+1))=1\}|\\
=&G_0-G_1,
\end{split}
\end{equation}
according to Lemma \ref{Lem1}, we have $f$ is balanced  if and only if $\sum_{u\in\{0,1\}^{n-t}}\Delta(u)=0$.
\fi


 
\end{proof}



%Intuitively, Theorem \ref{The7} states that a function $f$ of  DJ problem is balanced if the number of inputs for which the subfunctions of the two-by-two group are simultaneously 0 is equal to the number of inputs for which the subfunctions of the two-by-two group are simultaneously 1, otherwise it is constant.

In light of Theorem \ref{The7},  we have   Corollary \ref{Cor6} below, which provides a  sufficient  condition for determining whether a given Boolean function $f$ of  DJ problem is balanced by the absolute value of $\Delta(u)$.


\begin{corollary}\label{Cor6} Suppose Boolean function $f:\{0,1\}^n \rightarrow \{0,1\}$, satisfies that it is either constant or balanced.
 If $\exists$ $u\in\{0,1\}^{n-t}$ such that $|\Delta(u)|\neq 2^{t-1}$, then $f$ is balanced. %a string $s \in \{0,1\}^n$  with $s\neq 0^n$, such that $f(x) = f(y)$ if and only if $x = y$ or $x \oplus y = s$. 
%Then
%  $\forall u,v \in \{0,1\}^{n-t},S(u)=S(v)$ if and only if $u \oplus v = 0^{n-t}$ or $u \oplus v = s_1$, where $s=s_1s_2$.
\end{corollary}
%\begin{proof}
%\iffalse
%We prove the theorem by contradiction. 
%Suppose $\exists$ $u\in\{0,1\}^{n-t}$ such that $|\Delta(u)|\neq 2^{t-1}$, but $f$ is constant. 

%From $f$ being  constant, it follows that $\forall u\in\{0,1\}^{n-t}$, 
%\begin{equation}
%\sum_{w\in\{0,1\}^t}f(uw)=0
%\end{equation}
%or 
%\begin{equation}
%\sum_{w\in\{0,1\}^t}f(uw)=2^t. 
%\end{equation}

%According to  equation $(\ref{Delta1})$ or equation $(\ref{Delta2})$, we have $|\Delta(u)|= 2^{t-1}$, which is contrary to the assumption $|\Delta(u)|\neq 2^{t-1}$. 

%Therefore, if $\exists$ $u\in\{0,1\}^{n-t}$ such that $|\Delta(u)|\neq 2^{t-1}$, then $f$ is balanced.
%\fi
% If $\exists$ $u\in\{0,1\}^{n-t}$ such that $|\Delta(u)|\neq 2^{t-1}$, then for all $u \in \{0,1\}^{n-t}$, $\Delta(u) =2^{t-1}$ does not hold and for all $u \in \{0,1\}^{n-t}$, $\Delta(u) =-2^{t-1}$ does not hold. According to Theorem \ref{The7} it follows that $f$ is not constant. Therefore, $f$ is balanced.
%\end{proof}



%Intuitively, Corollary \ref{Cor6}  states that a function $f$ of  DJ problem is decided to be balanced if there is some input for which the values of multiple subfunctions are not simultaneously $0$ or simultaneously $1$.


\iffalse

The following Theorem \ref{The8} provides a sufficient and necessary condition for $\Delta(u)$ to determine whether $f$ is constant, which can be used to ensure the correctness of Algorithm \ref{algorithm5} after the second measurement.

\begin{theorem}\label{The8} Suppose function $f:\{0,1\}^n \rightarrow \{0,1\}$, satisfies that it is either constant or balanced, and it is divided into $2^t$ subfunctions $f_w$ $(\forall u \in \{0,1\}^{n-t}, w \in \{0,1\}^{t}, f_w(u)=f(uw))$. %Then $f$ is balanced if and only if $G_0=G_1=D/2$. 
Then $f$ is constant  if and only if for all $u \in \{0,1\}^{n-t}$, $\Delta(u) = 2^{t-1}$ or for all $u \in \{0,1\}^{n-t}$, $\Delta(u) = -2^{t-1}$.
\end{theorem}
\begin{proof}

(\romannumeral1) $\Longleftarrow$. If  for all $u \in \{0,1\}^{n-t}$, $\Delta(u) =2^{t-1}$, then according to equation (\ref{Delta1}), for all $u \in \{0,1\}^{n-t}$, we have 
\begin{equation}
\sum_{w\in\{0,1\}^{t-1}0}f(uw)\oplus f(u(w+1))=0
\end{equation}
and
\begin{equation}
\sum_{w\in\{0,1\}^{t-1}0}f(uw)\land f(u(w+1))=0. 
\end{equation}

So for all $x \in \{0,1\}^{n}$, $f(x)=0$, that is $f(x) \equiv 0$.



 If  for all $u \in \{0,1\}^{n-t}$, $\Delta(u) =-2^{t-1}$, then by equation (\ref{Delta2}), for all $u \in \{0,1\}^{n-t}$, we have 
 \begin{equation}
 \sum_{w\in\{0,1\}^{t-1}0}f(uw)\oplus f(u(w+1))=0 
 \end{equation}
 and
 \begin{equation}
 \sum_{w\in\{0,1\}^{t-1}0}(\lnot f(uw))\land (\lnot f(u(w+1)))=0. 
  \end{equation}
  
 So for all $x \in \{0,1\}^{n}$, $f(x)=1$, that is $f(x) \equiv 1$.
 
 Therefore, if for all $u \in \{0,1\}^{n-t}$, $\Delta(u) =2^{t-1}$ or for all $u \in \{0,1\}^{n-t}$, $\Delta(u) =-2^{t-1}$, then $f$ is constant.

(\romannumeral2) $\Longrightarrow$. If $f$ is constant, according to the definition of DJ problem, then we have $f(x) \equiv 0$ or $f(x) \equiv 1$. 


If $f(x) \equiv 0$, then for all $x \in \{0,1\}^{n}$, $f(x)=0$. So for all $u \in \{0,1\}^{n-t}$, we have 
 \begin{equation}
\sum_{w\in\{0,1\}^{t-1}0}f(uw)\oplus f(u(w+1))=0
\end{equation}
and
\begin{equation}
\sum_{w\in\{0,1\}^{t-1}0}f(uw)\land f(u(w+1))=0.  
\end{equation}

According to equation (\ref{Delta1}), for all $u \in \{0,1\}^{n-t}$, we have $\Delta(u) =2^{t-1}$.


If $f(x) \equiv 1$, then for all $x \in \{0,1\}^{n}$, $f(x)=1$. So for all $u \in \{0,1\}^{n-t}$, we have 
\begin{equation}
\sum_{w\in\{0,1\}^{t-1}0}f(uw)\oplus f(u(w+1))=0
\end{equation}
and
\begin{equation}
\sum_{w\in\{0,1\}^{t-1}0}(\lnot f(uw))\land (\lnot f(u(w+1)))=0.  
\end{equation}

By equation (\ref{Delta2}), for all $u \in \{0,1\}^{n-t}$, we have $\Delta(u) =-2^{t-1}$.

 Therefore, if $f$ is constant, then for all $u \in \{0,1\}^{n-t}$, $\Delta(u) =2^{t-1}$ or for all $u \in \{0,1\}^{n-t}$, $\Delta(u) =-2^{t-1}$.
 
 \end{proof}
 
 
 \fi
 
 
 %%%%%%以下是原算法4的部分%%%%%%%
 
 \iffalse
 
 
 In the following, we give the unitary operators involved in Algorithm 4. 


The effect of operator $A$ in FIG. \ref{algorithm_four_nodes (III)} is: $\forall a,b\in \{0,1\}$ and $c \in \{0,1\}^{2}$,
\begin{equation}
\begin{split}
A\ket{a}\ket{b}\ket{d}=\ket{a}\ket{b}|c\oplus(a+b)\rangle.
\end{split}
\end{equation}

Intuitively, the effect of $A$ is to perform add operations on the values in the first two registers, and xor to the third and fourth register. $V$ does not change the states of the first two registers (i.e. $\ket{a}$ and $\ket{b}$). We call these two registers the control registers of $A$.


The effect of operator $V$ in FIG. \ref{algorithm_four_nodes (III)} is: $\forall a,b,c \in \{0,1\}$ and $d \in \{0,1\}^{2}$,
\begin{equation}
\begin{split}
V\ket{a}\ket{b}\ket{c}\ket{d}=\ket{a}\ket{b}\Ket{c\oplus\left\lfloor\frac{1-{\rm sgn}(2-a-2b)}{2}\right\rfloor}|d\oplus|2-a-2b|\rangle.
\end{split}
\end{equation}

Intuitively, the effect of $V$ is to perform arithmetic operations on the values in the first two registers, and xor to the third and fourth register. $V$ does not change the states of the first two registers (i.e. $\ket{a}$ and $\ket{b}$). We call these two registers the control registers of $V$.


The effect of operator $R'$ in FIG. \ref{algorithm_four_nodes (III)} is: $\forall d \in \{0,1\}^2$ and $e \in \{0,1\}$,
\begin{equation}
\begin{split}
	R'\ket{d}\ket{e}=\ket{d}\left(\frac{d}{2}\ket{e}+\sqrt{1-\left(\frac{d}{2}\right)^2}\Ket{1\oplus e}\right).
\end{split}
\end{equation}

Intuitively, the effect of $R'$ is to perform the controlled rotation on the state in the second register based on the value in the first register. $R'$ does not change the states of the first register (i.e. $\ket{d}$ ). We call the register the control register of $R'$. 


 
 
 
\iffalse
(\romannumeral1) $\Longleftarrow$. If $\sum\nolimits_{u\in\{0,1\}^{n-t}} \Delta(u) =2^{n-1}$, according to equation $(\ref{Delta1})$, then we have 
\begin{equation}
\begin{split}
&\sum\nolimits_{u\in\{0,1\}^{n-t}} \Delta(u)\\
=&\sum\nolimits_{u\in\{0,1\}^{n-t}} (2^{t-1}-\sum_{w\in\{0,1\}^{t-1}0}f(uw)\oplus f(u(w+1))\\
  &-2\sum_{w\in\{0,1\}^{t-1}0}f(uw)\land f(u(w+1)))\\
=&2^{n-1},
\end{split}
\end{equation}
and since for all $u \in \{0,1\}^{n-t}$, $-2^{t-1}\leq\Delta(u)\leq 2^{t-1}$, therefore for all $u \in \{0,1\}^{n-t}$, $\Delta(u)=2^{t-1}$.

And since  for all $u \in \{0,1\}^{n-t}$, we have
\begin{equation}
\begin{split}
&\Delta(u)\\
=&2^{t-1}-\sum_{w\in\{0,1\}^{t-1}0}f(uw)\oplus f(u(w+1))\\
  &-2\sum_{w\in\{0,1\}^{t-1}0}f(uw)\land f(u(w+1)),\\
\end{split}
\end{equation}
therefor  for all $u \in \{0,1\}^{n-t}$, we have
\begin{equation}
\begin{split}
\sum_{w\in\{0,1\}^{t-1}0}f(uw)\oplus f(u(w+1))=0.
\end{split}
\end{equation}
\begin{equation}
\begin{split}
\sum_{w\in\{0,1\}^{t-1}0}f(uw)\land f(u(w+1))=0.
\end{split}
\end{equation}

Therefore, for all $u \in \{0,1\}^{n-t}$, we have
\begin{equation}
\begin{split}
&|\{w\in\{0,1\}^{t-1}0|f(uw)=f(u(w+1))=0\}|\\
=&2^{t-1}-(|\{w\in\{0,1\}^{t-1}0|f(uw)\oplus f(u(w+1))=1\}|\\
&+|\{w\in\{0,1\}^{t-1}0|f(uw)=f(u(w+1))=1\}|)\\
=&2^{t-1}-(\sum_{w\in\{0,1\}^{t-1}0}f(uw)\oplus f(u(w+1))\\
&+\sum_{w\in\{0,1\}^{t-1}0}f(uw)\land f(u(w+1)))\\
=&2^{t-1}.
\end{split}
\end{equation}

So $|\{x\in\{0,1\}^n|f(x)=1\}|=0$, that is $f(x) \equiv 0$.

If $\sum\nolimits_{u\in\{0,1\}^{n-t}} \Delta(u) =-2^{n-1}$, according to equation $(\ref{Delta2})$, then we have 
\begin{equation}
\begin{split}
&\sum\nolimits_{u\in\{0,1\}^{n-t}} \Delta(u)\\
  =&\sum\nolimits_{u\in\{0,1\}^{n-t}}(-2^{t-1}+\sum_{w\in\{0,1\}^{t-1}0}f(uw)\oplus f(u(w+1))\\
  &+2\sum_{w\in\{0,1\}^{t-1}0}(\lnot f(uw))\land(\lnot f(u(w+1))))\\
=&-2^{n-1},
\end{split}
\end{equation}
and since for all $u \in \{0,1\}^{n-t}$, $-2^{t-1}\leq\Delta(u)\leq 2^{t-1}$, therefore for all $u \in \{0,1\}^{n-t}$, $\Delta(u)=-2^{t-1}$.

And since  for all $u \in \{0,1\}^{n-t}$, we have
\begin{equation}
\begin{split}
&\Delta(u)\\
=&-2^{t-1}+\sum_{w\in\{0,1\}^{t-1}0}f(uw)\oplus f(u(w+1))\\
  &+2\sum_{w\in\{0,1\}^{t-1}0}(\lnot f(uw))\land(\lnot f(u(w+1))),\\
\end{split}
\end{equation}
therefor  for all $u \in \{0,1\}^{n-t}$, we have
\begin{equation}
\begin{split}
\sum_{w\in\{0,1\}^{t-1}0}f(uw)\oplus f(u(w+1))=0.
\end{split}
\end{equation}
\begin{equation}
\begin{split}
\sum_{w\in\{0,1\}^{t-1}0}(\lnot f(uw))\land(\lnot f(u(w+1))=0.
\end{split}
\end{equation}

Therefore, for all $u \in \{0,1\}^{n-t}$, we have
\begin{equation}
\begin{split}
&|\{w\in\{0,1\}^{t-1}0|f(uw)=f(u(w+1))=1\}|\\
=&2^{t-1}-(|\{w\in\{0,1\}^{t-1}0|f(uw)\oplus f(u(w+1))=1\}|\\
&+|\{w\in\{0,1\}^{t-1}0|f(uw)=f(u(w+1))=0\}|)\\
=&2^{t-1}-(\sum_{w\in\{0,1\}^{t-1}0}f(uw)\oplus f(u(w+1))\\
&+\sum_{w\in\{0,1\}^{t-1}0}(\lnot f(uw))\land(\lnot f(u(w+1)))\\
=&2^{t-1}.
\end{split}
\end{equation}

So $|\{x\in\{0,1\}^n|f(x)=1\}|=2^n$, that is $f(x) \equiv 1$.

Therefore, if $\sum\nolimits_{u\in\{0,1\}^{n-t}} \Delta(u) =\pm 2^{n-1}$, then $f$ is constant.

(\romannumeral2) $\Longrightarrow$. If $f$ is constant, according to the definition of DJ problem, then we have $f(x) \equiv 0$ or $f(x) \equiv 1$. 

If $f(x) \equiv 0$, then $|\{x\in\{0,1\}^n|f(x)=0\}|=2^n$. Therefore, for all $u \in \{0,1\}^{n-t}$, we have $|\{w\in\{0,1\}^{t-1}0|f(uw)=f(u(w+1))=0\}|=2^{t-1}$ and $|\{w\in\{0,1\}^{t-1}0|f(uw)=f(u(w+1))=1\}|=0$.


So according to the definition of $\Delta(u)$,we have
\begin{equation}
\begin{split}
&\sum\limits_{u\in\{0,1\}^{n-t}} \Delta(u)\\
=&\sum\limits_{u\in\{0,1\}^{n-t}} (|\{w\in\{0,1\}^{t-1}0|f(uw)=f(u(w+1))=0\}|\\
&\quad-|\{w\in\{0,1\}^{t-1}0|f(uw)=f(u(w+1))=1\}|)\\
=&\sum\limits_{u\in\{0,1\}^{n-t}}2^{t-1}\\
=&2^{n-1}.
\end{split}
\end{equation}



If $f(x) \equiv 1$, then $|\{x\in\{0,1\}^n|f(x)=1\}|=2^n$. Therefore, for all $u \in \{0,1\}^{n-t}$, we have $|\{w\in\{0,1\}^{t-1}0|f(uw)=f(u(w+1))=0\}|=0$ and $|\{w\in\{0,1\}^{t-1}0|f(uw)=f(u(w+1))=1\}|=2^{t-1}$.


So according to the definition of $\Delta(u)$,we have
\begin{equation}
\begin{split}
&\sum\limits_{u\in\{0,1\}^{n-t}} \Delta(u)\\
=&\sum\limits_{u\in\{0,1\}^{n-t}} (|\{w\in\{0,1\}^{t-1}0|f(uw)=f(u(w+1))=0\}|\\
&\quad-|\{w\in\{0,1\}^{t-1}0|f(uw)=f(u(w+1))=1\}|)\\
=&\sum\limits_{u\in\{0,1\}^{n-t}}-2^{t-1}\\
=&-2^{n-1}.
\end{split}
\end{equation}

Therefore, if $f$ is constant, then $\sum\nolimits_{u\in\{0,1\}^{n-t}} \Delta(u) =\pm 2^{n-1}$.

\fi




%\newpage

%\begin{center}
%\bf B1.\quad Algorithm 4
%\end{center}


\begin{figure}[H]
  \begin{minipage}{\linewidth}
	\label{algorithm4}
    \begin{algorithm}[H]
      \caption{Distributed quantum algorithm for DJ problem (four distributed computing nodes) (III)}
      \begin{algorithmic}[1]
        \State $|\Psi_0\rangle = |0^{n-2}\rangle|0^{14}\rangle$;

        \State $|\Psi_1\rangle = (H^{\otimes n-2}\otimes I^{\otimes {14}})|\Psi_0\rangle$; %=(H^{\otimes n-2}|0^{n-2}\rangle)|0^{14}\rangle=\frac{1}{\sqrt{2^{n-2}}}\sum_{u\in\{0,1\}^{n-2}}|u\rangle|0^{14}\rangle

        \State The 1st computing node and the 2nd computing node query their oracles under the control of the first quantum register: 
        $|\Psi_3\rangle% \frac{1}{\sqrt{2^{n-2}}}\sum_{u\in\{0,1\}^{n-2}}|u\rangle|f_{00}(u)\rangle|f_{01}(u)\rangle|0^{12}\rangle
        =\frac{1}{\sqrt{2^{n-2}}}\sum_{u\in\{0,1\}^{n-2}}|u\rangle|f(u00)\rangle|f(u01)\rangle|0^{12}\rangle$;

     
        
         \State The 4th quantum register performs two-bit controlled $X$ gate under the control of the 2nd and 3rd quantum registers: 
         $|\Psi_4\rangle=\frac{1}{\sqrt{2^{n-2}}}\sum_{u\in\{0,1\}^{n-2}}|u\rangle|f(u00)\rangle|f(u01)\rangle|f(u00)\land f(u01)\rangle|0^{11}\rangle$;
         
         
           \State The 3rd quantum register performs controlled $X$ gate under the control of the 2nd quantum register: 
           
           $|\Psi_5\rangle=\frac{1}{\sqrt{2^{n-2}}}\sum_{u\in\{0,1\}^{n-2}}|u\rangle|f(u00)\rangle|f(u00)\oplus f(u01)\rangle|f(u00)\land f(u01)\rangle|0^{11}\rangle$;
           
           
            %\State The 5th quantum register performs controlled $X$ gate under the control of the 3rd quantum register: $|\Psi_6\rangle=\frac{1}{\sqrt{2^{n-2}}}\sum_{u\in\{0,1\}^{n-2}}|u\rangle|f(u00)\rangle|f(u00)\oplus f(u01)\rangle|f(u00)\land f(u01)\rangle|f(u00)\oplus f(u01)\rangle|0^{10}\rangle$;
            
            
            
         \State  Perform the same operations on the 3rd  and the 4th computing node as on the 1st  and the 2nd computing node: 
         
         $|\Psi_6\rangle=\frac{1}{\sqrt{2^{n-2}}}\sum_{u\in\{0,1\}^{n-2}}|u\rangle|f(u00)\rangle|f(u00)\oplus f(u01)\rangle|f(u00)\land f(u01)\rangle$
         
         $\qquad\qquad|f(u10)\rangle|f(u10)\oplus f(u11)\rangle|f(u10)\land f(u11)\rangle|0^{8}\rangle$;
         
         %\State  Perform similar operations on the 3rd computing node and the 4th computing node as on the 1st  computing node and the 2nd computing node: $|\Psi_6\rangle=\frac{1}{\sqrt{2^{n-2}}}\sum_{u\in\{0,1\}^{n-2}}|u\rangle|f(u00)\rangle|f(u00)\oplus f(u01)\rangle|f(u00)\land f(u01)\rangle|f(u10)\rangle|f(u10)\oplus f(u11)\rangle|f(u10)\land f(u11)\rangle|0^{6}\rangle$;
        
        
          \State The 8th quantum register performs its own $A$ under the control of the 3rd and 6th quantum registers on $\Ket{\Psi_6}$ to get  $\Ket{\Psi_7}$;
          
          %$|\Psi_7\rangle=\frac{1}{\sqrt{2^{n-2}}}\sum_{u\in\{0,1\}^{n-2}}|u\rangle|f(u00)\rangle|f(u00)\oplus f(u01)\rangle|f(u00)\land f(u01)\rangle|f(u10)\rangle$
          
         % $\qquad\qquad|f(u10)\oplus f(u11)\rangle|f(u10)\land f(u11)\rangle|f(u00)\oplus f(u01)+f(u10)\oplus f(u11)\rangle|0^{6}\rangle$;
          
          
          \State The 9th quantum register performs its own $A$ under the control of the 4th and 7th quantum registers on $\Ket{\Psi_7}$ to get  $\Ket{\Psi_8}$; 
          
          %$|\Psi_8\rangle=\frac{1}{\sqrt{2^{n-2}}}\sum_{u\in\{0,1\}^{n-2}}|u\rangle|f(u00)\rangle|f(u00)\oplus f(u01)\rangle|f(u00)\land f(u01)\rangle|f(u10)\rangle$
          
          %$\qquad\qquad|f(u10)\oplus f(u11)\rangle|f(u10)\land f(u11)\rangle|f(u00)\oplus f(u01)+f(u10)\oplus f(u11)\rangle$
          
          %$\qquad\qquad|f(u00)\land f(u01)+f(u10)\land f(u11)\rangle|0^{4}\rangle$;
          
          
         
         
        
        
         \State $\Ket{\Psi_9}=\left(I^{\otimes n+4}\otimes V\otimes I\right)\Ket{\Psi_8}$;
         
         %The 10th quantum register performs its own $V$ under the control of the 8th and 9th quantum registers: 
         
         %$|\Psi_9\rangle=\frac{1}{\sqrt{2^{n-2}}}\sum_{u\in\{0,1\}^{n-2}}|u\rangle|f(u00)\rangle|f(u00)\oplus f(u01)\rangle|f(u00)\land f(u01)\rangle|f(u10)\rangle$
         
         %$\qquad\qquad|f(u10)\oplus f(u11)\rangle|f(u10)\land f(u11)\rangle|f(u00)\oplus f(u01)+f(u10)\oplus f(u11)\rangle$
         
        % $\qquad\qquad|f(u00)\land f(u01)+f(u10)\land f(u11)\rangle|\left\lfloor\frac{1-{\rm sgn}(\Delta(u))}{2}\right\rfloor\rangle||\Delta(u)|\rangle\ket{0}$;%, where $\Delta(u)=2-(f(u00)\oplus f(u01)+f(u10)\oplus f(u11))-2(f(u00)\land f(u01)+f(u10)\land f(u11))$;
        
        
        
        \State $\Ket{\Psi_{10}}=\left(I^{\otimes n+8}\otimes Z\otimes R'\right)\Ket{\Psi_9}$;
        %The quantum gate $Z$ acts on the 10th qubit, the last quantum register performs its own $R$ under the control of the bottom two quantum registers: 
        
        %$|\Psi_{10}\rangle=\frac{1}{\sqrt{2^{n-2}}}\sum_{u\in\{0,1\}^{n-2}}(-1)^{\left\lfloor\frac{1-{\rm sgn}(\Delta(u))}{2}\right\rfloor}|u\rangle|f(u00)\rangle|f(u00)\oplus f(u01)\rangle$
        
        %$\qquad\qquad|f(u00)\land f(u01)\rangle|f(u10)\rangle|f(u10)\oplus f(u11)\rangle|f(u10)\land f(u11)\rangle$
        
        %$\qquad\qquad|f(u00)\oplus f(u01)+f(u10)\oplus f(u11)\rangle|f(u00)\land f(u01)+f(u10)\land f(u11)\rangle$
        
        %$\qquad\qquad|\left\lfloor\frac{1-{\rm sgn}(\Delta(u))}{2}\right\rfloor\rangle||\Delta(u)|\rangle\left(\frac{|\Delta(u)|}{2}\ket{0}+\sqrt{1-\left(\frac{|\Delta(u)|}{2}\right)^2}\ket{1}\right)$;

       % \State The fourth quantum register performs its own $U$ under the control of the second and third  quantum registers: $|\Psi_6\rangle=\frac{1}{\sqrt{2^{n-1}}}\sum_{u\in\{0,1\}^{n-1}}(-1)^{\left\lfloor\frac{1-{\rm sgn}(\delta(u))}{2}\right\rfloor}|u\rangle|f(u0)\rangle|f(u1)\rangle|0^{3}\rangle(\frac{|\delta(u)|}{2}\ket{0}+\sqrt{1-(\frac{|\delta(u)|}{2})^2}\ket{1})$;
        
       % \State Each computing node queries its own oracle under the control of the first quantum register: $|\Psi_8\rangle=\frac{1}{\sqrt{2^{n-1}}}\sum_{u\in\{0,1\}^{n-1}}(-1)^{\left\lfloor\frac{1-{\rm sgn}(\delta(u))}{2}\right\rfloor}|u\rangle\ket{0}\ket{0}|0^{3}\rangle(\frac{|\delta(u)|}{2}\ket{0}+\sqrt{1-(\frac{|\delta(u)|}{2})^2}\ket{1})$;
       
       \State Uncomputing the quantum registers in the middle part:
       
        $|\Psi_{11}\rangle=\frac{1}{\sqrt{2^{n-2}}}\sum_{u\in\{0,1\}^{n-2}}(-1)^{\left\lfloor\frac{1-{\rm sgn}(\Delta(u))}{2}\right\rfloor}|u\rangle|0^{13}\rangle\left(\frac{|\Delta(u)|}{2}\ket{0}+\sqrt{1-\left(\frac{|\Delta(u)|}{2}\right)^2}\ket{1}\right)$

       \State  Measure the last quantum register, if the result is 1, output $f$ is balanced. If the result is 0: 
       
       $|\Psi_{12}\rangle=\frac{1}{\sqrt{\sum_{u\in\{0,1\}^{n-2}}\Delta^2(u)}}\sum_{u\in\{0,1\}^{n-2}}\Delta(u)|u\rangle|0^{14}\rangle$;

        \State $|\Psi_{13}\rangle=(H^{\otimes n-2}\otimes I^{\otimes {14}})|\Psi_{12}\rangle$;%=\frac{1}{\sqrt{\sum_{u\in\{0,1\}^{n-2}}\Delta^2(u)}}\sum_{u\in\{0,1\}^{n-2}}\Delta(u)\sum_{z\in\{0,1\}^{n-2}}\frac{1}{\sqrt{2^{n-2}}}(-1)^{u\cdot z}\ket{z}|0^{14}\rangle=\sum_{z\in\{0,1\}^{n-2}}\sum_{u\in\{0,1\}^{n-2}}\frac{\Delta(u)}{\sqrt{\sum_{u\in\{0,1\}^{n-2}}\Delta^2(u)}}\frac{(-1)^{u\cdot z}}{\sqrt{2^{n-2}}}\ket{z}|0^{14}\rangle$;

        \State Measure the first quantum register, if the result is not $0^{n-2}$, output $f$ is balanced. Otherwise, output $f$ is constant.
      \end{algorithmic}
    \end{algorithm}

  \end{minipage}
  \end{figure}



\begin{figure}[H]%参数[h]表示紧跟着文字
  \centering%居中
  \includegraphics[width=6.5in]{4.png}
  \caption{The circuit for the distributed quantum algorithm for DJ problem  (four   computing nodes) (III).}
  \label{algorithm_four_nodes (III)}
\end{figure}




%\begin{center}
%\bf B2.\quad Correctness analysis of Algorithm 4
%\end{center}

In the following, we prove the correctness of  Algorithm 4, we write out the state after the first step of the algorithm in FIG. \ref{algorithm_four_nodes (III)}.
\begin{equation}
\begin{split}
  |\Psi_1\rangle=\frac{1}{\sqrt{2^{n-2}}}\sum_{u\in\{0,1\}^{n-2}}|u\rangle|0^{14}\rangle.
\end{split} 
\end{equation} 

Then the algorithm queries the oracle $O_{f_{00}}$ and $O_{f_{01}}$, resulting in the following state:
\begin{equation}
\begin{split}
  |\Psi_3\rangle=&\frac{1}{\sqrt{2^{n-2}}}\sum_{u\in\{0,1\}^{n-2}}|u\rangle|f_{00}(u)\rangle|f_{01}(u)\rangle|0^{12}\rangle\\
   =&\frac{1}{\sqrt{2^{n-2}}}\sum_{u\in\{0,1\}^{n-2}}|u\rangle|f(u00)\rangle|f(u01)\rangle|0^{12}\rangle.
\end{split} 
\end{equation} 


The 4th quantum register performs two-bit controlled $X$ gate under the control of the 2nd and 3rd quantum registers, we get the state:
\begin{equation}
\begin{split}
|\Psi_4\rangle
=&\frac{1}{\sqrt{2^{n-2}}}\sum_{u\in\{0,1\}^{n-2}}|u\rangle|f(u00)\rangle|f(u01)\rangle|f(u00)\land f(u01)\rangle|0^{11}\rangle.
\end{split} 
\end{equation} 

The 3rd quantum register performs controlled $X$ gate under the control of the 2nd quantum register, we get the state:
\begin{equation}
\begin{split}
|\Psi_5\rangle
=&\frac{1}{\sqrt{2^{n-2}}}\sum_{u\in\{0,1\}^{n-2}}|u\rangle|f(u00)\rangle|f(u00)\oplus f(u01)\rangle|f(u00)\land f(u01)\rangle|0^{11}\rangle.
\end{split} 
\end{equation}    


 Perform the same operations on the remaining compute nodes  as on the 1st  computing node and the 2nd computing node, we have the following state:
\begin{equation}
\begin{split}
 |\Psi_6\rangle=&\frac{1}{\sqrt{2^{n-2}}}\sum_{u\in\{0,1\}^{n-2}}|u\rangle(\otimes_{w\in\{0,1\}0}|f_w(u)\rangle|f_w(u)\oplus f_{w+1}(u)\rangle|f_w(u)\land f_{w+1}(u)\rangle)|0^{8}\rangle\\
 =&\frac{1}{\sqrt{2^{n-2}}}\sum_{u\in\{0,1\}^{n-2}}|u\rangle(\otimes_{w\in\{0,1\}0}|f(uw)\rangle|f(uw)\oplus f(u(w+1))\rangle|f(uw)\land f(u(w+1))\rangle)|0^{8}\rangle.
\end{split} 
\end{equation}   



The $8$-th quantum register performs its own $A$ under the control of the $(\{3k|1\leq k\leq 2\})$-th quantum registers. Then we obtain the following state:
\begin{equation}
\begin{split}
|\Psi_7\rangle=&\frac{1}{\sqrt{2^{n-2}}}\sum_{u\in\{0,1\}^{n-2}}|u\rangle(\otimes_{w\in\{0,1\}0}|f(uw)\rangle|f(uw)\oplus f(u(w+1))\rangle|f(uw)\land f(u(w+1))\rangle)\\
&\qquad\Ket{\sum_{w\in\{0,1\}0}f(uw)\oplus f(u(w+1))}|0^{6}\rangle.
\end{split} 
\end{equation}   



The $9$-th quantum register performs its own $A$ under the control of the $(\{3k+1|1\leq k\leq 2\})$-th quantum registers. Then we obtain the following state:
\begin{equation}
\begin{split}
|\Psi_8\rangle=&\frac{1}{\sqrt{2^{n-2}}}\sum_{u\in\{0,1\}^{n-2}}|u\rangle(\otimes_{w\in\{0,1\}0}|f(uw)\rangle|f(uw)\oplus f(u(w+1))\rangle|f(uw)\land f(u(w+1))\rangle)\\
&|\sum_{w\in\{0,1\}0}f(uw)\oplus f(u(w+1))\rangle|\sum_{w\in\{0,1\}0}f(uw)\land f(u(w+1))\rangle|0^{4}\rangle.
\end{split} 
\end{equation}             
           

The $10$-th quantum register performs its own $V$ under the control of  $8$-th and the $9$-th quantum registers, we get the following state:
\begin{equation}
\begin{split}
|\Psi_9\rangle=&\frac{1}{\sqrt{2^{n-2}}}\sum_{u\in\{0,1\}^{n-2}}|u\rangle(\otimes_{w\in\{0,1\}0}|f(uw)\rangle|f(uw)\oplus f(u(w+1))\rangle|f(uw)\land f(u(w+1))\rangle)\\
&|\left\lfloor\frac{1-{\rm sgn}(\Delta(u))}{2}\right\rfloor\rangle||\Delta(u)|\rangle\ket{0}.
\end{split} 
\end{equation}   


Then the quantum gate $Z$ acts on the $10$-th qubit, the last quantum register performs its own $R$ under the control of the bottom $2$ quantum registers, we get the following state:
\begin{equation}
\begin{split}
|\Psi_{10}\rangle=&\frac{1}{\sqrt{2^{n-2}}}\sum_{u\in\{0,1\}^{n-t}}|u\rangle(\otimes_{w\in\{0,1\}0}|f(uw)\rangle|f(uw)\oplus f(u(w+1))\rangle|f(uw)\land f(u(w+1))\rangle)\\
&|\left\lfloor\frac{1-{\rm sgn}(\Delta(u))}{2}\right\rfloor\rangle||\Delta(u)|\rangle\left(\frac{|\Delta(u)|}{2}\ket{0}+\sqrt{1-\left(\frac{|\Delta(u)|}{2}\right)^2}\ket{1}\right).
\end{split} 
\end{equation}  


Uncomputing the middle $13$ quantum registers, resulting in the following state:
\begin{equation}
\begin{split}
|\Psi_{11}\rangle=&\frac{1}{\sqrt{2^{n-2}}}\sum_{u\in\{0,1\}^{n-2}}(-1)^{\left\lfloor\frac{1-{\rm sgn}(\Delta(u))}{2}\right\rfloor}|u\rangle|0^{13}\rangle\left(\frac{|\Delta(u)|}{2}\ket{0}+\sqrt{1-\left(\frac{|\Delta(u)|}{2}\right)^2}\ket{1}\right).
\end{split} 
\end{equation}  



By tracing out the states of the $13$ bit registers in the middle, we can get the state:
\begin{equation}
\begin{split}
&|\Psi_{11}'\rangle=\frac{1}{\sqrt{2^{n-2}}}\sum_{u\in\{0,1\}^{n-2}}(-1)^{\left\lfloor\frac{1-{\rm sgn}(\Delta(u))}{2}\right\rfloor}|u\rangle\otimes\left(\frac{|\Delta(u)|}{2}\ket{0}+\sqrt{1-\left(\frac{|\Delta(u)|}{2}\right)^2}\ket{1}\right).
\end{split}
\end{equation}














After measurement on the last register, if the result is $1$, then $\exists$ $u\in\{0,1\}^{n-2}$ such that $|\Delta(u)|\neq 2$. From Theorem \ref{The6}, we know that $f$ is balanced. If the measurement result is $0$, we get the state:
\begin{equation}
\begin{split}
|\Psi'_{12}\rangle=\sum_{u\in\{0,1\}^{n-2}}\frac{(-1)^{\left\lfloor\frac{1-{\rm sgn}(\Delta(u))}{2}\right\rfloor}|\Delta(u)|}{\sqrt{\sum\limits_{u\in\{0,1\}^{n-2}}|\Delta(u)|^2}}|u\rangle\ket{0}.
\end{split}
\end{equation}




By tracing out the state of the $1$ bit register in the last, we can get the state:
\begin{equation}
\begin{split}
|\Psi''_{12}\rangle=&\sum_{u\in\{0,1\}^{n-2}}\frac{(-1)^{\left\lfloor\frac{1-{\rm sgn}(\Delta(u))}{2}\right\rfloor}|\Delta(u)|}{\sqrt{\sum\limits_{u\in\{0,1\}^{n-2}}|\Delta(u)|^2}}|u\rangle\\
=&\sum_{u\in\{0,1\}^{n-2}}\frac{\Delta(u)}{\sqrt{\sum\limits_{u\in\{0,1\}^{n-2}}\Delta^2(u)}}|u\rangle.
\end{split}
\end{equation}


After Hadamard transformation on the first register, we can get the following state:
\begin{equation}
\begin{split}
	|\Psi'_{13}\rangle=&H^{\otimes n-2}|\Psi''_{12}\rangle\\
	=&\sum_{u,z\in\{0,1\}^{n-2}}\frac{\Delta(u)}{\sqrt{\sum\limits_{u\in\{0,1\}^{n-2}}\Delta^2(u)}}\frac{(-1)^{u\cdot z}}{\sqrt{2^{n-2}}}\ket{z}.
\end{split} 
\end{equation} 



The probability of measuring the first quantum register with the result of $0^{n-2}$ is
\begin{equation}
\left|\frac{1}{\sqrt{2^{n-2}\sum\limits_{u\in\{0,1\}^{n-2}}\Delta^2(u)}}\sum\limits_{u\in\{0,1\}^{n-2}}\Delta(u)\right|^2.
\end{equation}

%After measurement on the first register, according to Theorem \ref{The2}, if the result is not $0^{n-t}$, then $f$ is  balanced, otherwise $f$ is constant.

%In the step 12 of Algorithm 5, if the measurement of the last quantum register is $1$, then there $\exists$ $u\in\{0,1\}^{n-t}$ such that $|\Delta(u)|\neq 2^{t-1}$. According to Theorem \ref{The6}, it follows that $f$ is balanced.

%In the step 13 of Algorithm 5, the probability of measuring the first quantum register with the result of $0^{n-t}$ is
%\begin{equation}
%\left|\frac{1}{\sqrt{2^{n-t}\sum\limits_{u\in\{0,1\}^{n-t}}\Delta^2(u)}}\sum\limits_{u\in\{0,1\}^{n-t}}\Delta(u)\right|^2.
%\end{equation}

After measurement on the first register, according to Theorem \ref{The7} and Theorem \ref{The8}, if the result is not $0^{n-2}$, then $f$ is  balanced, otherwise $f$ is constant.


\fi



 %%%%%%以上是原算法4的部分%%%%%%%
 
 
 
 
 
 

%\newpage


%\begin{center}
%\bf C1.\quad Algorithm 5
%\end{center}







%%%%%%以下是原算法1的相关部分%%%%%%

\iffalse

In the following, we give the unitary operators involved in Algorithm \ref{algorithm2} .


The effect of operator $U$ in FIG. \ref{algorithm_two_nodes (I)} is: $\forall a,b,c \in \{0,1\}$ and $d \in \{0,1\}^{2}$,
\begin{equation}
\begin{split}U\ket{a}\ket{b}\ket{c}\ket{d}=\ket{a}\ket{b}|c\oplus\lfloor\frac{1-{\rm sgn}(2-2(a+b))}{2}\rfloor\rangle|d\oplus|2-2(a+b)|\rangle.
\end{split}
\end{equation}

Intuitively, the effect of $U$ is to perform arithmetic operations on the values in the first two registers, and xor to the third and fourth register. $U$ does not change the states of the first two registers (i.e. $\ket{a}$ and $\ket{b}$). %We call these two registers the control registers of $U$. %In order to show the control registers of $U$ in quantum circuit diagram clearer, we use $\Box $ to mark the control registers of $U$ in FIG. \ref{algorithm_two_nodes (I)}. Similarly, we use $\Box $ to mark the control registers of every oracle in FIG. \ref{algorithm_multiple_nodes (I)}, FIG. \ref{algorithm_two_nodes (II)}, FIG. \ref{algorithm_four_nodes (III)} and FIG. \ref{algorithm_multiple_nodes (III)}.

The effect of operator $R$ in FIG. \ref{algorithm_two_nodes (I)} is: $\forall d \in \{0,1\}^2$ and $e \in \{0,1\}$,
\begin{equation}
\begin{split}
	R\ket{d}\ket{e}=\ket{d}(\frac{d}{2}\ket{e}+\sqrt{1-(\frac{d}{2})^2}\Ket{1\oplus e}).
\end{split}
\end{equation}

Intuitively, the effect of $R$ is to perform the controlled rotation on the state in the second register based on the value in the first register. $R$ does not change the states of the first register (i.e. $\ket{d}$). %We call the register the control register of $R$. 

%\begin{center}
%\bf B1.\quad Algorithm \ref{algorithm2} 
%\end{center}

\begin{figure}[H]
  \begin{minipage}{\linewidth}
	\label{algorithm1}
    \begin{algorithm}[H]
      \caption{Distributed quantum algorithm for DJ problem (two distributed computing nodes) (I)}
      \begin{algorithmic}[1]
        \State $|\psi_0\rangle = |0^{n-1}\rangle\ket{0}\ket{0}|0^{4}\rangle$;

        \State $|\psi_1\rangle = \left(H^{\otimes n-1}\otimes I^{\otimes {6}}\right)|\psi_0\rangle$; %=(H^{\otimes n-1}|0^{n-1}\rangle)\ket{0}\ket{0}|0^{4}\rangle=\frac{1}{\sqrt{2^{n-1}}}\sum_{u\in\{0,1\}^{n-1}}|u\rangle\ket{0}\ket{0}|0^{4}\rangle$;

        \State Each computing node queries its own oracle under the control of the first quantum register: 
       
        $|\psi_3\rangle%=\frac{1}{\sqrt{2^{n-1}}}\sum_{u\in\{0,1\}^{n-1}}|u\rangle|f_0(u)\rangle|f_1(u)\rangle|0^{4}\rangle
        =\frac{1}{\sqrt{2^{n-1}}}\sum_{u\in\{0,1\}^{n-1}}|u\rangle|f(u0)\rangle|f(u1)\rangle|0^{4}\rangle$;

        \State The fourth quantum register performs its own $U$ under the control of the second and third quantum registers: 
        
        $|\psi_4\rangle=\frac{1}{\sqrt{2^{n-1}}}\sum_{u\in\{0,1\}^{n-1}}|u\rangle|f(u0)\rangle|f(u1)\rangle|\left\lfloor\frac{1-{\rm sgn}(\delta(u))}{2}\right\rfloor\rangle||\delta(u)|\rangle\ket{0}$;%, where $\delta(u)=2-2(f(u0)+f(u1))$;
        
        \State The quantum gate $Z$ acts on the fourth qubit, the last quantum register performs $R$ under the control of the 
        bottom two quantum registers: 
        
        $|\psi_5\rangle=\frac{1}{\sqrt{2^{n-1}}}\sum_{u\in\{0,1\}^{n-1}}(-1)^{\left\lfloor\frac{1-{\rm sgn}(\delta(u))}{2}\right\rfloor}|u\rangle|f(u0)\rangle|f(u1)\rangle|\left\lfloor\frac{1-{\rm sgn}(\delta(u))}{2}\right\rfloor\rangle||\delta(u)|\rangle$ 
        
        $\qquad\qquad\left(\frac{|\delta(u)|}{2}\ket{0}+\sqrt{1-\left(\frac{|\delta(u)|}{2}\right)^2}\ket{1}\right)$;

        \State The fourth quantum register performs its own $U$ under the control of the second and third  quantum registers: 
        
        $|\psi_6\rangle=\frac{1}{\sqrt{2^{n-1}}}\sum_{u\in\{0,1\}^{n-1}}(-1)^{\left\lfloor\frac{1-{\rm sgn}(\delta(u))}{2}\right\rfloor}|u\rangle|f(u0)\rangle|f(u1)\rangle|0^{3}\rangle\left(\frac{|\delta(u)|}{2}\ket{0}+\sqrt{1-\left(\frac{|\delta(u)|}{2}\right)^2}\ket{1}\right)$;
        
        \State Each computing node queries its own oracle under the control of the first quantum register:
        
         $|\psi_8\rangle=\frac{1}{\sqrt{2^{n-1}}}\sum_{u\in\{0,1\}^{n-1}}(-1)^{\left\lfloor\frac{1-{\rm sgn}(\delta(u))}{2}\right\rfloor}|u\rangle\ket{0}\ket{0}|0^{3}\rangle\left(\frac{|\delta(u)|}{2}\ket{0}+\sqrt{1-\left(\frac{|\delta(u)|}{2}\right)^2}\ket{1}\right)$;

       \State  Measure the last quantum register,  if the result is 1, output $f$ is balanced. If the result is 0: 
       
       $|\psi_9\rangle=%\frac{\frac{1}{\sqrt{2^{n-1}}}\sum_{u\in\{0,1\}^{n-1}}(-1)^{\left\lfloor\frac{1-{\rm sgn}(\delta(u))}{2}\right\rfloor}|u\rangle\ket{0}\ket{0}|0^{3}\rangle\frac{|\delta(u)|}{2}\ket{0}}{\left|\left|\frac{1}{\sqrt{2^{n-1}}}\sum_{u\in\{0,1\}^{n-1}}(-1)^{\left\lfloor\frac{1-{\rm sgn}(\delta(u))}{2}\right\rfloor}|u\rangle\ket{0}\ket{0}|0^{3}\rangle\frac{|\delta(u)|}{2}\ket{0}\right|\right|}=
       \frac{1}{\sqrt{\sum_{u\in\{0,1\}^{n-1}}\delta^2(u)}}\sum_{u\in\{0,1\}^{n-1}}\delta(u)|u\rangle|0^6\rangle$;

        \State $|\psi_{10}\rangle=\left(H^{\otimes n-1}\otimes I^{\otimes {6}}\right)|\psi_{9}\rangle$;%=\frac{1}{\sqrt{\sum_{u\in\{0,1\}^{n-1}}\delta^2(u)}}\sum_{u\in\{0,1\}^{n-1}}\delta(u)\sum_{z\in\{0,1\}^{n-1}}\frac{1}{\sqrt{2^{n-1}}}(-1)^{u\cdot z}\ket{z}|0^{6}\rangle=\sum_{z\in\{0,1\}^{n-1}}\sum_{u\in\{0,1\}^{n-1}}\frac{\delta(u)}{\sqrt{\sum_{u\in\{0,1\}^{n-1}}\delta^2(u)}}\frac{(-1)^{u\cdot z}}{\sqrt{2^{n-1}}}\ket{z}|0^{6}\rangle$;

        \State Measure the first quantum register, if the result is not $0^{n-1}$, output $f$ is balanced. Otherwise, output $f$ is constant.
      \end{algorithmic}
    \end{algorithm}

  \end{minipage}
  \end{figure}



\begin{figure}[H]%参数[h]表示紧跟着文字
  \centering%居中
  \includegraphics[width=6.5in]{1.png}
  \caption{The circuit for the distributed quantum algorithm for DJ problem  (two  computing nodes) (I).}
  \label{algorithm_two_nodes (I)}
\end{figure}




%\begin{center}
%\bf B2.\quad Correctness analysis of Algorithm \ref{algorithm2} 
%\end{center}

In the following, we prove the correctness of  Algorithm \ref{algorithm2} , we write out the state after the first step of  the algorithm in FIG. \ref{algorithm_two_nodes (I)}.
\begin{align}
  |\psi_1\rangle=\frac{1}{\sqrt{2^{n-1}}}\sum_{u\in\{0,1\}^{n-1}}|u\rangle\ket{0}\ket{0}|0^{4}\rangle
\end{align}

Then the algorithm querie the oracle $O_{f_{0}}$, resulting in the following state:
\begin{align}
  |\psi_2\rangle&=\left(O_{f_{0}}\frac{1}{\sqrt{2^{n-1}}}\sum_{u\in\{0,1\}^{n-1}}|u\rangle\ket{0}\right)\ket{0}|0^{4}\rangle\\
  &=\frac{1}{\sqrt{2^{n-1}}}\sum_{u\in\{0,1\}^{n-1}}|u\rangle|f_{0}(u)\rangle\ket{0}|0^{4}\rangle\\
  &=\frac{1}{\sqrt{2^{n-1}}}\sum_{u\in\{0,1\}^{n-1}}|u\rangle|f(u0)\rangle \ket{0}|0^{4}\rangle.
\end{align}

Then the algorithm querie the oracle $O_{f_{1}}$, resulting in the following state:
\begin{equation}
  |\psi_3\rangle=\frac{1}{\sqrt{2^{n-1}}}\sum_{u\in\{0,1\}^{n-1}}|u\rangle|f(u0)\rangle|f(u1)\rangle|0^{4}\rangle.
\end{equation}

After the action of the operator $U$, we have the following state:
\begin{equation}
\begin{split}
  &|\psi_4\rangle=\frac{1}{\sqrt{2^{n-1}}}\sum_{u\in\{0,1\}^{n-t}}|u\rangle|f(u0)\rangle|f(u1)\rangle\otimes\Ket{\left\lfloor\frac{1-{\rm sgn}(\delta(u))}{2}\right\rfloor}\Ket{|\delta(u)|}\ket{0}, 
\end{split}
\end{equation}
where $\delta(u)=2-2\sum_{w\in\{0,1\}}f(uw)$.

After the action of the operator $Z$ and $R$, we have the following state:
\begin{equation}
\begin{split}
  &|\psi_5\rangle=\frac{1}{\sqrt{2^{n-1}}}\sum_{u\in\{0,1\}^{n-1}}(-1)^{\left\lfloor\frac{1-{\rm sgn}(\delta(u))}{2}\right\rfloor}|u\rangle\otimes|f(u0)\rangle|f(u1)\rangle\Ket{\left\lfloor\frac{1-{\rm sgn}(\delta(u))}{2}\right\rfloor}\Ket{|\delta(u)|}\\
  &\qquad\quad\otimes\left(\frac{|\delta(u)|}{2}\ket{0}+\sqrt{1-\left(\frac{|\delta(u)|}{2}\right)^2}\ket{1}\right).
\end{split}
\end{equation}

After that, we use the $U$ operator again, and restore the status of the $3$ bit registers of  $U$ operator  to $\ket{0}$. Then we obtain the following state:
\begin{equation}
\begin{split}
  &|\psi_6\rangle=\frac{1}{\sqrt{2^{n-1}}}\sum_{u\in\{0,1\}^{n-1}}(-1)^{\left\lfloor\frac{1-{\rm sgn}(\delta(u))}{2}\right\rfloor}|u\rangle\otimes|f(u0)\rangle|f(u1)\rangle|0^{3}\rangle\otimes\left(\frac{|\delta(u)|}{2}\ket{0}+\sqrt{1-\left(\frac{|\delta(u)|}{2}\right)^2}\ket{1}\right).
\end{split}
\end{equation}


After that, we query each oracle again and restore the status of the $2$ bit registers to $\ket{0}$. Then we obtain the following state:
\begin{equation}
\begin{split}
&|\psi_8\rangle=\frac{1}{\sqrt{2^{n-1}}}\sum_{u\in\{0,1\}^{n-1}}(-1)^{\left\lfloor\frac{1-{\rm sgn}(\delta(u))}{2}\right\rfloor}|u\rangle|0^{5}\rangle\otimes(\frac{|\delta(u)|}{2}\ket{0}+\sqrt{1-\left(\frac{|\delta(u)|}{2}\right)^2}\ket{1}).
\end{split}
\end{equation}

By tracing out the states of the $5$ bit registers in the middle, we can get the state:
\begin{equation}
\begin{split}
&|\psi_8'\rangle=\frac{1}{\sqrt{2^{n-1}}}\sum_{u\in\{0,1\}^{n-1}}(-1)^{\left\lfloor\frac{1-{\rm sgn}(\delta(u))}{2}\right\rfloor}|u\rangle\otimes\left(\frac{|\delta(u)|}{2}\ket{0}+\sqrt{1-\left(\frac{|\delta(u)|}{2}\right)^2}\ket{1}\right).
\end{split}
\end{equation}

After measurement on the last register, if the result is $1$, then $\exists$ $u\in\{0,1\}^{n-1}$ such that $|\delta(u)|\neq 2$. From Theorem \ref{The1}, we know that $f$ is balanced. If the measurement result is $0$, we get the state:
\begin{equation}
\begin{split}
&|\psi_9'\rangle=\sum_{u\in\{0,1\}^{n-1}}\frac{(-1)^{\left\lfloor\frac{1-{\rm sgn}(\delta(u))}{2}\right\rfloor}|\delta(u)|}{\sqrt{\sum\limits_{u\in\{0,1\}^{n-1}}|\delta(u)|^2}}|u\rangle\ket{0}.
\end{split}
\end{equation}

By tracing out the state of the $1$ bit register in the last, we can get the state:
\begin{equation}
\begin{split}
|\psi_9''\rangle=&\sum_{u\in\{0,1\}^{n-1}}\frac{(-1)^{\left\lfloor\frac{1-{\rm sgn}(\delta(u))}{2}\right\rfloor}|\delta(u)|}{\sqrt{\sum\limits_{u\in\{0,1\}^{n-1}}|\delta(u)|^2}}|u\rangle\\
=&\sum_{u\in\{0,1\}^{n-1}}\frac{\delta(u)}{\sqrt{\sum\limits_{u\in\{0,1\}^{n-1}}\delta^2(u)}}|u\rangle.
\end{split}
\end{equation}


After Hadamard transformation on the first register, we can get the following state:
\begin{equation}
\begin{split}
	|\psi'_{10}\rangle=&H^{\otimes n-1}|\psi''_9\rangle\\
	=&\sum_{u,z\in\{0,1\}^{n-1}}\frac{\delta(u)}{\sqrt{\sum\limits_{u\in\{0,1\}^{n-1}}\delta^2(u)}}\frac{(-1)^{u\cdot z}}{\sqrt{2^{n-1}}}\ket{z}.
\end{split} 
\end{equation} 



The probability of measuring the first quantum register with the result of $0^{n-1}$ is
\begin{equation}
\left|\frac{1}{\sqrt{2^{n-1}\sum\limits_{u\in\{0,1\}^{n-1}}\delta^2(u)}}\sum\limits_{u\in\{0,1\}^{n-1}}\delta(u)\right|^2.
\end{equation}


After measurement on the first register, according to Theorem \ref{The2}, if the result is not $0^{n-1}$, then $f$ is  balanced, otherwise $f$ is constant. 

\fi

%%%%%%以上是原算法1的相关部分%%%%%%



\section{Design and analysis of Algorithm  \ref{algorithm3}} \label{sec:Distributed quantum algorithm for  DJ problem (Algorithm 1)}

 % We propose two methods to solve  DJ problem in distributed scenario, labeled as method (I) and method (II) respectively, where method (I) includes Algorithm \ref{algorithm2} , method (II) includes Algorithm \ref{algorithm3}  and Algorithm \ref{algorithm5} .
  
  The  idea of our algorithms is  first based on the  essential characteristics of DJ problem in  distributed scenario, and is insights from the distributed Simon's quantum algorithm \cite{Tan2022DQCSimon}, by using   the specific unitary operator  and  quantum teleportation to combine the corresponding oracles of multiple subfunctions. Then, the joint information between multiple subfunctions is extracted by the specific controlled rotation operator based on the technique of HHL algorithm \cite{HHL_2009}, and the quantum state consistent with the structure of  DJ problem is obtained. %Finally, we use  the structure to solve DJ problem exactly.
  
  
\iffalse
    We design three algorithms, namely  Algorithm \ref{algorithm3} , Algorithm \ref{algorithm2}  and Algorithm \ref{algorithm5} .
  Note that the oracle query of each quantum computing node in  Algorithm \ref{algorithm2}  and Algorithm \ref{algorithm5}   can actually be completed in parallel. With the help of auxiliary $\left(2^t-1\right)\left(n-t\right)$ qubits, we can change the state of the control register after the first Hadamard transformation $\frac{1}{\sqrt{2^{n-t}}}\sum_{u\in\{0,1\}^{n-t}}|u\rangle$ to $\frac{1}{\sqrt{2^{n-t}}}\sum_{u\in\{0,1\}^{n-t}}\underbrace{|u\rangle|u\rangle \ldots |u\rangle}_{2^t}$. %That is, we changed the control register from one group to the same $2^t$ groups. 
  In fact, after the first Hadamard transformation,  we can  teleport each group of $n-t$ control bits to every quantum computing node and use these to control the oracles of the computing nodes.
\fi



%\subsection{Distributed quantum algorithm for  DJ problem (\uppercase\expandafter{\romannumeral2})}


%\subsection{Algorithm \ref{algorithm3}}

 %Algorithm \ref{algorithm3} can only has the advantage of fewer quantum bits  and quantum gates required, and the corresponding quantum circuit is concise.

\subsection{Design of  Algorithm \ref{algorithm3}}


%\begin{center}
%\bf \uppercase\expandafter{\romannumeral2.}\quad SCHEME (\uppercase\expandafter{\romannumeral2})
%\end{center}


%\begin{center}
%\bf A.\quad Preliminaries 
%\end{center}






In the following, we describe the unitary operators involved in Algorithm \ref{algorithm3}. 


The operators $O_{f_0}$ and $O_{f_1}$ in  Algorithm \ref{algorithm3} are defined in equation (\ref{Of0}) and equation (\ref{Of1}), respectively. %as:
%\begin{align}
%O_{f_0}\ket{u}\ket{b}=&\ket{u}\ket{b\oplus f(u0)},\\
%O_{f_1}\ket{u}\ket{b}=&\ket{u}\ket{b\oplus f(u1)},
%\end{align}
% where $u\in\{0,1\}^{n-1}$ and $b\in\{0,1\}$.
 
The operator $Z$ in Algorithm \ref{algorithm3} is the usual Pauli matrix $Z$.

%\begin{center}
%\bf B.\quad Algorithm \ref{algorithm5} 
%\end{center}




\begin{figure}[H]
	\begin{minipage}{\linewidth}	
		\begin{algorithm}[H]	
			\caption{Distributed quantum algorithm for DJ problem (two distributed computing nodes)}
			\label{algorithm3}
			\begin{algorithmic}[1]
				\State $|\varphi_0\rangle = |0^{n-1}\rangle\ket{0}$;
				
				\State $|\varphi_1\rangle = \left(H^{\otimes n-1}\otimes I\right)|\psi_0\rangle$; %=\frac{1}{\sqrt{2^{n-1}}}\sum_{u\in\{0,1\}^{n-1}}|u\rangle\ket{0}$;
				
				\State %The computing node $f_0$ queries its own oracle under the control of the first quantum register: 				
				$|\varphi_2\rangle=O_{f_0}|\varphi_1\rangle$;%\frac{1}{\sqrt{2^{n-1}}}\sum_{u\in\{0,1\}^{n-1}}|u\rangle|f_0(u)\rangle=
				%\dfrac{1}{\sqrt{2^{n-1}}}\sum\limits_{u\in\{0,1\}^{n-1}}|u\rangle|f(u0)\rangle$;
				
				\State %Apply the quantum gate $Z$ to the second quantum register: 			
				$|\varphi_3\rangle=(I\otimes Z)|\varphi_2\rangle$;%\dfrac{1}{\sqrt{2^{n-1}}}\sum\limits_{u\in\{0,1\}^{n-1}}(-1)^{f(u0)}|u\rangle|f(u0)\rangle$;
				
				\State %The computing node $f_1$ queries its own oracle under the control of the first quantum register: 				
				$|\varphi_4\rangle=O_{f_1}|\varphi_3\rangle$;%\frac{1}{\sqrt{2^{n-1}}}\sum_{u\in\{0,1\}^{n-1}}(-1)^{f(u0)}|u\rangle|f(u0)\oplus f_1(u)\rangle=
				%\dfrac{1}{\sqrt{2^{n-1}}}\sum\limits_{u\in\{0,1\}^{n-1}}(-1)^{f(u0)}|u\rangle|f(u0)\oplus f(u1)\rangle$;%=\frac{1}{\sqrt{2^{n-1}}}\sum\limits_{\substack{u\in\{0,1\}^{n-1}\\ f(u0)\oplus f(u1)=0}}(-1)^{f(u0)}|u\rangle\ket{0}+\frac{1}{\sqrt{2^{n-1}}}\sum\limits_{\substack{u\in\{0,1\}^{n-1}\\ f(u0)\oplus f(u1)=1}}(-1)^{f(u0)}|u\rangle\ket{1}$;
				
				\State Measure the last qubit of $|\varphi_4\rangle$: if the result is 1, then output $f$ is balanced; if the result is 0, then denote the quantum state after measurement  as $\ket{\varphi_5}$;
				
				%$|\varphi_5\rangle=\dfrac{1}{\sqrt{M}}\sum\limits_{\substack{u\in\{0,1\}^{n-1}\\ f(u0)\oplus f(u1)=0}}(-1)^{f(u0)}|u\rangle\ket{0}$;%, where $M=|\{u\in\{0,1\}^{n-1}|f(u0)\oplus f(u1)=0\}|$;
				
				\State $|\varphi_6\rangle=\left(H^{\otimes n-1}\otimes I\right)|\varphi_5\rangle$;%=\frac{1}{\sqrt{2^{n-1}M}}\sum_{z\in\{0,1\}^{n-1}}\sum\limits_{\substack{u\in\{0,1\}^{n-1}\\ f(u0)\oplus f(u1)=0}}(-1)^{f(u0)+u\cdot z}\ket{z}\ket{0}$;
				
				\State Measure the first $n-1$ qubits of $\ket{\varphi_6}$:  if the result is   $0^{n-1}$, then output $f$ is constant; if the result is not $0^{n-1}$, then output $f$ is balanced.
			\end{algorithmic}
		\end{algorithm}
		
	
		\end{minipage}
	\end{figure}


		\begin{figure}[H]%参数[h]表示紧跟着文字
		\centering%居中
		\includegraphics[width=6.3in]{3.png}
		\caption{The circuit for the  distributed quantum algorithm for DJ problem (two computing nodes) (Algorithm \ref{algorithm3} ).}
		\label{algorithm_two_nodes (II)}
	\end{figure}




%\begin{center}
%\bf B.\quad Correctness analysis of Algorithm \ref{algorithm5} 
%\end{center}


\subsection{The correctness analysis of  Algorithm \ref{algorithm3}}




In the following, we prove the correctness of  Algorithm \ref{algorithm3}. The state after the first step of Algorithm \ref{algorithm3} is:
\begin{align}
  |\varphi_1\rangle&=\frac{1}{\sqrt{2^{n-1}}}\sum_{u\in\{0,1\}^{n-1}}|u\rangle\ket{0}.
\end{align}

 Algorithm \ref{algorithm3} then queries the oracle $O_{f_{0}}$, resulting in the following state:
\begin{equation}
\begin{split}
  |\varphi_2\rangle&=O_{f_0}|\varphi_1\rangle\\
  %&=O_{f_{0}}\left(\frac{1}{\sqrt{2^{n-t}}}\sum_{u\in\{0,1\}^{n-t}}|u\rangle\ket{0}\right)\\
  &=\frac{1}{\sqrt{2^{n-1}}}\sum_{u\in\{0,1\}^{n-1}}|u\rangle|f_{0}(u)\rangle\\ 
  %&=\frac{1}{\sqrt{2^{n-1}}}\sum_{u\in\{0,1\}^{n-1}}|u\rangle|f(u0)\rangle.
\end{split}
\end{equation}

Then, the quantum gate $Z$ on  $|\varphi_2\rangle$ is applied to get the following states:
\begin{equation}
\begin{split}
  |\varphi_3\rangle=&(I\otimes Z)|\varphi_2\rangle\\
  =&\frac{1}{\sqrt{2^{n-1}}}\sum_{u\in\{0,1\}^{n-1}}(-1)^{f_0(u)}|u\rangle|f_0(u)\rangle.
\end{split}
\end{equation}

After applying the operator $O_{f_1}$ on $|\varphi_3\rangle$, we have the following state:
\begin{equation}
\begin{split}
  |\varphi_4\rangle=&O_{f_1}|\varphi_3\rangle\\
  =&\frac{1}{\sqrt{2^{n-1}}}\sum_{u\in\{0,1\}^{n-1}}(-1)^{f_0(u)}|u\rangle|f_0(u)\oplus f_1(u)\rangle\\
  %=&\frac{1}{\sqrt{2^{n-1}}}\sum_{u\in\{0,1\}^{n-1}}(-1)^{f(u0)}|u\rangle|f(u0)\oplus f(u1)\rangle\\
  =&\frac{1}{\sqrt{2^{n-1}}}\sum\limits_{\substack{u\in\{0,1\}^{n-1}\\ f_0(u)\oplus f_1(u)=0}}(-1)^{f_0(u)}|u\rangle\ket{0}+\frac{1}{\sqrt{2^{n-1}}}\sum\limits_{\substack{u\in\{0,1\}^{n-1}\\ f_0(u)\oplus f_1(u)=1}}(-1)^{f_0(u)}|u\rangle\ket{1}.
\end{split}
\end{equation}




After measuring on the last qubit of $|\varphi_4\rangle$, if the result is $1$, then there $\exists$ $u\in\{0,1\}^{n-1}$ such that $f_0(u)\oplus f_1(u)=1$. From Corollary \ref{Cor3}, we know that $f$ is balanced. If the result is $0$, then we get the state:
\begin{equation}
\begin{split}
|\varphi_5\rangle=&\frac{1}{\sqrt{M}}\sum\limits_{\substack{u\in\{0,1\}^{n-1}\\ f_0(u)\oplus f_1(u)=0}}(-1)^{f_0(u)}|u\rangle\ket{0}, \\
\end{split}
\end{equation}
where $M=|\{u\in\{0,1\}^{n-1}|f_0(u)\oplus f_1(u)=0\}|$.

%By tracing out the state of the $1$ bit register in the last, we can get the state:
%\begin{equation}
%\begin{split}
%|\varphi_5'\rangle=&\frac{1}{\sqrt{M}}\sum\limits_{\substack{u\in\{0,1\}^{n-1}\\ f(u0)\oplus f(u1)=0}}(-1)^{f(u0)}|u\rangle.
%\end{split}
%\end{equation}

After Hadamard transformation on the first $n-1$ qubits of $|\varphi_5\rangle$, we obtain the following state:
\begin{equation}
\begin{split}
	|\varphi_6\rangle=&\left(H^{\otimes n-1}\otimes I\right)|\varphi_5\rangle\\
	=&\frac{1}{\sqrt{2^{n-1}M}}\sum\limits_{\substack{u,z\in\{0,1\}^{n-1}\\ f_0(u)\oplus f_1(u)=0}}(-1)^{f_0(u)+u\cdot z}\ket{z}\ket{0}.
\end{split}	
\end{equation} 




%In the step 6 of Algorithm \ref{algorithm5} , if the measurement of the second quantum register is $1$, then there $\exists$ $u\in\{0,1\}^{n-1}$ such that $f(u0)\oplus f(u1)=1$. According to Theorem \ref{The3}, it follows that $f$ is balanced.

The probability of measuring the first $n-1$ qubits of  $|\varphi_6\rangle$ with the result of $0^{n-1}$ is
\begin{equation}
\left|\frac{1}{\sqrt{2^{n-1}M}}\sum\limits_{\substack{u\in\{0,1\}^{n-1}\\ f_0(u)\oplus f_1(u)=0}}(-1)^{f_0(u)}\right|^2.
\end{equation}

After measuring on the first $n-1$ qubits of  $|\varphi_6\rangle$, according to Theorem \ref{The4}, %and Theorem \ref{The5} , 
if the result is  $0^{n-1}$, then $f$ is  constant, otherwise $f$ is balanced.

\subsection{Comparison with other algorithms}

%In the following, we compare Algorithm \ref{algorithm3} with other algorithms.

First, we compare  with the   distributed quantum algorithm for DJ problem proposed previously \cite{avron_quantum_2021}. 
For $t=1$, the algorithm in paper \cite{avron_quantum_2021} has certain error, but Algorithm \ref{algorithm3} can  solve it exactly.



Second, we compare  with distributed classical deterministic algorithm. Algorithm \ref{algorithm3} needs one query for each oracle  to solve DJ problem. However, distributed classical deterministic algorithm needs to query oracles $O(2^{n-1})$ times in the worst case. Therefore, Algorithm \ref{algorithm3} has the advantage of exponential acceleration compared with the   distributed classical deterministic algorithm.



Third, we compare with DJ algorithm. In DJ algorithm, the number of qubits required by the implementation circuit of oracle corresponding to Boolean function $f$ is $n+1$. In Algorithm \ref{algorithm3}, the number of qubits required by the implementation circuit of oracle corresponding to subfunctions $f_0$ and $f_1$ is $n$. %According to the paper \cite{Elementary_gates_1995}, the depth of the implemented oracle of the subfunctions $f_0$ and $f_1$ is smaller than the function $f$. %This is advantageous when the implementation of the oracle is the main overhead.
%According to the paper \cite{avron_quantum_2021}, the noise in the output scales exponentially with the depth of the circuit. Therefore, Algorithm \ref{algorithm3} has advantage against noise interference. 
%Furthermore, for  DJ problem in  the case of two distributed computing nodes, DJ algorithm will no longer be applicable. Algorithm \ref{algorithm3} is applicable. 

% In addition, the algorithm  in \cite{avron_quantum_2021} cannot deal with the case of $t > 1$. Our algorithm can handle it. Therefore, our distributed quantum algorithm has higher scalability.



%\newpage


%\subsection{Distributed quantum algorithm for  DJ problem (\uppercase\expandafter{\romannumeral3})}


%\begin{center}
%\bf \uppercase\expandafter{\romannumeral3.}\quad SCHEME (\uppercase\expandafter{\romannumeral3})
%\end{center}


%\begin{center}
%\bf A.\quad Preliminaries 
%\end{center}





\begin{remark}
However,  Algorithm \ref{algorithm3}  cannot exactly solve  DJ problem in  distributed scenario  with multiple computing nodes. In \ref{Distributed quantum algorithm for DJ problem with errors (four distributed computing nodes)}, we  give an example to demonstrate that the algorithm cannot exactly solve  DJ problem    for the case with four computing nodes.

\end{remark}


\section{Design and analysis of Algorithm  \ref{algorithm2}} \label{sec:Distributed quantum algorithm for  DJ problem (Algorithm 2)}



%\subsection{Algorithm \ref{algorithm2}}
%Distributed quantum algorithm for  DJ problem (\uppercase\expandafter{\romannumeral1})}

%\begin{center}
%\bf \uppercase\expandafter{\romannumeral1.}\quad SCHEME (\uppercase\expandafter{\romannumeral1})
%\end{center}

%\begin{center}
%\bf A.\quad Preliminaries 
%\end{center}

Since Algorithm  \ref{algorithm3} cannot exactly solve  DJ problem in  distributed scenario with multiple computing nodes, we use the new structural features of DJ problem  to design Algorithm \ref{algorithm2}. %The design of Algorithm \ref{algorithm2} makes full use of the essential structure of  DJ problem, which combines the design method of distributed Simon's quantum algorithm \cite{Tan2022DQCSimon} and HHL algorithm \cite{HHL_2009}. It is worth pointing out that Algorithm \ref{algorithm2} can solve the distributed DJ problem in the case of multiple computing nodes exactly.


%${\rm Node}\ 2\ :$



\subsection{Design of Algorithm \ref{algorithm2}}




Below we introduce related notation and  function that are used in Algorithm \ref{algorithm2}.% and Algorithm \ref{algorithm5}.



Let $[N]$ represent the set of integers $\{0,1,\cdots, 2^t-1\}$, ane let $BI:\{0,1\}^t \rightarrow [N]$ be the function to convert a binary string of $t$ bits to an equal decimal integer.


%Let $[I]$ represent the set of integers $\{-2^t,\cdots,-1,0,1,\cdots,2^t\}$, and let $IC:[I] \rightarrow \{0,1\}^{t+2}$ be the function to convert a decimal number into an equal binary string of $t+2$ bits  expressed in two's complement, and let $CI:\{0,1\}^{t+2}\rightarrow [I]$ be the function to convert a binary string of $t+2$ bits  expressed in two's complement into an equal  decimal number.

\iffalse
Let ${\rm sgn}:\mathbb{R}\rightarrow\{-1,0,1\}$ be the sign function as follows:
\begin{equation}
{\rm sgn}\ x=
\begin{cases}
1, & x>0,\\
0, & x=0,\\
-1, & x<0.
\end{cases}
 \end{equation}
 \fi
 
 %Let $|x|:\mathbb{R}\rightarrow\mathbb{R}$ be the absolute value function. 



%In the following, we describe the  operators involved in Algorithm \ref{algorithm2}.



Operator $O'_{f_w}$ in Algorithm \ref{algorithm2}
 is defined as: 
\begin{equation}
O'_{f_w}\ket{u}\ket{b}\ket{c}=\ket{u}\ket{b}\ket{c\oplus f_w(u)},
\end{equation}
where $w\in \{0,1\}^t$, $u\in\{0,1\}^{n-t}$, $b\in\{0,1\}^{BI(w)}$ and $c\in \{0,1\}$.

\iffalse
With the help of auxiliary $\left(2^t-1\right)\left(n-t\right)$ qubits, we can change the state of the control register in Algorithm \ref{algorithm2} after the first Hadamard transformation $\dfrac{1}{\sqrt{2^{n-t}}}\sum_{u\in\{0,1\}^{n-t}}|u\rangle$ to $\dfrac{1}{\sqrt{2^{n-t}}}\sum_{u\in\{0,1\}^{n-t}}\underbrace{|u\rangle|u\rangle \ldots |u\rangle}_{2^t}$. %That is, we changed the control register from one group to the same $2^t$ groups. 
  In fact, after the first Hadamard transformation,  we can  teleport each group of $n-t$ control bits to every quantum computing node and use these to control the oracles of the computing nodes. In this case, the operator $O_{f_w}$ can be transformed into $O'_{f_w}$ as: $\forall w\in \{0,1\}^t$, $u\in\{0,1\}^{n-t}$, $b\in\{0,1\}$,
\begin{equation}
O'_{f_w}\ket{u}\ket{b}=\ket{u}\ket{b\oplus f_w(u)}.
\end{equation}
\fi


%as follows:
%\begin{equation}
%|x|=
%\begin{cases}
%x, & x\geq0,\\
%x, & x<0.
%\end{cases}
% \end{equation}

Operator $U$ in Algorithm \ref{algorithm2} 
is defined as: %$\forall a_i\in \{0,1\}$, $b\in\{0,1\}$ and $c\in \{0,1\}^{t+1}$,
\begin{equation}\label{U}
U\left(\bigotimes_{i\in\{0,1\}^{t}}\ket{a_i}\right)|b\rangle
=\left(\bigotimes_{i\in\{0,1\}^{t}}\ket{a_i}\right)\Ket{b\oplus \left(2^{t}-2\sum\limits_{i\in\{0,1\}^{t}}a_i\right)},
\end{equation}
where  $a_i\in \{0,1\}$, $i\in \{0,1\}^t$ and $b\in \{0,1\}^{t+2}$.
In fact,  $U$ is  unitary.

%\begin{remark}
%The values in the quantum registers in this paper are expressed in two's complement.
%\end{remark}




Another  operator $R$ in Algorithm \ref{algorithm2} 
is defined as: %$\forall d \in \{0,1\}^{t+1}$ and $e\in\{0,1\}$,
\begin{equation}\label{R}
	R\ket{d}\ket{e}=\ket{d}\left(\frac{d}{2^t}\ket{e}+(-1)^{e}\cdot\sqrt{1-\left(\frac{d}{2^t}\right)^2}\Ket{1\oplus e}\right),
\end{equation}
where  $ d \in \{0,1\}^{t+2}$ and $e\in\{0,1\}$.
Also,  $R$ is unitary.

\begin{lemma}\label{Lem_U} 
The  operator $U$  in equation (\ref{U})  and  the  operator $R$  in equation (\ref{R}) are both unitary.
\end{lemma}
\begin{proof}
Suppose $\left(\bigotimes_{i\in\{0,1\}^{t}}\ket{a_i}\right)|b\rangle\neq  \left(\bigotimes_{i\in\{0,1\}^{t}}\ket{a'_i}\right)|b'\rangle$. Then 
\begin{equation}
\begin{split}
&\bra{b'}\left(\bigotimes_{i\in\{0,1\}^{t}}\bra{a'_i}\right)U^{\dagger}U\left(\bigotimes_{i\in\{0,1\}^{t}}\ket{a_i}\right)|b\rangle\\
=&\Bra{b'\oplus \left(2^{t}-2\sum\limits_{i\in\{0,1\}^{t}}a'_i\right)}\left(\bigotimes_{i\in\{0,1\}^{t}}\bra{a'_i}\right)\left(\bigotimes_{i\in\{0,1\}^{t}}\ket{a_i}\right)\Ket{b\oplus \left(2^{t}-2\sum\limits_{i\in\{0,1\}^{t}}a_i\right)}\\
=&0.
\end{split}
\end{equation}

Therefore, $U$ is unitary.

Suppose $\ket{d}\ket{e}\neq\ket{d'}\ket{e'}$. Then 
\begin{equation}
\begin{split}
&\bra{e'}\bra{d'}R^{\dagger}R\ket{d}\ket{e}\\
=&\left(\frac{d'}{2^t}\bra{e'}+(-1)^{e'}\cdot\sqrt{1-\left(\frac{d'}{2^t}\right)^2}\bra{1\oplus e'}\right)\braket{d'|d}\left(\frac{d}{2^t}\ket{e}+(-1)^{e}\cdot\sqrt{1-\left(\frac{d}{2^t}\right)^2}\Ket{1\oplus e}\right)\\
=&0.
\end{split}
\end{equation}
\iffalse
Suppose $\ket{d}\ket{1}\neq\ket{d'}\ket{1}$, then 
\begin{equation}
\begin{split}
&\bra{1}\bra{d'}R^{\dagger}R\ket{d}\ket{1}\\
=&\left(\frac{d'}{2^t}\bra{1}-\sqrt{1-\left(\frac{d'}{2^t}\right)^2}\bra{0}\right)\braket{d'|d}\left(\frac{d}{2^t}\ket{1}-\sqrt{1-\left(\frac{d}{2^t}\right)^2}\Ket{0}\right)\\
=&0.
\end{split}
\end{equation}
Since
\begin{equation}
\begin{split}
&\bra{1}\bra{d}R^{\dagger}R\ket{d}\ket{0}\\
=&\left(\frac{d}{2^t}\bra{1}-\sqrt{1-\left(\frac{d}{2^t}\right)^2}\bra{0}\right)\braket{d|d}\left(\frac{d}{2^t}\ket{0}+\sqrt{1-\left(\frac{d}{2^t}\right)^2}\Ket{1}\right)\\
=&0.
\end{split}
\end{equation}
\fi

Therefore, $R$  is unitary.
\end{proof}


%The effect of $U$ in Algorithm \ref{algorithm2}  is to perform arithmetic operations on the values in the $2^{t}$ control registers, and xor to the target register. 

%It is not difficult to be implemented for performing arithmetic operations on  multiple elements. 
%By virtue of using sorting network in \cite{paterson_improved_1990}\cite{ajtai19830}, $2^t$ elements can be sorted in $O(t)$ depth of comparators. Here comparator is a basic circuit module which is easy to be realized.



%Intuitively, the effect of $A$ is to perform add operations on the values in the first two registers, and xor to the third and fourth register. $V$ does not change the states of the first two registers (i.e. $\ket{a}$ and $\ket{b}$). We call these two registers the control registers of $A$.

\iffalse

\begin{lemma}\label{Lem_R} 
The operator $R$ in Algorithm \ref{algorithm2} is unitary.
\end{lemma}
\begin{proof}
Suppose $\ket{d}\ket{e}\neq\ket{d'}\ket{e'}$. Then 
\begin{equation}
\begin{split}
&\bra{e'}\bra{d'}R^{\dagger}R\ket{d}\ket{e}\\
=&\left(\frac{d'}{2^t}\bra{e'}+(-1)^{e'}\cdot\sqrt{1-\left(\frac{d'}{2^t}\right)^2}\bra{1\oplus e'}\right)\braket{d'|d}\left(\frac{d}{2^t}\ket{e}+(-1)^{e}\cdot\sqrt{1-\left(\frac{d}{2^t}\right)^2}\Ket{1\oplus e}\right)\\
=&0.
\end{split}
\end{equation}
\iffalse
Suppose $\ket{d}\ket{1}\neq\ket{d'}\ket{1}$, then 
\begin{equation}
\begin{split}
&\bra{1}\bra{d'}R^{\dagger}R\ket{d}\ket{1}\\
=&\left(\frac{d'}{2^t}\bra{1}-\sqrt{1-\left(\frac{d'}{2^t}\right)^2}\bra{0}\right)\braket{d'|d}\left(\frac{d}{2^t}\ket{1}-\sqrt{1-\left(\frac{d}{2^t}\right)^2}\Ket{0}\right)\\
=&0.
\end{split}
\end{equation}
Since
\begin{equation}
\begin{split}
&\bra{1}\bra{d}R^{\dagger}R\ket{d}\ket{0}\\
=&\left(\frac{d}{2^t}\bra{1}-\sqrt{1-\left(\frac{d}{2^t}\right)^2}\bra{0}\right)\braket{d|d}\left(\frac{d}{2^t}\ket{0}+\sqrt{1-\left(\frac{d}{2^t}\right)^2}\Ket{1}\right)\\
=&0.
\end{split}
\end{equation}
\fi
Therefore, the operator $R$ in Algorithm \ref{algorithm2} is unitary.
\end{proof}

\fi

%The effect of $R$ in Algorithm \ref{algorithm2}  is to perform controlled rotation on the the state in the last register based on the value in the $t+1$ control registers. 

%It is not difficult to be implemented for performing controlled rotation on  multiple elements. 





%\begin{center}
%\bf C1.\quad Algorithm \ref{algorithm3} 
%\end{center}



\begin{figure}[H]%参数[h]表示紧跟着文字
  \centering%居中
  \includegraphics[width=6.4in]{1.png}
  \caption{The circuit for the distributed quantum algorithm for DJ problem  (two computing nodes) (Algorithm \ref{algorithm2} ).}
  \label{algorithm_multiple_nodes (I)}
\end{figure}


\begin{figure}[H]
  \begin{minipage}{\linewidth}	
    \begin{algorithm}[H]
      \caption{Distributed quantum algorithm for DJ problem ($2^t$  distributed computing nodes)}
      \label{algorithm2}
      \begin{algorithmic}[1]
        %\State $\ket{\phi_0} = |0^{n        \State $|\phi_1\rangle = \left(H^{\otimes n-t}\otimes I^{\otimes {2^{t}+t+3}}\right)\ket{\phi_0}$; %=(H^{\otimes n-t}|0^{n-t}\rangle)(\bigotimes_{w \in \{0,1\}^t}\ket{0})|0^{t+3}\rangle=\frac{1}{\sqrt{2^{n-t}}}\sum_{u\in\{0,1\}^{n-t}}|u\rangle(\bigotimes_{w \in \{0,1\}^t}\ket{0})|0^{t+3}\rangle$;-t}\rangle\left(\bigotimes_{w \in \{0,1\}^t}\ket{0}\right)|0^{t+3}\rangle$;
        
        \State $\ket{\phi_0} = |0^{n-t}\rangle\ket{0^{2^{t}+t+3}}$;

         \State $|\phi_1\rangle=\left(H^{\otimes n-t}\otimes I^{\otimes {2^{t}+t+3}}\right)\ket{\phi_0}$;

        \State $\ket{\phi_2}=\prod\limits_{w\in\{0,1\}^t}\left(O'_{f_{w}}\otimes I^{\otimes 2^t+t+2-BI(w)}\right)\ket{\phi_1}$;
        
        %Each computing node queries its own oracle under the control of the first quantum register: 
        
        %$|\phi_3\rangle=%\frac{1}{\sqrt{2^{n-t}}}\sum_{u\in\{0,1\}^{n-t}}|u\rangle(\bigotimes_{w \in \{0,1\}^t}|f_w(u)\rangle)|0^{t+3}\rangle
        %\frac{1}{\sqrt{2^{n-t}}}\sum_{u\in\{0,1\}^{n-t}}|u\rangle\left(\bigotimes_{w \in \{0,1\}^t}|f(uw)\rangle\right)|0^{t+3}\rangle$;
        
        
         \State $\ket{\phi_3}=\left(I^{\otimes {n-t}}\otimes U\otimes I\right)\ket{\phi_2}$;
        
        \State $\ket{\phi_4}=\left(I^{\otimes {n-t+2^t}}\otimes R\right)\ket{\phi_3}$;
        
        \State $\ket{\phi_5}=\left(\prod\limits_{w\in\{0,1\}^t}\left(O'_{f_{w}}\otimes I^{\otimes 2^t+t+2-BI(w)}\right)\right)\left(I^{\otimes {n-t}}\otimes U\otimes I\right)\ket{\phi_4}$;
        
       % \State $\ket{\phi_6}=\prod\limits_{w\in\{0,1\}^t}\left(O_{f_{w}}\otimes I^{\otimes 2^t+t+2-BI(w)}\right)\ket{\phi_5}$;
        

       % \State The $(2^t+2)$-th quantum register performs its own $U$ under the control of the middle $2^t$ quantum registers: 
        
       % $|\phi_4\rangle=\frac{1}{\sqrt{2^{n-t}}}\sum_{u\in\{0,1\}^{n-t}}|u\rangle\left(\bigotimes_{w \in \{0,1\}^t}|f(uw)\rangle\right)\Ket{\left\lfloor\frac{1-{\rm sgn}(\delta(u))}{2}\right\rfloor}||\delta(u)|\rangle\ket{0}$;%, where $\delta(u)=2^t-2\sum_{w\in\{0,1\}^t}f(uw)$;
        
       % \State The quantum gate $Z$ acts on the $(2^t+2)$-th qubit, the last quantum register performs its own $R$ under the control of the bottom $t+1$ quantum registers:
        
        % $|\phi_5\rangle=\frac{1}{\sqrt{2^{n-t}}}\sum_{u\in\{0,1\}^{n-t}}(-1)^{\left\lfloor\frac{1-{\rm sgn}(\delta(u))}{2}\right\rfloor}|u\rangle\left(\bigotimes_{w \in \{0,1\}^t}|f(uw)\rangle\right)\Ket{\left\lfloor\frac{1-{\rm sgn}(\delta(u))}{2}\right\rfloor}||\delta(u)|\rangle$
         
        %$\qquad\qquad\left(\frac{|\delta(u)|}{2^t}\ket{0}+\sqrt{1-\left(\frac{|\delta(u)|}{2^t}\right)^2}\ket{1}\right)$;

       % \State Uncomputing the states of the $2^t+t+2$ bit registers in the middle, resulting in the following state:

%$|\phi_8\rangle=\frac{1}{\sqrt{2^{n-t}}}\sum_{u\in\{0,1\}^{n-t}}(-1)^{\left\lfloor\frac{1-{\rm sgn}(\delta(u))}{2}\right\rfloor}|u\rangle\left(\bigotimes_{w \in \{0,1\}^t}\ket{0}\right)|0^{t+2}\rangle$

%$\qquad\qquad\left(\frac{|\delta(u)|}{2^t}\ket{0}+\sqrt{1-\left(\frac{|\delta(u)|}{2^t}\right)^2}\ket{1}\right)$;

        %\State The $(2^t+2)$-th quantum register performs its own $U$ under the control of the middle $2^t$ quantum registers: 
        
        %$|\phi_6\rangle=\frac{1}{\sqrt{2^{n-t}}}\sum_{u\in\{0,1\}^{n-t}}(-1)^{\left\lfloor\frac{1-{\rm sgn}(\delta(u))}{2}\right\rfloor}|u\rangle\left(\bigotimes_{w \in \{0,1\}^t}|f(uw)\rangle\right)\Ket{0^{t+2}}$
        
       % $\qquad\qquad\left(\frac{|\delta(u)|}{2^t}\ket{0}+\sqrt{1-\left(\frac{|\delta(u)|}{2^t}\right)^2}\ket{1}\right)$;
        
        %\State Each computing node queries its own oracle under the control of the first quantum register: 
        
       % $|\phi_8\rangle=\frac{1}{\sqrt{2^{n-t}}}\sum_{u\in\{0,1\}^{n-t}}(-1)^{\left\lfloor\frac{1-{\rm sgn}(\delta(u))}{2}\right\rfloor}|u\rangle\left(\bigotimes_{w \in \{0,1\}^t}\ket{0}\right)|0^{t+2}\rangle\left(\frac{|\delta(u)|}{2^t}\ket{0}+\sqrt{1-\left(\frac{|\delta(u)|}{2^t}\right)^2}\ket{1}\right)$;

       %\State  Measure the last quantum register, if the result is 1, output $f$ is balanced. If the result is 0: 
       
      % $|\phi_9\rangle=%\sqrt{\frac{1}{\sum_{u\in\{0,1\}^{n-t}}\left(\frac{1}{\sqrt{2^{n-t}}}(-1)^{\left\lfloor\frac{1-{\rm sgn}(\delta(u))}{2}\right\rfloor}\frac{|\delta(u)|}{2^t}\right)^2}}\frac{1}{\sqrt{2^{n-t}}}\sum_{u\in\{0,1\}^{n-t}}(-1)^{\left\lfloor\frac{1-{\rm sgn}(\delta(u))}{2}\right\rfloor}|u\rangle(\bigotimes_{w \in \{0,1\}^t}\ket{0})|0^{t+2}\rangle\frac{|\delta(u)|}{2^t}\ket{0}=
      % \frac{1}{\sqrt{\sum_{u\in\{0,1\}^{n-t}}\delta^2(u)}}\sum_{u\in\{0,1\}^{n-t}}(-1)^{\left\lfloor\frac{1-{\rm sgn}(\delta(u))}{2}\right\rfloor}|\delta(u)||u\rangle|0^{2^t+t+3}\rangle$;
       
       
      \State  Measure the last qubit of  $\ket{\phi_5}$: if the result is 1, then output $f$ is balanced; if the result is 0, then denote the quantum state after measurement  as $\ket{\phi_6}$;
       
       

        \State $|\phi_{7}\rangle=%(H^{\otimes n-t}\otimes I^{\otimes {2^{t}+t+3}})|\phi_{9}\rangle=\frac{1}{\sqrt{\sum_{u\in\{0,1\}^{n-t}}\delta^2(u)}}\sum_{u\in\{0,1\}^{n-t}}\delta(u)\sum_{z\in\{0,1\}^{n-t}}\frac{1}{\sqrt{2^{n-t}}}(-1)^{u\cdot z}\ket{z}|0^{2^t+t+3}\rangle=
        \left(H^{\otimes n-t}\otimes I^{\otimes {2^t+t+3}}\right)|\phi_{6}\rangle$;

        \State Measure the first $n-t$ qubits of $\ket{\phi_7}$: if the result is  $0^{n-t}$, then output $f$ is constant;  if the result is not $0^{n-t}$, then output $f$ is balanced. 
      \end{algorithmic}
    \end{algorithm}

  \end{minipage}
  \end{figure}

%$O'_{f_1}$


\begin{figure}[H]%参数[h]表示紧跟着文字
  \centering%居中
  \includegraphics[width=6.4in]{2.png}
  \caption{The circuit for the distributed quantum algorithm for DJ problem  ($2^t$ computing nodes) (Algorithm \ref{algorithm2} ).}
  \label{algorithm_multiple_nodes (I)}
\end{figure}


%\begin{remark}
%Note that the oracle query of each quantum computing node in  Algorithm \ref{algorithm2}  can actually be completed in parallel. With the help of auxiliary $\left(2^t-1\right)\left(n-t\right)$ qubits, we can change the state of the control register after the first Hadamard transformation $\dfrac{1}{\sqrt{2^{n-t}}}\sum_{u\in\{0,1\}^{n-t}}|u\rangle$ to $\dfrac{1}{\sqrt{2^{n-t}}}\sum_{u\in\{0,1\}^{n-t}}\underbrace{|u\rangle|u\rangle \ldots |u\rangle}_{2^t}$. %That is, we changed the control register from one group to the same $2^t$ groups. 
  %Actually, after the first Hadamard transformation,  we can  teleport each group of $n-t$ control qubits to every quantum computing node and use these to control the oracles of the computing nodes. 
 % By teleporting control qubits, we can replace the query operator $O'_{f_w}$ with the query operator $O_{f_w}$, the query operator $O_{f_w}$ is defined in equation (\ref{Ofw}). %as: 
%\begin{equation}
%O_{f_w}\ket{u}\ket{b}=\ket{u}\ket{b\oplus f_w(u)},
%\end{equation}
%where $ w\in \{0,1\}^t$, $u\in\{0,1\}^{n-t}$, $b\in\{0,1\}$.
%It's clear that the qubit number of  query operator $O_{f_w}$ is $n-t+1$. After the action of the $2^t$ query operator $O'_{f_w}$, we can teleport the values of the $2^t$ subfunctions $f_w$ to the unitary operator $U$.

%\end{remark}



%$O'_{f_{1^t}}$


%${\rm Node}\ 2$

%\subsection{Analysis of Algorithm \ref{algorithm2}}
\subsection{Correctness analysis of  Algorithm \ref{algorithm2}}

%\begin{center}
%\bf C2.\quad Correctness analysis of Algorithm \ref{algorithm3} 
%\end{center}

In the following, we prove the correctness of  Algorithm \ref{algorithm2}.
%\begin{proof}
The state after the first step of  Algorithm \ref{algorithm2} is: %algorithm in FIG. \ref{algorithm_multiple_nodes (I)}.
\begin{equation}
\begin{split}
  |\phi_1\rangle=&\left(H^{\otimes n-t}\otimes I^{\otimes {2^{t}+t+3}}\right)\ket{\phi_0}\\
  %=&\left(H^{\otimes n-t}\otimes I^{\otimes {2^{t}+t+3}}\right)|0^{n-t}\rangle\ket{0^{2^{t}+t+3}}\\
  =&\frac{1}{\sqrt{2^{n-t}}}\sum_{u\in\{0,1\}^{n-t}}|u\rangle\ket{0^{2^{t}+t+3}}.
\end{split}
\end{equation}

Then Algorithm \ref{algorithm2}  queries the oracles $O'_{f_{w}}$ $\left(w\in\{0,1\}^t\right)$,  resulting in the following state:
%\begin{equation}
%\begin{split}
%  |\phi_2\rangle&=\left(O_{f_{0^t}}\frac{1}{\sqrt{2^{n-t}}}\sum_{u\in\{0,1\}^{n-t}}|u\rangle\ket{0}\right)\underbrace{\ket{0} \ldots\ket{0}}_{2^t-1}\Ket{0^{t+3}}\\
%  &=\frac{1}{\sqrt{2^{n-t}}}\sum_{u\in\{0,1\}^{n-t}}|u\rangle|f_{0^t}(u)\rangle \underbrace{\ket{0} \ldots\ket{0}}_{2^t-1}\Ket{0^{t+3}}\\
%  &=\frac{1}{\sqrt{2^{n-t}}}\sum_{u\in\{0,1\}^{n-t}}|u\rangle|f(u0^t)\rangle \underbrace{\ket{0} \ldots\ket{0}}_{2^t-1}\Ket{0^{t+3}}.
%\end{split}
%\end{equation}
%Then the algorithm queries each of the other oracles based on the circuit diagram FIG. \ref{algorithm_multiple_nodes (I)} to get the following state:
\begin{equation}
\begin{split}
  |\phi_2\rangle=&\prod\limits_{w\in\{0,1\}^t}\left(O'_{f_{w}}\otimes I^{\otimes 2^t+t+2-BI(w)}\right)\ket{\phi_1}\\
  =&\frac{1}{\sqrt{2^{n-t}}}\sum_{u\in\{0,1\}^{n-t}}|u\rangle\bigotimes\limits_{w\in\{0,1\}^t}\ket{f_w(u)}
  %\underbrace{|f(u0^t)\rangle \ldots|f(u1^t)\rangle}_{2^t}
  |0^{t+3}\rangle.
\end{split}
\end{equation}

After  applying  operator $U$, we have the following state:
\begin{equation}
\begin{split}
  |\phi_3\rangle=&\left(I^{\otimes {n-t}}\otimes U\otimes I\right)\ket{\phi_2}\\=&\frac{1}{\sqrt{2^{n-t}}}\sum_{u\in\{0,1\}^{n-t}}|u\rangle%\underbrace{|f(u0^t)\rangle \ldots|f(u1^t)\rangle}_{2^t}
  \bigotimes\limits_{w\in\{0,1\}^t}\ket{f_w(u)}\Ket{\delta(u)}\Ket{0}, 
\end{split}
\end{equation}
where $\delta(u)=2^t-2\sum_{w\in\{0,1\}^t}f_w(u)$.

After  applying operator $R$, we have the following state:
\begin{equation}
\begin{split}
  |\phi_4\rangle=&\left(I^{\otimes {n-t+2^t}}\otimes R\right)\ket{\phi_3}\\%\\=&\frac{1}{\sqrt{2^{n-t}}}\sum_{u\in\{0,1\}^{n-t}}|u\rangle\underbrace{|f(u0^t)\rangle \ldots|f(u1^t)\rangle}_{2^t}\Ket{\delta(u)}\\
%& \left(\frac{\delta(u)}{2^t}\ket{0}+\sqrt{1-\left(\frac{\delta(u)}{2^t}\right)^2}\ket{1}\right)\\
=&\frac{1}{\sqrt{2^{n-t}}}\sum_{u\in\{0,1\}^{n-t}}|u\rangle
%\underbrace{|f(u0^t)\rangle \ldots|f(u1^t)\rangle}_{2^t}
\bigotimes\limits_{w\in\{0,1\}^t}\ket{f_w(u)}\Ket{\delta(u)}
\left(\frac{\delta(u)}{2^t}\ket{0}+\sqrt{1-\left(\frac{\delta(u)}{2^t}\right)^2}\ket{1}\right).
\end{split}
\end{equation}

After  applying  operator $U$ and $O'_{f_{w}}$ $(w\in\{0,1\}^t)$, resulting in the following state:
%Uncomputing the states of the $2^t+t+2$ bit registers in the middle, resulting in the following state:
%After that, we use the $U$ operator again, and restore the status of the $t+2$ bit registers of  $U$ operator  to $\ket{0}$. Then we obtain the following state:
%\begin{equation}
%\begin{split}
%  &|\phi_6\rangle=\frac{1}{\sqrt{2^{n-t}}}\sum_{u\in\{0,1\}^{n-t}}(-1)^{\left\lfloor\frac{1-{\rm sgn}(\delta(u))}{2}\right\rfloor}|u\rangle\otimes\underbrace{|f(u0^t)\rangle \ldots|f(u1^t)\rangle}_{2^t}|0^{t+2}\rangle\otimes\left(\frac{|\delta(u)|}{2^t}\ket{0}+\sqrt{1-\left(\frac{|\delta(u)|}{2^t}\right)^2}\ket{1}\right).
%\end{split}
%\end{equation}
%After that, we query each oracle again and restore the status of the $2^t$ bit registers to $\ket{0}$. Then we obtain the following state:
\begin{equation}
\begin{split}
\ket{\phi_5}=&\left(\prod\limits_{w\in\{0,1\}^t}\left(O'_{f_{w}}\otimes I^{\otimes 2^t+t+2-BI(w)}\right)\right)\left(I^{\otimes {n-t}}\otimes U\otimes I\right)\ket{\phi_4}\\=&\frac{1}{\sqrt{2^{n-t}}}\sum_{u\in\{0,1\}^{n-t}}|u\rangle|0^{2^t+t+2}\rangle\left(\frac{\delta(u)}{2^t}\ket{0}+\sqrt{1-\left(\frac{\delta(u)}{2^t}\right)^2}\ket{1}\right).
\end{split}
\end{equation}

%By tracing out the states of the $2^t+t+2$ bit registers in the middle, we can get the state:
%\begin{equation}
%\begin{split}
%&|\phi_8'\rangle=\frac{1}{\sqrt{2^{n-t}}}\sum_{u\in\{0,1\}^{n-t}}(-1)^{\left\lfloor\frac{1-{\rm sgn}(\delta(u))}{2}\right\rfloor}|u\rangle\otimes\left(\frac{|\delta(u)|}{2^t}\ket{0}+\sqrt{1-\left(\frac{|\delta(u)|}{2^t}\right)^2}\ket{1}\right).
%\end{split}
%\end{equation}

After measuring  the last qubit of $\ket{\phi_5}$, if the result is $1$, then $\exists$ $u\in\{0,1\}^{n-t}$ such that $|\delta(u)|\neq 2^t$. From Corollary \ref{Cor1}, we know that $f$ is balanced. If the result is $0$, then we get the state:
\begin{equation}
\begin{split}
&|\phi_6\rangle=\sum_{u\in\{0,1\}^{n-t}}\frac{\delta(u)}{\sqrt{\sum\limits_{u\in\{0,1\}^{n-t}}\delta^2(u)}}|u\rangle\ket{0^{2^t+t+3}}.
\end{split}
\end{equation}

%By tracing out the state of the $1$ bit register in the last, we can get the state:
%\begin{equation}
%|\phi_9''\rangle=\sum_{u\in\{0,1\}^{n-t}}\frac{(-1)^{\left\lfloor\frac{1-{\rm sgn}(\delta(u))}{2}\right\rfloor}|\delta(u)|}{\sqrt{\sum\limits_{u\in\{0,1\}^{n-t}}|\delta(u)|^2}}|u\rangle
%=\sum_{u\in\{0,1\}^{n-t}}\frac{\delta(u)}{\sqrt{\sum\limits_{u\in\{0,1\}^{n-t}}\delta^2(u)}}|u\rangle.
%\end{equation}


With Hadamard transformation on  the first $n-t$ qubits of $\ket{\phi_6}$, we get the following state:
\begin{equation}
\begin{split}
	|\phi_{7}\rangle=&
        \left(H^{\otimes n-t}\otimes I^{\otimes {2^t+t+3}}\right)|\phi_{6}\rangle\\
	=&\sum_{u,z\in\{0,1\}^{n-t}}\frac{\delta(u)}{\sqrt{\sum\limits_{u\in\{0,1\}^{n-t}}\delta^2(u)}}\frac{(-1)^{u\cdot z}}{\sqrt{2^{n-t}}}\ket{z}\ket{0^{2^t+t+3}}.
	%=&\frac{1}{\sqrt{2^{n-t}}}\sum_{u\in\{0,1\}^{n-t}}(H^{\otimes n-t}|u\rangle)|S(u)\rangle\\
	%=&\frac{1}{\sqrt{2^{n-t+2}}}(\sum_{u\in\{0,1\}^{n-t}}(H^{\otimes n-t}|u\rangle)|S(u)\rangle\\
	%+&\sum_{u\in\{0,1\}^{n-t}}(H^{\otimes n-t}|u\rangle)|S(u)\rangle)\\
	%=&\frac{1}{\sqrt{2^{n-t+2}}}(\sum_{u\in\{0,1\}^{n-t}}(H^{\otimes n-t}|u\rangle)|S(u)\rangle\\
	%+&\sum_{u\in\{0,1\}^{n-t}}(H^{\otimes n-t}|u \oplus s_1\rangle)|S(u \oplus s_1)\rangle)\\
	%	=&\frac{1}{\sqrt{2^{n-t+2}}}(\sum_{u\in\{0,1\}^{n-t}}(H^{\otimes n-t}|u\rangle)|S(u)\rangle\\
	%+&\sum_{u\in\{0,1\}^{n-t}}(H^{\otimes n-t}|u \oplus s_1\rangle)|S(u)\rangle)\\
	%=&\frac{1}{\sqrt{2^{n-t+2}}}\sum_{u\in\{0,1\}^{n-t}}(H^{\otimes n-t}(|u\rangle+|u\oplus s_1\rangle))|S(u)\rangle\\
	%=&\frac{1}{\sqrt{2^{n-t+2}}}\sum_{u\in\{0,1\}^{n-t}}(\frac{1}{\sqrt{2^{n-t}}}\sum_{z\in\{0,1\}^{n%-t}}\\
	%&((-1)^{u \cdot z}+(-1)^{(u \oplus s_1) \cdot z})\ket{z})|S(u)\rangle\\
	%=&\frac{1}{\sqrt{2^{n-t+2}}}\sum_{u\in\{0,1\}^{n-t}}(\frac{1}{\sqrt{2^{n-t}}}\sum_{z\in\{0,1\}^{n-t}}\\
	%&(-1)^{u \cdot z}(1+(-1)^{ s_1 \cdot z})\ket{z})|S(u)\rangle.
\end{split}
\end{equation} 

%Note that if $s_1 \cdot z=1$ we have $1+(-1)^{ s_1 \cdot z}=0$ and the basis state $\ket{z}$ vanishes in the above state. If $s_1 \cdot z=0$, we have $1+(-1)^{ s_1 \cdot z}=2$, so we have:

%\begin{align*}
%	|\phi'_7\rangle=&\frac{1}{\sqrt{2^{n-t+2}}}\sum_{u\in\{0,1\}^{n-t}}(\frac{1}{\sqrt{2^{n-t}}}\sum_{z\in\{0,1\}^{n-t}}\\
%	&(-1)^{u \cdot z}(1+(-1)^{ s_1 \cdot z})\ket{z})|S(u)\rangle\\
%	=&\frac{1}{\sqrt{2^{n-t}}}\sum_{u\in\{0,1\}^{n-t}}(\frac{1}{\sqrt{2^{n-t}}}\sum_{z\in s_1^{\perp}}(-1)^{u \cdot z}\ket{z})|S(u)\rangle\\
%	=&\frac{1}{2^{n-t}}\sum_{u\in\{0,1\}^{n-t}}\sum_{z\in s_1^{\perp}}(-1)^{u \cdot z}\ket{z}|S(u)\rangle\\
%	=&\frac{1}{2^{n-t}}\sum_{z\in s_1^{\perp}}\ket{z}\sum_{u\in\{0,1\}^{n-t}}(-1)^{u \cdot z}|S(u)\rangle.
%\end{align*}


The probability of measuring the first $n-t$ qubits of $\ket{\phi_7}$ with the result of $0^{n-t}$ is
\begin{equation}
\left|\frac{1}{\sqrt{2^{n-t}\sum\limits_{u\in\{0,1\}^{n-t}}\delta^2(u)}}\sum\limits_{u\in\{0,1\}^{n-t}}\delta(u)\right|^2.
\end{equation}


After measuring the first $n-t$ qubits of $\ket{\phi_7}$, according to Theorem \ref{The2}, if the result is  $0^{n-t}$, then $f$ is  constant, otherwise $f$ is balanced. %we can get a string that is in $s_1^{\perp}$. After $O(n-t)$ repetitions of the above algorithm, we can obtain $O(n-t)$ elements in the $s_1^{\perp}$. Then, using the classical Gaussian elimination method, we can obtain $s_1$.

%If we have already found $s_1$, we can use Algorithm \ref{algorithm5}  to find out $s_2$. Since $f(s_10^t)=f((s_10^t)\oplus s)$, we have $f(s_10^t)=f(0^{n-t}s_2)$. So we can find a $v$ such that $f(s_10^t)=f(0^{n-t}v)$. Then we can obtain $s_2=v$. At last, we can obtain $s=s_1s_2$. 

%\end{proof}



%\newpage

\subsection{Comparison with other algorithms}


%In the following, we compare Algorithm \ref{algorithm2} with other algorithms.

%First, we compare  with the  distributed quantum algorithm for DJ problem proposed previously \cite{avron_quantum_2021}. 
%For $t=1$, the algorithm in paper \cite{avron_quantum_2021} has error, but Algorithm \ref{algorithm2}  can  solve it exactly. In addition, for $t>1$, the promotion algorithm  in \cite{avron_quantum_2021} also has error. Algorithm \ref{algorithm2} can handle it. Therefore, Algorithm \ref{algorithm2} has higher scalability.



We compare  with distributed classical deterministic algorithm. Algorithm \ref{algorithm2} needs two queries for each oracle  to solve DJ problem. However, distributed classical deterministic  algorithm needs to query oracles $O(2^{n-t})$ times in the worst case. Therefore, Algorithm \ref{algorithm2} has the advantage of exponential acceleration compared with the distributed classical deterministic algorithm.



Then, we compare  with DJ algorithm. In DJ algorithm, the number of qubits required by the implementation circuit of oracle corresponding to Boolean function $f$ is $n+1$. In Algorithm \ref{algorithm2}, the number of qubits required by the implementation circuit of oracle corresponding to subfunction $f_w$ is $n-t+1$. %According to the paper \cite{Elementary_gates_1995}, the depth of the implemented oracle of the subfunction $f_w$ is smaller than the function $f$. %This is advantageous when the implementation of the oracle is the main overhead.
%According to the paper \cite{avron_quantum_2021}, the noise in the output scales exponentially with the depth of the circuit. Therefore, Algorithm \ref{algorithm2}  has advantage against noise interference. 
%Furthermore, for  DJ problem in distributed scenario, DJ algorithm will no longer be applicable. Algorithm \ref{algorithm2} is applicable. 



\iffalse






\section{Distributed quantum algorithm for  DJ problem (Algorithm 1)} \label{sec:Distributed quantum algorithm for  DJ problem (Algorithm 1)}


\subsection{Distributed quantum algorithm for  DJ problem (\uppercase\expandafter{\romannumeral2})}





%\begin{center}
%\bf \uppercase\expandafter{\romannumeral2.}\quad SCHEME (\uppercase\expandafter{\romannumeral2})
%\end{center}


%\begin{center}
%\bf A.\quad Preliminaries 
%\end{center}



In the following, we first describe some definitions and theorems that are closely related to  DJ problem in distributed scenario and are used in the correctness analysis of Algorithm \ref{algorithm3} . 


The following Theorem \ref{The3} provides a sufficient condition for determining that $f$ is balanced, which can be used to ensure the correctness of Algorithm \ref{algorithm3} after the first measurement.

\begin{theorem}\label{The3} Suppose function $f:\{0,1\}^n \rightarrow \{0,1\}$, satisfies that it is either constant or balanced, it is divided into subfunctions $f_0$ and $f_1$. If $\exists$ $u\in\{0,1\}^{n-1}$ such that $f(u0)\oplus f(u1)=1$, then $f$ is balanced.
\end{theorem}
\begin{proof}
Suppose $\exists$ $u\in\{0,1\}^{n-1}$ such that $f(u0)\oplus f(u1)=1$, then $f$ is constant. 
If $f$ is  constant, then $\forall u\in\{0,1\}^{n-1}$, $f(u0)=f(u1)=0$ or $f(u0)=f(u1)=1$. So $f(u0)\oplus f(u1)=0$, which is contrary to the assumption $f(u0)\oplus f(u1)=1$. Therefore, if $\exists$ $u\in\{0,1\}^{n-1}$ such that $f(u0)\oplus f(u1)=1$, then $f$ is balanced.
\end{proof}


\begin{defi}
Suppose function $f:\{0,1\}^n \rightarrow \{0,1\}$, let $B_0=|\{u\in\{0,1\}^{n-1}|f(u0)\oplus f(u1)=0, f(u0)=0\}|$, $B_1=|\{u\in\{0,1\}^{n-1}|f(u0)\oplus f(u1)=0, f(u0)=1\}|$.
\end{defi}


\begin{defi}
 Suppose function $f:\{0,1\}^n \rightarrow \{0,1\}$, let $M=|\{u\in\{0,1\}^{n-1}|f(u0)\oplus f(u1)=0\}|$ $(0\leq M\leq 2^{n-1})$. 
\end{defi}

The following Theorem \ref{The4} provides a sufficient and necessary condition to determine that $f$ is balanced, which can be used to ensure the correctness of Algorithm \ref{algorithm3} after the second measurement.

\begin{theorem}\label{The4} Suppose function $f:\{0,1\}^n \rightarrow \{0,1\}$, satisfies that it is either constant or balanced, it is divided into subfunctions $f_0$ and $f_1$. Then $f$ is balanced if and only if $B_0=B_1=M/2$. %Then $f$ is constant  if and only if $|\{u\in\{0,1\}^{n-1}|f(u0)=0\}|=|\{u\in\{0,1\}^{n-1}|f(u1)=0\}|=2^{n-1}$ or $|\{u\in\{0,1\}^{n-1}|f(u0)=1\}|=|\{u\in\{0,1\}^{n-1}|f(u1)=1\}|=2^{n-1}$. 
\end{theorem}
\begin{proof}
%First, we prove that $f$ is balanced if and only if $B_0=B_1=M/2$.

(\romannumeral1) $\Longleftarrow$. Suppose $B_0=B_1=M/2$ $(0\leq M\leq 2^{n-1})$, then $f$ is constant. If $f$ is constant, then $f\equiv 0$ or $f\equiv 1$, that is $|\{u\in\{0,1\}^{n-1}|f(u0)=f(u1)=0\}|=2^{n-1}$ or $|\{u\in\{0,1\}^{n-1}|f(u0)=f(u1)=1\}|=2^{n-1}$. So $B_0=2^{n-1}$ or $B_1=2^{n-1}$, which is contrary to the assumption $B_0=B_1=M/2$ $(0\leq M\leq 2^{n-1})$. Therefore, if $B_0=B_1=M/2$, then $f$ is balanced, where $M=|\{u\in\{0,1\}^{n-1}|f(u0)\oplus f(u1)=0\}|$.


(\romannumeral2) $\Longrightarrow$. %If $f$ is balanced, according to the definition of DJ problem, then   $|\{x\in\{0,1\}^n|f(x)=0\}|=|\{x\in\{0,1\}^n|f(x)=1\}|=2^{n-1}$. 
Since 
\begin{equation}\label{thm4equ6}
M%=&|\{u\in\{0,1\}^{n-1}|f(u0)\oplus f(u1)=0\}|\\
=|\{u\in\{0,1\}^{n-1}|f(u0)= f(u1)=0\}|
+|\{u\in\{0,1\}^{n-1}|f(u0)= f(u1)=1\}|,
\end{equation}
 therefore %$|\{u\in\{0,1\}^{n-1}|f(u0)\oplus f(u1)=1\}|=2^{n-1}-M$, that is 
\begin{equation}\label{thm4equ3}
 |\{u\in\{0,1\}^{n-1}|f(u0)=0,f(u1)=1\}|+|\{u\in\{0,1\}^{n-1}|f(u0)=1,f(u1)=0\}|
 =2^{n-1}-M. 
\end{equation}

%We have 
%\begin{equation}
%\begin{split}
% &|\{u\in\{0,1\}^{n-1}|f(u0)= f(u1)=0\}|\\
% =&|\{u\in\{0,1\}^{n-1}|f(u0)=0\}|-|\{u\in\{0,1\}^{n-1}|f(u0)=0,f(u1)=1\}|\\
 %=&|\{u\in\{0,1\}^{n-1}|f(u1)=0\}|-|\{u\in\{0,1\}^{n-1}|f(u0)=1,f(u1)=0\}|. 
%\end{split}
%\end{equation}

Let 
 \begin{equation}
S_0=|\{u\in\{0,1\}^{n-1}|f(u0)=0\}|-|\{u\in\{0,1\}^{n-1}|f(u0)=0,f(u1)=1\}|.
\end{equation}
\begin{equation}
S_1=|\{u\in\{0,1\}^{n-1}|f(u1)=0\}|-|\{u\in\{0,1\}^{n-1}|f(u0)=1,f(u1)=0\}|. 
\end{equation}

Then 
\begin{equation}\label{thm4equ4}
S_0=S_1=|\{u\in\{0,1\}^{n-1}|f(u0)= f(u1)=0\}|.
\end{equation}

According to equation (\ref{thm4equ3}), we have
\begin{equation}
\begin{split}
S_0+S_1
%=&\left(|\{u\in\{0,1\}^{n-1}|f(u0)=0\}|-|\{u\in\{0,1\}^{n-1}|f(u0)=0,f(u1)=1\}|\right)\\
%&+\left(|\{u\in\{0,1\}^{n-1}|f(u1)=0\}|-|\{u\in\{0,1\}^{n-1}|f(u0)=1,f(u1)=0\}|\right)\\
=&\left(|\{u\in\{0,1\}^{n-1}|f(u0)=0\}|+|\{u\in\{0,1\}^{n-1}|f(u1)=0\}|\right)\\&-\left(|\{u\in\{0,1\}^{n-1}|f(u0)=0,f(u1)=1\}|+|\{u\in\{0,1\}^{n-1}|f(u0)=1,f(u1)=0\}|\right)\\
=&\left(|\{u\in\{0,1\}^{n-1}|f(u0)=0\}|+|\{u\in\{0,1\}^{n-1}|f(u1)=0\}|\right)-\left(2^{n-1}-M\right). 
\end{split}
\end{equation}

If $f$ is balanced, %according to the definition of DJ problem, then we have 
then$|\{x\in\{0,1\}^n|f(x)=0\}|=|\{x\in\{0,1\}^n|f(x)=1\}|=2^{n-1}$. So $|\{u\in\{0,1\}^{n-1}|f(u0)=0\}|+|\{u\in\{0,1\}^{n-1}|f(u1)=0\}|=|\{x\in\{0,1\}^n|f(x)=0\}|=2^{n-1}$. Therefore, we have
\begin{equation}\label{thm4equ5}
S_0+S_1
%=&\left(|\{u\in\{0,1\}^{n-1}|f(u0)=0\}|+|\{u\in\{0,1\}^{n-1}|f(u1)=0\}|\right)-\left(2^{n-1}-M\right)\\
=2^{n-1}-\left(2^{n-1}-M\right)
=M.
\end{equation}

According to equation (\ref{thm4equ4}) and equation (\ref{thm4equ5}), we have
%then from $S_0=S_1=|\{u\in\{0,1\}^{n-1}|f(u0)= f(u1)=0\}|$, we have 
\begin{equation}\label{thm4equ1}
S_0=S_1=|\{u\in\{0,1\}^{n-1}|f(u0)= f(u1)=0\}|=M/2. 
\end{equation}

According to equation (\ref{thm4equ6}) and equation (\ref{thm4equ1}), we have
%Further, from $M=|\{u\in\{0,1\}^{n-1}|f(u0)= f(u1)=0\}|+|\{u\in\{0,1\}^{n-1}|f(u0)= f(u1)=1\}|$, we have 
\begin{equation}\label{thm4equ2}
\begin{split}
|\{u\in\{0,1\}^{n-1}|f(u0)= f(u1)=1\}|=&M-|\{u\in\{0,1\}^{n-1}|f(u0)= f(u1)=0\}|\\
=&M-M/2
=M/2.
\end{split}
\end{equation}

%Since 
%\begin{equation}
%B_0=|\{u\in\{0,1\}^{n-1}|f(u0)\oplus f(u1)=0, f(u0)=0\}|
%=|\{u\in\{0,1\}^{n-1}|f(u0)= f(u1)=0\}|,
%\end{equation}
%\begin{equation}
%B_1=|\{u\in\{0,1\}^{n-1}|f(u0)\oplus f(u1)=0, f(u0)=1\}|\\
%=|\{u\in\{0,1\}^{n-1}|f(u0)= f(u1)=1\}|,
%\end{equation}
According to equation (\ref{thm4equ1}) and equation (\ref{thm4equ2}), we have $B_0=B_1=M/2$.

%$S_0=|\{u\in\{0,1\}^{n-1}|f(u0)=0\}|-|\{u\in\{0,1\}^{n-1}|f(u0)=0,f(u1)=1\}|=B_0$, $S_1=|\{u\in\{0,1\}^{n-1}|f(u1)=0\}|-|\{u\in\{0,1\}^{n-1}|f(u0)=1,f(u1)=0\}|=B_1$, we have $B_0=B_1=M/2$.



% According to equation $(\ref{delta1})$, then we have $\sum\nolimits_{u\in\{0,1\}^{n-t}} \delta(u)=\sum\nolimits_{u\in\{0,1\}^{n-t}} (2^t-2\sum_{w\in\{0,1\}^t}f(uw))=\sum\nolimits_{u\in\{0,1\}^{n-t}}2^t-2\sum\nolimits_{u,w\in\{0,1\}^{n-t}}f(uw)=2^n-2\sum\nolimits_{x\in\{0,1\}^{n}}f(x)=2^n-2|\{x\in\{0,1\}^n|f(x)=1\}|=0$.

%Second, we prove that $f$ is constant if and only if $|\{u\in\{0,1\}^{n-1}|f(u0)=0\}|=|\{u\in\{0,1\}^{n-1}|f(u1)=0\}|=2^{n-1}$ or $|\{u\in\{0,1\}^{n-1}|f(u0)=1\}|=|\{u\in\{0,1\}^{n-1}|f(u1)=1\}|=2^{n-1}$. 

%(3) $\Longleftarrow$. If $|\{u\in\{0,1\}^{n-1}|f(u0)=0\}|=|\{u\in\{0,1\}^{n-1}|f(u1)=0\}|=2^{n-1}$ or $|\{u\in\{0,1\}^{n-1}|f(u0)=1\}|=|\{u\in\{0,1\}^{n-1}|f(u1)=1\}|=2^{n-1}$, then $|\{x\in\{0,1\}^n|f(x)=1\}|=0$ or $2^n$, that is $f(x) \equiv 0$ or $f(x) \equiv 1$. Therefore, $f$ is constant.

%(4) $\Longrightarrow$. If $f$ is constant, according to the definition of DJ problem, then  $f(x) \equiv 0$ or $f(x) \equiv 1$. So $|\{x\in\{0,1\}^n|f(x)=1\}|=0$ or $2^n$. Therefore $|\{u\in\{0,1\}^{n-1}|f(u0)=0\}|=|\{u\in\{0,1\}^{n-1}|f(u1)=0\}|=2^{n-1}$ or $|\{u\in\{0,1\}^{n-1}|f(u0)=1\}|=|\{u\in\{0,1\}^{n-1}|f(u1)=1\}|=2^{n-1}$.

\end{proof}

The following Theorem \ref{The5} provides a sufficient and necessary condition to determine that $f$ is constant, which can be used to ensure the correctness of Algorithm \ref{algorithm3} after the second measurement.

\begin{theorem}\label{The5} Suppose function $f:\{0,1\}^n \rightarrow \{0,1\}$, satisfies that it is either constant or balanced, it is divided into subfunctions $f_0$ and $f_1$. Then $f$ is constant  if and only if $|\{u\in\{0,1\}^{n-1}|f(u0)=0\}|=|\{u\in\{0,1\}^{n-1}|f(u1)=0\}|=2^{n-1}$ or $|\{u\in\{0,1\}^{n-1}|f(u0)=1\}|=|\{u\in\{0,1\}^{n-1}|f(u1)=1\}|=2^{n-1}$. 
\end{theorem}
\begin{proof}
%Second, we prove that $f$ is constant if and only if $|\{u\in\{0,1\}^{n-1}|f(u0)=0\}|=|\{u\in\{0,1\}^{n-1}|f(u1)=0\}|=2^{n-1}$ or $|\{u\in\{0,1\}^{n-1}|f(u0)=1\}|=|\{u\in\{0,1\}^{n-1}|f(u1)=1\}|=2^{n-1}$. 

(\romannumeral1) $\Longleftarrow$. If $|\{u\in\{0,1\}^{n-1}|f(u0)=0\}|=|\{u\in\{0,1\}^{n-1}|f(u1)=0\}|=2^{n-1}$ or $|\{u\in\{0,1\}^{n-1}|f(u0)=1\}|=|\{u\in\{0,1\}^{n-1}|f(u1)=1\}|=2^{n-1}$, then $|\{x\in\{0,1\}^n|f(x)=1\}|=0$ or $2^n$, that is $f(x) \equiv 0$ or $f(x) \equiv 1$. Therefore, $f$ is constant.

(\romannumeral2) $\Longrightarrow$. If $f$ is constant, according to the definition of DJ problem, then  $f(x) \equiv 0$ or $f(x) \equiv 1$. So $|\{x\in\{0,1\}^n|f(x)=1\}|=0$ or $2^n$. Therefore $|\{u\in\{0,1\}^{n-1}|f(u0)=0\}|=|\{u\in\{0,1\}^{n-1}|f(u1)=0\}|=2^{n-1}$ or $|\{u\in\{0,1\}^{n-1}|f(u0)=1\}|=|\{u\in\{0,1\}^{n-1}|f(u1)=1\}|=2^{n-1}$.

\end{proof}





%\begin{center}
%\bf B.\quad Algorithm \ref{algorithm5} 
%\end{center}


\begin{figure}[H]
	\begin{minipage}{\linewidth}	
		\begin{algorithm}[H]	
			\caption{Distributed quantum algorithm for DJ problem (two distributed computing nodes) (II)}
			\label{algorithm3}
			\begin{algorithmic}[1]
				\State $|\varphi_0\rangle = |0^{n-1}\rangle\ket{0}$;
				
				\State $|\varphi_1\rangle = \left(H^{\otimes n-1}\otimes I\right)|\psi_0\rangle$; %=\frac{1}{\sqrt{2^{n-1}}}\sum_{u\in\{0,1\}^{n-1}}|u\rangle\ket{0}$;
				
				\State The computing node $f_0$ queries its own oracle under the control of the first quantum register: 
				
				$|\varphi_2\rangle=%\frac{1}{\sqrt{2^{n-1}}}\sum_{u\in\{0,1\}^{n-1}}|u\rangle|f_0(u)\rangle=
				\dfrac{1}{\sqrt{2^{n-1}}}\sum\limits_{u\in\{0,1\}^{n-1}}|u\rangle|f(u0)\rangle$;
				
				\State Apply the quantum gate $Z$ to the second quantum register: 
				
				$|\varphi_3\rangle=\dfrac{1}{\sqrt{2^{n-1}}}\sum\limits_{u\in\{0,1\}^{n-1}}(-1)^{f(u0)}|u\rangle|f(u0)\rangle$;
				
				\State The computing node $f_1$ queries its own oracle under the control of the first quantum register: 
				
				$|\varphi_4\rangle=%\frac{1}{\sqrt{2^{n-1}}}\sum_{u\in\{0,1\}^{n-1}}(-1)^{f(u0)}|u\rangle|f(u0)\oplus f_1(u)\rangle=
				\dfrac{1}{\sqrt{2^{n-1}}}\sum\limits_{u\in\{0,1\}^{n-1}}(-1)^{f(u0)}|u\rangle|f(u0)\oplus f(u1)\rangle$;%=\frac{1}{\sqrt{2^{n-1}}}\sum\limits_{\substack{u\in\{0,1\}^{n-1}\\ f(u0)\oplus f(u1)=0}}(-1)^{f(u0)}|u\rangle\ket{0}+\frac{1}{\sqrt{2^{n-1}}}\sum\limits_{\substack{u\in\{0,1\}^{n-1}\\ f(u0)\oplus f(u1)=1}}(-1)^{f(u0)}|u\rangle\ket{1}$;
				
				\State Measure the second quantum register, if the result is 1, output $f$ is balanced. If the result is 0: 
				
				$|\varphi_5\rangle=\dfrac{1}{\sqrt{M}}\sum\limits_{\substack{u\in\{0,1\}^{n-1}\\ f(u0)\oplus f(u1)=0}}(-1)^{f(u0)}|u\rangle\ket{0}$;%, where $M=|\{u\in\{0,1\}^{n-1}|f(u0)\oplus f(u1)=0\}|$;
				
				\State $|\varphi_6\rangle=\left(H^{\otimes n-1}\otimes I\right)|\varphi_5\rangle$;%=\frac{1}{\sqrt{2^{n-1}M}}\sum_{z\in\{0,1\}^{n-1}}\sum\limits_{\substack{u\in\{0,1\}^{n-1}\\ f(u0)\oplus f(u1)=0}}(-1)^{f(u0)+u\cdot z}\ket{z}\ket{0}$;
				
				\State Measure the first quantum register, if the result is not  $0^{n-1}$, output $f$ is balanced. Otherwise, output $f$ is constant.
			\end{algorithmic}
		\end{algorithm}
		
			
		\end{minipage}
	\end{figure}


	
	\begin{figure}[H]%参数[h]表示紧跟着文字
		\centering%居中
		\includegraphics[width=5.5in]{3.png}
		\caption{The circuit for the  distributed quantum algorithm for DJ problem ($2$ computing nodes) (Algorithm \ref{algorithm3} ).}
		\label{algorithm_two_nodes (II)}
	\end{figure}



%\begin{center}
%\bf B.\quad Correctness analysis of Algorithm \ref{algorithm5} 
%\end{center}

In the following, we prove the correctness of  Algorithm \ref{algorithm3} .

\begin{proof}
we write out the state after the first step of the algorithm in FIG. \ref{algorithm_two_nodes (II)}.
\begin{align}
  |\varphi_1\rangle&=\frac{1}{\sqrt{2^{n-1}}}\sum_{u\in\{0,1\}^{n-1}}|u\rangle\ket{0}.
\end{align}

The  algorithm then queries the oracle $O_{f_{0}}$, resulting in the following state:
\begin{equation}
\begin{split}
  |\varphi_2\rangle&=O_{f_{0}}\left(\frac{1}{\sqrt{2^{n-t}}}\sum_{u\in\{0,1\}^{n-t}}|u\rangle\ket{0}\right)\\
  &=\frac{1}{\sqrt{2^{n-t}}}\sum_{u\in\{0,1\}^{n-t}}|u\rangle|f_{0}(u)\rangle\\ 
  &=\frac{1}{\sqrt{2^{n-t}}}\sum_{u\in\{0,1\}^{n-t}}|u\rangle|f(u0)\rangle.
\end{split}
\end{equation}

The algorithm then apply the quantum gate $Z$ to the second quantum register to get the following states:
\begin{equation}
  |\varphi_3\rangle=\frac{1}{\sqrt{2^{n-t}}}\sum_{u\in\{0,1\}^{n-t}}(-1)^{f(u0)}|u\rangle|f(u0)\rangle.
\end{equation}

After the action of the operator $O_{f_1}$, we have the following state:
\begin{equation}
\begin{split}
  |\varphi_4\rangle=&\frac{1}{\sqrt{2^{n-1}}}\sum_{u\in\{0,1\}^{n-1}}(-1)^{f(u0)}|u\rangle|f(u0)\oplus f_1(u)\rangle\\
  =&\frac{1}{\sqrt{2^{n-1}}}\sum_{u\in\{0,1\}^{n-1}}(-1)^{f(u0)}|u\rangle|f(u0)\oplus f(u1)\rangle\\
  =&\frac{1}{\sqrt{2^{n-1}}}\sum\limits_{\substack{u\in\{0,1\}^{n-1}\\ f(u0)\oplus f(u1)=0}}(-1)^{f(u0)}|u\rangle\ket{0}+\frac{1}{\sqrt{2^{n-1}}}\sum\limits_{\substack{u\in\{0,1\}^{n-1}\\ f(u0)\oplus f(u1)=1}}(-1)^{f(u0)}|u\rangle\ket{1}.
\end{split}
\end{equation}




After measurement on the last register, if the result is $1$, then there $\exists$ $u\in\{0,1\}^{n-1}$ such that $f(u0)\oplus f(u1)=1$. From Theorem \ref{The3}, we know that $f$ is balanced. If the measurement result is $0$, we get the state:
\begin{equation}
\begin{split}
|\varphi_5\rangle=&\frac{1}{\sqrt{M}}\sum\limits_{\substack{u\in\{0,1\}^{n-1}\\ f(u0)\oplus f(u1)=0}}(-1)^{f(u0)}|u\rangle\ket{0}, \\
\end{split}
\end{equation}
where $M=|\{u\in\{0,1\}^{n-1}|f(u0)\oplus f(u1)=0\}|$.

By tracing out the state of the $1$ bit register in the last, we can get the state:
\begin{equation}
\begin{split}
|\varphi_5'\rangle=&\frac{1}{\sqrt{M}}\sum\limits_{\substack{u\in\{0,1\}^{n-1}\\ f(u0)\oplus f(u1)=0}}(-1)^{f(u0)}|u\rangle.
\end{split}
\end{equation}

After Hadamard transformation on the first register, we can get the following state:
\begin{equation}
	|\varphi'_6\rangle=\left(H^{\otimes n-1}\otimes I\right)|\varphi'_5\rangle
	=\frac{1}{\sqrt{2^{n-1}M}}\sum\limits_{\substack{z,u\in\{0,1\}^{n-1}\\ f(u0)\oplus f(u1)=0}}(-1)^{f(u0)+u\cdot z}\ket{z}\ket{0}.
\end{equation} 




%In the step 6 of Algorithm \ref{algorithm5} , if the measurement of the second quantum register is $1$, then there $\exists$ $u\in\{0,1\}^{n-1}$ such that $f(u0)\oplus f(u1)=1$. According to Theorem \ref{The3}, it follows that $f$ is balanced.

The probability of measuring the first quantum register with the result of $0^{n-1}$ is
\begin{equation}
\left|\frac{1}{\sqrt{2^{n-1}M}}\sum\limits_{\substack{u\in\{0,1\}^{n-1}\\ f(u0)\oplus f(u1)=0}}(-1)^{f(u0)}\right|^2.
\end{equation}

After measurement on the first register, according to Theorem \ref{The4} and Theorem \ref{The5}, if the result is not $0^{n-1}$, then $f$ is  balanced, otherwise $f$ is constant.

\end{proof}


%\newpage


%\subsection{Distributed quantum algorithm for  DJ problem (\uppercase\expandafter{\romannumeral3})}


%\begin{center}
%\bf \uppercase\expandafter{\romannumeral3.}\quad SCHEME (\uppercase\expandafter{\romannumeral3})
%\end{center}


%\begin{center}
%\bf A.\quad Preliminaries 
%\end{center}

\fi




%\subsection{Distributed quantum algorithm for  DJ problem (\uppercase\expandafter{\romannumeral2})}


\section{Design and analysis of Algorithm  \ref{algorithm5}} \label{sec:Distributed quantum algorithm for  DJ problem (Algorithm 3)}



%\subsection{Algorithm \ref{algorithm5}}

%Distributed quantum algorithm for  DJ problem (\uppercase\expandafter{\romannumeral1})}

%\begin{center}
%\bf \uppercase\expandafter{\romannumeral1.}\quad SCHEME (\uppercase\expandafter{\romannumeral1})
%\end{center}

%\begin{center}
%\bf A.\quad Preliminaries 
%\end{center}

We synthesised the structural features of  DJ problem in  distributed scenario used in Algorithm \ref{algorithm3} and Algorithm \ref{algorithm2}  to design Algorithm \ref{algorithm5}. Algorithm \ref{algorithm5} combines the design methods and advantages of Algorithm \ref{algorithm3} and Algorithm \ref{algorithm2}, but the number of qubits required to implement certain unitary operators in Algorithm \ref{algorithm5} is less than that in Algorithm \ref{algorithm2}.


\subsection{Design of Algorithm \ref{algorithm5}}





In the following, we describe the unitary operators involved in Algorithm \ref{algorithm5}. 

%Let $[N]$ represent the set of integers $\{0,1,\cdots, 2^t-1\}$, $BI:\{0,1\}^t \rightarrow [N]$ be the function to convert a binary string of $t$ bits to an equal decimal integer.


%The operator $O'_{f_w}$ in  Algorithm \ref{algorithm5} is the same as in Algorithm \ref{algorithm2}.

%: $\forall w\in \{0,1\}^t$, $u\in\{0,1\}^{n-t}$, $b\in\{0,1\}^{BI(w)}$ and $c\in \{0,1\}$,
%\begin{equation}
%O_{f_w}\ket{u}\ket{b}\ket{c}=\ket{u}\ket{b}\ket{c\oplus f_w(u)}.
%\end{equation}



Operator $O''_{f_{w'0}}$ in Algorithm \ref{algorithm5}
 is defined as: 
\begin{equation}
O''_{f_{w'0}}\ket{u}\ket{b}\ket{c}=\ket{u}\ket{b}\ket{c\oplus f_{w'0}(u)},
\end{equation}
where $w'\in \{0,1\}^{t-1}$, $u\in\{0,1\}^{n-t}$, $b\in\{0,1\}^{3\cdot BI(w'0)/2}$ and $c\in \{0,1\}$.


Operator $O''_{f_{w'1}}$ in Algorithm \ref{algorithm5}
 is defined as: 
\begin{equation}
O''_{f_{w'1}}\ket{u}\ket{b}\ket{c}=\ket{u}\ket{b}\ket{c\oplus f_{w'1}(u)},
\end{equation}
where $w'\in \{0,1\}^{t-1}$, $u\in\{0,1\}^{n-t}$, $b\in\{0,1\}^{3\cdot BI(w'0)/2+1}$ and $c\in \{0,1\}$. The function $BI$ is defined in Algorithm \ref{algorithm2}.




\iffalse
With the help of auxiliary $\left(2^t-1\right)\left(n-t\right)$ qubits, we can change the state of the control register in Algorithm \ref{algorithm5} after the first Hadamard transformation $\dfrac{1}{\sqrt{2^{n-t}}}\sum_{u\in\{0,1\}^{n-t}}|u\rangle$ to $\dfrac{1}{\sqrt{2^{n-t}}}\sum_{u\in\{0,1\}^{n-t}}\underbrace{|u\rangle|u\rangle \ldots |u\rangle}_{2^t}$. %That is, we changed the control register from one group to the same $2^t$ groups. 
  In fact, after the first Hadamard transformation,  we can  teleport each group of $n-t$ control bits to every quantum computing node and use these to control the oracles of the computing nodes. In this case, the operator $O_{f_w}$ can be transformed into $O'_{f_w}$ as: $\forall w\in \{0,1\}^t$, $u\in\{0,1\}^{n-t}$, $b\in\{0,1\}$,
\begin{equation}
O'_{f_w}\ket{u}\ket{b}=\ket{u}\ket{b\oplus f_w(u)}.
\end{equation}
\fi



Operator $A$ in  Algorithm \ref{algorithm5} is defined as: %$\forall a_i\in \{0,1\}$, $b\in\{0,1\}^2$, $c\in \{0,1\}$  and $d\in \{0,1\}^{t}$,
\begin{equation}\label{A}
A\left(\bigotimes_{i\in\{0,1\}^{t-1}\setminus\{1^{t-1}\}}\ket{a_i}\ket{b}\right)\ket{a_{1^{t-1}}}\ket{c}|d\rangle\\
=\left(\bigotimes_{i\in\{0,1\}^{t-1}\setminus\{1^{t-1}\}}\ket{a_i}\ket{b}\right)\ket{a_{1^{t-1}}}\ket{c}\Ket{d\oplus\sum_{i\in\{0,1\}^{t-1}}a_i},
\end{equation}
 where $a_i\in \{0,1\}$, $i\in \{0,1\}^{t-1}$, $b\in\{0,1\}^2$, $c\in \{0,1\}$  and $d\in \{0,1\}^{t}$.
%Similar to the proof of Lemma \ref{Lem_U} , it can be shown that $A$ is  unitary.



%The effect of $A$ in Algorithm \ref{algorithm5}  is to perform arithmetic operations on the values in the $2^{t-1}$ control registers, and xor to the target register. 

%It is not difficult to be implemented for performing arithmetic operations on  multiple elements. 
%By virtue of using sorting network in \cite{paterson_improved_1990}\cite{ajtai19830}, $2^t$ elements can be sorted in $O(t)$ depth of comparators. Here comparator is a basic circuit module which is easy to be realized.



%Intuitively, the effect of $A$ is to perform add operations on the values in the first two registers, and xor to the third and fourth register. $V$ does not change the states of the first two registers (i.e. $\ket{a}$ and $\ket{b}$). We call these two registers the control registers of $A$.


Operator $V$ in  Algorithm \ref{algorithm5} is defined as: %$\forall a,b \in \{0,1\}^t$, $c \in \{0,1\}$ and $d \in \{0,1\}^{t}$,
\begin{equation}\label{V}
\begin{split}
V\ket{a}\ket{b}%\ket{c}
\ket{c}
=&\ket{a}\ket{b}%\Ket{c\oplus\left\lfloor\frac{1-{\rm sgn}(2^{t-1}-a-2b)}{2}\right\rfloor}
\Ket{c\oplus\left(2^{t-1}-a-2b\right)},
\end{split}
\end{equation}
where $ a,b \in \{0,1\}^t$ %, $c \in \{0,1\}$ 
and $c \in \{0,1\}^{t+1}$.
%Similar to the proof of Lemma \ref{Lem_U} , it can be shown that $V$ is  unitary.

%Intuitively, the effect of $V$ is to perform arithmetic operations on the values in the first two registers, and xor to the third and fourth register. $V$ does not change the states of the first two registers (i.e. $\ket{a}$ and $\ket{b}$). We call these two registers the control registers of $V$.



%The effect of $V$ in Algorithm \ref{algorithm5}  is to perform arithmetic operations on the values in the $2$ control registers, and xor to the target register.

% It is not difficult to be implemented for performing arithmetic operations on  multiple elements. 
%By virtue of using sorting network in \cite{paterson_improved_1990}\cite{ajtai19830}, $2^t$ elements can be sorted in $O(t)$ depth of comparators. Here comparator is a basic circuit module which is easy to be realized.

Operator $R'$ in  Algorithm \ref{algorithm5} is defined as: 
\begin{equation}\label{R'}
	R'\ket{d}\ket{e}=\ket{d}\left(\frac{d}{2^{t-1}}\ket{e}+(-1)^{e}\cdot\sqrt{1-\left(\frac{d}{2^{t-1}}\right)^2}\Ket{1\oplus e}\right),
\end{equation}
where $ d \in \{0,1\}^{t+1}$ and $e\in\{0,1\}$.


\begin{lemma}\label{Lem_AVR} 
The operator $A$ in equation (\ref{A}), the operator $V$ in equation (\ref{V}) and the operator $R'$ in equation (\ref{R'}) are all unitary.
\end{lemma}
\begin{proof}
Similar to the proof of Lemma \ref{Lem_U}. %it can be shown that the operator $A$, the operator $V$ and the operator $R'$ are all unitary.
\end{proof}

\begin{figure}[H]%参数[h]表示紧跟着文字
  \centering%居中
  \includegraphics[width=6.4in]{4.png}
  \caption{The circuit for the distributed quantum algorithm for DJ problem  (four computing nodes) (Algorithm \ref{algorithm5} ).}
  \label{algorithm_multiple_nodes (III)}
\end{figure}









%$\rm Node\ 2^t$

%The effect of $R'$ in Algorithm \ref{algorithm5}  is to perform controlled rotation on the the state in the last register based on the value in the $t$ control registers.

%It is not difficult to be implemented for performing controlled rotation on  multiple elements. 

\iffalse
\begin{figure}[H]%参数[h]表示紧跟着文字
  \centering%居中
  \includegraphics[width=6.5in]{5.png}
  \caption{The circuit for the distributed quantum algorithm for DJ problem  ($2^t$ computing nodes) (II).}
  \label{algorithm_multiple_nodes (III)}
\end{figure}
\fi




\begin{figure}[H]
  \begin{minipage}{\linewidth}
    \begin{algorithm}[H]
      \caption{Distributed quantum algorithm for DJ problem ($2^t$  distributed computing nodes)}
      \label{algorithm5}
      \begin{algorithmic}[1]
        \State $|\Phi_0\rangle = |0^{n-t}\rangle|0^{3\cdot 2^{t-1}+3t+2}\rangle$;

        \State $\ket{\Phi_1} =\left(H^{\otimes n-t}\otimes I^{\otimes {3\cdot 2^{t-1}+3t+2}}\right)|\Phi_0\rangle$;% =(H^{\otimes n-t}|0^{n-t}\rangle)|0^{3\cdot 2^{t-1}+3t+2}\rangle=\frac{1}{\sqrt{2^{n-t}}}\sum_{u\in\{0,1\}^{n-t}}|u\rangle|0^{3\cdot 2^{t-1}+3t+2}\rangle$;

        \State% The 1st computing node and the 2nd computing node query their oracles under the control of the first quantum register: 
        $|\Phi_2\rangle=
        % \frac{1}{\sqrt{2^{n-t}}}\sum_{u\in\{0,1\}^{n-t}}|u\rangle|f_{0^t}(u)\rangle|f_{0^{t-1}1}(u)\rangle|0^{3\cdot 2^{t-1}+3t}\rangle=
        %\dfrac{1}{\sqrt{2^{n-t}}}\sum_{u\in\{0,1\}^{n-t}}|u\rangle|f(u0^t)\rangle|f(u0^{t-1}1)\rangle|0^{3\cdot 2^{t-1}+3t}\rangle$;
        \prod\limits_{w'\in\{0,1\}^{t-1}}\left[\left(I^{\otimes n-t+3\cdot BI(w'0)/2}\otimes{\rm CNOT}\otimes I^{\otimes 3\cdot2^{t-1}+3t-3\cdot BI(w'0)/2}\right)\right.$
        
       $\quad\left(I^{\otimes n-t+3\cdot BI(w'0)/2}\otimes{\rm CCNOT}\otimes I^{\otimes 3\cdot2^{t-1}+3t-3\cdot BI(w'0)/2-1}\right)$
       
       $\quad\left.\left(O''_{f_{w'1}}\otimes I^{\otimes 3\cdot2^{t-1}+3t-3\cdot BI(w'0)/2}\right)\left(O''_{f_{w'0}}\otimes I^{\otimes 3\cdot2^{t-1}+3t-3\cdot BI(w'0)/2+1}\right)\right]\ket{\Phi_1}$

     
        
         \State %The 4th quantum register performs two-bit controlled $X$ gate under the control of the 2nd and 3rd quantum registers: 
         $|\Phi_3\rangle=\prod\limits_{i=1}^2\left(I^{\otimes n-t+i}\otimes A\otimes I^{\otimes (3-i)t+2}\right)\ket{\Phi_2}$;
         
         
           %\State %The 3rd quantum register performs controlled $X$ gate under the control of the 2nd quantum register:      
           %$|\Phi_4\rangle=\left(I^{\otimes n-t}\otimes {\rm CNOT}\otimes I^{ \otimes 3\cdot 2^{t-1}+3t}\right)\ket{\Phi_3}$;
           
        
            
            
            
         %\State  Perform the same operations on the remaining compute nodes  as on the 1st  computing node and the 2nd computing node: 
         
%$|\Phi_5\rangle=%\frac{1}{\sqrt{2^{n-t}}}\sum_{u\in\{0,1\}^{n-t}}|u\rangle(\otimes_{w\in\{0,1\}^{t-1}0}|f_w(u)\rangle|f_w(u)\oplus f_{w+1}(u)\rangle|f_w(u)\land f_{w+1}(u)\rangle)\ket{0^{3t+2}}=
        % \dfrac{\sum\limits_{u\in\{0,1\}^{n-t}}|u\rangle}{\sqrt{2^{n-t}}}\left(\bigotimes\limits_{w\in\{0,1\}^{t-1}0}|f(uw)\rangle|f(uw)\oplus f(u(w+1))\rangle|f(uw)\land f(u(w+1))\rangle\right)\Ket{0^{3t+2}}$;
         
        %\State  Perform similar operations on the 3rd computing node and the 4th computing node as on the 1st  computing node and the 2nd computing node: $|\Psi_6\rangle=\frac{1}{\sqrt{2^{n-2}}}\sum_{u\in\{0,1\}^{n-2}}|u\rangle|f(u00)\rangle|f(u00)\oplus f(u01)\rangle|f(u00)\land f(u01)\rangle|f(u10)\rangle|f(u10)\oplus f(u11)\rangle|f(u10)\land f(u11)\rangle|0^{6}\rangle$;
        
        
          %\State The $(3\cdot 2^{t-1}+2)$-th quantum register performs its own $A$ under the control of the $(\{3k|1\leq k\leq 2^{t-1}\})$-th quantum registers on $\Ket{\Phi_5}$ to get $\Ket{\Phi_6}$;
          
          %$|\Phi_7\rangle=\frac{1}{\sqrt{2^{n-t}}}\sum_{u\in\{0,1\}^{n-t}}|u\rangle\left(\bigotimes_{w\in\{0,1\}^{t-1}0}|f(uw)\rangle|f(uw)\oplus f(u(w+1))\rangle|f(uw)\land f(u(w+1))\rangle\right)$
          
          %$\qquad\qquad|\sum_{w\in\{0,1\}^{t-1}0}f(uw)\oplus f(u(w+1))\rangle|0^{2t+2}\rangle$;
          
          
          %\State The 9th quantum register performs its own $A$ under the control of the 4th and 7th quantum registers: $|\Psi_8\rangle=\frac{1}{\sqrt{2^{n-2}}}\sum_{u\in\{0,1\}^{n-2}}|u\rangle|f(u00)\rangle|f(u00)\oplus f(u01)\rangle|f(u00)\land f(u01)\rangle|f(u10)\rangle|f(u10)\oplus f(u11)\rangle|f(u10)\land f(u11)\rangle|f(u00)\oplus f(u01)+f(u10)\oplus f(u11)\rangle|f(u00)\land f(u01)+f(u10)\land f(u11)\rangle|0^{4}\rangle$;
          
          % \State The $(3\cdot 2^{t-1}+3)$-th quantum register performs its own $A$ under the control of the $(\{3k+1|1\leq k\leq 2^{t-1}\})$-th quantum registers on $\Ket{\Phi_6}$ to get $\Ket{\Phi_7}$; 
           
           %$|\Phi_8\rangle=\frac{1}{\sqrt{2^{n-t}}}\sum_{u\in\{0,1\}^{n-t}}|u\rangle\left(\bigotimes_{w\in\{0,1\}^{t-1}0}|f(uw)\rangle|f(uw)\oplus f(u(w+1))\rangle|f(uw)\land f(u(w+1))\rangle\right)$
           
           %$\qquad\qquad|\sum_{w\in\{0,1\}^{t-1}0}f(uw)\oplus f(u(w+1))\rangle|\sum_{w\in\{0,1\}^{t-1}0}f(uw)\land f(u(w+1))\rangle|0^{t+2}\rangle$;
          
          
         
         
        
        
         \State $\Ket{\Phi_4}=\left(I^{\otimes n-t+3\cdot 2^{t-1}}\otimes V\otimes I\right)\Ket{\Phi_3}$;
         %The $(3\cdot 2^{t-1}+4)$-th quantum register performs its own $V$ under the control of  $(3\cdot 2^{t-1}+2)$-th and the $(3\cdot 2^{t-1}+3)$-th quantum registers: 
         
         %$|\Phi_9\rangle=\frac{1}{\sqrt{2^{n-t}}}\sum_{u\in\{0,1\}^{n-t}}|u\rangle\left(\bigotimes_{w\in\{0,1\}^{t-1}0}|f(uw)\rangle|f(uw)\oplus f(u(w+1))\rangle|f(uw)\land f(u(w+1))\rangle\right)$
         
         %$\qquad\qquad|\sum_{w\in\{0,1\}^{t-1}0}f(uw)\oplus f(u(w+1))\rangle|\sum_{w\in\{0,1\}^{t-1}0}f(uw)\land f(u(w+1))\rangle$
         
        % $\qquad\qquad|\left\lfloor\frac{1-{\rm sgn}(\Delta(u))}{2}\right\rfloor\rangle||\Delta(u)|\rangle\ket{0}$;
        
        
        
        \State$\Ket{\Phi_{5}}=\left(I^{\otimes n+3\cdot2^{t-1}+t}\otimes R'\right)\Ket{\Phi_4}$;
        %The quantum gate $Z$ acts on the $(3\cdot 2^{t-1}+4)$-th qubit, the last quantum register performs its own $R$ under the control of the bottom $t$ quantum registers: 
        
        %$|\Phi_{10}\rangle=\frac{1}{\sqrt{2^{n-t}}}\sum_{u\in\{0,1\}^{n-t}}|u\rangle\left(\bigotimes_{w\in\{0,1\}^{t-1}0}|f(uw)\rangle|f(uw)\oplus f(u(w+1))\rangle|f(uw)\land f(u(w+1))\rangle\right)$
        
         %$\qquad\qquad|\sum_{w\in\{0,1\}^{t-1}0}f(uw)\oplus f(u(w+1))\rangle|\sum_{w\in\{0,1\}^{t-1}0}f(uw)\land f(u(w+1))\rangle$
         
        %$\qquad\qquad|\left\lfloor\frac{1-{\rm sgn}(\Delta(u))}{2}\right\rfloor\rangle||\Delta(u)|\rangle\left(\frac{|\Delta(u)|}{2^{t-1}}\ket{0}+\sqrt{1-\left(\frac{|\Delta(u)|}{2^{t-1}}\right)^2}\ket{1}\right)$;

 
       
       \State %Uncomputing the states of the $3\cdot 2^{t-1}+3t+1$ bit registers in the middle:      
       $\ket{\Phi_6}=\prod\limits_{w'\in\{0,1\}^{t-1}}\left[\left(I^{\otimes n-t+3\cdot BI(w'0)/2}\otimes{\rm CNOT}\otimes I^{\otimes 3\cdot2^{t-1}+3t-3\cdot BI(w'0)/2}\right)\right.$
        
       $\quad\left(I^{\otimes n-t+3\cdot BI(w'0)/2}\otimes{\rm CCNOT}\otimes I^{\otimes 3\cdot2^{t-1}+3t-3\cdot BI(w'0)/2-1}\right)$
       
       $\quad\left.\left(O''_{f_{w'1}}\otimes I^{\otimes 3\cdot2^{t-1}+3t-3\cdot BI(w'0)/2}\right)\left(O''_{f_{w'0}}\otimes I^{\otimes 3\cdot2^{t-1}+3t-3\cdot BI(w'0)/2+1}\right)\right]$
       
       $\quad\prod\limits_{i=1}^2\left(I^{\otimes n-t+i}\otimes A\otimes I^{\otimes (3-i)t+2}\right)\left(I^{\otimes n-t+3\cdot 2^{t-1}}\otimes V\otimes I\right)\ket{\Phi_5}$;

       \State  Measure the last qubit of  $\ket{\Phi_6}$: if the result is 1, then output $f$ is balanced; if the result is 0, then denote the quantum state after measurement  as $\ket{\Phi_7}$;
       
        %Measure the last quantum register, if the result is 1, output $f$ is balanced. If the result is 0: 
       
       %$|\Phi_{11}\rangle=\dfrac{1}{\sqrt{\sum_{u\in\{0,1\}^{n-t}}\Delta^2(u)}}\sum_{u\in\{0,1\}^{n-t}}\Delta(u)|u\rangle|0^{3\cdot 2^{t-1}+3t+2}\rangle$;

        \State $|\Phi_{8}\rangle=\left(H^{\otimes n-t}\otimes I^{\otimes {3\cdot 2^{t-1}+3t+2}}\right)|\Phi_{7}\rangle$;
        %=\frac{1}{\sqrt{\sum_{u\in\{0,1\}^{n-t}}\Delta^2(u)}}\sum_{u\in\{0,1\}^{n-t}}\Delta(u)\sum_{z\in\{0,1\}^{n-t}}\frac{1}{\sqrt{2^{n-t}}}(-1)^{u\cdot z}\ket{z}|0^{3\cdot 2^{t-1}+3t+2}\rangle=
%\sum_{z\in\{0,1\}^{n-t}}
%\sum_{u,z\in\{0,1\}^{n-t}}\frac{\Delta(u)}{\sqrt{\sum_{u\in\{0,1\}^{n-t}}\Delta^2(u)}}\frac{(-1)^{u\cdot z}}{\sqrt{2^{n-t}}}\ket{z}|0^{3\cdot 2^{t-1}+3t+2}\rangle$;
%Measure the first $n-t$ qubits of $\ket{\phi_7}$, if the result is not $0^{n-t}$, output $f$ is balanced. Otherwise, output $f$ is constant.

        \State Measure the first $n-t$ qubits of $\ket{\Phi_8}$: if the result is  $0^{n-t}$, then output $f$ is constant;  if the result is not $0^{n-t}$, then output $f$ is balanced.
      \end{algorithmic}
    \end{algorithm}

  \end{minipage}
  \end{figure}





%$O'_{f_{11}}$





%$O'_{f_{1^{t}}}$



\iffalse

\begin{figure}[H]%参数[h]表示紧跟着文字
  \centering%居中
  \includegraphics[width=6.5in]{5.png}
  \caption{The circuit for the distributed quantum algorithm for DJ problem  ($2^t$ computing nodes) (Algorithm \ref{algorithm5} ).}
  \label{algorithm_multiple_nodes (III)}
\end{figure}

\fi

%\begin{center}
%\bf C2.\quad Correctness analysis of Algorithm 5
%\end{center}


%\subsection{Analysis of Algorithm \ref{algorithm5}}


\begin{figure}[H]%参数[h]表示紧跟着文字
  \centering%居中
  \includegraphics[width=6.5in]{5.png}
  \caption{The circuit for the distributed quantum algorithm for DJ problem  ($2^t$ computing nodes) (Algorithm \ref{algorithm5} ).}
  \label{algorithm_multiple_nodes (III)}
\end{figure}




%\begin{remark}

%Note that the oracle query of each quantum computing node in   Algorithm \ref{algorithm5}   can actually be completed in parallel. With the help of auxiliary $\left(2^t-1\right)\left(n-t\right)$ qubits, we can change the state of the control register after the first Hadamard transformation $\dfrac{1}{\sqrt{2^{n-t}}}\sum_{u\in\{0,1\}^{n-t}}|u\rangle$ to $\dfrac{1}{\sqrt{2^{n-t}}}\sum_{u\in\{0,1\}^{n-t}}\underbrace{|u\rangle|u\rangle \ldots |u\rangle}_{2^t}$. %That is, we changed the control register from one group to the same $2^t$ groups. 
  %Actually, after the first Hadamard transformation,  we can  teleport each group of $n-t$ control qubits to every quantum computing node and use these to control the oracles of the computing nodes. 
%  By teleporting control qubits, we can replace the query operator $O''_{f_{w'0}}$ and $O''_{f_{w'1}}$ with the query operator $O_{f_w}$, the query operator $O_{f_w}$ is defined in equation (\ref{Ofw}). %as: 
%\begin{equation}
%O_{f_w}\ket{u}\ket{b}=\ket{u}\ket{b\oplus f_w(u)},
%\end{equation}
%where $ w\in \{0,1\}^t$, $u\in\{0,1\}^{n-t}$, $b\in\{0,1\}$.
%It's clear that the qubit number of  query operator $O_{f_w}$ is $n-t+1$.

%In addition, with the help of auxiliary $2^{t-1}$ qubits, we can put  the $2^{t-1}$ control qubits $\ket{a_i}$ of  the operator $A$ together by teleportation, and replace  the operator $A$ with  the operator $A'$,  the operator $A'$ is defined as:
%\begin{equation}
%A'\left(\bigotimes_{i\in\{0,1\}^{t-1}}\ket{a_i}\right)|d\rangle\\
%=\left(\bigotimes_{i\in\{0,1\}^{t-1}}\ket{a_i}\right)%\Ket{d\oplus\sum_{i\in\{0,1\}^{t-1}}a_i},
%\end{equation}
%where $\forall a_i\in \{0,1\}$ and $d\in \{0,1\}^{t}$.
%It's clear that the qubit number of the opertor $A'$ is $2^{t-1}+t$, which is less than the qubit number of  the operator $A$ and qubit number of the operator $U$ in Algorithm \ref{algorithm2}.

%\end{remark}




\subsection{Correctness analysis of  Algorithm \ref{algorithm5}}

In the following, we prove the correctness of  Algorithm \ref{algorithm5}.
The state after the first step of  Algorithm \ref{algorithm5} is:
%\begin{proof}
%First, we write out the state after the first step of Algorithm \ref{algorithm5} .
\begin{equation}
\begin{split}
  \ket{\Phi_1}=&\left(H^{\otimes n-t}\otimes I^{\otimes {3\cdot 2^{t-1}+3t+2}}\right)|\Phi_0\rangle\\
  %=&\frac{1}{\sqrt{2^{n-t}}}\sum_{u\in\{0,1\}^{n-t}}|u\rangle\left(\bigotimes_{w \in \{0,1\}^{t-1}0}|0^3\rangle\right)\ket{0^{3t+2}}\\
  =&\frac{1}{\sqrt{2^{n-t}}}\sum_{u\in\{0,1\}^{n-t}}|u\rangle\ket{0^{3\cdot 2^{t-1}+3t+2}}.
\end{split} 
\end{equation} 

Then Algorithm \ref{algorithm5} queries the oracles $O''_{f_{w'0}}$ and $O''_{f_{w'1}}$ $(w'\in\{0,1\}^{t-1})$, and applies the operators ${\rm CCNOT}$ and ${\rm CNOT}$, resulting in the following state:
\begin{equation}
\begin{split}
  |\Phi_2\rangle=&\prod\limits_{w'\in\{0,1\}^{t-1}}\left[\left(I^{\otimes n-t+3\cdot BI(w'0)/2}\otimes{\rm CNOT}\otimes I^{\otimes 3\cdot2^{t-1}+3t-3\cdot BI(w'0)/2}\right)\right.\\
        &\left(I^{\otimes n-t+3\cdot BI(w'0)/2}\otimes{\rm CCNOT}\otimes I^{\otimes 3\cdot2^{t-1}+3t-3\cdot BI(w'0)/2-1}\right)\\
       &\left.\left(O''_{f_{w'1}}\otimes I^{\otimes 3\cdot2^{t-1}+3t-3\cdot BI(w'0)/2}\right)\left(O''_{f_{w'0}}\otimes I^{\otimes 3\cdot2^{t-1}+3t-3\cdot BI(w'0)/2+1}\right)\right]\ket{\Phi_1}\\
=&\frac{1}{\sqrt{2^{n-t}}}\sum_{u\in\{0,1\}^{n-t}}|u\rangle\left(\bigotimes_{w'\in\{0,1\}^{t-1}}|f_{w'0}(u)\rangle|f_{w'0}(u)\oplus f_{w'1}(u)\rangle|f_{w'0}(u)\land f_{w'1}(u)\rangle\right)\ket{0^{3t+2}}.
\end{split} 
\end{equation} 




\iffalse
The 4th quantum register performs two-bit controlled $X$ gate under the control of the 2nd and 3rd quantum registers, we get the state:
\begin{equation}
\begin{split}
|\Phi_3\rangle
=&\frac{1}{\sqrt{2^{n-t}}}\sum_{u\in\{0,1\}^{n-t}}|u\rangle|f(u0^t)\rangle|f(u0^{t-1}1)\rangle|f(u0^t)\land f(u0^{t-1}1)\rangle|0^{3\cdot 2^{t-1}+3t-1}\rangle.
\end{split} 
\end{equation} 

The 3rd quantum register performs controlled $X$ gate under the control of the 2nd quantum register, we get the state:
\begin{equation}
\begin{split}
|\Phi_4\rangle
=&\frac{1}{\sqrt{2^{n-t}}}\sum_{u\in\{0,1\}^{n-t}}|u\rangle|f(u0^t)\rangle|f(u0^t)\oplus f(u0^{t-1}1)\rangle|f(u0^t)\land f(u0^{t-1}1)\rangle|0^{3\cdot 2^{t-1}+3t-1}\rangle.
\end{split} 
\end{equation}    


 Perform the same operations on the remaining compute nodes  as on the 1st  computing node and the 2nd computing node, we have the following state:
\begin{equation}
\begin{split}
 |\Phi_5\rangle=&\frac{1}{\sqrt{2^{n-t}}}\sum_{u\in\{0,1\}^{n-t}}|u\rangle\left(\bigotimes_{w\in\{0,1\}^{t-1}0}|f_w(u)\rangle|f_w(u)\oplus f_{w+1}(u)\rangle|f_w(u)\land f_{w+1}(u)\rangle\right)\ket{0^{3t+2}}\\
 =&\frac{1}{\sqrt{2^{n-t}}}\sum_{u\in\{0,1\}^{n-t}}|u\rangle\left(\bigotimes_{w\in\{0,1\}^{t-1}0}|f(uw)\rangle|f(uw)\oplus f(u(w+1))\rangle|f(uw)\land f(u(w+1))\rangle\right)\ket{0^{3t+2}}.
\end{split} 
\end{equation}   

\fi


By applying the operator $A$ on $\ket{\Phi_2}$, we obtain the following state:
\begin{equation}
\begin{split}
|\Phi_3\rangle=&\prod\limits_{i=1}^2\left(I^{\otimes n-t+i}\otimes A\otimes I^{\otimes (3-i)t+2}\right)\ket{\Phi_2}\\
=&\frac{1}{\sqrt{2^{n-t}}}\sum_{u\in\{0,1\}^{n-t}}|u\rangle\left(\bigotimes_{w'\in\{0,1\}^{t-1}}|f_{w'0}(u)\rangle|f_{w'0}(u)\oplus f_{w'1}(u)\rangle|f_{w'0}(u)\land f_{w'1}(u)\rangle\right)\\
&\left|\sum_{w'\in\{0,1\}^{t-1}}f_{w'0}(u)\oplus f_{w'1}(u)\right\rangle\Ket{\sum_{w'\in\{0,1\}^{t-1}}f_{w'0}(u)\land f_{w'1}(u)}\Ket{0^{t+2}}.
\end{split} 
\end{equation}    

\iffalse
The $(3\cdot 2^{t-1}+2)$-th quantum register performs its own $A$ under the control of the $(\{3k|1\leq k\leq 2^{t-1}\})$-th quantum registers. Then we obtain the following state:
\begin{equation}
\begin{split}
|\Phi_6\rangle=&\frac{1}{\sqrt{2^{n-t}}}\sum_{u\in\{0,1\}^{n-t}}|u\rangle\left(\bigotimes_{w\in\{0,1\}^{t-1}0}|f(uw)\rangle|f(uw)\oplus f(u(w+1))\rangle|f(uw)\land f(u(w+1))\rangle\right)\\
&\left|\sum_{w\in\{0,1\}^{t-1}0}f(uw)\oplus f(u(w+1))\rangle|0^{2t+2}\right\rangle.
\end{split} 
\end{equation}   



The $(3\cdot 2^{t-1}+3)$-th quantum register performs its own $A$ under the control of the $(\{3k+1|1\leq k\leq 2^{t-1}\})$-th quantum registers. Then we obtain the following state:
\begin{equation}
\begin{split}
|\Phi_7\rangle=&\frac{1}{\sqrt{2^{n-t}}}\sum_{u\in\{0,1\}^{n-t}}|u\rangle\left(\bigotimes_{w\in\{0,1\}^{t-1}0}|f(uw)\rangle|f(uw)\oplus f(u(w+1))\rangle|f(uw)\land f(u(w+1))\rangle\right)\\
&\left|\sum_{w\in\{0,1\}^{t-1}0}f(uw)\oplus f(u(w+1))\right\rangle\Ket{\sum_{w\in\{0,1\}^{t-1}0}f(uw)\land f(u(w+1))}\Ket{0^{t+2}}.
\end{split} 
\end{equation}             

\fi



By applying the operator $V$ on $\ket{\Phi_3}$, we get the following state:
%The $(3\cdot 2^{t-1}+4)$-th quantum register performs its own $V$ under the control of  $(3\cdot 2^{t-1}+2)$-th and the $(3\cdot 2^{t-1}+3)$-th quantum registers, we get the following state:
\begin{equation}
\begin{split}
\Ket{\Phi_4}=&\left(I^{\otimes n-t+3\cdot 2^{t-1}}\otimes V\otimes I\right)\Ket{\Phi_3}\\
=&\frac{1}{\sqrt{2^{n-t}}}\sum_{u\in\{0,1\}^{n-t}}|u\rangle\left(\bigotimes_{w'\in\{0,1\}^{t-1}}|f_{w'0}(u)\rangle|f_{w'0}(u)\oplus f_{w'1}(u)\rangle|f_{w'0}(u)\land f_{w'1}(u)\rangle\right)\\
&\left|\sum_{w'\in\{0,1\}^{t-1}}f_{w'0}(u)\oplus f_{w'1}(u)\right\rangle\Ket{\sum_{w'\in\{0,1\}^{t-1}}f_{w'0}(u)\land f_{w'1}(u)}
\Ket{\Delta(u)}\Ket{0},
\end{split} 
\end{equation}   
where $\Delta(u)=2^{t-1}-\sum_{w'\in\{0,1\}^{t-1}}f_{w'0}(u)\oplus f_{w'1}(u)-2\sum_{w'\in\{0,1\}^{t-1}}f_{w'0}(u)\land f_{w'1}(u)$.




%Then the quantum gate $Z$ acts on the $(3\cdot 2^{t-1}+4)$-th qubit, the last quantum register performs its own $R'$ under the control of the bottom $t$ quantum registers, we get the following state:
Then we apply the operator  $R'$ on $\Ket{\Phi_4}$,  and the following state is obtained:
\begin{equation}
\begin{split}
\Ket{\Phi_{5}}=&\left(I^{\otimes n+3\cdot2^{t-1}+t}\otimes R'\right)\Ket{\Phi_4}\\
=&\frac{1}{\sqrt{2^{n-t}}}\sum_{u\in\{0,1\}^{n-t}}|u\rangle\left(\bigotimes_{w'\in\{0,1\}^{t-1}}|f_{w'0}(u)\rangle|f_{w'0}(u)\oplus f_{w'1}(u)\rangle|f_{w'0}(u)\land f_{w'1}(u)\rangle\right)\\
&\left|\sum_{w'\in\{0,1\}^{t-1}}f_{w'0}(u)\oplus f_{w'1}(u)\right\rangle\Ket{\sum_{w'\in\{0,1\}^{t-1}}f_{w'0}(u)\land f_{w'1}(u)}
\Ket{\Delta(u)}\\
&\left(\frac{\Delta(u)}{2^{t-1}}\ket{0}+\sqrt{1-\left(\frac{\Delta(u)}{2^{t-1}}\right)^2}\ket{1}\right).
\end{split} 
\end{equation}  

In addition, we restore  the middle $3\cdot 2^{t-1}+3t+1$ qubits of $\Ket{\Phi_{5}}$ to $\Ket{0^{3\cdot 2^{t-1}+3t+1}}$, resulting in the following state:
\begin{equation}
\begin{split}
\ket{\Phi_6}=&\prod\limits_{w'\in\{0,1\}^{t-1}}\left[\left(I^{\otimes n-t+3\cdot BI(w'0)/2}\otimes{\rm CNOT}\otimes I^{\otimes 3\cdot2^{t-1}+3t-3\cdot BI(w'0)/2}\right)\right.\\
        &\left(I^{\otimes n-t+3\cdot BI(w'0)/2}\otimes{\rm CCNOT}\otimes I^{\otimes 3\cdot2^{t-1}+3t-3\cdot BI(w'0)/2-1}\right)\\
       &\left.\left(O''_{f_{w'1}}\otimes I^{\otimes 3\cdot2^{t-1}+3t-3\cdot BI(w'0)/2}\right)\left(O''_{f_{w'0}}\otimes I^{\otimes 3\cdot2^{t-1}+3t-3\cdot BI(w'0)/2+1}\right)\right]\\
       &\quad\prod\limits_{i=1}^2\left(I^{\otimes n-t+i}\otimes A\otimes I^{\otimes (3-i)t+2}\right)\left(I^{\otimes n-t+3\cdot 2^{t-1}}\otimes V\otimes I\right)\ket{\Phi_5}\\
=&\frac{1}{\sqrt{2^{n-t}}}\sum_{u\in\{0,1\}^{n-t}}%(-1)^{\left\lfloor\frac{1-{\rm sgn}(\Delta(u))}{2}\right\rfloor}
|u\rangle\Ket{0^{3\cdot 2^{t-1}+3t+1}}\left(\frac{\Delta(u)}{2^{t-1}}\ket{0}+\sqrt{1-\left(\frac{\Delta(u)}{2^{t-1}}\right)^2}\ket{1}\right).
\end{split} 
\end{equation}  



%By tracing out the states of the $3\cdot 2^{t-1}+3t+1$ bit registers in the middle, we can get the state:
%\begin{equation}
%\begin{split}
%&|\Phi_{10}'\rangle=\frac{1}{\sqrt{2^{n-t}}}\sum_{u\in\{0,1\}^{n-t}}(-1)^{\left\lfloor\frac{1-{\rm sgn}(\Delta(u))}{2}\right\rfloor}|u\rangle\left(\frac{|\Delta(u)|}{2^{t-1}}\ket{0}+\sqrt{1-\left(\frac{|\Delta(u)|}{2^{t-1}}\right)^2}\ket{1}\right).
%\end{split}
%\end{equation}





After measurement on the last qubit of $\ket{\Phi_6}$, if the result is $1$, then $\exists$ $u\in\{0,1\}^{n-t}$ such that $|\Delta(u)|\neq 2^{t-1}$. From Corollary \ref{Cor6}, we know that $f$ is balanced. If the result is $0$, then we get the state:
%After measurement on the last register, if the result is $1$, then $\exists$ $u\in\{0,1\}^{n-t}$ such that $|\Delta(u)|\neq 2^{t-1}$. From Theorem \ref{The6}, we know that $f$ is balanced. If the measurement result is $0$, we get the state:
\begin{equation}
\begin{split}
|\Phi_{7}\rangle=\sum_{u\in\{0,1\}^{n-t}}\frac{\Delta(u)}{\sqrt{\sum\limits_{u\in\{0,1\}^{n-t}}\Delta^2(u)}}|u\rangle\ket{0^{3\cdot 2^{t-1}+3t+2}}.
\end{split}
\end{equation}




%By tracing out the state of the $1$ bit register in the last, we can get the state:
%\begin{equation}
%\begin{split}
%|\Phi''_{11}\rangle=&\sum_{u\in\{0,1\}^{n-t}}\frac{(-1)^{\left\lfloor\frac{1-{\rm sgn}(\Delta(u))}{2}\right\rfloor}|\Delta(u)|}{\sqrt{\sum\limits_{u\in\{0,1\}^{n-t}}|\Delta(u)|^2}}|u\rangle\\
%=&\sum_{u\in\{0,1\}^{n-t}}\frac{\Delta(u)}{\sqrt{\sum\limits_{u\in\{0,1\}^{n-t}}\Delta^2(u)}}|u\rangle.
%\end{split}
%\end{equation}









By using Hadamard transformation on  the first $n-t$ qubits of $\ket{\Phi_7}$, we get the following state:
\begin{equation}
\begin{split}
	|\Phi_{8}\rangle=&\left(H^{\otimes n-t}\otimes I^{\otimes {3\cdot 2^{t-1}+3t+2}}\right)|\Phi_{7}\rangle\\
	=&\sum_{u,z\in\{0,1\}^{n-t}}\frac{\Delta(u)}{\sqrt{\sum\limits_{u\in\{0,1\}^{n-t}}\Delta^2(u)}}\frac{(-1)^{u\cdot z}}{\sqrt{2^{n-t}}}\ket{z}\ket{0^{3\cdot 2^{t-1}+3t+2}}.
\end{split} 
\end{equation} 





The probability of measuring the first $n-t$ qubits of $\ket{\Phi_8}$ with the result of $0^{n-t}$ is
\begin{equation}
\left|\frac{1}{\sqrt{2^{n-t}\sum\limits_{u\in\{0,1\}^{n-t}}\Delta^2(u)}}\sum\limits_{u\in\{0,1\}^{n-t}}\Delta(u)\right|^2.
\end{equation}





%After measurement on the first register, according to Theorem \ref{The2}, if the result is not $0^{n-t}$, then $f$ is  balanced, otherwise $f$ is constant.

%In the step 12 of Algorithm 5, if the measurement of the last quantum register is $1$, then there $\exists$ $u\in\{0,1\}^{n-t}$ such that $|\Delta(u)|\neq 2^{t-1}$. According to Theorem \ref{The6}, it follows that $f$ is balanced.

%In the step 13 of Algorithm 5, the probability of measuring the first quantum register with the result of $0^{n-t}$ is
%\begin{equation}
%\left|\frac{1}{\sqrt{2^{n-t}\sum\limits_{u\in\{0,1\}^{n-t}}\Delta^2(u)}}\sum\limits_{u\in\{0,1\}^{n-t}}\Delta(u)\right|^2.
%\end{equation}

After measurement on the first $n-t$ qubits of $\ket{\Phi_8}$, according to Theorem \ref{The7}, % and Theorem \ref{The8}, 
if the result is  $0^{n-t}$, then $f$ is  constant, otherwise $f$ is balanced.

%\end{proof}
%\begin{center}
%\bf \uppercase\expandafter{\romannumeral 4.}\quad DISCUSSION AND CONCLUSION
%\end{center}

\subsection{Comparison with other algorithms}
Actually,  the comparisons of Algorithm \ref{algorithm5} with the   distributed quantum algorithm for DJ problem proposed previously \cite{avron_quantum_2021},  distributed classical deterministic algorithm, and DJ algorithm  are the same as that of Algorithm \ref{algorithm2}.

\iffalse
In the following, we compare Algorithm \ref{algorithm5} with other algorithms.

First, we compare with the   distributed quantum algorithm for DJ problem proposed previously \cite{avron_quantum_2021}. 
For $t=1$, the algorithm in paper \cite{avron_quantum_2021} can not solve DJ problem exactly. Algorithm \ref{algorithm5} can  solve it exactly. In addition, the algorithm  in \cite{avron_quantum_2021} does not consider the  case of $t > 1$. Algorithm \ref{algorithm5} can handle it. Therefore, Algorithm \ref{algorithm5} has higher scalability.



Second, we compare with distributed classical deterministic algorithm. Algorithm \ref{algorithm5} needs two query for each oracle  to solve DJ problem. However, distributed classical deterministic algorithm needs to query oracles $O(2^{n-t})$ times. Therefore, Algorithm \ref{algorithm5} has the advantage of exponential acceleration compared with the  distributed  classical  deterministic algorithm.



Third, we compare Algorithm \ref{algorithm5} with DJ algorithm. In DJ algorithm, the number of bits required by the implementation circuit of oracle corresponding to function $f$ is $n+1$. In Algorithm \ref{algorithm5}, the number of bits required by the implementation circuit of oracle corresponding to subfunction $f_w$ is $n-t+1$. %According to the paper \cite{Elementary_gates_1995}, the depth of the implemented oracle of the subfunction $f_w$ is smaller than the function $f$. %This is advantageous when the implementation of the oracle is the main overhead.
%According to the paper \cite{avron_quantum_2021}, the noise in the output scales exponentially with the depth of the circuit. Therefore, Algorithm \ref{algorithm2}  has advantage against noise interference. 
Furthermore, for  DJ problem in distributed scenario, DJ algorithm will no longer be applicable. Algorithm \ref{algorithm5} is applicable. 

\fi

In the following, we analyze and compare the three algorithms we designed.  
Compared  Algorithm \ref{algorithm3}  with  Algorithm \ref{algorithm2}, Algorithm \ref{algorithm3}  has the advantage that the number of quantum gates and qubits required by the circuit is reduced. Algorithm \ref{algorithm2}  has the advantage of scaling to multiple computing nodes. %, but it can only be decomposed into two computing nodes.  

Compared  Algorithm \ref{algorithm2}  with Algorithm \ref{algorithm5}, Algorithm \ref{algorithm2}  has the advantage of fewer total qubits and quantum gates. Algorithm \ref{algorithm5}  has the advantage that the qubit number for realizing some single unitary operators decreases. %, but the total number of qubits and quantum gates increases.  
For instance, the number of qubits required for the unitary operator $V$ in Algorithm \ref{algorithm5} is $3t+1$, while the number of qubits required for the unitary operator $U$ in Algorithm \ref{algorithm2} is $2^t+t+2$. 

Although the number of qubits required for the unitary operator $A$ in Algorithm \ref{algorithm5} is $3\cdot2^{t-1}+t-1$, actually,  with the help of auxiliary $2^{t-1}$ qubits, we can put  the $2^{t-1}$ control qubits $\ket{a_i}$ of  the operator $A$ together by teleportation, and replace  the operator $A$ with  the operator $A'$,  the operator $A'$ is defined as:
\begin{equation}
A'\left(\bigotimes_{i\in\{0,1\}^{t-1}}\ket{a_i}\right)|d\rangle\\
=\left(\bigotimes_{i\in\{0,1\}^{t-1}}\ket{a_i}\right)\Ket{d\oplus\sum_{i\in\{0,1\}^{t-1}}a_i},
\end{equation}
where $\forall a_i\in \{0,1\}$,  $i\in\{0,1\}^{t-1}$ and $d\in \{0,1\}^{t}$.
It is clear that the qubit number of $A'$ is $2^{t-1}+t$, which is less than the qubit number of  $A$ in Algorithm \ref{algorithm5} and qubit number of $U$ in Algorithm \ref{algorithm2}.


 The performance of our  algorithms is shown in TAB. \ref{Tab1}.

\setcounter{table}{0}

\begin{table}[H]
\begin{center}
\begin{tabular}{|l|c|c|c|}
\hline
\diagbox{\makecell[c]{Name}}{\makecell[c]{Index}} &\makecell[c]{Total number\\ of qubits} &\makecell[c]{Number of\\ quantum gates}&\makecell[c]{The number of qubits of \\ a single unitary operator} \\
\hline
 \quad Algorithm \ref{algorithm3} & $n$ & 5 & The qubit number of $Z$ is $1$ \\ 
\hline
 \quad Algorithm \ref{algorithm2} & $n+2^t+3$ & $2^{t+1}+6$ & \makecell[c]{The qubit number of $U$ is $2^t+t+2$\\ The qubit number of $R$ is $t+3$}\\  
 \hline
 \quad Algorithm \ref{algorithm5} & $n+3\cdot2^{t-1}+2t+2$ & $2^{t+2}+10$ & \makecell[c]{The qubit number of $A$ is $3\cdot2^{t-1}+t-1$\\ The qubit number of $V$ is $3t+1$\\ The qubit number of $R'$ is $t+2$} \\ 
\hline
\end{tabular}
\caption{The performance table of our algorithms.}\label{Tab1}
\end{center}
\end{table}

\iffalse
\begin{remark}

Note that the oracle query of each quantum computing node in   Algorithm \ref{algorithm5}   can actually be completed in parallel. With the help of auxiliary $\left(2^t-1\right)\left(n-t\right)$ qubits, we can change the state of the control register after the first Hadamard transformation $\dfrac{1}{\sqrt{2^{n-t}}}\sum_{u\in\{0,1\}^{n-t}}|u\rangle$ to $\dfrac{1}{\sqrt{2^{n-t}}}\sum_{u\in\{0,1\}^{n-t}}\underbrace{|u\rangle|u\rangle \ldots |u\rangle}_{2^t}$. %That is, we changed the control register from one group to the same $2^t$ groups. 
  Actually, after the first Hadamard transformation,  we can  teleport each group of $n-t$ control qubits to every quantum computing node and use these to control the oracles of the computing nodes. 
  By teleporting control qubits, we can replace the query operator $O''_{f_{w'0}}$ and $O''_{f_{w'1}}$ with the query operator $O_{f_w}$, the query operator $O_{f_w}$ is defined in equation (\ref{Ofw}). %as: 
%\begin{equation}
%O_{f_w}\ket{u}\ket{b}=\ket{u}\ket{b\oplus f_w(u)},
%\end{equation}
%where $ w\in \{0,1\}^t$, $u\in\{0,1\}^{n-t}$, $b\in\{0,1\}$.
It's clear that the qubit number of  query operator $O_{f_w}$ is $n-t+1$.

%In addition, with the help of auxiliary $2^{t-1}$ qubits, we can put  the $2^{t-1}$ control qubits $\ket{a_i}$ of  the operator $A$ together by teleportation, and replace  the operator $A$ with  the operator $A'$,  the operator $A'$ is defined as:
%\begin{equation}
%A'\left(\bigotimes_{i\in\{0,1\}^{t-1}}\ket{a_i}\right)|d\rangle\\
%=\left(\bigotimes_{i\in\{0,1\}^{t-1}}\ket{a_i}\right)%\Ket{d\oplus\sum_{i\in\{0,1\}^{t-1}}a_i},
%\end{equation}
%where $\forall a_i\in \{0,1\}$ and $d\in \{0,1\}^{t}$.
%It's clear that the qubit number of the opertor $A'$ is $2^{t-1}+t$, which is less than the qubit number of  the operator $A$ and qubit number of the operator $U$ in Algorithm \ref{algorithm2}.

\end{remark}
\fi

%\section{Comparison between our algorithms} \label{sec:complexity analysis}

%First, we compare our algorithms with DJ algorithm. In DJ algorithm, the number of bits required by the implementation circuit of oracle corresponding to function $f$ is $n+1$. In our algorithms, the number of bits required by the implementation circuit of oracle corresponding to subfunction $f_w$ is $n-t+1$. According to the paper \cite{Elementary_gates_1995}, the depth of the implemented oracle of the subfunction $f_w$ may be smaller than the function $f$. %This is advantageous when the implementation of the oracle is the main overhead.
%According to the paper \cite{avron_quantum_2021}, the noise in the output scales exponentially with the depth of the circuit. Therefore, our algorithms have advantages against noise interference. 
%Furthermore, for  DJ problem in distributed scenario, DJ algorithm will no longer be applicable. Our algorithm is applicable. 



%Second, we compare our algorithms with distributed classical deterministic algorithm. Our distributed quantum algorithm needs two queries for each oracle $O_{f_w}$ to solve DJ problem. However, distributed classical deterministic algorithm needs to query oracles $O(2^{n-t})$ times. Therefore, our algorithms have the advantage of exponential acceleration compared with the classical distributed algorithm.



%Third, we compare our algorithms with the best  distributed quantum algorithm for DJ problem proposed previously \cite{avron_quantum_2021}. 
%For $t=1$, the algorithm in paper \cite{avron_quantum_2021} can not solve DJ problem exactly. Our algorithm can  solve it exactly. In addition, the algorithm  in \cite{avron_quantum_2021} cannot deal with the case of $t > 1$. Our algorithm can handle it. Therefore, our distributed quantum algorithm has higher scalability.






 

%Although the number of qubits used may be increased in our algorithm, the circuit depth of our algorithm is decreased.This is advantageous when the implementation of the oracle is the main overhead. After the function in DJ problem is decomposed into subfunctions, the depth of the implemented oracle of the subfunctions may be smaller than the original function because the subfunctions are simpler than the original function. So the circuit depth may be further reduced. In addition

%In  DJ problem of the distributed scenario, DJ algorithm will no longer be applicable. Fortunately, our algorithm is applicable. 








%\textbf{Space complexity} 

%\textbf{Time complexity.} 

%\textbf{Circuit depth}.

%\textbf{Communication complexity.}






\iffalse

\section{Comparison with other algorithms} 
Our distributed quantum algorithm needs two queries for each oracle %$O_{f_w}$ 
to solve DJ problem. However, in order to solve DJ problem, classical distributed  algorithms need to query oracles $O(2^{n-t})$ times. Our algorithm has the advantage of exponential acceleration compared with the classical distributed algorithm.

For $t=1$, the algorithm in paper \cite{avron_quantum_2021} to solve DJ problem is not exact. Encouragingly, our algorithms has the advantage of exactness. In addition, the method in \cite{avron_quantum_2021} does not consider handling cases when $t > 1$. It is worth pointing out that we considered expanding to multiple compute nodes and  solving the DJ problem exactly.

%DJ problem is based on the quantum query model. %In the quantum query model, we are mainly concerned with reducing the number of oracle queries. In the distributed scenario, we want to minimize the depth of oracle queries. The original Simon's algorithm\cite{simon_power_1997} needs $O(n)$ queries to solve Simon's problem and our algorithm need $O(n-t)$ queries for each oracle. Our algorithm can reduce the depth of oracle queries compared to the original algorithm. Although the number of qubits used may increase in our algorithm, the query depth of our algorithm decreases. This is also significant in cases where the implementation of the oracle is the main overhead. 

After the original function is decomposed into subfunctions, because the subfunctions are simpler than the original function, the depth of the implemented oracle of subfunctions may be smaller than the original function. Thus, the depth of the circuit may be further reduced. In addition, in  DJ problem of the distributed scenario, DJ algorithm will no longer be applicable. Fortunately, our algorithm is still applicable.

\fi


%\section{Conclusion}\label{Sec7}

\section{Concluding remarks} \label{sec:conclusions}

\iffalse

Our distributed quantum algorithm needs two queries for each oracle %$O_{f_w}$ 
to solve DJ problem. However, in order to solve DJ problem, classical distributed  algorithms need to query oracles $O(2^{n-t})$ times. %Our algorithm has the advantage of exponential acceleration compared with the classical distributed algorithm.

For $t=1$, the algorithm in paper \cite{avron_quantum_2021} to solve DJ problem is not exact. Encouragingly, our algorithm has the advantage of exactness. In addition, the method in \cite{avron_quantum_2021} does not consider handling cases when $t > 1$. It is worth pointing out that we considered expanding to multiple compute nodes and  solving the DJ problem exactly.

%DJ problem is based on the quantum query model. %In the quantum query model, we are mainly concerned with reducing the number of oracle queries. In the distributed scenario, we want to minimize the depth of oracle queries. The original Simon's algorithm\cite{simon_power_1997} needs $O(n)$ queries to solve Simon's problem and our algorithm need $O(n-t)$ queries for each oracle. Our algorithm can reduce the depth of oracle queries compared to the original algorithm. Although the number of qubits used may increase in our algorithm, the query depth of our algorithm decreases. This is also significant in cases where the implementation of the oracle is the main overhead. 
After the original function is decomposed into subfunctions, because the subfunctions are simpler than the original function, the depth of the implemented oracle of subfunctions may be smaller than the original function. Thus, the depth of the circuit may be further reduced. In addition, in  DJ problem of the distributed scenario, DJ algorithm will no longer be applicable. Fortunately, our algorithm is still applicable.

\fi


In comparsion to quantum algorithms, dsitributed quantum algorithms usually have the advantages of less number of input qubits and circuit depth \cite{avron_quantum_2021,Elementary_gates_1995,beals_efficient_2013,Qiu2017DQC,Qiu22,Tan2022DQCSimon,Xiao2023DQAShor}, and this subject is therefore  important and practical. 
However, for designing efficient distributed quantum algorithms, the structure of problem to be solved should be clarified in distributed framework. 


	In this paper, we have discovered the essential structure of DJ problem in  distributed scenario by presenting a number of equivalence characterizations between a DJ problem being constant (balanced) and the properties of its subfunctions. These structure properties have provided fundamental ideas for designing distributed exact DJ algorithms. If these structure properties are ignored and we just use DJ algorithm to solve each subfunction, then the result is not exact and the error will be higher and higher with the increasing of number of subfunctions. 



By using the structure properties of DJ problem in distributed situation, we have designed three distributed exact quantum algorithms for solving DJ problem, that is,  Algorithm \ref{algorithm3}, Algorithm \ref{algorithm2} and Algorithm \ref{algorithm5}. Algorithm \ref{algorithm3} can only solve DJ problem in distributed condition with two subfunctions. So, we have  designed 
 Algorithm \ref{algorithm2}, and it can solve DJ problem in distributed situation   with multiple subfunctions. 




By combining   the  ideas and methods of Algorithm \ref{algorithm3} and Algorithm \ref{algorithm2}, we have further given  Algorithm \ref{algorithm5},
which also can solve DJ problem in distributed situation with multiple computing nodes, and some of its single quantum gates require less qubits than Algorithm \ref{algorithm2}.



These distributed exact DJ algorithms we have designed have the following advantages. First,  compared with distributed classical deterministic  algorithm, our algorithms have  exponential advantage in query complexity; second,  compared with  DJ algorithm, the single query operator in our algorithms requires fewer qubits, and the depth of circuit is reduced \cite{Elementary_gates_1995}, which has better anti-noise performance.

By the way, since each oracle in Algorithm \ref{algorithm2} and Algorithm \ref{algorithm5} is controlled by the same set of control bits and is serially connected,  Algorithm \ref{algorithm2} and Algorithm \ref{algorithm5} are serial quantum query algorithms.
In fact, the oracle query of each computing node in Algorithm \ref{algorithm2} and Algorithm \ref{algorithm5} can be completed in parallel. With the help of auxiliary $\left(2^t-1\right)\left(n-t\right)$ qubits, we can change the state
 (i.e. $\dfrac{1}{\sqrt{2^{n-t}}}\sum_{u\in\{0,1\}^{n-t}}|u\rangle$) of the control register after the first Hadamard transformation to $\dfrac{1}{\sqrt{2^{n-t}}}\sum_{u\in\{0,1\}^{n-t}}\underbrace{|u\rangle|u\rangle \ldots |u\rangle}_{2^t}$. That is to say, we can change the control register from one group to the same $2^t$ groups.

More exactly, after the first Hadamard transformation, we can teleport each group of $n-t$ control qubits to every computing node and use these to control the oracles of the computing nodes.
By teleporting control qubits, we can replace  $O'_{f_w}$ in Algorithm \ref{algorithm2} and  $O''_{f_w}$ in Algorithm \ref{algorithm5} by $O_{f_w}$.

It is clear that in general  the number of qubits needed for $O_{f_w}$ is $n-t+1$, which is less than those for $O'_{f_w}$ and $O''_{f_w}$, due to the fact that $O_{f_w}$ does not necessarily have to cross the line. Therefore, using quantum teleportation to transmit control bits would not only change Algorithm \ref{algorithm2} and Algorithm \ref{algorithm5} to parallel quantum query algorithms, but also reduce the number of qubits required for each of their oracles. However, the use of quantum teleportation may increase the communication complexity of algorithms.



In sequential researches,  we would like to study distributed quantum algorithms for solving generalized DJ problem, generalized Simon problem, and other hidden group problems. 












% then design the distributed quantum algorithms for solving DJ problem, which is closely related to the structure
%of DJ problem, and may provide some inspiration for
%the distributed quantum algorithm for generalized DJ problem.

% Through parallel processing of multiple distributed quantum computing nodes, the oracle of each subfunction requires fewer qubits. This reduces the depth of circuit complexity for each oracle, which helps reducing circuit noise, and makes it easier to be implemented %with exponential acceleration advantages 
%in the current NISQ era.

%Our distributed quantum algorithm has the advantage of exponential acceleration compared with the classical distributed algorithm. Compared with previous distributed quantum algorithms, our algorithm can achieve the advantage of exactness and high scalability.




\section*{Acknowledgements}This work is supported in part by the National Natural Science Foundation of China (Nos. 61876195, 61572532), and the Natural Science Foundation of Guangdong Province of China (No. 2017B030311011).



















\begin{thebibliography}{}

\bibitem{avron_quantum_2021}J. Avron, O. Casper, I. Rozen, Quantum advantage and noise reduction in distributed quantum computing, Physical Review A 104 (5) (2021) 052404.



\bibitem{Elementary_gates_1995} A. Barenco, C.H. Bennett, R. Cleve, D.P. DiVincenzo, N. Margolus, P. Shor, T. Sleator, J.A. Smolin, H. Weinfurter, Elementary gates for quantum computation, Physical Review A 52 (5) (1995) 3457.



\bibitem{bennett_teleporting_1993}C.H. Bennett, G. Brassard, C. Crepeau, R. Jozsa,
A. Peres, W.K. Wootters, Teleporting an unknown quantum state via dual classical and Einstein-Podolsky-Rosen channels, Physical Review Letters 70 (13) (1993) 1895.



\bibitem{goos_distributed_2003}H. Buhrman, H. Röhrig, Distributed quantum computing, in: International Symposium on Mathematical Foundations of Computer Science, 2003, pp. 1--20.



\bibitem{beals_efficient_2013}R. Beals, S. Brierley, O. Gray, A.W. Harrow, S. Kutin,
N. Linden, D. Shepherd, M. Stather, Efficient distributed quantum computing, Proceedings Royal Society A 469 (2153) (2013) 20120686.



\bibitem{caleffi_quantum_2018}M. Caleffi, A.S. Cacciapuoti, G. Bianchi, Quantum internet: from communication to distributed computing!, in: Proceedings of the 5th ACM International Conference on Nanoscale Computing and Communication, 2018, pp. 1--4.



\bibitem{deutsch_rapid_1992}D. Deutsch, R. Jozsa, Rapid solution of problems by quantum computation, Proceedings Royal Society A 439 (1907)
(1992) 553--558.



\bibitem{deutsch_quantum_1985}D. Deutsch, Quantum theory, the Church-Turing principle and the universal quantum computer, Proceedings Royal Society A 400 (1818) (1985) 97--117.



\bibitem{grover_fast_1996}L.K. {Grover}, A fast quantum mechanical algorithm for database search, in: Proceedings of the twenty-eighth annual ACM symposium on Theory of computing, 1996, pp. 212--219.



\bibitem{HHL_2009}A.W. Harrow, A. Hassidim, S. Lloyd, Quantum algorithm for linear systems of equations, Physical Review Letters 103 (15) (2009) 150502.



\bibitem{Implementation of the Deutsch-Jozsa Algorithm_2013} C.V. Kraus, P. Zoller, M.A. Baranov, Braiding of atomic Majorana fermions in wire networks and implementation of the Deutsch-Jozsa algorithm, Physical Review Letters 111 (20) (2013) 203001.



\bibitem{Qiu2017DQC}K. Li, D.W. Qiu, L.Z. Li, S.G. Zheng, Z.B. Rong, Application of distributed semi-quantum computing model in phase estimation, Information Processing Letters 120 (2017) 23--29.



\bibitem{nielsen_quantum_2010}M.A. {Nielsen}, I.L. {Chuang}, Quantum computation and quantum information, Cambridge University Press, Cambridge, 2000.



\bibitem{preskill_quantum_2018}J. Preskill, Quantum computing in the NISQ era and beyond, Quantum 2 (2018) 79.



\bibitem{Qiu22} D.W. Qiu, L. Luo, L.G. Xiao, Distributed Grover's algorithm,
arXiv: 2204.10487v3.



\bibitem{Generalized Deutsch-Jozsa proble_2018} D.W. Qiu, S.G. Zheng, Generalized Deutsch-Jozsa problem and the optimal quantum algorithm, Physical Review A 97 (6) (2018) 062331.



\bibitem{Revisiting Deutsch-Jozsa algorithm_2018} D.W. Qiu, S.G. Zheng, Revisiting Deutsch-Jozsa algorithm, Information and Computation 275 (2020) 104605.



\bibitem{simon_power_1997}D.R. Simon, On the power of quantum computation, SIAM Journal on Computing 26 (5) (1997) 1474--1483.



\bibitem{shor_polynomial-time_1997}P.W. {Shor}, Algorithms for quantum computation: discrete logarithms and
factoring, in: Proceedings of the 35th Annual Symposium on Foundations of Computer
Science, 1994, pp. 124--134.



\bibitem{Tan2022DQCSimon}J.W. Tan, L.G. Xiao, D.W. Qiu, L. Luo, P. Mateus, Distributed quantum algorithm for Simon's problem, Physical Review A 106 (3) (2022) 032417.



\bibitem{Xiao2023DQAShor} L.G. Xiao, D.W. Qiu, L. Luo, P. Mateus, Distributed Shor's algorithm, Quantum Information and Computation 23 (1\&2) (2023) 0027--0044.



\bibitem{Testing Boolean Functions Properties_2021} Z.W. Xie, D.W. Qiu, G.Y. Cai, J. Gruska, P. Mateus, Testing Boolean functions properties, Fundamenta Informaticae 182 (4) (2021) 321--344.



 

 











\iffalse


\bibitem{aspuru2005simulated}
A. Aspuru-Guzik, A.D. Dutoi, P. J. Love, M. Head-Gordon, Simulated quantum computation of molecular energies, Science, 309 (5741) (2005) 1704--1707.


\bibitem{avron2021quantum}
J. Avron, O. Casper, I. Rozen, Quantum advantage and noise reduction in distributed quantum computing, Physical Review A, 104 (5) (2021) 052404.


\bibitem{beals2013effcient}
R. Beals, S. Brierley, O. Gray, A. W. Harrow, S. Kutin, N. Linden, D. Shepherd, M. Stather, Efficient distributed quantum computing, Proceedings of the Royal Society A Mathematical Physical and Engineering Science, 469 (2153) (2013) 20120686.

\bibitem{beauregard2003circuit}
S. Beauregard, Circuit for Shor's algorithm using $2n+3$ qubits, Quantum Information and Computation, 3 (2) (2003) 175--185.

\bibitem{bennett1993teleporting}
C. Bennett, G. Brassard, C. Cr\'epeau, R. Jozsa, A. Peres, and W. K. Wootters, Teleporting an unknown quantum state via dual classical and Einstein-Podolsky-Rosen channels, Physiscal Review Letters, 70 (13) (1993) 1895--1899.

\bibitem{craig2021how}
C. Gidney, M. Ekera, How to factor 2048 bit RSA integers in 8 hours using 20 million noisy qubits, Quantum, 5 (2021) 433.


\bibitem{ni2007quantum}
N. Gisin, R. Thew, Quantum communication, Nature Photonics, 1 (3) (2007) 165--171.

\bibitem{haner2017factoring}
T. Haner, M. Roetteler, K. M. Svore, Factoring using $2n+2$ qubits with Toffoli based modular multiplication, Quantum Information and Computation, 17 (7-8) (2017) 673--684.

\bibitem{grover1996a}
L. K. {Grover}, A fast quantum mechanical algorithm for database search, in: Proceedings of the twenty-eighth annual ACM symposium on Theory of computing, 1996, pp. 212--219.

\bibitem{harrow2009quantum}
A. W. Harrow, A. Hassidim, S. Lloyd, Quantum algorithm for linear systems of equations, Physical Review Letters, 103 (15) (2009) 150502.

\bibitem{li2017application}
K. Li, D. Qiu, L. Li, S. Zheng, Z. Rong, Application of distributed semi-quantum computing model in phase estimation, Information Processing Letters 120 (2017) 23--29.

\bibitem{montanaro2016quantum}
A. Montanaro, Quantum algorithms: an overview, npj Quantum Information, 2 (2016) 15023.

\bibitem{nielsen2000quantum}
M. A. {Nielsen}, I. L. {Chuang}, Quantum Computation and Quantum Information, Cambridge University Press, Cambridge,  2000.

\bibitem{parker2000efficient}
S. Parker, M. B. Plenio, Efficient factorization with a single pure qubit and $\log N$ mixed qubits, Physical Review Letters, 85 (14) (2000) 3049--3052.

\bibitem{rosenberg2016solving}
G. Rosenberg, P. Haghnegahdar, P. Goddard, P. Carr, K. Wu, M. L. De Prado, Solving the optimal trading trajectory problem using a quantum annealer, IEEE Journal of Selected Topics in Signal Processing, 10 (6) (2016) 1053--1060.

\bibitem{shor1994algorithms}
P.W. {Shor}, Algorithms for quantum computation: discrete logarithms and
  factoring, in: Proceedings of the 35th Annual Symposium on Foundations of Computer
  Science, 1994, pp. 124--134.

\bibitem{shor1999polynomial}
P.W. {Shor}, Polynomial-time algorithms for prime factorization and discrete logarithms on a quantum computer, Siam Review 41 (2) (1999) 303--332.


\bibitem{yimsiriwattana2004distributed}
A. Yimsiriwattana, S.J. Lomonaco, Distributed quantum computing: a distributed Shor algorithm, Quantum Information and Computation II, 5436 (2004) 360--372.




%----------------------多行注释
\iffalse
\bibitem{ABG06}
A. Ambainis, M. Beaudry, M. Golovkins, A. Kikusts, M. Mercer, D. Therien,
Algebraic results on quantum automata, Theory of Computing Systems 39 (1) (2006) 165--188.

\bibitem{strengths}
A.~{Ambainis}, R.~{Freivalds}, 1-way quantum finite automata: strengths,
  weaknesses and generalizations, in: Proceedings 39th Annual Symposium on
  Foundations of Computer Science (Cat. No.98CB36280), 1998, pp. 332--341.

\bibitem{AW02}A. Ambainis,  J. Watrous, Two-way finite automata with quantum and classical
states, Theoretical Computer Science 287 (2002) 299-311.

\bibitem{bell2021on}
P.~C. {Bell}, M.~{Hirvensalo}, On injectivity of quantum finite automata,
  Journal of Computer and System Sciences 122 (2021) 19--33.

\bibitem{belovs2007multi}
A.~{Belovs}, A.~{Rosmanis}, J.~{Smotrovs}, Multi-letter reversible and quantum
  finite automata, in: DLT'07 Proceedings of the 11th international conference
  on Developments in language theory, 2007, pp. 60--71.

\bibitem{bertoni2003quantum}
A.~{Bertoni}, C.~{Mereghetti}, B.~{Palano}, Quantum computing: 1-way quantum
  automata, in: DLT'03 Proceedings of the 7th international conference on
  Developments in language theory, Vol. 2710, 2003, pp. 1--20.

\bibitem{bhatia2019quantum}
A.~S. {Bhatia}, A.~{Kumar}, Quantum finite automata: survey, status and
  research directions., arXiv preprint arXiv:1901.07992 (2019).

\bibitem{bianchi2014size}
M.~P. {Bianchi}, C.~{Mereghetti}, B.~{Palano}, Size lower bounds for quantum
  automata, Theoretical Computer Science 551 (2014) 102--115.

\bibitem{characterizations}
A.~{Brodsky}, N.~{Pippenger}, Characterizations of 1-way quantum finite
  automata, SIAM Journal on Computing 31~(5) (2002) 1456--1478.

\bibitem{golovkins2002probabilistic}
M.~{Golovkins}, M.~{Kravtsev}, Probabilistic reversible automata and quantum
  automata, in: COCOON '02 Proceedings of the 8th Annual International
  Conference on Computing and Combinatorics, 2002, pp. 574--583.

\bibitem{grover1996a}
L.~K. {Grover}, A fast quantum mechanical algorithm for database search, in:
  Proceedings of the twenty-eighth annual ACM symposium on Theory of Computing,
  1996, pp. 212--219.

%\bibitem{hopcroft1979introduction}
%J.~E. {Hopcroft}, R.~{Motwani}, {Rotwani}, J.~D. {Ullman}, Introduction to
%  Automata Theory, Languages, and Computation, 1979.

%\bibitem{hopcroft1979}
%J.~E. Hopcroft, J.~D. Ullman, Introduction To Automata Theory, Languages, And
%  Computation, 1st Edition, Addison-Wesley Longman Publishing Co., Inc., USA,
%  1990.

\bibitem{hopcroft1979}
J.E. Hopcroft, J.D. Ullman, Introduction to Automata Theory, Languages, and Computation, Addison-Wesley,
New York, 1979.

\bibitem{kondacs1997on}
A.~{Kondacs}, J.~{Watrous}, On the power of quantum finite state automata, in:
  Proceedings 38th Annual Symposium on Foundations of Computer Science, 1997,
  pp. 66--75.

\bibitem{li2016lower}
L.~{Li}, D.~{Qiu}, Lower bounds on the size of semi-quantum finite automata,
  Theoretical Computer Science 623 (2016) 75--82.

\bibitem{mereghetti2001note}
C.~{Mereghetti}, B.~{Palano}, G.~{Pighizzini}, Note on the succinctness of
  deterministic, nondeterministic, probabilistic and quantum finite automata,
  Theoretical Informatics and Applications 35~(5) (2001) 477--490.

\bibitem{moore2000quantum}
C.~{Moore}, J.~P. {Crutchfield}, Quantum automata and quantum grammars,
  Theoretical Computer Science 237~(1) (2000) 275--306.

\bibitem{nielsen2000quantum}
M.~A. {Nielsen}, I.~L. {Chuang}, Quantum Computation and Quantum Information,
  2000.

\bibitem{nishimura2009an}
H.~{Nishimura}, T.~{Yamakami}, An application of quantum finite automata to
  interactive proof systems, Journal of Computer and System Sciences 75~(4)
  (2009) 255--269.

\bibitem{Pas00} K. Paschen, Quantum finite automata using ancilla qubits,
University of Karlsruhe, Technical report, 2000.

\bibitem{QIU20151QFAC}
D.~{Qiu}, L.~{Li}, P.~{Mateus}, A.~{Sernadas}, Exponentially more concise
  quantum recognition of non-rmm regular languages, Journal of Computer and
  System Sciences 81~(2) (2015) 359--375.

\bibitem{QY09}
 D. Qiu, S. Yu, Hierarchy and equivalence of multi-letter quantum finite automata, Theoretical Computer Science 410(30-32) (2009) 3006-3017.


\bibitem{say2014quantum}
A.~C. {Say}, A.~{Yakaryilmaz}, Quantum finite automata: A modern introduction,
  Computing with New Resources (2014) 208--222.

\bibitem{shor1994algorithms}
P.W. {Shor}, Algorithms for quantum computation: discrete logarithms and
  factoring, in: Proceedings of the 35th Annual Symposium on Foundations of Computer
  Science, 1994, pp. 124--134.

%\bibitem{shor1999polynomial}
%P.W. {Shor}, Polynomial-time algorithms for prime factorization and discrete logarithms on a quantum computer, Siam Review 41~(2) (1999) 303--332.

\bibitem{tzeng1992a}
W.-G. {Tzeng}, A polynomial-time algorithm for the equivalence of probabilistic
  automata, SIAM Journal on Computing 21~(2) (1992) 216--227.

\bibitem{yakaryilmaz2010succinctness}
A.~{Yakaryilmaz}, A.~C.~C. {Say}, Succinctness of two-way probabilistic and
  quantum finite automata, Discrete Mathematics and Theoretical Computer
  Science 12~(4) (2010) 19--40.

\bibitem{yamakami2014one}
T.~{Yamakami}, One-way reversible and quantum finite automata with advice,
  Information and Computation 239 (2014) 122--148.
\fi
%---------------多行注释结束


\fi


\end{thebibliography}















\iffalse

\section*{Appendix A}

\setcounter{defi}{0}

\setcounter{theorem}{0}

\setcounter{lemma}{0}

\begin{defi}\label{deltadef}
Suppose function $f:\{0,1\}^n \rightarrow \{0,1\}$, for all $u \in \{0,1\}^{n-t}$, let  $\delta(u) = |\{w\in\{0,1\}^t|f(uw)=0\}|-|\{w\in\{0,1\}^t|f(uw)=1\}|$.%=2^t-|\{w\in\{0,1\}^t|f(uw)=1\}|-|\{w\in\{0,1\}^t|f(uw)=1\}|=2^t-2|\{w\in\{0,1\}^t|f(uw)=1\}|$.
\end{defi}



According to the definition of $\delta(u)$, we have
\begin{equation}\label{delta1}
\begin{split}
&\delta(u)\\
 =&|\{w\in\{0,1\}^t|f(uw)=0\}|\\
              &\quad-|\{w\in\{0,1\}^t|f(uw)=1\}|\\
  =&2^t-|\{w\in\{0,1\}^t|f(uw)=1\}|\\
  &\quad -|\{w\in\{0,1\}^t|f(uw)=1\}|\\
  =&2^t-2|\{w\in\{0,1\}^t|f(uw)=1\}|\\ 
  =&2^t-2\sum_{w\in\{0,1\}^t}f(uw).
\end{split}
\end{equation}



%Notice that there could be multiple identical elements in $G(u)$. 
%An example of $\delta(u)$ is shown in Appendix \ref{example}.



The following theorems concerning $\delta(u)$ are useful and important.





\begin{theorem}\label{The1} Suppose function $f:\{0,1\}^n \rightarrow \{0,1\}$, satisfies that it is either constant or balanced, it is divided into $2^t$ subfunctions $f_w$ $(\forall u \in \{0,1\}^{n-t}, w \in \{0,1\}^{t}, f_w(u)=f(uw))$. If $\exists$ $u\in\{0,1\}^{n-t}$ such that $|\delta(u)|\neq 2^t$, then $f$ is balanced. %a string $s \in \{0,1\}^n$  with $s\neq 0^n$, such that $f(x) = f(y)$ if and only if $x = y$ or $x \oplus y = s$. 
%Then
%  $\forall u,v \in \{0,1\}^{n-t},S(u)=S(v)$ if and only if $u \oplus v = 0^{n-t}$ or $u \oplus v = s_1$, where $s=s_1s_2$.
\end{theorem}

\begin{proof}
Suppose $\exists$ $u\in\{0,1\}^{n-t}$ such that $|\delta(u)|\neq 2^t$, then $f$ is constant. 
If $f$ is  constant, then $\forall u\in\{0,1\}^{n-t}$, $\sum_{w\in\{0,1\}^t}f(uw)=0$ or $\sum_{w\in\{0,1\}^t}f(uw)=2^t$. According to  equation $(\ref{delta1})$, then we have $|\delta(u)|= 2^t$, which is contrary to the assumption $|\delta(u)|\neq 2^t$. Therefore, if $\exists$ $u\in\{0,1\}^{n-t}$ such that $|\delta(u)|\neq 2^t$, then $f$ is balanced.
\end{proof}


\begin{theorem}\label{The2} Suppose function $f:\{0,1\}^n \rightarrow \{0,1\}$, satisfies that it is either constant or balanced, it is divided into $2^t$ subfunctions $f_w$ $(\forall u \in \{0,1\}^{n-t}, w \in \{0,1\}^{t}, f_w(u)=f(uw))$. Then $f$ is balanced if and only if $\sum\nolimits_{u\in\{0,1\}^{n-t}} \delta(u) = 0$. %Then $f$ is constant  if and only if $\sum\nolimits_{u\in\{0,1\}^{n-t}} \delta(u) =\pm 2^n$.
Then $f$ is constant  if and only if for all $u \in \{0,1\}^{n-t}$, $\delta(u) =2^t$ or for all $u \in \{0,1\}^{n-t}$, $\delta(u) =-2^t$. 
\end{theorem}


\begin{proof}
Firstly, we prove that $f$ is balanced if and only if $\sum\nolimits_{u\in\{0,1\}^{n-t}} \delta(u) = 0$.

(\romannumeral1) $\Longleftarrow$. If $\sum\nolimits_{u\in\{0,1\}^{n-t}} \delta(u)=0$, according to equation $(\ref{delta1})$, then we have 
\begin{equation}
\begin{split}
&\sum\nolimits_{u\in\{0,1\}^{n-t}} \delta(u)\\
=&\sum\nolimits_{u\in\{0,1\}^{n-t}} (2^t-2\sum_{w\in\{0,1\}^t}f(uw))\\
=&\sum\nolimits_{u\in\{0,1\}^{n-t}}2^t-2\sum\nolimits_{u,w\in\{0,1\}^{n-t}}f(uw)\\
=&2^n-2\sum\nolimits_{x\in\{0,1\}^{n}}f(x)\\
=&0, 
\end{split}
\end{equation}
that is $\sum\nolimits_{x\in\{0,1\}^{n}}f(x)=2^{n-1}$.

So we have 
\begin{equation}
\begin{split}
&|\{x\in\{0,1\}^n|f(x)=1\}|\\
=&\sum\nolimits_{x\in\{0,1\}^{n}}f(x)=2^{n-1}. 
\end{split}
\end{equation}
\begin{equation}
\begin{split}
&|\{x\in\{0,1\}^n|f(x)=0\}|\\
=&2^n-|\{x\in\{0,1\}^n|f(x)=1\}|\\
=&2^{n-1}. 
\end{split}
\end{equation}
Therefore, $f$ is balanced.

(\romannumeral2) $\Longrightarrow$. If $f$ is balanced, according to the definition of DJ problem, then  we have $|\{x\in\{0,1\}^n|f(x)=0\}|=|\{x\in\{0,1\}^n|f(x)=1\}|=2^{n-1}$. According to equation $(\ref{delta1})$, then we have 
\begin{equation}
\begin{split}
&\sum\nolimits_{u\in\{0,1\}^{n-t}} \delta(u)\\
=&\sum\nolimits_{u\in\{0,1\}^{n-t}} (2^t-2\sum_{w\in\{0,1\}^t}f(uw))\\
=&\sum\nolimits_{u\in\{0,1\}^{n-t}}2^t-2\sum\nolimits_{u,w\in\{0,1\}^{n-t}}f(uw)\\
=&2^n-2\sum\nolimits_{x\in\{0,1\}^{n}}f(x)\\
=&2^n-2|\{x\in\{0,1\}^n|f(x)=1\}|\\
=&0.
\end{split}
\end{equation}



Secondly, we prove that $f$ is constant if and only if $\sum\nolimits_{u\in\{0,1\}^{n-t}} \delta(u) = 2^n$. 



(\romannumeral3) $\Longleftarrow$. If $\sum\nolimits_{u\in\{0,1\}^{n-t}} \delta(u)=\pm 2^n$, according to equation $(\ref{delta1})$, then we have 
\begin{equation}
\begin{split}
&\sum\nolimits_{u\in\{0,1\}^{n-t}} \delta(u)\\
=&\sum\nolimits_{u\in\{0,1\}^{n-t}} (2^t-2\sum_{w\in\{0,1\}^t}f(uw))\\
=&\sum\nolimits_{u\in\{0,1\}^{n-t}}2^t-2\sum\nolimits_{u,w\in\{0,1\}^{n-t}}f(uw)\\
=&2^n-2\sum\nolimits_{x\in\{0,1\}^{n}}f(x)=\pm 2^n,
\end{split}
\end{equation}
that is $\sum\nolimits_{x\in\{0,1\}^{n}}f(x)=0$ or $2^n$. 

So $|\{x\in\{0,1\}^n|f(x)=1\}|=0$ or $2^n$, that is $f(x) \equiv 0$ or $f(x) \equiv 1$. Therefore, $f$ is constant.

(\romannumeral4) $\Longrightarrow$. If $f$ is constant, according to the definition of DJ problem, then we have $f(x) \equiv 0$ or $f(x) \equiv 1$. So $|\{x\in\{0,1\}^n|f(x)=1\}|=0$ or $2^n$, that is $\sum\nolimits_{x\in\{0,1\}^{n}}f(x)=0$ or $2^n$. According to equation $(\ref{delta1})$, then we have 
\begin{equation}
\begin{split}
&\sum\nolimits_{u\in\{0,1\}^{n-t}} \delta(u)\\
=&\sum\nolimits_{u\in\{0,1\}^{n-t}} (2^t-2\sum_{w\in\{0,1\}^t}f(uw))\\
=&\sum\nolimits_{u\in\{0,1\}^{n-t}}2^t-2\sum\nolimits_{u,w\in\{0,1\}^{n-t}}f(uw)\\
=&2^n-2\sum\nolimits_{x\in\{0,1\}^{n}}f(x)\\
=&\pm 2^n.
\end{split}
\end{equation}





Secondly, we prove that $f$ is constant if and only if for all $u \in \{0,1\}^{n-t}$, $\delta(u) =2^t$ or for all $u \in \{0,1\}^{n-t}$, $\delta(u) =-2^t$. 



(\romannumeral3) $\Longleftarrow$. If  for all $u \in \{0,1\}^{n-t}$, $\delta(u) =2^t$, then according to equation (\ref{delta1}), for all $u \in \{0,1\}^{n-t}$, we have $\sum_{w\in\{0,1\}^t}f(uw)=0$. So for all $x \in \{0,1\}^{n}$, $f(x)=0$, that is $f(x) \equiv 0$.





 If  for all $u \in \{0,1\}^{n-t}$, $\delta(u) =-2^t$, then according to equation (\ref{delta1}), for all $u \in \{0,1\}^{n-t}$, we have $\sum_{w\in\{0,1\}^t}f(uw)=2^t$. So for all $x \in \{0,1\}^{n}$, $f(x)=1$, that is $f(x) \equiv 1$.
 
 Therefore, if for all $u \in \{0,1\}^{n-t}$, $\delta(u) =2^t$ or for all $u \in \{0,1\}^{n-t}$, $\delta(u) =-2^t$, then $f$ is constant.

(\romannumeral4) $\Longrightarrow$. If $f$ is constant, according to the definition of DJ problem, then we have $f(x) \equiv 0$ or $f(x) \equiv 1$. 


If $f(x) \equiv 0$, then for all $x \in \{0,1\}^{n}$, $f(x)=0$. So for all $u \in \{0,1\}^{n-t}$, we have $\sum_{w\in\{0,1\}^t}f(uw)=0$.  According to equation (\ref{delta1}), for all $u \in \{0,1\}^{n-t}$, we have $\delta(u) =2^t$.


If $f(x) \equiv 1$, then for all $x \in \{0,1\}^{n}$, $f(x)=1$. So for all $u \in \{0,1\}^{n-t}$, we have $\sum_{w\in\{0,1\}^t}f(uw)=2^t$.  According to equation (\ref{delta1}), for all $u \in \{0,1\}^{n-t}$, we have $\delta(u) =-2^t$.

 Therefore, if $f$ is constant, then for all $u \in \{0,1\}^{n-t}$, $\delta(u) =2^t$ or for all $u \in \{0,1\}^{n-t}$, $\delta(u) =-2^t$.

\end{proof}






The following theorems %concerning DJ problem
 are profound and interesting.




\begin{theorem}\label{The3} Suppose function $f:\{0,1\}^n \rightarrow \{0,1\}$, satisfies that it is either constant or balanced, it is divided into subfunctions $f_0$ and $f_1$. If $\exists$ $u\in\{0,1\}^{n-1}$ such that $f(u0)\oplus f(u1)=1$, then $f$ is balanced.
\end{theorem}

\begin{proof}
Suppose $\exists$ $u\in\{0,1\}^{n-1}$ such that $f(u0)\oplus f(u1)=1$, then $f$ is constant. 
If $f$ is  constant, then $\forall u\in\{0,1\}^{n-1}$, $f(u0)=f(u1)=0$ or $f(u0)=f(u1)=1$. So $f(u0)\oplus f(u1)=0$, which is contrary to the assumption $f(u0)\oplus f(u1)=1$. Therefore, if $\exists$ $u\in\{0,1\}^{n-1}$ such that $f(u0)\oplus f(u1)=1$, then $f$ is balanced.
\end{proof}


\begin{defi}
Suppose function $f:\{0,1\}^n \rightarrow \{0,1\}$, let $B_0=|\{u\in\{0,1\}^{n-1}|f(u0)\oplus f(u1)=0, f(u0)=0\}|$, $B_1=|\{u\in\{0,1\}^{n-1}|f(u0)\oplus f(u1)=0, f(u0)=1\}|$.
\end{defi}


\begin{defi}
 Suppose function $f:\{0,1\}^n \rightarrow \{0,1\}$, let $M=|\{u\in\{0,1\}^{n-1}|f(u0)\oplus f(u1)=0\}|$ $(0\leq M\leq 2^{n-1})$. 
\end{defi}

\begin{theorem}\label{The4} Suppose function $f:\{0,1\}^n \rightarrow \{0,1\}$, satisfies that it is either constant or balanced, it is divided into subfunctions $f_0$ and $f_1$. Then $f$ is balanced if and only if $B_0=B_1=M/2$. %Then $f$ is constant  if and only if $|\{u\in\{0,1\}^{n-1}|f(u0)=0\}|=|\{u\in\{0,1\}^{n-1}|f(u1)=0\}|=2^{n-1}$ or $|\{u\in\{0,1\}^{n-1}|f(u0)=1\}|=|\{u\in\{0,1\}^{n-1}|f(u1)=1\}|=2^{n-1}$. 
\end{theorem}

\begin{proof}
%First, we prove that $f$ is balanced if and only if $B_0=B_1=M/2$.

(\romannumeral1) $\Longleftarrow$. Suppose $B_0=B_1=M/2$ $(0\leq M\leq 2^{n-1})$, then $f$ is constant. If $f$ is constant, then $f\equiv 0$ or $f\equiv 1$, that is $|\{u\in\{0,1\}^{n-1}|f(u0)=f(u1)=0\}|=2^{n-1}$ or $|\{u\in\{0,1\}^{n-1}|f(u0)=f(u1)=1\}|=2^{n-1}$. So $B_0=2^{n-1}$ or $B_1=2^{n-1}$, which is contrary to the assumption $B_0=B_1=M/2$ $(0\leq M\leq 2^{n-1})$. Therefore, if $B_0=B_1=M/2$, then $f$ is balanced, where $M=|\{u\in\{0,1\}^{n-1}|f(u0)\oplus f(u1)=0\}|$.


(\romannumeral2) $\Longrightarrow$. %If $f$ is balanced, according to the definition of DJ problem, then   $|\{x\in\{0,1\}^n|f(x)=0\}|=|\{x\in\{0,1\}^n|f(x)=1\}|=2^{n-1}$. 
Since 
\begin{equation}
\begin{split}
M=&|\{u\in\{0,1\}^{n-1}|f(u0)\oplus f(u1)=0\}|\\
=&|\{u\in\{0,1\}^{n-1}|f(u0)= f(u1)=0\}|\\
&+|\{u\in\{0,1\}^{n-1}|f(u0)= f(u1)=1\}|,
\end{split}
\end{equation}
 therefore $|\{u\in\{0,1\}^{n-1}|f(u0)\oplus f(u1)=1\}|=2^{n-1}-M$, that is 
\begin{equation}
\begin{split}
& |\{u\in\{0,1\}^{n-1}|f(u0)=0,f(u1)=1\}|\\
&+|\{u\in\{0,1\}^{n-1}|f(u0)=1,f(u1)=0\}|\\
 =&2^{n-1}-M. 
 \end{split}
\end{equation}

 We have 
 \begin{equation}
\begin{split}
 &|\{u\in\{0,1\}^{n-1}|f(u0)= f(u1)=0\}|\\
 =&|\{u\in\{0,1\}^{n-1}|f(u0)=0\}|\\
 &-|\{u\in\{0,1\}^{n-1}|f(u0)=0,f(u1)=1\}|\\
 =&|\{u\in\{0,1\}^{n-1}|f(u1)=0\}|\\
 &-|\{u\in\{0,1\}^{n-1}|f(u0)=1,f(u1)=0\}|. 
\end{split}
\end{equation}

Let 
 \begin{equation}
\begin{split}
S_0=&|\{u\in\{0,1\}^{n-1}|f(u0)=0\}|\\
&-|\{u\in\{0,1\}^{n-1}|f(u0)=0,f(u1)=1\}|.
\end{split}
\end{equation}
\begin{equation}
\begin{split}
S_1=&|\{u\in\{0,1\}^{n-1}|f(u1)=0\}|\\
&-|\{u\in\{0,1\}^{n-1}|f(u0)=1,f(u1)=0\}|. 
\end{split}
\end{equation}

Then 
\begin{equation}
\begin{split}
S_0=S_1=|\{u\in\{0,1\}^{n-1}|f(u0)= f(u1)=0\}|.
\end{split}
\end{equation}

Then 
\begin{equation}
\begin{split}
&S_0+S_1\\
=&(|\{u\in\{0,1\}^{n-1}|f(u0)=0\}|\\
&-|\{u\in\{0,1\}^{n-1}|f(u0)=0,f(u1)=1\}|)\\
&+(|\{u\in\{0,1\}^{n-1}|f(u1)=0\}|\\
&-|\{u\in\{0,1\}^{n-1}|f(u0)=1,f(u1)=0\}|)\\
=&(|\{u\in\{0,1\}^{n-1}|f(u0)=0\}|\\
&+|\{u\in\{0,1\}^{n-1}|f(u1)=0\}|)\\
&-(|\{u\in\{0,1\}^{n-1}|f(u0)=0,f(u1)=1\}|\\
&+|\{u\in\{0,1\}^{n-1}|f(u0)=1,f(u1)=0\}|)\\
=&(|\{u\in\{0,1\}^{n-1}|f(u0)=0\}|\\
&+|\{u\in\{0,1\}^{n-1}|f(u1)=0\}|)-(2^{n-1}-M). 
\end{split}
\end{equation}

If $f$ is balanced, according to the definition of DJ problem, then we have $|\{x\in\{0,1\}^n|f(x)=0\}|=|\{x\in\{0,1\}^n|f(x)=1\}|=2^{n-1}$. So $|\{u\in\{0,1\}^{n-1}|f(u0)=0\}|+|\{u\in\{0,1\}^{n-1}|f(u1)=0\}|=|\{x\in\{0,1\}^n|f(x)=0\}|=2^{n-1}$. Therefore, we have
\begin{equation}
\begin{split}
&S_0+S_1\\
=&(|\{u\in\{0,1\}^{n-1}|f(u0)=0\}|\\
&+|\{u\in\{0,1\}^{n-1}|f(u1)=0\}|)-(2^{n-1}-M)\\
=&2^{n-1}-(2^{n-1}-M)\\
=&M, 
\end{split}
\end{equation}
then from $S_0=S_1=|\{u\in\{0,1\}^{n-1}|f(u0)= f(u1)=0\}|$, we have $S_0=S_1=|\{u\in\{0,1\}^{n-1}|f(u0)= f(u1)=0\}|=M/2$. 


Further, from $M=|\{u\in\{0,1\}^{n-1}|f(u0)= f(u1)=0\}|+|\{u\in\{0,1\}^{n-1}|f(u0)= f(u1)=1\}|$, we have 
\begin{equation}
\begin{split}
&|\{u\in\{0,1\}^{n-1}|f(u0)= f(u1)=1\}|\\
=&M-|\{u\in\{0,1\}^{n-1}|f(u0)= f(u1)=0\}|\\
=&M-M/2\\
=&M/2.
\end{split}
\end{equation}

Since 
\begin{equation}
\begin{split}
B_0=&|\{u\in\{0,1\}^{n-1}|f(u0)\oplus f(u1)=0, f(u0)=0\}|\\
=&|\{u\in\{0,1\}^{n-1}|f(u0)= f(u1)=0\}|.
\end{split}
\end{equation}
\begin{equation}
\begin{split}
B_1=&|\{u\in\{0,1\}^{n-1}|f(u0)\oplus f(u1)=0, f(u0)=1\}|\\
=&|\{u\in\{0,1\}^{n-1}|f(u0)= f(u1)=1\}|.
\end{split}
\end{equation}
Therefore, we have $B_0=B_1=M/2$.

%$S_0=|\{u\in\{0,1\}^{n-1}|f(u0)=0\}|-|\{u\in\{0,1\}^{n-1}|f(u0)=0,f(u1)=1\}|=B_0$, $S_1=|\{u\in\{0,1\}^{n-1}|f(u1)=0\}|-|\{u\in\{0,1\}^{n-1}|f(u0)=1,f(u1)=0\}|=B_1$, we have $B_0=B_1=M/2$.



% According to equation $(\ref{delta1})$, then we have $\sum\nolimits_{u\in\{0,1\}^{n-t}} \delta(u)=\sum\nolimits_{u\in\{0,1\}^{n-t}} (2^t-2\sum_{w\in\{0,1\}^t}f(uw))=\sum\nolimits_{u\in\{0,1\}^{n-t}}2^t-2\sum\nolimits_{u,w\in\{0,1\}^{n-t}}f(uw)=2^n-2\sum\nolimits_{x\in\{0,1\}^{n}}f(x)=2^n-2|\{x\in\{0,1\}^n|f(x)=1\}|=0$.

%Second, we prove that $f$ is constant if and only if $|\{u\in\{0,1\}^{n-1}|f(u0)=0\}|=|\{u\in\{0,1\}^{n-1}|f(u1)=0\}|=2^{n-1}$ or $|\{u\in\{0,1\}^{n-1}|f(u0)=1\}|=|\{u\in\{0,1\}^{n-1}|f(u1)=1\}|=2^{n-1}$. 

%(3) $\Longleftarrow$. If $|\{u\in\{0,1\}^{n-1}|f(u0)=0\}|=|\{u\in\{0,1\}^{n-1}|f(u1)=0\}|=2^{n-1}$ or $|\{u\in\{0,1\}^{n-1}|f(u0)=1\}|=|\{u\in\{0,1\}^{n-1}|f(u1)=1\}|=2^{n-1}$, then $|\{x\in\{0,1\}^n|f(x)=1\}|=0$ or $2^n$, that is $f(x) \equiv 0$ or $f(x) \equiv 1$. Therefore, $f$ is constant.

%(4) $\Longrightarrow$. If $f$ is constant, according to the definition of DJ problem, then  $f(x) \equiv 0$ or $f(x) \equiv 1$. So $|\{x\in\{0,1\}^n|f(x)=1\}|=0$ or $2^n$. Therefore $|\{u\in\{0,1\}^{n-1}|f(u0)=0\}|=|\{u\in\{0,1\}^{n-1}|f(u1)=0\}|=2^{n-1}$ or $|\{u\in\{0,1\}^{n-1}|f(u0)=1\}|=|\{u\in\{0,1\}^{n-1}|f(u1)=1\}|=2^{n-1}$.

\end{proof}


\begin{theorem}\label{The5} Suppose function $f:\{0,1\}^n \rightarrow \{0,1\}$, satisfies that it is either constant or balanced, it is divided into subfunctions $f_0$ and $f_1$. Then $f$ is constant  if and only if $|\{u\in\{0,1\}^{n-1}|f(u0)=0\}|=|\{u\in\{0,1\}^{n-1}|f(u1)=0\}|=2^{n-1}$ or $|\{u\in\{0,1\}^{n-1}|f(u0)=1\}|=|\{u\in\{0,1\}^{n-1}|f(u1)=1\}|=2^{n-1}$. 
\end{theorem}

\begin{proof}
%Second, we prove that $f$ is constant if and only if $|\{u\in\{0,1\}^{n-1}|f(u0)=0\}|=|\{u\in\{0,1\}^{n-1}|f(u1)=0\}|=2^{n-1}$ or $|\{u\in\{0,1\}^{n-1}|f(u0)=1\}|=|\{u\in\{0,1\}^{n-1}|f(u1)=1\}|=2^{n-1}$. 

(\romannumeral1) $\Longleftarrow$. If $|\{u\in\{0,1\}^{n-1}|f(u0)=0\}|=|\{u\in\{0,1\}^{n-1}|f(u1)=0\}|=2^{n-1}$ or $|\{u\in\{0,1\}^{n-1}|f(u0)=1\}|=|\{u\in\{0,1\}^{n-1}|f(u1)=1\}|=2^{n-1}$, then $|\{x\in\{0,1\}^n|f(x)=1\}|=0$ or $2^n$, that is $f(x) \equiv 0$ or $f(x) \equiv 1$. Therefore, $f$ is constant.

(\romannumeral2) $\Longrightarrow$. If $f$ is constant, according to the definition of DJ problem, then  $f(x) \equiv 0$ or $f(x) \equiv 1$. So $|\{x\in\{0,1\}^n|f(x)=1\}|=0$ or $2^n$. Therefore $|\{u\in\{0,1\}^{n-1}|f(u0)=0\}|=|\{u\in\{0,1\}^{n-1}|f(u1)=0\}|=2^{n-1}$ or $|\{u\in\{0,1\}^{n-1}|f(u0)=1\}|=|\{u\in\{0,1\}^{n-1}|f(u1)=1\}|=2^{n-1}$.

\end{proof}


%Next, we further introduce some notations that will be used in this paper.

%\begin{defi}
%For all $u \in \{0,1\}^{n-t}$, let  $D(u) = |\{w\in\{0,1\}^{t-1}0|f(uw)\oplus f(u(w+1))=1\}|$.
%\end{defi}


\begin{defi}\label{Deltadef}
Suppose function $f:\{0,1\}^n \rightarrow \{0,1\}$, for all $u \in \{0,1\}^{n-t}$, let  $\Delta(u) = |\{w\in\{0,1\}^{t-1}0|f(uw)=f(u(w+1))=0\}|-|\{w\in\{0,1\}^{t-1}0|f(uw)=f(u(w+1))=1\}|$.%=2^t-|\{w\in\{0,1\}^t|f(uw)=1\}|-|\{w\in\{0,1\}^t|f(uw)=1\}|=2^t-2|\{w\in\{0,1\}^t|f(uw)=1\}|$.
\end{defi}


\iffalse

Since 
\begin{equation}
\begin{split}
 &|\{w\in\{0,1\}^{t-1}0|f(uw)\oplus f(u(w+1))=1\}|\\
 &+|\{w\in\{0,1\}^{t-1}0|f(uw)\oplus f(u(w+1))=0\}|\\
 = &|\{w\in\{0,1\}^{t-1}0|f(uw)\oplus f(u(w+1))=1\}|\\
    &+|\{w\in\{0,1\}^{t-1}0|f(uw)=f(u(w+1))=0\}|\\
    &+|\{w\in\{0,1\}^{t-1}0|f(uw)=f(u(w+1))=1\}|\\
= &2^{t-1},
\end{split}
\end{equation}
then according to the definition of $\Delta(u)$, for all $u \in \{0,1\}^{n-t}$, we have
\begin{equation}
\begin{split}
-2^{t-1}\leq\Delta(u)\leq 2^{t-1}.
\end{split}
\end{equation}

\fi


According to the definition of $\Delta(u)$,  for all $u \in \{0,1\}^{n-t}$, we have
\begin{equation}\label{Delta1}
\begin{split}
 &\quad\Delta(u)\\
 &=|\{w\in\{0,1\}^{t-1}0|f(uw)=f(u(w+1))=0\}|\\
  &\quad-|\{w\in\{0,1\}^{t-1}0|f(uw)=f(u(w+1))=1\}|\\
  &=(2^{t-1}-|\{w\in\{0,1\}^{t-1}0|f(uw)=f(u(w+1))=1\}|\\
  &\quad-|\{w\in\{0,1\}^{t-1}0|f(uw)\oplus f(u(w+1))=1\}|)\\
  &\quad- |\{w\in\{0,1\}^{t-1}0|f(uw)=f(u(w+1))=1\}|\\
  &=2^{t-1}-|\{w\in\{0,1\}^{t-1}0|f(uw)\oplus f(u(w+1))=1\}|\\
  &\quad-2|\{w\in\{0,1\}^{t-1}0|f(uw)=f(u(w+1))=1\}|\\
  &=2^{t-1}-\sum_{w\in\{0,1\}^{t-1}0}f(uw)\oplus f(u(w+1))\\
  &\quad-2\sum_{w\in\{0,1\}^{t-1}0}f(uw)\land f(u(w+1)).
\end{split}
\end{equation}

According to the definition of $\Delta(u)$, we also have
\begin{equation}\label{Delta2}
\begin{split}
 &\quad\Delta(u)\\
 &=|\{w\in\{0,1\}^{t-1}0|f(uw)=f(u(w+1))=0\}|\\
  &\quad-|\{w\in\{0,1\}^{t-1}0|f(uw)=f(u(w+1))=1\}|\\
  &=|\{w\in\{0,1\}^{t-1}0|f(uw)=f(u(w+1))=0\}|\\
  &\quad-(2^{t-1}-|\{w\in\{0,1\}^{t-1}0|f(uw)=f(u(w+1))=0\}|\\
  &\quad-|\{w\in\{0,1\}^{t-1}0|f(uw)\oplus f(u(w+1))=1\}|)\\
  &=-2^{t-1}+|\{w\in\{0,1\}^{t-1}0|f(uw)\oplus f(u(w+1))=1\}|\\
  &\quad+2|\{w\in\{0,1\}^{t-1}0|f(uw)=f(u(w+1))=0\}|\\
  &=-2^{t-1}+\sum_{w\in\{0,1\}^{t-1}0}f(uw)\oplus f(u(w+1))\\
  &\quad+2\sum_{w\in\{0,1\}^{t-1}0}(\lnot f(uw))\land(\lnot f(u(w+1))).
\end{split}
\end{equation}
%Notice that there could be multiple identical elements in $G(u)$. 
%An example of $\delta(u)$ is shown in Appendix \ref{example}.



The following theorems concerning $\Delta(u)$ are useful and important.

\begin{theorem}\label{The6} Suppose function $f:\{0,1\}^n \rightarrow \{0,1\}$, satisfies that it is either constant or balanced. If $\exists$ $u\in\{0,1\}^{n-t}$ such that $|\Delta(u)|\neq 2^{t-1}$, then $f$ is balanced. %a string $s \in \{0,1\}^n$  with $s\neq 0^n$, such that $f(x) = f(y)$ if and only if $x = y$ or $x \oplus y = s$. 
%Then
%  $\forall u,v \in \{0,1\}^{n-t},S(u)=S(v)$ if and only if $u \oplus v = 0^{n-t}$ or $u \oplus v = s_1$, where $s=s_1s_2$.
\end{theorem}

\begin{proof}
Suppose $\exists$ $u\in\{0,1\}^{n-t}$ such that $|\Delta(u)|\neq 2^{t-1}$, then $f$ is constant. 
If $f$ is  constant, then $\forall u\in\{0,1\}^{n-t}$, $\sum_{w\in\{0,1\}^t}f(uw)=0$ or $\sum_{w\in\{0,1\}^t}f(uw)=2^t$. According to  equation $(\ref{Delta1})$ or $(\ref{Delta2})$, then we have $|\Delta(u)|= 2^{t-1}$, which is contrary to the assumption $|\Delta(u)|\neq 2^{t-1}$. Therefore, if $\exists$ $u\in\{0,1\}^{n-t}$ such that $|\Delta(u)|\neq 2^{t-1}$, then $f$ is balanced.
\end{proof}



\begin{defi}
Suppose function $f:\{0,1\}^n \rightarrow \{0,1\}$, let $G_0=\sum\nolimits_{w\in\{0,1\}^{t-1}0}|\{u\in\{0,1\}^{n-t}|f(uw)\oplus f(u(w+1))=0, f(uw)=0\}|$, $G_1=\sum\nolimits_{w\in\{0,1\}^{t-1}0}|\{u\in\{0,1\}^{n-t}|f(uw)\oplus f(u(w+1))=0, f(uw)=1\}|$.
\end{defi}


\begin{defi}
Suppose function $f:\{0,1\}^n \rightarrow \{0,1\}$, let $D=\sum\nolimits_{w\in\{0,1\}^{t-1}0}|\{u\in\{0,1\}^{n-t}|f(uw)\oplus f(u(w+1))=0\}|$ $(0\leq D\leq 2^{n-1})$. 
\end{defi}


\begin{lemma}\label{Lem1} Suppose function $f:\{0,1\}^n \rightarrow \{0,1\}$, satisfies that it is either constant or balanced, it is divided into $2^t$ subfunctions $f_w$ $(\forall u \in \{0,1\}^{n-t}, w \in \{0,1\}^{t}, f_w(u)=f(uw))$ . Then $f$ is balanced if and only if $G_0=G_1=D/2$. %Then $f$ is constant  if and only if for all $w\in\{0,1\}^t$, $|\{u\in\{0,1\}^{n-t}|f(uw)=0\}|=2^{n-t}$ or $|\{u\in\{0,1\}^{n-t}|f(uw)=1\}|=2^{n-t}$.
\end{lemma}

\begin{proof}
%First, we prove that $f$ is balanced if and only if $G_0=G_1=D/2$.

(\romannumeral1) $\Longleftarrow$. Suppose $G_0=G_1=D/2$ $(0\leq D\leq 2^{n-1})$, then $f$ is constant. If $f$ is constant, then $f\equiv 0$ or $f\equiv 1$, that is $\sum\nolimits_{w\in\{0,1\}^{t-1}0}|\{u\in\{0,1\}^{n-t}|f(uw)\oplus f(u(w+1))=0, f(uw)=0\}|=2^{n-1}$ or $\sum\nolimits_{w\in\{0,1\}^{t-1}0}|\{u\in\{0,1\}^{n-t}|f(uw)\oplus f(u(w+1))=0, f(uw)=1\}|=2^{n-1}$. So $G_0=2^{n-1}$ or $G_1=2^{n-1}$, which is contrary to the assumption $G_0=G_1=D/2$ $(0\leq D\leq 2^{n-1})$. Therefore, if $G_0=G_1=D/2$, then $f$ is balanced, where $D=\sum\nolimits_{w\in\{0,1\}^{t-1}0}|\{u\in\{0,1\}^{n-t}|f(uw)\oplus f(u(w+1))=0\}|$.




(\romannumeral2) $\Longrightarrow$. %If $f$ is balanced, according to the definition of DJ problem, then   $|\{x\in\{0,1\}^n|f(x)=0\}|=|\{x\in\{0,1\}^n|f(x)=1\}|=2^{n-1}$. 
Since 
\begin{equation}
\begin{split}
&D\\
=&\sum\limits_{w\in\{0,1\}^{t-1}0}|\{u\in\{0,1\}^{n-t}|f(uw)\oplus f(u(w+1))=0\}|\\=&\sum\limits_{w\in\{0,1\}^{t-1}0}|\{u\in\{0,1\}^{n-t}|f(uw)= f(u(w+1))=0\}|\\
&+\sum\limits_{w\in\{0,1\}^{t-1}0}|\{u\in\{0,1\}^{n-t}|f(uw)=f(u(w+1))\\
&\qquad\qquad\qquad=1\}|, 
\end{split}
\end{equation}
therefore
\begin{equation}
\begin{split}
&\sum\limits_{w\in\{0,1\}^{t-1}0}|\{u\in\{0,1\}^{n-t}|f(uw)\oplus f(u(w+1))=1\}|\\
=&2^{n-1}-D, 
\end{split}
\end{equation}
that is 
\begin{equation}
\begin{split}
&\sum\limits_{w\in\{0,1\}^{t-1}0}|\{u\in\{0,1\}^{n-t}|f(uw)=0, \\
&\qquad\qquad\quad f(u(w+1))=1\}|\\
&+\sum\limits_{w\in\{0,1\}^{t-1}0}|\{u\in\{0,1\}^{n-t}|f(uw)=1, \\
&\qquad\qquad\qquad f(u(w+1))=0\}|\\
=&2^{n-1}-D. 
\end{split}
\end{equation}

We have 
\begin{equation}
\begin{split}
&\sum\limits_{w\in\{0,1\}^{t-1}0}|\{u\in\{0,1\}^{n-t}|f(uw)=f(u(w+1))=0\}|\\
=&\sum\limits_{w\in\{0,1\}^{t-1}0}|\{u\in\{0,1\}^{n-t}|f(uw)=0\}|\\
&-\sum\limits_{w\in\{0,1\}^{t-1}0}|\{u\in\{0,1\}^{n-t}|f(uw)=0,\\
&\qquad\qquad\qquad f(u(w+1))=1\}|\\
=&\sum\limits_{w\in\{0,1\}^{t-1}0}|\{u\in\{0,1\}^{n-t}|f(u(w+1))=0\}|\\
&-\sum\limits_{w\in\{0,1\}^{t-1}0}|\{u\in\{0,1\}^{n-t}|f(uw)=1,\\
&\qquad\qquad\qquad f(u(w+1))=0\}|.
\end{split}
\end{equation}


Let 
\begin{equation}
\begin{split}
T_0=&\sum\limits_{w\in\{0,1\}^{t-1}0}|\{u\in\{0,1\}^{n-t}|f(uw)=0\}|\\
&-\sum\limits_{w\in\{0,1\}^{t-1}0}|\{u\in\{0,1\}^{n-t}|f(uw)=0,\\
&\qquad\qquad\qquad f(u(w+1))=1\}|.
\end{split}
\end{equation}
\begin{equation}
\begin{split}
T_1=&\sum\limits_{w\in\{0,1\}^{t-1}0}|\{u\in\{0,1\}^{n-t}|f(u(w+1))=0\}|\\
&-\sum\limits_{w\in\{0,1\}^{t-1}0}|\{u\in\{0,1\}^{n-t}|f(uw)=1,\\
&\qquad\qquad\qquad f(u(w+1))=0\}|.
\end{split}
\end{equation}

Then
\begin{equation}
\begin{split}
T_0=&T_1\\
=&\sum\limits_{w\in\{0,1\}^{t-1}0}|\{u\in\{0,1\}^{n-t}|f(uw)=\\
&\qquad\qquad\quad f(u(w+1))=0\}|.
\end{split}
\end{equation}

Then 
\begin{equation}
\begin{split}
&T_0+T_1\\
=&\sum\limits_{w\in\{0,1\}^{t-1}0}|\{u\in\{0,1\}^{n-t}|f(uw)=0\}|\\
&-\sum\limits_{w\in\{0,1\}^{t-1}0}|\{u\in\{0,1\}^{n-t}|f(uw)=0,\\
&\qquad\qquad\qquad f(u(w+1))=1\}|\\
&+\sum\limits_{w\in\{0,1\}^{t-1}0}|\{u\in\{0,1\}^{n-t}|f(u(w+1))=0\}|\\
&-\sum\limits_{w\in\{0,1\}^{t-1}0}|\{u\in\{0,1\}^{n-t}|f(uw)=1,\\
&\qquad\qquad\qquad f(u(w+1))=0\}|.\\
=&(\sum\limits_{w\in\{0,1\}^{t-1}0}|\{u\in\{0,1\}^{n-t}|f(uw)=0\}|\\
&+\sum\limits_{w\in\{0,1\}^{t-1}0}|\{u\in\{0,1\}^{n-t}|f(u(w+1))=0\}|)\\
&-(\sum\limits_{w\in\{0,1\}^{t-1}0}|\{u\in\{0,1\}^{n-t}|f(uw)=0,\\
&\qquad\qquad\qquad f(u(w+1))=1\}|\\
&+\sum\limits_{w\in\{0,1\}^{t-1}0}|\{u\in\{0,1\}^{n-t}|f(uw)=1,\\
&\qquad\qquad\qquad f(u(w+1))=0\}|)\\
=&(\sum\limits_{w\in\{0,1\}^{t-1}0}|\{u\in\{0,1\}^{n-t}|f(uw)=0\}|\\
&+\sum\limits_{w\in\{0,1\}^{t-1}0}|\{u\in\{0,1\}^{n-t}|f(u(w+1))=0\}|)\\
&-(2^{n-1}-D). 
\end{split}
\end{equation}

If $f$ is balanced, according to the definition of DJ problem, then we have $|\{x\in\{0,1\}^n|f(x)=0\}|=|\{x\in\{0,1\}^n|f(x)=1\}|=2^{n-1}$. 

So 
\begin{equation}
\begin{split}
&\sum\limits_{w\in\{0,1\}^{t-1}0}|\{u\in\{0,1\}^{n-t}|f(uw)=0\}|\\
&+\sum\limits_{w\in\{0,1\}^{t-1}0}|\{u\in\{0,1\}^{n-t}|f(u(w+1))=0\}|\\
=&|\{x\in\{0,1\}^n|f(x)=0\}|\\
=&2^{n-1}.
\end{split}
\end{equation}

Therefore, we have
\begin{equation}
\begin{split}
&T_0+T_1\\
=&(\sum\limits_{w\in\{0,1\}^{t-1}0}|\{u\in\{0,1\}^{n-t}|f(uw)=0\}|\\
&+\sum\limits_{w\in\{0,1\}^{t-1}0}|\{u\in\{0,1\}^{n-t}|f(u(w+1))=0\}|)\\
&-(2^{n-1}-D)\\
=&2^{n-1}-(2^{n-1}-D)\\
=&D, 
\end{split}
\end{equation}
then from $T_0=T_1=\sum\limits_{w\in\{0,1\}^{t-1}0}|\{u\in\{0,1\}^{n-t}|f(uw)=f(u(w+1))=0\}|$, we have $T_0=T_1=\sum\limits_{w\in\{0,1\}^{t-1}0}|\{u\in\{0,1\}^{n-t}|f(uw)=f(u(w+1))=0\}|=D/2$. 


Further, from $D=\sum\limits_{w\in\{0,1\}^{t-1}0}|\{u\in\{0,1\}^{n-t}|f(uw)= f(u(w+1))=0\}|+\sum\limits_{w\in\{0,1\}^{t-1}0}|\{u\in\{0,1\}^{n-t}|f(uw)=f(u(w+1))=1\}|$, we have 
\begin{equation}
\begin{split}
&\sum\limits_{w\in\{0,1\}^{t-1}0}|\{u\in\{0,1\}^{n-t}|f(uw)=f(u(w+1))=1\}|\\
=&D-\sum\limits_{w\in\{0,1\}^{t-1}0}|\{u\in\{0,1\}^{n-t}|f(uw)=f(u(w+1))\\
&\qquad\qquad\qquad\quad=0\}|\\
=&D-D/2\\
=&D/2.
\end{split}
\end{equation}



Since 
\begin{equation}
\begin{split}
&G_0\\
=&\sum\limits_{w\in\{0,1\}^{t-1}0}|\{u\in\{0,1\}^{n-t}|f(uw)\oplus f(u(w+1))=0,\\ &\qquad\qquad\quad f(uw)=0\}|\\
=&\sum\limits_{w\in\{0,1\}^{t-1}0}|\{u\in\{0,1\}^{n-t}|f(uw)=f(u(w+1))=0\}|.
\end{split}
\end{equation}
\begin{equation}
\begin{split}
&G_1\\
=&\sum\limits_{w\in\{0,1\}^{t-1}0}|\{u\in\{0,1\}^{n-t}|f(uw)\oplus f(u(w+1))=0,\\ &\qquad\qquad\quad f(uw)=1\}|\\
=&\sum\limits_{w\in\{0,1\}^{t-1}0}|\{u\in\{0,1\}^{n-t}|f(uw)=f(u(w+1))=1\}|.
\end{split}
\end{equation}
Therefore, we have $G_0=G_1=D/2$.

%Let $S_0=|\{u\in\{0,1\}^{n-1}|f(u0)=0\}|-|\{u\in\{0,1\}^{n-1}|f(u0)=0,f(u1)=1\}|$, $S_1=|\{u\in\{0,1\}^{n-1}|f(u1)=0\}|-|\{u\in\{0,1\}^{n-1}|f(u0)=1,f(u1)=0\}|$. 

%Then $S_0+S_1=(|\{u\in\{0,1\}^{n-1}|f(u0)=0\}|-|\{u\in\{0,1\}^{n-1}|f(u0)=0,f(u1)=1\}|)+(|\{u\in\{0,1\}^{n-1}|f(u1)=0\}|-|\{u\in\{0,1\}^{n-1}|f(u0)=1,f(u1)=0\}|)=(|\{u\in\{0,1\}^{n-1}|f(u0)=0\}|+|\{u\in\{0,1\}^{n-1}|f(u1)=0\}|)-(|\{u\in\{0,1\}^{n-1}|f(u0)=0,f(u1)=1\}|+|\{u\in\{0,1\}^{n-1}|f(u0)=1,f(u1)=0\}|)=(|\{u\in\{0,1\}^{n-1}|f(u0)=0\}|+|\{u\in\{0,1\}^{n-1}|f(u1)=0\}|)-(2^{n-1}-M)$. 




%If $f$ is balanced, according to the definition of DJ problem, then we have $|\{x\in\{0,1\}^n|f(x)=0\}|=|\{x\in\{0,1\}^n|f(x)=1\}|=2^{n-1}$. So $|\{u\in\{0,1\}^{n-1}|f(u0)=0\}|+|\{u\in\{0,1\}^{n-1}|f(u1)=0\}|=|\{x\in\{0,1\}^n|f(x)=0\}|=2^{n-1}$. Therefore, $S_0+S_1=(|\{u\in\{0,1\}^{n-1}|f(u0)=0\}|+|\{u\in\{0,1\}^{n-1}|f(u1)=0\}|)-(2^{n-1}-M)=2^{n-1}-(2^{n-1}-M)=M$. Then from $S_0=S_1=|\{u\in\{0,1\}^{n-1}|f(u0)= f(u1)=0\}|$, we have $S_0=S_1=M/2$. 

%Further, from $S_0=|\{u\in\{0,1\}^{n-1}|f(u0)=0\}|-|\{u\in\{0,1\}^{n-1}|f(u0)=0,f(u1)=1\}|=B_0$, $S_1=|\{u\in\{0,1\}^{n-1}|f(u1)=0\}|-|\{u\in\{0,1\}^{n-1}|f(u0)=1,f(u1)=0\}|=B_1$, we have $B_0=B_1=M/2$.



% According to equation $(\ref{delta1})$, then we have $\sum\nolimits_{u\in\{0,1\}^{n-t}} \delta(u)=\sum\nolimits_{u\in\{0,1\}^{n-t}} (2^t-2\sum_{w\in\{0,1\}^t}f(uw))=\sum\nolimits_{u\in\{0,1\}^{n-t}}2^t-2\sum\nolimits_{u,w\in\{0,1\}^{n-t}}f(uw)=2^n-2\sum\nolimits_{x\in\{0,1\}^{n}}f(x)=2^n-2|\{x\in\{0,1\}^n|f(x)=1\}|=0$.

%Second, we prove that $f$ is constant if and only if $|\{u\in\{0,1\}^{n-1}|f(u0)=0\}|=|\{u\in\{0,1\}^{n-1}|f(u1)=0\}|=2^{n-1}$ or $|\{u\in\{0,1\}^{n-1}|f(u0)=1\}|=|\{u\in\{0,1\}^{n-1}|f(u1)=1\}|=2^{n-1}$. 

%(3) $\Longleftarrow$. If $|\{u\in\{0,1\}^{n-1}|f(u0)=0\}|=|\{u\in\{0,1\}^{n-1}|f(u1)=0\}|=2^{n-1}$ or $|\{u\in\{0,1\}^{n-1}|f(u0)=1\}|=|\{u\in\{0,1\}^{n-1}|f(u1)=1\}|=2^{n-1}$, then $|\{x\in\{0,1\}^n|f(x)=1\}|=0$ or $2^n$, that is $f(x) \equiv 0$ or $f(x) \equiv 1$. Therefore, $f$ is constant.

%(4) $\Longrightarrow$. If $f$ is constant, according to the definition of DJ problem, then  $f(x) \equiv 0$ or $f(x) \equiv 1$. So $|\{x\in\{0,1\}^n|f(x)=1\}|=0$ or $2^n$. Therefore $|\{u\in\{0,1\}^{n-1}|f(u0)=0\}|=|\{u\in\{0,1\}^{n-1}|f(u1)=0\}|=2^{n-1}$ or $|\{u\in\{0,1\}^{n-1}|f(u0)=1\}|=|\{u\in\{0,1\}^{n-1}|f(u1)=1\}|=2^{n-1}$.

\end{proof}


\begin{theorem}\label{The7} Suppose function $f:\{0,1\}^n \rightarrow \{0,1\}$, satisfies that it is either constant or balanced, it is divided into $2^t$ subfunctions $f_w$ $(\forall u \in \{0,1\}^{n-t}, w \in \{0,1\}^{t}, f_w(u)=f(uw))$. %Then $f$ is balanced if and only if $G_0=G_1=D/2$. 
Then $f$ is balanced  if and only if $\sum_{u\in\{0,1\}^{n-t}}\Delta(u)=0$.
\end{theorem}

\begin{proof}
Since
\begin{equation}
\begin{split}
&\sum_{u\in\{0,1\}^{n-t}}\Delta(u)\\
 =&\sum_{u\in\{0,1\}^{n-t}}(|\{w\in\{0,1\}^{t-1}0|f(uw)=f(u(w+1))=0\}|\\
&-|\{w\in\{0,1\}^{t-1}0|f(uw)=f(u(w+1))=1\}|)\\
=&\sum_{u\in\{0,1\}^{n-t}}|\{w\in\{0,1\}^{t-1}0|f(uw)=f(u(w+1))=0\}|\\
&-\sum_{u\in\{0,1\}^{n-t}}|\{w\in\{0,1\}^{t-1}0|f(uw)=f(u(w+1))=\\
&\qquad\qquad\qquad1\}|\\
=&\sum_{w\in\{0,1\}^{t-1}0}|\{u\in\{0,1\}^{n-t}|f(uw)=f(u(w+1))=0\}|\\
&-\sum_{w\in\{0,1\}^{t-1}0}|\{u\in\{0,1\}^{n-t}|f(uw)=f(u(w+1))=\\
&\qquad\qquad\qquad1\}|\\
&=G_0-G_1,
\end{split}
\end{equation}
 therefore, according to Lemma \ref{Lem1}, we have $f$ is balanced  if and only if $\sum_{u\in\{0,1\}^{n-t}}\Delta(u)=0$.

\end{proof}



\begin{theorem}\label{The8} Suppose function $f:\{0,1\}^n \rightarrow \{0,1\}$, satisfies that it is either constant or balanced, it is divided into $2^t$ subfunctions $f_w$ $(\forall u \in \{0,1\}^{n-t}, w \in \{0,1\}^{t}, f_w(u)=f(uw))$. %Then $f$ is balanced if and only if $G_0=G_1=D/2$. 
Then $f$ is constant  if and only if for all $u \in \{0,1\}^{n-t}$, $\Delta(u) = 2^{t-1}$ or $u \in \{0,1\}^{n-t}$, $\Delta(u) = -2^{t-1}$.
\end{theorem}

\begin{proof}

(\romannumeral1) $\Longleftarrow$. If  for all $u \in \{0,1\}^{n-t}$, $\Delta(u) =2^{t-1}$, then according to equation (\ref{Delta1}), for all $u \in \{0,1\}^{n-t}$, we have $\sum_{w\in\{0,1\}^{t-1}0}f(uw)\oplus f(u(w+1))=0$, $\sum_{w\in\{0,1\}^{t-1}0}f(uw)\land f(u(w+1))=0$. So for all $x \in \{0,1\}^{n}$, $f(x)=0$, that is $f(x) \equiv 0$.



 If  for all $u \in \{0,1\}^{n-t}$, $\Delta(u) =-2^{t-1}$, then according to equation (\ref{Delta2}), for all $u \in \{0,1\}^{n-t}$, we have $\sum_{w\in\{0,1\}^{t-1}0}f(uw)\oplus f(u(w+1))=0$, $\sum_{w\in\{0,1\}^{t-1}0}(\lnot f(uw))\land (\lnot f(u(w+1)))=0$. So for all $x \in \{0,1\}^{n}$, $f(x)=1$, that is $f(x) \equiv 1$.
 
 Therefore, if for all $u \in \{0,1\}^{n-t}$, $\Delta(u) =2^{t-1}$ or for all $u \in \{0,1\}^{n-t}$, $\Delta(u) =-2^{t-1}$, then $f$ is constant.

(\romannumeral2) $\Longrightarrow$. If $f$ is constant, according to the definition of DJ problem, then we have $f(x) \equiv 0$ or $f(x) \equiv 1$. 


If $f(x) \equiv 0$, then for all $x \in \{0,1\}^{n}$, $f(x)=0$. So for all $u \in \{0,1\}^{n-t}$, we have $\sum_{w\in\{0,1\}^{t-1}0}f(uw)\oplus f(u(w+1))=0$, $\sum_{w\in\{0,1\}^{t-1}0}f(uw)\land f(u(w+1))=0$.  According to equation (\ref{Delta1}), for all $u \in \{0,1\}^{n-t}$, we have $\Delta(u) =2^{t-1}$.


If $f(x) \equiv 1$, then for all $x \in \{0,1\}^{n}$, $f(x)=1$. So for all $u \in \{0,1\}^{n-t}$, we have $\sum_{w\in\{0,1\}^{t-1}0}f(uw)\oplus f(u(w+1))=0$, $\sum_{w\in\{0,1\}^{t-1}0}(\lnot f(uw))\land (\lnot f(u(w+1)))=0$.  According to equation (\ref{Delta2}), for all $u \in \{0,1\}^{n-t}$, we have $\Delta(u) =-2^{t-1}$.

 Therefore, if $f$ is constant, then for all $u \in \{0,1\}^{n-t}$, $\Delta(u) =2^{t-1}$ or for all $u \in \{0,1\}^{n-t}$, $\Delta(u) =-2^{t-1}$.
 

(\romannumeral1) $\Longleftarrow$. If $\sum\nolimits_{u\in\{0,1\}^{n-t}} \Delta(u) =2^{n-1}$, according to equation $(\ref{Delta1})$, then we have 
\begin{equation}
\begin{split}
&\sum\nolimits_{u\in\{0,1\}^{n-t}} \Delta(u)\\
=&\sum\nolimits_{u\in\{0,1\}^{n-t}} (2^{t-1}-\sum_{w\in\{0,1\}^{t-1}0}f(uw)\oplus f(u(w+1))\\
  &-2\sum_{w\in\{0,1\}^{t-1}0}f(uw)\land f(u(w+1)))\\
=&2^{n-1},
\end{split}
\end{equation}
and since for all $u \in \{0,1\}^{n-t}$, $-2^{t-1}\leq\Delta(u)\leq 2^{t-1}$, therefore for all $u \in \{0,1\}^{n-t}$, $\Delta(u)=2^{t-1}$.

And since  for all $u \in \{0,1\}^{n-t}$, we have
\begin{equation}
\begin{split}
&\Delta(u)\\
=&2^{t-1}-\sum_{w\in\{0,1\}^{t-1}0}f(uw)\oplus f(u(w+1))\\
  &-2\sum_{w\in\{0,1\}^{t-1}0}f(uw)\land f(u(w+1)),\\
\end{split}
\end{equation}
therefor  for all $u \in \{0,1\}^{n-t}$, we have
\begin{equation}
\begin{split}
\sum_{w\in\{0,1\}^{t-1}0}f(uw)\oplus f(u(w+1))=0.
\end{split}
\end{equation}
\begin{equation}
\begin{split}
\sum_{w\in\{0,1\}^{t-1}0}f(uw)\land f(u(w+1))=0.
\end{split}
\end{equation}

Therefore, for all $u \in \{0,1\}^{n-t}$, we have
\begin{equation}
\begin{split}
&|\{w\in\{0,1\}^{t-1}0|f(uw)=f(u(w+1))=0\}|\\
=&2^{t-1}-(|\{w\in\{0,1\}^{t-1}0|f(uw)\oplus f(u(w+1))=1\}|\\
&+|\{w\in\{0,1\}^{t-1}0|f(uw)=f(u(w+1))=1\}|)\\
=&2^{t-1}-(\sum_{w\in\{0,1\}^{t-1}0}f(uw)\oplus f(u(w+1))\\
&+\sum_{w\in\{0,1\}^{t-1}0}f(uw)\land f(u(w+1)))\\
=&2^{t-1}.
\end{split}
\end{equation}

So $|\{x\in\{0,1\}^n|f(x)=1\}|=0$, that is $f(x) \equiv 0$.

If $\sum\nolimits_{u\in\{0,1\}^{n-t}} \Delta(u) =-2^{n-1}$, according to equation $(\ref{Delta2})$, then we have 
\begin{equation}
\begin{split}
&\sum\nolimits_{u\in\{0,1\}^{n-t}} \Delta(u)\\
  =&\sum\nolimits_{u\in\{0,1\}^{n-t}}(-2^{t-1}+\sum_{w\in\{0,1\}^{t-1}0}f(uw)\oplus f(u(w+1))\\
  &+2\sum_{w\in\{0,1\}^{t-1}0}(\lnot f(uw))\land(\lnot f(u(w+1))))\\
=&-2^{n-1},
\end{split}
\end{equation}
and since for all $u \in \{0,1\}^{n-t}$, $-2^{t-1}\leq\Delta(u)\leq 2^{t-1}$, therefore for all $u \in \{0,1\}^{n-t}$, $\Delta(u)=-2^{t-1}$.

And since  for all $u \in \{0,1\}^{n-t}$, we have
\begin{equation}
\begin{split}
&\Delta(u)\\
=&-2^{t-1}+\sum_{w\in\{0,1\}^{t-1}0}f(uw)\oplus f(u(w+1))\\
  &+2\sum_{w\in\{0,1\}^{t-1}0}(\lnot f(uw))\land(\lnot f(u(w+1))),\\
\end{split}
\end{equation}
therefor  for all $u \in \{0,1\}^{n-t}$, we have
\begin{equation}
\begin{split}
\sum_{w\in\{0,1\}^{t-1}0}f(uw)\oplus f(u(w+1))=0.
\end{split}
\end{equation}
\begin{equation}
\begin{split}
\sum_{w\in\{0,1\}^{t-1}0}(\lnot f(uw))\land(\lnot f(u(w+1))=0.
\end{split}
\end{equation}

Therefore, for all $u \in \{0,1\}^{n-t}$, we have
\begin{equation}
\begin{split}
&|\{w\in\{0,1\}^{t-1}0|f(uw)=f(u(w+1))=1\}|\\
=&2^{t-1}-(|\{w\in\{0,1\}^{t-1}0|f(uw)\oplus f(u(w+1))=1\}|\\
&+|\{w\in\{0,1\}^{t-1}0|f(uw)=f(u(w+1))=0\}|)\\
=&2^{t-1}-(\sum_{w\in\{0,1\}^{t-1}0}f(uw)\oplus f(u(w+1))\\
&+\sum_{w\in\{0,1\}^{t-1}0}(\lnot f(uw))\land(\lnot f(u(w+1)))\\
=&2^{t-1}.
\end{split}
\end{equation}

So $|\{x\in\{0,1\}^n|f(x)=1\}|=2^n$, that is $f(x) \equiv 1$.

Therefore, if $\sum\nolimits_{u\in\{0,1\}^{n-t}} \Delta(u) =\pm 2^{n-1}$, then $f$ is constant.

(\romannumeral2) $\Longrightarrow$. If $f$ is constant, according to the definition of DJ problem, then we have $f(x) \equiv 0$ or $f(x) \equiv 1$. 

If $f(x) \equiv 0$, then $|\{x\in\{0,1\}^n|f(x)=0\}|=2^n$. Therefore, for all $u \in \{0,1\}^{n-t}$, we have $|\{w\in\{0,1\}^{t-1}0|f(uw)=f(u(w+1))=0\}|=2^{t-1}$ and $|\{w\in\{0,1\}^{t-1}0|f(uw)=f(u(w+1))=1\}|=0$.


So according to the definition of $\Delta(u)$,we have
\begin{equation}
\begin{split}
&\sum\limits_{u\in\{0,1\}^{n-t}} \Delta(u)\\
=&\sum\limits_{u\in\{0,1\}^{n-t}} (|\{w\in\{0,1\}^{t-1}0|f(uw)=f(u(w+1))=0\}|\\
&\quad-|\{w\in\{0,1\}^{t-1}0|f(uw)=f(u(w+1))=1\}|)\\
=&\sum\limits_{u\in\{0,1\}^{n-t}}2^{t-1}\\
=&2^{n-1}.
\end{split}
\end{equation}



If $f(x) \equiv 1$, then $|\{x\in\{0,1\}^n|f(x)=1\}|=2^n$. Therefore, for all $u \in \{0,1\}^{n-t}$, we have $|\{w\in\{0,1\}^{t-1}0|f(uw)=f(u(w+1))=0\}|=0$ and $|\{w\in\{0,1\}^{t-1}0|f(uw)=f(u(w+1))=1\}|=2^{t-1}$.


So according to the definition of $\Delta(u)$,we have
\begin{equation}
\begin{split}
&\sum\limits_{u\in\{0,1\}^{n-t}} \Delta(u)\\
=&\sum\limits_{u\in\{0,1\}^{n-t}} (|\{w\in\{0,1\}^{t-1}0|f(uw)=f(u(w+1))=0\}|\\
&\quad-|\{w\in\{0,1\}^{t-1}0|f(uw)=f(u(w+1))=1\}|)\\
=&\sum\limits_{u\in\{0,1\}^{n-t}}-2^{t-1}\\
=&-2^{n-1}.
\end{split}
\end{equation}

Therefore, if $f$ is constant, then $\sum\nolimits_{u\in\{0,1\}^{n-t}} \Delta(u) =\pm 2^{n-1}$.



\end{proof}


\textit{Correctness analysis of algorithms}.---The exact correctness analysis of our  algorithms are presented in Ref. \cite{SM}.
In the following, we  give an informal description of the correctness proof of Algorithm \ref{algorithm2} . The state after the third step of  Algorithm \ref{algorithm2}   in FIG. \ref{algorithm_two_nodes (I)} is shown below.%we write out the state after the third step of the algorithm in FIG. \ref{algorithm_two_nodes (I)}.
%\begin{align}
%  |\psi_1\rangle&=\frac{1}{\sqrt{2^{n-1}}}\sum_{u\in\{0,1\}^{n-1}}|u\rangle\ket{0}\ket{0}|0^{4}\rangle.
%\end{align}
%Then each computing node queries its own oracle under the control of the first quantum register:
\begin{align}
  |\psi_3\rangle=\frac{1}{\sqrt{2^{n-1}}}\sum_{u\in\{0,1\}^{n-1}}|u\rangle|f(u0)\rangle|f(u1)\rangle|0^{4}\rangle
\end{align}

After the action of the operator $U$, we have the following state:
\begin{equation}
\begin{split}
  |\psi_4\rangle&=\frac{1}{\sqrt{2^{n-1}}}\sum_{u\in\{0,1\}^{n-1}}|u\rangle|f(u0)\rangle|f(u1)\rangle\\
  &\qquad\qquad\qquad\qquad|\left\lfloor\frac{1-{\rm sgn}(\delta(u))}{2}\right\rfloor\rangle||\delta(u)|\rangle\ket{0},
\end{split}
\end{equation}
where $\delta(u)=2-2(f(u0)+f(u1))$.

After the action of the operator $Z$ and $R$, we have the following state:
\begin{equation}
\begin{split}
  &|\psi_5\rangle=\frac{1}{\sqrt{2^{n-1}}}\sum_{u\in\{0,1\}^{n-1}}(-1)^{\left\lfloor\frac{1-{\rm sgn}(\delta(u))}{2}\right\rfloor}|u\rangle|f(u0)\rangle|f(u1)\rangle\\
  &\qquad\qquad\qquad\qquad\qquad|\left\lfloor\frac{1-{\rm sgn}(\delta(u))}{2}\right\rfloor\rangle||\delta(u)|\rangle\\
 &\qquad\qquad\qquad\qquad\qquad(\frac{|\delta(u)|}{2}\ket{0}+\sqrt{1-(\frac{|\delta(u)|}{2})^2}\ket{1}).
\end{split}
\end{equation}

Uncomputing the quantum registers in the middle part. Then we obtain the following state:
\begin{equation}
\begin{split}
|\psi_8\rangle=&\frac{1}{\sqrt{2^{n-1}}}\sum_{u\in\{0,1\}^{n-1}}(-1)^{\left\lfloor\frac{1-{\rm sgn}(\delta(u))}{2}\right\rfloor}|u\rangle\ket{0}\ket{0}|0^{3}\rangle\\
&\qquad\qquad\qquad\qquad(\frac{|\delta(u)|}{2}\ket{0}+\sqrt{1-(\frac{|\delta(u)|}{2})^2}\ket{1})
\end{split}
\end{equation}


After measurement on the last register, if the result is $1$, then  $f$ is balanced. If the measurement result is $0$, after Hadamard transformation on the first register and measurement on the first register, if the result is not $0^{n-1}$, then $f$ is  balanced, otherwise $f$ is constant.



In this section, we prove the correctness of  Algorithm \ref{algorithm3} , Algorithm \ref{algorithm5}  and Algorithm 5.  


First, we prove the correctness of  Algorithm \ref{algorithm3} , we write out the state after the first step of the algorithm in FIG. \ref{algorithm_multiple_nodes (I)}.

\begin{align}
  |\phi_1\rangle&=\frac{1}{\sqrt{2^{n-t}}}\sum_{u\in\{0,1\}^{n-t}}|u\rangle(\bigotimes_{w \in \{0,1\}^t}\ket{0})|0^{t+3}\rangle\\
  &=\frac{1}{\sqrt{2^{n-t}}}\sum_{u\in\{0,1\}^{n-t}}|u\rangle\underbrace{\ket{0} \ldots\ket{0}}_{2^t}|0^{t+3}\rangle.
\end{align}

Then the algorithm querie the oracle $O_{f_{0^t}}$, resulting in the following state:

\begin{align}
  |\phi_2\rangle&=\left(O_{f_{0^t}}\frac{1}{\sqrt{2^{n-t}}}\sum_{u\in\{0,1\}^{n-t}}|u\rangle\ket{0}\right)\underbrace{\ket{0} \ldots\ket{0}}_{2^t-1}|0^{t+3}\rangle\\
  &=\frac{1}{\sqrt{2^{n-t}}}\sum_{u\in\{0,1\}^{n-t}}|u\rangle|f_{0^t}(u)\rangle \underbrace{\ket{0} \ldots\ket{0}}_{2^t-1}|0^{t+3}\rangle\\
  &=\frac{1}{\sqrt{2^{n-t}}}\sum_{u\in\{0,1\}^{n-t}}|u\rangle|f(u0^t)\rangle \underbrace{\ket{0} \ldots\ket{0}}_{2^t-1}|0^{t+3}\rangle.
\end{align}

Then the algorithm queries each of the other oracles based on the circuit diagram FIG. \ref{algorithm_multiple_nodes (I)} to get the following states:

\begin{equation}
  |\phi_3\rangle=\frac{1}{\sqrt{2^{n-t}}}\sum_{u\in\{0,1\}^{n-t}}|u\rangle\underbrace{|f(u0^t)\rangle \ldots|f(u1^t)\rangle}_{2^t}|0^{t+3}\rangle.
\end{equation}

After the action of the operator $U$, we have the following state:

\begin{equation}
\begin{split}
  &|\phi_4\rangle=\frac{1}{\sqrt{2^{n-t}}}\sum_{u\in\{0,1\}^{n-t}}|u\rangle\underbrace{|f(u0^t)\rangle \ldots|f(u1^t)\rangle}_{2^t}\\
 &\qquad\qquad\quad\qquad\quad \otimes|\left\lfloor\frac{1-{\rm sgn}(\delta(u))}{2}\right\rfloor\rangle||\delta(u)|\rangle\ket{0}, \\ \text{where}\ &\delta(u)=2^t-2\sum_{w\in\{0,1\}^t}f(uw).
\end{split}
\end{equation}

After the action of the operator $Z$ and $R$, we have the following state:

\begin{equation}
\begin{split}
  &|\phi_5\rangle=\frac{1}{\sqrt{2^{n-t}}}\sum_{u\in\{0,1\}^{n-t}}(-1)^{\left\lfloor\frac{1-{\rm sgn}(\delta(u))}{2}\right\rfloor}|u\rangle\\
  &\otimes\underbrace{|f(u0^t)\rangle \ldots|f(u1^t)\rangle}_{2^t}|\left\lfloor\frac{1-{\rm sgn}(\delta(u))}{2}\right\rfloor\rangle||\delta(u)|\rangle\\
  &\otimes(\frac{|\delta(u)|}{2^t}\ket{0}+\sqrt{1-(\frac{|\delta(u)|}{2^t})^2}\ket{1}).
\end{split}
\end{equation}

After that, we use the $U$ operator again, and restore the status of the $t+2$ bit registers of to $U$ operator $\ket{0}$. Then we obtain the following state:

\begin{equation}
\begin{split}
  &|\phi_6\rangle=\frac{1}{\sqrt{2^{n-t}}}\sum_{u\in\{0,1\}^{n-t}}(-1)^{\left\lfloor\frac{1-{\rm sgn}(\delta(u))}{2}\right\rfloor}|u\rangle\\
  &\qquad\qquad\quad\quad\otimes\underbrace{|f(u0^t)\rangle \ldots|f(u1^t)\rangle}_{2^t}|0^{t+2}\rangle\\
  &\qquad\qquad\quad\quad\otimes(\frac{|\delta(u)|}{2^t}\ket{0}+\sqrt{1-(\frac{|\delta(u)|}{2^t})^2}\ket{1}).
\end{split}
\end{equation}


After that, we query each oracle again and restore the status of the $2^t$ bit registers to $\ket{0}$. Then we obtain the following state:

\begin{equation}
\begin{split}
&|\phi_8\rangle=\frac{1}{\sqrt{2^{n-t}}}\sum_{u\in\{0,1\}^{n-t}}(-1)^{\left\lfloor\frac{1-{\rm sgn}(\delta(u))}{2}\right\rfloor}|u\rangle|0^{2^t+t+2}\rangle\\
  &\qquad\qquad\qquad\qquad\otimes(\frac{|\delta(u)|}{2^t}\ket{0}+\sqrt{1-(\frac{|\delta(u)|}{2^t})^2}\ket{1}).
\end{split}
\end{equation}

By tracing out the states of the $2^t+t+2$ bit registers in the middle, we can get the state:

\begin{equation}
\begin{split}
&|\phi_8'\rangle=\frac{1}{\sqrt{2^{n-t}}}\sum_{u\in\{0,1\}^{n-t}}(-1)^{\left\lfloor\frac{1-{\rm sgn}(\delta(u))}{2}\right\rfloor}|u\rangle\\
  &\qquad\qquad\qquad\qquad\otimes(\frac{|\delta(u)|}{2^t}\ket{0}+\sqrt{1-(\frac{|\delta(u)|}{2^t})^2}\ket{1}).
\end{split}
\end{equation}

After measurement on the last register, if the result is $1$, then $\exists$ $u\in\{0,1\}^{n-t}$ such that $|\delta(u)|\neq 2^t$. From Theorem \ref{The1}, we know that $f$ is balanced. If the measurement result is $0$, we get the state:

\begin{equation}
\begin{split}
&|\phi_9'\rangle=\sum_{u\in\{0,1\}^{n-t}}\frac{(-1)^{\left\lfloor\frac{1-{\rm sgn}(\delta(u))}{2}\right\rfloor}|\delta(u)|}{\sqrt{\sum\limits_{u\in\{0,1\}^{n-t}}|\delta(u)|^2}}|u\rangle\ket{0}.
\end{split}
\end{equation}

By tracing out the state of the $1$ bit register in the last, we can get the state:

\begin{equation}
\begin{split}
|\phi_9''\rangle=&\sum_{u\in\{0,1\}^{n-t}}\frac{(-1)^{\left\lfloor\frac{1-{\rm sgn}(\delta(u))}{2}\right\rfloor}|\delta(u)|}{\sqrt{\sum\limits_{u\in\{0,1\}^{n-t}}|\delta(u)|^2}}|u\rangle\\
&=\sum_{u\in\{0,1\}^{n-t}}\frac{\delta(u)}{\sqrt{\sum\limits_{u\in\{0,1\}^{n-t}}\delta^2(u)}}|u\rangle.
\end{split}
\end{equation}


After Hadamard transformation on the first register, we can get the following state:

\begin{equation}
\begin{split}
	|\phi_{10}\rangle=&H^{\otimes n-t}|\phi''_9\rangle\\
	=&\sum_{u,z\in\{0,1\}^{n-t}}\frac{\delta(u)}{\sqrt{\sum\limits_{u\in\{0,1\}^{n-t}}\delta^2(u)}}\frac{(-1)^{u\cdot z}}{\sqrt{2^{n-t}}}\ket{z}.
	%=&\frac{1}{\sqrt{2^{n-t}}}\sum_{u\in\{0,1\}^{n-t}}(H^{\otimes n-t}|u\rangle)|S(u)\rangle\\
	%=&\frac{1}{\sqrt{2^{n-t+2}}}(\sum_{u\in\{0,1\}^{n-t}}(H^{\otimes n-t}|u\rangle)|S(u)\rangle\\
	%+&\sum_{u\in\{0,1\}^{n-t}}(H^{\otimes n-t}|u\rangle)|S(u)\rangle)\\
	%=&\frac{1}{\sqrt{2^{n-t+2}}}(\sum_{u\in\{0,1\}^{n-t}}(H^{\otimes n-t}|u\rangle)|S(u)\rangle\\
	%+&\sum_{u\in\{0,1\}^{n-t}}(H^{\otimes n-t}|u \oplus s_1\rangle)|S(u \oplus s_1)\rangle)\\
	%	=&\frac{1}{\sqrt{2^{n-t+2}}}(\sum_{u\in\{0,1\}^{n-t}}(H^{\otimes n-t}|u\rangle)|S(u)\rangle\\
	%+&\sum_{u\in\{0,1\}^{n-t}}(H^{\otimes n-t}|u \oplus s_1\rangle)|S(u)\rangle)\\
	%=&\frac{1}{\sqrt{2^{n-t+2}}}\sum_{u\in\{0,1\}^{n-t}}(H^{\otimes n-t}(|u\rangle+|u\oplus s_1\rangle))|S(u)\rangle\\
	%=&\frac{1}{\sqrt{2^{n-t+2}}}\sum_{u\in\{0,1\}^{n-t}}(\frac{1}{\sqrt{2^{n-t}}}\sum_{z\in\{0,1\}^{n%-t}}\\
	%&((-1)^{u \cdot z}+(-1)^{(u \oplus s_1) \cdot z})\ket{z})|S(u)\rangle\\
	%=&\frac{1}{\sqrt{2^{n-t+2}}}\sum_{u\in\{0,1\}^{n-t}}(\frac{1}{\sqrt{2^{n-t}}}\sum_{z\in\{0,1\}^{n-t}}\\
	%&(-1)^{u \cdot z}(1+(-1)^{ s_1 \cdot z})\ket{z})|S(u)\rangle.
\end{split} 
\end{equation} 

%Note that if $s_1 \cdot z=1$ we have $1+(-1)^{ s_1 \cdot z}=0$ and the basis state $\ket{z}$ vanishes in the above state. If $s_1 \cdot z=0$, we have $1+(-1)^{ s_1 \cdot z}=2$, so we have:

%\begin{align*}
%	|\phi'_7\rangle=&\frac{1}{\sqrt{2^{n-t+2}}}\sum_{u\in\{0,1\}^{n-t}}(\frac{1}{\sqrt{2^{n-t}}}\sum_{z\in\{0,1\}^{n-t}}\\
%	&(-1)^{u \cdot z}(1+(-1)^{ s_1 \cdot z})\ket{z})|S(u)\rangle\\
%	=&\frac{1}{\sqrt{2^{n-t}}}\sum_{u\in\{0,1\}^{n-t}}(\frac{1}{\sqrt{2^{n-t}}}\sum_{z\in s_1^{\perp}}(-1)^{u \cdot z}\ket{z})|S(u)\rangle\\
%	=&\frac{1}{2^{n-t}}\sum_{u\in\{0,1\}^{n-t}}\sum_{z\in s_1^{\perp}}(-1)^{u \cdot z}\ket{z}|S(u)\rangle\\
%	=&\frac{1}{2^{n-t}}\sum_{z\in s_1^{\perp}}\ket{z}\sum_{u\in\{0,1\}^{n-t}}(-1)^{u \cdot z}|S(u)\rangle.
%\end{align*}


The probability of measuring the first quantum register with the result of $0^{n-t}$ is
\begin{equation}
\left|\frac{1}{\sqrt{2^{n-t}\sum\limits_{u\in\{0,1\}^{n-t}}\delta^2(u)}}\sum\limits_{u\in\{0,1\}^{n-t}}\delta(u)\right|^2.
\end{equation}


After measurement on the first register, according to Theorem \ref{The2}, if the result is not $0^{n-t}$, then $f$ is  balanced, otherwise $f$ is constant. %we can get a string that is in $s_1^{\perp}$. After $O(n-t)$ repetitions of the above algorithm, we can obtain $O(n-t)$ elements in the $s_1^{\perp}$. Then, using the classical Gaussian elimination method, we can obtain $s_1$.

%If we have already found $s_1$, we can use Algorithm \ref{algorithm5}  to find out $s_2$. Since $f(s_10^t)=f((s_10^t)\oplus s)$, we have $f(s_10^t)=f(0^{n-t}s_2)$. So we can find a $v$ such that $f(s_10^t)=f(0^{n-t}v)$. Then we can obtain $s_2=v$. At last, we can obtain $s=s_1s_2$. 








In the following, we prove the correctness of  Algorithm \ref{algorithm5} , we write out the state after the first step of the algorithm in FIG. \ref{algorithm_two_nodes (II)}.

\begin{align}
  |\varphi_1\rangle&=\frac{1}{\sqrt{2^{n-1}}}\sum_{u\in\{0,1\}^{n-1}}|u\rangle\ket{0}
\end{align}

The  algorithm then queries the oracle $O_{f_{0}}$, resulting in the following state:

\begin{equation}
\begin{split}
  |\varphi_2\rangle&=O_{f_{0}}\frac{1}{\sqrt{2^{n-t}}}\sum_{u\in\{0,1\}^{n-t}}|u\rangle\ket{0}\\
  &=\frac{1}{\sqrt{2^{n-t}}}\sum_{u\in\{0,1\}^{n-t}}|u\rangle|f_{0}(u)\rangle\\ 
  &=\frac{1}{\sqrt{2^{n-t}}}\sum_{u\in\{0,1\}^{n-t}}|u\rangle|f(u0)\rangle
\end{split}
\end{equation}

The algorithm then apply the quantum gate $Z$ to the second quantum register to get the following states:
\begin{equation}
  |\varphi_3\rangle=\frac{1}{\sqrt{2^{n-t}}}\sum_{u\in\{0,1\}^{n-t}}(-1)^{f(u0)}|u\rangle|f(u0)\rangle.
\end{equation}

After the action of the operator $O_{f_1}$, we have the following state:
\begin{equation}
\begin{split}
  |\varphi_4\rangle=&\frac{1}{\sqrt{2^{n-1}}}\sum_{u\in\{0,1\}^{n-1}}(-1)^{f(u0)}|u\rangle|f(u0)\oplus f_1(u)\rangle\\
  =&\frac{1}{\sqrt{2^{n-1}}}\sum_{u\in\{0,1\}^{n-1}}(-1)^{f(u0)}|u\rangle|f(u0)\oplus f(u1)\rangle\\
  =&\frac{1}{\sqrt{2^{n-1}}}\sum\limits_{\substack{u\in\{0,1\}^{n-1}\\ f(u0)\oplus f(u1)=0}}(-1)^{f(u0)}|u\rangle\ket{0}\\
  &+\frac{1}{\sqrt{2^{n-1}}}\sum\limits_{\substack{u\in\{0,1\}^{n-1}\\ f(u0)\oplus f(u1)=1}}(-1)^{f(u0)}|u\rangle\ket{1}
\end{split}
\end{equation}




After measurement on the last register, if the result is $1$, then there $\exists$ $u\in\{0,1\}^{n-1}$ such that $f(u0)\oplus f(u1)=1$. From Theorem \ref{The3}, we know that $f$ is balanced. If the measurement result is $0$, we get the state:
\begin{equation}
\begin{split}
|\varphi_5\rangle=&\frac{1}{\sqrt{M}}\sum\limits_{\substack{u\in\{0,1\}^{n-1}\\ f(u0)\oplus f(u1)=0}}(-1)^{f(u0)}|u\rangle\ket{0}, \\
\end{split}
\end{equation}
where $M=|\{u\in\{0,1\}^{n-1}|f(u0)\oplus f(u1)=0\}|$.

By tracing out the state of the $1$ bit register in the last, we can get the state:
\begin{equation}
\begin{split}
|\varphi_5'\rangle=&\frac{1}{\sqrt{M}}\sum\limits_{\substack{u\in\{0,1\}^{n-1}\\ f(u0)\oplus f(u1)=0}}(-1)^{f(u0)}|u\rangle.
\end{split}
\end{equation}

After Hadamard transformation on the first register, we can get the following state:
\begin{equation}
\begin{split}
	|\varphi'_6\rangle=&(H^{\otimes n-1}\otimes I)|\varphi'_5\rangle\\
	=&\frac{1}{\sqrt{2^{n-1}M}}\sum\limits_{\substack{z,u\in\{0,1\}^{n-1}\\ f(u0)\oplus f(u1)=0}}(-1)^{f(u0)+u\cdot z}\ket{z}\ket{0}
\end{split} 
\end{equation} 




%In the step 6 of Algorithm \ref{algorithm5} , if the measurement of the second quantum register is $1$, then there $\exists$ $u\in\{0,1\}^{n-1}$ such that $f(u0)\oplus f(u1)=1$. According to Theorem \ref{The3}, it follows that $f$ is balanced.

The probability of measuring the first quantum register with the result of $0^{n-1}$ is
\begin{equation}
\left|\frac{1}{\sqrt{2^{n-1}M}}\sum\limits_{\substack{u\in\{0,1\}^{n-1}\\ f(u0)\oplus f(u1)=0}}(-1)^{f(u0)}\right|^2.
\end{equation}

After measurement on the first register, according to Theorem \ref{The4} and Theorem \ref{The5}, if the result is not $0^{n-1}$, then $f$ is  balanced, otherwise $f$ is constant.







%%%%%%%%%%%%%%%2022/10/20%%%%%%%%%%%%%%%





In the end, we prove the correctness of  Algorithm 5, we write out the state after the first step of the algorithm in FIG. \ref{algorithm_multiple_nodes (III)}.
\begin{equation}
\begin{split}
  \ket{\Phi_1}&=\frac{1}{\sqrt{2^{n-t}}}\sum_{u\in\{0,1\}^{n-t}}|u\rangle(\bigotimes_{w \in \{0,1\}^{t-1}0}|0^3\rangle)\ket{0^{3t+2}}\\
  &=\frac{1}{\sqrt{2^{n-t}}}\sum_{u\in\{0,1\}^{n-t}}|u\rangle\underbrace{\ket{0} \ldots\ket{0}}_{3\cdot2^{t-1}}\ket{0^{3t+2}}.
\end{split} 
\end{equation} 

Then the algorithm queries the oracle $O_{f_{0^t}}$ and $O_{f_{0^{t-1}1}}$, resulting in the following state:
\begin{equation}
\begin{split}
  &|\Phi_3\rangle\\
  =& \frac{1}{\sqrt{2^{n-t}}}\sum_{u\in\{0,1\}^{n-t}}|u\rangle|f_{0^t}(u)\rangle|f_{0^{t-1}1}(u)\rangle|0^{3\cdot 2^{t-1}+3t}\rangle\\
  =&\frac{1}{\sqrt{2^{n-t}}}\sum_{u\in\{0,1\}^{n-t}}|u\rangle|f(u0^t)\rangle|f(u0^{t-1}1)\rangle|0^{3\cdot 2^{t-1}+3t}\rangle
\end{split} 
\end{equation} 


The 4th quantum register performs two-bit controlled $X$ gate under the control of the 2nd and 3rd quantum registers, we get the state:
\begin{equation}
\begin{split}
&|\Phi_4\rangle\\
=&\frac{1}{\sqrt{2^{n-t}}}\sum_{u\in\{0,1\}^{n-t}}|u\rangle|f(u0^t)\rangle|f(u0^{t-1}1)\rangle\\
&\qquad\qquad\qquad|f(u0^t)\land f(u0^{t-1}1)\rangle|0^{3\cdot 2^{t-1}+3t-1}\rangle
\end{split} 
\end{equation} 

The 3rd quantum register performs controlled $X$ gate under the control of the 2nd quantum register, we get the state:
\begin{equation}
\begin{split}
&|\Phi_5\rangle\\
=&\frac{1}{\sqrt{2^{n-t}}}\sum_{u\in\{0,1\}^{n-t}}|u\rangle|f(u0^t)\rangle|f(u0^t)\oplus f(u0^{t-1}1)\rangle\\
&\qquad\qquad\qquad|f(u0^t)\land f(u0^{t-1}1)\rangle|0^{3\cdot 2^{t-1}+3t-1}\rangle
\end{split} 
\end{equation}    


 Perform the same operations on the remaining compute nodes  as on the 1st  computing node and the 2nd computing node, we have the following state:
\begin{equation}
\begin{split}
 &|\Phi_6\rangle\\
 =&\frac{1}{\sqrt{2^{n-t}}}\sum_{u\in\{0,1\}^{n-t}}|u\rangle(\otimes_{w\in\{0,1\}^{t-1}0}|f_w(u)\rangle\\
 &|f_w(u)\oplus f_{w+1}(u)\rangle|f_w(u)\land f_{w+1}(u)\rangle)\ket{0^{3t+2}}\\
 =&\frac{1}{\sqrt{2^{n-t}}}\sum_{u\in\{0,1\}^{n-t}}|u\rangle(\otimes_{w\in\{0,1\}^{t-1}0}|f(uw)\rangle\\
 &|f(uw)\oplus f(u(w+1))\rangle|f(uw)\land f(u(w+1))\rangle)\ket{0^{3t+2}}
\end{split} 
\end{equation}   



The $(3\cdot 2^{t-1}+2)$-th quantum register performs its own $A$ under the control of the $(\{3k|1\leq k\leq 2^{t-1}\})$-th quantum registers. Then we obtain the following state:
\begin{equation}
\begin{split}
|\Phi_7\rangle=&\frac{1}{\sqrt{2^{n-t}}}\sum_{u\in\{0,1\}^{n-t}}|u\rangle(\otimes_{w\in\{0,1\}^{t-1}0}|f(uw)\rangle\\
&|f(uw)\oplus f(u(w+1))\rangle|f(uw)\land f(u(w+1))\rangle)\\
&|\sum_{w\in\{0,1\}^{t-1}0}f(uw)\oplus f(u(w+1))\rangle|0^{2t+2}\rangle.
\end{split} 
\end{equation}   



The $(3\cdot 2^{t-1}+3)$-th quantum register performs its own $A$ under the control of the $(\{3k+1|1\leq k\leq 2^{t-1}\})$-th quantum registers. Then we obtain the following state:
\begin{equation}
\begin{split}
|\Phi_8\rangle=&\frac{1}{\sqrt{2^{n-t}}}\sum_{u\in\{0,1\}^{n-t}}|u\rangle(\otimes_{w\in\{0,1\}^{t-1}0}|f(uw)\rangle\\
&|f(uw)\oplus f(u(w+1))\rangle|f(uw)\land f(u(w+1))\rangle)\\
&|\sum_{w\in\{0,1\}^{t-1}0}f(uw)\oplus f(u(w+1))\rangle\\
&|\sum_{w\in\{0,1\}^{t-1}0}f(uw)\land f(u(w+1))\rangle|0^{t+2}\rangle.
\end{split} 
\end{equation}             
           

The $(3\cdot 2^{t-1}+4)$-th quantum register performs its own $V$ under the control of  $(3\cdot 2^{t-1}+2)$-th and the $(3\cdot 2^{t-1}+3)$-th quantum registers, we get the following state:
\begin{equation}
\begin{split}
|\Phi_9\rangle=&\frac{1}{\sqrt{2^{n-t}}}\sum_{u\in\{0,1\}^{n-t}}|u\rangle(\otimes_{w\in\{0,1\}^{t-1}0}|f(uw)\rangle\\
&|f(uw)\oplus f(u(w+1))\rangle|f(uw)\land f(u(w+1))\rangle)\\
&|\left\lfloor\frac{1-{\rm sgn}(\Delta(u))}{2}\right\rfloor\rangle||\Delta(u)|\rangle\ket{0}.
\end{split} 
\end{equation}   


Then the quantum gate $Z$ acts on the $(3\cdot 2^{t-1}+4)$-th qubit, the last quantum register performs its own $R$ under the control of the bottom $t$ quantum registers, we get the following state:
\begin{equation}
\begin{split}
|\Phi_{10}\rangle=&\frac{1}{\sqrt{2^{n-t}}}\sum_{u\in\{0,1\}^{n-t}}|u\rangle(\otimes_{w\in\{0,1\}^{t-1}0}|f(uw)\rangle\\
&|f(uw)\oplus f(u(w+1))\rangle|f(uw)\land f(u(w+1))\rangle)\\
&|\left\lfloor\frac{1-{\rm sgn}(\Delta(u))}{2}\right\rfloor\rangle||\Delta(u)|\rangle\\
&(\frac{|\Delta(u)|}{2^{t-1}}\ket{0}+\sqrt{1-(\frac{|\Delta(u)|}{2^{t-1}})^2}\ket{1}).
\end{split} 
\end{equation}  


Uncomputing the middle $3\cdot2^{t-1}+3t+1$ quantum registers, resulting in the following state:
\begin{equation}
\begin{split}
|\Phi_{11}\rangle=&\frac{1}{\sqrt{2^{n-t}}}\sum_{u\in\{0,1\}^{n-t}}(-1)^{\left\lfloor\frac{1-{\rm sgn}(\Delta(u))}{2}\right\rfloor}|u\rangle|0^{3\cdot 2^{t-1}+3t+1}\rangle\\
&(\frac{|\Delta(u)|}{2^{t-1}}\ket{0}+\sqrt{1-(\frac{|\Delta(u)|}{2^{t-1}})^2}\ket{1}).
\end{split} 
\end{equation}  



By tracing out the states of the $3\cdot 2^{t-1}+3t+1$ bit registers in the middle, we can get the state:

\begin{equation}
\begin{split}
&|\Phi_{11}'\rangle=\frac{1}{\sqrt{2^{n-t}}}\sum_{u\in\{0,1\}^{n-t}}(-1)^{\left\lfloor\frac{1-{\rm sgn}(\Delta(u))}{2}\right\rfloor}|u\rangle\\
  &\qquad\qquad\qquad\qquad\otimes(\frac{|\Delta(u)|}{2^{t-1}}\ket{0}+\sqrt{1-(\frac{|\Delta(u)|}{2^{t-1}})^2}\ket{1}).
\end{split}
\end{equation}














After measurement on the last register, if the result is $1$, then $\exists$ $u\in\{0,1\}^{n-t}$ such that $|\Delta(u)|\neq 2^{t-1}$. From Theorem \ref{The6}, we know that $f$ is balanced. If the measurement result is $0$, we get the state:

\begin{equation}
\begin{split}
|\Phi'_{12}\rangle=\sum_{u\in\{0,1\}^{n-t}}\frac{(-1)^{\left\lfloor\frac{1-{\rm sgn}(\Delta(u))}{2}\right\rfloor}|\Delta(u)|}{\sqrt{\sum\limits_{u\in\{0,1\}^{n-t}}|\Delta(u)|^2}}|u\rangle\ket{0}.
\end{split}
\end{equation}




By tracing out the state of the $1$ bit register in the last, we can get the state:

\begin{equation}
\begin{split}
|\Phi''_{12}\rangle=&\sum_{u\in\{0,1\}^{n-t}}\frac{(-1)^{\left\lfloor\frac{1-{\rm sgn}(\Delta(u))}{2}\right\rfloor}|\Delta(u)|}{\sqrt{\sum\limits_{u\in\{0,1\}^{n-t}}|\Delta(u)|^2}}|u\rangle\\
&=\sum_{u\in\{0,1\}^{n-t}}\frac{\Delta(u)}{\sqrt{\sum\limits_{u\in\{0,1\}^{n-t}}\Delta^2(u)}}|u\rangle.
\end{split}
\end{equation}


After Hadamard transformation on the first register, we can get the following state:

\begin{equation}
\begin{split}
	|\Phi''_{13}\rangle=&H^{\otimes n-t}|\Phi''_{12}\rangle\\
	=&\sum_{u,z\in\{0,1\}^{n-t}}\frac{\Delta(u)}{\sqrt{\sum\limits_{u\in\{0,1\}^{n-t}}\Delta^2(u)}}\frac{(-1)^{u\cdot z}}{\sqrt{2^{n-t}}}\ket{z}.
\end{split} 
\end{equation} 



The probability of measuring the first quantum register with the result of $0^{n-t}$ is
\begin{equation}
\left|\frac{1}{\sqrt{2^{n-t}\sum\limits_{u\in\{0,1\}^{n-t}}\Delta^2(u)}}\sum\limits_{u\in\{0,1\}^{n-t}}\Delta(u)\right|^2.
\end{equation}

%After measurement on the first register, according to Theorem \ref{The2}, if the result is not $0^{n-t}$, then $f$ is  balanced, otherwise $f$ is constant.

%In the step 12 of Algorithm 5, if the measurement of the last quantum register is $1$, then there $\exists$ $u\in\{0,1\}^{n-t}$ such that $|\Delta(u)|\neq 2^{t-1}$. According to Theorem \ref{The6}, it follows that $f$ is balanced.

%In the step 13 of Algorithm 5, the probability of measuring the first quantum register with the result of $0^{n-t}$ is
%\begin{equation}
%\left|\frac{1}{\sqrt{2^{n-t}\sum\limits_{u\in\{0,1\}^{n-t}}\Delta^2(u)}}\sum\limits_{u\in\{0,1\}^{n-t}}\Delta(u)\right|^2.
%\end{equation}

After measurement on the first register, according to Theorem \ref{The7} and Theorem \ref{The8}, if the result is not $0^{n-t}$, then $f$ is  balanced, otherwise $f$ is constant.



\fi



\appendix
\section{DJ Algorithm}
\label{DJ Algorithm}




\begin{figure}[H]
	\begin{minipage}{\linewidth}	
		\begin{algorithm}[H]	
			\caption{DJ algorithm}
			\label{DJ algorithm}
			\begin{algorithmic}[1]
				\State 
				$|\psi_0\rangle=|0\rangle^{\otimes n}|1\rangle$;
							
				\State 
				$|\psi_1\rangle=H^{\otimes n+1}|\psi_0\rangle=\frac{1}{\sqrt{2^n}}\sum\limits_{x\in\{0 , 1\}^n}|x\rangle|-\rangle$;
								
				\State 			
				$|\psi_2\rangle=O_f|\psi_1\rangle=\frac{1}{\sqrt{2^n}}\sum\limits_{x\in\{0 , 1\}^n}|x\rangle\Big(\frac{|0\oplus f(x)\rangle-|1\oplus f(x)\rangle}{\sqrt{2}}\Big)=\frac{1}{\sqrt{2^n}}\sum\limits_{x\in\{0 , 1\}^n}(-1)^{f(x)}|x\rangle|-\rangle$;
							
				\State 			
				$|\psi_3\rangle=H^{\otimes n+1}|\psi_2\rangle=\sum\limits_{x,z\in\{0 , 1\}^n}\dfrac{(-1)^{x\cdot z+f(x)}}{2^n}|z\rangle|1\rangle$;
				
				\State Measure the first $n$ qubits of $\ket{\psi_3}$:  if the result is not  $0^{n}$, then output $f$ is balanced; if the result is $0^{n}$, then output $f$ is constant.
			\end{algorithmic}
		\end{algorithm}
		\end{minipage}
	\end{figure}

\begin{figure}[H]
\centering
\includegraphics[scale=0.45]{DJ_ALG.png}
\caption{The circuit for  DJ algorithm (Algorithm \ref{DJ algorithm} ).}
\label{fig:1}
\end{figure}






\section{Distributed DJ algorithm for multiple computing nodes with errors}
\label{Distributed DJ algorithm for multiple computing nodes with errors}






In the following, we first give the algorithm ${\rm DJ}_w$ that acts on  subfunction $f_w$.% and the  circuit that implements ${\rm DJ}_w$ algorithm.

%Given a Boolean function $f:\{0,1\}^n\rightarrow \{0,1\}$ of DJ problem,  decomposing it into $2^t$ subfunctions $f_w$ as in Equation (\ref{General method of function decomposition}).

\begin{figure}[H]
	\begin{minipage}{\linewidth}	
		\begin{algorithm}[H]	
			\caption{${\rm DJ}_w$ algorithm}
			\label{DJw algorithm}
			\begin{algorithmic}[1]
				\State 
				$|\psi'_0\rangle=|0^{n-t}\rangle|1\rangle$;
							
				\State 
				$|\psi'_1\rangle=H^{\otimes n-t+1}|\psi'_0\rangle=\frac{1}{\sqrt{2^{n-t}}}\sum\limits_{x\in\{0 , 1\}^{n-t}}|x\rangle|-\rangle$;
								
				\State 			
				$|\psi'_2\rangle=O_{f_w}|\psi'_1\rangle=\frac{1}{\sqrt{2^{n-t}}}\sum\limits_{x\in\{0 , 1\}^{n-t}}(-1)^{f_w(u)}|x\rangle|-\rangle$;
							
				\State 			
				$|\psi'_3\rangle=H^{\otimes n-t+1}|\psi'_2\rangle=\sum\limits_{x,z\in\{0 , 1\}^{n-t}}\dfrac{(-1)^{x\cdot z+f_w(u)}}{2^{n-t}}|z\rangle|1\rangle$;
				
				\State Measure the first $n-t$ qubits of $\ket{\psi'_3}$:  if the result is not  $0^{n-t}$, then output $f_w$ is not constant; if the result is $0^{n-t}$, then output $f_w$ is not balanced.
			\end{algorithmic}
		\end{algorithm}
		\end{minipage}
	\end{figure}

\iffalse
\begin{algorithm}[H]
	\caption{ ${\rm DJ}_i$ algorithm}%算法名字
	\LinesNumbered %要求显示行号
	\KwIn{A black box $O_{f_w}$ which performs the transformation
	$O_{f_w}|x\rangle|b\rangle\rightarrow|x\rangle|b\oplus f_w(u)\rangle$. %It is
%It is promised that $f$ is either constant or  balanced.%for all values of  $x$, or else $f(x)$ is balanced,
%that is, equal to $1$ for exactly half of all the possible $x$, and $0$ for the other half.
}%输入参数
	\KwOut{ If $f_w$ is constant, then output $0^{n-t}$. %If $f_w$ is not constant,then output $0^{n-t}$ or  not.
	}%输出
	
	
	$|\phi_0\rangle=|0\rangle^{\otimes(n-m)}|1\rangle$.%$1^{\circ}$  
	
	 $|\phi_1\rangle=H^{\otimes(n-m)}|\phi_0\rangle=\frac{1}{\sqrt{2^{n-t}}}\sum\limits_{x\in\{0 , 1\}^{n-t}}|x\rangle|-\rangle$.%$2^{\circ}$
	
	 $|\phi_2\rangle=O_{f_w}|\phi_1\rangle=\frac{1}{\sqrt{2^{n-t}}}\sum\limits_{x\in\{0 , 1\}^{n-t}}(-1)^{f_w(u)}|x\rangle|-\rangle$.%$3^{\circ}$
	
	 $|\phi_3\rangle=H^{\otimes (n-m+1)}|\phi_2\rangle=\sum\limits_{z\in\{0 , 1\}^{n-t}}\sum\limits_{x\in\{0 , 1\}^{n-t}}\dfrac{(-1)^{x\cdot z+f_w(u)}}{2^{n-t}}|z\rangle|1\rangle$.%=\left\{
	%\begin{array}{rcl}
	%	(-1)^{f(0^n)}|0^n\rangle|-\rangle, & & \emph{f}\ \text{为常值函数;}\\
	%	\sum\limits_{z\in\{0 , 1\}^n,z\neq 0^n}\sum\limits_{x\in\{0 , 1\}^n}\frac{(-1)^{x\cdot %z+f(x)}}{2^n}|z\rangle|-\rangle, & & \emph{f}\ \text{为平衡函数。}
	%\end{array} \right.$%$4^{\circ}$
	
	 Measure the first $n-m$ qubits.%$5^{\circ}$
\end{algorithm}
\fi



\begin{figure}[H]
\centering
\includegraphics[scale=0.65]{DJ_w_ALG.png}
\caption{The circuit for  ${\rm DJ}_w$ algorithm (Algorithm \ref{DJw algorithm} ).}
\label{fig:2}
\end{figure}

In the following, we design  Algorithm \ref{Distributed DJ algorithm for multiple computing nodes with errors_Algorithm flow}, which generalizes the algorithm in \cite{avron_quantum_2021} .


\begin{figure}[H]
	\begin{minipage}{\linewidth}	
		\begin{algorithm}[H]	
			\caption{Distributed DJ algorithm for multiple computing nodes with errors}
			\label{Distributed DJ algorithm for multiple computing nodes with errors_Algorithm flow}
			\begin{algorithmic}[1]
				\State 
				Decompose Boolean function $f:\{0,1\}^n\rightarrow \{0,1\}$ into $2^t$ subfunctions $f_w$ $(w\in\{0,1\}^t)$ as Equation (\ref{General method of function decomposition});
							
				\State 
				%For each  subfunction  $f_w$, 
				Apply the ${\rm DJ}_w$ algorithm to subfunction  $f_w$;
								
				\State 			
				Record  the measurement  result  of ${\rm DJ}_w$ algorithm as  $M_w$;
							
				\State 			
				If  there  is  $M_w$ that  is not  $0^{n-t}$,  then output  $f$ is   balanced;
				
				\State If  all  $M_w$ are $0^{n-t}$, then output  $f$  is  constant.
			\end{algorithmic}
		\end{algorithm}
		\end{minipage}
	\end{figure}

%The following proposition \ref{Distributed  DJ algorithm pro 1} analyses the error of Algorithm \ref{Distributed  DJ algorithm}.% , which is proved in  Appendix.

 %proposition 
In the following, we give the error analysis for Algorithm \ref{Distributed DJ algorithm for multiple computing nodes with errors_Algorithm flow}.

%\begin{proposition}\label{Distributed  DJ algorithm pro 1}
%Given a Boolean function $f:\{0,1\}^n\rightarrow \{0,1\}$ of DJ problem,  decompose $f$ into $2^t$ subfunctions $f_w:\{0,1\}^{n-t}\rightarrow \{0,1\}$ $(w\in\{0,1\}^t)$. Let $ DJ_w$ be the algorithm corresponding to the subfunctions $f_w$, $M_w$ be the measurement result of $DJ_w$ and $N=2^n$. The  probability that Algorithm \ref{Distributed DJ algorithm for multiple computing nodes with errors_Algorithm flow}  misidentify a balanced function as constant is at least  $0$.
%The error of  Algorithm \ref{Distributed DJ algorithm for multiple computing nodes with errors_Algorithm flow} is at most $\frac{2^t-1}{N-1}$.
%\end{proposition}

%\begin{proof}
%Record the number of its function value of $1$ be $l_w$, that is
Let
%\begin{equation}
$l_w=|\{u\in\{0,1\}^{n-t}|f_w(u)=1\}|$,
%\end{equation}
where $w\in\{0,1\}^t$.
%For the subfunction $f_w$, let the number of its function value equal to $1$ be a random variable $k_w$, where $k_w\in[ 0,2^{n-t}]$, then after using the corresponding ${\rm DJ}_i$ algorithm, the probability that the measurement result $M_w$ is $0^{n-t}$ is

The probability that $M_w=0^{n-t}$ in the case $l_w=k_w$ is
\begin{equation}\label{Miprob}
\begin{split}
&\Pr(M_w=0^{n-t}| l_w=k_w)\\
=&\left|\frac{1}{2^{n-t}}[k_w\cdot(-1)+(2^{n-t}-k_w)\cdot1]\right|^2\\
%=&\left(2^{1+t-n}k_w-1\right)^2\\
=&\left(\dfrac{2^{t+1}}{N}k_w-1\right)^2.
\end{split}
\end{equation}


%If $f$ is a constant  function, then all its subfunctions $f_w$ are constant  functions. From the formula $(\ref{Miprob})$, it can be obtained that for any $i\in[0,2^m- 1]$, the probability of $M_w=0^{n-t}$ is $1$. According to the step $5$  of the distributed $\rm DJ$ algorithm we designed, the probability of its correct determination is $1$.


The  probability that Algorithm \ref{Distributed DJ algorithm for multiple computing nodes with errors_Algorithm flow}  misidentifies a balanced function as constant is 
\begin{equation}\label{Distributed DJ algorithm for multiple computing nodes with errors_Algorithm_probability}
\begin{split}
&\Pr(f\ is\ \ balanced,M_w=0^{n-t},\forall w\in\{0,1\}^t)\\
=&\sum\limits_{\substack{\sum\limits_{w\in\{0,1\}^t}k_w=\frac{N}{2}\\ 0\leq k_w\leq\frac{N}{2^t}}}\Pr(l_w=k_w)\Pr(M_w=0^{n-t},\forall w\in\{0,1\}^t |l_w=k_w)\\
=&\sum\limits_{\substack{\sum\limits_{w\in\{0,1\}^t}k_w=\frac{N}{2}\\ 0\leq k_w\leq\frac{N}{2^t}\\ }}\frac{\dbinom{N/2^t}{k_0}\dbinom{N/2^t}{k_1}\cdots\dbinom{N/2^t}{k_{2^t-1}}}{\dbinom{N}{N/2}}\prod\limits_{w\in\{0,1\}^t}\left(\dfrac{2^{t+1}}{N}k_w-1\right)^2.\\
>0.
\end{split}
\end{equation}
%In fact, random vector $(k_0, k_1,\dots,k_{2^m-1})\sim H(N,N/2,N/2^m,N/2^m,\dots,N/2^m)$. 

%\end{proof}

\iffalse
For
$0\leq k_w\leq\frac{N}{2^t}$, 
we have
%\begin{equation}
$ 0\leq\left(\dfrac{2^{t+1}}{N}k_w-1\right)^2\leq1$.
% \end{equation}
 
Therefore, we have
\begin{equation}
\begin{split}
&\Pr(f\ is\ balanced,M_w=0^{n-t},\forall w\in\{0,1\}^t)\\
\leq&\sum\limits_{\substack{\sum\limits_{w\in\{0,1\}^t}k_w=\frac{N}{2}\\ 0\leq k_w\leq\frac{N}{2^t}\\ }}\frac{\dbinom{N/2^t}{k_0}\dbinom{N/2^t}{k_1}\cdots\dbinom{N/2^t}{k_{2^t-1}}}{\dbinom{N}{N/2}}\left(\dfrac{2^{t+1}}{N}k_0-1\right)^2\\
=&\left(\dfrac{2^{t+1}}{N}\right)^2\left(\sum_{k_0=0}^{N/2^t}\dfrac{\dbinom{N/2^t}{k_0}\dbinom{N-N/2^t}{N/2-k_0}}{\dbinom{N}{N/2}}\left(k_0-\dfrac{N}{2^{t+1}}\right)^2\right)\\
=&\left(\dfrac{2^{t+1}}{N}\right)^2\left[\left(\frac{N}{2^{t+1}}\right)^2\frac{2^t-1}{N-1}\right]\\
=&\frac{2^t-1}{N-1}.
\end{split}
\end{equation}

%It is clear that  the random variable $k_0$ obey the hypergeometric distribution $H(N,N/2,N/2^t)$. 

%, $\sum\limits_{k_0=0}^{N/2^t}\dfrac{\dbinom{N/2^t}{k_0}\dbinom{N-N/2^t}{N/2-k_0}}{\dbinom{N}{N/2}}\left(k_0-\dfrac{N}{2^{t+1}}\right)^2$ is the variance of the random variable $k_0$, where $k_0$ obey the hypergeometric distribution $H(N,N/2,N/2^t)$. %This is because according to the expectation formula of the hypergeometric distribution, we have
%\begin{equation}
%E(k_0)=\frac{(N/2)(N/2^m)}{N}=\frac{N}{2^{m+1}}.
%\end{equation}
%So we have
%\begin{equation}
%\begin{split}
%D(k_0)&=E(k_0-E(k_0))^2\\
%&=\sum\limits_{k_0=0}^{N/2^m}\dfrac{\dbinom{N/2^m}{k_0}\dbinom{N-N/2^m}{N/2-k_0}}{\dbinom{N}{N/2}}\left(k_0-\dfrac{N}{2^{m+1}}\right)^2.
%\end{split}
%\end{equation}

\iffalse
According to the variance formula of hypergeometric distribution , we have
\begin{equation}
\begin{split}
D(k_0)&=\sum_{k_0=0}^{N/2^t}\dfrac{\dbinom{N/2^t}{k_0}\dbinom{N-N/2^t}{N/2-k_0}}{\dbinom{N}{N/2}}\left(k_0-\dfrac{N}{2^{t+1}}\right)^2\\
&=\frac{\frac{N}{2}\frac{N}{2^t}(N-\frac{N}{2})(N-\frac{N}{2^t})}{N^2(N-1)}\\
&=\left(\frac{N}{2^{t+1}}\right)^2\frac{2^t-1}{N-1}.
\end{split}
\end{equation}

Therefore, we have
\begin{equation}\label{prob_error1}
\begin{split}
&\Pr(f\ is\ balanced,M_w=0^{n-t},\forall w\in\{0,1\}^t)\\
\leq&\left(\dfrac{2^{t+1}}{N}\right)^2\left(\sum_{k_0=0}^{N/2^t}\dfrac{\dbinom{N/2^t}{k_0}\dbinom{N-N/2^t}{N/2-k_0}}{\dbinom{N}{N/2}}\left(k_0-\dfrac{N}{2^{t+1}}\right)^2\right)\\
=&\frac{2^t-1}{N-1}.
\end{split}
\end{equation}
\fi
%Therefore, if $f$ is a balance function, then the probability of correct determination by the distributed $\rm DJ$ algorithm we designed is as follows:
%\begin{equation}
%\begin{split}
%&1-\Pr(f\ is\ a\ balance\ function,\forall M_w=0^{n-t},0\leq w\leq 2^{t}-1)\\
%\geq& 1-\frac{2^m-1}{N-1}\\
%=&\frac{N-2^m}{N-1}.
%\end{split}
%\end{equation}


\end{proof}

\fi



Although the algorithm in \cite{avron_quantum_2021} can be extended to the general case of multiple distributed computing nodes, i.e. Algorithm \ref{Distributed DJ algorithm for multiple computing nodes with errors_Algorithm flow}, it follows from equation \ref{Distributed DJ algorithm for multiple computing nodes with errors_Algorithm_probability} that the  probability that Algorithm \ref{Distributed DJ algorithm for multiple computing nodes with errors_Algorithm flow}  misidentifies a balanced function as constant is at least  $0$. % which is a result of the structure of the Boolean function of  DJ problem being broken.
% Therefore, we propose  distributed quantum algorithms for solving  DJ problem by using the essential structure of  DJ problem, which draws on the design ideas and methods of distributed Simon's quantum algorithm \cite{Tan2022DQCSimon} and HHL algorithm \cite{HHL_2009}.







\section{Examples of the structure of  DJ problem in  distributed scenario}
\label{Examples of the structure of  DJ problem in  distributed scenario}


\begin{example}
Given a Boolean function $f:\{0,1\}^3\rightarrow\{0,1\}$ for DJ problem, decompose $f$ into two subfunctions: $f_{0}$ and $f_{1}$, which are assumed to be as follows.
\iffalse
\begin{equation}
\begin{split}
&f_{00}(00)=1,\ f_{00}(01)=0,\ f_{00}(10)=0,\ f_{00}(11)=1;\\
&f_{01}(00)=0,\ f_{01}(01)=0,\ f_{01}(10)=1,\ f_{01}(11)=0;\\
&f_{10}(00)=0,\ f_{10}(01)=1,\ f_{10}(00)=0,\ f_{10}(11)=1;\\
&f_{11}(00)=1,\ f_{11}(01)=1,\ f_{11}(10)=1,\ f_{11}(11)=0.\\
\end{split}
\end{equation}
\fi
\begin{table}[H]
\begin{center}
\begin{tabular}{|l|c|c|}
\hline
\diagbox{\makecell[c]{$u$}}{\makecell[c]{$f_w(u)$}}{\makecell[c]{$w$}} &\makecell[c]{0} &\makecell[c]{1}\\
\hline
 \quad 00 &   1  &  0  \\ 
\hline
 \quad 01 &   0  &  0 \\  
 \hline
 \quad 10 &   0  &  1 \\ 
  \hline
 \quad 11 &   1  & 1 \\ 
\hline
\end{tabular}
\caption{Example of the structured table for  DJ problem in  distributed scenario with two subfunctions}\label{Tab_Example of a structured table for a DJ problem in a distributed scenario with two subfunctions}
\end{center}
\end{table}



It is clear that 
\begin{align}
B_{00}=&|\{u\in\{0,1\}^{2}|f_0(u)=f_1(u)=0\}|=1.\\
B_{11}=&|\{u\in\{0,1\}^{2}|f_0(u)= f_1(u)=1\}|=1.\\
 M=&|\{u\in\{0,1\}^{2}|f_0(u)\oplus f_1(u)=0\}|=2. 
\end{align}

Therefore 
\begin{equation}
B_{00}=B_{11}=M/2.
\end{equation}

From Theorem \ref{The4}, we can deduce that $f$ is balanced.


\end{example}


\begin{example}
Given a Boolean function $f:\{0,1\}^4\rightarrow\{0,1\}$ for DJ problem, decompose $f$ into four subfunctions: $f_{00}$, $f_{01}$, $f_{10}$ and $f_{11}$, which are assumed to be as follows.

\begin{table}[H]
\begin{center}
\begin{tabular}{|l|c|c|c|c|}
\hline
\diagbox{\makecell[c]{$u$}}{\makecell[c]{$f_w(u)$}}{\makecell[c]{$w$}} &\makecell[c]{00} &\makecell[c]{01}&\makecell[c]{10}&\makecell[c]{11} \\
\hline
 \quad 00 &   1  &  0 & 1  & 0 \\ 
\hline
 \quad 01 &   1  &  0 & 1 & 1\\  
 \hline
 \quad 10 &   0  &  1 & 0 & 0\\ 
  \hline
 \quad 11 &   1  & 1 &  0 &  0\\ 
\hline
\end{tabular}
\caption{Example of the structured table for  DJ problem in  distributed scenario with four subfunctions}\label{Tab_Example of a structured table for a DJ problem in a distributed scenario with four subfunctions}
\end{center}
\end{table}



It is clear that 
\begin{equation}
\delta(00)=0;\ \delta(01)=-2;\ \delta(10)=2;\ \delta(11)=0.
\end{equation}

Therefore 
\begin{equation}
\sum\limits_{u\in\{0,1\}^{2}} \delta(u) = 0.
\end{equation}

From Theorem \ref{The2}, we can deduce that $f$ is balanced.


\end{example}




\begin{example}
Given a Boolean function $f:\{0,1\}^4\rightarrow\{0,1\}$ for DJ problem, decompose $f$ into four subfunctions: $f_{00}$, $f_{01}$, $f_{10}$ and $f_{11}$, which are assumed to be as follows.

\begin{table}[H]
\begin{center}
\begin{tabular}{|l|c|c|c|c|}
\hline
\diagbox{\makecell[c]{$u$}}{\makecell[c]{$f_w(u)$}}{\makecell[c]{$w$}} &\makecell[c]{00} &\makecell[c]{01}&\makecell[c]{10}&\makecell[c]{11} \\
\hline
 \quad 00 &   1  &  1 & 0  & 0 \\ 
\hline
 \quad 01 &   1  &  0 & 1 & 1\\  
 \hline
 \quad 10 &   0  &  1 & 0 & 0\\ 
  \hline
 \quad 11 &   1  &  0 &  1 &  0\\ 
\hline
\end{tabular}
\caption{Example of the structured table for  DJ problem in  distributed scenario with four subfunctions}\label{Tab_Example of a structured table for a DJ problem in a distributed scenario with four subfunctions}
\end{center}
\end{table}



It is clear that 
\begin{equation}
\Delta(00)=0;\ \Delta(01)=-1;\ \Delta(10)=1;\ \Delta(11)=0.
\end{equation}

Therefore 
\begin{equation}
\sum\limits_{u\in\{0,1\}^{2}} \Delta(u) = 0.
\end{equation}

From Theorem \ref{The7}, we can deduce that $f$ is balanced.


\end{example}






\section{Distributed quantum algorithm for DJ problem with errors (four distributed computing nodes)}
\label{Distributed quantum algorithm for DJ problem with errors (four distributed computing nodes)}



\begin{figure}[H]
	\begin{minipage}{\linewidth}	
		\begin{algorithm}[H]	
			\caption{Distributed quantum algorithm for DJ problem with errors (four distributed computing nodes)}
			\label{algorithmDDJerror}
			\begin{algorithmic}[1]
				\State $|\varphi'_0\rangle = |0^{n-2}\rangle\ket{0^{n-2}}$;
				
				\State $|\varphi'_1\rangle = \left(H^{\otimes n-2}\otimes I\right)|\psi'_0\rangle$; 
				
				\State 		
				$|\varphi'_2\rangle=O_{f_{01}}O_{f_{00}}|\varphi'_1\rangle$;
				
				\State %Apply the quantum gate $Z$ to the second quantum register: 			
				$|\varphi'_3\rangle=(I\otimes Z)|\varphi'_2\rangle$;
				
				\State 		
				$|\varphi'_4\rangle=O_{f_{11}}O_{f_{10}}|\varphi'_3\rangle$;
				
				\State Measure the last qubit of $|\varphi'_4\rangle$: if the result is 1, then output $f$ is balanced; if the result is 0, then denote the quantum state after measurement  as $\ket{\varphi'_5}$;
						
				\State $|\varphi'_6\rangle=\left(H^{\otimes n-2}\otimes I\right)|\varphi'_5\rangle$;
				
				\State Measure the first $n-2$ qubits of $\ket{\varphi'_6}$:  if the result is not  $0^{n-2}$, then output $f$ is balanced; if the result is $0^{n-2}$, then output $f$ is constant.
			\end{algorithmic}
		\end{algorithm}			
		\end{minipage}
	\end{figure}



  \begin{figure}[H]%参数[h]表示紧跟着文字
		\centering%居中
		\includegraphics[width=6.3in]{DDJerrors.png}
		\caption{The circuit for the  distributed quantum algorithm for DJ problem with errors (four computing nodes) (Algorithm \ref{algorithmDDJerror} ).}
		\label{algorithmDDJerror_circuit}
	\end{figure}

In the following, we prove that  Algorithm \ref{algorithmDDJerror} is not exact and has error. The state after the first step of Algorithm \ref{algorithmDDJerror} is:
\begin{align}
  |\varphi'_1\rangle&=\frac{1}{\sqrt{2^{n-2}}}\sum_{u\in\{0,1\}^{n-2}}|u\rangle\ket{0}.
\end{align}

 Algorithm \ref{algorithmDDJerror} then queries the oracle $O_{f_{00}}$ and the oracle $O_{f_{01}}$, resulting in the following state:
\begin{equation}
\begin{split}
  |\varphi'_2\rangle&=O_{f_{01}}O_{f_{00}}|\varphi'_1\rangle\\
  %&=O_{f_{0}}\left(\frac{1}{\sqrt{2^{n-t}}}\sum_{u\in\{0,1\}^{n-t}}|u\rangle\ket{0}\right)\\
  &=\frac{1}{\sqrt{2^{n-2}}}\sum_{u\in\{0,1\}^{n-2}}|u\rangle|f_{00}(u)\oplus f_{01}(u)\rangle\\ 
  %&=\frac{1}{\sqrt{2^{n-2}}}\sum_{u\in\{0,1\}^{n-2}}|u\rangle|f(u00)\oplus f(u01)\rangle.
\end{split}
\end{equation}

Then, the quantum gate $Z$ on  $|\varphi'_2\rangle$ is applied to get the following states:
\begin{equation}
\begin{split}
  |\varphi'_3\rangle=&(I\otimes Z)|\varphi'_2\rangle\\
  =&\frac{1}{\sqrt{2^{n-2}}}\sum_{u\in\{0,1\}^{n-2}}(-1)^{f_{00}(u)\oplus f_{01}(u)}|u\rangle|f_{00}(u)\oplus f_{01}(u)\rangle.
\end{split}
\end{equation}

After applying the operator $O_{f_{10}}$ and $O_{f_{11}}$ on $|\varphi'_3\rangle$, we have the following state:
\begin{equation}
\begin{split}
  |\varphi'_4\rangle=&O_{f_{11}}O_{f_{10}}|\varphi'_3\rangle\\
  =&\frac{1}{\sqrt{2^{n-2}}}\sum_{u\in\{0,1\}^{n-2}}(-1)^{f_{00}(u)\oplus f_{01}(u)}|u\rangle|f_{00}(u)\oplus f_{01}(u) \oplus f_{10}(u)\oplus f_{11}(u) \rangle\\
  =&\frac{1}{\sqrt{2^{n-2}}}\sum\limits_{\substack{u\in\{0,1\}^{n-2}\\ f_{00}(u)\oplus f_{01}(u) \oplus f_{10}(u)\oplus f_{11}(u)=0}}(-1)^{f_{00}(u)\oplus f_{01}(u)}|u\rangle\ket{0}\\
  &+\frac{1}{\sqrt{2^{n-2}}}\sum\limits_{\substack{u\in\{0,1\}^{n-2}\\ f_{00}(u)\oplus f_{01}(u) \oplus f_{10}(u)\oplus f_{11}(u)=1}}(-1)^{f_{00}(u)\oplus f_{01}(u)}|u\rangle\ket{1}.
\end{split}
\end{equation}




After measuring on the last qubit of $|\varphi'_4\rangle$, if the result is $1$, then there $\exists$ $u\in\{0,1\}^{n-2}$ such that $f(u00)\oplus f(u01)\oplus f(u11)\oplus f(u10)=1$. Similar to the proof of  Corollary \ref{Cor3}, we know that $f$ is balanced. If the result is $0$, then we get the state:
\begin{equation}
\begin{split}
|\varphi'_5\rangle=&\frac{1}{\sqrt{M'}}\sum\limits_{\substack{u\in\{0,1\}^{n-2}\\ f_{00}(u)\oplus f_{01}(u) \oplus f_{10}(u)\oplus f_{11}(u)=0}}(-1)^{f_{00}(u)\oplus f_{01}(u)}|u\rangle\ket{0}, \\
\end{split}
\end{equation}
where $M'=|\{u\in\{0,1\}^{n-2}|f_{00}(u)\oplus f_{01}(u) \oplus f_{10}(u)\oplus f_{11}(u)=0\}|$.

%By tracing out the state of the $1$ bit register in the last, we can get the state:
%\begin{equation}
%\begin{split}
%|\varphi_5'\rangle=&\frac{1}{\sqrt{M}}\sum\limits_{\substack{u\in\{0,1\}^{n-1}\\ f(u0)\oplus f(u1)=0}}(-1)^{f(u0)}|u\rangle.
%\end{split}
%\end{equation}

After Hadamard transformation on the first $n-1$ qubits of $|\varphi'_5\rangle$, we get the following state:
\begin{equation}
\begin{split}
	|\varphi'_6\rangle=&\left(H^{\otimes n-2}\otimes I\right)|\varphi'_5\rangle\\
	=&\frac{1}{\sqrt{2^{n-2}M'}}\sum\limits_{\substack{u,z\in\{0,1\}^{n-2}\\ f_{00}(u)\oplus f_{01}(u) \oplus f_{10}(u)\oplus f_{11}(u)=0}}(-1)^{f_{00}(u)\oplus f_{01}(u)+u\cdot z}\ket{z}\ket{0}.
\end{split}	
\end{equation} 




%In the step 6 of Algorithm \ref{algorithm5} , if the measurement of the second quantum register is $1$, then there $\exists$ $u\in\{0,1\}^{n-1}$ such that $f(u0)\oplus f(u1)=1$. According to Theorem \ref{The3}, it follows that $f$ is balanced.

The probability of measuring the first $n-2$ qubits of  $|\varphi'_6\rangle$ with the result of $0^{n-2}$ is
\begin{equation}\label{algorithmDDJerrorprobability}
\left|\frac{1}{\sqrt{2^{n-2}M'}}\sum\limits_{\substack{u\in\{0,1\}^{n-2}\\ f_{00}(u)\oplus f_{01}(u) \oplus f_{10}(u)\oplus f_{11}(u)=0}}(-1)^{f_{00}(u)\oplus f_{01}(u)}\right|^2.
\end{equation}

%After measuring on the first $n-1$ qubits of  $|\varphi_6\rangle$, according to Theorem \ref{The4} and Theorem \ref{The5} , if the result is not $0^{n-1}$, then $f$ is  balanced, otherwise $f$ is constant.

In the following we give an example to demonstrate that Algorithm \ref{algorithmDDJerror} cannot exactly solve  DJ problem in  distributed scenario with four computing nodes.

\begin{example}
Given a Boolean function $f:\{0,1\}^4\rightarrow\{0,1\}$ for DJ problem, decompose $f$ into four subfunctions: $f_{00}$, $f_{01}$, $f_{10}$ and $f_{11}$, which are assumed to be as follows.
\iffalse
\begin{equation}
\begin{split}
&f_{00}(00)=1,\ f_{00}(01)=0,\ f_{00}(10)=0,\ f_{00}(11)=1;\\
&f_{01}(00)=0,\ f_{01}(01)=0,\ f_{01}(10)=1,\ f_{01}(11)=0;\\
&f_{10}(00)=0,\ f_{10}(01)=1,\ f_{10}(00)=0,\ f_{10}(11)=1;\\
&f_{11}(00)=1,\ f_{11}(01)=1,\ f_{11}(10)=1,\ f_{11}(11)=0.\\
\end{split}
\end{equation}
\fi
\begin{table}[H]
\begin{center}
\begin{tabular}{|l|c|c|c|c|}
\hline
\diagbox{\makecell[c]{$u$}}{\makecell[c]{$f_w(u)$}}{\makecell[c]{$w$}} &\makecell[c]{00} &\makecell[c]{01}&\makecell[c]{10}&\makecell[c]{11} \\
\hline
 \quad 00 &   1  &  0 & 0  & 1 \\ 
\hline
 \quad 01 &   0  &  0 & 1 & 1\\  
 \hline
 \quad 10 &   0  &  1 & 0 & 1\\ 
  \hline
 \quad 11 &   1  & 0 &  1&  0\\ 
\hline
\end{tabular}
\caption{Example of the structured table for  DJ problem in  distributed scenario with four subfunctions}\label{Tab_Example of a structured table for a DJ problem in a distributed scenario with four subfunctions}
\end{center}
\end{table}

For  Boolean function $f$, it is clear that
\begin{equation}
\begin{split}
M'%=&|\{u\in\{0,1\}^{n-2}|f(u00)\oplus f(u01)\oplus f(u10)\oplus f(u11)=0\}|\\
=&|\{u\in\{0,1\}^{2}|f_{00}(u)\oplus f_{01}(u) \oplus f_{10}(u)\oplus f_{11}(u)=0\}|\\
=&4.
\end{split}
\end{equation}

For  Boolean function $f$, running Algorithm \ref{algorithmDDJerror}, according to equation (\ref{algorithmDDJerrorprobability}), yields the probability of measuring the first $n-2$ qubits of  $|\varphi'_6\rangle$ with the result of $0^{n-2}$ is
\begin{equation}\label{algorithmDDJerrorprobabilityexample}
\begin{split}
&\left|\frac{1}{\sqrt{2^{2}\cdot 4}}\sum\limits_{\substack{u\in\{0,1\}^{2}\\ f_{00}(u)\oplus f_{01}(u) \oplus f_{10}(u)\oplus f_{11}(u)=0}}(-1)^{f_{00}(u)\oplus f_{01}(u)}\right|^2\\
=&\left|\frac{1}{\sqrt{2^{2}\cdot 4}}\left[(-1)^1+(-1)^0+(-1)^1+(-1)^1\right]\right|^2\\
=&\frac{1}{4}.
\end{split}
\end{equation}

Obviously,  $f$ is  balanced. However, by equation (\ref{algorithmDDJerrorprobabilityexample}), it follows that Algorithm \ref{algorithmDDJerror}  will output $f$ is constant with probability $\frac{1}{4}$, which is wrong.

\end{example}






\end{document}


