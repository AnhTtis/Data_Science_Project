  
%%%%%%%%%%%%%%%%%%%%%%%%%%%%%%%%%%%%%%%%%%%%%%%%%%%%%%
%
% COMPLETE VERSION 
%
%%%%%%%%%%%%%%%%%%%%%%%%%%%%%%%%%%%%%%%%%%%%%%%%%%%%%%
\documentclass[11pt]{amsart}
\usepackage[utf8]{inputenc}
\usepackage{fullpage}
\usepackage{enumerate}
%\usepackage{enumitem}
\usepackage[bbgreekl]{mathbbol}
\usepackage{amsmath,amssymb}
%\usepackage{authblk}\
\usepackage{color}
\usepackage{marginnote}
\usepackage[linesnumbered,vlined,ruled]{algorithm2e}
\usepackage{graphicx}
\usepackage{tikz}
\usepackage{dsfont}
\usepackage{graphicx}
\usepackage{placeins}
\usepackage{xcolor}
\usepackage{enumerate}
\usepackage[shortlabels]{enumitem}

\usepackage{bbm}

\usepackage{qsymbols}
\usepackage{graphicx}
\usepackage{placeins}
\usepackage{float}
\usepackage{subfigure}
\usepackage{parskip}


%\usepackage[bbgreekl]{mathbbol}
%\usepackage[bbgreekl]{mathbbol}
%\usepackage{amsfonts}

%\DeclareSymbolFontAlphabet{\mathbb}{AMSb}
%\DeclareSymbolFontAlphabet{\mathbbl}{bbold}

%\usepackage{showkeys}
\newtheorem{theorem}{Theorem}
\newtheorem{corollary}{Corollary}
\newtheorem{lemma}{Lemma}
\newtheorem{proposition}{Proposition}
\newtheorem{definition}{Definition}
\newtheorem{claim}{Claim}
\newtheorem{remark}{Remark}
%\newcommand{\qed}{\hfill $\Box$ \medbreak}
%\newenvironment{proof}{\noindent {\bf Proof.}}{\qed}
\newcommand{\mar}[1]{{\color{blue}#1}}
\newenvironment{proofsketch}{\noindent {\bf Proof sketch.}}{\qed}
\newenvironment{proofofclaim}{\noindent {\emph{Proof.}}}{\hfill $\diamond$ \medbreak}
\newenvironment{proofp}{\noindent {\bf Proof of Proposition 2.}}{\qed}
\newenvironment{prooft}{\noindent {\bf Proof of Theorem 2.}}{\qed}

\newcommand{\cupdot}{\mathbin{\mathaccent\cdot\cup}}
\newcommand{\dMAM}{\textsf{dMAM}}
\newcommand{\old}[1]{{\color{red} #1}} 
\newcommand{\id}{\mbox{\rm id}}
\newcommand{\E}{\mathbf{E}}
\newcommand{\Pc}{{\mathcal P}}
\newcommand{\X}{{\mathcal X}}
\newcommand{\HR}{{\mathcal H}}
\newcommand{\N}{{\mathbb N}}
\newcommand{\R}{{\mathbb R}}
\newcommand{\Z}{{\mathbb Z}}
\newcommand{\V}{{\mathbb V}}
\newcommand{\D}{{\mathbb D}}
\newcommand{\Cc}{{\mathcal C}}
\newcommand{\Sc}{{\mathcal S}}
\newcommand{\Congest}{{\rm \textsf{CONGEST}}}
\newcommand{\Local}{{\rm \textsf{LOCAL}}}
\newcommand{\seq}{{\rm \textsf{Seq}}}
\newcommand{\dist}{{\rm \textsf{Dist}}}
\newcommand{\ppr}{p^{1-\frac{\log \log \log p + 4}{\log \log p}}}
\newcommand{\Forb}{\mathrm{Forb}}
\newcommand{\supp}{\mathrm{supp}} 
\newcommand{\1}{\mathds{1}}
\newcommand{\B}{{\rho}}
\newcommand{\tz}{{z'}}
\newcommand{\ttz}{{z''}}
\DeclareMathOperator{\Deg}{Deg}

%\newtheorem{example}{Example}


%%%%%%%%%%%%%%%%%%%%%%%%%%%%%%%%%%%%%%%%%%%%%%%%%%%%%%
\begin{document}
%%%%%%%%%%%%%%%%%%%%%%%%%%%%%%%%%%%%%%%%%%%%%%%%%%%%%%

\title[Criteria for entropic curvature]{Criteria for entropic curvature \\ on graph spaces}
\author{Martin Rapaport, Paul-Marie Samson}
\thanks{This research is partly funded by the B\'ezout Labex, funded by ANR, reference ANR-10-LABX-58. The  author is supported by a grant of the Simone and Cino Del Duca Foundation.}
\address{Univ Gustave Eiffel, UPEM, Univ Paris Est Creteil, CNRS,
F-77447 Marne-la-Vall\'ee, France}
\address{Univ Gustave Eiffel, UPEM, Univ Paris Est Creteil, CNRS,
F-77447 Marne-la-Vall\'ee, France}
\email{ martin.rapaport@univ-eiffel.fr, paul-marie.samson@univ-eiffel.fr}
\keywords{Displacement convexity property, entropic curvature, Ricci curvature,  optimal transport, graphs, lattices, discrete spaces, Schr\"odinger bridges, Bonnet-Myers theorem, transport-entropy inequalities, Poincar\'e inequality, logarithmic-Sobolev inequality, Prékopa-Leindler inequalities, Ising model, discrete hypercube}
\subjclass{60E15, 32F32 and 39A12}
\date{\today}
\begin{abstract} This paper presents local criteria for lower bounds on  entropic curvature of \textit{graph spaces} along Schr\"odinger bridges at zero temperature, according to the definition given by the second named author in \cite{Sam21}, in the continuity of the work by C. L\'eonard \cite{Leo17} inspired by the Lott-Sturm-Villani theory.

A \textit{graph space} is {defined as} a quadruple $(\X,d,L,m)$ where $\X$ is the set of vertices of a graph, $d$ is the combinatorial distance of the graph and $m$ is a reversible reference measure with respect to a generator $L$ of a Markov semi-group. The criteria are given by local optimization problems on balls of radius two,  depending only on the generator $L$ and on the discrete structure of these balls. General tensorization properties of the criteria are  presented for the study of {the Cartesian } product of graphs.

 This approach is robust since it applies to a wide range of graph spaces and also for any measure $m$, including measures with interaction potential like Ising Models. We introduce a very large class of graph{s},  named {\it structured graphs}, for which the local criteria give non-negative entropic curvature for the uniform measure.  A Bonnet-Myers type of theorem also ensures that any such graph with positive entropic curvature is finite.
 
 On any \textit{graph space}  $(\X,d,L,m)$ with $m(\X)<\infty$, positive entropic curvature provide{s}  transport-entropy inequalities with  so-called weak optimal transport costs, as well as   Poincaré or modified logarithmic-Sobolev inequalities  for the renormalized probability measure  $\mu=m/m(\X)$. These functional inequalities are {well} known to be related to refined concentration properties of the measure $\mu$, speed of convergence of semi-groups  to the measure $\mu$ or bounds on its mixing time. For certain specific graph spaces, the local criteria are optimal and imply sharp functional inequalities.

  We also present examples of graphs with  negative curvature. Some comparisons of our results with other notions of curvature are established, such as  %Erbar-Maas entropic curvature \cite{EM12}, 
  Bakry-Emery curvature conditions \cite{CKL21,CLP20}, Ollivier or Lin-Lu-Yau's curvature \cite{Oll09,LLY11}.
\end{abstract}
\maketitle
%\author[1]{Martin Rapaport \thanks{Additional support from ANR project ..., and from INRIA project ...}}
%\author[2]{Paul-Marie Samson }



%\affil[1,2]{LAMA UMR8050, Universit{\'e} Gustave-Eiffel, France}
%\affil[2]{Universit{\'e} Gustave-Eiffel, France}

%\date{\today}
%\address{P.-M. Samson, LAMA, Univ Gustave Eiffel, UPEM, Univ Paris Est Creteil, CNRS,
%F-77447 Marne-la-Vall\'ee, France}
%\email{ paul-marie.samson@univ-eiffel.fr}
%\keywords{Displacement convexity property, Ricci curvature, graphs, Schr\"odinger bridges, transport-entropy inequalities, concentration of measure, Poincar\'e inequality, logarithmic-Sobolev inequality}
%\subjclass{60E15, 32F32 and 39A12}
%\thanks{This research is partly funded by the B\'ezout Labex, funded by ANR, reference ANR-10-LABX-58. The  author is supported by a grant of the Simone and Cino Del Duca Foundation.}
%\maketitle




 
 %%%%%%%%%%%%%%%%%%%%%%%%%%%%%%%%%%%%%%%%%%%%%


 
 
\section{Introduction}
Let us present the framework of this research, in keeping with the seminal papers  \cite{Leo17,Sam21}. Let $\X$ be the set of vertices of a connected undirected graph $G=(\X,E)$ where $E\subset \X\times \X$ denotes the set of edges, without multiple edges and without loops. Two vertices $x$ and $y$ are {\it neighbours} if $(x,y)\in E$, we  write $x\sim y$ in this case. Let $d$ denote the combinatorial graph distance, so that $d(x,y)=1$ if and only if  $x\sim y$.
The graph $G$ is supposed to be \textit{locally finite}, that is, the \textit{vertex degree} $\text{Deg}(x):=\sum_{y\sim x} 1$ of any $x\in \mathcal{X}$ is finite. The \textit{maximal degree} of the graph is denoted by   $\text{Deg}_{\text{max}}:=\sup_{x\in \mathcal{X}}\text{Deg}(x)\in \N\cup\{+\infty\}$. A {\it discrete geodesic path} $\alpha$ joining two  vertices $x$ and $y$ is a sequence of  neighbours of minimal size $d=d(x,y)$:  $\alpha=(z_0,\ldots,z_d)$ with $z_0=x$ and $z_d=y$ and for any $i\in[d]:=\{1,\cdots ,d\}$, $(z_{i-1},z_i)\in E$. In the sequel,  $z\in\alpha$ means that there exists $i\in\{0,\ldots, d\}$ such that $z=z_i$, and  $(z,w)\in \alpha$ means that there exists $0\leq i<j\leq \ell$ such that $z=z_i$ and $w=z_j$. Let $G(x,y)$ be the set of all geodesic paths joining $x$ to $y$, and let   $[x,y]$ be the set of all vertices that belong to a geodesic from $x$ to $y$,
\[[x,y]:=\big\{z\in \X\,\big|\, z\in \alpha,\alpha\in G(x,y)\big\},\]  
$]x,y[:=[x,y]\setminus\{x,y\}$ and $[x,y[:=[x,y]\setminus \{y\}$. More generally, given two subsets  $A$ and $B$ of $\X$, one defines
\[[A,B]:=\bigcup_{x\in A,y\in B} [x,y]\quad \mbox{and} \quad ]A,B[:=\bigcup_{x\in A,y\in B} ]x,y[.\]

The set $\X$ is endowed with the $\sigma$-algebra generated by singletons. Let $\mathcal{P}(\X)$ denote the set of probabilities on $\X$, and $\mathcal{P}_b(\X)$ the subset of probabilities  with bounded support. The subset of probability measures $\mu$ satisfying $\int d(x_0,y) d\mu(y)<\infty$ for some $x_0\in\X$, denoted by $\Pc_1(\X)$, can be   endowed with the $W_1$-Wasserstein distance defined as usual by 
\begin{eqnarray}\label{defW1}
W_1(\nu_0,\nu_1):=\inf_{\pi\in \Pi(\nu_0,\nu_1)}\iint d(x,y)\,d\pi(x,y), \qquad \nu_0,\nu_1\in \Pc_1(\X),
\end{eqnarray}
where 
$\Pi(\nu_0,\nu_1)$ is the set of probability measures on the product space $\X\times \X$ with first marginal $\nu_0$ and second marginal $\nu_1$. We call {\it $W_1$-optimal coupling} of $\nu_0$ and $\nu_1$ any coupling $\pi\in\Pi(\nu_0,\nu_1)$ that achieves the infimum in \eqref{defW1}. For any non negative measure $M$ on a measurable space $\mathcal Y$, $\supp(M)$ denotes the support of this measure.  For further use, note that any measure $\pi\in\Pi(\nu_0,\nu_1)$ admits the two  following decompositions : for any $(x,y)\in \supp(\pi)$, 
 \[ \pi(x,y)=\nu_0(x)\,{\pi}_{_\rightarrow}(y|x)= \nu_1(y)\,{\pi}_{_\leftarrow}(x|y),\]
 defining thus two Markov kernels ${\pi}_{_\rightarrow}$ and ${\pi}_{_\leftarrow}$. 

On a discrete space $\X$, recall that any generator $L$ acting on functions from $\X$ to $\R$ is entirely given by the jump rates from $x\in \X$ to $y\in \X$ denoted by $L(x,y)$, $L(x,y):=L\delta_y(x)$ with $\delta_y(y)=1$ and $\delta_y(z)=0$ for $z\neq y$.

In this paper, we call \textit{graph space} any locally finite graph $G$ as above endowed with a reference measure $m$ on $\X$ and a generator $L$ satisfying the following two conditions :
\begin{itemize}
\item[$\cdot$] The measure $m$ is reversible with respect to $L$,  
$m(x)L(x,y)=m(y)L(y,x)$ for any $x,y\in\X$.
\item[$\cdot$] 
For any $x,y\in\X$, one has 
\begin{equation}\label{PropL}
L(x,y)>0 \quad\mbox{ if and only if }\quad  d(x,y)=1,
\end{equation}
(and $ L(x,x):=- \sum_{y\in \X, y\neq x} L(x,y)$). 
\end{itemize}
By definition, for any $x,y\in \X$ and $k\in \N^*$, one denotes  \[L^k(x,y):=\sum_{z_1, z_2, \ldots,z_{k-1}\in\X} L(x, z_1)L(z_1,z_2)\cdots L(z_{k-1},y).\] The property \eqref{PropL} ensures that for any $x,y\in \X$,
\[ L^{d(x,y)}(x,y)=\sum_{\alpha\in G(x,y)} L(\alpha), \quad \mbox{where} \quad L(\alpha):= L(z_0, z_1)\cdots L(z_{d(x,y)-1},z_{d(x,y)}),\]
for any $\alpha=(z_0,z_1,\ldots, z_{d(x,y)})\in G(x,y)$.

As defined in \cite{Sam21}, given  $\nu_0,\nu_1\in \mathcal{P}_b(\X)$, {\it a Schr\"odinger bridge at zero temperature}, denoted by $(\widehat \nu_t)_{t\in [0,1]}$ in the present paper, is a particular $W_1$ constant speed geodesic between $\nu_0$ and $\nu_1$ on $\Pc(\X)$, namely,  $ \widehat \nu_0= \nu_0$, $ \widehat \nu_1= \nu_1$, and for any $0\leq s\leq t\leq 1$,
 \[W_1\big(\widehat \nu_t,\widehat \nu_s\big)=(t-s) W_1(\nu_0,\nu_1).\]
Such a  path is obtained from a mixture of Schr\"odinger bridges,  by a slowing down procedure as a temperature term goes to zero due to C. Léonard \cite[Theorem 2.1]{Leo16} (see also \cite{Sam21}). These geodesic paths are mixture of $W_1$-constant speed geodesics $\nu_t^{x,y}$ between the Dirac measures $\delta_x$ at $x\in \supp(\nu_0)$ and $\delta_y$ at $y\in\supp(\nu_1)$, according to a coupling $\widehat \pi\in \Pi(\nu_0,\nu_1)$. Observe that given  bounded marginals $\nu_0$ and $\nu_1$, the L\'eonard slowing down procedure selects a single  coupling $\widehat \pi\in \Pi(\nu_0,\nu_1)$ if some conditions are satisfied on the underlying space \cite[Result 0.3]{Leo16}.  The main property of $\widehat \pi$ is to be a $W_1$-optimal coupling.  As explained in \cite{Sam21}, the structure of Schr\"odinger bridges at zero temperature that we also consider in this paper is the following: 
for any $z\in \X$
\begin{equation}\label{defhatnut}
\widehat \nu_t(z):=\sum_{x,y\in \X} \nu_t^{x,y}(z) \,\widehat \pi(x,y),
\end{equation}
with for any $x,y\in \X$,
\begin{equation}\label{pathdirac}
\nu_t^{x,y}(z):=
\1_{[x,y]}(z)\, r(x,z,z,y)\,\B_t^{d(x,y)}(d(x,z)),
\end{equation}
where for $x,z,v,y\in\X$,
\begin{equation}\label{defrpont}
 r(x,z,v,y):= \frac{L^{d(x,z)}(x,z) L^{d(v,y)}(v,y)}{L^{d(x,y)}(x,y)},
 \end{equation}
and $\B_t^d$ denotes the binomial law with parameter $t\in [0,1]$, $d\in \N$ :
\[ \B_t^d(k):= \binom{d}{k}\, t^k(1-t)^{d-k},\quad k\in\{0,\ldots,d\},\]
 with the binomial coefficient $\binom{d}{k}:=\frac{d!}{k!(d-k)!}$.
 All along the paper one omits the dependence in $\nu_0$ and $\nu_1$ of $(\widehat \nu_t)_{t\in [0,1]}$ and $\widehat \pi$ to lighten the notations.

In the paper \cite{Sam21}, by analogy of the definition of entropic curvature due to Lott-Sturm-Villani on Riemannian manifold or more generally  on geodesic spaces \cite{LV09,Stu06a,Vil09}, this property is expressed on a  graph space $(\X,d,L,m)$ in terms of a convexity property of the relative entropy along  Schr\"odinger bridges at zero temperature.
By definition,  the  {\it relative entropy} of a probability measure $q$ on a measurable space $\mathcal Y$ with respect to a probability  measure   $r\in \Pc(\mathcal Y)$  is  given by 
 \[\HR(q|r):= \int_{\mathcal Y} \log(dq/dr) \,dq\qquad \in [0, \infty], \]         
if $q$ is absolutely  continuous with respect to $r$ and    $\HR(q|r):=+\infty$ otherwise. As recalled in \cite{Sam21}, this definition extends to $\sigma$-finite non-negative measures $r$ adding  weak conditions on $q\in \Pc(\mathcal Y)$, and in that case $\HR(q|r)\in (-\infty, \infty]$.
%The following general convexity property of entropy was already introduced in \cite{Sam21}. 
\begin{definition}\label{defcourb}\cite{Sam21} On the graph space $(\X,d,m,L)$, one says that  the 
  relative entropy is $C$-displa-cement convex where $C=(C_t)_{t\in[0,1]}$,  if for any probability measures $\nu_0,\nu_1\in \mathcal{P}_b(\X)$, there exists a 
  Schr\"odinger bridge at zero temperature $(\widehat \nu_t)_{t\in [0,1]}$ whose structure is given by  \eqref{defhatnut}, and  such that 
   for any $t\in(0,1)$, 
 \begin{eqnarray}\label{deplacebis}
\HR(\widehat \nu_t|m)\leq (1-t) \HR(\nu_0|m)+t \,\HR(\nu_1|m)- \frac{t(1-t)}2C_t(\widehat \pi),
\end{eqnarray}
where   $\widehat \pi$ is the $W_1$-optimal coupling  between $\nu_0$ and $\nu_1$ that appears in  \eqref{defhatnut}.
\end{definition}
Observe that such a property on graphs has been first proposed by   M. Erbar and J. Maas \cite{Maa11,EM12,EM14} where the cost $C_t(\widehat \pi)$ is replaced by $\kappa {\mathcal W}_2^2(\nu_0,\nu_1)$, with $\kappa\in \R$ and ${\mathcal W}_2$  an abstract Wasserstein distance on  $\Pc(\X)$.  In that framework, Schr\"odinger bridges at zero temperature are also replaced by a ${\mathcal W}_2$-geodesic, and the best constant $\kappa$ represents the so-called entropic curvature of the space.  Actually, the distance ${\mathcal W}_2$ is defined using a discrete type of Benamou-Brenier formula in order to  provide a Riemannian  structure for the probability space $\Pc(\X)$. This distance ${\mathcal W}_2$ is greater than $\sqrt 2 W_1$, however ${\mathcal W}_2$ can not be  expressed as a minimum of a cost  among  transference plans $\pi$ as in the definition \eqref{defW1} of $W_1$. As we will see further in this paper, one of the main advantages of the Schr\"odinger method initiated in \cite{Sam21}, is to capture such types of costs and also to be able to  introduce new interesting ones. These transport costs, also called {\it weak optimal transport costs}, appear in the literature to describe refined concentration phenomena in discrete spaces (see \cite{GRST14bis}). 

\tableofcontents
\newpage
\section{Main results for any graph space}
In this part, we focus on the convexity property \eqref{deplacebis} for very general graph spaces, which means without a particular geometric structure. We introduce a uniform local assumption on the graph space  $(\X,d,m,L)$ under which the  cost $C_t(\widehat \pi)$ can be replaced by $\kappa\, T_2(\widehat \pi)$  with 
\[ T_2(\widehat \pi):=  \iint  d(x,y)(d(x,y)-1)\, d {\widehat \pi}(x,y).\]
%where $\widehat \pi$ is the $W_1$-optimal coupling that appears in the definition \eqref{defhatnut} of the Schr\"odinger bridge at zero temperature $(\widehat \nu_t)_{t\in [0,1]}$ from $\nu_0$ to $\nu_1$.  
%As the graph is equipped with the counting measure, this assumption can be  interpreted as a local geometric property on balls of radius 2. Let us note that the \textit{Bakry-Émery curvature-dimension conditions} developed in the context of graphs \cite{CKL21,CLP20} also depends locally on the structure of balls of radius 2.  
By definition, we call  {\it entropic curvature} of the graph space $(\X,d,m,L)$ denoted by $\kappa$ the supremum of $k\in \R$ such that the $C$-displacement convexity property $\eqref{deplacebis}$ holds with $C_t=k \,T_2$. Observe that if $\kappa=+\infty$ then $\eqref{deplacebis}$ ensures that $T_2(\widehat \pi)=0$ since $H(\widehat \nu_t|m)<+\infty$ for any $\nu_0,\nu_1\in \Pc_b(\X)$. 
As a convention $\kappa T_2(\widehat \pi)=0$ if $\kappa=+\infty$. 
When $\kappa>0$ (respectively $\kappa\geq 0$), one says that the space $(\X,d,m,L)$ has positive entropic curvature (respectively non-negative entropic curvature).  
More generally, given a family of  cost functions $c=(c_t)_{t\in (0,1)}$, $c_t:\N\to \R$, we call  {\it $T_c$-entropic curvature} of the graph space $(\X,d,m,L)$ the best constant $\kappa_c\in \R\cup \{+\infty\}$ such that the $C$-displacement convexity property $\eqref{deplacebis}$ holds with $C_t=\kappa_c \,T_{c_t}$ with 
\[ T_{c_t}(\widehat \pi):=  \iint  c_t(d(x,y))\, d {\widehat \pi}(x,y).\]

Similarly, let us also introduce a definition of {\it ${\widetilde T}$-entropic curvature} as the best constant $\widetilde{\kappa}\in \R$ such that \eqref{deplacebis} holds with
$C_t= \widetilde{\kappa} \widetilde{T} \hspace{0.1cm},$
where
\[\widetilde{T}(\widehat \pi):=  \int  \left(\int d(x,w)\, d {\widehat \pi}_{_\rightarrow}(w|x)\right)^2d\nu_0(x)+ \int \left(\int d(w,y) \,d {\widehat \pi}_{_\leftarrow}(w|y)\right)^2 d\nu_1(y).\]
Let us note that as soon as $\X $ is not reduced to a singleton, there always exist $\nu_0$ and $\nu_1\in 
{\mathcal P}_b(\X)$ such that $\widetilde{T}(\widehat \pi)>0$ and therefore $\widetilde \kappa<+\infty$.

Analogously,  we also call {\it $W_1$-entropic curvature} of the graph space $(\X,d,m,L)$ the best constant $\kappa_1\in \R$ such that \eqref{deplacebis} holds with
$C_t = \kappa_1 \,W_1^2.$
By the Cauchy-Schwarz inequality and since $\widehat \pi$ is a $W_1$-optimal coupling, one has $\widetilde{T}(\widehat \pi)\geq 2\, W_1^2(\nu_0,\nu_1)$ and therefore, if $\widetilde{\kappa}\geq 0$ then one has  
\[\kappa_1\geq 2\,\widetilde{\kappa}.\]
Further in this introduction, one gives examples of graphs for which this inequality is strict.


For $z\in \X$, let $B_1(z):=\{w\in \X\,|\, d(z,w)\leq 1\}$ denotes \textit{the ball of radius one} centered at $z$, and  for $k=1$ or $k=2$  let \textit{the combinatorial sphere} $S_{k}(z)$ denotes the set of vertices at distance $k$ from $z$
\[S_k(z):=\Big\{w\in \X\,\Big|\, d(z,w)=k\Big\}.\]
Given a vertex $z\in\X$ and a 
 subset
$W\subset S_2(z)$, %and satisfying
%\begin{equation}\label{condVW}ll
%]z,W[\subset V,
%\end{equation}
one introduces a non-negative key quantity that will be used to locally lower bound the entropic curvature of the space
\begin{align}\label{defR_2}
K_L(z,W):=\sup  \Biggl\{  \sum_{\ttz\in W} L^2(z,\ttz) \prod_{\tz\in  ]z,\ttz[}\left(\frac{\alpha(\tz)}{L(z,\tz)}\right)^{\frac{2L(z,\tz)L(\tz,\ttz)}{L^2(z,\ttz)}}\Biggl| \nonumber
\qquad\qquad\qquad\qquad\\
\qquad\qquad\qquad\qquad\qquad\qquad\qquad{\alpha}=(\alpha(v))_{v\in ]z,W[}\in \mathbb{R}_{+}^{]z,W[},\sum_{v \in ]z,W[} \alpha(v)=1 \Biggr\} .
\end{align}
To simplify the notations, one  omits the dependence in $L$ and notices $K=K_L$ when there is no possible confusions.  
In this definition as in the all paper,  we use the convention that a sum indexed by an empty set is $0$. Therefore,      $K(z,W)=0$ holds if and only if $W=\emptyset$. 
One may easily check that given $z$ the quantity $K(z,W)$ is increasing in $W$. Namely,  for  $W\subset W'\subset S_2(z)$, it holds 
\begin{equation}\label{enfin}
K(z,W) \leq K(z,W')\leq  K\big(z,S_2(z)\big).
\end{equation}


For more comprehension about this quantity, consider the special case where $m=m_0$ is the counting measure. The counting measure  is reversible with respect to the generator $L_0$ defined by $L_0(x,y)=1$ if and only if  $d(x,y)=1$. Let $|A|$ denote the cardinal of any finite set $A\subset \X$.  Then \eqref{defR_2} becomes 
\begin{equation}\label{defRbis}
K_0(z,W):=K_{L_0}(z,W)=\sup_{\alpha\in \mathbb{R}_{+}^{]z,W[}} \Biggl\{ \sum_{\ttz\in W} \big|]z,\ttz[\big| \Big(\prod_{\tz\in  ]z,\ttz[} {\alpha(\tz)}\Big)^{\frac{2}{|]z,\ttz[|}}\,\Bigg|\,\sum_{v \in ]z,W[} \alpha(v)=1 \Biggr\}.
\end{equation}
Note that given a vertex $z\in \X $, the existence of edges between vertices exclusively within  $S_ {1}(z)$ or $S_{2}(z)$ respectively does not change the value of $K\big(z, S_{2}(z)\big)$.
Here is the main result of this paper that will be analyzed for a variety of graph spaces in this work. Its proof is given in Appendix B.
\begin{theorem}\label{thmprinc} Let $(\X,d,m,L)$ be a graph space. Let  \[K=K_L:=\sup_{z\in\X} K_L\big(z,S_2(z)\big) .\] 
Then  the entropic curvature $\kappa$ of the space $(\X,d,m,L)$  is bounded from below by $r=r^L:=-2\log K$ if $K>0$, and $\kappa=+\infty$ if $K=0$. 
\end{theorem}


{\bf Comments:}
\begin{enumerate}[label=(\roman*)]
\item Observe that $K=0$   if and only if  for all $z\in \X$, $S_2(z)=\emptyset$, which means that $G=(\X,E)$ is a complete graph. 
\item 
For $z\in \X$, let us define 
\begin{equation}\label{defr}
   r(z)=r^L(z):=-2 \log K\big(z,S_2(z)\big). 
\end{equation}
Theorem \ref{thmprinc} ensures that $\kappa\geq r=\min_{z\in\X} r(z)$,
and therefore the quantity
$r(z)$ can be interpreted as a local lower  bound on entropic curvature at the vertex $z$.

This quantity only depends on the structure of the ball of radius 2 centered at $z$ and the value of $L$ on this ball. Therefore, as the graph is equipped with the counting measure ($m=m_0$ and $L=L_0$), since $r(z)$ only depends on the structure of the ball $B_{2}(z)$ lower bound on these local quantities can be  interpreted as a geometric property of the balls of radius 2.% 2 which we will denote for every $z\in \X$ as $B_{2}(z)$. 

\item For more comprehension, let us present a simple necessary condition for positive curvature, $r>0$, as $m=m_0$ is the counting measure. According to \eqref{defRbis}, if for some $z_0\in \X$, there exists $z''_0\in S_2(z_0)$ such that $\big|]z_0,z''_0[\big|=1$, or equivalently $]z_0,z''_0[=\{z'_0\}$, then by choosing the function $\alpha=\delta_{z'_0}$, one gets 
$K\big(z_0,S_2(z_0)\big)\geq 1$ and therefore $r\leq 0$.
 Therefore, if the space $(\X,d,m_0,L_0)$ has positive curvature, then necessarily  for any $z,\ttz\in \X$ with $d(z,\ttz)=2$,   $\big|]z,\ttz[\big|\geq 2$.
 In other words, one needs that there is always at least 2 midpoints between two vertices at distance 2. 

 \color{black}
 


 
 \item A \textit{graph isomorphism} between $G=(V,E)$ and 
 $G=(V^\prime,E^\prime)$ is a bijection $\phi: V\rightarrow V^\prime$ such that $(u,v)\in E$ if and only if $\big(\phi(u),\phi(v)\big)\in E^\prime$. A \textit{graph automorphism} is a graph isomorphism from a graph $G$ to itself. A graph $G=(\mathcal{X},E)$ is said to be vertex-transitive if every pair of vertices is equivalent under some element of the automorphism group. Informally, it is a graph where no vertex can be distinguished from another.
 Some notable examples of vertex-transitive graphs are the  \textit{Cayley graphs} or the \textit{Petersen graph}. If a graph $G=(\mathcal{X},E)$ is vertex-transitive $B_{2}(z)$ is isomorphic to $B_{2}(z^\prime) $ for every $z,z^\prime\in \mathcal{X}$. Therefore if $G=(\mathcal{X},E)$ is vertex-transitive, for any generator $L$, the value of the constant $r$ as well as the value of the constants $r_1,\overline{r},\widetilde{r}$ (see Theorem \ref{thmprincbis})  does not depend on the chosen vertex $z\in \X$.

 \end{enumerate}


\color{black}

If $r(z)> 0$ for all $z\in \X$, then the graph space $(\X,d,m,L)$ has non-negative entropic curvature, i.e. $\kappa\geq 0$, and the following result states that the space has also non-negative $\widetilde{T}$ and $W_1$-entropic curvature .   %Since $\kappa_1\geq 2\,\widetilde{\kappa}$, one has $\kappa_1\geq r \geq 0$ \textcolor{blue}{Doute-lien}. This non negative lower-bound on the $W_1$-entropic curvature can be actually improved using the fact that the coupling measure $\widehat{\pi}$ in the definition  \eqref{defhatnut} of the Schr\"odinger bridge at zero temperature, is a $W_1$-optimal coupling. 
%It is known that a coupling $\widehat{\pi}$ is a $W_1$-optimal coupling if and only if its support $\widehat{S}:=\supp(\widehat{\pi})$ is a $d$-cyclically monotone (see \cite[Theorem 5.10]{Vil09}). This property provides a specific  structure for the Schr\"odinger bridge at zero temperature as recalled at the beginning of the proof of Theorem \ref{thmprinc} in Appendix. 
%Actually for $r\geq 0$, Theorem \ref{thmprinc} can be complemented as follows.

\begin{theorem}\label{thmprincbis} Let  $(\X,d,m,L)$ be a graph space and such that for all $z\in \X$, $K\big(z,S_2(z)\big)<1$. 
\begin{enumerate}[label=(\roman*)]
\item The $\widetilde{T}$-entropic curvature $\widetilde{\kappa}$ of  $(\X,d,m,L)$ is   bounded from below by $\widetilde{r}= \widetilde{r}^L:=1-K$.
\item The $W_1$-entropic curvature $\kappa_1$ of  $(\X,d,m,L)$ is lower bounded by $4r_1$ with $r_1= r_1^L:=\inf_{z\in \X} r_1(z)$ and 
\begin{equation}\label{defr_1}
r_1(z)=r_1^L(z):=\bigg(\sup_{V_+, V_-,W_+,W_-} \Big\{ \frac{\1_{V_+\neq \emptyset}   }{1-K(z,W_+)}+\frac{\1_{V_-\neq \emptyset}}{1-K(z,W_-)} \Big\}\bigg)^{-1},
\end{equation}
where the supremum runs over all subsets $V_+,V_-\subset S_1(z)$, $(V_+,V_-)\neq (\emptyset,\emptyset)$, and all subsets $W_+,W_-\subset S_2(z)$ with $V_+\supset]z,W_+[$ and $V_-\supset]z,W_-[$ 
satisfying 
\begin{equation}\label{condsupl}
\forall x\in  V_-\cup W_-,\quad \forall y \in  V_+\cup W_+,\qquad z\in [x,y].
\end{equation}
\item For any $t\in(0,1)$ and any integer $d$ let
\begin{equation}\label{defvt}
  u_t(d)
:=\frac{d(d-1)}2\Big[\1_{d=2}+\1_{d=3}\Big]+\frac{\1_{d\geq 4}}2 \Big[d(d-1) +
\sum_{k=2}^{d-2} 
\sqrt{k(k-1)(d-k)(d-k-1)}\frac{\rho_t^d(k)}{t(1-t)}\Big] 
\end{equation}
and 
\begin{equation}\label{defcout2}
\overline{c}_t(d):=\int_0^1 u_s(d)\,q_t(s)\,ds,
\end{equation}
where $q_t$ is the kernel on $[0,1]$ defined by \[q_t(s)=\frac{2s}t \1_{[0,t]}(s)+ \frac{2(1-s)}{1-t} \1_{[t,1]}(s),\qquad s\in[0,1].\] 
%\[c_2(d):= \max\left[d(d-2\log d-2), \frac{d(d-1)}2\right].\]
For $\overline{c}=(\overline{c}_t)_{t\in(0,1)}$ the $T_{\overline{c}}$-entropic curvature $\kappa_{\overline{c}}$ of  $(\X,d,m,L)$ is lower bounded by $ 4\overline{r}$, where
 $\overline{r}=\overline{r}^L:=\inf_{z\in \X} \overline{r}(z)$ with $\overline{r}(z)=+\infty$ if $S_2(z)=\emptyset$ and otherwise 
\begin{equation}\label{defr_2}
\overline{r}(z)=\overline{r}^L(z):=\bigg(\sup_{W_+,W_-} \Big\{ \frac{\1_{W_+\neq \emptyset}   }{-\log K(z,W_+)}+\frac{\1_{W_-\neq \emptyset}}{-\log K(z,W_-)} \Big\}\bigg)^{-1},
\end{equation}
where the supremum runs over all subsets $W_+,W_-\subset S_2(z)$, $(W_+,W_-)\neq (\emptyset,\emptyset)$,
satisfying 
\begin{equation}\label{condsuplbis}
\forall x\in  W_-,\quad \forall y \in   W_+,\qquad z\in [x,y].
\end{equation}



\end{enumerate}
\end{theorem}
The proof of this result is postponed in Appendix B.
\newpage
{\bf Comments:}
\begin{enumerate}[label=(\roman*)]
\item Since $\kappa_1\geq 2\,\widetilde{\kappa}$, the first result of this theorem implies $\kappa_1\geq 2(1-K) \geq 0$. This non negative lower-bound on $\kappa_1$ is  actually improved by  $4r_1$. Indeed, the monotonicity property \eqref{enfin} implies  
\[0<2\big(1-K(z,S_2(z))\big) =4 \left(\frac1{1-K(z,S_2(z))}+ \frac1{1-K(z,S_2(z))}\right)^{-1}\leq 4 r_1(z)\leq 4\big(1-K(z,S_2(z))\big),\]
and therefore $0\leq 2(1-e^{-r/2})\leq 4r_1\leq 4(1-e^{-r/2}) \leq  2r$. 
As exposed in the next section, for the discrete hypercube $\{0,1\}^n$ %and for the complete graph, 
$\kappa_1=4r_1=4(1-K)$ is  the optimal lower bound for the $W_1$-entropic curvature, and  $\widetilde{\kappa}=1-K$ is the optimal $\widetilde{T}$-entropic curvature too.

The proof of $\kappa_1\geq 4r_1$ uses the fact that the coupling measure $\widehat{\pi}$ in the definition  \eqref{defhatnut} %of the Schr\"odinger bridge at zero temperature, 
is a $W_1$-optimal coupling. 
It is known that a coupling $\widehat{\pi}$ is a $W_1$-optimal coupling if and only if its support $\widehat{S}:=\supp(\widehat{\pi})$ is a $d$-cyclically monotone (see \cite[Theorem 5.10]{Vil09}). This property provides a specific  structure of the Schr\"odinger bridge at zero temperature as recalled at the beginning of the proof of Theorem \ref{thmprinc} in Appendix, which allows to reach the lower bound $4r_1$. This argument  is also a key element of the proof of  $\kappa_{\overline{c}}\geq  4\overline{r}$.



\item  \color{black} Choosing $x\in V_-$ and $y\in V_+$, condition \eqref{condsupl} implies $d(x,y)= 2$ since $z\in [x,y]$ and therefore $d(V_-,V_+)= 2$. Analogously, one has $d(W_-,W_+)=4$ and $d(V_-,W_+)=d(V_+,W_-)= 3$ if $V_-,V_+,W_-,W_+$ are not empty sets. It follows that $V_+\cap V_-=\emptyset$ and $W_+\cap W_-=\emptyset$. This observation also holds considering condition \eqref{condsuplbis}, namely for $W_-\neq \emptyset$ and $W_+\neq \emptyset$, $d(W_-,W_+)=4$, $W_-\cap W_+=\emptyset$, but also $d\big(]z,W_-[,]z,W_+[\big)=2$, $]z,W_-[\cap ]z,W_+[=\emptyset$.
\item About the last result of Theorem \ref{thmprincbis}, let us first give estimates of the family of cost functions $\overline{c}=(\overline{c}_t)_{t\in(0,1)}$. 
Since
\[\sum_{k=2}^{d-2} 
{k(d-k)}\frac{\rho_t^d(k)}{t(1-t)}= \sum_{k=0}^{d} 
{k(d-k)}\frac{\rho_t^d(k)}{t(1-t)}-(d-1)\big(t^{d-2}+(1-t)^{d-2}\big)\leq d(d-1),\]
%\[\sum_{k=2}^{d-2} 
%{k(d-k)}\frac{\rho_t^d(k)}{t(1-t)}=d(d-1)-d \big[t^{d-2}+(1-t)^{d-2}\leq d(d-\big],\]
one easily checks that for any integer $d$,
\[u_t(d)\leq d(d-1)\quad \mbox{and therefore}\quad \overline{c}_t(d)\leq d(d-1).\]
Easy computations also give 
\[\sum_{k=2}^{d-2} 
{(k-1)(d-k-1)}\frac{\rho_t^d(k)}{t(1-t)}=d(d-1)-(d-1)\frac {1-t^{d}-(1-t)^{d}}{t(1-t)},\]
that implies 
\[u_t(d)\geq \frac{d(d-1)}2\Big[\1_{d=2}+\1_{d=3}\Big]+\1_{d\geq 4} \Big[d(d-1)-(d-1)\,\frac {1-t^{d}-(1-t)^{d}}{2t(1-t)} \Big]\geq \frac{d(d-1)}2.\]  
As a consequence for $d\geq 4$, one has
\[\overline{c}_t(d)\geq d(d-1)-\frac{d-1}2\int_0^1\gamma_s(d)\, q_t(s) ds,\]
with for $d\geq 2$
\[\gamma_s(d):=\frac {1-s^{d}-(1-s)^{d}}{s(1-s)}=\sum_{k=0}^{d-2} \big((1-s)^k+s^k\big).\]
Since 
\begin{align}\label{subtile}
\nonumber\int_0^1\gamma_s(d)\, q_t(s)\, ds&=2\sum_{k=0}^{d-2}\Big(\frac1{k+1}-\frac1d\Big)\big(t^k+(1-t)^k\big)\\
&\nonumber\leq 2\sum_{k=0}^{d-2} \frac1{k+1}=2\1_{d=2}+ 3\1_{d=3} + \1_{d\geq 4} \Big(3+2\sum_{k=2}^{d-2} \frac1{k+1}\Big)\\
&\leq 2\1_{d=2}+ 3\1_{d=3} + \1_{d\geq 4} \Big(3+2\log\frac{d-2}2\Big),
\end{align}
we get for any $t\in(0,1)$, $\overline{c}_t\geq \overline{c}_*$ where the cost function $\overline{c}_*$ is given by
\begin{equation}\label{defcbar}
  \overline{c}_*(d):=\frac{d(d-1)}2\1_{d<4}+\1_{d\geq 4} \Big[d(d-1)-(d-1)\Big(\frac{3}{2} +\log\frac{d-2}{2}\Big)\Big], \quad d\in \N.  
\end{equation}
As a main property for the cost $\overline{c}_*$, for large values of $d$, $\overline{c}_*(d)$ is of order $d^2$, and for any $d\in\N$, $\overline{c}_*(d)\geq \frac{d(d-1)}2$ (which follows from the inequality $\log u\leq u-1, u>0$).


\item Concerning the constant $\overline{r}$, the monotonicity property \eqref{enfin} also gives \[0<-2\log K(z,S_2(z)) =4 \left(\frac1{-\log K(z,S_2(z))}+ \frac1{-\log K(z,S_2(z))}\right)^{-1}\leq 4 \overline{r}(z)\leq -4\log K(z,S_2(z)),\]
and therefore $0\leq r\leq 4\overline{r}\leq 2r$. As we will see, on the discrete hypercube $\{0,1\}^n$,  $4\overline{r}$ and $2r$ are both of order $4/n$ for large $n$. Moreover, due to the central limit theorem, 
 the lower bound $4\overline{r}$ of the $ T_{\overline{c}}$-entropic curvature is   asymptotically sharp in $n$. On the discrete hypercube, we also have $\overline{r}=1/n$ which is similar to $\overline{r}$. Actually in many cases  $r_1$ and $\overline{r}$ are of the same order since  the inequality $\log u\leq u-1, u>0,$ provides
 \[\overline{r}(z)\geq \bigg(\sup_{W_+,W_-} \Big\{ \frac{\1_{W_+\neq \emptyset}   }{1- K(z,W_+)}+\frac{\1_{W_-\neq \emptyset}}{1- K(z,W_-)} \Big\}\bigg)^{-1}:=\overline{r}'(z),\]
 The expression of $\overline{r}'(z)$ on the right-hand side is very similar to the one of $r_1(z)$.
Actually using the concavity of the function $g:x\in]1,+\infty]\mapsto \Big[-\log(1-1/x)\Big]^{-1}$, the last inequality can be improved, namely
\[\overline{r}(z)\geq -\frac12 \log\big(1-2\,\overline{r}'(z)\big).\]
\end{enumerate}
 
Applying Theorem 2.1 of \cite{Sam21}, one gets the following curved Prékopa-Leindler type of inequality on discrete spaces with entropic curvature bounded from below. 
\begin{theorem}\label{PL} Let $(\mathcal{X},d,m,L)$ be a graph space. Given a family of  cost functions $c=(c_t)_{t\in (0,1)}$, $c_t:\N\to \R$, assume that the  $T_c$-entropic curvature $\kappa_c$ of the discrete space $(\X,d,m,L)$ is bounded from below, $\kappa_c>-\infty$.  
 If $f,g,h$ are real functions on $\mathcal{X}$ satisfying for all $x,y\in \X$,
\begin{equation*}
(1-t)f(x)+tg(y)\leq \int h\, d \nu_t^{x,y}+\frac{\kappa_c}2\,t(1-t)\,c\big(d(x,y)\big),
\end{equation*}
then 
\begin{equation*}
\Big( \int e^{f} dm\Big)^{1-t} \Big(\int e^{g}dm \Big)^{t}\leq \int e^{h} dm \hspace{0.1cm} . 
\end{equation*}
\end{theorem}
\noindent According to Theorem \ref{thmprinc}, this result applies replacing  $\kappa_{c}$ by $r=-2\log K$, and with $c_t(d)=d(d-1), d\in \N$ for any $t\in(0,1)$.
Let us note that $\kappa_{c}$ does not need to be positive. According to Theorem \ref{thmprincbis}, it also applies if $r(z)> 0$ for all $z\in \X$ replacing $\kappa_{c}$ by $4\overline{r}$ and with the family of cost functions $\overline{c}=(\overline{c}_t)_{t\in (0,1)}$ given by \eqref{defcout2}.  

%\color{red} PROPRIETE DE CONVEXITE A RETRAVAILLER SUR $\{0,1\}^n$ et $Z^n$.

%It is well known that the usual Prékopa-Leindler inequality on $\R^n$ is stable under log-concave perturbation on $\R^n$ of the reference measure. Following this idea, let us say that a real function $r$ on $(\X,d,m,L)$ is $c$-semi-convex with $c\in \R$ if for any 
%$x,y\in \X$ and any $t\in (0,1)$,
%\[ \int r(z)\, d\nu_t^{x,y}(z)\leq (1-t)r(x)+t r(y)+c\,\frac{t(1-t)}{2 } \,d(x,y)\big(d(x,y)-1\big).\]
%If this property holds with $c=0$, one says that $r$ is convex. 
%As mentioned in Lemma \ref{conv} given  in Appendix, this semi-convexity property is equivalent to the following local convexity property, namely  for any $z,\ttz\in \X$ with $d(z,\ttz)=2$
%\[\sum_{\tz\in S_1(z)\cap [z,\ttz]} \left(r(\ttz)+r(z)-2r(\tz)\right) \frac{L(z,\tz)L(\tz,\ttz)}{L^2(z,\ttz)}\geq - c.\]


%Applying Theorem \ref{PL}, one easily get that if $(\mathcal{X},d,m,L)$ has entropic curvature bounded from below by  $\kappa\in \R$, and if $r$ is a $c$-semi convex function, then  for any real functions $f,g,h$ on $\mathcal{X}$ satisfying for all $x,y\in \X$,
%\begin{equation*}
%(1-t)f(x)+tg(y)\leq \int h\, d \nu_t^{x,y}+\frac{\kappa-c}2\,t(1-t)\,d(x,y)(d(x,y)-1), 
%\end{equation*}
%one has 
%\begin{equation*}
%\Big( \int e^{f} dm_r\Big)^{1-t} \Big(\int e^{g}dm_r \Big)^{t}\leq \int e^{h} dm_r \hspace{0.1cm}, 
%\end{equation*}
%where $m_r$ denotes the measure on $\X$ with density $e^{-r}$ with respect to $m$, $m_r:=e^{-r}m$.
% \color{black}
 
 As for other notions of discrete curvature such as the coarse Ricci curvature \cite[Proposition 23]{Oll09}, Lin-Lu-Yau curvature \cite[Theorem 4.1]{LLY11}, curvature-dimension conditions
 \cite[Theorem 6,3]{FS18}-\cite[Theorem 2.1]{LMP18}, and entropic curvature by Erbar-Maas \cite[Theorem 1.3]{Kam20}, one may prove a Bonnet-Myers theorem for graphs that ensures that $\X$ is finite under the hypothesis of finiteness of the maximal degree, of positive entropic curvature and of mild assumptions on the measure $m$. The proof of the next theorem is given in Appendix B.

\begin{theorem}\label{BM} Let $(\X,d,m,L)$ be a graph space. We assume that $\textnormal{Deg}_{\max}<\infty$  and that the measure $m$ is bounded and bounded away from 0 :
\begin{equation*}\label{condm}
\sup_{x\in \X} m(x)<\infty,\qquad \inf_{x\in \X} m(x)>0.
\end{equation*}
If the entropic curvature $\kappa$ of  $(\mathcal{X},d,m,L)$ is positive  then the diameter of the space $\X$ is bounded and therefore $\X$ is finite. More precisely one has  
\begin{equation*}
\textnormal{Diam}(\mathcal{X})\leq\frac{8\log \Big(\textnormal{Deg}_{\max}\, \frac{\sup_{x\in\X} m(x)}{\inf_{x\in \X}m(x)}\Big)}{\kappa}+1,
\end{equation*}
where $\textnormal{Diam}(\mathcal{X}):=\sup_{x,y} d(x,y)$.
Same type of results hold if the $\widetilde{T}$-entropic curvature or the $W_1$-entropic curvature of the space is positive.
\end{theorem}

Assume that $m(\X)<\infty$ and let $\mu:=m/m(\X)$ be the associated  normalized probability measure. %As the space $(\X,d,m,L)$ has  positive entropic curvature and satisfies the $(G)$-conditions, 
%Theorem \ref{BM} ensures that $\X$ is finite and therefore $m(\X)<\infty$. It follows that the normalized probability measure $\mu=m/m(\X)$ is well defined. Moreover, s\textnormal{Deg}_{\max} \textnormal{Deg}_{\max}
Since for any $\nu\in \Pc(\X)$, $\HR(\nu|m)=\HR(\nu|\mu)-\log m(\X)$,  the convexity property \eqref{deplacebis} holds for the measure $m$ if and only if it  holds for the probability measure $\mu$.
If $r>0$ then  Theorem \ref{thmprinc} and Theorem \ref{thmprincbis} imply that the space has positive curvature, namely $\kappa_1>0, \kappa>0$, $\widetilde\kappa>0$ and $\kappa_{\overline{c}}>0$. Since by Jensen's inequality $\HR(\widehat{\nu}_t|\mu)\geq 0$, one gets   for any $t\in (0,1)$, for any probability measures $\nu_0,\nu_1\in \mathcal{P}_b(\X)$ 
\[\frac 12 \max\Big(\kappa_1\,W_1^2(\nu_0,\nu_1), \widetilde\kappa \,\widetilde{T}(\widehat\pi) , \kappa\, {T}_2(\widehat\pi), \kappa_{\overline{c}}\, {T}_{\overline{c}_t}(\widehat\pi)\Big)\leq \frac{1}{t}\HR(\nu_0|\mu)+\frac{1}{1-t} \HR(\nu_1|\mu),\]
where $\widehat\pi\in \Pi(\nu_0,\nu_1)$ is a $W_1$-optimal coupling.
Let us define 
\[\widetilde{T}(\nu_0,\nu_1):=\inf_{\pi\in \Pi(\nu_0,\nu_1)} \widetilde{T}(\pi),\]
and  similarly $T_2(\nu_0,\nu_1)$ and $T_{\overline{c}_t}(\nu_0,\nu_1)$. Since $\overline{c}_t\geq \overline{c}_*$ for any $t\in(0,1)$, optimizing over all $t\in (0,1)$ in the above transport-entropy inequality  one gets the following result.

\begin{corollary}\label{Transport} Let   $(\X,d,m,L)$ be a graph space with positive curvature, namely  $\kappa_1>0, \kappa>0$, $\widetilde\kappa>0$ and $\kappa_{\overline{c}}>0$. If $m(\X)<\infty$ 
then the probability measure $\mu=m/m(\X)$ satisfies the following transport-entropy inequality, for any probability measures $\nu_0,\nu_1\in \mathcal{P}_b(\X)$  
\begin{multline*}
\frac{1}{2} \max\Big(\kappa_1\,W_1^2(\nu_0,\nu_1),\widetilde\kappa\,\widetilde{T}(\nu_0,\nu_1)  , \kappa\,T_2(\nu_0,\nu_1),\kappa_{\overline{c}}\, T_{\overline{c}_*}(\nu_0,\nu_1)   \Big)\ \leq \Big(\sqrt{\HR(\nu_0|\mu)}+\sqrt{\HR(\nu_1|\mu)}\Big)^2.
\end{multline*}
\end{corollary}

\begin{remark} The above transport-entropy inequalities also provide bounds on the diameter $\textnormal{Diam}(\mathcal{X})$ of the space $\X$ when the probability measure $\mu$ is bounded away from 0.
Choosing Dirac measures for $\nu_{0}$ and $\nu_{1}$, one gets 
\[\textnormal{Diam}(\mathcal{X})\leq \sqrt{-\,\frac{8\log \inf_{x\in \X} \mu(x) }{\kappa_{1}}}.\]
In some cases this upper bound is very accurate. For example, for the $n$-dimensional hypercube $\{0,1\}^{n}$ for which $\textnormal{Diam}(\mathcal{X})=n$, endowed with the uniform probability measure \textnormal{(}$\mu(x)=1/2^n$ for all $x\in\X$\textnormal{)}, the  right hand side of this inequality is $n \sqrt{2\ln(2)}$ since $\kappa_{1}=\frac{4}{n}$ \textnormal{(}as we will show in section 6.1\textnormal{)}.
\end{remark}

As the space has positive $\widetilde{T}$-entropic curvature $\widetilde{\kappa}$, following the ideas of the seminal work \cite{GRST14}, the probability measure $\mu$ also satisfies a modified logarithmic-Sobolev inequality, and therefore a discrete Poincaré inequality. 
Recall that for any positive function $f$ on $\X$,  
 the entropy of $f$ with respect to $\mu$ is given by
\[{\rm Ent}_\mu(f):=\mu( f\log f )-\mu(f)\log \mu(f)=\HR(\nu|\mu),\]
where $\mu(f):=\int f d\mu$ and $\nu$ is the probability measure on $\X$ with density $f/\mu(f)$ with respect to $\mu$.  The variance with respect to $\mu$ of any function  $f:\X\to \R$  is 
\[{\rm Var}_\mu(f)=\mu(f^2)-\mu(f)^2.\]
\begin{theorem}\label{Logsob}
Let $(\X,d,m,L)$ be a graph space, with positive ${\widetilde T}$-entropic curvature $\widetilde{\kappa}$ and such that $m(\X)<\infty$.  
Then the probability measure $\mu:=m/m(\X)$ satisfies the following modified logarithmic-Sobolev inequality, for any bounded function $f:\X\to [0,+\infty)$,
\begin{equation}\label{logsob}
{\rm Ent}_\mu(f)\leq \frac{1}{2\widetilde{\kappa}} \int \max_{x', x'\sim x}\left[\log f(x)-\log f(x')\right]_+^2 f(x)\,d\mu(x),
\end{equation}
where $[a]_+=\max(0,a)$, $a\in\R$.
It follows that  the probability measure $\mu$ also satisfies  the following Poincaré type of inequality, for any real bounded  function $g:\X\to \R$, 
\[{\rm Var}_\mu(g)\leq \frac{1}{\widetilde{\kappa}} 
\int \max_{x', x'\sim x}\left[g(x)-g(x')\right]_+^2 d\mu(x)
.\]
\end{theorem}
\noindent The proof of this result is given in Appendix B.      According to Theorem \ref{thmprincbis}, this result applies as soon as the space $(\X,d,m,L)$ satisfies $r>0$ since $\widetilde{\kappa}\geq r/2$.

{\bf Comments:}
\begin{enumerate}[label=(\roman*)]
\item
Actually, according to the proof of Theorem \ref{Logsob}, one may prove  the following improvement of the above modified modified logarithmic-Sobolev inequality : 
\begin{multline*} \label{logsob}
{\rm Ent}_\mu(f)\leq \frac{1}{2\widetilde{\kappa}} \int \max_{x', x'\sim x}\left[\log f(x)-\log f(x')\right]_+^2 f(x)\,d\mu(x)
\\-\frac{\widetilde{\kappa}}{2} \mu(f) \int \left(\int d(x,y) \,d\widehat \pi_\leftarrow(x|y) \right)^2 \,d\nu(y)
\end{multline*} 
where $\nu$ is the probability measure with density $f/\mu(f)$ with respect to $\mu$ and $\widehat \pi$ is a $W_1$-optimal coupling between $\mu$ and $\nu$.
Therefore Cauchy-Schwarz inequality also provides 
\[
{\rm Ent}_\mu(f)\leq \frac{1}{2\widetilde{\kappa}} \int \max_{x', x'\sim x}\left[\log f(x)-\log f(x')\right]_+^2 f(x)\,d\mu(x)\\
-\frac{\widetilde{\kappa}}{2} \mu(f) W_1^2(\nu,\mu).\]
%$\Delta$ on the set of real functions on $\X$ given by 
%\item \textcolor{blue}{Il me semble qu'il y a une erreur ici. J'ecris dans le suivant item, je crois qu'on etait en train de regarder le mauvais operateur }

%Let us consider the case where $m=\mu_{0}$ the counting measure with the generator $L=L_{0}$ associated and let us suppose that the space $(\X,d,m_{0},L_{0})$ has strictly positive ${\widetilde T}$ entropic curvature $\widetilde{\kappa}$. By Bonnet-Myers Theorem, we know that $\X$ is finite. Let $\mu_{0}$ denote the uniform probability measure on $\X$ and let $\lambda$ denote the \textit{spectral gap} of the graph $(\X,d)$ that is the least non zero eigenvalue of the \textit{Discrete Laplacian} operator  denoted $\Delta$ defined for any $g\in \mathbb{R}^{\X}$ as follows
%\[\Delta g(x)=\sum_{x',x'\sim x} \big(g(x)-g(x')\big).\]
%By the variational principle and by the discrete Green formula
%we have 
%\[\lambda=\inf_{g\perp 1} \frac{(\Delta g,g)}{(g,g)}= \inf_{g\perp 1} \frac{\int_{\mathcal{X}} g\Delta g\, d\mu_0}{{\rm Var}_{\mu_0}(g)}=\inf_{g\perp 1} \frac{\frac12\sum_{x,x', x'\sim x}\big(g(x)-g(x')\big)^2 \mu_0(x)}{{\rm Var}_{\mu_0}(g)}\] 

%where the inner product is defined for any two functions $f,g\in {\mathbb{R}}^\mathcal{X}$ as $(f,g):=\sum_{x\in \X} f(x)g(x) \mu_{0}(x)$.

%Easy computations give
%\[\int g\Delta g\, d\mu_0= \frac12\sum_{x,x', x'\sim x}\big(g(x)-g(x')\big)^2 \mu_0(x).\]
%As a consequence, since  
%\[\sum_{x\in \mathcal{X}} \max_{x',x'\sim x} (g(x)-g(x'))_{+}^{2}\mu_{0}(x) \leq \sum_{{x,x', x'\sim x}}  \big[g(x)-g(x')\big]_+^{2}\mu_0(x)=\int_{\mathcal{X}} g\Delta g \,d\mu_0,\] 
%Theorem \ref{Logsob} gives $\lambda\geq \widetilde{\kappa}_{2}$. 
%For many graphs  this lower estimate is not sharp because of the strong approximation of the last inequality.  
%For instance for the unnormalized Laplacian on $\{0,1\}^{n}$, $\lambda=2$ and  $\widetilde{\kappa}_{2}=\frac{1}{n}$.


%We explain below how to improve this estimate by considering another type of curvature constant for graphs with geometric structures such as  Cayley graphs.



\item Suppose that the space is equipped with the counting measure, $(\X,d,m,L)=(\X,d,m_{0},L_{0})$, and has strictly positive entropic curvature $\widetilde{\kappa}$. Bonnet-Myers Theorem \ref{BM} ensures that $\X$ is finite. Let $\mu_{0}:=m_0/m_0(\X)$  and let $\Delta$ be the unnormalized \textit{discrete Laplacian} defined for any real function $f$ on $\X$ as follows 
\[\Delta f(z)=L_{0}f(z):=\sum_{z',z'\sim z} \big(f(z')-f(z)\big).\] 
Let $\mathcal{L}$ be the operator defined as $\mathcal{L}=-\Delta$ which is positive definite. Let $\lambda_{1}$ denote the \textit{spectral gap} of the graph $(\X,d)$ that is the least non zero eigenvalue  of $\mathcal{L}$.
By the variation formula and Green formula,
\[\lambda_{1}=\inf_{\int_{\mathcal{X}} fd\mu_{0}=0}  \frac{\int_{\mathcal{X}} f \mathcal{L}fd\mu_{0}}{\int_{\mathcal{X}} f^{2}d\mu_{0}}=\inf_{\sum_{z\in \X} f(z)=0} \frac{1}{2} \frac{\sum_{z,z'\in \X} \big(f(z)-f(z')\big)^2 \mu_0(x)}{{\rm Var}_{\mu_0}(f)}\] 
%where the inner product is defined as $\langle f,g \rangle :=\sum_{z\in \X} f(z)g(z) \mu_{0}(z)$ for any two real functions $f,g$ on $\X$.
As a consequence, since  
\[\sum_{z\in \mathcal{X}} \max_{z',z'\sim x} [f(z)-f(z')]_{+}^{2}\mu_{0}(z) \leq \sum_{{z,z', z'\sim z}}  \big[f(z)-f(z')\big]_+^{2}\mu_0(z)=\frac{1}{2}\int_{\mathcal{X}} f \mathcal{L}f  \,d\mu_0,\] 
Theorem \ref{Logsob} gives $\lambda_{1}\geq 2\widetilde{\kappa}$. 
For many graphs  this lower estimate is not sharp \label{spectral1}.  For instance  on $\{0,1\}^{n}$, $\lambda_{1}=2$ and  we will prove in section \ref{sectionhypecube} that $\widetilde{\kappa}=\frac{1}{n}$. We explain below how to improve this estimate by considering another type of curvature constant for graphs with geometric structures such as  Cayley graphs.


%\textcolor{blue}{Explication: 
%$\Delta_{N}f(x)=\frac{1}{deg(x)}\sum_{y:y\sim x} f(y)-f(x)$ and 
%$Jf(x)=-\Delta_{N}f(x)=f(x)-\frac{1}{deg(x)}\sum_{y:y\sim x} f(y)$ 
%Proposition: Let $(V,^{n},E_{n})=(V,E)^{\square{n}}$ . If 
%$\{\lambda_{k}\}_{k=0}^{N-1}$ is a sequence of eigenvalues of 
%$J$ for $(V,E)$ then the eigenvalues of $J$ for $(V,^{n},E_{n})$
%are given by $\frac{\alpha_{k_{1}}+\alpha_{k_{2}}\ldots \alpha_{k_{n}}}{n}$ 
%for all $k_{i}\in \{0,1,\ldots n-1\}$ where each eigenvalue is counted with its multiplicity . 
%Since $\{0,1\}$ has eigenvalues $\lambda_{1}=0$ and $\lambda_{2}=2$ then for $\{0,1\}^{n}$ :$0$ eig with mult. 
%${n}\choose{n}$, $\frac{2}{n}$ eig with mult.${n}\choose{1}$ etc
%$2$ eig with mult ${n}\choose{n}$ So $\lambda_{2}=\frac{2}{n}$ and for unnormalized $\mathcal{L}$ it is just $2$.}

%\textcolor{blue}{Remarque:
%Si on normalise $\Delta$ on obtiendrai $\frac{2}{n}\geq \frac{1}{n}$... mais je crois qu'on peut pas normaliser car on utilise $L_{0}$}


\end{enumerate}


\section{Refined results  for structured graph spaces}
 
Let us now consider classes of graphs with more structure in order to propose refinements of the results of the last part. 
%Let us introduce \textit{structured graphs} and the so-called \textit{Ricci flat} graphs.

\begin{definition}
We say that a graph $(\X,E)$ is {\it structured} if there exists a finite set $\Sc$ of maps $\sigma:\X\to  \X$, with the following properties
\begin{enumerate}[label=(\roman*)]
    \item For any $z\in \X$ and any $\sigma\in S$, $d\big(z,\sigma(z)\big)\leq 1$.
    \item If  $z$ and $\tz$ are  neighbours in $\X$, then there exists a single $\sigma\in S$ such that $\tz=\sigma(z)$. One defines 
    \[ \Sc_z:=\big\{\sigma\in \Sc\,\big|\, \sigma(z)\sim z\big\},\]
    so that  
    \[S_1(z):=\big\{\sigma(z)\,\big|\,\sigma\in \Sc_z\big\} \quad\mbox{and}\quad|S_1(z)|=|\Sc_z|.\]
    
    \item For any $z\in \X$ and for any  $\tau\in \Sc$, setting 
    \[\Sc_z^{\tau\rightarrow \cdot}:= \big\{\sigma\in \Sc_{\tau(z)}\,\big|\,d\big(z,\sigma(\tau(z))\big)=2\big\}\quad \mbox{and}\quad \Sc_z^{\cdot\rightarrow \tau}:= \big\{\sigma\in \Sc_{z}\,\big|\,d\big(z,\tau(\sigma(z))\big)=2\big\},\]
    $\Sc_z^{\tau\rightarrow \cdot}$ is empty if and only if $\Sc_z^{\cdot\rightarrow \tau}$ is empty.
    \item For any $z\in \X$ and for any  $\tau\in \Sc$, if $\Sc_z^{\cdot\rightarrow \tau}\neq \emptyset$ then there exists a one to one map $\psi:\Sc_z^{\cdot\rightarrow \tau}\to \Sc_z^{\tau\rightarrow \cdot} $ such that for all $\sigma\in \Sc_z^{\cdot\rightarrow \tau}$,
    \[ \tau(\sigma(z))=\psi(\sigma)(\tau(z)).\]
\end{enumerate}
\end{definition}
%\textcolor{blue}{Ne faudrait pas t-il demontrer comme c'est fait pour les ricci flat que le produit de deux structured graphs est un structured graph? }

Let us introduce some paradigmatic examples of structured graphs.\

{\it Example 1 (Cayley graph).}\label{ex1} 
 Let $(G,\ast)$ be a finite group and let $\Sc$ be a subset of generators of the group (which does not contain the neutral element of the group denoted by $e$ in order to avoid loops in the consequent graph).  The group $G$ and a subset $\Sc$ determine a \textit{Cayley graph} $(\X,E)$ as follows: $\X=G$ and $x\sim y$ for $x,y\in \X$ if and only if $y=x\ast s$ for some $s\in \Sc$. Let us consider Cayley graphs that satisfy certain conditions: $\Sc=\Sc_{e}=\Sc_{g}$ for all $g\in G$ and   for all $g,h\in \Sc$, $ghg^{-1}\in \Sc$ . Then $(G,\ast)$ is a structured graph with set of moves $\Sc$. Indeed, the first three axioms are obviously satisfied and $\psi(h):=g\ast h\ast g^{-1}\in \Sc $ is a one to one map and satisfies for any $l\in G$, $g\ast h\ast l=\psi(h)\ast g\ast l$. The next three examples can be seen as Cayley graphs that satisfy the above hypotheses.\
%For instance, the transposition model on the \textit{symmetric group} $S_{n}$ is a structured graph. 

{\it Example 2.}\label{ex2} 
Let $\X=\{0,1\}^n$ be \textit{the discrete hypercube}. It is a structured graph 
with set of moves \[\Sc:=\big\{\sigma_i\,\big|\,i\in [n]\big\},\]
where for any $i\in [n]$, $\sigma_i(z)$ is defined by flipping the $i$'s coordinate of $z\in \{0,1\}^n$. The discrete hypercube will be endowed with the \textit{Hamming distance}  : $d(x,y)=\sum_{i=1}^{n} \1_{x_{i}\neq y_{i}}$ for $x,y\in \X$.\

{\it Example 3.}\label{ex3}
The \textit{lattice} $\X=\Z^n$ is a structured graph with set of moves 
$\Sc:=\{\sigma_{i+},\sigma_{i-}\,|\,i\in [n]\}$ where for any $z\in \Z^n$
$\sigma_{i+}(z)=z+e_i$, $\sigma_{i-}(z)=z-e_i$, and $(e_1,\ldots,e_n)$ is the canonical base of $\R^n$.
The graph distance is given by 
$d(x,y):=\sum_{i=1}^{n} |x_{i}-y_{i}|,\hspace{0.2cm} \forall x,y\in \X$.


 

%The so called {\it Ricci flat graphs} are  examples of {\it structured graphs}.k
%The concept of Ricci flat graphs was first introduced by Chung and Yau for the study of logarithmic Harnack inequalities on graphs in \cite{CY96} and recently revisited in \cite{CKKLP21}. These graphs generalize the Cayley graphs of Abelian groups. 
%\begin{definition}
%Let $(\X,E)$ be a $D$-regular graph. We say that $z\in \X$ is \textit{Ricci-flat} if there exist some maps $\sigma_{i}:B_{1}(z)\rightarrow \X$,  $1\leq i\leq D$, with the following properties
%\begin{itemize}
%    \item  $\sigma_{i}(z^{\prime})\sim z^{\prime} \hspace{0.2cm} \text{for all } z^{\prime}\in B_{1}(z)$ ,
%    \item $\sigma_{i}(z)\neq \sigma_{j}(z) \hspace{0.1cm} \text{if} \hspace{0.1cm} i\neq j$ ,
%    \item $S_1(\sigma_{i}(z))= \sigma_{i}(S_1(z))$ for any  $i\in [D]$.
%\end{itemize}
%A graph $(\X,E)$ is said to be \textit{Ricci flat} if it is Ricci flat for every $z\in \X$.
%\end{definition}
%With the above notations,  a Ricci flat graph is a structured graph with $\Sc:=\{\sigma_i\,|\, i\in[D]\}$. Indeed, given $\sigma_i\in \Sc$,
%by the third property of Ricci flat graphs it follows that $\sigma_{i}(\sigma_{j}(z))\in %S_{1}(\sigma_{i}(z))\cap S_{1}(\sigma_{j}(z))$ and thus it is immediate that there exists a one to one map $\phi:[D]\to [D]$ such that $\sigma_i(\sigma_j(z))=\sigma_{\phi(j)}(\sigma_i(z))$. Since $d\big(z,\sigma_i(\sigma_j(z))\big)=2$ if and only if $d\big(z,\sigma_{\phi(j)}(\sigma_i(z))\big)=2$, it follows that the map $\psi:\sigma_j\to \sigma_{\phi_j}$ is one to one from $\Sc_z^{\cdot\rightarrow \sigma_i}$ to $\Sc_z^{\sigma_i\rightarrow \cdot}$, and for any $\sigma_j\in \Sc_z^{\cdot\rightarrow \sigma_i}$, $\sigma_i(\sigma_j(z))=\psi(\sigma_j)(\sigma_i(z))$.

{\it Example 4 (Transposition model).}\label{ex4}
Let $S_{n}$ be the \textit{symmetric group} consisting of all bijective maps $\sigma: [n]\rightarrow [n]$. For any $z\in S_n$ and $\{i,j\} \subset [n]$, $i\neq j$, let $\sigma_{ij}(z)$ be  the neighbour of $z$  that differs from $z$ by a \textit{transposition} $(ij), \hspace{0.1cm} \sigma_{ij}(z):=z(ij).$ The graph distance between two elements of $x$ and $y$ of $S_{n}$, is the minimal number of transpositions $ \tau_1,..., \tau_k$ such that $x \tau_1\cdots \tau_k=y.$ The transposition model on $S_{n}$ is a structured graph with $\Sc:=\{\sigma_{ij}\hspace{0.1cm}|\hspace{0.1cm} \{i,j\}\subset [n]\}$.\

{\it Example 5 (Bernoulli-Laplace model).} Let $\X=\X_{m}$ be the slice of the discrete hypercube $\{0,1\}^n$ ,
$ \X_{m}:=\left\{x\in \{0,1\}^{n}\,\big| \,x_1+\ldots +x_n=m\right\}.$
For any $\{i,j\}\subset [n]$, let $\sigma_{ij}:\X_{m}\to\X_{m}$ denote the one to one functions that exchanges the value of coordinate $i$ with the one of coordinate $j$, namely for any $z=(z_1,\ldots,z_n)\in \X_{m} $ 
\begin{equation*}
\big(\sigma_{ij}(z)\big)_j:=z_i,\qquad \big(\sigma_{ij}(z)\big)_i:=z_j,
\end{equation*}
and for any $k\in[n]\setminus\{i,j\}$,
$\big(\sigma_{ij}(z)\big)_k:=z_k.$ Two vertices in $\X_{m}$ are declared neighbours if they differ by exactly two coordinates and $d(x,y):=\frac12\sum_{i=1}^{n} \1_{x_{i}\neq y_{i}}$ for $x,y\in \X_m$.
The Bernoulli-Laplace model is a structured graph with 
$\Sc:=\{\sigma_{ij}\hspace{0.1cm}|\hspace{0.1cm} \{i,j\}\subset [n]\}$.\

There is a notion related to structured graphs, the so-called \textit{Ricci flat graphs}. The concept of Ricci flat graphs was first introduced by Chung and Yau for the study of logarithmic Harnack inequalities on graphs in \cite{CY96} and recently revisited in \cite{CKKLP21}. These graphs generalize the Cayley graphs of Abelian groups. 
\begin{definition}[\cite{CY96}]
Let $(\X,E)$ be a $D$-regular graph. We say that $z\in \X$ is \textit{Ricci-flat} if there exist some maps $\sigma_{i}:B_{1}(z)\rightarrow \X$,  $1\leq i\leq D$, with the following properties
\begin{enumerate}[label=(\roman*)]
    \item  $\sigma_{i}(z^{\prime})\sim z^{\prime} \hspace{0.2cm} \text{for all } z^{\prime}\in B_{1}(z)$ ,
    \item $\sigma_{i}(z)\neq \sigma_{j}(z) \hspace{0.1cm} \text{if} \hspace{0.1cm} i\neq j$ ,
    \item $S_1(\sigma_{i}(z))= \sigma_{i}(S_1(z))$ for any  $i\in [D]$.
\end{enumerate}
A graph $(\X,E)$ is said to be \textit{Ricci flat} if it is Ricci flat for every $z\in \X$.
\end{definition}
%Let $(\X,E)$ be a Ricci flat graph.

In the case that the maps $\sigma_{i}:B_{1}(z)\rightarrow \X$ do not depend on a chosen vertex $z\in \X$, Ricci flat graphs are examples of structured graphs. Indeed under this condition, a Ricci flat graph is a structured graph with $\Sc:=\{\sigma_i\,|\, i\in[D]\}$. Given $\sigma_i\in \Sc$,
by the third property of Ricci flat graphs it follows that $\sigma_{i}(\sigma_{j}(z))\in S_{1}(\sigma_{i}(z))\cap S_{1}(\sigma_{j}(z))$ and thus it is immediate that there exists a one to one map $\phi:[D]\to [D]$ such that $\sigma_i(\sigma_j(z))=\sigma_{\phi(j)}(\sigma_i(z))$. Since $d\big(z,\sigma_i(\sigma_j(z))\big)=2$ if and only if $d\big(z,\sigma_{\phi(j)}(\sigma_i(z))\big)=2$, it follows that the map $\psi:\sigma_j\to \sigma_{\phi_j}$ is one to one from $\Sc_z^{\cdot\rightarrow \sigma_i}$ to $\Sc_z^{\sigma_i\rightarrow \cdot}$, and for any $\sigma_j\in \Sc_z^{\cdot\rightarrow \sigma_i}$, $\sigma_i(\sigma_j(z))=\psi(\sigma_j)(\sigma_i(z))$.

%there exists $i'\in [D]$  $x=\sigma_{\phi(i')}(\sigma_i(x))=\sigma_i(\sigma_{i'}(x))$. 
%Ricci flat graphs having no triangle (see \cite{}), it follows that 
%\[\Sc_x^{\cdot\rightarrow \sigma_i}= \Sc\setminus\{\sigma_{i'}\}\quad \mbox{and}\quad \Sc_x^{\sigma_i\rightarrow \cdot}=\Sc\setminus\{\sigma_{\phi(i')}\}.\]
%The map $\psi:\sigma_j\to \sigma_{\phi_j}$ is one to one from $\Sc_x^{\cdot\rightarrow \sigma_i}$ to $\Sc_x^{\sigma_i\rightarrow \cdot}$, and for any $\sigma_j\in \Sc_x^{\cdot\rightarrow \sigma_i}$, $\sigma_i(\sigma_j(x))=\psi(\sigma_j)(\sigma_i(x))$.

%For instance, the discrete hypercube $\X=\{0,1\}^n$ is a structured graph 
%with set of moves 
%\[\Sc:=\big\{\sigma_i\,\big|\,i\in [n]\big\},\]
%where for any $i\in [n]$, $\sigma_i(z)$ is defined by flipping the $i$'s coordinate of $z\in \{0,1\}^n$. 
%Let $(G,\ast)$ be a finite group and let $\Sc$ be a generator of the group. 
%Let us suppose that 
%$\Sc=\Sc_{e}=\Sc_{g}$ for all $g\in G$ (where $e$ is the identity element) and  that for all $g,h\in \Sc$, $ghg^{-1}\in \Sc$ . Then $(G,\ast)$ is a structured graph. Indeed, the first three axioms are obviously satisfied and $\psi(h):=g\ast h\ast g^{-1}\in \Sc $ is a one to one map and satisfies for any $l\in G$, $g\ast h\ast l=\psi(h)\ast g\ast l$.\ %For instance, the transposition model on the \textit{symmetric group} $S_{n}$ is a structured graph. 

%Let $S_{n}$ the \textit{symmetric group} consisting of all bijective maps $\sigma: [n]\rightarrow [n]$. For any $z\in S_n$ and $\{i,j\} \subset [n]$, $i\neq j$, let $\sigma_{ij}(z)$ be  the neighbour of $z$  that differs from $z$ by a \textit{transposition} $(ij), \hspace{0.1cm} \sigma_{ij}(z):=z(ij).$ The graph distance between two elements of $x$ and $y$ of $S_{n}$, is the \textit{transposition distance} $d(x,y)$ that corresponds to the minimal number of transpositions $ \tau_1,..., \tau_k$ such that $x \tau_1\cdots \tau_k=y$. In that way, the transposition model on $S_{n}$ is a structured graph with $\Sc:=\{ \text{all transpositions} \hspace{0.1cm} \sigma_{ij} | \sigma_{ij}: [n]\rightarrow [n]\}$.\

%Let us introduce the \textit{Bernoulli-Laplace model}. Let $\X=\X_{K}$, the slice of the discrete hypercube $\{0,1\}^n$ ,
%$ \X_{K}:=\left\{x\in \{0,1\}^{n}\,\big| \,x_1+\ldots +x_n=K\right\}.$
%For any $\{i,j\}\subset [n]$, let $\sigma_{ij}:\X_{K}\to\X_{K}$ denote the one to one functions that exchanges the value of coordinate $i$ with the one of coordinate $j$, namely for any $z=(z_1,\ldots,z_n)\in \X_{K} $ 
%\begin{equation*}
%\big(\sigma_{ij}(z)\big)_j:=z_i,\qquad \big(\sigma_{ij}(z)\big)_i:=z_j,
%\end{equation*}
%and for any $k\in[n]\setminus\{i,j\}$,
%$\big(\sigma_{ij}(z)\big)_k:=z_k$. Two vertices in $\X_{K}$ are declared neighbours if they differ by exactly two coordinates.
%In that way, the Bernoulli-Laplace model is a structured graph with 
%$\Sc=\{\sigma_{ij}:\X_{K}\rightarrow \X_{K}|\big(\sigma_{ij}(z)\big)_j=z_i, \big(\sigma_{ij}(z)\big)_i=z_j \hspace{0.1cm} \text{and for any} \hspace{0.1cm} k\in[n]\setminus\{i,j\},
%\big(\sigma_{ij}(z)\big)_k=z_k\}$.



Ricci flat graphs have non negative 
\textit{Bakry-\'Emery} curvature as well as non negative \textit{Ollivier curvature} or \textit{Lin-Lu-Yau} curvature \cite{KKRT16,CKKLP21}. Actually the same kind of property holds for the entropic curvature for any structured graph.


%Let us assume that a graph $(\X,E)$ is Ricci flat. Let $S$ be the set of allowed moves from any vertex of $B_1(z)$. In that way, any vertex may depends on $z$.
 %Given $\sigma,\tau\in S_z$ with $\sigma\neq \tau$, we know that $\sigma(z)$ is a neighbour of $\tau(\sigma(z))$ and this is also the case for $\tau(z)$. Indeed, according to the last  property of the above definition, there exists a single $\tau'\in S_z\setminus\{\sigma\}$ such that  $\tau(\sigma(z))=\sigma(\tau'(z))$. In other words, if $\sigma\in S_z$ and $\tau\in S_{\sigma(z)}$, then $\tau\in S_z$. 
 %\textcolor{red}{Question/Possible reformulation : Est ce que vous etes en train de dire la chose suivante (je crois que oui)
 %$\sigma_{i}\sigma_{j}(z)\in S_{1}(\sigma_{i}(z))\cap S_{1}(\sigma_{j}(z))$ ?
 %En effet, $\sigma_{i}\sigma_{j}z=\sigma_{i}(\sigma_{j}z)\in S_{1}(\sigma_{j}(z))$ et 
 %$S_{1}(\sigma_{i}(z))=\sigma_{i}S_{1}(z)\ni \sigma_{i}\sigma_{j}(z)$ }



\begin{theorem}\label{thmstructure}
Let $(\X,E)$ be a structured graph associated with a finite set of moves $\Sc:=\big\{\sigma_i\,\big|\,i\in [n]\big\}$. The lower bound $r=r_0$ of the entropic curvature $\kappa$ of the space $(\X,d,m_0,L_0)$ given by Theorem \ref{thmprinc} is non negative. 

Moreover, given $z\in \X$, if for any $\sigma\in \Sc$, $d\big(z,\sigma(\sigma(z))\big)\leq 1$ then 
\[r(z)=r_0(z)=-2\log K_0\big(z,S_2(z)\big)\geq -2\log\big(1-1/|S_1(z)|\big)>\frac2{|S_1(z)|}.\]
 \end{theorem}
The proof of this general result is postponed in Appendix B.
In the following examples, we do not mention the discrete hypercube $\{0,1\}^{n}$, the lattice $\mathbb{Z}^{n}$  which will be discussed in detail in section 6, by considering also perturbations of the counting measure $m_0$ on these graphs.
%The discrete hypercube, the lattice $\mathbb{Z}^{n}$ and their perturbation results will be treated in detail in section 4. 
%In particular, one precisely computes for these specific examples of structured graphs and their perturbation (\textit{Ising model}),  the lower bounds $r$, $r_1$, $\widetilde \overline{r}$ and $\overline{r}$ of the $W_1$, $\widetilde{T}$ and $\widetilde T_{c_2}$ entropic curvature given by Theorem \ref{thmprincbis}. 

 %\begin{corollary}\


\textit{Example 1  (Cayley graph).}
Since the group $(G,\ast)$ is a structured graph, it has non negative entropic curvature. Moreover, if $s=s^{-1} $ for all $s\in \Sc$ then $d\big(z,s\ast s \ast z\big)=0$ for all $z\in G$ and then  
\begin{equation*}
r_0(z)>\frac{2}{|S_{1}(z)|}=\frac{2}{|\Sc|} \hspace{0.1cm} .
\end{equation*}

{\it Example 4 $($Transposition model $)$.}
 For all $z\in S_{n}$ let us note that 
\[S_1(z)=\Big\{\sigma_{ij}(z)\,\Big |\, \{i,j\}\in I\Big \}\quad \mbox{with}\quad I=\Big\{\{i,j\}\,\Big|\,1\leq i<j\leq n\Big\}.\] 
Also, $d\big(z,\sigma_{ij}(\sigma_{ij}(z))\big)=0$ for all $\{i,j\}\in I$ . Thus,
\begin{equation*}
r_0(z)>\frac{2}{|S_{1}(z)|}=\frac{4}{n(n-1)} \hspace{0.1cm}.
\end{equation*}
%\textcolor{blue}{Remarque dans Discrete Ricci bounds for Bernoulli-Laplace and Random transpositions, Erbar-Maas-Tetali obtienent quelque chose de moins fort. Theoreme 5.1 :$Ricc\geq \frac{4}{n(n-1)}$ }

%\item
{\it Example 5 $($Bernoulli-Laplace model $)$.} 
For all $z\in \X_{m}$, denoting $J_0(z):=\{i\in [n]\,|\, z_i=0\}$ and $J_1(z):=\{i\in [n]\,|\, z_i=1\}$, one has 
$S_1(z)=\Big\{\sigma_{ij}(z)\,\Big |\, i\in J_0(z), j\in J_1(z)\Big \}$. Moreover, $d\big(z,\sigma_{ij}(\sigma_{ij}(z))\big)=0$ for all $\{i,j\}\in [n]$ . Thus,
\begin{equation*}
r_0(z)>\frac{2}{|S_{1}(z)|}=\frac{2}{m(n-m)} \hspace{0.1cm} .
\end{equation*}
%\textcolor{blue}{Remarque dans le mm article, eux ils trouvent pour ce modele, $Ricc\geq \frac{n+2}{2k(n-k)}$ }

%\item The discrete hypercube and the lattice $\mathbb{Z}^{n}$ and their perturbation results will be treated in detail in section 4.
 
%\end{itemize}
%\end{corollary}  
   

For structured graphs, one  introduces another type of transportation cost $\widetilde T_2$  comparable to $\widetilde{T}$,  related to refined modified logarithmic Sobolev inequalities, as for the cost $\widetilde{T}$ in Theorem~\ref{Logsob}.



Given two probability measures $\nu_0,\nu_1\in \Pc(\X)$, for any coupling measure $\pi\in \Pi(\nu_0,\nu_1)$, let us define 
\[\widetilde T_2( \pi):=\int\sum_{\sigma\in \Sc_x} \Pi^\sigma_\rightarrow(x)^2d\nu_0(x)+\int\sum_{\sigma\in \Sc_y} \Pi^\sigma_\leftarrow(y)^2d\nu_1(y),
\]
with
\begin{align*}
 \Pi^\sigma_\rightarrow(x)&:=\int \1_{\sigma(x)\in ]x,y]} \,d(x,y)\, r(x,\sigma(x),\sigma(x),y)\,d\pi_{_\rightarrow}(y|x),\\ 
 \Pi^\sigma_\leftarrow(y)&:=\int \1_{\sigma(y)\in ]y,x]} \,d(x,y)\, r(y,\sigma(y),\sigma(y),x)\,d\pi_{_\leftarrow}(x|y).
\end{align*} 

As an example, on the discrete hypercube $\X=\{0,1\}^n$, since $L_{0}^{d(x,y)}(x,y)=d(x,y)!$ and $\sigma_i(x)\in ]x,y]$ if and only if $x_i\neq y_i$ for any $x,y\in \{0,1\}^n$. Therefore,  a simple expression holds for the cost  $\widetilde T_2(\pi)$ in that case, namely 
\begin{equation}\label{Pisigma}
\Pi^{\sigma_{i}}_{\rightarrow}(x)=\int \1_{x_{i}\neq y_{i}}d\pi_{_\rightarrow}(y|x)  \quad \mbox{and}\quad \Pi^{\sigma_{i}}_{\leftarrow}(x)=\int \1_{x_{i}\neq y_{i}}d\pi_{_\leftarrow}(x|y). 
\end{equation} 
Such a type of cost has been first introduced by K. Marton \cite{Mar96} to get refined concentration properties on bounded spaces related to the one reached by M. Talagrand with the so-called convex-hull method (see \cite[Section 4]{Tal95}). Actually, the definition of $\widetilde T_2$ on any structure graph can be interpreted as an extension  of the transportation costs introduced by K. Marton and  M. Talagrand on the hypercube. These costs belong to a larger class of costs named {\it weak transport costs} introduced in the paper \cite{GRST14}.    


Observing that $\sum_{\sigma\in S_x}r(x,\sigma(x),\sigma(x),y)=1$, by the Cauchy-Schwarz inequality, one has 
\begin{align*}
    \widetilde{T}( \pi)&\geq \widetilde T_2( \pi)\geq \int\frac1{|\Sc_x|}\left( \int d(x,y)\, \sum_{\sigma\in \Sc_x}r(x,\sigma(x),\sigma(x),y)\,d\pi_{_\rightarrow}(y|x)\right)^2d\nu_0(x)\\
&\qquad+\int\frac1{|\Sc_y|}  \left(\int d(x,y)\,\sum_{\sigma\in \Sc_y} r(y,\sigma(y),\sigma(y),x)\,d\pi_{_\leftarrow}(x|y)\right)^2d\nu_1(y)
\\&\geq \frac{\widetilde{T}(\widehat \pi)}{\sup_{x\in \X} |\Sc_x|}\geq \frac{\widetilde{T}(\widehat \pi)}{|\Sc|}.
\end{align*}
By definition, let us call {\it $\widetilde T_2$-entropic curvature} of the discrete space $(\X,d,m,L)$ the best constant $\widetilde\kappa_2\in \R$ such that \eqref{deplacebis} holds with
$C_t = \widetilde\kappa_2 \,\widetilde T_2.$
As a consequence of the last inequality , if $\widetilde \kappa_2\geq 0 $ or $\widetilde \kappa\geq 0 $ then $\widetilde \kappa_2\geq \widetilde \kappa\geq \widetilde \kappa_2/\sup_x |\Sc_x|$.

For a better lower-estimate of $\widetilde \kappa_2$,  one introduces a new  quantity denoted by  $\widetilde{K}(z,W)$ defined for any $z\in \X$ and $W\subset S_2(z)$. Namely let $\widetilde{K}(z,\emptyset):=0$ and for $W\neq \emptyset$ , let 
\begin{align}\label{defKtilde}
\nonumber\widetilde{K}(z,W)&=\widetilde{K}_L(z,W):=\sup  \Biggl\{  \sum_{\ttz\in W} L^2(z,\ttz) \prod_{\tz \in ]z,\ttz[}\left(\frac{\beta(\tz)}{\big(L(z,\tz)\big)^2}\right)^{\frac{L(z,\tz)L(\tz,\ttz)}{L^2(z,\ttz)}}
\\&
-\sum_{(\tz,w')\in ]z,W[^2, \tz\neq w'} \sqrt{\beta(\tz)}\sqrt{\beta(w')}\, \Bigg|\,{\beta}=(\beta(\tz))_{\tz\in ]z,W[}\in \mathbb{R}_{+}^{]z,W[},\sum_{\tz\in ]z,W[} \beta(\tz)=1 \Biggr\}.
\end{align}
For a structured graph, this quantity can also be  expressed as follows,
\begin{align*}
\widetilde{K}_L(z,W)&=\widetilde{K}(z,W):=\sup  \Biggl\{  \sum_{\ttz\in W} L^2(z,\ttz) \prod_{\sigma\in \Sc_{]z,\ttz[}}\left(\frac{\beta(\sigma)}{\big(L(z,\sigma(z))\big)^2}\right)^{\frac{L(z,\sigma(z))L(\sigma(z),\ttz)}{L^2(z,\ttz)}}\!\!\!
\\&\nonumber
-\sum_{(\sigma,\tau)\in \Sc_{]z,W[}^2, \sigma\neq \tau} \sqrt{\beta(\sigma)}\sqrt{\beta(\tau)}\, \Bigg|\,{\beta}=(\beta(\sigma))_{\sigma\in \Sc_{]z,W[}}\in \mathbb{R}_{+}^{\Sc_{]z,W[}},\sum_{\sigma\in \Sc_{]z,W[}} \beta(\sigma)=1 \Biggr\},
\end{align*}
where for any subset $W\subset S_2(z)$, $\Sc_{]z,W[}:=\big\{\sigma\in S\,\big|\, \sigma(z)\in ]z,W[\big\}$. If $m=m_0$ and $L=L_0$, then we write $ \widetilde K_{0}(z,W):=\widetilde K_{L_0}(z,W)$.
\begin{theorem}\label{Thmstructure}
Let $\X$ be a structured graph associated to a set of moves $\Sc$, and 
assume that $(\X,d,m,L)$ is a graph space.  For any $z\in \X$, let us define $\widetilde r_2=\widetilde r_2^L:=\inf_{z\in \X}\widetilde r_2(z) $, with
\begin{equation}\label{defr_3}
\widetilde r_2(z)=\widetilde r_2^L(z):=1-\widetilde{K}_L(z), \quad \mbox{and}\quad  \widetilde{K}_L(z):=\sup_{W\in S_2(z)} \widetilde{K}_L(z,W).
\end{equation}
For any $z\in \X$, one has
\begin{eqnarray}\label{rr_3} 1-K(z,S_2(z))\leq  \widetilde r_2(z)\leq |S_1(z)|\,\big(1-K(z,S_2(z))\big).
\end{eqnarray}

\begin{enumerate}[label=(\roman*)]
%\begin{enumerate}[(i)]


\item If the generator $L$ satisfies for any $z\in \X$ and any $\sigma,\tau \in \Sc$   with $d\big(z,\tau(\sigma(z))\big)=2$ 
\begin{equation}\label{condL}
L\big(z,\sigma(z)\big)L\big(\sigma(z),\tau(\sigma(z))\big)=L\big(z,\tau(z)\big)L\big(\tau(z),\tau(\sigma(z))\big),
\end{equation}
then the
$\widetilde{T}_2$-entropic curvature $\widetilde{\kappa}_2$ of  $(\X,d,m,L)$ is   bounded from below by $\widetilde r_2\geq 0$.
\item Assume that  \eqref{condL} holds and moreover that for any $z\in \X$ and any  $\sigma\in \Sc$,
\begin{equation}\label{restr}
d\big(z,\sigma(\sigma(z))\big)\leq 1
\end{equation}
then the above result can be improved replacing the curvature cost $C_t(\widehat \pi)=\widetilde r_2 \widetilde T_2(\widehat\pi)$ in the $C$-displacement convexity property of entropy \eqref{deplacebis} by  the cost $C_t(\widehat \pi)=\widetilde r_2\widetilde C_t^1(\widehat \pi)$, $t\in (0,1)$, where for any $D\geq1$, the cost  $\widetilde C_t^D(\widehat \pi)$ is given by  
\[\widetilde C_t^D(\widehat \pi):=\int \sum_{\sigma\in \Sc} D^2 h_t\left(\frac{\Pi^\sigma_\rightarrow(x)}{D}\right) d\nu_0(x)+\int \sum_{\sigma\in \Sc} D^2 h_{1-t}\left(\frac{\Pi^\sigma_\leftarrow(y)}D\right) d\nu_1(y),\]
where  for any $u\geq 0$, 
\[
h_t(u):=\frac{ th(u)-h(tu)}{t(1-t)},\qquad\mbox{with} \qquad h(u)=
\left\{\begin{array}{ll}
2\left[(1-u)\log(1-u)+u\right] &\mbox{ for } \;0\leq u\leq 1,
\\
+\infty &\mbox{ for } \;u>1.
\end{array}\right.
\]
Assume that $D:=\textnormal{Diam}(\mathcal{X})<\infty$. If condition \eqref{restr} is not satisfied, then  the $C$-displacement convexity property of entropy \eqref{deplacebis} also holds with the cost $C_t(\widehat \pi)=\widetilde r_2\widetilde C_t^D(\widehat \pi)$, $t\in (0,1)$.   
\end{enumerate}
\end{theorem}

The proof of this Theorem is given in Appendix B.

{\bf Comments:}
\begin{enumerate}[label=(\roman*)]

\item Condition \eqref{condL} makes sense since if $d\big(z,\tau(\sigma(z))\big)=2$ then $\sigma(z)\in ]z,\tau(\sigma(z))[$, and according to the definition of structured graph there exists $\psi(\sigma)\in \Sc_{\tau(z)}$ such that  
$\tau(\sigma(z))=\psi(\sigma)(\tau(z))$, and therefore  $\tau(z)\in ]z,\tau(\sigma(z))[.$ %\textcolor{blue}{Note that by the argument below, every structured graph has girth less than or equal to 4.}
 Condition \eqref{condL} actually provides needed properties to bound from below $\widetilde{\kappa}_3$ by $\widetilde r_2$, which are collected in  Lemma \ref{lemL} in Appendix A.
\item The second part of the theorem improves its first part
since for any $t\in (0,1), u\geq 0, h_t(u)\geq u^2$, and therefore $C_t^D(\widehat \pi)\geq \widetilde T_2(\widehat \pi)$. This improvement is useful in particular when considering the derived modified logarithmic Sobolev inequalities. It allows to reach smaller discrete Dirichlet forms in the right-hand side of the modified Sobolev inequality (see the comments of Theorem \ref{Logsobbis} below).
\item Condition \eqref{restr} implies that any discrete geodesic $(z_0,\ldots,z_d)$  from $z_0=z$ to any point $z_d\in \X$  ($d=d(z, z_d)$) does not use any move $\sigma\in \Sc$ more than one time. Indeed, if $(z_0,\ldots, z_d) $ is such that for some $0\leq k<\ell\leq d-1$, $z_{k+1}=\sigma(z_k)$ and $z_{\ell+1}=\sigma(z_\ell)$, then  Lemma \ref{lemL} implies that $(z_0,\sigma(z_0), \sigma(\sigma(z_0)), \ldots,\sigma(  \sigma(z_\ell)), z_{\ell+2},\ldots,z_d)$ is also a geodesic. Therefore $d\big(z_0, \sigma(\sigma(z_0))\big)=2$ which is a contradiction with condition \eqref{restr}. 
\end{enumerate}
As for $\widetilde{T}$-entropic curvature, positive $\widetilde T_2$-entropic also provides transport entropy inequalities and also  modified logarithmic-Sobolev  and Poincaré inequalities.
For any $\sigma\in \Sc$, and $g:\X\to \X$, let 
\[\partial_\sigma g(z):=g(\sigma(z))-g(z),\qquad z\in \X.\]

\newpage
\begin{theorem}\label{Logsobbis}
Let $\X$ be a structured graph described by a set of moves $\Sc$ and 
 $(\X,d,m,L)$ be a graph space with $m(\X)<+\infty$. Let $\mu:=m/m(\X)$.
\begin{enumerate}[label=(\roman*)]
 \item If the $\widetilde T_2$-entropic curvature $\widetilde \kappa_2$ of the space $(\X,d,m,L)$ is positive, then  $\mu$ satisfies the following transport-entropy inequality, for any probability measures $\nu_0$ and $\nu_1$ on $\mathcal{P}_b(\X)$
 \begin{equation*}
\frac{\widetilde\kappa_2}{2} \,\widetilde{T}_2(\nu_0,\nu_1)  \leq \Big(\sqrt{\HR(\nu_0|\mu)}+\sqrt{\HR(\nu_1|\mu)}\Big)^2,
\end{equation*}
 with $\widetilde{T}_2(\nu_0,\nu_1):=\inf_{\pi\in \Pi(\nu_0,\nu_1)} \widetilde{T}_2(\pi)$.
 \item
 If the $\widetilde T_2$-entropic curvature $\widetilde \kappa_2$ of the space $(\X,d,m,L)$ is positive, then  $\mu$ satisfies the following modified logarithmic-Sobolev inequality, for any bounded function $f:\X\to [0,+\infty)$, 
 \begin{equation}\label{logsobT3}
{\rm Ent}_\mu(f)\leq \frac{1}{2\widetilde \kappa_2} \int \sum_{\sigma\in \Sc}  [\partial_\sigma \log f]_-^2  f\,d\mu \hspace{0.2cm} .
\end{equation}
 \item
 If the $C$-displacement convexity property of entropy \eqref{deplacebis} holds with  the cost $C_t(\widehat \pi)= \widetilde{\kappa}_2\widetilde C_t^D(\widehat \pi)$  given in Theorem \ref{Thmstructure} for some $D\geq 1$
then for any bounded function $f:\X\to [0,+\infty)$,
\begin{equation}\label{logsobbis}
{\rm Ent}_\mu(f)\leq  \int \sum_{\sigma\in \Sc} \frac{\widetilde \kappa_2 D^2}2 \, h^*\left(\frac{2}{D\widetilde\kappa_2} [\partial_\sigma \log f]_-\right)  f\,d\mu,
\end{equation}
where $[a]_-=\max(0,-a)$, $a\in\R$ and $\frac{1}2 h^*(2v)=e^{-v}+v-1$, $v\geq 0$.
 \end{enumerate}
In any case, it follows that   $\mu$ also satisfies  the following Poincaré  inequality, for any real bounded function $g:\X\to \R$,
 \begin{equation}\label{poinT3}
{\rm Var}_\mu(g)\leq \frac{1}{2\widetilde{\kappa}_2} 
\int \sum_{\sigma\in \Sc}(\partial_\sigma g)^2 d\mu
.
\end{equation}
\end{theorem}
The proof of the transport entropy inequality is identical to the one of Corollary \ref{Transport}. Proofs of  modified logarithmic Sobolev inequalities  and the Poincaré inequality are given  together with the one of Theorem \ref{Logsob} in Appendix B.

{\bf Comments:}
\begin{enumerate}[label=(\roman*)]

%\begin{itemize}
\item Let us give few  inequalities which are useful to compare the discrete Dirichlet forms on the right-hand side of \eqref{logsobT3} and \eqref{logsobbis} with other  discrete Dirichlet forms. 
Let $\alpha,a,b$ be positive real numbers with $a\geq b$, one easily proves that
\[\frac12 h^*\big(2\alpha[\log a-\log b]\big)a\leq  \frac{\alpha^2}2 [\log a-\log b]^2 a
\]
If $\alpha\leq 1$, then the  convexity property of the function $h^*$ implies
\[
  \frac12 h^*\big(2\alpha[\log a-\log b]\big)a\leq \frac\alpha2 h^*\big(2[\log a-\log b]\big)a 
  =\alpha\left( a[\log a-\log b]-[a-b]\right)
\]
and if  $\alpha\geq 1$, then  the decreasing monotonicity property of the function $u\in (0,+\infty)\to \frac1{2u^2} h^*(2u)$ gives 
\[\frac12 h^*\big(2\alpha[\log a-\log b]\big)a\leq\frac{\alpha^2}2 h^*(2[\log a-\log b])a.\]
As a consequence, for any $\alpha>0$,
\begin{align*}
\frac12 h^*\big(2\alpha[\log a-\log b]\big)a&\leq \frac{\alpha\max(1,\alpha)}2 \,h^*(2[\log a-\log b])a \\
&\leq \alpha\max(1,\alpha) \min\left\{[\log a-\log b][a-b],\frac{[a-b]^2}{2b} \right\}.
\end{align*}
Applying these inequalities  with $\alpha=1/(D\widetilde\kappa_2)$, $a=f(x)$ and $b=f(\sigma(x))$, one gets
the following comparisons
\begin{align*}\int \sum_{\sigma\in \Sc} \frac{\widetilde \kappa_2 D^2}2 \, h^*\left(\frac{2}{D\widetilde\kappa_2} [\partial_\sigma  \log f]_-\right)  f\,d\mu
&\leq \min\left\{\frac1{2\widetilde \kappa_2} \int \sum_{\sigma\in \Sc} [\partial_\sigma \log f ]_-^2 f\,d\mu,\right.\\
&\qquad \frac 12 \max\left(D, \frac 1{\widetilde \kappa_2}\right) \int \sum_{\sigma\in \Sc} \frac{[\partial_\sigma f(x)]_-^2}{f(\sigma(x))} \,d\mu(x),\\
&\qquad \left.\max\left(D, \frac 1{\widetilde \kappa_2}\right) \int \sum_{\sigma\in \Sc} {[\partial_\sigma \log f]_-[\partial_\sigma f]_-} \,d\mu\right\}
\end{align*}
In particular, it follows that \eqref{logsobbis} is a refinement of \eqref{logsobT3}.
 
\item 
With respect to the spectral gap comparisons we made in the comments of Theorem \ref{Logsob} for generic graphs, the above Theorem makes it possible to obtain more precise estimates regarding the spectral gap for structured graphs. Indeed, thanks to the Poincar{\'e} inequality \eqref{poinT3}  one obtains that $\lambda_{1}\geq 2 \widetilde{\kappa}_{2}$ and in many cases $\lambda_{1}$ and $\widetilde{\kappa}_{2}$ are of the same order of magnitude. For example in the case of $\{0,1\}^{n}$, $\lambda_{1}=2$ and $\widetilde{\kappa}_{2}\geq 1$ (as we shall see in section 6.1) and thus we obtain sharp estimates in this case.
%The above theorem makes it possible to obtain bounds for the spectral gap of structured graphs. More precise lower bounds regarding the spectral gap are obtained than for generic graphs. 

%\textcolor{blue}{On sait ceci car $$\widetilde{r}_{3}=1$.C' est on si  $\widetilde{\kappa}_{3}>1$? }


\end{enumerate}

\section{Perturbation results}

\subsection{Perturbation with a potential}\label{sectionpertu}
Let $(\X,d,m,L)$ be a graph space satisfying a $C$-displacement convexity property of entropy \eqref{deplacebis}. Let $m_v$ denote the measure with density $e^{-v}$ with respect to $m$, where $v:\X\to \R$ is a potential. We want to understand how the  constants $r,r_1,\overline{r},\widetilde{r},\widetilde r_2$ defined in introduction are  modified if the measure $m$ is replaced by  a perturbation $m_v$ of the measure $m$ in the  $C$-displacement convexity property of the relative entropy  along the Schr\"odinger bridges at zero temperature of the space $(\X,d,m,L)$. The results of this part are  used in section \ref{examplesZcube}  for measures with interaction potential in particular on the discrete hypercube and on the lattice $\Z^n$.

Since for any probability measure $\nu\in \Pc(\X)$ absolutely continuous with respect to $m$,
\begin{equation}\label{Hm0mv}
H(\nu|m_v)=H(\nu|m)+\int v\, d\nu,
\end{equation}
convexity properties of $t\in(0,1)\to H(\widehat \nu_t|m_v)$ may follow from convexity properties of $t\in(0,1)\to H(\widehat \nu_t|m)$ and convexity properties of $\psi:t\in(0,1)\to \int v\, d\widehat \nu_t$.
According to Lemma \ref{deriveseconde}, assuming $\nu_0$ and $\nu_1$ have bounded support,
\begin{equation}\label{automne}
\psi''(t)%:=\left(\int V\, d\widehat \nu_t\right)^{''}
=\sum_{(x,y)\in\X^2}\left(\int v\, d \nu_t^{x,y}\right)^{''}\widehat \pi(x,y)\\
=\sum_{(x,y)\in\X^2}
d(x,y)\big(d(x,y)-1\big)\, D_t v(x,y)
\widehat \pi(x,y)
\end{equation}
with 
\begin{equation}\label{mulh}
D_t v(x,y):=\sum_{(z,\ttz)\in [x,y], d(z,\ttz)=2} Dv(z,\ttz) \, L^2(z,\ttz)\, r(x,z,\ttz,y)\, \rho_t^{d(x,y)-2}(d(x,z)),
\end{equation}
and for $z,\ttz\in \X$ with $d(z,\ttz)=2$,
\[ Dv(z,\ttz):=\sum_{\tz\in  ]z,\ttz[} \left(v(\ttz)+v(z)-2v(\tz)\right)\frac{L(z,\tz)L(\tz,\ttz)}{L^2(z,\ttz)}.\]
%Setting
%\[\D v:=\sup\big\{D_s v(x,y)\,\big|\,x,y\in \X, s\in (0,1)\big\},\]
As a consequence one has 
\begin{align*}
\psi(t)&=(1-t)\psi(0)+t\psi(1)- \frac{t(1-t)}2 \int_0^1\psi''(s)q_t(s)\, ds\\
&=(1-t)\psi(0)+t\psi(1)- \frac{t(1-t)}2 \iint c^v_t(x,y)\,d\widehat\pi(x,y),
\end{align*}
with 
\[c^v_t(x,y):=d(x,y)\big(d(x,y)-1\big)\, \int_0^1 D_s v(x,y) q_t(s)\, ds.\]
This gives the following result.
\begin{theorem}\label{entropicperturb}
Let $(\X,d,m,L)$ be a graph space. Assume that a   $C$-displacement convexity property of entropy \eqref{deplacebis} holds.
Given a  potential $v:\X\to \R$, let $m_v$ denote the measure with density $e^{-v}$ with respect to $m$. 
Then the relative entropy with respect to $m_v$, $\nu\in \Pc(\X)\mapsto H(\nu|m_v)$, satisfies the $C^v$-displacement convexity property \eqref{deplacebis} along the Schr\"odinger bridge at zero temperature of the space $(\X,d,m,L)$, with for any $t\in (0,1)$,
\[C^v_t(\widehat \pi)=C_t(\widehat \pi) +\iint c_t^v(x,y)\, d\widehat \pi(x,y) .\]
\end{theorem}

Another way to  get $C^v$-displacement convexity property with the measure $m_v$ is to consider the generator $L_v$  defined by 
\[L_v(x,y)=e^{\frac12(v(x)-v(y))} L(x,y),\qquad x,y\in \X, x\neq y.\]
One easily checks that the measure $m_v$ is reversible with respect to $L_v$ and that the space $(\X,d,m_v,L_v)$ is a graph space.
%satisfies the $(G)$-conditions. 
Moreover, since for any $x,y\in \X$,
\[L_v^{d(x,y)}(x,y)=e^{\frac12(v(x)-v(y))} L^{d(x,y)}(x,y),\]
the Schr\"odinger briges at zero temperature of the space $(\X,d,m_v,L_v)$ are the same as the one of the space $(\X,d,m,L)$. Indeed for any $x,y\in \X$ and $z\in [x,y]$, the quantity
\[r(x,z,z,y)=\frac{L_v^{d(x,z)}(x,z) L_v^{d(z,y)}(z,y)}{L_v^{d(x,y)}(x,y)}=\frac{L^{d(x,z)}(x,z) L^{d(z,y)}(z,y)}{L^{d(x,y)}(x,y)},\]
does not depend on the potential $v$ and therefore the Schr\"odinger briges $\nu_t^{x,y}$  between Dirac measure on the space $(\X,d,m_v,L_v)$ are the same as the one of the space $(\X,d,m,L)$.
Any result we get on the graph space $(\X,d,m_v,L_v)$ on the lower bound on entropic curvature from Theorems \ref{thmprinc}, \ref{thmprincbis} and \ref{Thmstructure} can be interpreted as a perturbation result of the same result on  $(\X,d,m,L)$.

For further use, let us just simplify the definition of the key quantities $K^v:=K_{L_v}$ and $\widetilde K^v:=\widetilde K_{L_v}$ on the space  $(\X,d,m_v,L_v)$. 
Observing that 
\[Dv(z,\ttz)=2 \sum_{\tz\in ]z,\ttz[} \Big(\log \frac{L_v^2(z,\ttz)}{L^2(z,\ttz)}-2\log \frac{L_v(z,\tz)}{L(z,\tz)}\Big) \, \frac{L_v(z,\tz)L_v(\tz, \ttz)}{L_v^2(z,\ttz)},\]
according to \eqref{defR_2} and to \eqref{defKtilde}, one has for any $z\in\X$ and  $W\subset S_2(z)$,
\begin{align}\label{defKLv}
    K^v\big(z,W\big)&=\sup_\alpha \Biggl\{  \sum_{\ttz\in W} e^{-Dv(z,\ttz)/2} L^2(z,\ttz) \prod_{\tz\in  ]z,\ttz[}\left(\frac{\alpha(\tz)}{L(z,\tz)}\right)^{\frac{2L(z,\tz)L(\tz,\ttz)}{L^2(z,\ttz)}}\Biggl\},
%    \nonumber
%    &\leq K\big(z,S_2(z)\big)\, \sup_{z,\ttz, d(z,\ttz)=2} e^{-Dv(z,\ttz)/2}\, . 
\end{align}
where the supremum is over all ${\alpha}=(\alpha(v))_{v\in ]z,W[}\in \mathbb{R}_{+}^{]z,W[}$ such that $\sum_{v \in ]z,W[} \alpha(v)=1$, and $\widetilde K^v(z)=\sup_{W\subset S_2(z)} \widetilde{K}^v(z,W)$  with
\begin{multline}\label{defKvtilde}
\widetilde{K}^v(z,W):=\sup  \Biggl\{ \sum_{\ttz\in W}e^{-Dv(z,\ttz)/2} L^2(z,\ttz) \prod_{\sigma\in \Sc_{]z,\ttz[}}\left(\frac{\beta(\sigma)}{\big(L(z,\sigma(z))\big)^2}\right)^{\frac{L(z,\sigma(z))L(\sigma(z),\ttz)}{L^2(z,\ttz)}}\\-\sum_{(\sigma,\tau)\in \Sc_{]z,W[}^2, \sigma\neq \tau} \sqrt{\beta(\sigma)}\sqrt{\beta(\tau)}\Biggl\},
\end{multline}
where the supremum is over all ${\beta}=(\beta(\sigma))_{\sigma\in \Sc_{]z,W[}}\in \mathbb{R}_{+}^{\Sc_{]z,W[}}$ such that $\sum_{\sigma\in \Sc_{]z,W[}} \beta(\sigma)=1$.



\subsection{Restriction to  convex subsets}\label{sectionrestconv}
In this section, we consider another type of perturbation result, when the measure $m$ is restricted to a convex subset $\Cc$ of $\X$.
\begin{definition}\label{defconvset} On a graph space $(\X,d,m,L)$, a subset $\Cc$ of $\X$ is convex if for any $x,y\in \Cc$,
\[[x,y]\subset \Cc.\]
\end{definition}
Let $(\X,d,m,L)$ be a graph space. Given a subset $\Cc$ of $\X$, let $(\Cc,d_\Cc, m_\Cc,L_\Cc)$ denotes the graph space restricted to $\Cc$ defined by :  $m_\Cc=\1_\Cc m$, for any $x,y\in \X_\Cc$, $x\neq y$, $d_\Cc(x,y):=1$ if and only if $d(x,y)=1$,  and  
$L_\Cc(x,y):=L(x,y)$. One easily checks that the space $(\Cc,d_\Cc, m_\Cc,L_\Cc)$ is a graph space.

If $\Cc$ is a convex subset of $\X$, then the set of discrete geodesics on $(\Cc,d_\Cc, m_\Cc,L_\Cc)$ between two vertices $x$ and $y$ of $\Cc$ is the same as the one on $(\X,d,m,L)$. Since $L_\Cc(\gamma)=L(\gamma)$  for any discrete geodesic $\gamma$ between  $x$ and $y$, it follows that the Schr\"odinger bridge at zero temperature between the Dirac measure at $x$ and the Dirac measure at $y$ is the same on the space $(\Cc,d_\Cc, m_\Cc,L_\Cc)$ as on the space $(\Cc,d_\Cc, m_\Cc,L_\Cc)$.  Therefore this observation also holds  for any Schr\"odinger bridge at zero temperature between two probability measures $\nu_0$ and $\nu_1$ on $\Cc$. This remark implies the following result.
\begin{theorem}\label{thmrestconv}
Let $(\X,d,m,L)$ be a graph space and let $\Cc$ be a convex subset of $\X$. If the relative entropy with respect to $m$ satisfies a $C$-displacement convexity property \eqref{deplacebis} on the  space $(\X,d,m,L)$, then the same property holds for the relative entropy with respect to $m_\Cc$  on the space $(\Cc,d_\Cc, m_\Cc,L_\Cc)$. 
\end{theorem}



\section{Tensorization properties}
%In this part, we will study the \textit{tensorization properties} of the entropic curvature in general spaces as well as in structured spaces. The properties of tensorization related to displacement of convexity have already been mentioned in \cite{Sam21} .\\ 
In this part, we will study the \textit{tensorization properties} of the constants $r$ and $\widetilde r_2$ with respect to the usual \textit{Cartesian product of graphs}. For the sake of completeness let us recall the standard definition of Cartesian product of graphs. The Cartesian product of two graphs $(\mathcal{X}_1, E_{1})$ and $ (\mathcal{X}_2, E_{2})$ endowed with distances $d_1$ and $d_2$ respectively is a graph 
\begin{equation*}
(\X,E)=(\mathcal{X}_1, E_{1})\square (\mathcal{X}_2,E_{2}) ,
\end{equation*}
where $\X=\mathcal{X}_1 \times \mathcal{X}_2 $ and the set of edges $E$ is defined by 
\[ (x_1,x_2)\sim (x^\prime_1, x^\prime_2) \hspace{0.2cm } \text{if} \hspace{0.2cm}
\begin{cases}
    \text{either} \hspace{0.2cm} x_1\sim x^\prime_1 \hspace{0.2cm}  \text{and} \hspace{0.2cm} x_2=x^\prime_2 \hspace{0.06cm} , \\
    \text{or}  \hspace{0.2cm}  x_2\sim x^\prime_2  \hspace{0.2cm} \text{and}  \hspace{0.2cm} x_1=x^\prime_1 \hspace{0.1cm}.
\end{cases}
\]
As a consequence if $d_1$, respectively $d_2$, denotes the graph distance on the graph $(\mathcal{X}_1, E_{1})$, respectively $(\mathcal{X}_2, E_{2})$, then the graph distance $d:=d_1\square d_2$ on $(\mathcal{X}, E)$ is given by 
\[d\big((x_1,x_2),(y_1,y_2)\big):=d_1(x_1,y_1)+d_2(x_2,y_2), \qquad  (x_1,x_2),(y_1,y_2)\in \X.\]

Similarly, one defines the  Cartesian product of  two graph spaces $(\mathcal{X}_1,d_1,m_{1},L_{1})$, $(\mathcal{X}_2,d_2,m_{2},L_{2})$ the graph space 
\[(\mathcal{X}_1,d_1,m_{1},L_{1})\square(\mathcal{X}_2,d_2,m_{2},L_{2}):=(\X_1\times \X_2, d_1\square d_2, m_1\times m_2, L_1\oplus L_2),\]
where the generator $L=L_1\oplus L_2$ on $\X=\X_1\times \X_2$ is given by: for $z=(z_1,z_2),z'=(z_1',z_2')\in \X$
\[L(z,z^{\prime}) := \hspace{0.2cm }\begin{cases}
0\hspace{0.2cm}  \text{if} \hspace{0.2cm} d(z,z')\geq 2,\\
   L_{1}(z_1,z^{\prime}_1) \hspace{0.2cm}  \text{if} \hspace{0.2cm} z_1\sim z^{\prime}_1 \hspace{0.2cm} \text{and} \hspace{0.2cm}   z'=(z_1',z_2) \hspace{0.06cm} ,\\
   L_{2}(z_2,z_2^{\prime}) \hspace{0.2cm}   \text{if} \hspace{0.2cm} z_2\sim z_2^{\prime} \hspace{0.2cm} \text{and} \hspace{0.2cm}   z'=(z_1,z_2') \hspace{0.06cm} ,\\
 -\Big( \sum_{z_1',z_1'\sim z_1} L_{1}(z_1,z^{\prime}_1) +\sum_{z_2', z_2'\sim z_2} L_{2}(z_2,z_2')\Big)\hspace{0.2cm} \text{if} \hspace{0.2cm}   z=z^{\prime} \hspace{0.06cm}.
\end{cases}\]
Since for $i=1,2$ the measures $m_i$ is reversible with respect to the generators $L_i$, the product measure $m$ is also reversible with respect to $L$. This definition can be iterated to define the product of  a finite sequences of graph spaces. 
The following theorem states the tensorisation properties of the lower bounds $r_1$ and $\widetilde r_2$ on entropic curvature introduced in Theorem \ref{thmprinc} and Theorem \ref{Thmstructure}.  The proof is postponed in Appendix B.

\begin{theorem}\label{ttens}
Let $(\X, d,m,L)$ be the cartesian product of $n$ graph spaces $(\X_{i},d_i,m_{i},L_{i}), i\in [n]$,
\[(\X, d,m,L):=\big( \mathcal{X}_{i}\times\cdots\times \X_n,d_1\square\cdots  \square d_n ,m_1\otimes\cdots\otimes m_n, L_1\oplus\cdots \oplus  L_n\big).\]
For $z=(z_1.\ldots ,z_n)\in \X$, let $r(z)$ denotes the quantity \eqref{defr} defined on the space $(\X, d,m,L)$ and for $i\in [n]$, let $r(z_i)$ be the same quantity defined on the space $(\X_i, d_i,m_i,L_i)$. Identically one denotes $\widetilde r_2(z)$, $\widetilde r_2(z_i), i\in [n]$ the quantities whose definition is given by \eqref{defr_3}. If $\min \big({r}(z_{1}),\ldots ,{r}(z_{n})  \big)\leq 0$, then 
\[r(z)\geq \min \big({r}(z_{1}),\ldots ,{r}(z_{n})  \big),\]
and if $\min \big({r}(z_{1}),\ldots ,{r}(z_{n})  \big)\geq 0$
then 
\[r(z)\geq -2\log \Big(1-\frac1n\big(1-e^{-\min({r}(z_{1}),\ldots ,{r}(z_{n}))/2}\big)\Big)\geq \frac1n \min \big({r}(z_{1}),\ldots ,{r}(z_{n})  \big).\]
We also have 
\[\widetilde r_2(z)\geq \min \big(\widetilde{r}_{2}(z_{1}),\ldots ,\widetilde{r}_{2}(z_{n})  \big).\]
\end{theorem}

{\bf Comments:} If $\X_1=\cdots=\X_n=\{0,1\}$ is the two points space equipped with the counting measure $m_1=\cdots=m_n=m_0$ then the graph space $(\X,d,m,L)$ is the discrete hypercube studied in section \ref{sectionhypecube} equipped with the counting measure also, for which we prove that for any $z\in \X$
\[r(z)=-2\log(1-1/n)\quad   \mbox{for} \;n\geq 2, \quad \widetilde r_2(z)=1 \quad\mbox{for} \;n\geq 1,\] 
and $r(z)=+\infty$ for $n=1$. 
It follows that the last two  tensorisation results of Theorem \ref{ttens} can not be improved.     


%\newpage
%\begin{proof}[Proof of Proposition \ref{proptens}]
%The proof of both inequalities related to tensorization is straightforward computation. Let $z=(z_{1},z_{2})$ be an arbitrary vertex in $V$. Let us remark that 
%\begin{equation*}
%S_{2}(z)=\{ ( z_{1}^{\prime\prime} , z_{2}), (z_{1},z_{2}^{\prime\prime}), (z_{1}^\prime, z_{2}^\prime)\hspace{0.1cm} \text{such that} \hspace{0.1cm} d_{1}(z_{1},z_{1}^{\prime\prime})=d_{2}(z_{2}, z_{2}^{\prime\prime})=2,\hspace{0.1cm}d_{1}(z_{1},z_{1}^\prime)=d_{2}(z_{2},z_{2}^\prime)=1 \}.
%\end{equation*}
%Thus,
%\begin{align*}
%K(z,S_{2}(z))&=\sup_\alpha \Bigg(  \sum_{z_{1}^{\prime \prime} \in S_{2}(z_{1})} 
%L_{1}(z_{1},z_{1}^{\prime \prime})^{2}
%\Big(\prod_{z_{1}^{\prime} \in S_1(z_{1})\cap [z_{1},z_{1}^{\prime \prime} ]} {\alpha(z_{1}^{\prime},z_{2})}\Big)^{\frac{2L_{1}(z_{1},z_{1}^{\prime})L_{1}(z_{1}^{\prime},z_{1}^{\prime \prime})}{L_{1}^{2}(z_{1},z_{1}^{\prime \prime}) }} \\
%&+ \sum_{z_{2}^{\prime \prime} \in S_{2}(z_{2})} L_{2}(z_{2},z_{2}^{\prime \prime})^{2}   \Big(\prod_{z_{2}^{\prime} \in S_1(z_{2})\cap [z_{2},z_{2}^{\prime \prime} ]} {\alpha(z_{1},z_{2}^{\prime})}\Big)^{\frac{2L_{2}(z_{2},z_{2}^{\prime})L_{2}(z_{2}^{\prime},z_{2}^{\prime \prime})}{L_{2}^{2}(z_{2},z_{2}^{\prime \prime}) }} \\
%&+ 2\sum_{z_{1}^\prime\sim z_{1}, z_{2}^\prime\sim z_{2}}
%L_{1}(z_{1},z_{1}^{\prime})L_{2}(z_{2},z_{2}^{\prime})
%\alpha(z_{1}^\prime,z_{2})\alpha(z_{1},z_{2}^\prime) \Big) \Bigg)\\
%&\leq  \sup_{\alpha}  \Bigg(  \sum_{z_{1}^{\prime \prime} %\in S_{2}(z_{1})}
%L_{1}(z_{1},z_{1}^{\prime \prime})^{2}
%\Big(\prod_{z_{1}^{\prime} \in S_1(z_{1})\cap [z_{1},z_{1}^{\prime \prime} ]} {\alpha(z_{1}^{\prime},z_{2})}\Big)^{\frac{2L_{1}(z_{1},z_{1}^{\prime})L_{1}(z_{1}^{\prime},z_{1}^{\prime \prime})}{L_{1}^{2}(z_{1},z_{1}^{\prime \prime}) }} \Bigg)\\
%&+ \sup_{\alpha} \Bigg( \sum_{z_{2}^{\prime \prime} \in S_{2}(z_{2})} L_{2}(z_{2},z_{2}^{\prime \prime})^{2}   \Big(\prod_{z_{2}^{\prime} \in S_1(z_{2})\cap [z_{2},z_{2}^{\prime \prime} ]} {\alpha(z_{1},z_{2}^{\prime})}\Big)^{\frac{2L_{2}(z_{2},z_{2}^{\prime})L_{2}(z_{2}^{\prime},z_{2}^{\prime \prime})}{L_{2}^{2}(z_{2},z_{2}^{\prime \prime}) }} \Bigg) \\
%&+2 \sup_{\alpha}\Bigg( \sum_{z_{1}^\prime\sim z_{1}, z_{2}^\prime\sim z_{2}}
%L_{1}(z_{1},z_{1}^{\prime})L_{2}(z_{2},z_{2}^{\prime})
%\alpha(z_{1}^\prime,z_{2})\alpha(z_{1},z_{2}^\prime) \Big) \Bigg)  \\
%&=K(z_{1},S_{2}(z_{1}))+K(z_{2},S_{2}(z_{2}))+
%2\sup_{\alpha} \Big(\sum_{z_{1}^\prime\sim z_{1}, z_{2}^\prime\sim z_{2}}
%L_{1}(z_{1},z_{1}^{\prime})L_{2}(z_{2},z_{2}^{\prime})
%\alpha(z_{1}^\prime,z_{2})\alpha(z_{1},z_{2}^\prime) \Big) \Big)
%\end{align*}

%Note that the last term is simplified to
%\begin{align*}
%&2\sup_{\alpha} \Big(\sum_{z_{1}^\prime\sim z_{1}, z_{2}^\prime\sim z_{2}}
%L_{1}(z_{1},z_{1}^{\prime})L_{2}(z_{2},z_{2}^{\prime})
%\alpha(z_{1}^\prime,z_{2})\alpha(z_{1},z_{2}^\prime) \Big) \Big)\\
%&=2\sup_{\alpha_{1}+\alpha_{2}=1}\Bigg(\sup_{\sum \alpha(z_{1}^{\prime},z_{2})=\alpha_{1}}\sum_{z_{1}^{\prime}\sim z_{1}} L_{1}(z_{1},z_{1}^{\prime})\alpha(z_{1}^{\prime},z_{2})\sup_{\sum_{z_{2}^{\prime}\sim z_{2}}\alpha(z_{1},z_{2}^{\prime})=\alpha_{2}} \sum_{z_{2}^{\prime}\sim z_{2}} L_{2}(z_{2},z_{2}^{\prime})\alpha(z_{2},z_{2}^{\prime})\Bigg)\\
%&=\frac{1}{2} \max_{z_{1}^{\prime}\sim z_{1}} L_{1}(z_{1},z_{1}^{\prime})
%\max_{z_{2}^{\prime}\sim z_{2}}L_{2}(z_{2},z_{2}^{\prime})
%\end{align*}
%\end{proof}






 

\section{Entropic curvature on the discrete hypercube and on the lattice $\Z^n$}\label{examplesZcube} 
This part is devoted to applications of the perturbation results of section \ref{sectionpertu} for specific structure graphs : the discrete hypercube $\{0,1\}^n$ and  the lattice $\Z^n$ endowed with a measure $m_{v}$ with density $e^{-v}$ with respect to the counting measure $m_0$ on the set of vertices. We analyze the constants $r^v=r^{L_v},r_1^v=r_1^{L_v},\overline{r}^v=\overline{r}^{L_v},$ and $\widetilde r_2^v=\widetilde r_2^{L_v}$ that bound from below the different types of entropic curvature of the space $(\X,d,m_v,L_v)$, defined in section \ref{sectionpertu}. One of the goals is to derive  functional inequalities for the measure $m_v$ or its associated normalized probability measure $\mu_v:=m_v/m_v(\X)$, by applying then Theorem \ref{PL}, Corollary \ref{Transport}, Theorem \ref{Logsob} and Theorem \ref{Logsobbis}.  

\subsection{The discrete hypercube and the Ising model}\label{sectionhypecube}
As mentioned in the introduction, the discrete hypercube is a structured graph with set of moves $\Sc:=\{\sigma_i\,|\,i\in[n]\}$ where $\sigma_i(x)$ is defined by flipping the $i$'s coordinate of $x\in\X=\{0,1\}^n$.
In this part  $\mu_0$ is the uniform probability measure on $\{0,1\}^n$. Given $v:\{0,1\}^n\to \R$,  $m_v=e^{-v}m_0$ is a perturbation of the counting measure $m_0$ on $\{0,1\}^n$.

\begin{remark}
The lower bounds $r,r_1, \overline{r},\widetilde{r},\widetilde{r}_2$ on entropic curvature computed in this section  for the hypercube are the same for any graph whose local structure is the one of the hypercube. As pointed out in \cite{LMP17} , the hypercube is not determined by its local structure. Indeed, Laborde and Hebbare \cite{HL82} showed that the conjecture according to which every bipartite, regular graph satisfying that all balls of radius 2 are isomorphic to those of the hypercube is necessarily the hypercube is false. 
\end{remark}

Since for any $i\in[n]$ and $z\in \X$, $\sigma_i(\sigma_i(x))=x$, Theorem \ref{thmstructure} gives $K_0(z,S_2(z))\leq 1-1/n$ for any $z\in \X$. Actually, by choosing $\alpha(\tz)=1/n$ for any $\tz \in S_1(z)$ in the definition of $K_0(z,S_2(z))$, one exactly gets $K_0(z,S_1(z))=1-1/n$ and therefore according to its definition \eqref{defr}, $r=-2\log(1-1/n)$. Theorem \ref{thmprincbis} also provides the lower bound $\widetilde{r}=1/n$ for the $\widetilde{T}$-entropic curvature of the space.

Let us now compute the lower bound $r_1$ of the $W_1$-entropic curvature for the discrete hypercube. We need to estimate $r_1(z)$ for any  $z\in \X$. Notice that 
\[S_2(z):=\Big\{\sigma_j\sigma_i(z)\,\Big|\, (i,j)\in I\Big\}\qquad \mbox{with}\quad I=\Big\{(i,j)\,\Big|\,1\leq i<j\leq n\Big\}.\]
Given $W\subset S_2(z)$,  $W=\Big\{\sigma_j\sigma_i(z)\,\Big|\, (i,j)\in A\Big\}$ for some $A\subset I$, and setting
\begin{equation}\label{hard} A^1:=\Big \{i\in[n]\,\Big|\,\exists j\in [n], (i,j)\in A \;\mbox{ or } \; (j,i)\in A\Big\},
\end{equation}
the  expression \eqref{defRbis} provides using Cauchy-Schwarz inequality
\begin{equation}\label{automnebis}
K_0(z,W):=\sup_\alpha \sum_{(i,j)\in A} 2 \alpha_i\alpha_j\leq \sup_\alpha \sum_{(i,j)\in A^1\times A^1,i\neq j}  \alpha_i\alpha_j\leq 1-\frac{1}{|A^1|},
\end{equation}
where the supremum runs over all vectors $\alpha=(\alpha_1,\ldots,\alpha_n)$ with positive coordinate satisfying $\alpha_1+\cdots+\alpha_n=1$.
Let  $V_+,V_-\subset S_1(z)$, $(V_+,V_-)\neq (\emptyset,\emptyset)$ and  $W_+,W_-\subset S_2(z)$ with $V_+\supset]z,W_+[$ and $V_-\supset]z,W_-[$ 
satisfying \eqref{condsupl}. Note that $|]z,W[|=|A^1|$. Following the above notations for $W$ and $A^1$, one denotes  by $A_+^1$ and $A_-^1$ the sets related to $W_+$ and $W_-$. If $W_+\neq \emptyset$ and $W_-\neq \emptyset$, then $V_+\neq \emptyset$ and $V_-\neq \emptyset$ with $V_+\cap V_-=\emptyset$ and therefore \eqref{automnebis} gives
\[\frac{\1_{V_+\neq \emptyset}}{1-K_0(z,W_+)}+\frac{\1_{V_-\neq \emptyset}}{1-K_0(z,W_-)} \\
\leq |A^1_+|+|A^1_-|\leq  |V_+|+|V_-|\leq n.
\]
If  $W_+=\emptyset$ and $W_-\neq \emptyset$ then since $V_+\cap V_-=\emptyset$ one has 
\[ \frac{\1_{V_+\neq \emptyset}}{1-K_0(z,W_+)}+\frac{\1_{V_-\neq \emptyset}}{1-K_0(z,W_-)}
=\1_{V_+\neq \emptyset} + \frac{1}{1-K_0(z,W_-)}\leq \1_{V_+\neq \emptyset}+|V_-|\leq n,\]
and if $(W_+,W_-)=(\emptyset,\emptyset)$ then 
\[\frac{\1_{V_+\neq \emptyset}}{1-K_0(z,W_+)}+\frac{\1_{V_-\neq \emptyset}}{1-K_0(z,W_-)} = \1_{V_+\neq \emptyset}+\1_{V_-\neq \emptyset}\leq n.\]
Thus one gets  $r_1(z)\geq 1/n$ and according to Theorem \ref{thmprincbis} the $W_1$-entropic curvature is bounded from below by $4r_1\geq 4/n$. Asymptotically as $n$ goes to $+\infty$, this lower bound  is the best  one may expect (see \cite[Corollary 4.5]{GRST14}).

The estimate of the lower bound $\overline{r}=\min_{z\in \X} \overline{r}(z)$ on the $T_{\overline{c}}$-entropic curvature of the space is similar ($\overline{c}=(\overline{c}_t)_{t\in (0,1)}$ with $\overline{c}_t$ defined by \eqref{defcout2}).  Let $W_+,W_-\subset S_2(z)$ with  $(W_+,W_-)\neq (\emptyset,\emptyset)$, and  
satisfying \eqref{condsuplbis}. If $W_+\neq \emptyset$ and $W_-\neq \emptyset$ then inequality \eqref{automnebis} gives \begin{align*}\frac{\1_{W_+\neq \emptyset}}{-\log K_0(z,W_+)}+\frac{\1_{W_-\neq \emptyset}}{-\log K_0(z,W_-)}
&\leq \frac{\1_{W_+\neq \emptyset}}{1-K_0(z,W_+)}+\frac{\1_{W_-\neq \emptyset}}{1-K_0(z,W_-)} \\
&\leq |A^1_+|+|A^1_-|=  |]z,W_+[|+|]z,W_-[|\leq n.
\end{align*}
If $W_+\neq \emptyset$ and $W_-=\emptyset$,  then 
\[\frac{\1_{W_+\neq \emptyset}}{1-K_0(z,W_+)}+\frac{\1_{W_-\neq \emptyset}}{1-K_0(z,W_-)}=\frac{1}{1-K_0(z,W_+)}\leq \frac{1}{1-K_0(z,S_2(z))}=n.\]
In any cases $\overline{r}(z)\geq 1/n$ and Theorem \ref{thmprincbis} ensures that the $T_{\overline{c}}$-entropic curvature of the hypercube is bounded from below by $4\overline{r}\geq 4/n$. We know from \cite{Sam21} that this lower-bound is asymptotically optimal in $n$. Indeed, one may recover the optimal $T_2$-transport entropy inequality for the standard Gaussian measure on $\R$ from the transport entropy inequality with cost $T_{\overline{c}}$ derived from this entropic lower bound (see  \cite[Lemma 4.1]{Sam21}). 

Let us now compute $\widetilde r_2(z)=1-\sup_{W\subset S_2(z)} \widetilde{K}_0(z,W)$.
Using the above notations, for $z\in \X$, one has 
\[\widetilde{K}_0(z,W)=\widetilde{K}(z,W):=\sup_\beta  \Biggl\{  \sum_{(i,j)\in A} 2 \sqrt{\beta_i}\sqrt{\beta_j}
-\sum_{(i,j)\in A^1\times A^1, i\neq j} \sqrt{\beta_i}\sqrt{\beta_j}\Biggr\},\]
where the supremum runs over all $\beta=(\beta_i)_{i\in A^1}$ such that $\beta_i\geq 0$ and $\sum_{i\in A^1} \beta_i=1$. Obviously $\widetilde{K}_0(z,W)=0$ and therefore $\widetilde r_2(z)=1$. 

Theorem \ref{Thmstructure} indicates that the $\widetilde T_2$ entropic curvature of the discrete hypercube is bounded from below by $\widetilde r_2=1$ and given that $\lambda_{1}=2$ in this case one gets that $\widetilde{\kappa}_{3}=1$. 

Since  $\sigma_i(\sigma_i(x))=x$ for any $i\in[n]$ and $z\in \X$, each moves $\sigma_i$ is used at most one time along any discrete geodesic. It follows that the $C$-displacement convexity property \eqref{deplacebis} holds with the cost $C_t(\widehat \pi)=\widetilde C_t^1(\widehat \pi)$, $t\in (0,1)$. 
%Observing that $\sigma_i(x)\in]x,y]$ or $\sigma_i(y)\in]y,x]$ if and only if $x_i\neq y_i$, easy computations give that 
As previously mentioned, the quantities $\Pi^{\sigma_i}_\rightarrow(x)$ and 
$\Pi^{\sigma_i}_\rightarrow(y)$ involved in the definition of the costs $\widetilde T_2$ and $\widetilde C_t^1(\widehat \pi)$ are given by \eqref{Pisigma}.
%\begin{align*}
%\[\Pi^{\sigma_i}_\rightarrow(x):=\int \1_{x_i\neq y_i} \,d\pi_{_\rightarrow}(y|x)\quad \mbox{and}\quad  
%\ \Pi^{\sigma_i}_\leftarrow(y):=\int \1_{x_i\neq y_i} %\,d\pi_{_\leftarrow}(x|y).\]
%\end{align*} 
%\begin{equation*}
%\Pi^{\sigma_{i}}_{\rightarrow}(x)=\int \1_{x_{i}\neq y_{i}}d\pi_{_\rightarrow}(y|x)  \quad \mbox{and}\quad \Pi^{\sigma_{i}}_{\leftarrow}(x)=\int \1_{x_{i}\neq y_{i}}d\pi_{_\leftarrow}(x|y). 
%\end{equation*} 
Thus, one exactly recovers the results of \cite[Theorem 2.5]{Sam21} for the uniform probability measure $\mu_0$ on the discrete hypercube. 
Applying Theorem \ref{Logsobbis}, one exactly recovers the modified logarithmic Sobolev inequalities given in \cite[Comments (d) of Theorem 2.5]{Sam21} for $\mu_0$. 

We want now to go a step further by considering perturbation measures $m_v$ of $m_0$. In order to apply Theorem \ref{entropicperturb}, one needs to estimate the quantity $D_sv(x,y)$ given  by \eqref{mulh} for any $x,y\in \{0,1\}^n$. For that purpose let us introduce some kind of discrete hessian for the potential $v$.
For any $z\in\{0,1\}^n$ ($n\geq 2$), and $\{i,j\}\subset [n]$, 
one denotes
\[z_{\overline{ij}}:=(z_1,\ldots,z_{i-1},z_{i+1},\ldots, z_{j-1},z_{j+1},\ldots z_n)\in \{0,1\}^{n-2},\]
and one uses the notation $z_{\overline{ij}}z_iz_j=z$.
 Let $Hv(z)$ denote the symmetric matrix with off-diagonal entries and for $i\neq j$, \[(Hv(z))_{ij}:=\partial_{ij}^2v(z_{\overline{ij}}),\]
where 
\[\partial_{ij}^2v(z_{\overline{ij}}):=v(z_{\overline{ij}}11)+v(z_{\overline{ij}}00)-v(z_{\overline{ij}}01)-v(z_{\overline{ij}}10).\]
%This matrix can be interpreted as a discrete form of  Hessian for the function $v$. 
The minimum and maximum eigenvalues of the symmetric matrix $Hv(z)$ are denoted, respectively, by $\lambda_{\rm min}(Hv(z))$ and $\lambda_{\rm max}(Hv(z))$. 
%Let also $\lambda_{\rm min}(Hv(z))$ and $\lambda_{\rm max}(Hv(z))$ respectively denote  the smaller and the bigger eigenvalue of the symmetric matrix $Hv(z)$.
We know that $\lambda_{\rm max}(Hv(z))\geq 0$ and $\lambda_{\rm min}(Hv(z))\leq 0$ since the matrix $Hv(z)$ has off diagonal. Let  also
\[\lambda_{\rm max}^\infty(Hv):=\max_{z\in\X}  \lambda_{\rm max}(Hv(z)) \quad \mbox{and}\quad \lambda_{\rm min}^\infty(Hv):=\min_{z\in\X}  \lambda_{\rm min}(Hv(z)).\] 
\begin{lemma}\label{Dvcube}
Let $v:\{0,1\}^n\to \R$ be a potential. If for any $z\in\{0,1\}^n$, the matrix $Hv(z)=V$ does not depend on $z$, then setting $v_{ij}:=\partial_{ij}^2v(z_{\overline{ij}})$,   one has for any $x,y\in \X$ with $d(x,y)\geq 2$,
\[\int_0^1 D_sv(x,y)\,q_t(s)\, ds= \frac{2\sum_{\{i,j\}\subset [n] }(x_i-y_i)(x_j-y_j)V_{ij}} {(d(x,y)(d(x,y)-1)}  \geq \frac{\lambda_{\min}(V)}{ d(x,y)-1}.\]
In any other cases we also have
\[\int_0^1 D_sv(x,y)\,q_t(s)\, ds\geq \frac{\lambda_{\rm min}^\infty(Hv)}{d(x,y)-1} \sum_{k=1}^{d(x,y)-1}\frac1k\geq \lambda_{\rm min}^\infty(Hv).\]
\end{lemma}
As an example, let $v$ be the potential  defined by 
\begin{equation}\label{vacances}
v(z)=\sum_{i\in [n]} u_i z_i +\frac{1}{2} \sum_{i,j\in [n], i\neq j} V_{ij}\, z_iz_j,
\end{equation}
where $u=(u_1, \ldots, u_n)\in \R^n$ and  $V=(V_{ij})_{i,j\in [n]}$ is a symmetric matrix of real coefficients  with off diagonal.  In that case $Hv(z)= V$ for any $z\in \X$, and  
Theorem \ref{entropicperturb} and Lemma \ref{Dvcube} imply that  the relative entropy with respect to $m_v$ satisfies the $C^v$-displacement convexity property \eqref{deplacebis}
 with for any $t\in (0,1)$,
 \begin{align*}
 C^v_t(\widehat \pi)&\geq \iint \Big(\frac4n c_t(d(x,y))+{\lambda_{ \min}(V)}\,d(x,y)\1_{d(x,y)\geq 2} \Big)d\widehat\pi(x,y) \\
 &\geq \iint d(x,y)\Big(\frac2n (d(x,y)-1) +\lambda_{\min}(V)\Big)\1_{d(x,y)\geq 2}\,d\widehat\pi(x,y) 
 \end{align*} 
Observe that one needs $|\lambda_{\min}(V)|=-\lambda_{\min}(V)$ smaller than constant over $n$ to get positive curvature from the last estimates. This condition is very strong in high dimension as regard to the condition we will get now  by applying directly Theorem \ref{thmprinc} on the space $(\X, d,L_v,m_v)$ as explained in section \ref{sectionpertu}.

For that purpose, let us first observe  that for any  $z\in\{0,1\}^n$ and any  $i\neq j $, 
\begin{align}\label{Dvijcube}
Dv(z,\sigma_i\sigma_j(z))&= v(\sigma_i\sigma_j(z))+v(z)-v(\sigma_i(z))-v(\sigma_j(z))\\
&=(2z_i-1)(2z_j-1)\,\partial_{ij}^2v(z_{\overline{ij}}).%=(2\overline z_i-1)(2\overline z_j-1)\,\partial_{ij}^2v(z_{\overline{ij}}).
\end{align}
Therefore, applying Theorem \ref{thmprinc}, the entropic curvature $\kappa^v$ of the space $(\X, d,L_v,m_v)$ is bounded from below by $r^v=-2\log\big(\max_{z\in\{0,1\}^n} K^v(z,S_2(z))\big)$ with, according to \eqref{defKLv}, 
\[K^v(z, S_2(z)):=\sup_\alpha\Big\{2\sum_{\{i,j\}\subset [n]} e^{-(2z_i-1)(2z_j-1)\,\partial_{ij}^2v(z_{\overline{ij}})/2} \alpha_i\alpha_j\Big\}.\]
In order to give estimates of this key quantity, let us introduce some notations.
For any $z\in\{0,1\}^n$, let $|Hv(z)|$ denotes the symmetric matrix with off diagonal and with coefficients $(|Hv(z)|)_{ij}:=|(Hv(z))_{ij}|$, $\{i,j\}\subset[n]$.  Setting 
\[|Hv(z)|_{\max}:=\max_{\{i,j\}\subset [n]} |(Hv(z))_{ij}|\quad \mbox{and}\quad |Hv|_{\max,\infty} :=\sup_{z\in \{0,1\}^n} |Hv(z)|_{\max},\]
since $Hv(z)$ is an off diagonal symmetric matrix, one easily checks that 
\[\big|\lambda_{\min}(Hv(z))\big|\leq \lambda_{\max}|Hv(z)|,\qquad  \big|\lambda_{\min}^\infty(Hv)\big|\leq \lambda_{\max}^\infty|Hv|,\]
and 
\[|Hv(z)|_{\max}\leq\min\big\{\big|\lambda_{\min}|Hv(z)|\big|,\lambda_{\max}|Hv(z)|)\big\},\qquad |Hv|_{\max,\infty}\leq\min\big\{|\lambda_{\min}^\infty(Hv)|,\lambda_{\max}^\infty(Hv)\big\}.\]

\begin{lemma}\label{lemcubeK_v} With the above notations, let 
\[r(v):= 1+ \frac{\lambda_{\min}^\infty(Hv)}2 - \frac{\lambda_{\max}^\infty|Hv|}2 \,k\Big(\frac{|Hv|_{\max,\infty}}{2}\Big) ,\]
with $k(s):=\frac1s(e^s-s-1)$, $s>0$.
For any $z\in \{0,1\}^n$ and for any $W\subset S_2(z)$, one has
\[ K^v(z, W)\leq 1-\inf_{\alpha}\Big\{r(v)\sum_{i\in A^1}\alpha_i^2\Big\}, \]
where the subset of indices $A^1\subset[n]$ is given by \eqref{hard}, and the infimum runs over all $\alpha$ with positive coordinates $\alpha_i$ satisfying $\sum_{i\in A^1}\alpha_i=1$. Moreover if $r(v)>0$, then it holds 
\[-\frac{\lambda_{\min}^\infty(Hv)}2\leq \widetilde K^v\leq -\frac{\lambda_{\min}^\infty(Hv)}2 + k\Big(\frac{|Hv|_{\max,\infty}}{2}\Big)= 1- r(v).\]
\end{lemma}
The proof of this lemma is postponed in Appendix B. Since $A^1=[n]$ for $W=S_2(z)$, the upper estimate of $K^v(z, W)$  with $W=S_2(z)$  implies that : for $r(v)\leq 0$ the  entropic curvature $\kappa^v$ of the space $(\X, d,L_v,m_v)$ satisfies 
\[\kappa^v\geq r^v\geq -2\log(1-r(v))\geq 2r(v),\]
and for $r(v)> 0$ one has 
\[\kappa^v\geq -2\log\Big(1-\frac{r(v)}n\Big)\geq \frac2n\,r(v)>0.\]

Applying also Theorem \ref{thmprincbis}  and Theorem \ref{Thmstructure} together with Lemma \ref{lemcubeK_v} we get the next result. 

\begin{proposition}\label{prophypercube}
On the discrete hypercube  $\X=\{0,1\}^n$, let $m_v$ denotes the measure with density $e^{-v}$ with respect to the counting measure $m_0$, with $v:\{0,1\}^n\to \R$. This measure is reversible with respect to the generator $L_v$ given by  
$L_v(x,y)=e^{\frac12(v(x)-v(y))}L_0(x,y)$ for any $x\neq y$. 
Let us assume that  for any $z\in \{0,1\}^n$, $K^v(z, S_2(z))<1$.
Then,  denoting $\kappa_1^v$ (respectively $\tilde\kappa^v$, $\overline{\kappa}^v$,$\tilde\kappa_2^v$) the $W_1$-entropic curvature of the space $(\X, d,L_v,m_v)$ (respectively the $\widetilde{T}$, $T_{\overline c}$ and $\widetilde T_2$-entropic curvature of the space), one has 
\[\kappa_1^v\geq 4 r_1^v,\quad \tilde\kappa^v \geq r^v/2, \quad  \overline{\kappa}^v \geq 4 \overline{r}^v,\quad  \widetilde\kappa_2^v\geq\widetilde r_2^v,\]
with, if $r(z)>0$, 
\[r_1^v\geq \frac{r(v)}n,\quad  r^v/2\geq-\log\Big(1-\frac{r(v)}n\Big)\geq  \frac{r(v)}n, \quad \overline{r}^v\geq \frac{r(v)}n,\quad  \widetilde r_2^v\geq r(v).\]
\end{proposition}
The inequality $\widetilde r_2^v=1-\widetilde K^v\geq r(v)$ is a consequence of the second part of Lemma \ref{lemcubeK_v}.  The lower bounds on $r_1^v$ and $\overline{r}^v$ follows from the first part of Lemma \ref{lemcubeK_v}, by adapting the  arguments we have used at the beginning of this section in order to estimate $r_1$ and $\overline{r}$. It suffices to observe that Lemma \ref{lemcubeK_v} implies that if $r(v)>0$ then for any $z\in\{0,1\}^n$ and any $W\subset S_2(z)$,
\[  K^v(z, W)\leq 1-r(v)\inf_{\alpha}\Big\{\sum_{i\in A^1}\alpha_i^2\Big\}=1-\frac{r(v)}{|A^1|}.\]
Then, the details of the proofs are left to the reader.

{\bf Comments:}
\begin{enumerate}[label=(\roman*)]
\item As example, if  the potential $v$ is given by \eqref{vacances} then 
\[r(v)=1+ \frac{\lambda_{\min}(V)}2 - \frac{\lambda_{\max}(|V|)}2 \,k\Big(\frac{|V|_{\max}}{2}\Big) .\]
where $V$ is a symmetric matrix with off diagonal.

If $V=0$, then $r(v)=1$ and $\mu_v$  is the product of Bernoulli measures with parameter $p_i=\frac{e^{u_i}}{1+e^{u_i}}$. Therefore,   all the lower bounds we get on entropic curvature are the same as for the uniform probability measure $\mu_0$. 

%If $-\beta \lambda_{\min}(\Sigma)\leq 2/n$, then inequality \eqref{condising} is satisfied. 
%And the inequality $1-\Big(1-\frac1n\Big)e^{\alpha/n}\geq \frac{1-\alpha}n$ for $n\geq 2$ and $\alpha\in [0,1]$, implies \[r_n(v)\geq \frac1n + \frac{\beta \lambda_{\min}(\Sigma)}2.\]  


\item Applying Theorem \ref{PL}, Corollary \ref{Transport}, Theorem \ref{Logsob} and Theorem \ref{Logsobbis} also provide new functional inequalities for the measure $m_v$ or $\mu_v=m_v/m_v\big(\{0,1\}^n\big)$ on the discrete cube.
In particular, if 
\begin{equation}\label{asump}
\lambda_{\max}(|V|) \,k\Big(\frac{|V|_{\max}}{2}\Big)-\lambda_{\min}(V)<2,
\end{equation}
then $\widetilde\kappa^v_2\geq r(v)>0$ and  according to  Theorem \ref{Logsobbis}, the measure $\mu_v$ satisfies the following modified 
logarithmic Sobolev inequality, for any positive function $f$ on $\{0,1\}^n$,
\begin{equation*}
{\rm Ent}_{\mu_v}(f)\leq  \int \sum_{i\in[n]} \frac{ r(v) }2 \, h^*\left(\frac{2}{r(v)} [\partial_{\sigma_i} \log f]_-\right)  f\,d\mu_v\leq \frac 1{r(v)} \int \sum_{i\in[n]} {[\partial_{\sigma_i} \log f]_-[\partial_{\sigma_i} f]_-} \,d\mu,
\end{equation*}
and also    the following Poincaré  inequality, for any real bounded function $g:\X\to \R$, 
\[{\rm Var}_{\mu_v}(g)\leq \frac{1}{2 r(v)} 
\int \sum_{i\in[n]}(\partial_{\sigma_i} g)^2 d\mu_v
.\]
By a simple change of variable these results can be transposed on the set $\{-1,1\}^n$ instead of $\{0,1\}^n$, in order to compare this result with the one of  Bauerschmidt-Bodineau \cite{BB19} and Eldan-Koehler-Zeitouni \cite{EKZ22}.   Their condition lies on the difference between the largest and the smallest eigenvalue of the symmetric matrix $V$, which is comparable to \eqref{asump}.
\end{enumerate}






 \subsection{The lattice $\Z^n$}
 
 
 
 In this part, let $m_0$ be the counting measure on $\X:=\Z^n$. Recall that the lattice $\Z^n$ is a structured graph with set of moves $\Sc:=\{\sigma_{i+},\sigma_{i-}\,|\, i\in[n]\}$ with 
 $\sigma_{i+}(z)=z+e_i$ and $\sigma_{i-}(z)=z-e_i$ for any $z\in\Z^n$. By Theorem \ref{thmstructure}, it has non-zero entropic curvature and since it is not a finite graph by Bonnet-Myers Theorem \ref{BM}, $\kappa=0$. 
 As an illustrative example, it is easy to see that 
 \[K_{0}\big(z,S_2(z)\big)= \sup_\alpha\Bigg[\sum_{i=1}^n \big(\alpha_{i+}^2+\alpha_{i-}^2\big) + \sum_{1\leq i<j\leq n}  2(\alpha_{i-}\alpha_{j-}+\alpha_{i-}\alpha_{j+}+ \alpha_{i+}\alpha_{j-}+\alpha_{i+}\alpha_{j+}\big) \Bigg]=1,\] 
 where the supremum runs over all $\alpha=(\alpha_{i+},\alpha_{i-})_{i\in [n]}$ with non negatives coordinates satisfying $\sum_{i\in [n]}(\alpha_{i+}+\alpha_{i-})=1$.

Let  $m_v$ be a measure  with  potential   $v:\Z^n\to\R$  with respect to $m_0$, $m_v=e^{-v} m_0$. 
As in the case of the discrete hypercube,  let us  define coefficients  that can be interpreted as local second partial derivatives. For any $i\in [n]$, let 
\[\partial^2_{ii}v(z):=v(z+e_i)+v(z-e_i)-2 v(z),\]
and for $\{i,j\}\subset [n]$,
\[\partial^2_{ij}v(z):=v(z+e_i+e_j)+v(z)-v(z+e_i)-v(z+e_j).\]
One may easily check that for any $i,j\in[n]$  and any $\varepsilon_i\in\{-1,1\}$, 
\begin{equation}\label{DvZn}
Dv(z,z+\varepsilon_i e_i+\varepsilon_j e_j)=\varepsilon_i\varepsilon_j \partial_{ij}v(z\wedge(z+\varepsilon_i e_i+\varepsilon_j e_j)),
\end{equation}
where for $z,w\in \Z^n$, $z\wedge w$ is the vector with coordinates $\min(z_i,w_i),i\in[n]$.

As an example, if  the potential $v$ is given by the sum of a quadratic and a linear form,
\begin{equation}\label{vquad}
    v(z):=\sum_{i\in [n]} u_i z_i +\frac{1}{2} \sum_{(i,j)\in [n]^2} V_{ij}\, z_iz_j,\quad z\in \Z^n,
\end{equation}
with $u=(u_1, \ldots, u_n)\in \R^n$ and  $V=(V_{ij})_{i,j\in [n]}$  a symmetric matrix of real coefficients, then easy computations give  
\[\partial_{ij}v(z\wedge(z+\varepsilon_i e_i+\varepsilon_j e_j))=V_{ij}\quad \mbox{and}\quad Dv(z,z+\varepsilon_i e_i+\varepsilon_j e_j)=\varepsilon_i\varepsilon_j V_{ij}\]
for any $i,j\in[n]$  and any $\varepsilon_i\in\{-1,1\}$.


Applying Theorem \ref{thmprinc}, we know that  the entropic curvature $\kappa^v$ of the space $(\Z^n, d,L_v,m_v)$ is bounded from below by $r^v=-2\log\big(\max_{z\in\{0,1\}^n} K^v(z,S_2(z))\big)$ with, according to \eqref{defKLv}, 
\begin{multline*}
K^v(z, S_2(z)):=\sup_\alpha\Big\{2\sum_{\{i,j\}\subset[n]}
\sum_{\varepsilon_i,\varepsilon_j\in\{-1,+1\}}  e^{-\varepsilon_i\varepsilon_j\,\partial_{ij}v(z\wedge(z+\varepsilon_i e_i+\varepsilon_j e_j))/2} \alpha_{\varepsilon_i}\alpha_{\varepsilon_j}\\
+ \sum_{i\in [n]} \sum_{\varepsilon_i\in\{-1,+1\}}
e^{-\partial_{ii}v(z\wedge(z+2\varepsilon_i ))/2} \alpha_{\varepsilon_i}^2 \Big\},
\end{multline*}
where the supremum runs over all vectors $\alpha$ with non-negative $\alpha_{i+},\alpha_{i-},i\in [n]$ such that $\sum_{i\in[n]} (\alpha_{i+}+\alpha_{i-})=1$, and with the notation $\alpha_{\varepsilon_i}=\alpha_{i+}$ if $\varepsilon_i=1$ and $\alpha_{\varepsilon_i}=\alpha_{i-}$ if $\varepsilon_i=-1$.

The next lemma gives one way to upper bound $K^v(z, S_2(z))$ in  general.
\begin{lemma}\label{KLvZ} Given a potential $v:\Z^n\to \R$, for any $z\in \Z^n$, let $Av(z)$  denotes the $n$ by $n$ symmetric matrix  defined by 
\[(Av(z))_{ii}:=e^{-a_{ii}(z)/2}-1,\quad i\in[n],\]  
with $a_{ii}(z):=\min\big(\partial_{ii}v(z),\partial_{ii}v(z-2e_i)\big), i\in[n]$ and 
\[(Av(z))_{ij}:=e^{a_{ij}(z)/2}-1,\quad \{i,j\}\subset[n],\]
with $\displaystyle a_{ij}(z):=\max\big\{-\partial_{ij}v(z), -\partial_{ij}v(z-e_i-e_j),\partial_{ij}v(z-e_i),\partial_{ij}v(z-e_j)\big\}.$\\
If $\lambda_{\max}(Av(z))\leq 0$, then one  has 
\[K^v(z, S_2(z))\leq 1+\frac{\lambda_{\max}(Av(z))}n \quad\mbox{and}\quad 
\widetilde K^v(z)=\sup_{W\subset S_2(z)}\widetilde K^v(z,W)\leq 1+  \lambda_{\max}(Av(z)) .\]
\end{lemma}
The hypothesis $\lambda_{\max}^\infty(Av):=\sup_{z\in \Z^n}\lambda_{\max}(Av(z)) \leq 0$ is not empty. For example,  if  $v$ is given by \eqref{vquad} then for any $z\in \Z^n$, one has $(Av(z))_{ii}=e^{-V_{ii}/2}-1$ and $(Av(z))_{ij}=e^{|V_{ij}|/2}-1$.  Clearly if for any $i\in [n]$, $V_{ii}>0$ and for any $\{i,j\}\subset[n]$, $V_{ij}=0$, then 
\[\lambda_{\max}^\infty(Av)\leq -\big(1-e^{-\,\min_{i\in[n]}(V_{ii})/2}\big)<0.\]
By a continuity argument if for any $i\in [n]$, $V_{ii}>0$ and the values of $|V_{ij}|$, $\{i,j\}\subset[n]$, are sufficiently small then  
$\lambda_{\max}^\infty(Av)<0$. 

For instance, assume that $V_{i,j}\geq 0$ for all $\{i,j\}\subset [n]$ and that the matrix $-Av(z)$ is \textit{diagonally dominant}, that is 
\[|(-Av(z))_{ii}|\geq \sum_{j,j\neq i} |(-Av(z))_{ij}|,\quad \forall i\in [n],\] 
or equivalently 
\begin{eqnarray}\label{diagdom}
n\geq e^{\frac{-V_{ii}}{2}}+\sum_{j\neq i}e^{\frac{V_{ij}}{2}},\quad \forall i\in [n].
\end{eqnarray}
As a consequence of the Gershgorin's circle theorem the biggest eigenvalue of  a diagonally dominant matrix with non negative diagonal entries is non-positive and therefore $\lambda_{\max}^\infty(Av)\leq 0$. Observe that the above inequality implies the  diagonal dominance of the matrix $V$, namely  $V_{ii}\geq \sum_{j\neq i} V_{ij}$ for all $i\in [n]$, and therefore $\lambda_{\max}(V)\leq 0$. 
And conversely, for sufficiently small and non-negative $V_{ij}$ (for all $\{i,j\}\subset [n]$),  the  inequality \eqref{diagdom} is also close to the  diagonal dominance property of the matrix $V$. As a comment,  the diagonal dominance of the matrix $V$ is known to be equivalent to a \textit{discrete midpoint convexity} property of the quadratic form $v(z)=\sum_{i,j\in [n], i\neq j} V_{ij}\, z_iz_j$ on $\Z^n$ (see details in \cite[Theorem 9]{TT21}). 

Note that the bounds on $K^v\big(z, S_2(z)\big)$ and $\widetilde K^v(z)$ given in Lemma \ref{KLvZ} can be  improved if the potential $v$ is given by \eqref{vquad} by introducing other matrices, and even more if the matrix $V$ is specified. Applying Theorem \ref{thmprinc}, Theorem \ref{thmprincbis} and Theorem \ref{Thmstructure} provides the following non-negative lower bounds on entropic curvatures  if $\lambda_{\max}^\infty(Av)\leq 0$.

%\textcolor{blue}{Deux remarques (lien potentiel $v$ et convexit{\'e} Murota). }
%\textcolor{blue}{
%Preliminaries.
%Let $f(x)=x^{T}Qx$ where $Q$ is an $n$ by $n$ symmetric matrix
%and $x\in \mathbb{Z}^{n}$. }\

%\textcolor{blue}{
%\textit{Lemma 7.1} Let $f(x)=x^{T}Qx$, the discrete midpoint convexity ($f(x)+f(y)\geq f(\lfloor(\frac{x+y}{2})\rfloor)+f(\lceil(\frac{x+y}{2})\rceil) $ for $x,y\in \mathbb{Z}^{n}$ is equivalent to
%\begin{equation*}
%    z^{T}Qz\geq 1_{J}^{T}Q1_{J}
%\end{equation*}
%for $z=x-y$ where $J:=\{i: z_{i} \hspace{0.1cm} \text(is) \hspace{0.1cm} \text{odd})\}$. Hence, $f$ is locally (resp. globally discrete midpoint convex iff the equation above hold for all $z\in \mathbb{Z}^{n}$ with $\mid \mid z\mid \mid_{\infty}=2$ (resp. $\mid \mid z\mid \mid_{\infty}\geq 2$.}

%\textcolor{blue}{
%\textit{Proof.} With the use of identities 
%\begin{equation*}
%\frac{1}{2}( \lfloor(\frac{x+y}{2})\rfloor+\lceil(\frac{x+y}{2})\rceil)=\frac{x+y}{2},
%\end{equation*}
%\begin{equation*}
%    f(x)+f(y)-2f(\frac{x+y}{2})=\frac{1}{2}(x-y)^{T}Q(x-y),
%\end{equation*}
%one can rewrite the discrete midpoint convexity to
%\begin{equation*}
%    (x-y)^{T}Q(x-y)\geq (\lceil(\frac{x+y}{2})\rceil)-\lfloor(\frac{x+y}{2})\rfloor)^{T}Q (\lceil(\frac{x+y}{2})\rceil)-\lfloor(\frac{x+y}{2})\rfloor).
%\end{equation*}}
%\textcolor{blue}{
%The substitution of $x-y=z$ and $(\lceil(\frac{x+y}{2})\rceil)-\lfloor(\frac{x+y}{2})\rfloor=1_{J}$ yields the result.} 

%\textcolor{blue}{
%\textit{Diagonal dominance of $Q$}:
%\begin{equation*}
%q_{ii}\geq \sum_{j\neq i} \mid q_{ij} \mid \hspace{0.2cm} (i=1,...,n).
%\end{equation*}
%}

%\textcolor{blue}{
%\textit{Proposition 7.2} If $Q$ is diagonally dominant then $f(x)=x^{T}Qx$ is \textcolor{red}{integrally convex ??} $*$ . The converse is also true if 
%$n\leq 2$.}\

%\textcolor{blue}{\textit{Proposition 7.3} $f(x)=x^{T}Qx$ is $L$-convex if and only if $Q$ is diagonally dominant and $q_{ij}\leq 0$ for all $i\neq j$.}\

%\textcolor{blue}{\textit{Proposition 7.5} For $n=2$, 
%$f(x)=x^{T}Qx$ is locally discrete midpoint convex iff 
%$q_{11}\geq \mid q_{12}\mid $ and $q_{22}\geq \mid q_{12}\mid$. %\textit{Proof.} Suppose that $f$ is locally discrete midpoint convex. By inequality of lemma, for 
%$z=(2,1)$ and $z=(2,-1)$ we obtain $q_{11}+q_{12}\geq 0$ and $q_{11}-q_{12}\geq 0$ i.e. $q_{11}-\mid q_{12}\mid \geq 0$. By %symmetry, $q_{22}-\mid q_{21}\mid \geq 0$. Conversely, these conditions imply the one of the lemma for all $z$ with $\mid \mid_{\infty} z\mid \mid_{\infty}=2$. }\

%\textcolor{blue}{\textit{Proposition 7.6} For $n=2$, 
%$f(x)=x^{T}Qx$ is globally discrete midpoint convex iff
%$q_{11}\geq \mid q_{12}\mid $, $q_{22}\geq \mid q_{12}\mid$ and %$q_{11}+q_{22}\geq \frac{5}{2} q_{12}$.}\




%\textcolor{blue}{
%\textit{Proposition 7.7} $f(x)=x^{T}Qx$ is globally discrete midpoint convex if 
%\begin{equation*}
%    \lambda_{min}^{Q}\geq \frac{n-1}{n+3} \lambda_{\max}^{Q}. 
%\end{equation*}
%}
%\textcolor{blue}{Two remarks:
%\begin{itemize}
%    \item Following the example $(39)$ if $\Sigma_{ii}>0$ for all 
%    $i\in [n]$ and for any $\{i,j\}\subset [n]$, $\Sigma_{ij}=0$ then $\Sigma$ is diagonally dominant . Thus by Proposition 7.2, the form $(39)$ is integrally convex . (La partie lin{\'e}aire je suis sur n embete pas ...).
%    \item If $n<9$ and $\frac{\lambda_{\max}(\Sigma)}{\lambda_{\min}(\Sigma)}<\frac{3}{2}$ then
 %   \begin{equation*}
 %       \frac{\lambda_{\max}(\Sigma)}{\lambda_{\min}(\Sigma)}< \frac{n+3}{n-1}
 %   \end{equation*}
 %   and by Proposition 7.7 the quadratic form $(39)$ 
 %   $x^{T}\Sigma x$ is globally discrete midpoint convex. (Il faut peut etre normaliser car la forme $(39)$ est $\frac{1}{2}x^{T}\Sigma x$ )
%\end{itemize}
%}

%\textcolor{blue}{$*$ For a point $x\in \mathbb{Z}^{n}$
%we consider its integer neighborhood $\frac{x+y}{2}$. 
%$N(x)=\{z\in \mathbb{Z}^{n}: \mid x_{i}-z_{i}\mid <1 (i=1...n)\}$ and the value at x of the convex envelope $f^{\sim}$ of $f$ can be represented as follows as 
%\begin{equation*}
%    f^{\sim}(x)=\min \{\sum_{z\in N(x)} \lambda_{z}f(z): \sum_{z\in N(x)}\lambda_{z}z=x , (\lambda_{z})\in \Lambda(x)\} (x\in \mathbb{R}^{n})
%\end{equation*}
%where $\Lambda(x)$ denotes the set of coefficients $(\lambda_{z}: z\in N(x))\in \mathbb{R}^{N(x)})$ for convex combinations indexed by $N(x)$ . A function $f:\mathbb{Z}^{n}\rightarrow \mathbb{R}\cup \{+\infty\}$ is called integrally convex if its local convex envelope $f^{\sim}:\mathbb{R}^{n}\rightarrow \mathbb{R}\cup \{+\infty\}$ is (globally) convex in the ordinary sense.}
%\textcolor{blue}{The following example is an example of a convex-extensible function that is not integrally convex.Example : Let $f:\mathbb{Z}^{2}\rightarrow \mathbb{R}$ be defined by  $f(x_{1},x_{2})=\mid 2x_{1}-x_{2}\mid$ for all $(x_{1},x_{2})\in \mathbb{Z}^{2}.$ This function is convex extensible with the convex envelope is given by $\hat{f}(x_{1},x_{2})=\mid 2x_{1}-x_{2}\mid$ for all $(x_{1},x_{2})\in \mathbb{R}^{2}$. For 
%$y=(1/2,1)$ we have 
%$N(y)=\{(0,1),(1,1)\}$ and their local convex extension $f^{\sim}$ of $f$ around y is given by 
%$f^{\sim}(1/2,1)=\frac{f(0,1)+f(1,1)}{2}=1$.
%On the other hand $y$ is the midpoint of $(0,0)$and $(1,2)$ ...this shows that $f^{\sim}$ is not convex . }
%\textcolor{blue}{Remark: weak discrete midpoint convexity
%\begin{equation*}
%    f(x)+f(y)\geq 2f^{\sim}(\frac{x+y}{2})
%\end{equation*}
%is a indeed a weak condition than discrete midpoint convexity since $\frac{1}{2}( \lfloor(\frac{x+y}{2})\rfloor+\lceil(\frac{x+y}{2})\rceil)=\frac{x+y}{2}$}

%\textcolor{blue}{Characterizations of integrally convex functions in terms of an inequality of the form 
%\begin{equation*}
%f^{\sim}(\frac{x+y}{2})\leq \frac{1}{2}(f(x)+f(y)) .
%\end{equation*}
%Proposition. If $f$ is integrally convex, then the inequality above holds. Proof.Obvious. 
%Il y a une recripoque si le domaine de $f$ est integrally convex (cad, si l'indicatrice du domaine est integrally convex)}


%\textcolor{blue}{Synthese. If $v=z^{T}\Sigma z:\mathbb{Z}^{n}\rightarrow \mathbb{R}\cup \{+\infty\}$ and $\Sigma$ is diagonally dominant then $v$ is integrally convex and thus satisfies the weak discrete midpoint convexity 
%\begin{equation*}
%    v(x)+v(y)\geq 2v^{\sim}(\frac{x+y}{2}) \hspace{0.2cm} \forall x,y\in \mathbb{Z}^{n}
%\end{equation*}
%}

%\textcolor{blue}{Murota {\'e}crit : "If $Q$ is diagonally dominant then $f(x)=x^{T}Qx$ is integrally convex. The converse is also true if $n\leq 2$. Recently it has been shown that the diagonal dominance of $Q$ is equivalent to the directed midpoint convexity  of $f(x)=x^{T}Qx$". Il semble que cela est plus fort car dans l'article %en question c 'est ecrit que les fonctions (DDM) est un sous ensemble %des integrally convex functions}.
%\textcolor{blue}{Directed Discrete midpoint convexity. For an ordered pair $(x,y)$ of $x,y\in \mathbb{Z}^{n}$, $\mu(x,y)\in \mathbb{Z}^{n}$ is defined by
%$\mu(x,y)_{i}=\lceil(\frac{x_{i}+y_{i}}{2})\rceil$ if $x_{i}\geq y_{i}$ and %$\mu(x,y)_{i}=\lfloor(\frac{x_{i}+y_{i}}{2})\rfloor$ if 
%$x_{i}<y_{i}$. %A function $f:\mathbb{Z}^{n}\rightarrow \mathbb{R}\cup %\{+\infty\}$ satisfies directed discrete midpoint convexity (DDM-convexity) if 
%\begin{equation*}
%    f(x)+f(y)\geq f(\mu(x,y))+f(\mu(y,x)) 
%\end{equation*}
%for all $x,y\in \mathbb{Z}^{n}$. $S\subset \mathbb{Z}^{n}$ is called %a directed discrete if midpoint if its indicator function is %DDM-convex, that is if
%$x,y\in S\implies \mu(x,y),\mu(y,x)\in S$ holds.
%}
%\textcolor{blue}{Theorem. For a quadratic function 
%function $f(x)=x^{T}Qx( x\in \mathbb{Z}^{n})$ with a symmetric %matrix $Q$, f is DDM-convex if and only if Q is diagonally %dominant.(je n ai pas lu la preuve)}


\begin{proposition} With the above notations, if the graph space $(\Z^n,d,L_v,m_v)$ with $m_v=e^{-v} m_0$ is such that the potential $v$ satisfies $\lambda_{\max}^\infty(Av)\leq 0$, then the entropic curvature $\kappa^v$, the $W_1$-entropic curvature $\kappa^v_1$, the $\widetilde{T}$-entropic curvature $\widetilde\kappa^v$ and the $\widetilde T_2$-entropic curvature $\widetilde\kappa^v_2$ of this space are non-negative, namely
\[\kappa^v\geq 2\log\big(1+{\lambda_{\max}^\infty(Av)}/n\big)\geq -\frac{2\lambda_{\max}^\infty(Av)}n\qquad
\kappa^v_1\geq -\frac{2\lambda_{\max}^\infty(Av)}n,\]
\[\widetilde\kappa^v\geq -\frac{\lambda_{\max}^\infty(Av)}n,\qquad \widetilde\kappa_2^v\geq -\lambda_{\max}^\infty(Av).\]
\end{proposition}
For any integers $d, k_1,\ldots ,k_n$ such that $d=k_1+\cdot +k_n$, let $\binom{d }{k_1,\ldots,k_n}=\frac{n!}{k_1!\cdots k_n!}$ denote the multinomial coefficient. If $\lambda_{\max}^\infty(Av)< 0$ then 
Theorem \ref{PL} applies with cost $c(d)=d(d-1)$ and $\kappa_c=\kappa^v$  and provides a Prékopa-Leindler type of inequality for the measure $m_v$ with the Schr\"odinger bridge between Dirac measures at $x$ and $y$ in $\Z^n$ given by 
 \begin{align*}
  \nu_t^{x,y}(z)&=\frac{\binom{d(x,z)}{|z_1-x_1|,\ldots,|z_n-x_n|}\binom{d(z,y)}{|y_1-z_1|,\ldots,|y_n-z_n|}}{ \binom{d(x,y)}{|y_1-x_1|,\ldots,|y_n-x_n|} } \binom{d(x,y)}{d(x,z)} \;t^{d(x,z)} (1-t)^{d(z,y)} \1_{[x,y]}(z)\\
  &=\binom{|y_1-x_1|}{|z_1-x_1|} \cdots \binom{|y_n-x_n|}{|z_n-x_n|}\;t^{d(x,z)} (1-t)^{d(z,y)} \1_{[x,y]}(z),\qquad z\in \Z^n,
  \end{align*}
since 
\[
L_0^{d(x,y)}(x,y)= \# G(x,y)= \binom{d(x,y)}{|y_1-x_1|,\ldots,|y_n-x_n|}.
\]

Assuming moreover that $m_v(\Z^n)<\infty$, Corollary \ref{Transport} and the first part of Theorem \ref{Logsobbis} provide transport-entropy inequalities for the probability measure $\mu_v=m_v/m_v(\Z^n)$ involving the costs $W_1,T_2,\widetilde{T}$ and $\widetilde T_2$ given by 
\begin{multline*}\widetilde T_2(\pi):=\sum_{i\in [n]} \Big(\int [y_i-x_i]_+ \,d\pi_{_\rightarrow}(y|x)\Big)^2 d\nu_0(x)+\sum_{i\in [n]}\Big(\int [y_i-x_i]_- \,d\pi_{_\rightarrow}(y|x)\Big)^2 d\nu_0(x)\\
 + \sum_{i\in [n]} \Big(\int [x_i-y_i]_+ \,d\pi_{_\leftarrow}(x|y)\Big)^2 d\nu_1(y)+\sum_{i\in [n]}\Big(\int [x_i-y_i]_- \,d\pi_{_\leftarrow}(x|y)\Big)^2 d\nu_1(y),
 \end{multline*}
 for any $\pi\in \Pi(\nu_0,\nu_1)$.
 This expression is a consequence of the following identities, for any $x,y\in \Z^n$,
\[\1_{\sigma_{i+}(x)\in]x,y]}r(x,\sigma_{i+}(x),\sigma_{i+}(x), y)=[y_i-x_i]_+ \quad\mbox{and}\quad\1_{\sigma_{i+}(x)\in]x,y]}r(x,\sigma_{i-}(x),\sigma_{i-}(x), y)=[y_i-x_i]_-.\]

Theorem \ref{Logsobbis} also provides a modified logarithmic Sobolev inequality and a Poincaré inequality for the probability measure $\mu_v$. Namely if $\lambda_{\max}^\infty(Av)< 0$, then $\widetilde \kappa_2>-\lambda_{\max}^\infty(Av)$ and for any bounded function $f:\Z^n\to [0,+\infty)$, 
 \begin{equation*}
{\rm Ent}_{\mu_v}(f)\leq \frac{1}{2\widetilde \kappa_2} \int \sum_{i\in [n]}   \big([\partial_{\sigma_{i+}} \log f]_-^2 + [\partial_{\sigma_{i-}} \log f]_-^2\big) f\,d\mu_v,
\end{equation*}
and for any real bounded function $g:\Z^n\to \R$, 
\[{\rm Var}_{\mu_v}(g)\leq \frac{1}{2\widetilde{\kappa}_2} 
\int \sum_{i\in[n]}\big((\partial_{\sigma_i+} g)^2+(\partial_{\sigma_i-} g)^2 \big)\,d\mu_v
.\]


 





%\textcolor{blue}{Let the positive and negative supports of $z\in \mathbb{Z}^{N}$ defined as $supp^{+}(z)=\{i\mid z_{i}>0\}$ and 
%$supp^{-}(z)=\{j\mid z_{j}<0\}$
%. Kazuo Murota %introduced the notion of 
%$M^{\natural}$-concave functions on $\mathbb{Z}^{N}$.
%\begin{definition}[Axiom of exchange property]
%A function $f:\mathbb{Z}^{N}\rightarrow \mathbb{R}\cup\{-\infty\}$ with $dom(f)\neq\emptyset$ is $M^{\natural}$-concave if for any 
%$x,y\in \mathbb{Z}^{N}$ and $i\in supp^{+}(x-y)$ there exists some $j\in supp^{-}(x-y)$ such that 
%\begin{equation*}
%    f(x)+f(y)\leq f(x-e_{i}+e_{j})+f(y+e_{i}-e_{j}).
%\end{equation*}
%\end{definition}
%The local exchange property admits a reformulation via a Discrete Hessian .
%\begin{definition}
%For $x\in \mathbb{Z}^{N}$ and $i,j\in N$
%\begin{equation*}
%    H_{ij}(x)=f(x+e_{i}+e_{j})-f(x+e_{i})-f(x+e_{j})+f(x).
%\end{equation*}
%Let $H_{f}(x)=(H_{ij}(x)\mid i,j\in N)$ be the  matrix consisting of those components. 
%\end{definition}
%M-concave functions can be defined via some properties of the discrete Hessian.
%\textbf{Theorem:}\\A function $f:\mathbb{Z}^{n}\rightarrow$ is $M^{\natural}$-concave if and only if the discrete Hessian matrix $H_{f}(x)=(H_{ij}(x))$ satisfies the following conditions for each $x\in \mathbb{Z}^{N}$
%\begin{equation*}
 %   H_{ij}(x)\leq 0 \hspace{0.4cm} \text{for any} (i,j) \hspace{0.1cm},
%\end{equation*}
%\begin{equation*}
% H_{ij}(x)\leq \max(H_{ik}(x),H_{jk}(x)) \hspace{0.2cm} if \hspace{0.2cm} \{i,j\}\cap \{k\}=\emptyset.
%\end{equation*}
%It is known $[]$ that an $M^{\natural}$-concave function 
%$f:\mathbb{Z}^{N}\rightarrow \mathbb{R}\cup \{-\infty\}$ is submodular on the integer lattice. The reciprocal is not true, $M^{\natural}$-concave functions form a proper subclass of submodular functions on $\mathbb{Z}^{n}$. By Theorem $[]$ , any $M^{\natural}$-concave function is concave extensible. Example: For a quadratic function $f:\mathbb{Z}^{N}\rightarrow \mathbb{R}$ defined by $f(x)=\sum_{i=1}^{n}\sum_{j=1}^{n}a_{ij}x_{i}x_{j}$ with $a_{ij}=a_{ji}$ for all $i,j\in [N]$, $H_{ij}(x)=2a_{ij}$. So, by the Theorem a quadratic function $f$ is 
%$M^{\natural}$ concave  if and only if 
%$a_{ij}\leq 0$  for all $(i,j)$ and $a_{ij}\leq %\max(a_{ik},a_{jk})$ if $\{i,j\}\cap \{k\}=\emptyset $ .
%Je viens de faire que du recopiage mais ce que je %vois c est qu'il existe une hessiene discrete pour %$\mathbb{Z}^{N}$ qui a des bonnes proprietes. Elle %est egale (sauf un moins a) %$H=-\partial^{2}_{i+j+}v(z)$ mais cela seulement %dans ce cas .  }




%\subsection{Slices of the hypercube or the Bernoulli-Laplace model}
% Let  $\mu_0$ be  the uniform probability measure on $\X=\X_\kappa$, the slice of the discrete hypercube $\{0,1\}^n$ of order $k\in[n-1]$,
%\[ \X_\kappa:=\left\{x=(x_1,\ldots,x_n)\in \{0,1\}\,\big| \,x_1+\ldots +x_n=\kappa\right\}.\]
%For any $z\in \X_\kappa$, one has $\mu_0(z)=1/|\X_\kappa|=1/\binom{\kappa}{n}$.
%For any $\{i,j\}\subset [n]$, let $\sigma_{ij}:\X_\kappa\to\X_\kappa$ denote the one to one functions that exchanges the value of coordinate $i$ with the one of coordinate $j$: namely, for any $z=(z_1,\ldots,z_n)\in \X_\kappa $ 
%\[(\sigma_{ij}(z))_j:=z_i,\qquad (\sigma_{ij}(z))_i:=z_j,\]
%and for any $k\in[n]\setminus\{i,j\}$,
%$(\sigma_{ij}(z))_k=z_k$.
%We note $J_0(z):=\{i\in [n]\,|\, z_i=0\}$ and $J_1(z):=\{i\in [n]\,|\, z_i=1\}$.
%For  any $i\in J_0(z)$ and $j\in J_1(z)$,    $\sigma_{ij}(z)$ denotes the neighbour of $z$ in $\X_\kappa$  defined by
%\[\left(\sigma_{ij}(z)\right)_i=1,\quad\left(\sigma_{ij}(z)\right)_j=0,\]
%and for any $\ell\in[n]\setminus\{i,j\}$, $\left(\sigma_{ij}(z)\right)_\ell=z_\ell$.
%One has 
%\[S_1(z)=\Big\{\sigma_{ij}(z)\,\Big |\, i\in J_0(z), j\in J_1(z)\Big \},\]
%and 
%\[S_2(z)=\Big\{\sigma_{kl}\sigma_{ij}(z)\,\Big |\, i,k\in J_0(z),i<k, j,l\in J_1(z), j<l\Big \}.\]
%Notice that for $\kappa=1$ or $\kappa=n-1$, the set $\V(z)$ is empty since in these cases, $\X_\kappa$ is the complete graph with $n$ vertices. 
%Let $I_0:=\{(i,k)|i,k\in J_0(z),i<k\}$ and $I_1:=\{(j,l)|j,l\in J_1(z),j<l\}$. Let us consider $V\subset S_1(z)$ and $W\subset S_2(z)$ such that \eqref{condVW} holds. Assume $W\neq \emptyset$, then there exist non empty subsets $A_0\subset I_0$, $A_1\subset I_1$ and  $C\subset A_0\times A_1$ such that 
%\[W=\Big\{\sigma_{kl}\sigma_{ij}(z)\,\Big|\, ((i,k),(j,l))\in C \Big\}.\]
%Since for $(i,k)\in A_0, (j,l)\in A_1$, 
%\[[z, \sigma_{kl}\sigma_{ij}(z)]\cap S_1(z)=\{\sigma_{kl}(z),\sigma_{ij}(z),\sigma_{il}(z), \sigma_{kj}(z)\},\]
%one has 
%\[V\supset\Big\{\sigma_{kl}(z),\sigma_{ij}(z),\sigma_{il}(z), \sigma_{kj}(z)\,\Big|\, ((i,k),(j,l))\in C\Big\},\] 
%and \eqref{defRbis} gives 
%\[K(z,V,W):=\sup_\alpha \sum_{((i,k),(j,l))\in C} 4 \left(\alpha_{ij} \alpha_{kl}\alpha_{il} \alpha_{kj} \right)^{1/2},\]
%where the supremum runs over all vectors $\alpha=(\alpha_{ij})_{(i,j)\in J_0(z)\times J_1(z)}$ with non negative coordinates such that $\sum_{i\in J_0(z)}\sum_{j\in J_1(z)}\alpha_{ij}=1$.
%By denoting $B_0:=\bigcup_{(i,k)\in A_0}\{i,k\}\neq \emptyset$, and $B_1:=\bigcup_{(j,l)\in A_1}\{j,l\}\neq \emptyset$, one gets
%\begin{align*}
%K(z,V,W)&\leq \sup_\alpha \sum_{(i,k)\in A_0,(j,l)\in A_1} 4 \left(\alpha_{ij} \alpha_{kl}\alpha_{il} \alpha_{kj} \right)^{1/2}\\
%&=\sup_\alpha  \sum_{(j,l)\in B_1\times B_1, j\neq l}\Bigg[\Big(\sum_{i\in B_0} (\alpha_{ij} \alpha_{il})^{1/2}\Big)^2-\sum_{i\in B_0} \alpha_{ij} \alpha_{il}\Bigg]
%\end{align*}
%where the supremum runs over all vectors $\alpha=(\alpha_{ij})_{(i,j)\in B_0\times B_1}$ with non negative coordinates such that $\sum_{i\in B_0}\sum_{j\in B_1}\alpha_{ij}=1$.
%Applying Cauchy-Schwarz inequality, it follows that
%\begin{align*}
%K(z,V,W)&
%\leq \sup_\alpha  \sum_{(j,l)\in B_1\times B_1, j\neq l}\Bigg[\Big(\sum_{i\in B_0} \alpha_{ij}\Big)\Big(\sum_{i\in B_0} \alpha_{il}\Big)-\sum_{i\in B_0} \alpha_{ij} \alpha_{il}\Bigg]\\
%&=\sup_\alpha  \Bigg[\Big(\sum_{(i,j)\in B_0\times B_1}\alpha_{ij}\Big)^2-\sum_{j\in B_1}\Big(\sum_{i\in B_0} \alpha_{ij}\Big)^2  - \sum_{i\in B_0}\Big(\sum_{j\in B_1} \alpha_{ij}\Big)^2  + \sum_{(i,j)\in B_0\times B_1}\alpha_{ij}^2\Bigg]\\
%&=1-\inf_\alpha\Bigg[\sum_{j\in B_1}\Big(\sum_{i\in B_0} \alpha_{ij}\Big)^2  + \sum_{i\in B_0}\Big(\sum_{j\in B_1} \alpha_{ij}\Big)^2  - \sum_{(i,j)\in B_0\times B_1}\alpha_{ij}^2\Bigg]\\
%&\leq 1-\inf_\alpha \max\Bigg[\sum_{j\in B_1}\Big(\sum_{i\in B_0} \alpha_{ij}\Big)^2, \sum_{i\in B_0}\Big(\sum_{j\in B_1} \alpha_{ij}\Big)^2\Bigg]\\
%&\leq 1-\max\Big[\frac{1}{|B_1|},\frac{1}{|B_0|}\Big].
%\end{align*}
%where we use again Cauchy-Schwarz inequality for the last inequality. Since $|B_1|\leq |J_1(z)|\leq \kappa$ and $|B_0|\leq |J_0(z)|\leq n- \kappa$, one gets \[ K\leq 1-\max\Big[\frac{1}{\kappa},\frac{1}{n-\kappa}\Big] .\]

%In order to estimate constant $C$, assume first that in the definition \eqref{defC}, $W_+$ and $W_-$ are two non-empty subsets of $S_2(z)$. 
%There exist $A_{0+}\subset I_0$, $A_{1+}\subset I_1$. $A_{0-}\subset I_0$, $A_{1-}\subset I_1$ and some  subsets $C_+\subset A_{0+}\times A_{1+}$, $C_-\subset A_{0-}\times A_{1-}$ such that 
%\[W_+=\Big\{\sigma_{kl}\sigma_{ij}(z)\,\Big|\, ((i,j),(k,l))\in C_+ \Big\}\quad\mbox{and}\quad
%W_-=\Big\{\sigma_{kl}\sigma_{ij}(z)\,\Big|\, ((i,j),(k,l))\in C_- \Big\}.\]
%As above, defining the sets $B_{0-}$ (respectively  $B_{1-}, B_{0+}, B_{1+}$) as the union of the projections of $A_{0-}$  (respectively  $A_{1-}, A_{0+}, A_{1+}$) on the first and  the second coordinate, condition  \eqref{condVW} applied with $(V_+,W_+)$ and $(V_-,W_-)$, together with $V_-\cap V_+=\emptyset$,  ensure that $B_{0-}\cap B_{0+}=\emptyset$ and $B_{1-}\cap B_{1+}=\emptyset$. Indeed, if for example $B_{0-}\cap B_{0+}\neq\emptyset$ then there exists $i,k,k'\in J_0(z)$ and $j,j',l,l'$ (not necessarily all distinct) such that $((i,j),(k,l))\in C_+$ and $((i,j'),(k',l'))\in C_-$. Condition \eqref{condVW} together with $V_-\cap V_+=\emptyset$ imply $z\in[\sigma_{ij}(z),\sigma_{ij'}(z)]$. This is impossible since on the graph $\X_\kappa$, $d(\sigma_{ij}(z),\sigma_{ij'}(z))\leq 1$. Using the above estimate of the function $K$, one gets
%\begin{align*}
%\frac{\1_{V_+\neq \emptyset}}{1-K(z,V_+,W_+)}+\frac{\1_{V_-\neq \emptyset}}{1-K(z,V_-,W_-)}&\leq   \min\big(|B_{0+}|,|B_{1+}|\big)+\min\big(|B_{0-}|,|B_{1-}|\big)\\
%&\leq \min\big(|B_{1-}|+|B_{1+}|,|B_{0-}|+|B_{0+}| \big)\\
%&\leq \min(\kappa,n-\kappa).
%\end{align*}
%If in the definition \eqref{defC}, one  assume that $W_+=\emptyset$ and $W_-\neq \emptyset$ then since $V_+\cap V_-=\emptyset$ one gets 
%\begin{align*}
%\frac{\1_{V_+\neq \emptyset}}{1-K(z,V_+,W_+)}+\frac{\1_{V_-\neq \emptyset}}{1-K(z,V_-,W_-)}&\leq  \1_{V_+\neq \emptyset} + \min\big(|B_{0-}|,|B_{1-}|\big)\\
%&= \min\big(\1_{V_+\neq \emptyset}+|B_{1-}|,\1_{V_+\neq \emptyset}+|B_{0-}|\big)\\
%&\leq \min(\kappa,n-\kappa).
%\end{align*}
%It remains to consider the case where $W_+=W_-=\emptyset$. Observe that if $\kappa\notin\{1,n-1\}$, then   $\1_{V_+\neq \emptyset}+\1_{V_-\neq \emptyset}\leq 2\leq \min(\kappa,n-\kappa)$. If  $\kappa\in\{1,n-1\}$, then $\X_\kappa$  is the complete graph and therefore for any $x,y\in \X_\kappa$, $d(x,y)\leq 1$. The condition $d(V_+, V_-)\geq 2$ for $V_+\neq \emptyset$ and , $V_-\neq \emptyset$ implies that either $V_+= \emptyset$ or $V_-= \emptyset$ in the definition \eqref{defC}, it follows that $\1_{V_+\neq \emptyset}+\1_{V_-\neq \emptyset}\leq 1\leq \min(\kappa,n-\kappa)$.

%As a conclusion for any subsets $V_-,V_+,W_-,W_+$ satisfying the conditions of the definition \eqref{defC} of $C(z)$, one has
%\[\frac{\1_{V_+\neq \emptyset}}{1-K(z,V_+,W_+)}+\frac{\1_{V_-\neq \emptyset}}{1-K(z,V_-,W_-)}\leq \min(\kappa,n-\kappa).\]
%It follows that 
%\[C\geq \max\Big[\frac{1}{\kappa},\frac{1}{n-\kappa}\Big].\]
%This lower bound is optimal according to \cite[Proposition 4.1]{GRST14}.

%\subsection{The transposition model}
% Let $\X=S_n$ denotes the group of permutations on the set $[n]$ ($n\geq 2$), and $\mu_0$ denotes  the uniform probability measure on $S_n$, $\mu_0(z)=1/n!$ for any $z\in S_n$.  For any $z\in S_n$ and $\{i,j\} \subset [n]$, $i\neq j$, let $\sigma_{ij}(z)$ be   the neighbour of $z$  that differs from $z$ by a transposition $(ij)$
% \[\sigma_{ij}(z):=z\,(ij).\]
% The map $\sigma_{ij}:S_n\to S_n$ is one to one.
%Thus,   the graph distance between two permutations $x$ and $y$ is the  \textit{ transposition distance} $d(x,y)$ that corresponds to the minimal number of transpositions $ \tau_1,..., \tau_k$ such that $x \tau_1\cdots \tau_k=y$.
%Given $z\in S_n$, one has 
%\[S_1(z)=\Big\{\sigma_{ij}(z)\,\Big |\, \{i,j\}\in I\Big \}\quad \mbox{with}\quad I=\Big\{\{i,j\}\,\Big|\,1\leq i<j\leq n\Big\}.\]
%and $S_2(z)=\V_1(z)\cup\V_2(z)$ with 
%\[\V_1(z)=\Big\{\sigma_{kl}\sigma_{ij}(z)\,\Big |\, (\{i,j\},\{k,l\})\in \mathbb{I}\Big\},\]
%with 
%\[ \mathbb{I}:=\Big\{(\{i,j\},\{k,l\})\,\Big|\,\{i,j\},\{k,l\}\in I, \{i,j\}\cap\{k,l\}=\emptyset \Big \},\]
%and 
%\[\V_2(z)=\Big\{z(ijk), z(ikj)\,\Big |\, (i,j,k)\in \mathbb{J} \}\quad \mbox{with}\quad \mathbb{J}=\Big\{(i,j,k)\,\Big|\,1\leq i<j<k \leq n\Big\},\]
%where $(ijk)$ denotes a cycle of length 3 in $S_n$.
%Let us consider $V\subset S_1(z)$ and $W\subset S_2(z)$, $W\neq \emptyset$,  such that \eqref{condVW} holds.  There exists $A\subset \mathbb{I}$ and  $B\subset \mathbb{J}$, $B'\subset \mathbb{J}$, such that 
%\[W=\Big\{\sigma_{kl}\sigma_{ij}(z)\,\Big |\, (\{i,j\},\{k,l\})\in A\Big\}\cup \Big\{z(ijk), z(i'k'j')\,\Big |\, (i,j,k)\in B,  (i',j',k')\in B'  \Big\}.\]
%Since for $((i,j),(k,l))\in \mathbb{I}$, 
%\[[z,\sigma_{kl}\sigma_{ij}(z)]\cap S_1(z)=\big\{\sigma_{kl}(z),\sigma_{ij}(z)\big\},\]
%and for $(i,j,k)\in \mathbb{J}$,
%\[ [z,z(ijk)]\cap S_1(z)=\big\{\sigma_{ij}(z),\sigma_{jk}(z), \sigma_{ik}(z)\big\},\]
%it follows that 
%\[V\supset \Big\{\sigma_{kl}(z),\sigma_{ij}(z)\,\Big|\,(\{i,j\},\{k,l\})\in A\Big\}\\\cup \Big\{\sigma_{ij}(z),\sigma_{jk}(z), \sigma_{ik}(z) \,\Big|\, (i,j,k)\in B\cup B' \Big\},\]
%and
%\[
%K(z,V,W)
%= \sup_\alpha \Bigg[ \frac12\sum_{(\{i,j\},\{k,l\})\in A}2 \alpha_{ij}\alpha_{kl}+ \sum_{(i,j,k)\in B} 3(\alpha_{ij}\alpha_{jk}\alpha_{ik})^{2/3} + \sum_{(i,j,k)\in B'} 3(\alpha_{ij}\alpha_{jk}\alpha_{ik})^{2/3}\Bigg],
%\]
%where the supremum runs over all vectors $\alpha=(\alpha_{ij})_{\{i,j\}\in I}$ with non negative coordinates with $\sum_{\{i,j\}\in I}\alpha_{ij}=1$.
%Let us define  \[E=\Big(\bigcup_{(\{i,j\},\{k,l\})\in A} \big\{\{i,j\},\{k,l\}\big\}\Big)\cup\Big(\bigcup_{(i,j,k)\in B\cup B'} \big\{\{i,j\},\{i,k\},\{j,l\}\big\}\Big),
%\] and 
%\[D= \Big\{\{i,j,k\}\,\Big|\,\{i,j\}\in E, \{i,k\}\in E, \{j,k\}\in E\Big\}.\]
%The set $E$ is not empty since $W\neq \emptyset$.
%After rearrangements of the sums and using  the following arithmetic-geometric mean inequality $(\beta_1\beta_2\beta_3)^{1/3}\leq \frac13(\beta_1+\beta_2+\beta_3)$, $\beta_1,\beta_2,\beta_3>0$, one gets
%\begin{align*}
%K(z,V,W)
%&\leq  \sup_\alpha \Bigg[ \sum_{(\{i,j\},\{k,l\})\in E\times E, \{i,j\}\cap \{k,l\}=\emptyset}  \alpha_{ij}\alpha_{kl}+ 6 \sum_{\{i,j,k\}\in D} (\alpha_{ij}\alpha_{jk}\alpha_{ik})^{2/3} \Bigg],\\
%&= \sup_\alpha \Bigg[ \sum_{((i,j),(k,l))\in \mathbb I}2 \alpha_{ij}\alpha_{kl}+ \sum_{(i,j,k)\in \mathbb J} 6(\alpha_{ij}\alpha_{jk}\alpha_{ik})^{2/3} \Bigg]\\
%&\leq \sup_\alpha \Bigg[ \Big(\sum_{\{i,j\}\in E } \alpha_{ij}\Big)^2 - \sum_{\{i,j\}\in E } \alpha_{ij}^2 - \sum_{\{i,j,k\}\in D}  2\left (\alpha_{ij}\alpha_{jk}+ \alpha_{ij}\alpha_{ik}+\alpha_{ik}\alpha_{jk}\right) \\
%&\qquad\qquad \qquad \qquad\qquad \qquad \qquad\qquad \qquad + \sum_{\{i,j,k\}\in D} 6(\alpha_{ij}\alpha_{jk}\alpha_{ik})^{2/3}  \Bigg]\\
%&\leq 1-\inf_\alpha \sum_{\{i,j\}\in E } \alpha_{ij}^2\leq 1-\frac1{|E|},
%\end{align*}
%where the last inequality follows from Cauchy-Schwarz inequality.

%Choosing $V=S_1(z)$ and $W=S_2(z)$, its provides  
%\begin{equation*}
%K(z,S_1(z), S_2(z))
%= \sup_\alpha \Bigg[ \sum_{((i,j),(k,l))\in \mathbb I}2 \alpha_{ij}\alpha_{kl}+ \sum_{(i,j,k)\in \mathbb J} 6(\alpha_{ij}\alpha_{jk}\alpha_{ik})^{2/3} \Bigg]\leq 1-\frac1{|I|}= 1-\frac2{n(n-1)}
%\end{equation*}
%Actually, this bound is sharp since one has equality  for the vector $\alpha$ with all coordinates $\alpha_{ij}=1/|I|$. Therefore one has $K(z,S_1(z),S_2(z))=1-\frac2{n(n-1)}=K$ for any $z\in S_n$, and  \[\kappa\geq-2\log\Big(1-\frac2{n(n-1)}\Big)\geq \frac4{n(n-1)}\qquad \mbox{and}\quad \widetilde\kappa\geq \frac2{n(n-1)}.\]

%In order to estimate the constant $C$, let $V_+,V_-\subset S_1(z)$, $W_+,W_-\subset S_2(z)$ be some sets satisfying the needed conditions in the definition \eqref{defC} of $C(z)$.  Assume first that $W_+\neq \emptyset$ and $W_-\neq \emptyset$. Using similar notations as above, one defines a set $E_+\subset I$ associated to $W_+$ and a set $E_-\subset I$ associated to $W_-$.
% One easily shows that $E_+\cap E_-=\emptyset$ (otherwise there exists $\{i,j\}\in I$ such that $z\in[\sigma_{ij}(z),\sigma_{ij}(z)]$).
%It follows that 
%\[\frac{\1_{V_+\neq \emptyset}}{1-K(z,V_+,W_+)}+\frac{\1_{V_-\neq \emptyset}}{1-K(z,V_-,W_-)}\leq \max_{E_+,E_-\subset I, E_+\cap E_-=\emptyset}\Big\{{|E_+|}+{|E_-|}\Big\}\leq \frac2{n(n-1)}.\]
%If $W_+= \emptyset$, $V_+\neq \emptyset$ and  $W_-\neq \emptyset$ then since $V_-\cap V_+=\emptyset$, one has $|E_-|\leq S_1(z) -1=\frac{n(n-1)}{2} -1$. Therefore one gets 
%\[\frac{\1_{V_+\neq \emptyset}}{1-K(z,V_+,W_+)}+\frac{\1_{V_-\neq \emptyset}}{1-K(z,V_-,W_-)}=1+\frac{1}{1-K(z,V_-,W_-)}\leq 1+|E_-|\leq \frac2{n(n-1)}.\]
%Finally, if $W_+= W_-=\emptyset$, for $n\geq 3$, 
%\[\frac{\1_{V_+\neq \emptyset}}{1-K(z,V_+,W_+)}+\frac{\1_{V_-\neq \emptyset}}{1-K(z,V_-,W_-)}=\1_{V_+\neq \emptyset}+\1_{V_-\neq \emptyset}\leq \frac2{n(n-1)},\]
%and for $n=2$ either $V_+=\emptyset$ or $V_-=\emptyset$ so that the above inequality still holds. 
%It follows that one gets $C(z)\geq \frac{2}{n(n-1)}$ and the $W_1$-entropic curvature $\kappa_1$ of the space $(S_n,d,\mu_0,L_0)$ is bounded from below by
%$\frac{8}{n(n-1)}$.

%For the notion of entropic curvature developed by M. Erbar and J. Maas, it has been shown that the curvature term for the symmetric group \cite[Theorem 5.1]{EMT15}
%is lower bounded by $\frac{4}{n(n-1)}$. We obtain a result of the same order. It is still an open problem whether one can reach a lower bound  of order $\frac{1}{n}$, that would implies  known functional inequalities such as modified logarithmic Sobolev inequalities (see \cite{GQ03}) or transport entropy inequalities (see \cite{Sam17})  with the right order of magnitude for the constants.

%\subsection{The binomial law on $\{0,1,\ldots ,N\}$ }
%Let $\mathcal{X}=\{0,1,\ldots ,N\}$ and let $m=\mu$ denote the binomial distribution of parameter $p\in(0,1)$.
%Let us consider the generator $L$ reversible with respect to $\mu$ defined as  \[L(z,z+1)=p(1-\frac{z}{N}) \quad \mbox{and}\quad L(z,z-1)=(1-p)\frac{z}{N} \]
%for every $z\in \mathcal{X}$. For $z=0$, one has $S_{1}(0)=\{1\}$, $S_{2}(0)=\{2\}$ and 
%\[
%K(0,S_{1}(0),S_{2}(0))=\sup_{0\leq \alpha_1\leq 1} \Bigg( L^{2}(0,2)\Big(\frac{ \alpha_{1}}{L(0,1)}\Big)^{2}\Bigg)=\frac{L^{2}(0,2)}{L(0,1)^2 }= 1-\frac{1}{N}.\]
%For  $z=1$, one has  $S_{1}(1)=\{0,2\}$, $S_{2}(1)=\{3\}$ and 
%\[
%K(1,S_{1}(1),S_{2}(1))=\sup_{0\leq \alpha_2\leq 1} \Bigg(L^{2}(1,3)\Big(\frac{\alpha_{2}}{L(1,2)}\Big)^{2} \Bigg)=\frac{L^{2}(1,3)}{L(1,2)^2}=\frac{N-2}{N-1}.\]
%By symmetry, one  gets identically
%\[K(N,S_{1}(N),S_{2}(N))= 1-\frac{1}{N}\quad \mbox{and}  \quad 
%K(1,S_{1}(1),S_{2}(1))=\frac{N-2}{N-1}.\]
%Let $z\in \mathcal{X}\setminus \{0,1,N,N-1\}$, then $S_{1}(z)=\{z-1,z+1\}$, $S_{2}(z)=\{z-2,z+2\}$ and 
%\begin{align*}K(z,S_{1}(z),S_{2}(z))&=\sup_{\alpha_{z-1},\alpha_{z+1}, %\alpha_{z-1}+\alpha_{z+1}\leq 1} \Bigg( L^{2}(z,z+2) %\Big(\frac{\alpha_{z+1}}{L(z,z+1)}\Big)^{2}+
%L^{2}(z,z-2) \Big(\frac{\alpha_{z-1}}{L(z,z-1)}\Big)^{2} \Bigg)\\
%&=\max\Big\{\frac{N-z-1}{N-z},\frac{z-1}{z}\Big\}.
%\end{align*}
%It follows that $K=\max_{z\in\X} K(z,S_{1}(z),S_{2}(z))= 1-1/N$.

%Considering different cases, one easily show that for any $z\in\X$, $C(z)\geq %1/N$, and therefore the $W_1$-entropic curvature of the space is bounded from %below by $4/N$.


%\subsection{The multinomial law }



%Let $\mathcal{X}:=\{(x_{1},\ldots x_{d}), x_{i}\in \N, \sum_{i=1}^{d}x_{i}=N\} $ and let 
%$\mu$ the multinomial distribution, for any $x\in \X$,
%\[\mu(x):=\frac{N!}{d^{N}\prod_{i=1}^d x_{i}!}.\]
%Two different points $x$ and $y$ of $\X$ are neighbours if there exists $\{i,j\}\subset [d]$ such that for any $k\in[d]\setminus\{i,j\}$, $x_k=y_k$, and either $(y_i,y_j)=(x_i+1,x_j-1)$ and one denotes $x_{i+j-}:=y$, either  $(y_i,y_j)=(x_i-1,x_j+1)$ and one denotes $x_{j+i-}:=y$. %Observe that according to this definition $x_{i-j+}=x_{j_+i-}$. 
%Let us consider the generator $L$ reversible with respect to $\mu$ defined by $L(x,y)=x_i$ if there exists $j\neq i$ such that $y=x_{j+i-}$ and $L(x,y)=0$ for any other $y\neq x$.
%To simplify the notations, for any $i,j,l,k\subset [d]$ that all differ and for any $z\in \X$ one denotes
%\[z_{i_{++}j_{--}}:=(z_{i_{+}j_{-}})_{i_{+}j_{-}},\quad %z_{i_{++}j_{-}k_{-}}:=(z_{i_{+}j_{-}})_{i_{+}k_{-}},\]
%\[z_{i_{+}l_{+}j_{--}}:=(z_{i_{+}j_{-}})_{i_{+}k_{-}},\quad z_{i_{+}l_{+}j_{-}k_{-}}:=(z_{i_{+}j_{-}})_{l_{+}k_{-}},\]
%which are all points of $S_2(z)$. Namely, one has
% \[S_{2}(z)=S_{2}^{1}(z)\cup S_{2}^{2}(z)\cup S_{2}^{3}(z)\cup S_{2}^{4}(z) .\]
% with 
% \begin{align*}
% S_{2}^{1}(z)&:=\big\{z_{i_{++}j_{--}}\in \X \,|\,(i,j)\subset [d]^2, i\neq j\big\} ,\\
%S_{2}^{2}(z)&:=\big\{z_{i_{++}j_{-}k_{-}}\in \X\,|\, \{j,k\}\subset [d],i\in[d]\setminus\{j,k\} \big\}  ,\\
%S_{2}^{3}(z)&:=\big\{z_{i_{+}l_{+}j_{--}}\in \X \,|\, j\in[d],\{i,l\}\subset [d]\setminus \{j\}\big\}  ,\\
% S_{2}^{4}(z)&:=\big\{z_{i_{+}l_{+}j_{-}k_{-}}\in \X \,|\, \{j,k\}\subset [d] ,\{i,l\}\subset [d]\setminus\{j,k\}\big\}.
%\end{align*}
%For $\alpha:S_1(z)\to \R_+$, let us note $\alpha_{i+j-}=\alpha(z_{i+j-})$ and for $W\subset S_2(z)$,
%\[R(W):=\sum_{\ttz\in W} L^2(z,\ttz) \prod_{\tz\in S_1(z)\cap [z,\ttz]}\left(\frac{\alpha(\tz)}{L(z,\tz)}\right)^{\frac{2L(z,\tz)L(\tz,\ttz)}{L^2(z,\ttz)}}.\]
%One has 
%\begin{align*}
%R(S_{2}^{1}(z))&=\sum_{(i,j)\subset [d]^2, i\neq j, z_{i_{++}j_{--}}\in \X} \alpha_{i+j-}^2 \frac{z_j-1}{z_j},\\
%R(S_{2}^{2}(z))&=\sum_{\{j,k\}\subset [d],i\in[d]\setminus\{j,k\}, z_{i_{++}j_{-}k_{-}}\in \X} 2\,\alpha_{i_{+}j_{-}}\alpha_{i_{+}k_{-}},\\
%R(S_{2}^{3}(z))&=\sum_{j\in[d],\{i,l\}\subset [d]\setminus \{j\},z_{i_{+}l_{+}j_{--}}\in \X }  2\,\frac{z_{j}-1}{z_{j}}\alpha_{i_{+}j_{-}}\alpha_{l_{+}j_{-}},\\
%R(S_{2}^{4}(z))&=\sum_{\{j,k\}\subset [d] ,\{i,l\}\subset [d]\setminus\{j,k\},z_{i_{+}l_{+}j_{-}k_{-}}\in \X } 4 \sqrt{ \alpha_{i_{+}j_{-}}\alpha_{i_{+}k_{-}}\alpha_{l_{+}j_{-}}\alpha_{l_{+}k_{-}}},  
%\end{align*}
%and therefore
%\[
%K(z,S_{1}(z),S_{2}(z)):=\sup_{ \alpha_{i+,j-},(i,j)\subset [d]^2, i\neq j} \left\{R(S_{2}^{1}(z))+R(S_{2}^{2}(z))+R(S_{2}^{3}(z))+R(S_{2}^{4}(z))\right\},\]
%where the supremum runs over all non negative $\alpha_{i+,j-}$ satisfying $\sum_{(i,j)\subset [d]^2, i\neq j} \alpha_{i+,j-}=1$.



%Let us show the converse inequality. 
%Given $z\in \X$, one easily checks that 
%\[R(S_{2}^{1}(z))+R(S_{2}^{3}(z))\leq \sum_{j\in[d]} \frac{z_j-1}{z_j}\Big( \sum_{i\in  [d]\setminus\{j\}}\alpha_{i+j-}\Big)^2,\]
%and also applying Cauchy-Schwarz inequality  
%\begin{align*}
%R(S_{2}^{2}(z))+R(S_{2}^{4}(z))&\leq 2 \sum_{\{j,k\}\subset [d]}\Big(\sum_{i\in[d]\setminus\{j,k\}} \sqrt{\alpha_{i+j-}\alpha_{i+k-}}\Big)^2\\
%&\leq 2 \sum_{\{j,k\}\subset [d]}\Big(\sum_{i\in[d]\setminus\{j,k\}} \alpha_{i+j-}\Big) \Big(\sum_{i\in[d]\setminus\{j,k\}} \alpha_{i+k-}\Big)\\
%&\leq 2 \sum_{\{j,k\}\subset [d]}\Big(\sum_{i\in[d]\setminus\{j\}} \alpha_{i+j-}\Big) \Big(\sum_{i\in[d]\setminus\{k\}} \alpha_{i+k-}\Big)
%\end{align*}
%For any $j\in[d]$, setting $A_j=\sum_{i\in  [d]\setminus\{j\}}\alpha_{i+j-}$, one has $\sum_{j\in [d]} A_j= 1$, and from the above estimates, 
%\begin{align*}
%&R(S_{2}^{1}(z))+R(S_{2}^{2}(z))+R(S_{2}^{3}(z))+R(S_{2}^{4}(z)\\
%&\leq \sum_{j\in[d]} \frac{z_j-1}{z_j}\,A_j^2+2 \sum_{\{j,k\}\subset [d]} A_jA_k\\
%&= \Big(\sum_{j\in[d]} A_j\Big)^2-\sum_{j\in[d]} \frac{A_j^2}{z_j}\\
%&\leq \Big(1- \frac1N\Big) \Big(\sum_{j\in[d]} A_j\Big)^2\leq1- \frac1N
%\end{align*}
%where we have used Cauchy-Schwarz inequality for the last inequality and the fact that $\sum_{j\in[d]}z_j=N$. It follows that $K\leq 1-1/N$.

%Choosing $z_0=(0,0,\ldots,N)\in \X$, one has 
%\[ R(S_{2}^{1}(z_0))=
%\sum_{i\in [d-1]} \alpha_{i_{+}d_{-}}^{2} \frac{z_{d}-1}{z_{d}} =\sum_{i\in [d-1]}\frac{N-1}{N}\alpha_{i_{+}d_{-}}^{2} , \]

%\[ R(S_{2}^{3}(z_0))%=\sum_{\{i,k\}\subset [d-1],i\neq k} 2z_{d}(z_{d}-1)\frac{\alpha_{i_{+}d_{-}}}{z_{d}}\frac{\alpha_{k_{+}d_{-}}}{z_{d}}
%=\sum_{\{i,l\}\subset[d-1],j\neq k} %2\,\frac{N-1}{N}\alpha_{i_{+}d_{-}}\alpha_{l_{+}d_{-}}, \] 
%and \[ R(S_{2}^{2}(z_0))=R(S_{2}^{4}(z_0))=0. \] 
%It follows that 
%\begin{align*}
%K&\geq K(z_0,S_{1}(z_0),S_{2}(z_0))\\&=\frac{N-1}{N}\sup_{ \alpha_{i+,j-},(i,j)\subset [d]^2, i\neq j} \Big\{\sum_{i\in [d-1]}\alpha_{i_{+}d_{-}}^{2} + 2 \sum_{\{i,l\}\subset[d-1]} \alpha_{i_{+}d_{-}}\alpha_{l_{+}d_{-}}\Big\}\\
%&=\frac{N-1}{N}\sup_{ \alpha_{i+,j-},(i,j)\subset [d]^2, i\neq j} \Big(\sum_{i\in [d-1]}\alpha_{i_{+}d_{-}}\Big)^2=1-\frac1N
%\end{align*}
%and therefore $K=1-1/N$.
%Thus,
%\begin{align*}
%R(z,S_{1}(z),S_{2}(z)) &=\sum_{(i,j),i\neq j} \frac{z_{j}-1}{z_{j}}+\alpha_{i_{+}j_{-}}^{2}+
%2\sum_{i=0}^{d}\sum_{\{j,k\}\subset \{0,\ldots d\}\setminus \{i\}} \alpha_{i_{+}j_{-}}\alpha_{i_{+}k_{-}} \\
%&+2 \sum_{i=0}^{d}\sum_{(j,k)\subset \{0,\ldots d\}\setminus \{i\}}  \frac{z_{i}-1}{z_{i}}\alpha_{i_{-}j_{+}}\alpha_{i_{-}k_{+}}\\
%&+ \sum_{\{i,j\}\subset \{0,\ldots, d\}}\sum_{\{k,l\}\subset \{0,\ldots, d\}\setminus \{i\}} \sqrt{ \alpha_{i_{-}k_{+}}\alpha_{i_{-}l_{+}}\alpha_{j_{-}k_{+}}\alpha_{j_{-}l_{+}}}  
%\end{align*}
%and since $\frac{z_{j}-1}{z_{j}}\leq \frac{N-1}{N}$ %one concludes that $R=\frac{N-1}{N}$.
%\subsection{\textcolor{red}{Recherche encore de la deuxieme direction de la proposition}}

%The concept of Ricci flat graphs was introduced by Chung and Yau in \cite{CY96} and recently revisited in \cite{CKKLP21}. These graphs generalize the Cayley graphs of Abelian groups.

%\begin{definition}
%Let $G=(\X,E)$ be a $d$-regular graph. We say that $z\in \X$ is \textit{Ricci-flat} if there exist maps $\eta_{i}:B_{1}(z)\rightarrow \X$ for $1\leq i\leq d$ with the following properties
%\begin{itemize}
 %   \item $\eta_{i}(v)\sim v \hspace{0.2cm} \text{for all } v\in B_{1}(z)$ ,
 %   \item $\eta_{i}(v)\neq \eta_{j}(v) \hspace{0.1cm} \text{if} \hspace{0.1cm} i\neq j$ ,
 %   \item $\bigcup_{j}\eta_{j}(\eta_{i}(z))=\bigcup_{j} \eta_{i}(\eta_{j}(z)) \hspace{0.2cm} \text{for all}\hspace{0.1cm} i,j\in [d]$.
%\end{itemize}
%A graph $G=(\X,E)$ is said to be \textit{Ricci flat} if it is Ricci flat for every $z\in \X$.

%\end{definition}
%Note that the lattice example $\mathbb{Z}^{n}$ studied in detail in subsection 3.1, corresponds to the case where $\eta_{i}(z)=z+e_{i}$.\

%\begin{definition}
%Let $z\in \X$. For $W\subset S_{2}(z)$, let $V_{W}\subset S_{1}(z)$ be the set defined as
%\begin{equation*}
 %   V_{W}:=]z,W[ .
%\end{equation*}
%Let $\mathcal{W}=\{W_{i}\}_{i=1}^{d}\subset S_{2}(z)$ with $d\in \mathbb{N}$. We say that the family $\mathcal{W}=\{W_{i}\}_{i=1}^{d}$ is \textit{admissible} with respect to the vertex $z$ if the family $\mathcal{W}$ satisfies the following conditions
%\begin{itemize}
 %   \item $S_{2}(z)=\bigcup_{i}^{d} W_{i}$ ,
 %   \item $V_{W_{i}}\neq V_{W_{j}} \hspace{0.1cm} \text{if} \hspace{0.1cm} i\neq j, \text{for} \hspace{0.1cm} i,j\in [d]$, 
 %   \item $\text{For all} \hspace{0.2cm} W^{\prime}\subset W,\hspace{0.1cm} V_{W^{\prime}}=V_{W}$.
%\end{itemize}
%\end{definition}
%For the sake of simplicity, in what follows we will write $V_{i}$ instead of %$V_{W_{i}}$.
%\begin{proposition}
%Let $z\in \X$. Let a family $\mathcal{W}=\{ W_{i}\}_{i=1}^{d}$  admissible with respect to the vertex $z$ . Then, 
%\begin{equation*} \label{growth}
%|V_{i}| \geq |W_{i}| \hspace{0.2cm} \forall i\in [d]
%\end{equation*}
%if and only if $r(z)\geq 0$.
%\end{proposition}

%Let $z\in \X$ arbitrary. First, if $r(z)\geq 0$. Then 
%\begin{equation*}
%K(z,S_{2}(z))=\max_{\alpha}    \sum_{i=1}^{d}|V_{i}||W_{i}|\Big(\prod_{i\in V_{i}}\alpha_{i}\Big)^{\frac{2}{|V_{i}|}} \leq 1 .
%\end{equation*}
%Thus, choosing $(\alpha_{i})_{i}$ as $\frac{1}{|V_{i}|}$ for $i\in [d]$ one immediately gets 
%\begin{equation*}
%    1\geq K(z,S_{2}(z))\geq \max_{i\in [d]} \Big(\frac{|W_{i}|}{|V_{i}|}\Big) .
%\end{equation*}


  

%Let suppose now that a family $\mathcal{W}=\{W_{i}\}_{i=1}^{d}$ is admissible with respect to the vertex $z$ and $|V_{i}| \geq |W_{i}|$ for all $i\in [d]$.
%Let suppose in a first moment that 
%\begin{equation}\label{nointer}  
%    V_{i}\cap V_{j}=\emptyset \hspace{0.1cm} \text{for all} \hspace{0.1cm} i,j\in [d] ,
%\end{equation}
%in what follows we will see that our analysis can be reduced to this case. 

%By the arithmetic-geometric inequality and by \eqref{nointer},
%\begin{align*}
%K(z,S_{2}(z))&= \sum_{i=1}^{d}|V_{i}||W_{i}|\Big(\prod_{i\in V_{i}}\alpha_{i}\Big)^{\frac{2}{|V_{i}|}} \\
%&\leq \sum_{i=1}^{d}|V_{i}||W_{i}|\frac{ \Big(\sum_{i\in V_{i}} \alpha_{i}\Big)^{2}}{|V_{i}|^{2}}  \\
%&\leq \max_{i\in [d]} \frac{|W_{i}|}{|V_{i}|}\sum_{i=1}^{d}\sum_{i\in V_{i}}\alpha_{i}=\max_{i\in [d]} \frac{|W_{i}|}{|V_{i}|} \\
%&\leq 1 \hspace{0.2cm} .
%\end{align*}

%{\bf Comment:}
%\begin{itemize}
 %   \item As for the Bakry-\'Emery curvature as for the %Ollivier curvature \cite{KKRT16,CKKLP21} we obtain %analogous results for the Entropic curvature.

%\end{itemize}




%Let $Cay(\Gamma,S)$ the Cayley graph of an abelian group $\Gamma$ with generating set $S$. Let us consider the generator $L_{0}$ reversible with  respect to the counting measure defined as follows 
%\[L_{0}(g,gs)=1 \quad \mbox{for all $g\in \Gamma$ and $s\in (S\cup S^{-1}$})\setminus \{e\} \quad .\]
%For convenience, let $S=\{s_{1},s_{2}\ldots s_{n}\}$.  Since any Cayley graph is vertex transitive it suffices to study any arbitrary element $g$ of the group $\Gamma$. Let us note that $S_{1}(g)=\{gs_{i},gs_{i}^{-1}\} \hspace{0.2cm} \forall i\in [n]$ and
%$S_{2}(g)=\{gs_{i}^{2},gs_{i}^{-2},gs_{i}s_{j},gs_{i}^{-1}s_{j%}^{-1}\}$. Let %$\alpha_{i+}:=\alpha(gs_{i}),\alpha_{i-}:=\alp%ha(gs_{i}^{-1})$. In that way, 
%\begin{equation*}
%K(g,S_{1}(g),S_{2}(g))=\sup_{\sum_{i=1}^{n}\alpha_{i}=1} \sum_{i=1}^{n}\alpha_{i+}^{2}+\sum_{i=1}^{n}\alpha_{i-}^{2}+\sum_{i+\neq j+} 2\alpha_{i+}\alpha_{j+}+\sum_{i-\neq j-} 2\alpha_{i-}\alpha_{j-}=1.
%\end{equation*}
 %It follows immediately that the entropic curvature is greater than or equal to zero. This is consistent with the analogous results
%for other curvatures such as the Bakry-\'Emery curvature developed in \cite[Theorem 2.3]{KKRT16}. 
%\textit{hamiltonian} graphs 
%Therefore, the Cayley graphs of finitely generated abelian groups are entropically \textit{flat} 
%This is consistent with the analogous result
%for other curvatures such as the Bakry-\'Emery curvature developed in \cite[Theorem 2.3]{KKRT16}. 

\section{Non positively curved graphs}

%In this section we will give examples and applications of entropic curvature for graphs. Firstly, we will exemplify positive entropic curvature, in particular, we will be interested in a \textit{family of graphs} whose curvature is strictly positive which we will denote $\mathcal{T}$. Subsequently, we will give examples of non-positively curved graphs with an emphasis on graphs called \textit{geodetic graphs}. \

In this section we will give examples of graphs whose entropic curvature is negative. In particular, we will concentrate on the notion of \textit{geodetic graphs} and see how this notion is closely related to negative  curvature. In everything that follows the dynamics are determined by the generator $L_{0}$ with reversible measure associated the counting measure $m_{0}$. In that way, we will always use the formulation developed in \eqref{defRbis}. 
%This section is motivated by Ore's question \cite{Oys87} about a characterization of \textit{geodetic graphs}, which has recently been studied in \cite{Lin21}. Indeed we will see that negative entropic curvature characterizes in one direction \textit{geodetic graphs}. More precisely, if a graph is \textit{geodetic} and has a diameter greater than or equal to two then its entropic curvature is non positive. Let us introduce \textit{geodetic graphs}. In this way we will see that the condition of being a \textit{geodetic graph}  is a sufficient condition to obtain negative entropic curvature. First we will introduce and analyze \textit{geodetic graphs}. Subsequently, we will study the entropic curvature that is negative for a class of graphs denoted as \textit{antitrees}s 
Let us  recall the definition of \textit{geodetic graphs} introduced by Ore (see \cite{Oys87}).
\begin{definition}
A graph $G=(\mathcal{X},E)$ is called geodetic if for every two vertices $u$ and $v$ in $G$ there exists a unique 
geodesic connecting $u$ and $v$.
\end{definition}
For example, every tree, every complete graph, every odd-length cycle and the Petersen graph are geodetic graphs.

The following proposition states that if a graph has diameter greater or equal to two and is geodetic then its entropic curvature given by Theorem 1 is negative.


\begin{proposition}\label{propgeodetic}
For the class of spaces $(\X,d,L_{0},m_{0})$ where $G=(\mathcal{X},E)$ is a geodetic graph with diameter greater or equal to 2 the entropic curvature is always non positive. Moreover, one has  \[r=-2\log(K_{0})=-2\log\max_{z\in \mathcal{X}}\Big(\max_{z^{\prime},\hspace{0.1cm} z^{\prime}\sim z}\textnormal{Deg}(z^{\prime})-1\Big)\geq 2\Big(2-\max_{z^{\prime},\hspace{0.1cm} z^{\prime}\sim z}\textnormal{Deg}(z^{\prime})\Big).\]
\end{proposition}
The proof of this proposition is postponed in Appendix B.

%\begin{proof}
%It was proved in \cite{PS82} that a graph $G$ of finite diameter $d$ is geodetic if and only if for every $z\in \mathcal{X}$, each vertex of $S_{r}(z)$ is adjacent to a unique vertex in $S_{r-1}(z)$ for each $0 \leq r \leq d$. Let $G=(\mathcal{X},E)$ be a geodetic graph with diameter greater or equal to 2 and $z_{0}\in \mathcal{X}$. Simply taking $r=2$, we obtain that there exists $z_{0}^{\prime \prime}\in S_{2}(z_{0})$ such that
%$|S_{1}(z_{0})\cap [z_{0},z_{0}^{\prime \prime}]| =1$ . %Therefore, we immediately conclude that $K\geq 1$. Moreover, for every $z\in \mathcal{X}$
%\begin{equation*}
%K_{0}(z,S_{2}(z))=\sup_{\alpha} \sum_{\tz\sim \ttz, \hspace{0.1cm} d(z,\ttz)=2} \alpha(\tz)^{2}= \sup_{\alpha} \sum_{\tz,\hspace{0.1cm} \tz\sim z}(\textnormal{Deg}(z^{\prime})-1)\alpha(z^{\prime})^{2}=\max_{z^{\prime},\hspace{0.1cm} z^{\prime}\sim z}\textnormal{Deg}(z^{\prime})-1 \hspace{0.1cm} ,
%\end{equation*}
%and the conclusion follows.

%\end{proof}


%\subsubsection{The example of the Petersen graph}.
%\begin{figure}[!h]
%\centering
%\includegraphics[scale=0.3]{P.png}
%\caption{The values of 1-$K(z,S_{1}(z),S_{2}(z))$ for each vertex $z$ in the \textit{Petersen graph}} 
%\label{Figure1.1}
%\end{figure}
%\FloatBarrier







%\subsubsection{The example of the \textit{infinite binomial tree}}.
%\begin{figure}[!h]
%\centering
%\includegraphics[scale=0.2]{infbi.png}
%\caption{The values of 1-$K(z,S_{1}(z),S_{2}(z))$ for each vertex $z$ in the \textit{infinite binomial tree}} 
%\label{Figure1.1}
%\end{figure}
%\FloatBarrier



{\bf Comments:}
\begin{enumerate}[(i)]

 \item Let us observe that the fact that $r\leq 0 $ is consistent to the geometry of the underlying generic geodetic graph.
 \item One of Ore's questions focuses on the characterization of geodetic graphs which has recently been studied in \cite{Lin21}. Of course, note that if a space $(\mathcal{X},d,m_{0},L_{0})$ has zero or negative  entropic curvature it does not imply that the underlying graph is geodetic. Indeed, it was shown in \cite{Sam21}, that the circle $\mathbb{Z}/N\mathbb{Z}$ of any length and parity satisfy null entropic curvature. Also, note that the hexagonal tiling of the plane is not a geodetic graph, however locally it looks like a 3-regular tree and has negative entropic curvature, i.e. for every vertex $z$ in the  hexagonal tiling $r(z)=-2$.

 
 
 


 
 
\end{enumerate}



\section{Some combinatorial implications of the entropic curvature and a few comparisons with other notions of curvature}

In this section we will discuss some combinatorial implications of the entropic curvature. As in the previous section, we will use the dynamics determined by the generator $L_{0}$. Certain general conditions will be obtained for positive entropic curvature. Also, some comparisons with two other notions of curvatures will be discussed. In particular, we will establish a relation with the \textit{Lin-Lu-Yau curvature} and make some comparative remarks with respect to the \textit{Bakry-Émery curvature}.
 \\

\paragraph{ \bf General conditions for positive entropic curvature.}\

Let us introduce certain local combinatorial conditions to obtain positive entropic curvature for a vertex $z\in \X$. 
To do so, let us introduce the notion of \textit{admissible family}.
 
Let $z\in \X$. % For $W\subset S_{2}(z)$, let $V_{W}\subset S_{1}(z)$ be the set defined as
%\begin{equation*}
%    V_{W}:=]z,W[ .
%\end{equation*}
Let $\mathcal{W}=\{W_{i}\}_{i=1}^{d}\subset S_{2}(z)$ with $d\in \mathbb{N}$. We say that the family $\mathcal{W}=\{W_{i}\}_{i=1}^{d}$ is \textit{admissible} with respect to the vertex $z$ if the family $\mathcal{W}$ satisfies the following conditions
\begin{enumerate}[label=(\roman*)]
    \item $S_{2}(z)=\bigcup_{i}^{d} W_{i}$ ,
   % \item $V_{W_{i}}\neq V_{W_{j}} \hspace{0.1cm} \text{if} \hspace{0.1cm} i\neq j, \text{for} \hspace{0.1cm} i,j\in [d]$,
   \item $]z,W_{i}[\neq ]z,W_{j} [\hspace{0.1cm} \text{if} \hspace{0.1cm} i\neq j, \text{for} \hspace{0.1cm} i,j\in [d]$,
    %\item $\text{For all} \hspace{0.2cm} W^{\prime}\subset W,\hspace{0.1cm} V_{W^{\prime}}=V_{W}$.
    \item \text{For all} $z^{\prime \prime}\in W_{i}, \hspace{0.2cm} ]z,z^{\prime \prime}[=]z,W_{i}[ \hspace{0.2cm} \text{for all} \hspace{0.1cm} i\in [d]$ .
\end{enumerate}

Clearly, an admissible family always exists. 
%For the sake of simplicity, in what follows we will write $V_{i}$ instead of $]z,W_{i}[$.%For the sake of simplicity, in what follows we will write $V_{i}$ instead of $]z,W_{i}[$.

\begin{proposition}\label{combfam}
Let $z\in \X$. Let a family $\mathcal{W}=\{ W_{i}\}_{i=1}^{d}$  admissible with respect to the vertex $z$ . 
If $r(z)\geq 0$ then for all $i\in [d]$, 
\begin{equation*} \label{growth}
\big|]z,W_{i}[\big| \geq \big|W_{i}\big|,
\end{equation*}
and if 
\begin{equation*}  
 \big|]z,W_{i}[\big| \geq \big|W_{i}\big| \hspace{0.15cm} \text{and} \hspace{0.15cm} ]z,W_{i}[\cap ]z,W_{j}[=\emptyset \qquad \mbox{ for all } \hspace{0.1cm} i,j\in [d], i\neq j ,
\end{equation*} then $r(z)\geq 0$. 
\end{proposition}
The proof of this proposition is postponed in Appendix B.

{\bf Comment:}
It is conjectured that the hypothesis that $]z,W_{i}[\cap ]z,W_{j}[=\emptyset \hspace{0.1cm} \text{for all} \hspace{0.1cm} i,j\in [d]$, $i\neq j$, can be removed. It is for now an open problem.\\



%Let $z\in \X$ arbitrary. First, if $r(z)\geq 0$. Then 
%\begin{equation*}
%K(z,S_{2}(z))=\max_{\alpha}    \sum_{i=1}^{d}|V_{i}||W_{i}|\Big(\prod_{i\in V_{i}}\alpha_{i}\Big)^{\frac{2}{|V_{i}|}} \leq 1 .
%\end{equation*}
%Thus, choosing $(\alpha_{i})_{i}$ as $\frac{1}{|V_{i}|}$ for $i\in [d]$ one immediately gets 
%\begin{equation*}
 %   1\geq K(z,S_{2}(z))\geq \max_{i\in [d]} \Big(\frac{|W_{i}|}{|V_{i}|}\Big) .
%\end{equation*}


  

%Let suppose now that a family $\mathcal{W}=\{W_{i}\}_{i=1}^{d}$ is admissible with respect to the vertex $z$ and $|V_{i}| \geq |W_{i}|$ for all $i\in [d]$.
%Let suppose in a first moment that 
%\begin{equation}\label{nointer}  
 %   V_{i}\cap V_{j}=\emptyset \hspace{0.1cm} \text{for all} \hspace{0.1cm} i,j\in [d] ,
%\end{equation}
%in what follows we will see that our analysis can be reduced to this %case. 

%By the arithmetic-geometric inequality and by \eqref{nointer},
%\begin{align*}
%K(z,S_{2}(z))&= \sum_{i=1}^{d}|V_{i}||W_{i}|\Big(\prod_{i\in %V_{i}}\alpha_{i}\Big)^{\frac{2}{|V_{i}|}} \\
%&\leq \sum_{i=1}^{d}|V_{i}||W_{i}|\frac{ \Big(\sum_{i\in V_{i}} \alpha_{i}\Big)^{2}}{|V_{i}|^{2}}  \\
%&\leq \max_{i\in [d]} \frac{|W_{i}|}{|V_{i}|}\sum_{i=1}^{d}\sum_{i\in V_{i}}\alpha_{i}=\max_{i\in [d]} \frac{|W_{i}|}{|V_{i}|} \\
%&\leq 1 \hspace{0.2cm} .
%\end{align*}



\paragraph{ \bf Some relations between entropic curvature and the Lin-Lu-Yau curvature.}\

The Lin-Lu-Yau curvature is a modified notion of the coarse Ollivier's Ricci curvature introduced by Lin, Lu and Yau in \cite{LLY11}.  In \cite{MW19}, Florentin M{\"u}nch and Rados{\l}aw K.Wojciechowski, generalized the notion of Lin-Lu-Yau curvature  for any graph Laplacian.
\begin{definition}[Lin-Lu-Yau Ricci curvature]
Given $G=(\mathcal{X},E)$ a graph endowed with its graph distance $d$ and with a Markov chain defined by $m:=\{m_{z}(\cdot)\}_{z\in \mathcal{X}}$. 
%The \textit{Ollivier's coarse Ricci curvature along the edge $\{x,y\}$} is defined as
%\begin{equation*}
%\mathcal{K}_{Oll}(x,y):=1-\frac{W_{1}
%(\eta_{x},\eta_{y})}{d(x,y)}\hspace{0.1cm} , 
%\end{equation*}
%with the markov kernel $\eta$ defined as follows
%\begin{equation*}
%\eta_{x}(y)=
%\begin{cases}
%\frac{1}{\textnormal{Deg}(x)} \hspace{0.1cm} \text{if} \hspace{0.1cm} y\sim x \hspace{0.1cm} ,\\
%0 \hspace{0.1cm} \text{otherwise} .
%\end{cases}
%\end{equation*}
%The \textit{Ollivier's coarse Ricci curvature} of $(G,d,\eta)$ is defined as
%\begin{equation*}
%\mathcal{K}_{Oll}:=\inf_{x,y\in \mathcal{X}} \mathcal{K}_{Oll}(x,y) \hspace{0.1cm} .
%\end{equation*}
For $0\leq \alpha<1$, the \textit{$\alpha$-lazy random walk} $m_{x}^{\alpha}$ associated to the graph Laplacian $L_{0}$ is defined as
\begin{equation*}
m_{x}^{\alpha}(y)=
\begin{cases}
\frac{\alpha}{\Deg_{\text{max}}} \hspace{0.2cm} \text{if} \hspace{0.2cm} x\sim y ,\\
 1- \alpha \frac{\Deg(x)}{\Deg_{\text{max}}}
\hspace{0.2cm} \text{if} \hspace{0.2cm} y= x ,\\
0 \hspace{0.2cm} \text{otherwise}.
\end{cases}
\end{equation*}


For every $x,y\in \mathcal{X}$, one defines
\begin{equation*}\kappa_{\alpha}(x,y):=1-\frac{W_{1}(m_{x}^{\alpha},m_{y}^{\alpha})}{d(x,y)}.
\end{equation*}

As shown in \cite{MW19}, the limit as $\alpha\rightarrow 0$ exists for any graph Laplacian and therefore one can define the Ricci Lin-Lu-Yau curvature along the edge $\{x,y\}$ denoted as $\kappa_{LLY}(x,y)$ by
\begin{equation*}
\kappa_{LLY}(x,y):=\lim_{\alpha \rightarrow 0} \frac{\kappa_{\alpha}(x,y)}{\alpha} .
\end{equation*}
\end{definition}


 
%In the following proposition, a global relationship between entropic curvature and the Lin-Lu-Yau curvature is established via the graph-theoretical notion of \textit{girth}.
The following proposition establishes a link between the Lin-Lu-Yau curvature and  the graph-theoretical notion of \textit{girth}. 
Let us first introduce the concept of girth of a graph.
%whose proof has the same structure as in %\cite[Theorem 2.b(ii)]{HS13}.
%To this end, we will look at combinatorial implications of negative entropic curvature.



\begin{definition}
The girth of a graph $G=(\mathcal{X},E)$, denoted in what follows $g(G)$ is the length of the shortest cycle contained in $G$. Acyclic graphs are considered to have infinite girth.
\end{definition}
Adapting the proof \cite[Theorem 2.b(ii)]{HS13}, we obtain the following proposition with the measures $m_{x}^{\alpha}$ associated to the generator  $L_{0}$. Its proof of this proposition is postponed in Appendix B.

%\begin{remark}
%Let  $G=(\mathcal{X},E)$ be a graph . If $g(G)\geq 5$, then there cannot be two midpoints between two vertices at distance two and thus as already noted in the introduction for all $z\in \mathcal{X}$, $r(z)\leq 0$. 
%\end{remark}

\begin{proposition}\label{propcycles}
Let $G=(\mathcal{X},E)$ be a graph . If for all $x,y\in \mathcal{X}$ with $d(x,y)=1$ 
\begin{equation*}
\kappa_{LLY}(x,y)<\frac{6-\Deg(x)-\Deg(y)}{\Deg_{\max}}
\end{equation*}
then $g(G)\geq 5$. 
%and thus 
%\begin{equation*}
%    \forall z\in \mathcal{X},\hspace{0.2cm} r(z)\leq 0 .
%\end{equation*}
\end{proposition}

%\begin{proof}
%Let suppose that $g(G)<5$ then there exist $x^{\prime}\sim x$ and 
%$y^{\prime}\sim y$ such that $d(x^{\prime},y^{\prime})\leq 1$.Without loss of generality, let suppose that $\text{Deg}(y)\geq \text{Deg}(x)$. Let us define the following transport $\pi$ between $m_{x}^{\alpha}$ and $m_{y}^{\alpha}$ as follows 
%\begin{itemize}
%\item $\pi(x,x)=\pi(y,y)=\pi(x^{\prime},y^{\prime})=\frac{\alpha}{\text{Deg}_{\max}},$

%\item $\pi(y,x)=1-\frac{\alpha \text{Deg}(y)}{\text{Deg}_{\max}}-\frac{\alpha}{\text{Deg}_{\max}},$

%\item $\pi(S_{1}(y)\setminus (\{x\}\cup \{y^{\prime}\},x)=\frac{(\text{Deg}(y)-\text{Deg}(x))\alpha}{\text{Deg}_{\max}}, $

%\item $\pi(S_{1}(y)\setminus (\{x\}\cup \{y^{\prime}\}),S_{1}(x)\setminus \{y\}\cup \{x^{\prime}\}))=\frac{(\text{Deg}(x)-2)\alpha}{\text{Deg}_{\max}}$.
%\end{itemize}

%Since $W_{1}(m_{x}^{\alpha},m_{y}^{\alpha})=1-\kappa_{\alpha}(x,y)$ one obtains that $\kappa_{LLY}(x,y)\geq \frac{6-\text{Deg}(y)-\text{Deg}(x)}{\text{Deg}_{\max}}$ which is a contradiction.
\begin{remark}
Let  $G=(\mathcal{X},E)$ be a graph . If $g(G)\geq 5$, then there cannot be two midpoints between two vertices at distance two and thus as already noted in the introduction for all $z\in \mathcal{X}$, $r(z)\leq 0$.

\end{remark}
Thanks to the above remark we immediately obtain the following corollary which establishes a global relationship between the entropic curvature and the Lin-Lu-Yau curvature.
\begin{corollary}
Let $G=(\mathcal{X},E)$ be a graph. If for all $z\in \X$
    $r(z)> 0$,
then for all $x,y\in \mathcal{X}$ with $d(x,y)=1$,
\begin{equation*}
\kappa_{LLY}(x,y)\geq \frac{6-\Deg(x)-\Deg(y)}{\Deg_{\max}} \hspace{0.1cm} .
\end{equation*}

%If for all $x,y\in \mathcal{X}$ with $d(x,y)=1$ 
%\begin{equation*}
%\kappa_{LLY}(x,y)<\frac{6-\Deg(x)-\Deg(y)}{\Deg_{\max}}
%\end{equation*}
%then $g(G)\geq 5$. 
%and thus 
%\begin{equation*}
%    \forall z\in \mathcal{X},\hspace{0.2cm} r(z)\leq 0 .
%\end{equation*}

\end{corollary}





%\end{proof}

%We obtain global relations between \textit{entropic curvature} and \textit{Ollivier's Ricci curvature} (see Appendix) for  $D$-regular graphs and we do not obtain relations between \textit{entropic curvature} and \textit{Ollivier's Ricci curvature} or \textit{Lin-Lu-Yau curvature} (see Appendix) locally.\\

%In \cite[Theorem 2.b(ii)]{HS13}, the authors show that if the Ollivier's Ricci curvature denoted by $K_{\text{Oll}}$ defined via the simple neighbour random walk  is bounded from above then the girth of a graph is greater or equal to five and thus we obtain negative entropic negative curvature. The kernel defined by the neighbour random walk  coincides with the dynamics defined by $L_{0}$ for the case of $D$-regular graphs. Thus, by their result, if 
%\begin{equation*}
%K_{\text{0ll}}(x,y)<-2+\frac{6}{D}  \hspace{0.1cm} \forall x,y\in \mathcal{X} \hspace{0.2cm} \text{then} \hspace{0.2cm} K(z,S_{2}(z))>1 \hspace{0.2cm} \forall z\in \mathcal{X} .
%\end{equation*}

 
 %In \cite{MW19}, Florentin M{\"u}nch and Rados{\l}aw K.Wojciechowski, generalized the notion of Lin-Lu-Yau curvature  for any graph Laplacian. The Lin-Lu-Yau curvature associated to $L_{0}$ is defined in the appendix. 
 %{\bf Comment:}
 
 %Let us note, that if \textit{Lin-Lu-Yau curvature} is strictly positive it does not imply that the entropic curvature is positive. 
 If the Lin-Lu-Yau curvature is strictly positive on an edge $\{x,y\}$, it does not imply that $r(x)>0$ nor that $r(y)>0$ . For this purpose, let us consider the following graph called the \textit{windmill graph} $W_{d}(4,2)$ consisting of 2 copies of the complete graph $K_{4}$ at a shared universal vertex (see Figure~1).  
 \vspace{-1 cm}
\begin{figure}[!ht]
\centering
\includegraphics[scale=0.3]{windflow.pdf}
\vspace{- 1,7 cm}
\caption{The value of \textit{Lin-Lu-Yau Ricci curvature} according to the graph Laplacian $L_{0}$ for the edge $(x,y)\in W_{d}(4,2)$ is 2/3.} 
\label{Figure1.1}
\end{figure}
\FloatBarrier
  
%This graph has strictly positive Lin-Lu-Yau curvature on the edge $\{x,y\}$. 
Indeed, it is easy to verify that
$W_{1}(m_{x}^{\alpha},m_{y}^{\alpha})=1-\frac{\alpha}{2}-\frac{\alpha}{6}$. Therefore, $\kappa_{LLY}(x,y)=\lim_{\alpha\rightarrow 0} \frac{\kappa_{\alpha}(x,y)}{\alpha}=\frac{2}{3}.$
However, $\sup\{r(x),r(y)\}<0$.\\





%\paragraph{ \bf Some qualitative comparisons between the entropic curvature and the \textit{Bakry-Émery curvature}.}\
%\\

\paragraph{ \bf Some comparative aspects between entropic curvature and the \textit{Bakry-Émery curvature}.}\
\\

There are strong relationships between the entropic curvature and the \textit{Bakry-Émery curvature}. The notion of Bakry-Émery curvature was first introduced by Bakry and \'Emery in \cite{BE85}. The Bakry-Émery curvature is motivated by the Bochner's identity in Riemannian Geometry and has been extensively studied in discrete spaces recently  \cite{CKL21,CLP20}. Let us make two qualitative remarks on the similarities with respect to the local structure and the negativity of curvature for both notions.\

First, let us note that similarly to the entropic curvature in the case where $L=L_{0}$, recent works about the \textit{Bakry-Émery curvature-dimension conditions} developed in the context of graphs \cite{CKL21,CLP20} show that the Bakry-Émery conditions are also related to the local structure of balls of radius 2.
Indeed, in their setting \cite{CKL21}, for a locally finite graph $G=(\mathcal{X},E)$ \textit{the curvature matrix} for a vertex $z\in \mathcal{X}$ is completely determined by the ball of radius 2 around $z$ denoted as $B_{2}(z)$. More precisely it is determined by \textit{the incomplete ball of radius 2 around $z$}, that is, the graph induced by $B_{2}(z)$ removing all edges connecting vertices within $S_{2}(z)$. Interestingly, in our context our criteria for lower bounds on entropic curvature also depends  on balls of radius 2 and its value is invariant with respect to the existence of edges connecting vertices within $S_{2}(z)$.\

Second, let us note the relation between the negativity of the curvature for both concepts. Let $G=(\X,E)$ be a graph. If the \textit{the punctured 2-ball}  defined in \cite{CLP20} as $B_{2}^{o}(z):=B_{2}(z)\setminus \{z\}=S_{1}(z)\cup S_{2}(z)$ has more than one connected component then the Bakry-{\'Emery} curvature criterion %$CD(\kappa,\infty)$
at the vertex $z$ is always satisfied  with curvature term $\kappa_{BE}<0$ with five exceptions \cite[Theorem 6.4]{CLP20}. Note that similarly in this configuration, the lower bound on entropic curvature $r$ is always less or equal to zero. Indeed, let $z\in(\mathcal{X},E)$ such that $B_{2}^{o}(z)$ has more than one connected component then by choosing two neighbours of $z$ in two different connected components of $B_{2}^{o}(z)$ one obtains that  there exists a unique midpoint between them (which is $z$), and thus $r\leq 0$. \
%Note that this is a sufficient but not necessary condition to have non-positive curvature in both settings. For example the Petersen graph does not satisfy the local disconnectedness property for any vertex and has negative entropic curvature as detailed in section 5 as well as negative curvature for $CD(\kappa,\infty)$ with $\kappa \cite[Example 5.16]{CLP20}.\\


%Firstly, let us  make two qualitative comparative remarks related to the local structure and the negativity of curvature for both notions .\


%For the sake of completeness, let us introduce the Bakry-Émery curvature. The notion of Bakry-Émery curvature was first introduced by Bakry and \'Emery in \cite{BE85}. The Bakry-Émery curvature is motivated by the Bochner's identity in Riemannian Geometry and has been extensively studied in discrete spaces recently  \cite{CKL21,CLP20}.
%To define the Bakry's Émery's curvature dimension inequality, we will briefly introduce the bilinear forms $\Gamma$ and $\Gamma_{2}$ respectively,
%\begin{align*}
%    2\Gamma_{2}(f,g)&:=\Delta(fg)-f\Delta g-g\Delta f,\\
%    2\Gamma_{2}(f,g)&:=\Delta(\Gamma(f,g))-\Gamma(f,\Delta g)-\Gamma(g,\Delta f)
%\end{align*}
%where $\Delta$ is the discrete Laplace operator. As a convention $\Gamma(f,f)$ and $\Gamma_{2}(f,f)$ are denoted by $\Gamma(f)$ and $\Gamma_{2}(f)$ respectively. 
%\begin{definition}\cite{YS10,Sch99}
%Let $G=(\mathcal{X},E)$ be a graph. Let $\kappa\in \mathbb{R}$ and $N\in (0,\infty]$. 
%A vertex $x\in \mathcal{X}$ satisfies the Bakry-Émery's curvature-dimension inequality $CD(\kappa,N)$, if for any $f:\mathcal{X}\rightarrow \mathbb{R}$
%\begin{equation*}
%    \Gamma_{2}(f)(x)\geq \frac{1}{N}(\Delta f(x))^{2}+\kappa\Gamma(f)(x),
%\end{equation*}
%where $N$ is called a dimension parameter and $\kappa$ is regarded as a lower Ricci bound at $x\in \X$.  
%\end{definition}
 
 %\subparagraph{\bf A few words about the local structure of the Bakry-Émery curvature.} 
%Let us mention two qualitative similarities between the entropic curvature and the Bakry-Émery curvature in graphs. 




 
 
%\subparagraph{ \bf Structured graphs and Bakry-{\'E}mery curvature.}\
 
Finally, let us recall that according to Theorem \ref{thmstructure}, any structured graph has non negative entropic curvature. Moreover, if it satisfies a commutativity assumption with respect to the set of moves $\Sc$, then every $z\in \X$ verifies the Bakry-{\'E}mery curvature condition $CD(\kappa_{BE},\infty)$ 
%$CD(\kappa,\infty)$
with curvature term $\kappa_{BE}\geq 0$. For the sake of completeness, the definition of the Bakry-{\'E}mery curvature condition $CD(\kappa_{BE},\infty)$ is recalled in Appendix B, together with  the proof of the next result, adapted in the context of structured graphs from a proof in   \cite{CY96,YS10} revisited in \cite{CKKLP21}. 


 \begin{proposition}\label{BE}
 Let $G=(\X,E)$ be a structured graph with a finite set $\Sc$ of maps and let us suppose that moves in $\Sc$ commute, that is, $\sigma\tau=\tau\sigma$ for all
 $\sigma,\tau \in \Sc$ then the Bakry-{\'E}mery curvature criterion $CD(0,\infty)$ is satisfied for every $z\in \X$.
 \end{proposition}
% The proof of this proposition is postponed in Appendix B.\
 
 
 {\bf Comment:}  The commutativity condition is not a necessary condition for structured graphs  to satisfy the $CD(\kappa_{BE},\infty)$  criterion with  $\kappa_{BE}\geq 0$. Indeed, the Bernoulli-Laplace model corresponds to a non-commutative structured graph and satisfies the $CD(\kappa_{BE},\infty)$  criterion with positive curvature term $\kappa_{BE}$ as shown in \cite[Theorem 2.7]{KKRT16}.
 %\subparagraph{ \bf Structured graphs and Bakry-{\'E}mery curvature.}\
 %As for the entropic curvature (Theorem 6), it can be shown that structured graphs %have non nonnegative curvature in the Bakry-{\'E}mery sense.
 %\textcolor{blue}{Ne faudrait il pas etudier la courbure d'Ollivier-Lin Lu-Yau  %pour ces graphes? }
 
%\begin{proposition}\label{structuredbe}
%Let $(\X,E)$ be a structured graph associated with a finite set of moves $\Sc$. Let $z\in \X$ then $CD(0,\infty)$ holds at $z$.
%\end{proposition}
%\begin{proof}
%\Sc_z^{\tau\rightarrow \cdot}
%\Sc_z^{\cdot\rightarrow \tau}
%To prove the statement, following the ideas of \cite[Proposition 2.3]{Sch99}, it is enough to show that $\partial_{\sigma_{\tau}}$ for any $\tau\in \Sc$ commutes with the Laplacian operator $\Delta$. Let $[L,K]$ be the commutator defined as $LK-KL$ for any suitable operators $K$ and $L$.
%Let $z\in \X$, $\tau\in \Sc$ and for any $g\in B_{2}(z)$
%\begin{align*}
%[\Delta,\partial_{\sigma_{\tau}}]&=\sum_{\sigma\in \Sc} \Big(g(\tau(\sigma(z))-g(\sigma(\tau(z))\Big)\\
%&=\sum_{\sigma\in \Sc_z^{\cdot\rightarrow \tau}}g(\tau(\sigma(z))+\sum_{\sigma\in \Sc,z\sim \tau(\sigma(z))}g(\tau(\sigma(z))\\ 
%&-\sum_{\sigma\in \Sc_z^{\tau\rightarrow \cdot}} g(\sigma(\tau(z))-\sum_{\sigma\in \Sc,z\sim \sigma(\tau(z))} g(\sigma(\tau(z))  \\
%&=\sum_{\psi(\sigma)\in \Sc_z^{\tau\rightarrow \cdot}}g(\psi(\sigma)\tau(z))-
%\sum_{\sigma\in \Sc_z^{\tau\rightarrow \cdot}} g(\sigma(\tau(z))\\
%&+\sum_{\sigma\in \Sc,z\sim \tau(\sigma(z))}g(\tau(\sigma(z))-\sum_{\sigma\in \Sc,z\sim \sigma(\tau(z))} g(\sigma(\tau(z)) \\
%&=\sum_{\sigma\in \Sc,z\sim \tau(\sigma(z))}g(\tau(\sigma(z))-\sum_{\sigma\in \Sc,z\sim \sigma(\tau(z))} g(\sigma(\tau(z)) ,
%\end{align*}
%where in the penultimate line we have used that the fact that $\psi$ is a one to one map between $\Sc_z^{\tau\rightarrow \cdot}$ and $\Sc_z^{\cdot\rightarrow \tau}$.
%We claim that $\sum_{\sigma\in \Sc,z\sim \tau(\sigma(z))}g(\tau(\sigma(z))-\sum_{\sigma\in \Sc,z\sim \sigma(\tau(z))} g(\sigma(\tau(z))$ is zero. 
%For this purpose, let $\sigma^{\prime}\in \Sc$ be such that $\tau(\sigma^{\prime}(z))\sim z$  and such that $\sigma^{\prime}(tau(z))\sim z$ , we claim that then $\tau(\sigma^{\prime}(z))=\sigma^{\prime}(tau(z))$, which would prove the first claim. We proceed by contradiction. 
%Let suppose that $\tau(\sigma^{\prime}(z))\neq \sigma^{\prime}(tau(z))$. 
%In what follows we will use Lemma 3$(i)$ for structured graphs. 
%Since $(z,\tau(\sigma^{\prime}(z)))\in G(z,\tau(\sigma^{\prime}(z)))$, $(\tau(z),\tau(\sigma^{\prime}(z)))\in G(z,\tau(\sigma^{\prime}(z))$ which implies that $\tau(z)=z$. Likewise, $\sigma^{\prime}(z)=z$ .
%Thus, $\sigma^{\prime}(\tau(z))=\sigma^{\prime}(z)=z$ and $\tau(\sigma^{\prime}(z))=\tau(z)=z$ which contradicts the assumption that $\tau(\sigma^{\prime}(z))=\sigma^{\prime}(\tau(z))$.

%\end{proof}
 
 %Let $(\X,E)$ be a structured graph associated with a finite set of moves$\Sc:=\big\{\sigma_i\,\big|\,i\in [n]\big\}$ then $\partial_{\sigma_{j}}$ forany $j\in [n]$ commutes with the Laplacian operator $\Delta$. 

%Let us observe that every structured graph $(\X,E)$ associated with a finite set of moves $\Sc:=\big\{\sigma_i\,\big|\,i\in [n]\big\}$ has always non negative curvature in the Bakry-{\'E}mery sense. For this, thanks to \cite[Proposition 2.3]{Sch99}, it suffices to show that the operator $\partial_{\sigma_{j}}$ for any $j\in [n]$ commutes with the Laplacian operator $\Delta$. Let $[L,K]$ be the commutator defined as $LK-KL$ for any suitable operators $K$ and $L$.}
%Let $z\in \X$, then
%\begin{equation*}
%[\Delta,\partial_{\sigma_{j}}]g(z)=\Delta (\partial_{\sigma_{j}}g)(z)-\partial_{\sigma_{j}}(\Delta g)(z)=\sum_{i=1}^{n}g\big(\sigma_{j}(\sigma_{i}(z))\big)-g\big(\sigma_{i}(\sigma_{j}(z))\big)=\sum_{i=1}^{n}g\big(\psi(\sigma_{i})(\sigma_{j}(z))\big)-g\big(\sigma_{i}(\sigma_{j}(z))\big)=0
%\end{equation*}
%where in the last equation we have used the fact that $\psi(\sigma_{i})=\sigma_{l}$ for some $l\in [n]$ and  $\psi(\sigma_{i})\neq \psi(\sigma_{j})$ for some $j,l\in [n]$ distinct since $\psi$ is a one to one map. By the Proposition already mentioned the result follows.}
 
 




\section{Appendix A}

\begin{lemma}\label{lemL} Let $\X$ be  a structured graph associated to a set of moves $\Sc$. %Assume that $L$ is a generator on $\X$ satisfying condition \eqref{condL}. 
Then the following properties hold :
\begin{enumerate}[label=(\roman*)]
\item \label{1} Given $d\in \N$, $\tau \in \Sc$ and $\alpha_0,\ldots,\alpha_d\in \X$, if $\big(\alpha_0,\ldots,\alpha_d, \tau(\alpha_d)\big)\in G\big(\alpha_0,\tau(\alpha_d)\big)$ then for any $k\in\{0,\ldots,d\}$ one has $\big(\alpha_0,\ldots , \alpha_k,\tau(\alpha_k),\ldots,\tau(\alpha_d)\big)\in G\big(\alpha_0,\tau(\alpha_d)\big)$.
If moreover the generator $L$  on $\X$ satisfies condition \eqref{condL}, then  one has 
 \[L\big(\alpha_0,\ldots,\alpha_d, \tau(\alpha_d)\big)=L\big(\alpha_0,\ldots , \alpha_k,\tau(\alpha_k),\ldots,\tau(\alpha_d)\big).\]

\item \label{3} For any $x,y,z\in \X$ and $\tau\in \Sc$ with $\tau(z)\sim z$, if  $(z,\tau(z))\in [x,y]$ then  $\tau(x)\in ]x,y]$ and $\tau(z)\in [\tau(x),y]$.

\item \label{6} Let $z\in \X$ and $\tau,\sigma_1,\sigma_2\in \Sc $ such that 
$d\big(z,\tau(\sigma_1(z))\big)=2 $ and $\tau(\sigma_1(z))=\tau(\sigma_2(z))$. Then one has $\sigma_1=\sigma_2$.


\item \label{2}
Let $d\in \N^*$, $\tau \in \Sc$ and $\alpha_0,\beta_2,\ldots \beta_{d+1}\in \X$. If $(\alpha_0, \tau(\alpha_0), \beta_2, \ldots ,\beta_{d+1})\in G(\alpha_0, \beta_{d+1})$ then for any $k\in \{0,\ldots, d\}$, there exists a single $(\alpha_0,\alpha_1,\ldots, \alpha_k)\in \X^{k+1}$ such that \[(\alpha_0,\alpha_1,\ldots, \alpha_k,\tau(\alpha_k), \beta_{k+2},\ldots, \beta_{d+1})\in G(\alpha_0, \beta_{d+1}),\] and for any $\ell\in [k]$, $\beta_{\ell+1}=\tau(\alpha_\ell)$.

\item \label{4} Given $d\in \N^*$, $\tau \in \Sc$ and $\alpha_0\in \X$, let 
\[{\mathcal Y}(\alpha_0,\tau ,d):=\big\{(\alpha_1,\ldots,\alpha_{d+1})\in \X^{d+1}\,\big|\, (\alpha_0,\ldots,\alpha_d,\alpha_{d+1})\in G(\alpha_0,\alpha_{d+1}), \alpha_{d+1}=\tau (\alpha_d)\big\},\]
and 
\[{\mathcal W}(\alpha_0,\tau ,d):=\big\{(\beta_2,\ldots,\beta_{d+1})\in \X^{d}\,\big|\, (\alpha_0,\tau(\alpha_0),\beta_2,\ldots,\beta_{d+1})\in G(\alpha_0,\beta_{d+1})\big\}.\]
The map $\Psi:(\alpha_1,\ldots, \alpha_{d+1})\mapsto (\tau(\alpha_1),\ldots, \tau(\alpha_d))$ is one to one from the set ${\mathcal Y}(\alpha_0,\tau ,d)$  to the set ${\mathcal W}(\alpha_0,\tau ,d)$.
\item \label{5} Assume that $L$ is a generator on $\X$ satisfying condition \eqref{condL}. 
 Let $x,y\in \X$ and $\tau\in \Sc$ such that $\tau(x)\in ]x,y]$ and let  $k\in \{0,\ldots,d-1\}$ where $d=d(x,y)-1$. Then  one has 
 \[\sum_{z\in \X,d(x,z)=k,(z,\tau(z))\in[x,y] } L^k(x,z)L(z,\tau(z))L^{d-k-1}(\tau(z),y)=L(x,\tau(x)) L^{d-1}(\tau(x), y).\]
%\item \label{6} For any $x,y\in \X$ and $\sigma\in \Sc$ with  $z\in [x,\sigma(z)[$ 
%    \[L^{d(x,z)}(x,z)L(z,\sigma(z))=L(x,\sigma(x))L^{d(\sigma(x),z)}(\sigma(x),z).\]

%\item \label{1} Let $\sigma\in \Sc$ and $d\geq 2$ an integer. %Let  $\alpha:=(z_0,z_1,\ldots, z_{d})$ be a discrete geodesic path from $z_0$ to $z_{d}$ drawn on the graph $\X_S$ such that  $z_d=\sigma(z_{d-1})$ (for any $k\in\{0,\ldots,d-1\}$, there exists $\tau\in S$ such that $\tau(z_k)=z_{k+1}$). Then the  path  $\beta:=(z_0,\sigma(z_0),\sigma(z_1),\ldots,\sigma(z_{d-1}))$ is also a discrete path in $\X$ from $z_0$ to $z_d$ and 
%The following are equivalent 
%\begin{itemize}
%\item 
%$(z_0,z_1,\ldots, z_{d}) \in G(z_0,z_d)$ {with} $z_{d}=\sigma(z_{d-1})$,
%\item $ d(z_0,z_d)=d$, $L(z_0,z_1,\ldots, z_{d})>0$  and $L(z_0,z_1,\ldots, z_{d}) =L(z_0,\sigma(z_0),\sigma(z_1),\ldots,\sigma(z_{d-1}))$ 
%\item $(z_0,\sigma(z_0),\sigma(z_1),\ldots,\sigma(z_{d-1}))\in G(z_0,z_d).$
%\end{itemize}

%    \item \label{2} For any $x,y,z\in \X$ and $\sigma\in \Sc$ with $\sigma(z)\sim z$,  $(z,\sigma(z))\in [x,y]$ if and only if $\sigma(x)\in ]x,y]$ and $\sigma(z)\in [\sigma(x),y]$.
    
\end{enumerate}
\end{lemma}
\begin{proof} 
The proof of \ref{1} is by induction over $k\in \{0,\ldots,d\}$. The property holds for $k=d$ by assumption. Assume that for some fixed $k\in [d]$,   $\big(\alpha_0,\ldots , \alpha_k,\tau(\alpha_k),\ldots,\tau(\alpha_d)\big)$ is a discrete geodesic. Then since $d(\alpha_{k-1},\alpha_k)=1$, there exists a single  $\sigma_k\in \Sc$ such that 
$\sigma_k(\alpha_{k-1})=\alpha_k$. Since $d\big(\alpha_{k-1},\tau(\alpha_k)\big)=d\big(\alpha_{k-1},\tau(\sigma_k(\alpha_{k-1}))\big)=2$, 
according to the definition of structured graphs, there exists $\psi(\sigma_k)\in \Sc_{\tau(\alpha_{k-1})}$ such that 
\[\tau\big(\sigma_k(\alpha_{k-1})\big)=\psi(\sigma_k)(\tau(\alpha_{k-1})).\]
Therefore one has $\tau(\alpha_{k-1})\in]\alpha_{k-1},\tau(\alpha_k)[$ and  $(\alpha_0,\ldots , \alpha_{k-1},\tau(\alpha_{k-1}),\ldots,\tau(\alpha_d))$ is a discrete geodesic. If moreover condition \eqref{condL} holds, then by induction hypothesis
\begin{align*}
&L\big(\alpha_0,\ldots , \alpha_{k-1},\tau(\alpha_{k-1}),\ldots,\tau(\alpha_d)\big)\\
&=L\big(\alpha_0,\ldots , \alpha_{k-1}\big)L\big(\alpha_{k-1},\tau(\alpha_{k-1})\big)L\big(\tau(\alpha_{k-1}),\tau(\sigma_k(\alpha_{k-1})) \big)L\big(\tau(\alpha_k),\ldots, \tau(\alpha_d)\big)\\
&=L\big(\alpha_0,\ldots , \alpha_{k-1}\big)L\big(\alpha_{k-1},\sigma_k(\alpha_{k-1})\big)L\big(\sigma_k(\alpha_{k-1}),\tau(\sigma_k(\alpha_{k-1})) \big)L\big(\tau(\alpha_k),\ldots, \tau(\alpha_d)\big)\\
&=L\big(\alpha_0,\ldots , \alpha_k,\tau(\alpha_k),\ldots,\tau(\alpha_d)\big)\\
&=L\big(\alpha_0,\ldots,\alpha_d, \tau(\alpha_d)\big).
\end{align*}

Item \ref{3} is an easy consequence of \ref{1}. Indeed, if $(z,\tau(z))\in [x,y]$ then, setting $d=d(x,y)$ and $k=d(x,z)$, there exists $(\alpha_0,\alpha_1,\ldots,\alpha_k, \tau(\alpha_k), \beta_{k+2}, \ldots, \beta_d)\in G(x,y)$ with $\alpha_k=z$. Item \ref{1} implies that \[(x,\tau(x),\tau(\alpha_1),\ldots, \tau(\alpha_k), \beta_{k+2}, \ldots, \beta_d)=(\alpha_0,\tau(\alpha_0),\ldots, \tau(\alpha_k), \beta_{k+2}, \ldots, \beta_d)\in G(x,y),\]
and therefore $\tau(x)\in ]x,y]$ and $\tau(z)\in[\tau(x),y]$.

For the proof of \ref{6}, let $z\in\X$ and $\sigma_1,\sigma_2,\tau \in \Sc$ such that $d\big(z,\tau(\sigma_1(z))\big)=2$. If $\tau(\sigma_1(z))=\tau(\sigma_2(z))$ then according to the definition of structured graphs $\psi(\sigma_1)(\tau(z))=\psi(\sigma_2)(\tau(z))$. It follows that $\psi(\sigma_1)=\psi(\sigma_2)$ and therefore $\sigma_1=\sigma_2$ since the map $\psi$ is one to one. 

The proof of \ref{2} is  by induction over $k\in \{0,\ldots,d\}$. The property holds for $k=0$ by assumption. Assume that for a fixed $k\in\{0,\ldots,d-1\}$, there exists a single $(\alpha_0,\alpha_1,\ldots, \alpha_k)\in \X^{k+1}$ such that \[(\alpha_0,\alpha_1,\ldots, \alpha_k,\tau(\alpha_k), \beta_{k+2},\ldots, \beta_{d+1})\in G(\alpha_0, \beta_{d+1}),\] and for any $\ell\in [k]$, $\beta_{\ell+1}=\tau(\alpha_\ell)$. According to the definition of structured graphs, since $d(\tau(\alpha_k), \beta_{k+2})=1$, there exists a single $\sigma_{k+1}'\in \Sc_{\tau(\alpha_k)}$ such that $\beta_{k+2}=\sigma_{k+1}'(\tau(\alpha_k))$, and since $d\big(\alpha_k,\sigma_{k+1}'(\tau(\alpha_k))\big)=2$, one has
\[\beta_{k+2}=\sigma_{k+1}'(\tau(\alpha_k))=\tau\big(\psi^{-1}(\sigma_{k+1}')(\alpha_k)\big).\]
Setting $\alpha_{k+1}=\psi^{-1}(\sigma_{k+1}')(\alpha_k)$, one has $\beta_{k+2}=\tau(\alpha_{k+1})$ and since $\alpha_{k+1}\in ]\alpha_k,\beta_{k+2}[$ it follows that  
\[(\alpha_0,\alpha_1,\ldots, \alpha_{k+1},\tau(\alpha_{k+1}), \beta_{k+3},\ldots, \beta_{d+1})\in G(\alpha_0, \beta_{d+1}).\] 
Moreover if $\alpha'_{k+1}$ is such that $\beta_{k+2}=\tau(\alpha_{k+1}')$ and 
\[(\alpha_0,\alpha_1,\ldots,\alpha_k, \alpha_{k+1}',\tau(\alpha_{k+1}'), \beta_{k+3},\ldots, \beta_{d+1})\in G(\alpha_0, \beta_{d+1}),\]
then there exists $\sigma_{k+1}$ such that $\alpha_{k+1}'=\sigma_{k+1}(\alpha_k)$. Applying  \ref{6} %, one has $\beta_{k+2}=\tau(\sigma_{k+1}(\alpha_k))=\tau((\psi^{-1}(\sigma_{k+1}')(\alpha_k))$ but also $\beta_{k+2}=\psi(\sigma_{k+1})(\tau(\alpha_k))=\sigma_{k+1}'(\tau(\alpha_k))$. 
it follows that $\psi(\sigma_{k+1})=\sigma_{k+1}'$ and therefore 
\[\alpha_{k+1}'=\sigma_{k+1}(\alpha_k)=\psi^{-1}(\sigma_{k+1}')(\alpha_k)=\alpha_{k+1}.\]
This ends the proof of \ref{2}.

We now turn to the proof of \ref{4}. Let $d\in \N^*$, $\tau \in \Sc$ and $\alpha_0\in \X$. If $(\alpha_1,\ldots,\alpha_{d+1})\in {\mathcal Y}(\alpha_0,\tau, d)$ then according to \ref{1} with $k=0$, one has 
\[(\alpha_0,\tau(\alpha_0),\ldots,\tau(\alpha_d))\in G(\alpha_0,\tau(\alpha_d)),\]
and therefore $\Psi(\alpha_1,\ldots,\alpha_{d+1})\in {\mathcal W}(\alpha_0,\tau, d)$.
Conversely if $(\beta_2,\ldots,\beta_{d+1})\in {\mathcal W}(\alpha_0,\tau, d)$, then 
\[(\alpha_0,\tau(\alpha_0),\beta_2,\ldots,\beta_{d+1})\in G(\alpha_0,\beta_{d+1}),\]
and according to \ref{2} for $k=d$, there exists a single $(\alpha_0,\ldots, \alpha_d)$ such that 
\[(\alpha_0,\alpha_1,\ldots,\alpha_d, \tau(\alpha_d))\in  G(\alpha_0,\beta_{d+1}),\]
and for all $\ell\in[d]$, $\beta_{\ell+1}=\tau(\alpha_\ell)$.
Therefore, there exists a single  $(\alpha_1,\ldots, \alpha_d)\in {\mathcal Y}(\alpha_0,\tau, d)$ such that $\psi(\alpha_1,\ldots, \alpha_d)=(\beta_2,\ldots ,\beta_{d+1})$.

For the proof of \ref{5}, let $L$ be a generator on $\X$ satisfying condition \eqref{condL}, let $x,y\in \X$ and $\tau\in \Sc$ such that $\tau(x)\in ]x,y]$ and let  $k\in \{0,\ldots,d-1\}$ where $d=d(x,y)-1$. By definition,  one has 
\begin{multline*}
 \sum_{z\in \X,d(x,z)=k,(z,\tau(z))\in[x,y] } L^k(x,z)L(z,\tau(z))L^{d-k-1}(\tau(z),y)\\
 =\sum_{(\alpha_1,\ldots, \alpha_{k+1})\in {\mathcal Y}(x,\tau,k)}\sum_{\gamma\in G(\tau(\alpha_k),y)} L(x,\alpha_1,\ldots, \alpha_{k+1})\, L(\gamma).
\end{multline*}
Applying \ref{1} and then \ref{4} it follows that 
\begin{align*}
 \sum_{z\in \X,d(x,z)=k,(z,\tau(z))\in[x,y] }& L^k(x,z)L(z,\tau(z))L^{d-k-1}(\tau(z),y)\\
 &=\sum_{(\alpha_1,\ldots, \alpha_k)\in {\mathcal Y}(x,\tau,k)}\sum_{\gamma\in G(\tau(\alpha_k),y)} L(x,\tau(x),\tau(\alpha_1),\ldots, \tau(\alpha_k))\, L(\gamma)\\
 &=\sum_{(\beta_2,\ldots,\beta_{k+1})\in {\mathcal W}(x,\tau,k)} \sum_{\gamma\in G(\beta_{k+1},y)} L(x,\tau(x),\beta_2,\ldots, \beta_{k+1})\, L(\gamma)\\
 &=L(x,\tau(x))\sum_{\gamma'\in G(\tau(x),y)} L(\gamma')=L(x,\tau(x))L^{d-1}(\tau(x),y).
\end{align*}
The proof of Lemma \ref{lemL} is completed.
%The proof of equivalences in \eqref{1} is by induction  over $d\geq 2$. Condition \eqref{condL} ensures that equivalences hold  for $d=2$ since according  to the $C$-conditions, $L(z,\tz)>0$ if and only if $z\sim \tz$. Assume that equivalences  hold for some fixed integer $d\geq 2$. If      $\alpha=(z_0,z_1,\ldots, z_{d+1})\in G(z_0,z_{d+1})$ with $\sigma(z_d)=z_{d+1}$, then obviously $d(z_0,z_{d+1})=d+1$ and therefore, for any $k\in\{0,\ldots d\}$, $d(z_k,z_{k+1})=1$. Since $L(z,\tz)>0$ if and only if $z\sim \tz$, it follows that $L(\alpha)>0$.  Applying the induction hypothesis and then  condition \eqref{condL} gives
%\begin{align*}
%L(\alpha)&=L(z_0,z_1)L\big((z_1,\ldots,z_{d+1})\big)\\
%&=L(z_0,z_1)L\big((z_1,\sigma(z_1),\ldots,\sigma(z_{d}))\big)\\
%&=L(z_0,z_1)L(z_1,\sigma(z_1))L\big((\sigma(z_1),\ldots,\sigma(z_{d}))\big)\\
%&=L(z_0,\sigma(z_0))L(\sigma(z_0),\sigma(z_1))L\big((\sigma(z_1),\ldots,\sigma(z_{d}))\big)=L(\beta).
%\end{align*}
%As a consequence $L(\beta)>0$ and it follows that  $\beta\in G(z_0,z_{d+1})$. Conversely one identically proves that $\beta\in G(z_0,z_{d+1})$ implies  $L(\alpha)=L(\beta)$ and $\alpha\in G(z_0,z_{d+1})$, from the same identities due to the induction hypothesis. This ends the proof of \eqref{1}.
%Let $x,y,z\in \X$ and $\sigma\in \Sc$ with $\sigma(z)\sim z$. One has $(z,\sigma(z))\in[x,y]$ if and only if there exists $(z_0,z_1,\ldots, z_{d})\in G(x,y)$ ($d=d(x,y)$, $z_0=x$, $z_d=y$) and $k\in \{0,\ldots d-1\}$ such that $z=z_k$ $z_{k+1}=\sigma(z)$. By using \eqref{1}, one easily check that this is equivalent to the fact that for some $k\in \{0,\ldots d-1\}$,
%\[(z_0,\sigma(z_0),\sigma(z_1),\ldots,\sigma(z_k),z_{k+2},\ldots, z_d) \in G(x,y).\] 
%This exactly means that  $\sigma(z_0)=\sigma(x)\in ]x,y]$ and $\sigma(z_k)=\sigma(z)\in [\sigma(x),y]$, which ends the proof of \eqref{2}.
%The proof of \eqref{3} easily follows from \eqref{1}. Let $x,z\in \X$ and $\sigma\in \Sc$ with  $z\in [x,\sigma(z)[$.  The expected inequality is obvious for $x=z$ and equivalent to \eqref{condL} for $d(x,z) =1$. Assume that $d=d(x,z)\geq 2$. According to \eqref{1}  one has 
%\begin{align*}
%  L^{d(x,z)}(x,z)L(z,\sigma(z))&= \sum_{\alpha=(z_0,\ldots,z_{d+1})\in G(x,\sigma(z)),z_d=z ,z_{d+1}=\sigma(z)} L(\alpha)\\
%  &=\sum_{\beta=(z_0,\sigma(z_0),\ldots,\sigma(z_{d}))\in G(x,\sigma(z)),z_d=z} L(\beta)\\
%  &=L(x,\sigma(x))\sum_{z_1,\ldots, z_{d-1}\in \X_S}L(\sigma(x),\sigma(z_1))\cdots L(\sigma(z_{d-1}),\sigma(z))\\
%  &=L(x,\sigma(x))\sum_{z'_1,\ldots, z'_{d-1}\in \X_S}L(\sigma(x),z'_1)\cdots L(z'_{d-1},\sigma(z))\\
%  &=L(x,\sigma(x))L^{d(\sigma(x),\sigma(z))}(\sigma(x),\sigma(z)).
%\end{align*}
%where the last equalities holds since  $d(\sigma(x),\sigma(z))=d$ and since for any $w\in \X$, $S_1(\sigma(w))=\sigma(S_1(w))\subset \sigma(\X)$.
\end{proof}




\begin{lemma}\label{deriveseconde} Let   $(\X,d,m,L)$ be a graph space. Let $v:\X\to \R$ be a bounded function and given  $x,y\in\X$  let 
\[R(t)=\int v \, d\nu_t^{x,y},\qquad t\in (0,1),\]
where $(\nu_t^{x,y})_{t\in [0,1]}$ is the Schr\"odinger path between Dirac measures at $x$ and $y$ defined  by \eqref{pathdirac}. 
One has for any $t\in[0,1]$,
\[R''(t):=
d(x,y)\big(d(x,y)-1\big)\, D_t v(x,y),
\]
with 
\begin{multline*}
  D_t v(x,y):=\sum_{(z,\ttz)\in [x,y], d(z,\ttz)=2} \left[\sum_{\tz\in  ]z,\ttz[} \left(v(\ttz)+v(z)-2v(\tz)\right) L(z,\tz)L(\tz,\ttz) \right] \\r(x,z,\ttz,y)\, \rho_t^{d(x,y)-2}(d(x,z))  \end{multline*}
\end{lemma}

\begin{proof}[Proof of Lemma \ref{deriveseconde}]
Let $d:=d(x,y)$. For $t\in [0,1]$, one has 
\[R(t)=\sum_{k=0}^d\rho_t^d(k) R_{k},\quad\mbox{
with}\quad 
R_{k}:=\sum_{z\in [x,y], d(x,z)=k} v(z) \,\frac{L^{d(x,z)} (x,z)L^{d(z,y)}(z,y)}{L^d(x,y)}\]
Simple computations give for any $t\in [0,1]$,
\[R'(t)=d\,\sum_{k=0}^{d-1}\rho_t^{d-1}(k) \big(R_{k+1}-R_{k}\big),\]
and therefore
\[R''(t)=d(d-1)\,\sum_{k=0}^{d-2}\rho_t^{d-2}(k) \big(R_{k+2}+R_{k}-2R_{k+1}\big).\]
Then the result follows  observing that for any  $k\in\{0,\ldots ,d-2\}$, 
\begin{align*}
&R_{k+2}+R_{k}-2R_{k+1}\\
&= \sum_{(z,\tz,\ttz)\in [x,y], d(x,z)=k} \left(v(\ttz)+v(z)-2v(\tz)\right) \frac{L^{d(x,z)}(x,z)L(z,\tz)L(\tz,\ttz)L^{d(\ttz,y)}(\ttz,y)}{L^{d}(x,y)}\\
&= \sum_{(z,\ttz)\in [x,y] d(x,z)=k, d(z,\ttz)=2} \left[\sum_{\tz\in  ]z,\ttz[} \left(v(\ttz)+v(z)-2v(\tz)\right) L(z,\tz)L(\tz,\ttz) \right] r(x,z,\ttz,y) .
\end{align*}
\end{proof}




\section{Appendix B}
\subsection{Proofs of Theorem \ref{thmprinc}, Theorem \ref{thmprincbis} and  Theorem \ref{Thmstructure}
}
\begin{proof}[Proof of Theorem \ref{thmprinc}]
Theorem \ref{thmprinc} is a consequence of Lemma 3.1 and Theorem  3.5 of \cite{Sam21}. 
The results of this paper \cite{Sam21} are given for graph spaces and the two  following additional assumptions : 
the measure $m$ is uniformly upper bounded and lower bounded away from 0,
\begin{equation*}
\sup_{x\in \X} m(x)<\infty,\qquad \inf_{x\in \X} m(x)>0,
\end{equation*}
and the generator $L$ is uniformly upper bounded, and uniformly lower bounded away from zero on the set of neighbours,
\[\sup_{x\in \X} |L(x,x)|<\infty,\qquad \inf_{x,y\in \X, d(x,y)=1} L(x,y)>0.\]
These conditions are not  in the setting of Theorem \ref{thmprinc}. To overcome this difficulty, one will consider a well chosen space $(\Cc,d,L_\Cc, m_\Cc)$ defined as the restriction of the   
space $(\X,d,m,L)$ to a well chosen finite  convex subset $\Cc$ as defined in section \ref{sectionrestconv}. 



Let $\nu_0$ and $\nu_1$ be two probability measures on $\X$ with   bounded support. Since each vertex has bounded degree, there exists a finite convex subset $\Cc$ of $\X$ that contains all the balls of radius 2 with center in the finite subset $[\supp(\nu_0),\supp(\nu_1)]$. Choose for example the convex subset $\Cc$ with minimal elements. Let $(\widehat \nu_t)_{t\in[0,1]}$ denotes the Schr\"odinger bridge at zero temperature selected from the slowing down procedure on the space $(\Cc,d,L_\Cc, m_\Cc)$. As explained in \cite{Sam21} there exists  a $W_1$-optimal coupling $\widehat \pi$ with marginals $\nu_0$ and $\nu_1$ such that the expression of $(\widehat \nu_t)_{t\in[0,1]}$ is given by \eqref{defhatnut} on the space $(\Cc,d,L_\Cc, m_\Cc)$. Due to the
assumption on the subset $\Cc$, for any $(x,y)\in \supp \widehat\pi$, the set $[x,y]$ is the same on the space $(\X,d,m,L)$ and on the space $(\Cc,d,L_\Cc, m_\Cc)$. Moreover, since $L_\Cc(x,y)=L(x,y)$ for $x\neq y$, the expression of $r(x,z,z,y)$ for $z\in[x,y]$ and $(x,y)\in \supp (\widehat\pi)$ is also the same on $(\X,d,m,L)$ and on $(\Cc,d,L_\Cc, m_\Cc)$.  Therefore the expression of the Schr\"odinger bridges between Dirac measure 
$\delta_x$ and $\delta_y$ is for us given by \eqref{pathdirac} does not depend on the chosen convex subset $\Cc$. Up to now we are working on $(\Cc,d,L_\Cc, m_\Cc)$ but for most of all expressions we write, there is no dependence in $\Cc$ or the subset $\Cc$ may be replaced by $\X$. 

For any $t\in (0,1)$, the support of  $\widehat  \nu_t$, denoted by $\widehat Z$ for simplicity sake,  is given by 
\[\widehat Z:=\supp(\widehat  \nu_t)=\bigcup_{(x,y)\in \supp(\widehat \pi)}[x,y].\]
For any $z\in \widehat Z$, let
\[V_{_\rightarrow}(z):=\Big\{\tz\in S_1(z)\,\Big|\, (z,\tz)\in C_\rightarrow\Big\} \quad\mbox{ and }\quad V_{_\leftarrow}(z):=\Big\{\tz\in S_1(z)\,\Big|\, (z,\tz)\in C_\leftarrow\Big\},\]
and similarly 
\[\V_{_\rightarrow}(z):=\Big\{\tz,\in S_2(z)\,\Big|\, (z,\tz)\in C_{_\rightarrow}\Big\} \quad\mbox{ and }\quad\V_{_\leftarrow}(z):=\Big\{\tz,\in S_2(z)\,\Big|\, (z,\tz)\in C_{_\leftarrow}\Big\},\] 
where 
\[C_{_\rightarrow}:=\Big\{(z,w)\in\X\times \X\,\Big|\,z\neq w, \exists (x,y)\in \supp(\widehat{\pi}), (z,w)\in [x,y]\Big\},\]
and 
\[C_{_\leftarrow}:=\Big\{(z,w)\in\X\times \X\,\Big|\, (w,z) \in C_{_\rightarrow} \Big\}.\] 
 As explained in \cite{Sam21},  $C_{_\rightarrow}$ and $C_{_\leftarrow}$ are disjoint sets which implies that  $V_{_\rightarrow}(z)\cap V_{_\leftarrow}(z)=\emptyset $, and also  $\V_{_\rightarrow}(z)\cap \V_{_\leftarrow}(z)=\emptyset $.  
 
 For any $z\in \widehat Z$
let 
 \[ \widehat Y_z:=\Big\{y\in\supp(\nu_1)\,\Big|\, \exists x\in \X,  (x,y)\in \widehat{\pi}, z\in[x,y]\Big\},\]
 and identically let 
 \[ \widehat X_z:=\Big\{x\in\supp(\nu_0)\,\Big|\, \exists y\in \X,  (x,y)\in \widehat{\pi}, z\in[x,y]\Big\}.\]
 For $y\in \supp(\nu_1)$, $z\in \X$ and $t\in [0,1]$, the quantity \begin{equation}\label{a_t}
 a_t(z,y) :=\int\nu_t^{w,y}(z) \,d\widehat{\pi}_{_\leftarrow}(w|y),
 \end{equation}
 is positive if and only if $z\in \widehat Z$ and 
 $y\in \widehat Y_z$. Identically, for  $x\in \supp(\nu_0)$,  $z\in \X$ and $t\in [0,1]$, the quantity 
 \[b_t(z,x):=\int \nu_t^{x,w}(z) \,d\widehat{\pi}_{_\rightarrow}(w|x),\]
  is positive if and only if $z\in \widehat Z$ and $x\in \widehat X_z$. Actually $a_t$ and $b_t$ represent conditional laws, $\sum_{z\in \X}a_t(z,y)=\sum_{z\in \X}b_t(z,x) =1$.
  
  
For $t\in[0,1]$, $z\in \widehat Z$,   $\tz\in S_1(z)$ and $y\in\supp(\nu_1)$, let 
\begin{equation}\label{a_t'}
{\mathrm a}_t(z,\tz,y):= \sum_{w\in \X, (z,\tz)\in[y,w]}  
r(y,z,\tz,w) \,d(y,w)\, \B_t^{d(y,w)-1}(d(z,w)-1)\,\widehat{\pi}_{_\leftarrow}(w|y),
\end{equation}
and for any $x\in \supp(\nu_0)$, let 
\begin{equation*}
{\mathrm b}_t(z,\tz,x):=\sum_{w\in \X, (z,\tz)\in[x,w]}  
\,r(x,z,\tz,w) \,d(x,w)\, \B_t^{d(x,w)-1}(d(x,z))\,\widehat{\pi}_{_\rightarrow}(w|x).
\end{equation*}
where the function $r$ is given by \eqref{defrpont}.
For $t\in(0,1)$, the quantity ${\mathrm a}_t(z,\tz,y)$ is positive if and only if $\tz\in V_{_\leftarrow}(z)$ and 
$y\in \widehat Y_{(z,\tz)}$ with
 \[ \widehat Y_{(z,\tz)}=\Big\{y\in\supp(\nu_1)\,\Big|\, \exists x\in \X,  (x,y)\in \widehat{\pi}, (z,\tz)\in[y,x]\Big\}\subset \widehat Y_z\cap  \widehat Y_{\tz} ,\] 
 According to \cite[Lemma 3.4]{Sam21}, given $z\in \widehat Z$ and $\tz\in V_{_\leftarrow}(z)$  the ratio ${\mathrm a}_t(z,\tz,y)/a_t(z,y)$ does not depend on $y\in \widehat Y_{(z,\tz)}$. Therefore, for any $z\in \X$
 and $\tz\in S_1(z)$, one may define 
 \[A_t(z,\tz):=\left\{\begin{array}{ll}
  \frac{{\mathrm a}_t(z,\tz,y)}{a_t(z,y)}& \mbox{ for } (z,\tz) \in \widehat Z\times V_{_\leftarrow}(z) \mbox{ and } y\in \widehat Y_{(z,\tz)}\neq \emptyset, \\
  0 & \mbox{ otherwise.} 
 \end{array}\right.\] 
 Identically, for $t\in(0,1)$, the quantity ${\mathbbm b}_t(z,\ttz,x)$ is positive if and only if $\tz\in V_{_\rightarrow}(z)$ and $x\in X_{(z,\tz)}$ with
 \[ \widehat X_{(z,\tz)}=\Big\{y\in\supp(\nu_1)\,\Big|\, \exists x\in \X,  (x,y)\in \widehat{\pi}^0, (z,\tz)\in[x,y]\Big\}\subset \widehat X_z\cap  \widehat X_{\tz},\]
 and according to \cite[Lemma 3.4]{Sam21},  the ratio ${\mathrm b}_t(z,\tz,x)/b_t(z,x)$ does not depend on
$x\in X_{(z,\tz)}$. Therefore, for any $z\in \X$
 and $\tz\in S_1(z)$, one  defines 
 \[B_t(z,\tz):=\left\{\begin{array}{ll}
  \frac{{\mathrm b}_t(z,\tz,x)}{b_t(z,x)}& \mbox{ for } (z,\tz) \in \widehat Z\times V_{_\rightarrow}(z) \mbox{ and } x\in \widehat X_{(z,\tz)}\neq \emptyset, \\
  0 & \mbox{ otherwise.} 
 \end{array}\right.\] 
 Observe that by reversibility, for any $(z,\tz)\in C_{_{\rightarrow}}$ with $d(z,\tz)=1$,
 \begin{align*}
   B_t(z,\tz)L(z,\tz)\widehat{\nu}_t(z)& =\sum_{x\in \widehat X_{(z,\tz)}} B_t(z,\tz)L(z,\tz) b_t(z,x)\nu_0(x)\\
   &=\sum_{x\in \widehat X_{(z,\tz)}} b_t(z,\tz,x)L(z,\tz) \nu_0(x)\\
   &=\sum_{(x,y)\in \supp(\widehat \pi), (z,\tz)\in[x,y]} r(x,z,\tz, y)L(z,\tz) d(x,y) \B_t^{d(x,y)-1}(d(x,z))\,\widehat{\pi}(x,y)\\
   &=\sum_{(x,y)\in \supp(\widehat \pi), (\tz,z)\in[y,x]} r(y,\tz,z, x)L(\tz,z) d(y,x) \B_t^{d(y,x)-1}(d(x,\tz)-1)\,\widehat{\pi}(x,y)\\
   &=\sum_{y\in \widehat Y_{(\tz,z)}} a_t(\tz,z,y)L(\tz,z) \nu_1(y)\\
   &=A_t(\tz,z)L(\tz,z)\widehat{\nu}_t(\tz).
 \end{align*}

For $t\in[0,1]$,  $z\in \widehat Z$, $\ttz\in S_2(z)$  and  $y\in\supp(\nu_1)$, define also 
 \begin{equation}\label{a_t''}
{\mathbbm a}_t(z,\ttz,y):= \!\!\!\!\!\sum_{w\in \X, (z,\ttz)\in[y,w]}  
\!\!\!\!\!r(y,z,\ttz,w) \,d(y,w)(d(y,w)-1)\, \B_t^{d(y,w)-2}(d(z,w)-2)\,\widehat{\pi}_{_\leftarrow}(w|y),
\end{equation}
and for $x\in\supp(\nu_0)$
\begin{eqnarray*}
 {\mathbbm b}_t(z,\ttz,x):=\!\!\!\!\!\sum_{w\in \X, (z,\ttz)\in[x,w]} \!\!\!\!\! 
r(x,z,\ttz,w) \,d(x,w)(d(x,w)-1)\, \B_t^{d(x,w)-2}(d(x,z))
\,\widehat{\pi}_{_\rightarrow}(w|x).
\end{eqnarray*}
For $t\in(0,1)$, we also have ${\mathbbm a}_t(z,\ttz,y)>0$ if and only if $\ttz\in \V_{_\leftarrow}(z)$ and 
$y\in \widehat Y_{(z,\ttz)}$, 
and ${\mathbbm b}_t(z,\ttz,x)>0$ if and only if $\ttz\in \V_{_\rightarrow}(z)$ and 
$x\in  \widehat X_{(z,\ttz)}$. Since according to \cite[Lemma 3.4]{Sam21}, the ratio ${\mathbbm a}_t(z,\ttz,y)/a_t(z,y)$ does not depend on $y\in \widehat Y_{(z,\ttz)}$, and the ratio ${\mathbbm b}_t(z,\ttz,x)/b_t(z,x)$ does not depend on
$x\in X_{(z,\tz)}$. Therefore one may define for any $z\in \X$ and $\ttz\in S_2(z)$, 
\[{\mathbb A}_t(z,\ttz):=\left\{\begin{array}{ll}
  \frac{{\mathbbm a}_t(z,\ttz,y)}{a_t(z,y)}& \mbox{ for } (z,\ttz) \in \widehat Z\times \V_{_\leftarrow}(z) \mbox{ and } y\in \widehat Y_{(z,\ttz)}\neq \emptyset, \\
  0 & \mbox{ otherwise,} 
 \end{array}\right.\] 
and
\[{\mathbb B}_t(z,\ttz):=\left\{\begin{array}{ll}
  \frac{{\mathbbm b}_t(z,\ttz,x)}{b_t(z,x)}& \mbox{ for } (z,\ttz) \in \widehat Z\times \V_{_\rightarrow}(z) \mbox{ and } x\in \widehat X_{(z,\ttz)}\neq \emptyset, \\
  0 & \mbox{ otherwise.} 
 \end{array}\right.\] 
One also observe that  for $t\in(0,1)$ and $z\in\widehat Z$, if $\ttz\in \V_{_\leftarrow}(z)$ (or equivalently  ${\mathbb A}_t(z,\ttz)>0$), then   $A_t(z,\tz)>0$ for any $\tz\in S_1(z)$ with  $\tz\sim \ttz$ (since $\tz\in V_{_\leftarrow}(z))$. Therefore for any $t\in(0,1)$, $z\in \widehat Z$, $\tz\in S_1(z),\ttz \in S_2(z)$, one has $(A_t(z,\tz),{\mathbb A}_t(z,\ttz))\in (0,+\infty)\times [0,+\infty)\cup\{(0,0)\}$.   Identically, one has $(B_t(z,\tz),{\mathbb B}_t(z,\ttz))\in (0,+\infty)\times [0,+\infty)\cup\{(0,0)\}$. 
 
 As above, one simply check that by reversibility, for any $(z,\ttz)\in C_{_{\rightarrow}}$ with $d(z,\ttz)=2$,
 \begin{align}\label{RelAB}
   {\mathbb B}_t(z,\ttz)L^2(z,\ttz)\widehat{\nu}_t(z)={\mathbb A}_t(\ttz,z)L^2(\ttz,z)\widehat{\nu}_t(\ttz). 
 \end{align}
 
One will apply the following theorem which is a direct result of Lemma 3.1 and the main Theorem 3.5 of \cite{Sam21}. For $z\in \widehat Z$ and $t\in (0,1)$, let 
\begin{multline*}
H_t(z):=\Big(\sum_{\tz\in V_{_\leftarrow}(z)}
A_t(z,\tz) \,L(z,\tz)\Big)^2\\  
+ \sum_{\tz\in V_{_\leftarrow}(z),\, \ttz\in \V_{_\leftarrow}(z), \,\tz\sim \ttz} \rho \Big(A_t^2(z,\tz),{\mathbb A_t}(z,\ttz)\Big)  \, L(\tz, \ttz)L(z,\tz),
\end{multline*}
and let
\begin{multline*}
K_t(z):=\Big(\sum_{\tz\in V_{_\rightarrow}(z)}
B_t(z,\tz) \,L(z,\tz)\Big)^2 \\+  \sum_{\tz\in V_{_\rightarrow}(z),\, \ttz\in \V_{_\rightarrow}(z),\, \tz\sim \ttz}  \rho\Big(B_t^2(z,\tz),\mathbbm{B}_t(z,\ttz)\Big)  \,
 L(\tz,\ttz)L(z,\tz),
 \end{multline*}
where the function $\rho:(0,+\infty)\times[0,+\infty)\cup\{(0,0)\}\to \R$ is defined by 
\begin{equation*}%\label{defG}
\rho(a,b):=\left(\log b-\log a-1\right) b,  \qquad a>0, b>0,
\end{equation*}
and $\rho(a,0)=0$ for $a\geq 0$.
According to \cite[Lemma 3.1]{Sam21} and \cite[Theorem 3.5]{Sam21} the following result holds.
\begin{theorem}\label{thmsam21}  We assume that the discrete space $(\X,d,m,L)$ is a graph space. 
Let $(\widehat \nu_t)_{t\in [0,1]}$ be the Schr\"odinger bridge at zero temperature
between two  probability measures $\nu_0,\nu_1\in \Pc(\X)$ with bounded support defined above and given by \eqref{defhatnut}. For any $t\in (0,1)$, let $q_t$ be the kernel on $[0,1]$ defined by \[q_t(s)=\frac{2s}t \1_{[0,t]}(s)+ \frac{2(1-s)}{1-t} \1_{[t,1]}(s),\qquad s\in[0,1].\] 
Then, one has 
\[(1-t)H(\nu_0|m)+tH(\nu_1|m)-H(\widehat \nu_t|m)\geq \int_0^1 \left(\int \big(H_s+K_s\big)  \,d \widehat\nu_s\right) q_t(s)\,ds.\]
As a consequence if there exists a real function $\zeta:[0,1]\to \R$ 
such that for any $s\in(0,1)$,
\begin{equation*}
\int \big(H_s+K_s\big)  \,d \widehat\nu_s \geq \zeta(s),
\end{equation*}
and if $\zeta q_t$ is integrable with respect to the Lebesgue measure on $[0,1]$, then the convexity property of entropy \eqref{deplacebis} holds with, for any $t\in (0,1)$,  
 \[
 C_t(\widehat \pi)= \int_0^1 \zeta(s) q_t(s)\, ds.\]
 \end{theorem}
 Observe that  if  $\zeta$ is a constant function then the cost $C_t(\widehat \pi)$ is equal to this constant since $\int_0^1 q_t(s)\, ds=1$. And if $\zeta=\xi''$ where $\xi$ is a real continuous functions   on $[0,1]$, twice differentiable on $(0,1)$, then one has
 \[
 C_t(\widehat \pi)= \frac2{t(1-t)}\Big[
(1-t)\xi(0)+t\xi(1)-\xi(t)\Big].\]
 Using  the following  identity, for any integer $N$, for any $b\geq 0$,  and any positive $L_1,\ldots, L_N, a_1,\ldots, a_N$,
 \[\sum_{i=1}^N  \rho(a_i^2,b) L_i = L \,\rho\Big(\prod_{i=1}^N a_i^{2L_i/L},b\Big), \qquad \mbox{with} \quad L=\sum_{i=1}^N L_i,\]
 one gets for any $z\in \widehat Z$,
  \begin{align}\label{epuise0}
H_t(z)&=\Big(\sum_{\tz\in V_{_\leftarrow}(z)}
A_t(z,\tz) \,L(z,\tz)\Big)^2\nonumber\\  
&\qquad\qquad+ \sum_{ \ttz\in \V_{_\leftarrow}(z)} \quad \sum_{\tz\in ]z,\ttz[ } \rho \Big(A_t^2(z,\tz),{\mathbb A_t}(z,\ttz)\Big)  \, L(\tz, \ttz)L(z,\tz)\nonumber\\
&=\Big(\sum_{\tz\in V_{_\leftarrow}(z)}
A_t(z,\tz) \,L(z,\tz)\Big)^2 \nonumber\\  
&\qquad\quad+ \sum_{ \ttz\in \V_{_\leftarrow}(z)} L^2(z,\ttz) \,\rho \Bigg(\prod_{\tz\in ]z,\ttz[ } A_t(z,\tz)^{\frac{2L(z,\tz)L(\tz, \ttz)}{ L^2(z,\ttz)}},{\mathbb A_t}(z,\ttz)\Bigg) \nonumber \\
&\geq \Big(\sum_{\tz\in V_{_\leftarrow}(z)}
A_t(z,\tz) \,L(z,\tz)\Big)^2 \\
&\qquad+ \rho \Bigg(\sum_{ \ttz\in \V_{_\leftarrow}(z)} L^2(z,\ttz)\prod_{\tz\in ]z,\ttz[} A_t(z,\tz)^{\frac{2L(z,\tz)L(\tz, \ttz)}{ L^2(z,\ttz)}},\sum_{ \ttz\in \V_{_\leftarrow}(z)} L^2(z,\ttz){\mathbb A_t}(z,\ttz)\Bigg),\nonumber 
\end{align}
 where for the last inequality we  use the convexity property of the function $\rho$, and the identity $\rho(\lambda a,\lambda b)=\lambda \rho(a,b)$, $a>0,b,\lambda\geq 0$.
 
Since the function $a\to \rho(a,b)$ is decreasing on $(0,+\infty)$ for any $b\geq 0$, setting 
\[\overline{\mathbb A}_t(z):=\sum_{ \ttz\in \V_{_\leftarrow}(z)} {\mathbb A_t}(z,\ttz)\,  L^2(z,\ttz)\quad \mbox{
and}\quad \overline{A}_t(z):=\sum_{\tz\in V_{_\leftarrow}(z)}
A_t(z,\tz) \,L(z,\tz),\]
and according to the definition \eqref{defR_2} of $K(z,\V_{_\leftarrow}(z))$ and since 
\[\overline{A}_t(z)\geq \sum_{\tz\in ]z,\V_{_\leftarrow}(z)[}
A_t(z,\tz) \,L(z,\tz),\]
one gets 
\begin{equation}\label{epuise}
H_t(z)\geq \overline{A}^2_t(z)+ \rho\Big(K\big(z,\V_{_\leftarrow}(z)\big)\, \overline{A}_t^2(z), \overline{\mathbb A}_t(z)\Big).
\end{equation}
One may identically prove that for any $z\in\widehat Z$,
\begin{equation}\label{epuisebis}
K_t(z)\geq \overline{B}^2_t(z)+ \rho\Big(K\big(z,\V_{_\rightarrow}(z)\big)\, \overline{B}_t^2(z), \overline{\mathbb B}_t(z)\Big),%\1_{D_{_\rightarrow}}(z),
\end{equation}
with
\[\overline{\mathbb B}_t(z):=\sum_{ \ttz\in \V_{_\rightarrow}(z)} {\mathbb B_t}(z,\ttz)\,  L^2(z,\ttz),\quad \mbox{
and}\quad \overline{B}_t(z):=\sum_{\tz\in V_{_\rightarrow}(z)}
B_t(z,\tz) \,L(z,\tz).\]
Applying the inequality 
\[\rho(K a,b)= \rho(a,b)-b\log K\geq -a -b\log K,\]
for $K,a>0 ,b\geq 0$, 
and integrating \eqref{epuise} and \eqref{epuisebis} with respect to $\widehat{\nu}_t$, it follows that 
\begin{align}\label{suiteT2}
   \int \big(H_t+K_t\big)  \,d \widehat\nu_t  &\geq \int -\overline{\mathbb A}_t(z) \,\log  K\big(z,\V_{_\leftarrow}(z)\big) -\overline{\mathbb B}_t(z) \,\log  K\big(z,\V_{_\rightarrow}(z)\big)\,d \widehat\nu_t(z)\\
   &\geq - \int \log K(z, S_2(z)) \big(\overline{\mathbb A}_t+ \overline{\mathbb B}_t\big)\,d \widehat\nu_t\nonumber 
   \geq -\log K \int \big(\overline{\mathbb A}_t+ \overline{\mathbb B}_t\big)\,d \widehat\nu_t
\end{align}
Observe that in the last inequalities the definitions of $S_2(z)$ and the constant $K$ should be first given on the space  $(\Cc,d,L_\Cc, m_\Cc)$. But according to the choose of  $\Cc$ the structure of the  balls of radius 2 centered at $z\in \widehat Z$ are the same on $(\Cc,d,L_\Cc, m_\Cc)$ as on $(\X,d,L, m)$. Therefore the definition of the constant $K$ does not depend on the chosen $\Cc$. The proof of Theorem \ref{thmprinc} ends by applying Theorem \ref{thmsam21} and since, according to \eqref{a_t''},
\begin{align*}
&\int  \overline{{\mathbbm A}}_t  \,d \widehat\nu_t =\int \sum_{z\in\widehat Z} \sum_{\ttz\in \V_{_\leftarrow}(z)}
\frac{{\mathbbm a}_t(z,\ttz,y)}{a_t(z,y)} \,L^2(z,\ttz) a_t(z,y)\,d \nu_1(y)\\
&= \iint  \sum_{(z,\ttz), (z,\ttz)\in [y,w]}  
r(y,z,\ttz,w)L^2(z,\ttz) \,d(y,w)(d(y,w)-1)\, \B_t^{d(y,w)-2}(d(z,w)-2)\,d\widehat{\pi}(w,y)\\
&= \iint \sum_{k=2}^{d(y,w)} \Big(\sum_{ (z,\ttz)\in[y,w], \ttz\in \V_{_\leftarrow}(z), d(z,w)=k} r(y,z,\ttz,w)  L^2(z,\ttz) \Big) \\
&\qquad\qquad\qquad\qquad\qquad\qquad\qquad\qquad\qquad\B_t^{d(y,w)-2}(k-2) \,d(y,w)(d(y,w)-1)\,d\widehat{\pi}(w,y)\\
&= \iint \sum_{k=2}^{d(y,w)} \B_t^{d(y,w)-2}(k-2) \,d(y,w)(d(y,w)-1)\,d\widehat{\pi}(w,y)=T_2(\widehat\pi).\\
%&=\iint d(y,w)(d(y,w)-1) \,d\widehat{\pi}(w,y)=T_2(\widehat\pi).
\end{align*}
From the identity \eqref{RelAB} it follows that $\displaystyle\int  \overline{{\mathbbm B}}_t  \,d \widehat\nu_t =\int  \overline{{\mathbbm A}}_t  \,d \widehat\nu_t=T_2(\widehat\pi)$.
\end{proof}


\begin{proof}[Proof of Theorem \ref{thmprincbis}] One assume that for any $z\in\X$, $K(z,S_2(z))<1$ and therefore $K\leq 1$. 
Applying the inequality 
$\rho(K a,b)\geq -Ka,$
for $K,a>0 ,b\geq 0$, 
\eqref{epuise} and \eqref{epuisebis} provide
\begin{align}\label{versw1}
\int \big(H_t+K_t\big)  \,d \widehat\nu_t  &\geq  \int \big[1-K\big(z,\V_{_\leftarrow}(z)\big)\big]\1_{V_{_\leftarrow}(z)\neq \emptyset}\,  \overline{A}_t^2(z)\,d \widehat\nu_t(z)\\
&\nonumber\qquad\qquad+\int\big[1-K\big(z,\V_{_\rightarrow}(z)\big)\big]\1_{V_{_\rightarrow}(z)\neq \emptyset}\, \overline{B}_t^2(z)  \,d \widehat\nu_t(z)\\
&\geq  (1-K) \int \big(\overline{A}_t^2+\overline{B}_t^2 \big) \,d \widehat\nu_t.\nonumber
\end{align}
 By Cauchy-Schwarz inequality, one has 
\begin{align*}
\int  \overline{ A}_t ^2 \,d \widehat\nu_t=\int \sum_{z\in\widehat Z} \Big(\sum_{\tz\in V_{_\leftarrow}(z)}
\frac{{\mathrm a}_t(z,\tz,y)}{a_t(z,y)} \,L(z,\tz)\Big)^2 a_t(z,y)\,d\nu_1(y)\geq \int  \Big(\sum_{z\in\widehat Z} \sum_{\tz\in V_{_\leftarrow}(z)}
{\mathrm a}_t(z,\tz,y)\,L(z,\tz)\Big)^2\,d \nu_1(y).
 \end{align*}
 Moreover,  according to \eqref{a_t'}, easy computations give
 \begin{align*}
 &\sum_{z\in\widehat Z}\;\; \sum_{\tz\in V_{_\leftarrow}(z)}
{\mathrm a}_t(z,\tz,y)\,L(z,\tz)\\
&= \sum_{w\in \X}\;\;\sum_{ (z,\tz)\in[y,w], \tz\in V_{_\leftarrow}(z)}  
r(y,z,\tz,w) \,d(y,w)\, \B_t^{d(y,w)-1}(d(z,w)-1)\,\widehat{\pi}_{_\leftarrow}(w|y) L(z,\tz)\\
&= \sum_{w\in \X}\sum_{k=1}^{d(y,w)} \Big(\sum_{ (z,\tz)\in[y,w], \tz\in V_{_\leftarrow}(z), d(z,w)=k} r(y,z,\tz,w)  L(z,\tz) \Big) \B_t^{d(y,w)-1}(k-1) \,d(y,w)\,\widehat{\pi}_{_\leftarrow}(w|y)\\
&= \sum_{w\in \X}\sum_{k=1}^{d(y,w)} \B_t^{d(y,w)-1}(k-1) \,d(y,w)\,\widehat{\pi}_{_\leftarrow}(w|y)
= \sum_{w\in \X}  \,d(y,w)\,\widehat{\pi}_{_\leftarrow}(w|y),\end{align*}
and therefore $\displaystyle\int  \overline{ A}_t ^2 \,d \widehat\nu_t\geq \int  \Big(\int d(y,w)\,d\widehat{\pi}_{_\leftarrow}(w|y)\Big)^2\,d \nu_1(y)$. Identically, one gets $\displaystyle\int  \overline{ B}_t ^2 \,d \widehat\nu_t\geq \int  \Big(\int d(x,w)\,d\widehat{\pi}_{_\rightarrow}(w|x)\Big)^2\,d \nu_0(x)$.
Since $1-K=r/2\geq 0$, it follows that 
\[
\int \big(H_t+K_t\big)  \,d \widehat\nu_t 
\geq \frac{r}2 \,\widetilde{T}(\widehat \pi).
\]
Applying then   Theorem \ref{thmsam21} provides $C_t(\widehat\pi)\geq {r}\, \widetilde{T}(\widehat \pi)/2$ which ends the proof of the first part of Theorem \ref{thmprincbis}.



Due to the above computations, we know that  $\int \overline{A}_t \,d \widehat\nu_t=\int \overline{B}_t \,d \widehat\nu_t= W_1(\nu_0,\nu_1)$. Therefore, starting again 
 from \eqref{versw1} and applying Cauchy-Schwarz inequality yields 
\begin{align*}
&\int \big(H_t+K_t\big)  \,d \widehat\nu_t\\
&\geq  W_1^2(\nu_0,\nu_1) \left(\frac{1}{ \int \big[1-K(z,\V_{_\leftarrow}(z))\big]^{-1} \1_{V_{_\leftarrow}(z)\neq \emptyset}\,d \widehat\nu_t(z)}+\frac{1} {\int \big[1-K(z, \V_{_\rightarrow}(z))\big]^{-1}  \1_{V_{_\rightarrow}(z)\neq \emptyset}\,d \widehat\nu_t(z)}\right).
\end{align*}
According to \cite[Lemma 4.3]{Sam21}, we know that if $(z,z_+)\in C_{_\rightarrow}$ and $(z,z_-)\in C_{_\leftarrow}$ then $z\in [z_-,z_+]$. As a consequence, the subsets  $V_{_\leftarrow}(z),V_{_\rightarrow}(z)$ of $S_1(z)$ and $\V_{_\leftarrow}(z),\V_{_\rightarrow}(z)$ of $S_2(z)$ satisfy condition \eqref{condsupl}. Therefore, from the definition \eqref{defr_1} of the constant $r_1$ given in Theorem \ref{thmprinc}, and the identity 
$\inf_{u,v>0, u+v\leq r^{-1}}\left\{ \frac{1}u+\frac{1}v\right\}= 4r, r> 0$,
it follows that 
\[ \int \big(H_t+K_t\big) \,d \widehat\nu_t \geq 4r_1 W_1^2(\nu_0,\nu_1).\]
Applying Theorem \ref{thmsam21} provides $C_t(\widehat \pi)\geq 4r_1\,W_1^2(\nu_0,\nu_1)$, which ends the proof of the second part of Theorem \ref{thmprincbis}.  

In order to prove the last part of Theorem \ref{thmprincbis}, one extends  to any graphs ideas from the proof of \cite[Theorem 2.5]{Sam21} on the discrete hypercube. Coming back to \eqref{suiteT2}, one has
\begin{align*}
   \int \big(H_t+K_t\big)  \,d \widehat\nu_t & \geq \int -\log K\big(z,\V_{_\leftarrow}(z)\big)\,\overline{\mathbb A}_t(z)  -\log K\big(z,\V_{_\rightarrow}(z)\big)\, \overline{\mathbb B}_t(z) \,d \widehat\nu_t(z)\\
   &=\int \sum_{z\in\widehat Z} \sum_{\ttz\in \V_{_\leftarrow}(z)}
-\log K\big(z,\V_{_\leftarrow}(z)\big)\, \1_{\V_{_\leftarrow}(z)\neq \emptyset}\, {\mathbbm a}_t(z,\ttz,y)\,L^2(z,\ttz) \,d \nu_1(y)
\\
&\qquad +\int \sum_{z\in\widehat Z} \sum_{\ttz\in \V_{_\rightarrow}(z)}
-\log K\big(z,\V_{_\rightarrow}(z)\big) \,\1_{\V_{_\rightarrow}(z)\neq \emptyset}\, {\mathbbm b}_t(z,\ttz,x)\,L^2(z,\ttz) \,d \nu_0(x)\\
&= \iint C_t(x,y) \,d\widehat \pi(x,y),
\end{align*}
with, setting $d(x,y)=d$, \begin{align*}
C_t(x,y)&:=\sum_{z\in[x,y]} -\log K\big(z,\V_{_\leftarrow}(z)\big)\,\1_{\V_{_\leftarrow}(z)\neq \emptyset} \,r(x,z,z,y) \,d(d-1)\, \B_t^{d-2}(d(x,z)-2) \\
&\qquad\qquad+\sum_{z\in[x,y]}  -\log K\big(z,\V_{_\rightarrow}(z)\big)\,\1_{\V_{_\rightarrow}(z)\neq \emptyset} \,r(x,z,z,y) \,d(d-1)\, \B_t^{d-2}(d(x,z))\\
%=\sum_{z\in [x,y]} \big[1-K\big(z,\V_{_\leftarrow}(z)\big)\big] \1_{\V_{_\leftarrow}(z)\neq \emptyset}  \frac{d(x,z)(d(x,z)-1)}{t^2}\,{\nu_t}\!^{x,y}(z)\\
%&\qquad\qquad\qquad\qquad+\sum_{z\in [x,y]}\big[1-K\big(z,\V_{_\rightarrow}(z)\big)\big]\1_{\V_{_\rightarrow}(z)\neq \emptyset}  \frac{d(z,y)(d(z,y)-1)}{ (1-t)^2}\,  {\nu_t}\!^{x,y}(z)\\
&=\sum_{k=0}^{d}
\sum_{z\in [x,y], d(x,z)=k} -\log K\big(z,\V_{_\leftarrow}(z)\big) \1_{\V_{_\leftarrow}(z)\neq \emptyset} \,r(x,z,z,y)  \,\frac{k(k-1)}{t^2}\,{\rho_t^d}(k)\\
&\qquad+\sum_{k=0}^{d}
\sum_{z\in [x,y], d(x,z)=k}-\log K\big(z,\V_{_\rightarrow}(z)\big)\1_{\V_{_\rightarrow}(z)\neq \emptyset} \,r(x,z,z,y)  \,  \frac{(d-k)(d-k-1)}{ (1-t)^2}\,  {\rho_t^d}(k)
\end{align*} 
A   lower bound on $C_t(x,y)$ as a function of $d=d(x,y)$ can be obtained as follows.
%from the definition of $K$. After easy computations, one gets 
%\[c_t(x,y)\geq 2(1-K) d(x,y)(d(x,y)-1).\]
%In order to get a second lower bound involving constant $\overline{r}$, one  first rewrites   $c_t(x,y)$,  applies Cauchy-Schwarz inequality, and then uses the identity $\inf_{u>0,v>0, u+v \leq r^{-1}}\left\{ \frac{a^2}u +\frac{b^2}{v}\right\}= r(a+b)^2, \, a,b\geq 0$, with the definition \eqref{defr_2} of  constant $\overline{r}$, namely setting $d=d(x,y)$,
\begin{align*}
&C_t(x,y)\\
&\geq -\log K(z,S_2(z))\Big(\frac{d(d-1)}{t^2}\,\rho_t^d(d)+\frac{(d-1)(d-2)}{t^2}\,\rho_t^d(d-1)\\&\qquad\qquad\qquad\qquad\qquad\qquad\qquad\qquad\qquad+\frac{d(d-1)}{(1-t)^2}\,\rho_t^d(0)+\frac{(d-1)(d-2)}{(1-t)^2}\,\rho_t^d(1)\Big)\\
&+\1_{d\geq 4} \sum_{k=0}^{d} \sum_{z\in [x,y], d(x,z)=k} 
\Big(-\log K\big(z,\V_{_\leftarrow}(z)\big) \frac{k(k-1)}{t^2}  -\log K\big(z,\V_{_\rightarrow}(z)\big) \frac{(d-k)(d-k-1)}{ (1-t)^2}\Big)\\
&\qquad\qquad\qquad\qquad\qquad\qquad\qquad\qquad\qquad\qquad\qquad\qquad\qquad\qquad\qquad\, r(x,z,z,y) \,{\rho_t^d}(k)\\
&\geq \overline{r} d(d-1)\Big(t^{d-2}+(1-t)^{d-2}+(d-2) t^{d-3}(1-t)+(d-2)(1-t)^{d-3}t\Big)\\
&\qquad\qquad\qquad+ \overline{r} \1_{d\geq 4} %\inf_{z\in [x,y],2\leq d(x,z)\leq d-2} \Big[\big[-\log K\big(z,\V_{_\leftarrow}(z)\big)\big]^{-1}+\big[-\log K\big(z,\V_{_\rightarrow}(z)\big)\big]^{-1}\Big]^{-1} \\
\sum_{k=2}^{d-2} 
\left( \frac{\sqrt{k(k-1)}}{t} + \frac{\sqrt{(d-k)(d-k-1)}}{ 1-t}\right)^2{\rho_t^d}(k)
 \end{align*}
 where  for the last inequalities we use the fact that $\sum_{z\in[x,y],d(x,z)=k} r(x,z,z,y)=1$ and the inequality
 \[aA+bA=(a^{-1}+b^{-1})^{-1}\Big(\frac A\alpha+\frac B\beta\Big)\geq (a^{-1}+b^{-1})^{-1}\Big(\sqrt A+\sqrt B\Big)^2,\]
 with $a=-\log K\big(z,\V_{_\leftarrow}(z)>0$, $b=-\log K\big(z,\V_{_\rightarrow}(z)>0$, $A=\frac{k(k-1)}{t^2}$, $B=\frac{(d-k)(d-k-1)}{ (1-t)^2}$, $\alpha=\frac{a^{-1}}{a^{-1}+b^{-1}}$, $\beta=\frac{b^{-1}}{a^{-1}+b^{-1}}$, so that $\alpha+\beta=1$ and according to the definition of the constant $\overline{r}$, $(a^{-1}+b^{-1})^{-1}\geq \overline{r}$
 (since for $z\in [x,y],2\leq d(x,z)\leq d-2$, $\V_{_\leftarrow}(z)\neq \emptyset$ and $\V_{_\rightarrow}(z)\neq \emptyset$).
 
 Observing that  \begin{align*}\sum_{k=2}^{d-2}k(k-1){\rho_t^d}(k)&=\sum_{k=0}^{d}k(k-1){\rho_t^d}(k)-d(d-1)\big[t^d+(d-2)t^{d-1}(1-t)\big]\\&=d(d-1)\big[t^2-t^d-(d-2)t^{d-1}(1-t)\big] ,
\end{align*}
and according to the definition \eqref{defvt} of $u_t(d)$, one gets 
\begin{align*}
&d(d-1)\Big(t^{d-2}+(1-t)^{d-2}+(d-2) t^{d-3}(1-t)+(d-2)(1-t)^{d-3}t\Big)\\
&\qquad\qquad+  \1_{d\geq 4} %\inf_{z\in [x,y],2\leq d(x,z)\leq d-2} \Big[\big[-\log K\big(z,\V_{_\leftarrow}(z)\big)\big]^{-1}+\big[-\log K\big(z,\V_{_\rightarrow}(z)\big)\big]^{-1}\Big]^{-1} \\
\sum_{k=2}^{d-2} 
\left( \frac{\sqrt{k(k-1)}}{t} + \frac{\sqrt{(d-k)(d-k-1)}}{ 1-t}\right)^2{\rho_t^d}(k)=4u_t(d),
%v_s(d)&=\frac{d(d-1)}2\Big[\1_{d=2}+\frac32\,\1_{d=3}\Big]\\
%&\qquad\qquad+\frac{\1_{d\geq 4}}2 \Big[d(d-1) +
%\sum_{k=2}^{d-2} 
%\sqrt{k(k-1)(d-k)(d-k-1)}\frac{\rho_t^d(k)}{t(1-t)}\Big].\\
%&\geq \frac{d(d-1)}2.
\end{align*}
and therefore 
$C_t(x,y)\geq 4\overline{r} u_t(d)$. Then Theorem \ref{thmsam21} ensures that the displacement convexity property holds with 
\begin{equation*}
C_t(\widehat \pi)=4\overline{r}\iint \overline{c}_t\big(d(x,y)\big)\,d\widehat \pi(x,y),
\end{equation*}
with $\overline{c}_t(d):=\int_0^1 u_s(d)\, q_t(s)\,ds $.
%\end{align*}
%We know that $r\geq 2\overline{r}$ and the following estimate is  proved in \cite{Sam21}, for any integer $d\geq 1$
%\[\int_0^1 v_s(d(x,y))\, q_t(s)\,ds \geq d(d-2\log d-2).\]
%It follows that 
%\[C_t(\widehat \pi)\geq 4\overline{r} \int c_2(d(x,y)) \,d\widehat \pi (x,y),\]
%where $c_2(d)=\max\left[d(d-2\log d-2, \frac{d(d-1)}2\right] $, which ends the proof of Theorem \ref{thmprincbis}.
\end{proof}


%\begin{proof}[Proof of Theorem \ref{thmRicci flat}]
%Let $z\in \X$ arbitrary, let $z^{\prime \prime}\in S_{2}(z)$ and let %suppose without loss of generality that %$S_{1}(z)=\{\sigma_{i}(z)\}_{i=1}^{n}$. Let $\mathcal{N}_{z^{\prime %\prime}}:=\{(i,j): \sigma_{i}\sigma_{j}(z)=z^{\prime \prime }\}$. Let us %remark that if $(i,j),(h,l)\in \mathcal{N}_{z^{\prime \prime}}$ and if %$(i,j)\neq (k,l)$ then $\{i,j\}\cap \{k,l\}=\emptyset$. The set %$\mathcal{N}_{z^{\prime \prime}}$ can be decomposed as follows 
%\begin{equation*}
 %   \mathcal{N}_{z^{\prime \prime}}=\mathcal{N}_{z^{\prime \prime}}^{1}\cup \mathcal{N}_{z^{\prime \prime }}^{2}
%\end{equation*}

%where $\mathcal{N}_{z^{\prime \prime}}^{1}:=\{(i,i): %\sigma_{i}^{2}(z)=z^{\prime \prime} \} $ and
%$\mathcal{N}_{z^{\prime \prime}}^{2}:=\{(i,j), i\neq j : %\sigma_{i}\sigma_{j}(z)=z^{\prime \prime}\}$.
%Let $n_{1}(z^{\prime \prime}):=|\mathcal{N}_{z^{\prime \prime}}^{1}|$ %and 
%$n_{2}(z^{\prime \prime}):=\frac{| \mathcal{N}_{z^{\prime %\prime}}^{2}|}{2}$. Since $\mathcal{N}_{z^{\prime \prime }}^{2}$ and %$\mathcal{N}_{z^{\prime \prime}}^{2}$ are disjoint sets, %$|\mathcal{N}_{z^{\prime \prime}}|=n_{1}(z^{\prime \prime})+ %n_{2}(z^{\prime \prime})$. Using the arithmetic-geometric inequality %twice,
%\begin{align*}
%K(z,S_{2}(z))&=\sup_{\alpha} \sum_{z^{\prime \prime}\in %S_{2}(z)}\Big(n_{1}(z^{\prime \prime})+2n_{2}(z^{\prime \prime})\Big) %\Big( \prod_{\{i,j\},(i,j)\in \mathcal{N}_{z^{\prime %\prime}}^{2}}\alpha_{i}\alpha_{j}
%\prod_{k,(k,k)\in \mathcal{N}_{z^{\prime %\prime}}^{1}}\alpha_{k}\Big)^{\frac{2}{n_{1}(z^{\prime %\prime})+2n_{2}(z^{\prime \prime})}}\\
%&=\sup_{\alpha} \sum_{z^{\prime \prime}\in S_{2}(z)} \Biggl\{ %\Big(n_{1}(z^{\prime \prime})+2n_{2}(z^{\prime \prime})\Big)\\
%&\Bigg(\Big(\prod_{\{i,j\},(i,j)\in \mathcal{N}_{z^{\prime %\prime}}^{2}}\alpha_{i}\alpha_{j}\Big)^{\frac{1}{n^{2}(z^{\prime %\prime})}}\Bigg)^{\frac{2n_{2}(z^{\prime \prime})}{n_{1}(z^{\prime %\prime})+2n_{2}(z^{\prime \prime})}}
%\Bigg(\Big(\prod_{k,(k,k)\in \mathcal{N}_{z^{\prime %\prime}}^{1}}\alpha_{k}^{2}\Big)^{\frac{1}{n_{1}(z^{\prime %\prime})}}\Big)\Bigg)^{\frac{n_{1}(z^{\prime \prime})}{n_{1}(z^{\prime %\prime})+2n_{2}(z^{\prime \prime})}}\Biggr\}\\
%&\leq \sup_{\alpha}\sum_{z^{\prime \prime}\in %S_{2}(z)}\Bigg(2n_{2}(z^{\prime \prime})  \prod_{\{i,j\},(i,j)\in %\mathcal{N}_{z^{\prime \prime}}^{2}}\Big(\alpha_{i}\alpha_{j}\Big)^{\frac{1}{n_{2}(z^{\prime %\prime})}}+ n_{1}(z^{\prime \prime})\Big(\prod_{k,(k,k)\in %\mathcal{N}_{z^{\prime %\prime}}^{1}}\alpha_{k}^{2}\Big)^{\frac{1}{n_{1}(z^{\prime %\prime})}}\Big) \Bigg)\\
%&\leq \sum_{z^{\prime \prime}\in S_{2}(z)}\Bigg( \sum_{\{i,j\},(i,j)\in %\mathcal{N}_{z^{\prime %\prime}}^{2}}2\alpha_{i}\alpha_{j}+\sum_{k,(k,k)\in %\mathcal{N}_{z^{\prime \prime}}^{1}}\alpha_{k}^{2} \Bigg)\\
%&\leq \sum_{(i,j)\subset [n]}2\alpha_{i}\alpha_{j}+\sum_{k\in %[n]}\alpha_{k}^{2}=\Big(\sum_{i\in [n]}\alpha_{i}\Big)^{2}\\
%&\leq 1 .
%\end{align*}
%If the moves $\{\sigma_{i}\}_{i=1}^{n}$ have the additional property of %reflexivity that is $\sigma_{i}^{2}(z)=z$ for all $i\in [n]$ then the %term $\sum_{k,(k,k)\in \mathcal{N}_{z^{\prime %\prime}}^{1}}\alpha_{k}^{2}$ is zero and by the Cauchy-Schwarz %ineqaulity $K(z,S_{2}(z))\leq \frac{1}{n}$ and hence the conclusion %follows.

%\end{proof}

\begin{proof}[Proof of Theorem \ref{Thmstructure}] We start with the proof of inequality \eqref{rr_3}. Let $z\in \X$. According to \eqref{defKtilde} and \eqref{defr_3}
\begin{align}\label{vent}
   \widetilde r_2(z)&=\inf_{W\subset S_2(z)}\left\{1-\widetilde K_L(z,W)\right\}\nonumber\\
   &=\inf_{W\subset S_2(z)}
   \inf_\beta  \Biggl\{ \left(\sum_{\sigma\in \Sc_{]z,W[}} \sqrt{\beta(\sigma)}\right)^2 -\sum_{\ttz\in W} L^2(z,\ttz) \prod_{\sigma\in \Sc_{]z,\ttz[}}\left(\frac{\beta(\sigma)}{\big(L(z,\sigma(z))\big)^2}\right)^{\frac{L(z,\sigma(z))L(\sigma(z),\ttz)}{L^2(z,\ttz)}}\nonumber\Biggr\}\\
   %&\qquad\qquad\qquad\qquad\qquad\qquad\qquad\qquad\, \Bigg|\,{\beta}=(\beta(\sigma))_{\sigma\in S_{]z,W[}}\in \mathbb{R}_{+}^{S_{]z,W[}},\sum_{\sigma\in S_{]z,W[}} \beta(\sigma)=1 \Biggr\}\nonumber\\
   &=\inf_{W\subset S_2(z)}
   \inf_\beta  \Biggl\{ \left(\sum_{\sigma\in \Sc_{]z,W[}} \sqrt{\beta(\sigma)}\right)^2 -\sum_{\ttz\in W} L^2(z,\ttz) \prod_{\sigma\in \Sc_{]z,\ttz[}}\left(\frac{\beta(\sigma)}{\big(L(z,\sigma(z))\big)^2}\right)^{\frac{L(z,\sigma(z))L(\sigma(z),\ttz)}{L^2(z,\ttz)}}\nonumber\Biggr\}\\
   %&\qquad\qquad\qquad\qquad\qquad\qquad\qquad\qquad\, \Bigg|\,{\beta}\in \mathbb{R}_{+}^{S_{]z,W[}},\sum_{\sigma\in S_{]z,W[}} \beta(\sigma)=1 \Biggr\}\nonumber\\
   &=\inf_{W\subset S_2(z)}
   \inf  \Biggl\{ \left(\sum_{\sigma\in \Sc_{]z,W[}} \sqrt{\beta(\sigma)}\right)^2 \left[1-\sum_{\ttz\in W} L^2(z,\ttz) \prod_{\sigma\in \Sc_{]z,\ttz[}}\left(\frac{\alpha(\sigma)}{L(z,\sigma(z))}\right)^{\frac{2L(z,\sigma(z))L(\sigma(z),\ttz)}{L^2(z,\ttz)}}\right]\nonumber\\
   &\qquad\qquad\qquad\qquad\qquad\qquad\qquad\qquad\, \Bigg|\,{\beta}\in \mathbb{R}_{+}^{\Sc_{]z,W[}},\sum_{\sigma\in \Sc_{]z,W[}} \beta(\sigma)=1,\alpha(\sigma):=\frac{\sqrt{\beta(\sigma)}}{\sum_{\sigma\in \Sc_{]z,W[}} \sqrt{\beta(\sigma)}}  \Biggr\}
\end{align}
According to the definition of $K(z, W)$, one gets 
\begin{align*}
    \widetilde r_2(z)&\geq \inf_{W\subset S_2(z)} \left[\inf  \Biggl\{ \left(\sum_{\sigma\in \Sc_{]z,W[}} \sqrt{\beta(\sigma)}\right)^2\,\Bigg|\,{\beta}\in \mathbb{R}_{+}^{\Sc_{]z,W[}},\sum_{\sigma\in \Sc_{]z,W[}} \beta(\sigma)=1\Biggr\}\Big(1-K(z,W)\Big)\right]\\
    &=\inf_{W\subset S_2(z)}\Big(1-K(z,W)\Big) =1-K(z,S_2(z))
\end{align*}
If $r(z)\geq 0$, by the Cauchy Schwarz inequality, \eqref{vent} provides
\begin{align*}
   \widetilde r_2(z)&\leq \inf_{W\subset S_2(z)}
   \inf  \Biggl\{ \big| \Sc_{]z,W[}\big| \, \left[1-\sum_{\ttz\in W} L^2(z,\ttz) \prod_{\sigma\in \Sc_{]z,\ttz[}}\left(\frac{\alpha(\sigma)}{L(z,\sigma(z))}\right)^{\frac{2L(z,\sigma(z))L(\sigma(z),\ttz)}{L^2(z,\ttz)}}\right]\nonumber\\
   &\qquad\qquad\qquad\qquad\qquad\qquad\, \Bigg|\,{\beta}\in \mathbb{R}_{+}^{\Sc_{]z,W[}},\sum_{\sigma\in S_{]z,W[}} \beta(\sigma)=1,\alpha(\sigma):=\frac{\sqrt{\beta(\sigma)}}{\sum_{\sigma)\in \Sc_{]z,W[}} \sqrt{\beta(\sigma)}}  \Biggr\}\\
   &\leq \big|S_1(z)\big|\inf_{W\subset S_2(z)}
   \inf  \Biggl\{ 1-\sum_{\ttz\in W} L^2(z,\ttz) \prod_{\sigma\in S_{]z,\ttz[}}\left(\frac{\alpha(\sigma)}{L(z,\sigma(z))}\right)^{\frac{2L(z,\sigma(z))L(\sigma(z),\ttz)}{L^2(z,\ttz)}}\nonumber\\
   &\qquad\qquad\qquad\qquad\qquad\qquad\, \Bigg|\,{\beta}\in \mathbb{R}_{+}^{S_{]z,W[}},\sum_{\sigma\in \Sc_{]z,W[}} \beta(\sigma)=1,\alpha(\sigma):=\frac{\sqrt{\beta(\sigma)}}{\sum_{\sigma\in \Sc_{]z,W[}} \sqrt{\beta(\sigma)}}  \Biggr\}\\
   &= \big|S_1(z)\big| \inf_{W\subset S_2(z)}\Big(1-K(z,W)\Big) =\big|S_1(z)|\big(1-K(z,S_2(z))\big).
\end{align*}
This ends the proof of inequality \eqref{rr_3}.



The proof of the lower bound $\widetilde r_2$ of the $\widetilde T_2$-entropic curvature of the space is similar to the one of Theorem \ref{thmprinc} or   Theorem \ref{thmprincbis}.  
Starting again from inequality \eqref{epuise0} and setting   
\[\widetilde A_t^2(z):=\sum_{\sigma\in \Sc,\sigma(z)\in V_{_\leftarrow}(z)}\Big(
A_t(z,\sigma(z)) \,L(z,\sigma(z))\Big)^2,\]
one gets 
\begin{align}\label{epuise1}
H_t(z)&\nonumber\geq \widetilde A_t^2(z)+ \sum_{\sigma,\tau\in \Sc,\sigma\neq \tau, \sigma(z),\tau(z)\in V_{_\leftarrow}(z)} A_t(z,\sigma(z)) \,L(z,\sigma(z)) A_t(z,\tau(z)) \,L(z,\tau(z))\\
&\nonumber\qquad+\rho \bigg(\sum_{ \ttz\in \V_{_\leftarrow}(z)} L^2(z,\ttz)\prod_{\tz\in [z,\ttz] \cap S_1(z)} A_t(z,\tz)^{\frac{2L(z,\tz)L(\tz, \ttz)}{ L^2(z,\ttz)}},\overline{\mathbb A}_t(z)\bigg)\\
&= \widetilde A_t^2(z)\Big[1+\sum_{\sigma,\tau\in \Sc_{]z,\V_{_\leftarrow}(z)[},\sigma\neq \tau} \sqrt{\beta(\sigma,z)}\, \sqrt{\beta(\tau,z)}\Big]\\
\nonumber&\qquad+ \rho \bigg(\widetilde A_t^2(z) \sum_{ \ttz\in \V_{_\leftarrow}(z)} L^2(z,\ttz)\prod_{\sigma \in \Sc_{]z,\ttz[}} \bigg(\frac{\beta(\sigma,z)}{\big(L(z,\sigma(z))\big)^2}\bigg)^{\frac{L(z,\sigma(z))L(\sigma(z), \ttz)}{ L^2(z,\ttz)}},\overline{\mathbb A}_t(z)\bigg)
\end{align}
where for any $z\in\X$ and  $\sigma\in  \Sc_{]z,\V_{_\leftarrow}(z)[}$,  $\beta(\sigma,z):=\frac{\left(
A_t(z,\sigma(z)) \,L(z,\sigma(z))\right)^2}{\widetilde A_t^2(z)}.$
Using the inequality $\rho(a,b)\geq -a$, it follows that 
\begin{align*}
H_t(z)&\geq \widetilde A_t^2(z)
\Biggl[1+  \sum_{\sigma,\tau\in \Sc_{]z,\V_{_\leftarrow}(z)[},\sigma\neq \tau} \sqrt{\beta(\sigma,z)}\, \sqrt{\beta(\tau,z)} \\ 
&\qquad\qquad - \sum_{ \ttz\in \V_{_\leftarrow}(z)} L^2(z,\ttz)\prod_{\sigma \in \Sc_{]z,\ttz[}} \bigg(\frac{\beta(\sigma,z)}{\big(L(z,\sigma(z))\big)^2}\bigg)^{\frac{L(z,\sigma(z))L(\sigma(z), \ttz)}{ L^2(z,\ttz)}}\Biggr]\nonumber,
\end{align*}
and therefore,
from the definition of $\widetilde K(z,\V_{_\leftarrow}(z))$, 
\begin{equation}\label{burnout}H_t(z)\geq (1-\widetilde K(z,\V_{_\leftarrow}(z)) \widetilde A_t^2(z).
\end{equation}
According to the definition of constant $\widetilde r_2$, one has 
\begin{align}\label{coudeville}
\int H_t \, d\widehat \nu_t &\geq \widetilde r_2 \int \widetilde A_t^2(z) \, d\widehat \nu_t(z)\nonumber\\
&=\widetilde r_2\sum_{\sigma\in \Sc}\int \sum_{z\in \X}\frac{\big(a_t(z,\sigma(z),y)L(z,\sigma(z))\big)^2}{a_t(z,y)}\,\1_{z\in [y,\sigma(z)[} \,d\nu_1(y)\nonumber\\
&\geq\widetilde r_2\sum_{\sigma\in \Sc}\int \frac{\Big(\sum_{z\in \X,z\in [y,\sigma(z)[ } a_t(z,\sigma(z),y)L(z,\sigma(z))\Big)^2}{\sum_{z\in \X, z\in [y,\sigma(z)[ } a_t(z,y)}\,d\nu_1(y)
\end{align}
where the last inequality holds if $\widetilde r_2\geq 0$ by applying Cauchy-Schwarz inequality.

According to the  definition of $a_t(z,\sigma(z),y)$ given by \eqref{a_t'}, one has 
\begin{align*}
    &\sum_{z\in \X,z\in [y,\sigma(z)[ } a_t(z,\sigma(z),y)L(z,\sigma(z))\\&=\sum_{w\in \X}\sum_{z\in\X}\1_{(z,\sigma(z))\in [y,w]}\,d(y,w) r(y,z,\sigma(z),w)L(z,\sigma(z))\,\rho_t^{d(y,w)-1}(d(z,w)-1)\,\widehat \pi_{\leftarrow}(w|y)\\
    &=\sum_{w\in \X}  \sum_{k=0}^{d(w,y)-1} d(y,w)\rho_t^{d(y,w)-1}(d(y,w)-1-k) \widehat \,\pi_{\leftarrow}(w|y) \!\!\!\!\!\!\!\sum_{z\in \X, d(y,z)=k, (z,\sigma(z))\in [y,w]}\!\!\!\!\!\!\! r(y,z,\sigma(z),w)L(z,\sigma(z))
\end{align*}
From Lemma \ref{lemL} \ref{3}, we know that if $(z,\sigma(z))\in [y,w]$ then $\sigma(y)\in]y,w]$,  and $\sigma(z)\in [\sigma(y),w]$, and   Lemma \ref{lemL} \ref{5} implies 
\begin{align*}
\sum_{z\in \X, d(y,z)=k, (z,\sigma(z))\in [y,w]}& r(y,z,\sigma(z),w)L(z,\sigma(z))\\
&=\frac{\sum_{z\in \X, d(y,z)=k, (z,\sigma(z))\in [y,w]} L^{d(y,z)}(y,z)L(z,\sigma(z))L^{d(\sigma(z),w)}(\sigma(z),w)}{L^{d(y,w)}(y,w)}\\
&=\frac{L(y,\sigma(y))L^{d(y,w)-1}(\sigma(y),w)}{L^{d(y,w)}(y,w)}=r(y,\sigma(y),\sigma(y),w)
\end{align*}
It follows that
\begin{align*}
    \sum_{z\in \X,z\in [y,\sigma(z)[ } a_t(z,\sigma(z),y)L(z,\sigma(z))&=\sum_{w\in \X}\1_{\sigma(y)\in]y,w]}d(y,w) r(y,\sigma(y),\sigma(y),w)\,\widehat\pi_{_\leftarrow}(w|y)=\Pi^\sigma_\leftarrow(y).
\end{align*} 
Observing that 
\begin{equation}\label{obvious}
\sum_{z\in \X, z\in [y,\sigma(z)[ } a_t(z,y)\leq 1,
\end{equation}
inequality \eqref{coudeville} therefore provides
\[\int H_t \, d\widehat \nu_t \geq \widetilde r_2\int\sum_{\sigma\in \Sc} \Pi^\sigma_\leftarrow(y)^2d\nu_1(y).\]
We similarly prove that 
$\int K_t \, d\widehat \nu_t \geq \widetilde r_2\int\sum_{\sigma\in \Sc} \Pi^\sigma_\rightarrow(x)^2d\nu_0(x)$,
and thus we get 
\[\int 
(H_t+K_t) \, d\widehat \nu_t \geq \widetilde r_2\, \widetilde{T}_3(\widehat \pi).\]
The proof of the first part of Theorem \ref{Thmstructure}  ends by applying Theorem \ref{thmsam21}.

Let now assume that condition \eqref{restr} also holds, the second part of Theorem \ref{Thmstructure} will follows from improving the trivial bound \eqref{obvious}.
One has 
\begin{align}\label{long}
   &\sum_{z\in \X, z\in [y,\sigma(z)[ } a_t(z,y)=1-\sum_{z\in \X, z\not\in [y,\sigma(z)[ } a_t(z,y)\nonumber\\
   &\leq 1-\sum_{x\in\X}\1_{\sigma(y)\in]y,x]}\sum_{z\in [y,x], z\not\in [y,\sigma(z)[ } r(y,z,z,y)\rho_t^{d(x,y)}(d(x,z)) \,\widehat \pi_{_\leftarrow}(x|y)\\
   \nonumber &=1-\sum_{x\in\X}\frac{\1_{\sigma(y)\in]y,x]}}{L^{d(x,y)}(x,y)} \sum_{k=0}^{d(x,y)}\rho_t^{d(x,y)}(k)\left(\sum_{z\in [y,x],d(y,z)=d(x,y)-k, z\not\in [y,\sigma(z)[ }\quad
   \sum_{\gamma\in G(y,x), z\in \gamma} L(\gamma)\right) \widehat \pi_{_\leftarrow}(x|y)
\end{align}
Given $\sigma\in \Sc$ the set of geodesics from $y$ to $x$ contains the set of geodesics using the move $\sigma$, more precisely, setting $d(x,y)=d$
\[G(y,x)\supset \bigcup_{\ell=0}^{d-1}G_{\sigma,\ell}(y,x),\quad \mbox{with}\quad G_{\sigma,\ell}(y,x):=\big\{\gamma=(z_0,\ldots,z_d)\in G(y,x)\,\big|\,z_{\ell+1}=\sigma(z_l)\big\}.\]
Observe that according to Lemma \ref{lemL} \ref{5}, if $\sigma(y)\in ]y,x]$ then 
for any $\ell\in\{0,\ldots, d-1\}$
\[\sum_{\gamma\in G_{\sigma,\ell}(y,x)}L(\gamma)=L(y,\sigma(y))L^{d(\sigma(y),x)}(\sigma(y),x).\]
According to assumption \eqref{restr}, for $\ell\neq \ell'$, $G_{\sigma,\ell}(y,x)$ and $G_{\sigma,\ell'}(y,x)$ are disjoints sets, and therefore
\begin{align*}
\sum_{z\in [y,x],d(y,z)=d(x,y)-k, z\not\in [y,\sigma(z)[ }&\quad
   \sum_{\gamma\in G(y,x), z\in \gamma} L(\gamma)
   \\&\geq \sum_{\ell=0}^{d-1}\quad\sum_{z\in [y,x],d(y,z)=d(x,y)-k, z\not\in [y,\sigma(z)[ }\quad
   \sum_{\gamma\in G_{\sigma,\ell}(y,x), z\in \gamma}  L(\gamma) \\
   &=\sum_{\ell=0}^{d-1} \quad
   \sum_{\gamma=(z_0,\ldots,z_d)\in G_{\sigma,\ell}(y,x)} \1_{z_{d-k}\not\in [y,\sigma(z_{d-k})[ } L(\gamma) 
\end{align*}
Assume that $\sigma(y)\in]y,x]$ and let $\gamma=(z_0,\ldots,z_d)\in G_{\sigma,\ell}(y,x)$ with $z_{d-k}\not\in [y,\sigma(z_{d-k})[$. Observe first that $k\neq d$ (otherwise $y=z_0\not\in [y,\sigma(z_0)[=[y,\sigma(y)[$ which is impossible). If $0\leq \ell<d-k$ then $\gamma=(z_0,\ldots,z_\ell,z_{\ell+1},\ldots,z_{d-k},\ldots,z_d)$ with  $z_{\ell+1}=\sigma(z_\ell)$. It follows that necessarily $z_{d-k}\not\in [y,\sigma(z_{d-k})[$, since otherwise $(z_0,\ldots,z_\ell,z_{\ell+1},z_{d-k},\sigma(z_{d-k}))$ is a geodesic from $z_0$ to $\sigma(z_{d-k})$ that uses the move $\sigma$ twice. It follows that 
\[\sum_{\gamma=(z_0,\ldots,z_d)\in G_{\sigma,\ell}(y,x)} \1_{z_{d-k}\not\in [y,\sigma(z_{d-k})[ } L(\gamma)=\sum_{\gamma=(z_0,\ldots,z_d)\in G_{\sigma,\ell}(y,x)}  L(\gamma) = L(y,\sigma(y))L^{d(\sigma(y),x)}(\sigma(y),x).\]
Assume now that  $ d-k\leq\ell\leq d-1$, then   $\gamma=(z_0,\ldots,z_{d-k},\ldots, z_\ell,z_{\ell+1},\ldots,z_d)$ with  $z_{\ell+1}=\sigma(z_\ell)$. According to Lemma \ref{lemL} \ref{1}, $(z_0,\ldots,z_{d-k},\sigma(z_{d-k}),\ldots, \sigma(z_\ell),z_{\ell+2},\ldots,z_d)$ is also a geodesic in $G(y,x)$ and therefore  $z_{d-k}\in [y,\sigma(z_{d-k})[$.
As a consequence \[\sum_{\gamma=(z_0,\ldots,z_d)\in G_{\sigma,\ell}(y,x)} \1_{z_{d-k}\not\in [y,\sigma(z_{d-k})[ } L(\gamma)=0.\]
Finally, if $\sigma(y)\in]y,x]$, one gets for any fixed $k\in\{0,\ldots,d-1\}$,
\begin{align*}
\sum_{z\in [y,x],d(y,z)=d(x,y)-k, z\not\in [y,\sigma(z)[ }\quad
   \sum_{\gamma\in G(y,x), z\in \gamma} L(\gamma)
%\sum_{\ell=0}^{d-1} \quad
%   \sum_{\gamma=(z_0,\ldots,z_d)\in G_{\sigma,\ell}(y,x)} \1_{z_{d-k}\not\in [y,\sigma(z_{d-k})[ } L(\gamma) 
&\geq\sum_{d=0}^{d-k-1}L(y,\sigma(y))L^{d(\sigma(y),x)}(\sigma(y),x)\\
   &=(d-k)\, L(y,\sigma(y))L^{d(\sigma(y),x)}(\sigma(y),x).
   \end{align*}
Observe that if assumption \eqref{restr} is not fulfilled, using the fact that $G(y,x)\supset G_{\sigma,\ell}(y,x)$, one identically gets 
\[\sum_{z\in [y,x],d(y,z)=d(x,y)-k, z\not\in [y,\sigma(z)[ }\quad
   \sum_{\gamma\in G(y,x), z\in \gamma} L(\gamma)\geq L(y,\sigma(y))L^{d(\sigma(y),x)}(\sigma(y),x).\]
As a consequence, if assumption \eqref{restr} is satisfied,  \eqref{long} provides 
\begin{align*}
\sum_{z\in \X, z\in [y,\sigma(z)[ } a_t(z,y)&\leq 1-\sum_{x\in\X}{\1_{\sigma(y)\in]y,x]}}r(y,\sigma(y),\sigma(y),x) \sum_{k=0}^{d(x,y)-1}\rho_t^{d(x,y)}(k)\big(d(x,y)-k\big) \widehat \pi_{_\leftarrow}(x|y)\\
&=1-(1-t)\Pi^\sigma_\leftarrow(y), 
\end{align*}
and if assumption \eqref{restr} is not fulfilled,  \eqref{long} implies 
\begin{align*}
\sum_{z\in \X, z\in [y,\sigma(z)[ } a_t(z,y)&\leq 1-\sum_{x\in\X}\1_{\sigma(y)\in]y,x]} r(y,\sigma(y),\sigma(y),x) \sum_{k=0}^{d(x,y)-1}\rho_t^{d(x,y)}(k)\, \widehat \pi_{_\leftarrow}(x|y)\\
&=1-\sum_{x\in\X}\1_{\sigma(y)\in]y,x]} r(y,\sigma(y),\sigma(y),x) \big(1-t^{d(x,y)}\big)\, \widehat \pi_{_\leftarrow}(x|y)\\
&\leq 1-(1-t) \sum_{x\in\X}\1_{\sigma(y)\in]y,x]} r(y,\sigma(y),\sigma(y),x) \, \widehat \pi_{_\leftarrow}(x|y)\\
&\leq1-{(1-t)}\frac{\Pi^\sigma_\leftarrow(y)}{\textnormal{Diam}(\mathcal{X})},
\end{align*}
Setting $D=1$ if assumption \eqref{restr} holds and $D=\textnormal{Diam}(\mathcal{X})$ otherwise, inequality \eqref{coudeville} then provides
\[\int H_t \, d\widehat \nu_t \geq \widetilde r_2\int\sum_{\sigma\in \Sc} \frac{\Pi^\sigma_\leftarrow(y)^2}{1-(1-t)\frac{\Pi^\sigma_\leftarrow(y)}D}d\nu_1(y)= \xi_\leftarrow''(t),\]
with
\[ \xi_\leftarrow(t):=\frac{\widetilde r_2 D^2}2 \int \sum_{\sigma\in \Sc} h\Big((1-t) \,\frac{\Pi^\sigma_\leftarrow(y)}D\Big) \,d\nu_1(y).\]
One identically  proves that 
\[\int K_t \, d\widehat \nu_t \geq  \xi_\rightarrow''(t),\]
with
\[ \xi_\rightarrow(t):=\frac{\widetilde r_2 D^2}2 \int \sum_{\sigma\in \Sc} h\Big(t \,\frac{\Pi^\sigma_\rightarrow(x)}D\Big) \,d\nu_0(x).\]
The proof of the second part of Theorem \ref{Thmstructure} ends by applying Theorem \ref{thmsam21}.
\end{proof}


\subsection{Proofs of Theorem \ref{BM}, Theorem \ref{thmstructure}, Theorem \ref{Logsob}, Theorem \ref{Logsobbis} and Theorem \ref{ttens}}

\begin{proof}[Proof of Theorem \ref{BM}]
Let $x,y$ be distinct vertices in $\mathcal{X}$. By definition of entropic curvature we have 
\begin{equation}\label{bmyers}
\HR(\widehat \nu_t|m)\leq (1-t) \HR(\nu_0|m)+t \HR(\nu_1|m)- \kappa\, \frac{t(1-t)}{2}\,T_2(\widehat \pi) \hspace{0.1cm} ,
\end{equation}
for any $\nu_{0},\nu_{1}\in \Pc(\X)$ .
Let $\nu_{0}=\delta_{x}$ and $\nu_{1}=\delta_{y}$ then \eqref{bmyers} becomes 
\begin{align*}
d(x,y)(d(x,y)-1)&\leq \frac{-2}{\kappa t(1-t)}\HR(\widehat \nu_t|m) +\frac2{\kappa t} \log \frac{1}{m(x)} +\frac2{\kappa(1-t)} \log \frac{1}{m(y)}\\
&\leq \frac{2}{\kappa t(1-t)}\left(-\HR(\widehat \nu_t|m) + \log\frac1{\inf_{x\in \X}m(x)}\right)
\end{align*}
Furthermore by Jensen inequality we have
\begin{equation*}
-\HR(\widehat \nu_t|m)\leq \log m(\supp(\widehat \nu_t))\leq \log \Big(|\supp(\widehat \nu_t)| \sup_{x\in\X} m(x)\Big)\hspace{0.1cm},
\end{equation*}
and since  $|supp(\widehat \nu_t)|\leq \big(\textnormal{Deg}_{\max}\big)^{d(x,y)}$ one finally gets 
\begin{equation*}
d(x,y)\leq \frac{2}{\kappa t(1-t)}\log \Big(\textnormal{Deg}_{\max} \frac{\sup_{x\in\X} m(x)}{\inf_{x\in \X}m(x)}\Big) +1 <\infty \hspace{0.1cm} .
\end{equation*}
Choosing $t=1/2$ and maximizing over $x$ and $y$ ends the proof of Theorem \ref{BM}.
\end{proof}



\begin{proof}[Proof of Theorem \ref{thmstructure}]
According to Theorem \ref{thmprinc}, in order to prove that a structured graph $(\X,d,m_0,L_0)$  with associated finite set of moves $\Sc$ has non negative entropic curvature, it suffices to show that for any $z\in \X$, $K_0(z,S_2(z))\leq 1$.
Let $z\in \X$ be a fixed vertex, and for any $\ttz\in S_2(z)$ let  
\[ U(\ttz):=\big\{(\tau,\sigma)\in \Sc \,\big|\,\tau(\sigma(z))=\ttz\}.\]
Each couple $(\tau,\sigma)$   can be associated to a single geodesic $(z,\sigma(z),\tau(\sigma(z)))$ from $z$ to $\ttz$. Obviously for  $w''\in S_{2}(z)$ with $w''\neq \ttz$, the sets $U(\ttz)$ and $U(w'')$ are disjoints.

According to the definition of structured graphs, if $(\tau,\sigma)\in U(\ttz)$ then $(\psi(\sigma),\tau)\in U(\ttz)$ where $\psi:\Sc_z^{\cdot\rightarrow \tau}\to \Sc_z^{\tau\rightarrow \cdot}$ is a fixed one to one map. As a consequence,  given $(\sigma_2,\sigma_1)\in U(\ttz)$, one may construct by induction a sequence  $(\sigma_{k+1},\sigma_k)$, $k\in \N^*$, of elements in $U(\ttz)$ defined by $\sigma_{k+1}=\psi_{k-1}(\sigma_k)$ for all $k\geq 2$ with $\psi_{k-1}:\Sc_z^{\cdot\rightarrow \sigma_{k-1}}\to \Sc_z^{\sigma_{k-1}\rightarrow \cdot}$. 
Let us define 
\[\overline{(\sigma_2,\sigma_1)}:=\big\{(\sigma_{k+1},\sigma_k)\,\big|\, k\geq 1\big\}.\]
Since $\Sc$ is finite, there exists $k\geq 2$ and $j\leq k$ such that $\sigma_{k+1}=\sigma_j$.
Let 
\[\ell:=\min\big\{k\geq 1\,\big|\,\exists j\in\{1,\ldots,k\}, \sigma_{k+1}=\sigma_j\big\}.\]
 The maps $\sigma_1,\sigma_2,\ldots, \sigma_\ell$ all differs. Let $j\in [\ell]$ such that $\sigma_j=\sigma_{\ell+1}$. Let us prove that $j=1$. If $j\geq 2$ then $\sigma_{\ell+1}(\sigma_\ell(z))=\ttz=\sigma_j(\sigma_{j-1}(z))=\sigma_{\ell+1}(\sigma_{j-1}(z))$.  Lemma \ref{lemL} \ref{6} implies $\sigma_\ell=\sigma_{j-1}$ which contradicts the definition of $\ell$. 
It follows that $j=1$, i. e.  $\sigma_{\ell+1}=\sigma_1$. As a consequence,  one has $\sigma_{\ell+2}=\psi_\ell(\sigma_{\ell+1})=\psi_\ell(\sigma_{1})$ with $\sigma_{\ell+2} (\sigma_{1}(z))=\sigma_{\ell+2} (\sigma_{\ell+1}(z))=\ttz=\sigma_2(\sigma_1(z))$. Therefore  $\sigma_{\ell+2}=\sigma_2$ and $\sigma_2=\psi_\ell(\sigma_1)$. By induction it follows that 
\[\overline{(\sigma_2,\sigma_1)}:=\big\{(\sigma_{2},\sigma_1),(\sigma_{3},\sigma_2), \ldots,(\sigma_{\ell+1},\sigma_\ell)\big\}.\]
Then one easily checks that for any $(\sigma_{k+1},\sigma_k)\in \overline{(\sigma_2,\sigma_1)}$, one has $\overline{(\sigma_{k+1},\sigma_k)}= \overline{(\sigma_2,\sigma_1)}$. It follows that the set 
\[{\mathcal C}(\ttz)=\big\{\overline{(\tau,\sigma)}\,\big|\, (\tau,\sigma)\in U(\ttz)\big\},\]
is a partition of $U(\ttz)$.

For $c=\overline{(\sigma_2,\sigma_1)}$ as above,  one denotes by
\[s(c):=\{\sigma_1(z),\ldots,\sigma_\ell(z)\}\subset ]z,\ttz[.\]
Observe that $c$ and $s(c)$ have same number of elements. 
We claim that all sets $s(c)$ are pairwise disjoints. Indeed, note that if $c'\in {\mathcal C}(\ttz)$ with $s(c)\cap s(c')\neq \emptyset$, then there exist $\sigma,\tau,\tau'\in \Sc$ such that $\sigma(z),\tau(z)\in s(c)$, $\sigma(z),\tau'(z)\in s(c')$ and $\tau(\sigma(z))=\ttz=\tau'(\sigma(z))$. It follows that  $\tau=\tau'$,  $(\tau,\sigma)\in c\cap c'$ and therefore $c=c'$, $s(c)=s(c')$. 
Since ${\mathcal C}(\ttz)$ is a partition of $U(\ttz)$, we finally get that the collection of sets  
$\{s(c)\,|\,c\in {\mathcal C}(\ttz)\}$ is also a partition of $]z,\ttz[$.

Let $\alpha:S_1(z)\to \R^+$ such that $\sum_{\tz\in S_1(z)} \alpha(\tz)=1$.  Given $\ttz\in S_2(z)$, applying the arithmetic-geometric mean inequality gives
\begin{align*}
    \big|]z,\ttz[\big| \Big(\prod_{\tz\in  ]z,\ttz[} {\alpha(\tz)}\Big)^{\frac{2}{|]z,\ttz[|}}&=\Big(\sum_{c\in {\mathcal C}(\ttz)}|s(c)|\Big)\prod_{c\in {\mathcal C}(\ttz)} \Big(\prod_{\tz\in s(c)} \alpha(z')\Big)^{\frac{2}{\sum_{c\in {\mathcal C}(\ttz)}|s(c)|}}\\
    &\leq \sum_{c\in {\mathcal C}(\ttz)} |s(c)| \Big(\prod_{\tz\in s(c)} \alpha(z')\Big)^{\frac{2}{|s(c)|}}
\end{align*}
Since $|s(c)|=|c|$, observing that 
\[\Big(\prod_{\tz\in s(c)} \alpha(z')\Big)^2= \prod_{(\tau,\sigma)\in c} \alpha(\tau(z))\alpha(\sigma(z)),\]
and applying again the 
arithmetic-geometric mean inequality, one gets 
\[\big|]z,\ttz[\big| \Big(\prod_{\tz\in  ]z,\ttz[} {\alpha(\tz)}\Big)^{\frac{2}{|]z,\ttz[|}} \leq \sum_{c\in {\mathcal C}(\ttz)} \sum_{(\tau,\sigma)\in c} \alpha(\tau(z))\alpha(\sigma(z)).\]

Given $c\in {\mathcal C}(\ttz)$, either $|c|=1=|s(c)|$ and there exists $\tau\in \Sc$ such that $c=\big\{(\tau(z),\tau(z))\}$ and $d(z,\tau(\tau(z)))=2$, either $|c|\geq 2$ and for any $(\tau,\sigma)\in c$, $\tau(z)\neq \sigma(z)$.  
As a consequence, setting 
\[s_1(\ttz):=\bigcup_{c\in {\mathcal C}(\ttz), |c|=1} s(c),\]
one has 
\[\sum_{c\in {\mathcal C}(\ttz)} \sum_{(\tau,\sigma)\in c} \alpha(\tau(z))\alpha(\sigma(z))\leq \sum_{\tz\in s_1(\ttz)} \alpha(\tz)^2  +\sum_{c\in {\mathcal C}(\ttz),|c|\geq 2} \sum_{(\tau,\sigma)\in c} \alpha(\tau(z))\alpha(\sigma(z)).\] 
and therefore 
\begin{align*}
    \sum_{\ttz\in S_2(z)} \big|]z,\ttz[\big| \Big(\prod_{\tz\in  ]z,\ttz[} {\alpha(\tz)}\Big)^{\frac{2}{|]z,\ttz[|}} &\leq \sum_{\ttz\in S_2(z)}\,\, \sum_{\tz\in s_1(\ttz)} \alpha(\tz)^2  \\
    &\qquad +\sum_{\ttz\in S_2(z)}\,\,\sum_{c\in {\mathcal C}(\ttz),|c|\geq 2} \sum_{(\tau,\sigma)\in c} \alpha(\tau(z))\alpha(\sigma(z))\\
    &\leq \sum_{\tz \in S_1(z)} \alpha(\tz)^2 + \sum_{((\tz,w')\in S_1(z),w'\neq \tz} \alpha(\tz)\alpha(w')\\
    &=\Big(\sum_{\tz \in S_1(z)} \alpha(\tz)\Big)^2=1,
\end{align*}
where the last inequality is a consequence of the fact that for $\ttz\neq w''$, one has  $U(\ttz)\cap U(w'')=\emptyset$ and also  $s_1(\ttz)\cap s_1(w'')=\emptyset$. Then according to the definition of $K_0(z,S_2(z))$, one has $K_0(z,S_2(z))\leq 1$.

If for any $\sigma\in \Sc$, $d(z,\sigma(\sigma(z)))\leq 1$, then for any $\ttz\in S_2(z)$, $s_1(\ttz)=\emptyset$.
Applying Cauchy Schwarz inequality, the above estimates provide
\begin{align*}
\sum_{\ttz\in S_2(z)} \big|]z,\ttz[\big| \Big(\prod_{\tz\in  ]z,\ttz[} {\alpha(\tz)}\Big)^{\frac{2}{|]z,\ttz[|}} &\leq \sum_{((\tz,w')\in S_1(z),w'\neq \tz} \alpha(\tz)\alpha(w')\\
&= \Big(\sum_{\tz \in S_1(z)} \alpha(\tz)\Big)^2-\sum_{\tz \in S_1(z)} \alpha(\tz)^2\\
&\leq 1-\frac1{|S_1(z)|},
\end{align*}
and therefore $K_0(z,S_2(z))\leq 1-1/|S_1(z)|$. The last statement of Theorem \ref{thmstructure} then follows from Theorem \ref{thmprinc}.
\end{proof}

\begin{proof}[Proofs of Theorem \ref{Logsob} and Theorem \ref{Logsobbis}]
% Since the space   $(\X,d,m,L)$  satisfies the $(G)$-conditions, with positive ${\widetilde T}$-entropic curvature $\widetilde{\kappa}$,  we know from Bonnet-Myers Theorem \ref{BM} that $\X$ is finite. 
Let $\nu_0$ and $\nu_1$ be  probability measures with same convex bounded support $\Cc\subset \X$ and respective densities $f_0$ and $f_1$ with respect to the measure $m$. Modified logarithmic Sobolev inequalities will follow from the convexity property 
\begin{equation}\label{versHWI}
 \HR(\nu_0|m)\leq - \frac{ \HR(\widehat \nu_t|m)-\HR(\nu_0|m)}{t} + \HR(\nu_1|m)- \frac12 (1-t)C_t(\widehat\pi), \qquad t\in(0,1),
\end{equation}
as $t$ goes to zero, with $C_t(\widehat\pi)={\widetilde{\kappa}}  \,\widetilde{T}(\widehat \pi)$ in Theorem \ref{Logsob},  and $C_t(\widehat\pi)={\widetilde{\kappa}_3}  \,\widetilde{T}_3(\widehat \pi)$ or $C_t(\widehat\pi)={\widetilde{\kappa}_3}  \,\widetilde{C}_t(\widehat \pi)$ in Theorem \ref{Logsobbis}.   

Let us start with the proof of Theorem \ref{Logsob}. Observing  that for any $x,y\in \X$ and $z\in[x,y]$,
\begin{multline*}\partial_t \nu_t^{x,y}(z)_{|t=0}=\frac{L^{d(x,z)}(x,z)L^{d(z,y)}(z,y)}{L^{d(x,y)}(x,y)}\,\binom{d(x,y)}{d(x,z)} \left( \1_{[x,y]}(z)\1_{z\sim x}-d(x,y)\1_{x=z}\right)\\
=\sum_{x'\in S_1(x)\cap[x,y]}  d(x,y) 
\frac{L(x,x')L^{d(x',y)}(x',y)}{L^{d(x,y)}(x,y) }\left(\delta_{x'}(z)-\delta_x(z)\right)
\end{multline*}
and since for any $t\in[0,1]$ the finite convex subset $\Cc$ is the support of $\widehat \nu_t$, one gets 
\begin{align}\label{tempete}
&\partial_t \HR(\widehat \nu_t|m)_{|t=0}\nonumber=\sum_{z\in \Cc} \partial_t \widehat\nu_t(z)_{|t=0} \log f_0(z)\\
&=\nonumber\sum_{z\in \Cc}  \sum_{x,y\in \Cc}  \sum_{x'\in S_1(x)\cap[x,y]} d(x,y) \frac{L(x,x')L^{d(x',y)}(x',y)}{L^{d(x,y)}(x,y) }\left(\delta_{x'}(z)-\delta_x(z)\right) \log f_0(z) \,\widehat \pi(x,y)\\
&= \sum_{x,y\in \Cc}\sum_{x'\in S_1(x)\cap[x,y]}\left(\log f_0(x')-\log f_0(x)\right) d(x,y) \frac{L(x,x')L^{d(x',y)}(x',y)}{L^{d(x,y)}(x,y) } \widehat \pi(x,y)\\
&\nonumber\geq -\sum_{x\in\Cc} \max_{x',x'\sim x}[\log f_0(x)-\log f_0(x')]_+ \Big(\sum_{y\in \Cc} d(x,y) \widehat \pi_\rightarrow(y|x) \Big) \,\nu_0(x)\\
&\nonumber\geq - \frac1{2 \widetilde{\kappa}} \sum_{x\in\Cc} \max_{x',x'\sim x}[\log f_0(x)-\log f_0(x')]_+^2 \nu_0(x) - \frac{\widetilde{\kappa}}2 \sum_{x\in\Cc} \Big(\sum_{y\in \Cc} d(x,y) \widehat \pi_\rightarrow(y|x) \Big)^2 \,\nu_0(x),
\end{align}
where for the last inequality, one uses the inequality $ab\leq a^2/2+b^2/2$, $a,b\in \R$. 
Therefore, from the definition of $\widetilde{T}(\widehat \pi)$, \eqref{versHWI} implies as $t$ goes to zero 
\begin{multline*}
\HR(\nu_0|m)\leq  \frac1{2 \widetilde{\kappa}} \sum_{x\in\Cc} \max_{x',x'\sim x}[\log f_0(x)-\log f_0(x')]_+^2 \nu_0(x) +\HR(\nu_1|m) \\- \frac{\widetilde{\kappa}}2 \sum_{y\in\Cc} \Big(\sum_{x\in \Cc} d(x,y) \widehat \pi_\leftarrow(x|y) \Big)^2 \,\nu_1(y) 
\end{multline*}
By choosing $\nu_1=\mu_\Cc:=\frac{\1_\Cc m}{m(\Cc)}$ and $f_0:=\frac{f\1_\Cc}{m(f\1_\Cc)}$, it gives 
%\[
%\HR(\nu_0|\mu_\Cc)\leq  \frac1{2 \widetilde{\kappa}} \sum_{x\in\Cc} \max_{x',x'\sim x}[\log f_0(x)-\log f_0(x')]_+^2 \nu_0(x)  - \frac{\widetilde{\kappa}}2 \sum_{y\in\Cc} \left(\sum_{x\in \Cc} d(x,y) \widehat \pi_\leftarrow(x|y) \right)^2 \,\mu_\Cc(y) 
%.\]
%Choosing $f_0:=\frac{f\1_\Cc}{m(f\1_\Cc)}$, one gets  
\begin{align*}
    {\rm Ent}_{\mu_\Cc}(f)&\leq  \frac1{2 \widetilde{\kappa}} \sum_{x\in\Cc} \max_{x',x'\sim x}[\log f(x)-\log f(x')]_+^2 f(x) \,\mu_\Cc(x)  \\
    &\qquad\qquad\qquad\qquad-  \mu_\Cc(f) \frac{\widetilde{\kappa}}2 \sum_{y\in\Cc} \Big(\sum_{x\in \Cc} d(x,y) \widehat \pi_\leftarrow(x|y) \Big)^2 \,\mu_\Cc(y) \\
    &\leq \frac1{2 \widetilde{\kappa}} \int \max_{x',x'\sim x}[\log f(x)-\log f(x')]_+^2 f(x) \,d\mu_\Cc(x)
\end{align*}
Applying this inequality with  $\Cc=\Cc_n$ where $(\Cc_n)$ is an increasing sequence of convex subsets with $\bigcup_{n}\Cc_n=\X$, the  monotone convergence theorem provides the expected  modified logarithmic Sobolev inequality \eqref{logsob} for $\mu$ since $f$ is bounded and $m(\X)<+\infty$.
%\[{\rm Ent}_{\mu}(f)\leq  \frac1{2 \widetilde{\kappa}} \sum_{x\in\Cc} \max_{x',x'\sim x}[\log f(x)-\log f(x')]_+^2 f(x) \,\mu(x)  -  \mu(f) \frac{\widetilde{\kappa}}2 \sum_{y\in\Cc} \left(\sum_{x\in \Cc} d(x,y) \widehat \pi_\leftarrow(x|y) \right)^2 \,\mu(y) 
%.\] 
%This inequality  provides the modified logarithmic Sobolev inequality \eqref{logsob} by applying it to the measure $\nu_0=\nu:=f\mu/\mu(f)$ and by using the homogeneity property of entropy ${\rm Ent}_\mu(\lambda f)=\lambda   {\rm Ent}_\mu(f)$, $\lambda>0$.

Let $g:\X\to \R$ be a bounded function such that $\mu(g)=0$. As usual, applying \eqref{logsob} to the function $f=1+\varepsilon g $ where $\varepsilon$ is a sufficiently small parameter so that $f>0$, a  Taylor expansion as $\varepsilon$ goes to zero gives
\[\frac{\varepsilon^2}{2}\mu(g^2) +\circ(\varepsilon^2)\leq \frac{\varepsilon^2}{2\widetilde{\kappa}} 
\int \max_{x', x'\sim x}\left[g(x)-g(x')\right]_+^2 d\mu(x)+\circ(\varepsilon^2)
.\]
It provides the  Poincaré inequality of Theorem \ref{Logsob} as $\varepsilon$ goes to zero.

The proof of Theorem \ref{Logsobbis} is similar. Starting again from equality \eqref{tempete}, one has
\begin{align*}
\partial_t \HR(\widehat \nu_t|m)_{|t=0}
&=\sum_{x,y\in \Cc} \sum_{\sigma\in \Sc,  \sigma(x)\in ]x,y]} \partial_\sigma \log f_0(x) \, \frac{L(x,\sigma(x))L^{d(\sigma(x),y)}(\sigma(x),y)}{L^{d(x,y)}(x,y) } \widehat \pi(x,y)\\
&\geq -\sum_{x\in \Cc} \sum_{\sigma\in \Sc} \left[\partial_\sigma \log f_0(x)\right]_-\Pi^\sigma_\rightarrow(x)\,\nu_0(x) \\
&\geq - \frac1{2\widetilde  \kappa_2 } \sum_{x\in \Cc} \sum_{\sigma\in \Sc} 
[\partial_\sigma( \log f_0)(x)]_-^2 \nu_0(x) - \frac{\widetilde  \kappa_2}2 \sum_{x\in\Cc} \sum_{\sigma\in \Sc}\left(\Pi^\sigma_\rightarrow(x)\right)^2\,\nu_0(x)
\end{align*}
The above inequality together with  \eqref{versHWI}  imply as $t$ goes to zero 
\begin{equation*}
\HR(\nu_0|m)\leq \frac1{2\widetilde  \kappa_2 } \int \sum_{\sigma\in \Sc} [\partial_\sigma( \log f)]_-^2 d\nu_0 +\HR(\nu_1|m) - \frac{\widetilde  \kappa_2}2 \int \sum_{\sigma\in \Sc} \left(\Pi^\sigma_\leftarrow(y)\right)^2 d\nu_1(y).
\end{equation*}
Then the end of  the  proof of the first part of Theorem \ref{Logsobbis}  is similar to the one of Theorem \ref{Logsob} with approximation's arguments. For the proof of its second part, one uses the inequality
\begin{align*}
\partial_t \HR(\widehat \nu_t|m)_{|t=0}
&\geq -\sum_{x\in \Cc} \sum_{\sigma\in \Sc} \left[\partial_\sigma \log f_0(x)\right]_-\Pi^\sigma_\rightarrow(x)\,\nu_0(x) \\
&\geq -  \sum_{x\in\Cc} \sum_{\sigma\in \Sc} \frac{\widetilde \kappa_2 D^2}2 \, h^*\left(\frac{2}{D\widetilde\kappa_2} [\partial_\sigma( \log f)(x)]_-\right)  \nu_0(x) -  \sum_{x\in\Cc} \sum_{\sigma\in \Sc}\frac{\widetilde{\kappa}_2 D^2}2h\left(\frac{\Pi^\sigma_\rightarrow(x)}D\right)\,\nu_0(x), 
\end{align*}
where $h^*(v):=\sup_{0\leq u<1}\big\{uv-h(u)\big\}=2\left(e^{-v/2}+v/2-1\right), v\geq 0$.
Since 
\[\lim_{t\to 0} \widetilde C_t^D(\widehat \pi)= \int \sum_{\sigma\in \Sc} D^2 h\left(\frac{\Pi^\sigma_\rightarrow(x)}{D}\right) d\nu_0(x)+\int \sum_{\sigma\in \Sc} D^2 h_{1}\left(\frac{\Pi^\sigma_\leftarrow(y)}D\right) d\nu_1(y), \]
with $h_1(u):=uh'(u)-h(u)=2(-u-\log(1-u))$, $u\in[0,1)$, 
as before inequality  \eqref{versHWI}  implies as $t$ goes to zero 
\begin{equation*}
\HR(\nu_0|m)\leq  \int \sum_{\sigma\in \Sc} \frac{\widetilde \kappa_2 D^2}2 \, h^*\left(\frac{2}{D\widetilde\kappa_2} [\partial_\sigma( \log f)]_-\right)  d\nu_0 +\HR(\nu_1|m) - \int \sum_{\sigma\in \Sc} D^2 h_{1}\left(\frac{\Pi^\sigma_\leftarrow(y)}D\right) d\nu_1(y).
\end{equation*}
The proof of the second part of Theorem \ref{Logsobbis}  ends as the one of Theorem \ref{Logsob}.
\end{proof}



% \subsubsection{Proof of Theorem \ref{ttens}}

\begin{proof}[Proof of Theorem \ref{ttens}]
Let $(\mathcal{X}_{i},d_{\X_{i}},\mu_{i},L_{i})$, %$(\mathcal{Y},d_{\mathcal{Y}},L_{2},\mu_{2})$ 
be graph spaces and let $\big(\prod_{i=1}^{n} \mathcal{X}_{i},d_{\X ^{\square{n}}},\mu_{i}^{\otimes n} ,\oplus_{i=1}^{n} L_{i}\big)$ be the product graph space. Let $n\geq 2$ and let $z\in \prod_{i=1}^{n} \mathcal{X}_{i}$. Let us compute $K_{\oplus_{i=1}^{n} L_{i}}\big(z,S_{2}(z)\big) $.
According to the structure of Cartesian product of graphs, one has
\begin{multline*}
    K_{\oplus_{i=1}^{n} L_{i}} \big(z,S_{2}(z)\big)=\sup_\alpha \Big\{  \sum_{i=1}^{n}\sum_{z_{i}^{\prime \prime} \in S_{2}(z_{i})} 
L_{i}^{2}(z_{i},z_{i}^{\prime \prime})
\Big(\prod_{z_{i}^{\prime} \in ]z_{i},z_{i}^{\prime \prime}[}  \frac{\alpha_i(z_{i}^{\prime})}{L_{i}(z_{i},z_{i}^{\prime})}\Big)^{\frac{2L_{i}(z_{i},z_{i}^{\prime})L_{i}(z_{i}^{\prime},z_{i}^{\prime \prime})}{L_{i}^{2}(z_{i},z_{i}^{\prime \prime}) }} 
\\+ 2 \sum_{\{i,j\}\subset[n]} \sum_{z_{i}^\prime,z_{i}^\prime\sim z_{i}}\sum_{ z_{j}^\prime,z_{j}^\prime\sim z_{j}}\alpha_i(z_{i}^\prime)\alpha_j(z_{j}^\prime) \Big\},
\end{multline*}
where the supremum runs over all non negative vector $\alpha$ with coordinates $\alpha_i(z_i')$, $i\in[n]$, $z_i'\sim z_i$, such that $\sum_{i\in [n]}\sum_{z_{i}^\prime,z_{i}^\prime\sim z_{i}} \alpha_i(z_i')=1$. Setting $\alpha_i=\sum_{z_{i}^\prime,z_{i}^\prime\sim z_{i}} \alpha_i(z_i')$, and according to the definition of $K_{L_{i}}\big(z_{i},S_{2}(z_{i})\big)$, it follows that
\begin{align*}
K_{\oplus_{i=1}^{n} L_{i}} \big(z,S_{2}(z)\big)&=\sup_\alpha 
\Big\{  \sum_{i=1}^{n} \alpha_i^2  K_{L_{i}}\big(z_{i},S_{2}(z_{i})+ 2 \sum_{\{i,j\}\subset[n]} \alpha_i\alpha_j
\Big\}\\
&=1-\inf_\alpha\Big\{\sum_{i=1}^{n} \alpha_i^2 \big(1- K_{L_{i}}\big(z_{i},S_{2}(z_{i})\big)\Big\}\\
&\leq 1-\inf_\alpha\Big\{\big(1-\max_{i\in[n]}K_{L_{i}}\big(z_{i},S_{2}(z_{i})\big)\sum_{i=1}^{n} \alpha_i^2 \Big\},
\end{align*}
where the supremum is over all $\alpha=(\alpha_1,\ldots,\alpha_n)\in \R_+^n$ with $\sum_{i=1}^{n} \alpha_i=1$. The expected result follows from the sign of $\big(1-\max_{i\in[n]}K_{L_{i}}\big(z_{i},S_{2}(z_{i})\big)$ and since $\inf_\alpha\Big\{\sum_{i=1}^{n} \alpha_i^2 \Big\}=1/n$ and  $\sup_\alpha\Big\{\sum_{i=1}^{n} \alpha_i^2 \Big\}=1$.
This concludes the first part of Theorem \ref{ttens}.


Let us now study the constant $\widetilde r_2(z)$. By easy induction arguments, it suffices to get the result for $n=2$. Let $W\subset S_2(z)$. The set $]z,W[$ is the disjoint union of the two sets 
\[]z,W[^{1}:=\{z^{\prime} \in ]z,W[\hspace{0.06cm}| \hspace{0.06cm} z^{\prime}=(z_{1}^{\prime},z_{2}), z_{1}\sim z_{1}^{\prime} \}\quad \mbox{and} 
\quad ]z,W[^{2}:=\{ z^{\prime}\in ]z,W[ \hspace{0.06cm}| \hspace{0.06cm} 
z^{\prime}=(z_{1},z_{2}^{\prime}), z_{2}\sim z_{2}^{\prime} \}.\] 
Let $V_1:=\big\{z_1'\,\big|\,(z_1'.z_2)\in ]z,W[^{1}\big\}$ and $V_2:=\big\{z_2'\,\big|\,(z_1.z_2')\in ]z,W[^{2}\big\}$.
Similarly, the set $W$ is the disjoint union of the three following  sets
\[\overline{W}_1:=\big\{(z_1'',z_2)\,|\, d_{\X_1}(z_1,z_1'')=2\big \},\quad
 \overline{W}_2:=\big\{(z_1,z_2'')\,|\, d_{\X_2}(z_2,z_2'')=2\big \},\]
 \[\mbox{and}\quad \overline{W}_3:=\big\{(z_1',z_2')\,|\, z_1\sim z_1', z_2\sim z_2'\big \}.\]
 One has
 \[]z,\overline{W}_1[\cup]z,\overline{W}_2[\cup]z,\overline{W}_3[=]z,W[^{1}\cup]z,W[^{2}.\]
 If $W_1:=\big\{z_1''\,\big|\,(z_1'',z_2)\in \overline{W}_1\big\}$ and $W_2:=\big\{z_2''\,\big|\,(z_1,z_2'')\in \overline{W}_{2}\big\}$, then
one has $]z_1,W_1[\subset V_1$ and $]z_2,W_2[\subset V_2$.
With the above notations and according to the definition \eqref{defKtilde} of $\widetilde K_L(z,W)$,  the structure of product of graphs gives 
\begin{align*}
   1-\widetilde K_{L_1\oplus L_2}(z,W)
   &=
   \inf_\beta  \Biggl\{ \bigg(\sum_{z_1'\in V_1} \sqrt{\beta_1(z^{\prime}_1)} +\sum_{z_2' \in V_2} \sqrt{\beta_2(z^{\prime}_2)}\bigg)^{2} -2\sum_{{(z_1',z_2') \in \overline{W}_3}} \sqrt{\beta_1(z^{\prime}_1)} \sqrt{\beta_2(z^{\prime}_2)}\\
   &\qquad\qquad-\sum_{z_1''\in W_1} L^{2}_1(z_1,z''_1) \prod_{z_1',\in ]z,z_1''[}\bigg(\frac{\beta_1(z^{\prime}_1)}{\big(L_1(z_1,z^{\prime}_1)\big)^{2}}\bigg)^{\frac{L_1(z_1,z^{\prime}_1)L_1(z_1^{\prime},z''_1)}{L^{2}_1(z_1,z''_1)}}\nonumber\\
   &\qquad\qquad-\sum_{z_2''\in W_2} L^{2}_2(z_2,z''_2) \prod_{z_2',\in ]z_2,z_2''[}\bigg(\frac{\beta_2(z^{\prime}_2)}{\big(L_2(z,z^{\prime}_2)\big)^{2}}\bigg)^{\frac{L_2(z_2,z^{\prime}_2)L_2(z_2^{\prime},z''_2)}{L^{2}_2(z_2,z''_2)}}\nonumber\Biggr\}, 
\end{align*}
where the supremum runs over all non negative vector $\beta$ with coordinates $\beta_i(z_i')$, $i\in[2]$, $z_i'\in V_i$, such that $\sum_{i\in [n]}\sum_{z_{i}^\prime\in V_i} \beta_i(z_i')=1$.
 Since $\overline W_3\subset V_1\times V_2$, it follows  
\begin{align*}
   1-\widetilde K_{L_1\oplus L_2}(z,W)&\geq
   \inf_\beta  \Biggl\{\left(\sum_{z_1' \in V_1} \sqrt{\beta_1(z^{\prime}_1)}\right)^{2}-\sum_{z_1''\in W_1} L^{2}_1(z_1,z''_1) \prod_{z_1',\in ]z_1,z_1''[}\bigg(\frac{\beta_1(z^{\prime}_1)}{\big(L_1(z_1,z^{\prime}_1)\big)^{2}}\bigg)^{\frac{L_1(z_1,z^{\prime}_1)L_1(z_1^{\prime},z''_1)}{L^{2}_1(z_1,z''_1)}}\\
   &+\bigg(\sum_{z_2' \in V_2} \sqrt{\beta_2(z^{\prime}_2)}\bigg)^{2}-\sum_{z_2''\in W_2} L^{2}_2(z_2,z''_2) \prod_{z_2',\in ]z_2,z_2''[}\bigg(\frac{\beta_2(z^{\prime}_2)}{\big(L_2(z_2,z^{\prime}_2)\big)^{2}}\bigg)^{\frac{L_2(z_2,z^{\prime}_2)L_2(z_2^{\prime},z''_2)}{L^{2}_2(z_2,z''_2)}}\Biggl\}
\end{align*}
For $i\in[2]$, $]z_i,W_i[\subset V_i$. Therefore, setting $\beta_i=\sum_{z_{i}^\prime\in V_i} \beta_i(z_i')$, from the definition of $\widetilde K_{L_i}(z_i,W_i)$ one gets 
\begin{align*}
   1-\widetilde K_{L_1\oplus L_2}(z,W)&\geq
   \inf_{\beta_1+\beta_2=1}  \Big\{\beta_1 \big(1-\widetilde K_{L_1}(z_1,W_1)\big) +\beta_2 \big(1-\widetilde K_{L_2}(z_2,W_2)\big)\Big\}\\ 
   &=\min\big(1-\widetilde K_{L_1}(z_1,W_1), 1-\widetilde K_{L_2}(z_2,W_2)\big) \geq \min\big(\widetilde r_1(z_1), \widetilde r_2(z_2)\big). 
\end{align*}
 The second part of  Theorem \ref{ttens} then follows optimizing over all $W\subset S_2(z)$. 
\end{proof}


\subsubsection{Proofs of Lemma \ref{Dvcube}, Lemma \ref{lemcubeK_v} and Lemma \ref{KLvZ}}


\begin{proof}[Proof of Lemma \ref{Dvcube} ]

We want to bound from below  the quantity defined by \eqref{mulh}, $D_t v(x,y)$, for any $x,y\in \{0,1\}^n$ with $d=d(x,y)\geq 2$. The identity \eqref{Dvijcube} provides 
\begin{align*} 
&D_t v(x,y)\\&=2\sum_{ z\in[x,y]}\;\; \sum_{\{i,j\}\subset[n],(z,\sigma_i\sigma_j(z))\in[x,y]}(2z_i-1)(2z_j-1)\,\partial_{ij}^2 v(z_{\overline{ij}}) \,r(x,z,\sigma_i\sigma_j(z),y)\rho_t^{d-2}(d(x,z)).
\end{align*}
If $\sigma_i\sigma_j(z))\in[x,y]$ then $(2z_i-1)=x_i-y_i$ and $(2z_j-1)=x_j-y_j$. As a consequence, if $\partial_{ij}^2 v(z_{\overline{ij}}):=v_{ij}$ does not depend on $z_{\overline{ij}}$, one has 
\begin{align*}
D_t v(x,y)&=2\sum_{\{i,j\}\subset[n]} (x_i-y_i)(x_j-y_j)v_{ij}\sum_{ z\in[x,y],(z,\sigma_i\sigma_j(z))\in[x,y]}r(x,z,\sigma_i\sigma_j(z),y)\rho_t^{d-2}(d(x,z)) \\
&=\frac2{d(d-1)}  \sum_{\{i,j\}\subset[n]} (x_i-y_i)(x_j-y_j)v_{ij},
\end{align*}
which ends the proof of the first part of Lemma \ref{Dvcube}.
In any case, when $\partial_{ij}^2 v(z_{\overline{ij}})$ depends on $z_{\overline{ij}}$, we also have 
\begin{equation*}
D_t v(x,y)
=\sum_{k=0}^{d-2} \ell_{t}^{x,y}(k)\frac{k!(d-2-k)!}{d!} \,\rho_t^{d-2}(k)= \sum_{k=1}^{d-1} \ell_{t}^{x,y}(k-1)\frac{(k-1)!(d-k-1)!}{d!} \,\rho_t^{d-2}(k-1),
\end{equation*}
with 
\begin{align*} \ell_{t}^{x,y}(k)&:=2 \sum_{z\in [x,y], d(x,z)=k}\sum_{ \{i,j\}\subset[n],(z,\sigma_i\sigma_j(z))\in[x,y]}(2z_i-1)(2z_j-1)\,\partial_{ij}^2 v(z_{\overline{ij}})\\
&=2 \sum_{z\in [x,y], d(x,z)=k} \sum_{ \{i,j\}\subset[n]} (2z_i-1)\1_{z_i\neq y_i}(2z_j-1)\1_{z_j\neq y_j} \partial_{ij}^2 v(z_{\overline{ij}}) ,
\end{align*}
or by symmetry,
\[\ell_{t}^{x,y}(k)=2 \sum_{z\in [x,y], d(x,z)=k+2}\sum_{ \{i,j\}\subset[n]}(2z_i-1)\1_{z_i\neq x_i}(2z_j-1)\1_{z_j\neq x_j} \partial_{ij}^2 v(z_{\overline{ij}}).\]

It follows that  for $k\in\{1,\ldots , d-1\}$,
\begin{align*}
\ell_{t}^{x,y}(k-1)&\geq \sum_{z\in [x,y], d(x,z)=k-1} \lambda_{\min}(Hv(z))  \sum_{i\in [n]} \1_{z_i\neq y_i}(2z_i-1)^2\\
&\geq\lambda_{\min}^\infty(Hv)\, (d-k+1) \,\frac{d!}{(k-1)!(d-k+1)!} \,
\end{align*}
and by symmetry
\[\ell_{t}^{x,y}(k-1)\geq \lambda_{\min}^\infty(Hv)\, (k+1) \,\frac{d!}{(k+1)!(d-k-1)!},\]
Since $Hv(z)$ has  off-diagonal entries,  $\lambda_{\min}^\infty(Hv)\leq \lambda_{\min}(Hv(z))\leq 0$ and therefore we get   
\begin{align*}
D_t v(x,y)&\geq \lambda_{\min}^\infty(Hv) \sum_{k=1}^{d-1} \min\{1/k,1/(d-k)\}\,\rho_t^{d-2}(k-1)\\
&=\lambda_{\min}^\infty(Hv)  \sum_{k=1}^{d-1} \min\{d-k,k\}\,\frac{\rho_t^{d}(k)}{t(1-t)d(d-1)}\\
&\geq \frac{\lambda_{\min}^\infty(Hv)}{2(d-1)}\, \frac{1-\rho_t^{d}(0)-\rho_t^{d}(n)}{t(1-t)}=\frac{\lambda_{\min}^\infty(Hv)}{2(d-1)}\,
\gamma_t(d). 
\end{align*}
Then inequality \eqref{subtile} provides the expected result, 
\[\int_0^1 D_s v(x,y)\,q_t(s)\,ds\geq \frac{\lambda_{\min}^\infty(Hv)}{2(d-1)}\int_0^1 \gamma_s(d)\,ds\geq  \frac{\lambda_{\min}^\infty(Hv)}{d-1}\,{\sum_{k=1}^{d-1}\frac1k}\geq \lambda_{\min}^\infty(Hv).\]
\end{proof}
\begin{proof}[Proof of Lemma \ref{lemcubeK_v}] Let $z\in \{0,1\}^n$ and $W\subset S_2(z)$.
Using the definition of the subset of indices $A^1$ associated to the set $W$ and given by
\eqref{hard}, the quantity $K^v(s,W)$ given by \eqref{defKvtilde} can be written as
\[K^v(s,W)=\sup_{\alpha}\Big\{ 2\sum_{\{i,j\}\subset A^1}  e^{-(2z_i-1)(2z_j-1)\,\partial_{ij}^2v(z_{\overline{ij}})/2} \alpha_i\alpha_j\Big\},\]
where the infimum runs over all $\alpha=(\alpha_i)_{i\in A^1}$ with positive coordinates $\alpha_i$ satisfying $\sum_{i\in A^1}\alpha_i=1$.
 The upper bound on $K^v(s,W)$ is a consequence of the inequality $e^s\leq 1+s+|s|k(|s|)$, $s\in \R$. It provides 
\begin{align*}
&2\sum_{\{i,j\}\subset A^1}  e^{-(2z_i-1)(2z_j-1)\,\partial_{ij}^2v(z_{\overline{ij}})/2} \alpha_i\alpha_j\\&\leq2\sum_{\{i,j\}\subset A^1} \Big(1-\frac12 (2z_i-1)(2z_j-1)\,\partial_{ij}^2v(z_{\overline{ij}})\Big) \alpha_i\alpha_j  +k\big(|Hv|_{\max,\infty}/2\big)\sum_{\{i,j\}\subset A^1} \big|\partial_{ij}^2v(z_{\overline{ij}})\big| \alpha_i\alpha_j
\\
&\leq 1- \Big(\sum_{i\in A^1} \alpha_i^2\Big) \Big[1+\frac{\lambda_{\min}(Hv(z))}2-k\big(|Hv|_{\max,\infty}/2\big) \frac{\lambda_{\max}(|Hv|(z))}2\Big]&\\
&\leq 1- \Big(\sum_{i\in A^1} \alpha_i^2\Big) r(v),  
\end{align*}
which ends the proof of the first part of Lemma \ref{lemcubeK_v}.

For the second part of Lemma \ref{lemcubeK_v}. The proof is similar for the upper bound of $\widetilde  K^v(z)=\sup_{W\in S_2(z)}\widetilde  K^v(z,W)$ with according to \eqref{defKvtilde}
\[\widetilde  K^v(s,W)=\sup_{\beta}\Big\{ 2\sum_{\{i,j\}\subset A^1}  \Big(e^{-(2z_i-1)(2z_j-1)\,\partial_{ij}^2v(z_{\overline{ij}})/2}-1\Big) \sqrt{\beta_i}\sqrt{\beta_j}\Big\},\]
where the infimum runs over all $\beta=(\beta_i)_{i\in A^1}$ with positive coordinates $\beta_i$ satisfying $\sum_{i\in A^1}\beta_i=1$.
As above it follows that 
\begin{align*}
&2\sum_{\{i,j\}\subset A^1} \Big(e^{-(2z_i-1)(2z_j-1)\,\partial_{ij}^2v(z_{\overline{ij}})/2}-1\Big) \sqrt{\beta_i}\sqrt{\beta_j} 
\\&\leq-\sum_{\{i,j\}\subset A^1}  (2z_i-1)(2z_j-1)\,\partial_{ij}^2v(z_{\overline{ij}})\sqrt{\beta_i}\sqrt{\beta_j}   +k\big(|Hv|_{\max,\infty}/2\big)\sum_{\{i,j\}\subset A^1} \big|\partial_{ij}^2v(z_{\overline{ij}})\big| \sqrt{\beta_i}\sqrt{\beta_j} \\
&\leq -\frac{\lambda_{\min}(Hv(z))}2+k\big(|Hv|_{\max,\infty}/2\big) \frac{\lambda_{\max}(|Hv|(z))}2\leq 1-r(v),
\end{align*}
which implies the expected upper bound on $\widetilde K^v=\sup_{z\in \{0,1\}^n}\widetilde K^v(z)$.
For the lower bound on $\widetilde K^v$, since for any $z\in\{0,1\}^n$,  $A^1=[n]$ for $W=S_2(z)$, the inequality $e^s-1\geq s$ gives 
 \begin{align*}
\widetilde K^v(z)&\geq \widetilde K^v(s,S_2(z))=\sup_{\beta}\Big\{ 2\sum_{\{i,j\}\subset [n]}  \Big(e^{-(2z_i-1)(2z_j-1)\,\partial_{ij}^2v(z_{\overline{ij}})/2}-1\Big) \sqrt{\beta_i}\sqrt{\beta_j}\Big\}\\
&\geq \sup_{\beta}\Big\{-2\sum_{\{i,j\}\subset [n]} (2z_i-1)(2z_j-1)\,\partial_{ij}^2v(z_{\overline{ij}})\sqrt{\beta_i}\sqrt{\beta_j}\\
&\geq -\lambda_{\min}(Hv(z)).
\end{align*}
It follows that $\widetilde K^v\geq -\lambda_{\min}^\infty(Hv)$. 
\end{proof}



\begin{proof}[Proof of Lemma \ref{KLvZ}]
We want to upper bound the quantity $K^v(z, S_2(z))$ for any $z\in \Z^n$ whose expression is given by \eqref{defKLv}. According to the structure of the lattice $\Z^n$ and from the identity \eqref{DvZn}, one has for any $z\in \Z^n$,
\begin{align*}
   &K^v(z, S_2(z)):=\sup_\alpha\Big\{ 2 \sum_{\{i,j\}\subset[n]} \Big(\alpha_{i+}\alpha_{j+}e^{-\partial_{ij}v(z)/2}+\alpha_{i-}\alpha_{j-}e^{-\partial_{ij}v(z-e_i-e_j)/2} +\alpha_{i+}\alpha_{j-}e^{\partial_{ij}v(z-e_j)/2}\\
   &\qquad\qquad\qquad\qquad\qquad\qquad+\alpha_{i-}\alpha_{j+}e^{\partial_{ij}v(z-e_i)/2}\Big) 
   %\\&\qquad\qquad\qquad\qquad\qquad\qquad\qquad\qquad\qquad\qquad 
   +\sum_{i\in[n]} \Big(\alpha_{i+}^2 e^{-\partial_{ii}v(z)/2}+ \alpha_{i-}^2 e^{-\partial_{ii}v(z-2e_i)/2}\Big)\Big\},  
\end{align*}
where the supremum runs over all vectors $\alpha$ with non-negative coordinates $\alpha_{i+},\alpha_{i-}$ satisfying $\sum_{i\in[n]} (\alpha_{i+}+\alpha_{i-})=1$.
According to the definition of the matrix $Av(z)$ in Lemma \ref{KLvZ}, one has 
\begin{align*}
   &K^v(z, S_2(z))\\&\leq \sup_\alpha\Big\{ 2 \sum_{\{i,j\}\subset[n]}  \big(\alpha_{i+}+\alpha_{i-}\big)\big(\alpha_{j+}+\alpha_{j-}\big)\big((Av(z))_{ij} +1\big)  +\sum_{i\in[n]} \big(\alpha_{i+}+ \alpha_{i-}\big)^2\big((Av(z))_{ii} +1\big)\}\\
   &\leq 1 +\lambda_{\max}\big(Av(z)\big)\sum_{i\in[n]} \big(\alpha_{i+}+ \alpha_{i-}\big)^2\\
   &\leq 1+\frac{\lambda_{\max}\big(Av(z)\big)}n,
\end{align*}
where the last inequality is a consequence of Cauchy-Schwarz inequality if $\lambda_{\max}\big(A(z)\big)\leq 0$.

We want now to upper bound  $\widetilde K^v(z, W)$ for any $W\subset S_2(z)$. According to \eqref{defKtilde} it can be expressed as follows 
\begin{align*}
\widetilde{K}^v(z,W)&=\sup  \Biggl\{  e^{-D v(z,\ttz)/2}\sum_{\ttz\in W} |]z,\ttz[| \Big(\prod_{\sigma\in \Sc_{]z,\ttz[}} {\beta(\sigma)}\Big)^{\frac1{|]z,\ttz[|}}-\sum_{(\sigma,\tau)\in \Sc_{]z,W[}^2, \sigma\neq \tau} \sqrt{\beta(\sigma)}\sqrt{\beta(\tau)}
 \\&\qquad\qquad\qquad\qquad\qquad\qquad\Bigg|\,{\beta}=(\beta(\sigma))_{\sigma\in \Sc_{]z,W[}}\in \mathbb{R}_{+}^{\Sc_{]z,W[}},\sum_{\sigma\in \Sc_{]z,W[}} \beta(\sigma)=1 \Biggr\}.
\end{align*}
Given $W\subset S_2(z)$, there exist  subsets  $I_+,I_-\subset[n]$, $J_+, J_-\subset\{(i,j)\in[n]\times [n]\,|\, i< j\}$ and $K\subset\{(i,j)\in[n]\times [n]\,|\, i\neq j\}$, such that 
\begin{multline*}
    W=\big\{z+2e_i\,\big|\,i \in I_+\big\}\cup\big\{z-2e_i\,\big|\,i \in I_-\big\}\cup \big\{z+e_i+e_j\,\big|\, (i,j)\in J_+\big\}\\
    \cup\big\{z-e_i-e_j\,\big|\, (i,j)\in J_-\big\}\cup\big\{z+e_i-e_j\,\big|\, (i,j)\in K\big\}.
\end{multline*}
Setting $L_+=I_+\cup \{i\,|\, \exists k\in [n], (i,k)\in J_+ \cup  K\mbox{ or } (k,i)\in J_+\}$ and $L_-=I_-\cup \{j\,|\, \exists k\in [n], (k,j)\in J_- \cup  K\mbox{ or } (j,k)\in J_-\}$, the identity \eqref{DvZn} gives 
\begin{align*}
\widetilde{K}^v(z,W)&=\sup_\beta 
\Big\{ 2 \sum_{(i,j)\in J_+} \sqrt{\beta_{i+}}\sqrt{\beta_{j+}}e^{-\partial_{ij}v(z)/2} +2\sum_{(i,j)\in J_-} \sqrt{\beta_{i-}}\sqrt{\beta_{j-}}e^{-\partial_{ij}v(z-e_i-e_j)/2} \\&\qquad+2 \sum_{(i,j)\in K} \sqrt{\beta_{i+}}\sqrt{\beta_{j-}}e^{\partial_{ij}v(z-e_j)/2}+ \sum_{i\in I_+} \beta_{i+} e^{-\partial_{ii}v(z)/2} +  \sum_{i\in I_-} \beta_{i-} e^{-\partial_{ii}v(z-2e_i)/2}\\
&\qquad-\sum_{(i,j)\in L_+, i\neq j} \sqrt{\beta_{i+}}\sqrt{\beta_{j+}} -\sum_{(i,j)\in L_-, i\neq j} \sqrt{\beta_{i-}}\sqrt{\beta_{j-}} -2\sum_{(i,j)\in L_+\times L_-} \sqrt{\beta_{i+}}\sqrt{\beta_{j-}}\Big\},
\end{align*}
where the supremum runs over all vectors $\beta$ with non-negative coordinates $\beta_{i+}, i\in L_+,\beta_{j-}, j\in L_-$ satisfying $\sum_{i\in L+}\beta_{i+}+\sum_{j\in L-}\beta_{j-}=1$.
The definition of the matrix $A(z)$ then provides
\begin{align*}
&\widetilde{K}^v(z,W)\leq 1+\sup_\beta 
\Big\{ 2 \sum_{\{i,j\}\subset L_+} \sqrt{\beta_{i+}}\sqrt{\beta_{j+}}((Av(z))_{ij}+1) +2\sum_{\{i,j\}\subset L_-} \sqrt{\beta_{i-}}\sqrt{\beta_{j-}}((Av(z))_{ij}+1) \\&\qquad+2 \sum_{(i,j)\in L_+\times L_-, i\neq j} \sqrt{\beta_{i+}}\sqrt{\beta_{j-}}((Av(z))_{ij}+1)+ \sum_{i\in L_+} \beta_{i+} (Av(z))_{ii} +  \sum_{i\in L_-} \beta_{i-} (Av(z))_{ii}\\
&\qquad-\sum_{(i,j)\in L_+, i\neq j} \sqrt{\beta_{i+}}\sqrt{\beta_{j+}} -\sum_{(i,j)\in L_-, i\neq j} \sqrt{\beta_{i-}}\sqrt{\beta_{j-}} -2\sum_{(i,j)\in L_+\times L_-} \sqrt{\beta_{i+}}\sqrt{\beta_{j-}}\Big\}\\
&=1+ \sup_\beta 
\Big\{ 2 \sum_{\{i,j\}\subset L_+} \sqrt{\beta_{i+}}\sqrt{\beta_{j+}}(Av(z))_{ij} +2\sum_{\{i,j\}\subset L_-} \sqrt{\beta_{i-}}\sqrt{\beta_{j-}}(Av(z))_{ij}\\&\qquad+2 \sum_{(i,j)\in L_+\times L_-,i\neq j} \sqrt{\beta_{i+}}\sqrt{\beta_{j-}}(Av(z))_{ij}+ \sum_{i\in L_+} \beta_{i+} (Av(z))_{ii} +  \sum_{i\in L_-} \beta_{i-} (Av(z))_{ii}-2\sum_{i\in L_+\cap L_-} \sqrt{\beta_{i+}}\sqrt{\beta_{i-}}\Big\}
\end{align*}
Since $-1\leq A_{ii}(z)$ it follows that 
\begin{align*}
\widetilde{K}^v(z,W)&\leq 1+\sup_\beta 
\Big\{2 \sum_{\{i,j\}\subset [n], i\neq j} \big(\sqrt{\beta_{i+}}\1_{i\in L_+}+\sqrt{\beta_{i-}}\1_{i\in L_-}\big)\big(\sqrt{\beta_{j+}}\1_{j\in L_+}+\sqrt{\beta_{j-}}\1_{j\in L_-}\big)A_{ij}(z)\\
&\qquad +\sum_{i\in [n]} \big(\sqrt{\beta_{i+}}\1_{i\in L_+}+\sqrt{\beta_{i-}}\1_{i\in L_-}\big)^2 A_{ii}(z)\Big\}\\
&\leq 1+\lambda_{\max}\big(Av(z)\big) \sum_{i\in [n]} \big(\sqrt{\beta_{i+}}\1_{i\in L_+}+\sqrt{\beta_{i-}}\1_{i\in L_-}\big)^2\\
&\leq 1+ \lambda_{\max}\big(Av(z)\big),
\end{align*}
where the last inequality holds since $\lambda_{\max}\leq 0$.
Finally we get   $\widetilde{K}^v(z)=\sup_{W\subset S_2(z)}\widetilde{K}^v(z,W)\leq 1+ \lambda_{\max}\big(Av(z)\big)$. It  ends the proof of Lemma \ref{KLvZ}.
\end{proof}


\subsection{Proofs of Proposition \ref{propgeodetic}, Proposition \ref{combfam}, Proposition \ref{propcycles} and Proposition \ref{BE} }
\begin{proof}[Proof of Proposition \ref{propgeodetic}]
Recall that $S_{r}(z)$ denotes the combinatorial sphere of raduis $r\in \N$. It was proved in \cite{PS82} that a graph $G$ of finite diameter $d$ is geodetic if and only if for every $z\in \mathcal{X}$, each vertex of $S_{r}(z)$ is adjacent to a unique vertex in $S_{r-1}(z)$ for each $0 \leq r \leq d$. Let $G=(\mathcal{X},E)$ be a geodetic graph with diameter greater or equal to 2 and $z_{0}\in \mathcal{X}$. By choosing $r=2$, we obtain that there exists $z_{0}^{\prime \prime}\in S_{2}(z_{0})$ such that
$|]z_{0},z_{0}^{\prime \prime}[|=1$. Therefore, we immediately conclude that $K\geq 1$. Moreover, for every $z\in \mathcal{X}$
\begin{equation*}
K_{0}(z,S_{2}(z))=\sup_{\alpha} \sum_{\tz\sim \ttz, \hspace{0.1cm} d(z,\ttz)=2} \alpha(\tz)^{2}= \sup_{\alpha} \sum_{\tz,\hspace{0.1cm} \tz\sim z}(\textnormal{Deg}(z^{\prime})-1)\alpha(z^{\prime})^{2}=\max_{z^{\prime},\hspace{0.1cm} z^{\prime}\sim z}\textnormal{Deg}(z^{\prime})-1 \hspace{0.1cm} ,
\end{equation*}
and the conclusion follows.
\end{proof}




  
\begin{proof}[Proof of Proposition \ref{combfam}]
Let $z\in \X$. In order to simplify the notations, one denotes $V_{i}:=]z,W_{i}[$. We first remark that if $\mathcal{W}$ is  admissible for the vertex $z$ then $K_{0}\big(z,S_{2}(z)\big)$ admits the following reformulation 
\begin{equation*}
K_{0}\big(z,S_{2}(z)\big)=\max_{\alpha}    \sum_{i=1}^{d}|V_{i}||W_{i}|\Big(\prod_{i\in V_{i}}\alpha_{i}\Big)^{\frac{2}{|V_{i}|}} \hspace{0.1cm} .
\end{equation*}
If $r(z)=-2\log K_{0}\big(z,S_{2}(z)\big) \geq 0$ then it follows that   
\begin{equation*}
\max_{\alpha}    \sum_{i=1}^{d}|V_{i}||W_{i}|\Big(\prod_{i\in V_{i}}\alpha_{i}\Big)^{\frac{2}{|V_{i}|}} \leq 1 \hspace{0.1cm} .
\end{equation*}

%\textcolor{blue}{Mal expliqué}

Choosing for some $i\in [d]$, $(\alpha_{i})_{i}$ as follows
\begin{equation*}
\alpha_{i}=
\begin{cases}
\frac{1}{|V_{i}|} \hspace{0.2cm} \text{if} \hspace{0.2cm} i\in V_{i} ,\\
0 \hspace{0.2cm} \text{otherwise},
\end{cases}
\end{equation*}
 one immediately gets 
\begin{equation*}
    1\geq K_{0}\big(z,S_{2}(z)\big) \geq \frac{|W_{i}|}{|V_{i}|} \hspace{0.2cm} .
\end{equation*}
% \textcolor{blue}{Conclusion ?}
Since, this argument holds for any $i\in [d]$, the first implication is proved.

If $|V_{i}| \geq |W_{i}|$ and $V_{i}\cap V_{j}=\emptyset $ for all $i,j\in [d]$ by the arithmetic-geometric inequality one obtains
\begin{align*}
K_{0}\big(z,S_{2}(z)\big)&= \sum_{i=1}^{d}|V_{i}||W_{i}|\Big(\prod_{i\in V_{i}}\alpha_{i}\Big)^{\frac{2}{|V_{i}|}} \\
&\leq \sum_{i=1}^{d}|V_{i}||W_{i}|\frac{ \Big(\sum_{i\in V_{i}} \alpha_{i}\Big)^{2}}{|V_{i}|^{2}}  \\
&\leq \max_{i\in [d]} \frac{|W_{i}|}{|V_{i}|}\sum_{i=1}^{d}\sum_{i\in V_{i}}\alpha_{i}=\max_{i\in [d]} \frac{|W_{i}|}{|V_{i}|} \\
&\leq 1 \hspace{0.2cm} .
\end{align*}
\end{proof}

\begin{proof}[Proof of Proposition \ref{propcycles}]
Let suppose that $g(G)<5$ then there exist $x^{\prime}\sim x$ and 
$y^{\prime}\sim y$ such that $d(x^{\prime},y^{\prime})\leq 1$. Without loss of generality, let suppose that $\text{Deg}(y)\geq \text{Deg}(x)$. Let us define the following transport $\pi$ between $m_{x}^{\alpha}$ and $m_{y}^{\alpha}$ as follows (as shown in Figure 2)
\[\pi(x,x)=\pi(y,y)=\pi(y^{\prime},x^{\prime})=\frac{\alpha}{\text{Deg}_{\max}},\qquad
\pi\Big(S_{1}(y)\setminus \{x,y^{\prime}\},S_{1}(x)\setminus  \{y,x^{\prime}\}\Big)=\frac{(\text{Deg}(x)-2)\alpha}{\text{Deg}_{\max}},\]
\[\pi(y,x)=1-\frac{\alpha \text{Deg}(y)}{\text{Deg}_{\max}}-\frac{\alpha}{\text{Deg}_{\max}},
\qquad \pi\Big(S_{1}(y)\setminus\{x,y^{\prime}\},x\Big)=\frac{(\text{Deg}(y)-\text{Deg}(x))\alpha}{\text{Deg}_{\max}} . \]
Since,
\begin{align*}
W_{1}(m_{x}^{\alpha},m_{y}^{\alpha})&=1-\kappa_{\alpha}(x,y)\\
&\leq \frac{\alpha}{\text{Deg}_{\max}}+3\frac{(\text{Deg}(x)-2)\alpha}{\text{Deg}_{\max}}+1-\frac{\alpha \text{Deg}(y)}{\text{Deg}_{\max}}-\frac{\alpha}{\text{Deg}_{\max}}+2\frac{(\text{Deg}(y)-\text{Deg}(x))\alpha}{\text{Deg}_{\max}},
\end{align*}
one obtains that $\kappa_{LLY}(x,y)\geq \frac{6-\text{Deg}(y)-\text{Deg}(x)}{\text{Deg}_{\max}}$ which is a contradiction.

\begin{figure}[!ht]
\vspace{-0,5 cm}
\centering
\includegraphics[scale=0.4]{plan.pdf}
\caption{Transport plan $\pi$ between $m_{x}^{\alpha}$ and $m_{y}^{\alpha}$}
\vspace{-1 cm}
\label{Figure1.2}
\end{figure}
\FloatBarrier
\end{proof}

\begin{proof}[Proof of Proposition \ref{BE}]
%For the sake of completeness, let us introduce the Bakry-Émery curvature. 
%The notion of Bakry-Émery curvature was first introduced by Bakry and \'Emery in \cite{BE85}. The Bakry-Émery curvature is motivated by the Bochner's identity in Riemannian Geometry and has been extensively studied in discrete spaces recently  \cite{CKL21,CLP20}.
For the proof it is necessary to recall the definition of the Bakry-Émery curvature condition. Let us briefly introduce the bilinear forms $\Gamma$ and $\Gamma_{2}$ respectively,
\begin{align*}
    2\Gamma_{2}(f,g)&:=\Delta(fg)-f\Delta g-g\Delta f,\\
    2\Gamma_{2}(f,g)&:=\Delta(\Gamma(f,g))-\Gamma(f,\Delta g)-\Gamma(g,\Delta f)
\end{align*}
where $\Delta$ is the discrete Laplace operator. As a convention $\Gamma(f):=\Gamma(f,f)$ and $\Gamma_{2}(f):=\Gamma_{2}(f,f)$.
\begin{definition}\cite[Bakry-Émery curvature condition]{YS10,Sch99}
Let $G=(\mathcal{X},E)$ be a graph. Let $\kappa_{BE}\in \mathbb{R}$ and $N\in (0,\infty]$. 
A vertex $z\in \mathcal{X}$ satisfies the Bakry-Émery curvature-dimension inequality $CD(\kappa_{BE},N)$, if for any $f:\mathcal{X}\rightarrow \mathbb{R}$
\begin{equation*}
    \Gamma_{2}(f)(z)\geq \frac{1}{N}(\Delta f(x))^{2}+\kappa_{BE}\Gamma(f)(z),
\end{equation*}
where $N$ is  a dimension parameter and $\kappa_{BE}$ is regarded as a lower Ricci bound at $z\in \X$.  
\end{definition}






Let $(\X,E)$ be a structured graph. We want to prove that any vertex $z$ satisfies the Bakry-Émery curvature-dimension inequality $CD(0,\infty)$. For the sake of simplicity let 
$\Sc:=\{\sigma_i\,|\, i\in[n]\}$. From the definition, 
$2\Gamma_{2}(f)(z):=\Delta \Gamma(f)(z)-2\Gamma(f,\Delta f)(z)$. From the same computations as in \cite{CY96,YS10,CKKLP21} we get 
\begin{align*}
\Delta \Gamma(f)(z)&=\sum_{i,\sigma_{i} z\sim z}\big( \Gamma(f)(\sigma_{i} z)-\Gamma(f)(z) \big)\\
%&=\frac{1}{2}\sum_{\sigma_{i}x\sim x}\big( \sum_{\sigma_{j}x\sim x}(f(\sigma_{j}\sigma_{i}x)-f(\sigma_{i}x))^{2}-\sum_{\sigma_{j}x\sim x} (f(\sigma_{j}x)-f(x))^{2}\big)\\
&=\frac{1}{2} \sum_{i,\sigma_{i}z\sim z}\sum_{j,\sigma_{j}z\sim z}\big(f(\sigma_{j}\sigma_{i}z)-f(\sigma_{i}z)-f(\sigma_{j}z)+f(z)\big)^{2}\\
&+\sum_{i,\sigma_{i}z\sim z}\sum_{j,\sigma_{j}z\sim x}(f(\sigma_{j}z)-f(z))(f(\sigma_{j}\sigma_{i}z)-f(\sigma_{i}z)-f(\sigma_{j}z)+f(z))
\end{align*}
where we have used the identity $A^{2}-B^{2}=(A-B)^{2}+2B(A-B)$.
On the other hand
\begin{align*}
   -2\Gamma(f,\Delta f)(x)&=-\sum_{j,\sigma_{j}z\sim z}(f(\sigma_{j}z)-f(z))(\Delta f(\sigma_{j}z)-\Delta f(z))\\
   &=-\sum_{i,\sigma_{i}z\sim z}\sum_{j,\sigma_{j}z\sim z} (f(\sigma_{j}z)-f(z))(f(\sigma_{i}\sigma_{j}z)-f(\sigma_{i}z)-f(\sigma_{j}z)+f(z))
  % &=-\sum_{\sigma_{i}x\sim x}\sum_{\sigma_{j}x\sim x} (f(\sigma_{j}x)-f(x))(f(\sigma_{j}\sigma_{i}x)-f(\sigma_{j}x)
  %-f(\sigma_{i}x)+f(x))
\end{align*}

Using our assumption that $\sigma_{i}\sigma_{j}z=\sigma_{j}\sigma_{i}z$ for all $z\in \X$ and for all $i,j\in [n]$, summing up we get
\begin{equation*}
    2\Gamma_{2}(f)(z)=\frac{1}{2}\sum_{i,\sigma_{i}z\sim z}\sum_{j,\sigma_{j}z\sim z} (f(\sigma_{j}\sigma_{i}z)-f(\sigma_{i}z)-f(\sigma_{j}z)+f(z))^{2}\geq 0 \hspace{0.1cm} .
\end{equation*}
%\textcolor{blue}{Ce n est pas ma demostration , je l ai escrite en integralite mais il ne faudra pas. Il y a une remarque de Klartag-Kozma-Salli et Tetali Discrete curvature on abelian groups qui me fait penser que la commutativite est necessaire }
\end{proof}
%\begin{proof}[Proof of Proposition \ref{structuredbe}]
%\Sc_z^{\tau\rightarrow \cdot}
%\Sc_z^{\cdot\rightarrow \tau}
%To prove the statement, following the ideas of \cite[Proposition 2.3]{Sch99}, it is enough to show that $\partial_{\sigma_{\tau}}$ for any $\tau\in \Sc$ commutes with the Laplacian operator $\Delta$. Let $[L,K]$ be the commutator defined as $LK-KL$ for any suitable operators $K$ and $L$.
%Let $z\in \X$, $\tau\in \Sc$ and for any $g\in B_{2}(z)$,
%\begin{align*}
%[\Delta,\partial_{\sigma_{\tau}}]g(z)&=\sum_{\sigma\in \Sc} \Big(g(\tau(\sigma(z)))-g(\sigma(\tau(z)))\Big)\\
%&=\sum_{\sigma\in \Sc_z^{\cdot\rightarrow \tau}}g\big(\tau(\sigma(z))\big)+\sum_{\sigma\in \Sc,z\sim \tau(\sigma(z))}g\big(\tau(\sigma(z))\big)\\ 
%&-\sum_{\sigma\in \Sc_z^{\tau\rightarrow \cdot}} g\big(\sigma(\tau(z))\big)-\sum_{\sigma\in \Sc,z\sim \sigma(\tau(z))} g\big(\sigma(\tau(z))\big)  \\
%&=\sum_{\psi(\sigma)\in \Sc_z^{\tau\rightarrow \cdot}}g\big(\psi(\sigma)(\tau(z))\big)-
%\sum_{\sigma\in \Sc_z^{\tau\rightarrow \cdot}} g\big(\sigma(\tau(z))\big)\\
%&+\sum_{\sigma\in \Sc,z\sim \tau(\sigma(z))}g\big(\tau(\sigma(z))\big)-\sum_{\sigma\in \Sc,z\sim \sigma(\tau(z))} g\big(\sigma(\tau(z))\big) \\
%&=\sum_{\sigma\in \Sc,z\sim \tau(\sigma(z))}g\big(\tau(\sigma(z))\big)-\sum_{\sigma\in \Sc,z\sim \sigma(\tau(z))} g\big(\sigma(\tau(z))\big) \hspace{0.2cm} ,
%\end{align*}
%where  we have used that $\psi:\Sc_z^{\cdot\rightarrow \tau}\to %\Sc_z^{\tau\rightarrow \cdot} $ is a one to one map.
%We claim that $\sum_{\sigma\in \Sc,z\sim %\tau(\sigma(z))}g\big(\tau(\sigma(z))\big)-\sum_{\sigma\in \Sc,z\sim %\sigma(\tau(z))} g\big(\sigma(\tau(z))\big)$ is zero. 
%For this purpose, let $\sigma^{\prime}\in \Sc$ be such that $z\sim \tau(\sigma^{\prime}(z))$  and such that $z\sim \sigma^{\prime}(\tau(z))$ , we will show that $\tau(\sigma^{\prime}(z))=\sigma^{\prime}(\tau(z))$, which would prove the first claim. We proceed by contradiction. 
%Let suppose that $\tau(\sigma^{\prime}(z))\neq \sigma^{\prime}(\tau(z))$. By definition of structured graphs and the fact that $z\sim \tau(\sigma^{\prime}(z))$ and $z\sim \sigma^{\prime}(\tau(z))$, there are two triangles : $[z,\tau(z),\sigma^{\prime}(\tau(z))]$ and $[z,\sigma^{\prime}(z),\tau(\sigma^{\prime}(z))]$.
%In what follows we will use Lemma 3$(i)$ for structured graphs. 
%Since $\big(z,\tau(\sigma^{\prime}(z))\big)\in G\big(z,\tau(\sigma^{\prime}(z))\big)$, $(\tau(z),\tau(\sigma^{\prime}(z)))\in G\big(z,\tau(\sigma^{\prime}(z))\big)$ which implies that $\tau(z)=z$. Likewise, $\sigma^{\prime}(z)=z$ .
%Thus, $\sigma^{\prime}(\tau(z))=\sigma^{\prime}(z)=z$ and %$\tau(\sigma^{\prime}(z))=\tau(z)=z$ which contradicts the assumption that %$\tau(\sigma^{\prime}(z))\neq \sigma^{\prime}(\tau(z))$. Therefore, %$[\Delta,\partial_{\sigma_{\tau}}]=0$.

%\end{proof}

%\textcolor{blue}{
%If $H(\nu_{t}| m)\leq (1-t)H(\nu_{0}| %m)+tH(\nu_{1}| %m)-\kappa_{1}\frac{t(1-t)}{2}W_{1}^{2}(\nu_{0},\nu_%{1})$ with $\kappa_{1}>0$.
%By Pinsker's inequality, %$||\nu_{t}-m||_{TV}^{2}\leq
%2H(\nu_{t}\mid m)$ thus we get
%\begin{equation*}
%    \frac{1}{2}||\nu_{t}-m||_{TV}^{2}\leq (1-t)H(\nu_{0}\mid m)+tH(\nu_{1}\mid m)-\kappa_{1}\frac{t(1-t)}{2}W_{1}^{2}(\nu_{0},\nu_{1}).
%\end{equation*}}
%\textcolor{blue}{
%Choosing $\nu_{0}=\delta_{x}$ and $\nu_{1}=\delta_{y}$ and $t=\frac{1}{2}$ we get
%\begin{equation*}
%    \frac{\kappa_{1}}{8}d(x,y)^{2}\leq \frac{1}{2}\log(\frac{1}{m(x)})+\frac{1}{2}\log(\frac{1}{m(y)})-\frac{1}{2}|| \nu_{1/2}-m||_{TV}^{2} .
%\end{equation*}}
%\textcolor{blue}{Maximizing over $x,y$ we get 
%\begin{equation*}
 %   \textnormal{Diam}(\X)\leq \sqrt{\frac{4\big(\log(\frac{1}{m(x)}+\log(\frac{1}{m(y)})-|| \nu_{1/2}-m||_{TV}^{2}\big)}{\kappa_{1}}} .
%\end{equation*}}
%\textcolor{blue}{Ou encore on pourrait trouver una majoration de $||\nu_{t}-m||_{TV}$ .}


%%%%%%%%%%%%%%%%%%%%%%%%%%%%%%%%%%%%%%%%%
%%%%%%%%%%%%%%%%%%%%%%%%%%%%%%%%%%%%%%%%%%
%\newpage
% 1
% \newpage
 
%\subsection{Related work}

%\paragraph{ \bf Bakry-{\'E}mery curvature-dimension conditions.} 

%Let us note that  similarly, recent works about the \textit{Bakry-Émery curvature-dimension conditions} developed in the context of graphs \cite{CKL21,CLP20} show that the Bakry-Émery conditions are also related to the local structure of balls of radius 2.
%Indeed, in their setting \cite{CKL21}, for a locally finite graph $G=(\mathcal{X},E)$ \textit{the curvature matrix} for a vertex $z\in \mathcal{X}$ is completely determined by the ball of radius 2 around $z$ denoted as $B_{2}(z)$. More precisely it is determined by \textit{the incomplete ball of radius 2 around $z$}, that is, the graph induced by $B_{2}(z)$ removing all edges connecting vertices within $S_{2}(z)$. Interestingly, as already stated, in our context the entropic curvature also depends only on balls of radius 2 and its value is invariant with respect to the existence of edges connecting vertices within $S_{2}(z)$.
 
 %For the curvature criterion $CD(K,\infty)$, the curvature at vertices is always negative (with four exceptions) if \textit{the punctured 2-ball}  defined in \cite{CLP20} as $B_{2}^{o}(x):=B_{2}(x)\setminus \{x\}$ has more than one connected component \cite[Theorem 6.4]{CLP20}. Let us note that this condition of local disconnectedness also applies in our setting. Indeed, let $z\in(\mathcal{X},E)$ such that $B_{2}^{o}(z)$ has more than one connected component then by choosing two neighbours of $z$ in two different connected components of $B_{2}^{o}(z)$ one obtains that there exists a unique midpoint between them (which is $z$), hence the above applies. Note that this is a sufficient but not necessary condition to have non-positive curvature in both settings. For example the Petersen graph does not satisfy the local disconnectedness property for any vertex and has negative entropic curvature as detailed in section 3.2 as well as negative curvature for $CD(K,\infty)$ \cite[Example 5.16]{CLP20}.\\



%In contrast to the Bakry-{\'E}mery criterion in graphs, in particular the condition $CD(K,\infty)$ \cite[Theorem 1.2]{KKRT16} also revisited in \cite[Corollary 3.3]{CLP20}, the upper bound on the entropic curvature for a vertex $z\in \mathcal{X}$ is not related to the triangles adjacent to it. In fact
%\begin{equation*}
%1-K(z,S_{1}(z),S_{2}(z)) \leq 1-\sup_{z''\in S_{2}(z)}\frac{\prod_{z'\in S_{1}(z)\cap[z,z'']}\Big(L(z,z')L(z',z'')\Big)^{2 \frac{L(z,z')L(z',z'')}{L^{2}(z,z'')}}}{L(z,z'')} .
%\end{equation*}
%In the particular case, where $L=L_{0}$ the above equation becomes
%\begin{equation*}
%1-K(z,S_{1}(z),S_{2}(z))\leq 1-\sup_{z''\in S_{2}(z)}\frac{1}{\mid S_{1}(z)\cap [z,z'']\mid} .
%\end{equation*}
%\begin{proof}
%For convenience, let us denote 
%\begin{equation*}
%\mathcal{L}(z,z',z''):=\frac{L(z,z')L(z',z'')}{L^{2}(z,z'')} .
%\end{equation*}

%Firstly, let us bound the following quantity, using the change of variable
%$\alpha(z')=e^{h(z')}\mu(z')$ with $h$ positive,

%\begin{align*}
%\sup_{\sum_{z'}\alpha(z')\leq 1} \sum_{z'\in S_{1}(z)\cap[z,z'']}\log \alpha(z')\mathcal{L}(z,z',z'') &\geq \sup_{h\geq 0,\int e^{h}d\mu\leq 1} \Big( \sum_{z'\in S_{1}(z)\cap[z,z'']} h(z')\frac{\mathcal{L}(z,z',z'')}{\mu(z')}\mu(z')\Big)\\
%&+ \sum_{z'\in S_{1}(z)\cap[z,z'']}\log \mu(z')\mathcal{L}(z,z',z'') .
%\end{align*}

%By the variational formula, for every positive  probability measure $\nu$ and every positive function $f:\mathcal{X}\rightarrow \mathbb{R}$,
%\begin{equation*}
%\text{Ent}_{\nu}(f)=\sup_{\int_{X}e^{g}d\nu\leq 1}\int fg d\nu . 
%\end{equation*}
%Therefore, 
%\begin{align*}
%\sup_{\sum_{z'\in S_{1}(z)\cap[z,z'']}\alpha(z')\leq 1} \sum_{z'\in S_{1}(z)\cap[z,z'']}\log \alpha(z')\mathcal{L}(z,z',z'') &\geq \text{Ent}_{\mu}\Big(\frac{\mathcal{L}(z,z',z'')}{\mu(z')}\Big)+\sum_{z'\in S_{1}(z)\cap[z,z'']}\log \mu(z') \mathcal{L}(z,z',z'') \\
%& = \sum_{z'\in S_{1}(z)\cap[z,z'']} \mathcal{L}(z,z',z'')\log \mathcal{L}(z,z',z'').
%\end{align*}

%Thus, 
%\begin{align*}
%\sup_{\sum_{\alpha(z')}\leq 1} L(z,z'')\Big(\prod_{\sum_{z'\in S_{1}(z)\cap[z,z'']}}\alpha(z')\Big)^{2\mathcal{L}(z,z',z'')}&\geq L(z,z'')\exp\Big(2 \sum_{z'\in S_{1}(z)\cap[z,z'']}(\mathcal{L}(z,z',z'')\log \mathcal{L}(z,z',z'')\Big) \\
%&=L(z,z'')\exp \Bigg(\log \prod_{z'\in S_{1}(z)\cap[z,z'']}\Big( \mathcal{L}(z,z',z'')^{2\mathcal{L}(z,z',z'')}\Big) \Bigg)\\
%&=\frac{\prod_{z'\in S_{1}(z)\cap[z,z'']}\Big(L(z,z')L(z',z'')\Big)^{2\mathcal{L}(z,z',z'')}}{L(z,z'')} \hspace{0.2cm}.
%\end{align*}

%\end{proof}

 

 
 %\paragraph{ \bf Ollivier's Ricci curvature Lin-Lu-Yau and Lin-Lu-Yau's Ricci curvature.}
 
%We obtain global relations between \textit{entropic curvature} and \textit{Ollivier's Ricci curvature} (see Appendix) for  $D$-regular graphs and we do not obtain relations between \textit{entropic curvature} and \textit{Ollivier's Ricci curvature} or \textit{Lin-Lu-Yau curvature} (see Appendix) locally.\\

%In \cite[Theorem 2.b(ii)]{HS13}, the authors show that if the Ollivier's Ricci curvature denoted by $K_{\text{Oll}}$ defined via the simple neighbour random walk  is bounded from above then the girth of a graph is greater or equal to five and thus we obtain negative entropic negative curvature. The kernel defined by the neighbour random walk  coincides with the dynamics defined by $L_{0}$ for the case of $D$-regular graphs. Thus, by their result, if 
%\begin{equation*}
%K_{\text{0ll}}(x,y)<-2+\frac{6}{D}  \hspace{0.1cm} \forall x,y\in \mathcal{X} \hspace{0.2cm} \text{then} \hspace{0.2cm} K(z,S_{2}(z))>1 \hspace{0.2cm} \forall z\in \mathcal{X} .
%\end{equation*}

 
 %In \cite{MW19}, Florentin M{\"u}nch and Rados{\l}aw K.Wojciechowski, generalized the notion of Lin-Lu-Yau curvature  for any graph Laplacian. The Lin-Lu-Yau curvature associated to $L_{0}$ is defined in the appendix. Let us note, that if \textit{Lin-Lu-Yau curvature} is strictly positive it does not imply that the entropic curvature is positive. To be more precise, if the Lin-Lu-Yau curvature is positive on the edge $\{x,y\}$, it does not imply that $K(x,S_{1}(x),S_{2}(x))<1$ nor that $K(y,S_{1}(y),S_{2}(y))<1$.For this purpose, let us consider the following graph called the \textit{windmill graph} $W_{d}(4,2)$ consisting of 2 copies of the complete graph $K_{4}$ at a shared universal vertex  (see Figure 1).
  
%\begin{figure}[!ht]
%\centering
%\includegraphics[scale=0.2]{windflow.png}
%\caption{The value of \textit{Lin-Lu-Yau Ricci curvature} according to the graph Laplacian $L_{0}$ for the edge $(x,y)\in W_{d}(4,2)$} 
%\label{Figure1.1}
%\end{figure}
%\FloatBarrier
  
%This graph has strictly positive \textit{Lin-Lu-Yau curvature} on the edge $\{x,y\}$. 
%Indeed, it is easy to verify that
%\begin{equation*}
%W_{1}(m_{x}^{\alpha},m_{y}^{\alpha})=1-\frac{\alpha}{2}-\frac{\alpha}{6} .
%\end{equation*}
%Therefore, $K_{LLY}(x,y)=\lim_{\alpha\rightarrow 0} %\frac{K_{\alpha}(x,y)}{\alpha}=\frac{2}{3}.$

%However,
%$\sup\{K(x,S_{1}(x),S_{2}(x)),K(y,S_{1}(y),S_{2}(y)%\}>1$.





%\section{Applications and examples}

%In this section we will give examples and applications of entropic curvature for graphs. Firstly, we will exemplify positive entropic curvature, in particular, we will be interested in a \textit{family of graphs} whose curvature is strictly positive which we will denote $\mathcal{T}$. Subsequently, we will give examples of non-positively curved graphs with an emphasis on graphs called \textit{geodetic graphs}. \

%In everything that follows the dynamics will be determined by the generator $L_{0}$ with reversible measure associated $m_{0}$. Therefore, we will always use the formulation developed in \eqref{defRbis}. Let us remark that the operator $L_{0}$ as defined in the introduction corresponds to the usual \textit{Discrete Laplacian} $\Delta f(x)=\sum_{y\sim x} (f(y)-f(x))$ defined for any $f\in \mathbb{R}^{\mathcal{X}}$.

%\subsection{Positive entropic curvature in the class $\mathcal{T}$}

%Our aim in this section is to study the entropic curvature on a specific class of graphs which will be denoted \textit{class of graphs} $\mathcal{T}$. 

%\begin{definition}

% Let $\mathcal{X}$ be a graph belonging to the class $\mathcal{T}$.


%\begin{enumerate}[(i)]

%\item For every $z\in \mathcal{X}$, the restriction of the graph $\mathcal{X}$ to the ball $B_{2}(z):=\{k | d(z,k)\leq 2\}$ denoted as $\mathcal{X}\restriction_{B(z,2)}$ after removing all edges between vertices exclusively in $S_{1}(z)$ or $S_{2}(z)$ respectively is a bipartite graph $G=(U,V,E)$ such that:
%\begin{enumerate}
 %   \item 
 %   \[U=\{z\}\cup S_{2}(z) \quad \mbox{and}\quad V=S_{1}(z)\] with $S_{1}(z)=\{v_{1},v_{2},\cdots v_{m}\}$ and 
   % $S_{2}(z)=\{z_{1},z_{2},\cdots z_{n}\}$ .
  %  \item For every $v\in V$
  %  \[\{z,v\}\in E \hspace{0.1cm}.\] 
  %  \item Let us denote by 
  %  \[\mathcal{N}(z_{i}):=\{v\in V| (z_{i},v)\in E\} \] then 
  %  \[\max_{v_{i}\in \mathcal{N}(z_{i})}i= \min_{i\in \mathcal{N}(z_{i+1})}i \in E\]  for $1\leq i\leq n-1$ and \[\mathcal{N}(z_{1})\cap \mathcal{N}(z_{n})=\{v_{1}\} \hspace{0.1cm} . \]  
  %  The following figure illustrates the above condition.
%\begin{figure}[!ht]
%\centering
%\includegraphics[scale=0.4]{bipartite.png}
%\caption{Geometrical condition for restricted graphs $G$ in $\mathcal{T}$}
%\label{Figure1.1}
%\end{figure}
%\FloatBarrier
%\end{enumerate}
  

  

%\item For all $z\in \mathcal{X}$ and $z''\in S_{2}(z)$ one has
%$|S_{1}(z)\cap S_{1}(z'')|\geq 2$.

%\begin{figure}[!ht]
%\centering
%\includegraphics[scale=0.4]{Dibujo1.png}
%\caption{Example of forbidden configuration violating property $(i)$}
%\label{Figure1.1}
%\end{figure}
%\FloatBarrier



%\item For all $z\in \mathcal{X}$ and $z'\in S_{1}(z)$ one has 
%$|\V(z)\cap V(z')|\le2$.
%$|S_{2}(z)\cap S_{1}(z')|\leq 2$.


%\begin{figure}[!ht]
%\centering
%\includegraphics[scale=0.4]{Dibujo2.png}
%\caption{Example of forbidden configuration violating property $(ii)$}
%\label{Figure1.2}
%\end{figure}
%\FloatBarrier

%\item Let $z\in \mathcal{X}$ arbitrary. For all $z_{1}^{\prime \prime}\neq z_{2}^{\prime \prime}\in S_{2}(z)$ one has
%$$\mid S_{1}(z)\cap S_{1}(z_{1}^{\prime \prime})\cap S_{1}(z_{2}^{\prime \prime}) \mid \leq 1$
%\begin{figure}[!ht]
%\centering
%\includegraphics[scale=0.4]{Dibujo3.png}
%\caption{Example of forbidden configuration violating property $(iii)$}
%\label{Figure1.2}
%\end{figure}
%\FloatBarrier

%\end{enumerate}


%\end{definition}
%\begin{definition}

%The class of graphs that satisfy the following four properties is denoted $\mathcal{T}$.

%For all $z\in \mathcal{X}$,
%let $S_{1}(z)=\{z_{1},z_{2},\ldots z_{n}\}$ and $S_{2}(z)=\{v_{1},v_{2},\ldots v_{m}\}$.
%Let $N_{i}(z):=\{z_{i}\in S_{1}(z) : [z,z_{i}]\in [z,v_{i}] \hspace{0.1cm} \text{for some} \hspace{0.1cm} i\in [k]\} \hspace{0.1cm}$ and $U_{i}(z):=\{z_{i}\in S_{1}(z) \mid z_{i}\in [v_{i},v_{i+1}]\}$. A graph $G=(\mathcal{X},E)$ belongs to $\mathcal{T}$ if and only if the restriction of the graph $\mathcal{X}$ to the ball $B_{2}(z)$ denoted as $\mathcal{X}\mid_{B_{2}(z)}$ satisfies the following properties:

%\begin{enumerate}[(i)]

%\item If $z_{i}\in [z,v_{i}]$ then $z_{i}\sim v_{i}$ \hspace{0.1cm} ,

%\item For all $z\in \mathcal{X}$, $N_{i}(z)\cap %N_{i+1}(z)=U_{i}(z)\hspace{0.1cm} \forall 1\leq i\leq %k-1 \hspace{0.1cm},\text{and} \hspace{0.1cm} %N_{k}(z)\cap N_{1}(z)=U_{k}(z)$,

%\item For all $z\in \mathcal{X}$, $|N_{i}(z)| \geq 2$ \hspace{0.1cm}, 

%\item For all $z\in \mathcal{X}$, $|U_{i}(z)|:=p_{i} \geq 1$\hspace{0.1cm}.

%\end{enumerate}
%The following figure illustrates the above condition.
%\begin{figure}[h!]
%\centering
%\includegraphics[scale=0.3]{nuevo3.png}
%\caption{Structure of the neighborhood of radius 2 of an arbitrary vertex $z\in G$ with $G\in \mathcal{T}$}
%\label{Figure1.2}
%\end{figure}
%\FloatBarrier



%\end{definition}

%\begin{definition}

%The class of graphs that satisfy the following two properties is denoted $\mathcal{T}$.

%A graph $G=(\mathcal{X},E)$ belongs to $\mathcal{T}$ if and only if for every $z\in \X$ the restriction of $\mathcal{X}$ to $B_{2}(z)$ denoted as $\mathcal{X}\mid_{B_{2}(z)}$ satisfies the following properties:

%\begin{enumerate}[(i)]

%\item Let $S_{1}(z)=\{z_{0},z_{2},\ldots ,z_{n}\}$ and $S_{2}(z)=\{v_{0},v_{2},\ldots ,v_{m-1}\}$ then $S_{1}(z)=\bigcup_{i=0}^{m-1} N_{i}(z)$ such that 
%\begin{enumerate}
%\item $N_{i}(z)\cap N_{j}(z)=\emptyset$, \hspace{0.1cm} 
%$\forall i,j\in \mathbb{Z}/m\mathbb{Z}$ such that $\min\Big((j-i)\hspace{0.1cm} \text{mod}\hspace{0.1cm} m, (i-j)\hspace{0.1cm} \text{mod}\hspace{0.1cm} m \Big)>1$ .

%\item $U_{m}(z):=N_{m}(z)\cap N_{1}(z)\neq \emptyset$, 
%\item $N_{i}(z)\cap N_{j}(z)=\emptyset$, otherwise.
%\end{enumerate}

%\item For all $i\in [m-1]$, for all $z_{i}\in N_{i}(z)$, $z_{i}\sim v_{i}$. 
%\end{enumerate}
%\end{definition}

%The following figure illustrates the above condition.
%\begin{figure}[!ht]
%\centering
%\includegraphics[scale=0.3]{nuevo22.png}
%\caption{Structure of the neighborhood of radius 2 of an arbitrary vertex $z\in G$ with $G\in \mathcal{T}$}
%\label{Figure1.2}
%\end{figure}
%\FloatBarrier

%\begin{definition}
%Let $n\in \mathbb{N}$ and let define a graph $\bar{G}=(\bar{X},\bar{E})$ in the following way.
%Let \begin{equation*}
%\bar{X}:=\{z\}\cup \Big(\bigcup\limits_{j=1}^{n}B_{j}\Big)\cup S_{2} \hspace{0.1cm},
%\end{equation*}
%where $S_{2}:=\{z''_{1},z''_{2},\ldots z''_{n}\}$, the sets $B_{j}$'s are non-empty, pairwise disjoint and admit the following decomposition
%\begin{equation*}
%B_{j}=B_{j}^{1}\cup B_{j}^{2} \hspace{0.2cm} \forall j\in[n] .
%\end{equation*}
%Let define $\bar{E}$ as follows
%\begin{enumerate}[(i)]
%\item
%\begin{equation*}
%\forall j\in [n], \forall z^{\prime}\in B_{j}, \{z,z^{\prime}\}\in \bar{E} \hspace{0.1cm} ,
%\end{equation*}
%\item 
%\begin{equation*}
%\forall j\in [n],\forall z^{\prime}\in B_{j}, \{z^{\prime},z''_{j}\}\in \bar{E} \hspace{0.1cm},
%\end{equation*}
%\item 
%\begin{equation*}
%\forall j\in[n-1], \forall z^{\prime}\in B_{j}^{2}, \{z^{\prime},z''_{j+1}\}\in \bar{E} \hspace{0.2cm} \text{and} \hspace{0.2cm} \forall z^{\prime}\in B_{n}^{2},\{z^{\prime},v_{1}\}\in \bar{E} \hspace{0.1cm}, 
%\end{equation*}
%\item 
%\begin{equation*}
%\forall j\in [n],|B_{j}^{2}| \geq 2 \hspace{0.1cm} .
%\end{equation*}
%\end{enumerate}

%A graph $G=(\mathcal{X},E)$ belongs to $\mathcal{T}$ if and only if the restriction of the graph $G$ to the ball of radius 2 centered at any $z\in \mathcal{X}$ is isomorphic to $\bar{G}$ for some $n\in \mathbb{N}$. 
%\end{definition}

%The following figure illustrates the above geometrical condition.
%\begin{figure}[h!]
%\centering
%\includegraphics[scale=0.3]{nuevo14.png}
%\caption{Structure of the neighborhood of radius 2 of an arbitrary vertex $z\in G$ with $G\in \mathcal{T}$}
%\label{Figure1.2}
%\end{figure}
%\FloatBarrier



 %The following result states that the class %$\mathcal{T}$ has positive entropic curvature, that %is, $(G,m_{0},d,L_{0})$ has positive entropic %curvature for every $G\in \mathcal{T}$.
%\newpage

%\begin{proposition}
%If $\X$ is a graph belonging to the class $\mathcal{T}$, then it has  positive 
%entropic curvature, namely
%\[\kappa_1\geq 2(1-K), \quad \kappa\geq -2\log K, \quad\widetilde \kappa\geq1-K,\] 
%with 
%\[ K\geq  \min\Big(\frac{2}{\min_{j} \mid I_{j}\mid},
%\frac{\sqrt{2}}{2} \Big)<1 \]
%\end{proposition}



%Let us firstly describe the geometric nature of the graph $G\in \mathcal{T}$. Let $z$ be any vertex of $G$ with $n$ neighbours which we will denote in cyclical clockwise order $\{1,2\cdots ,n\}$.
%Let $S_{2}(z)= \{z_{j}\}_{j=1}^{k}$ with $j\in [k]$ and let 
%\begin{equation*}
%I_{j}:=\{z^{\prime}\in S_{1}(z) : [z,z^{\prime}]\in [z,z_{j}] \hspace{0.1cm} \text{for some} \hspace{0.1cm} j\in [k]\} \hspace{0.1cm}.
%\end{equation*}

%Since $G\in \mathcal{T}$, we will have the following cyclic structure $I_{1}=\{1,\cdots ,i_{1}\}, I_{2}=\{i_{1},\cdots ,i_{2}\},\\ \cdots I_{k-1}=\{i_{k-1},\cdots ,i_{k}\}$ and finally 
%$I_{k}=\{i_{k},\cdots ,n,1\}$ where $i_{1}<i_{2},\cdots <i_{k}\in \{1,2,\cdots ,n\}$. Let us remark also that 
%$|S_{1}(z) \cap [z,z_{j}]|=|I_{j}|$. The following figure illustrates the geometric structure noted above.

%\begin{figure}[!ht]
%\centering
%\includegraphics[scale=0.5]{Dibujo44.png}
%\caption{Structure of the neighborhood of radius 2 of an arbitrary vertex $z\in G$ with $G\in \mathcal{T}$}
%\label{Figure1.4}
%\end{figure}
%\FloatBarrier


%Let us consider the case where all $\mid I_{j}\mid =2$. Then, we have the following  lemma .

%For the proof of the Proposition 2 we will need the following elementary lemma.

%\begin{lemma}
%Let $\alpha_1,\dots,\alpha_n\ge0$ such that $\sum_{i=1}^{n}\alpha_{i}=1$. Then, if $n>3$
%\begin{equation*}
%\sum_{i=1}^{n-1}\alpha_{i}\alpha_{i+1}+\alpha_{n}\alpha_{1}\leq \frac{1}{4} \hspace{0.1cm} ,
%\end{equation*}
%and for $n=3$
%\begin{equation*}
%\alpha_{1}\alpha_{2}+\alpha_{2}\alpha_{3}+\alpha_{3}\alpha_{1}\leq \frac{1}{3} \hspace{0.1cm}.
%\end{equation*}

%\begin{proof}
%If $n>3$  is an even integer, then the inequality $4ab\leq (a+b)^2$ implies  
%\begin{equation*}
%\sum_{i=1}^{n-1}\alpha_{i}\alpha_{i+1}+\alpha_{n}\alpha_{1}\leq \Big( \sum_{i\,\text{ even}} \alpha_{i}\Big)\Big(\sum_{i\,\text{  odd}} \alpha_{i} \Big) \leq \frac{1}{4} \hspace{0.1cm} .
%\end{equation*}
%If $n>3$ is odd, by the cyclicity of the sum, one may assume without loss of generality $\alpha_{1}\leq \alpha_{2}$. It follows that 
%\begin{equation*}
%\sum_{i=1}^{n-1}\alpha_{i}\alpha_{i+1}+\alpha_{n}\alpha_{1}\leq \sum_{i=1}^{n-1}\alpha_{i}\alpha_{i+1}+\alpha_{n}\alpha_{2}\leq \Big(\sum_{i\,\text{ even}} \alpha_{i} \Big) \Big(\sum_{i\,\text{ odd}} \alpha_{j}\Big)\leq\frac14 \hspace{0.1cm} .
%\end{equation*}
%For $n=3$, Cauchy-Schwarz inequality implies
%\[\alpha_{1}\alpha_{2}+\alpha_{2}\alpha_{3}+\alpha_{3}\alpha_{1} =\frac12(\alpha_1+\alpha_2+\alpha_3)^2-\frac12(\alpha_1^2+\alpha^2_2+\alpha_3^2)\leq \frac12- \frac16 (\alpha_1+\alpha_2+\alpha_3)^2=\frac13.
%\]
%\end{proof}

%\end{lemma}



%\begin{proofp}

%By the arithmetic-geometric inequality, and since %$\sum_{j=1}^{k}\sum_{l\in I_{j}} \alpha_{l}\leq 2$ 
%Let $G\in \mathcal{T}$ . 
%Let consider $z\in G$ arbitrary and let suppose %without loss of generality that\\
%$|\{i:N_{i}(z)\neq \emptyset \}|=k$ with $k\in %\mathbb{N}$.

%Since 
%\begin{equation*}
%\sum_{j=1}^{k}\sum_{l\in N_{j}}\alpha_{l}=
%\Big(\sum_{i\notin U_{j},1\leq j\leq k} \alpha_{i}+\sum_{i\in U_{j},j\in [k]}\alpha_{i}\Big)+\sum_{i\in U_{j},j\in [k]}\alpha_{i}\leq 2
%\end{equation*}
%by the arithmetic-geometric inequality, we  immediately get
%\begin{equation}\label{fproof}
%K(z,S_{1}(z),S_{2}(z))=\sum_{j=1}^{k} \mid N_{j}\mid 
%\Big(\prod_{l\in N_{j}} \alpha_{l}\Big)^{\frac{2}{\mid N_{j}\mid}}
%\leq \frac{2}{\min_{j} \mid N_{j} \mid} \hspace{0.1cm} .
%\end{equation}

%Therefore Proposition 2 is trivially proved for the case where $N_{j}$  has cardinality greater than 2 for some $j\in [k]$, with curvature given by $\kappa\geq -2\log\Big(\frac{2}{\min_{j} |I_{j}|}\Big)$. Let $T_{j}:=N_{j}\setminus U_{j}$, then $K(z,S_{1}(z),S_{2}(z))=\sum_{j=1}^{k} (|T_{j}| +\mid U_{j}\mid ) \Big( \prod_{l\in T_{j}\cup \{i_{j}\}} \alpha_{l} \Big)^{\frac{2}{|T_{j}| +\mid U_{j}\mid }}$. 
%There are two cases to consider.
%In the particular case where $|\{j: \mid N_{j}|=2\}| =2$ one bounds directly $K(z,S_{1}(z),S_{2}(z))$ using \eqref{fproof} by $\frac{5}{6}$. 
%Let now prove Proposition 2 in full generality, that is in the case that 
%$|\{j: \mid T_{j}\mid=2\}| \geq 3$
% $\min_{j} |N_{j}|=2$. The main idea of the proof is to reduce ourselves to the case of cyclic chains of length two.



%By the arithmetic-geometric inequality 
%\begin{equation*}
%K(z,S_{1}(z),S_{2}(z))\leq \sum_{j=1}^{k} \Big(\sum_{l\in T_{j}\cup U_{j}} \alpha_{l} \Big) \Big( \prod_{l\in T_{j}\cup U_{j}} \alpha_{l} \Big)^{\frac{1}{\mid T_{j} \mid +\mid U_{j} \mid}} .
%\end{equation*}

%Since $ \Big( \prod_{l\in T_{j}\cup U_{j}}\alpha_{l}\Big)^{\frac{1}{\mid T_{j}\mid +\mid U_{j}\mid}}\leq \Big( \max_{l\in T_{j}\cup U_{j}} \alpha_{l}\alpha_{l+1} \Big)^{\frac{1}{2}}$,
%by the Cauchy-Schwarz inequality 
%\begin{equation*}
%K(z,S_{1}(z),S_{2}(z)) \leq \Big( \sum_{j=1}^{k} \sum_{l\in T_{j}\cup U_{j}} \alpha_{l} \Big)^{\frac{1}{2}} \Big( \sum_{j=1}^{k} \max_{l\in T_{j}\cup U_{j}} \alpha_{l}\alpha_{l+1} \sum_{l\in T_{j}\cup U_{j}} \alpha_{l} \Big)^{\frac{1}{2}} . 
%\end{equation*}

%Noticing that $\sum_{j=1}^{k}\sum_{l\in T_{j}\cup U_{j}}\alpha_{l} \leq 2$ and $\sum_{l\in T_{j}\cup U_{j}}\alpha_{l}  \leq 1$,
%\begin{equation*}
%K(z,S_{1}(z),S_{2}(z)) \leq \sqrt{2  \sum_{j=1}^{k} \max_{l\in T_{j}\cup U_{j}} \alpha_{l}\alpha_{l+1} } \hspace{0.1cm} .
%\end{equation*}

%Therefore, using the Lemma 1,
%\begin{equation*}
%K\leq \sqrt{2 \sum_{j=1}^{k} \sum_{l} \alpha_{l}\alpha_{l+1}}\leq \min \Big(\frac{\sqrt{2}}{2}, \frac{\sqrt{6}}{3},\frac{5}{6}\Big)=\frac{\sqrt{2}}{2}<1 \hspace{0.1cm} 
%\end{equation*}

%\end{proofp}

%\begin{remark}
%The argument above easily can be extended to %the case where the intersections between sets %$I_{j}$ and $I_{j+1}$ for $1\leq j\leq k$ have %cardinality larger than 1.
%Let $U_{j}:=I_{j}\cap I_{j+1}$ for $1\leq j\leq %k$, $|U_{j}|=p$ with $p\in \mathbb{N}$ and %$U_{j}\cap U_{j^{\prime}}=\emptyset$ for $j\neq %j^{\prime}$.

%Then 
%$\sum_{j=1}^{k}\sum_{l\in I_{j}}\alpha_{l}=
%\Big(\sum_{i\notin U_{j},1\leq j\leq k} %\alpha_{i}+\sum_{i\in U_{j},j\in %[k]}\alpha_{i}\Big)+\sum_{i\in U_{j},j\in %[k]}\alpha_{i}\leq 2 $ and the same reasoning %as before follows.
%\end{remark}

%\subsubsection{A family of examples: %\textit{triangulations} of bounded degree}

%A main example of the class of graphs $\mathcal{T}$ are the \textit{triangulations} with bounded degree. 

%\begin{definition}
%We say that $\Gamma=(\mathcal{X},E)$ is a triangulation if it is a \textit{planar} graph such that all its faces are triangular, that is, have three edges. 

%\end{definition}

%\begin{remark}
%This definition is equivalent to the notion of maximal planar graphs in the sense of inclusion and since every maximal planar graph is 3-connected, by the Steinitz theorem \cite{Ste22} every triangulation $\Gamma$ can be represented as the graph of a convex polyhedron.
%\end{remark}
%Let us introduce $\Gamma_{d}$ as a triangulation such that every vertex has degree less or equal to $d$ with $d\in \{4,5,6\}$. Let $\Gamma\in \Gamma_{d}$ with $d\in \{4,5,6\}$. 
%Let $K_{2}(d):=1-\sup_{z\in \Gamma_{d}}K(z,S_{1}(z),S_{2}(z))$. 
%It is easily verified that $K_{2}(4)\geq \frac{2}{3}$, $K_{2}(5)\geq \frac{1}{3}$ and $K_{2}(6)\geq 0$. 
%\end{proposition}

%\begin{remark}
%By the entropic Bonnet-Myers Theorem $(\Gamma,d,m_{0},L_{0})$ has finite diameter for any $\Gamma=(\X,E)\in \Gamma_{d}$ with $d\in \{4,5\}$. Let us note that by a direct argument, using Euler's formula, the property of triangulations $2|E|=3|F|$ and the handshaking lemma it follows immediately that if $\Gamma\in \Gamma_{4}$ then the number of vertices of $\Gamma$ is less or equal to 6 and if $\Gamma\in \Gamma_{5}$ then the number of vertices of $\Gamma$ is less or equal to 12.  
%\end{remark}


%Let $\Gamma=(\mathcal{X},E)$ be a triangulation. Let $F$ be the set of faces of $\Gamma$ and let $f\in F$. The \textit{degree face} of $f$ is defined as
%\begin{equation*}
%\textnormal{Deg}(f):=|\text{boundary edges of f} |=| 
%\text{boundary vertices of f}| \hspace{0.1cm}.
%\end{equation*}

%The \textit{combinatorial curvature} was first introduced by M.Gromov in \cite{Gro87} for the study of hyperbolic groups.

%\begin{definition}
%The \textit{combinatorial curvature} $\mathcal{K}:\mathcal{X}\rightarrow \mathbb{R}$ is defined as 
%\begin{equation*}
%\mathcal{K}(v)=1-\frac{\textnormal{Deg}(v)}{2}+\sum_{f\in F,v\in F} \frac{1}{\textnormal{Deg}(f)} \hspace{0.1cm} .
%\end{equation*}

%\end{definition}

%There is a relationship between the entropic curvature and the combinatorial curvature in the setting of triangulations. Indeed, it is direct by definition of combinatorial curvature that $\mathcal{K}(v)\geq 1-\frac{d}{6} \hspace{0.2cm} \text{for all} \hspace{0.2cm} v\in \Gamma_{d}$. This implies that there is a decay of the combinatorial curvature as well as for the entropic curvature.

 



%\subsubsection{Two examples: the Octahedral graph and the Icosahedral graph}


%The \textit{Octahedral graph} is the 6-node 12-edge Platonic graph having the connectivity of the Octahedron. Let us note that the Octahedral graph denoted as $\mathcal{O}$ belongs to $\Gamma_{4}$. Let $z$ be an arbitrary vertex of $\mathcal{O}$, let $S_{1}(z)=\{z_{i}\}_{i=1}^{5}$ and let $z^{\prime \prime}$ be such that $S_{2}(z)=\{z^{\prime \prime}\}$ as in the embedding below

%\begin{figure}[!ht]
%\centering
%\includegraphics[scale=0.4]{Dibujo5.png}
%\caption{An embedding of the Octahedral graph} 
%\label{Figure1.1}
%\end{figure}
%\FloatBarrier

%By \eqref{defRbis} one has to solve $K(z,S_{1}(z),S_{2}(z))=\max_{\alpha_{i}} 4\sqrt{\alpha_{1}\alpha_{2}\alpha_{3}\alpha_{4}}$. By the arithmetic-geometric inequality one has 
%$4\sqrt{\alpha_{1}\alpha_{2}\alpha_{3}\alpha_{4}}\l%eq \frac{1}{4}$ and since this value is attained one concludes that $K=\frac{1}{4}$. %$C_{2}(\mathcal{O}):=1-\sup_{z\in \mathcal{O}}R(z,S_{2}(z))=\frac{3}{4}$.

%The \textit{Icosahedral graph} is the Platonic graph whose nodes have the connectivity of the Icosahedron. Let us observe that the Icosahedral graph denoted as $\mathcal{I}$ belongs to $\Gamma_{5}$. Let $z$ be an arbitrary vertex of $\mathcal{I}$ and let us denote $\{z_{i}\}_{i=1}^{5}$ the neighbours of $z$ in clockwise order and $z^{ij}$ be such that  $z^{ij}\in S_{2}(z), [z,z_{i}]\in [z,z^{ij}], [z,z_{j}]\in [z,z^{ij}]$ for $i,j\in \{1,2,3,4,5\}$ as in the embedding below 
%\begin{figure}[!ht]
%\centering
%\includegraphics[scale=0.2]{Icosahedralff.png}
%\caption{An embedding of the Icosahedral graph } 
%\label{Figure1.1}
%\end{figure}
%\FloatBarrier
%\nopagebreak

%By \eqref{defRbis}, one has to solve %$K(z,S_{1}(z),S_{2}(z))=\max_{\alpha_{i}}2\Big(\sum%_{i=1}^{4}\alpha_{i}\alpha_{i+1}
%+\alpha_{1}\alpha_{5} \Big)$ Let %$I=\sum_{i=1}^{4}\alpha_{i}\alpha_{i+1} %+\alpha_{1}\alpha_{5}$. Without loss of generality, %let us suppose that $\alpha_{3}\leq \alpha_{4}$. %Then, by arithmetic geometric inequality
%\begin{equation*}
%I\leq (\alpha_{1}+\alpha_{4})(\alpha_{2}+\alpha_{3}+\alpha_{5})\leq \Big( \frac{\sum_{i}\alpha_{i}}{2} \Big)^{2}=\frac{1}{4} .
%\end{equation*}
%Since this value is attained %$R(z,S_{2}(z))=\frac{1}{2}$ 
%one concludes that $K=\frac{1}{2}$. %$C_{2}(\mathcal{I}):=1-\sup_{z\in \mathcal{I}}R(z,S_{2}(z))=\frac{1}{2}$.


%\subsection{Non-positively curved graphs}







%entropic curvature on geodetic graphs

%In everything that follows the dynamics will be determined by the generator $L_{0}$ with reversible measure associated $m_{0}$. Therefore, we will always use the formulation developed in \eqref{defRbis}. Let us remark that the operator $L_{0}$ as defined in the introduction corresponds to the usual \textit{Discrete Laplacian} $\Delta f(x)=\sum_{y\sim x} (f(y)-f(x))$ defined for any $f\in \mathbb{R}^{\mathcal{X}}$.

%This section is motivated by Ore's question \cite{Oys87} about a characterization of \textit{geodetic graphs}, which has recently been studied in \cite{Lin21}. Indeed we will see that negative entropic curvature characterizes in one direction \textit{geodetic graphs}. More precisely, if a graph is \textit{geodetic} and has a diameter greater than or equal to two then its entropic curvature is non positive. Let us introduce \textit{geodetic graphs}. In this way we will see that the condition of being a \textit{geodetic graph}  is a sufficient condition to obtain negative entropic curvature. First we will introduce and analyze \textit{geodetic graphs}. Subsequently, we will study the entropic curvature that is negative for a class of graphs denoted as \textit{antitrees} .


%\begin{definition}
%A graph $G=(\mathcal{X},E)$ is called geodetic if for every two vertices $u$ and $v$ in $G$ there exists a unique 
%geodesic connecting $u$ and $v$.
%\end{definition}

%For example, every tree, every complete graph, every odd-length cycle and the \textit{Petersen graph} are geodetic graphs.

%\begin{proposition}
%For the class of spaces $(G,d,L_{0},m_{0})$ where $G=(\mathcal{X},E)$ is a geodetic graph with diameter greater or equal to 2 the entropic curvature is always non positive. Moreover, $K=\max_{z\in \mathcal{X}}\Big(\max_{z^{\prime},z^{\prime}\sim z}\textnormal{Deg}(z^{\prime})-1\Big)$.
%\end{proposition}

%\begin{proof}
%It was proved in \cite{PS82} that a graph $G$ of finite diameter $d$ is geodetic if and only if for every $z\in \mathcal{X}$, each vertex of $S_{r}(z)$ is adjacent to a unique vertex in $S_{r-1}(z)$ for each $0 \leq r \leq d$. Let $G=(\mathcal{X},E)$ be a geodetic graph with diameter greater or equal to 2 and $z_{0}\in \mathcal{X}$. Simply taking $r=2$, we obtain that there exists $z_{0}^{\prime \prime}\in S_{2}(z_{0})$ such that
%$|S_{1}(z_{0})\cap [z_{0},z_{0}^{\prime \prime}]| %=1$ . Therefore, we immediately conclude that $K\geq 1$. Moreover, for every $z\in \mathcal{X}$
%\begin{equation*}
%K(z,S_{1}(z),S_{2}(z))=\sup_{\alpha} \sum_{z^{\prime}\sim z^{\prime \prime},d(z^{\prime\prime},z)=2} \alpha(z^{\prime})^{2}= \sup_{\alpha} \sum_{z^{\prime},z^{\prime}\sim z}(\textnormal{Deg}(z^{\prime})-1)\alpha(z^{\prime})^{2}=\max_{z^{\prime},z^{\prime}\sim z}\textnormal{Deg}(z^{\prime})-1 \hspace{0.1cm} .
%\end{equation*}

%\end{proof}


%\subsubsection{The example of the Petersen graph}.
%\begin{figure}[!h]
%\centering
%\includegraphics[scale=0.3]{P.png}
%\caption{The values of 1-$K(z,S_{1}(z),S_{2}(z))$ for each vertex $z$ in the \textit{Petersen graph}} 
%\label{Figure1.1}
%\end{figure}
%\FloatBarrier







%\subsubsection{The example of the \textit{infinite binomial tree}}.
%\begin{figure}[!h]
%\centering
%\includegraphics[scale=0.2]{infbi.png}
%\caption{The values of 1-$K(z,S_{1}(z),S_{2}(z))$ for each vertex $z$ in the \textit{infinite binomial tree}} 
%\label{Figure1.1}
%\end{figure}
%\FloatBarrier



%\begin{remark}
%\begin{enumerate}[(i)]

 %\item Let us observe that the fact that $K\geq 1$ is consistent to the geometry of the underlying generic geodetic graph.
 %\item Note that if a space $(\mathcal{X},d,m,L)$ has zero or negative  entropic curvature it does not imply that the underlying graph is geodetic. Indeed, it was shown in \cite{Sam21}, that the circle $\mathbb{Z}/N\mathbb{Z}$ endowed with a uniform measure of any length and parity satisfy null entropic curvature. Also, note that the hexagonal tiling of the plane is not a geodetic graph, however locally it looks like a 3-regular tree and has negative entropic curvature, i.e., for every vertex $z$ in the  hexagonal tiling %$1-K(z,S_{1}(z),S_{2}(z))=-1$.

 
 
 


 
 
%\end{enumerate}


%\end{remark}

%In \cite{CLM20} David Cushing, Shiping Liu, Florentin M\"{u}nch, and Norbert Peyerimhoff proved that certain infinite graphs satisfy positive curvature in the sense of Ollivier as well as for normalized and  non normalized Bakry-{\'E}mery curvature under some growth hypothesis. 
% \textit{Antitrees} were introduced by Radoslaw Wojciechowski and firstly appeared in the litterature in \cite{KLW13}

%\textit{Antitrees} as in \cite{CLM20} are defined as follows:
%\begin{itemize}

%\item there is a first generation $V_{1}$ and any vertex in $V_{1}$ is connected to all vertices in $V_{2}$ and no vertices in $V_{k}$ 
%with $k\geq 3$, 

%\item any vertex $x\in V_{k}$, $k\geq 2$ is connected to all vertices in $V_{k-1}$ and $V_{k+1}$ and no vertices 
%in $V_{l}$, $\mid k-l\mid \geq 2$.  

%\end{itemize}

%Let an infinite sequence $(a_{k})_{1\leq k\leq \infty}$ such that $\mid V_{k}\mid =a_{k}$, $\forall k\in \mathbb{N}$. 
%The \textit{antitree} is denoted by $\mathcal{AT}$. %Let us study the entropic curvature of $\mathcal{AT}(\{k\})$.
%In the following, we will consider 
%that $V_{n}=K_{n}$ for all $n\geq 1$, i.e. all vertices of the same generation are connected. 
%Let denote an \textit{antitree} by $\mathcal{AT}$.

%\begin{remark}
%The definition allows the existence of edges between vertices of the same generation. 
%\end{remark}

%\begin{proposition} If the antitree $\mathcal{AT}$ admits positive growth, i.e., if $\mid V_{k+1} \mid>\mid V_{k} \mid \forall k\in \mathbb{N}$ then the antitree has negative entropic curvature. If $\mid V_{k+1} \mid=\mid V_{k} \mid \forall k\in \mathbb{N}$ then the antitree has zero entropic curvature. 
%\end{proposition}

%\begin{proof}

%Let $z\in \mathcal{AT}$. Let $n\in \mathbb{N}$ and let $z\in V_{n}$. Let us remark that
%$S_{1}(z)=V_{n-1}\cup V_{n+1}$ and $S_{2}(z)=V_{n-2}\cup V_{n+2}$. 

%Therefore by \eqref{defRbis}, 
%\begin{align*}
%K(z,S_{1}(z),S_{2}(z))&\leq \sup_{\alpha} \Bigg(\sum_{z^{''}\in V_{n+2}} \mid V_{n+1}\mid \Big(\prod_{z^{'}\in V_{n+1}}\alpha(z^{'})\Big)^{
%\frac{2}{\mid V_{n+1}\mid}}+\sum_{z^{''}\in V_{n-2}}\mid V_{n-1} \mid \Big(\prod_{z^{'}\in V_{n-1}}\alpha(z^{'})\Big)^{\frac{2}{\mid V_{n-1}\mid}} \Bigg) \\
%&\leq \mid V_{n+2} \mid \mid V_{n+1}\mid \Big(\frac{\sum_{z^{'}\in V_{n+1}} \alpha(z^{'})}{\mid V_{n+1}\mid} \Big)^{2}+\mid V_{n-2}\mid \mid V_{n-1} \mid \Big( \frac{\sum_{z^{'}\in V_{n-1}} \alpha(z^{'})}{\mid V_{n-1}\mid}\Big)^{2}  \\
%&\leq \frac{\mid V_{n+2}\mid}{\mid V_{n+1}\mid}\Big(\sum_{z^{'}\in V_{n+1}}\alpha(z^{'})\Big)^{2}+\frac{\mid V_{n-2}\mid }{\mid V_{n-1}\mid}\Big(\sum_{z^{'}\in V_{n-1}} \alpha(z^{'})\Big)^{2} \\
%&\leq \max\Big( \frac{\mid V_{n+2}\mid}{\mid V_{n+1}\mid},\frac{\mid V_{n-2}\mid}{\mid V_{n-1}\mid}\Big).
%\end{align*}

%Since this bound is attained, 
%\begin{equation*}
%K(z,S_{1}(z),S_{2}(z))=\max\Big( \frac{\mid %V_{n+2}\mid}{\mid V_{n+1}\mid},\frac{\mid V_{n-2}\mid}{\mid V_{n-1}\mid}\Big) .
%\end{equation*}

%\end{proof}

%\begin{remark}
%In contrast to \cite{CLM20}, where $V_{n}=K_{n} \hspace{0.1cm} \forall n\in \mathbb{N}$, with $K_{n}$ referring to a complete graph on $n$ vertices, the entropic curvature of the antitrees is not strictly positive for any strictly positive growth.   
%\end{remark}





%\section{Appendix}

%\subsection{Lin-Lu-Yau Ricci curvature}


%Given $G=(\mathcal{X},E)$ a graph endowed with its graph distance $d$ and with a Markov chain defined by $m:=\{m_{z}(\cdot)\}_{z\in \mathcal{X}}$. 
%The \textit{Ollivier's coarse Ricci curvature along the edge $\{x,y\}$} is defined as
%\begin{equation*}
%\mathcal{K}_{Oll}(x,y):=1-\frac{W_{1}
%(\eta_{x},\eta_{y})}{d(x,y)}\hspace{0.1cm} , 
%\end{equation*}
%with the markov kernel $\eta$ defined as follows
%\begin{equation*}
%\eta_{x}(y)=
%\begin{cases}
%\frac{1}{\textnormal{Deg}(x)} \hspace{0.1cm} \text{if} \hspace{0.1cm} y\sim x \hspace{0.1cm} ,\\
%0 \hspace{0.1cm} \text{otherwise} .
%\end{cases}
%\end{equation*}
%The \textit{Ollivier's coarse Ricci curvature} of $(G,d,\eta)$ is defined as
%\begin{equation*}
%\mathcal{K}_{Oll}:=\inf_{x,y\in \mathcal{X}} \mathcal{K}_{Oll}(x,y) \hspace{0.1cm} .
%\end{equation*}
%For $0\leq \alpha<1$, the \textit{$\alpha$-lazy random walk} $m_{x}^{\alpha}$ associated to the graph Laplacian $L_{0}$ is defined as
%\begin{equation*}
%m_{x}^{\alpha}(y)=
%\begin{cases}
%\frac{\alpha}{\text{Deg}_{\text{max}}} \hspace{0.2cm} \text{if} \hspace{0.2cm} x\sim y ,\\
% 1- \alpha \frac{\text{Deg}(x)}{\text{Deg}_{\text{max}}}
%\hspace{0.2cm} \text{if} \hspace{0.2cm} y= x ,\\
%0 \hspace{0.2cm} \text{otherwise}.
%\end{cases}
%\end{equation*}


%For every $x,y\in \mathcal{X}$, one defines
%\begin{equation*}
%K_{\alpha}(x,y):=1-\frac{W_{1}(m_{x}^{\alpha},m_{y}^{\alpha})}{d(x,y)}.
%\end{equation*}

%As shown in \cite{MW19}, the limit as $\alpha\rightarrow 0$ exists for any graph Laplacian and therefore one can define the Ricci Lin-Lu-Yau curvature along the edge $\{x,y\}$ denoted as $K_{LLY}(x,y)$ by
%\begin{equation*}
%K_{LLY}(x,y):=\lim_{\alpha \rightarrow 0} \frac{K_{\alpha}(x,y)}{\alpha} .
%\end{equation*}

%\begin{lemma}
%For every $z\in \mathcal{X}$ and $z''\in S_{2}(z)$ such that $d(z,z'')=2$, let $S_{1}^{z,z^{''}}:=S_{1}(z)\cap [z,z'']$.
%If the entopic curvature is strictly positive then 
%\begin{equation*}
%\mid S_{1}^{z,z''}\mid > \mid \{w''\in S_{2}(z)\mid S_{1}^{z,w''}=S_{1}^{z,z''}\} \mid
%\end{equation*}
%\end{lemma}

%\begin{proof}

%Let $z\in \mathcal{X}$ and $z_{0}''\in S_{2}(z)$. %Let 
%$\alpha(z'):=\frac{1}{\mid S_{1}^{z,z_{0}''}\mid} %\mathds{1}_{z'\in S_{1}^{z,z_{0}''}}$ then,  

%\begin{align*}
%K(z,S_{1}(z),S_{2}(z))&=\sup_{\alpha}\sum_{z''} \mid S_{1}^{z,z''}\mid \Big(\prod_{z'\in S_{1}^{z,z''}} \alpha(z') \Big)^{\frac{2}{\mid S_{1}(z,z'')\mid}} \\
%&\geq  \sum_{z'',S_{1}^{z,z''}=S_{1}^{z,z_{0}''}}\frac{1}{ \mid S_{1}^{z,z_{0}''}\mid } \\
%&= \frac{1}{ \mid S_{1}^{z,z_{0}''}\mid } \mid \{z''\in S_{2}(z) ,S_{1}^{z,z''}=S_{1}^{z,z_{0}''}\}\mid ,
%\end{align*}
%Since $K(z,S_{1}(z),S_{2}(z))<1$ then 
%\begin{equation*}
%\mid S_{1}^{z,z_{0}''}\mid > \mid \{z''\in S_{2}(z),S_{1}^{z,z''}=S_{1}^{z,z_{0}''}\} ,
%\end{equation*}
% and the conclusion follows.

%\end{proof}




%\begin{lemma}\label{conv} Let   $(\X,d,m,L)$ be a space satisfying the $(G)$-conditions. Let $c\in \R$ and $r:\X\to \R$. The following assertions  are equivalent :
%\begin{enumerate}[(i)]
%\item For any $t\in [0,1]$  and any $x,y\in \X$,
%\[ \int r(z)\, d\nu_t^{x,y}(z)\leq (1-t)r(x)+t r(y)+c\,\frac{t(1-t)}{2 } \,d(x,y)\big(d(x,y)-1\big).\]
%\item For any  $z,\ttz\in \X$ with $d(z,\ttz)=2$, 
%\[\sum_{\tz\in S_1(z)\cap [z,\ttz]} \left(r(\ttz)+r(z)-2r(\tz)\right) \frac{L(z,\tz)L(\tz,\ttz)}{L^2(z,\ttz)}\geq -c.\]
%\end{enumerate}
%\end{lemma}
%\begin{proof}[Proof of Lemma \ref{conv}]
%Let $z,\ttz\in \X$ with $d(z,\ttz)=2$.  Applying $(i)$ with $x=z$ dans $y=\ttz$ provides for any $t\in [0,1]$
%\[(1-t)^2 r(z)+ 2t(1-t)\sum_{\tz\in S_1(z)\cap [z,\ttz]} r(\tz) \,\frac{L(z,\tz)L(\tz,\ttz)}{L^2(z,\ttz)} +t^2 r(\ttz)\leq (1-t)r(z)+t r(\ttz)+ct(1-t),\]
%or equivalently 
%\[t(1-t)\left[\sum_{\tz\in S_1(z)\cap [z,\ttz]} \left(r(\ttz)+r(z)-2r(\tz)\right) \frac{L(z,\tz)L(\tz,\ttz)}{L^2(z,\ttz)}\right]\geq -ct(1-t).\]
%This implies $(ii)$ 

%Conversely, let $x,y\in \X$ and $d:=d(x,y)$ and for $t\in [0,1]$, 
%\[R(t):=\int r(z)\, d\nu_t^{x,y}(z)=\sum_{k=0}^d\rho_t^d(k) R^{xy}_{k},\]
%where 
%\[R^{xy}_{k}:=\sum_{z\in [x,y], d(x,z)=k} r(z) \,\frac{L^{d(x,z)} (x,z)L^{d(z,y)}(z,y)}{L^d(x,y)}\]

%Simple computations give for any $t\in [0,1]$,
%\[R'(t)=d\,\sum_{k=0}^{d-1}\rho_t^{d-1}(k) \big(R^{xy}_{k+1}-R^{xy}_{k}\big),\]
%and therefore
%\[R''(t)=d(d-1)\,\sum_{k=0}^{d-2}\rho_t^{d-2}(k) \big(R^{xy}_{k+2}+R^{xy}_{k}-2R^{xy}_{k+1}\big).\]
%Then, observing that for any  $k\in\{0,\ldots ,d-2\}$, 
%\begin{align*}
%&R^{xy}_{k+2}+R^{xy}_{k}-2R^{xy}_{k+1}\\
%&= \sum_{(z,\tz,\ttz)\in [x,y], d(x,z)=k} \left(r(\ttz)+r(z)-2r(\tz)\right) \frac{L^{d(x,z)}(x,z)L(z,\tz)L(\tz,\ttz)L^{d(\ttz,y)}(\ttz,y)}{L^{d}(x,y)}\\
%&= \sum_{(z,\ttz)\in [x,y] d(x,z)=k, d(z,\ttz)=2} \left[\sum_{\tz\in S_1(z)\cap [z,\ttz]} \left(r(\ttz)+r(z)-2r(\tz)\right) \frac{L(z,\tz)L(\tz,\ttz)}{L^2(z,\ttz)}\right] \\&\qquad\qquad\qquad\qquad\qquad\qquad\qquad\qquad\qquad\qquad\frac{L^{d(x,z)}(x,z)L^2(z,\ttz)L^{d(\ttz,y)}(\ttz,y)}{L^{d}(x,y)},
%\end{align*}
%one gets, together with  $(ii)$, for any $t\in[0,1]$,
%\[R''(t)\geq -c \,d(d-1).\]
%It easily follows that for any $t\in[0,1]$
%\[ R(t)\leq (1-t)R(0)+tR(1)+c d(d-1) \, \frac{t(1-t)}2,\]
%which is exactly $(i)$. 
%\end{proof}




%\section{AUTRES}

%\begin{proof}[Proof of Lemma \ref{Dvcube} (r\'esultat moins bon)]
%We want to bound from below  the quantity $\D v$ defined by \eqref{defDv} where $v:\X\to \R$ is a potential function. Let $w=-v$.

%First remark that for any  $z\in\{0,1\}^n$ and any  $i< j $, 
%\begin{align*}
%Dw(\sigma_i(z),\sigma_j(z))&= w(\sigma_j(z))+w(\sigma_i(z))-w(z)-w(\sigma_i\sigma_j(z))\\
%&=(2z_i-1)(2z_j-1)\,\partial_{ij}^2(v (z_{\overline{ij}}).
%\end{align*}
%As a consequence it follows that for any $x,y\in \X$ with $d=d(x,y)\geq 2$ and any $t\in[0,1]$, 
%\begin{align*} 
%&D_t w(x,y)\\
%&:=\sum_{ z\in[x,y]}\; \sum_{(i,j)\in[n]\times [n]}(2z_i-1)\1_{\sigma_i(z)\in[x,z]}(2z_j-1)\1_{\sigma_j(z)\in[x,z]}\,\partial_{ij}^2v(z_{\overline{ij}}) \,r(x,\sigma_i(z),\sigma_j(z),y)\\
%&\qquad\qquad\qquad\qquad\qquad\qquad\qquad\qquad\qquad\qquad\qquad\qquad\qquad\qquad\qquad\rho_t^{d-2}(d(x,z)-1)\\
%&=\sum_{k=1}^{d-1}\sum_{ z\in[x,y], d(x,z)=k}\; \sum_{i\in I(x,z)}\sum_{j\in I(z,y)} (2y_i-1)(2x_j-1)\,\partial_{ij}^2v(z_{\overline{ij}})\,\frac{(k-1)!(d-k-1)!}{d!}\,\rho_t^{d-2}(k-1),
%\end{align*}
%where $I(x,z):=\big\{i\in[n]\,|\, x_i\neq z_i\big\}$. Since $I(x,z)$ and $I(z,y)$ are disjoints sets of indices, we also have
%\begin{multline*}D_t w(x,y)=\sum_{k=1}^{d-1}\sum_{ z\in[x,y], d(x,z)=k}\; \sum_{(i,j)\in[n]\times [n]} (2y_i-1)\1_{i\in I(x,z)}(2x_j-1)\1_{j\in I(z,y)}\,\big(Hv(z)-\lambda_{\rm min}(Hv(z))\big)_{ij}\\
%\qquad\qquad\qquad\qquad\qquad\qquad\qquad\qquad\qquad\qquad\qquad\qquad\qquad\frac{(k-1)!(d-k-1)!}{d!}\,\rho_t^{d-2}(k-1).
%\end{multline*}

%For any $n$ by $n$ matrix $\Sigma$, let $\|\Sigma\|_{op}$ denote the operator norm of $\Sigma$.
%Observing that for $z\in[x,y]$ with $d(x,z)=k$,
%\begin{align*}
%&\sum_{(i,j)\in[n]\times [n]} (2y_i-1)\1_{i\in I(x,z)}(2x_j-1)\1_{j\in I(z,y)}\,\big(Hv(z)-\lambda_{\rm min}(Hv(z))\big)_{ij}\\
%&\leq 
%\|Hv(z)-\lambda_{\rm min}(Hv(z))\|_{op} \Big(\sum_{i\in I(x,z)}(2y_i-1)^2\Big)^{1/2}\Big(\sum_{j\in I(z,y)}(2x_j-1)^2\Big)^{1/2}\\
%&\leq \Delta^\infty_\lambda(Hv)\sqrt{k(d-k)}\leq \frac {d\, \Delta^\infty_\lambda(Hv)} 2,
%\end{align*}
%one gets 
%\begin{align*}
%D_t w(x,y)&\leq \frac {d\,\Delta^\infty_\lambda(Hv) } 2\, \sum_{k=1}^{d-1}\sum_{ z\in[x,y], d(x,z)=k}\frac{(k-1)!(d-k-1)!}{d!}\,\rho_t^{d-2}(k-1)\\
%&=\frac {\Delta^\infty_\lambda(Hv) } {2(d-1)}\,\frac {1-\rho_t^d(d)-\rho_t^d(0)}{t(1-t)}=\frac {\Delta^\infty_\lambda(Hv) } {2(d-1)}\,\gamma_t(d). 
%\end{align*}
%For $d=2 $ it provides $D_t w(x,y)\leq \Delta^\infty_\lambda(Hw)$, and for $d\geq 3$,
%Then inequality \eqref{subtile} ensures that 
%\[\int_0^1 D_s w(x,y)\,q_t(s)\,ds\leq \frac{\Delta^\infty_\lambda(Hv)}{2(d-1)}\int_0^1 \gamma_s(d)\,ds\leq  {\Delta^\infty_\lambda(Hv)}\,\frac{\sum_{k=1}^{d-1}\frac1k}{d-1}\leq \Delta^\infty_\lambda(Hv).\]
%This inequality provides the expected result 
%Since $-\D v=\sup\Big\{\int_0^1 D_sw(x,y)\,q_t(s)\,ds\,\Big|\, x,y\in\X\Big\}$, this inequality provides 
%\[\D v\geq -{\Delta^\infty_\lambda(Hv)}\,\frac{\sum_{k=1}^{d-1}\frac1k}{d-1}\geq -\Delta^\infty_\lambda(Hv).\] 
% \end{proof}


%\begin{proof}[Proof of Lemma \ref{Dvcube} (autre preuve plus longue)]

%We want to bound from above  the quantity $\D w$ defined by \eqref{defDv}. 

%First note that for any  $z\in\{0,1\}^n$ and any  $i< j $, 
%\begin{align*}
%Dw(z,\sigma_i\sigma_j(z))&= w(\sigma_i\sigma_j(z))+w(z)-w(\sigma_i(z))-w(\sigma_j(z))\\
%&=(2z_i-1)(2z_j-1)\,\partial_{ij}^2(-w)(z_{\overline{ij}}).
%\end{align*}
%As a consequence it follows that for any $x,y\in \X$ with $d=d(x,y)\geq 2$ and any $t\in[0,1]$, 
%\begin{align*} 
%&D_t w(x,y):=2\sum_{ z\in[x,y]}\;\; \sum_{\{i,j\}\subset[n],(z,\sigma_i\sigma_j(z))\in[x,y]}(2z_i-1)(2z_j-1)\,\partial_{ij}^2(-w)(z_{\overline{ij}}) \,r(x,z,\sigma_i\sigma_j(z),y)\rho_t^{d-2}(d(x,z)),\\
%&=2\sum_{k=0}^{d-2}\sum_{z\in [x,y], d(x,z)=k} \Big(\sum_{\{i,j\}\subset [n], (z,\sigma_i\sigma_j(z))\in[x,y]} (2z_i-1)(2z_j-1)\,\partial_{ij}^2(-w)(z_{\overline{ij}})\Big) \frac{k!(d-2-k)!}{d!} \,\rho_t^{d-2}(k)\\
%&=\frac{1}{d(d-1)}\sum_{k=0}^{d-2} \ell_{t}^{x,y}(k),
%\end{align*}
%with for $k\in \{0,\ldots, d-2\}$,
%\[\ell_{t}^{x,y}(k):=2 \sum_{z\in [x,y], d(x,z)=k}\Big( \sum_{\{i,j\}\subset [n], (z,\sigma_i\sigma_j(z))\in[x,y]} (2z_i-1)(2z_j-1)\,\partial_{ij}^2(-w)(z_{\overline{ij}})\Big) t^{k}(1-t)^{d-k-2}.\]

%If $d(x,y)=2$ then $\{i\in[n]\,|\,x_i\neq y_i\}=\{i_0,j_0\}$ and $(z,\sigma_i\sigma_j(z))\in[x,y]$ if and only if $\{i,j\}=\{i_0,j_0\}$ and $z=x,\sigma_i\sigma_j(z)=y$. In that case, one has 
%\begin{align*} 
%2 \sum_{\{i,j\}\subset [n], (z,\sigma_i\sigma_j(z))\in[x,y]} (2z_i-1)(2z_j-1)\,\partial_{ij}^2(-w)(z_{\overline{ij}})&= 2(2x_{i_0}-1)(2x_{j_0}-1)\,\partial_{i_0j_0}^2(-w)(x_{\overline{i_0j_0}})\\
%&\leq 2| \partial_{i_0j_0}^2(-w)(x_{\overline{i_0j_0}})| \leq {\|H(-w)(x)\|_{op}} \leq \|H(-w)\|_{op,\infty}.
%\end{align*}
%and therefore $\ell_t^{x,y}(0)\leq \|H(-w)\|_{op,\infty}$ and $D_t w(x,y)\leq \|H(-w)\|_{op,\infty}/2 $.

%If $d(x,y)=3$ then $\{i\in[n]\,|\,x_i\neq y_i\}=\{i_0,j_0,k_0\}$ and $(z,\sigma_i\sigma_j(z))\in[x,y]$ implies $d(x,z)=1$ or $d(x,z)=0$. Assume $d(x,z)=1$ and $z=\sigma_{i_0}(x)$. In that case $(z,\sigma_i\sigma_j(z))\in[x,y]$ if and only if $\{i,j\}=\{j_0,k_0\}$ and $z=\sigma_{i_0}(x),\sigma_i\sigma_j(z)=y$ and therefore 
%\begin{align*} 
%2 \sum_{\{i,j\}\subset [n], (z,\sigma_i\sigma_j(z))\in[x,y]} (2z_i-1)(2z_j-1)\,\partial_{ij}^2(-w)(z_{\overline{ij}})&= 2(2x_{j_0}-1)(2x_{k_0}-1)\,\partial_{j_0k_0}^2(-w)(\sigma_{i_0}(x)_{\overline{j_0k_0}})\\
%&\leq 2| \partial_{j_0k_0}^2(-w)(\sigma_{i_0}(x)_{\overline{j_0k_0}})| \leq {\|H(-w)(\sigma_{i_0}(x))\|_{op}} \leq \|H(-w)\|_{op,\infty},
%\end{align*}
%and $\ell_t(1)\leq 3t \|H(-w)\|_{op,\infty}$.
%Now assume that   $d(x,z)=0$, $z=x$ 
%then
%\begin{align*}
%\ell_t(0)&\leq 2\Big(| \partial_{i_0j_0}^2(-w)(x_{\overline{j_0k_0}})|+| \partial_{j_0k_0}^2(-w)(x_{\overline{i_0k_0}})|+| \partial_{j_0k_0}^2(-w)(x_{\overline{j_0k_0}})|\Big)(1-t)\\
%&\leq 3(1-t)\|H(-w)\|_{op,\infty},
%\end{align*}
%This finally provides 
%\[D_tw(x,y)\leq \frac16(\ell_t(0)+\ell_t(1))\leq \frac{\|H(-w)\|_{op,\infty}}{2}.\] 

%Assume now that  $d(x,y)\geq 4$. Observing that 
%\[2 \sum_{\{i,j\}\subset [n], (z,\sigma_i\sigma_j(z))\in[x,y]} (2z_i-1)(2z_j-1)\,\partial_{ij}^2(-w)(z_{\overline{ij}})\leq \|H(-w)(z)\|_{op}\sum_{i\in [n], z_i\neq y_i} (2z_i-1)^2=\|H(-w)(z)\|_{op} d(z,y)\] 
%one gets \[\ell_{t}^{x,y}(k)\leq \frac{\|H(-w)\|_{op,\infty} }{d-(k+1)} \,  \frac{d!}{k!(d-2-k)!} t^{k}(1-t)^{d-k-2}.\] By symmetry, we also have 
%\begin{align*}
%\ell_{t}^{x,y}(k)&=2\sum_{w\in [x,y], d(x,w)=k+2}\Big( \sum_{\{i,j\}\subset [n], (\sigma_i\sigma_j(w),w)\in[x,y]} (2\overline{w_i}-1)(2\overline{w_j}-1)\,\partial_{ij}^2(-w)(w_{\overline{ij}})\Big) t^{k}(1-t)^{d-k-2}\\
%&=2\sum_{w\in [x,y], d(x,w)=k+2}\Big( \sum_{\{i,j\}\subset [n], (\sigma_i\sigma_j(w),w)\in[x,y]} (2{w_i}-1)(2{w_j}-1)\,\partial_{ij}^2(-w)(w_{\overline{ij}})\Big) t^{k}(1-t)^{d-k-2}\\
%&\leq \|H(-w)\|_{op,\infty} (k+2) \frac{d!}{(k+2)!(d-2-k)!} t^k (1-t)^{d-k-2}\\
%&=\frac{\|H(-w)\|_{op,\infty} }{k+1} \,  \frac{d!}{k!(d-2-k)!} \,t^{k}(1-t)^{d-k-2}.
%\end{align*}
%As a consequence, we finally get 
%\[\ell_{t}^{x,y}(k)\leq \|H(-w)\|_{op,\infty}\,\frac{d(d-1)\,\rho_t^{d-2}(k) }{\max\big(k+1,d-(k+1)\big)} \; =\|H(-w)\|_{op,\infty}\,\min\big(k+1,d-(k+1)\big)\,\frac{\rho_t^{d}(k+1) }{t(1-t)},\]
%and therefore
%\begin{equation*}\label{vite}
%D_t w(x,y)\leq\frac{\|H(-w)\|_{op,\infty}}{d(d-1)}\sum_{k=1}^{d-1}\min\big(k,d-k\big)\,\frac{\rho_t^{d}(k) }{t(1-t)}\leq
%\, \frac{\|H(-w)\|_{op,\infty}}{2(d-1)}
%\,\frac{1-\rho_t^{d}(0) -\rho_t^{d}(d)}{t(1-t)}
%\end{equation*}
%Then inequality \eqref{subtile} ensures that 
%\[\int_0^1 D_s v(x,y)\,q_t(s)\,ds\leq \frac{\|H(-w)\|_{op,\infty}}{2(d-1)}\Big(3+2\log\frac{d-2}2\Big)\leq\frac{\|H(-w)\|_{op,\infty}}2.\]
% Since this inequality actually holds  for any $x,y\in \{0,1\}^n$, this yields $\D w\leq $ $\int_0^1 D_w v(x,y)\,q_t(s)\,ds\leq \|H(-w)\|_{op,\infty}/2$
% \end{proof}
%One of the most famous inequality in mathematics is the \textit{arithmetic-geometric mean inequality} which states that for every positive integer $n$ and 
%$x_{1},x_{2},\cdots x_{n}>0$ one has
%\begin{equation*}
%\frac{x_{1}+x_{2}+\cdots x_{n}}{n}\geq \sqrt[n]{x_{1}\cdot x_{2} \cdots x_{n}} \hspace{0.1cm} .
%\end{equation*}

%\begin{definition}
%The \textit{Elementary symmetric polynomials} in $x_{1}, x_{2},...x_{n}$ are defined as 
%\begin{equation*}
%e_{k}(x_{1},x_{2},\cdots ,x_{n}):=\sum_{I\subset \{1,2\cdots n\}, \mid I \mid=k} \prod_{i\in I} x_{i}
%\end{equation*}
%for $1\leq k\leq n$.  In general, %$e_{1}(x_{1},x_{2},\cdots, %x_{n})=x_{1}+x_{2}+\cdots+x_{n}$ and 
%$e_{n}(x_{1},x_{2},\cdots ,x_{n})=x_{1}\cdot x_{2} \cdots x_{n}$. The elementary symmetric polynomials interpolate between the sum of $n$ numbers and the product of $n$ numbers.
%Each $e_{k}(x_{1},x_{2},...x_{n})$ is the sum of %$\binom{n}{k}$ terms. 

%\end{definition}

%\begin{definition}
%The $k-th$ elementary symmetric polynomials of %$x_{1},x_{2} \cdots x_{n}$ are defined as
%\begin{equation*}
%E_{k}(x_{1},x_{2},\cdots x_{n}):=\frac{ e_{k}(x_{1},x_{2},\cdots x_{n})}{\binom{n}{k}} .
%\end{equation*}
%\end{definition}





%The \textit{Mclaurin's inequality} states the following, for every positive $x_{1},x_{2},\cdots x_{n}$
%\begin{equation*}
%E_{1}(x_{1},x_{2},x_{3},...x_{n}) \geq \sqrt{E_{2}(x_{1},x_{2},\cdots x_{n})} \geq \sqrt[3]{E_{3}(x_{1},x_{2},\cdots x_{n})}\geq \cdots \geq \sqrt[n]{E_{n}(x_{1},x_{2},\cdots x_{n})} . 
%\end{equation*}

%\subsection{Ollivier's coarse Ricci curvature}

%Let $G=(\mathcal{X},E)$ be a graph endowed with its graph distance $d$ with a Markov chain determined by $n$. Let $x,y\in \mathcal{X}$ be two distinct vertices. 
%The \textit{Ollivier's coarse Ricci curvature along the edge $\{x,y\}$} is defined as
%\begin{equation*}
%\mathcal{K}_{Oll}(x,y):=1-\frac{W_{1}(n_{x},n_{y})}{d(x,y)}\hspace{0.1cm} , 
%\end{equation*}
%with the markov kernel $n$ defined as follows
%\begin{equation*}
%n_{x}(y)=
%\begin{cases}
%\frac{1}{\textnormal{Deg}(x)} \hspace{0.1cm} \text{if} \hspace{0.1cm} y\sim x \hspace{0.1cm} ,\\
%0 \hspace{0.1cm} \text{otherwise} .
%\end{cases}
%\end{equation*}
%The \textit{Ollivier's coarse Ricci curvature} of $(G,d,n)$ is defined as
%\begin{equation*}
%\mathcal{K}_{Oll}:=\inf_{x,y\in \mathcal{X}} \mathcal{K}_{Oll}(x,y) \hspace{0.1cm} .
%\end{equation*}







%\begin{remark}
% Let us mention that Theorem 3 implies the entropic Bonnet-Myers Theorem. Indeed, let $x_{0}$ and $y_{0}$ be arbitrary vertices in $\mathcal{X}$. %Choosing, the functions f,g,h respectively as 
%$$
%f(x) = \left\{
 %   \begin{array}{ll}
  %      0 & \mbox{if} \hspace{0.1cm} x=x_{0}\\
  %      -\infty & \mbox{otherwise}
  %  \end{array}
%\right.
%$$
%$$
%g(y) = \left\{
 %   \begin{array}{ll}
  %      0 & \mbox{if} \hspace{0.1cm} y=y_{0}\\
   %     -\infty & \mbox{otherwise}
%    \end{array}
%\right.
%$$
%and  
%\begin{equation*}
%h(z)=-\ln(\widehat\nu_{t}^{x_{0}, y_{0}}(z))-\kappa \frac{t(1-t)}{2}d(x_{0},y_{0})(d(x_{0},y_{0})-1),
%\end{equation*}
%and applying Theorem 3 we obtain the desired result. 

%\end{remark}







%%%%%%%%%%%%%%%%%%%%%%%%%%%%%%%%%%%%%%%%%%%%%%%%%%%%%%
%
% COMPLETE VERSION 
%
%%%%%%%%%%%%%%%%%%%%%%%%%%%%%%%%%%%%%%%%%%%%%%%%%%%%%%

%%%%%%%%%%%%%%%%%%%%%%%%%%%%%%%%%%%%%%%%%%%%%%%%%%%%%%





%\begin{proof}
%\mar{Let $x\in \mathcal{X}$, then by the harmonicity of $f$
%\begin{equation*}
%Lf(x)=\sum_{y\in \mathcal{X}}L(x,y)f(y)=0 \hspace{0.1cm} ,
%\end{equation*}
%and by assumption $L(x,y)>0$ if and only if $x\sim y$ therefore there exists $y\sim x$ such that $f(y)\geq f(x)$. Otherwise 
%$Lf(x)=\sum_{y\in \mathcal{X}} L(x,y)f(y)<f(x)\sum_{y\in X} L(x,y)=0 .$
%Inductively there exists $(x_{n})_{n}\in \mathcal{X}$ such that 
%$f(x_{n+1})\geq f(x_{n})$. 
%If we have a Poincar{\'e} inequality as the following one ... and since $\mid \nabla f \mid \in L^{1}(\mathcal{X},m)$
%\begin{equation*}
%Var_{m}(f)\leq (1+C_{2}) \int \mid \nabla f \mid^{2} dm <\infty ,
%\end{equation*}
%where $Var_{m}(f)=\int f^{2}dm-\Big(\int fdm\Big)^{2} $.
%Therefore, $f\in L^{2}(\mathcal{X},m)$, ie 
%\begin{equation*}
%\sum_{x\in \mathcal{X}} f^{2}(x)m(x)<\infty .
%\end{equation*}
%This would imply that $f$ is the zero function.}
%\end{proof}
%\textcolor{blue}{Remarque/Question... peut etre sans aucune importance
%$X_{t}\sim \nu_{t},X_{0}\sim \nu_{0}, X_{1}\sim \nu_{1} $ random variables in the space of of paths.
%$H(\nu_{t}\mid m)=log(\mid \X \mid)-H(X_{t})$ where $H$ is the entropy functional.(je crois que cela marche meme si $\X$ infini mais numerable).
%The displacement inequality can be rewritten as 
%\begin{equation*}
%    H(X_{t})\geq (1-t)H(X_{0})+tH(X_{1})+K\frac{t(1-t)}{2}(E(d(X_{0},X_{1})))^{2} ,
%\end{equation*}
%\begin{equation*}
%    N(X_{t})\geq N(X_{0})^{1-t}N(X_{1})^{t}e^{2t(1-t)E(d(X_{0},X_{1}))^{2}}
%\end{equation*}
%Cela ressemble beaucoup a une entropie power inequality (EPI), 
%$N(X+Y)\geq N(X)+N(Y)$.
%Ceci a peut etre une intuition en termes de theorie de l information %si $\kappa$ est grand on a toute l information et si $\kappa$ tens %tends vers -infini on a aucune information para rapport a $X_{t}$
%}


%%%%%%%%%%%%%%%%%%%%%%%%%%%%%%%%%%%%%%%%%%%




\bibliographystyle{plain}
\bibliography{criteria-curvature}

\end{document}



%%%%%%%%%%%%%%%%%%%%%%%%%%%%%

THEOREME 10 ABANDONNE
%%%%%%%%%%%%%%%%%%%%%%%%%%

\begin{equation}\label{defDv}
    c_t^v(x,y)\geq  d(x,y)(d(x,y)-1)\,\D v\quad\mbox{with}\quad \D v:=\inf\Big\{\int_0^1 D_sv(x,y)\,q_t(s)\,ds\,\Big|\, x,y\in\X\Big\},
\end{equation}
it follows that Theorem \ref{entropicperturb} together with Theorem \ref{thmprinc} applied on the space $(\X,d,m,L)$ provide the following lower bound for the entropic curvature $\kappa^v$ of the space  $(\X,d,m_v,L_v)$, \begin{equation}\label{lowerkappav}\kappa^v\geq \kappa+\D v\geq -2\log K_L+\D v\geq -2\log K_L + \inf_{z,\ttz, d(z,\ttz)=2} Dv(z,\ttz),
\end{equation}
where the last inequality is a consequence of \eqref{mulh} and the identity \[\sum_{(z,\ttz)\in [x,y], d(z,\ttz)=2} L^2(z,\ttz)\, r(x,z,\ttz,y)\, \rho_t^{d(x,y)-2}(d(x,z))=1.\]

If one applies Theorem \ref{thmprinc} on the space $(\X,d,m_v,L_v)$, we also get 
$\kappa^v \geq -2 \log K_{L_v}$,
and therefore one has 
\[\kappa^v\geq \max\big(-2 \log K_{L_v}, -2\log K_L+\D v\big).\]
Observe that since 
\[Dv(z,\ttz)=2 \sum_{\tz\in ]z,\ttz[} \Big(\log \frac{L_v^2(z,\ttz)}{L^2(z,\ttz)}-2\log \frac{L_v(z,\tz)}{L(z,\tz)}\Big) \, \frac{L_v(z,\tz)L_v(\tz, \ttz)}{L_v^2(z,\ttz)},\]
the definition of $K_{L_v}\big(z,S_2(z)\big)$ is also given by 
\begin{align}\label{defKLv}
    K_{L_v}\big(z,S_2(z)\big)&=\sup_\alpha \Biggl\{  \sum_{\ttz\in S_2(z)} e^{-Dv(z,\ttz)/2} L^2(z,\ttz) \prod_{\tz\in  ]z,\ttz[}\left(\frac{\alpha(\tz)}{L(z,\tz)}\right)^{\frac{2L(z,\tz)L(\tz,\ttz)}{L^2(z,\ttz)}}\Biggl\} \\
    \nonumber
    &\leq K\big(z,S_2(z)\big)\, \sup_{z,\ttz, d(z,\ttz)=2} e^{-Dv(z,\ttz)/2}\, . 
\end{align}
and therefore 
\[-2 \log K_{L_v}\geq -2\log K+\inf_{z,\ttz, d(z,\ttz)=2} Dv(z,\ttz).\]
However the lower bounds $-2 \log K_{L_v}$ and $(-2\log K_L+\D v)$ are not comparable in full generality as we will see with examples on the discrete hypercube $\{0,1\}^n$ or on the lattice $\Z^n$ in section \ref{examplesZcube}. 

%This lower bound on the curvature $\kappa_v$ is smaller than the one given by \eqref{lowerkappav} since    
%\[\inf_{z,\ttz, d(z,\ttz)=2} Dv(z,\ttz)\leq \D v.\]
%We will see further  how to take advantage of the lower bounds involving the quantity $\D v$, in  particular on the discrete hypercube graph.


We know that if  $-2 \log K_{L_v}{\geq 0}$ then Theorem \ref{thmprincbis} and Theorem \ref{Thmstructure} provide 
lower bounds for the other types of  entropic curvature of the space $(\X,d,m_v,L_v)$. 
Identically, if $-2\log K_L+\D v\geq 0$ then %, following the proofs of Theorem \ref{thmprincbis} and Theorem \ref{Thmstructure}, 
one gets other entropic curvature  lower bounds  involving $\D v$. 

\begin{theorem}\label{entropicperturbbis}
Let $(\X,d,m,L)$ be a graph space. Given  a potential $v:\X\to \R$, let $(\X,d,m_v,L_v)$ be the graph space defined as above. If for all $z\in \X$, \[K_L(z,S_2(z)) \,e^{-\D v/2}<1,\] then the following holds :
\begin{itemize}
\item The $\widetilde{T}$-entropic curvature $\widetilde{\kappa}^v$ of  $(\X,d,m_v,L_v)$ is   bounded from below by $\widetilde r^v_2:=1-K_Le^{-\D v/2}$.
\item The $W_1$-entropic curvature $\kappa_1^v$ of  $(\X,d,m_v,L_v)$ is lower bounded by $4r^v_1$ where $r^v_1:=\inf_{z\in \X} r^v_1(z)$ and  $r_1^v(z)$ is defined as in \eqref{defr_1} replacing  $K_L(z,W)$ by $K_L(z,W)e^{-\D v/2}$ for any $z\in\X$ and any $W\subset S_2(z)$.

\item For $c=(c_t)_{t\in(0,1)}$ defined by \eqref{defcout2} the  $T_{c}$-entropic curvature $\kappa_{c}^v$ of  $(\X,d,m_v,L_v)$ is lower bounded by $4r^v_2$ where
 $r^v_2:=\inf_{z\in \X} r^v_2(z)$. The quantity
 $\overline{r}^v(z)$ is defined as in \eqref{defr_2} by replacing  $K_L(z,W)$ by $K_L(z,W)e^{-\D v/2}$ for any $z\in\X$ and $W\subset S_2(z)$.
 \item
Assume moreover that $\X$ is a structured graph associated to a set of moves $\Sc$. Let $\widetilde{K}^v(z,\emptyset):=0$, and for $W\subset S_2(z)$, $W\neq \emptyset$, let 
\begin{align}\label{defKtildev}
\widetilde{K}^v(z,W)&:=\sup  \Biggl\{  e^{-\D v/2}\sum_{\ttz\in W} L^2(z,\ttz) \prod_{\sigma\in \Sc_{]z,\ttz[}}\left(\frac{\beta(\sigma)}{\big(L(z,\sigma(z))\big)^2}\right)^{\frac{L(z,\sigma(z))L(\sigma(z),\ttz)}{L^2(z,\ttz)}}\\&-\sum_{(\sigma,\tau)\in \Sc_{]z,W[}^2, \sigma\neq \tau} \sqrt{\beta(\sigma)}\sqrt{\beta(\tau)}
\nonumber
 \Bigg|\,{\beta}=(\beta(\sigma))_{\sigma\in \Sc_{]z,W[}}\in \mathbb{R}_{+}^{\Sc_{]z,W[}},\sum_{\sigma\in \Sc_{]z,W[}} \beta(\sigma)=1 \Biggr\}.
\end{align}
For any $z\in \X$, let us define $\widetilde r_2^v:=\inf_{z\in \X}\widetilde r_2^v(z) $, with
\begin{equation*}
\widetilde r_2^v(z):=1-\widetilde{K}^v(z), \quad \mbox{and}\quad  \widetilde{K}^v(z):=\sup_{W\in S_2(z)} \widetilde{K}^v(z,W).
\end{equation*}
One has, for any $z\in \X$,
\[ 1-K_L(z,S_2(z))e^{-\D v/2}\leq \widetilde r_2^v(z)\leq |S_1(z)|\,\big(1-K_L(z,S_2(z))e^{-\D v/2}\big).\]
 %Assume that  for all $z\in \X$, \[K(z,S_2(z))e^{\D v/2}\leq 1. \]
\begin{enumerate}[label=\roman*.]
\item  If the generator $L$ satisfies condition \eqref{condL}
%If  for any $z\in \X$ and any $\sigma,\tau \in S$   such that  $d(z,\sigma\tau(z))=2$, one has \begin{equation}\label{condLv}
%\tau(z)\in ]z,\sigma\tau(z)[. \end{equation}
then the
$\widetilde{T}_3$-entropic curvature $\widetilde{\kappa}^v_3$ of the space   $(\X,d,m_v,L_v)$ is   bounded from below by $\widetilde r^v_3$.
\item Assume that  the generator $L$ satisfies condition \eqref{condL}, and assume moreover that any move $\sigma\in \Sc$ can be used at most one time along each geodesic. 
Then the above result can be improved replacing the curvature cost $C_t(\widehat \pi)=\widetilde r_2^v\, \widetilde T_2(\widehat\pi)$ in 
%the $C$-displacement convexity property of entropy
\eqref{deplacebis} (with the measure $m_v$) by  the cost $C_t(\widehat \pi)=\widetilde r_2^v\,\widetilde C_t^1(\widehat \pi)$, $t\in (0,1)$, where for any $D\geq1$ (the cost  $\widetilde C_t^D(\widehat \pi)$ is defined in Theorem \ref{Thmstructure}).


Assume that $D=\textnormal{Diam}(\mathcal{X})<\infty$. If some moves $\sigma\in S$ can be used more than one time along  geodesics  the $C$-displacement convexity property of entropy \eqref{deplacebis} (with the measure $m_v$) also holds with the cost $C_t(\widehat \pi)=\widetilde r_2^v\,\widetilde C_t^D(\widehat \pi)$, $t\in (0,1)$.   
\end{enumerate}
\end{itemize}
\end{theorem}
The proof of this Theorem is postponed in Appendix B.

\begin{proof}[Proof of Theorem \ref{entropicperturbbis}] As in the proof of Theorem \ref{thmprinc}, in order to apply Theorem \ref{thmsam21}, given two probability measures $\nu_0$ and $\nu_1$ with bounded support, one considers a finite convex subset $\Cc$ of $\X$ that contains all the balls of radius 2 with center in the finite subset $[\supp(\nu_0),\supp(\nu_1)]$. 
According to equalities \eqref{Hm0mv} and \eqref{automne} and applying  Theorem \ref{thmsam21}  on the space $(\Cc,d,m_{\Cc},L_\Cc)$, one gets 
\begin{align*}
    (1-t)H(\nu_0|m_v)&+tH(\nu_0|m_v)-H(\widehat \nu_t|m_v)\\&=(1-t)H(\nu_0|m)+tH(\nu_0|m)-H(\widehat \nu_t|m)\\
    &\qquad\qquad\qquad\qquad+\int_0^1
    \Big(\sum_{(x,y)\in\X^2}
d(x,y)\big(d(x,y)-1\big)\, D_s v(x,y)\,\widehat \pi(x,y)\Big)q_t(s)\,ds
    %&\geq \int_0^1 \left(\int (H_s+ K_s) d\widehat\nu_s\right)q_t(s) \,ds -\int_0^1
    %\Big(\sum_{(x,y)\in\X^2}
%d(x,y)\big(d(x,y)-1\big)\, D_s v(x,y)\,\widehat \pi(x,y)\Big)q_t(s)\,ds
\\
& \geq \int_0^1 \left(\int \big(H_s+ K_s\big) d\widehat\nu_s+\D v \, T_2(\widehat\pi)\right)q_t(s) \,ds 
\end{align*}
As mentioned in the proof of Theorem \ref{thmprinc} for any $s\in (0,1)$, $T_2(\widehat\pi)=\int  \overline{{\mathbbm B}}_s  \,d \widehat\nu_s =\int  \overline{{\mathbbm A}}_s  \,d \widehat\nu_s$. One may observe that the quantity  $\overline{{\mathbbm A}}_s$ is the same on the space $(\X, d,m,L)$ as on the space $(\X, d,m_v,L_v)$. This is also the case for the quantities $\overline{{\mathbbm B}}_s$, $\overline{A}^2_s$,  $\overline{B}^2_s$ and $\widetilde A_s^2$. It follows that
\[(1-t)H(\nu_0|m_v)+tH(\nu_1|m_v)-H(\widehat \nu_t|m_v)\geq 
\int_0^1 \left(\int \Big(H_s+\frac{\D v}2\, \overline{{\mathbbm A}}_s+ K_s+\frac{\D v}2\, \overline{{\mathbbm B}}_s\Big) d\widehat\nu_s\right)q_t(s) \,ds\]
As in the proof of Theorem \ref{thmprincbis},  entropic curvature lower bounds will follows by bounding from bellow  $H_t+\frac{\D v}2\,\overline{{\mathbbm A}}_t$ and similarly $K_t+\frac{\D v}2 \,\overline{{\mathbbm B}}_t$ for any $t\in(0,1)$.
According to \eqref{epuise} and \eqref{epuisebis}  used on the space $(\X, d,m,L)$ \eqref{epuisebis}, one has for any $z\in \widehat Z$, 
\begin{align*}
 H_t(z)+\frac{\D v}2\,\overline{{\mathbbm A}}_t(z)&\geq \overline{A}^2_t(z)+ \rho\Big(K\big(z,\V_{_\leftarrow}(z)\big)\, \overline{A}_t^2(z), \overline{\mathbb A}_t(z)\Big)+\frac{\D v}2\,\overline{{\mathbbm A}}_t(z)\\
 &=\overline{A}^2_t(z)+\rho\Big(K\big(z,\V_{_\leftarrow}(z)\big)e^{-\D v/2}\, \overline{A}_t^2(z), \overline{\mathbb A}_t(z)\Big),
 \end{align*}
and similarly
\[K_t(z)+\frac{\D v}2\,\overline{{\mathbbm B}}_t(z)\geq \overline{B}^2_t(z)+\rho\Big(K\big(z,\V_{_\leftarrow}(z)\big)e^{-\D v/2}\, \overline{B}_t^2(z), \overline{\mathbb B}_t(z)\Big).\]
The last two inequalities provide exactly the lower bounds of \eqref{epuise} and \eqref{epuisebis} replacing  
$K(z,W)$ by $K(z,W)\,e^{-\D v/2}$ for  $W\subset S_2(z)$. Then the proof of the three first items of Theorem \ref{entropicperturbbis} is the same as the one of Theorem \ref{thmprincbis}.

In the last item  of Theorem \ref{thmprincbis}, $\X$ is a structured graph. 
According to \eqref{epuise1}  used on the space $(\Cc, d,m_\Cc,L_\Cc)$  one has for any $z\in \widehat Z$, 
\begin{align*}
 &H_t(z)+\frac{\D v}2\, \overline{{\mathbbm A}}_t(z)\\&\geq \widetilde A_t^2(z)\left[1+\sum_{\sigma,\tau\in \Sc_{]z,\V_{_\leftarrow}(z)[},\sigma\neq \tau} \sqrt{\beta(\sigma,z)}\, \sqrt{\beta(\tau,z)}\right]+\frac{\D v}2\overline{{\mathbbm A}}_t(z)\\
\nonumber&\qquad+ \rho\Bigg(\widetilde A_t^2(z) \sum_{ \ttz\in \V_{_\leftarrow}(z)} L^2(z,\ttz)\prod_{\sigma \in \Sc_{]z,\ttz[}} \left(\frac{\beta(\sigma,z)}{\big(L(z,\sigma(z))\big)^2}\right)^{\frac{L(z,\sigma(z))L(\sigma(z), \ttz)}{ L^2(z,\ttz)}},\overline{\mathbb A}_t(z)\Bigg)\\
 &=\widetilde A_t^2(z)\left[1+\sum_{\sigma,\tau\in \Sc_{]z,\V_{_\leftarrow}(z)[},\sigma\neq \tau} \sqrt{\beta(\sigma,z)}\, \sqrt{\beta(\tau,z)}\right]\\
\nonumber&\qquad+ \rho\Bigg(\widetilde A_t^2(z) e^{-\D v/2} \sum_{ \ttz\in \V_{_\leftarrow}(z)} L^2(z,\ttz)\prod_{\sigma \in \Sc_{]z,\ttz[}} \left(\frac{\beta(\sigma,z)}{\big(L(z,\sigma(z))\big)^2}\right)^{\frac{L(z,\sigma(z))L(\sigma(z), \ttz)}{ L^2(z,\ttz)}},\overline{\mathbb A}_t(z)\Bigg)\\
&\geq \widetilde A_t^2(z)\Biggl[1+\sum_{\sigma,\tau\in \Sc_{]z,\V_{_\leftarrow}(z)[},\sigma\neq \tau} \sqrt{\beta(\sigma,z)}\, \sqrt{\beta(\tau,z)}\\
&\qquad\qquad\qquad-e^{-\D v/2} \sum_{ \ttz\in \V_{_\leftarrow}(z)} L^2(z,\ttz)\prod_{\sigma \in \Sc_{]z,\ttz[}} \left(\frac{\beta(\sigma,z)}{\big(L(z,\sigma(z))\big)^2}\right)^{\frac{L(z,\sigma(z))L(\sigma(z), \ttz)}{ L^2(z,\ttz)}}\Biggl]\\
&\geq \big(1-\widetilde{K}^v(z,\V_{_\leftarrow}(z))\big)\widetilde A_t^2(z)
 \end{align*}
 This inequality provides the same type of lower bound as in  inequality \eqref{burnout}. Then the end of the proof of the second part of Theorem \ref{entropicperturbbis} exactly follows the one of Theorem \ref{Thmstructure}.
\end{proof}

%%%%%%%%%%%%%%%%%%%%%%%%%%%%
Conseequences du THEOREME 10 ABANDONNE SUR LE CUBE
%%%%%%%%%%%%%%%%%%%%%


\begin{proposition}\label{prophypercube}
Let $\X=\{0,1\}^n$ be the discrete hypercube and let $m_v$ denotes the measure with density $e^{-v}$ with respect to the counting measure $m_0$, with $v:\X\to \R$. This measure is reversible with respect to the generator $L_v$ given by  
$L_v(x,y)=e^{\frac12(v(x)-v(y))}L_0(x,y)$ for any $x\neq y$.  Let 
\[r_n(v):= 1-(1-1/n)e^{-\lambda_{\rm min}^\infty(Hv)/2}.\] 
The entropic curvature $\kappa^v$ 
 of the space $(\X,d,m_v,L_v)$ is bounded from below by \[-2\log(1-r_n(v))=-2\log(1-1/n)+\lambda_{\rm min}^\infty(Hv).\] If moreover 
 \begin{equation}\label{condising}
-\lambda_{\rm min}^\infty(Hv) < -2\log(1-1/n),
\end{equation}
or equivalently $r_n(v)>0$, then using the notations of Theorem
\ref{entropicperturbbis}, entropic curvatures of the space $(\X,d,m_v,L_v)$ are bounded from below as follows 
\[\widetilde{\kappa}_2^v\geq r_n(v), \quad\kappa_1^v\geq 4 r_n(v), \quad \kappa_{c}^v\geq \frac4n+2\lambda_{\rm min}^\infty(Hv) , \quad \widetilde \kappa^v_3\geq n\,r_n(v),\]
where we replace condition \eqref{condising} by $-\lambda_{\rm min}^\infty(Hv)<2/n$ to get the lower bound on $\kappa_{c}^v$.
Moreover the $C$-displacement convexity property of entropy
\eqref{deplacebis}  holds with the measure $m_v$ and the cost 
$C_t(\widehat \pi)=n\,c_n(v)\,\widetilde C_t^1(\widehat \pi)$, $t\in (0,1)$. 
\end{proposition}
The lower bounds on $\kappa^v$ and $\widetilde{\kappa}_2^v$ are easy consequences of  $\D v\geq \lambda_{\rm min}^\infty(Hv)$. 

Adapting the  arguments we have used at the beginning of this section to  estimate $r_1,\overline{r}$ on the space $(\X,d,m_0,L_0)$,   one obtains the lower bounds on $\kappa_1^v$ and $\kappa_{c}^v$.  One just applies the inequality $g(|V|_+)+g(|V_-|)\leq g(|V|_++|V_-|)\leq g(n)$ with $g(h):=\frac h{h-(h-1)e^{-\D v/2}}$, $h\geq 1$  to get the lower bound on $\kappa_1^v$, and with $g(h)=(1+\D v/2)^{-1}\frac h{h-(h-1)(1+\D v/2)^{-1}}$, $h\geq 1$  to get the lower bound on $\kappa_c^v$.

The proof of the lower bound on $\widetilde \kappa^v_3$ is a consequence of the identity $\sup_\beta\sum_{(i,j)\in A^1\times A^1, i\neq j} \sqrt{\beta_i}\sqrt{\beta_j}=|A^1|-1$. The details of all these proofs are left to the reader.



{\bf Comments:}
\begin{itemize}
\item As an example, let $v$ be the potential  defined by 
\begin{equation}\label{vacances}
v(z)=\sum_{i\in [n]} u_i z_i +\frac{\beta}{2} \sum_{i,j\in [n], i\neq j} \Sigma_{ij}\, z_iz_j,
\end{equation}
where $u=(u_1, \ldots, u_n)\in \R^n$ and  $\Sigma=(\Sigma_{ij})_{i,j\in [n]}$ is a symmetric matrix of real coefficients  with $0$ diagonal.  Since for any $z\in \X$, $Hv(z)= \beta\,\Sigma$, one has $\lambda_{\rm min}^\infty(Hv)={\beta} \,\lambda_{\min}(\Sigma)$.

If $\beta=0$, then $r_n(v)=1/n$ and $\mu_v$  is the product of Bernoulli measures with parameter $p_i=\frac{e^{u_i}}{1+e^{u_i}}$ and  all the lower bounds we get are the same as for the uniform probability measure $\mu_0$. 

If $-\beta \lambda_{\min}(\Sigma)\leq 2/n$, then inequality \eqref{condising} is satisfied. 
And the inequality $1-\Big(1-\frac1n\Big)e^{\alpha/n}\geq \frac{1-\alpha}n$ for $n\geq 2$ and $\alpha\in [0,1]$, implies \[r_n(v)\geq \frac1n + \frac{\beta \lambda_{\min}(\Sigma)}2.\]  


\item Applying Theorem \ref{PL}, Corollary \ref{Transport}, Theorem \ref{Logsob} and Theorem \ref{Logsobbis} also provide new functional inequalities for the measure $m_v$ or $\mu_v$ on the discrete cube.
In particular if $\lambda_{\rm min}^\infty(Hv) >-2/n\geq 2\log(1-1/n)$, then \[\widetilde\kappa^v_3\geq n\,r_n(v)\geq 1+\frac{n\lambda_{\rm min}^\infty(Hv)}2>0.\] Therefore  according to  Theorem \ref{Logsobbis}, the measure $\mu_v$ satisfies the following modified 
logarithmic Sobolev inequality, for any positive function $f$ on $\{0,1\}^n$,
\begin{equation*}
{\rm Ent}_{\mu_v}(f)\leq  \int \sum_{i\in[n]} \frac{ \widetilde\kappa^v_3 }2 \, h^*\left(\frac{2}{\widetilde\kappa^v_3} [\partial_{\sigma_i} \log f]_-\right)  f\,d\mu_v\leq \frac 1{\widetilde \kappa^v_3} \int \sum_{i\in[n]} {[\partial_{\sigma_i} \log f]_-[\partial_{\sigma_i} f]_-} \,d\mu,
\end{equation*}
and also    the following Poincaré  inequality, for any real bounded function $g:\X\to \R$, 
\[{\rm Var}_{\mu_v}(g)\leq \frac{1}{2\widetilde\kappa^v_3} 
\int \sum_{i\in[n]}(\partial_{\sigma_i} g)^2 d\mu_v
.\]
By a simple change of variable these results can be transposed on the set $\{-1,1\}^n$ instead of $\{0,1\}^n$, in order to compare this result with the one of  Bauerschmidt-Bodineau \cite{BB19} and Eldan-Koehler-Zeitouni \cite{EKZ22}.   Their condition lies on the difference between the largest and the smallest eigenvalue of the symmetric matrix $\Sigma$, namely $\lambda_{\max}(\Sigma)-\lambda_{\min}(\Sigma)<1$. Our result depends only on the smallest eigenvalue but we loose a factor $n$.

\end{itemize}


%%%%%%%%%%%%%%%%%%%%%%%
Autour du Lemme 3 avec Dv abandonné sur Z^n
%%%%%%%%%%%%%%%%%%%%%%%%%%%%%
For any $z,w\in \Z^n$, let $z\wedge w$ denote the vector of $\Z^n$ with coordinates $\min(z_i,w_i)$, $i\in [n]$.
Given $\varepsilon \in \{-1,+1\}^n$
let $H^\varepsilon v(z)$ denote the $n$ by $n$ symmetric matrix   given by 
\[(H^\varepsilon v(z))_{i,i}:=\partial^2_{ii}v(z),\]
and for $\{i,j\}\subset [n]$,
\[(H^\varepsilon v(z))_{i,j}:=\partial^2_{ij}v(z\wedge(z+\varepsilon_ie_i+\varepsilon_j e_j) ).\]
As an example, if  the potential $v$ is given by the sum of a quadratic and a linear form,
\begin{equation}\label{vquad}
    v(z):=\sum_{i\in [n]} u_i z_i +\frac{1}{2} \sum_{(i,j)\in [n]^2} \Sigma_{ij}\, z_iz_j,\quad z\in \Z^n,
\end{equation}
with $u=(u_1, \ldots, u_n)\in \R^n$ and  $\Sigma=(\Sigma_{ij})_{i,j\in [n]}$  a symmetric matrix of real coefficients, then easy computations give  $H^\varepsilon v(z)=\Sigma$ for any $z\in \Z^n$ and any $\varepsilon \in \{-1,+1\}^n$.

Given $z\in \mathbb{Z}^{n}$, one defines 
\[\lambda_{\min}(H^\bullet v(z)):=\min_{\varepsilon\in\{-1,1\}^n} \lambda_{\min}(H^\varepsilon v(z)),\]
%where $\lambda_{\min}(\Sigma)$ denotes the smallest eigenvalue of any real symmetric matrix $\Sigma$ \textcolor{blue}{Peut etre il faudrait changer car $\Sigma$ fait penser a la forme quadrartique ou non ?}
and  
\[\lambda_{\min}^\infty(H^\bullet v):=\inf_{z\in \Z^n} \lambda_{\min}(H^\bullet v(z)).\]
Let 
also 
\[\delta_{ii}^\infty(H^\bullet v):=\sup_{z\in \Z^n} \max_{\varepsilon\in\{-1,1\}^n} \Big[\partial^2_{ii}v(z)-\lambda_{\min}\big(H^\varepsilon v(z)\big)\Big].\] 

Since   the space $(\Z^n,d,L_0,m_0)$ has zero entropic curvature, by Theorem \ref{entropicperturb}, the relative entropy  $ H(\cdot|m_v)$ satisfies the $C^v$-displacement convexity property \eqref{deplacebis} 
%along  Schr\"odinger bridges at zero temperature of the space $(\Z^n,d,m_0,L_0)$, 
with for any $t\in (0,1)$,
\[C^v_t(\widehat \pi)=\iint c_t^v(x,y)\, d\widehat \pi(x,y)=\iint d(x,y)\big(d(x,y)-1\big) \Big(\int_0^1 D_s v(x,y) q_t(s)\, ds\Big)  d\widehat \pi(x,y) \geq \D v\, T_2(\widehat{\pi}).\]
Moreover, according to the comments of section \ref{sectionpertu}, setting 
\[L_v(x,y)=e^{\frac12(v(x)-v(y))} L_0(x,y),\qquad x,y\in \Cc, x\neq y,\]
 the entropic curvature $\kappa^v$ of the space $(\Z^n,d,L_v,m_{v})$ is bounded from below by $\D v$.
 As for the discrete hypercube, let us  present  estimates of $D_t v(x,y)$ for $x,y\in \Z^n$.
\begin{lemma}\label{LemDvZn}
Given a bounded potential $v:\Z^n\to \R$, one has  for any $x,y\in \Z^n$ and any $t\in(0,1)$ 
\[D_tv(x,y) \geq \frac{\lambda_{\min}^\infty(H^\bullet v)}{d(d-1)}\sum_{i=1}^n  d_i(d_i-1) -\frac{\alpha(t,d)}{d(d-1)} \sum_{i=1}^n \delta_{ii}^\infty(H^\bullet v)\,d_i,\]
where $d_i:=|y_i-x_i|$,  $d:=\sum_{i\in[n]} d_i$ and 
\[\alpha(t,d):= \min\Big(\frac{1-t^{d-1}}{1-t}\,,\, \frac{1-(1-t)^{d-1}}{t}\Big)\in[1, 2].\]
\end{lemma}
Let us discuss the sign of the lower bound given by Lemma \ref{LemDvZn} that can be also rewritten  as 
\[\frac{\lambda_{\min}^\infty(H^\bullet v)}{d(d-1)}\sum_{i=1}^n  d_i\Big(d_i-1-\frac{\alpha(t,d)\delta_{ii}^\infty(H^\bullet v)}{{\lambda_{\min}^\infty(H^\bullet v)}} \Big).\]
Assume that $\lambda_{\min}^\infty(H^\bullet v)>0$. This lower bound is non negative as soon as for any $i\in [n]$, $d_i\geq 2$ and 
\[\frac{\alpha(t,d)\delta_{ii}^\infty(H^\bullet v)}{{\lambda_{\min}^\infty(H^\bullet v)}}<1.\]
 For example, as the potential $v$ is given by \eqref{vquad}, this last condition holds if $\lambda_{\max}(\Sigma)/\lambda_{\min}(\Sigma)<3/2$ where $\lambda_{\max}(\Sigma)$ is the maximum eigenvalue of $\Sigma$. Indeed, one has 
\[\frac{\alpha(t,d)\delta_{ii}^\infty(H^\bullet v)}{{\lambda_{\min}^\infty(H^\bullet v)}}\leq 2\Big(\frac{\lambda_{\max}^\infty(H^\bullet v)}{\lambda_{\min}^\infty(H^\bullet v)}-1\Big),\]
where $\lambda_{\max}^\infty(H^\bullet v):=-\lambda_{\min}^\infty(-H^\bullet v)$.

In any case the  lower bound given by Lemma \ref{LemDvZn} is non positive if for all $i\in[n]$, $d_i\leq 1$, and it can not be improved for small values of $d_i$. Indeed, assume that the potential   $v$ is given by \eqref{vquad} with 
$\Sigma_{i_0,j_0}\neq 0$ for some $i_0,j_0\in[n]$, $i_0\neq j_0$. By choosing $y=x+e_{i_0}+e_{j_0}$ one has $d_{i_0}=d_{j_0}=1$ and  $D_tv(x,y)= Dv(x,x+e_{i_0}+e_{j_0})= \Sigma_{i_0,j_0}$, and choosing $y=x+e_{i_0}-e_{j_0}$ one has $d_{i_0}=d_{j_0}=1$ and $D_tv(x,x+e_{i_0}+e_{j_0})= Dv(x,x+e_{i_0}-e_{j_0})=- \Sigma_{i_0,j_0}$. Therefore   $\D v<0$ as soon as there exists a coefficient $\Sigma_{i_0,j_0}\neq 0$. We learn from this example that we can not expect to derive  a non-negative lower bound on the entropic curvature $\kappa^v$ of the space  $(\X,d,L_v,m_{v})$ without strong restrictions on $v$ from the estimate $\kappa^v\geq \D v$, as we do on the discrete hypercube.

\begin{proof}[Proof of Lemma \ref{LemDvZn}] Let $x,y$ be two fixed point of $\Z^n$. Given a potential $v:\Z^n\to \R$, we want to bound from below the quantity 
\[D_tv(x,y):=\sum_{(z,\ttz)\in [x,y], d(z,\ttz)=2} Dv(z,\ttz) \, L^2(z,\ttz)\, r(x,z,\ttz,y)\, \rho_t^{d(x,y)-2}(d(x,z)).\]
Let ${{\epsilon}}$ denotes the vector with coordinates $\varepsilon_i\in\{-1,0,1\}$, $i\in [n]$ defined by $\varepsilon_i=1$ if $y_i>x_i$,  $\varepsilon_i=-1$ if  $x_i>y_i$, and $\varepsilon_i=0$ if $x_i=y_i$.  One also denotes $d:=d(x,y)=d_1+\ldots+d_n$ and $\mathbb d=(d_1,\ldots,d_n)$ with $d_i=|y_i-x_i|$, and similarly for any $z,w\in\Z$,
${\mathbb d}(z,w):=\big(|w_1-z_1|,\ldots,|w_n-z_n|)$.
For $\mathbb{k}\in \N^n$, we write $\mathbb{k}\leq \mathbb{d}$ if $k_i\leq d_i$ for all $i\in[n]$.
Let $(e_1,\ldots,e_n)$ denote the canonical bases of $\R^n$.
Observing that $(z,\ttz)\in [x,y]$ with $d(z,\ttz)=2$ if and only if there exists $i\in [n]$ such that $\ttz=z+2\varepsilon_i e_i$, or there exists $\{i,j\}\subset [n]$ such that $\ttz=z+\varepsilon_i e_i+\varepsilon_j e_j$, one gets 
\begin{align*}
    D_tv(x,y)&=\sum_{k=0}^{d-2}\quad \sum_{\mathbb{k}=(k_1,\ldots,k_n)\in \N^n,\mathbb{k}\leq \mathbb{d}}
    \1_{k=k_1+\cdots+k_n} \1_{z=x+{\mathbb{k}}} \\
    &\qquad\quad \Bigg(\sum_{\{i,j\}\subset[n], k_i\leq d_i-1, k_j\leq d_j-1}Dv(z,z+\varepsilon_i e_i+\varepsilon_j e_j)\, 2\, r(x,z,z+\varepsilon_i e_i+\varepsilon_j e_j,y)\\ 
    &\qquad\qquad\qquad+\sum_{i\in[n], k_i\leq d_i-2}Dv(z,z+2\varepsilon_i e_i)\,  r(x,z,z+2\varepsilon_i e_i,y)\Bigg)\rho_t^{d-2}(k)
\end{align*}
%For any $\mathbb{k}\in\N^n$ and $k:=k_1+\cdots+k_n$, recall that  $\binom{k}{\mathbb k}$ denote the multinomial coefficient, $\binom{k}{\mathbb k}:=\frac {k!}{k_1!\ldots k_n!}$. 

Recall that  for any $z,w\in \Z^n$, $L_0^{d(z,w)}(z,w)=\binom{d(z,w)}{\mathbb{d}(z,w)}$, where $\binom{d(z,w)}{\mathbb{d}(z,w)}$ denotes the multinomial coefficient. One easily checks that for $z=x+\mathbb{k}$ with $k=k_1+\cdots+k_n$ with 
$k_i\leq d_i-1, k_j\leq d_j-1$, one has 
\[r(x,z,z+\varepsilon_i e_i+\varepsilon_j e_j,y)=\frac{(d_i-k_i)(d_j-k_j)}{(d-k)(d-k-1)}\,r(x,z,z,y),\]
and 
\begin{equation}\label{DvZn}
Dv(z,z+\varepsilon_i e_i+\varepsilon_j e_j)=\varepsilon_i\varepsilon_j \partial_{ij}v(z\wedge(z+\varepsilon_i e_i+\varepsilon_j e_j))=\varepsilon_i\varepsilon_j (H^\epsilon v(z))_{ij}.
\end{equation}
Identically, one checks that  for $k_i\leq d_i-2$,
\[r(x,z,z+2\varepsilon_i e_i,y)=\frac{(d_i-k_i)(d_i-k_i-1)}{(d-k)(d-k-1)}\,r(x,z,z,y),\]
and 
$Dv(z,z+2\varepsilon_i e_i)= \partial_{ii} v(z)$.
It follows that 
%\begin{align*}
\[    D_tv(x,y)=\sum_{k=0}^{d-2}\quad \sum_{\mathbb{k}=(k_1,\ldots,k_n)\in \N^n,\mathbb{k}\leq \mathbb{d}}
    \1_{k=k_1+\cdots+k_n} \1_{z=x+{\mathbb{k}}} \,r(x,z,z,y)\frac{\rho_t^{d-2}(k)}{(d-k)(d-k-1)} \,\ell_t(\mathbb k)\]
with    
   \begin{align*} 
    \ell_t(\mathbb k)&:=
    \sum_{(i,j)\in[n]\times [n], i\neq j} (d_i-k_i)(d_j-k_j)\varepsilon_i\varepsilon_j (H^\epsilon v(z))_{ij}  +\sum_{i\in[n]}\partial_{ii} v(z)\,  (d_i-k_i)(d_i-k_i-1)
    \\
    &\geq 
    \lambda_{\min}\big(H^\epsilon v(z)\big) \Big(\sum_{i\in [n]}  (d_i-k_i)^2\Big)  -\sum_{i\in[n]}\partial_{ii} v(z)\,  (d_i-k_i)\\
    &\geq \lambda_{\min}^\infty\big(H^\bullet v\big) \Big(\sum_{i\in [n]}  (d_i-k_i)(d_i-k_i-1)\Big)  - \sum_{i\in[n]} \delta_{ii}^\infty\big(H^\bullet v\big)(d_i-k_i).\\
\end{align*}
Since for $z=x+\mathbb{k}$ with $k=k_1+\cdots+k_n$,
\[r(x,z,z,y)\frac{\rho_t^{d-2}(k)}{(d-k)(d-k-1)}=\frac{\rho_t^{d_1}(k_1)\cdots \rho_t^{d_n}(k_n)}{d(d-1)(1-t)^2}, \]
one gets  
\begin{align*}
&\sum_{k=0}^{d-2}\quad \sum_{\mathbb{k}=(k_1,\ldots,k_n)\in \N^n,\mathbb{k}\leq \mathbb{d}}
    \1_{k=k_1+\cdots+k_n} \1_{z=x+{\mathbb{k}}} (d_1-k_1)(d_1-k_1-1)\,r(x,z,z,y)\frac{\rho_t^{d-2}(k)}{(d-k)(d-k-1)}\\
    &=\sum_{k_1=0}^{d_1-2}\sum_{k_2=0}^{d_2}\cdots \sum_{k_n=0}^{d_n} \Big(\sum_{k=0}^{d-2} \1_{k=k_1+\cdots+k_n} \Big) (d_1-k_1)(d_1-k_1-1)\,\frac{\rho_t^{d_1}(k_1)\cdots \rho_t^{d_n}(k_n)}{d(d-1)(1-t)^2}\\
    &=\sum_{k_1=0}^{d_1-2} (d_1-k_1)(d_1-k_1-1) \,\frac{\rho_t^{d_1}(k_1)}{d(d-1)(1-t)^2}=\frac{d_1(d_1-1)}{d(d-1)}.
\end{align*}
and similarly, 
\begin{align*}
&\sum_{k=0}^{d-2}\quad \sum_{\mathbb{k}=(k_1,\ldots,k_n)\in \N^n,\mathbb{k}\leq \mathbb{d}}
    \1_{k=k_1+\cdots+k_n} \1_{z=x+{\mathbb{k}}} (d_1-k_1)\,r(x,z,z,y)\frac{\rho_t^{d-2}(k)}{(d-k)(d-k-1)}\\
    &=\sum_{k_1=0}^{d_1-1}\sum_{k_2=0}^{d_2}\cdots \sum_{k_n=0}^{d_n} \Big(\sum_{k=0}^{d-2} \1_{k=k_1+\cdots+k_n} \Big) (d_1-k_1)\,\frac{\rho_t^{d_1}(k_1)\cdots \rho_t^{d_n}(k_n)}{d(d-1)(1-t)^2}\\
    &=\sum_{k_1=0}^{d_1-1} \sum_{k_2=0}^{d_2}\cdots \sum_{k_n=0}^{d_n} \big(1-\1_{k_1+\cdots+k_n=d-1}\big)(d_1-k_1)\,\frac{\rho_t^{d_1}(k_1)\cdots \rho_t^{d_n}(k_n)}{d(d-1)(1-t)^2}\\
    &=\Big(\sum_{k_1=0}^{d_1-1} (d_1-k_1)\,\frac{\rho_t^{d_1}(k_1)}{d(d-1)(1-t)^2}\Big) -\frac{\rho_t^{d_1}(d_1-1)\rho_t^{d_2}(d_2)\cdots \rho_t^{d_n}(d_n)}{d(d-1)(1-t)^2}=\frac{d_1(1-t^{d-1})}{d(d-1)(1-t)}.
\end{align*}
These results also holds replacing $d_1$ by $d_i$, $i\in[n]$ and  provide 
\[D_tv(x,y)\geq \frac{1}{d(d-1)}\sum_{i=1}^n \Big[ \lambda_{\min}^\infty(H^\bullet v) d_i(d_i-1) -\delta_{ii}^\infty(H^\bullet v)\,\frac{1-t^{d-1}}{1-t}\,d_i\Big].\]
By symmetry, since $D_tv(x,y)=D_tv(y,x)$, the same lower bound holds replacing $t$ by $1-t$. The proof of Lemma \ref{LemDvZn} ends  by optimizing over $t\in(0,1)$ and observing that $\delta_{ii}^\infty(H^\bullet v)\geq 0$.
\end{proof}

%%%%%%%%%%%%%%%%%%%%%%%%%
ANCIENNE PREUVE DES THEOREMES PRINCIPAUX AVEC HYPOTHESE SUPLEMENTAIRE SUR L

%%%%%%%%%%%%%%%%%%%%%%%%%%%%%%%%%%%%%

\begin{proof}[Proof of Theorem \ref{thmprinc}]
Theorem \ref{thmprinc} is a consequence of Lemma 3.1 and Theorem  3.5 of \cite{Sam21}. 
To avoid repetitions and for the convenience of the reader, we do not enter the details of these results and just recall the needed notations of \cite{Sam21}. 

Let $\nu_0$ and $\nu_1$ be two probability measures on $\X$ with   bounded support and let $\widehat \pi\in\Pi(\nu_0,\nu_1)$ be the $W_1$-optimal coupling that appears in the definition \eqref{defhatnut} of the bridge $(\widehat \nu_t)_{t\in[0,1]}$. 
For any $t\in (0,1)$, the support of  $\widehat  \nu_t$, denoted by $\widehat Z$ for simplicity sake,  is given by 
\[\widehat Z:=\supp(\widehat  \nu_t)=\bigcup_{(x,y)\in \supp(\widehat \pi)}[x,y].\]
For any $z\in \widehat Z$, let
\[V_{_\rightarrow}(z):=\Big\{\tz\in S_1(z)\,\Big|\, (z,\tz)\in C_\rightarrow\Big\} \quad\mbox{ and }\quad V_{_\leftarrow}(z):=\Big\{\tz\in S_1(z)\,\Big|\, (z,\tz)\in C_\leftarrow\Big\},\]
and similarly 
\[\V_{_\rightarrow}(z):=\Big\{\tz,\in S_2(z)\,\Big|\, (z,\tz)\in C_{_\rightarrow}\Big\} \quad\mbox{ and }\quad\V_{_\leftarrow}(z):=\Big\{\tz,\in S_2(z)\,\Big|\, (z,\tz)\in C_{_\leftarrow}\Big\},\] 
where 
\[C_{_\rightarrow}:=\Big\{(z,w)\in\X\times \X\,\Big|\,z\neq w, \exists (x,y)\in \supp(\widehat{\pi}), (z,w)\in [x,y]\Big\},\]
and 
\[C_{_\leftarrow}:=\Big\{(z,w)\in\X\times \X\,\Big|\, (w,z) \in C_{_\rightarrow} \Big\}.\] 
 As explained in \cite{Sam21},  $C_{_\rightarrow}$ and $C_{_\leftarrow}$ are disjoint sets which implies that  $V_{_\rightarrow}(z)\cap V_{_\leftarrow}(z)=\emptyset $, and also  $\V_{_\rightarrow}(z)\cap \V_{_\leftarrow}(z)=\emptyset $.  
 
 For any $z\in \widehat Z$
let 
 \[ \widehat Y_z:=\Big\{y\in\supp(\nu_1)\,\Big|\, \exists x\in \X,  (x,y)\in \widehat{\pi}, z\in[x,y]\Big\},\]
 and identically let 
 \[ \widehat X_z:=\Big\{x\in\supp(\nu_0)\,\Big|\, \exists y\in \X,  (x,y)\in \widehat{\pi}, z\in[x,y]\Big\}.\]
 For $y\in \supp(\nu_1)$, $z\in \X$ and $t\in [0,1]$, the quantity \begin{equation}\label{a_t}
 a_t(z,y) :=\int\nu_t^{w,y}(z) \,d\widehat{\pi}_{_\leftarrow}(w|y),
 \end{equation}
 is positive if and only if $z\in \widehat Z$ and 
 $y\in \widehat Y_z$. Identically, for  $x\in \supp(\nu_0)$,  $z\in \X$ and $t\in [0,1]$, the quantity 
 \[b_t(z,x):=\int \nu_t^{x,w}(z) \,d\widehat{\pi}_{_\rightarrow}(w|x),\]
  is positive if and only if $z\in \widehat Z$ and $x\in \widehat X_z$. Actually $a_t$ and $b_t$ represent conditional laws, $\sum_{z\in \X}a_t(z,y)=\sum_{z\in \X}b_t(z,x) =1$.
  
  
For $t\in[0,1]$, $z\in \widehat Z$,   $\tz\in S_1(z)$ and $y\in\supp(\nu_1)$, let 
\begin{equation}\label{a_t'}
{\mathrm a}_t(z,\tz,y):= \sum_{w\in \X, (z,\tz)\in[y,w]}  
r(y,z,\tz,w) \,d(y,w)\, \B_t^{d(y,w)-1}(d(z,w)-1)\,\widehat{\pi}_{_\leftarrow}(w|y),
\end{equation}
and for any $x\in \supp(\nu_0)$, let 
\begin{equation*}
{\mathrm b}_t(z,\tz,x):=\sum_{w\in \X, (z,\tz)\in[x,w]}  
\,r(x,z,\tz,w) \,d(x,w)\, \B_t^{d(x,w)-1}(d(x,z))\,\widehat{\pi}_{_\rightarrow}(w|x).
\end{equation*}
where the function $r$ is given by \eqref{defr}.
For $t\in(0,1)$, the quantity ${\mathrm a}_t(z,\tz,y)$ is positive if and only if $\tz\in V_{_\leftarrow}(z)$ and 
$y\in \widehat Y_{(z,\tz)}$ with
 \[ \widehat Y_{(z,\tz)}=\Big\{y\in\supp(\nu_1)\,\Big|\, \exists x\in \X,  (x,y)\in \widehat{\pi}, (z,\tz)\in[y,x]\Big\}\subset \widehat Y_z\cap  \widehat Y_{\tz} ,\] 
 According to \cite[Lemma 3.4]{Sam21}, given $z\in \widehat Z$ and $\tz\in V_{_\leftarrow}(z)$  the ratio ${\mathrm a}_t(z,\tz,y)/a_t(z,y)$ does not depend on $y\in \widehat Y_{(z,\tz)}$. Therefore, for any $z\in \X$
 and $\tz\in S_1(z)$, one may define 
 \[A_t(z,\tz):=\left\{\begin{array}{ll}
  \frac{{\mathrm a}_t(z,\tz,y)}{a_t(z,y)}& \mbox{ for } (z,\tz) \in \widehat Z\times V_{_\leftarrow}(z) \mbox{ and } y\in \widehat Y_{(z,\tz)}\neq \emptyset, \\
  0 & \mbox{ otherwise.} 
 \end{array}\right.\] 
 Identically, for $t\in(0,1)$, the quantity ${\mathbbm b}_t(z,\ttz,x)$ is positive if and only if $\tz\in V_{_\rightarrow}(z)$ and $x\in X_{(z,\tz)}$ with
 \[ \widehat X_{(z,\tz)}=\Big\{y\in\supp(\nu_1)\,\Big|\, \exists x\in \X,  (x,y)\in \widehat{\pi}^0, (z,\tz)\in[x,y]\Big\}\subset \widehat X_z\cap  \widehat X_{\tz},\]
 and according to \cite[Lemma 3.4]{Sam21},  the ratio ${\mathrm b}_t(z,\tz,x)/b_t(z,x)$ does not depend on
$x\in X_{(z,\tz)}$. Therefore, for any $z\in \X$
 and $\tz\in S_1(z)$, one  defines 
 \[B_t(z,\tz):=\left\{\begin{array}{ll}
  \frac{{\mathrm b}_t(z,\tz,x)}{b_t(z,x)}& \mbox{ for } (z,\tz) \in \widehat Z\times V_{_\rightarrow}(z) \mbox{ and } x\in \widehat X_{(z,\tz)}\neq \emptyset, \\
  0 & \mbox{ otherwise.} 
 \end{array}\right.\] 
 Observe that by reversibility, for any $(z,\tz)\in C_{_{\rightarrow}}$ with $d(z,\tz)=1$,
 \begin{align*}
   B_t(z,\tz)L(z,\tz)\widehat{\nu}_t(z)& =\sum_{x\in \widehat X_{(z,\tz)}} B_t(z,\tz)L(z,\tz) b_t(z,x)\nu_0(x)\\
   &=\sum_{x\in \widehat X_{(z,\tz)}} b_t(z,\tz,x)L(z,\tz) \nu_0(x)\\
   &=\sum_{(x,y)\in \supp(\widehat \pi), (z,\tz)\in[x,y]} r(x,z,\tz, y)L(z,\tz) d(x,y) \B_t^{d(x,y)-1}(d(x,z))\,\widehat{\pi}(x,y)\\
   &=\sum_{(x,y)\in \supp(\widehat \pi), (\tz,z)\in[y,x]} r(y,\tz,z, x)L(\tz,z) d(y,x) \B_t^{d(y,x)-1}(d(x,\tz)-1)\,\widehat{\pi}(x,y)\\
   &=\sum_{y\in \widehat Y_{(\tz,z)}} a_t(\tz,z,y)L(\tz,z) \nu_1(y)\\
   &=A_t(\tz,z)L(\tz,z)\widehat{\nu}_t(\tz).
 \end{align*}

For $t\in[0,1]$,  $z\in \widehat Z$, $\ttz\in S_2(z)$  and  $y\in\supp(\nu_1)$, define also 
 \begin{equation}\label{a_t''}
{\mathbbm a}_t(z,\ttz,y):= \!\!\!\!\!\sum_{w\in \X, (z,\ttz)\in[y,w]}  
\!\!\!\!\!r(y,z,\ttz,w) \,d(y,w)(d(y,w)-1)\, \B_t^{d(y,w)-2}(d(z,w)-2)\,\widehat{\pi}_{_\leftarrow}(w|y),
\end{equation}
and for $x\in\supp(\nu_0)$
\begin{eqnarray*}
 {\mathbbm b}_t(z,\ttz,x):=\!\!\!\!\!\sum_{w\in \X, (z,\ttz)\in[x,w]} \!\!\!\!\! 
r(x,z,\ttz,w) \,d(x,w)(d(x,w)-1)\, \B_t^{d(x,w)-2}(d(x,z))
\,\widehat{\pi}_{_\rightarrow}(w|x).
\end{eqnarray*}
For $t\in(0,1)$, we also have ${\mathbbm a}_t(z,\ttz,y)>0$ if and only if $\ttz\in \V_{_\leftarrow}(z)$ and 
$y\in \widehat Y_{(z,\ttz)}$, 
and ${\mathbbm b}_t(z,\ttz,x)>0$ if and only if $\ttz\in \V_{_\rightarrow}(z)$ and 
$x\in  \widehat X_{(z,\ttz)}$. Since according to \cite[Lemma 3.4]{Sam21}, the ratio ${\mathbbm a}_t(z,\ttz,y)/a_t(z,y)$ does not depend on $y\in \widehat Y_{(z,\ttz)}$, and the ratio ${\mathbbm b}_t(z,\ttz,x)//b_t(z,x)$ does not depend on
$x\in X_{(z,\tz)}$. Therefore one may define for any $z\in \X$ and $\ttz\in S_2(z)$, 
\[{\mathbb A}_t(z,\ttz):=\left\{\begin{array}{ll}
  \frac{{\mathbbm a}_t(z,\ttz,y)}{a_t(z,y)}& \mbox{ for } (z,\ttz) \in \widehat Z\times \V_{_\leftarrow}(z) \mbox{ and } y\in \widehat Y_{(z,\ttz)}\neq \emptyset, \\
  0 & \mbox{ otherwise,} 
 \end{array}\right.\] 
and
\[{\mathbb B}_t(z,\ttz):=\left\{\begin{array}{ll}
  \frac{{\mathbbm b}_t(z,\ttz,x)}{b_t(z,x)}& \mbox{ for } (z,\ttz) \in \widehat Z\times \V_{_\rightarrow}(z) \mbox{ and } x\in \widehat X_{(z,\ttz)}\neq \emptyset, \\
  0 & \mbox{ otherwise.} 
 \end{array}\right.\] 
One also observe that  for $t\in(0,1)$ and $z\in\widehat Z$, if $\ttz\in \V_{_\leftarrow}(z)$ (or equivalently  ${\mathbb A}_t(z,\ttz)>0$), then   $A_t(z,\tz)>0$ for any $\tz\in S_1(z)$ with  $\tz\sim \ttz$ (since $\tz\in V_{_\leftarrow}(z))$. Therefore for any $t\in(0,1)$, $z\in \widehat Z$, $\tz\in S_1(z),\ttz \in S_2(z)$, one has $(A_t(z,\tz),{\mathbb A}_t(z,\ttz))\in (0,+\infty)\times [0,+\infty)\cup\{(0,0)\}$.   Identically, one has $(B_t(z,\tz),{\mathbb B}_t(z,\ttz))\in (0,+\infty)\times [0,+\infty)\cup\{(0,0)\}$. 
 
 As above, one simply check that by reversibility, for any $(z,\ttz)\in C_{_{\rightarrow}}$ with $d(z,\ttz)=2$,
 \begin{align}\label{RelAB}
   {\mathbb B}_t(z,\ttz)L^2(z,\ttz)\widehat{\nu}_t(z)={\mathbb A}_t(\ttz,z)L^2(\ttz,z)\widehat{\nu}_t(\ttz). 
 \end{align}
 
One will apply the following theorem which is a direct result of Lemma 3.1 and the main Theorem 3.5 of \cite{Sam21}. For $z\in \widehat Z$ and $t\in (0,1)$, let 
\begin{multline*}
H_t(z):=\Big(\sum_{\tz\in V_{_\leftarrow}(z)}
A_t(z,\tz) \,L(z,\tz)\Big)^2\\  
+ \sum_{\tz\in V_{_\leftarrow}(z),\, \ttz\in \V_{_\leftarrow}(z), \,\tz\sim \ttz} \rho \Big(A_t^2(z,\tz),{\mathbb A_t}(z,\ttz)\Big)  \, L(\tz, \ttz)L(z,\tz),
\end{multline*}
and let
\begin{multline*}
K_t(z):=\Big(\sum_{\tz\in V_{_\rightarrow}(z)}
B_t(z,\tz) \,L(z,\tz)\Big)^2 \\+  \sum_{\tz\in V_{_\rightarrow}(z),\, \ttz\in \V_{_\rightarrow}(z),\, \tz\sim \ttz}  \rho\Big(B_t^2(z,\tz),\mathbbm{B}_t(z,\ttz)\Big)  \,
 L(\tz,\ttz)L(z,\tz),
 \end{multline*}
where the function $\rho:(0,+\infty)\times[0,+\infty)\cup\{(0,0)\}\to \R$ is defined by 
\begin{equation*}%\label{defG}
\rho(a,b):=\left(\log b-\log a-1\right) b,  \qquad a>0, b>0,
\end{equation*}
and $\rho(a,0)=0$ for $a\geq 0$.
According to \cite[Lemma 3.1]{Sam21} and \cite[Theorem 3.5]{Sam21} the following result holds.
\begin{theorem}\label{thmsam21bis}  We assume that the discrete space $(\X,d,m,L)$ is a graph space. 
Let $(\widehat \nu_t)_{t\in [0,1]}$ be the Schr\"odinger bridge at zero temperature
between two  probability measures $\nu_0,\nu_1\in \Pc(X)$ with bounded support given by \eqref{defhatnut}. For any $t\in (0,1)$, let $q_t$ be the kernel on $[0,1]$ defined by \[q_t(s)=\frac{2s}t \1_{[0,t]}(s)+ \frac{2(1-s)}{1-t} \1_{[t,1]}(s),\qquad s\in[0,1].\] 
Then, one has 
\[(1-t)H(\nu_0|m)+tH(\nu_1|m)-H(\widehat \nu_t|m)\geq \int_0^1 \left(\int \big(H_s+K_s\big)  \,d \widehat\nu_s\right) q_t(s)\,ds.\]
As a consequence if there exists a real function $\zeta:[0,1]\to \R$ 
such that for any $s\in(0,1)$,
\begin{equation*}
\int \big(H_s+K_s\big)  \,d \widehat\nu_s \geq \zeta(s),
\end{equation*}
and if $\zeta q_t$ is integrable with respect to the Lebesgue measure on $[0,1]$, then the convexity property of entropy \eqref{deplacebis} holds with, for any $t\in (0,1)$,  
 \[
 C_t(\widehat \pi)= \int_0^1 \zeta(s) q_t(s)\, ds.\]
 \end{theorem}
 Observe that  when  $\zeta$ is a constant function then the cost $C_t(\widehat \pi)$ is equal to this constant since $\int_0^1 q_t(s)\, ds=1$. And when $\zeta=\xi''$ where $\xi$ is a real continuous functions   on $[0,1]$, twice differentiable on $(0,1)$, then
 \[
 C_t(\widehat \pi)= \frac2{t(1-t)}\Big[
(1-t)\xi(0)+t\xi(1)-\xi(t)\Big].\]
 Using  the following  identity, for any integer $N$, for any $b\geq 0$,  and any positive $L_1,\ldots, L_N, a_1,\ldots, a_N$,
 \[\sum_{i=1}^N  \rho(a_i^2,b) L_i = L \,\rho\Big(\prod_{i=1}^N a_i^{2L_i/L},b\Big), \qquad \mbox{with} \quad L=\sum_{i=1}^N L_i,\]
 one gets
  \begin{align}\label{epuise0}
H_t(z)&=\Big(\sum_{\tz\in V_{_\leftarrow}(z)}
A_t(z,\tz) \,L(z,\tz)\Big)^2\nonumber\\  
&\qquad\qquad+ \sum_{ \ttz\in \V_{_\leftarrow}(z)} \quad \sum_{\tz\in [z,\ttz] \cap S_1(z)} \rho \Big(A_t^2(z,\tz),{\mathbb A_t}(z,\ttz)\Big)  \, L(\tz, \ttz)L(z,\tz)\nonumber\\
&=\Big(\sum_{\tz\in V_{_\leftarrow}(z)}
A_t(z,\tz) \,L(z,\tz)\Big)^2 \nonumber\\  
&\qquad\quad+ \sum_{ \ttz\in \V_{_\leftarrow}(z)} L^2(z,\ttz) \,\rho \Bigg(\prod_{\tz\in [z,\ttz] \cap S_1(z)} A_t(z,\tz)^{\frac{2L(z,\tz)L(\tz, \ttz)}{ L^2(z,\ttz)}},{\mathbb A_t}(z,\ttz)\Bigg) \nonumber \\
&\geq \Big(\sum_{\tz\in V_{_\leftarrow}(z)}
A_t(z,\tz) \,L(z,\tz)\Big)^2 \\
&\qquad+ \rho \Bigg(\sum_{ \ttz\in \V_{_\leftarrow}(z)} L^2(z,\ttz)\prod_{\tz\in [z,\ttz] \cap S_1(z)} A_t(z,\tz)^{\frac{2L(z,\tz)L(\tz, \ttz)}{ L^2(z,\ttz)}},\sum_{ \ttz\in \V_{_\leftarrow}(z)} L^2(z,\ttz){\mathbb A_t}(z,\ttz)\Bigg),\nonumber 
\end{align}
 where for the last inequality we  use the convexity property of the function $\rho$, and the identity $\rho(\lambda a,\lambda b)=\lambda \rho(a,b)$, $a>0,b,\lambda\geq 0$.
 
Since the function $a\to \rho(a,b)$ is decreasing on $(0,+\infty)$ for any $b\geq 0$, setting 
\[\overline{\mathbb A}_t(z):=\sum_{ \ttz\in \V_{_\leftarrow}(z)} {\mathbb A_t}(z,\ttz)\,  L^2(z,\ttz)\quad \mbox{
and}\quad \overline{A}_t(z):=\sum_{\tz\in V_{_\leftarrow}(z)}
A_t(z,\tz) \,L(z,\tz),\]
and according to the definition \eqref{defR_2} of $K(z,\V_{_\leftarrow}(z))$ and since 
\[\overline{A}_t(z)\geq \sum_{\tz\in ]z,\V_{_\leftarrow}(z)[}
A_t(z,\tz) \,L(z,\tz),\]
one gets 
\begin{equation}\label{epuise}
H_t(z)\geq \overline{A}^2_t(z)+ \rho\Big(K\big(z,\V_{_\leftarrow}(z)\big)\, \overline{A}_t^2(z), \overline{\mathbb A}_t(z)\Big).
\end{equation}
One may identically prove that for any $z\in\widehat Z$,
\begin{equation}\label{epuisebis}
K_t(z)\geq \overline{B}^2_t(z)+ \rho\Big(K\big(z,\V_{_\rightarrow}(z)\big)\, \overline{B}_t^2(z), \overline{\mathbb B}_t(z)\Big),%\1_{D_{_\rightarrow}}(z),
\end{equation}
with
\[\overline{\mathbb B}_t(z):=\sum_{ \ttz\in \V_{_\rightarrow}(z)} {\mathbb B_t}(z,\ttz)\,  L^2(z,\ttz),\quad \mbox{
and}\quad \overline{B}_t(z):=\sum_{\tz\in V_{_\rightarrow}(z)}
B_t(z,\tz) \,L(z,\tz).\]
Applying the inequality 
\[\rho(K a,b)= \rho(a,b)-b\log K\geq -a -b\log K,\]
for $K,a>0 ,b\geq 0$, 
and integrating \eqref{epuise} and \eqref{epuisebis} with respect to $\widehat{\nu}_t$, it follows that 
\begin{align}\label{suiteT2}
   \int \big(H_t+K_t\big)  \,d \widehat\nu_t  \geq &\int -\overline{\mathbb A}_t(z) \,\log  K\big(z,\V_{_\leftarrow}(z)\big) -\overline{\mathbb B}_t(z) \,\log  K\big(z,\V_{_\rightarrow}(z)\big)\,d \widehat\nu_t(z)\\
   &\geq -\log K \int \big(\overline{\mathbb A}_t+ \overline{\mathbb B}_t\big)\,d \widehat\nu_t.\nonumber
\end{align}
The proof of Theorem \ref{thmprinc} ends by applying Theorem \ref{thmsam21} and since, according to \eqref{a_t''},
\begin{align*}
&\int  \overline{{\mathbbm A}}_t  \,d \widehat\nu_t =\int \sum_{z\in\widehat Z} \sum_{\ttz\in \V_{_\leftarrow}(z)}
\frac{{\mathbbm a}_t(z,\ttz,y)}{a_t(z,y)} \,L^2(z,\ttz) a_t(z,y)\,d \nu_1(y)\\
&= \iint  \sum_{(z,\ttz), (z,\ttz)\in [y,w]}  
r(y,z,\ttz,w)L^2(z,\ttz) \,d(y,w)(d(y,w)-1)\, \B_t^{d(y,w)-2}(d(z,w)-2)\,d\widehat{\pi}(w,y)\\
&= \iint \sum_{k=2}^{d(y,w)} \Big(\sum_{ (z,\ttz)\in[y,w], \ttz\in \V_{_\leftarrow}(z), d(z,w)=k} r(y,z,\ttz,w)  L^2(z,\ttz) \Big) \\
&\qquad\qquad\qquad\qquad\qquad\qquad\qquad\qquad\qquad\B_t^{d(y,w)-2}(k-2) \,d(y,w)(d(y,w)-1)\,d\widehat{\pi}(w,y)\\
&= \iint \sum_{k=2}^{d(y,w)} \B_t^{d(y,w)-2}(k-2) \,d(y,w)(d(y,w)-1)\,d\widehat{\pi}(w,y)\\
&=\iint d(y,w)(d(y,w)-1) \,d\widehat{\pi}(w,y)=T_2(\widehat\pi).
\end{align*}
From the identity \eqref{RelAB} it follows that $\displaystyle\int  \overline{{\mathbbm B}}_t  \,d \widehat\nu_t =\int  \overline{{\mathbbm A}}_t  \,d \widehat\nu_t=T_2(\widehat\pi)$.
\end{proof}



\end{document}
%If $d(x,y)=2$ then $\{i\in[n]\,|\,x_i\neq y_i\}=\{i_0,j_0\}$ and $(z,\sigma_i\sigma_j(z))\in[x,y]$ if and only if $\{i,j\}=\{i_0,j_0\}$ and $z=x,\sigma_i\sigma_j(z)=y$. In that case, one has 
%\begin{align*} 
%2 \sum_{\{i,j\}\subset [n], (z,\sigma_i\sigma_j(z))\in[x,y]} (2z_i-1)(2z_j-1)\,\partial_{ij}^2v(z_{\overline{ij}})&= 2(2x_{i_0}-1)(2x_{j_0}-1)\,\partial_{i_0j_0}^2v(x_{\overline{i_0j_0}})\\
%&\leq 2| \partial_{i_0j_0}^2v(x_{\overline{i_0j_0}})| \leq {\|Hv(x)\|_{op}} \leq \|Hv\|_{op,\infty}.
%\end{align*}
%and therefore $\ell_t^{x,y}(0)\leq \|Hv\|_{op,\infty}$ and $D_tv(x,y)\leq \|Hv\|_{op,\infty} $.

%If $d(x,y)=3$ then $\{i\in[n]\,|\,x_i\neq y_i\}=\{i_0,j_0,k_0\}$ and $(z,\sigma_i\sigma_j(z))\in[x,y]$ implies $d(x,z)=1$ or $d(x,z)=0$. Assume $d(x,z)=1$ and $z=\sigma_{i_0}(x)$. In that case $(z,\sigma_i\sigma_j(z))\in[x,y]$ if and only if $\{i,j\}=\{j_0,k_0\}$ and $z=\sigma_{i_0}(x),\sigma_i\sigma_j(z)=y$ and therefore 
%\begin{align*} 
%2 \sum_{\{i,j\}\subset [n], (z,\sigma_i\sigma_j(z))\in[x,y]} (2z_i-1)(2z_j-1)\,\partial_{ij}^2v(z_{\overline{ij}})&= 2(2x_{j_0}-1)(2x_{k_0}-1)\,\partial_{j_0k_0}^2v(\sigma_{i_0}(x)_{\overline{j_0k_0}})\\
%&\leq 2| \partial_{j_0k_0}^2v(\sigma_{i_0}(x)_{\overline{j_0k_0}})| \leq {\|Hv(\sigma_{i_0}(x))\|_{op}} \leq \|Hv\|_{op,\infty},
%\end{align*}
%and $\ell_t(1)\leq 3t \|Hv\|_{op,\infty}$.
%Now assume that   $d(x,z)=0$, $z=x$ 
%then
%\begin{align*}
%\ell_t(0)&\leq 2\Big(| \partial_{i_0j_0}^2v(x_{\overline{j_0k_0}})|+| \partial_{j_0k_0}^2v(x_{\overline{i_0k_0}})|+| \partial_{j_0k_0}^2v(x_{\overline{j_0k_0}})|\Big)(1-t)\\
%\leq 3(1-t)\|Hv\|_{op,\infty},
%\end{align*}
%This finally provides  


