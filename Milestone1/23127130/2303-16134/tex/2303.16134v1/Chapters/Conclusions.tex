% !TEX TS-program = pdflatex
% !TEX root = ../ArsClassica.tex

%************************************************
\chapter{Conclusions}
\label{chp:conclusions}
%************************************************

\begin{flushright}
\itshape
So long, and thanks for all the fish. \\
\medskip
--- Douglas Adams, The Hitchhiker's Guide to the Galaxy
\end{flushright}

 
\lstset{numbers=left,
    numberstyle=\scriptsize,
    stepnumber=1,
    numbersep=8pt
}    

% Intro paragraphs
Over the past couple of decades, Bayesian inference has been established as the standard mathematical framework for conducting scientific inference in the physical sciences. This progress has been largely facilitated by the recent advances in computer technology and probabilistic computational methods. However, the specific characteristics of the mathematical models and available data used in astronomy and cosmology still pose significant challenges for existing computational tools.

From the perspective of theoretical modelling, many astrophysical models involve computationally expensive operations which are almost always non--differentiable. This limits the potential range of application of a plethora of MCMC methods, particularly those that rely on the use of the gradient of the posterior density function or are unable to scale to a large number of parallel CPUs. On the other hand, the commonly sparse nature of the available data often induces a level of multimodality in the studied posterior distributions. The existence of multiple modes in the posterior distribution can hinder the sampling procedure of most computational tools and in the case of most MCMC methods, make the results unreliable. This thesis has introduced two methods and their software implementations that were specifically designed with this kind of challenge in mind.

% Ensemble slice sampling
In Chapter \ref{chp:ess} we introduced \textit{Ensemble Slice Sampling (ESS)}, a method that extends the applicability of the univariate slice sampler to multivariate target distributions, by utilising an ensemble of parallel walkers. The method requires minimal tuning and no gradient information, demonstrates affine--invariant sampling performance, and is trivially parallelisable to a large number of CPUs. Chapter \ref{chp:zeus} presents \texttt{zeus}, an open--source \texttt{Python} implementation of ESS. Compared to the popular MCMC sampler \texttt{emcee}, the sampling efficiency of \texttt{zeus} scales more favourably with the total number of dimensions. Furthermore, the generated Markov chains exhibit substantially lower autocorrelation levels for a wide range of target distributions and the method generally requires significantly fewer walkers than \texttt{emcee}. Finally, in the problems of BAO and exoplanet parameter estimation, \texttt{zeus} is $9$ and $29$ times more efficient than the competition, respectively.

% Preconditioned Monte Carlo
Chapter \ref{chp:pmc} is devoted to \textit{Preconditioned Monte Carlo (PMC)}, a novel Monte Carlo method for sampling from posteriors with non--trivial geometry (i.e. non--linear correlations, multimodality). PMC utilises a Normalising Flow (NF) transformation in order to precondition the target distribution by approximately removing the correlations between its parameters. PMC then relies on a \textit{Sequential Monte Carlo (SMC)} in order to produce posterior samples and an estimate of the model evidence. Empirical tests validate the high sampling efficiency of PMC. In the cases of primordial feature analysis and gravitational wave inference, PMC is approximately $50$ and $25$ times faster respectively than nested sampling. Finally, Chapter \ref{chp:pocomc} offers a short overview of \texttt{pocoMC}, an open--source \texttt{Python} implementation of PMC. The basic principles of PMC are presented along with the various options and features provided in the package. In terms of parallelisation, \texttt{pocoMC} manifests linear scaling up to thousands of CPUs.

% Future outlook
The methods introduced in the aforementioned chapters aim to address the various computational challenges currently presented by modern astrophysical models and data. Despite their empirical success, as demonstrated by the provided tests and their adoption by the astronomical community, their application in higher dimensions (e.g. $D>100$) is still hindered by the \textit{curse of dimensionality}. In the future, in order to accommodate for subtle effects present in the data, astrophysical models will necessarily become increasingly complicated. As a response, sampling methods such as the ones presented in this thesis will have to evolve in order to cope with the additional computational challenges. A possible avenue of future research could be the self--supervised construction of surrogate models (e.g. emulators) for either the likelihood function, posterior density, or model, thus enabling the use of gradient--based MCMC methods in the context of advanced schemes such as PMC. We sincerely hope that, in the meantime, methods and packages such as ESS $\&$ PMC, and \texttt{zeus} $\&$ \texttt{pocoMC} will prove useful to the astronomical community by facilitating the next generation of Bayesian data analyses.