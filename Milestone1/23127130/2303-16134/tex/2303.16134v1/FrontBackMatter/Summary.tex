% !TEX TS-program = pdflatex
% !TEX root = ../ArsClassica.tex

%*******************************************************
% Lay Summary
%*******************************************************
\pdfbookmark{Lay summary}{Lay summary}
\addcontentsline{toc}{chapter}{Lay summary}

\chapter*{Lay summary}

\looseness=-1 Over the past few decades, the volume of astronomical and cosmological data has increased substantially. In response to that, a variety of astrophysical models have been proposed to explain the plethora of observations. As the information provided by the data is always incomplete and uncertain, inferring the properties of a model, including the values of its parameters, given the observed data, generally requires us to reason in the face of uncertainty. In the context of \textit{Bayesian inference}, uncertainty is represented by the notion of probability. One usually starts by quantifying their state of knowledge about the possible values of the model parameters \textit{prior} to seeing the data, in the form of a probability distribution called the \textit{prior}. The next step is to use the so--called \textit{Bayes' theorem} in order to update one's degree of belief about the model parameters given the available data. The outcome of this updating process is the \textit{posterior} probability distribution of the model parameters given the data which quantifies the plausibility of different parameter values.

Approximating the \textit{posterior} generally requires the use of probabilistic computational methods. Standard practice in astronomy often employs conventional computational tools (e.g. \textit{Markov chain Monte Carlo}) despite their specific theoretical limitations or narrow range of validity. The aim of this thesis is to first introduce the basic principles of Bayesian inference along with the basic methods used for Bayesian computation and then present two novel algorithms and their respective software implementations. A common element of these newly developed tools is their ability to exploit the available information about the geometry of the posterior in order to approximate it more quickly. Finally, both methods are able to benefit from the possible availability of multiple CPUs in order to accelerate their computation.