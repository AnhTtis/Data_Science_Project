\documentclass[lettersize,journal]{IEEEtran}
\usepackage{amsmath,amsfonts}
\usepackage{algorithmic}
\usepackage{array}
\usepackage[caption=false,font=normalsize,labelfont=sf,textfont=sf]{subfig}
\usepackage{textcomp}
\usepackage{stfloats}
\usepackage{url}
\usepackage{verbatim}
\usepackage{graphicx}
\hyphenation{op-tical net-works semi-conduc-tor IEEE-Xplore}
\def\BibTeX{{\rm B\kern-.05em{\sc i\kern-.025em b}\kern-.08em
    T\kern-.1667em\lower.7ex\hbox{E}\kern-.125emX}}
\usepackage{balance}
\usepackage{booktabs}
\usepackage{multirow}
\usepackage{multicol}
\usepackage{color,xcolor}
\usepackage{amssymb}
\usepackage[ruled, vlined]{algorithm2e}

\newcommand*{\red}{\textcolor{black}}
%\newcommand*{\red}{\textcolor{black}}

\begin{document}

\title{Cross-Modal Causal Representation Learning for Radiology Report Generation}

\author{Weixing~Chen, Yang~Liu,~\IEEEmembership{Member,~IEEE}, Ce~Wang, Jiarui~Zhu, Guanbin~Li,~\IEEEmembership{Member,~IEEE}, \\Cheng-Lin Liu,~\IEEEmembership{Fellow,~IEEE}, and~Liang~Lin,~\IEEEmembership{Fellow,~IEEE}
\thanks{This work is supported in part by the National Key R\&D Program of China under Grant No.2021ZD0111601, in part by the National Natural Science Foundation of China under Grant No.62436009, No. 62322608, and No.62301532, in part by the Guangdong Basic and Applied Basic Research Foundation under Grant No.2025A1515011874 and No.2023A1515011530. (\emph{Corresponding author: Yang Liu.})} 
\thanks{Weixing Chen, Yang Liu, Guanbin Li and Liang Lin are with the School
of Computer Science and Engineering, Sun Yat-sen University, China, and Guangdong Key Laboratory of Big Data Analysis and Processing, Guangzhou, China. \protect
(E-mail: chenwx228@mail2.sysu.edu.cn, liuy856@mail.sysu.edu.cn, liguanbin@mail.sysu.edu.cn, linliang@ieee.org)}
\thanks{Ce Wang is with the School of Science, Sun Yat-sen University, Shenzhen, 510275, China. (E-mail: fogever@icloud.com)}
\thanks{Jiarui Zhu is with the Hong Kong Polytechnic University. (E-mail: jiarui.zhu@connect.polyu.hk)}
%\thanks{Shen Zhao is with the School of Intelligent Systems Engineering, Sun Yat-sen University, China. (E-mail: z-s-06@163.com)}
\thanks{Cheng-Lin Liu is with the State Key Laboratory of Multimodal Artificial Intelligence Systems, Institute of Automation, Chinese Academy of Sciences, Beijing, China, and also with the School of Artificial Intelligence, University of Chinese Academy of Sciences, Beijing, China. (liucl@nlpr.ia.ac.cn)}
}

%\markboth{IEEE Transactions on Image Processing}%
%{How to Use the IEEEtran \LaTeX \ Templates}

\maketitle
% Radiological Cross-modal Alignment and Reconstruction Enhanced, RadCARE

\begin{abstract}
 \red{Radiology Report Generation (RRG) is essential for computer-aided diagnosis and medication guidance, which can relieve the heavy burden of radiologists by automatically generating the corresponding radiology reports according to the given radiology image.
 However, generating accurate lesion descriptions remains challenging due to spurious correlations from visual-linguistic biases and inherent limitations of radiological imaging, such as low resolution and noise interference.
 To address these issues, we propose a two-stage framework named Cross-Modal Causal Representation Learning (CMCRL), consisting of the Radiological Cross-modal Alignment and Reconstruction Enhanced (RadCARE) pre-training and the Visual-Linguistic Causal Intervention (VLCI) fine-tuning.
 %% 预训练
 In the pre-training stage, RadCARE introduces a degradation-aware masked image restoration strategy tailored for radiological images, which reconstructs high-resolution patches from low-resolution inputs to mitigate noise and detail loss. Combined with a multiway architecture and four adaptive training strategies (e.g., text postfix generation with degraded images and text prefixes), RadCARE establishes robust cross-modal correlations even with incomplete data.  
 %% 微调
 In the \red{VLCI} phase, we deploy causal front-door intervention through two modules: the Visual Deconfounding Module (VDM) disentangles local-global features without fine-grained annotations, while the Linguistic Deconfounding Module (LDM) eliminates context bias without external terminology databases. Experiments on IU-Xray and MIMIC-CXR show that our CMCRL pipeline significantly outperforms state-of-the-art methods, with ablation studies confirming the necessity of both stages. Code and models are available at \url{https://github.com/WissingChen/CMCRL}.}
\end{abstract}

\begin{IEEEkeywords}
Radiology Report Generation, causality, visual-language pre-training.
\end{IEEEkeywords}


\section{Introduction}


\begin{figure}[t]
 \centering
    \includegraphics[width=1\linewidth]{figure/1.pdf}
    \caption{The example of visual-linguistic spurious correlation of {RRG}{. (a) There exists data bias in the training set, where the combination of ``No Effusion" and ``Normal Heart" occurs significantly more frequently than others. This leads the model to correlate the ``Normal Heart" with the conclusion of ``No Effusion", forming a spurious correlation. (b) Consequently, the non-causal model does not truly rely on pulmonary information to determine the presence of effusion. Instead, it arrives at the conclusion of ``No Effusion" based on the presence of a ``Normal Heart". (c) In contrast, our \red{CMCRL}, by extracting critical visual features and applying them to causal front-door intervention, enables the model to identify features that are causally related to ``Pleural Effusion" and produce the correct report.}
    }
\label{fig:intro_bias_info}
\end{figure}

% 定义任务,任务的难点
% 现有方法的做法,不足
% 引出因果,现有因果方法存在不足,我们可以解决
% 引出预训练,虽然我们的方法不依赖额外的数据,但是需要有一个良好的表征
% 方法概述,先预训练,在因果微调
% 贡献
\IEEEPARstart{R}{adiology} images are widely used in clinical procedures \cite{yu2022crosslink}, providing significant evidence for disease analysis and radiological diagnosis~\cite{zhou2021review}. Nevertheless, observing suspicious lesions and writing a coherent diagnosis report is time-consuming, even for experienced radiologists. Furthermore, inexperienced radiologists often fail to capture tiny abnormalities due to the high requirement for clinical knowledge. To relieve this issue, Radiology Report Generation (RRG) has emerged and attracted growing interest in recent years~\cite {tanida2023interactive}. RRG extracts features from radiology images and generates the corresponding reports, which is similar to image captioning~\cite{nguyen2022effective}. However, the current RRG faces three challenges that are significantly different from the image captioning task:
\textbf{1)} Longer sentence generation (60-100 tokens) tends to accumulate a larger bias, whereas image captions typically consist of fewer than 20 tokens~\cite{chen2020generating}, 
\textbf{2)} The necessity to capture all key regions in the radiology image (i.e., abnormalities and diagnostic evidence) and a low tolerance for factually incorrect information~\cite{tanida2023interactive}, 
\textbf{3)} More complex linguistic and visual semantic patterns require a proficient understanding of radiological information, whereas entities in natural images are diverse and easily distinguishable~\cite{zhang2020radiology, zhou2021review}. 
Therefore, these challenges impose significant limitations on modeling visual-linguistic interactions and learning informative cross-modal representations for accurate RRG~\cite{chen2020generating,chen2023cross}.



To tackle the aforementioned challenges, current RRG methods have made significant efforts, such as the memory-driven module for longer sentence generation~\cite{chen2020generating}, additional knowledge for more accurate description~\cite{liu2021exploring}, and contrast with normal samples (i.e., images without lesion areas) for the capture of abnormalities (i.e., lesion areas within images)~\cite{liu2021contrastive}. Most of the previous RRG methods aim to capture the latent subtle differences in images (visual biases) and learn a concise set of key descriptions in the text (linguistic biases) for accurate long-sequence generation. However, these methods usually focus on training computationally expensive models based on a large number of region-level annotations~\cite{tanida2023interactive} and task-specific knowledge\footnote{These methods build the template or knowledge database laboriously, making it hard to transfer those approaches directly to other datasets~\cite{yang2022knowledge}.}, rather than focusing on identifying and mitigating biases inherent in the data itself.

As shown in Fig \ref{fig:intro_bias_info} (a-b), the significant visual and linguistic biases in numerous image-text data build a spurious correlation between ``Normal Heart" and ``No Effusion", leading to the incorrect and unreliable reports. To mitigate the cross-modal bias and encourage the model to infer based on the corresponding visual basis, as shown in Fig \ref{fig:intro_bias_info} (c), causal inference \cite{pearl2016causal} has shown promising performance in several visual-linguistic tasks \cite{liu2022causal}. However, directly applying existing causal methods to the RRG task may yield unsatisfactory results due to the unobservable bias in the visual and linguistic domain and the complex visual-linguistic interaction in radiology images and textual reports. Although back-door intervention can cut off the shortcut path, it requires approximating the observable bias using a well-trained visual object detector or a well-constructed linguistic dictionary. Fortunately, causal front-door intervention gives a feasible way to mitigate visual-linguistic spurious correlations without the calculation of unobservable bias. With causal front-door intervention, we can eliminate the spurious cross-modal correlations effectively by introducing an additional mediator~\cite{liu2023cross, yang2021causal} and generate an accurate description of ``Pleural Effusion". The mediator can be assumed as the feature of ``Pleural" in different findings (e.g., ``Normal Heart") to estimate the sub-distribution of ``Pleural Effusion"~\cite{yang2021deconfounded}. However, the accurate and reasonable acquisition of the mediator is challenging, especially without the support of additional radiological knowledge. 

\red{To effectively obtain reliable estimates of mediator, well-aligned radiological multi-modal representations are crucial. However, existing multi-modal approaches are limited by singular pre-training strategies and fail to fully leverage data with missing modalities within datasets~\cite{liu2023m3ae}. Moreover, given the issues of low-resolution inputs and noise interference present in radiological data, directly adopting multi-modal pre-training models designed for natural image tasks does not effectively capture and align multi-modal features~\cite{he2022masked, wang2021simvlm}. This further restricts the performance of subsequent causal interventions.}

\red{Motivated by the characteristic multi-modal radiological data and effectiveness of causal inference in deconfounding the cross-modal bias, we introduce a Cross-Modal Causal Representation Learning (CMCRL) framework for RRG. This framework includes a pre-training model specifically designed for radiological data, termed Radiological Cross-modal Alignment and Reconstruction Enhanced (RadCARE), and a lightweight cross-modal causal intervention model without requiring an observable bias assumption, named Visual-Linguistic Causal Intervention (VLCI), to mitigate biases in visual and linguistic data.
Our RadCARE framework introduces a degradation-aware masked image restoration strategy that reconstructs high-resolution patches from degraded inputs, explicitly mitigating detail loss while preserving anatomical coherence. This is coupled with a multi-way architecture enabling four adaptive training strategies, including postfix text generation via prefix text or multi-modal data, and masked image restoration via degraded images or combined with complete text.
Our VLCI framework contains the visual deconfounding module (\textbf{VDM}) and linguistic deconfounding module (\textbf{LDM}) based on the causal front-door intervention paradigm. In VDM, the visual mediator is constructed by local detail information (e.g., lung texture) and global contour (e.g., lung contour) from radiology images, to disentangle the visual features. The linguistic confounders can be eliminated by the LDM, which estimates the change in the probability of word embedding caused by visual details and linguistic context. 
In summary, our main contributions are listed as follows}:

\begin{itemize}
    \item \red{To alleviate the problem of unpaired data and capture detailed features when pre-training cross-modal data, we propose RadCARE that integrates degradation-aware masked image restoration with postfix text generation for pre-training in various data situations (e.g., unpaired, single modality), which is efficient and easy to implement.}
    \item  \red{To mitigate cross-modal biases, we propose visual-linguistic causal intervention modules VDM and LDM, integrated into the VLCI without requiring additional data or a well-trained semantic extractor for guidance. 
    % 更应该强调独特的地方而不是这俩模块干了什么?
    % The VDM aims to disentangle the region-based image features, and the LDM aims to eliminate the spurious correlations caused by the entangled visual-linguistic embedding.
    } 
    % 为什么不需要额外的标注数据
    \item \red{We propose a CMCRL framework for RRG, which introduces mediators without additional knowledge, to deconfound the visual-linguistic features by causal front-door intervention. To the best of our knowledge, we are the first to conduct a cross-modal causal intervention for RRG. Experiments show that we achieve state-of-the-art performance on IU-Xray and MIMIC-CXR datasets.}
\end{itemize}

\section{Related Work}

\subsection{Image Captioning}
% \red{Image captioning aims to understand image information and describe it in text, which mainly adopts the encoder-decoder framework~\cite{devlin2018bert}. Recent work has achieved significant success, generally improving model performance from three aspects: visual representation learning, linguistic comprehension and generation, and training strategy~\cite{stefanini2022show}. The global features extractor from CNNs leads to excessive compression of information and lacks granularity~\cite{karpathy2015deep}, thus, visual saliency can further improve performance via the spatial relation of regional features and rely on the integration of visual and semantic data to improve performance~\cite{yu2018topic,jiang2022visual}. As for the linguistics features, generating the captioning from coarse to fine~\cite{wang2017skeleton} and integrating multi-modal information via multiple layers of LSTM~\cite{xian2019self} achieve a promising performance, but are limited by the training efficiency and expression ability. With the rise of transformer-based models~\cite{devlin2018bert}, intrinsic information of each modality is learned through self-attention, while cross-attention enables the learning of multi-modal information, thereby effectively enhancing the model's performance~\cite{nguyen2022grit, liu2022show}. Furthermore, the performance improvements can be achieved by the formulation of training strategies, including the learning order of the samples~\cite{zhou2019re}, reinforcement learning~\cite{nguyen2022effective}, and visual-linguistic pre-training~\cite{yu2022coca, wang2021simvlm}. Compared with the image captioning approaches, the {RRG} has similar structures~\cite{stefanini2022show}. Nevertheless, image captioning usually generates a single sentence to describe the main entities, while the {RRG} focuses on potential subtle lesion areas in radiology images and generates longer sentences from more sophisticated visual-linguistic semantics.}

Image captioning, based on the encoder-decoder framework, aims to describe image content in text. Recent advancements have enhanced performance in three areas: visual representation learning, linguistic generation, and training strategies~\cite{stefanini2022show}. While CNN-based global feature extraction compresses information excessively~\cite{karpathy2015deep}, integrating visual saliency through regional features and combining visual-semantic data improves results~\cite{jiang2022visual}. In linguistic processing, coarse-to-fine captioning~\cite{wang2017skeleton} and multi-layer LSTM integration~\cite{xian2019self} have shown promise but are limited by training efficiency. Transformer-based models~\cite{devlin2018bert} leverage self-attention for modality learning and cross-attention for multi-modal integration, significantly boosting performance~\cite{nguyen2022grit, liu2022show}. Performance is further improved by training strategies such as sample order optimization~\cite{zhou2019re}, reinforcement learning~\cite{nguyen2022effective}, and visual-linguistic pre-training~\cite{yu2022coca, wang2021simvlm}. 

\red{With the proliferation of Large Language Models (LLMs), image captioning achieved significant breakthroughs in both model architecture and generation capabilities. The V2L-Tokenizer~\cite{zhu2024beyond} model demonstrated SOTA performance across multiple tasks through a non-finetuning approach. Meanwhile, prompt learning offered new insights into the controllability of image captioning~\cite{liu2024improved, chen2024sharegpt4v}. Additionally, models like GPT-4o\cite{hurst2024gpt} and Qwen2-VL\cite{bai2023qwen} integrated advanced visual processing capabilities with powerful language generation, achieving remarkable results without extensive fine-tuning.} 
Compared to image captioning, RRG shares similar structures~\cite{stefanini2022show} but focuses on subtle lesion detection in radiology images, generating longer and more complex descriptions.

\subsection{Causal Inference}
Causality offers a robust approach to mitigating the visual-linguistic bias stemming from the heterogeneity of multi-modal data by eliminating spurious correlations~\cite{pearl2016causal,liu2022causal}. Causal inference~\cite{liu2022causal} estimates hidden causal effects within distributions, significantly improving model generalization by addressing confounders through back-door, front-door, or counterfactual interventions. This method enhances performance in tasks like image classification~\cite{yue2020interventional}, semantic segmentation~\cite{miao2023caussl}, visual feature representation~\cite{liu2022contextual, wang2021causal}, image captioning~\cite{liu2022show}, and visual question answering (VQA)\cite{liu2023cross}. For instance, Wang et al.\cite{wang2020visual} applied back-door intervention to improve Faster R-CNN, which subsequently enhanced performance in VQA~\cite{zang2023discovering} and image captioning~\cite{liu2022show}. Additionally, Liu et al. proposed an event-level causal VQA model using front-door intervention, which leverages attention to integrate local and global causality-aware visual-linguistic representations~\cite{liu2023cross}. This approach also addresses confounding by simulating causal interventions based on human priors~\cite{yang2023good}.

Since confounders are often unobservable, front-door and counterfactual interventions are particularly effective. These interventions can capture hidden confounders by integrating causal mechanisms into attention modules through cross-sample attention or self-annotation~\cite{hu2021causal, wang2021causal, xue2023variational}. 
\red{In the era of LLMs, causal learning has emerged as a key direction for enhancing both the generalization ability and interpretability. On one hand, by integrating causality into the processes of pre-training, fine-tuning, and inference, LLMs can better handle complex tasks, while also improving their performance in multi-modal scenarios~\cite{zhao2024causal, chu2024causal}. On the other hand, causal learning enables LLMs to move beyond mere statistical correlations and understand the underlying causal mechanisms in the data, thereby generating more reliable and ethically aligned outputs~\cite{huang2025causality}.}
Unlike prior work focusing on VQA or image captioning, we tackle the more complex task of RRG, which involves modeling intricate visual-linguistic interactions in radiology images and textual reports~\cite{zhao2018causaltriad, miao2023caussl, li2023causally}. Our proposed front-door causal intervention method addresses and eliminates both visual and linguistic spurious correlations.

% \begin{figure*}[h!]
%    \centering
%   %\fbox{\rule{0pt}{2in} \rule{0.9\linewidth}{0pt}}
% \includegraphics[width=1\linewidth]{figure/method_vlci.pdf}
%     \vspace{-15pt}
%    \caption{
%    The pipeline of the VLCI. The mechanism of front-door causal intervention via the structural causal model (SCM) (a) is implemented in the example from Fig.~\ref{fig:intro_bias_info}, in contrast to the non-causal model.
%    Our approach follows a two-stage pipeline: 1) Visual Linguistic Pre-training (VLP) (c) utilizes the Multiway Transformer (b) to learn multi-modal context and concepts, and achieve effective cross-modal alignment. 2) Visual-Linguistic Causal Intervention (VLCI) (d) consists of a pre-trained transformer, a Visual Deconfounding Module (VDM), and a Linguistic Deconfounding Module (LDM),
%    which is implemented after VLP to mitigate the cross-modal bias.
%    The image embedding module employs the initial three blocks of ResNet101.
%    Specifically, the VDM explores visual bias using local sampling and global sampling. The LDM estimates linguistic bias using a vocabulary dictionary and visual features.}
%        \vspace{-10pt}
%    \label{fig:method_vlci}
% \end{figure*}

\subsection{Radiology Report Generation}
%Recently, {RRG} methods have followed the works of image captioning and have shown remarkable performance. However, the abnormal descriptions and lesion regions in patient samples only constitute a small portion, leading to the visual-linguistic bias in the {RRG} task. To address this issue, knowledge-aware methods prevail because they can explore and distill the knowledge to accurately identify abnormalities and describe them using appropriate terminology. Specifically, the bias from language can be mitigated by utilizing the general graph~\cite{zhang2020radiology} as prior knowledge~\cite{liu2021exploring, huang2023kiut}. However, fixed and limited knowledge is insufficient to address complex radiological problems. Therefore, Yang et al.~\cite{yang2022knowledge} and Li et al.~\cite{li2023dynamic} employed RadGraph~\cite{jain2021radgraph} for knowledge retrieval and dynamic knowledge construction. Additionally, utilizing a well-trained label extractor or manually annotated region labels for classification tasks, and further incorporating them to assist in report generation, can also be considered as approaches leveraging external knowledge~\cite{you2021aligntransformer, wang2022cross, tanida2023interactive}.

% Although the knowledge-based model is a promising direction, acquiring knowledge is expensive and challenging to transfer to other tasks. Therefore, CA~\cite{liu2021contrastive} and CMCL~\cite{liu2022competence} explored the potential abnormal regions by comparing them with normal samples. Moreover, knowledge can be retrieved or constructed using templates corresponding to similar images~\cite{yang2022knowledge, li2023dynamic}, which effectively mitigates the issue of representation distortion caused by excessive focus on knowledge. Additionally, the template-based approach requires several samples for comparison, memory-driven transformers can store the visual~\cite{nooralahzadeh2021progressive} and linguistic~\cite{chen2020generating} context without templates. However, considering only the context would lead to cross-modal bias. Thus, the models align the cross-modal information in the embedding space and incorporate it as memory, which can be retrieved by visual-linguistic features~\cite{chen2022cross, wang2022cross}. The visual-linguistic bias hinders the robustness and reliability of {RRG}. To mitigate the cross-modal bias, we introduce a lightweight VLCI approach that mitigates cross-modal confounders, uncovers the true cross-modal causality through causal front-door intervention, and reduces the need for additional annotation when discovering abnormalities.


% 增加完实验再回头补
Recently, RRG methods, inspired by image captioning, have achieved notable success. However, abnormal descriptions and lesion regions make up only a small portion of patient samples, leading to visual-linguistic bias. Knowledge-aware approaches address this by leveraging external knowledge to identify abnormalities and describe them accurately~\cite{zhang2020radiology, liu2021exploring}. Although fixed knowledge is limited for complex radiological cases, dynamic knowledge construction using RadGraph~\cite{jain2021radgraph} has been proposed~\cite{yang2022knowledge, li2023dynamic}. Furthermore, external knowledge can be integrated through label extractors or manually annotated regions to enhance report generation~\cite{you2021aligntransformer, wang2022cross, tanida2023interactive}. 

While knowledge-based models show promise, acquiring knowledge is costly and difficult to generalize. To mitigate this, CA~\cite{liu2021contrastive}, CMCL~\cite{liu2022competence}, CAMANet~\cite{wang2024camanet} explored abnormal regions by comparing them with normal samples. Templates based on similar images also help reduce representation distortion~\cite{yang2022knowledge, li2023dynamic}. Memory-driven transformers can store visual~\cite{nooralahzadeh2021progressive,divya2024memory} and linguistic~\cite{chen2020generating} contexts without templates but risk cross-modal bias. To address this, models align cross-modal data in embedding spaces for retrieval~\cite{chen2022cross, wang2022cross}. 
\red{With the advancement of multi-modal large language models (MLLMs), their robust visual reasoning capabilities, internal knowledge, and language generation skills have significantly enhanced performance in radiology report generation tasks. However, the inherent tendency of large models to produce hallucinations continues to pose challenges for this task. Consequently, some methods have focused on developing approaches that enable large models to generate descriptions consistent with the findings in radiological images~\cite{liu2024context, jin2024promptmrg, wang2023r2gengpt}. } 
To further reduce cross-modal bias, we introduce a lightweight VLCI approach, which applies causal front-door intervention to uncover true cross-modal causality and reduce reliance on additional annotations.



\section{Methodology} % (3.5)
% 1. Overview

In this section, we first introduce the RadCARE framework for pre-training, followed by the two essential cross-modal causal intervention modules, i.e., the Visual Deconfounding Module (VDM) and the Linguistic Deconfounding Module (LDM). Then, we introduce how to integrate VDM and LDM into the VLCI for cross-modal causal intervention and complete the CMCRL framework.

\subsection{Overview}

% 标准模型
\red{
A typical RRG model can generate a series of radiological findings and insights given a radiological image. Upon receiving the Chest X-ray image $I\in\mathbb{R}^{C\times H\times W}$, the model first utilizes a visual feature extractor to obtain $h_v$, which guides the generation of the next word $w_i \in R$ from the prefix text embedding $h_w$, as shown in Fig.~\ref{fig:method_overview} (a). 
% 说明混淆因子
However, due to the presence of confounders $Z$ in both the visual and linguistic modalities, the non-causal model may capture the spurious correlations between ``Normal Heart" and ``No Effusion", which causes the neglect of ``Pleural Effusion" accompanied by ``Normal Heart", as shown in Fig.~\ref{fig:method_overview} (c). 
% 说明缓解方法
Fortunately, the mediators $M$ can cut off the link between $Z$ and $F$, and $F=\{h_v,h_w\}$ is denoted the multi-modal feature. Mediators enable the intervention and adjustment of the relation between $F$ and $R$, thereby revealing the true causal relation between the two variables. 
Thus, we perform a causal intervention on the multi-modal feature prior to the decoder to estimate the deconfounded probability of the correct word, as shown in Fig.~\ref{fig:method_overview} (b). 
%This estimation is achieved by aggregating the probabilities of each sub-distribution of mediators, as illustrated in Fig.~\ref{fig:method_overview} (c).
Nevertheless, due to the absence of elaborate datasets and a well-trained feature extractor, we conduct a causal front-door intervention to eliminate the spurious correlations via mediators $M$. However, the estimation of both confounders and the mediator requires sufficient prior information, i.e., various visual-linguistic concepts. Therefore, we leverage the RadCARE to construct the correlation between the visual contexts and linguistic concepts before conducting the causal intervention using VLCI.
}

\begin{figure}
    \centering
    \includegraphics[width=1\linewidth]{figure/2.pdf}
\caption{\red{Comparison of the pipeline for (a) Baseline, i.e., non-causal model, and (b) our CMCRL framework. 
    %The mechanism of front-door causal intervention via the structural causal model (SCM) (c) is implemented in the example from Fig.~\ref{fig:intro_bias_info}, in contrast to the non-causal model.
    }}
        \label{fig:method_overview}
\end{figure}

\subsection{RadCARE}


In the radiological pre-training framework, there exist two difficulties: (1) the unpaired data that only has a single modality is hard to be utilized in supervised learning, and (2) heterogeneous data that makes it difficult to distinguish the region features because the morphology of the same lesion varies greatly~\cite{zhou2021review}. 
To address these challenges and provide fine-grained region features without region labels, we employ a visual-linguistic pre-training approach to learn and align visual-linguistic information, as shown in Fig.~\ref{fig:method_overview} (b). 
\red{
Unlike existing methods that simply combine Masked Language Modeling (MLM) with Masked Image Modeling (MIM) \cite{zhao2023mamo} or rely on contrastive learning for cross-modal alignment \cite{xvlm,singh2022flava,varma2023villa}, 
% 强调独特贡献
we integrate postfix text generation task with degradation-aware masked image restoration task to achieve efficient cross-modal representation learning and alignment. 
To address the issue of missing modalities, we implement alternating training using multiple data input schemes, surpassing existing approaches that rely on a single training strategy. 
Moreover, to effectively capture the subtle details in radiological images, we employ a degradation-aware mechanism for pixel-level cross-modal semantic learning, thereby enhancing the alignment of visual and linguistic information.
}

\begin{figure}
    \centering
    \includegraphics[width=1\linewidth]{figure/3.pdf}   
    \caption{\red{(a) shows four training strategies of our RadCARE, (b) demonstrates the detail of the multiway transformer blocks.}}
        \label{fig:method_radcare}
\end{figure}

% For language concepts, we employ Prefix Language Modeling (\red{PrefixLM}) to align with the visual modality utilizing bidirectional attention in image and prefix text.
% Meanwhile, to acquire visual context, we utilize degradation-aware Masked Image Modeling (MIM) for pixel-level semantic learning to capture more detailed features. 

%we utilize degradation of the image and masking the features extracted by CNN. 
% In the radiological pre-training framework, there exist two difficulties: (1) The unpaired data that only has a single modality is hard to be utilized in supervised learning, (2) heterogeneous data that makes it difficult to distinguish the region feature because the morphology of the same lesion varies greatly~\cite{zhou2021review}.
% Since the cross-modal pre-training provides fine-grained regional features without regional label~\cite{xvlm}, we utilize \red{PrefixLM} and MIM in linguistic and visual modeling to deal with unpaired data.

% 所以结合多模态预训练,实现V-T, T-V, V-V, T-T这几种情况
\red{Specifically, RadCARE introduces four strategies for pre-training as shown in Fig~\ref{fig:method_radcare} (a), including 
1) generation of postfix text with text prefixes as input, 
2) restoration of the masked image with both complete text and degraded images as input,
3) generation of postfix text with degraded images and text prefixes as input, and
4) restoration of the masked image with degraded images as input.
}
To enable flexible switching between various strategies, we use a multiway transformer to extract multi-modal features and two linear layers to solve \red{text generation and image restoration tasks}, respectively~\cite{wang2021simvlm, he2022masked}, as shown in Fig\ref{fig:method_overview} (b). In each block of the multiway encoder, the attention layer is weight-shared while the two feed-forward layers handle the corresponding modal features respectively~\cite{wang2022image}. Similarly, each block of the multiway decoder consists of a weight-shared self-attention layer, a pool of feed-forward networks used for different modalities, as shown in Fig.~\ref{fig:method_radcare} (b). Additionally, the noise present in the MIMIC-CXR dataset, such as tasks requiring differentiation between the left and right lung despite being given lateral views, hinders the model's ability to learn robustly \cite{tanno2019learning, wang2024robust}. To address this issue, we combined two datasets for pre-training, as the IU-Xray dataset contains both lateral and frontal views.

% SimVLM -> Prefix Language Model
\subsubsection{\red{Postfix Text Generation}} Motivated by the work of SimVLM~\cite{wang2021simvlm}, we extract image features from the first three blocks of ResNet101~\cite{he2016deep} as prefix tokens for the \red{RadCARE}. Simultaneously, the text is randomly divided into two parts, with one part generated by another under the guidance of the obtained image tokens. \red{In cases where the corresponding image is missing, the \red{RadCARE} can still be trained using only the text.} Let $h_v \in \mathbb{R}^{\frac{HW}{P^2}\times d}$ denote the image token extracted from the raw image $I$, where $P$ represents the patch size and $d$ is the embedding size. Then, $\{{w}_{{np}}, \ldots, {w}_{n}\}$ represents the postfix sequence following the textual description $h_w$ of length $n_p\ge 0$. Thus, the formulation is as follows:
\begin{equation}
    \mathcal{L}_{\textrm{\red{text}}}(\theta) = -\sum^{n}_{i=n_p}logP_\theta(w_{i}|h_v, h_{w_{<n_p}}),
\end{equation}
Here, $\theta$ denotes the trainable parameters of the model, $h_v$ represents the visual embedding with a trainable 2D positional encoding, $h_w$ is learned based on a fixed vocabulary and serves as the prefix received by the encoder, and $n$ denotes the length of the report.

%Motivated by SimVLM~\cite{wang2021simvlm}, we extract image features from the first three blocks of ResNet101~\cite{he2016deep} as prefix tokens in \red{PrefixLM}. Simultaneously, the text is divided randomly into two parts, of which one is generated by another under the guidance of the obtained image tokens. When the corresponding image is absent, the \red{PrefixLM} can also be trained with only text modality, which is the same as SimVLM. 
%Assume that $h_v\in\mathbb{R}^{\frac{HW}{P^2}\times d}$ is denoted as the image token extracted by the raw image $I$, where $P$ is the patch size, and $d$ is the embedding size. 
% Assume that $h_v\in\mathbb{R}^{k\times d}$ is denoted the image token extracted by the raw image $I$, where $k=\frac{HW}{P^2}$ is the length of the image tokens, $P$ is the patch size, and $d$ is the embedding size. 
%Then% $\{w_{i},\}_{i={L}_{p}}^{L}$ 
%$ \{{w}_{{n}_{p}}, \ldots, {w}_{n}  \}$
%is the postfix sequence after the textual description $h_w$ of length $n_p\ge 0$. Thus, the formulation is as follows:
% In this proxy task, we considered images as prefixes and the training objective becomes:
%\begin{equation}
%    \mathcal{L}_{\textrm{\red{PrefixLM}}}(\theta) = -\sum^{n}_{i=n_p}logP_\theta(w_{i}|h_v, h_{w_{<n_p}}),
%\end{equation}
%where $\theta$ is the trainable parameters of the model, $h_v$ is the visual embedding with a trainable 2D positional encoding, $h_w$ is learned for a fixed vocabulary and received by the encoder as the prefix, and $n$ is the report length. 

% Mask Image Model
\subsubsection{\red{Masked Image Restoration}} {To handle unpaired images, we leverage the MIM paradigm~\cite{he2022masked}. 
Furthermore, since \red{masked image restoration} can be trained using pairwise data~\cite{geng2022multimodal}, the missing semantics of masked images can be complemented by text, thereby enhancing cross-modal association. Additionally, we learn radiological image representations by reconstructing high-resolution patches from low-resolution inputs, which can encode more local information into latent embeddings~\cite{zhou2023advancing}.
Consequently, we sub-sample the images for degradation before the visual embedding and reconstruct the masked visual token extracted from CNNs by incorporating the semantics of both the unmasked visual token and the linguistic token.
This degradation-aware approach enables us to capture subtle differences in the dataset~\cite{huang2021gloria,cheng2023prior}.} The objective of \red{masked image restoration} can be formulated as:
\begin{equation}
    \mathcal{L}_{\textrm{\red{image}}}(\theta) = P_\theta(h_{vm}|h_{vv}, h_w),
\end{equation}
where ${h}_{vm}$ represents the masked visual tokens extracted by the ResNet backbone, ${h}_{vv}$ refers to the unmasked tokens, and $h_w$ corresponds to the word tokens of the entire report. %Subsequently, the ResNet and the multi-modal transformer encoder are employed as the Visual-Textual Representation Learning Model (VRLM) in downstream tasks.

% To further deal with unpaired images, we take advantage of \textbf{MIM}
%To deal with unpaired images like MAE~\cite{he2022masked}, we take advantage of the MIM paradigm. Additionally, since the MIM is trained with pairwise data, the missing semantics of masked images can be provided by text to enhance the cross-modal association~\cite{geng2022multi-modal}. 
%Thus, we reconstruct the masked visual token via the semantics of the unmasking visual token and linguistic token, which can learn the tiny difference in the dataset~\cite{seo2022masked}. 
%The target of the MIM can be formulated as follows:
%\begin{equation}
%    \mathcal{L}_{\textrm{MIM}}(\theta) = P_\theta(h_{vm}|h_{vv}, h_w),
%\end{equation}
%where ${h}_{vm}$ denotes the masked visual tokens extracted by the ResNet backbone, ${h}_{vv}$ is the unmasked tokens, and $h_w$ denotes the word tokens of the whole report. Then the ResNet and the multi-modal transformer encoder are utilized as VRLM in the downstream tasks.

\subsection{Visual-Linguistic Causal Intervention}
% Causal (重点是,mediator具体是什么)
% 在得到多模态的特征概念后,我们需要对其进行因果干预的微调。
% 开始推导公式同时表明与级联式特征提取的不同,我们是对原有特征的分布进行调整,后者是从原有特征中进一步提取特征。
% VDM采用全局与局部的特征融合作为mediator,全局信息相当于病灶的轮廓与位置等信息,而局部信息是当下最关键的几处位置的细节,如肺部阴影下的血管。
% LDM采用的mediator是通过全token与当前的视觉局部特征进行
After the \red{RadCARE}, the learned visual-linguistic feature encoders still contain visual and linguistic biases from cross-modal confounders~\cite{yang2021deconfounded}. 
Therefore, \red{we further employ visual-linguistic causal intervention (VLCI) to discover the causal effect between visual and linguistic modalities during RRG}, as shown in Fig.~\ref{fig:method_overview} (b).

\subsubsection{Preliminaries}
To clarify the mechanism of causal intervention, we introduce Pearl’s Structural Causal Model (SCM)~\cite{pearl2016causal}, as shown in Fig.~\ref{fig:method_scm}. The SCM is a mathematical framework used in causal inference to model the relations among variables and determine cause and effect relations. The SCM uses directed acyclic graphs (DAGs) to represent causal systems, where each variable is a node and the arrows represent causal relations between the variables.

% 还在改
\begin{figure}[t]
  \centering
  \includegraphics[width=1\linewidth]{figure/4.pdf}

  \caption{
  {The structural causal model (SCM) in (a) illustrates that $Y$ is directly caused by $X$ and indirectly influenced by confounders $Z$, as shown in (b). To address the confounding effect of $Z$, either by estimating the observable confounders $Z$ or by introducing a mediator $M$, the back-door and front-door interventions can be applied to block the path from $Z \to X$, as demonstrated in (c) and (d). In our proposed approach (e), we decompose the cross-modal confounders into visual ($Z_v$) and linguistic ($Z_l$) components. The front-door causal intervention is implemented by the mediators $M_v$ and $M_l$, effectively blocking the paths $Z_v \to h_v$ and $Z_l \to h_w$ in (f).}}

  \label{fig:method_scm}
\end{figure}

In Fig.~\ref{fig:method_scm}(a), the chain structure $X \to Y$ represents the output $Y$ is affected by the input $X$, formulated as $P(Y|X)$. But the confounders $Z$ caused by data bias would lead to a spurious correlation, as shown in Fig.~\ref{fig:method_scm}(b). 
This graph indicates that two causes (both X and Z) lead to the common effect or outcome (Y), such as the cross-modal confounders of ``Normal Heart" and cross-modal feature of ``Pleural" leading to a confounded estimation of ``No Effusion" (the spurious correlation is ``Normal Heart" and ``No Effusion"). In this case, we formulate $P(Y|X)$ as:
\begin{equation}
\begin{aligned}
    P(Y|X) = \sum_{z} P(Y|X, Z=z)P(Z=z|X),
\end{aligned}
\end{equation}
where the confounders $Z$ generally brings about the observational bias via $P(z|X)$~\cite{liu2022show}.
%%%%%%%%%%%%%%%%%%%%%%%%%%%%%%%%%%%%%%%%%%%%%%%%%%%%%%%%%
% 后门干预 -> 前门干预
To alleviate this issue, back-door causal intervention can be implemented by introducing the do calculus $do(\cdot)$~\cite{liu2022show, liu2023cross, lopez2017discovering, qi2020two}. \red{In Fig.~\ref{fig:method_scm} (c), w}e can calculate the probabilities of observable confounders and block the back-door path $Z \to F$, the interventional probability is as follows:
\begin{equation}
\begin{aligned}
    P(Y|do(X))=\sum_{z} P(Y|X, Z=z)P(Z=z),
    \label{eq:back_door_do}
\end{aligned}
\end{equation}
where $Z$ can be the learned RoI features of the heart and the attended word feature of ``Normal Heart". \red{Based on the conditional probabilities following intervention, it can estimate the true causal effects of each factor in the absence of confounding influences. These individual causal effects can then be aggregated to derive the overall causal effect under the given input conditions.} However, the back-door causal intervention is limited by the observability of confounders, and the front-door causal intervention provides a method of deconfounding in Fig.~\ref{fig:method_scm} (d). To eliminate the unobservable confounder, we introduce the mediator $M$ to cut off the link $X \gets Z \to Y$. The total probability $P(Y|do(X))$ can be represented as the following summation:
\begin{equation}
\begin{aligned}
    P(Y|do(X){)}=\sum_{m}P(Y|do(X),M=m)P(M=m|do(X)),
    \label{eq:front_door_do_1}
\end{aligned}
\end{equation}
where $M$ is introduced by $X$ without the back-door path. Thus, the intervention probability is equal to the conditional probability in the path $X \to M$~\cite{liu2023cross}. Besides, there is no direct causal path between $X$ and $Y$. In this way, the introduced summation in Eq.~(\ref{eq:front_door_do_1}) can be reformulated as:
\begin{equation}
\begin{aligned}
    &P(Y|do(X){)}\\
    &=\sum_{m}P(Y|do(X), do(M=m))P(M=m|X=x)\\
    &=\sum_{m}P(Y|do(M=m))P(M=m|X=x).
    \label{eq:front_door_do_2}
\end{aligned}
\end{equation}
To estimate $P(Y|do(M=m))$, we can apply the back-door intervention to cut off the link $M \gets X \gets Z \to Y$~\cite{liu2022show}. Therefore, we have the intervention probability formulation:
\begin{equation}
\begin{aligned}
    &P(Y|do(M=m))\\
    &=\sum_{\hat{x}}P(Y|do(M=m),X=\hat{x})P(X=\hat{x}|do(M=m))\\
    &=\sum_{\hat{x}}P(X=\hat{x})P(Y|X=\hat{x},M=m),
    \label{eq:front_door_do_3}
\end{aligned}
\end{equation}
where $\hat{x}$ is the selected features from $X$ not caused by $M$. Finally, we can calculate Eq.~(\ref{eq:front_door_do_2}) by applying Eq.~({\ref{eq:front_door_do_3}}):
\begin{equation}
\begin{aligned}
    &P(Y|do(M=m))=\\
    &\sum_{m}P(M=m|X=x)\sum_{\hat{x}}P(X=\hat{x})P(Y|X=\hat{x},M=m).
    \label{eq:front_door_do}
\end{aligned}
\end{equation}
However, existing front-door causal intervention methods employ diverse approaches for estimating mediators and conducting interventions. 
% 说明e
In {RRG}, we construct the structural causal model (SCM) similar to Image Caption~\cite{liu2022show}, assuming that $h_v$ is the visual feature, $h_w$ is the linguistic feature of the attended word, and $h$ is the fusion feature from $h_v$ and $h_w$ ($h_v \to h \gets h_w$), leading to the generation of report $R$. 
Moreover, $h_v$ would be influenced by $h_w$ through cross-attention and makes $h_v \gets h_w$, as shown in Fig.~\ref{fig:method_scm} (e).

\red{Previous method  ~\cite{liu2022show} employed a pre-constructed semantic dictionary for backdoor intervention, without accounting for complex representations found in radiological imaging. Differently, we construct mediators internally within the model to simulate front-door causal interventions, as illustrated in Figure~\ref{fig:method_vlci}. We further mine and construct mediators separately from visual and linguistic aspects by leveraging semantic correlations obtained during the pre-training phase. Finally, we estimate the actual causal effects by conditioning each mediator through mediator-based sub-distributions. In contrast, the baseline merely relies on the confounded distribution for estimation. To address this, we introduce the Front-Door Intervention Module (FIM) based on Equation (\ref{eq:front_door_do}) to disentangle the effects and construct the Visual Mediator Module (VDM) and Linguistic Mediator Module (LDM) separately.}
% Following that, we divide the mediator into visual and linguistic modalities and introduce a Front-door Intervention Module (FIM) based on Equation~(\ref{eq:front_door_do}) to deconfound them.

% by dividing it into visual and linguistic modalities, as illustrated in Fig.~\ref{fig:method_scm} (e). Consequently, we introduce a Front-door Intervention Module (FIM) based on Equation~(\ref{eq:front_door_do}) to deconfound visual and linguistic modalities.

% TODO 从这里开始改

\subsubsection{Front-door Intervention Module (FIM)}

\begin{figure}[t]
 \centering
    \includegraphics[width=1\linewidth]{figure/5.pdf}
 
    \caption{Illustration of the Front-door Intervention Module (FIM), which consists of two Attention Fusion layers with non-parameters.
    }
    \label{fig:method_fim}
\end{figure}

In Fig.~\ref{fig:method_scm} (e), the causal effects $h_v\to R$ and $h_w\to R$ are affected by the confounders $Z=\{Z_v,Z_l\}$ from back-door paths $h_v \gets Z_v\to R$ and $h_w \gets Z_l\to R$  ~\cite{liu2022show}, respectively. In our SCM, the non-interventional prediction can be expressed as:
\begin{equation}
\begin{aligned}
    &P(R|I) = P(R|h_v, h_w)\\
    &=\sum_{i=1}^{n}\sum_{z} P(w_{i}|h_v, h_w, Z=z)P(Z=z|h_v, h_w),
    \label{eq:mrg_non_do}
\end{aligned}
\end{equation}
% The confounders and mediator
where $Z$ brings the spurious correlation via $P(Z=z|h_v, h_w)$, leading to incorrect reports.
$h_v$ is the visual token from visual embedding, and $h_w$ is the linguistic token of the attended word from linguistic embedding.

Taking Fig.~\ref{fig:intro_bias_info} as an example, when $P(Z=\textrm{``Normal~Heart''}|h_v=\textrm{``Heart"}, h_w=\textrm{``Normal''})$ is large while $P(Z=\textrm{``Cardiomegaly"}|h_v=\textrm{``Heart''}, h_w=\textrm{``Normal"})$ is small in the training set, it tends to enlarge $P(R=\textrm{``No~Effusion"}|h_v, h_w, Z=\textrm{``Normal~Heart"})$ in the testing set. 

%shown in Fig.~\ref{fig:intro_bias_info}.
To mitigate visual-linguistic confounders and uncover the true causal structure, we apply causal front-door intervention by introducing mediator $M_v$ and $M_l$, respectively, as shown in Fig.~\ref{fig:method_scm} (f).
Generally, $Z_v$ is unobservable without a well-trained object detector, and the back-door path $h_v\gets Z_v \to R$ can be blocked by $M_v$ via learning the true causal effect $h_v \to M_v \to h\to R$. Similarly, the intervention on the back-door path $h_v\gets h_w\gets Z_l\to R$ can be implemented by calculating the $M_l$ without well-constructed confounders dictionaries. Thus, we can formulate Eq.~(\ref{eq:mrg_non_do}) as:
\begin{equation}
\begin{aligned}
    P(R|do(I))=P(R|do(h_v), do(h_w))
    \label{eq:mrg_do}
\end{aligned}
\end{equation}

To further estimate Eq.~(\ref{eq:mrg_do}) with the deep learning framework using Eq.~(\ref{eq:front_door_do}), we adopt Normalized Weighted Geometric Mean (NWGM)~\cite{xu2015show}
as:
\begin{equation}
\begin{aligned}
    P(R|do(h_v), do(h_w)) \approx \textrm{Softmax}(g(h_w, h_v, \hat{M_v}, \hat{M_l})).
    \label{eq:done}
\end{aligned}
\end{equation}
where $g(\cdot)$ denote the network mapping functions {including the transformer decoder and a linear layer}, $\hat{M_v}$ and $\hat{M_l}$ denote the estimations of $M_v$ and $M_l$ via VDM and LDM. 
% TODO 为什么这么设计FIM?
In Fig.~\ref{fig:method_fim}, the FIM consists of two Attention Fusion layers and it is integrated into VDM and LDM. Different from the cascaded transformer~\cite{nguyen2022grit}, FIM does not have optimized parameters. It relies on extracted features and intervenes using estimated mediators of the corresponding modalities. This approach ensures that the estimation of mediators is entirely dependent on the sampling in VDM and LDM rather than FIM, achieving learning of mediators.

%%%%%% 以直观的方式说明一下干预的操作,也就图6


\begin{figure}[t]
 \centering
    \includegraphics[width=1\linewidth]{figure/6.pdf}

    \caption{Illustration of the Visual Deconfound Module (VDM) and linguistic Deconfound Module (LDM).
    }
    \label{fig:method_vdm_ldm}
\end{figure}


\subsubsection{Visual Deconfounding Module (VDM)}
% 为什么要用local和gobal
% local 的high attn token 代表了局部的细节特征,global是类似心脏轮廓和位置的全局性的特征
% 心脏轮廓影响了胸积液的判断,同时肺部的纹理细节也可以有助于判断
% In Fig.~\ref{fig:method_vlci}(d), we calculate the visual mediator $M_v$ via local feature $h_{vl}$ from local sampling and global feature $h_{vg}$ from global sampling. The $h_{vl}$ is denoted as the local detail information acquired from Local Sampling, while the $h_{vg}$ is the contours and position feature acquired from Global Sampling~\cite{sun2022lesion}. 
% For instance, the contour of the heart affects the determination of pleural effusion, and the texture of the lungs can also be the basis of detection.
In Fig.~\ref{fig:method_vdm_ldm}, the visual mediator $M_v$ is calculated using local features $h_{vl}$ obtained from local sampling and global features $h_{vg}$ obtained from global sampling.
The features $h_{vl}$ represent local details acquired from Local Sampling, while $h_{vg}$ represents contour and position features obtained from Global Sampling~\cite{sun2022lesion}.
For example, the contour of the heart influences the determination of pleural effusion, and the texture of the lungs can also serve as a basis for detection.

\noindent\textbf{Local Sampling.} 
\red{We leverage the attention accumulated from the encoder to select the top $k$ tokens~\cite{he2022transfg}.} These selected visual tokens with high attention correspond to the report's keywords as $h_{vl}\in\mathbb{R}^{k\times d}$, where $k=6$ for each attention head, and $d$ is the dimension of the transformer. Subsequently, $h_{vl}$ is further enhanced using CaaM \cite{wang2021causal}, which excavates the local internal relations. Specifically, these highly attended tokens are computed not only with self-attention but also with negative attention scores, aiming to achieve more robust features. The purpose of $h_{vl}$ is to capture crucial local details in the image, which serve as the key basis for further processing.

% Inspire by TransFG \cite{he2022transfg}, we use the attention accumulated from the encoder to select top $k$ tokens. These selected visual tokens with high attention can correspond to the keywords of the report as $h_{vl}\in\mathbb{R}^{k\times d}$, where $k=6$ for each head of attention, and $d$ is the dimension of the transformer. Then, $h_{vl}$ is enhanced via CaaM \cite{wang2021causal}, which further excavates the local internal relations. Specifically, these highly attended tokens in CaaM are computed not only with self-attention but also with negative attention scores, aiming to achieve more robust features. The $h_{vl}$ aims to obtain local critical details in the image, which can be used as the key basis.

\noindent\textbf{Global Sampling.} The global sampling is implemented by Down Sampling Transformer block, in which the $14\times 14$ visual tokens are down-sampled to $7\times 7$ as $h_{vg}\in\mathbb{R}^{49\times d}$. Max pooling in this block can better retain the global structure information in the image as the general features of the data itself. We formulate the operation as follows:
\begin{equation}
    h_{vg} = W[\textrm{P}(h_v) + \textrm{Attn}(\textrm{P}(\textrm{LN}(h_v))],
    \label{eq:dst}
\end{equation}
where P is the 2d max pooling layer, LN is layer normalization, Attn is the 2d relative attention \cite{dai2021coatnet}, and $W$ denotes the weights of the linear layer.

\noindent\textbf{Local-Global Fuse Module.} 
Finally, the $h_{vl}$ is integrated with $h_{vg}$ to enhance local details with global structural information via Local-Global Fuse Module formulated as Eq.~(\ref{eq:LGFM}), namely mediator $M_v$. 
\begin{equation}
    M_v=\textrm{FFN}([\textrm{MHA}(h_{vl}, h_{vl}, h_{vl}), \textrm{MHA}(h_{vl}, h_{vg}, h_{vg})])
    \label{eq:LGFM}
\end{equation}
where MHA and FFN are the Multi-Head Attention layer and Feed-Forward Network layer, respectively. $[\cdot,\cdot]$ denotes concatenation.

\begin{figure}[t]
  \centering
  \includegraphics[width=1\linewidth]{figure/7.pdf}

  \caption{
  \red{Demonstration of the causal intervention mechanism, we integrate visual local-global features as visual mediators and combined visual local features with the vocabulary as linguistic mediators. Through their sub-distributions, we further estimate the causal effects in the RRG task.}}
 
  \label{fig:method_vlci}
\end{figure}

\subsubsection{Linguistic Deconfounding Module (LDM)}
% Textual Confounder
For linguistic deconfounding, we have some observations from $h_v \gets h_w\gets Z_l\to R$ (Fig.~\ref{fig:method_scm} (e)): (1) the linguistic contexts can affect the generation of the next word, and (2) the attended word features affect the attended visual features via cross-attention~\cite{liu2022show}.
% why, 因为文本的上下文和来自视觉的高注意力特征都会影响文本的生成
% 为什么在embedding space做,词频的差异也会带来很大的距离偏差,因此词向量的距离就不能很好地代表语义相关性
Additionally, the difference in word frequency brings a large distance deviation in embedding space, so the distance of word vectors cannot represent semantic relevance well~\cite{li2020sentence}. 

\noindent\textbf{Visual-Linguistic Fuse Module.} 
Thus, we calculate the linguistic mediator $M_l$ in embedding space via all vocabularies from the tokenizer as the global features and use $h_{vl}$ obtained from the VDM, which estimates the current word frequency to adjust the distribution of $h_w$, see in Fig.~\ref{fig:method_vdm_ldm}. 
% formulation
\begin{equation}
\begin{aligned}
   &h^{'}_{vl} = \textrm{FFN}(\textrm{MHA}(h_{vl}, \hat{w}, \hat{w})); \\
   &M_t = \textrm{FFN}(\textrm{MHA}(h^{'}_{vl}, h_{vl}, h_{vl}))
\end{aligned}
\end{equation}
where $\hat{w}$ denotes all word tokens from the tokenizer. Then, we build the causal effect $ h_w \to M_l \to h_v \to M_v \to h \to R$ to cut off the back-door path $h_v \gets h_w\gets Z_l \to R$ via $M_l$.

In Fig.~\ref{fig:method_vlci}, the deconfounded visual and linguistic features are fed to the decoder to learn fused cross-modal features. The output layer is a linear projection with softmax operation, mapping probability into the dimensional equal to the vocabulary size. Finally, the training target is to minimize the negative log-likelihood loss according to Eq.~(\ref{eq:mrg_do}) and Eq.~(\ref{eq:done}):
\begin{equation}
\begin{aligned}
    &\mathcal{L}_{\textrm{nll}}(\theta) = -\sum_{i=1}^{n}log (\textrm{Softmax}(g(h_{w_{<i}}, h_v, \hat{M_v}, \hat{M_l})))
\end{aligned}
\end{equation}
where $n$ is the length of the report and $h_{w_{<i}}$ is the prefix text when estimating the word $w_i$.
\section{Experiment} %(7)
% 1. 实验设置,参数说明,CE Metric的说明(重点)
% 2. Main Result,这里只对比非知识图谱的方法,同时强调我们方法的轻量与快速。
% 3. 医学上的分析需要围绕CE Metric上进行,目前也是同规模最优的,同时超越了部分基于知识的方法。
% 4. ablation,同时在开源的方法中,我们即插即用的因果模块可以进一步提升原有模型 (MAE和\red{PrefixLM}的ablation不是重点,舍去)
% 5. 在image caption的任务中同样有效
% 6. 进一步分析可视化的结果,以及生成的例子

\subsection{Experimental settings}


% model scale
\begin{table}[]\renewcommand\arraystretch{0.9}
  \centering
  \setlength{\tabcolsep}{0.7mm}{
    \caption{The details of \red{CMCRL} and several comparison RRG models, the \#Enc and \#Dec denote the number of transformer layers in the encoder and decoder, respectively. 
    The marker $\clubsuit$ means 2 Contrastive Attentions, and $\spadesuit$ means Hierarchical LSTM. 
    The backbone of \red{CMCRL} is the first three blocks of Resnet101. 
    Besides, we show the employed model boosting modules, including the knowledge-aware module $\mathcal{K}$, template retrieval module $\mathcal{T}$, and memory-drive module $\mathcal{M}$.
    Additionally, we adopt two scales of the \red{CMCRL}, $\textbf{\red{CMCRL}}_3$ for the IU-Xray dataset and $\textbf{\red{CMCRL}}_6$ for the MIMIC-CXR dataset.
    }

  \label{tab:model_scale}
  \begin{tabular}{@{}lllccccc@{}}
    \toprule
    &Method              & Visual Embedding &  \#Enc  &   \#Dec & $\mathcal{K}$ & $\mathcal{T}$ & $\mathcal{M}$ \\
    \midrule
    \multirow{5}{*}{\rotatebox{90}{Light}}
    &R2Gen\cite{chen2020generating}               & Resnet101         & 3 & 3 &  &  & $\surd$\\
    % chen2022cross
    &R2GenCMN\cite{chen2022cross}               & Resnet101         & 3 & 3 &  &  & $\surd$\\
    &CMCL\cite{liu2022competence}                & Resnet50          & - & $\spadesuit$ &  &  &  \\
    &PPKED\cite{liu2021exploring}               & Resnet152         & 2 & 1 & $\surd$ & $\surd$ &   \\
    &CA\cite{liu2021contrastive}                  & Resnet50          & $\clubsuit$ & $\spadesuit$ &   & $\surd$ &  \\
    &AlignTransformer\cite{you2021aligntransformer}    & Resnet50          & 3 & 3 & $\surd$  &   & $\surd$  \\
    &MMTN\cite{cao2023mmtn}                & DenseNet121              & 3 & 3 & $\surd$ &   & $\surd$ \\
    &\red{CAMANet~\cite{wang2024camanet}}&  \red{DenseNet121}              & \red{3} & \red{3} & &  &  \\
    &\red{SSVE~\cite{divya2024memory}}&  \red{ResNet101}              & \red{3} & \red{3} & &  & \red{$\surd$} \\
    \midrule
    \multirow{3}{*}{\rotatebox{90}{Heavy}}
    &M2TR\cite{nooralahzadeh2021progressive}                & Densenet151       & 6 & 12 &   &  & $\surd$\\
    %Self-boost          & Resnet101         & 12 & $\spadesuit$ &   &   &    \\
    &MGSK\cite{yang2022knowledge}             & Resnet101         & 12 & 3 & $\surd$ & $\surd$ &  \\
    %&MSAT\cite{wang2022medical}                & CLIP              & 6 & 6 & $\surd$ &   & $\surd$ \\
    &RAMT\cite{zhang2023semi}                & DenseNet121              & 6 & 6 & $\surd$ &   &  \\
    &DCL\cite{li2023dynamic}                & ViT Linear Projection           & 25 & 1 & $\surd$ & $\surd$  &  \\
    &\red{Med-LLM\cite{liu2024context}}& \red{MedCLIP-ViT} & \red{Q-former} & \red{32} & & \\
    \midrule
    &$\textbf{\red{CMCRL}}_3$                & Resnet101*        & 3 & 3 &  &  &  \\
    &$\textbf{\red{CMCRL}}_6$                & Resnet101*        & 6 & 6 &  &  &  \\
    \bottomrule
  \end{tabular}}

\end{table}



%\subsection{Datasets, Metrics and Settings}
\subsubsection{Datasets}
%\noindent 
\noindent \textbf{IU-Xray}~\cite{jamiaocv080}, also known as the Indiana University Chest X-ray Collection, is a publicly available radiological dataset widely used to evaluate the performance of {RRG} methods. It comprises 7,470 chest images and 3,955 corresponding reports, with an average report length of 30. To maintain consistency, we follow the setting used in R2Gen~\cite{chen2020generating} and partition the dataset into training, validation, and testing sets at a ratio of 7:1:2, ensuring that there is no overlap in patients. We tokenize the words with more than 3 occurrences and set the max length as 60. Note that we adopt two images (frontal and lateral views) as input for each sample.
% with 2,955 image-text pairs

\noindent \textbf{MIMIC-CXR}~\cite{johnson2019mimic} is a large-scale chest radiological dataset, with 377,110 images and 227,835 corresponding reports, of which the average length of reports is 48 in the training/val set and 61 in testing set. We use the official paradigm, the dataset is divided into the training set with 368,960 images and 222,758 reports, the validation set with 2,991 images and 1,808 reports, and the testing set with 5,159 images and 3,269 reports. Different from the IU-Xray dataset, MIMIC-CXR has instances of a single modality (only image or report) as well as multiple images corresponding to one report. Therefore, we use a single image as the input and tokenize the words with more than 10 occurrences, and set the max length as 80. 
% with 279,733 image-report pairs


\begin{table}[t]
  \centering
  \setlength{\tabcolsep}{1mm}{
    \caption{Comparison of computational cost between \red{CMCRL} with R2Gen and R2GenCMN. The $1^{st}$ and $2^{nd}$ best results are bolded and underlined, respectively.}

  \label{tab:model_cost}
  \begin{tabular}{@{}lllll@{}}
    \toprule
    Method          & Inference Time (s)  & Params (M) & FLOPs (TFLOPs) & BLEU-4\\
    \midrule
    R2Gen         & 97.14 & 78.07 & \textbf{3.66} & {10.3}\\ 
    R2GenCMN  & \textbf{40.25} & \textbf{58.65}& 12.98 & {\underline{10.5}}\\ 
    \red{CMCRL}       & \underline{97.10} & \underline{69.41} & \underline{10.50} & {\textbf{11.9}}\\ 
    \bottomrule
  \end{tabular}}
\end{table}

% main
\begin{table*}
  \centering
  \setlength{\tabcolsep}{4mm}{
    \caption{The performances of \red{CMCRL} and other methods on IU-Xray and MIMIC-CXR datasets. The ${1}^{st}$ and ${2}^{nd}$ best results are bolded and underlined, respectively.
    For some methods, the results are missing and denoted by a ``-". 
  }

  \label{tab:main_result}
  \begin{tabular}{@{}lccccccccc@{}}
    \toprule
    Method             & BLEU-1 & BLEU-2 & BLEU-3 & BLEU-4  & Rouge-L & METEOR & Precision & Recall & F1\\
    \midrule
    \multicolumn{10}{c}{\textbf{IU-Xray Dataset}} \\
    \midrule
    %\multirow{5}{*}{\rotatebox{0}{\scriptsize Lightweight}}
                    R2Gen\cite{chen2020generating}              & {47.0} & {30.4} & {21.9} & {16.5}    & {37.1} & {18.7} & - & - & - \\ 
                    CMCL\cite{liu2022competence}               & {47.3} & {30.5} & {21.7} & {16.2}    & {37.8} & {18.6} & - & - & - \\ 
                    PPKED\cite{liu2021exploring}              & {48.3} & {31.5} & {22.4} & {16.8}   & {37.6} & {19.0} & - & - & - \\
                    CA\cite{liu2021contrastive}                & {49.2} & {31.4} & {22.2} & {16.9}   & {38.1} & {19.3} & - & - & - \\
                    AlignTransformer\cite{you2021aligntransformer}   & {48.4} & {31.3} & {22.5} & {17.3}    & {37.9} & {\underline{20.4}} & - & - & - \\
    %\cmidrule{1-11}
    %\multirow{2}{*}{\rotatebox{0}{\scriptsize Heavyweight}}
                    M2TR\cite{nooralahzadeh2021progressive}               & {48.6} & {31.7} & {23.2} & {17.3}   & {\underline{39.0}} & {19.2} & - & - & - \\
                    MGSK\cite{yang2022knowledge}             & {\underline{49.6}} & {\underline{32.7}} & {\underline{23.8}} & {\underline{17.8}} & {38.1} & - & - & - & - \\

                    RAMT\cite{zhang2023semi}          & {48.2} & {31.0} & {22.1} & {16.5}  & {37.7} & {19.5} & - & - & - \\
                    MMTN\cite{cao2023mmtn}            & {48.6} & {32.1} & {23.2} & {17.5}  & {37.5} & - & - & - & - \\
                    %&Self-boost\#         & 0.487 & \textbf{0.346} & \textbf{0.270} & \textbf{0.208} &  \textbf{0.452} & 0.359 & /\\
    %\cmidrule{1-12}
                    % XPRONET & \textbf{0.525} & \underline{0.357} & \textbf{0.262} & \textbf{0.199} & - & \textbf{0.411} & \underline{0.220} & - & - & /\\
                    DCL\cite{li2023dynamic} & - & - & - & {16.3} & {38.3} & {19.3} & - & - & - \\
                    % KiUT & \textbf{0.525} & \textbf{0.360} & \textbf{0.251} & 0.185 & - & \textbf{0.409} & \textbf{0.242}\\
                    \red{SSVE\cite{divya2024memory}} & \red{49.2} & \red{32.1} & \red{23.3} & \red{18.0} & \red{37.9} & \red{20.3} & \red{-} & \red{-} & \red{-} \\
                    \red{Med-LLM\cite{liu2024context}} & \red{-} & \red{-} & \red{-} & \red{16.8} & \red{38.1} & \red{20.9} & \red{-} & \red{-} & \red{-} \\
                    $\textbf{\red{CMCRL}}_3 \textbf{(ours)}$               & {\textbf{50.5}} & {\textbf{33.4}} &  {\textbf{24.5}} & {\textbf{19.0}} & {\textbf{39.4}} & {\textbf{21.0}} & - & - & - \\\hline
    \midrule
    \multicolumn{10}{c}{\textbf{MIMIC-CXR Dataset}} \\
    \midrule
    %\multirow{5}{*}{\rotatebox{0}{\scriptsize Lightweight}}
                    R2Gen\cite{chen2020generating}                   & {35.3} & {21.8} & {14.5} & {10.3} &  {27.7}  & {14.2} & {33.3} & {27.3} & {27.6}\\ 
                    R2GenCMN\cite{chen2022cross}                     & {35.6} & {21.9} & {14.7} & {10.5} &  {27.8}  & {14.1} & {33.4} & {27.5} & {27.8}\\ 
                    CMCL\cite{liu2022competence}                     & {33.4} & {21.7} & {14.0} & {09.7} &  {28.1} & {13.3} & - & - & - \\ 
                    PPKED\cite{liu2021exploring}                     & {36.0} & {22.4} & {14.9} & {10.6} &  {\textbf{28.4}} & {14.9} & - & - & - \\
                    CA\cite{liu2021contrastive}                      & {35.0} & {21.9} & {15.2} & {10.9} &  {\underline{28.3}} & {15.1} & {35.2} & {29.8} & {30.3}\\
                    AlignTransformer\cite{you2021aligntransformer}   & {37.8} & {23.5} & {15.6} & {11.2} &  {\underline{28.3}} & {\underline{15.8}} & - & - & - \\
                    M2TR\cite{nooralahzadeh2021progressive}          & {37.8} & {23.2} & {15.4} & {10.7} &  {27.2} & {14.5} & {24.0} & {\textbf{42.8}} & {30.8}\\
                    MGSK\cite{yang2022knowledge}                     & {36.3} & {22.8} & {15.6} & {11.5} &  {\textbf{28.4}} & -           & {45.8} & {34.8} & {37.1} \\
                    RAMT\cite{zhang2023semi}                         & {36.2} & {22.9} & {15.7} & {11.3} &  {\textbf{28.4}} & {15.3}  & {38.0} & {34.2} & {33.5} \\
                    MMTN\cite{cao2023mmtn}                           & {\underline{37.9}} & {\underline{23.8}} & {\underline{15.9}} & {\underline{11.6}} & {\underline{28.3}} & {\textbf{16.1}} & - & - & - \\
                    % XPRONET & 0.344 & 0.215 & 0.146 & 0.105 & - & 0.279 & 0.138 & - & - & /\\
                    DCL\cite{li2023dynamic}  & - & - & - & {10.9} & {\textbf{28.4}} & {15.0} & {\underline{47.1}} & {\underline{35.2}} & {\underline{37.3}}\\
                    % PromptMRG &
                    % SSVE & 37.3 & 23.0 & 16.1 & 11.2 & \\
                    \red{CAMANet\cite{wang2024camanet}} & \red{37.4} & \red{23.0} & \red{15.5} & \red{11.2} & \red{27.9} & \red{14.5} & \red{48.3} & \red{32.3} & \red{38.7} \\
                    % KiUT & \underline{0.393} & \underline{0.243} & \underline{0.159} & 0.113 & - & \textbf{0.285} & \textbf{0.160}& 0.371 & 0.318 & 0.321\\
    %\cmidrule{1-11}
                    $\textbf{\red{CMCRL}}_6 \textbf{(ours)}$              & {\textbf{40.0}} & {\textbf{24.5}} & {\textbf{16.5}} & {\textbf{11.9}} & {28.0} & {15.0} & {\textbf{48.9}} & {34.0} & {\textbf{40.1}}\\
    \bottomrule
  \end{tabular}}
\end{table*}


\subsubsection{Baseline Models}
We compare the proposed \red{CMCRL} model with several state-of-the-art {RRG} models, including R2Gen \cite{chen2020generating}, R2GenCMN \cite{chen2022cross}, CMCL \cite{liu2022competence}, PPKED \cite{liu2021exploring}, CA \cite{liu2021contrastive}, AlignTransformer \cite{you2021aligntransformer}, M2TR \cite{nooralahzadeh2021progressive}, RAMT \cite{zhang2023semi}, MMTN \cite{cao2023mmtn}, MGSK \cite{yang2022knowledge}, DCL \cite{li2023dynamic}, \red{SSVE\cite{divya2024memory}, CAMANet\cite{wang2024camanet}, and Med-LMM\cite{liu2024context}}. These models are categorized into lightweight and heavyweight models. The lightweight models comprise no more than 3-layer modules for both the encoder and the decoder, as detailed in Table~\ref{tab:model_scale}. Most of these models incorporate various modules to enhance their performance, such as the knowledge-aware module, template retrieval module, and memory-drive module, which can be computationally expensive.  It is important to note that the total parameters of our \red{CMCRL} are less than the R2Gen, while \red{CMCRL} is more efficient by eliminating the recursive memory calculation dependency, as shown in Table~\ref{tab:model_cost}.


\subsubsection{Evaluation Metrics} {We adopt the widely used natural language generation (NLG) metrics, including 
BLEU~\cite{papineni2002bleu}, a metric evaluates how similar the generated text is to reference texts by calculating n-gram precision.
ROUGE-L~\cite{rouge2004package} evaluates both precision and recall and also considers synonyms, stemming, and paraphrasing.
METEOR~\cite{banerjee2005meteor} focuses on the longest common subsequence between the generated text and the reference text, capturing fluency and coherence.
CIDEr~\cite{vedantam2015cider} considers term frequency-inverse document frequency (TF-IDF) weighting to reduce the impact of overly common words. %However, we excluded the use of the CIDEr metric in the section on Quantitative Analysis, as more than half of the methods did not adopt it.
Since the {RRG} specifically focuses on the abnormality detection accuracy rather than the text fluency and similarity with the real report, we further adopt clinical efficacy (CE) metrics~\cite{chen2020generating, liu2021contrastive, nooralahzadeh2021progressive, yang2022knowledge}. It is calculated by the labels extracted from CheXpert~\cite{irvin2019chexpert}. 
Specifically, the extracted positive labels are considered as positives, while the non-positive labels (negative, not mentioned, and uncertain) are treated as negatives. Using this approach, we calculate the micro-precision, micro-recall, and micro-F1 scores between the labels from reference reports and generated reports.}


\subsubsection{Implementation Settings} We use the first three blocks of ResNet101~\cite{he2016deep} to extract 1,024 feature maps, which are projected into 512 maps of size $14 \times 14$. 
The dimension of the transformer layers and the number of attention heads are fixed to 512 and 8, respectively. 
In the IU-Xray dataset (referred to as \red{CMCRL}$_3$), both the encoder and decoder consist of 3 layers, whereas in the MIMIC-CXR dataset (referred to as \red{CMCRL}$_6$), both the encoder and decoder comprise 6 layers, as illustrated in Table~\ref{tab:model_scale}. The variation in the number of layers can be attributed to the significantly larger size of the MIMIC-CXR dataset in comparison to the IU-Xray dataset.
% A. framework of VLP, B. Mask rate C. Setting of Baseline
During pre-training, we utilize a dataset that combines IU-Xray and MIMIC-CXR datasets, resulting in 4,347 unique words. We perform fine-tuning on these two datasets separately with the same tokenizer as \red{RadCARE}. The batch size is set to 64 during pre-training and 16 during fine-tuning.
In the pre-training stage, we adopt an image mask rate of 85\% for \red{masked image restoration}.  
The \red{RadCARE} is trained using the AdamW optimizer with a warm-up step of 10\% of the total training steps, and the peak learning rate is set to 5e-4.  The weight decay of the optimizer is set to 1e-2, and the total epochs are set to 30 in pre-training.
%, and set to 100 and 30 for the IU-Xray and MIMIC-CXR datasets, vely. 
In the fine-tuning stage, the model is fine-tuned using the Adam optimizer with an initial learning rate of 1e-5 and a weight decay of 5e-5 for 10 epochs on both the IU-Xray and MIMIC-CXR datasets.


\subsection{Quantitative Analysis}
% 总体性能 解释CIDEr差一点
% NLG
\subsubsection{NLG Metric}
As shown in Table~\ref{tab:main_result}, our \red{CMCRL} outperforms nearly all the {RRG} methods. Specifically, compared with the lightweight model MMTN~\cite{cao2023mmtn}, \red{CMCRL}$_3$ significantly improves the BLEU-1 metric by 1.9\% on the IU-Xray dataset. 
Similarly, when compared with the heavyweight model RAMT~\cite{zhang2023semi}, \red{CMCRL}$_6$ achieves a substantial 3.8\% boost in the BLEU-1 metric on the MIMIC-CXR dataset. This significant improvement in the BLEU-1 metric highlights our method's ability to select words more precisely.  Notably, on the IU-Xray dataset, our tokenizer incorporates words from the MIMIC-CXR dataset, leading to even greater estimation challenges.
To assess text fluency more effectively, we also consider the precise match of four consecutive words using the BLEU-4 metric. Our \red{CMCRL} demonstrates a 1.5\% improvement over MMTN on the IU-Xray dataset and a 0.6\% improvement over RAMT on the MIMIC-CXR dataset.  Notably, the longer descriptions in the MIMIC-CXR dataset present visual-linguistic data bias and significant spurious correlations among multiple words. Our \red{CMCRL} effectively leverages cross-modal causal intervention to achieve performance improvement.

In contrast to BLEU, Rouge-L focuses more on the structural and sequential similarity of sentences. Our method demonstrates significant superiority on the IU-Xray dataset while exhibiting slightly lower performance compared to DCL~\cite{li2023dynamic} on the MIMIC-CXR dataset. However, Table~\ref{tab:main_result} reveals that all methods achieve Rouge-L scores around 0.280 on the MIMIC-CXR dataset. This observation suggests that, when generating long sentences, the differences in structure and sequence with the reference reports are similar across various methods. As a result, this metric may not be a sensitive evaluation criterion for this dataset. 

Similarly, the METEOR metric considers synonyms and phrase reordering, emphasizing diverse matches of phrases and vocabulary. As indicated in Table~\ref{tab:main_result}, \red{CMCRL} achieves the best performance in terms of the METEOR metric on the IU-Xray dataset, but it falls short compared to MMTN on the MIMIC-CXR dataset. This suggests that there are some limitations in matching phrase and vocabulary diversity in our approach. By referring to Table~\ref{tab:model_scale}, we observe that both the best-performing MMTN and AlignTransformer benefit from knowledge supplementation and the assistance of a memory module, potentially enhancing the model's ability to estimate synonyms more effectively. While our approach relies on learning semantics solely from the linear projection space of tokens, which may have limitations.

% Moreover, CIDEr metric emphasize the importance of more significant and information-rich words in the generated text, rather than common words, posing a challenge for long sequence reports containing medical terminology. on the MIMIC-CXR dataset, VLCI outperforms two open-source non-knowledge-based methods (R2Gen and R2GenCMN) but falls behind knowledge-based methods (PPKED, MGSK, DCL).  We speculate that domain-specific knowledge with specialized concepts can assist the model in generating more accurate disease descriptions and achieving superior performance, even if these descriptions may be less common.  Our VLCI focuses on causal intervention within cross-modal data without relying on external knowledge, making it unable to accurately capture domain-specific terminology.
% the performance of VLCI is lower than that of DCL on MIMIC-CXR. This metric assesses the similarity of the entire report, posing challenges for long-sequence reports with radiological terminology. Thus, the knowledge-based approach (PPKED, MGSK, DCL) with professional concepts can generate more precise disease descriptions and achieve superior performance, while VLCI focuses on causal intervention within cross-modal data without relying on external knowledge. 



\subsubsection{CE Metric}
The purpose of {RRG} is to alleviate the burden on radiological professionals and provide precise diagnostic evidence. Therefore,  the CE metric is more appropriate for evaluating the clinical significance of {RRG} methods compared to NLG metrics, as it specifically assesses the accuracy of abnormality classification. 
The CE metric is only applied to the MIMIC-CXR dataset because the label extractor (CheXpert)~\cite{irvin2019chexpert} is specially designed for MIMIC-CXR to obtain class labels. Compared with the lightweight R2Gen in Table~\ref{tab:main_result}, \red{CMCRL} demonstrates a remarkable improvement of 15.6\% in Precision, 6.7\% in Recall, and 12.5\% in F1-Score. This validates that \red{CMCRL} can provide a more accurate clinic diagnosis rather than merely generating fluent reports. Additionally, our \red{CMCRL} outperforms the state-of-the-art method DCL in terms of Precision score and F1 score. Notably, when compared to several knowledge-based methods, our approach achieves a more efficient clinical evidence generation by relying solely on causal intervention without requiring additional knowledge assistance. Nevertheless, the Recall score is relatively lower compared to M2TR in Table~\ref{tab:main_result}, indicating that the dataset may contain extreme categories. M2TR employs staged generation, which leverages the preliminary concepts generated in the first stage, enabling effective anomaly detection. However, this process can also lead to an increase in false positives, consequently reducing the Precision score.

\begin{figure}[t]
  \centering
  \includegraphics[width=1\linewidth]{figure/8.pdf}

  \caption{Evaluation of abnormality classification results (accuracy) on MIMIC-CXR. The baseline model is the transformer without causal intervention.}
  \label{fig:experiment_abnormality}
\end{figure}


\subsection{Qualitative Analysis}
% abnormalities detection

\subsubsection{Abnormalities Detection}
To further validate the assistance of causal intervention (i.e., alleviate the
burden on radiological professionals and provide precise diagnostic evidence) in our method, we extract 14 categories of labels from the reports generated by the baseline and \red{CMCRL}, and evaluated their accuracy, as shown in Fig.~\ref{fig:experiment_abnormality}. Our approach achieves significant performance improvements in all categories, particularly in ``Edema" ($0.509 \to 0.840$) and ``Enlarge Cardiomediastinum" ($0.473 \to 0.842$). This is because our \red{CMCRL} explores sufficient visual information and further produces more accurate and less biased descriptions by cross-modal causal intervention than the Transformer baseline. 
However, the estimation of some categories remains ambiguous, e.g., ``Lung Opacity". It reveals that \red{CMCRL} can provide a comprehensive consideration of various radiologic signs to detect the abnormality but give less improvement for the single source abnormality. For example, whether ``Edema" is caused by the heart has different radiologic signs, while the increase in lung density can be considered as ``Lung Opacity". Thus, \red{CMCRL} can capture the abnormality with complex causes more effectively, where exists more spurious correlations. Besides, Fig.~\ref{fig:experiment_abnormality} shows the unavailability of causal intervention in independent abnormalities, e.g. ``Support Devices".

\subsubsection{Causal Consistency}
{To assess whether \red{CMCRL} utilized appropriate visual evidence for its inference, we develop a questionnaire and distributed it to human experts for evaluation. We select 10 data categories, each containing 10 samples, resulting in a total of 100 instances. Visual attention maps are generated based on the descriptions in the reports, where attention is captured through the cumulative response of entire sentences. These visualizations are compiled into a questionnaire (as shown in Figure \ref{fig:stat_result} (a-c)) and distributed to 12 experienced radiological professionals. They evaluate the plausibility of the results based on the original images, attention maps, and corresponding sentences.}

{As shown in Fig \ref{fig:stat_result} (d), our approach achieves an overall reasonableness rating of 76.3\%, indicating that most of the samples are accepted by experts. Moreover, due to the low image resolution (224*224), the experts find it challenging to make judgments. Furthermore, the results for Cardiomegaly, Pneumothorax, and Support Devices exhibited significant consistency with human priors. This underscores the effectiveness of our approach in uncovering latent causal relations rather than relying on spurious correlations for judgment. Additionally, we intend to further advance the field in interpretable and trustworthy reasoning, with the goal of leveraging explicit causal relations for inference in the future.}

\begin{figure}[!t]
 \centering
    \includegraphics[width=1\linewidth]{figure/9.pdf}
    \caption{{
    A sample from the questionnaire. (a) is the generation report, (b) is the input image and (c) is the heatmap from cross-attention layer. (d) is the human expert evaluation results of \red{CMCRL}.
    }}
    \label{fig:stat_result}
\end{figure}


%%%%%%%%%%%%%%%%%%%%%%%%%%%%%%%%%%%%%%%%%%%%%%%%%%%%%%%%%%%%%%%%%%%%%%%%%%%%%%%%%%%%%%%%%
\subsubsection{Case Study}
We further conduct the qualitative analysis on the MIMIC-CXR dataset via three intuitive generated examples of the baseline and the \red{CMCRL} in Fig.~\ref{fig:experiment_result}.
% visual
In Fig.~\ref{fig:experiment_result} (a), the reference report contains four abnormalities. However, the baseline model neglects all abnormalities, while \red{CMCRL} accurately identifies all abnormalities. This indicates that our VDM comprehensively captures all essential visual features, crucial for {RRG}.
% linguistic
Fig.~\ref{fig:experiment_result} (b) shows an example where the same visual region is simultaneously discovered by the baseline and the \red{CMCRL} but leads to different descriptions. Our \red{CMCRL} can accurately describe the heart, while the baseline is uncertain and even has a miscalculation of pneumonia. 
It shows that LDM can alleviate the semantic bias caused by word frequency in word embedding space.
%
Fig.~\ref{fig:experiment_result} (c) illustrates a complex causal graph, where ``atelectasis" and ``edema" could also be causes of ``cardiomegaly." However, the baseline fails to correctly consider the causes of ``cardiomegaly" and erroneously captures these pieces of evidence. In contrast, \red{CMCRL} leverages the causal intervention module to disentangle these confounders, enabling careful consideration of various pieces of evidence for an accurate judgment.
% 对报告的可视化做了删改,所以看不出流畅程度
% Compared to the Baseline, our VLCI can generate a more fluent report and indicates the normality. 
%Both the Baseline and VLCI can generate a fluent report and indicates the normality. But VLCI fails to capture the peribronchial opacities, which are the radiologic signs between ``Clear Lung" and ``Consolidation". This is because the ``Lung Opacity" only changes the pulmonary density and it is difficult to be discovered determinedly.

\begin{figure*}[h]
 \centering
    \includegraphics[width=1\linewidth]{figure/10.pdf}
    \caption{{The results of the Non-causal Model (Baseline) and \red{CMCRL} models on the MIMIC-CXR dataset are presented in (a-c), and (d) shows the false sample in \red{CMCRL}. Thirteen kinds of abnormalities are marked with different markers and colors. Note that keywords in the reports are also marked with different markers and colors. Correctly identified abnormalities are marked in the corresponding color, while other descriptions in bold, italics, and underscores are incorrect. Descriptions marked only with underscores indicate repeated words. 
    }}
    \label{fig:experiment_result}
\end{figure*}


In Fig.~\ref{fig:experiment_result} (d), the ground truth indicates the presence of hydropneumothorax, a condition characterized by the simultaneous presence of gas and fluid in the chest, whereas pleural effusion contains only fluid. Although \red{CMCRL} correctly identifies the presence of gas and fluid in the chest, leading to the diagnosis of pneumothorax, it erroneously estimates pleural effusion due to insufficient relevant knowledge. In this example, our method produces inaccurate text and fails to identify pneumonia and lung consolidation. Moreover, while VDM and LDM excel at recognizing visual and language concepts, detecting highly specialized concepts with latent relations not explicitly present in the data presents challenges.

\begin{table}
  \centering
  \setlength{\tabcolsep}{2mm}{
    \caption{Ablation Result of DenseNet121 backbone.}

  \label{tab:ablation_densenet}
  \begin{tabular}{@{}lllll@{}}
    \toprule
    Method on IU-Xray          & BLEU-4  & Rouge-L & METEOR\\
    \midrule
    ResNet101         & {14.8} & {34.5} & {18.0}\\ 
    ResNet101 +  VLCI  & {\textbf{19.0} (+4.2)} & {\textbf{39.4} (+4.9)}& {\textbf{21.0} (+3.0)}\\ 
    DenseNet121       & {16.4} & {36.1} & {18.3}\\ 
    DenseNet121 +  VLCI& {\textbf{17.6} (+1.2)} & {\textbf{39.4} (+3.3)} & {\textbf{19.5} (+1.2)}\\ 
    \midrule
    Method on MIMIC-CXR       & BLEU-4  & CIDEr & METEOR\\
    \midrule
    ResNet101& {10.1} & {13.0} & {13.5}\\ 
    ResNet101 +  VLCI& {\textbf{11.9} (+1.8)} & {\textbf{19.0} (+6)}& {\textbf{15.0} (+1.5)}\\ 
    DenseNet121& {9.1} & {11.1} & {12.7}\\ 
    DenseNet121 +  VLCI & {\textbf{11.3} (+2.2)} & {\textbf{13.4} (+2.3)} & {\textbf{14.9} (+2.2)}\\ 
    \bottomrule
  \end{tabular}}
\end{table}

% VLP mask
\begin{table}[t]
  \centering
  \setlength{\tabcolsep}{3mm}{
  \caption{We evaluated the performance of various masking ratios for \red{masked image restoration} on the IU-Xray dataset. We pre-trained the \red{RadCARE} model for 100 epochs and then fine-tuned it in the baseline (non-causal model) for an additional 5 epochs.}

  \label{tab:experiment_vlp_mask}
  \begin{tabular}{@{}lcccc@{}}
    \toprule
    Masking Ratio           & BLEU-1  & BLEU-4  & CIDEr & Rouge-L \\
    \midrule
    Baseline            & {43.3} & {14.8} & {50.1} & {34.5} \\ 
    w/ 75\%             & {45.0} & {16.0} & {48.6} & {\textbf{36.0}} \\ 
    w/ 85\%             & {\textbf{45.2}} & {\textbf{16.1}} & {\textbf{52.2}} & {35.1} \\
    w/ 95\%             & {43.2} & {15.3} & {46.0} & {34.6} \\ 
    \bottomrule
  \end{tabular}}

\end{table}


\subsection{Ablation Studies}
\subsubsection{Effectiveness of VLCI}
In Table~\ref{tab:ablation_densenet}, we adopt the same setting as ResNet101 to conduct the ablation experiments using DenseNet121 as the backbone. We perform \red{RadCARE} and fine-tuning with causal intervention on IU-Xray and MIMIC-CXR datasets. Due to the small scale of the IU-Xray dataset and the simplicity of report content, the persuasiveness of the CIDEr metric on this dataset is limited. We evaluated BLEU-4, Rouge-L, and METEOR on the IU-Xray dataset, and the results indicate that our method still achieves a significant improvement with DenseNet121. However, we observed that the performance of models using DenseNet121 as the backbone is notably inferior to those using ResNet101. We speculate that while DenseNet121 is efficient in feature extraction due to its dense connections, it may not always provide the best performance for tasks involving specific types of radiological images or requiring deeper feature extraction capabilities. Additionally, we conducted experiments on the larger MIMIC-CXR dataset, using BLEU-4, CIDEr, and METEOR as evaluation metrics. The results further confirm our speculations and validate the effectiveness of our approach.

% VLP ablation
\begin{table}[!t]
  \centering
  \setlength{\tabcolsep}{1mm}{
    \caption{The performance of different pre-training methods on IU-Xray, the result marked by * means fine-tuning with 10 epochs, while the rest only use the encoder with 100 epochs. \red{``text" is denoted as postfix text generation and ``image" is masked image restoration}. The result marked by $\dagger$ is from \cite{Chen_2022_CVPR}.}
  \label{tab:ablation_vlp}
  \begin{tabular}{@{}lccc@{}}
    \toprule
    Method           & BLEU-4  & CIDEr & Rouge-L \\
    \midrule
    Baseline                  & {14.8} & {50.1} & {34.5} \\ 
    w/ MAE                    & {15.4} & {48.6} & {36.0} \\ 
    w/ VisualGPT$\dagger$     & {15.9} & {49.7} & {\textbf{37.4}} \\ 
    w/ MIM              & {16.2} & {\underline{60.2}} & {36.2} \\
    w/ \red{TEXT}              & {\underline{16.5}} & {53.8} & {\underline{36.5}} \\ 
    w/ \red{TEXT+IMAGE}          & {16.0} & {43.1} & {36.4} \\ 
    \midrule
    w/ \red{TEXT}*             & {15.1} & {39.9} & {34.9} \\ 
    w/ \red{TEXT+IMAGE}*         & {16.1} & {52.2} & {35.1} \\ 
    \textbf{w/ \red{RadCARE}}* (Ours)   & {\textbf{17.0}} & {\textbf{63.1}} & {36.3} \\ 
    \bottomrule
  \end{tabular}}
\end{table}

% main ablation
\begin{table}
  \centering
  \setlength{\tabcolsep}{0.7mm}{
    \caption{Ablation analysis of our \red{CMCRL}. The Baseline is implemented by the transformer. The marker at the Baseline (non-causal model) and R2Gen~\cite{chen2020generating} means the operation in the brackets.}
          \vspace{-10pt}
  \label{tab:ablation_baseline}
  \begin{tabular}{@{}llccc@{}}
    \toprule
    Dataset     & Method        & BLEU-4  & CIDEr & Rouge-L \\
    \midrule
                & Baseline                                             & {14.8} & {50.1} & {34.5} \\ 
                & Baseline$^{w\blacklozenge}$ (w/ VDM)                 & {16.0} & {52.1} & {36.4}\\
                & Baseline$^{w\bullet}$ (w/ LDM)                       & {15.5} & {50.9} & {36.1}\\
                & Baseline$^{w\blacklozenge \bullet} $ (w/ VDM\&LDM)   & {16.3} & {54.4} & {36.1} \\ 
                \cmidrule{2-5}
                & R2Gen                                                & {16.5} & {49.3} & {36.0} \\
                & R2Gen$^{w\blacklozenge}$ (w/ VDM)                    & {17.1} & {55.3} & {37.0} \\
    IU-Xray     & R2Gen$^{w\bullet}$ (w/ LDM)                          & {16.6} & {54.6} & {36.0} \\
                & R2Gen$^{w\blacklozenge\bullet}$ (w/ VDM\&LDM)        & {17.3} & {\underline{62.8}} & {36.8} \\
                \cmidrule{2-5}
                & Baseline$^{w\bigstar}$ (w/ \red{RadCARE})                       & {17.0} & {\textbf{63.1}} & {36.3} \\ 
                & Baseline$^{w\bigstar\blacklozenge}$ (w/ \red{RadCARE}\&VDM)     & {17.4}  &  {52.3} & {37.4}  \\
                & Baseline$^{w\bigstar \bullet}$ (w/ \red{RadCARE}\&LDM)          &  {\underline{17.8}} &  {57.3} & {\underline{37.8}}  \\
                & \red{CMCRL}          & {\textbf{19.0}} & {59.2} & {\textbf{39.4}} \\
    \midrule
                & Baseline                                             & {10.1} & {13.0} & {27.0} \\ 
                & Baseline$^{w\blacklozenge}$ (w/ VDM)                 & {10.3} & {14.4} & {27.2}\\
                & Baseline$^{w\bullet}$ (w/ LDM)                       & {6.9} & {7.1} & {22.4}\\
                & Baseline$^{w\blacklozenge \bullet} $ (w/ VDM\&LDM)   & {7.0} & {7.4} & {23.0} \\
                \cmidrule{2-5}
                & R2Gen                                                & {10.3} & {16.8} & {\underline{27.8}}   \\
                & R2Gen$^{w\blacklozenge}$ (w/ VDM)                    & {10.6} & {17.1} & {27.7}  \\
    MIMIC-CXR   & R2Gen$^{w\bullet}$ (w/ LDM)                          & {9.1} & {13.6}  & {25.6}  \\
                & R2Gen$^{w\blacklozenge\bullet}$ (w/ VDM\&LDM)        & {10.0} & {14.3} & {26.4}\\
                \cmidrule{2-5}
                & Baseline$^{w\bigstar}$ (w/ \red{RadCARE})                      & {10.6} & {15.1} & {\underline{27.8}} \\ 
                & Baseline$^{w\bigstar\blacklozenge}$ (w/ \red{RadCARE}\&VDM)    & {11.0} & {\underline{17.7}} & {\textbf{28.0}} \\
                & Baseline$^{w\bigstar \bullet}$ (w/ \red{RadCARE}\&LDM)         & {\underline{11.5}} & {15.7} & {27.7} \\
                & \red{CMCRL}                                                 & {\textbf{11.9}} & {\textbf{19.0}} & {\textbf{28.0}} \\
    \bottomrule
  \end{tabular}}
\end{table}


\subsubsection{Effectiveness of \red{RadCARE}}

In Table~\ref{tab:experiment_vlp_mask}, we conduct ablation experiments to assess the impact of masking ratios on model performance, and the results are presented. Our \red{RadCARE} model achieved the best performance with a higher masking ratio of 85\%, which is in contrast to the optimal masking ratio of 75\% reported by MAE~\cite{he2022masked}. We attribute this difference to the cross-modal information correlations, where the masked information can be reconstructed by visible features from both language and images. Furthermore, \red{RadCARE} tends to learn general features from the masked modality at higher masking ratios, while distinguishable features can be extracted by the complete information from another modality. To explore whether increasing the masking ratio further would further improve the performance, we experimented with a higher masking ratio of 95\%. However, the decreased results in Table~\ref{tab:experiment_vlp_mask} indicate that this approach leads to excessive information loss.



In Table~\ref{tab:ablation_vlp}, we make a comparison with different pre-training methods. It shows that the cross-modal pre-training method has a more robust representation ability than the \red{masked image restoration} with single-modality. However, the Rouge-L metric in VisualGPT surpasses ours, possibly due to its exclusive pre-training of the text decoder, enabling more concentrated learning of the intricate structure in radiological reports.
\red{Additionally, our cross-modal pre-training via postfix text generation task achieves comparable performance to the model that only fine-tunes the encoder, while ours fine-tunes the whole model with fewer epochs. }
Moreover, our \red{RadCARE} adopts the degradation images as input, which facilitates the extraction of visual details in the downstream task.

 
% Slight improvements from VDM and LDM
Furthermore, in Table~\ref{tab:ablation_baseline}, Baseline$^{w\blacklozenge \bullet}$ is significantly worse than baseline on MIMIC-CXR, e.g., 10.1 $\to$ 7.0 for BLEU-4, while still keeping performance improvement on IU-Xray dataset. % compared with the baseline. 
This validates the significant feature complexity from the large-scale MIMIC-CXR dataset leads to unstable probability distribution estimation with causal intervention. 
% 为了关联,得到先验分布. mimic的数据集非常大且复杂以至于很难直接在intervention中得到先验分布。而相对于iu-xray这种小数据集,由于data bias反而容易得到关联。
Meanwhile, the \red{RadCARE} can substantially boost the performance of the baseline, e.g., 14.8 $\to$ 17.0, 10.1 $\to$ 10.6 for BLEU-4 on IU-Xray and MIMIC-CXR datasets, respectively. The improvement is caused by the learned comprehensive concepts and context in the pre-training and the cross-modal features alignment stage, which shows the importance of \red{RadCARE}. Similarly, The Rough-L is also barely improved due to the features' complexity and long sequence from the MIMIC-CXR dataset. For example, although AlignTransformer achieves the same score of the Rough-L as CA on the MIMIC-CXR, it outperforms CA on all other metrics.


\subsubsection{Effectiveness of Causal Intervention}
% 与VLP相比,VLP + VDM 小数据集提升明显,但大的少;大数据集文本复杂;有视觉信息,但是不能准确描述
% 与VLP相比,VLP + LDM 大数据集提升明显,但小的少;大数据集文本复杂;更好的描述,但是缺少准确的视觉信息
% VLCI 小数据集的CIDEr明显不如 VDM + LDM;因为小数据集的文本相对单一且短,虽然报告整体显示但是用词准确度不如
\noindent \textbf{VDM}. %To disentangle visual features, we implement the visual deconfounding via the mediator that fuses local detail information and global contour position. 
In Table~\ref{tab:ablation_baseline}, Baseline$^{w\blacklozenge}$ and R2Gen$^{w\blacklozenge}$ can boost the performance compared to Baseline and R2Gen, which demonstrates the validity of the VDM. 
% CIDEr metric?
However, the improvement of BLEU-4 between Baseline$^{w\bigstar \blacklozenge}$ and Baseline$^{w\bigstar}$ on the IU-Xray dataset is more significant than that on the MIMIC-CXR dataset. This is because the VDM can discover essential visual information, but the report of the MIMIC-CXR is more complex and the model fails to generate accurate descriptions. The performance degradation of CIDEr can further illustrate it.

In Fig.~\ref{fig:experiment_local_vis}, the encoder of the non-causal model exhibits limited attention to all potential abnormal regions.  Instead, it excessively focuses on the base of the lung, possibly due to the dataset's high prevalence of lung-related diseases. In contrast, the attention map from the \red{CMCRL} encoder can truly focus on the dominant regions that may indicate abnormalities, including the entire lung lobe, carina, and pleura, rather than false correlations with biased visual concepts. This confirms the semantic sensitivity of the VDM, which captures dominant visual content by performing visual causal interventions.
%In Fig.~\ref{fig:experiment_local_vis}, the attention map from the encoder of our VLCI can truly focus on the dominated area of possible abnormalities rather than spurious correlations with biased visual concepts. This validates that the VDM is semantics-sensitive to capture dominant visual content by conducting visual causal intervention.

\begin{figure}[t]
 \centering
    %\fbox{\rule{0pt}{2in} \rule{0.9\linewidth}{0pt}}
    \includegraphics[width=1\linewidth]{figure/11.pdf}
    \caption{The visualization of the attention map. (a) is an example from the MIMIC-CXR dataset that the colored text should be discovered in the marked region of the image. The images in (b) are the attention maps of the non-causal model and our \red{CMCRL}, respectively. The tag ``Enc" means the accumulated attention maps from the encoder for the selected local features, and ``Dec" is the response to the ``pleural" (decoder output).}
    \label{fig:experiment_local_vis}
\end{figure}

% 句子描述更准确
\noindent \textbf{LDM}. %To generate accurate reports after perceiving all essential visual features, the LDM can mitigate spurious correlations caused by cross-modal bias and adjust the semantic embedding space. 
Compared to the VDM, the LDM plays a more significant role in {RRG} because the sophisticated linguistic semantic patterns within reports are entangled and biased that require elaborate linguistic deconfounding. 
In Table~\ref{tab:ablation_baseline}, the performance drops without LDM, e.g., 11.9 $\to$ 11.0 for the BLEU-4 metric on the MIMIC-CXR dataset. This shows the importance of adjusting semantic relevance in word embedding space. 
Compared with the baseline, the performance improvement of Baseline$^{w\bigstar \bullet}$ on the MIMIC-CXR dataset demonstrates that the LDM can generate more accurate reports even with biased visual information. 
% CIDEr 的问题
However, the CIDEr metric on the IU-Xray dataset shows the effectiveness of the combination of VDM and LDM, while ILVD obtains a lower score. This is due to the worse diversity on the IU-Xray dataset, where Baseline$^{w\blacklozenge \bullet}$ and R2Gen$^{w\blacklozenge \bullet}$ can get higher CIDEr but lower BLEU-4 with inadequate multi-modal feature correlation.
In Fig.~\ref{fig:experiment_local_vis} (b), the attention map of the decoder in the non-causal model exhibits evident redundant responses, with high attention widely distributed in the upper part of the lung, especially the carina. 
The \red{CMCRL}, in contrast, can capture dominant semantic information in a coarse-to-fine manner, refining it from the potential abnormal regions that receive extensive attention in the encoder to the bilateral thorax. The high attention on the carina may be attributed to the presence of a support device that could increase cardiac load, cause vascular occlusion or congestion, leading to changes in intrathoracic pressure and eventually resulting in pleural effusion.
These findings indicate that the LDM can capture more discriminative semantic information from the linguistic modality through linguistic front-door interventions.

%In Fig.~\ref{fig:experiment_local_vis} (b), the attention map of the baseline decoder shows an obvious redundancy response, while VLCI can capture dominated semantic information in a coarse-to-fine manner, which is more related to the abnormalities. These results show that LDM can capture more discriminative semantic information from linguistic modality by linguistic front-door intervention. 



%-------------------------------------------------------------------------
\section{Conclusion}

\red{In this paper, we \red{propose the Cross-Modal Causal Representation Learning (CMCRL)} framework for {RRG}, to deconfound visual-linguistic confounders by causal intervention. To alleviate the problem of unpaired visual-linguistic data when pre-training, we \red{propose RadCARE} for cross-modal pre-training, \red{integrating postfix text generation and degradation-aware masked image restoration task}. To mitigate cross-modal confounders and discover the true cross-modal causality, we propose visual-linguistic causal front-door intervention modules VDM and LDM\red{, which are integrated into the visual-Linguistic Causal Intervention (VLCI) model}. Experiments on IU-Xray and MIMIC-CXR datasets show that our \red{CMCRL} can effectively mitigate visual-linguistic bias and outperforms the state-of-the-art methods. The lower computational cost and faster inference speed of \red{CMCRL} promote its clinical application.  \red{In future work, we aim to further refine our method to better align with advanced Large Language Models (LLMs) and apply it to a broader range of medical imaging modalities, including CT, MRI, and digital pathology slides, among others. Additionally, to enhance the interpretability of causal reasoning, we will integrate the inferential capabilities of LLMs with their internal knowledge for causal inference. This may involve leveraging existing multimodal data to construct retrieval-augmented generation (RAG) tools or employing multi-agent systems to mine patient causal information from multidimensional medical records. We believe our work will inspire more causal reasoning methods within the realm of Radiology Report Generation (RRG) and extend its impact to other areas of medical imaging.}}


{\small
\bibliography{ref}
\bibliographystyle{IEEEtran}
}

\begin{IEEEbiography}[{\includegraphics[width=1in,height=1.25in,clip,keepaspectratio]{bio/WeixingChen}}]{Weixing Chen} has received the B.S. degree from the college of Medicine and Biological Information Engineering, Northeastern University, in 2020 and M.S. degree from Shenzhen Institute of Advanced Technology, Chinese Academy of Sciences in 2023. He is currently a Ph.D. student at the School of Computer Science and Engineering, Sun Yat-sen University. His main interests include medical image analysis, multi-modal learning, and causal relation discovery. He has been serving as a reviewer for numerous academic journals and conferences such as TNNLS, NeurIPS, ICML, ICLR, MICCAI, and ACM MM.
\end{IEEEbiography}

\begin{IEEEbiography}[{\includegraphics[width=1in,height=1.25in,clip,keepaspectratio]{bio/YangLiu}}]{Yang Liu}(M'21) is currently an associate professor working at the School of Computer Science and Engineering, Sun Yat-sen University. He received his Ph.D. degree from Xidian University in 2019. His current research interests include multi-modal reasoning, causality learning and embodied AI. He is the recipient of the First Prize of the Third Guangdong Province Young Computer Science Academic Show. He has authorized and co-authorized more than 40 papers in top-tier academic journals and conferences such as TPAMI, TIP, TKDE, T-MECH, CVPR, ICCV, IJCAI, and ACM MM.
\end{IEEEbiography}

\begin{IEEEbiography}[{\includegraphics[width=1in,height=1.25in,clip,keepaspectratio]{bio/CeWang}}]{Ce Wang} has received the B.S. degree from the Department of Mathematics, Jilin University, in 2015 and Ph.D. degree from the Center for Combinatorics, Nankai University in 2020. From 2020 to 2023, he worked as a postdoctoral researcher in the Institute of Computing Technology, Chinese Academy of Sciences. From 2023 to 2024, He worked as a postdoctoral researcher in the department of ECE, HKUST. He is currently an Associate Professor with the Sun Yat-sen University, Shenzhen, Guangdong, China. His main interests include image and video processing, computational imaging, medical image analysis, and explainable healthcare AI. He has been serving as a reviewer for numerous academic journals and conferences such as TMI, MIA, TCSVT, NIPS, ICLR, MICCAI and ACM MM.
\end{IEEEbiography}

\begin{IEEEbiography}[{\includegraphics[width=1in,height=1.25in,clip,keepaspectratio]{bio/JiaruiZhu}}]{Jiarui Zhu} is currently a Ph.D student in Department of Health Technology and Informatics, the Hong Kong Polytechnic University. He received his B.S. degree from the college of Medicine and Biological Information Engineering, Northeastern University in 2020 and Msc degree in Medical Physics from Hong Kong Polytechnic University in 2023. His current research interests include 3D gaussian representations and few-shot learning in medical imaging processing.
\end{IEEEbiography}

\begin{IEEEbiography}[{\includegraphics[width=1in,height=1.25in,clip,keepaspectratio]{bio/GuanbinLi}}]{Guanbin Li}(M'15) is currently a professor in School of Computer Science and Engineering, Sun Yat-Sen University. He received his PhD degree from the University of Hong Kong in 2016. His current research interests include computer vision, image processing, and deep learning. He is a recipient of ICCV 2019 Best Paper Nomination Award. He has authorized and co-authorized on more than 100 papers in top-tier academic journals and conferences. He serves as an area chair for the conference of VISAPP. He has been serving as a reviewer for numerous academic journals and conferences such as TPAMI, IJCV, TIP, TMM, TCyb, CVPR, ICCV, ECCV and NeurIPS.
\end{IEEEbiography}

\begin{IEEEbiography}[{\includegraphics[width=1in,height=1.25in,clip,keepaspectratio]{bio/Cheng-LinLiu.pdf}}]{Cheng-Lin Liu}(Fellow, IEEE) received the BS, ME, and PhD degrees from Wuhan University, Beijing University of Technology, and the Institute of Automation of Chinese Academy of Sciences, in 1989,
1992 and 1995, respectively. He was a postdoctoral fellow with the Korea Advanced Institute of Science and Technology (KAIST) and later with the Tokyo University of Agriculture and Technology from 1996 to 1999. From 1999 to 2004, he was a research staff member and later a senior researcher with the Central Research Laboratory, Hitachi, Ltd., Tokyo, Japan.
Since 2005, he has been a professor with the National Laboratory of Pattern Recognition (NLPR), Institute of Automation of Chinese Academy of Sciences, Beijing, China. His research interests include pattern recognition, machine learning, document analysis and recognition. He has published more than 400
technical papers in prestigious international journals and conferences. He is an associate editor-in-chief of Pattern Recognition Journal and Acta Automatica
Sinica and is on the editorial board of several international and domestic journals. He is a fellow of the IAPR, the CAA, and CAAI.
\end{IEEEbiography}

\begin{IEEEbiography}[{\includegraphics[width=1in,height=1.25in,clip,keepaspectratio]{bio/LiangLin}}]{Liang Lin}(Fellow, IEEE) is currently a Full Professor with Sun Yat-sen University, Guangzhou, China. From 2008 to 2010, he was a Postdoctoral Fellow
with the University of California, Los Angeles, Los
Angeles, CA, USA. From 2016 to 2018, he led the SenseTime R\&D teams to develop cutting-edge and deliverable solutions for computer vision, data analysis and mining, and intelligent robotic systems. He
has authored or coauthored more than 100 papers
in top-tier academic journals and conferences, such
as 15 papers in IEEE TRANSACTIONS ON PATTERN
ANALYSIS AND MACHINE I NTELLIGENCE and International Journal of Computer Vision, and more than 60 papers in CVPR, ICCV, NeurIPS, and IJCAI.
He was an Associate Editor for IEEE TRANSACTIONS ON MULTIMEDIA , IEEE TRANSACTIONS ON NEURAL NETWORKS AND LEARNING SYSTEMS, and was an Area/Session Chair for numerous conferences, such as CVPR, ICCV, AAAI,
ICME, and ICMR. He was the recipient of the Annual Best Paper Award by Pattern Recognition (Elsevier) in 2018, Best Paper Diamond Award at IEEE ICME 2017, Best Paper Runner-Up Award at ACM NPAR 2010, Google Faculty Award in 2012, Best Student Paper Award at IEEE ICME 2014, and Hong
Kong Scholars Award in 2014. He is a Fellow of IAPR, AAIA, and IET.
\end{IEEEbiography}

%\begin{IEEEbiography}[{\includegraphics[width=1in,height=1.25in,clip,keepaspectratio]{}}]{IEEE Publications Technology Team}
%In this paragraph you can place your educational, professional background and research and other interests.\end{IEEEbiography}


\end{document}


