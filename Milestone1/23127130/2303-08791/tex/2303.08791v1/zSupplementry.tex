\documentclass[%preprint,
superscriptaddress,
%groupedaddress,
%unsortedaddress,
%runinaddress,
%frontmatterverbose, 
%preprint,
%preprintnumbers,
%nofootinbib,
%nobibnotes,
%bibnotes,
 amsmath,amssymb,
 %aps,
%pra,
%prb,
%rmp,
%prstab,
%prstper,
%floatfix,
]{revtex4-2}

%\usepackage[italian,english]{babel}
\usepackage[T1]{fontenc}
\usepackage[utf8]{inputenc}
\usepackage[margin=1in]{geometry} 
\usepackage{amsmath,amsthm,amssymb}
\usepackage{siunitx}
%\usepackage{subfigure}
\usepackage{graphicx}
\usepackage{textcomp}
\usepackage{xcolor}
\usepackage{gensymb}
\usepackage{booktabs}
\usepackage{soul}
\usepackage{hyperref}
\usepackage{bm}
\usepackage{enumitem}
\usepackage{chemformula}
\usepackage{diagbox} 
\usepackage{appendix}
\usepackage{fixltx2e}
\usepackage{listings}
\usepackage{color, colortbl}
\definecolor{Gray}{gray}{0.9}
\usepackage{subcaption}
\usepackage{rotating}
\usepackage{fancyhdr}
\usepackage{emptypage} 
\usepackage{colorprofiles}
\usepackage{multirow}
\usepackage{xr}

\pagestyle{fancy}  
\fancyfoot[C]{\thepage}
\fancyhead{}
\renewcommand{\headrulewidth}{0pt}


\setcounter{figure}{0}
\renewcommand{\thefigure}{S\arabic{figure}}

\setcounter{section}{0}
\renewcommand{\thesection}{S\arabic{section}}


\makeatletter
\newcommand*{\addFileDependency}[1]{% argument=file name and extension
  \typeout{(#1)}
  \@addtofilelist{#1}
  \IfFileExists{#1}{}{\typeout{No file #1.}}
}
\makeatother

\newcommand*{\myexternaldocument}[1]{%
    \externaldocument{#1}%
    \addFileDependency{#1.tex}%
    \addFileDependency{#1.aux}%
}

\myexternaldocument{main}

\begin{document}

\title{Supplementary information for\\ Halide perovskite artificial solids as a new platform to simulate collective phenomena in doped Mott insulators}% Force line breaks with \\

\author{Alessandra Milloch}
\email[]{alessandra.milloch@unicatt.it}
\affiliation{Department of Mathematics and Physics, Università Cattolica del Sacro Cuore, Brescia I-25133, Italy}
\affiliation{ILAMP (Interdisciplinary Laboratories for Advanced
Materials Physics), Università Cattolica del Sacro Cuore, Brescia I-25133, Italy}
\affiliation{Department of Physics and Astronomy, KU Leuven, B-3001 Leuven, Belgium}

\author{Umberto Filippi}
\affiliation{Italian Institute of Technology (IIT), Genova 16163, Italy}

\author{Paolo Franceschini}
\affiliation{CNR-INO (National Institute of Optics), via Branze 45, 25123 Brescia, Italy}

\author{Michele Galvani}
\affiliation{Department of Mathematics and Physics, Università Cattolica del Sacro Cuore, Brescia I-25133, Italy}

\author{Selene Mor}
\affiliation{Department of Mathematics and Physics, Università Cattolica del Sacro Cuore, Brescia I-25133, Italy}
\affiliation{ILAMP (Interdisciplinary Laboratories for Advanced
Materials Physics), Università Cattolica del Sacro Cuore, Brescia I-25133, Italy}

\author{Stefania Pagliara}
\affiliation{Department of Mathematics and Physics, Università Cattolica del Sacro Cuore, Brescia I-25133, Italy}
\affiliation{ILAMP (Interdisciplinary Laboratories for Advanced
Materials Physics), Università Cattolica del Sacro Cuore, Brescia I-25133, Italy}

\author{Gabriele Ferrini}
\affiliation{Department of Mathematics and Physics, Università Cattolica del Sacro Cuore, Brescia I-25133, Italy}
\affiliation{ILAMP (Interdisciplinary Laboratories for Advanced
Materials Physics), Università Cattolica del Sacro Cuore, Brescia I-25133, Italy}

\author{Francesco Banfi}
\affiliation{FemtoNanoOptics group, Université de Lyon, CNRS, Université Claude Bernard Lyon 1, Institut Lumière Matière, F-69622 Villeurbanne, France}

\author{Massimo Capone}
\affiliation{International School for Advanced Studies (SISSA), via Bonomea 265, Trieste}

\author{Dmitry Baranov}
\affiliation{Italian Institute of Technology (IIT), Genova 16163, Italy}
\affiliation{Division of Chemical Physics, Department of Chemistry, Lund University, P.O. Box 124, SE-221 00 Lund, Sweden}

\author{Liberato Manna}
\affiliation{Italian Institute of Technology (IIT), Genova 16163, Italy}

\author{Claudio Giannetti}
\email[]{claudio.giannetti@unicatt.it}
\affiliation{Department of Mathematics and Physics, Università Cattolica del Sacro Cuore, Brescia I-25133, Italy}
\affiliation{ILAMP (Interdisciplinary Laboratories for Advanced
Materials Physics), Università Cattolica del Sacro Cuore, Brescia I-25133, Italy}
\affiliation{CNR-INO (National Institute of Optics), via Branze 45, 25123 Brescia, Italy}

\maketitle


\section{Sample preparation}
\subsection{Synthesis of cesium oleate stock solution}
The cesium oleate stock solution was prepared by loading 0.4 g of cesium carbonate (\ch{Cs_2CO_3} , 99$\%$), 1.75 ml of oleic acid (OA, 90$\%$) and 15 ml octadecene-1 (ODE, 90$\%$)  into a 40 ml vial and placing it into an aluminum block preheated at 120 °C on top of a hotplate. The mixture was stirred under N$_2$ flow and its temperature was kept between $\sim$110-115 °C until all visible solid disappeared ($\sim$60 minutes). The mixture was eventually cooled down on a room temperature hotplate under stirring.
\subsection{Synthesis of \ch{CsPbBr_3} nanocubes}
\ch{CsPbBr_3} nanocubes (NCs) were synthesized following the method reported in Ref. \cite{baranov2019investigation} and Ref. \cite{toso2019wide} with minor variations.
In a 20 ml vial, 72 mg of lead (II) bromide (\ch{PbBr_2},
$\geqslant$ 98$\%$) were combined with 5 ml of ODE, 0.5 ml of oleylamine (OLAm, 70$\%$) and 0.05 ml of OA. The vial was equipped with a magnetic stirrer and placed into an aluminum block preheated at 185 °C on top of a hotplate. As the temperature of the mixture reached 170 °C, the vial was lifted from the block and fixed with a clamp above the hotplate and as soon as it cooled down to 160 °C, 0.5 ml of the cesium oleate stock solution were swiftly injected. The reaction was quenched after $\sim$ 7 seconds with a cold water bath ($\sim$ 20 °C) under stirring. The crude solution was centrifuged for 3 minutes at 4000 rpm and the supernatant was discarded. Then the solution was centrifuged again at 4000 rpm for 3 minutes to discard any remaining liquid, with the vials in the centrifuge oriented such that the precipitate was pointing outwards. The remaining liquid was removed with a 100 $\micro$l mechanical micropipette, and the step was repeated again, this time using a cotton swab to collect the liquid. The remaining solid was dissolved in 1 ml of toluene (99.7$\%$) and filtered through 0.45 $\micro$m hydrophobic PTFE syringe filter to eliminate any residual aggregates.\\
The solution for photoluminescence (PL) and absorbance measurements was prepared by diluting 6 $\micro$l of the NC solution in a 2994 $\micro$l of toluene in a quartz cuvette.
NC batches were considered suitable for self-assembly experiments when the full width at half maximum (FWHM) of the PL spectrum was < 90 meV. Absorbance was measured through 10 mm path length, and the NC concentration was calculated from Beer's law by using absorbance at 335 nm and a published size-dependent extinction coefficient \cite{de2016highly}. Typical concentration varied in the range of $\sim$ 4-4.5 $\micro$M.
\subsection{Preparation of \ch{CsPbBr_3} nanocubes superlattice and disordered film}
Self-assembly of \ch{CsPbBr_3} NCs was carried out using a slow solvent evaporation method on 1 cm $\times$ 1 cm monocrystalline silicon (Si) and glass substrates (figure \ref{SI - fig: sl_film}). The substrates were cleaned by rinsing with methanol and 2-propanol and dried by gentle tapping with an absorbing paper tissue and by blowing compressed air. \\
Three substrates were placed inside a Petri dish and 30 $\micro$l of the stock solution of \ch{CsPbBr_3} NCs in toluene were drop-casted on each of them. The solvent was allowed to slowly evaporate ($\sim$12 hours), after which films were considered ready to be used for experiments.\\
The disordered films of \ch{CsPbBr_3} NCs were instead prepared by collecting the precipitate of the crude solution after the three centrifugation steps and by mechanically scrambling and pressing it with a plastic scoop and eventually spreading it on 0.5 cm $\times$ 1 cm Si or glass substrates (figure \ref{SI - fig: dis_film}).\\
Samples thickness was measured from SEM images of middle film profiles obtained by breaking in half replicas of samples used for experiments. Estimated sample thickness for the SL films was 7 $\micro$m $\leqslant thickness_{SL} \leqslant$ 15 $\micro$m, and for the disordered films was 10 $\micro$m $\leqslant thickness_{dis} \leqslant$ 20 $\micro$m.\\
\begin{figure}[h!]
     \centering
    \begin{subfigure}[t]{0.45\textwidth}
    \includegraphics[width=1\textwidth]{SI_sl_film5.jpg}
    \caption[Specimen fixed on the paper graph mask.]{}
\label{SI - fig: sl_film}
    \end{subfigure}$\,\,\,\,\,\,\,\,\,\,\,\,\,\,\,\,\,\,\,\,\,\,\,\,$
    \begin{subfigure}[t]{0.45\textwidth}
    \includegraphics[width=1\textwidth]{SI_dis_film4.jpg}
    \caption[Specimen fixed on the paper graph mask.]{}
    \label{SI - fig: dis_film}
    \end{subfigure}
    \caption{(a) Schematic representation of SLs formation during slow solvent evaporation and of (b) randomly oriented NCs on a silicon substrate.}
    \label{SI - fig: film}
\end{figure}
%\FloatBarrier

\section{Characterization of SL and disordered films}
\subsection{Photoluminescence and absorbance}
The samples were optically characterized by absorption and photoluminescence (PL) spectra, acquired with a Cary 300 spectrophotometer and with a Cary Eclipse spectrofluorometer,  respectively. The equilibrium PL and absorption spectra of NCs dispersed in toluene and of NC deposited on a glass substrate are displayed in figure \ref{SI - fig: PL dis} and \ref{SI - fig: PL SL} for disordered NC and superlattices respectively. In both cases, we observe a PL red-shift ($\Delta$PL$_{NC} =$ 2.414 eV - 2.382 eV = 32 meV for the disordered NCs sample and $\Delta$PL$_{SL}$ = 2.414 eV - 2.386 eV = 28 meV for the SLs sample) and a PL peak broadening ($\Delta \text{FWHM}_{NC} = 142 ~\text{meV} - 86 ~\text{meV} = 55 ~\text{meV}$ for disordered NCs and $\Delta\text{FWHM}_{SL}$ = 116 meV - 86 meV = 30 meV for SLs) for the deposited films compared to NCs in toluene dispersion, consistently with previous observations \cite{baranov2019investigation,kovalenko2017lead,nagaoka2017nanocube,van2018cuboidal,tong2018spontaneous}.\\
\begin{figure}[h!]
     \centering
    \begin{subfigure}[t]{0.45\textwidth}
    \includegraphics[width=1\textwidth]{SI_PL_dis.pdf}
    \caption[Specimen fixed on the paper graph mask.]{}
\label{SI - fig: PL dis}
    \end{subfigure}
    \begin{subfigure}[t]{0.45\textwidth}
    \includegraphics[width=1\textwidth]{SI_PL_SL.pdf}
    \caption[Specimen fixed on the paper graph mask.]{}
    \label{SI - fig: PL SL}
\end{subfigure}
    \caption{(a) Room temperature PL and absorption spectra of NCs dispersed in toluene (black) and of disordered NCs on a glass substrate  (red). (b) Room temperature PL and absorption spectra of NCs dispersed in toluene and of NC SLs deposited on a glass substrate.}
\label{SI - fig: PL}    
\end{figure}
%\FloatBarrier

\subsection{X-ray diffraction}
X-ray diffraction (XRD) patterns collected from disordered NCs film and superlattice sample are shown in figure \ref{SI - fig: xld}. The XRD data were measured with a PANalytical Empyrean diffractometer in a parallel beam configuration, equipped with a Cu K$\alpha$ ($\lambda$ = 1.5406 \r{A}) ceramic X-ray tube operating at 45 kV and 40 mA, 1 mm wide incident and receiving slits, and a 40 mA PIXcel3D 2×2 two-dimensional detector.\\
For disordered NCs, the XRD pattern shows the Bragg reflections ascribed to the orthorhombic structure of \ch{CsPbBr_3} \cite{lopez2020crystal} and all the peaks expected from a film of randomly oriented NCs \cite{toso2019wide}. Conversely, for the SL sample, we observe the presence of two strong peaks at $2\theta$ $\sim$ 15$^\circ$ and $2\theta$ $\sim$ 30.5$^\circ$. As described in detail in previous works \cite{toso2019wide,toso2021multilayer}, this XRD pattern originates from a close-packing of the cubes in the plane parallel to the substrate, which leads to the enhanced Bragg reflections from (110), (1$\Bar{1}$0) and (002) ($2\theta$ $\sim$ 15$^\circ$) and (220), (2$\Bar{2}$0) and (004) ($2\theta$ $\sim$ 30.5$^\circ$) planes of the orthorombic unit cell of \ch{CsPbBr_3}. The precise periodicity between NCs inside SLs produces a phase modulation on the diffracted X-rays, causing additional interference which produces the fine structure for the $2\theta$ $\sim$ 15$^\circ$ peak, observable in figure \ref{SI - fig: sl_xrd} (black solid line).
\begin{figure}[h!]
    \centering
     \centering
    \begin{subfigure}[t]{\textwidth}
    \includegraphics[width=1\textwidth]{aged_XRD2.png} %SI_xrd_dis.png
    \caption[Specimen fixed on the paper graph mask.]{}
    \label{SI - fig: dis_xrd}
    \end{subfigure}
    \begin{subfigure}[t]{\textwidth}
    \includegraphics[width=1\textwidth]{SI_xrd_sl.png}
    \caption[Specimen fixed on the paper graph mask.]{}
    \label{SI - fig: sl_xrd}
    \end{subfigure}
    \caption{XRD patterns of disordered NCs film (a) and superlattice film (b).}
    %Both in the fresh and aged films it's possible to observe rectangula SLs, suggesting that aging is not affecting the microscopic structural order of SLs.}
    \label{SI - fig: xld}
\end{figure}
%\FloatBarrier

\subsection{Optical microscopy}
Optical microscope images of the NCs films, acquired on a ZETA-20 true color 3D optical profiler, are displayed in figure \ref{SI - fig: optical microscopy}. For the disordered NCs film (fig. \ref{SI - fig: optical microscopy dis}), the image shows a roughly homogeneous spatial distribution of \ch{CsPbBr_3} NCs; no well-defined micron-sized rectangular structure, that typically corresponds to SLs, is observable in the
sample, suggesting that mechanical scrambling produces largely disordered NC film. In the SL sample (fig. \ref{SI - fig: optical microscopy SL}), the NCs are assembled into aggregates of rectangular shape and size between 1 and 10 \textmu m, each corresponding to a NC superlattice. 

\begin{figure}[h!]
    \centering
    \begin{subfigure}[t]{0.45\textwidth}
    \includegraphics[width=1\textwidth]{glass-100X-disordered.png}%SI_dis_optical.jpg
    \caption[Specimen fixed on the paper graph mask.]{}
    \label{SI - fig: optical microscopy dis}
    \end{subfigure}
    \begin{subfigure}[t]{0.45\textwidth}
    \includegraphics[width=1\textwidth]{OLAm_toluene_pre_3d.png} %SI_sl_fresh_optical222.jpg
    \caption[Specimen fixed on the paper graph mask.]{}
    \label{SI - fig: optical microscopy SL}
    \end{subfigure}
    \caption{Optical microscopy images of disordered NCs film (a) and superlattice film (b).}
    %Both in the fresh and aged films it's possible to observe rectangula SLs, suggesting that aging is not affecting the microscopic structural order of SLs.}
    \label{SI - fig: optical microscopy}
\end{figure}
%\FloatBarrier

\section{Sample aging}
Pump-probe experiments were performed on samples with various ages, from fresh ($\sim$ day 4) to $\sim$ 90 days old ones. The data presented in the main text were acquired on 30-35 days old disordered NCs sample and 85-92 days old SL sample. When not used for experiments the films were stored in a vacuum desiccator cabinet at $\sim$1 mbar. Aging effects are investigated by means of XRD and optical microscopy. 


\subsection{Disordered NCs film}
Figure \ref{SI - fig: dis_xrd} shows the XRD pattern of the fresh (day 1, black solid line) and aged (day 17, red solid line) disordered film. The latter measurement was done after the film had been mounted in the cryostat employed for pump-probe experiments and had been subjected to ambient pressure-to-vacuum cycles and to room temperature-to-17 K cycles. As observable, both XRD patterns show,  with different ratios in the peak intensities, the Bragg reflections ascribed to the orthorhombic structure of \ch{CsPbBr_3}, as expected for a film of disordered NCs. Nevertheless, for the day 1 measurement, the peak at $2\theta$ $\sim$ 15$^\circ$ shows a splitting, which suggests that when the film was deposited, NCs assembled and arranged in a periodic way in some form of super-structures. The size of these superstructures must be $\leqslant$ 1 $\micro$m, since they are not observable with optical microscopy (figure \ref{SI - fig: optical microscopy dis}) and neither with SEM, where just a rough surface is noticeable. After 17 days and after being subjected to several vacuum and temperature cycles, the splitting at the $2\theta$ $\sim$ 15$^\circ$ peak disappears, suggesting that the periodic arrangement is lost. Since all the pump and probe measurements on disordered film were conducted after this last XRD measurement, we can assume that the disordered film we measured is made of randomly oriented and randomly packed NCs.




\subsection{SL film}
Figure \ref{SI - fig: sl_xrd} shows XRD patterns of fresh (day 1, black solid line) and aged (day 80, red solid film) films of densely packed \ch{CsPbBr_3} NC SLs. The XRD pattern of the fresh SL film was collected immediately after the evaporation of the solvent ($\approx$ 12 hours after dropcasting solution on Si substrate), while the XRD pattern of the aged SL film was collected after pump and probe experiments, i.e. after the sample had undergone several cycles of vacuum and of cryogenic temperatures. \\
We observe that, also in the aged sample, the main diffraction contribution comes from two peaks ($2\theta$ $\sim$ 15$^\circ$ and $2\theta$ $\sim$ 30$^\circ$), but the fine structure for the $2\theta$ $\sim$ 15$^\circ$ peak is lost upon aging. This superlattice interference is closely related to the statistical fluctuation of the NC spacing, indicated by $\sigma_L$. It represents the stacking disorder of NCs and its effect on the diffraction profile is to smear fringes, making them disappear when its value becomes too high, leaving just one peak profile dependent on atomic plane periodicity and NC thickness. \\
Table \ref{SI - tab: fit} and figure \ref{SI - fig: fit_XRD} summarize the results obtained by fitting the XRD pattern of the fresh and aged films, by means of the multilayer diffraction method described in Ref. \cite{toso2021multilayer}. After 90 days, an increase of $\sigma_L$ from 1.321 \r{A} to 2.322 \r{A} (relative increase of $\sim$ 75$\%$) is observed. This is consistent with the XRD pattern of the aged film shown in figure \ref{SI - fig: sl_xrd}, which displays no fringes at the $2\theta$ $\sim$ 15$^\circ$ peak. 
A decrease of the interparticle spacing L is also observed after 90 days, from 34.2 \r{A} to 29.4 \r{A}. This is attributed to the removal of entrapped solvent (toluene) molecules during the application of vacuum: if we consider the volume of toluene molecule, which is equal to $\sim$177 \r{A}$^3$, and we approximate the volume to the one of a cube, we get an edge equal to $\sim$5.6 \r{A}. That value is comparable to the contraction of L from fresh to aged sample (4.8 \r{A}) obtained from the fit.
It is noteworthy that the average number of atomic planes per NC, N, does not change according to the fit. That implies NCs do not undergo a significant coarsening. \\
Control experiments (not shown) were performed to verify that after applying vacuum the only change in the results of XRD pattern fitting is the decrease of interparticle spacing L due to solvent removal and, subsequently, to verify that upon cooling the sample (to 84 K in this case) no change in L and $\sigma_L$ was observed once the sample was returned to room temperature.\\
These considerations allow us to claim that aging of these samples does not lead to aggregation or side reactions that could form other products \cite{baranov2020aging}. The only aging effect appears to be the increase in disorder manifested as an increase in the inter-NC spacing fluctuation (as depicted in figure \ref{SI - fig: SL_spacing}), due to NC and ligand rearrangement. 
Nevertheless, despite the increased disorder, the aged film is still made of close-packed monodispersed NCs lying in planes parallel to the substrate. The images of the aged film obtained from optical microscopy and scanning electron microscopy (SEM, performed on a JEOL JSM-6490LA microscope) are shown in figure \ref{SI - fig:fresh-aged}b,d and prove that the morphology of aged samples remains similar to fresh ones (fig. \ref{SI - fig:fresh-aged}a,c), with no noticeable impurities on the surface of SLs.\\


\begin{figure}[h!]
    \centering
    \begin{subfigure}[t]{0.48\textwidth}
    \includegraphics[width=1\textwidth]{SI_sl_fresh_optical222.jpg}
    \caption[Specimen fixed on the paper graph mask.]{}
    \label{tensile_test_spec}
    \end{subfigure}
    \begin{subfigure}[t]{0.48\textwidth}
    \includegraphics[width=1\textwidth]{SI_sl_aged_optical22.jpg}
    \caption[Specimen fixed on the paper graph mask.]{}
    \label{tensile_test_spec}
    \end{subfigure}
    \begin{subfigure}[t]{0.48\textwidth}
    \includegraphics[width=1\textwidth]{SI_sl_fresh_sem2.png}
    \caption[Specimen fixed on the paper graph mask.]{}
\label{tensile_test_spec}
    \end{subfigure}
    \begin{subfigure}[t]{0.48\textwidth}
    \includegraphics[width=1\textwidth]{SI_sl_aged_sem2.jpg}
    \caption[Specimen fixed on the paper graph mask.]{}
    \label{tensile_test_spec}
    \end{subfigure}
    \caption{ Optical microscope and SEM images of fresh (a, c) and aged (b, d) SL films respectively.}
    %Both in the fresh and aged films it's possible to observe rectangula SLs, suggesting that aging is not affecting the microscopic structural order of SLs.}
    \label{SI - fig:fresh-aged}
\end{figure}
%\FloatBarrier


\begin{center}
\begin{table}[h]
\caption{\textbf{Fitting parameter of multilayer diffraction model.} Results of XRD pattern fitting  of fresh and aged films shown in figure \ref{SI - fig: fit_XRD}.}
    \centering
\begin{tabular}{c | c | c | c}
\toprule[1.25pt]
    \bf{Parameter} & \bf{Definition} & \bf{Day 1}   & \bf{Day 80} \\ \midrule[1.25pt] \rowcolor[gray]{.95}
d [\r{A}]  & Lattice constant &  5.830       &    5.829    \\ 
\midrule
L [\r{A}]  &  Interparticle spacing  &   34.186      &     29.384                              \\\rowcolor[gray]{.95}
\midrule
$\sigma_L$ [\r{A}]  & Interparticle spacing fluctuation &  1.321     &   2.322                              \\
\midrule
N [atomic planes]  & Number of atomic planes for NC &  14.394   &      14.306                           \\ \rowcolor[gray]{.95}
\midrule
$\sigma_N$ [atomic planes]  & Size distribution of NCs & 1.760       &   2.150                               \\  
\midrule
q-zero correction [\r{A}]  & Correction of diffractometer misalignment&  0.004          &    0.012                               \\ \rowcolor[gray]{.95}
\midrule
NC edge [nm]  & (N-1)$\times$d &  7.8 $\pm$ 0.3 &  7.7 $\pm$ 0.4\\
\bottomrule[1.25pt]
%
\end{tabular}
\label{SI - tab: fit}
\end{table}
%\FloatBarrier
\begin{figure}[h!]
    \centering    
    \begin{subfigure}[t]{\textwidth}
    \includegraphics[width=0.49\textwidth]{SI_fit_fresh1.png}
    \includegraphics[width=0.49\textwidth]{SI_fit_fresh2.png}
    \caption[Specimen fixed on the paper graph mask.]{}
    \label{SI - fig: fit_fresh}
    \end{subfigure}
    \begin{subfigure}[t]{\textwidth}
    \includegraphics[width=0.49\textwidth]{SI_fit_aged1.png}
    \includegraphics[width=0.49\textwidth]{SI_fit_aged22.png}
    \caption[Specimen fixed on the paper graph mask.]{}
    \label{SI - fig: fit_aged}
    \end{subfigure}
    \caption{Fitting of $2\theta$ $\sim$ 15$^\circ$ and $2\theta$ $\sim$ 30.5$^\circ$ (q $\sim$ 1.065, 2.111 \r{A}$^{-1}$ respectively, q$\,\, = (4\pi/\lambda_{exc})$sin$(\theta)$) XRD peaks of fresh (a) and aged (b) films performed by means of the multilayer diffraction model described in Ref. \cite{toso2021multilayer}.}
    \label{SI - fig: fit_XRD}
\end{figure}
%\FloatBarrier
\end{center}





\begin{figure}[h!]
     \centering
    \includegraphics[width=0.9\textwidth]{SI_SL_fresh_aged2.jpg}
    \caption{Cartoon of the main effect of aging that causes a decrease of NC spacing in SLs and an increase of NC spacing fluctuations. The NC close packing arrangement and preferential orientation is nevertheless preserved.}
    %Aging consequence: the main effect of aging on SLs is to decrease NC spacing and to increase NC spacing fluctuation. Nevertheless, NCs preserve the close packing arrangement and preferential orientation.}
    \label{SI - fig: SL_spacing}
\end{figure}
%\FloatBarrier






\clearpage
\section{Equilibrium optical properties}
\label{SI - sec: equilibrium optical properties}


Analysis of the pump-probe data by means of differential fitting of $\Delta R/R$ spectra requires a parametrization of the \ch{CsPbBr_3} optical constants at low temperature (T). We therefore build the real and imaginary parts of the low-T refractive index by employing a suitable Kramers-Kr\"onig consistent model, where the relevant parameters are chosen in agreement with the energy values obtained from the experimental absorbance measured at room temperature (fig. \ref{SI - fig: PL SL}) and with the temperature dependent trends reported in literature. 

The real part of the optical conductivity, $\sigma_1$, is obtained as the sum of a sigmoid function, accounting for the conduction band edge, a Drude-Lorentz oscillator, describing the exciton resonance, and a background component modelled through a high-energy Drude-Lorentz oscillator (figure \ref{SI - fig: equilibrium optical properties}a). The conduction band gap energy is red-shifted by $\sim$80 meV as compared to room temperature (fig. \ref{SI - fig: PL SL}) and the amplitude of the exciton peak is increased compared to the edge amplitude, consistently with the temperature dependence of \ch{CsPbBr_3} absorption spectra reported in literature \cite{guo2019dynamic,shcherbakov2021temperature}. The exciton binding is fixed at 43 meV, in agreement with literature reports \cite{protesescu2015nanocrystals,li2016temperature} and experimental absorbance (fig. \ref{SI - fig: PL SL}). The imaginary part $\sigma_2$ is then computed from Kramers-Kro\"nig relations. The resulting real and imaginary parts of the refractive index ($n$ and $k$, respectively) are plotted in figure \ref{SI - fig: equilibrium optical properties}b. We note that the results of the differential fitting procedure reported in the main text and in Sec. \ref{S_pumpprobe} are robust if small changes of the equilibrium dielectric function are introduced. 
\begin{figure}[h!]
\centering
\includegraphics[width=16cm]{SI_Equilirbium_optical_properties.pdf}
\caption{a) Parametrized real part of the optical conductivity at low temperature (black line). Filled areas represent the different components included in $\sigma_1$. b) Real 
($n$) and imaginary ($k$) parts of the parameterized low temperature refractive index.}
\label{SI - fig: equilibrium optical properties}
\end{figure}

\section{Fit of pump-probe data}
\label{S_pumpprobe}
For each measured pump-probe time delay $\Delta t$ between 4 ps and 130 ps, the spectrally resolved reflectivity variation $\Delta R/R$ is fitted with a function
\begin{equation}
    \frac{\Delta R}{R} = \frac{R_{outeq}-R_{eq}}{R_{eq}}
\end{equation}
where $R_{eq}$ is the equilibrium reflectivity estimated as $((1-n)^2+k^2)/((1+n)^2+k^2)$, $n$ and $k$ being the real and imaginary parts of the refractive index plotted in figure \ref{SI - fig: equilibrium optical properties}b. $R_{eq}$ is evaluated as the normal incidence bulk reflectivity because the light penetration depth in the photon energy range of interest is $\lesssim$ 1 \textmu m, which is smaller than the film overall thickness (on the order of $\sim$ 10 \textmu m).  $R_{outeq}$ is the out-of-equilibrium reflectivity, computed with the same procedure employed for $R_{eq}$ and containing the fit parameters. The free parameters for the differential fit are the out-of-equilibrium free-carriers edge energy position, the sigmoid edge width, the exciton energy and the exciton spectral weight, which represents the smallest subset of free parameters necessary to reproduce the out of equilibrium reflectivity at all fluences and all timescales. The data collected from NC superlattices require to include in the out-of-equilibrium optical conductivity also an additional feature at energy $\sim$2.36 eV, which is smaller then the exciton resonance. Since, at long time delays ($\Delta t \gtrsim$ 50 ps), the pump-probe map in figure 2b shows two distinct peaks between 2.30 eV and 2.37 eV, the experimental data is best described by modelling the new feature below band gap as a sum of two Drude-Lorentz oscillators centered at slightly different energies. This double-peak behaviour likely originates from local inhomogeneities of the samples within the photo-excited area, where superlattices of different lateral size or assembled from NCs of different size can be present and contribute to the measured signal. Figure \ref{SI - fig: fit high F} a and b show the the best differential fit at short (5 ps) and long (89 ps) time delays respectively, as reference examples. Fig. \ref{SI - fig: fit high F} c and d report the various components of the corresponding out-of-equilibrium optical conductivity (solid lines) obtained from the best fit to the data. 

\begin{figure}[h!]
\centering
\includegraphics[width=14.5cm]{SI_FitHighF.pdf}
\caption{Differential fit of $\Delta R/R$ data measured on NCs superlattices at 17 K with 230 \textmu J/cm$^2$ excitation fluence. a) and b) show the fitted spectra at 5 ps and 89 ps respectively. c) and d) report the equilibrium (dashed lines) and out of equilibrium (solid lines) components of the optical conductivity.}
\label{SI - fig: fit high F}
\end{figure}


\newpage
The output parameters of the fit performed at all time delays are displayed in figure \ref{SI - fig: fit high F dynamics}, where the blue solid lines represent the values at equilibrium and the red markers are the out-of-equilibrium values extracted from the fit. The top row reports the free-carriers sigmoid edge parameters: the edge amplitude (fig. \ref{SI - fig: fit high F dynamics}a)) is kept constants at all $\Delta t$; the red-shift (fig. \ref{SI - fig: fit high F dynamics}b)) observed after pump excitation decays with two distinct dynamics, a fast one ($\sim$ 20 ps) and a lower one taking place over hundreds of ps timescale; the edge width displays a broadening in the first $\sim$ 30 ps. The second row in fig. \ref{SI - fig: fit high F dynamics} shows the dynamics of the parameters of the Drude-Lorentz oscillator relative to the excitonic line, namely the exciton energy (fig. \ref{SI - fig: fit high F dynamics}d)), the plasma frequency $\omega_p$ (fig. \ref{SI - fig: fit high F dynamics}e)) and the exciton width (fig. \ref{SI - fig: fit high F dynamics}f). The exciton width does not show significant changes during the relaxation dynamics and is therefore kept constant at all $\Delta t$ to increase the stability of the fitting procedure. The two bottom rows in figure \ref{SI - fig: fit high F dynamics} report the parameters relative to the Drude-Lorentz oscillators appearing out of equilibrium and associated to the emergence of the cooperative behaviour discussed in the main text. Whereas one single new oscillator is needed to qualitatively reproduce the spectral feature around 2.36 eV at short time delays, two distinct oscillators allow to fit the more structured response observed at long time delays. For consistency, we employed the same model with two new oscillators also at short time delays, where the effect of two distinguished components is covered by the free-carriers edge shift and therefore the error-bars of the associated parameters are larger. 

It is  relevant to point out that the details of the model employed to fit the $\Delta R/R$ have only a marginal effect on the overall result of the fitting procedure. In addition to what presented in figure \ref{SI - fig: fit high F dynamics}, we tested out other models that involved either only one new oscillator, no free-carriers edge width variation, a narrower edge width at equilibrium, or different starting points for the parameters in figures \ref{SI - fig: fit high F dynamics}g)-l). Whatever the fit function employed, the extracted parameters and their relative dynamics display always compatible values and similar trends: the red-shift of the free-carriers edge decaying with two distinct dynamics, the blue-shift and the quench of the exciton recovering to the equilibrium values over $\sim$100 ps and the appearance of new oscillators that are red-shifted and narrower than the equilibrium excitonic peak. 

Inspection of the spectral weight transfer between the exciton resonance and the free-carrier band is performed by computing the spectral weight variation $\Delta$SW of the different components coming into play. 
Figure \ref{SI - fig: SW} reports $\Delta$SW of each component, estimated as the integral of the optical conductivity $\sigma_1$ over the energy axis. We observe that the SW lost after photo-excitation by the exciton peak is re-distributed between the new oscillators and free-carrier states in conduction band. Overall, the spectral weight is conserved, as shown by the black markers in figure \ref{SI - fig: SW} representing the sum of all SW variation contributions, which is compatible with zero at all time delays. 


\begin{figure}[h!]
\centering
\includegraphics[width=16cm]{SI_FitHighF_dynamics.pdf}
\caption{$\Delta R/R$  fit output parameters for the superlattice NCs sample at 230 \textmu J/cm$^2$ excitation fluence. a), b) and c) are the free-carriers edge amplitude, center energy and width, respectively. d), e) and f) represent the exciton parameters: exciton energy, plasma frequency and width, respectively. g)-l) display the Drude-Lorentz model parameters for the new oscillators.}
\label{SI - fig: fit high F dynamics}
\end{figure}

\begin{figure}[h!]
\centering
\includegraphics[width=8cm]{SI_SW.pdf}
\caption{Spectral weight variation of the exciton (blue), new oscillators (green), and conduction band (red) after pump excitation. Black points represent the sum of the four contributions, indicating SW conservation.}
\label{SI - fig: SW}
\end{figure}

\newpage
The same fitting procedure is applied to the data collected on the same superlattice sample with lower excitation fluence (30 \textmu J/cm$^2$), plotted in figure \ref{SI - fig: low F}. The fit results are displayed in figure \ref{SI - fig: fit low F} for two time delays (5 ps and 89 ps) and in figure \ref{SI - fig: fit low F dynamics} (fit output parameters for all time delays). In this lower fluence excitation scheme the free-electron states edge red-shift is smaller and relaxes with a slow dynamics of 130 ps decay time. The exciton does not show any energy shift, but its spectral weight decreases and is transferred to new oscillators appearing out of equilibrium. Since the free-electrons edge amplitude and width as well as the exciton position and width do not show any significant variation during the relaxation dynamics, they are kept fixed to increase the fit stability.

\begin{figure}[h!]
\centering
\includegraphics[width=8cm]{SI_LowFMap.pdf}
\caption{Ultrafast transient reflectivity of \ch{CsPbBr_3} superlattice sample measured at 17 K and low excitation fluence (30 \textmu J/cm$^2$).}
\label{SI - fig: low F}
\end{figure}

\begin{figure}[h!]
\centering
\includegraphics[width=14.5cm]{SI_FitLowF.pdf}
\caption{Differential fit of $\Delta R/R$ data measured on NCs superlattices at 17 K with 30 \textmu J/cm$^2$ excitation fluence. a) and b) show the fitted spectra at 5 ps and 89 ps respectively. c) and d) report the equilibrium (dashed lines) and out of equilibrium (solid lines) components of the optical conductivity.}
\label{SI - fig: fit low F}
\end{figure}

\begin{figure}[h!]
\centering
\includegraphics[width=16cm]{SI_FitLowF_dynamics.pdf}
\caption{$\Delta R/R$  fit output parameters for the superlattice NCs sample at 30 \textmu J/cm$^2$ excitation fluence. a), b) and c) are the free-carriers edge amplitude, center energy and width, respectively. d), e) and f) represent the exciton parameters: exciton energy, plasma frequency and width, respectively. g)-l) display the Drude-Lorentz model parameters for the new oscillators.}
\label{SI - fig: fit low F dynamics}
\end{figure}

\newpage
For the disordered NCs sample ($\Delta R/R$ data in fig. 2, the fit output is reported in figures \ref{SI - fig: fit dis}  and \ref{SI - fig: fit dis dynamics}. 
In this case, no additional oscillator is needed to reproduce the out of equilibrium optical conductivity. After pump excitation, a red-shift of the free-carriers edge decaying over a $\sim$ 20 ps timescale is revealed, along with a broadening of the edge width and a decrease of the exciton spectral weight. No significant variation of the free-electrons edge amplitude and of the exciton position and width is observed. 

\begin{figure}[h!]
\centering
\includegraphics[width=14.5cm]{SI_FitDis.pdf}
\caption{Differential fit of $\Delta R/R$ data measured on disordered NCs at 17 K. a) and b) show the fitted spectra at 4 ps and 84 ps respectively. c) and d) report the equilibrium (dashed lines) and out of equilibrium (solid lines) components of the optical conductivity.}
\label{SI - fig: fit dis}
\end{figure}

\begin{figure}[h!]
\centering
\includegraphics[width=16cm]{SI_FitDis_dynamics.pdf}
\caption{$\Delta R/R$  fit output parameters for the disordered NCs sample. a), b) and c) are the free-carriers edge amplitude, center energy and width, respectively. d), e) and f) represent the exciton parameters: exciton energy, plasma frequency and width, respectively.}
\label{SI - fig: fit dis dynamics}
\end{figure}

\clearpage
\section{Fluorescence}
Figure \ref{SI - fig: fluorescence} displays the  spectrum of the stray light coming at the detector when the pump pulse only impinges onto the sample. The narrow high energy component is associated to the scattering of the pump beam (blue region in Fig. S9). The radiation detected in the spectral range between 2.30 eV and 2.38 eV (orange region in Fig. S9) comes from the \ch{CsPbBr_3} superlattice sample and is associated to fluorescence emission.
\begin{figure}[h!]
\centering
\includegraphics[width=8cm]{SI_fluorescence.pdf}
\caption{Spectrum of the stray light detected when the sample is illuminated by the pump beam only.}
\label{SI - fig: fluorescence}
\end{figure}


\section{Exciton density}
\label{section: mean N}
In order to estimate the number of excitons generated by the pump pulse in each individual nanocube, we start from the equilibrium optical properties of \ch{CsPbBr_3} (figure \ref{SI - fig: equilibrium optical properties}) at the pump photon energy and calculate the reflectivity $R$, which results being $\approx12\%$. Since the penetration depth is much smaller than the sample thickess, we assume that all the radiation that is not reflected is absorbed. The number of absorbed photons per unit area is then given by
\begin{equation}
    N_{\gamma/area} = \frac{F}{\hbar \omega} (1-R)
\end{equation}
where $\hbar \omega$ is the photon energy and $F$ is the incident fluence. The incident photons are absorbed within a thickness of 135 nm, which corresponds to the penetration depth $l_p$ at the pump photon energy. 
The number of absorbed photons per unit volume is then estimated from
\begin{equation}
    N_{\gamma/volume} = \frac{N_{\gamma/area}}{l_p}.
\end{equation}
Lastly, the number of absorbed photons per nanocube can be obtained by taking into account the volume of each NC, modelled as a cube of $L $ = 8 nm side: 

\begin{equation}
    N_{\gamma/NC} = N_{\gamma/volume} L^3.
\end{equation}

We now assume that each absorbed photon produces an electronic excitation, of which $75\%$ are electron-hole bound states and $25\%$ are free carriers. These values are estimated from the contributions of exciton and band edge to the equilibrium optical conductivity at 17 K. Given the values of pump fluence employed in our experiment, we estimate to generate a number of excitons in each NC that can range between $0$ and $25$. Taking into account that the number of perovskite unit cells within each NC is $L^3/a^3$, $a$ = 5.83 \r{A}, this excitation regime corresponds to a photodoping level up to $\sim1\%$.

%\newpage

\bibliography{Refs}
\end{document}