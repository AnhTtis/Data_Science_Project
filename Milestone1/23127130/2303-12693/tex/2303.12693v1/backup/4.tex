

\documentclass[letterpaper, 12pt, journal, twoside]{support/IEEEtran}
\usepackage[fleqn]{amsmath}
\usepackage{times}
\usepackage[pdftex]{graphicx}
\usepackage{subfigure}
\usepackage{amsmath,amssymb,amsopn,amstext,amsfonts}
\usepackage{cancel}
\usepackage[noadjust]{cite}
\usepackage{soul}
\usepackage{caption}
\captionsetup{font={small}}

\captionsetup[figure]{labelfont={},textfont={}}


\usepackage{balance}
\usepackage{color}
\usepackage{mathtools}
% \usepackage{algorithm}
% \usepackage{algorithmic}
\usepackage{bm}
%\newtheorem{theorem}{Theorem}
\usepackage{ diagbox}
\usepackage{float}
\usepackage{epstopdf}
\usepackage{url}
\usepackage{multirow}
\usepackage{tikz}
\usepackage{subeqnarray}
\usepackage{cases}
\usepackage{booktabs}
\usepackage[linkcolor=black,citecolor=black,urlcolor=black,colorlinks=true]{hyperref}
\usepackage{algorithm}
\usepackage[noend]{algpseudocode}
\newtheorem{myTheo}{Theorem}
%\newtheorem{thm}{Theorem}[section] %如果不采用章节号做前缀,则不用[section]
\newtheorem{myDef}{Definition} %这句定义使得defn环境和thm共享编号
\newtheorem{lemma}{Lemma} %这句定义使得lem环境和thm共享编号
\newtheorem{myCollo}{Corollary}
\newtheorem{remark}{Remark}
%\newtheorem{lemma}{Lemma}
\newtheorem{myPro}{Proposition}
\newtheorem{assumption}{Assumption}
\newtheorem{example}{Example}
\soulregister\cite7
\soulregister\citep7
\soulregister\citet7
\soulregister\ref7
\soulregister\it7
\soulregister\pageref7

\bibliographystyle{support/IEEEtran}

\newcommand\px{\mathrel{/\mkern-5mu/}}  %平行
\newcommand{\ann}[1]{%
    \begin{tikzpicture}[remember picture, baseline=-0.75ex]%
        \node[coordinate] (inText) {};%
    \end{tikzpicture}%
    \marginpar{%
        \renewcommand{\baselinestretch}{1.0}%
        \begin{tikzpicture}[remember picture]%
            \definecolor{orange}{rgb}{1,0.5,0}%
            \draw node[fill=red!20,rounded corners,text width=\marginparwidth] (inNote){\footnotesize#1};%
    \end{tikzpicture}%
    }%
    \begin{tikzpicture}[remember picture, overlay]%
        \draw[draw = orange, thick]
            ([yshift=-0.2cm] inText)
                -| ([xshift=-0.2cm] inNote.west)
                -| (inNote.west);%
    \end{tikzpicture}%
}%

\graphicspath{{figures/}}
\DeclareGraphicsExtensions{.pdf,.png,.jpg,.eps}
\IEEEoverridecommandlockouts
%\overrideIEEEmargins

\title{\LARGE \bf Resilient Output Containment Control of Heterogeneous Multi-agent Systems against Composite Attacks: A Novel Digital Twin Approach}

%\title{Distributed Optimization in Prescribed-Time: Theory and Experiment}%
\author{
  \vskip 1em
  { 
  Xin Gong, \emph{Graduate Student Member, IEEE}, 
	Yukang Cui, \emph{Member, IEEE},
  Lingbo Cao
  }

  \thanks{
    This work was partially supported by the National Natural Science Foundation of China under Grant 61903258, 61973156, 61603180, Qatar National Research Fund NPRP12C-0814-190012. %(\emph{Corresponding author: Yukang Cui.}) %the National Natural Science Foundation of China under Grant 61903258

X. Gong is with the Department of Mechanical Engineering, The University of Hong Kong, Pokfulam Road, Hong Kong (e-mail: {\tt\small gongxin@connect.hku.hk}).


Y. Cui and T. Wang are with the College of Mechatronics and Control Engineering, Shenzhen University, Shenzhen, 518060, China (e-mail: {\tt\small cuiyukang,szuwtn@gmail.com}).


  
%J. He is with the Department of Mechanical Engineering, The University of Hong Kong, Pokfulam Road, Hong Kong (e-mail: {\tt\small esmehe@connect.hku.hk}). 

%X. Gong is with the Department of Mechanical Engineering, The University of Hong Kong, Pokfulam Road, Hong Kong, and also with the College of Mechatronics and Control Engineering, Shenzhen University, Shenzhen 518060, China. (e-mail: {\tt\small gongxin@connect.hku.hk}).
%China, and also
%with the Department of Mechanical Engineering, University of Hong Kong,
%Hong Kong
    
  }
%\thanks{$^{*}$ means the corresponding author.}
}

%\maketitle
%\author{}%\vspace{-0.0cm}
%%\thanks{This work was partially supported by.}% <-this % stops a space
%\thanks{$^{*}$These authors contribute equally and share the first authorship.}
%\thanks{$^{1}$Author is with the Group Robotics with Intelligent Planning (GRIP) Lab, Department of Mechanical Engineering, University of Hong Kong, Hong Kong,
%   {\tt\small gongxin@connect.hku.hk}}
%\thanks{Digital Object Identifier (DOI): see the top of this page.}
%\vspace{-0.5cm}}

% The note headers
%\markboth{Journal of \LaTeX\ Class Files,~Vol.~14, No.~8, August~2015}%
%{Shell \MakeLowercase{\textit{et al.}}: Bare Demo of IEEEtran.cls for IEEE Journals}
\markboth{IEEE Transactions on ...}{GONG \MakeLowercase{\textit{et al.}}: Resilient Output Containment Control of Heterogeneous MAS}%{He \MakeLowercase{\textit{et al.}}: Resilient Path Planning of UAVs against Covert Attacks on UWB Sensors}



\begin{document}
  \maketitle
  \begin{abstract}
    This brief deals with the distributed consensus observer design problem for high-order integrator multi-agent systems on directed graphs, which intends to estimate the leader state accurately in a prescribed time interval. A new kind of distributed prescribed-time observers (DPTO) on directed graphs is first formulated for the followers, which is implemented in a cascading manner. Then, the prescribed-time zero-error estimation performance of the above DPTO is guaranteed for both time-invariant and time-varying directed interaction topologies, based on strictly Lyapunov stability analysis and mathematical induction method. Finally, the practicability and validity of this new distributed observer are illustrated via a numerical simulation example.
\end{abstract}
\begin{IEEEkeywords}
  Consensus observer, Directed graphs, High-order Multi-agent systems, Prescribed-time stability
% Periodic positive systems, hyper-pyramid,
% reachable set estimation, S-procedure, state-feedback control.
%Formation-containment control,  high-order multi-agent systems,  observer-type protocols,  time-varying formation configuration
\end{IEEEkeywords}
\section{Introduction}
\IEEEPARstart{T}{he} last decade has witnessed substantial progresses contributed by various industries (see \cite{ xu2020distributed, liang2016leader, hua2017distributed, de2014controlling} and references therein) on distributed coordination of multi-agent systems (MAS). 








Two inevitable yet challenging difficulties arise when designing distributed finite-time observers for MAS:

\begin{enumerate}
  \item How can all followers obtain finite-time zero-error converge since the actual states of each pinned leader are only available to only a portion of followers, especially on a directed topology?
  \item How to regulate the consensus observation time arbitrarily despite the influences of the initial states of the MAS and network algebraic connectivity, particularly for high-order MAS on large directed networks?
\end{enumerate}





\begin{enumerate}
\item  A new kind of cascaded DPTO, based on a \textbf{hybrid constant and time-varying feedback}, is developed for the agents on directed graphs, which achieves the distributed accurate estimation on each order of leader state in a cascaded manner. 
\item \textbf{Zero-error converge on time-invariant/varying directed graphs}: In contrast to the previous work \cite{zuo2019distributed} which only guarantees finite-time attractiveness of a fixed error bound, we manage to regulate the observation error on directed graphs into zero in a finite-time sense. It is further found that the feasibility of this DPTO can be extended to some time-varying directed graphs.
%The difficulties caused by the asymmetrical Laplacian matrix under the circumstance of single-way directed communication topology are circumvented in the frameworks of distributed prescribed-time fault-tolerant control.
\item \textbf{Prescribed-time converge}: This DPTO could achieve distributed zero-error estimation in a predefined-time manner, whose needed time interval is independent of the initial states of all agents and network algebraic connectivity. Thus, the observer design procedure is much more easily-grasped for the new users than that in \cite[Theorem 2]{zuo2019distributed}.
\end{enumerate}



\noindent\textbf{Notations:}
In this brief, $\boldsymbol{1}_m$ (or $\boldsymbol{0}_m$) denotes a column vector of size $m$ filled with $1$ (respectively, 0). Denote the index set of sequential integers as $\textbf{I}[m,n]=\{m,m+1,\ldots~,n\}$ where $m<n$ are two natural numbers. Define the set of real numbers, positive real numbers and nonnegative real numbers as $\mathbb{R}$, $\mathbb{R}_{>0}$ and $\mathbb{R}_{\geq 0}$, respectively. ${\rm {\rm blkdiag}}({b})$ means a {\rm blkdiag}onal matrix whose {\rm blkdiag}onal elements equal to a given vector ${b}$.
For a given symmetric matrix $A\in \mathbb{R}^{n\times n}$, its spectrum can be sorted as: $\lambda_1(A)\leq\lambda_2(A) \leq\ldots \leq\lambda_n(A)$.%; moreover, $A>0$ means that $\lambda_1(A)>0$.
%For a time-varying function $x(t): \mathbb{R}_{\geq 0 }\mapsto \mathbb{R}$, denote that $\sup_{t\in [t_0, t_1]} x(t) $ and $\inf_{t\in [t_0, t_1]} x(t)$ as the upper bound and lower bound of $x(t)$ over the time interval $[t_0, t_1]$, respectively. Moreover, denote that $\|x(t)\|_{[t_0, t_1]} =\sup_{t\in [t_0, t_1]} \|x(t)\| $. Define that $L_{\infty}:=\{x(t)|x(t): \mathbb{R}_{\geq 0 }\mapsto \mathbb{R}^n,\ \|x(t)\|_{[t_0, t_1]}<\infty\}$. In the following sections, $x(t) \in L_{\infty}$, $t\in [t_0, t_1]$, represents that the variable $x$ is uniformly bounded over $[t_0, t_1]$.   %$A>eq 0$ (or $A> 0$) denotes that $A$ is a nonnegative matrix (positive matrix, respectively), which means all elements of $A$ are nonnegative (positive, respectively).
 %${\rm span}(x)$ denotes the span vector of a given vector $x=[p_1, p_2,\ldots~, p_n]^{\mathrm{T}}\in \mathbb{R}^n$.

\label{introduction}


\section{Preliminaries}\label{section2}

%\subsection{Notations}




{\color{black}
\subsection{Graph Theory}
In this article, $M$ Leaders and $N$ followers are consisidered on a directed graph $\mathcal{G}$.
The sets of leaders and followers are defined as $\mathcal{L}$ and $\mathcal{F}$ , respectively.
The directed graph of followers  can be represented by the subgraph                              
$\mathcal{G}_f =(\mathcal{V}, \mathcal{E}, \boldsymbol{A} )$  with the node set 
$\mathcal{V}=\{ 1, 2, \ldots~ , N \}$, the edge set
$\mathcal{E} \subset \mathcal{V} \times \mathcal{V}=\{(v_j,\ v_i)\mid\ v_i,\ v_j \in \mathcal{V}\}$ 
, and the associated adjacency matrix $\boldsymbol{A}=[a_{ij}] \in \mathbb{R}^{N\times N} $ .
The weight of the edge $(v_j,\ v_i)$ is denoted by $a_{ij}$ with $a_{ij} > 0$ if $(v_j,\ v_i) \in \mathcal{E}$ 
otherwise $a_{ij} = 0$. The neighbor set of node $v_i$ is represented by $\mathcal{N}_{i}=\{v_{j}\in \mathcal{V}\mid (v_j,\ v_i)\in \mathcal{E} \}$. Define the Laplacian matrix as 
$L=\mathcal{D}-\mathcal{A}  \in \mathbb{R}^{N\times N}$ 
with $\mathcal{D}={\rm blkdiag}(d_i) \in \mathbb{R}^{N\times N}$ where $d_i=\sum_{j \in \mathcal{F}} a_{ij}$ 
is the weight in-degree of node $v_i$.
The leader has no incoming edges and thus exhibits an autonomous behavior, 
while the follower has incoming edges and receives neighbor(including leader and follower) information 
directly. 
The interactions among the leaders and the followers are represented by 
$G_{ik}=${\rm blkdiag}($g_{ik}$) $\in \mathbb{R}^{N\times N}$  while $g_{ik}$ is the weight of the path from 
$i$th leader to kth follower.  And $g_{ik} = 1$ If there is a direct path from $i$th leader 
to $k$th follower , $g_{ik} > 0$, otherwise $g_{ik} = 0$.
}

\subsection{Some Useful Lemmas and Definition}
\begin{myDef}\label{def41}
  For the $i$th follower, the system accomplishes containment if there exists series of $\alpha_{\cdot i}$,
   which satisfy $\sum _{k \in \mathcal{L}} \alpha_{k i} =1$ to ensure following equation hold:
   \begin{equation}
    {\rm  lim}_{t\rightarrow \infty } (y_i(t)-\sum _{k\in \mathcal{L}} \alpha_{k i}y_k(t))=0,
   \end{equation}
   where $i \in \textbf{I}[1,N]$.
\end{myDef}
\begin{lemma}[Bellman-Gronwall Lemma \cite{lewis2003} ] \label{Bellman-Gronwall Lemma 2}
    Assuming $\Phi : [T_a,T_b] \rightarrow \mathbb{R}$ is a  continuous function,$C:[T_a,T_b] \rightarrow \mathbb{R}$ is nonnegative and integrable,$B \geq 0$ is a constant, and
    \begin{equation}
    \Phi(t) \leq B + \int_{0}^{t} C(\tau)\Phi(\tau) \,d\tau ,t \in [T_a,T_b] ,
    \end{equation}
    then we obtain that 
    $$\Phi(t) \leq B e^{\int_{0}^{t} C(\tau) \,d\tau }$$ for all $t \in [T_a,T_b]$.
\end{lemma}





\begin{lemma}[{\cite[Lemma 1]{cai2017 }}]\label{Lemma 1}
    Consider the following system
    $$\dot{x}=\epsilon F x +F_1(t)x+F_2(t)$$
    where $ x \in \mathbb{R}^{n\times n}$ , $F \in \mathbb{R}^{n\times n}$
    is Hurwitz, $\epsilon > 0$, $F_1(t) \in \mathbb{R}^{n\times n}$
    and $F_2(t) \in \mathbb{R}^{n}$ are bounded and continuous for all $t \geq  t_0$. We have (i) if
    $F_1(t)$, $F_2(t) \rightarrow 0 $ as  $t \rightarrow  \infty$ (exponentially), then for any $x(t_0)$ and
    any $\epsilon > 0$, $x(t) \rightarrow 0$ as $ t \rightarrow \infty$ (exponentially); 
    (ii) if $F_1(t) = 0$, $F_2(t)$ decays to zero exponentially at the rate of $\alpha$,
     and $\epsilon \geq \frac{\alpha }{\alpha_F} $, where $\alpha_F=\min(R(\sigma(-F)))$, 
     then, for any $ x(t_0)$, $x(t) \rightarrow 0 $ as $ t \rightarrow \infty$
    exponentially at the rate of $\alpha$.
\end{lemma}


\section{SYSTEM SETUP AND PROBLEM FORMULATION}\label{section3}
In this section, a new problem called resilient containment of
MAS group against composite attacks is proposed. First, the model of the MAS group is formulated and some basic definitions of the composite attacks is given.
{\color{black}
\subsection{MAS group Model}
In the framework of containment control, we consider a group of $N+M$ MAS, which can be divided into two groups:

1)$M$ leaders are the roots of directed graph $\mathcal{G}$, which have no neighbors. Define the index set of leaders as $\mathcal{L}= \textbf{I}[N+1,N+M]$.

2)$N$ followers who coordinate with their neighbors to achieve the containment set
of the above leader. Define the index set of followers
as $\mathcal{F} = \textbf{I}[1, N]$.

Similar to many existing works[]-[], we consider the following dynamics of leader :
\begin{equation}\label{EQ1}
\begin{cases}
\dot{x}_k=S x_k,\\
y_k=R x_k,
\end{cases}
\end{equation}
where $x_k\in \mathbb{R}^q$ and $y_k\in \mathbb{R}^p$ are system states and reference output of the $k$th leader, respectively.

The dynamics of each follower is given by 
\begin{equation}\label{EQ2}
  \begin{cases}
  \dot{x}_i=A_i x_i + B_i \bar{u}_i,\\
  y_i=C_i x_i,
  \end{cases}
\end{equation}
where $x_i\in \mathbb{R}^{ni}$, $u_i\in \mathbb{R}^{mi}$ and $y_i\in \mathbb{R}^p$ are 
system state, control input and output of the $i$th follower, respectively. For convenience, the notation $'(t)'$ can be omitted in the following discussion. We make the following assumptions about the agents and the communication network.

\begin{assumption}\label{assumption 2}
  The directed graph $\mathcal{G}$ contains a spanning tree with
the leader as its root.
\end{assumption}

\begin{assumption}\label{assumption 4}
  The real parts of the eigenvalues of $S$ are non-negative.
\end{assumption}




\begin{assumption}\label{assumption 3}
 The pair $(A_i, B_i)$ is stabilizable for $i \in \textbf{I}[1,N]$.
\end{assumption}



\begin{assumption}\label{assumption 5}
For all $\lambda \in \sigma(S)$, where $S$ represents the spectrum of $S$,
  \begin{equation}
   {\rm rank} \left[
      \begin{array}{c|c}
     A_i-\lambda I_{n_i} &  B_i  \\ \hline
    C_i  & 0   \\
      \end{array}
      \right]=n_i+p.
  \end{equation}
\end{assumption}

\begin{assumption}
  The graph $\mathcal{G}$ is strongly connected.
\end{assumption}


}

{\color{black}
\subsection{ Attack Descriptions}
In this work, we consider the multi-agent-systems consisting of cooperative agents with potential malicious attackers. As shown in the figure.1, the attackers will use four kinds of attacks to compromise the containment performance of the MASs:

1)DoS attacks: The communication graphs among agents (both in TL and CPL)  denied by attacker;

2)AAs:the motor inputs infiltrated by attacker to destroy the input signal of the agent;

3)FDI attacks:  the exchanging information among agents distort by attack;

4)CAs: mislead its downstream agents by disguising as a leader .

To resist the composite attack, we introduced a new layer named TL with the same communication topology as CPL, which yet have greater security and less physical meanings. Therefore, this TL could effectively against most of the above attacks. With the introduction of TL,the resilient control scheme can be decoupled to defend against DoS attacks on TL and defend against   potential unbounded AAs on CPL. The following two small subsections give the definitions and essential constraints for the DoS attacks and AAs, respectively.

1)Dos attacks: DOS attack refers to a type of attack where an adversary presents some or all components of a control system.It can affect the measurement and control channels simultaneously, resulting in the loss of data availability. Suppose that attackers can attack the communication network in a varing active period. Then it has to stop the attack activity and shift to a sleep period to reserve energy for the next attacks. Assume that there exists a $l \in \mathbb{N}$ , define $\{t_l \}_{l \in \mathbb{N}}$ and $\{\Delta_* \}_{l \in \mathbb{N}}$  as the start time and the duration time of the $l$th attack sequence of DoS attacks, that is , the $l$th DoS attack time-interval is $A_l = [t_l , t_l +\Delta_* )$ with $t_{l+1} > t_l +\Delta_* $ for all $l \in \mathbb{N}$. Therefore, for all $t\geq \tau \in \mathbb{R}$, the sets of time instants where the communication network is under Dos attacks are represent by
\begin{equation}
    S_A(\tau,t) = \cup A_l \cap [\tau , t],l\in \mathbb{N},
\end{equation}
and the sets of time instants where the communication network is allowed are 
\begin{equation}
    S_N(\tau,t) = [\tau,t] S_A / (\tau,t).
\end{equation}

\begin{myDef} [{Attack Frequency \cite{feng2017}   }]
For any $\delta_2 > \delta_1 \geq t_0$, let $N_a(\delta_1,\delta_2)$ represent the number of DoS attacks in $[\delta_1,\delta_2)$. Therefore, $F_a(\delta_1,\delta_2)= \frac{N_a(\delta_1,\delta_2)}{\delta_2 - \delta_1}$ is defined as the attack frequency at $[\delta_1,\delta_2)$ for all $\delta_2 > \delta_1 \geq t_0$.
\end{myDef}

\begin{myDef} [{ Attack Duration \cite{feng2017}  }]
For any $\delta_2 > \delta_1 \geq t_0$, let $T_a(\delta_1,\delta_2)$ represent the total time interval of DoS attack on multi-agent systems during  $[\delta_1,\delta_2)$. The attack duration over $[\delta_1,\delta_2)$ is defined as: there exist constants $\tau_G > 1$and $T_0 > 0$ such that
  \begin{equation}
    T_a(\delta_1,\delta_2) \leq T_0 + \frac{\delta_2-\delta_1}{\tau_G}. 
  \end{equation}
\end{myDef}

2) Unbounded Actuation Attacks:
For each follower, the system input is under unknown actuator fault, which is described as
\begin{equation}
    \bar{u}_i=u_i+d_i, \forall i  \in \mathcal{F},
\end{equation}
where $d_i$ denotes the unknown actuator fault caused in actuator channels. That is, the ture values of $\bar{u}_i$ and $d_i$ are unknown and we can only measure the damaged control input information $\bar{u}_i$.


\begin{assumption}
  The actuator attack $d_i$ is unbounded and its derivative $\dot{d}_i$ is bounded.
\end{assumption}

}






\subsection{ Problem Formulation}


%{\color{red}
Define the following local output formation containment error:
\begin{equation}\label{EQ xi}
    \xi_i = \sum_{j\in \mathcal{F}} a_{ij}(y_j -y_i) +\sum_{k \in \mathcal{L}} g_{ik}(y_k - y_i).
\end{equation}
The global form of (\ref{EQ xi}) is written as 
\begin{equation}
    \xi = - \sum_{k \in \mathcal{L}}(\Phi_k \otimes I_p)(y -  \underline{y}_k),
\end{equation}
where $\Phi_k = (\frac{1}{m} \mathcal{L } + G_{ik})$, $\xi = [\xi_1^T,\xi_2^T,\dots,\xi_n^T]^T$, $y=[y_1^T,y_2^T,\dots,y_n^T]^T$, and $\underline{y}_k = (l_n \otimes y_k)$.
\begin{lemma}
    Under Assumption 1, the matrixs $\Phi_k$ and $\sum_{k \in \mathcal{L}}\Phi_k $ are positive-definite and non-singular. Moreover, both $(\Phi_k)^{-1}$ and $(\sum_{k \in \mathcal{L}} \Phi_k)^{-1}$ are non-negative. 
\end{lemma}




Define the following global output containment error:
\begin{equation}
e= y - (\sum_{r\in \mathcal{L} }(\Phi_r \otimes I_p))^{-1} \sum_{k \in \mathcal{L} } (\Phi_k \otimes I_p) \underline{y}_k,
\end{equation}
where $e=[e_i^T,e_2^T,\dots,e_n^T]^T$ and $\xi = -\sum_{k \in \mathcal{L}}(\Phi \otimes I_p )e$.

\begin{lemma}[{\cite[Lemma 1]{zuo2019}}]
    Under Assumption 1, the output containment control objective in (\ref{def41}) is achieved if $\lim_{t \rightarrow \infty} e = 0$.
\end{lemma}



\noindent \textbf{Problem ACMCA} (Attack-resilient containment control of MASs
against Composite Attacks): The resilient containment control problem is to design the input $u_i$ in (1) for each follower , such that $\lim_{t \rightarrow \infty}e= 0$ in (10) with the case of unknown leader dynamics and under unknown unbounded cyber-attacks and network DoS attackers, i.e., the trajectories of each follower converges into a point in the dynamic convex hull spanned by trajectories of multiple leaders.


%}


































\section{Main Results}

Since the leader output $y_k(t)$ are only available to its neighbors and $S$,$R$ are unknown for all agents, the full distributed observers are proposed to estimate the unknown matrix $S$ and $R$ for all agents under the effect of Dos attacks. Then, a fully distributed virtual resilient layer is proposed to estimate the state of followers containment. Finally, new adaptive resilient state estimators and controllers are proposed to resist the influence of both DoS attacks and actuator faults.


\subsection{ Fully Distributed Observers to Estimate Leader States and Dynamics}
 In this section, we design distributed leader states and dynamics observers that are independent of the global graph topology and the global leader information.
  
  To facilitate the analysis, define the leader dynamics in (2) as follows:
  \begin{equation}
      \Upsilon =[S;R]\in \mathbb{R} ^{(p+q)\times q}
  \end{equation}
and its estimations be splited in two parts as follows:
 \begin{equation}
      \hat{\Upsilon } _{0i}(t)=[\hat{S}_{0i}(t);\hat{R}_{0i}(t)]\in \mathbb{R} ^{(p+q)\times q}
  \end{equation} 
  \begin{equation}
     \hat{\Upsilon } _{i}(t)=[\hat{S}_{i}(t);\hat{R}_{i}(t)]\in \mathbb{R} ^{(p+q)\times q},
  \end{equation} 
where $\hat{\Upsilon } _{0i}(t)$ and $\hat{\Upsilon } _{i}(t)$ will be updated by (19)and (20), and it converge to $\Upsilon$ at different rates.





%{\color{blue}
\begin{myTheo}\label{Theorem 1}
    Consider a group of $M$ leaders and $N$ followers with dynamic in (\ref{EQ1}) and (\ref{EQ2}) . Suppose that Assumption 2 and Assumption 6 holds . The problem of the leader unknown dynamics under the Dos attack is solved by the following dynamic estimates $\hat{\Upsilon } _{0i}(t)$ and 
$\hat{\Upsilon } _{i}(t)$ :
\begin{equation}\label{EQ15}
   \dot{\hat{\Upsilon }} _{0i}(t)=\sum_{j \in \mathcal{F}} w_{ij}(\hat{\Upsilon } _{0j}(t)-\hat{\Upsilon } _{0i}(t)) + \sum_{k  \in \mathcal L}w_{ik}(\Upsilon-\hat{\Upsilon } _{0i}(t)) ,\forall i \in \mathcal{F} ,
\end{equation}
\begin{equation}\label{EQ16}
    \dot{\hat{\Upsilon }} _{i}(t)=\left\lVert \hat{\Upsilon } _{0i}(t) \right\rVert_2(\hat{\Upsilon } _{0i}(t)-\hat{\Upsilon }_{i}(t)) +\sum_{j\in \mathcal{F}} w_{ij}(\hat{\Upsilon } _{j}(t)-\hat{\Upsilon } _{i}(t)) +\sum_{k\in \mathcal{L}}w_{ik}(\Upsilon-\hat{\Upsilon } _{i}(t)), \forall i \in \mathcal{F} .
\end{equation}
\end{myTheo}
%}

%{\color{blue}
\textbf{Proof.} 
Since we only use relative neighborhood information to estimate leader dynamics, the leader estimator will suffer the influence of Dos attacks. $w_{ij}(t)$ and $w_{ik}(t)$ is a designed weight for $i,j \in \mathcal{F}$ and $k\in \mathcal{L}$. For the denied
communication link , $w_{ij}=0$ and $w_{ik}=0$; And for for the  normal communication link, $w_{ij}=a_{ij}$ and $w_{ik}=g_{ik}$.

\textbf{Step 1:}
Let
$\tilde{\Upsilon}_{0i}(t)=\Upsilon-\hat{\Upsilon}_{0i}  (t)$,
then
\begin{equation}\label{EQ17}
    \dot{\tilde{\Upsilon}} _{0i} (t)
=\dot{\Upsilon}-\dot{\Upsilon}_{0i} (t) 
=-\sum_{j = 1}^{N} w_{ij}(\hat{\Upsilon } _{0j}(t)-\hat{\Upsilon } _{0i}(t)) +\sum_{k\in \mathcal{L}}w_{ik}(\Upsilon-\hat{\Upsilon } _{0i}(t)).
\end{equation}
 So, for the normal communication, the global estimation dynamics error  of system (\ref{EQ16}) can be written as
\begin{equation}\label{EQ 18}
    \dot{\tilde{\Upsilon}}_0 (t) = -\sum_{k\in \mathcal{L}}\Phi_k \otimes I_{p+q}\tilde{\Upsilon}_0(t) , t \geq t_0.
\end{equation}
where $\Phi_k= \frac{1}{M}L + G_k$, $\tilde{\Upsilon}_0(t)= [\tilde{\Upsilon}_{01}(t),\tilde{\Upsilon}_{01}(t),\dots,\tilde{\Upsilon}_{0N}(t)]$. 

And for the denied communication, we can see that 
\begin{equation}\label{EQ 19}
    \dot{\tilde{\Upsilon}}_0 (t) = 0, t\geq t_0.
\end{equation}
So, we can conclude (\ref{EQ 18}) and (\ref{EQ 19}) that
\begin{equation}
\begin{aligned}
  \tilde{\Upsilon}_0(t) &\leq \tilde{\Upsilon}_0(t_0)e^{-\sigma_{\rm max}(\Phi_k)\left\lvert S_A(t_0,\infty)\right\rvert } \\
  &\leq \tilde{\Upsilon}_0(t_0)e^{-\sigma_{\rm max}(\Phi_k)\left\lvert t - t_0- S_N(t_0,\infty)\right\rvert }.
\end{aligned}
\end{equation}
Under Assumption \ref{assumption 2} and the Lemma 6 of \cite{haghshenas2015}, all the eigenvalues of $\Phi_k$ have positive real parts. 
Therefore, we can conclude  that $lim_{t\rightarrow \infty }  \tilde{\Upsilon}_{0i}(t) = 0 $ for $i=I[1,N]$, exponentially.

\textbf{Step 2:}
Define the dynamics error $\tilde{\Upsilon}_i(t)$ as $\tilde{\Upsilon}_{i}(t)=\Upsilon-\hat{\Upsilon}_{i}(t)$ .
The inverse of $\tilde{\Upsilon}_i$ as the following
\begin{equation}
    \dot{\tilde{\Upsilon}} _{i}(t)
=\dot{\Upsilon}-\dot{\Upsilon}_{i}(t)
=-\left\lVert \hat{\Upsilon } _{0i}(t) \right\rVert_2(\hat{\Upsilon } _{0i}(t)-\hat{\Upsilon } _{i}(t)) - \sum_{j = 1}^{N} w_{ij}(\hat{\Upsilon } _{j}(t)-\hat{\Upsilon } _{i}(t)) +\sum_{k\in \mathcal{L}}w_{ik}(\Upsilon-\hat{\Upsilon } _{i}(t)).
\end{equation}
Then the estimated global dynamics error of $\tilde{\Upsilon}_i$ is
\begin{equation}
    \dot{\tilde{\Upsilon}}(t) =-(\sum_{k\in \mathcal{L}}\Phi_k^w \otimes I_{p+q} + {\rm blkdiag}(\left\lVert \hat{\Upsilon } _{0i}(t) \right\rVert_2) \otimes I_{p+q})\tilde{\Upsilon}(t)
+{\rm blkdiag}(\left\lVert \hat{\Upsilon } _{0i}(t) \right\rVert_2)\otimes I_{p+q} \tilde{\Upsilon}_0(t).
\end{equation}
where $\Phi_k^w (t) = \begin{cases}
 0 ,t \in S_A, \\ \Phi_k,t \in S_N,
\end{cases}$ and $\tilde{\Upsilon}_0(t)= [\tilde{\Upsilon}_{01}(t),\tilde{\Upsilon}_{01}(t),\dots,\tilde{\Upsilon}_{0N}(t)]$.
Consider the impact of Dos attacks, there exists the following relationship:
\begin{equation}
    \dot{\tilde{\Upsilon}}(t) = \begin{cases}
     -(\sum_{k\in \mathcal{L}}\Phi_k \otimes I_{p+q} + {\rm blkdiag}(\left\lVert \hat{\Upsilon } _{0i}(t) \right\rVert_2) \otimes I_{p+q})\tilde{\Upsilon}
+{\rm blkdiag}(\left\lVert \hat{\Upsilon } _{0i}(t) \right\rVert_2)\otimes I_{p+q} \tilde{\Upsilon}_0(t), t \in S_N(t_0,\infty).\\
-{\rm blkdiag}(\left\lVert \hat{\Upsilon } _{0i}(t) \right\rVert_2)\otimes I_{p+q} \tilde{\Upsilon}
+{\rm blkdiag}(\left\lVert \hat{\Upsilon } _{0i}(t) \right\rVert_2)\otimes I_{p+q} \tilde{\Upsilon}_0(t), t \in S_A(t_0,\infty).
    \end{cases}
\end{equation}
Then, according to the comparison method, we can obtain that
\begin{equation}
\begin{aligned}
    \dot{\tilde{\Upsilon}} (t)
    &\leq -{\rm blkdiag}(\left\lVert \hat{\Upsilon } _{0i}(t) \right\rVert_2)\otimes I_{p+q} \tilde{\Upsilon}(t)
+{\rm blkdiag}(\left\lVert \hat{\Upsilon } _{0i}(t) \right\rVert_2)\otimes I_{p+q} \tilde{\Upsilon}_0(t)
,t \in (t_0,\infty).
\end{aligned}
\end{equation}
based on Lemma \ref{Bellman-Gronwall Lemma 2}, one can show that
\begin{equation}
    \tilde{\Upsilon}(t) \leq (\tilde{\Upsilon}(t_0) + \int_{t_0} ^{t} {\rm blkdiag}(\left\lVert \hat{\Upsilon } _{0i}(t) \right\rVert_2)\otimes I_{p+q} \tilde{\Upsilon}_0(t) \,d\tau) e^{- {\rm max}(\left\lVert \hat{\Upsilon } _{0i}(t) \right\rVert_2)(t-t_0)}.
\end{equation}
where ${\rm min}(\left\lVert \hat{\Upsilon } _{0i}(t) \right\rVert_2) $ is the minimum value of $\left\lVert \hat{\Upsilon } _{0i}(t) \right\rVert_2 $ for $i\in\textbf{I}[1,N]$.
By $\left\lVert \hat{\Upsilon } _{0i}(t) \right\rVert_2 \geq 0 $ , $\tilde{\Upsilon}_0(t)$ converge to 0 exponentially, we can obtain that 
$lim_{t\rightarrow \infty }  \tilde{\Upsilon} = 0 $ exponentially. $\hfill \hfill \blacksquare $

%}







\subsection{ Distributed Resilient Estimator Design}
 In this section, A fully distributed virtual resilient layer is proposed to resist DoS attacks,
consider the following fully distributed virtual resilient layer: 
\begin{equation}\label{equation 200}
  \dot{z}_i=\hat{S}_i z_i -\chi (\sum _{j \in \mathcal{F} }w_{ij}(z_i-z_j)+\sum_{k \in \mathcal{L}}w_{ik}(z_i-x_k)),
\end{equation}
%{\color{blue}
where $z_i$ is the local state of the virtual layer and $\chi  >  0$ is the estimator gain designed in Theorem \ref{Theorem 2}. 
%}
The global state of virtual resilient layer   can be written as
\begin{equation}
  \dot{z}= \hat{S}_dz-\chi (\sum_{k \in \mathcal{L}}(\Psi_k^w \otimes I_p )(z-\underline{x}_k)), 
\end{equation}
where $\hat{S}_d={\rm blkdiag}(\hat{S}_i) $, $z=[z_1,z_2,\dots,z_N]$, $\underline{x}_k=l_n \otimes x_k$ and $\chi  >  0$ is the estimator gain designed in Theorem \ref{Theorem 2}.

Define the global virtual resilient layer state estimation error
\begin{equation}
\begin{aligned}
  \tilde{z} 
  &=z-(\sum_{r \in \mathcal{L}}(\Psi_r \otimes I_p ))^{-1} \sum_{k \in \mathcal{L}}(\Psi_k \otimes I_p ) \underline{x}_k ,\\
\end{aligned}
\end{equation}
then, for the normal communication, we have
%
%{\color{red}
\begin{equation}\label{EQ26}
    \begin{aligned}
      \dot{\tilde{z}}
       &=\hat{S}_dz-\chi (\sum_{k \in \mathcal{L}}(\Psi_k \otimes I_p )(z-\underline{x}_k))-
  (\sum_{r \in \mathcal{L}}(\Psi_r \otimes I_p ))^{-1}\sum_{k \in \mathcal{L}}(\Psi_k \otimes I_p ) (I_n \otimes S) \underline{x}_k \\
      &=\hat{S}_dz- (I_n \otimes S)z+ (I_n \otimes S)z 
  -(I_n \otimes S)(\sum_{r \in \mathcal{L}}(\Psi_r \otimes I_p ))^{-1} \sum_{k \in \mathcal{L}}(\Psi_k \otimes I_p )  \underline{x}_k  +M \\
  & -\chi \sum_{k \in \mathcal{L}}(\Psi_k \otimes I_p )(z-(\sum_{rr \in \mathcal{L}}(\Psi_{rr} \otimes I_p ))^{-1} (\sum_{kk \in \mathcal{L}}(\Psi_{kk} \otimes I_p )\underline{x}_{kk}+
  (\sum_{rr \in \mathcal{L}}(\Psi_{rr} \otimes I_p )^{-1} (\sum_{kk \in \mathcal{L}}(\Psi_{kk} \otimes I_p )\underline{x}_{kk}-\underline{x}_k)) \\
  &=\tilde{S}_dz+(I_n \otimes S) \tilde{z}-\chi\sum_{k \in \mathcal{L}}(\Psi_k \otimes I_p ) \tilde{z} -
  \chi(\sum_{kk \in \mathcal{L}}(\Psi_{kk} \otimes I_p ) \underline{x}_{kk}- \sum_{k \in \mathcal{L}}(\Psi_k \otimes I_p )\underline{x}_k) +M\\
  & =(I_n \otimes S) \tilde{z}-\chi\sum_{k \in \mathcal{L}}(\Psi_k \otimes I_p ) \tilde{z}+ \tilde{S}_d \tilde{z} + F_2^x(t) ,
    \end{aligned}
\end{equation}
where  $F_2^x(t)=\tilde{S}_d(\sum_{r \in \mathcal{L}}(\Psi_r \otimes I_p ))^{-1} \sum_{k \in \mathcal{L}}(\Psi_k \otimes I_p ) \underline{x}_k +M$ and
$M = (\sum_{r \in \mathcal{L}}(\Psi_r \otimes I_p ))^{-1}\sum_{k \in \mathcal{L}}(\Psi_k \otimes I_p ) (I_n \otimes S) \underline{x}_k-
(I_n \otimes S) (\sum_{r \in \mathcal{L}}(\Psi_r \otimes I_p ))^{-1}\sum_{k \in \mathcal{L}}(\Psi_k \otimes I_p ) \underline{x}_k$ and $\tilde{S}_d={\rm blkdiag}(\tilde{S}_i)$ for $i\in \mathcal{F}$.

For the denied communication, we have 
\begin{equation}\label{EQ27}
    \begin{aligned}
      \dot{\tilde{z}}
       &=\hat{S}_dz-
  (\sum_{r \in \mathcal{L}}(\Psi_r \otimes I_p ))^{-1}\sum_{k \in \mathcal{L}}(\Psi_k \otimes I_p ) (I_n \otimes S) \underline{x}_k \\
      &=\hat{S}_dz- (I_n \otimes S)z+ (I_n \otimes S)z 
  -(I_n \otimes S)(\sum_{r \in \mathcal{L}}(\Psi_r \otimes I_p ))^{-1} \sum_{k \in \mathcal{L}}(\Psi_k \otimes I_p )  \underline{x}_k  +M \\
  &=\tilde{S}_dz+(I_n \otimes S) \tilde{z} +M\\
  & =(I_n \otimes S) \tilde{z}+ \tilde{S}_d \tilde{z} + F_2^x(t) ,
    \end{aligned}
\end{equation}
So, we can conclude (\ref{EQ26}) and (\ref{EQ27}) that
\begin{equation}\label{EQ32}
    \dot{\tilde{z}}_i = \begin{cases}
     (I_n \otimes S) \tilde{z}-\chi\sum_{k \in \mathcal{L}}(\Psi_k \otimes I_p ) \tilde{z}+ \tilde{S}_d \tilde{z} + F_2^x(t), t \in S_N(t_0,t) \\
     (I_n \otimes S) \tilde{z}+ \tilde{S}_d \tilde{z} + F_2^x(t) , t \in S_A(t_0,t).
    \end{cases}
\end{equation}
Let 
\begin{equation}
   M= \sum_{k \in \mathcal{L}} M_k= \sum_{k \in \mathcal{L}} ((\sum_{r \in \mathcal{L}}(\Psi_r \otimes I_p ))^{-1} (\Psi_k \otimes I_p ) (I_n \otimes S)
-(I_n \otimes S) (\sum_{r \in \mathcal{L}}(\Psi_r \otimes I_p ))^{-1}(\Psi_k \otimes I_p )) \underline{x}_k .
\end{equation}


By the Kronecker product property $(P \otimes Q)(Y \otimes Z) =(PY)\otimes(QZ) $, we can obtain that 
\begin{equation}
\begin{aligned}
  & (I_N \otimes S)(\sum_{r \in \mathcal{L}} \Psi_r \otimes I_p)^{-1} (\Psi_k \otimes I_p) \\
   &= (I_N \otimes S)((\sum_{r \in \mathcal{L}} \Psi_r)^{-1} \Psi_k) \otimes I_p) \\
   &=(I_N \times (\sum_{r \in \mathcal{L}} \Psi_r)^{-1} \Psi_k))\otimes(S \times I_p) \\
   &= (\sum_{r \in \mathcal{L}} \Psi_r \otimes I_p)^{-1} (\Psi_k \otimes I_p)  (I_N \otimes S) .\\
\end{aligned}
\end{equation}


 We can show that $M_k=0$ and obtain that 
 \begin{equation}
     M=\sum_{k \in \mathcal{L}}M_k =0.
 \end{equation}





%}



Then $F_2^x(t)=\tilde{S}_d(\sum_{r \in \mathcal{L}}(\Psi_r \otimes I_p ))^{-1} \sum_{k \in \mathcal{L}}(\Psi_k \otimes I_p ) \underline{x}_k$ and $\lim_{t \rightarrow \infty}F_2^x(t)=0$ is exponentially at the rate of $\tilde{S}_d$.


{\color{blue}
Consider the following system
\begin{equation}\label{EQF3}
    \dot{\tilde{z}}=F_3(t)\tilde{z}+F_4(t),
\end{equation}
the integral of (\ref{EQF3}) show as the following:
\begin{equation}
    \tilde{z}(t)=(\tilde{z}(t_0)+ \int_{t_0}^{t} F_4(\tau) \, d\tau) +\int_{t_0}^{t} F_3(\tau)\tilde{z}(\tau) \, d\tau,
\end{equation}
recalling the Lemma \ref{Bellman-Gronwall Lemma 2} , we can obtain that 
\begin{equation}
    \tilde{z}(t) \leq (\tilde{z}(t_0)+ \int_{t_0}^{t_{\rm max}} F_4(\tau) \, d\tau )e^{\int_{t_0}^{t} F_3(\tau) \, d\tau}.
\end{equation}

Based on (36)~(38), we can obtain the following inequality from (\ref{EQ32})
\begin{equation}
    \tilde{z}(t) \leq \begin{cases}
     e^{\int_{t_{2k}}^{t} \hat{S}_d -\bar{\chi}\, d\tau}(\tilde{z}(t_{2k}) +\int_{t_{2k}}^{t_{\rm max}} F_2^x(\tau) \, d\tau) , t \in [t_{2k},t_{2k+1}) \\
     e^{\int_{t_{2k+1}}^{t} \hat{S}_d \, d\tau}(\tilde{z}(t_{2k+1}) +\int_{t_{2k+1}}^{t_{\rm max}} F_2^x(\tau) \, d\tau) , t \in [t_{2k+1},t_{2k+2}).
    \end{cases}
\end{equation}
where $\bar{\chi}= \chi\sum_{k \in \mathcal{L}}(\Psi_k \otimes I_p ) $.

{\color{blue}

\begin{myTheo}\label{Theorem 2}
    Consider the MASs \ref{EQ1}-\ref{EQ2} suffered from DoS attacks, which satisfy Assumption 1,
    and the DoS attack satisfying Definition 2 and Definition 3. 
    There exists $\chi >0$, $\eta^* \in (0,||\bar{\chi}||_2)$ such that the duration and frequency of the attack satisfy
    \begin{equation}\label{Dos1}
        \frac{1}{\tau} < 1-\frac{\sigma_{\rm max}(S_d)+\eta^*+\lambda}{\sigma_{\rm min}(\bar{\chi})}
    \end{equation}
    and 
    \begin{equation}\label{Dos2}
        \frac{n_a(t_0,t)}{t-t_0} \leq \frac{\eta^*}{||\bar{\chi}||_2 \Delta_*}
    \end{equation}
    where $S_d=I_N \otimes S$, $\lambda={\rm min}(||\hat{\Upsilon}_{0i}(t)||)$. Then it can be guaranteed that the estimation error $\tilde{z}$ exponentially converges to zero under DoS attacks.
\end{myTheo}

\textbf{Proof.} 
Nest, we will prove that the virtual layer achieve containment under the Dos attacks.
For clarity, we first redefine the set $S_A[t_0, \infty)$ as $S_A[t_0, \infty)= \bigcup_{k=0,1,2,\dots} [t_{2k+1},t_{2k+2})$, where $t_{2k+1}$ and $t_{2k+2}$ indicate the time instants that attacks start and end, respectively.
Then, the set $S_N[t_0,\infty)$ can be redefined as $S_N[t_0,\infty)$ as $S_N[t_0, \infty)= [t_0,t_1)\bigcup_{k=1,2,\dots} [t_{2k},t_{2k+1})$ .

For convenient, we described the two situation that MAS is with/without attacks as follows:
\begin{equation}
a(k)=\begin{cases}
 \int_{t_k}^{t_{k+1}} \hat{S}_d - \bar{\chi} \,d\tau,k=2i, \\
 \int_{t_k^{t_{k+1}}} \hat{S}_d \,d\tau,k=2i+1.
\end{cases}
\end{equation}
When $t\in [t_{2k},t_{2k+1})$ and letting $D_{2k}(t,r)=\int_{t_{2k}}^{t_{\rm max}} \hat{S}_d-\bar{\chi} \, d\tau +\sum_{m=r}^{2k} a(m-1)$, we obtain

\begin{equation}
\begin{aligned}
    \tilde{z} 
    &\leq  e^{\int_{t_{0}}^{t} \hat{S}_d(\tau) \, d\tau-\bar{\chi}|S_N|}\tilde{z}(t_0) \\
   & +\int_{t_0}^{t_1}e^{ D_{2k}(t,2)+ \int_{\tau}^{t_1} \hat{S}_d-\bar{\chi} \,d\tau^*  }  \, d\tau \int_{\tau}^{t_1}  F_2^x(\tau^*) \,   d\tau^* \\ 
   & +\int_{t_1}^{t_2} e^{ D_{2k}(t,3)+ \int_{\tau}^{t_2} \hat{S}_d\,d\tau^* }  \, d\tau \int_{\tau}^{t_2}  F_2^x(\tau^*) \, d\tau^*  \\
   &+ \dots \\
   & +\int_{t_{2k-1}}^{t_{2k}} e^{ D_{2k}(t,2k)+ \int_{\tau}^{t_{2k}} \hat{S}_d  \,d\tau^* }    \, d\tau  \int_{\tau}^{t_{2k}}  F_2^x(\tau^*) \, d\tau^* \\
   & +\int_{t_{2k}}^{t_{\rm max}} e^{  \int_{\tau}^{t_{\rm max}} \hat{S}_d-\bar{\chi}\,d\tau^* }  \, d\tau  \int_{\tau}^{t_{\rm max}}  F_2^x(\tau^*)\, d\tau^* \\
\end{aligned}
\end{equation}
to proof $\tilde{z}$ converges to $0$ exponentially, we use the mean value theorem of integrals to deal with error term $F_2^x$.
\begin{align}
    &\int_{t_{2k}}^{t_{\rm max}} F_2^x(\tau) \,d\tau \leq  \int_{t_{2k}}^{t_{\rm max}} a_f e^{-\lambda \frac{t_{\rm max}+t_{2k}}{2}} \,d\tau \\
    & \int_{t_{2k-1}}^{2k} F_2^x(\tau) \, d\tau \leq \int_{t_{2k-1}}^{t_{2k}} a_f e^{-\lambda\frac{t_{2k}+t_{2k-1}}{2}} \,d\tau \leq  \int_{t_{2k-1}}^{t_{2k}} a_f 
    e^{-\lambda(\frac{t_{\rm max}+t_{2k}}{2})+\lambda \frac{t_{\rm max}-t_{2k-1}}{2}} \,d\tau \\
    & \int_{t_{2k-2}}^{t_{2k-1}} F_2^x(\tau) \, d\tau \leq \int_{t_{2k-2}}^{t_{2k-1}} a_f e^{-\lambda\frac{t_{2k-1}+t_{2k-2}}{2}} \,d\tau \leq  \int_{t_{2k-2}}^{t_{2k-1}} a_f e^{-\lambda(\frac{t_{2k}+t_{2k-1}}{2})+ \lambda \frac{t_{2k}-t_{2k-2}}{2}} \,d\tau \\
    &\leq \int_{t_{2k-2}}^{t_{2k-1}} a_f e^{-\lambda(\frac{t_{\rm max}+t_{2k}}{2})+\lambda \frac{t_{\rm max}-t_{2k-1}}{2}+ \lambda \frac{t_{2k}-t_{2k-2}}{2}} \,d\tau
    \leq \int_{t_{2k-2}}^{t_{2k-1}} a_f e^{\lambda(t_{\rm max}-t_{2k-2})}e^{-\lambda \frac{t_{\rm max}+t_{2k}}{2}} \,d\tau \\
    & \dots \\
    & \int_{t_{0}}^{t_1} F_2^x(\tau) \, d\tau \leq \int_{t_{0}}^{t_1} a_f e^{-\lambda\frac{t_{1}+t_{0}}{2}} \,d\tau  \leq \int_{t_{0}}^{t_1} a_f e^{\lambda(t_{\rm max}-t_0)}e^{-\lambda \frac{t_{\rm max}+t_{2k}}{2}} \,d\tau \\
    & \int_{\tau}^{t_i} F_2^x(\tau^*) \, d\tau^* \leq \int_{\tau}^{t_i} a_f e^{\lambda (t_{\rm max}-\tau)}e^{-\lambda \frac{t_{\rm max}+t_{2k}}{2}} \, d\tau^* \leq (t_{\rm max} - \tau)a_f e^{\lambda (t_{\rm max}-\tau)}e^{-\lambda \frac{t_{\rm max}+t_{2k}}{2}}
\end{align}
for $i=I[1,2k]$.

Then, we have
\begin{equation}
\begin{aligned}
    \tilde{z} 
    &\leq  e^{\int_{t_{0}}^{t} \hat{S}_d(\tau) \, d\tau-\bar{\chi}|S_N|}\tilde{z}(t_0) \\
   & +\int_{t_0}^{t_1}e^{ D_{2k}(t,2)+\int_{\tau}^{t_1}\hat{S}_d-\bar{\chi}\, d\tau^*}\,d \tau  a_f e^{-\lambda \frac{t_{\rm max}+t_{2k}}{2}} \int_{t_0}^{t_1}e^{\lambda(t_{\rm max}-\tau)}\, d\tau \\
   & +\int_{t_1}^{t_2}e^{ D_{2k}(t,3)+\int_{\tau}^{t_2}\hat{S}_d\, d\tau^*}\,d \tau a_f e^{-\lambda \frac{t_{\rm max}+t_{2k}}{2}} \int_{t_1}^{t_2}e^{\lambda(t_{\rm max}-\tau)}\, d\tau \\
   &+ \dots \\
   & +\int_{t_{2k-1}}^{t_{2k}}e^{ D_{2k}(t,2k)+\int_{\tau}^{t_{2k}}\hat{S}_d\, d\tau^*}\,d \tau a_f e^{-\lambda \frac{t_{\rm max}+t_{2k}}{2}} \int_{t_{2k-1}}^{t_{2k}}e^{\lambda(t_{\rm max}-\tau)}\, d\tau \\
   & +\int_{t_{2k}}^{t_{\rm max}}e^{\int_{\tau}^{t}\hat{S}_d-\bar{\chi}\, d\tau^*}\,d \tau a_f e^{-\lambda \frac{t_{\rm max}+t_{2k}}{2}} \int_{t_{2k}}^{t_{\rm max}}e^{\lambda(t_{\rm max}-\tau)}\, d\tau \\
\end{aligned}
\end{equation}

Similar to the case that $t\in [t_{2k},t_{2k+1})$, one has the following inequality:
\begin{equation}
\begin{aligned}
    \tilde{z} 
    &\leq  e^{\int_{t_{0}}^{t} \hat{S}_d(\tau) \, d\tau-\bar{\chi}|S_N|}\tilde{z}(t_0) \\
   & +\int_{t_0}^{t_1}e^{ D_{2k+1}(t,2)+\int_{\tau}^{t_1}\hat{S}_d-\bar{\chi}\, d\tau^*}\,d \tau t_{\rm max}a_f e^{-\lambda \frac{t_{\rm max}+t_{2k+1}}{2}} \int_{t_0}^{t_1}e^{\lambda(t_{\rm max}-\tau)}\, d\tau \\
   & +\int_{t_1}^{t_2}e^{ D_{2k+1}(t,3)+\int_{\tau}^{t_2}\hat{S}_d\, d\tau^*}\,d \tau t_{\rm max}a_f e^{-\lambda \frac{t_{\rm max}+t_{2k+1}}{2}} \int_{t_1}^{t_2}e^{\lambda(t_{\rm max}-\tau)}\, d\tau \\
   &+ \dots \\
   & +\int_{t_{2k}}^{t_{2k+1}}e^{ D_{2k+1}(t,2k+1)+\int_{\tau}^{t_{2k}}\hat{S}_d -\bar{\chi} \, d\tau^*}\,d \tau t_{\rm max}a_f e^{-\lambda \frac{t_{\rm max}+t_{2k+1}}{2}} \int_{t_{2k}}^{t_{2k+1}}e^{\lambda(t_{\rm max}-\tau)}\, d\tau \\
   & +\int_{t_{2k+1}}^{t}e^{\int_{\tau}^{t}\hat{S}_d \, d\tau^*}\,d \tau t_{\rm max}a_f e^{-\lambda \frac{t_{\rm max}+t_{2k+1}}{2}} \int_{t_{2k+1}}^{t_{\rm max}}e^{\lambda(t_{\rm max}-\tau)}\, d\tau \\
\end{aligned}
\end{equation}
for $t\in[t_{2k+1},t_{2k+2})$ with $D_{2k+1}(t,r)= \int_{t_{2k+1}}^{t} \hat{S}_d \, d\tau +\sum_{m=r}^{2k+1} a(m-1)$ .

from (52) and (53) , one has

\begin{equation}
\begin{aligned}
   \tilde{z} & \leq e^{\int_{t_0}^{t} \hat{S}_d(\tau) \, d\tau-\bar{\chi}|S_N|}\tilde{z}(t_0) +    a_f e^{-\lambda \frac{t_{2k}+t_{\rm max}}{2}} \int_{t_0}^{t_{\rm max}} e^{ \int_{\tau}^{t_{\rm max}} \hat{S}_d(\tau^*) \, d\tau^*-\bar{\chi}|S_N(\tau,t_{\rm max})| +\lambda I_{N\times p}(t_{\rm max}-\tau)} \,d\tau \\
   % & \leq e^{\int_{t_0}^{t} \hat{S}_d(\tau) \, d\tau-\bar{\chi}|S_N|}\tilde{z}(t_0) +   c_f e^{-\lambda \frac{\sigma_{2i}+t}{2}} \int_{t_0}^{t} e^{ \int_{t_0}^{t} \hat{S}_d(\tau^*) \, d\tau^*-\bar{\chi}|S_N(t_0,t)| +\lambda(t-t_0)} \,d\tau .\\
\end{aligned}
\end{equation}
with

\begin{equation}
    \begin{aligned}
    &\int_{t_0}^{t} \hat{S}_d(\tau) \, d\tau-\bar{\chi}|S_N|\\
    & \leq \int_{t_0}^{t} I_N \otimes S + \tilde{S}_d(\tau) \, d\tau-\bar{\chi}|S_N(t_0,t)| \\
    & \leq  S_d (t-t_0)+||\int_{t_0}^{t}  \tilde{S}_d(\tau) \, d\tau|| -\bar{\chi}(t-t_0-|S_A(t_0,t)|) \\
    &\leq S_d (t-t_0)+||\int_{t_0}^{t}  \tilde{S}_d(\tau) \, d\tau|| - \bar{\chi}(t-t_0) + \bar{\chi}(\frac{t-t_0}{\tau_a} +S_0 +(1+n_a(t_0,t))\Delta_*)\\
    &\leq S_d (t-t_0)+||\int_{t_0}^{t}  \tilde{S}_d(\tau) \, d\tau|| -\bar{\chi}(t-t_0)
    +\frac{\bar{\chi}}{\tau_a}(t-t_0) +\bar{\chi}(S_0+\Delta_*) + \bar{\chi} n_a(t_0,t)\Delta_* \\
    & \leq (S_d-\bar{\chi}+\frac{\bar{\chi}}{\tau_a}+ \eta^*)(t-t_0) +||\int_{t_0}^{t}  \tilde{S}_d(\tau) \, d\tau||+ \bar{\chi}(S_0+\Delta_*)
    \end{aligned}
\end{equation}
where $c_f=a_f {\rm max}({c_k})$ , $S_d=I_N\otimes S$.

According to (\ref{Dos1}), (\ref{Dos2}) and (54), we can obtain that
\begin{equation}
\begin{aligned}
   \tilde{z}(t) 
   &\leq c_{ss} e^{-\eta(t-t_0)} \tilde{z}(t_0) +  \int_{t_0}^{t_{\rm max}} c_{ss} e^{-(\eta-\lambda)(t_{\rm max}-\tau)} \, d\tau \\
   &\leq c_{ss} e^{-\eta(t-t_0)} \tilde{z}(t_0)+ \frac{c_f c_{ss}}{\eta-\lambda} e^{-\lambda \frac{t_{2k}+t_{\rm max}}{2}} (1-e^{-(\eta-\lambda)(t_{\rm max}-t_0)})
\end{aligned}
\end{equation}

where $c_{ss}=e^{||\int_{t_0}^{t}  \tilde{S}_d(\tau) \, d\tau||+ \bar{\chi}(S_0+\Delta_*)}$, and 
%$\eta^*(t-t_0)=\bar{\chi} n_a(t_0,t)\Delta_*$.
 $\eta=(\bar{\chi}- \frac{\bar{\chi}}{\tau_a}-S_d- \eta^*)>0$,  by $\tilde{S}_d$ converge to $0$ exponentially, we can obtain that $c_{ss}$ is bounded. 
Obviously, it can conclude that $\tilde{z}$ exponentially converges to zero  exponentially. $\hfill \hfill \blacksquare $

}
}











%}













\subsection{ Full Distributed Output Regulator Equation Solves}


\begin{myTheo}
 suppose the Assumption 2,3,4 hold, if the estimated solutions to the output regulator equations in (8), including 
$\hat{\Delta}_{ji}$ for $j$ = $0, 1$, and $\hat{\Delta}_{i}$ , are adaptively solved as follows:
\begin{equation}\label{equation 310}
    \dot{\hat{\Delta}}_{0i} = -\mu \hat{\Phi }^T_i(\hat{\Phi }_i \hat{\Delta}_{0i}-\hat{\mathcal{R}}_i)
\end{equation}
\begin{equation}
    \dot{\hat{\Delta}}_{1i} = \left\lVert \hat{\Upsilon}_i\right\rVert_2(\hat{\Delta}_{0i}- \hat{\Delta}_{1i})-\mu\hat{\Phi }^T_i(\hat{\Phi }_i \hat{\Delta}_{1i}-\hat{\mathcal{R}}_i)
\end{equation}
\begin{equation}
   \dot{\hat{\Delta}}_{i} = \left\lVert \hat{\Upsilon}_i\right\rVert_2(\hat{\Delta}_{1i}- \hat{\Delta}_{i})-\mu\hat{\Phi }^T_i(\hat{\Phi }_i \hat{\Delta}_{i}-\hat{\mathcal{R} }_i) ,
\end{equation}

where $\mu > 0$, $\hat{\Delta}_{ji}={\rm vec}(\hat{Y}_{ji})$ for $j=0,1$ , $\hat{d}_i={\rm vec}(\hat{Y}_i), \hat{\phi}_i=(I_q \otimes  A_{1i}-\hat{S}_i^T \otimes A_{2i})$, $\hat{R}_i={\rm vec}(\mathcal{\hat{R}}^{\ast}_i)$; $\hat{Y}_{ji}=[\hat{\Pi}_{ji}^T$, $\hat{\Gamma}_{ji}^T]^T$ for $j=0,1$ , $\hat{Y_{i}}=[\hat{\Pi}_{i}^T$, $\hat{\Gamma}_{i}^T]^T$, $\hat{R}^{\ast}_i=[0,\hat{R}_i^T]^T$, $
A_{1i}=\left[
  \begin{array}{cc}
  A_i &  B_i  \\
  C_i &  0   \\
  \end{array}
  \right]$, $A_{2i}=
    \left[
  \begin{array}{cc}
   I_{n_i} & 0  \\
   0 &  0   \\
  \end{array}
  \right]$.
\end{myTheo}
\textbf{Proof.}In Theorem 1, we realize that leader dynamic estimation is time-varying, so the output solution regulator equation estimation influenced by leader dynamic estimation is also time-varying. Next, we begin proof that the output solution regulator equation estimation Converges exponentially to the output solution regulator equation.

\textbf{Step 1:}Firstly, we proof that $\hat{\Delta}_{0i}$ converges to $\Delta$ at an exponential rate.

From Assumption \ref{assumption 5} , we can get that the following output regulator equation:
 \begin{equation}\label{EQ10}
     \begin{cases}
      A_i\Pi_i+B_i\Gamma_i=\Pi_i S \\
      C_i\Pi_i = R
     \end{cases}
 \end{equation}
have solution matrices  $\Pi_i$ and $\Gamma_i$ for $i=I[1,N]$ .

Rewriting the standard output regulation equation (\ref{EQ10}) yields as follow:
\begin{equation}\label{EQ51}
 \left[
  \begin{array}{cc}
  A_i &  B_i  \\
  C_i &  0   \\
  \end{array}
  \right]
   \left[
    \begin{array}{cc}
   \Pi_i \\
   \Gamma_i \\
    \end{array}
    \right]
    I_q -
    \left[
  \begin{array}{cc}
   I_{n_i} & 0  \\
   0 &  0   \\
  \end{array}
  \right]
   \left[
    \begin{array}{cc}
\Pi_i\\
\Gamma_i \\
    \end{array}
    \right] S
    =
    \left[\begin{array}{cc}
         0  \\
         R
    \end{array}\right].
\end{equation}
%
Reformulating the equation (\ref{EQ51}) as follow:
\begin{equation}
    A_{1i}Y I_q -A_{2i}Y S=R_i^{\ast},
\end{equation}
where $A_{1i}=\left[
  \begin{array}{cc}
  A_i &  B_i  \\
  C_i &  0   \\
  \end{array}
  \right],
  A_{2i}=\left[
  \begin{array}{cc}
   I_{n_i} & 0  \\
   0 &  0   \\
  \end{array}
  \right],
  Y= \left[
    \begin{array}{cc}
\Pi_i\\
\Gamma_i \\
    \end{array}
    \right], $ and $R_i^{\ast}= \left[\begin{array}{cc}
         0  \\
         R
    \end{array}\right] $. Then the standard form of the linear equation in Equation (\ref{EQ51}) can be rewritten as:
\begin{equation}
    \Phi_i \Delta=\mathcal{R}_i,
\end{equation}
where $\phi_i=(I_q \otimes  A_{1i}-S^T \otimes A_{2i})$, $\Delta_i={\rm vec}( \left[
    \begin{array}{cc}
\Pi_i\\
\Gamma_i \\
    \end{array}
    \right])$, 
    $\mathcal{R}_i={\rm vec}(\mathcal{R}^{\ast})$, $R^{\ast}=\left[
    \begin{array}{cc}
0\\
R \\
    \end{array}
    \right]$.
\begin{equation}
\begin{aligned}
\dot{\hat{\Delta}}_{0i}
&= -\mu \hat{\Phi }^T_i(\hat{\Phi }_i \hat{\Delta}_{0i}-\hat{\mathcal{R}}_i) \\
&=-\mu \hat{\Phi}^T_i \hat{\Phi}_i \hat{\Delta}_{0i}+ \mu \hat{\Phi}_i \hat{\mathcal{R}_i} \\
&=-\mu \Phi_i^T \Phi_i \hat{\Delta}_{0i} +\mu \Phi_i^T \Phi_i \hat{\Delta}_{0i} -\mu \hat{\Phi}_i^T  \hat{\Phi}_i \hat{\Delta}_{0i} +\mu \hat{\Phi}_i^T \hat{\mathcal{R}_i} -\mu \Phi_i^T \hat{\mathcal{R}_i} + \mu \Phi_i^T \hat{\mathcal{R}_i} -\mu \Phi_i^T \mathcal{R}_i + \mu \Phi_i^T \mathcal{R}_i \\
&=-\mu \Phi_i^T \Phi_i \hat{\Delta}_{0i} + \mu (\Phi_i^T \Phi_i - \hat{\Phi}_i^T \hat{\Phi}_i) \hat{\Delta}_{0i} + \mu (\hat{\Phi}_i^T - \Phi_i^T) \hat{\mathcal{R}_i} + \mu \Phi_i^T (\hat{\mathcal{R}_i}-\mathcal{R}_i) + \mu \Phi_i^T \mathcal{R}_i \\
&=-\mu \Phi_i^T \Phi_i \hat{\Delta}_{0i} + \mu \Phi_i^T \mathcal{R}_i +d_1(t) ,\\
\end{aligned}
\end{equation}
where  $d_1(t)= -\mu (\hat{\Phi}_i^T\hat{\Phi}_i - \Phi_i^T\Phi_i) \hat{\Delta}_{0i} + \mu \tilde{\Phi}_i^T \hat{\mathcal{R}}_i + \mu \Phi_i^T \tilde{\mathcal{R}}_i$ . Then, since the origin of $\dot{\hat{\Delta}}_{0i}=-\mu \Phi_i^T\Phi_i \hat{\Delta}_{0i}$ is exponentially stable , system (\ref{equation 310}) is input-to-stable with $d_1(t)+ \mu \Phi_i^T \mathcal{R}_i$ as the input.
$\tilde{\Phi}_i=\hat{\Phi}_i-\Phi_i=\tilde{S}^T \otimes A_{2i}$, $\tilde{\mathcal{R}}={\rm vec}(\left[\begin{array}{cc}
    0  \\
    \tilde{R}_i
    \end{array}
    \right])$,
so $\lim _{t \rightarrow \infty}d_1(t) =0 $ exponentially at the same rate of $\tilde{\Upsilon}$.

\textbf{Step 2:}To prepare the proof for the next step, we use the solution of Step 1 to calculate the convergence rates of $\mu\hat{\Phi}^T_i(\hat{\Phi}_i \Delta_i-\hat{\mathcal{R} }_i)$.

Let $\tilde{\Delta}_{0i}=\Delta - \Delta_{0i}$, The time derivative of  $\tilde{\Delta}_{0i}$
can be computed via
\begin{equation}
\begin{aligned}
    \dot{\tilde{\Delta}}_{0i}
    =-\mu \Phi_i^T \Phi_i \tilde{\Delta}_{0i} - \mu \Phi_i^T \Phi_i \Delta+\mu \Phi_i^T \mathcal{R}_i +d_1(t)
    =-\mu \Phi_i^T \Phi_i \tilde{\Delta}_{0i} +d_1(t),
\end{aligned}
\end{equation}
 so   $\lim _{t \rightarrow \infty }\tilde{\Delta}_{0i}=0$ exponentially. In particular, if $\mu \geq \frac{\alpha_G}{ \sigma_{\rm min}(\Phi_i^T\Phi_i)}$($\mu$ is sufficiently large and $\alpha_G$ is the exponential convergence rate of $\tilde{\Upsilon}$ ), $\lim _{t \rightarrow \infty }\tilde{\Delta}_{0i}=0$ exponentially at the rate of $\alpha_G$.
 
 By
 \begin{equation}
 \dot{\tilde{\Delta}}_{0i}
=\dot{\Delta}-\dot{\hat{\Delta}}_{0i}
=-\mu\hat{\Phi}^T_i\hat{\Phi}_i\tilde{\Delta}_{0i}+\mu\hat{\Phi}^T_i(\hat{\Phi}_i \Delta_i-\hat{\mathcal{R} }_i),
 \end{equation}
coverage to $0$ exponentially at the rate of $\alpha_G$ and $ \sigma_{\rm min}(\mu\hat{\Phi}^T_i\hat{\Phi}_i)) \geq \alpha_G$, we can get that
$lim_{t\rightarrow \infty} \mu\hat{\Phi}^T_i(\hat{\Phi}_i \Delta_i-\hat{\mathcal{R} }_i)=0$ exponentially at the rate of $\alpha_G$ at least.

\textbf{Step 3:}Now, we proof that $\hat{\Delta}_{1i}$ and $\hat{\Delta}_i$ converge to $\Delta$, respectively.

Let $\tilde{\Delta}_{1i}= \Delta - \Delta_{1i}$, then, The time derivative of  $\tilde{\Delta}_{1i}$
can be computed via
\begin{equation}
    \dot{\tilde{\Delta}}_{1i}
=\dot{\Delta}-\dot{\hat{\Delta}}_{1i}
=-(\left\lVert \hat{\Upsilon}_i\right\rVert_2I_{p^2+pq}+ \mu\hat{\Phi}^T_i\hat{\Phi}_i)\tilde{\Delta}_{1i}+\left\lVert \hat{\Upsilon}_i\right\rVert_2\tilde{\Delta}_{0i} +\mu\hat{\Phi}^T_i(\hat{\Phi}_i \Delta_i-\hat{\mathcal{R} }_i),
\end{equation}
$ \sigma_{\rm min}(\left\lVert \hat{\Upsilon}_i\right\rVert_2I_{p^2+pq}+ \mu\hat{\Phi}^T_i\hat{\Phi}_i) > \alpha_G$, $\left\lVert \hat{\Upsilon}_i\right\rVert_2\tilde{\Delta}_{0i} +\mu\hat{\Phi}^T_i(\hat{\Phi}_i \Delta_i-\hat{\mathcal{R} }_i)$ coverage to $0$ exponentially at the rate of $\alpha_G$.
By Lemma1, we obtain $lim_{t\rightarrow \infty }  \tilde{\Delta}_{1i} = 0$ at the rate of $\alpha_G$,


Let $\tilde{\Delta}_{i}= \Delta - \Delta_{i}$ , as the same proof of $\tilde{\Delta}_{1i}$, it easy to get that $lim_{t\rightarrow \infty }  \tilde{\Delta}_{i} = 0$ exponentially at the rate of $\alpha_G$, the proof is completed. $\hfill \hfill \blacksquare $
%
\begin{lemma}[\cite{chen2019},Lemma 4]\label{Lemma 5}
    The adaptive distributed leader dynamics observers in (\ref{EQ16}) ensure $ \left\lVert \tilde{\Delta}_i \right\rVert \left\lVert x_k\right\rVert  $ and $\dot{\hat{\Pi}}_i x_k $  exponentially converge to zero.
\end{lemma}






\subsection{ Distributed Resilient Controller Design}

%{\color{blue}

In this section, we propose a fully distributed observer containment control method to solve the containment problem. considering  the full distributed resilient control protocols as follows:
\begin{align}
 & u_i=\hat{\Gamma}_i z_i +K_i \epsilon_i -\hat{d}_i  \\  
 & \epsilon_i=\hat{x}_i - \hat{\Pi}_i z_i\\ \label{EQ58}
& \dot{\hat{x}}_i=A_i \hat{x}_i +B_i(\hat{\Gamma}_i z_i +K_i \epsilon_i)+G_i(C_i\hat{x}_i - y_i) \\ 
&\hat{d}_i=\frac{B_i^T C_i^T P_i \tilde{y}_i}{\left\lVert \tilde{y}_i^T P_i C_i B_i \right\rVert +e^{-\beta_i t}} \hat{\rho_i} \\ \label{EQ65}
&\dot{\hat{\rho}}_i=\left\lVert \tilde{y}_i^T P_i C_i B_i \right\rVert
\end{align}
where $\tilde{y}_i=y_i - C_i \hat{x}_i$, $\hat{d}_i$ is an adaptive compensational signal, $\hat{\rho}_i$ is an adaptive updating parameter and the controller gain $K_i$ is designed as 
\begin{equation}
    K_i=-R_i^{-1}B_i^T P_i
\end{equation}
where $P_i$ is the solution to
\begin{equation}
    A_i^T P_i + P_i A_i + Q_i -P_i B_i R_i^{-1} B_i^T P_i =0.
\end{equation}
%

Since the state $x_i$ is unknown for other agent, we design a new state observer which design as (68)~(70) to observe the state $x_i$, which used output feedback control. This observer takes into account the unbounded attack that the agents may be suffered.
Next, we show that the observer state error converges exponentially to 0.

Define the estimator state error $\tilde{x}_i= x_i-\hat{x}_i$, from (\ref{EQ2}) and (\ref{EQ58}),  we obtain that
\begin{equation}
\begin{aligned}
        \dot{\tilde{x}}_i  
        &= A_i x_i +B_i(u_i +d_i) - A_i x_i -B_i(\hat{\Gamma}_i z_i +K_i \epsilon_i) -G(C_i \hat{x_i} - y_i)\\
        &=(A_i+ G_i C_i) \tilde{x}_i+ B_i \tilde{d}_i.
\end{aligned}
\end{equation}
where $\tilde{d}_i = d_i - \hat{d}_i$.

Define the following Lyapunov function:
\begin{equation}
    V_i^x =\tilde{x}_i^T C_i^T P_i C_i \tilde{x}_i,
\end{equation}
and its times derivative is presented as follows:
\begin{equation}\label{EQ74}
    \dot{V}_i^x = 2 \tilde{x}_i^T C_i^T P_i C_i (A_i+ G_i C_i) \tilde{x}_i + 2 \tilde{y}_i^T P_i C_i B_i \tilde{d}_i,
\end{equation}
%}
%{\color{red}
noting that
\begin{equation}
\begin{aligned}
\tilde{y}_i^T P_i C_i B_i\tilde{d}_i
        &= \tilde{y}_i^T P_i C_i B_i d_i -\frac{\left\lVert \tilde{y}_i^T P_i C_i B_i \right\rVert^2}{ \left\lVert \tilde{y}_i^T P_i C_i B_i \right\rVert +e^{-\beta_i t}} \hat{\rho_i} \\
& \leq  \left\lVert \tilde{y}_i^T P_i C_i B_i\right\rVert \left\lVert d_i \right\rVert - \frac{\left\lVert \tilde{y}_i^T P_i C_i B_i \right\rVert^2  }{\left\lVert \tilde{y}_i^T P_i B_i \right\rVert + e^{-\beta_i t}} \hat{\rho_i} \\
& \leq \frac{\left\lVert \tilde{y}_i^T P_i C_i B_i \right\rVert ^2 (  \left\lVert d_i \right\rVert-  \hat{\rho}_i)+ \left\lVert \tilde{y}_i^T P_i C_i B_i \right\rVert\left\lVert d_i \right\rVert e^{-\beta_i t}}{\left\lVert \tilde{y}_i^T P_i C_i B_i \right\rVert +e^{-\beta_i t} } .
\end{aligned}
\end{equation}
Noting that $d\left\lVert d_i \right\rVert/dt $ is bounded,  so, $\left\lVert d_i \right\rVert e^{-\beta_i t} \rightarrow 0$ . Choose $\left\lVert \tilde{y}_i^T P_i C_i B_i \right\rVert \geq d\left\lVert d_i \right\rVert/dt $, that is, $d\left\lVert d_i \right\rVert/dt - \dot{\hat{\rho}}_i < 0$. Then , $ \exists t_2 > 0$ such that for all $t \geq t_2$ , we have 
\begin{equation}\label{EQ76}
    \tilde{y}_i^T P_i C_i B_i \tilde{d}_i < 0  .
\end{equation}
%
Substituting (\ref{EQ76}) into (\ref{EQ74}) yields
\begin{equation}\label{EQ77}
\begin{aligned}
      \dot{V}_i^x 
      & \leq 2 \tilde{x}_i^T C_i^T P_i C_i (A_i+ G_i C_i) \tilde{x}_i \\
      &\leq 2\sigma_{\rm max}(A_i + G_iC_i) V_i^x,
\end{aligned}
\end{equation}
Solving inequality (\ref{EQ77}), we get
\begin{equation}
    V_i^x(t) \leq V(t_2)e^{2\sigma_{\rm max}(A_i + G_i C_i)(t-t_2)} ,\forall t \geq t_2.
\end{equation}
{\color{blue}
Since ($C_i,A_i$) is detectable, let $G_i$ be such that $A_i +G_i C_i$ is Hurwitz. It is obvious that $V_i^x$ is unbounded. So, $\tilde{x}_i$ are also unbounded.
}
\begin{myTheo}
 Consider  heterogeneous MAS consisting of   $M$ leader (\ref{EQ1}) and $N$ followers (\ref{EQ2}) with unbounded faults. Under Assumptions 1~6, the Problem ACMCA is solved by designing the dynamic estimation (19)~(20),the fully distributed virtual resilient layer(30),the standard output regulation equation (58),the  output  solutionregulator  equation  estimation(55)~(57) and the fully distributed controller consisting of (66)~(70).
\end{myTheo}
\textbf{Proof.}
Firstly, we need prove the state tracking error(67) coverage to 0.

The derivative of $\epsilon_i$ is presented as follows:
  \begin{equation}\label{EQ63}
  \begin{aligned}
      \dot{\epsilon}_i
      &= A_i \hat{x}_i + B_i u_i +B_i \hat{d}_i -\dot{\hat{\Pi}}_i  z_i - \hat{\Pi}_i(\hat{S}_i z_i - \chi (\sum _{j \in \mathcal{F} }a_{ij}(z_i-z_j)+\sum_{k \in \mathcal{L}}g_{ik}(z_i-x_k)))
      +G_i C_i (\hat{x}_i - x_i)\\
      &=(A_i+B_i K_i)\epsilon_i  - \dot{\hat{\Pi}}_i z_i +\chi \hat{\Pi}_i  (\sum _{j \in \mathcal{F} }a_{ij}(z_i-z_j)+\sum_{k \in \mathcal{L}}g_{ik}(z_i-x_k))
       +G_i C_i (\hat{x}_i - x_i).
  \end{aligned}
  \end{equation}
%  
 the globally state tracking error of (\ref{EQ63}) is
 \begin{equation}
 \begin{aligned}
  \dot{\epsilon}
  &={\rm blkdiag}(A_i+B_iK_i)\epsilon -{\rm blkdiag}(\dot{\hat{\Pi}}_i)z + \chi {\rm blkdiag}(\hat{\Pi}_i)(\sum_{k \in \mathcal{L}}(\Psi_k \otimes I_p )(z-\underline{x}_k))
  +{\rm blkdiag}(G_i C_i) (\hat{x}-x)\\
  &={\rm blkdiag}(A_i+B_i K_i)\epsilon -{\rm blkdiag}(\dot{\hat{\Pi}}_i)(\tilde{z}+ (\sum_{r \in \mathcal{L}}(\Psi_r \otimes I_p ))^{-1} \sum_{k \in \mathcal{L}}(\Psi_k \otimes I_p ) \underline{x}_k) + \chi {\rm blkdiag}(\hat{\Pi}_i)\sum_{k \in \mathcal{L}}(\Psi_k \otimes I_p )\tilde{z} \\
  & +{\rm blkdiag}(G_i C_i) (\hat{x}-x).\\
\end{aligned}
 \end{equation}
%


Consider the following Lyapunov function candidate :
\begin{equation} 
    V= \epsilon ^T P \epsilon,
\end{equation}
and its time derivate is given as follows: 
\begin{equation}\label{EQ80}
\begin{aligned}
    \dot{V}
    &= 2\epsilon^T P \dot{\epsilon}_i \\
    &=-\epsilon^T {\rm blkdiag}(Q_i) \epsilon +2\epsilon^T P ({\rm blkdiag}(\dot{\hat{\Pi}}_i)+  \chi {\rm blkdiag}(\hat{\Pi}_i)\sum_{k \in \mathcal{L}}(\Psi_k \otimes I_p ) )\tilde{z} \\
    & + 2\epsilon^T P {\rm blkdiag}(\dot{\hat{\Pi}}_i)(\sum_{r \in \mathcal{L}}(\Psi_r \otimes I_p ))^{-1} \sum_{k \in \mathcal{L}}(\Psi_k \otimes I_p ) \underline{x}_k - 2\epsilon^T P {\rm blkdiag}(G_i C_i)\tilde{x}\\
    &\leq -\sigma_{min}(Q) \left\lVert \epsilon\right\rVert^2  +2\left\lVert \epsilon^T\right\rVert \left\lVert P\right\rVert_2  \left\lVert {\rm blkdiag}(\dot{\hat{\Pi}}_i)+  \chi {\rm blkdiag}(\hat{\Pi}_i)\sum_{k \in \mathcal{L}}(\Psi_k \otimes I_p )  \right\rVert  \left\lVert \tilde{z}\right\rVert\\ 
 & +2\left\lVert \epsilon^T\right\rVert \left\lVert P\right\rVert_2   \left\lVert  
  (\sum_{r \in \mathcal{L}}(\Psi_r \otimes I_p ))^{-1}\sum_{k \in \mathcal{L}}(\Psi_k \otimes I_p ) {\rm blkdiag}(\dot{\hat{\Pi}}_i) \underline{x}_k\right\rVert + 2\left\lVert \epsilon^T\right\rVert \left\lVert P\right\rVert_2 \left\lVert {\rm blkdiag}(G_i C_i)\right\rVert \left\lVert \tilde{x} \right\rVert, \\
\end{aligned}
\end{equation}
where $Q={\rm blkdiag}(Q_i)$.
%
%
%
%

$\dot{\hat{\Pi}}_i x_k$ converge to zero exponentially and $\Phi_k$ is positive, so there exist positove constants $ V_{\Pi}$ and $ \alpha_{\Pi}$ such that
the following holds :
\begin{equation}\label{EQ81}
  \left\lVert  (\sum_{r \in \mathcal{L}}(\Psi_r \otimes I_p ))^{-1}\sum_{k \in \mathcal{L}}(\Psi_k \otimes I_p ) {\rm blkdiag}(\dot{\hat{\Pi}}_i) \underline{x}_k\right\rVert 
  \leq  V_{\Pi} \exp (-\alpha_{\Pi}),
\end{equation}
use Young's inequality , we have
\begin{equation}\label{EQ61}
\begin{aligned}
    & 2\left\lVert \epsilon^T\right\rVert \left\lVert P\right\rVert_2   \left\lVert  {\rm blkdiag}(\dot{\hat{\Pi}}_i) 
  (\sum_{r \in \mathcal{L}}(\Psi_r \otimes I_p ))^{-1}\sum_{k \in \mathcal{L}}(\Psi_k \otimes I_p ) \underline{x}_k\right\rVert \\
  & \leq \left\lVert \epsilon\right\rVert^2 + \left\lVert P\right\rVert_2  ^2 \left\lVert {\rm blkdiag}(\dot{\hat{\Pi}}_i) 
  (\sum_{r \in \mathcal{L}}(\Psi_r \otimes I_p ))^{-1}\sum_{k \in \mathcal{L}}(\Psi_k \otimes I_p ) \underline{x}_k\right\rVert^2 \\
  & \leq (\frac{1}{6} \sigma_{min}(Q) - \frac{1}{3}\beta_{V1} )\left\lVert \epsilon_i\right\rVert ^2 +\frac{\left\lVert P\right\rVert_2  ^2}{ (\frac{1}{6} \sigma_{min}(Q) - \frac{1}{3}\beta_{V1} )} V_{\Pi}^2 \exp (-2\alpha_{\Pi})\\
  & \leq
  (\frac{1}{6} \sigma_{min}(Q) - \frac{1}{3}\beta_{V1} )\left\lVert \epsilon_i\right\rVert ^2 +\beta_{v21}e^{-2\alpha_{v1}t} .
\end{aligned}
\end{equation}
From Lemma \ref{Lemma 5}, we can obtain that $\dot{\hat{\Pi}}$ converge to 0 exponentially. By $\tilde{z}$ and $\tilde{x}$ also converge to zero exponentially, we similarly obtain that
\begin{equation}\label{EQ62}
\begin{aligned}
   & 2\left\lVert \epsilon^T\right\rVert \left\lVert P\right\rVert_2  \left\lVert {\rm blkdiag}(\dot{\hat{\Pi}}_i)+  \chi {\rm blkdiag}(\hat{\Pi}_i)\sum_{k \in \mathcal{L}}(\Psi_k \otimes I_p )  \right\rVert  \left\lVert \tilde{z}\right\rVert \\
  & \leq (\frac{1}{6} \sigma_{min}(Q) - \frac{1}{3}\beta_{V1} )\left\lVert \epsilon\right\rVert ^2+\beta_{v22}e^{-2\alpha_{v2}t}.
\end{aligned}
\end{equation}
and
\begin{equation}\label{EQ68}
\begin{aligned}
   & 2\left\lVert \epsilon^T\right\rVert \left\lVert P\right\rVert_2 \left\lVert {\rm blkdiag}(G_i C_i)\right\rVert \left\lVert \tilde{x} \right\rVert, \\
  & \leq (\frac{1}{6} \sigma_{min}(Q) - \frac{1}{3}\beta_{V1} )\left\lVert \epsilon\right\rVert ^2+\beta_{v23}e^{-2\alpha_{v3}t}.
\end{aligned}
\end{equation}
Substituting (\ref{EQ61}) (\ref{EQ62}) and (\ref{EQ68}) into (\ref{EQ80}) yields the following:
\begin{equation}\label{EQ70}
  \dot{V} \leq -(\beta_{V1}+\frac{1}{2} \sigma_{min}(Q) ) \epsilon^T \epsilon +\beta_{V2}e^{\alpha_{V1}t}.
\end{equation}
Solving (\ref{EQ70}) yields the following:
\begin{equation}
\begin{aligned}
     V(t) 
    & \leq V(0) - \int_{0}^{t} (\beta_{V1}+\frac{1}{2} \sigma_{min}(Q) ) \epsilon^T\epsilon \,d\tau + \int_{0}^{t} \beta_{V2}e^{\alpha_{V1}t} \,d\tau ,\\
\end{aligned}
\end{equation}
then
\begin{equation}\label{EQ72}
    \epsilon^T \epsilon \leq -\int_{0}^{t} \frac{1}{\sigma_{min}(P)} \beta_{V3} \epsilon^T\epsilon \,d\tau + \bar{B}.
\end{equation}
where $\beta_{V3}= \beta_{V1}+\frac{1}{2} \sigma_{min}(Q) $ and $ \bar{B}=V(0)-\int_{0}^{t} \beta_{V2}e^{\alpha_{V1}t} \,d\tau $ are define as a bounded constant.
Recalling Bellman-Gronwall Lemma, (\ref{EQ72}) is rewritten as follows:

\begin{equation}
  \left\lVert \epsilon \right\rVert \leq \sqrt{\bar{B}} e^{-\frac{\beta_{V3}t}{2\sigma_{min}(P)} },
\end{equation}
it show that $\epsilon$ converge to 0 exponentially.
%
%
Hence, the global output synchronization error satisfieds that
\begin{equation}
\begin{aligned}
e &= y - (\sum_{r\in \mathcal{L} }(\Phi_r \otimes I_p))^{-1} \sum_{k \in \mathcal{L} } (\Phi_k \otimes I_p) \underline{y}_k \\
&= {\rm blkdiag}(C_i)(x - \hat{x}) + {\rm blkdiag}(C_i)\hat{x} -{\rm blkdiag}(C_i \hat{\Pi}_i)z +{\rm blkdiag}(C_i \hat{\Pi}_i)z -{\rm blkdiag}(R_i)z +{\rm blkdiag}(R_i) \tilde{z} \\
&= {\rm blkdiag}(C_i)\tilde{x} + {\rm blkdiag}(C_i)\epsilon  -{\rm blkdiag}(\tilde{R}_i)z +{\rm blkdiag}(R_i) \tilde{z}.
\end{aligned}
\end{equation}
{\color{blue}
Since $\epsilon$, $\tilde{R}_i$, $\tilde{z}$  converge to 0  exponentially and $\tilde{x}$ is  unbounded which have been proofed before, it is obviously that the global output containment error $e$ is unbounded. 
}
The whole proof is completed.
$\hfill \hfill \blacksquare $


%}








\section{Numerical Simulation}\label{SecSm}

% \begin{figure}[!htbp]
% %\begin{minipage}[t]{1\linewidth}
% \centering
% \includegraphics[width=0.6\textwidth]{4Ag.pdf}
% \caption{Time-varying directed communication topology among all agents}
% \label{fig:figure1}
% \end{figure}



%{\color{blue}
%\begin{figure}[htbp]
%\centering
%\subfigure[Performance of observer w.r.t. the leader]{
%\begin{minipage}[t]{0.475\textwidth}
%\centering
%\includegraphics[width=0.85\textwidth]{pic/pobs.eps}
%%\caption{fig1}
%\end{minipage}\label{fig:figure2:1}
%}
%%\hspace{-0.1in}
%\subfigure[Performance of observer w.r.t. the first leader]{
%\begin{minipage}[t]{0.475\textwidth}
%\centering
%\includegraphics[width=0.85\textwidth]{pic/vobs.eps}
%%\caption{fig2}
%\end{minipage}\label{fig:figure2:2}
%}\\%
%\centering
%\caption{Performance of two observers}
%\label{fig:figure2}
%\end{figure}












\section{Conclusion}
We investigate the distributed prescribed-time consensus observer for multi-agent systems with
high-order integrator dynamics and directed topologies in this brief.  To our best knowledge, the DPTO on time-invariant/varying digraphs with prescribed-time zero-error converge has been formulated for the first time, which could achieve distributed estimation w.r.t. the leader state within an arbitrary time interval predefined by the users. An illustrative simulation example has been conducted, which confirms the prescribed-time performance of the above DPTO. Future works will consider using this prescribed-time observer to deal with distributed fault-tolerant control problems \cite{hua2017distributed, xu2020distributed}. %\cite{3xiao2021distributed} 

\section*{Appendix}

 

\

% Proof: Consider the Lyapunov function candidate
% $$
% V_{1}=\frac{1}{2} \sum_{i=1}^{N} \xi_{i}^{T} P \xi_{i}+\sum_{i=1}^{N} \sum_{j=1, j \neq i}^{N} \frac{\left(c_{i j}-\alpha\right)^{2}}{8 \kappa_{i j}}
% $$
% where $\alpha$ is a positive constant that is to be determined later. Evidently, $V_{1}$ is positive definite. The time derivative of $V_{1}$ along the trajectory of (5) is given by
% $$
% \begin{aligned}
% \dot{V}_{1}=& \sum_{i=1}^{N} \xi_{i}^{T} P \dot{\xi}_{i}+\sum_{i=1}^{N} \sum_{j=1, j \neq i}^{N} \frac{c_{i j}-\alpha}{4 \kappa_{i j}} \dot{c}_{i j} \\
% =& \sum_{i=1}^{N} \xi_{i}^{T} P A \xi_{i}+\sum_{i=1}^{N} \xi_{i}^{T} P B K \sum_{j=1}^{N} c_{i j} a_{i j}\left(\tilde{x}_{i}-\tilde{x}_{j}\right) \\
% &+\sum_{i=1}^{N} \sum_{j=1, j \neq i}^{N} \frac{c_{i j}-\alpha}{4 \kappa_{i j}} \dot{c}_{i j}
% \end{aligned}
% $$
% Since $a_{i j}=a_{j i}$ and $c_{i j}(t)=c_{j i}(t)$, it can be easily verified that
% $$
% \begin{array}{rl}
% \sum_{i=1}^{N} \xi_{i}^{T} & P B K \sum_{j=1}^{N} c_{i j} a_{i j}\left(\tilde{x}_{i}-\tilde{x}_{j}\right) \\
% & =-\frac{1}{2} \sum_{i=1}^{N} \sum_{j=1}^{N} c_{i j} a_{i j}\left(\xi_{i}-\xi_{j}\right)^{T} \Gamma\left(\tilde{x}_{i}-\tilde{x}_{j}\right)
% \end{array}
% $$

 \bibliography{PIDFR}
 
 
{\color{blue}

\begin{myTheo}\label{Theorem 2}
    Consider the MASs \ref{EQ1}-\ref{EQ2} suffered from DoS attacks, which satisfy Assumption 1,
    and the DoS attack satisfying Definition 2 and Definition 3. 
    There exists $\chi >0$, $\eta^* \in (0,||\bar{\chi}||_2)$ such that the duration and frequency of the attack satisfy
    \begin{equation}\label{Dos1}
        \frac{1}{\tau} < 1-\frac{\sigma_{\rm max}(S_d)+\eta^*+\lambda}{\sigma_{\rm min}(\bar{\chi})}
    \end{equation}
    and 
    \begin{equation}\label{Dos2}
        \frac{n_a(t_0,t)}{t-t_0} \leq \frac{\eta^*}{||\bar{\chi}||_2 \Delta_*}
    \end{equation}
    where $S_d=I_N \otimes S$, $\lambda={\rm min}(||\hat{\Upsilon}_{0i}(t)||)$. Then it can be guaranteed that the estimation error $\tilde{z}$ exponentially converges to zero under DoS attacks.
\end{myTheo}

\textbf{Proof.} 
Nest, we will prove that the virtual layer achieve containment under the Dos attacks.
For clarity, we first redefine the set $S_A[t_0, \infty)$ as $S_A[t_0, \infty)= \bigcup_{k=0,1,2,\dots} [t_{2k+1},t_{2k+2})$, where $t_{2k+1}$ and $t_{2k+2}$ indicate the time instants that attacks start and end, respectively.
Then, the set $S_N[t_0,\infty)$ can be redefined as $S_N[t_0,\infty)$ as $S_N[t_0, \infty)= [t_0,t_1)\bigcup_{k=1,2,\dots} [t_{2k},t_{2k+1})$ .

For convenient, we described the two situation that MAS is with/without attacks as follows:
\begin{equation}
a(k)=\begin{cases}
 \int_{t_k}^{t_{k+1}} \hat{S}_d - \bar{\chi} \,d\tau,k=2i, \\
 \int_{t_k^{t_{k+1}}} \hat{S}_d \,d\tau,k=2i+1.
\end{cases}
\end{equation}
When $t\in [t_{2k},t_{2k+1})$ and letting $D_{2k}(t,r)=\int_{t_{2k}}^{t} \hat{S}_d-\bar{\chi} \, d\tau +\sum_{m=r}^{2k} a(m-1)$, we obtain

\begin{equation}
\begin{aligned}
    \tilde{z} (t)
    &=  e^{\int_{t_{0}}^{t} \hat{S}_d(\tau) \, d\tau-\bar{\chi}|S_N|}\tilde{z}(t_0) \\
   & +\int_{t_0}^{t_1} F_2^x(\tau)  e^{ D_{2k}(t,2)+ \int_{\tau}^{t_1} \hat{S}_d-\bar{\chi} \,d\tau^*  }  \, d\tau   \\ 
   & +\int_{t_1}^{t_2} F_2^x(\tau) e^{ D_{2k}(t,3)+ \int_{\tau}^{t_2} \hat{S}_d\,d\tau^* }  \, d\tau   \\
   &+ \dots \\
   & +\int_{t_{2k-1}}^{t_{2k}} F_2^x(\tau) e^{ D_{2k}(t,2k)+ \int_{\tau}^{t_{2k}} \hat{S}_d  \,d\tau^* }    \, d\tau   \\
   & +\int_{t_{2k}}^{t} F_2^x(\tau) e^{  \int_{\tau}^{t} \hat{S}_d-\bar{\chi}\,d\tau^* }  \, d\tau   \\
\end{aligned}
\end{equation}
to proof $\tilde{z}$ converges to $0$ exponentially, we use the mean value theorem of integrals to deal with error term $F_2^x$.
\begin{equation}
F_2^x(\tau)=a_f e^{-\lambda \tau}=a_f e^{-\lambda t} e^{\lambda(t-\tau)}
\end{equation}

Then, we have
\begin{equation}
\begin{aligned}
    \tilde{z} (t)
    &= e^{\lambda(t-\tau)} e^{\int_{t_{0}}^{t} \hat{S}_d(\tau) \, d\tau-\bar{\chi}|S_N|}\tilde{z}(t_0) \\
   & +\int_{t_0}^{t_1} a_f e^{-\lambda t} e^{ D_{2k}(t,2)+\int_{\tau}^{t_1}\hat{S}_d-\bar{\chi}\, d\tau^* +\lambda(t-\tau)}\,d \tau    \\
   & +\int_{t_1}^{t_2} a_f e^{-\lambda t} e^{ D_{2k}(t,3)+\int_{\tau}^{t_2}\hat{S}_d\, d\tau^* +\lambda(t-\tau) }\,d \tau  \\
   &+ \dots \\
   & +\int_{t_{2k-1}}^{t_{2k}} a_f e^{-\lambda t} e^{ D_{2k}(t,2k)+\int_{\tau}^{t_{2k}}\hat{S}_d\, d\tau^*   +\lambda(t-\tau)}\,d \tau  \int_{t_{2k-1}}^{t_{2k}}e^{\lambda(t-\tau)}\, d\tau \\
   & +\int_{t_{2k}}^{t} a_f e^{-\lambda t} e^{\int_{\tau}^{t}\hat{S}_d-\bar{\chi}\, d\tau^* +\lambda(t-\tau)}\,d \tau  \\
\end{aligned}
\end{equation}

Similar to the case that $t\in [t_{2k},t_{2k+1})$, one has the following inequality:
\begin{equation}
\begin{aligned}
    \tilde{z} 
    &= e^{\int_{t_{0}}^{t} \hat{S}_d(\tau) \, d\tau-\bar{\chi}|S_N|}\tilde{z}(t_0) \\
   & +\int_{t_0}^{t_1} a_f e^{-\lambda t} e^{ D_{2k+1}(t,2)+\int_{\tau}^{t_1}\hat{S}_d-\bar{\chi}\, d\tau^*  +\lambda(t-\tau)}\,d \tau \\
   & +\int_{t_1}^{t_2} a_f e^{-\lambda t} e^{ D_{2k+1}(t,3)+\int_{\tau}^{t_2}\hat{S}_d\, d\tau^*  +\lambda(t-\tau)}\,d \tau  \\
   &+ \dots \\
   & +\int_{t_{2k}}^{t_{2k+1}} a_f e^{-\lambda t} e^{ D_{2k+1}(t,2k+1)+\int_{\tau}^{t_{2k}}\hat{S}_d -\bar{\chi} \, d\tau^* +\lambda(t-\tau)}\,d \tau \\
   & +\int_{t_{2k+1}}^{t} a_f e^{-\lambda t} e^{\int_{\tau}^{t}\hat{S}_d \, d\tau^*  +\lambda(t-\tau)}\,d \tau \\
\end{aligned}
\end{equation}
for $t\in[t_{2k+1},t_{2k+2})$ with $D_{2k+1}(t,r)= \int_{t_{2k+1}}^{t} \hat{S}_d \, d\tau +\sum_{m=r}^{2k+1} a(m-1)$ .

from (52) and (53) , one has

\begin{equation}
\begin{aligned}
   \tilde{z} & = e^{\int_{t_0}^{t} \hat{S}_d(\tau) \, d\tau-\bar{\chi}|S_N|}\tilde{z}(t_0) +    a_f e^{-\lambda t} \int_{t_0}^{t} e^{ \int_{\tau}^{t} \hat{S}_d(\tau^*) \, d\tau^*-\bar{\chi}|S_N(\tau,t)| +\lambda I_{N\times p}(t-\tau)} \,d\tau \\
   % & \leq e^{\int_{t_0}^{t} \hat{S}_d(\tau) \, d\tau-\bar{\chi}|S_N|}\tilde{z}(t_0) +   c_f e^{-\lambda \frac{\sigma_{2i}+t}{2}} \int_{t_0}^{t} e^{ \int_{t_0}^{t} \hat{S}_d(\tau^*) \, d\tau^*-\bar{\chi}|S_N(t_0,t)| +\lambda(t-t_0)} \,d\tau .\\
\end{aligned}
\end{equation}
with

\begin{equation}
    \begin{aligned}
    &\int_{t_0}^{t} \hat{S}_d(\tau) \, d\tau-\bar{\chi}|S_N|\\
    & \leq \int_{t_0}^{t} I_N \otimes S + \tilde{S}_d(\tau) \, d\tau-\bar{\chi}|S_N(t_0,t)| \\
    & \leq  S_d (t-t_0)+||\int_{t_0}^{t}  \tilde{S}_d(\tau) \, d\tau|| -\bar{\chi}(t-t_0-|S_A(t_0,t)|) \\
    &\leq S_d (t-t_0)+||\int_{t_0}^{t}  \tilde{S}_d(\tau) \, d\tau|| - \bar{\chi}(t-t_0) + \bar{\chi}(\frac{t-t_0}{\tau_a} +S_0 +(1+n_a(t_0,t))\Delta_*)\\
    &\leq S_d (t-t_0)+||\int_{t_0}^{t}  \tilde{S}_d(\tau) \, d\tau|| -\bar{\chi}(t-t_0)
    +\frac{\bar{\chi}}{\tau_a}(t-t_0) +\bar{\chi}(S_0+\Delta_*) + \bar{\chi} n_a(t_0,t)\Delta_* \\
    & \leq (S_d-\bar{\chi}+\frac{\bar{\chi}}{\tau_a}+ \eta^*)(t-t_0) +||\int_{t_0}^{t}  \tilde{S}_d(\tau) \, d\tau||+ \bar{\chi}(S_0+\Delta_*)
    \end{aligned}
\end{equation}
where $c_f=a_f {\rm max}({c_k})$ , $S_d=I_N\otimes S$.

According to (\ref{Dos1}), (\ref{Dos2}) and (54), we can obtain that
\begin{equation}
\begin{aligned}
   \tilde{z}(t) 
   &\leq c_{ss} e^{-\eta(t-t_0)} \tilde{z}(t_0) +  \int_{t_0}^{t} c_{ss} e^{-(\eta-\lambda)(t-\tau)} \, d\tau \\
   &\leq c_{ss} e^{-\eta(t-t_0)} \tilde{z}(t_0)+ \frac{c_f c_{ss}}{\eta-\lambda} e^{-\lambda t} (1-e^{-(\eta-\lambda)(t-t_0)})
\end{aligned}
\end{equation}

where $c_{ss}=e^{||\int_{t_0}^{t}  \tilde{S}_d(\tau) \, d\tau||+ \bar{\chi}(S_0+\Delta_*)}$, and 
%$\eta^*(t-t_0)=\bar{\chi} n_a(t_0,t)\Delta_*$.
 $\eta=(\bar{\chi}- \frac{\bar{\chi}}{\tau_a}-S_d- \eta^*)>0$,  by $\tilde{S}_d$ converge to $0$ exponentially, we can obtain that $c_{ss}$ is bounded. 
Obviously, it can conclude that $\tilde{z}$ exponentially converges to zero  exponentially. $\hfill \hfill \blacksquare $

}



 
 
 


 
  \end{document}\grid


