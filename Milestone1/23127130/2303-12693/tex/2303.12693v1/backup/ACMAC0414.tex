
\documentclass[letterpaper, 12pt, journal, twoside]{support/IEEEtran}
\usepackage[fleqn]{amsmath}
\usepackage{times}
\usepackage[pdftex]{graphicx}
\usepackage{subfigure}
\usepackage{amsmath,amssymb,amsopn,amstext,amsfonts}
\usepackage{cancel}
\usepackage[noadjust]{cite}
\usepackage{soul}
\usepackage{caption}
\captionsetup{font={small}}

\captionsetup[figure]{labelfont={},textfont={}}


\usepackage{balance}
\usepackage{color}
\usepackage{mathtools}
% \usepackage{algorithm}
% \usepackage{algorithmic}
\usepackage{bm}
%\newtheorem{theorem}{Theorem}
\usepackage{ diagbox}
\usepackage{float}
\usepackage{epstopdf}
\usepackage{url}
\usepackage{multirow}
\usepackage{tikz}
\usepackage{subeqnarray}
\usepackage{cases}
\usepackage{booktabs}
\usepackage[linkcolor=black,citecolor=black,urlcolor=black,colorlinks=true]{hyperref}
\usepackage{algorithm}
\usepackage[noend]{algpseudocode}
\newtheorem{myTheo}{Theorem}
%\newtheorem{thm}{Theorem}[section] %如果不采用章节号做前缀,则不用[section]
\newtheorem{myDef}{Definition} %这句定义使得defn环境和thm共享编号
\newtheorem{Lemma}{Lemma} %这句定义使得lem环境和thm共享编号
\newtheorem{myCollo}{Corollary}
\newtheorem{remark}{Remark}
%\newtheorem{Lemma}{Lemma}
\newtheorem{myPro}{Proposition}
\newtheorem{assumption}{Assumption}
\newtheorem{example}{Example}
\soulregister\cite7
\soulregister\citep7
\soulregister\citet7
\soulregister\ref7
\soulregister\it7
\soulregister\pageref7

\bibliographystyle{support/IEEEtran}

\newcommand\px{\mathrel{/\mkern-5mu/}}  %平行
\newcommand{\ann}[1]{%
    \begin{tikzpicture}[remember picture, baseline=-0.75ex]%
        \node[coordinate] (inText) {};%
    \end{tikzpicture}%
    \marginpar{%
        \renewcommand{\baselinestretch}{1.0}%
        \begin{tikzpicture}[remember picture]%
            \definecolor{orange}{rgb}{1,0.5,0}%
            \draw node[fill=red!20,rounded corners,text width=\marginparwidth] (inNote){\footnotesize#1};%
    \end{tikzpicture}%
    }%
    \begin{tikzpicture}[remember picture, overlay]%
        \draw[draw = orange, thick]
            ([yshift=-0.2cm] inText)
                -| ([xshift=-0.2cm] inNote.west)
                -| (inNote.west);%
    \end{tikzpicture}%
}%

\graphicspath{{figures/}}
\DeclareGraphicsExtensions{.pdf,.png,.jpg,.eps}
\IEEEoverridecommandlockouts
%\overrideIEEEmargins

\title{\LARGE \bf Resilient Output Containment Control of Heterogeneous Multi-agent Systems against Composite Attacks: A Novel Digital Twin Approach}

%\title{Distributed Optimization in Prescribed-Time: Theory and Experiment}%
\author{
  \vskip 1em
  { 
  Xin Gong, \emph{Graduate Student Member, IEEE}, 
	Yukang Cui, \emph{Member, IEEE},
  Lingbo Cao
  }

  \thanks{
    This work was partially supported by the National Natural Science Foundation of China under Grant 61903258, 61973156, 61603180, Qatar National Research Fund NPRP12C-0814-190012. %(\emph{Corresponding author: Yukang Cui.}) %the National Natural Science Foundation of China under Grant 61903258

X. Gong is with the Department of Mechanical Engineering, The University of Hong Kong, Pokfulam Road, Hong Kong (e-mail: {\tt\small gongxin@connect.hku.hk}).


Y. Cui and Lingbo. Cao are with the College of Mechatronics and Control Engineering, Shenzhen University, Shenzhen, 518060, China (e-mail: {\tt\small cuiyukang,lingbcao@gmail.com}).


  
%J. He is with the Department of Mechanical Engineering, The University of Hong Kong, Pokfulam Road, Hong Kong (e-mail: {\tt\small esmehe@connect.hku.hk}). 

%X. Gong is with the Department of Mechanical Engineering, The University of Hong Kong, Pokfulam Road, Hong Kong, and also with the College of Mechatronics and Control Engineering, Shenzhen University, Shenzhen 518060, China. (e-mail: {\tt\small gongxin@connect.hku.hk}).
%China, and also
%with the Department of Mechanical Engineering, University of Hong Kong,
%Hong Kong
    
  }
%\thanks{$^{*}$ means the corresponding author.}
}

%\maketitle
%\author{}%\vspace{-0.0cm}
%%\thanks{This work was partially supported by.}% <-this % stops a space
%\thanks{$^{*}$These authors contribute equally and share the first authorship.}
%\thanks{$^{1}$Author is with the Group Robotics with Intelligent Planning (GRIP) Lab, Department of Mechanical Engineering, University of Hong Kong, Hong Kong,
%   {\tt\small gongxin@connect.hku.hk}}
%\thanks{Digital Object Identifier (DOI): see the top of this page.}
%\vspace{-0.5cm}}

% The note headers
%\markboth{Journal of \LaTeX\ Class Files,~Vol.~14, No.~8, August~2015}%
%{Shell \MakeLowercase{\textit{et al.}}: Bare Demo of IEEEtran.cls for IEEE Journals}
\markboth{IEEE Transactions on ...}{GONG \MakeLowercase{\textit{et al.}}: Resilient Output Containment Control of Heterogeneous MAS}%{He \MakeLowercase{\textit{et al.}}: Resilient Path Planning of UAVs against Covert Attacks on UWB Sensors}



\begin{document}
  \maketitle
  \begin{abstract}
 This work studies the distributed resilient output containment control of heterogeneous multi-agent systems against composite attacks, including Denial-of-Services (DoS) attacks, false-data injection attacks, camouflage attacks,and actuation attacks (AAs). Inspired by the concept of Digital Twin, a Twin Layer (TL) with  higher security and privacy is used to decouple the above Problem into the defense protocol against DoS attacks on the TL and  the defense protocol against AAs on the Cyber-Physical Layer (CPL). First, consider the case that not any follower knows the leaders dynamics, we provide distributed observers to estimate the dynamics of leaders for each follower. Then, distributed observers are provided to predict the trajectories of followers which used the estimates of leaders dynamics by the above observers to against DoS attacks. Moreover, we give the estimated solutions to the output regular equations by estimating the  leaders dynamics. 
On the CPL, decentralized adaptive attack-resilient control schemes are provided to deal with the unbounded AAs. Furthermore, the output containment error could achieve uniformly ultimately bounded (UUB) convergence by the above control protocols and the boundary of output containment error can be given explicitly. Finally, the simulation has shown the effectiveness of these proposed control protocols.
\end{abstract}
\begin{IEEEkeywords}
  Containment, Directed graphs, High-order Multi-agent systems, Twin Layer
% Periodic positive systems, hyper-pyramid,
% reachable set estimation, S-procedure, state-feedback control.
%Formation-containment control,  high-order multi-agent systems,  observer-type protocols,  time-varying formation configuration
\end{IEEEkeywords}
\section{Introduction}
\IEEEPARstart{D}{ISTRIBUTED}  cooperative control of multi-agent systems (MASs) has attracted extensive attention over the past decade due to its broad applications in physics, biology, social activity, and engineering. As one of the most general phenomena in the area of cooperative control of MAS, consensus has attracted extensive attention in recent years.
Consensus Problems can be categorized into two classes, that is, leaderless consensus and leader–follower consensus for a single leader \cite{fax2004information,lewis2013cooperative,choi2009distributed,olfati2004consensus,olfati2007consensus,zhao2016distributed}.
Those work considered consensus for the homogeneous case, where all agents and the leader have the same dynamics.
However, in some practical applications, the individual systems may have different dynamics, and in practice their states may have different dimensions.
Thus, recent work examines consensus in heterogeneous MAS networks where the dynamics of agents may be different.
Since heterogeneous MAS can have different state dimensions, the state consensus is meaningless for heterogeneous MAS. So, the output consensus is considered, where the output is assumed to have the same dimension.

The output consensus problem of heterogeneous MAS with a single leader has been widely studied in past decades \cite{wieland2011internal,su2011cooperative,su2012cooperative,kim2010output,wang2010distributed,lunze2012synchronization,yaghmaie2016output}. The agent dynamics considered in all are heterogeneous with different system matrices and possibly different state dimensions.
There are generally two types of control protocols, related to the output regulator equations, are provided to deal with the output consensus problem of linear heterogeneous MAS, that is, feedforward method and internal model principle. 
The feedforward method studied in \cite{wieland2011internal,su2011cooperative,su2012cooperative,kim2010output} requires a feedback gain to create the closed-loop system matrix Hurwitz, and a feedforward gain  drive the trajectory of the closed-loop system toward a subspace that make the regulated output zero. The feedforward gain is a linear function of the solution of the output regulator equations.  The internal model principle studied in \cite{wang2010distributed,lunze2012synchronization,yaghmaie2016output} because it is robust against  plant parameters variation. A dynamic compensator is designed for each agent, which includes a dynamic p-copy internal model, where p is the agent's output.


For the case of multiple leaders, the output consensus is called containment control problem. It is
need to drive each follower into the convex hull spanned by the multiple leaders.
 Containment control problems are motivated by some natural phenomena and have important practical applications, such as it ensuring that the vehicle can enter the safety zone surrounded by the leaders. And in this work, we work toward the resilient  control schemes of output containment of heterogeneous MASs against composite attacks. 

In the past few years, the containment problem has been studied for heterogeneous MAS, where all the agents have
different dynamics \cite{haghshenas2015,zuo2017output,zuo2019,gong2020,zuo2020,lu2021,zuo2021 }. Haghshenas et al. \cite{haghshenas2015} and Zuo et al. \cite{zuo2017output} considered the  containment control problem of heterogeneous linear multi-agent systems. Zuo et al. \cite{zuo2019} considered the  containment control problem of heterogeneous linear multi-agent systems against malicious attacks. Gong et al. \cite{gong2020}  presented the  observer-type protocol to solve the  formation-containment control problem of  MAS, Zuo et al. \cite{zuo2020,lu2021} presented the control scheme for the formation-containment control under the unknown actuator attacks (AAs) and unknown input, respectively. And in this work, we work toward the resilient  control schemes of output containment of heterogeneous MASs against composite attacks.


In addition to AAs, there exist other attacks that may affect the security, robustness, resiliency, safety and information integrity of the MASs, such as Denial-of Service (DoS) attacks \cite{fengz2017,zhangd2019,yed2019,deng2021,yang2020}, replay attacks \cite{stux2011}, false-data injection (FDI) attacks \cite{zuo2021}, camouflage attacks (CAs) \cite{zuo2019F}, and two main methods have been proposed to deal with the above attacks. The first method is diagnosis-based defense \cite{mana2014,teix2014} which need to  detect, identify, and remove (or overcome) the compromised agents. While the approach is effective to various types of attacks, however, the premise of this method is that more than a half of neighbors of each agents should be intact. The second method is attack-resilient defense \cite{hc2017,Gusr2018,Moghadam2018} which can maintain an acceptable system performance under malicious attacks without the above premise.
But the second method lacks generality, that is, most of the existing works \cite{hc2017,Gusr2018,Moghadam2018} need to know the dynamic of leaders, which might not be known for each follower in some application, to design the control law, and it could only cope with one kind of attacks.
In this work, the recent sprungup hierarchical control approach [25], [27], [28] is  used to enrich the applicability of the second approach. This work consider four kinds of attacks: DoS attacks, FDI attacks, AAs, and CAs. A Twin Layer (TL) is provided to against the CAs and FDI attacks. Moreover, the TL could decouple the defense scheme against composite attacks into the task of against DoS attacks on the TL    and the task of against potentially unbounded AAs  task on the CyberPhysical Layer (CPL). In the following two paragraphs, we briefly discuss the advanced technology of the above tasks.

 Resilient control protocols of MASs against DoS attacks have been investigated in \cite{fengz2017,zhangd2019,yed2019,deng2021,yang2020}. Feng et al. \cite{feng2017} achieves the leader-follower consensus through its latest state and the  leader's dynamics under the DoS attack. Yang et al. \cite{yang2020} designed a dual-terminal event-triggered mechanism against DoS attacks for linear leader-following MASs. Deng et al. \cite{deng2021} proposed a distributed resilient learning algorithm to learn the unknown dynamic model under the influence of DoS attacks. For the case that the dynamic model of the leader is unknown for all followers, Cai et al. and Chen et al. \cite{cai2017,chen2019} used a adaptive distributed observer to estimated the leader dynamics and state for the MAS with single leader. Herein, motivated by the above works \cite{fengz2017,zhangd2019,yed2019,deng2021,yang2020} and \cite{cai2017,chen2019}, we formulate a distributed TL with distributed resilient observers and estimators. Since the leaders dynamic is unknown for all followers, we used adaptive observers to estimate the leaders dynamic. Then, distributed  estimators are provided to predict the trajectory of containment. Moreover, the estimated solutions of the regulator equations are given to deal with the state tracking errors.
 
 Resilient control protocols of MASs against AAs have been studied in \cite{chen2019,zuo2020,lu2021}.  A pioneering work  \cite{de2014resilient} presented a distributed resilient adaptive control scheme against attacks  in the form of exogenous disturbances and interagent uncertainties. Xie et al. \cite{xie2017decentralized} deal with the decentralized adaptive fault-olerant control problem for a class of uncertain large-scale interconnected systems with external disturbances and AAs. Moreover, Jin et al, \cite{jin2017adaptive} provided an adaptive controller to guaranteed that the closed-loop dynamical system against time-varying adversarial sensor and actuator attacks could achieve uniformly ultimately bounded (UUB) performance. It should be noted that the above work \cite{de2014resilient,xie2017decentralized,jin2017adaptive} could only deal with bounded AAs (or disturbances) which cannot maximize their destructive capacity indefinitely. Recently, Zuo et al. \cite{zuo2020} provided a adaptive control scheme for MAS against unbounded AAs. Based on the above work, we provided an adaptive protocol, featured by a soft-sign function, to deal with unbounded AAs on the CPL. Futhermore, the upper bound of the UUB boundary is depicted explicitly, which facilitates the design of other users.


 %In order to solve the problem of output containment of heterogeneous MASs against composite attacks consisting of the above four kinds of attacks. Based on the second method, we used  distributed observers to estimate the dynamics of leaders  for each follower, then a double-layer resilient control scheme is proposed to cope with the composite attacks consisting of the above four kinds of attacks. The double-layer consists of a Twin layer (TL) and a Cyber-Physical Layer (CPL). The TL could cope with the CAs and FDI attacks  since the information transmitted on the CPL is not used in the hierarchal control scheme. Then, the TL decouple the defense scheme against composite attacks into the task of against DoS attacks   and the task of against AAs  task. The task of against DoS attacks  could be dealt with on the TL and the AAs attacks could be coped with on the CPL.
 
  %Our main contributions are summarised as follows. First,   a double-layer resilient control scheme is proposed to deal with the output containment of heterogeneous MAS against composite attacks. Next, we consider the special case with the dynamics of leaders is unknown for each follower and design observers to estimate the dynamics of leaders. Moreover, different from the existing work Deng et al. [12] which the estimator of leader state start to work need the necessary time to finish the  estimation of leaders dynamics, the TL in our work could work in all time. Then, we consider the unbounded AAs and give the boundary of UUB of the output containment.
 The above discussions indicate that the problem of output containment of heterogeneous MASs against composite attacks has not
been solved or studied. In this work, we provide a hierarchical control scheme to deal with the above problem.
%More specifically, the main contributions and characteristics of this work can be summarized as follows: 
Moreover, Our main contributions are summarised as follows:
\begin{enumerate}
  \item A double-layer resilient control scheme is proposed to deal with the output containment of heterogeneous MAS against composite attacks.
  \item we consider the special case with the dynamics of leaders is unknown for each follower and design observers to estimate the dynamics of leaders. Moreover, different from the existing work Deng et al. [12] which the estimator of leader state start to work need the necessary time to finish the  estimation of leaders dynamics, the TL in our work could work in all time.
  \item we consider the unbounded AAs and give the upper boundary of UUB of the output containment.  
\end{enumerate}




\noindent\textbf{Notations:}
 In this work, $I_r$ is the identity matrix with compatible dimension.
 $\boldsymbol{1}_m$ (or $\boldsymbol{0}_m$) denotes a column vector of size $m$ filled with $1$ (respectively, 0). Denote the index set of sequential integers as $\textbf{I}[m,n]=\{m,m+1,\ldots~,n\}$, where $m<n$ are two natural numbers. Define that the set of real numbers and the set of natural numbers as $\mathbb{R}$ and $\mathbb{N}$, respectively. Define ${\rm {\rm blkdiag}}(A_1,A_2,\dots,A_N)$ as a block diagonal matrix whose principal diagonal elements equal to given matrices $A_1,A_2,\dots,A_N$. $\sigma_{\rm min}(X)$, $\sigma_{\rm max}(X)$ and $\sigma(X)$ denote the minimum singular value, maximum singular value and  the spectrum of matrix $X$, respectively. $\lambda_{\rm min}(X)$ and $\lambda_{\rm max}(X)$ represent the minimum eigenvalue, maximum eigenvalue of $X$. Herein, $||\cdot||$ denotes the Euclidean norm, and $\otimes$  denotes the Kronecker product.
%; moreover, $A>0$ means that $\lambda_1(A)>0$.
%For a time-varying function $x(t): \mathbb{R}_{\geq 0 }\mapsto \mathbb{R}$, denote that $\sup_{t\in [t_0, t_1]} x(t) $ and $\inf_{t\in [t_0, t_1]} x(t)$ as the upper bound and lower bound of $x(t)$ over the time interval $[t_0, t_1]$, respectively. Moreover, denote that $\|x(t)\|_{[t_0, t_1]} =\sup_{t\in [t_0, t_1]} \|x(t)\| $. Define that $L_{\infty}:=\{x(t)|x(t): \mathbb{R}_{\geq 0 }\mapsto \mathbb{R}^n,\ \|x(t)\|_{[t_0, t_1]}<\infty\}$. In the following sections, $x(t) \in L_{\infty}$, $t\in [t_0, t_1]$, represents that the variable $x$ is uniformly bounded over $[t_0, t_1]$.   %$A>eq 0$ (or $A> 0$) denotes that $A$ is a nonnegative matrix (positive matrix, respectively), which means all elements of $A$ are nonnegative (positive, respectively).
 %${\rm span}(x)$ denotes the span vector of a given vector $x=[p_1, p_2,\ldots~, p_n]^{\mathrm{T}}\in \mathbb{R}^n$.

\label{introduction}


\section{Preliminaries}\label{section2}

%\subsection{Notations}




{\color{black}
\subsection{Graph Theory}
 We use a directed graph to represent the interaction among agents. For a MAS consisting of $n$ agents, the graph                              
$\mathcal{G} $  can be expressed by a pair $(\mathcal{V}, \mathcal{E}, \mathcal{A} )$, where $\mathcal{V}=\{ 1, 2, \ldots~, N \}$ is the node set, 
$\mathcal{E} \subset \mathcal{V} \times \mathcal{V}=\{(v_j,\ v_i)\mid v_i,\ v_j \in \mathcal{V}\}$ is the edge set, and $\mathcal{A}=[a_{ij}] \in \mathbb{R}^{N\times N} $ is the associated adjacency matrix.
The weight of the edge $(v_j,\ v_i)$ is denoted by $a_{ij}$ with $a_{ij} > 0$ if $(v_j,\ v_i) \in \mathcal{E}$ 
otherwise $a_{ij} = 0$. The neighbor set of node $v_i$ is represented by $\mathcal{N}_{i}=\{v_{j}\in \mathcal{V}\mid (v_j,\ v_i)\in \mathcal{E} \}$. Define the Laplacian matrix as 
$L=\mathcal{D}-\mathcal{A}  \in \mathbb{R}^{N\times N}$ 
with $\mathcal{D}={\rm blkdiag}(d_i) \in \mathbb{R}^{N\times N}$ where $d_i=\sum_{j \in \mathcal{F}} a_{ij}$ 
is the weight in-degree of node $v_i$.
A directed is \emph{strongly connected} if there is a path from $v_i$ to $v_j$ for any pair of nodes $[v_i,v_j]$.



}

\subsection{Some Useful Lemmas and Definition}

\begin{Lemma}[Bellman-Gronwall Lemma \cite{lewis2003}] \label{Bellman-Gronwall Lemma 2}
    Assuming $\Phi : [T_a,T_b] \rightarrow \mathbb{R}$ is a nonnegative continuous function, $\alpha:[T_a,T_b] \rightarrow \mathbb{R}$ is integrable, $\kappa \geq 0$ is a constant, and
    \begin{equation}
    \Phi(t) \leq \kappa + \int_{0}^{t} \alpha (\tau)\Phi(\tau) \,d\tau,t \in [T_a,T_b],
    \end{equation}
    then we obtain that 
    $$\Phi(t) \leq \kappa e^{\int_{0}^{t} \alpha(\tau) \,d\tau }$$ for all $t \in [T_a,T_b]$.
\end{Lemma}






\begin{Lemma}[{\cite[Lemma 1]{cai2017 }}]\label{Lemma 1}
    Consider the following system
    $$\dot{x}=\epsilon F x +F_1(t)x+F_2(t)$$
    where $ x \in \mathbb{R}^{n\times n}$, $F \in \mathbb{R}^{n\times n}$
    is Hurwitz, $\epsilon > 0$, $F_1(t) \in \mathbb{R}^{n\times n}$
    and $F_2^x(t) \in \mathbb{R}^{n}$ are bounded and continuous for all $t \geq  t_0$. We have (i) if
    $F_1(t)$, $F_2(t) \rightarrow 0 $ as  $t \rightarrow  \infty$ (exponentially), then for any $x(t_0)$ and
    any $\epsilon > 0$, $x(t) \rightarrow 0$ as $ t \rightarrow \infty$ (exponentially); 
    (ii) if $F_1(t) = 0$, $F_2(t)$ decays to zero exponentially at the rate of $\alpha$,
     and $\epsilon \geq \frac{\alpha }{\alpha_F} $, where $\alpha_F=\min({\rm Re}(\sigma(-F)))$, 
     then, for any $ x(t_0)$, $x(t) \rightarrow 0 $ as $ t \rightarrow \infty$
    exponentially at the rate of $\alpha$.
\end{Lemma}


\section{System Setup and Problem Formulation}\label{section3}
In this section, a new problem called resilient containment of
MAS group against composite attacks is proposed. First, the model of the MAS group is formulated and some basic definitions of the composite attacks is given as follows.
{\color{black}
\subsection{MAS group Model}
In the framework of containment control, we consider a group of $N+M$ MAS, which can be divided into two groups:

1) $M$ leaders are the roots of directed graph $\mathcal{G}$, which have no neighbors. Define the index set of leaders as $\mathcal{L}= \textbf{I}[N+1,N+M]$.

2) $N$ followers who coordinate with their neighbors to achieve the containment set
of the above leader. Define the index set of followers
as $\mathcal{F} = \textbf{I}[1, N]$.

Similar to many existing works \cite{zuo2020,zuo2021,haghshenas2015}, we consider the following dynamics of leader :
\begin{equation}\label{EQ1}
\begin{cases}
\dot{x}_k=S x_k,\\
y_k=R x_k,
\end{cases}
\end{equation}
where $x_k\in \mathbb{R}^q$ and $y_k\in \mathbb{R}^p$ are system states and reference output of the $k$th leader, respectively.

The dynamics of each follower is given by 
\begin{equation}\label{EQ2}
  \begin{cases}
  \dot{x}_i=A_i x_i + B_i \bar{u}_i,\\
  y_i=C_i x_i,
  \end{cases}
\end{equation}
where $x_i\in \mathbb{R}^{ni}$, $u_i\in \mathbb{R}^{mi}$ and $y_i\in \mathbb{R}^p$ are 
system state, control input and output of the $i$th follower, respectively. For convenience, the notation $'(t)'$ can be omitted in the following discussion. We make the following assumptions about the agents and the communication network.

{\color{blue}
\begin{assumption}\label{assumption 1}
There exists a directed path from at least one leader to each follower $i\in\mathcal{F}$ in the graph $\mathcal{G}$.
\end{assumption}

\begin{assumption}\label{assumption 2}
  The real parts of the eigenvalues of $S$ are non-negative.
\end{assumption}




\begin{assumption}\label{assumption 3}
 The pair $(A_i, B_i)$ is  stabilizable and  the pair $(A_i, C_i)$ is detectable for $i \in \mathcal{F}$.
\end{assumption}



\begin{assumption}\label{assumption 4}
For all $\lambda \in \sigma(S)$, where $\sigma(S)$ represents the spectrum of $S$,
  \begin{equation}
   {\rm rank} \left[
      \begin{array}{c|c}
     A_i-\lambda I_{n_i} &  B_i  \\ \hline
    C_i  & 0   \\
      \end{array}
      \right]=n_i+p,
      i \in \mathcal{F}.
  \end{equation}
\end{assumption}

}

{\color{blue}
\begin{remark}
Assumptions 1,3 and 4 are standard in the classic output regulation problem, Assumption 2 is avoid the trivial case of stable $S$.  The modes associated with the eigenvalues of $S$ with negative real parts will exponentially decay to zero, and do not affect the asymptotic behavior of the closed-loop system. $\hfill \hfill \square $
\end{remark}
}



\begin{myDef}\label{def41}
  For the $i$th follower, the system accomplishes containment if there exists series of $\alpha_{\cdot i}$,
   which satisfy $\sum _{k \in \mathcal{L}} \alpha_{k i} =1$ to ensure following equation hold:
   \begin{equation}
     {\rm  lim}_{t\rightarrow \infty } (y_i(t)-\sum _{k\in \mathcal{L}} \alpha_{k i}y_k(t))=0,
   \end{equation}
   where $i \in \textbf{I}[1,N]$.
\end{myDef}

}


\subsection{ Attack Descriptions}
In this work, we consider the MASs consisting of cooperative agents with potential malicious attackers. As shown in the Fig. \ref{fig:figure0}, the attackers will use four kinds of attacks to compromise the containment performance of the MASs:

\begin{figure}[!]
  %\begin{minipage}[t]{1\linewidth}
  \centering
  \includegraphics[width=0.5\textwidth]{pic/TPs.png}
  \caption{ Resilient MASs against Composite Attacks: A double-layer framework.}
  \label{fig:figure0}
\end{figure}



1) DoS attacks (Denial-of-Services Attacks): The communication graphs among agents (both in TL and CPL)  denied by attacker;

2) AAs (Actuation Attacks):the motor inputs infiltrated by attacker to destroy the input signal of the agent;

3) FDI attacks (Fault injection Attacks):  the exchanging information among agents distort by attack;

4) CAs (Camoufalge Attacks): the attacks mislead its downstream agents by disguising as a leader.

To resist the composite attack, we introduced a new layer named TL with the same communication topology as CPL, which yet have greater security and less physical meanings. Therefore, this TL could effectively against most of the above attacks. With the introduction of TL,the resilient control scheme can be decoupled to defend against DoS attacks on TL and defend against   potential unbounded AAs on CPL. The following two small subsections give the definitions and essential constraints for the DoS attacks and AAs, respectively.

1) DoS attacks: DoS attack refers to a type of attack where an adversary presents some or all components of a control system. It can affect the measurement and control channels simultaneously, resulting in the loss of data availability. Suppose that attackers can attack the communication network in a varing active period. Then it has to stop the attack activity and shift to a sleep period to reserve energy for the next attacks. Assume that there exists a $l \in \mathbb{N}$, define $\{t_l \}_{l \in \mathbb{N}}$ and $\{\Delta_* \}_{l \in \mathbb{N}}$  as the start time and the duration time of the $l$th attack sequence of DoS attacks, that is, the $l$th DoS attack time-interval is $A_l = [t_l, t_l +\Delta_* )$ with $t_{l+1} > t_l +\Delta_* $ for all $l \in \mathbb{N}$. Therefore, for all $t\geq \tau \in \mathbb{R}$, the set of time instants where the communication network is under DoS attacks are represent by
\begin{equation}
    \Xi_A(\tau,t) = \cup A_l \cap [\tau, t],l\in \mathbb{N},
\end{equation}
and the set of time instants where the communication network is allowed are 
\begin{equation}
    \Xi_N(\tau,t) = [\tau,t] / \Xi_A(\tau,t).
\end{equation}

\begin{myDef} [{Attack Frequency \cite{feng2017}}]
For any $\delta_2 > \delta_1 \geq t_0$, let $N_a(t_1,t_2)$ represent the number of DoS attacks in $[t_1,t_2)$. Therefore, $F_a(t_1,t_2)= \frac{N_a(t_1,t_2)}{t_2 - t_1}$ is defined as the attack frequency at $[t_1,t_2)$ for all $t_2 > t_1 \geq t_0$.
\end{myDef}

\begin{myDef} [{Attack Duration \cite{feng2017}}]\label{Def3}
For any $t_2 > t_1 \geq t_0$, let $T_a(t_1,t_2)$ represent the total time interval of DoS attack on multi-agent systems during  $[t_1,t_2)$. The attack duration over $[t_1,t_2)$ is defined as: there exist constants $\tau_a > 1$ and $T_0 > 0$ such that
  \begin{equation}
    T_a(t_1,t_2) \leq T_0 + \frac{t_2-t_1}{\tau_a}. 
  \end{equation}
\end{myDef}

2) Unbounded Actuation Attacks:
For each follower, the system input is under unknown actuator fault, which is described as
\begin{equation}
    \bar{u}_i=u_i+\chi_i, \forall i  \in \mathcal{F},
\end{equation}
where $\chi_i$ denotes the unknown actuator fault caused in actuator channels. That is, the ture values of $\bar{u}_i$ and $\chi_i$ are unknown and we can only measure the damaged control input information $\bar{u}_i$.


\begin{assumption}\label{assumption 5}
  The actuator attack $\chi_i$ is unbounded and its derivative $\dot{\chi}_i$ is bounded by $\bar{d}$.
\end{assumption}

{\color{black}
\begin{remark}
In contrast to the works \cite{deng2021} and \cite{chen2019}, which only consider the bounded AAs, this work tackles with unbounded AAs under the Assumption \ref{assumption 5}. In the case when the derivative of the attack signal is unbounded, that is, the attack signal increases at a extreme speed, the MAS can reject the signal by removing the excessively large value, which can be easily detected.
$\hfill \hfill \square$
\end{remark}

}





\subsection{ Problem Formulation}


%{\color{red}
Under the condition that there exists no attack, the local neighboring relative output containment information is represented as:
\begin{equation}\label{EQ xi}
    \xi_i = \sum_{j\in \mathcal{F}} a_{ij}(y_j -y_i) +\sum_{k \in \mathcal{L}} g_{ik}(y_k - y_i),
\end{equation}
where $a_{ij}$ is the weight of edge $(v_i,v_j)$ in the graph $\mathcal{G}_f$, and $g_{ik}$ is  the  weight  of  the  path  from  $i$th  leader  to  $k$th  follower.

The global form of (\ref{EQ xi}) is written as 
\begin{equation}
    \xi = - \sum_{k \in \mathcal{L}}(\Psi_k \otimes I_p)(y -  \underline{y}_k),
\end{equation}
where $\Psi_k = (\frac{1}{m} L_f + G_{ik})$ with $L_f$ is the Laplacian matrix of communication digraph $\mathcal{G}_f$ and $G_{ik}={\rm diag}(g_{ik})$, $\xi = [\xi_1^{\mathrm{ T}},\xi_2^{\mathrm{T}},\dots,\xi_n^{\mathrm{T}}]^{\mathrm{T}}$, $y=[y_1^{\mathrm{T}},y_2^{\mathrm{T}},\dots,y_n^{\mathrm{T}}]^{\mathrm{T}}$, $\underline{y}_k = (l_n \otimes y_k)$.

\begin{Lemma}[\cite{haghshenas2015}] \label{Lemma3}
    Suppose Assumption 1 holds. then the matrices $\Psi_k$ and $\bar{\Psi}_L= \sum_{k \in \mathcal{L}} \Psi_k$ are positive-definite and non-singular. Therefore, both $(\Psi_k)^{-1}$ and $(\bar{\Psi}_L)^{-1}= (\sum_{k \in \mathcal{L}} \Psi_k)^{-1} $ are non-singular.
\end{Lemma}


 From Haghshenas et al. \cite{haghshenas2015} and Zuo et al. \cite{zuo2019},  the global output containment error can be written as:
\begin{equation}
e= y - (\sum_{r\in \mathcal{L} }(\Psi_r \otimes I_p))^{-1} \sum_{k \in \mathcal{L} } (\Psi_k \otimes I_p) \underline{y}_k,
\end{equation}
where $e=[e_i^{\mathrm{T}},e_2^{\mathrm{T}},\dots,e_n^{\mathrm{T}}]^{\mathrm{T}}$ and $\bar{\xi} = -\sum_{k \in \mathcal{L}}(\Psi_k \otimes I_p )e$.


\begin{Lemma}
    Define that $\Theta={\rm diag}(v)$ where $v=(\bar{\Psi}_L)^{-1} l_N$. Under Assumption 1 and Lemma \ref{Lemma3}, the matrix $\Omega=\Theta \bar{\Psi}_L +\bar{\Psi}_L^{\textbf{T}} \Theta >0$ is a positive diagonal matrix. 
\end{Lemma}


\begin{Lemma}[{\cite[Lemma 1]{zuo2019}}]
    Under Assumption 1, the output containment control objective in (\ref{def41}) is achieved if ${\rm lim}_{t \rightarrow \infty} e = 0$.
\end{Lemma}
\begin{myDef}[\cite{khalil2002nonlinear}]
 The signal $x(t)$ is said to be UUB with the ultimate bound $b$, if there exist positive constants $b$ and $c$, independent of $t_0 \geq 0$, and for every $a \in (0, c)$, there is $T_1 = T_1(a, b) \geq 0$, independent of $t_0$, such that
 \begin{equation}
     ||x(t)||\leq a \Rightarrow ||x(t)|| \leq b,\forall t \geq t_0+T_1.
 \end{equation}
\end{myDef}



Consider the composite attacks in  Subsection III-B, the local neighboring relative output containment information is rewritten as
\begin{equation}\label{EQ xi2}
    \bar{\xi}_i = \sum_{j\in \mathcal{F}} d_{ij}(\bar{y}_{i,j} -\bar{y}_{i,i}) +\sum_{k \in \mathcal{L}} d_{ik}(\bar{y}_{i,k} - \bar{y}_{i,i})+\sum_{l\in {\mathcal{C}}} c_{il}(y_l-\bar{y}_{i,i}),
\end{equation}
where $d_{ij}(t)$ and $d_{ik}(t)$ are the edge weights which influence by the DoS attacks. For the denied communication link, $d_{ij}(t)=0$ and $d_{ik}(t)=0$; for the normal communication link, $d_{ij}(t)=a_{ij}$ and $d_{ik}(t)=g_{ik}$. And $\bar{y}_{i,j}$ (or $\bar{y}_{i,i}$) represent the information flow of $y_j$ to the $i$th follower, compromised by the FDI attacks. $\mathcal{C}$ is the index set of all camouflage attackers. $c_{il}$ present the edge weight for $l$th camouflage attacks to the $i$th follower. From equation (\ref{EQ xi2}), it can see that the attackers could disrupt the communication between agents and distort the actuation inputs of all followers.  



Next, we introduce the attack-resilient containment control problem.

\noindent \textbf{Problem ACMCA} (Attack-resilient Containment control of MASs
against Composite Attacks): The resilient containment control problem is to design the input $u_i$ in (3) for each follower, such that $e$ in (15) is UUB under the composite attacks, i.e., the trajectories of each follower converges into a point in the dynamic convex hull spanned by trajectories of multiple leaders.


%}


































\section{Main Results}


In this section, a double-layer resilient control scheme is used to solve the \textbf{Problem ACMCA}. First, a distributed virtual resilient TL, which can resist the FDI attacks, CAs, and AAs naturally, is proposed to against DoS attacks. Since the dynamics of leader does not be known for each followers, we consider the distributed observers to estimate the dynamics of leaders for all agents under the effect of DoS attacks. Next, distributed estimators is proposed to estimate the containment trajectory of followers  under DoS attacker. Then, we  use the estimation of leaders dynamics in Theorem 1 to get the solution of output regulator equation. Finally, a adaptive controllers are proposed to resist unbounded AAs in CPL.


\subsection{Distributed Observers to Estimate Leaders Dynamics}
 In this section, we design distributed observers  to estimate the dynamics of leaders which only used the neighbor information.
  

 To facilitate the analysis, define the leader dynamics in (2) as follows:
  \begin{equation}
      \Upsilon =[S;R]\in \mathbb{R} ^{(p+q)\times q}
  \end{equation}
and its estimations can be defined as follows:
 \begin{align} 
     &\hat{\Upsilon } _{i}(t)=[\hat{S}_{i}(t);\hat{R}_{i}(t)]\in \mathbb{R} ^{(p+q)\times q},
  \end{align} 
where $\hat{\Upsilon } _{i}(t)$ will be updated by (\ref{EQ15}), and it converge to $\Upsilon$ exponentially.




{\color{blue}
\begin{myTheo}\label{Theorem 1}
    Consider a group of $M$ leaders and $N$ followers with dynamic in (\ref{EQ1}) and (\ref{EQ2}). Suppose that Assumptions (\ref{assumption 1}) holds. The problem of the leader unknown dynamics under the DoS attack is solved by the following dynamic estimates 
$\hat{\Upsilon } _{i}(t)$ :
\begin{equation}\label{EQ15}
   \dot{\hat{\Upsilon }} _{i}(t)=   \mu_1 (\sum_{j \in \mathcal{F}} d_{ij}(\hat{\Upsilon } _{j}(t)-\hat{\Upsilon } _{i}(t)) + \sum_{k  \in \mathcal L}d_{ik}(\Upsilon-\hat{\Upsilon } _{i}(t))),i=\textbf{I}[1,N],
\end{equation}
where $\mu_1 > \frac{\sigma_{\rm max}(S)}{ \lambda_{\rm min}(\Omega \Theta^{-1} ) )(1-\frac{1}{\tau_a})}$ is the estimator gain with $\bar{\Psi}_{L}=\sum_{k\in \mathcal{L}}\Psi_k $.
\end{myTheo}

%{\color{blue}
\textbf{Proof.} 
Form the (\ref{EQ15}), it can seen that We only used the  relative neighborhood information to estimate leader dynamics, therefore,  the leader observer will suffer the influence of DoS attacks.

\textbf{Step 1:}
Let
$\tilde{\Upsilon}_{i}(t)=\Upsilon-\hat{\Upsilon}_{i}  (t)$,
then
\begin{equation}\label{EQ17}
    \dot{\tilde{\Upsilon}} _{i} (t)
=\dot{\Upsilon}-\dot{\hat{\Upsilon}}_{i} (t) 
=-\mu_1 (\sum_{j = 1}^{N} d_{ij}(\hat{\Upsilon } _{j}(t)-\hat{\Upsilon } _{i}(t)) +\sum_{k\in \mathcal{L}}d_{ik}(\Upsilon-\hat{\Upsilon } _{i}(t))).
\end{equation}
the global estimation dynamics error  of system (\ref{EQ15}) can be written as
\begin{equation}\label{EQ 18}
    \dot{\tilde{\Upsilon}}  (t) = -  \mu_1  (\sum_{k\in \mathcal{L}}\Psi_k^D \otimes I_{p+q}\tilde{\Upsilon} (t)) =-  \mu_1 (\bar{\Psi}_{L}^D \otimes I_{p+q}) \tilde{\Upsilon} (t), t \geq t_0.
\end{equation}
where $\bar{\Psi}^D_{L}=\sum_{k\in \mathcal{L}}\Psi_k^D $, $\Psi_k^D (t) = \begin{cases}
 0,t \in \Xi_A, \\ \Psi_k,t \in \Xi_N,
\end{cases}$, and $\tilde{\Upsilon} (t)= [\tilde{\Upsilon}_{1}(t),\tilde{\Upsilon}_{1}(t),\dots,\tilde{\Upsilon}_{N}(t)]$. 
Form (\ref{EQ 18}), we have ${\rm vec }(\dot{\tilde{\Upsilon}}) =-\mu_1(I_{q}\otimes \bar{\Psi}^D_{L} \otimes I_{p+q} ){\rm vec }(\tilde{\Upsilon}) $.
Consider the following Lyapunov function
\begin{equation}
    V_1(t)={\rm vec}(\tilde{\Upsilon})^{\textbf{T}} (\Theta \otimes \bar{P}_1) {\rm vec}(\tilde{\Upsilon}),
\end{equation}

The derivative of $V_1(t)$ can be written as follows
\begin{equation}
    \begin{aligned}
    \dot{V}_1 &= {\rm vec}(\tilde{\Upsilon})^{\textbf{T}}( -\mu_1 ((\bar{\Psi}^D_{L})^{\textbf{T}} \Theta + \Theta \bar{\Psi}^D_{L} )\otimes \bar{P}_1) {\rm vec}(\tilde{\Upsilon}) \\
    &\leq  -\mu_1 \lambda_{\rm min}(\Omega^D \Theta^{-1} ) V_1,
    \end{aligned}
\end{equation}
where $ \Omega^D=(\bar{\Psi}^D_{L})^{\textbf{T}} \Theta + \Theta \bar{\Psi}^D_{L} $.
then, we have
\begin{equation}
    \begin{aligned}
    &V_1(t)\leq e^{-\mu_1 \lambda_{\rm min}(\Omega^D \Theta^{-1} )(t-t_0)} V_1(t_0) \\
    &{\rm vec} (\tilde{\Upsilon}) \leq \frac{V_1(t_0)}{\lambda_{\rm min}(\Theta \otimes \bar{P}_1)} e^{-\mu_1 \lambda_{\rm min}(\Omega^D \Theta^{-1} ) (t-t_0)} \\
    \end{aligned}
\end{equation}
from Definition \ref{Def3}, one has 
\begin{equation}
\begin{aligned}
  \bar{\Psi}_{L}^D(t-t_0) &= \bar{\Psi}_{L}\left\lvert \Xi_N(t_0,t)\right\rvert \\
  &=\bar{\Psi}_{L}|t-t_0-\Xi_A(t_0,t)|\\
  & \geq \bar{\Psi}_{L}(t-t_0-(T_0+ \frac{t-t_0}{\tau_a}))\\
  &\geq \bar{\Psi}_{L}((1-\frac{1}{\tau_a})(t-t_0)+T_0) \\
  &\geq (1-\frac{1}{\tau_a}) \bar{\Psi}_{L} (t-t_0)  , t\geq t_0,
\end{aligned}
\end{equation}
then, it easy obtain that

Therefore, we can conclude that $\lambda_{\rm min}(\Omega^D \Theta^{-1} ) (t-t_0) \geq (1-\frac{1}{\tau_a}) \lambda_{\rm min}(\Omega \Theta^{-1} ) (t-t_0) $
\begin{equation}
    {\rm vec} (\tilde{\Upsilon}) \leq c_{\Upsilon} e^{-\alpha_{\Psi}(t-t_0)}, 
\end{equation}
where $c_{\Upsilon}= \frac{V_1(t_0)}{\lambda_{\rm min}(\Theta \otimes \bar{P}_1)} $, and $\alpha_{\Psi}= \mu_1 (1-\frac{1}{\tau_a}) \lambda_{\rm min}(\Omega \Theta^{-1} ) )  $, which implies that ${\rm vec}(\tilde{\Upsilon})$ and $\tilde{\Upsilon}$ converge to zero exponentially at the rate of $\alpha_{\Psi}$ at least. Moreover, by $\alpha_{\Psi}>\sigma_{\rm max}(S)$, $\tilde{S}_i x_k$ converges to zero exponentially for $i\in \mathcal{F},k\in \mathcal{L}$. 
 $\hfill \hfill \blacksquare $
}
{\color{blue}
\begin{remark}
In practice, there exist kinds of noises and attacks among agents, so, we  reconstruct the leaders dynamic for all followers by the observers (\ref{EQ15}). The estimator gain $\mu_1$ is designed to guarantee that the convergence rate of the estimated value $\tilde{S}$ is greater than the divergence rate of the leader state. Moreover, it guarantees the possible of convergence of the containment error. 
And compared with the existing work \cite{chen2019}, which  only used the observer in the normal communication for consensus, this work consider the heterogeneous output containment with the DoS attack. 
$\hfill \hfill \square $
\end{remark}

}
%}







\subsection{ Distributed Resilient State Estimator Design}
Based on the estimated matrix $\hat{S}_i$ for each agent, a distributed resilient estimator is proposed in this section to estimate the state $x_k$ of the leaders in the presence of DoS attacks.
Consider the following  distributed virtual resilient layer: 
\begin{equation}\label{equation 200}
  \dot{z}_i=\hat{S}_i z_i -\mu_2 G (\sum _{j \in \mathcal{F} }d_{ij}(z_i-z_j)+\sum_{k \in \mathcal{L}}d_{ik}(z_i-x_k)),
\end{equation}
%{\color{blue}
where $z_i$ is the local state of the virtual layer, $\mu_2  >  0$ is the coupling gain that can be choose by the user, and $G$ is the observer gain to be determined later. 
%}
The global state of virtual resilient layer   can be written as
\begin{equation}
  \dot{z}= \hat{S}_b z-\mu_2 (I_N \otimes G) (\sum_{k \in \mathcal{L}}(\Psi_k^D \otimes I_p )(z-\underline{x}_k)), 
\end{equation}
where $\hat{S}_b={\rm blkdiag}(\hat{S}_i) $, $z=[z_1,z_2,\dots,z_N]$, $\underline{x}_k=l_n \otimes x_k$ and $\mu_2  >  0$ is the estimator gain designed in Theorem \ref{Theorem 2}.

Define the global virtual resilient layer state estimation error
\begin{equation}
\begin{aligned}
  \tilde{z} 
  &=z-(\sum_{r \in \mathcal{L}}(\Psi_r \otimes I_p ))^{-1} \sum_{k \in \mathcal{L}}(\Psi_k\otimes I_p ) \underline{x}_k,\\
\end{aligned}
\end{equation}
then, for the normal communication, we have
%
%{\color{red}
\begin{equation}\label{EQ26}
    \begin{aligned}
      \dot{\tilde{z}}
       &=\hat{S}_bz-\mu_2 (I_N \otimes G) (\sum_{k \in \mathcal{L}}(\Psi_k \otimes I_p )(z-\underline{x}_k))-
  (\sum_{r \in \mathcal{L}}(\Psi_r \otimes I_p ))^{-1}\sum_{k \in \mathcal{L}}(\Psi_k \otimes I_p ) (I_n \otimes S) \underline{x}_k \\
      &=\hat{S}_bz- (I_n \otimes S)z+ (I_n \otimes S)z 
  -(I_n \otimes S)(\sum_{r \in \mathcal{L}}(\Psi_r \otimes I_p ))^{-1} \sum_{k \in \mathcal{L}}(\Psi_k \otimes I_p )  \underline{x}_k  +M \\
  & -\mu_2 (I_N \otimes G) \sum_{k \in \mathcal{L}}(\Psi_k \otimes I_p )(z-(\sum_{rr \in \mathcal{L}}(\Psi_{rr} \otimes I_p ))^{-1} (\sum_{kk \in \mathcal{L}}(\Psi_{kk} \otimes I_p )\underline{x}_{kk}+
  (\sum_{rr \in \mathcal{L}}(\Psi_{rr} \otimes I_p )^{-1} (\sum_{kk \in \mathcal{L}}(\Psi_{kk} \otimes I_p )\underline{x}_{kk}-\underline{x}_k)) \\
  &=\tilde{S}_bz+(I_n \otimes S) \tilde{z}-\mu_2 (I_N \otimes G)\sum_{k \in \mathcal{L}}(\Psi_k \otimes I_p ) \tilde{z} -
  \mu_2 (I_N \otimes G)(\sum_{kk \in \mathcal{L}}(\Psi_{kk} \otimes I_p ) \underline{x}_{kk}- \sum_{k \in \mathcal{L}}(\Psi_k \otimes I_p )\underline{x}_k) +M\\
  & =(I_n \otimes S) \tilde{z}-\mu_2 (I_N \otimes G)\sum_{k \in \mathcal{L}}(\Psi_k \otimes I_p ) \tilde{z}+ \tilde{S}_b \tilde{z} + F_2^x(t),
    \end{aligned}
\end{equation}
where  $F_2^x(t)=\tilde{S}_b(\sum_{r \in \mathcal{L}}(\Psi_r \otimes I_p ))^{-1} \sum_{k \in \mathcal{L}}(\Psi_k \otimes I_p ) \underline{x}_k +M$ and
$M = (\sum_{r \in \mathcal{L}}(\Psi_r \otimes I_p ))^{-1}\sum_{k \in \mathcal{L}}(\Psi_k \otimes I_p ) (I_n \otimes S) \underline{x}_k-
(I_n \otimes S) (\sum_{r \in \mathcal{L}}(\Psi_r \otimes I_p ))^{-1}\sum_{k \in \mathcal{L}}(\Psi_k \otimes I_p ) \underline{x}_k$ and $\tilde{S}_b={\rm blkdiag}(\tilde{S}_i)$ for $i\in \mathcal{F}$.

For the denied communication, we have 

\begin{equation}\label{EQ27}
    \begin{aligned}
      \dot{\tilde{z}}
       &=\hat{S}_bz-
  (\sum_{r \in \mathcal{L}}(\Psi_r \otimes I_p ))^{-1}\sum_{k \in \mathcal{L}}(\Psi_k \otimes I_p ) (I_n \otimes S) \underline{x}_k \\
      &=\hat{S}_bz- (I_n \otimes S)z+ (I_n \otimes S)z 
  -(I_n \otimes S)(\sum_{r \in \mathcal{L}}(\Psi_r \otimes I_p ))^{-1} \sum_{k \in \mathcal{L}}(\Psi_k \otimes I_p )  \underline{x}_k  +M \\
  &=\tilde{S}_bz+(I_n \otimes S) \tilde{z} +M\\
  & =(I_n \otimes S) \tilde{z}+ \tilde{S}_b \tilde{z} + F_2^x(t),
    \end{aligned}
\end{equation}
So, we can conclude (\ref{EQ26}) and (\ref{EQ27}) that
\begin{equation}\label{EQ32}
    \dot{\tilde{z}} = \begin{cases}
     \hat{S}_b \tilde{z}-\mu_2 (I_N \otimes G)\sum_{k \in \mathcal{L}}(\Psi_k \otimes I_p ) \tilde{z} + F_2^x(t), t \in \Xi_N(t_0,t) \\
     \hat{S}_b \tilde{z} + F_2^x(t), t \in \Xi_A(t_0,t).
    \end{cases}
\end{equation}

{\color{blue}

\begin{Lemma}
Under Lemma 3 and the Kronecker product property $(P \otimes Q)(Y \otimes Z) =(PY)\otimes(QZ) $, it easy to show that 
$(\sum_{r \in \mathcal{L}}(\Psi_r \otimes I_p ))^{-1}\sum_{k \in \mathcal{L}}(\Psi_k \otimes I_p ) (I_n \otimes S) \underline{x}_k=
(I_n \otimes S) (\sum_{r \in \mathcal{L}}(\Psi_r \otimes I_p ))^{-1}\sum_{k \in \mathcal{L}}(\Psi_k \otimes I_p ) \underline{x}_k$. 

\textbf{Proof:}
Let 
\begin{equation}
   M= \sum_{k \in \mathcal{L}} M_k= \sum_{k \in \mathcal{L}} ((\sum_{r \in \mathcal{L}}(\Psi_r \otimes I_p ))^{-1} (\Psi_k \otimes I_p ) (I_n \otimes S)
-(I_n \otimes S) (\sum_{r \in \mathcal{L}}(\Psi_r \otimes I_p ))^{-1}(\Psi_k \otimes I_p )) \underline{x}_k.
\end{equation}


By the Kronecker product property $(P \otimes Q)(Y \otimes Z) =(PY)\otimes(QZ) $, we can obtain that 
\begin{equation}
\begin{aligned}
  & (I_N \otimes S)(\sum_{r \in \mathcal{L}} \Psi_r \otimes I_p)^{-1} (\Psi_k \otimes I_p) \\
   &= (I_N \otimes S)((\sum_{r \in \mathcal{L}} \Psi_r)^{-1} \Psi_k) \otimes I_p) \\
   &=(I_N \times (\sum_{r \in \mathcal{L}} \Psi_r)^{-1} \Psi_k))\otimes(S \times I_p) \\
   &= (\sum_{r \in \mathcal{L}} \Psi_r \otimes I_p)^{-1} (\Psi_k \otimes I_p)  (I_N \otimes S).\\
\end{aligned}
\end{equation}


 We can show that $M_k=0$ and obtain that 
 \begin{equation}
     M=\sum_{k \in \mathcal{L}}M_k =0.
 \end{equation}
 This Proof is completed.
$\hfill \hfill \blacksquare $
\end{Lemma}
}



%}



Then $F_2^x(t)=\tilde{S}_b(\sum_{r \in \mathcal{L}}(\Psi_r \otimes I_p ))^{-1} \sum_{k \in \mathcal{L}}(\Psi_k \otimes I_p ) \underline{x}_k$ and ${\rm  lim}_{t \rightarrow \infty}F_2^x(t)=0$ is exponentially at the same rate of $\tilde{S}_b$.


{\color{black}





{\color{blue}

\begin{myTheo}\label{Theorem 2}
    Consider the MASs \ref{EQ1}-\ref{EQ2} suffered from DoS attacks, which satisfy Assumption 1, 
    and the DoS attack satisfying Definition 2 and Definition 3.  There exist scalars 
    $\tilde{\alpha}_1>0$, $\tilde{\alpha}_2>0$ and $\mu_2>0$ and  positive symmetric define matrices $\bar{P}_2 >0$ such that:
    \begin{align}
    \bar{P}_2 S+S^{\textbf{T}}\bar{P}_2 - \bar{P}_2^{\textbf{T}} \mu_2^2 \bar{P}_2 + \tilde{\alpha}_1 \bar{P}_2 =0 \\
    \bar{P}_2 S+S^{\textbf{T}}\bar{P}_2  - \tilde{\alpha}_2 \bar{P}_2 \leq 0 \\
    \frac{1}{\tau_a} < \frac{\alpha_1}{\alpha_1+\alpha_2} \label{EQ36}
    \end{align}
%where  $\bar{\Psi}_{L}=\sum_{k \in \mathcal{L}}(\Psi_k ) $,
    where  $\alpha_1=\tilde{\alpha}_1-k_1 ||\Theta \otimes \bar{P}_2||$, $\alpha_2=\tilde{\alpha}_2+k_1 ||\Theta \otimes \bar{P}_2||$, $k_1>0$, and $G=\mu_2 \lambda_{\rm max}(\Omega^{-1} \Theta) \bar{P}_2$.
    Then it can be guaranteed that the estimation error $\tilde{z}$ exponentially converges to zero under DoS attacks.
\end{myTheo}

\textbf{Proof.} 
Nest, we will prove that the virtual layer achieve containment under the DoS attacks.
For clarity, we first redefine the set $\Xi_A[t_0, \infty)$ as $\Xi_A[t_0, \infty)= \bigcup_{k=0,1,2,\dots} [t_{2k+1},t_{2k+2})$, where $t_{2k+1}$ and $t_{2k+2}$ indicate the time instants that attacks start and end, respectively.
Then, the set $\Xi_N[t_0,\infty)$ can be redefined as $\Xi_N[t_0, \infty)= [t_0,t_1)\bigcup_{k=1,2,\dots} [t_{2k},t_{2k+1})$.

Consider the following Lyapunov function candidate:
\begin{equation}
   V_2(t)=\tilde{z}^{\textbf{T}} (\Theta \otimes \bar{P}_2) \tilde{z}.
\end{equation}



when the communication is  normal, computing the time-derivative of $V_1(t)$ yields that:
\begin{equation}
\begin{aligned}
\dot{V}_2
&=\tilde{z}^{\textbf{T}}(\Theta \otimes (\bar{P}_2 S + S^{\textbf{T}}\bar{P}_2))\tilde{z} - \tilde{z}^{\textbf{T}}
((\bar{\Psi}^{\textbf{T}}_L \Theta  +\Theta  \bar{\Psi}_L) \otimes \lambda_{\rm max}(\Omega^{-1} \Theta) \mu_2^2 \bar{P}_2^2 )\tilde{z} \\
&+\tilde{z}^{\textbf{T}} (\Theta \otimes (\bar{P}_2 \tilde{S}_i + \tilde{S}_i^{\textbf{T}}\bar{P}_2) ) \tilde{z}  +2\tilde{z}^{\textbf{T}} (\Theta \otimes \bar{P}_2) F_2^x \\
& \leq \tilde{z}^{\textbf{T}} (\Theta \otimes (\bar{P}_2 S + S^{\textbf{T}}\bar{P}_2) - \Theta \otimes \mu_2^2 \bar{P}_2^2)\tilde{z} \\
& +\tilde{z}^{\textbf{T}} ((\Theta \otimes \bar{P}_2) \tilde{S}_b + \tilde{S}_b^{\textbf{T}} (\Theta \otimes \bar{P}_2)) \tilde{z} +2\tilde{z}^{\textbf{T}} (\Theta \otimes \bar{P}_2) F_2^x
\end{aligned}
\end{equation}
By Young's inequality, there exist scalars $k_1>0$ yield that
\begin{align} \label{EQ39}
    2\tilde{z}^{\textbf{T}}(\Theta \otimes \bar{P}_2) F_2^x \leq k_1 \tilde{z}^{\textbf{T}} (\Theta \otimes \bar{P}_2)^2 \tilde{z} + \frac{1}{k_1} ||F_2^x||^2 
\end{align}
then 
\begin{equation}
    \dot{V}_2\leq -\tilde{\alpha}_1 V_2 + 2  ||\tilde{S}_b|| V_2 +k_1 ||\Theta \otimes \bar{P}_2||V_2 +\frac{1}{k_1} ||F_2^x||^2
\end{equation}
with $\tilde{S}_b$ and $F_2^x$ converge to $0$ exponentially, we can obtain that
\begin{equation} \label{EQ45}
    \dot{V}_2\leq - (\alpha_1+ a_s e^{-b_s t})V_2 + \varphi(t)
\end{equation}
where $a_s>0$, $b_s>0$, and $\varphi(t)=a_f e^{-b_f}$ with $a_f>0$, and $b_f>0$. 

When the communication is denied, the time-derivative of $V_1(t)$ yields that

\begin{equation}\label{EQ46}
\begin{aligned}
\dot{V}_2
&=\tilde{z}^{\textbf{T}}(I_N \otimes (\bar{P}_2 S + S^{\textbf{T}}\bar{P}_2))\tilde{z} 
+ \tilde{z}^{\textbf{T}}
((\Theta \otimes \bar{P}_2) \tilde{S}_b + \tilde{S}_b^{\textbf{T}} (\Theta \otimes \bar{P}_2)) \tilde{z} +2\tilde{z}^{\textbf{T}} (\Theta \otimes \bar{P}_2) F_2^x \\
& \leq \tilde{\alpha}_2 V_2 + \frac{||\tilde{S}_b||^2}{k_1 \sigma_{\rm min}(\Theta \bar{P}_2) }  V_2 +k_1 ||\Theta \otimes \bar{P}_2||V_2 +\frac{1}{k_1} ||F_2^x||^2 \\
 & \leq (\alpha_2+ a_s e^{-b_s t}) V_2 + \varphi(t) \\
\end{aligned}
\end{equation}
Solved the inequality (\ref{EQ45}) and (\ref{EQ46}), we can obtain the following inequality
\begin{equation} \label{EQ47}
   V(t)\leq \begin{cases}
    e^{\int_{t_{2k}}^{t}-\alpha_1+ a_s e^{-b_s t} \, d\tau}V_2(t_{2k}) + \int_{t_{2k}}^{t} e^{\int_{\tau}^{t}-\alpha_1+ a_s e^{-b_s s} \, ds} \varphi(\tau) \, d\tau , t\in [t_{2k},t_{2k+1})\\
    e^{\int_{t_{2k+1}}^{t}\alpha_2+ a_s e^{-b_s t} \, d\tau} V_2(t_{2k+1}) + \int_{t_{2k+1}}^{t} e^{\int_{\tau}^{t}\alpha_2+ a_s e^{-b_s s} \, ds} \varphi(\tau) \, d\tau , t\in [t_{2k+1},t_{2k+2}) .   
   \end{cases}
\end{equation}
 
let $\alpha=\begin{cases}
 -\alpha_1, t\in[t_{2k},t_{2k+1}) \\
 \alpha_2,t \in[t_{2k+1},t_{2k+2})\\
\end{cases}$, then we could reconstruct the (\ref{EQ47}) as follows
\begin{equation}\label{EQ46}
    v(t) \leq  e^{\int_{t_0}^{t}\alpha + a_s e^{-b_s t} \, d\tau}V_2(t_{0}) + \int_{t_{0}}^{t} e^{\int_{\tau}^{t}-\alpha_1+ a_s e^{-b_s s} \, ds} \varphi(\tau) \, d\tau .
\end{equation}

For $t\in \Xi_{N}[t_0,t)$, it yields from (\ref{EQ46}) that
\begin{equation}\label{EQ48}
\begin{aligned}
   V_2(t) &\leq e^{\int_{t_0}^{t}+ a_s e^{-b_s t}\, d\tau} e^{\int_{t_0}^{t}\alpha  \, d\tau}V_2(t_{0}) + \int_{t_{0}}^{t}          a_f e^{-b_f+ \int_{\tau}^{t} a_s e^{-b_s s} \, ds }  e^{\int_{\tau}^{t}\alpha \, ds}  \, d\tau \\
    &\leq e^{-\frac{a_s}{b_s}(e^{-b_s t} -e^{-b_s t_0})}  e^{-\alpha_1|\Xi_N(t_0,t)|+\alpha_2|\Xi_A(t_0,t)|}V_1(t_0)+\int_{t_0}^{t} a_f e^{-b_f \tau-\frac{a_s}{b_s}(e^{-b_s t} -e^{-b_s \tau}} e^{-\alpha_1|\Xi_N(\tau,t)|+\alpha_2|\Xi_A(\tau,t)|} \, d \tau .
\end{aligned}
\end{equation}
Similarly, it can be easily conclude that (\ref{EQ48}) holds for $t\in [t_{2k+1},t_{2k+2})$, too. With the Definition (\ref{Def3}), we have
\begin{equation}\label{EQ49}
\begin{aligned}
    -\alpha_1|\Xi_N(t_0,t)+\alpha_2|\Xi_A(t_0,t)| 
    &= -\alpha_1 (t-t_0-|\Xi_A(t_0,t)|) +\alpha_2|\Xi_A(t_0,t)| \\
    &\leq -\alpha_1 (t-t_0)+(\alpha_1+\alpha_2)(T_0+\frac{t-t_0}{\tau_a}) \\
    & \leq -\eta(t-t_0)+(\alpha_1+\alpha_2)T_0
\end{aligned}
\end{equation}
where $\eta =(\alpha_1  - \frac{\alpha_1+\alpha_2}{\tau} )$.
Substitute (\ref{EQ49}) into (\ref{EQ48}), we have 
\begin{equation}
    \begin{aligned}
   V_2(t)
    & \leq e^{-\frac{a_s}{b_s}(e^{-b_s t}-e^{-b_s t_0})+(\alpha_1+\alpha_2)(T_0)}V_1(t_0)e^{-\eta(t-t_0)} \\
    &+\frac{a_f}{-b_f-\eta}  e^{-\frac{a_s}{b_s}(e^{-b_s t}-e^{-b_s t_0})+(\alpha_1+\alpha_2)T_0} (e^{-b_f t}-e^{-\eta t+(-b_f +\eta)t_0}) \\
   & \leq c_1 e^{-\eta(t-t_0)}+c_2 e^{-b_f t}
    \end{aligned}
\end{equation}
with $c_1=e^{\frac{a_s}{b_s}e^{-b_s t_0}+(\alpha_1+\alpha_2)T_0}(V_1(t_0)-\frac{a_f}{-b_f+\eta}e^{-b_f t_0})$ and $c_2= \frac{a_f}{-b_f+\eta} e^{\frac{a_s}{b_s}e^{-b_s t_0}+(\alpha_1+\alpha_2)T_0}$.
From (\ref{EQ36}), we could get that $\eta>0$. Therefore, it obvious that $V_2(t)$ converge to $0$ exponentially.







{\color{blue}
\begin{remark}
Here, we have proof the $\tilde{z}$ converges to $0$ which is bounded. Then the inequality (\ref{EQ39}) can rewritten as $ 2\tilde{z}^{\textbf{T}}(\Theta \otimes \bar{P}_2) F_2^x \leq \varphi^*(t)$, which show that $k_1$ is not necessary and there exist $\alpha_1=\tilde{\alpha}_1$ and $\alpha_2=\tilde{\alpha}_2$.
$\hfill \hfill \square $
\end{remark}
}



\begin{remark}
Compare with the existing work \cite{deng2021} which the leader state  estimator start to work after finishing the leader dynamic estimation, this work does not need to wait for the leader state estimator to estimate an accurate state and the TL begin to work at the first time which is more applicable to the actual situation.  Moreover, Deng et al. \cite{deng2021} only dealt with the case of a single leader with constant trajectory, we consider the more general case of output containment with multiple  leaders. And different from the existing work \cite{yang2020} which can only obtain the bounded state tracking error under DoS attack, the state tracking error of TL in our work converges to zero exponentially which have more conservative result.
$\hfill \hfill \square $
\end{remark}













}
}











%}













\subsection{ Distributed Output Regulator Equation Solvers}

Since the leaders dynamics and information are unknown for some followers. Therefore,  the output regulator equation which need to know the global information as Zuo et al.  \cite{zuo2020} cannot be used here. With the estimated leader dynamics converge to the actual leader dynamics in Theorem 1, the output regulator of estimated leader dynamics can be solved by the following Theorem.

{\color{blue}
\begin{myTheo}
 suppose the Assumptions \ref{assumption 1} and \ref{assumption 4} hold, the estimated solutions $\hat{\Delta}_{i}$ to the output regulator equations in (\ref{EQ10}) are solved as follows:
\begin{align}
    &\dot{\hat{\Delta}}_{i} = - \mu_3 \hat{\Phi }^{\mathrm{T}}_i(\hat{\Phi }_i \hat{\Delta}_{i}-\hat{\mathcal{R}}_i) \label{EQ52}
\end{align}

%\begin{equation}
%   \dot{\hat{\Delta}}_{i} = - u \hat{\Phi }^{\mathrm{T}}_i(\hat{\Phi }_i \hat{\Delta}_{i}-\hat{\mathcal{R}}_i)
%\end{equation}

where $\hat{\Delta}_i={\rm vec}(\hat{Y}_i), \hat{\Phi}_i=(I_q \otimes  M_i-\hat{S}_i^{\mathrm{T}} \otimes N_i)$, $\hat{R}_i={\rm vec}(\mathcal{\hat{R}}^{\ast}_i)$;  $\hat{Y_{i}}=[\hat{\Pi}_{i}^{\mathrm{T}}$, $\hat{\Gamma}_{i}^{\mathrm{T}}]^{\mathrm{T}}$, $\hat{R}^{\ast}_i=[0,\hat{R}_i^{\mathrm{T}}]^{\mathrm{T}}$, $
M_i=\left[
  \begin{array}{cc}
  A_i &  B_i  \\
  C_i &  0   \\
  \end{array}
  \right]$, $N_i=
    \left[
  \begin{array}{cc}
   I_{n_i} & 0  \\
   0 &  0   \\
  \end{array}
  \right]$, and $\mu_3 > \frac{\sigma_{\rm max}(S)}{\lambda_{\rm min}(\Phi_i^{\mathrm{T}} \Phi_i ))}$ with $\Phi_i=(I_q \otimes  M_i-S^{\mathrm{T}} \otimes N_i)$.
\end{myTheo}
\textbf{Proof.}
In Theorem 1, we realize that leader dynamic estimation is time-varying, so the output regulator equation  influenced by leader dynamic estimation is also time-varying. Next, we begin proof that the estimated solution of regulator equation  Converges to the solution of standard  regulator equation exponentially.

\textbf{Step 1:}
First, we proof that $\hat{\Delta}_{i}$ converges to $\Delta_i$ at an exponential rate.

From Assumption \ref{assumption 4}, we can get that the following output regulator equation:
 \begin{equation}\label{EQ10}
     \begin{cases}
      A_i\Pi_i+B_i\Gamma_i=\Pi_i S \\
      C_i\Pi_i = R
     \end{cases}
 \end{equation}
have solution matrices  $\Pi_i$ and $\Gamma_i$ for $i=\textbf{I}[1,N]$.

Rewriting the standard output regulator equation (\ref{EQ10}) yields as follow:
\begin{equation}\label{EQ51}
 \left[
  \begin{array}{cc}
  A_i &  B_i  \\
  C_i &  0   \\
  \end{array}
  \right]
   \left[
    \begin{array}{cc}
   \Pi_i \\
   \Gamma_i \\
    \end{array}
    \right]
    I_q -
    \left[
  \begin{array}{cc}
   I_{n_i} & 0  \\
   0 &  0   \\
  \end{array}
  \right]
   \left[
    \begin{array}{cc}
\Pi_i\\
\Gamma_i \\
    \end{array}
    \right] S
    =
    \left[\begin{array}{cc}
         0  \\
         R
    \end{array}\right].
\end{equation}

Reformulating the equation (\ref{EQ51}) as follow:
\begin{equation}
    M_iY I_q -N_iY S=R_i^{\ast},
\end{equation}
where $M_i=\left[
  \begin{array}{cc}
  A_i &  B_i  \\
  C_i &  0   \\
  \end{array}
  \right],
  N_i=\left[
  \begin{array}{cc}
   I_{n_i} & 0  \\
   0 &  0   \\
  \end{array}
  \right],
  Y_i= \left[
    \begin{array}{cc}
\Pi_i\\
\Gamma_i \\
    \end{array}
    \right], $ and $R_i^{\ast}= \left[\begin{array}{cc}
         0  \\
         R
    \end{array}\right] $.  By Theorem 1.9 of Huang et al. \cite{huang2004}, the standard form of the linear equation in Equation (\ref{EQ51}) can be rewritten as:
\begin{equation}
    \Phi_i \Delta_i=\mathcal{R}_i,
\end{equation}
where $\Phi_i=(I_q \otimes  M_i-S^{\mathrm{T}} \otimes N_i)$, $\Delta_i={\rm vec}( \left[
    \begin{array}{cc}
\Pi_i\\
\Gamma_i \\
    \end{array}
    \right])$, 
    $\mathcal{R}_i={\rm vec}(\mathcal{R}^{\ast})$, $R^{\ast}=\left[
    \begin{array}{cc}
0\\
R \\
    \end{array}
    \right]$.
Noting that 
\begin{equation}
\begin{aligned}
\dot{\hat{\Delta}}_{i}
&= - \mu_3 \hat{\Phi }^{\mathrm{T}}_i(\hat{\Phi }_i \hat{\Delta}_{i}-\hat{\mathcal{R}}_i) \\
&=- \mu_3  \hat{\Phi}^{\mathrm{T}}_i \hat{\Phi}_i \hat{\Delta}_{i}+  \mu_3 \hat{\Phi}_i \hat{\mathcal{R}_i} \\
&=-  \mu_3\Phi_i^{\mathrm{T}} \Phi_i \hat{\Delta}_{i} + \mu_3 \Phi_i^{\mathrm{T}} \Phi_i \hat{\Delta}_{i} - \mu_3 \hat{\Phi}_i^{\mathrm{T}}  \hat{\Phi}_i \hat{\Delta}_{i} +  \mu_3 \hat{\Phi}_i^{\mathrm{T}} \hat{\mathcal{R}_i} - \mu_3 \Phi_i^{\mathrm{T}} \hat{\mathcal{R}_i} +  \mu_3 \Phi_i^{\mathrm{T}} \hat{\mathcal{R}_i} - \mu_3 \Phi_i^{\mathrm{T}} \mathcal{R}_i +  \mu_3 \Phi_i^{\mathrm{T}} \mathcal{R}_i \\
&=-  \mu_3\Phi_i^{\mathrm{T}} \Phi_i \hat{\Delta}_{i} +  \mu_3 (\Phi_i^{\mathrm{T}} \Phi_i - \hat{\Phi}_i^{\mathrm{T}} \hat{\Phi}_i) \hat{\Delta}_{i} +   \mu_3 (\hat{\Phi}_i^{\mathrm{T}} - \Phi_i^{\mathrm{T}}) \hat{\mathcal{R}_i} +  \mu_3 \Phi_i^{\mathrm{T}} (\hat{\mathcal{R}_i}-\mathcal{R}_i) +  \mu_3 \Phi_i^{\mathrm{T}} \mathcal{R}_i \\
&=- \mu_3 \Phi_i^{\mathrm{T}} \Phi_i \hat{\Delta}_{i} + \mu_3  \Phi_i^{\mathrm{T}} \mathcal{R}_i +d_i(t),\\
\end{aligned}
\end{equation}
where  $d_i(t)= - \mu_3 (\hat{\Phi}_i^{\mathrm{T}}\hat{\Phi}_i - \Phi_i^{\mathrm{T}}\Phi_i) \hat{\Delta}_{i} + \mu_3  \tilde{\Phi}_i^{\mathrm{T}} \hat{\mathcal{R}}_i +         \mu_3 \Phi_i^{\mathrm{T}} \tilde{\mathcal{R}}_i$ with $\tilde{\Phi}_i=\hat{\Phi}_i-\Phi_i=\tilde{S}_i^{\mathrm{T}} \otimes N_i$ and $\tilde{\mathcal{R}}={\rm vec}(\left[\begin{array}{cc}
    0  \\
    \tilde{R}_i
    \end{array}
    \right])$.
It obvious that $\lim _{t \rightarrow \infty}d_i(t) =0 $ exponentially at rate of $\alpha_{\Psi}$. 

Let $\tilde{\Delta}_{i}=\Delta_i - \hat{\Delta}_{i}$, The time derivative of  $\tilde{\Delta}_{i}$
can be computed via
\begin{equation}\label{EQ53}
\begin{aligned}
    \dot{\tilde{\Delta}}_{i}
    =- \mu_3 \Phi_i^{\mathrm{T}} \Phi_i \tilde{\Delta}_{i} -  \mu_3 \Phi_i^{\mathrm{T}} \Phi_i \Delta_i+  \mu_3\Phi_i^{\mathrm{T}} \mathcal{R}_i +d_i(t)
    =- \mu_3 \Phi_i^{\mathrm{T}} \Phi_i \tilde{\Delta}_{i} +d_i(t),
\end{aligned}
\end{equation}
solve the equation (\ref{EQ53}), we have 
\begin{equation}
    \tilde{\Delta}_i(t)=\tilde{\Delta}_i(t_0)e^{-\mu_3 \Phi_i^{\mathrm{T}} \Phi_i(t-t_0)}+\int_{t_0}^{t} d_i(\tau) e^{ \mu_3 \Phi_i^{\mathrm{T}} \Phi_i\tau} \, d\tau e^{- \mu_3 \Phi_i^{\mathrm{T}} \Phi_i (t-t_0)},
\end{equation}
 since $ \Phi_i^{\mathrm{T}} \Phi_i$ is  positive, and $d_i(t)$ converge to $0$ at rate of $\alpha_{\Psi}$, use Lemma \ref{Lemma 1} we can obtain that  $\lim _{t \rightarrow \infty }\tilde{\Delta}_{i}=0$ exponentially. Moreover, with $\mu_3 > \frac{\sigma_{\rm max}(S)}{\lambda_{\rm min}(\Phi_i^{\mathrm{T}} \Phi_i )}$, the exponential convergence rate of $\lim _{t \rightarrow \infty }\tilde{\Delta}_{i}=0$ is bigger than $\sigma_{\rm max}(S)$.
  $\hfill \hfill \blacksquare $
%
\begin{Lemma}\label{Lemma 5}%[\cite{chen2019}]
    The distributed leader dynamics observers in (\ref{EQ15}) ensure $ \dot{\hat{\Pi}}_i $ and $\dot{\hat{\Pi}}_i  x_k$   converge to zero exponentially.
\end{Lemma}
}

{\color{blue}
\begin{remark}
  The rate of convergence of $\dot{\hat{\Pi}}_i $ is at least as $\dot{\tilde{\Delta}}_i$, so, the exponential convergence rate of $\dot{\hat{\Pi}}_i $ is bigger than $\sigma_{\rm max}(S)$ by $\mu_3 > \frac{\sigma_{\rm max}(S)}{\lambda_{\rm min}(\Phi_i^{\mathrm{T}} \Phi_i )}$. Then, it obvious that $\dot{\hat{\Pi}}_i  x_k$ converges to zero exponentially. 
  %Compare with the existing work \cite{cai2017} that the gain $\mu_3 > \frac{\alpha_{\Psi}}{\lambda_{\rm min}(\Phi_i^{\mathrm{T}} \Phi_i )}$, our work have more conservative result.
$\hfill \hfill  \square $
\end{remark}
}





\subsection{Adaptive Distributed Resilient Controller Design}

%{\color{blue}

Define the following state tracking error :
 \begin{equation}
      \epsilon_i=x_i - \hat{\Pi}_i z_i\\ \label{EQ58}
 \end{equation}
then considering  the decentralized adaptive attack-resilient control protocols as follows:
\begin{align}
 & u_i=\hat{\Gamma}_i z_i +K_i \epsilon_i -\hat{\chi}_i  \\  
&\hat{\chi}_i=\frac{B_i^{\mathrm{T}}  P_i \epsilon_i}{\left\lVert \epsilon_i^{\mathrm{T}} P_i B_i \right\rVert +\omega} \hat{\rho_i} \\ \label{EQ65}
&\dot{\hat{\rho}}_i=\begin{cases}
 \left\lVert \epsilon_i^{\mathrm{T}} P_i B_i \right\rVert +2\omega, \mbox{if}  \left\lVert \epsilon_i^{\mathrm{T}} P_i B_i \right\rVert \geq \bar{d} \\
 \left\lVert \epsilon_i^{\mathrm{T}} P_i B_i \right\rVert +2\omega \frac{\left\lVert \epsilon_i^{\mathrm{T}} P_i B_i \right\rVert}{\bar{d}}, \mbox{ otherwise},
\end{cases}
\end{align}
where  $\hat{\chi}_i$ is an adaptive compensational signal, $\hat{\rho}_i$ is an adaptive updating parameter and the controller gain $K_i$ is designed as 
\begin{equation}\label{EQ0968}
    K_i=-R_i^{-1}B_i^{\mathrm{T}} P_i
\end{equation}
where $P_i$ is the solution to
\begin{equation}
    A_i^{\mathrm{T}} P_i + P_i A_i + Q_i -P_i B_i R_i^{-1} B_i^{\mathrm{T}} P_i =0.
\end{equation}
%


\begin{myTheo}
 Consider  heterogeneous MAS consisting of   $M$ leader (\ref{EQ1}) and $N$ followers (\ref{EQ2}) with unbounded faults. Under Assumptions \ref{assumption 1}-\ref{assumption 5}, the \textbf{Problem ACMCA} is solved by designing the leaders dynamics observer (\ref{EQ15}), the distributed state estimator (\ref{equation 200}), the estimated output  solution regulator  equation (\ref{EQ52}) and the  distributed controller consisting of (56)-(60).
\end{myTheo}
\textbf{Proof.}
First, we need to prove the state tracking error (\ref{EQ58}) is bounded.

The derivative of $\epsilon_i$ is presented as follows:
  \begin{equation}\label{EQ63}
  \begin{aligned}
      \dot{\epsilon}_i
      &= A_i x_i + B_i u_i +B_i \chi_i -\dot{\hat{\Pi}}_i  z_i - \hat{\Pi}_i(\hat{S}_i z_i - \mu_2 (\sum _{j \in \mathcal{F} }d_{ij}(z_i-z_j)+\sum_{k \in \mathcal{L}}d_{ik}(z_i-x_k)))\\
      &=(A_i+B_i K_i)\epsilon_i  - \dot{\hat{\Pi}}_i z_i +\mu_2 \hat{\Pi}_i  (\sum _{j \in \mathcal{F} }d_{ij}(z_i-z_j)+\sum_{k \in \mathcal{L}}d_{ik}(z_i-x_k))+B_i \tilde{\chi}_i,
  \end{aligned}
  \end{equation}
where $\tilde{\chi}_i=\chi_i-\hat{\chi}_i$.
 
 The global state tracking error of (\ref{EQ63}) is
 \begin{equation}
 \begin{aligned}
  \dot{\epsilon}
  &={\rm blkdiag}(A_i+B_iK_i)\epsilon -{\rm blkdiag}(\dot{\hat{\Pi}}_i)z + \mu_2 {\rm blkdiag}(\hat{\Pi}_i)(\sum_{k \in \mathcal{L}}(\Psi_k^D \otimes I_p )(z-\underline{x}_k))+{\rm blkdiag}(B_i)\tilde{\chi}
 \\  
  &={\rm blkdiag}(A_i+B_i K_i)\epsilon -\dot{\hat{\Pi}}_b(\tilde{z}+ (\sum_{r \in \mathcal{L}}(\Psi_r \otimes I_p ))^{-1} \sum_{k \in \mathcal{L}}(\Psi_k \otimes I_p ) \underline{x}_k) \\
    & + \mu_2 \hat{\Pi}_b\sum_{k \in \mathcal{L}}(\Psi_k^D \otimes I_p )\tilde{z} +      \mu_2 \hat{\Pi}_b\sum_{k \in \mathcal{L}}(\Psi_k^D \otimes I_p )((\sum_{r \in \mathcal{L}}(\Psi_r \otimes I_p ))^{-1} \sum_{k \in \mathcal{L}}(\Psi_k \otimes I_p ) \underline{x}_k) -  \underline{x}_k))        +        B_b \tilde{\chi} \\
   &={\rm blkdiag}(A_i+B_i K_i)\epsilon -\dot{\hat{\Pi}}_b(\tilde{z}+ (\sum_{r \in \mathcal{L}}(\Psi_r \otimes I_p ))^{-1} \sum_{k \in \mathcal{L}}(\Psi_k \otimes I_p ) \underline{x}_k) + \mu_2 \hat{\Pi}_b\sum_{k \in \mathcal{L}}(\Psi_k^D \otimes I_p )\tilde{z} +B_b \tilde{\chi}
,\\
\end{aligned}
 \end{equation}
 where $\dot{\hat{\Pi}}_b={\rm blkdiag}(\dot{\hat{\Pi}}_i)$, $\hat{\Pi}_b= {\rm blkdiag}(\hat{\Pi}_i)$ for $i=\textbf{I}[1,N]$ and $\epsilon=[\epsilon_1^{\mathrm{T}} \epsilon_2^{\mathrm{T}} \dots \epsilon_N^{\mathrm{T}}]^{\mathrm{T}}$, $\tilde{\chi}=[\tilde{\chi}_1^{\mathrm{T}} \tilde{\chi}_2^{\mathrm{T}} \dots \tilde{\chi}_N^{\mathrm{T}}]^{\mathrm{T}}$.
%
%


Consider the following Lyapunov function candidate:
\begin{equation} 
    V= \epsilon ^{\mathrm{T}} P_b \epsilon,
\end{equation}
where $P_b={\rm blkdiag}(P_i)$
and its time derivate is given as follows: 
\begin{equation}\label{EQ80}
\begin{aligned}
    \dot{V}
    &= 2\epsilon^{\mathrm{T}} P_b \dot{\epsilon}_i \\
    &=-\epsilon^{\mathrm{T}} {\rm blkdiag}(Q_i) \epsilon +2\epsilon^{\mathrm{T}} P_b (\dot{\hat{\Pi}}_b+  \mu_2 \hat{\Pi}_b\sum_{k \in \mathcal{L}}(\Psi_k^D \otimes I_p ) )\tilde{z} \\
    & + 2\epsilon^{\mathrm{T}} P_b \dot{\hat{\Pi}}_b(\sum_{r \in \mathcal{L}}(\Psi_r \otimes I_p ))^{-1} \sum_{k \in \mathcal{L}}(\Psi_k \otimes I_p ) \underline{x}_k + 2\epsilon^{\mathrm{T}} P_b B_b\tilde{\chi}\\
    &\leq -\sigma_{min}(Q_b) \left\lVert \epsilon\right\rVert^2  +2\left\lVert \epsilon^{\mathrm{T}}\right\rVert \left\lVert P_b\right\rVert  \left\lVert \dot{\hat{\Pi}}_b+  \mu_2 \hat{\Pi}_b\sum_{k \in \mathcal{L}}(\Psi_k^D \otimes I_p )  \right\rVert  \left\lVert \tilde{z}\right\rVert\\ 
 & +2\left\lVert \epsilon^{\mathrm{T}}\right\rVert \left\lVert P_b\right\rVert  ||\dot{\hat{\Pi}}_b 
  (\sum_{r \in \mathcal{L}}(\Psi_r \otimes I_p ))^{-1}\sum_{k \in \mathcal{L}}(\Psi_k \otimes I_p )  \underline{x}_k|| + 2\epsilon^{\mathrm{T}} P_b B_b\tilde{\chi}, \\
\end{aligned}
\end{equation}
where $Q_b={\rm blkdiag}(Q_i)$.
%
%
%
%

By $\dot{\hat{\Pi}}_i $ and $ \dot{\hat{\Pi}}_i x_k $ converge to zero exponentially in Lemma 6,  there exist positive constants $ V_{\Pi}$ and $ \alpha_{\Pi}$ such that
the following holds:
\begin{equation}\label{EQ81}
    \left\lVert  \dot{\hat{\Pi}}_b 
  (\sum_{r \in \mathcal{L}}(\Psi_r \otimes I_p ))^{-1}\sum_{k \in \mathcal{L}}(\Psi_k \otimes I_p ) \underline{x}_k\right\rVert
  \leq  V_{\Pi} \exp (-\alpha_{\Pi}),
\end{equation}
use Young's inequality, we have
\begin{equation}\label{EQ61}
\begin{aligned}
    & 2\left\lVert \epsilon^{\mathrm{T}}\right\rVert \left\lVert P_b\right\rVert   \left\lVert  \dot{\hat{\Pi}}_b 
  (\sum_{r \in \mathcal{L}}(\Psi_r \otimes I_p ))^{-1}\sum_{k \in \mathcal{L}}(\Psi_k \otimes I_p ) \underline{x}_k\right\rVert \\
  & \leq \left\lVert \epsilon\right\rVert^2 + \left\lVert P_b\right\rVert  ^2 \left\lVert \dot{\hat{\Pi}}_b 
  (\sum_{r \in \mathcal{L}}(\Psi_r \otimes I_p ))^{-1}\sum_{k \in \mathcal{L}}(\Psi_k \otimes I_p ) \underline{x}_k\right\rVert^2 \\
  & \leq (\frac{1}{4} \sigma_{min}(Q_b) - \frac{1}{2}\beta_{V1} )\left\lVert \epsilon_i\right\rVert ^2 +\frac{\left\lVert P\right\rVert  ^2}{ (\frac{1}{4} \sigma_{min}(Q_b) - \frac{1}{2}\beta_{V1} )} V_{\Pi}^2 \exp (-2\alpha_{\Pi})\\
  & \leq
  (\frac{1}{4} \sigma_{min}(Q_b) - \frac{1}{2}\beta_{V1} )\left\lVert \epsilon_i\right\rVert ^2 +\beta_{v21}e^{-2\alpha_{v1}t}.
\end{aligned}
\end{equation}
From Lemma \ref{Lemma 5}, we can obtain that $\dot{\hat{\Pi}}$ converge to 0 exponentially. By $\tilde{z}$ also converge to zero exponentially, we similarly obtain that
\begin{equation}\label{EQ62}
\begin{aligned}
   & 2\left\lVert \epsilon^{\mathrm{T}}\right\rVert \left\lVert P\right\rVert  \left\lVert \dot{\hat{\Pi}}_b+  \mu_2 \hat{\Pi}_b\sum_{k \in \mathcal{L}}(\Psi_k \otimes I_p )  \right\rVert  \left\lVert \tilde{z}\right\rVert \\
  & \leq (\frac{1}{4} \sigma_{min}(Q_b) - \frac{1}{2}\beta_{V1} )\left\lVert \epsilon\right\rVert ^2+\beta_{v22}e^{-2\alpha_{v2}t}.
\end{aligned}
\end{equation}

Next, noting that
\begin{equation}
\begin{aligned}
\epsilon_i^{\mathrm{T}} P_i  B_i\tilde{\chi}_i
        &= \epsilon_i^{\mathrm{T}} P_i  B_i \chi_i -\frac{\left\lVert \epsilon_i^{\mathrm{T}} P_i B_i \right\rVert^2}{ \left\lVert \epsilon_i^{\mathrm{T}} P_i B_i \right\rVert +\omega} \hat{\rho_i} \\
& \leq  \left\lVert \epsilon_i^{\mathrm{T}} P_i  B_i\right\rVert \left\lVert \chi_i \right\rVert - \frac{\left\lVert \tilde{y}_i^{\mathrm{T}} P_i C_i B_i \right\rVert^2  }{\left\lVert \epsilon_i^{\mathrm{T}} P_i B_i \right\rVert + \omega} \hat{\rho_i} \\
& = \frac{\left\lVert \epsilon_i^{\mathrm{T}} P_i B_i \right\rVert ^2 (  \left\lVert \chi_i \right\rVert-  \hat{\rho}_i)+ \left\lVert \epsilon_i^{\mathrm{T}} P_i B_i \right\rVert\left\lVert \chi_i \right\rVert \omega}{\left\lVert \epsilon_i^{\mathrm{T}} P_i B_i \right\rVert +\omega } \\
&=\frac{\left\lVert \epsilon_i^{\mathrm{T}} P_i B_i \right\rVert ^2 (\frac{\left\lVert \epsilon_i^{\mathrm{T}} P_i B_i \right\rVert +\omega}{\left\lVert \epsilon_i^{\mathrm{T}} P_i B_i \right\rVert}||\chi_i||-\hat{\rho}_i)}{\left\lVert \epsilon_i^{\mathrm{T}} P_i B_i \right\rVert +\omega}.
\end{aligned}
\end{equation}
Noting that $d\left\lVert \chi_i \right\rVert/dt$ is bounded,  so, if $\left\lVert \epsilon_i^{\mathrm{T}} P_i B_i \right\rVert\geq \bar{d} \geq \frac{d||\chi_i||}{d t}$, that is, $\frac{\bar{d}+\omega}{\bar{d}} \frac{d||\chi_i||}{dt} -\dot{\hat{\rho}} \leq \bar{d}+\omega-\dot{\hat{\rho}} \leq -\omega < 0$. Then, there exists $t_2 > 0$ such that for all $t \geq t_2$, we have 
\begin{equation}\label{EQ76}
     (\frac{\left\lVert \epsilon_i^{\mathrm{T}} P_i B_i \right\rVert +\omega}{\left\lVert \epsilon_i^{\mathrm{T}} P_i B_i \right\rVert}||\chi_i||-\hat{\rho}_i) \leq (\frac{\bar{d}+\omega}{\kappa}||\chi_i||-\hat{\rho}_i) <0.
\end{equation}
Thus, we can obtain that $\epsilon_i^{\mathrm{T}} P_i  B_i\tilde{\chi}_i <0$ and $\epsilon^{\mathrm{T}} P_b  B_b\tilde{\chi}$ over $t \in[t_2,\infty)$.

From (\ref{EQ81}) to (\ref{EQ76}), it yields
\begin{equation}\label{EQ70}
  \dot{V} \leq -(\beta_{V1}+\frac{1}{2} \sigma_{min}(Q_b) ) \epsilon^{\mathrm{T}} \epsilon +\beta_{V2}e^{\alpha_{V1}t}.
\end{equation}
Solving (\ref{EQ70}) yields the following:
\begin{equation}
\begin{aligned}
     V(t) 
    & \leq V(0) - \int_{0}^{t} (\beta_{V1}+\frac{1}{2} \sigma_{min}(Q_b) ) \epsilon^{\mathrm{T}}\epsilon \,d\tau + \int_{0}^{t} \beta_{V2}e^{\alpha_{V1}t} \,d\tau,\\
\end{aligned}
\end{equation}
then
\begin{equation}\label{EQ72}
    \epsilon^{\mathrm{T}} \epsilon \leq -\int_{0}^{t} \frac{1}{\sigma_{min}(P)} \beta_{V3} \epsilon^{\mathrm{T}}\epsilon \,d\tau + \bar{B}.
\end{equation}
where $\beta_{V3}= \beta_{V1}+\frac{1}{2} \sigma_{min}(Q_b) $ and $ \bar{B}=V(0)-\int_{0}^{t} \beta_{V2}e^{\alpha_{V1}t} \,d\tau $ are define as a bounded constant.
Recalling Bellman-Gronwall Lemma, (\ref{EQ72}) is rewritten as follows:

\begin{equation}
  \left\lVert \epsilon \right\rVert \leq \sqrt{\bar{B}} e^{-\frac{\beta_{V3}t}{2\sigma_{min}(P)} },
\end{equation}
conclude (73) and (77), we can that $\epsilon$ is bounded by $\bar{\epsilon}$, where $\bar{\epsilon}=[\bar{\epsilon}_1^{\mathrm{T}} \bar{\epsilon}_2^{\mathrm{T}} \dots \bar{\epsilon}_N^{\mathrm{T}}]^{\mathrm{T}}$ with $||\bar{\epsilon}_i||=\frac{\bar{d}}{||P_i B_i||}$.
%
%


%}
%{\color{red}







Hence, similar to Lemma 5, the global output synchronization error satisfied that
{\color{black}
\begin{equation}
\begin{aligned}
e &= y - (\sum_{r\in \mathcal{L} }(\Phi_r \otimes I_p))^{-1} \sum_{k \in \mathcal{L} } (\Psi_k \otimes I_p) \underline{y}_k \\
&=\bar{y}-(I_N \otimes R)(z-\tilde{z})\\
&= {\rm blkdiag}(C_i)x   -{\rm blkdiag}(C_i \hat{\Pi}_i)z +{\rm blkdiag}(C_i \hat{\Pi}_i)z -(I_N \otimes R)z +(I_N \otimes R) \tilde{z} \\
&=  {\rm blkdiag}(C_i)\epsilon  -{\rm blkdiag}(\tilde{R}_i)z +(I_N \otimes R) \tilde{z}.
\end{aligned}
\end{equation}
}
%{\color{blue}
Since $\epsilon$, $\tilde{R}_i$, $\tilde{z}$  converge to 0  exponentially which have been proofed before, it is obviously that the global output containment error $e$ is bounded by $\bar{e}$ with $\bar{e}=[\bar{e}_1^{\mathrm{T}} \bar{e}_2^{\mathrm{T}} \dots \bar{e}_i^{\mathrm{T}}]^{\mathrm{T}}$ and $\bar{e}_i=\frac{\bar{d} ||C_i||}{||\bar{P}_i B_i||}$ for $i= \textbf{I}[1,N]$. 
%}
The whole proof is completed.
$\hfill \hfill \blacksquare $

\begin{remark}
Compared with \cite{chen2019} which only solved the tracking problem with a single leader against bounded attacks, we consider more challenge work with multiple leaders output containment problem against unbounded attacks. Moreover, the boundary of the output containment error is given, that is, $ \frac{\bar{d} ||C_i||}{||\bar{P}_i B_i||}$. It's helpful for designers to pre-estimate of the controller.
$\hfill \hfill  \square $
\end{remark}








\section{Numerical Simulation}\label{SecSm}
\begin{figure}[!]
  %\begin{minipage}[t]{1\linewidth}
  \centering
  \includegraphics[width=0.2\textwidth]{pic/tp1.png}
  \caption{Topology graph.}
  \label{fig:figure1}
\end{figure}


In this section, we consider  a multi-group system consisting of seven agents (three leaders 5,6,7 and four followers 1,2,3,4) with the  corresponding graph show in Fig.2. We assume that $a_{ij}=1$ ($g_{ik}=1$) if there exist a path from node $j$ ($k$) to node $i$.
Assuming the dynamics of leaders are described by 
$$
S
=
 \left[
  \begin{array}{cc}
 0  & -2  \\
 1 &  0   \\ 
  \end{array}
  \right],R
  =
   \left[
    \begin{array}{cc}
   1 &  0    \\
   0  & 1  \\ 
    \end{array}
    \right],
$$
and the dynamics of follows are given by
$$
A_1
=
 \left[
  \begin{array}{cc}
 3  & -2  \\
 1 &  -2   \\ 
  \end{array}
  \right],B_1
  =
   \left[
    \begin{array}{cc}
   1.8 &  -1    \\
   2  & 3  \\ 
    \end{array}
    \right],C_1
  =
   \left[
    \begin{array}{cc}
   -0.5 &  1    \\
  2  & -1.5  \\ 
    \end{array}
    \right],
$$
$$
A_2
=
 \left[
  \begin{array}{cc}
 0.6  & -1  \\
 1 &  -2   \\ 
  \end{array}
  \right],B_2
  =
   \left[
    \begin{array}{cc}
   1 &  -2    \\
   1.9  & 4  \\ 
    \end{array}
    \right],C_2
  =
   \left[
    \begin{array}{cc}
   -0.5 &  1    \\
  1.5  & 1.4  \\ 
    \end{array}
    \right],
$$
$$
A_3=A_4
=
 \left[
  \begin{array}{ccc}
 0  &  1  &0\\
 0 &  0 & 1  \\ 
  0 &  0 & -2  \\ 
  \end{array}
  \right],B_3
  =B_4=
   \left[
    \begin{array}{ccc}
   6 &  0    \\
   0  & 1  \\ 
   1  & 0 \\ 
    \end{array}
    \right],C_3=C_4
  =
   \left[
    \begin{array}{ccc}
   0.5 &  -0.5  & 0.5  \\
  -0.5  & -0.5  &0.5\\ 
    \end{array}
    \right].
$$

The DoS attack periods are given as $[0.5+2k,1.92+2k)s$ for $k=0,1,2,\dots$, which satisfied Assumption 3. The MAS can againt the CAs and FDI attacks since the information transmitted on the CPL is not used in the hierarchal control scheme.
The AAs  are designed as $\chi_1=\chi_2=\chi_3=0.01 \times [2t, t]^{\mathrm{T}}$ and $\chi_4=-0.01\times[2t, t]^{\mathrm{T}}$. 

\begin{figure}[!]
  %\begin{minipage}[t]{1\linewidth}
  \centering
  \includegraphics[width=0.5\textwidth]{pic/Upsilon0420s0.eps}
  \caption{The estimations of leader dynamics in Theorem 1: the shadow areas denote time intervals against DoS attacks.}
  \label{fig:figure1}
\end{figure}

\begin{figure*}[htbp]
  \centering
  \subfigure[State estimation of TL $z_i(1)$]{
  \begin{minipage}[t]{0.4\textwidth}
  \centering
  \includegraphics[width=1\textwidth]{pic/TLzx0420s0.eps}
  \label{fig:figure2:1}
  %\caption{fig1}
  \end{minipage}%
  }%
  %\hspace{-0.2in}
  \subfigure[State estimation of TL $z_i(2)$]{
  \begin{minipage}[t]{0.4\textwidth}
  \centering
  \includegraphics[width=1\textwidth]{pic/TLzy0420s0.eps}
  \label{fig:figure2:2}
  %\caption{fig2}
  \end{minipage}%
  }%
  \centering
  \caption{Performance of the TL: the shadow areas denote time intervals against DoS attacks.}
  \label{fig:figure2}
  \end{figure*}



\begin{figure}[!]
  %\begin{minipage}[t]{1\linewidth}
  \centering
  \includegraphics[width=0.5\textwidth]{pic/Delta0420s0.eps}
  \caption{The estimated solution of the output regulating equation in Theorem 3: the shadow areas denote time intervals against DoS attacks.}
  \label{fig:figure3}
\end{figure}
  

\begin{figure}[!]
  %\begin{minipage}[t]{1\linewidth}
  \centering
  \includegraphics[width=0.5\textwidth]{pic/track0420s0.eps}
  \caption{ Output trajectories of the leaders and followers.}
  \label{fig:figure4}
\end{figure}




\begin{figure*}[htbp]
  \centering
  \subfigure[Output containment error $e_i(1)$]{
  \begin{minipage}[t]{0.4\textwidth}
  \centering
  \includegraphics[width=1\textwidth]{pic/ex0420s0.eps}
  \label{fig:figure5:1}
  %\caption{fig1}
  \end{minipage}%
  }%
  %\hspace{-0.2in}
  \subfigure[Output containment error $e_i(2)$]{
  \begin{minipage}[t]{0.4\textwidth}
  \centering
  \includegraphics[width=1\textwidth]{pic/ey0420s0.eps}
  \label{fig:figure5:2}
  %\caption{fig2}
  \end{minipage}%
  }%
  \centering
  \caption{The output containment errors: the blue shadow areas denote the UUB bound computed by $\frac{\bar{d}||C_i||}{||P_i||||B_i||}$ (choose $\bar{d}=0.1$ and  ${ \rm max}(e_i(1))= {\rm max}(e_i(2)) = 0.1808$ for $i\in \mathcal{F}$).}
  \label{fig:figure5}
  \end{figure*}
  



Chosen the gains in \ref{EQ15}, \ref{equation 200} and \ref{EQ52} as $\mu_1=6$, $\mu_2=0.5$ and $\mu_3=22$, respectively. It can be obtain by Theorem that $P2=[13.6669,-2.5130;-2.5130,17.5331]$ and $G=[5.2433,-0.9641;-0.9641,6.7265]$. By employing the above parameters, the trajectories of leaders dynamics estimation are shown in Fig \ref{fig:figure1}, the estimate of leaders dynamics $||\hat{\Upsilon}_i||$ against DoS attacks  remained stable after $3s$ with the value equal to the value of  leaders dynamics. The state estimation errors of TL are given in Fig. \ref{fig:figure2} which can be seen the performance of TL is stable under the DoS attacks.
The trajectories of  the solution errors of output regulating equations are given in Fig. \ref{fig:figure3}  which validate Theorem  3.
 And to solve the AAs on the CPL, we design the controller gain in (\ref{EQ0968}) as 
$$
K_1
=
 \left[
  \begin{array}{cc}
 -1.71  & 0.07  \\
 2.31  &  -1.12   \\ 
  \end{array}
  \right],K_2
  =
   \left[
    \begin{array}{cc}
   -0.62 &  -0.29   \\
   1.16  & -0.56  \\ 
    \end{array}
    \right],K_3=K_4
  =
   \left[
    \begin{array}{ccc}
   -1.00 &  -0.19  & -0.08  \\
  0.02  & -0.96  &-0.30\\ 
    \end{array}
    \right].
$$






Now, we can solve the ACMCA problem by Theorem 4.  The output trajectories of all the agents are showed in the Fig. \ref{fig:figure4}. It can be seen that followers eventually enter the convex hull formed by the leaders. The output containment errors of followers are shown in Fig. \ref{fig:figure5} which can be seen that the local output errors $e_i(t)$ stay within a small boundary.  These show that the  output  containment problem subject to composite attacks is solved.


% \begin{figure}[!htbp]
% %\begin{minipage}[t]{1\linewidth}
% \centering
% \includegraphics[width=0.6\textwidth]{4Ag.pdf}
% \caption{Time-varying directed communication topology among all agents}
% \label{fig:figure1}
% \end{figure}



%{\color{blue}
%\begin{figure}[htbp]
%\centering
%\subfigure[Performance of observer w.r.t. the leader]{
%\begin{minipage}[t]{0.475\textwidth}
%\centering
%\includegraphics[width=0.85\textwidth]{pic/pobs.eps}
%%\caption{fig1}
%\end{minipage}\label{fig:figure2:1}
%}
%%\hspace{-0.1in}
%\subfigure[Performance of observer w.r.t. the first leader]{
%\begin{minipage}[t]{0.475\textwidth}
%\centering
%\includegraphics[width=0.85\textwidth]{pic/vobs.eps}
%%\caption{fig2}
%\end{minipage}\label{fig:figure2:2}
%}\\%
%\centering
%\caption{Performance of two observers}
%\label{fig:figure2}
%\end{figure}














\section{Conclusion}
The distributed resilient output containment control of heterogeneous multi-agent systems subject to composite Attacks has been solved in this work. Inspired by the concept of hierarchical protocol, a TL, which is resilient to most attacks (including the CAs, FDI attacks, and AAs), is provided to decouples the defense strategy into two sub-problems. The first sub-problem is the defense against DoS attacks on the TL and the second sub-problem is the defense against unbounded AAs on the CPL.
For the first problem, we used the distributed observers of leaders dynamic to cope with the case that follower does not know the leaders dynamics. Then, the distributed  estimators is provided to predict the trajectories of followers. Moreover, solvers are used to calculate the estimated solutions of the regulator equations. For the second problem, the adaptive distributed control scheme is provided  to solve the unbounded AAs on the CPL. We have prove the UUB convergence of the above adaptive controller and given the error bound explicitly. Finally, the simulations illustrate the effectiveness  of these proposed methods.




 

\

% Proof: Consider the Lyapunov function candidate
% $$
% V_{1}=\frac{1}{2} \sum_{i=1}^{N} \xi_{i}^{T} P \xi_{i}+\sum_{i=1}^{N} \sum_{j=1, j \neq i}^{N} \frac{\left(c_{i j}-\alpha\right)^{2}}{8 \kappa_{i j}}
% $$
% where $\alpha$ is a positive constant that is to be determined later. Evidently, $V_{1}$ is positive definite. The time derivative of $V_{1}$ along the trajectory of (5) is given by
% $$
% \begin{aligned}
% \dot{V}_{1}=& \sum_{i=1}^{N} \xi_{i}^{T} P \dot{\xi}_{i}+\sum_{i=1}^{N} \sum_{j=1, j \neq i}^{N} \frac{c_{i j}-\alpha}{4 \kappa_{i j}} \dot{c}_{i j} \\
% =& \sum_{i=1}^{N} \xi_{i}^{T} P A \xi_{i}+\sum_{i=1}^{N} \xi_{i}^{T} P B K \sum_{j=1}^{N} c_{i j} a_{i j}\left(\tilde{x}_{i}-\tilde{x}_{j}\right) \\
% &+\sum_{i=1}^{N} \sum_{j=1, j \neq i}^{N} \frac{c_{i j}-\alpha}{4 \kappa_{i j}} \dot{c}_{i j}
% \end{aligned}
% $$
% Since $a_{i j}=a_{j i}$ and $c_{i j}(t)=c_{j i}(t)$, it can be easily verified that
% $$
% \begin{array}{rl}
% \sum_{i=1}^{N} \xi_{i}^{T} & P B K \sum_{j=1}^{N} c_{i j} a_{i j}\left(\tilde{x}_{i}-\tilde{x}_{j}\right) \\
% & =-\frac{1}{2} \sum_{i=1}^{N} \sum_{j=1}^{N} c_{i j} a_{i j}\left(\xi_{i}-\xi_{j}\right)^{T} \Gamma\left(\tilde{x}_{i}-\tilde{x}_{j}\right)
% \end{array}
% $$

\bibliography{PIDFR}
\end{document}\grid