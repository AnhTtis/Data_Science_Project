

\documentclass[letterpaper, 12pt, journal, twoside]{support/IEEEtran}
\usepackage[fleqn]{amsmath}
\usepackage{times}
\usepackage[pdftex]{graphicx}
\usepackage{subfigure}
\usepackage{amsmath,amssymb,amsopn,amstext,amsfonts}
\usepackage{cancel}
\usepackage[noadjust]{cite}
\usepackage{soul}
\usepackage{caption}
\captionsetup{font={small}}

\captionsetup[figure]{labelfont={},textfont={}}


\usepackage{balance}
\usepackage{color}
\usepackage{mathtools}
% \usepackage{algorithm}
% \usepackage{algorithmic}
\usepackage{bm}
%\newtheorem{theorem}{Theorem}
\usepackage{diagbox}
\usepackage{float}
\usepackage{epstopdf}
\usepackage{url}
\usepackage{multirow}
\usepackage{tikz}
\usepackage{subeqnarray}
\usepackage{cases}
\usepackage{booktabs}
\usepackage[linkcolor=black,citecolor=black,urlcolor=black,colorlinks=true]{hyperref}
\usepackage{algorithm}
\usepackage[noend]{algpseudocode}
\newtheorem{myTheo}{Theorem}
%\newtheorem{thm}{Theorem}[section] %如果不采用章节号做前缀,则不用[section]
\newtheorem{myDef}{Definition} %这句定义使得defn环境和thm共享编号
\newtheorem{lemma}{Lemma} %这句定义使得lem环境和thm共享编号
\newtheorem{myCollo}{Corollary}
\newtheorem{remark}{Remark}
%\newtheorem{lemma}{Lemma}
\newtheorem{myPro}{Proposition}
\newtheorem{assumption}{Assumption}
\newtheorem{example}{Example}
\soulregister\cite7
\soulregister\citep7
\soulregister\citet7
\soulregister\ref7
\soulregister\it7
\soulregister\pageref7

\bibliographystyle{support/IEEEtran}

\newcommand\px{\mathrel{/\mkern-5mu/}}  %平行
\newcommand{\ann}[1]{%
    \begin{tikzpicture}[remember picture, baseline=-0.75ex]%
        \node[coordinate] (inText) {};%
    \end{tikzpicture}%
    \marginpar{%
        \renewcommand{\baselinestretch}{1.0}%
        \begin{tikzpicture}[remember picture]%
            \definecolor{orange}{rgb}{1,0.5,0}%
            \draw node[fill=red!20,rounded corners,text width=\marginparwidth] (inNote){\footnotesize#1};%
    \end{tikzpicture}%
    }%
    \begin{tikzpicture}[remember picture, overlay]%
        \draw[draw = orange, thick]
            ([yshift=-0.2cm] inText)
                -| ([xshift=-0.2cm] inNote.west)
                -| (inNote.west);%
    \end{tikzpicture}%
}%

\graphicspath{{figures/}}
\DeclareGraphicsExtensions{.pdf,.png,.jpg,.eps}
\IEEEoverridecommandlockouts
%\overrideIEEEmargins

\title{\LARGE \bf Resilient Output Containment Control of Heterogeneous Multi-agent Systems against Composite Attacks: A Novel Digital Twin Approach}

%\title{Distributed Optimization in Prescribed-Time: Theory and Experiment}%
\author{
  \vskip 1em
  { 
  Xin Gong, \emph{Graduate Student Member, IEEE}, 
	Yukang Cui, \emph{Member, IEEE},
  Lingbo Cao
  }

  \thanks{
    This work was partially supported by the National Natural Science Foundation of China under Grant 61903258, 61973156, 61603180, Qatar National Research Fund NPRP12C-0814-190012. %(\emph{Corresponding author: Yukang Cui.}) %the National Natural Science Foundation of China under Grant 61903258

X. Gong is with the Department of Mechanical Engineering, The University of Hong Kong, Pokfulam Road, Hong Kong (e-mail: {\tt\small gongxin@connect.hku.hk}).


Y. Cui and T. Wang are with the College of Mechatronics and Control Engineering, Shenzhen University, Shenzhen, 518060, China (e-mail: {\tt\small cuiyukang,szuwtn@gmail.com}).


  
%J. He is with the Department of Mechanical Engineering, The University of Hong Kong, Pokfulam Road, Hong Kong (e-mail: {\tt\small esmehe@connect.hku.hk}). 

%X. Gong is with the Department of Mechanical Engineering, The University of Hong Kong, Pokfulam Road, Hong Kong, and also with the College of Mechatronics and Control Engineering, Shenzhen University, Shenzhen 518060, China. (e-mail: {\tt\small gongxin@connect.hku.hk}).
%China, and also
%with the Department of Mechanical Engineering, University of Hong Kong,
%Hong Kong
    
  }
%\thanks{$^{*}$ means the corresponding author.}
}

%\maketitle
%\author{}%\vspace{-0.0cm}
%%\thanks{This work was partially supported by.}% <-this % stops a space
%\thanks{$^{*}$These authors contribute equally and share the first authorship.}
%\thanks{$^{1}$Author is with the Group Robotics with Intelligent Planning (GRIP) Lab, Department of Mechanical Engineering, University of Hong Kong, Hong Kong,
%   {\tt\small gongxin@connect.hku.hk}}
%\thanks{Digital Object Identifier (DOI): see the top of this page.}
%\vspace{-0.5cm}}

% The note headers
%\markboth{Journal of \LaTeX\ Class Files,~Vol.~14, No.~8, August~2015}%
%{Shell \MakeLowercase{\textit{et al.}}: Bare Demo of IEEEtran.cls for IEEE Journals}
\markboth{IEEE Transactions on ...}{GONG \MakeLowercase{\textit{et al.}}: Resilient Output Containment Control of Heterogeneous MAS}%{He \MakeLowercase{\textit{et al.}}: Resilient Path Planning of UAVs against Covert Attacks on UWB Sensors}



\begin{document}
  \maketitle
  \begin{abstract}
    This brief deals with the distributed consensus observer design problem for high-order integrator multi-agent systems on directed graphs, which intends to estimate the leader state accurately in a prescribed time interval. A new kind of distributed prescribed-time observers (DPTO) on directed graphs is first formulated for the followers, which is implemented in a cascading manner. Then, the prescribed-time zero-error estimation performance of the above DPTO is guaranteed for both time-invariant and time-varying directed interaction topologies, based on strictly Lyapunov stability analysis and mathematical induction method. Finally, the practicability and validity of this new distributed observer are illustrated via a numerical simulation example.
\end{abstract}
\begin{IEEEkeywords}
  Consensus observer, Directed graphs, High-order Multi-agent systems, Prescribed-time stability
% Periodic positive systems, hyper-pyramid,
% reachable set estimation, S-procedure, state-feedback control.
%Formation-containment control,  high-order multi-agent systems,  observer-type protocols,  time-varying formation configuration
\end{IEEEkeywords}
\section{Introduction}
\IEEEPARstart{T}{he} last decade has witnessed substantial progresses contributed by various industries (see \cite{ xu2020distributed, liang2016leader, hua2017distributed, de2014controlling} and references therein) on distributed coordination of multi-agent systems (MAS). 
In this brief, we consider the leader-following scenarios \cite{liang2016leader} rather than leaderless ones \cite{hua2017distributed}, where only a small portion of followers is  pinned (directly connected) to the leader. As pointed out in \cite{de2014controlling}, the formation error (consensus error) suffers from biased measuring noise among robots. Moreover, the actuation faults may also propagate among the MAS networks, which pose a non-negligible threat to the collective control of MAS. An alternative and easily realized method to overcome the above harmful measuring noise and fault propagation is introducing a distributed observer for each agent and reassigning the control input according to the estimated information. Similar to the distributed consensus observer defined in \cite{zuo2019distributed}, herein, we intend to design a kind of distributed prescribed-time observers (DPTO), which could reconstruct the leader state for all followers, especially these unpinned ones. Notice that the above distributed observers are different from the traditional Luenberger observers \cite{9311845} focusing on reconstruction on the unmeasured states. Also, this kind of distributed observers \cite{de2014controlling,fu2017finite,fu2016fixed,zuo2017fixed2,zuo2019distributed} could eradicate the notorious communication loop problem \cite{khoo2014multi} encountered in traditional convey communication mechanism.




One of the key properties of a superior distributed observer is to achieve distributed estimation quickly with little constraint on the communication topologies and high accuracy. Unlike the distributed observer in \cite{de2014controlling} which only owns asymptotical convergence, Fu \emph{et. al.} first proposed a kind of distributed fixed-time observer for first-order \cite{fu2017finite} and second-order MAS \cite{fu2016fixed}, respectively. However, the above two works employ two fractional power feedback terms, that is, $\frac{p}{q}$ and  $2-\frac{p}{q}$ with $q>p>0$, which are too cumbersome to extend to high-order dynamics. Zuo \emph{et. al.} further employed single fractional power feedback ($\gamma>1$) in the design of distributed finite-time observer, which achieved the distributed fixed-time estimation w.r.t. the leader state on both undirected graphs \cite{zuo2017fixed2,zuo2019distributed} and directed ones \cite{zuo2019distributed}. As pointed in \cite{zuo2019distributed}, the proof based on the symmetry of Laplacian matrices is not applicable for directed topologies. Thus, it is not trivial to study the distributed finite-time observer on directed graphs. Notice that the fixed-time zero-error convergence is only guaranteed on undirected topologies \cite[Theorem 1]{zuo2019distributed}, while only fixed-time attractiveness of an error domain is proven in  \cite[Theorem 2]{zuo2019distributed}. Thus, it remains a big challenge to design a distributed finite-time zero-error observer on general directed topologies. Moreover, the needed time interval to achieve distributed observation in the above works \cite{fu2017finite,fu2016fixed,zuo2017fixed2,zuo2019distributed} depends on the settings of initial conditions and the network algebraic connectivity \cite{wu2005algebraic}, which put barriers in the way of their applications.







Two inevitable yet challenging difficulties arise when designing distributed finite-time observers for MAS:

\begin{enumerate}
  \item How can all followers obtain finite-time zero-error convergence since the actual states of each pinned leader are only available to only a portion of followers, especially on a directed topology?
  \item How to regulate the consensus observation time arbitrarily despite the influences of the initial states of the MAS and network algebraic connectivity, particularly for high-order MAS on large directed networks?
\end{enumerate}




Fortunately, recently proposed prescribed-time protocols in \cite{wang2018prescribed, gong2020distributed2} shed light on the solution of the above two issues. Motivated by \cite{wang2018prescribed, gong2020distributed2}, we formulate a kind of DPTO for each follower, featured by a new hybrid constant and time-varying scaling function. This design is fundamentally different from the previous distributed finite-time observer in \cite{fu2017finite,fu2016fixed,zuo2017fixed2,zuo2019distributed} based fractional power feedback, which has the following main contributions and characteristics:



\begin{enumerate}
\item  A new kind of cascaded DPTO, based on a \textbf{hybrid constant and time-varying feedback}, is developed for the agents on directed graphs, which achieves the distributed accurate estimation on each order of leader state in a cascaded manner. 
\item \textbf{Zero-error convergence on time-invariant/varying directed graphs}: In contrast to the previous work \cite{zuo2019distributed} which only guarantees finite-time attractiveness of a fixed error bound, we manage to regulate the observation error on directed graphs into zero in a finite-time sense. It is further found that the feasibility of this DPTO can be extended to some time-varying directed graphs.
%The difficulties caused by the asymmetrical Laplacian matrix under the circumstance of single-way directed communication topology are circumvented in the frameworks of distributed prescribed-time fault-tolerant control.
\item \textbf{Prescribed-time convergence}: This DPTO could achieve distributed zero-error estimation in a predefined-time manner, whose needed time interval is independent of the initial states of all agents and network algebraic connectivity. Thus, the observer design procedure is much more easily-grasped for the new users than that in \cite[Theorem 2]{zuo2019distributed}.
\end{enumerate}



\noindent\textbf{Notations:}
In this brief, $\boldsymbol{1}_m$ (or $\boldsymbol{0}_m$) denotes a column vector of size $m$ filled with $1$ (respectively, 0). Denote the index set of sequential integers as $\textbf{I}[m,n]=\{m,m+1,\ldots~,n\}$ where $m<n$ are two natural numbers. Define the set of real numbers, positive real numbers and nonnegative real numbers as $\mathbb{R}$, $\mathbb{R}_{>0}$ and $\mathbb{R}_{\geq 0}$, respectively. ${\rm diag}({b})$ means a diagonal matrix whose diagonal elements equal to a given vector ${b}$.
For a given symmetric matrix $A\in \mathbb{R}^{n\times n}$, its spectrum can be sorted as: $\lambda_1(A)\leq\lambda_2(A) \leq\ldots \leq\lambda_n(A)$.%; moreover, $A>0$ means that $\lambda_1(A)>0$.
%For a time-varying function $x(t): \mathbb{R}_{\geq 0 }\mapsto \mathbb{R}$, denote that $\sup_{t\in [t_0, t_1]} x(t) $ and $\inf_{t\in [t_0, t_1]} x(t)$ as the upper bound and lower bound of $x(t)$ over the time interval $[t_0, t_1]$, respectively. Moreover, denote that $\|x(t)\|_{[t_0, t_1]} =\sup_{t\in [t_0, t_1]} \|x(t)\| $. Define that $L_{\infty}:=\{x(t)|x(t): \mathbb{R}_{\geq 0 }\mapsto \mathbb{R}^n,\ \|x(t)\|_{[t_0, t_1]}<\infty\}$. In the following sections, $x(t) \in L_{\infty}$, $t\in [t_0, t_1]$, represents that the variable $x$ is uniformly bounded over $[t_0, t_1]$.   %$A\textgreatereq 0$ (or $A\textgreater 0$) denotes that $A$ is a nonnegative matrix (positive matrix, respectively), which means all elements of $A$ are nonnegative (positive, respectively).
 %${\rm span}(x)$ denotes the span vector of a given vector $x=[p_1, p_2,\ldots~, p_n]^{\mathrm{T}}\in \mathbb{R}^n$.

\label{introduction}


\section{Preliminaries}\label{section2}

%\subsection{Notations}




{\color{blue}
\subsection{Graph Theory}
In this article, $M$ Leaders and $N$ followers are consisidered on a directed graph $\mathcal{G}$.
The sets of leaders and followers are defined as $\mathcal{L}$ and $\mathcal{F}$ , respectively.
The directed graph of followers  can be represented by the subgraph                              
$\mathcal{G}_f =(\mathcal{V}, \mathcal{E}, \boldsymbol{A} )$  with the node set 
$\mathcal{V}=\{ 1, 2, \ldots~ , N \}$, the edge set
$\mathcal{E} \subset \mathcal{V} \times \mathcal{V}=\{(v_j,\ v_i)\mid\ v_i,\ v_j \in \mathcal{V}\}$ 
, and the associated adjacency matrix $\boldsymbol{A}=[a_{ij}] \in \mathbb{R}^{N\times N} $ .
The weight of the edge $(v_j,\ v_i)$ is denoted by $a_{ij}$ with $a_{ij} \textgreater 0$ if $(v_j,\ v_i) \in \mathcal{E}$ 
otherwise $a_{ij} = 0$. The neighbor set of node $v_i$ is represented by $\mathcal{N}_{i}=\{v_{j}\in \mathcal{V}\mid (v_j,\ v_i)\in \mathcal{E} \}$. Define the Laplacian matrix as 
$L=\mathcal{D}-\mathcal{A}  \in \mathbb{R}^{N\times N}$ 
with $\mathcal{D}=diag(d_i) \in \mathbb{R}^{N\times N}$ where $d_i=\sum_{j \in \mathcal{F}} a_{ij}$ 
is the weight in-degree of node $v_i$.
The leader has no incoming edges and thus exhibits an autonomous behavior, 
while the follower has incoming edges and receives neighbor(including leader and follower) information 
directly. 
The interactions among the leaders and the followers are represented by 
$G_{ik}=$diag($g_{ik}$) $\in \mathbb{R}^{N\times N}$  while $g_{ik}$ is the weight of the path from 
$i$th leader to kth follower.  And $g_{ik} = 1$ If there is a direct path from $i$th leader 
to $k$th follower , $g_{ik} \textgreater 0$, otherwise $g_{ik} = 0$.
}





{\color{blue}
\subsection{System Description}
Consider a group of $M$ leaders with the following dynamics :
\begin{equation}\label{EQ1}
\begin{cases}
\dot{x}_k=S x_k,\\
y_k=R x_k,
\end{cases}
\end{equation}
where $x_k\in \mathbb{R}^q$ and $y_k\in \mathbb{R}^p$ are system states and reference output, respectively.

The dynamic of each follower is given by 
\begin{equation}\label{EQ2}
  \begin{cases}
  \dot{x}_i=A_i x_i + B_i u_i,\\
  y_i=C_i x_i,
  \end{cases}
\end{equation}
where $x_i\in \mathbb{R}^{n_i}$, $u_i\in \mathbb{R}^{m_i}$ and $y_i\in \mathbb{R}^p$ are 
system state, control input and output,respectively.

For each follower, the system input is under unknown actuator fault, which is described as
\begin{equation}
    \overline{u}_i=u_i+d_i.
\end{equation}
where $d_i$ denotes the unknown actuator fault caused in actuator channels. That is, the ture values of $\overline{u}_i$ and $d_i$ are unknown and we can only measure the damaged control input information $\overline{u}_i$.


\begin{assumption}
  The actuator attack $d_i$ is unbounded and its derivative $\dot{d}_i$ is bounded.
\end{assumption}

}

{\color{blue}
\subsection{ Denial-of-Service Attacks}
DOS attack refers to a type of attack where an adversary presents some or all components of a control system.It can affect the measurement and control channels simultaneously, resulting in the loss of data availability.Suppose that attackers can attack the communication network in a varing active period. Then it has to stop the attack activity and shift to a sleep period to reserve energy for the next attacks. Assume that there exists a $l \in \mathbb{N}$ , define $\{t_l \}_{l \in \mathbb{N}}$ and $\{s_l \}_{l \in \mathbb{N}}$  as the start time and the duration time of the $l$th attack sequence of DoS attacks, that is , the $l$th DoS attack time-interval is $A_l = [t_l , t_l + s_l )$ with $t_{l+1} \textgreater t_l +s_l $ for all $l \in \mathbb{N}$. Therefore, for all $t\geq \tau \in \mathbb{R}$, the sets of time instants where the communication network is under Dos attacks are represent by
\begin{equation}
    S_A(\tau,t) = \cup A_l \cap [\tau , t],l\in \mathbb{N},
\end{equation}
and the sets of time instants where the communication network is allowed are 
\begin{equation}
    S_N(\tau,t) = [\tau,t] S_A / (\tau,t).
\end{equation}

\begin{myDef} \cite{feng2017}  (Attack Frequency)
For any $\delta_2 \textgreater \delta_1 \geq t_0$, let $N_a(\delta_1,\delta_2)$ represent the number of DoS attacks in $[\delta_1,\delta_2)$. Therefore, $F_a(\delta_1,\delta_2)= \frac{N_a(\delta_1,\delta_2)}{\delta_2 - \delta_1}$ is defined as the attack frequency at $[\delta_1,\delta_2)$ for all $\delta_2 \textgreater \delta_1 \geq t_0$.
\end{myDef}

\begin{myDef} \cite{feng2017}  (Attack Duration)
For any $\delta_2 \textgreater \delta_1 \geq t_0$, let $T_a(\delta_1,\delta_2)$ represent the total time interval of DoS attack on multi-agent systems during  $[\delta_1,\delta_2)$. The attack duration over $[\delta_1,\delta_2)$ is defined as: there exist constants $\tau_G \textgreater 1$and $T_0 \textgreater 0$ such that
  \begin{equation}
    T_a(\delta_1,\delta_2) \leq T_0 + \frac{\delta_2-\delta_1}{\tau_G}. 
  \end{equation}
\end{myDef}
}






\subsection{ Problem Formulation}
\begin{myDef}\label{def41}
  For the $ith$ follower, the system accomplishes containment if there exists series of $\alpha_{\cdot i}$,
   which satisfy $\sum _{k \in \mathcal{L}} \alpha_{k i} =1$ to let following equation hold:
   \begin{equation}
     {\rm  lim}_{t\rightarrow \infty } (y_i(t)-\sum _{k\in \mathcal{L}} \alpha_{k i}y_k(t))=0
   \end{equation}
   where $i \in$ { 1, 2,...,N }.
\end{myDef}

{\color{blue}
Define the following local output formation containment error:
\begin{equation}\label{EQ xi}
    \xi_i = \sum_{j\in \mathcal{F}} a_{ij}(y_j -y_i) +\sum_{k \in \mathcal{L}} g_{ik}(y_k - y_i).
\end{equation}
The global form of (\ref{EQ xi}) is written as 
\begin{equation}
    \xi = - \sum_{k \in \mathcal{L}}(\Phi_k \otimes I_p)(y -  \underline{y}_k).
\end{equation}
where $\Phi_k = (\frac{1}{m} \mathcal{L } + G_{ik})$, $\xi = [\xi_1^T,\xi_2^T,\dots,\xi_n^T]^T$, $y=[y_1^T,y_2^T,\dots,y_n^T]^T$, and $\underline{y}_k = (l_n \otimes y_k)$.
\begin{lemma}
    Under Assumption 1, the matrixs $\Phi_k$ and $\sum_{k \in \mathcal{L}} $ are positive-definite and non-singular. Moreover, both $(\Phi_k)^{-1}$ and $(\sum_{k \in \mathcal{L}} \Phi_k)^{-1}$ are non-negative. 
\end{lemma}
Define the following global output containment error:
\begin{equation}
e= y - (\sum_{r\in \mathcal{L} }(\Phi_r \otimes I_p))^{-1} \sum_{k \in \mathcal{L} } (\Phi_k \otimes I_p) \underline{y}_k.
\end{equation}
where $e=[e_i^T,e_2^T,\dots,e_n^T]^T$ and $\xi = -\sum_{k \in \mathcal{L}}(\Phi \otimes I_p )e$.

\noindent \textbf{Problem DPTOHD}: The resilient containment control problem is to design the input $u_i$ in (1) for each follower , such that ${\rm  lim}_{t \rightarrow \infty}e= 0$ in (10) with the case of unknown leader dynamics and under unknown unbounded cyber-attacks and network DoS attackers, i.e., the trajectories of each follower converges into a point in the dynamic convex hull spanned by trajectories of multiple leaders.


}


{\color{black}
\begin{assumption}\label{assumption 2}
  The directed graph G contains a spanning tree with
the leader as its root, and the pinning gains $g_{ik}$ between leader and follower
are identical for all $k \in \mathcal{L}$ .
\end{assumption}
}
\begin{assumption}\label{assumption 3}
  $(A_i, B_i)$ is stabilizable and $(A_i,C_i)$ is
detectable for i = 1, 2,$\cdots$ ,N
\end{assumption}

\begin{assumption}\label{assumption 4}
  The real parts of the eigenvalues of S are
non-negative. rank(R) = q.
\end{assumption}

\begin{assumption}\label{assumption 5}
For all $\lambda \in \sigma(S)$, where $S$ represents the spectrum of $S$,
  \begin{equation}
    rank \left[
      \begin{array}{ccc}
     A_i-\lambda I_{n_i} &  B_i  \\
    C_i  & 0   \\
      \end{array}
      \right]=n_i+p.
  \end{equation}
\end{assumption}

\begin{assumption}
  The graph $\mathcal{G}$ is strongly connected.
\end{assumption}
































\section{Main Results}




\subsection{ Fully Distributed Observers to Estimate Leader States and Dynamics}
 In this section, we design distributed leader states and dynamics observers that are independent of the global graph topology and the global leader information.
  
  To facilitate the analysis, let the leader dynamics in (2) be rewritten as follows:
  \begin{equation}
      \Upsilon =[S;R]\in \mathbb{R} ^{(p+q)\times q}
  \end{equation}
and its estimations be splited in two parts as follows:
 \begin{equation}
      \hat{\Upsilon } _{0i}=[\hat{S}_{0i};\hat{R}_{0i}]\in \mathbb{R} ^{(p+q)\times q}
  \end{equation} 
  \begin{equation}
     \hat{\Upsilon } _{i}=[\hat{S}_{i};\hat{R}_{i}]\in \mathbb{R} ^{(p+q)\times q}
  \end{equation} 
where $\hat{\Upsilon } _{0i}$and $\hat{\Upsilon } _{i}$will be updated by ()and ()and converge to $\Upsilon$ at different rates.



\begin{lemma}[{\cite[Lemma 1]{cai2017 }}]\label{Lemma 1}
    Consider the following system
    $$\dot{x}=\epsilon F x +F_1(t)x+F_2(t)$$
    where $ x \in \mathbb{R}^{n\times n} , F \in \mathbb{R}^{n\times n}$
    is Hurwitz, $\epsilon \textgreater 0, F_1(t) \in \mathbb{R}^{n\times n}$
    and $F_2(t) \in \mathbb{R}^{n}$ are bounded and continuous for all $t \geq  t_0$. We have (i) if
    $F1(t), F2(t) \rightarrow 0 $ as  $t \rightarrow  \infty$ (exponentially), then for any $x(t_0)$ and
    any $\epsilon \textgreater 0, x(t) \rightarrow 0$ as $ t \rightarrow \infty$ (exponentially); 
    (ii) if $F_1(t) = 0$, $F_2(t)$ decays to zero exponentially at the rate of $\alpha$,
     and $\epsilon \geq \frac{\alpha }{\alpha_F} $, where $\alpha_F=\min(R(\sigma(-F)))$, 
     then, for any $ x(t0), x(t) \rightarrow 0 $ as $ t \rightarrow \infty$
    exponentially at the rate of $\alpha$.
\end{lemma}




\begin{myTheo}\label{Theorem 1}
    Suppose that Assumption 3 holds. Let the dynamic estimates $\hat{\Upsilon } _{0i}$ and 
$\hat{\Upsilon } _{i}$ in (\ref{EQ15}) and (\ref{EQ16}) be updated as follows:
\begin{equation}\label{EQ15}
   \dot{\hat{\Upsilon }} _{0i}=\sum_{j \in \mathcal{F}} a_{ij}(\hat{\Upsilon } _{0j}-\hat{\Upsilon } _{0i}) + \sum_{k  \in \mathcal L}g_{ik}(\Upsilon-\hat{\Upsilon } _{0i}) ,
\end{equation}
\begin{equation}\label{EQ16}
    \dot{\hat{\Upsilon }} _{i}=\left\lVert \hat{\Upsilon } _{0i} \right\rVert_F(\hat{\Upsilon } _{0i}-\hat{\Upsilon } _{i}) +\sum_{j\in \mathcal{F}} a_{ij}(\hat{\Upsilon } _{j}-\hat{\Upsilon } _{i}) +\sum_{k\in \mathcal{L}}g_{ik}(\Upsilon-\hat{\Upsilon } _{i}).
\end{equation}
Then, the estimated leader dynamics $\hat{\Upsilon}_{0i}$ and $\hat{\Upsilon}$, exponentially converge to the actual leader dynamics $\Upsilon_i$ for i = 1,...,N.
\end{myTheo}
\textbf{Proof.} 
\textbf{Step 1:}
Define
$\tilde{\Upsilon}_{0i}=\Upsilon-\hat{\Upsilon}_{0i}$,
then
\begin{equation}\label{EQ17}
    \dot{\tilde{\Upsilon}} _{0i}
=\dot{\Upsilon}-\dot{\Upsilon}_{0i}
=-\sum_{j = 1}^{N} a_{ij}(\hat{\Upsilon } _{0j}-\hat{\Upsilon } _{0i}) +\sum_{k\in \mathcal{L}}g_{ik}(\Upsilon-\hat{\Upsilon } _{0i}),
\end{equation}
the global of system (\ref{EQ16}) can be written as
\begin{equation}
    \dot{\tilde{\Upsilon}}_0 = -\sum_{k\in \mathcal{L}}\Phi_k \otimes I_{p+q}\tilde{\Upsilon}_0 ,
\end{equation}
or a more standard from as follower
\begin{equation}
    vec(\dot{\tilde{\Upsilon}}_0) = -I_{q}\otimes \Phi_k \otimes I_{p+q}vec(\tilde{\Upsilon}_0).
\end{equation}
Under Assumption \ref{assumption 2}, by Lemma 6 of \cite{haghshenas2015}, all the eigenvalues of $\Phi_k$ have positive real parts \footnote{remains to be proof}. 
Therefore, ${\rm  lim}_{t\rightarrow \infty }  vec(\tilde{\Upsilon}_0) = 0 $ exponentially, which implies, for i=1,$\dots$,N,${\rm  lim}_{t\rightarrow \infty }  \tilde{\Upsilon}_{0i} = 0 $ exponentially.let $ \alpha_G=min(\mathfrak{R} (\sigma(\Phi_k)))$,for all $t\geq 0,\left\lVert \tilde{\Upsilon}_{0i}(t) \right\rVert \leq \beta_G \left\lVert {\Upsilon}_{0i}(0) \right\rVert \exp{-\alpha_Gt}$ for some $\beta_G \textgreater 0$.

\textbf{Step 2:}
Define $\tilde{\Upsilon}_i$ as $\tilde{\Upsilon}_{i}=\Upsilon-\hat{\Upsilon}_{i}$ .
The inverse of $\tilde{\Upsilon}_i$ as the following
\begin{equation}
    \dot{\tilde{\Upsilon}} _{i}
=\dot{\Upsilon}-\dot{\Upsilon}_{i}
=-\left\lVert \hat{\Upsilon } _{0i} \right\rVert_F(\hat{\Upsilon } _{0i}-\hat{\Upsilon } _{i}) - \sum_{j = 1}^{N} a_{ij}(\hat{\Upsilon } _{j}-\hat{\Upsilon } _{i}) +\sum_{k\in \mathcal{L}}g_{ik}(\Upsilon-\hat{\Upsilon } _{i}).
\end{equation}
Then the global of $\tilde{\Upsilon}_i$ is
\begin{equation}
    \dot{\tilde{\Upsilon}} =-(\sum_{k\in \mathcal{L}}\Phi_k \otimes I_{p+q} + \left\lVert \hat{\Upsilon } _{0i} \right\rVert_F \otimes I_{p+q})\tilde{\Upsilon}
+\left\lVert \hat{\Upsilon } _{0i} \right\rVert_F \tilde{\Upsilon}_0.
\end{equation}
Due to all the eigenvalues of $\Phi_k$ have positive real parts and $\left\lVert \hat{\Upsilon } _{0i} \right\rVert_F \geq 0$, ${\rm  lim}_{t\rightarrow \infty }  \tilde{\Upsilon}_{0} = 0 $ exponentially,based on the Lemma \ref{Lemma 1} ,
${\rm  lim}_{t\rightarrow \infty }  \tilde{\Upsilon} = 0 $ at the rate of $\alpha_G$.








\subsection{ Distributed Resilient Estimator Design}
In this section, we propose a fully distributed observer containment control method to sulve Problem 1. For this purpose, we consider the following fully distributed virtual resilient layer
$z_i$ and $z_k$ is updated by the distributed resilient estimator.


\begin{equation}\label{equation 200}
  \dot{z}_i=\hat{S}_i z_i -\chi (\sum _{j \in \mathcal{F} }w_{ij}(z_i-z_j)+\sum_{k \in \mathcal{L}}w_{ik}(z_i-x_k)).
\end{equation}
The global of (\ref{equation 200}) can be written as
\begin{equation}
  \dot{z}=diag(\hat{S}_i)z-\chi (\sum_{k \in \mathcal{L}}(\Psi_k \otimes I_p )(z-\underline{x}_k)),
\end{equation}

where $\underline{x}_k=l_n \otimes x_k$ and $\chi  \textgreater  0$ is the estimator gain designed in Theorem \ref{Theorem 2}.
$T_i$ represents the time that the ith agent receives the successful
learning commands from all its neighbors. $w_{ij}(t)$ is a designed
weight. $w_{ij}(t) = a_{ij}$ if the communication link $C(i, j)$ works
normally at time t and $w_{ij}(t) = 0$, otherwise. Based on $w_{ij}(t)$,
the Laplacian matrix $\mathcal{L} (t) = [\mathcal{L} _{i j}(t)]$ is redefined as $\mathcal{L}_{mf}(t) = -w_{mf}(t)$ if 
$m \neq  f$ and$ \mathcal{L}_{mm} (t) = \sum_{m \neq  f} \omega_{mf}(t) $. Finally, $G_{ik}$ redefine $G_{ik}^w=diag(w_{ij})$ with $w_{ik}(t) = g_{ik}$ if
the communication from the leader to the ith agent works
normally at time t and $w_{ik}(t) = 0$, otherwise. Now we only
consider the case that $t \geq 0 $ with $0$ being the time that all
agents finished the estimate of the unknown matrix $\Upsilon$.


Define 
\begin{equation}
\begin{aligned}
  \tilde{z} 
  &=z-(\sum_{r \in \mathcal{L}}(\Psi_r \otimes I_p ))^{-1} \sum_{k \in \mathcal{L}}(\Psi_k \otimes I_p ) \underline{x}_k ,\\
\end{aligned}
\end{equation}
then








{\color{blue}
\begin{equation}\label{EQ26}
    \begin{aligned}
      \dot{\tilde{z}}
       &=diag(\hat{S}_i)z-\chi (\sum_{k \in \mathcal{L}}(\Psi_k \otimes I_p )(z-\underline{x}_k))-
  (\sum_{r \in \mathcal{L}}(\Psi_r \otimes I_p ))^{-1}\sum_{k \in \mathcal{L}}(\Psi_k \otimes I_p ) (I_n \otimes S) \underline{x}_k \\
      &=diag(\hat{S}_i)z- (I_n \otimes S)z+ (I_n \otimes S)z 
  -(I_n \otimes S)(\sum_{r \in \mathcal{L}}(\Psi_r \otimes I_p ))^{-1} \sum_{k \in \mathcal{L}}(\Psi_k \otimes I_p )  \underline{x}_k  +M \\
  & -\chi \sum_{k \in \mathcal{L}}(\Psi_k \otimes I_p )(z-(\sum_{rr \in \mathcal{L}}(\Psi_{rr} \otimes I_p ))^{-1} (\sum_{kk \in \mathcal{L}}(\Psi_{kk} \otimes I_p )\underline{x}_{kk}+
  (\sum_{rr \in \mathcal{L}}(\Psi_{rr} \otimes I_p )^{-1} (\sum_{kk \in \mathcal{L}}(\Psi_{kk} \otimes I_p )\underline{x}_{kk}-\underline{x}_k)) \\
  &=diag{\tilde{S}}z+(I_n \otimes S) \tilde{z}-\chi\sum_{k \in \mathcal{L}}(\Psi_k \otimes I_p ) \tilde{z} -
  \chi(\sum_{kk \in \mathcal{L}}(\Psi_{kk} \otimes I_p ) \underline{x}_{kk}- \sum_{k \in \mathcal{L}}(\Psi_k \otimes I_p )\underline{x}_k) +M\\
  & =(I_n \otimes S) \tilde{z}-\chi\sum_{k \in \mathcal{L}}(\Psi_k \otimes I_p ) \tilde{z}+F_2(t) .
    \end{aligned}
\end{equation}
  
where $F_2(t)=diag(\tilde{S}_i)z +M$ and
$M = (\sum_{r \in \mathcal{L}}(\Psi_r \otimes I_p ))^{-1}\sum_{k \in \mathcal{L}}(\Psi_k \otimes I_p ) (I_n \otimes S) \underline{x}_k-
(I_n \otimes S) (\sum_{r \in \mathcal{L}}(\Psi_r \otimes I_p ))^{-1}\sum_{k \in \mathcal{L}}(\Psi_k \otimes I_p ) \underline{x}_k$.

Define 
\begin{equation}
    M_k=((\sum_{r \in \mathcal{L}}(\Psi_r \otimes I_p ))^{-1} (\Psi_k \otimes I_p ) (I_n \otimes S)
-(I_n \otimes S) (\sum_{r \in \mathcal{L}}(\Psi_r \otimes I_p ))^{-1}(\Psi_k \otimes I_p )) \underline{x}_k ,
\end{equation}

{\color{blue}
By the Kronecker product property $(P \otimes Q)(Y \otimes Z) =(PY)\otimes(QZ) $, we can obtain that 
\begin{equation}
\begin{aligned}
  & (I_N \otimes S)(\sum_{r \in \mathcal{L}} \Psi_r \otimes I_p)^{-1} (\Psi_k \otimes I_p) \\
   &= (I_N \otimes S)((\sum_{r \in \mathcal{L}} \Psi_r)^{-1} \Psi_k) \otimes I_p) \\
   &=(I_N \times (\sum_{r \in \mathcal{L}} \Psi_r)^{-1} \Psi_k))\otimes(S \times I_p) \\
   &= (\sum_{r \in \mathcal{L}} \Psi_r \otimes I_p)^{-1} (\Psi_k \otimes I_p)  (I_N \otimes S) .\\
\end{aligned}
\end{equation}
}

 we can show that $M_k=0$ and obtain that 
 \begin{equation}
     M=\sum_{k \in \mathcal{L}}M_k =0.
 \end{equation}





}



 then ${\rm  lim}_{t \rightarrow \infty}F(2)=0$ is exponentially at the rate of $\tilde{S}$.


\begin{lemma}(Bellman-Gronwall Lemma inference) \label{Bellman-Gronwall Lemma 1}
     Assuming $a : [t_0,\infty] \rightarrow \mathbb{R}$ is continuous and differentiable ,$w:[t_0,\infty] \rightarrow \mathbb{R}$ and $v:[t_0,\infty] \rightarrow \mathbb{R}$ are continuous,$v \geq 0$ ,$a(t_0)\in \mathbb{R}$ if
    \begin{equation}
    a(t) \leq a(t_0) + \int_{t_0}^{t} v(\tau)a(\tau) +w(\tau) \,d\tau ,t \geq t_0,
    \end{equation}
    or
    \begin{equation}
        \dot{a}(t) \leq a(t) v(t) + w(t),t \geq t_0,
    \end{equation}
    then we obtain that 
    \begin{equation}
       a(t) \leq (a(t_0) + \int_{t_0}^{t} w(t)\,d\tau) e^{\int_{t_0}^{t}
        v(t) \,d\tau}.
    \end{equation}
\end{lemma}


\begin{lemma}(Bellman-Gronwall Lemma 1) \label{Bellman-Gronwall Lemma 1}
    \footnote{inference of Bellman-Gronwall Lemma} Assuming $u : [t_0,\infty] \rightarrow \mathbb{R}$ is continuous and differentiable ,$w:[t_0,\infty] \rightarrow \mathbb{R}$ and $v:[t_0,\infty] \rightarrow \mathbb{R}$ are continuous,$v \geq 0$ ,$M\in \mathbb{R}$ if
    \begin{equation}
    u(t) \leq M + \int_{t_0}^{t} v(\tau)u(\tau) +w(\tau) \,d\tau ,t \geq t_0,
    \end{equation}
    or
    \begin{equation}
        \dot{u}(t) \leq u(t) v(t) + w(t),u(t_0) \leq M,t \geq t_0.
    \end{equation}
    then we obtain that 
    \begin{equation}
        u(t) \leq (M + \int_{t_0}^{t} w(t)\,d\tau) \exp(\int_{t_0}^{t}
        v(t) \,d\tau).
    \end{equation}
\end{lemma}

\begin{lemma}(Bellman-Gronwall Lemma 2) \label{Bellman-Gronwall Lemma 2}
    Assuming $\Phi : [T_a,T_b] \rightarrow \mathbb{R}$ is a  continuous function,$C:[T_a,T_b] \rightarrow \mathbb{R}$ is nonnegative and integrable,$B \geq 0$ is a constant,and
    \begin{equation}
    \Phi(t) \leq B + \int_{0}^{t} C(\tau)\Phi(\tau) \,d\tau ,t \in [T_a,T_b] ,
    \end{equation}
    then we obtain that 
    $$\Phi(t) \leq B \exp(\int_{0}^{t} C(\tau) \,d\tau )$$ for all $t \in [T_a,T_b]$.
\end{lemma}



\begin{myTheo}\label{Theorem 2}
    Consider the MASs \ref{EQ1}-\ref{EQ2} under DoS attacks,which satisfy Assumption 1. If the conditions in Lemma 7 are
satisfied and there exist scalars $\alpha \textgreater  0, \beta \textgreater 0$ and $\chi \textgreater 0$ and
a matrix $ P \textgreater 0$ such that the conditions 1)$ P S + S^T P - \beta P <  0$; 2) $P S + S^T P - (\frac{\chi}{N^2})P +\alpha P < 0$;
 3) $\alpha +\beta < \alpha_{\tau}$ hold, then it can be guaranteed that the estimation error
$\tilde{z}$ exponentially converges to zero in the presence of DoS attacks.
\end{myTheo}

Define the following Lyapunov function
\begin{equation}
    V_1=\tilde{z}(I \otimes P) \tilde{z},
\end{equation}

the derivative of the Lyapunov function is
\begin{equation}
\begin{aligned}
    \dot{V_1}
    &=2 \tilde{z}^T (I \otimes PS) \tilde{z} -2\chi \tilde{z}^T (I \otimes P) \sum_{k \in \mathcal{L}}(\Psi_k \otimes I_p ) \tilde{z} +2 \tilde{z}^T (I \otimes PS) F_2(t).\\
\end{aligned}
\end{equation}
1)for any $t \in S_N(0,\infty)$,there exists an interval $[\delta_{2i},\delta_{2i+1})$ such that
$t\in [\delta_{2i},\delta_{2i+1})$.In this case,we have
\begin{equation}
    \dot{V_1}
    =\tilde{z}^T (I \otimes (PS+S^T P)) \tilde{z} -2\chi \tilde{z}^T (I \otimes P) \sum_{k \in \mathcal{L}}(\Psi_k \otimes I_p ) \tilde{z} +2 \tilde{z}^T (I \otimes PS) F_2(t).
\end{equation}
Then
\begin{equation}
    \dot{V_1}\leq -\alpha V +2 \tilde{z}^T (I \otimes PS) F_2(t),
\end{equation}
based on the Lemma \ref{Bellman-Gronwall Lemma 1},we can obtain that \footnote{the value range of Lemma use $[t_0,t_1)$ replace to $[t_0,\infty)$ }
\begin{equation}\label{EQ21}
    V_1(t)\leq \exp(-\alpha(t-\delta_{2i})) V_{1a}(\delta_{2i}) +\exp(-\alpha(t-\delta_{2i})) \int_{\sigma_{2i}}^{t} 2 \tilde{z}^T (I \otimes PS) F_2 (\tau) \,d\tau .
\end{equation}

2)for any $t \in S_A(0,\infty)$,there exists an interval $[\delta_{2i+1},\delta_{2i+2})$ such that
$t\in [\delta_{2i+1},\delta_{2i+2})$.In this case,we have
\begin{equation}
    \dot{V}_{1}
    =\tilde{z}^T (I \otimes (PS+S^T P)) \tilde{z}  +2 \tilde{z}^T (I \otimes PS) F_2(t) ,
\end{equation}
then
\begin{equation}\label{EQ23}
    V_1(t)\leq \exp(\beta(t-\delta_{2i+1})) V_{1b}(\delta_{2i+1}) + \exp(\beta(t-\delta_{2i+1})) \int_{\sigma_{2i+1}}^{t} 2 \tilde{z}^T (I \otimes PS) F_2 (\tau) \,d\tau .
\end{equation}

we now conclude from (\ref{EQ21}) and (\ref{EQ23}) that  \footnote{summarize one Lemma or write out the detailed derivation}

\begin{equation}
V_1(t)\leq 
\begin{cases}
 \exp(-\alpha(t-\delta_{2i})) V_{1a}(\delta_{2i}) +\exp(-\alpha(t-\delta_{2i})) \int_{\sigma_{2i}}^{t} 2 \tilde{z}^T (I \otimes PS) F_2 (\tau) \,d\tau ,
 t\in [\delta_{2i},\delta_{2i+1}),\\
 \exp(\beta(t-\delta_{2i+1})) V_{1b}(\delta_{2i+1}) +\exp(-\alpha(t-\delta_{2i})) \int_{\sigma_{2i+1}}^{t} 2 \tilde{z}^T (I \otimes PS) F_2 (\tau) \,d\tau,
~~~t\in [\delta_{2i+1},\delta_{2i+2}).
\end{cases}
\end{equation}
Case 1: if $t\in [\delta_{2i},\delta_{2i+1})$, it has 
\begin{equation}
    \begin{aligned}
      V_1(t)  
      & \leq exp(-\alpha(t-\delta_{2i})) V_{1a}(\delta_{2i}) +\exp(-\alpha(t-\delta_{2i})) \int_{\sigma_{2i}}^{t} 2 \tilde{z}^T (I \otimes PS) F_2 (\tau) \,d\tau \\
      & \leq exp(-\alpha(t-\delta_{2i})) V_{1b}(\delta_{2i}^-) +\exp(-\alpha(t-\delta_{2i})) \int_{\sigma_{2i}}^{t} 2 \tilde{z}^T (I \otimes PS) F_2 (\tau) \,d\tau \\
      & \leq exp(-\alpha(t-\delta_{2i})) \exp(\beta(t-\delta_{2i-1})) V_{1b}(\delta_{2i-1}) \\
     & +exp(-\alpha(t-\delta_{2i}))  \exp(\beta(t-\delta_{2i-1})) \int_{\sigma_{2i+1}}^{t} 2 \tilde{z}^T (I \otimes PS) F_2 (\tau) \,d\tau \\
     & +\exp(-\alpha(t-\delta_{2i})) \int_{\sigma_{2i}}^{t} 2 \tilde{z}^T (I \otimes PS) F_2 (\tau) \,d\tau \\
     & \leq \dots \\
     & \leq \exp(-\alpha \left\lvert S_N(t_0,t) \right\rvert ) \exp(\beta \left\lvert S_A(t_0,t) \right\rvert ) V(t_{0}) \\
    & +\int_{t0}^{t} 2 \exp(-\alpha \left\lvert S_N(t_0,t) \right\rvert ) \exp(\beta \left\lvert S_A(t_0,t) \right\rvert )  \tilde{z}^T (I \otimes PS) F_2 (\tau) \,d\tau .\\
    \end{aligned}
\end{equation}

Case 2: if $t\in [\delta_{2i+},\delta_{2i+2})$, it has 
\begin{equation}
    \begin{aligned}
      V_1(t)  & \leq exp(\beta(t-\delta_{2i+1})) V_{1b}(\delta_{2i+1}) +\exp(-\alpha(t-\delta_{2i})) \int_{\sigma_{2i+1}}^{t} 2 \tilde{z}^T (I \otimes PS) F_2 (\tau) \,d\tau, \\
      & \leq exp(\beta(t-\delta_{2i+1})) exp(-\alpha(t-\delta_{2i})) V_{1a}(\delta_{2i}) \\
     & +exp(\beta(t-\delta_{2i+1}))  exp(-\alpha(t-\delta_{2i})) \int_{\sigma_{2i+1}}^{t} 2 \tilde{z}^T (I \otimes PS) F_2 (\tau) \,d\tau \\
     & +\exp(\beta(t-\delta_{2i+1})) \int_{\sigma_{2i}}^{t} 2 \tilde{z}^T (I \otimes PS) F_2 (\tau) \,d\tau \\
     & \leq \dots \\
     & \leq \exp(-\alpha \left\lvert S_N(t_0,t) \right\rvert ) \exp(\beta \left\lvert S_A(t_0,t) \right\rvert ) V(t_{0}) \\
    & +\int_{t0}^{t} 2 \exp(-\alpha \left\lvert S_N(t_0,t) \right\rvert ) \exp(\beta \left\lvert S_A(t_0,t) \right\rvert )  \tilde{z}^T (I \otimes PS) F_2 (\tau) \,d\tau .\\
    \end{aligned}
\end{equation}

Therefore, we can conclude that
\begin{equation}
\begin{aligned}
    V_1(t) 
    & \leq \exp(-\alpha \left\lvert S_N(t_0,t) \right\rvert ) \exp(\beta \left\lvert S_A(t_0,t) \right\rvert ) V(t_0) \\
    & +\int_{t_0}^{t} 2 \exp(-\alpha \left\lvert S_N(t_0,t) \right\rvert ) \exp(\beta \left\lvert S_A(t_0,t) \right\rvert )  \tilde{z}^T (I \otimes PS) F_2 (\tau) \,d\tau  , \forall t \geq t_0.
\end{aligned}
\end{equation}
where $\mathcal{A}=\exp(\overline{\alpha}_0)\exp(\beta+\alpha)\zeta_G V(t_0)$ and $\overline{\alpha}=\alpha -(\alpha+\beta)/ \tau_G$.
with $F_2(t)$ converges to 0 exponentially and $\overline{\alpha } \textgreater 0$
we can conclude that $\tilde{z}$ exponentially converges to zero expentionally.













\subsection{ Distributed Resilient Controller Design}


Define the containment estimated output error as
\begin{equation} \label{EQ e_i}
    \tilde{e}_i= C_i x_i - \tilde{R}_i z_i ,
\end{equation}

the control protocols as follows:\footnote{$\hat{d}_i$ need to update}
\begin{equation}
    \begin{aligned}
    & \eta_i = \theta \eta_i +F \hat{u}_i \\
    & \zeta_i=\overline{A}_{ic} \zeta_i + \overline{B}_{ic} \tilde{e}_i\\
    & u_i = G \eta_i + \hat{u}_i - \hat{d}_i \\
    & \hat{u}_i =\overline{C}_{ic} \zeta_i + \overline{D}_{ic}\tilde{e} \\
    &\hat{d}_i=\frac{B_i^T P_i \tilde{x}_{ci}}{\left\lVert \tilde{x}_{ci}^T P_i B_i \right\rVert +\exp(-\beta_i t)} \hat{\rho_i} \\
    &\dot{\hat{\rho}}_i=\left\lVert \tilde{x}_{ci}^T P_i B_i \right\rVert.\\
    \end{aligned}
\end{equation}



which contains the internal model as \footnote{Whether internal mold parameters need to be changed }
\begin{equation}
    \begin{aligned}
    & \eta_i = \theta \eta_i +F \hat{u}_i \\
    & u_i = G \eta_i + \hat{u}_i - \hat{d}_i \\
    \end{aligned}
\end{equation}
and the stabilizer as
\begin{equation}
    \begin{aligned}
    & \zeta_i=\overline{A}_{ic} \zeta_i + \overline{B}_{ic} \tilde{e}_i\\
    & \hat{u}_i =\overline{C}_{ic} \zeta_i + \overline{D}_{ic}\tilde{e} \\
    \end{aligned}
\end{equation}
 $\tilde{x}_{ci}$  and $P_i$ are designed in (\ref{EQ x_ci}) and (\ref{EQ V})

There exists a matrix $F$ such that $\theta - F G$ is Hurwitz if Assumption 3-5 hold .
then we can get the following augmented system 
\begin{equation} \label{EQ x_ci}
    \dot{x}_{ci} = A_{ci} x_{ci} + B_{ci}z_i + \overline{B}_i \tilde{d}_i
\end{equation}
where $x_{ci}= \left[
    \begin{array}{ccc}
x_i\\
\eta_i \\
\zeta_i \\
    \end{array}
    \right]  $ , $A_{ci}=\left[
    \begin{array}{ccc}
A_i +B_i \overline{D}_{ic} C_i & B_iG & B_i \overline{C}_{ic}\\
F \overline{D}_{ic} C_i  & \theta & F \overline{C}_{ic} \\
\overline{B}_{ic}C_i & 0 & \overline{A}_{ic}
    \end{array}
    \right] $,
    $B_{ic} = \left[
    \begin{array}{ccc}
-B_i \overline{D}_{ic} \tilde{R}_i \\
-F \overline{D}_{ic} \tilde{R}_i \\
- \overline{B}_{ic} \tilde{R}_i \\
    \end{array}
    \right] $, $\overline{B}_i=\left[
    \begin{array}{ccc}
Bi_i\\
0\\
0\\
    \end{array}
    \right] $,$\tilde{d}_i = d_i - \hat{d}_i$.
    
By choosing suitable matrices $\overline{A}_{ic},\overline{B}_{ic},\overline{C}_{ic},\overline{D}_{ic}$ such that  the matrix $A_{ci}$ is Hurwitz. From (\ref{EQ x_ci}) and (\ref{EQ e_i}),we obtain the following estimated regulator output equation:
\begin{equation}\label{EQ estimated output equation}
     \begin{cases}
      A_{ci} \hat{\Pi}_i+B_{ci}=\hat{\Pi}_i \tilde{S} \\
      C_i \hat{\Pi}_{ix} = \tilde{R}_i
     \end{cases}
 \end{equation}
 where $\Pi_i=[\Pi_{ix} \Pi_{i\eta} \Pi_{\zeta}]^T$.
 
 \begin{myTheo}
 suppose the Assumption 2,3,4 hold ture, the estimated output equation (\ref{EQ estimated output equation}) are adaptively solved as follows:
\begin{equation}\label{equation 310}
    \dot{\hat{\Delta}}_{0i} = -\mu \hat{\Phi }^T_i(\hat{\Phi }_i \hat{\Delta}_{0i}-\hat{\mathcal{R}}_i)
\end{equation}

\footnote{Whether to remove $\Delta_{0i}$ and $\Delta_{1i}$ only retain the $\Delta_i$} 


\begin{equation} 
    \dot{\hat{\Delta}}_{1i} = \left\lVert \hat{\Upsilon}_i\right\rVert_F(\hat{\Delta}_{0i}- \hat{\Delta}_{1i})-\mu\hat{\Phi }^T_i(\hat{\Phi }_i \hat{\Delta}_{1i}-\hat{\mathcal{R}}_i)
\end{equation}
\begin{equation}
   \dot{\hat{\Delta}}_{i} = \left\lVert \hat{\Upsilon}_i\right\rVert_F(\hat{\Delta}_{1i}- \hat{\Delta}_{i})-\mu\hat{\Phi }^T_i(\hat{\Phi }_i \hat{\Delta}_{i}-\hat{\mathcal{R} }_i) 
\end{equation}

where $\mu \textgreater 0$ $\hat{\Delta}_{ji}=vec(\hat{Y}_{ji})$ for $j=0,1$ ,$\hat{d}_i=vec(\hat{Y}_i), \hat{\phi}_i=(I_q \otimes  A_{1i}-\hat{S}_i^T \otimes A_{2i}),\hat{R}_i=vec(\mathcal{\hat{R}}^{\ast}_i);\hat{Y_{ji}}=[\hat{\Pi}_{ji}^T,I_p]$ for $j=0,1$ ,$\hat{Y_{i}}=[\hat{\Pi}_{i}^T,I_p]^T,\hat{R}^{\ast}_i=[0,\hat{R}_i^T]^T,
A_{1i}=\left[
  \begin{array}{cc}
  A_{ci} &  B_{ci}  \\
  C_{ix}&  0   \\
  \end{array}
  \right],A_{2i}=
    \left[
  \begin{array}{cc}
   I_{ci} & 0  \\
   0 &  0   \\
  \end{array}
  \right]$,$C_{ix}=[C_i,O,O]$.Then, $\tilde{\Delta }_{ji}$ for j = 0, 1, and $\tilde{\Delta }_{i}(t)$ exponentially converge to zero.
\end{myTheo}

Proof:The  standard output regulator equation of (\ref{EQ estimated output equation}) with standard $S$ and $R$ as follows:
\begin{equation}
   \begin{cases}
      A_{ci} \Pi_i+B_{ci}=\hat{\Pi}_i S \\
      C_i \Pi_{ix} = R
     \end{cases}
\end{equation}

then we have 
\begin{equation*}
 \left[
  \begin{array}{cc}
  A_{ci} &  B_{ci}  \\
  C_{ix} &  O   \\
  \end{array}
  \right]
   \left[
    \begin{array}{cc}
   \Pi_i \\
   I_p \\
    \end{array}
    \right]
    I_q -
    \left[
  \begin{array}{cc}
   I_{ci} & 0  \\
   0 &  0   \\
  \end{array}
  \right]
   \left[
    \begin{array}{cc}
\Pi_i\\
I_p \\
    \end{array}
    \right] S
    =
    \left[\begin{array}{cc}
         O  \\
         R
    \end{array}\right]
\end{equation*}

\begin{equation}
    A_{1i}Y I_q -A_{2i}Y S=R_i^{\ast}
\end{equation}

\begin{equation}
    \Phi_i \Delta=\mathcal{R}_i
\end{equation}
where $\phi_i=(I_q \otimes  A_{1i}-S^T \otimes A_{2i})$,$\Delta_i=vec( \left[
    \begin{array}{cc}
\Pi_i\\
I_p \\
    \end{array}
    \right])$
    $\mathcal{R}_i=vec(\mathcal{R}^{\ast})$,$R^{\ast}=\left[
    \begin{array}{cc}
O\\
R \\
    \end{array}
    \right]$
\begin{equation}
\begin{aligned}
\dot{\hat{\Delta}}_{0i}
&= -\mu \hat{\Phi }^T_i(\hat{\Phi }_i \hat{\Delta}_{0i}-\hat{\mathcal{R}}_i) \\
&=-\mu \hat{\Phi}^T_i \hat{\Phi}_i \hat{\Delta}_{0i}+ \mu \hat{\Phi}_i \hat{\mathcal{R}_i} \\
&=-\mu \Phi_i^T \Phi_i \hat{\Delta}_{0i} +\mu \Phi_i^T \Phi_i \hat{\Delta}_{0i} -\mu \hat{\Phi}_i^T  \hat{\Phi}_i \hat{\Delta}_{0i} +\mu \hat{\Phi}_i^T \hat{\mathcal{R}_i} -\mu \Phi_i^T \hat{\mathcal{R}_i} + \mu \Phi_i^T \hat{\mathcal{R}_i} -\mu \Phi_i^T \mathcal{R}_i + \mu \Phi_i^T \mathcal{R}_i \\
&=-\mu \Phi_i^T \Phi_i \hat{\Delta}_{0i} + \mu (\Phi_i^T \Phi_i - \hat{\Phi}_i^T \hat{\Phi}_i) \hat{\Delta}_{0i} + \mu (\hat{\Phi}_i^T - \Phi_i^T) \hat{\mathcal{R}_i} + \mu \Phi_i^T (\hat{\mathcal{R}_i}-\mathcal{R}_i) + \mu \Phi_i^T \mathcal{R}_i \\
&=-\mu \Phi_i^T \Phi_i \hat{\Delta}_{0i} + \mu \Phi_i^T +d(t) \\
\end{aligned}
\end{equation}
where  $d(t)= -\mu (\hat{\Phi}_i^T\hat{\Phi}_i - \Phi_i^T\Phi_i) \hat{\Delta}_{0i} + \mu \tilde{\Phi}_i^T \hat{\mathcal{R}}_i + \mu \Phi_i^T \tilde{\mathcal{R}}_i$ .Then,since the origin of $\dot{\hat{\Delta}}_{0i}=-\mu \Phi_i^T\Phi_i \hat{\Delta}_{0i}$ is exponentially stable .System (\ref{equation 310}) is input-to-stable with $d(t)+ \mu \Phi_i^T \mathcal{R}_i$ as the input.
$\tilde{\Phi}_i=\hat{\Phi}_i-\Phi_i=\tilde{S}_i^T \otimes A_{2i}$,$\tilde{\mathcal{R}}=vec(\left[\begin{array}{cc}
    0  \\
    \tilde{R}_i
    \end{array}
    \right])$
so $\lim _{t \rightarrow \infty}d(t) =0 $ exponentially at the same rate of $\tilde{\Upsilon}$.
let $\tilde{\Delta}_{0i}=\Delta - \Delta_{0i}$
\begin{equation}
\begin{aligned}
    \dot{\tilde{\Delta}}_{0i}
    =-\mu \Phi_i^T \Phi_i \tilde{\Delta}_{0i} - \mu \Phi_i^T \Delta+\mu \Phi_i^T \mathcal{R}_i +d(t)
    =-\mu \Phi_i^T \Phi_i \tilde{\Delta}_{0i} +d(t)
\end{aligned}
\end{equation}
 so   $\lim _{t \rightarrow \infty }\tilde{\Delta}_{0i}=0$ exponentially.In particular,if $\mu \geq \frac{\alpha_G}{min \sigma(\Phi_i^T\Phi_i)}$($\mu$ is sufficiently large),$\lim _{t \rightarrow \infty }\tilde{\Delta}_{0i}=0$ at the rate of $\alpha_G$.
 \begin{equation}
 \dot{\tilde{\Delta}}_{0i}
=\dot{\Delta}-\dot{\hat{\Delta}}_{0i}
=-\mu\hat{\Phi}^T_i\hat{\Phi}_i\tilde{\Delta}_{0i}+\mu\hat{\Phi}^T_i(\hat{\Phi}_i \Delta_i-\hat{\mathcal{R} }_i)
 \end{equation}
then if $\mu$ is sufficiently large ,$min(\mathcal{R} (\sigma(\mu\hat{\Phi}^T_i\hat{\Phi}_i))) \geq \alpha_G$,
and 
${\rm  lim}_{t\rightarrow \infty} \mu\hat{\Phi}^T_i(\hat{\Phi}_i \Delta_i-\hat{\mathcal{R} }_i)=0$ at the rate of $\alpha_G$ at least.

\begin{equation}
    \dot{\tilde{\Delta}}_{1i}
=\dot{\Delta}-\dot{\hat{\Delta}}_{1i}
=-(\left\lVert \hat{\Upsilon}_i\right\rVert_FI_{p^2+pq}+ \mu\hat{\Phi}^T_i\hat{\Phi}_i)\tilde{\Delta}_{1i}+\left\lVert \hat{\Upsilon}_i\right\rVert_F\tilde{\Delta}_{0i} +\mu\hat{\Phi}^T_i(\hat{\Phi}_i \Delta_i-\hat{\mathcal{R} }_i)
\end{equation}
By Lemma1,we obtain ${\rm  lim}_{t\rightarrow \infty }  \tilde{\Delta}_{1i} = 0$ at the rate of $\alpha_G$,
also ${\rm  lim}_{t\rightarrow \infty }  \dot{\tilde{\Delta}}_{i} = 0$ at the rate of $\alpha_G$,the proof is completed.

\begin{lemma}
    The adaptive distributed leader dynamics observers in (\ref{EQ16}) ensure $ \left\lVert \tilde{\Delta}_i \right\rVert \left\lVert x_k\right\rVert  $ and $\dot{\hat{\Pi}}_i x_k $ exponentially coverage to zero.
\end{lemma}







  

Define 
\begin{equation}\label{EQ x_ci}
    \tilde{x}_{ci}= x_{ci} - \hat{\Pi}_iz_i
\end{equation}
 the derivative of (\ref{EQ x_ci}) as follows:
\begin{equation}\label{EQ tilde{x_ci}}
    \begin{aligned}
      \dot{\tilde{x}}_{ci} 
      &= A_{ci} +B_{ci} +\overline{B}_i \tilde{d}_i -\dot{\hat{\Pi}}_i z_i -\chi (\sum _{j \in \mathcal{F} }w_{ij}(z_i-z_j)+\sum_{k \in \mathcal{L}}w_{ik}(z_i-x_k)) \\
      &=A_{ci} x_{ci} - A_{ci} \hat{\Pi}_i + \overline{B}_i \tilde{d}_i -\dot{\hat{\Pi}}_i z_i -\chi (\sum _{j \in \mathcal{F} }w_{ij}(z_i-z_j)+\sum_{k \in \mathcal{L}}w_{ik}(z_i-x_k)) \\
      &=A_{ci} \tilde{x}_{ci}  + \overline{B}_i \tilde{d}_i -\dot{\hat{\Pi}}_i z_i -\chi (\sum _{j \in \mathcal{F} }w_{ij}(z_i-z_j)+\sum_{k \in \mathcal{L}}w_{ik}(z_i-x_k)) \\
    \end{aligned}
\end{equation}
the global of (\ref{EQ tilde{x_ci}}) can be written as
\begin{equation}
    \begin{aligned}
      \dot{\tilde{x}}_c 
      &= diag (A_{ci}) \tilde{x}_c +diag(B_{ci})\tilde{d} -diag(\dot{\hat{\Pi}}_i)z - -\chi(\sum_{k \in \mathcal{L}}(\Psi_k \otimes I_p )\tilde{z} 
+\sum_{kk \in \mathcal{L}}(\Psi_kk \otimes I_p)\underline{x}_k-\sum_{k \in \mathcal{L}}(\Psi_k \otimes I_p)\underline{x}_k) \\
      &= diag (A_{ci}) \tilde{x}_c +diag(B_{ci})\tilde{d} -diag(\dot{\hat{\Pi}}_i)\tilde{z}-    diag(\dot{\hat{\Pi}}_i)(\sum_{r \in \mathcal{L}}(\Psi_r \otimes I_p ))^{-1} \sum_{k \in \mathcal{L}}(\Psi_k \otimes I_p ) \underline{x}_k  - \chi \sum_{k \in \mathcal{L}} \Phi_k \tilde{z}\\
      &=diag (A_{ci}) \tilde{x}_c +diag(B_{ci})\tilde{d} -diag(\dot{\hat{\Pi}}_i)\tilde{z}-    \frac{1}{M}diag(\dot{\hat{\Pi}}_i)\sum_{k \in \mathcal{L}}\underline{x}_k  - \chi M \Phi_k \tilde{z}\\
    \end{aligned}
\end{equation}
where $\tilde{x}_{c}=[\tilde{x}_{c1}, \tilde{x}_{c2}, \dots ,\tilde{x}_{cn} ]^T$ ,$x_{c}=[x_{c1}, x_{c2} ,\dots ,x_{cn}]^T$, $\tilde{d}=[\tilde{d}_1,\tilde{d}_2,\dots, \tilde{d}_n]$.



By $A_{ci}$ is Hurwitz , there exist $P_i A_{ci} + A_{ci}^T P_i \leq - Q_i $ with $P_i = P_i^T $ is positive-definite matrix.
Consider the following Lyapunov function candidate :
\begin{equation} \label{EQ V}
    V_2= \tilde{x}_{c} ^T P \tilde{x}_{c}
\end{equation}
where $P=diag(P_i)$ and its time derivate is given as follows:
\begin{equation}
\begin{aligned}
    \dot{V_2}
    &= 2\tilde{x}_{c}^T P \dot{\tilde{x}}_c \\
    &=\tilde{x}_{c}^T (P diag(A_{ci}) + diag(A_{ci}^T)P )  \tilde{x}_{c} +2\tilde{x}_{c}^T P diag(\overline{B}_i)\tilde{d}   \\
    & -2\tilde{x}_{c}^T P (diag(\dot{\hat{\Pi}}_i)+   \chi M \Phi_k   ) \tilde{z} - \frac{2}{M} \tilde{x}_{c}^T P diag(\dot{\hat{\Pi}}_i)\sum_{k \in \mathcal{L}}\underline{x}_k  \\
    &\leq - \tilde{x}_{c}^T \sigma_{min}(Q) \tilde{z}  +2\tilde{x}_{c}^T P diag(\overline{B}_i)\tilde{d}  +2\left\lVert \tilde{x}_c ^T\right\rVert \left\lVert P\right\rVert_F  \left\lVert (diag(\dot{\hat{\Pi}}_i)+   \chi M \Phi_k   ) \tilde{z} \right\rVert\\ 
 & +\frac{2}{M} \left\lVert \tilde{x}_{c}^T\right\rVert \left\lVert P\right\rVert_F   \left\lVert diag(\dot{\hat{\Pi}}_i)\sum_{k \in \mathcal{L}}\underline{x}_k\right\rVert  \\
\end{aligned}
\end{equation}
noting that
\begin{equation}
\begin{aligned}
\tilde{x}_{ci}^T P_i B_i\tilde{d}_i
&=\tilde{x}_{ci}^T P_i B_i d_i -\frac{\left\lVert \tilde{x}_{ci}^T P_i B_i \right\rVert^2}{ \left\lVert \tilde{x}_{ci}^T P_i B_i \right\rVert +\exp(-\beta_i t) \hat{\rho_i}} \\
& \leq  \left\lVert \tilde{x}_{ci}^T P_i B_i\right\rVert \left\lVert d_i \right\rVert - \frac{\left\lVert \tilde{x}_{ci}^T P_i B_i \right\rVert^2 }{\left\lVert \tilde{x}_{ci}^T P_i B_i \right\rVert + \exp(-\beta_i t) \hat{\rho_i}} \\
& \leq \frac{\left\lVert \tilde{x}_{ci}^T P_i B_i \right\rVert^2 (\left\lVert d_i \right\rVert- \hat{\rho}_i)+ \left\lVert \tilde{x}_{ci}^T P_i B_i \right\rVert\left\lVert d_i \right\rVert \exp(-\beta_i t)}{\left\lVert \tilde{x}_{ci}^T P_i B_i \right\rVert +\exp(-\beta_i t) \hat{\rho_i}} 
\end{aligned}
\end{equation}

Noting that $d\left\lVert d_i \right\rVert/dt $ is bounded, so,$\left\lVert \tilde{x}_{ci}^T P_i B_i \right\rVert +\exp(-\beta_i t) \rightarrow 0$ . Choose $\left\lVert \tilde{x}_{ci}^T P_i B_i \right\rVert \geq d\left\lVert d_i \right\rVert/dt $, that is,$d\left\lVert d_i \right\rVert/dt - \dot{\hat{\rho}}_i < 0$. Then ,$ \exists t_2 \textgreater 0$ such that for all $t \geq t_2$ ,we have 
\begin{equation}
    \tilde{x}_{ci}^T P_i B_i \tilde{d}_i < 0,
\end{equation}
Therefore, $ \tilde{x}_{c}^T P diag(B_i) \tilde{d} < 0$.

$\dot{\hat{\Pi}}_i x_k$ and $\tilde{z}$ converge to zero exponentially ,so there exist positove constants $ V_{\Pi}k$ and $ \alpha_{Vk}$ such that
the following holds ture :
\begin{equation}
  \left\lVert diag(\dot{\hat{\Pi}}_i)\sum_{k \in \mathcal{L}}\underline{x}_k\right\rVert 
  \leq  V_{\Pi}k \exp (\alpha_{Vk})
\end{equation}

\begin{equation}\label{EQ61}
  \frac{2}{M}\left\lVert \tilde{x}_{c}^T\right\rVert \left\lVert P\right\rVert_F   \left\lVert diag(\dot{\hat{\Pi}}_i)\sum_{k \in \mathcal{L}}\underline{x}_k\right\rVert
  \leq (\frac{1}{4} \sigma_{min}(Q) - \frac{1}{2}\beta_{V1} )\left\lVert \tilde{x}_{c}\right\rVert ^2 +\beta_{v21}\exp(-2\alpha_vt)
\end{equation}
by $\tilde{z}$ exponentially coverage to zero.
we similarly also obtain that   \footnote{is it right by $\tilde{z}$ exponential convergence}
\begin{equation}\label{EQ62} 
  2\left\lVert \tilde{x}_c^T\right\rVert \left\lVert P\right\rVert_F  \left\lVert (diag(\dot{\hat{\Pi}}_i)+   \chi M \Phi_k   ) \tilde{z} \right\rVert
  \leq (\frac{1}{4} \sigma_{min}(Q) - \frac{1}{2}\beta_{V1} )\left\lVert \tilde{x}_c\right\rVert ^2+\beta_{v22}\exp(-2\alpha_{\xi}t-2\alpha_{\Pi}t)
\end{equation}

we conclude from (\ref{EQ61}) and (\ref{EQ62}),that
\begin{equation}
  \dot{V_2} \leq -(\beta_{V1}+\frac{1}{2} \sigma_{min}(Q) ) \tilde{x}_c^T \tilde{x}_c +\beta_{V2}\exp(\alpha_{V1}t)
\end{equation}

\begin{equation}
\begin{aligned}
     V_2(t) 
    & \leq V(0) - \int_{0}^{t} \beta_{V3} \tilde{x}_c^T\tilde{x}_c \,d\tau + \int_{0}^{t} \beta_{V2}\exp(\alpha_{V1}t) \,d\tau \\
\end{aligned}
\end{equation}

\begin{equation}
    \tilde{x}_{c}^T \tilde{x}_{c} \leq -\int_{0}^{t} \frac{1}{\sigma_{min}(P)} \beta_{V3} \tilde{x}_c^T\tilde{x}_c \,d\tau + \overline{D}_i
\end{equation}
where $\overline{D}_i$ is define as a bounded constant.
Recalling  Lemma \ref{Bellman-Gronwall Lemma 2}

\begin{equation}
  \left\lVert \tilde{x}_c \right\rVert \leq \sqrt{\overline{B}} \exp(-\frac{1}{2\sigma_{min}(P)}\beta_{V3}t )
\end{equation}
so $\tilde{z}$ is  exponential convergence,it easy to see that $\tilde{x}_i=x_i-\hat{\Pi}_i z_i$ is also exponential convergence.

\begin{equation}
\begin{aligned}
  e_i
  &=C_i x_i -R(z_i-\tilde{z}_i) \\
  &=C_i(x_i-\hat{\Pi}_{ix} z_i)+(C\hat{\Pi}_{xi}-R)z_i +R \tilde{z}_i\\
  &=C_i \tilde{x}_{i}-\tilde{R}z_i+\tilde{z}_i
\end{aligned}
\end{equation}
by $\tilde{x}_i$, $\tilde{R}$ and $\tilde{z}_i$ coverage exponentially,we can obtain that $e_i$ coverage exponentially.








\section{Numerical Simulation}\label{SecSm}

% \begin{figure}[!htbp]
% %\begin{minipage}[t]{1\linewidth}
% \centering
% \includegraphics[width=0.6\textwidth]{4Ag.pdf}
% \caption{Time-varying directed communication topology among all agents}
% \label{fig:figure1}
% \end{figure}



%{\color{blue}
%\begin{figure}[htbp]
%\centering
%\subfigure[Performance of observer w.r.t. the leader]{
%\begin{minipage}[t]{0.475\textwidth}
%\centering
%\includegraphics[width=0.85\textwidth]{pic/pobs.eps}
%%\caption{fig1}
%\end{minipage}\label{fig:figure2:1}
%}
%%\hspace{-0.1in}
%\subfigure[Performance of observer w.r.t. the first leader]{
%\begin{minipage}[t]{0.475\textwidth}
%\centering
%\includegraphics[width=0.85\textwidth]{pic/vobs.eps}
%%\caption{fig2}
%\end{minipage}\label{fig:figure2:2}
%}\\%
%\centering
%\caption{Performance of two observers}
%\label{fig:figure2}
%\end{figure}












\section{Conclusion}
We investigate the distributed prescribed-time consensus observer for multi-agent systems with
high-order integrator dynamics and directed topologies in this brief.  To our best knowledge, the DPTO on time-invariant/varying digraphs with prescribed-time zero-error convergence has been formulated for the first time, which could achieve distributed estimation w.r.t. the leader state within an arbitrary time interval predefined by the users. An illustrative simulation example has been conducted, which confirms the prescribed-time performance of the above DPTO. Future works will consider using this prescribed-time observer to deal with distributed fault-tolerant control problems \cite{hua2017distributed, xu2020distributed}. %\cite{3xiao2021distributed} 

\section*{Appendix}

 

\

% Proof: Consider the Lyapunov function candidate
% $$
% V_{1}=\frac{1}{2} \sum_{i=1}^{N} \xi_{i}^{T} P \xi_{i}+\sum_{i=1}^{N} \sum_{j=1, j \neq i}^{N} \frac{\left(c_{i j}-\alpha\right)^{2}}{8 \kappa_{i j}}
% $$
% where $\alpha$ is a positive constant that is to be determined later. Evidently, $V_{1}$ is positive definite. The time derivative of $V_{1}$ along the trajectory of (5) is given by
% $$
% \begin{aligned}
% \dot{V}_{1}=& \sum_{i=1}^{N} \xi_{i}^{T} P \dot{\xi}_{i}+\sum_{i=1}^{N} \sum_{j=1, j \neq i}^{N} \frac{c_{i j}-\alpha}{4 \kappa_{i j}} \dot{c}_{i j} \\
% =& \sum_{i=1}^{N} \xi_{i}^{T} P A \xi_{i}+\sum_{i=1}^{N} \xi_{i}^{T} P B K \sum_{j=1}^{N} c_{i j} a_{i j}\left(\tilde{x}_{i}-\tilde{x}_{j}\right) \\
% &+\sum_{i=1}^{N} \sum_{j=1, j \neq i}^{N} \frac{c_{i j}-\alpha}{4 \kappa_{i j}} \dot{c}_{i j}
% \end{aligned}
% $$
% Since $a_{i j}=a_{j i}$ and $c_{i j}(t)=c_{j i}(t)$, it can be easily verified that
% $$
% \begin{array}{rl}
% \sum_{i=1}^{N} \xi_{i}^{T} & P B K \sum_{j=1}^{N} c_{i j} a_{i j}\left(\tilde{x}_{i}-\tilde{x}_{j}\right) \\
% & =-\frac{1}{2} \sum_{i=1}^{N} \sum_{j=1}^{N} c_{i j} a_{i j}\left(\xi_{i}-\xi_{j}\right)^{T} \Gamma\left(\tilde{x}_{i}-\tilde{x}_{j}\right)
% \end{array}
% $$

 \bibliography{PIDFR}
 
 
 
 
 


 
  \end{document}\grid
