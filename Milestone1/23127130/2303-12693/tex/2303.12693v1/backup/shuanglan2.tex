\documentclass[letterpaper, journal, twoside, 10pt,twocolumn]{support/IEEEtran}
%\documentclass[letterpaper, 12pt, journal, twoside]{support/IEEEtran}
\usepackage[fleqn]{amsmath}
\usepackage{times}
\usepackage[pdftex]{graphicx}
\usepackage{subfigure}
\usepackage{amsmath,amssymb,amsopn,amstext,amsfonts}
\usepackage{cancel}
\usepackage[noadjust]{cite}
\usepackage{soul}
\usepackage{caption}
\captionsetup{font={small}}

\captionsetup[figure]{labelfont={},textfont={}}


\usepackage{balance}
\usepackage{color}
\usepackage{mathtools}
% \usepackage{algorithm}
% \usepackage{algorithmic}
\usepackage{bm}
%\newtheorem{theorem}{Theorem}
\usepackage{ diagbox}
\usepackage{float}
\usepackage{epstopdf}
\usepackage{url}
\usepackage{multirow}
\usepackage{tikz}
\usepackage{subeqnarray}
\usepackage{cases}
\usepackage{booktabs}
\usepackage[linkcolor=black,citecolor=black,urlcolor=black,colorlinks=true]{hyperref}
\usepackage{algorithm}
\usepackage[noend]{algpseudocode}
\newtheorem{myTheo}{Theorem}
%\newtheorem{thm}{Theorem}[section] %如果不采用章节号做前缀,则不用[section]
\newtheorem{myDef}{Definition} %这句定义使得defn环境和thm共享编号
\newtheorem{Lemma}{Lemma} %这句定义使得lem环境和thm共享编号
\newtheorem{myCollo}{Corollary}
\newtheorem{remark}{Remark}
%\newtheorem{Lemma}{Lemma}
\newtheorem{myPro}{Proposition}
\newtheorem{assumption}{Assumption}
\newtheorem{example}{Example}
\soulregister\cite7
\soulregister\citep7
\soulregister\citet7
\soulregister\ref7
\soulregister\it7
\soulregister\pageref7

\bibliographystyle{support/IEEEtran}

\newcommand\px{\mathrel{/\mkern-5mu/}}  %平行
\newcommand{\ann}[1]{%
    \begin{tikzpicture}[remember picture, baseline=-0.75ex]%
        \node[coordinate] (inText) {};%
    \end{tikzpicture}%
    \marginpar{%
        \renewcommand{\baselinestretch}{1.0}%
        \begin{tikzpicture}[remember picture]%
            \definecolor{orange}{rgb}{1,0.5,0}%
            \draw node[fill=red!20,rounded corners,text width=\marginparwidth] (inNote){\footnotesize#1};%
    \end{tikzpicture}%
    }%
    \begin{tikzpicture}[remember picture, overlay]%
        \draw[draw = orange, thick]
            ([yshift=-0.2cm] inText)
                -| ([xshift=-0.2cm] inNote.west)
                -| (inNote.west);%
    \end{tikzpicture}%
}%

\graphicspath{{figures/}}
\DeclareGraphicsExtensions{.pdf,.png,.jpg,.eps}
\IEEEoverridecommandlockouts
%\overrideIEEEmargins

\title{\LARGE \bf Resilient Output Containment Control of Heterogeneous Multi-agent Systems against Composite Attacks: A Novel Digital Twin Approach}

%\title{Distributed Optimization in Prescribed-Time: Theory and Experiment}%
\author{
  \vskip 1em
  { 
  Xin Gong, \emph{Graduate Student Member, IEEE}, 
	Yukang Cui, \emph{Member, IEEE},
  Lingbo Cao
  }

  \thanks{
    This work was partially supported by the National Natural Science Foundation of China under Grant 61903258, Guangdong Basic and Applied Basic Research Foundation 2022A1515010234 and the Project of Department of Education of Guangdong Province 2022KTSCX105. %(\emph{Corresponding author: Yukang Cui.}) %the National Natural Science Foundation of China under Grant 61903258

X. Gong is with the Department of Mechanical Engineering, The University of Hong Kong, Pokfulam Road, Hong Kong (e-mail: {\tt\small gongxin@connect.hku.hk}).


Y. Cui and Lingbo. Cao are with the College of Mechatronics and Control Engineering, Shenzhen University, Shenzhen, 518060, China (e-mail: {\tt\small cuiyukang,lingbcao@gmail.com}).


  
%J. He is with the Department of Mechanical Engineering, The University of Hong Kong, Pokfulam Road, Hong Kong (e-mail: {\tt\small esmehe@connect.hku.hk}). 

%X. Gong is with the Department of Mechanical Engineering, The University of Hong Kong, Pokfulam Road, Hong Kong, and also with the College of Mechatronics and Control Engineering, Shenzhen University, Shenzhen 518060, China. (e-mail: {\tt\small gongxin@connect.hku.hk}).
%China, and also
%with the Department of Mechanical Engineering, University of Hong Kong,
%Hong Kong
    
  }
%\thanks{$^{*}$ means the corresponding author.}
}

%\maketitle
%\author{}%\vspace{-0.0cm}
%%\thanks{This work was partially supported by.}% <-this % stops a space
%\thanks{$^{*}$These authors contribute equally and share the first authorship.}
%\thanks{$^{1}$Author is with the Group Robotics with Intelligent Planning (GRIP) Lab, Department of Mechanical Engineering, University of Hong Kong, Hong Kong,
%   {\tt\small gongxin@connect.hku.hk}}
%\thanks{Digital Object Identifier (DOI): see the top of this page.}
%\vspace{-0.5cm}}

% The note headers
%\markboth{Journal of \LaTeX\ Class Files,~Vol.~14, No.~8, August~2015}%
%{Shell \MakeLowercase{\textit{et al.}}: Bare Demo of IEEEtran.cls for IEEE Journals}
\markboth{IEEE Transactions on ...}{GONG \MakeLowercase{\textit{et al.}}: Resilient Output Containment Control of Heterogeneous MAS}%{He \MakeLowercase{\textit{et al.}}: Resilient Path Planning of UAVs against Covert Attacks on UWB Sensors}



\begin{document}
  \maketitle
  \begin{abstract}
 This work studies the distributed resilient output containment control of heterogeneous multi-agent systems against composite attacks, including Denial of Services (DoS) attacks, false-data injection (FDI) attacks, camouflage attacks (CAs), and actuation attacks (AAs). Inspired by Digital Twin, a twin layer (TL) with  higher security and privacy is used to decouple the above problem into two subproblems: defense protocols against DoS attacks on the TL and  defense protocols against AAs on the cyber-physical layer (CPL). First, considering TL  modeling errors, we provide distributed observers to remold the leader dynamics for each follower. Then, distributed estimators use the remolded leader dynamics  to estimate the states of followers under DoS attacks. On the CPL, we give a solver that addresses output regulation equations according to the remolded leader dynamics. Moreover,
 distributed adaptive attack-resilient control schemes that addresses unbounded AAs are provided. Furthermore, we apply the above control protocols to prove that the followers can achieve uniformly ultimately bounded (UUB) convergence, and the upper bound of the UUB convergence is determined explicitly. Finally, simulations are provided to show the effectiveness of the proposed control protocols.
\end{abstract}
\begin{IEEEkeywords}
Composite attacks, Containment, Directed graphs, High-order multi-agent systems, Twin Layer 
% Periodic positive systems, hyper-pyramid,
% reachable set estimation, S-procedure, state-feedback control.
%Formation-containment control,  high-order multi-agent systems,  observer-type protocols,  time-varying formation configuration
\end{IEEEkeywords}
\section{Introduction}
\IEEEPARstart{D}{ISTRIBUTED}  cooperative control of multi-agent systems (MASs) has attracted extensive attention over the last decade due to its broad applications in    satellite formation \cite{scharnagl2019combining}, mobile multirobots \cite{wang2017adaptive}, and smart grids \cite{ansari2016multi}.
%  multi-robot cooperative control, ,
%  mobile multirobot [1], distributed microgrid [2], and vehicle platoon 
% physics, biology, social activity, and engineering. 
As one of the most general phenomena in the area of cooperative control of MASs,
consensus problems can be categorized into two classes, that is, leaderless consensus \cite{fax2004information,ren2007information,olfati2007consensus,wieland2011internal} and leader–follower consensus \cite{hong2013distributed,zhao2016distributed,su2011cooperative,su2012cooperative}.
{\color{black}
%\cite{,,kim2010output,wang2010distributed,lunze2012synchronization,yaghmaie2016output}
Their control schemes drive all agents to reach specified ideal trajectories. In recent years, scholars have introduced multiple leaders into consensus problems, and the control target of multiple leaders consensus problems is to drive all followers into the convex hull formed by the leaders. This multiple leaders consensus problem is also known as containment problem \cite{haghshenas2015,zuo2017output,zuo2019,zuo2020,ma2021observer}. Among them, Haghshenas et al. \cite{haghshenas2015} and Zuo et al. \cite{zuo2017output} solved the  containment control problem of heterogeneous linear multi-agent systems with an output regular equation. Zuo et al. \cite{zuo2019} considered the  containment control problem of heterogeneous linear multi-agent systems against  camouflage attacks (CAs).  Zuo et al. \cite{zuo2020} presented an adaptive compensator against unknown actuator attacks (AAs). Lu et al. \cite{lu2021} considered containment control problems against false-data injection (FDI) attacks. Ma et al. \cite{ma2021observer} solved the containment control problem against denial-of-service (DoS) attacks by developing an event-triggered observer. However, the above containment control works only considered the case of no attack or only one type of attack. 
Considering the vulnerability of MASs, the situation is more complex in reality, and these systems may face multiple attacks at the same time. Therefore, this work considers the multi-agent elastic output containment control problem against composite attacks, including DoS attacks, CAs, FDI attacks and AAs.


In recent years, digital twins \cite{zhu2020,gehrmann2019digital} have been widely used in various fields, including industry \cite{abburu2020cognitwin}, aerospace
%\cite{salinger2020hardware}
\cite{liu2022intelligent}, and medicine \cite{voigt2021digital}. Digital twin technology involves the real-time mapping of objects in virtual space. Inspired by this technology, this work designs a double-layer control structure consisting of TL and CPL. The TL is a virtual mapping of the CPL that has the same communication topology as the CPL and can interact with the CPL in real time. Moreover, the TL has high privacy and security, which can effectively address error information interference in communication networks, such as CAs and FDI attacks. Therefore, the main work of this paper is to design a corresponding control framework that resists DoS attacks
and AAs. Since the TL is only affected by DoS attacks, the main work of this paper can be divided into two parts: a design that resists DoS attacks on the TL and a design that resists AAs on the CPL. 

Resilient control protocols of MASs against DoS attacks were investigated in \cite{fengz2017,zhangd2019,yed2019,deng2021,yang2020}. Feng et al. \cite{fengz2017} considered the leader-follower consensus through its latest state and leader dynamics during DoS attacks. Yang et al. \cite{yang2020} designed a dual-terminal event-triggered mechanism against DoS attacks for linear leader-following MASs.
However, the methods proposed in the above works need to know the leader dynamics for each follower, which is unrealistic in some cases. To address this issue, Cai et al. \cite{cai2017} and Chen et al. \cite{chen2019} used adaptive distributed observers to estimate the leader dynamics and states for an MAS with a single leader. Deng et al. \cite{deng2021} proposed a distributed resilient learning algorithm for learning unknown dynamic models. However, these works did not  consider leader estimation errors and DoS attacks simultaneously. Thus, motivated by the above works, we consider a TL with leader modeling errors, which is inevitable in online modeling. This is an interesting and challenging problem, because leader modeling errors affect follower state estimation errors and output containment errors. To address this problem, we establish a double-TL structure that handles modeling errors and DoS attacks.
On the first TL, distributed observers are established to remold the leader dynamics for each follower. Then, on the second TL, distributed state estimators are used to construct the states of followers.
 
 
}


Resilient control protocols of MASs against AAs were studied in \cite{chen2019,zuo2020,lu2021}. One pioneering work  \cite{de2014resilient} presented a distributed resilient adaptive control scheme against attacks  with exogenous disturbances and interagent uncertainties. Xie et al. \cite{xie2017decentralized} dealt with the decentralized adaptive fault-tolerant control problem for a class of uncertain large-scale interconnected systems with external disturbances and AAs. Moreover, Jin et al. \cite{jin2017adaptive} provided an adaptive controller to guarantee that the closed-loop dynamic system against time-varying adversarial sensor and actuator attacks could achieve uniformly ultimately bounded (UUB) performance. It should be noted that the above works \cite{de2014resilient,xie2017decentralized,jin2017adaptive} could only deal with bounded AAs (or disturbances), which cannot maximize their destructive capacity indefinitely. Recently, Zuo et al. \cite{zuo2020} provided an adaptive control scheme for MAS against unbounded AAs.
Based on the above works, we present a double-CPL scheme to address the output containment problem against AAs. On the first CPL, we provide the output regulator equation solver to deal with the output containment of  heterogeneous MASs. On the second CPL, we use distributed attack-resilient protocols with adaptive compensation signals to resist unbounded AAs. Moreover, in contrast to  \cite{zuo2020}, our adaptive compensation signal includes a soft-sign function, and the containment error bound can be determined explicitly.


The above discussions indicate that the output containment control problem of heterogeneous MASs while determining TL molding errors and handing composite attacks has not yet been solved or studied. Thus, in this work, we provide a hierarchical control scheme to address the above problem.
Moreover, our main contributions can be summarized as follows:
\begin{enumerate}
  \item In contrast to previous works, our work considers the digital twin modeling error; thus, the TL needs to consider both DoS attacks and modeling errors, and the CPL needs to consider output regulation errors and AA errors, thereby increasing the complexity and practicality of the analysis. Moreover, compared with previous work \cite{deng2021}, in which the state estimator requires the necessary time to estimate the leader dynamics, the leader dynamic observers and follower state estimators on TL work synchronously.
  \item A TL-CPL double-layer control structure is introduced, and a double-TL is established in the control framework. The TL has higher security than CPL. Moreover, The TL can effectively defend most of the attacks, such as  CAs and FDI attacks. Moreover, due to the existence of the TL, the resilient control scheme can be decoupled into the defense against DoS attacks on the TL and the defense against potentially unbounded AAs on the CPL.
  \item Compared with \cite{zuo2020}, our work employs an adaptive compensation signal against unbounded AAs, that includes UUB convergence and chattering-free characteristics. Moreover, the containment output error bound of the UUB performance is given explicitly.  
\end{enumerate}




\noindent\textbf{Notations:}
In this work, $I_r$ is the identity matrix with compatible dimensions.
$\boldsymbol{1}_m$ (or $\boldsymbol{0}_m$) denotes a column vector of size $m$ filled with $1$ ($0$, respectively). Denote the index set of sequential integers as $\textbf{I}[m,n]=\{m,m+1,\ldots~,n\}$, where $m<n$ are two natural numbers. Define the set of real numbers and the set of natural numbers as $\mathbb{R}$ and $\mathbb{N}$, respectively. For any matrix $A\in \mathbb{R}^{M\times N}$, ${\rm vec }(A)= {\rm col}(A_1,A_2,\dots,A_N)$, where $A_i\in \mathbb{R}^{M} $ is the $i$th column of $A$.
Define ${\rm {\rm blkdiag}}(A_1,A_2,\dots,A_N)$ as a block diagonal matrix whose principal diagonal elements equal to given matrices $A_1,A_2,\dots,A_N$. $\sigma_{\rm min}(A)$, $\sigma_{\rm max}(A)$ and $\sigma(A)$ denote the minimum singular value, maximum singular value and  the spectrum of matrix $A$, respectively. $\lambda_{\rm min}(A)$ and $\lambda_{\rm max}(A)$ represent the minimum and maximum eigenvalue of $A$, respectively. Herein, $||\cdot||$ denotes the Euclidean norm, and $\otimes$  denotes the Kronecker product.
%; moreover, $A>0$ means that $\lambda_1(A)>0$.
%For a time-varying function $x(t): \mathbb{R}_{\geq 0 }\mapsto \mathbb{R}$, denote that $\sup_{t\in [t_0, t_1]} x(t) $ and $\inf_{t\in [t_0, t_1]} x(t)$ as the upper bound and lower bound of $x(t)$ over the time interval $[t_0, t_1]$, respectively. Moreover, denote that $\|x(t)\|_{[t_0, t_1]} =\sup_{t\in [t_0, t_1]} \|x(t)\| $. Define that $L_{\infty}:=\{x(t)|x(t): \mathbb{R}_{\geq 0 }\mapsto \mathbb{R}^n,\ \|x(t)\|_{[t_0, t_1]}<\infty\}$. In the following sections, $x(t) \in L_{\infty}$, $t\in [t_0, t_1]$, represents that the variable $x$ is uniformly bounded over $[t_0, t_1]$.   %$A>eq 0$ (or $A> 0$) denotes that $A$ is a nonnegative matrix (positive matrix, respectively), which means all elements of $A$ are nonnegative (positive, respectively).
 %${\rm span}(x)$ denotes the span vector of a given vector $x=[p_1, p_2,\ldots~, p_n]^{\mathrm{T}}\in \mathbb{R}^n$.

\label{introduction}


\section{Preliminaries}\label{section2}

%\subsection{Notations}




{\color{black}
\subsection{Graph Theory}
We use a directed graph to represent the interaction among agents. For an MAS consisting of $n$ agents, the graph                  
$\mathcal{G} $  can be expressed by a pair $(\mathcal{V}, \mathcal{E}, \mathcal{A} )$, where $\mathcal{V}=\{ 1, 2, \ldots~, N \}$ is the node set, 
$\mathcal{E} \subset \mathcal{V} \times \mathcal{V}=\{(v_j,\ v_i)\mid v_i,\ v_j \in \mathcal{V}\}$ is the edge set, and $\mathcal{A}=[a_{ij}] \in \mathbb{R}^{N\times N} $ is the associated adjacency matrix.
The weight of the edge $(v_j,\ v_i)$ is denoted by $a_{ij}$ with $a_{ij} > 0$ if $(v_j,\ v_i) \in \mathcal{E}$ and $a_{ij} = 0$
otherwise. The neighbor set of node $v_i$ is represented by $\mathcal{N}_{i}=\{v_{j}\in \mathcal{V}\mid (v_j,\ v_i)\in \mathcal{E} \}$. Define the Laplacian matrix as 
$L=\mathcal{D}-\mathcal{A}  \in \mathbb{R}^{N\times N}$ 
with $\mathcal{D}={\rm blkdiag}(d_i) \in \mathbb{R}^{N\times N}$ where $d_i=\sum_{j \in \mathcal{F}} a_{ij}$ 
is the weight in-degree of node $v_i$.
%A directed graph is \emph{strongly connected} if there is a path from $v_i$ to $v_j$ for any pair of nodes $(v_i,v_j)$.
}
\subsection{Some Useful Lemmas and Definition}
\begin{Lemma}[Bellman-Gronwall Lemma \cite{lewis2003}] \label{Bellman-Gronwall Lemma 2}
    Assume that $\Phi : [T_a,T_b] \rightarrow \mathbb{R}$ is a nonnegative continuous function, $\alpha:[T_a,T_b] \rightarrow \mathbb{R}$ is integrable, $\kappa \geq 0$ is a constant, and there has
    \begin{equation}
    \Phi(t) \leq \kappa + \int_{0}^{t} \alpha (\tau)\Phi(\tau) \,d\tau,t \in [T_a,T_b],
    \end{equation}
    then, we have
    $$\Phi(t) \leq \kappa e^{\int_{0}^{t} \alpha(\tau) \,d\tau }$$ for all $t \in [T_a,T_b]$.
\end{Lemma}








\section{System Setup and Problem Formulation}\label{section3}
In this section, a new problem called resilient containment of
MAS group against composite attacks is proposed. First, a model of the MAS group is established, and some composite attacks are defined as follows.
{\color{black}
\subsection{MAS group Model}
In the framework of containment control, we consider a group of $n+m$ MASs, which can be divided into two groups:

1) The $M$ leaders are the roots of the directed graph $\mathcal{G}$ and have no neighbors. We define the index set of leaders as $\mathcal{L}= \textbf{I}[n+1,n+m]$.

2) The $N$ followers coordinate with their neighbors to achieve the containment set
of the above leaders. We define the index set of the followers
as $\mathcal{F} = \textbf{I}[1, n]$.

Similar to previous works \cite{zuo2020,zuo2021,haghshenas2015}, we consider the following leader dynamics :
\begin{equation}\label{EQ1}
\begin{cases}
\dot{x}_k=S x_k,\\
y_k=R x_k,
\end{cases}
\end{equation}
where $x_k\in \mathbb{R}^q$ and $y_k\in \mathbb{R}^p$ are the system state and reference output of the $k$th leader, respectively.

The dynamics of each follower are given by 
\begin{equation}\label{EQ2}
  \begin{cases}
  \dot{x}_i=A_i x_i + B_i u_i,\\
  y_i=C_i x_i,
  \end{cases}
\end{equation}
where $x_i\in \mathbb{R}^{ni}$, $u_i\in \mathbb{R}^{mi}$ and $y_i\in \mathbb{R}^p$ are 
the system state, control input and output of the $i$th follower, respectively. For convenience, the notation $'(t)'$ can be omitted in the following discussion. We make the following assumptions about the agents and communication network.

{\color{black}
\begin{assumption}\label{assumption 1}
There exists a directed path from at least one leader to each follower $i\in\mathcal{F}$ in the graph $\mathcal{G}$.
\end{assumption}

\begin{assumption}\label{assumption 2}
  The real parts of the eigenvalues of $S$ are nonnegative.
\end{assumption}




\begin{assumption}\label{assumption 3}
 The pair $(A_i, B_i)$ is  stabilizable and  the pair $(A_i, C_i)$ is detectable for $i \in \mathcal{F}$.
\end{assumption}



\begin{assumption}\label{assumption 4}
For all $\lambda \in \sigma(S)$, where $\sigma(S)$ represents the spectrum of $S$,
  \begin{equation}
   {\rm rank} \left[
      \begin{array}{c|c}
     A_i-\lambda I_{n_i} &  B_i  \\ \hline
    C_i  & 0   \\
      \end{array}
      \right]=n_i+p,
      i \in \mathcal{F}.
  \end{equation}
\end{assumption}

}


\begin{remark}
Assumptions 1, 3 and 4 are standard in the classic output regulation problem, and Assumption 2 prevents the trivial case of a stable $S$. The modes associated with the eigenvalues of $S$ with negative real parts decay exponentially to zero and therefore do not affect the asymptotic behavior of the closed-loop system. $\hfill \hfill \square $
\end{remark}




\begin{myDef}\label{def41}
  For the $i$th follower, the system achieves containment if there exists a series of $\alpha_{\cdot i}$,
   that satisfy $\sum _{k \in \mathcal{L}} \alpha_{k i} =1$, thereby ensuring that the following equation holds:
   \begin{equation}
     {\rm  lim}_{t\rightarrow \infty } (y_i(t)-\sum _{k\in \mathcal{L}} \alpha_{k i}y_k(t))=0,
   \end{equation}
   where $i \in \textbf{I}[1,n]$.
\end{myDef}

}


\subsection{ Attack Descriptions}
In this work, we consider the MASs consisting of cooperative agents with potential malicious attackers. As shown in Fig. \ref{fig:figure0}, the attackers use four kinds of attacks to compromise the containment performance of the MASs:

\begin{figure}[!]
  %\begin{minipage}[t]{1\linewidth}
  \centering
  \includegraphics[width=0.5\textwidth]{pic/TPs.png}
  \caption{Resilient MASs against composite attacks: A double-layer framework.}
  \label{fig:figure0}
\end{figure}




1) DoS attacks (denial-of-services attacks): The communication graphs among the agents (in both TL and CPL) are denied by the attackers;

2) AAs (actuation attacks): The motor inputs are infiltrated by the attackers to destroy the input signals of the agent;

3) FDI attacks (fault-data injection attacks): The information exchanged among the agents is distorted by the attackers;

4) CAs (camouflage attacks): The attackers mislead downstream agents by disguising themselves as leaders.

To resist the composite attacks, we introduce a new layer named TL. the same communication topology as the CPL with greater security and fewer physical meanings. Therefore, this TL can effectively resist most of the above attacks. With the introduction of TL, a decoupled resilient control scheme can be introduced to defend against DoS attacks on TL and potential unbounded AAs on CPL. The following subsections present the definitions and essential constraints of the DoS attacks and AAs.

1) DoS attacks: DoS attacks refer to attacks where an adversary attacks some or all components of a control system. Dos attacks can affect the measurement and control channels simultaneously, resulting in the loss of data availability. Assume that there exists a $l \in \mathbb{N}$ and define $\{t_l \}_{l \in \mathbb{N}}$ and $\{\Delta_* \}_{l \in \mathbb{N}}$  as the start time and the duration time of the $l$th attack sequence of a DoS attack, that is, the $l$th DoS attack time-interval is $A_l = [t_l, t_l +\Delta_* )$, where $t_{l+1} > t_l +\Delta_* $ for all $l \in \mathbb{N}$. Therefore, for all $t\geq \tau \in \mathbb{R}$, the set of time instants where the communication network under DoS attacks can be represent by
\begin{equation}
    \Xi_A(\tau,t) = \cup A_l \cap [\tau, t],l\in \mathbb{N},
\end{equation}
and the set of time instants of  the denied communication network can be defined as 
\begin{equation}
    \Xi_N(\tau,t) = [\tau,t] / \Xi_A(\tau,t).
\end{equation}

\begin{myDef} [{Attack Frequency \cite{fengz2017}}]
For any $\delta_2 > \delta_1 \geq t_0$, let $N_a(t_1,t_2)$ represent the number of DoS attacks in $[t_1,t_2)$. Therefore, $F_a(t_1,t_2)= \frac{N_a(t_1,t_2)}{t_2 - t_1}$ can be defined as the attack frequency during $[t_1,t_2)$ for all $t_2 > t_1 \geq t_0$.
\end{myDef}

\begin{myDef} [{Attack Duration \cite{fengz2017}}]\label{Def3}
For any $t_2 > t_1 \geq t_0$, let $T_a(t_1,t_2)$ represent the total time interval of DoS attacks on multi-agent systems during $[t_1,t_2)$. The attack duration over $[t_1,t_2)$ is defined as: there exist constants $\tau_a > 1$ and $T_0 > 0$ such that
  \begin{equation}
    T_a(t_1,t_2) \leq T_0 + \frac{t_2-t_1}{\tau_a}. 
  \end{equation}
\end{myDef}

2) Unbounded Actuation Attacks:
For each follower, the system input is under unknown actuator faults, which is described as
\begin{equation}
    \bar{u}_i=u_i+\chi_i, \forall i  \in \mathcal{F},
\end{equation}
where $\chi_i$ denotes the unknown unbounded attack signal caused by actuator faults. Thus, the true values of $u_i$ and $\chi_i$ are unknown, and we can only measure the damaged control input information $\bar{u}_i$.


\begin{assumption}\label{assumption 5}
The actuator attack signal $\chi_i$ is unbounded and its derivative $\dot{\chi}_i$ is bounded by $\bar{d}$.
\end{assumption}

%{\color{black}
\begin{remark}
In contrast to the works \cite{deng2021} and \cite{chen2019}, which only considered bounded AAs, this work tackles with unbounded AAs under the Assumption \ref{assumption 5}. In the case when the derivative of the attack signal is unbounded, that is, the attack signal increases at a extreme speed, the MAS can reject the signal by removing the excessively large derivative values, which can be easily detected.
$\hfill \hfill \square$
\end{remark}

%}





\subsection{ Problem Formulation}


%{\color{red}
Under the condition that there exists no attack, the output containment error of $i$th follower is represented as:
\begin{equation}\label{EQ xi}
    \xi_i = \sum_{j\in \mathcal{F}} a_{ij}(y_j -y_i) +\sum_{k \in \mathcal{L}} g_{ik}(y_k - y_i),
\end{equation}
where $a_{ij}$ is the weight of edge $(v_i,v_j)$ in the graph $\mathcal{G}_f$
($\mathcal{G}_f$ is the  communication graph of all followers),
 and $g_{ik}$ is  the  weight  of  the  path  from  $i$th  leader  to  $k$th  follower.

The global form of (\ref{EQ xi}) is written as 
\begin{equation}
    \xi = - \sum_{k \in \mathcal{L}}(\Psi_k \otimes I_p)(y -  \underline{y}_k),
\end{equation}
where $\Psi_k = (\frac{1}{m} L_f + G_{ik})$ with $L_f$ is the Laplacian matrix of communication digraph $\mathcal{G}_f$ and $G_{ik}={\rm diag}(g_{ik})$, $\xi = [\xi_1^{\mathrm{ T}},\xi_2^{\mathrm{T}},\dots,\xi_n^{\mathrm{T}}]^{\mathrm{T}}$, $y=[y_1^{\mathrm{T}},y_2^{\mathrm{T}},\dots,y_n^{\mathrm{T}}]^{\mathrm{T}}$, $\underline{y}_k = (l_n \otimes y_k)$.

\begin{Lemma}[\cite{haghshenas2015}] \label{Lemma3}
Suppose Assumption 1 holds. Then, the matrices $\Psi_k$ and $\bar{\Psi}_L= \sum_{k \in \mathcal{L}} \Psi_k$ are positive-definite and non-singular. Therefore, both $(\Psi_k)^{-1}$ and $(\bar{\Psi}_L)^{-1}= (\sum_{k \in \mathcal{L}} \Psi_k)^{-1} $ are non-singular.
\end{Lemma}

The global output containment error can be written as \cite{zuo2019}:
\begin{equation}
e= y - (\sum_{r\in \mathcal{L} }(\Psi_r \otimes I_p))^{-1} \sum_{k \in \mathcal{L} } (\Psi_k \otimes I_p) \underline{y}_k,
\end{equation}
where $e=[e_i^{\mathrm{T}},e_2^{\mathrm{T}},\dots,e_n^{\mathrm{T}}]^{\mathrm{T}}$ and $\xi = -\sum_{k \in \mathcal{L}}(\Psi_k \otimes I_p )e$.


\begin{Lemma}
    Define that $\Theta={\rm diag}(v)$ where $v=(\bar{\Psi}_L)^{-1} \boldsymbol{1}_N$. Under Assumption 1 and Lemma \ref{Lemma3}, the matrix $\Omega=\Theta \bar{\Psi}_L +\bar{\Psi}_L^{\mathrm{T}} \Theta >0$ is a positive diagonal matrix. 
\end{Lemma}


\begin{Lemma}[{\cite[Lemma 1]{zuo2019}}]
    Under Assumption 1, the output containment control objective in  (5) is achieved if ${\rm lim}_{t \rightarrow \infty} e = 0$.
\end{Lemma}
\begin{myDef}[\cite{khalil2002nonlinear}]
 The signal $x(t)$ is said to be UUB with the ultimate bound $b$, if there exist positive constants $b$ and $c$, independent of $t_0 \geq 0$, and for every $a \in (0, c)$, there is $T_1 = T_1(a, b) \geq 0$, independent of $t_0$, such that
 \begin{equation}
     ||x(t)||\leq a \Rightarrow ||x(t)|| \leq b,\forall t \geq t_0+T_1.
 \end{equation}
\end{myDef}



Considering the composite attacks discussed in  Subsection III-B, the local neighboring relative output containment information is rewritten as
\begin{equation}\label{EQ xi2}
    \bar{\xi}_i = \sum_{j\in \mathcal{F}} d_{ij}(\bar{y}_{i,j} -\bar{y}_{i,i}) +\sum_{k \in \mathcal{L}} d_{ik}(\bar{y}_{i,k} - \bar{y}_{i,i})+\sum_{l\in {\mathcal{C}}} c_{il}(y_l-\bar{y}_{i,i}),
\end{equation}
where $d_{ij}(t)$ and $d_{ik}(t)$ are the edge weights that are influence by the DoS attacks. For the denied communication, $d_{ij}(t)=0$ and $d_{ik}(t)=0$; for the normal communication, $d_{ij}(t)=a_{ij}$ and $d_{ik}(t)=g_{ik}$. And $\bar{y}_{i,j}$ (or $\bar{y}_{i,i}$) represent the information flow of $y_j$ (or $y_i$) to the $i$th follower, compromised by the FDI attacks. $\mathcal{C}$ is the index set of all camouflage attackers. $c_{il}$ present the edge weight for $l$th camouflage attacks to the $i$th follower. From Equation (\ref{EQ xi2}), it can see that the attackers could disrupt the communication between agents and distort the actuation inputs of all followers.  


Based on the discussion aforementioned, TL can effectively defend against most attacks. Then, the attack-resilient containment control problem is introduced as follows.
%Next, we introduce the attack-resilient containment control problem.

\noindent \textbf{Problem ACMCA} (Attack-resilient Containment control of MASs
against Composite Attacks): The resilient containment control problem is to design the input $u_i$ in (3) for each follower, such that output containment error $e$ in (12) is UUB under the composite attacks, i.e., the trajectories of each follower converge into a point in the dynamic convex hull spanned by the trajectories of multiple leaders.


%}


































\section{Main Results}


In this section, a double-layer resilient control scheme is used to solve the \textbf{Problem ACMCA}. First, a distributed virtual resilient TL, which can resist the FDI attacks, CAs, and AAs naturally, is proposed to against DoS attacks. Considering the molding errors of TL, we use distributed observers to remold the leader dynamics for all agents under DoS attacks. Next, distributed estimators are proposed to estimate the followers states under DoS attacks. Then, we  use the remolded leader dynamics proposed in Theorem 1 to calculate the solution of output regulator equation. Finally, adaptive controllers are proposed to resist unbounded AAs on the CPL. The specific control framework is shown in Fig. \ref{fig:figure0c}.

\begin{figure}[!]
  %\begin{minipage}[t]{1\linewidth}
  \centering
  \includegraphics[width=0.5\textwidth]{picnew2/cxkt2.png}
  \caption{The control framework of ACMCA.}
  \label{fig:figure0c}
\end{figure}

\subsection{Distributed Observers to Remold leader dynamics}
 In this section, we design distributed observers  to remold the leader dynamics in the first TL.
  
To facilitate the analysis, define the leader dynamics in (2) as follows:
  \begin{equation}
      \Upsilon =[S;R]\in \mathbb{R} ^{(p+q)\times q}
  \end{equation}
and the remolded leader dynamics can be defined as follows:
 \begin{align} \label{EQUpsilonhat}
     &\hat{\Upsilon } _{i}(t)=[\hat{S}_{i}(t);\hat{R}_{i}(t)]\in \mathbb{R} ^{(p+q)\times q},
  \end{align} 
where $\hat{\Upsilon } _{i}(t)$ converges to $\Upsilon$ exponentially by Theorem 1.


\begin{myTheo}\label{Theorem 1}
    Consider a group of $M$ leaders and $N$ followers with dynamic in (\ref{EQ1}) and (\ref{EQ2}). Suppose that Assumption \ref{assumption 1} holds. Let the remolded leader dynamics $\hat{\Upsilon } _{i}(t)$ in (\ref{EQUpsilonhat}) on the TL be updated as follows:
\begin{align}\label{EQ15}
   \dot{\hat{\Upsilon }} _{i}(t)=   \mu_1 (\sum_{j \in \mathcal{F}} d_{ij}(\hat{\Upsilon } _{j}(t) \!-\! \hat{\Upsilon } _{i}(t))
   &\! +\! \sum_{k  \in \mathcal L}d_{ik}(\Upsilon\!-\!\hat{\Upsilon } _{i}(t))),\nonumber \\
  & i=\textbf{I}[1,n],
\end{align}
where $\mu_1 > \frac{\sigma_{\rm max}(S)}{ \lambda_{\rm min}(\Omega \Theta^{-1} ) )(1-\frac{1}{\tau_a})}$ is the estimator gain with $\bar{\Psi}_{L}=\sum_{k\in \mathcal{L}}\Psi_k $. Then, the remolded leader dynamics $\hat{\Upsilon }_i$ coverage to $\Upsilon$ exponentially.
\end{myTheo}

%{\color{blue}
\textbf{Proof.} 
Form (\ref{EQ15}), it can see that we only use the  relative neighborhood information to estimate leader dynamics, therefore,  the leader observer will suffer the influence of DoS attacks.

\textbf{Step 1:}
Define
$\tilde{\Upsilon}_{i}(t)=\Upsilon-\hat{\Upsilon}_{i}  (t)$ as modeling error,
form equation (\ref{EQ15}), we have
\begin{align}\label{EQ17}
    \dot{\tilde{\Upsilon}} _{i} (t)
=&\dot{\Upsilon}-\dot{\hat{\Upsilon}}_{i} (t) \nonumber \\
=&-\mu_1 (\sum_{j = 1}^{N} d_{ij}(\hat{\Upsilon } _{j}(t)-\hat{\Upsilon } _{i}(t)) \nonumber\\
&+\sum_{k\in \mathcal{L}}d_{ik}(\Upsilon-\hat{\Upsilon } _{i}(t))).
\end{align}

then, the global of (\ref{EQ17}) can be written as
\begin{align}\label{EQ 18}
    \dot{\tilde{\Upsilon}}  (t) =& -  \mu_1  (\sum_{k\in \mathcal{L}}\Psi_k^D \otimes I_{p+q}\tilde{\Upsilon} (t))\nonumber\\
    =&-  \mu_1 (\bar{\Psi}_{L}^D \otimes I_{p+q}) \tilde{\Upsilon} (t), t \geq t_0.
\end{align}
where $\bar{\Psi}^D_{L}=\sum_{k\in \mathcal{L}}\Psi_k^D $, $\Psi_k^D (t) = \begin{cases}
 0,t \in \Xi_A, \\ \Psi_k,t \in \Xi_N,
\end{cases}$, and $\tilde{\Upsilon} (t)= [\tilde{\Upsilon}_{1}(t),\tilde{\Upsilon}_{2}(t),\dots,\tilde{\Upsilon}_{N}(t)]$. 
Form (\ref{EQ 18}), we have ${\rm vec }(\dot{\tilde{\Upsilon}}) =-\mu_1(I_{q}\otimes \bar{\Psi}^D_{L} \otimes I_{p+q} ){\rm vec }(\tilde{\Upsilon}) $.
Consider the following Lyapunov function
\begin{equation}
    V_1(t)={\rm vec}(\tilde{\Upsilon})^{\mathrm{T}} (I_{q}\otimes \Theta \otimes \bar{P}_1) {\rm vec}(\tilde{\Upsilon}),
\end{equation}

The derivative of $V_1(t)$ can be written as follows
\begin{equation}
    \begin{aligned}
    \dot{V}_1 =& {\rm vec}(\tilde{\Upsilon})^{\mathrm{T}}( -\mu_1 I_{q}\otimes ((\bar{\Psi}^D_{L})^{\mathrm{T}} \Theta + \Theta \bar{\Psi}^D_{L} )\otimes \bar{P}_1) {\rm vec}(\tilde{\Upsilon}) \\
    \leq&  -\mu_1 \lambda_{\rm min}(\Omega^D \Theta^{-1} ) V_1,
    \end{aligned}
\end{equation}
where $ \Omega^D=(\bar{\Psi}^D_{L})^{\mathrm{T}} \Theta + \Theta \bar{\Psi}^D_{L} $.
Then, we have
\begin{equation}
    \begin{aligned}
    &V_1(t)\leq e^{-\mu_1 \lambda_{\rm min}(\Omega^D \Theta^{-1} )(t-t_0)} V_1(t_0), \\
    &{\rm vec} (\tilde{\Upsilon}) \leq \frac{V_1(t_0)}{\lambda_{\rm min}(\Theta \otimes \bar{P}_1)} e^{-\mu_1 \lambda_{\rm min}(\Omega^D \Theta^{-1} ) (t-t_0)}. \\
    \end{aligned}
\end{equation}

From Definition \ref{Def3}, one has 
\begin{equation}
\begin{aligned}
  \bar{\Psi}_{L}^D(t-t_0) =& \bar{\Psi}_{L}\left\lvert \Xi_N(t_0,t)\right\rvert \\
  =&\bar{\Psi}_{L}|t-t_0-\Xi_A(t_0,t)|\\
  \geq & \bar{\Psi}_{L}(t-t_0-(T_0+ \frac{t-t_0}{\tau_a}))\\
  %\geq & \bar{\Psi}_{L}((1-\frac{1}{\tau_a})(t-t_0)+T_0) \\
  \geq & (1-\frac{1}{\tau_a}) \bar{\Psi}_{L} (t-t_0)  , t\geq t_0,
\end{aligned}
\end{equation}
then, it easy obtain that  $\lambda_{\rm min}(\Omega^D \Theta^{-1} ) (t-t_0) \geq (1-\frac{1}{\tau_a}) \lambda_{\rm min}(\Omega \Theta^{-1} ) (t-t_0).$

Therefore, we can conclude that
\begin{equation}
    {\rm vec} (\tilde{\Upsilon}) \leq c_{\Upsilon} e^{-\alpha_{\Psi}(t-t_0)}, 
\end{equation}
where $c_{\Upsilon}= \frac{V_1(t_0)}{\lambda_{\rm min}(\Theta \otimes \bar{P}_1)} $, and $\alpha_{\Psi}= \mu_1 (1-\frac{1}{\tau_a}) \lambda_{\rm min}(\Omega \Theta^{-1} )  $, which implies that ${\rm vec}(\tilde{\Upsilon})$ and $\tilde{\Upsilon}$ converge to zero exponentially at the rate of $\alpha_{\Psi}$ at least. Moreover, by $\alpha_{\Psi}>\sigma_{\rm max}(S)$, $\tilde{S}_i x_k$ converges to zero exponentially for $i\in \mathcal{F},k\in \mathcal{L}$. 
 $\hfill \hfill \blacksquare $


\begin{remark}
In practice, there exist kinds of noises and attacks among agents, so, we  reconstruct the leader dynamics for all followers by the observers (\ref{EQ15}). The observer gain $\mu_1$ is designed to guarantee that the convergence rate of the estimated value $\tilde{S}$ is greater than the divergence rate of the leader state. Moreover, it guarantees the possible of convergence of the containment error. 
And compared with the existing work \cite{chen2019}, which  only used the observer in the consensus of single leader under normal communication, this work consider the heterogeneous output containment under DoS attacks. 
$\hfill \hfill \square $
\end{remark}


%}







\subsection{ Distributed Resilient State Estimator Design}
Based on the remodeled leader dynamics, a distributed resilient estimator is proposed to estimate the follower state  under DoS attacks on TL.
Consider the following  distributed  estimator in the second TL: 
\begin{equation}\label{equation 200}
  \dot{z}_i=\hat{S}_i z_i -\mu_2 G (\sum _{j \in \mathcal{F} }d_{ij}(z_i-z_j)+\sum_{k \in \mathcal{L}}d_{ik}(z_i-x_k)),
\end{equation}
where $z_i$ is the $i$th follower state estimation on the TL, $\mu_2  >  0$ is the coupling gain that can be choose by the user, and $G$ is the estimator gain to be determined later. 
The global state of TL   can be written as
\begin{equation}
  \dot{z}= \hat{S}_b z-\mu_2 (I_N \otimes G) (\sum_{k \in \mathcal{L}}(\Psi_k^D \otimes I_p )(z-\underline{x}_k)), 
\end{equation}
where $\hat{S}_b={\rm blkdiag}(\hat{S}_i) $, $z=[z_1,z_2,\dots,z_N]$, $\underline{x}_k=l_n \otimes x_k$.

Define the global state estimation error of TL as
\begin{equation}
\begin{aligned}
  \tilde{z} 
  =&z-(\sum_{r \in \mathcal{L}}(\Psi_r \otimes I_p ))^{-1} \sum_{k \in \mathcal{L}}(\Psi_k\otimes I_p ) \underline{x}_k,\\
\end{aligned}
\end{equation}
then, for the normal communication, we have
%
%{\color{red}
\begin{equation}\label{EQ26}
    \begin{aligned}
      \dot{\tilde{z}}
       =&\hat{S}_bz-\mu_2 (I_N \otimes G) (\sum_{k \in \mathcal{L}}(\Psi_k \otimes I_p )(z-\underline{x}_k)) \\
       &-
  (\sum_{r \in \mathcal{L}}(\Psi_r \otimes I_p ))^{-1}\sum_{k \in \mathcal{L}}(\Psi_k \otimes I_p ) (I_n \otimes S) \underline{x}_k \\
      =&\hat{S}_bz- (I_n \otimes S)z+ (I_n \otimes S)z \\
  & -\mu_2 (I_N \otimes G) \sum_{k \in \mathcal{L}}(\Psi_k \otimes I_p )\\
  &(z-(\sum_{rr \in \mathcal{L}}(\Psi_{rr} \otimes I_p ))^{-1} (\sum_{kk \in \mathcal{L}}(\Psi_{kk} \otimes I_p )\underline{x}_{kk}\\
  &+
  (\sum_{rr \in \mathcal{L}} (\Psi_{rr} \otimes I_p )^{-1} (\sum_{kk \in \mathcal{L}}(\Psi_{kk} \otimes I_p )\underline{x}_{kk}-\underline{x}_k)) \\
   &-(I_n \otimes S)(\sum_{r \in \mathcal{L}}(\Psi_r \otimes I_p ))^{-1} \sum_{k \in \mathcal{L}}(\Psi_k \otimes I_p )  \underline{x}_k  +M \\
  =&\tilde{S}_bz+(I_n \otimes S) \tilde{z}-\mu_2 (I_N \otimes G)\sum_{k \in \mathcal{L}}(\Psi_k \otimes I_p ) \tilde{z} \\
  &-
  \mu_2 (I_N \otimes G)(\sum_{kk \in \mathcal{L}} \! (\Psi_{kk} \otimes I_p ) \underline{x}_{kk}\\
  &-\! \sum_{k \in \mathcal{L}} \! (\Psi_k \otimes I_p )\underline{x}_k) \!+\!M\\
  =&(I_n \otimes S) \tilde{z} \!-\!\mu_2 (I_N \otimes G)\sum_{k \in \mathcal{L}}(\Psi_k \otimes I_p ) \tilde{z}\!+\! \tilde{S}_b \tilde{z} \!+\! F_2^x(t),
    \end{aligned}
\end{equation}
where  $F_2^x(t)=\tilde{S}_b(\sum_{r \in \mathcal{L}}(\Psi_r \otimes I_p ))^{-1} \sum_{k \in \mathcal{L}}(\Psi_k \otimes I_p ) \underline{x}_k +M$ and
$M = (\sum_{r \in \mathcal{L}}(\Psi_r \otimes I_p ))^{-1}\sum_{k \in \mathcal{L}}(\Psi_k \otimes I_p ) (I_n \otimes S) \underline{x}_k-
(I_n \otimes S) (\sum_{r \in \mathcal{L}}(\Psi_r \otimes I_p ))^{-1}\sum_{k \in \mathcal{L}}(\Psi_k \otimes I_p ) \underline{x}_k$ and $\tilde{S}_b={\rm blkdiag}(\tilde{S}_i)$ for $i\in \mathcal{F}$.

For the denied communication, we have 
\begin{equation}\label{EQ27}
    \begin{aligned}
      \dot{\tilde{z}}
       =&\hat{S}_bz-
  (\sum_{r \in \mathcal{L}}(\Psi_r \otimes I_p ))^{-1}\sum_{k \in \mathcal{L}}(\Psi_k \otimes I_p ) (I_n \otimes S) \underline{x}_k \\
      =&\hat{S}_bz- (I_n \otimes S)z+ (I_n \otimes S)z \\
  &\!-\!(I_n \otimes S)(\sum_{r \in \mathcal{L}}(\Psi_r \otimes I_p ))^{-1} \sum_{k \in \mathcal{L}}(\Psi_k \otimes I_p )  \underline{x}_k  +M \\
  %=&\tilde{S}_bz+(I_n \otimes S) \tilde{z} +M\\
  =&(I_n \otimes S) \tilde{z}+ \tilde{S}_b \tilde{z} + F_2^x(t).
    \end{aligned}
\end{equation}
So, we can conclude (\ref{EQ26}) and (\ref{EQ27}) that
\begin{equation}\label{EQ32}
    \dot{\tilde{z}} = \begin{cases}
    \begin{aligned}
    \hat{S}_b \tilde{z}-\mu_2 (I_N \otimes G)\sum_{k \in \mathcal{L}}
    (\Psi_k \otimes I_p )  \tilde{z} &+ F_2^x(t),\\
    t \in \Xi_N(t_0,t), 
    \end{aligned} \\
     \hat{S}_b \tilde{z} + F_2^x(t), t \in \Xi_A(t_0,t).
    \end{cases}
\end{equation}



\begin{Lemma}[\cite{haghshenas2015}]
Under Lemma 2 and the Kronecker product property $(P \otimes Q)(Y \otimes Z) =(PY)\otimes(QZ) $, it easy to show that 
$(\sum_{r \in \mathcal{L}}(\Psi_r \otimes I_p ))^{-1}\sum_{k \in \mathcal{L}}(\Psi_k \otimes I_p ) (I_n \otimes S) \underline{x}_k=
(I_n \otimes S) (\sum_{r \in \mathcal{L}}(\Psi_r \otimes I_p ))^{-1}\sum_{k \in \mathcal{L}}(\Psi_k \otimes I_p ) \underline{x}_k$. 

\textbf{Proof:}
Let 
\begin{equation}
\begin{aligned}
M= \sum_{k \in \mathcal{L}} M_k
=& \sum_{k \in \mathcal{L}} ((\sum_{r \in \mathcal{L}}(\Psi_r \otimes I_p ))^{-1} (\Psi_k \otimes I_p ) (I_n \otimes S) \\
&-(I_n \otimes S) (\sum_{r \in \mathcal{L}}(\Psi_r \otimes I_p ))^{-1}(\Psi_k \otimes I_p )) \underline{x}_k.
\end{aligned}
\end{equation}


By the Kronecker product property $(P \otimes Q)(Y \otimes Z) =(PY)\otimes(QZ) $, we can obtain that 
\begin{equation}
\begin{aligned}
  & (I_N \otimes S)(\sum_{r \in \mathcal{L}} \Psi_r \otimes I_p)^{-1} (\Psi_k \otimes I_p) \\
   =& (I_N \otimes S)((\sum_{r \in \mathcal{L}} \Psi_r)^{-1} \Psi_k) \otimes I_p) \\
   =&(I_N \times (\sum_{r \in \mathcal{L}} \Psi_r)^{-1} \Psi_k))\otimes(S \times I_p) \\
   =& (\sum_{r \in \mathcal{L}} \Psi_r \otimes I_p)^{-1} (\Psi_k \otimes I_p)  (I_N \otimes S).\\
\end{aligned}
\end{equation}


 We can show that $M_k=0$ and obtain that 
 \begin{equation}
     M=\sum_{k \in \mathcal{L}}M_k =0.
 \end{equation}
 
 This Proof is completed.
$\hfill \hfill \blacksquare $
\end{Lemma}




%}



Then $F_2^x(t)=\tilde{S}_b(\sum_{r \in \mathcal{L}}(\Psi_r \otimes I_p ))^{-1} \sum_{k \in \mathcal{L}}(\Psi_k \otimes I_p ) \underline{x}_k$ and ${\rm  lim}_{t \rightarrow \infty}F_2^x(t)=0$ is exponentially at the same rate of $\tilde{S}_b$.










\begin{myTheo}\label{Theorem 2}
    Consider the MASs (\ref{EQ1})-(\ref{EQ2}) against DoS attacks, which satisfy Assumption 1, 
    and the DoS attacks satisfying Definition 2 and Definition 3.  There exist scalars 
    $\tilde{\alpha}_1>0$, $\tilde{\alpha}_2>0$, $\mu_2>0$ and  positive symmetric define matrices $\bar{P}_2 >0$ such that:
    \begin{align}
    \bar{P}_2 S+S^{\mathrm{T}}\bar{P}_2 - \bar{P}_2^{\mathrm{T}} \mu_2^2 \bar{P}_2 + \tilde{\alpha}_1 \bar{P}_2 =0 \\
    \bar{P}_2 S+S^{\mathrm{T}}\bar{P}_2  - \tilde{\alpha}_2 \bar{P}_2 \leq 0 \\
    \frac{1}{\tau_a} < \frac{\alpha_1}{\alpha_1+\alpha_2} \label{EQ36}
    \end{align}
%where  $\bar{\Psi}_{L}=\sum_{k \in \mathcal{L}}(\Psi_k ) $,
    where  $\alpha_1=\tilde{\alpha}_1-k_1 ||\Theta \otimes \bar{P}_2||$, $\alpha_2=\tilde{\alpha}_2+k_1 ||\Theta \otimes \bar{P}_2||$ with $k_1>0$, and $G=\mu_2 \lambda_{\rm max}(\Omega^{-1} \Theta) \bar{P}_2$.
    Then it can be guaranteed that the estimation error $\tilde{z}$ exponentially converges to $0$ under DoS attacks.
\end{myTheo}

\textbf{Proof.} 
Nest, we will prove that the TL achieve containment under DoS attacks.
For clarity, we first redefine the set $\Xi_A[t_0, \infty)$ as $\Xi_A[t_0, \infty)= \bigcup_{k=0,1,2,\dots} [t_{2k+1},t_{2k+2})$, where $t_{2k+1}$ and $t_{2k+2}$ indicate the time instants that attacks start and end, respectively.
Then, the set $\Xi_N[t_0,\infty)$ can be redefined as $\Xi_N[t_0, \infty)= [t_0,t_1)\bigcup_{k=1,2,\dots} [t_{2k},t_{2k+1})$.

Consider the following Lyapunov function candidate:
\begin{equation}
   V_2(t)=\tilde{z}^{\mathrm{T}} (\Theta \otimes \bar{P}_2) \tilde{z}.
\end{equation}
when the communication is  normal, compute the time-derivative of $V_1(t)$ yields that:
\begin{equation}
\begin{aligned}
\dot{V}_2
=&\tilde{z}^{\mathrm{T}}(\Theta \otimes (\bar{P}_2 S + S^{\mathrm{T}}\bar{P}_2))\tilde{z} \\
&- \tilde{z}^{\mathrm{T}}
((\bar{\Psi}^{\mathrm{T}}_L \Theta  +\Theta  \bar{\Psi}_L) \otimes \lambda_{\rm max}(\Omega^{-1} \Theta) \mu_2^2 \bar{P}_2^2 )\tilde{z} \\
&+\tilde{z}^{\mathrm{T}} (\Theta \otimes (\bar{P}_2 \tilde{S}_i + \tilde{S}_i^{\mathrm{T}}\bar{P}_2) ) \tilde{z}  +2\tilde{z}^{\mathrm{T}} (\Theta \otimes \bar{P}_2) F_2^x \\
\leq& \tilde{z}^{\mathrm{T}} (\Theta \otimes (\bar{P}_2 S + S^{\mathrm{T}}\bar{P}_2) - \Theta \otimes \mu_2^2 \bar{P}_2^2)\tilde{z} \\
& +\tilde{z}^{\mathrm{T}} ((\Theta \otimes \bar{P}_2) \tilde{S}_b + \tilde{S}_b^{\mathrm{T}} (\Theta \otimes \bar{P}_2)) \tilde{z} +2\tilde{z}^{\mathrm{T}} (\Theta \otimes \bar{P}_2) F_2^x.
\end{aligned}
\end{equation}

By Young's inequality, there exists scalar $k_1>0$ yields that
\begin{align} \label{EQ39}
    2\tilde{z}^{\mathrm{T}}(\Theta \otimes \bar{P}_2) F_2^x \leq k_1 \tilde{z}^{\mathrm{T}} (\Theta \otimes \bar{P}_2)^2 \tilde{z} + \frac{1}{k_1} ||F_2^x||^2 ,
\end{align}
then 
\begin{equation}
    \dot{V}_2\leq -\tilde{\alpha}_1 V_2 + 2  ||\tilde{S}_b|| V_2 +k_1 ||\Theta \otimes \bar{P}_2||V_2 +\frac{1}{k_1} ||F_2^x||^2.
\end{equation}
With $\tilde{S}_b$ and $F_2^x$ converge to $0$ exponentially, we can obtain that
\begin{equation} \label{EQ45}
    \dot{V}_2\leq - (\alpha_1+ a_s e^{-b_s t})V_2 + \varphi(t),
\end{equation}
where $a_s>0$, $b_s>0$, and $\varphi(t)=a_f e^{-b_f}$ with $a_f>0$, and $b_f>0$. 

When the communication is denied, the time-derivative of $V_1(t)$ yields that

\begin{equation}\label{EQ42}
\begin{aligned}
\dot{V}_2
=&\tilde{z}^{\mathrm{T}}(I_N \otimes (\bar{P}_2 S + S^{\mathrm{T}}\bar{P}_2))\tilde{z} 
+ \tilde{z}^{\mathrm{T}}
((\Theta \otimes \bar{P}_2) \tilde{S}_b \\
&+ \tilde{S}_b^{\mathrm{T}} (\Theta \otimes \bar{P}_2)) \tilde{z} +2\tilde{z}^{\mathrm{T}} (\Theta \otimes \bar{P}_2) F_2^x \\
\leq& \tilde{\alpha}_2 V_2 + \frac{||\tilde{S}_b||^2}{k_1 \sigma_{\rm min}(\Theta \bar{P}_2) }  V_2 +k_1 ||\Theta \otimes \bar{P}_2||V_2 +\frac{1}{k_1} ||F_2^x||^2 \\
 \leq& (\alpha_2+ a_s e^{-b_s t}) V_2 + \varphi(t). \\
\end{aligned}
\end{equation}

Solved the inequality (\ref{EQ45}) and (\ref{EQ42}), we can obtain the following inequality
\begin{equation} \label{EQ47}
   V_2(t)\leq \begin{cases}\begin{aligned}
   & e^{\int_{t_{2k}}^{t}-\alpha_1+ a_s e^{-b_s t} \, d\tau}V_2(t_{2k}) \\
   &+ \int_{t_{2k}}^{t} e^{\int_{\tau}^{t}-\alpha_1+ a_s e^{-b_s s} \, ds} \varphi(\tau) \, d\tau , t\in [t_{2k},t_{2k+1}),
   \end{aligned}\\
   \begin{aligned}
    & e^{\int_{t_{2k+1}}^{t}\alpha_2+ a_s e^{-b_s t} \, d\tau} V_2(t_{2k+1}) \\
    &\!+ \!\int_{t_{2k+1}}^{t}\! e^{\int_{\tau}^{t}\!\alpha_2+ a_s e^{-b_s s}\! \, ds} \varphi(\tau)\! \, d\tau , t\in [t_{2k+1},t_{2k+2}) .
   \end{aligned}
   \end{cases}
\end{equation}
 
Let $\alpha=\begin{cases}
 -\alpha_1, t\in[t_{2k},t_{2k+1}) \\
 \alpha_2,t \in[t_{2k+1},t_{2k+2})\\
\end{cases}$, then we could reconstruct the (\ref{EQ47}) as follows
\begin{equation}\label{EQ46}
    V_2(t) \leq  e^{\int_{t_0}^{t}\alpha + a_s e^{-b_s t} \, d\tau}V_2(t_{0}) + \int_{t_{0}}^{t} e^{\int_{\tau}^{t}\alpha+ a_s e^{-b_s s} \, ds} \varphi(\tau) \, d\tau .
\end{equation}

For $t\in \Xi_{N}[t_0,t)$, it yields from (\ref{EQ46}) that
\begin{equation}\label{EQ548}
\begin{aligned}
   V_2(t) \leq& e^{\int_{t_0}^{t} a_s e^{-b_s t}\, d\tau} e^{\int_{t_0}^{t}\alpha  \, d\tau}V_2(t_{0}) \\
   &+ \int_{t_{0}}^{t}          a_f e^{-b_f+ \int_{\tau}^{t} a_s e^{-b_s s} \, d s }  e^{\int_{\tau}^{t}\alpha \, ds}  \, d\tau \\
    \leq& e^{-\frac{a_s}{b_s}(e^{-b_s t} -e^{-b_s t_0})}  e^{-\alpha_1|\Xi_N(t_0,t)|+\alpha_2|\Xi_A(t_0,t)|}V_2(t_0)\\
    &+\int_{t_0}^{t}\! a_f \! e^ { \!-\! b_f \tau\!-\!\frac{a_s}{b_s}(e^{\!-\!b_s t}\! -\!e^{-b_s \tau}\!)} e^{\!-\!\alpha_1|\Xi_N(\tau,t)|\!+\!\alpha_2|\Xi_A(\tau,t)|} \, d \tau .
\end{aligned}
\end{equation}

Similarly, it can be easily conclude that (\ref{EQ548}) holds for $t\in [t_{2k+1},t_{2k+2})$, too. With the Definition \ref{Def3}, we have
\begin{equation}\label{EQ49}
\begin{aligned}
   & -\alpha_1|\Xi_N(t_0,t)+\alpha_2|\Xi_A(t_0,t)| \\
    =& -\alpha_1 (t-t_0-|\Xi_A(t_0,t)|) +\alpha_2|\Xi_A(t_0,t)| \\
    \leq& -\alpha_1 (t-t_0)+(\alpha_1+\alpha_2)(T_0+\frac{t-t_0}{\tau_a}) \\
    \leq& -\eta(t-t_0)+(\alpha_1+\alpha_2)T_0,
\end{aligned}
\end{equation}
where $\eta =(\alpha_1  - \frac{\alpha_1+\alpha_2}{\tau} )$.
Substitute (\ref{EQ49}) into (\ref{EQ548}), we have 
\begin{equation}
    \begin{aligned}
   V_2(t)
    \leq& e^{-\frac{a_s}{b_s}(e^{-b_s t}-e^{-b_s t_0})+(\alpha_1+\alpha_2)(T_0)}V_2(t_0)e^{-\eta(t-t_0)} \\
    &+\frac{a_f}{-b_f-\eta}  e^{-\frac{a_s}{b_s}(e^{-b_s t}-e^{-b_s t_0})+(\alpha_1+\alpha_2)T_0} \\
    &(e^{-b_f t}-e^{-\eta t+(-b_f +\eta)t_0}) \\
   \leq& c_1 e^{-\eta(t-t_0)}+c_2 e^{-b_f t}
    \end{aligned}
\end{equation}
with $c_1=e^{\frac{a_s}{b_s}e^{-b_s t_0}+(\alpha_1+\alpha_2)T_0}(V_2(t_0)-\frac{a_f}{-b_f+\eta}e^{-b_f t_0})$ and $c_2= \frac{a_f}{-b_f+\eta} e^{\frac{a_s}{b_s}e^{-b_s t_0}+(\alpha_1+\alpha_2)T_0}$.
From (\ref{EQ36}), we could get that $\eta>0$. Therefore, it obvious that $V_2(t)$ converges to $0$ exponentially.








Here, we have proof the $\tilde{z}$ converges to $0$ which is bounded. Then the inequality (\ref{EQ39}) can be rewritten as $ 2\tilde{z}^{\mathrm{T}}(\Theta \otimes \bar{P}_2) F_2^x \leq \varphi^*(t)$ with $ \varphi^*(t)$ converge to 0 exponentially, which show that $k_1$ is not necessary and there exist $\alpha_1=\tilde{\alpha}_1$ and $\alpha_2=\tilde{\alpha}_2$.





\begin{remark}
Compare with the existing work \cite{deng2021} which the state  observer start to work after finishing the leader dynamic estimation, this work does not need the necessary time for the leader dynamic observer to remold an accurate dynamic and the state estimator begin to work at the first time which is more applicable to the actual situation.  Moreover, Deng et al. \cite{deng2021} only dealt with  the problem of single leader consensus, we consider the more general case of output containment with multiple  leaders. In addition, different from the existing work \cite{yang2020} which can only obtain the bounded state tracking error under DoS attacks, the state error of TL in our work converges to zero exponentially which have more conservative result.
$\hfill \hfill \square $
\end{remark}


























%}













\subsection{ Distributed Output Regulator Equation Solvers}
Our control protocol needs to use the output regulator equations to provide appropriate feedforward control to achieve output containment. But the output regulator equations needs to know the dynamic of the leader system. Since the MAS is fragile, there are many disturbs in the information transmission channel of CPL. 
Therefore, we cannot directly use the leader dynamics matrices $S$ and $R$ as \cite{zuo2020} here. Since the leader dynamics were remolded in TL, we use the remolding dynamic matrices $\hat{S}_i$ and $\hat{R}_i$ to calculate the solutions of the output regulator equations. Then, the output regulator equations can be solved by the following Theorem on the first CPL.
{\color{black}
\begin{myTheo}
 suppose the Assumptions \ref{assumption 1} and \ref{assumption 4} hold, the estimated solutions $\hat{\Delta}_{i}$ to the output regulator equations in (\ref{EQ10}) are solved as follows:
\begin{align} \label{EQ48}
    &\dot{\hat{\Delta}}_{i} = - \mu_3 \hat{\Phi }^{\mathrm{T}}_i(\hat{\Phi }_i \hat{\Delta}_{i}-\hat{\mathcal{R}}_i),
\end{align}
%\begin{equation}
%   \dot{\hat{\Delta}}_{i} = - u \hat{\Phi }^{\mathrm{T}}_i(\hat{\Phi }_i \hat{\Delta}_{i}-\hat{\mathcal{R}}_i)
%\end{equation}
where $\hat{\Delta}_i={\rm vec}(\hat{Y}_i), \hat{\Phi}_i=(I_q \otimes  M_i-\hat{S}_i^{\mathrm{T}} \otimes N_i)$, $\hat{R}_i={\rm vec}(\mathcal{\hat{R}}^{\ast}_i)$;  $\hat{Y_{i}}=[\hat{\Pi}_{i}^{\mathrm{T}}$, $\hat{\Gamma}_{i}^{\mathrm{T}}]^{\mathrm{T}}$, $\hat{R}^{\ast}_i=[0,\hat{R}_i^{\mathrm{T}}]^{\mathrm{T}}$, $
M_i=\left[
  \begin{array}{cc}
  A_i &  B_i  \\
  C_i &  0   \\
  \end{array}
  \right]$, $N_i=
    \left[
  \begin{array}{cc}
   I_{n_i} & 0  \\
   0 &  0   \\
  \end{array}
  \right]$, and $\mu_3 > \frac{\sigma_{\rm max}(S)}{\lambda_{\rm min}(\Phi_i^{\mathrm{T}} \Phi_i )}$ with $\Phi_i=(I_q \otimes  M_i-S^{\mathrm{T}} \otimes N_i)$.
\end{myTheo}

\textbf{Proof.}
In Theorem 1, we realize that leader dynamics estimations are time-varying, so the output regulator equations  influenced by leader dynamics estimations are also time-varying. Next, we  proof that the estimated solutions of the output regulator equations  converge to the solutions of the standard output regulator equations exponentially.

Next, we proof that $\hat{\Delta}_{i}$ converges to $\Delta_i$ at an exponential rate with $\Delta_i$ is the solution of the standard output regulator equation defined in (\ref{EQ52}).}

From Assumption \ref{assumption 4}, we can get that the following output regulator equation:
 \begin{equation}\label{EQ10}
     \begin{cases}
      A_i\Pi_i+B_i\Gamma_i=\Pi_i S \\
      C_i\Pi_i = R
     \end{cases}
 \end{equation}
have a pair unique solution matrices  $\Pi_i$ and $\Gamma_i$ for $i=\textbf{I}[1,n]$.

Rewriting the standard output regulator equation (\ref{EQ10}) yields as follow:
\begin{equation}\label{EQ51}
 \left[
  \begin{array}{cc}
  A_i &  B_i  \\
  C_i &  0   \\
  \end{array}
  \right]
   \left[
    \begin{array}{cc}
   \Pi_i \\
   \Gamma_i \\
    \end{array}
    \right]
    I_q -
    \left[
  \begin{array}{cc}
   I_{n_i} & 0  \\
   0 &  0   \\
  \end{array}
  \right]
   \left[
    \begin{array}{cc}
\Pi_i\\
\Gamma_i \\
    \end{array}
    \right] S
    =
    \left[\begin{array}{cc}
         0  \\
         R
    \end{array}\right].
\end{equation}

Reformulating the equation (\ref{EQ51}) as follow:
\begin{equation}\label{EQ511}
    M_iY I_q -N_iY S=R_i^{\ast},
\end{equation}
where $M_i=\left[
  \begin{array}{cc}
  A_i &  B_i  \\
  C_i &  0   \\
  \end{array}
  \right],
  N_i=\left[
  \begin{array}{cc}
   I_{n_i} & 0  \\
   0 &  0   \\
  \end{array}
  \right],
  Y_i= \left[
    \begin{array}{cc}
\Pi_i\\
\Gamma_i \\
    \end{array}
    \right], $ and $R_i^{\ast}= \left[\begin{array}{cc}
         0  \\
         R
    \end{array}\right] $.  By Theorem 1.9 of Huang et al. \cite{huang2004}, the standard form of the linear equation in Equation (\ref{EQ511}) can be rewritten as:
\begin{equation}\label{EQ52}
    \Phi_i \Delta_i=\mathcal{R}_i,
\end{equation}
where $\Phi_i=(I_q \otimes  M_i-S^{\mathrm{T}} \otimes N_i)$, $\Delta_i={\rm vec}( Y_i)$, 
    $\mathcal{R}_i={\rm vec}(\mathcal{R}^{\ast})$, $R^{\ast}=\left[
    \begin{array}{cc}
0\\
R \\
    \end{array}
    \right]$.
Noting that 
\begin{equation}
\begin{aligned}
\dot{\hat{\Delta}}_{i}
=& - \mu_3 \hat{\Phi }^{\mathrm{T}}_i(\hat{\Phi }_i \hat{\Delta}_{i}-\hat{\mathcal{R}}_i) \\
=&- \mu_3  \hat{\Phi}^{\mathrm{T}}_i \hat{\Phi}_i \hat{\Delta}_{i}+  \mu_3 \hat{\Phi}_i \hat{\mathcal{R}_i} \\
=&-\!  \mu_3\Phi_i^{\mathrm{T}}\! \Phi_i \hat{\Delta}_{i} \!+\! \mu_3 \Phi_i^{\mathrm{T}} \!\Phi_i \hat{\Delta}_{i} \!-\! \mu_3 \hat{\Phi}_i^{\mathrm{T}}  \hat{\Phi}_i \hat{\Delta}_{i}\! +\!  \mu_3 \hat{\Phi}_i^{\mathrm{T}} \hat{\mathcal{R}_i} \\
&- \mu_3 \Phi_i^{\mathrm{T}} \hat{\mathcal{R}_i} +  \mu_3 \Phi_i^{\mathrm{T}} \hat{\mathcal{R}_i} - \mu_3 \Phi_i^{\mathrm{T}} \mathcal{R}_i +  \mu_3 \Phi_i^{\mathrm{T}} \mathcal{R}_i \\
=&-  \mu_3\Phi_i^{\mathrm{T}} \Phi_i \hat{\Delta}_{i} +  \mu_3 (\Phi_i^{\mathrm{T}} \Phi_i - \hat{\Phi}_i^{\mathrm{T}} \hat{\Phi}_i) \hat{\Delta}_{i}\\
&+   \mu_3 (\hat{\Phi}_i^{\mathrm{T}} - \Phi_i^{\mathrm{T}}) \hat{\mathcal{R}_i} 
+  \mu_3 \Phi_i^{\mathrm{T}} (\hat{\mathcal{R}_i}-\mathcal{R}_i) +  \mu_3 \Phi_i^{\mathrm{T}} \mathcal{R}_i \\
=&- \mu_3 \Phi_i^{\mathrm{T}} \Phi_i \hat{\Delta}_{i} + \mu_3  \Phi_i^{\mathrm{T}} \mathcal{R}_i +d_i(t),\\
\end{aligned}
\end{equation}
where  $d_i(t)= - \mu_3 (\hat{\Phi}_i^{\mathrm{T}}\hat{\Phi}_i - \Phi_i^{\mathrm{T}}\Phi_i) \hat{\Delta}_{i} + \mu_3  \tilde{\Phi}_i^{\mathrm{T}} \hat{\mathcal{R}}_i +         \mu_3 \Phi_i^{\mathrm{T}} \tilde{\mathcal{R}}_i$ with $\tilde{\Phi}_i=\hat{\Phi}_i-\Phi_i=\tilde{S}_i^{\mathrm{T}} \otimes N_i$ and $\tilde{\mathcal{R}}={\rm vec}(\left[\begin{array}{cc}
    0  \\
    \tilde{R}_i
    \end{array}
    \right])$.
It obvious that $\lim _{t \rightarrow \infty}d_i(t) =0 $ exponentially at rate of $\alpha_{\Psi}$. 

Let $\tilde{\Delta}_{i}=\Delta_i - \hat{\Delta}_{i}$, the time derivative of  $\tilde{\Delta}_{i}$
can be computed via
\begin{equation}\label{EQ53}
\begin{aligned}
    \dot{\tilde{\Delta}}_{i}
    =&- \mu_3 \Phi_i^{\mathrm{T}} \Phi_i \tilde{\Delta}_{i} -  \mu_3 \Phi_i^{\mathrm{T}} \Phi_i \Delta_i+  \mu_3\Phi_i^{\mathrm{T}} \mathcal{R}_i +d_i(t) \\
   =&- \mu_3 \Phi_i^{\mathrm{T}} \Phi_i \tilde{\Delta}_{i} +d_i(t),
\end{aligned}
\end{equation}
solve the equation (\ref{EQ53}), we have 
\begin{equation}
\begin{aligned}
 \tilde{\Delta}_i(t)=\tilde{\Delta}_i(t_0)e^{-\mu_3 \Phi_i^{\mathrm{T}} \Phi_i(t-t_0)}+\int_{t_0}^{t} d_i(\tau) e^{- \mu_3 \Phi_i^{\mathrm{T}} \Phi_i(t -\tau)} \, d\tau .
\end{aligned}
\end{equation}

 Since $ \Phi_i^{\mathrm{T}} \Phi_i$ is  positive, and $d_i(t)$ converge to $0$ at rate of $\alpha_{\Psi}$, it can obtain that  $\lim _{t \rightarrow \infty }\tilde{\Delta}_{i}=0$ exponentially. Moreover, with $\mu_3 > \frac{\sigma_{\rm max}(S)}{\lambda_{\rm min}(\Phi_i^{\mathrm{T}} \Phi_i )}$, the exponential convergence rate of $\lim _{t \rightarrow \infty }\tilde{\Delta}_{i}=0$ is bigger than $\sigma_{\rm max}(S)$.
  $\hfill \hfill \blacksquare $
%
\begin{Lemma}\label{Lemma 5}%[\cite{chen2019}]
    The distributed leader dynamics observers in (\ref{EQ15}) ensure $ \dot{\hat{\Pi}}_i $ and $\dot{\hat{\Pi}}_i  x_k$   converge to zero exponentially.
\end{Lemma}



\begin{remark}
  As a part of $\dot{\hat{\Delta}}_i$, the convergence rate of $\dot{\hat{\Pi}}_i $ is at least as $\dot{\tilde{\Delta}}_i$, so, the exponential convergence rate of $\dot{\hat{\Pi}}_i $ is bigger than $\sigma_{\rm max}(S)$ by $\mu_3 > \frac{\sigma_{\rm max}(S)}{\lambda_{\rm min}(\Phi_i^{\mathrm{T}} \Phi_i )}$. Then, it obvious that $\dot{\hat{\Pi}}_i  x_k$ converges to zero exponentially. 
  %Compare with the existing work \cite{cai2017} that the gain $\mu_3 > \frac{\alpha_{\Psi}}{\lambda_{\rm min}(\Phi_i^{\mathrm{T}} \Phi_i )}$, our work have more conservative result.
$\hfill \hfill  \square $
\end{remark}






\subsection{Adaptive Distributed Resilient Controller Design}



Define the following state tracking error :
 \begin{equation}
      \epsilon_i=x_i - \hat{\Pi}_i z_i,\\ \label{EQ58}
 \end{equation}
then consider the decentralized adaptive attack-resilient control protocols in the second CPL as follows:
\begin{align}
 & u_i=\hat{\Gamma}_i z_i +K_i \epsilon_i -\hat{\chi}_i , \\  
&\hat{\chi}_i=\frac{B_i^{\mathrm{T}}  P_i \epsilon_i}{\left\lVert \epsilon_i^{\mathrm{T}} P_i B_i \right\rVert +\omega} \hat{\rho_i}, \\ \label{EQ65}
&\dot{\hat{\rho}}_i=\begin{cases}
 \left\lVert \epsilon_i^{\mathrm{T}} P_i B_i \right\rVert +2\omega,& \mbox{if}  \left\lVert \epsilon_i^{\mathrm{T}} P_i B_i \right\rVert \geq \bar{d}, \\
 \left\lVert \epsilon_i^{\mathrm{T}} P_i B_i \right\rVert +2\omega \frac{\left\lVert \epsilon_i^{\mathrm{T}} P_i B_i \right\rVert}{\bar{d}},& \mbox{ otherwise},
\end{cases}
\end{align}
where  $\hat{\chi}_i$ is an adaptive compensation signal, $\hat{\rho}_i$ is an adaptive updating parameter and the controller gain $K_i$ is designed as 
\begin{equation}\label{EQ0968}
    K_i=-R_i^{-1}B_i^{\mathrm{T}} P_i,
\end{equation}
where $P_i$ is the solution to
\begin{equation}
    A_i^{\mathrm{T}} P_i + P_i A_i + Q_i -P_i B_i R_i^{-1} B_i^{\mathrm{T}} P_i =0.
\end{equation}
%


\begin{myTheo}
 Consider  heterogeneous MAS consisting of   $M$ leaders (\ref{EQ1}) and $N$ followers (\ref{EQ2}) with unbounded faults. Under Assumptions \ref{assumption 1}-\ref{assumption 5}, the \textbf{Problem ACMCA} is solved by designing the leader dynamics observer (\ref{EQ15}), the distributed state estimator (\ref{equation 200}), the distributed output regulator  equation solver (\ref{EQ48}) and the  distributed adaptive controller consisting of (56)-(61).
\end{myTheo}
\textbf{Proof.}
In Theorems 1 and 2, we have proved that the TL can resist DoS attacks in which the frequency of  DoS attack satisfies Theorem 2. In the following, we will show that the state tracking error (\ref{EQ58}) is UUB under unbounded AAs.

The derivative of $\epsilon_i$ is presented as follows:
  \begin{equation}\label{EQ63}
  \begin{aligned}
      \dot{\epsilon}_i
      =& A_i x_i + B_i u_i +B_i \chi_i -\dot{\hat{\Pi}}_i  z_i \\
      &- \hat{\Pi}_i(\hat{S}_i z_i - \mu_2 (\sum _{j \in \mathcal{F} }d_{ij}(z_i-z_j)+\sum_{k \in \mathcal{L}}d_{ik}(z_i-x_k)))\\
      =&(A_i+B_i K_i)\epsilon_i  - \dot{\hat{\Pi}}_i z_i\\
      &+\mu_2 \hat{\Pi}_i  (\sum _{j \in \mathcal{F} }d_{ij}(z_i-z_j)+\sum_{k \in \mathcal{L}}d_{ik}(z_i-x_k))+B_i \tilde{\chi}_i,
  \end{aligned}
  \end{equation}
where $\tilde{\chi}_i=\chi_i-\hat{\chi}_i$.
 
 The global state tracking error of (\ref{EQ63}) is
 \begin{equation}
 \begin{aligned}
  \dot{\epsilon}
  =&{\rm blkdiag}(A_i+B_iK_i)\epsilon -{\rm blkdiag}(\dot{\hat{\Pi}}_i)z \\
  &+\! \mu_2 {\rm blkdiag}(\hat{\Pi}_i)(\sum_{k \in \mathcal{L}}\!(\Psi_k^D \!\otimes\! I_p )(z\!-\!\underline{x}_k))\!+\!{\rm blkdiag}(B_i)\tilde{\chi}
 \\  
  =&\bar{A}_b\epsilon -\dot{\hat{\Pi}}_b(\tilde{z}+ (\sum_{r \in \mathcal{L}}(\Psi_r \otimes I_p ))^{-1} \sum_{k \in \mathcal{L}}(\Psi_k \otimes I_p ) \underline{x}_k) \\
    & + \mu_2 \hat{\Pi}_b\sum_{k \in \mathcal{L}}(\Psi_k^D \otimes I_p )\tilde{z} 
    +      \mu_2 \hat{\Pi}_b\sum_{k \in \mathcal{L}}(\Psi_k^D \otimes I_p )\\
    &((\sum_{r \in \mathcal{L}}(\Psi_r \otimes I_p ))^{-1} \sum_{k \in \mathcal{L}}(\Psi_k \otimes I_p ) \underline{x}_k) -  \underline{x}_k))     +        B_b \tilde{\chi}    \\
   =&\bar{A}_b\epsilon -\dot{\hat{\Pi}}_b(\tilde{z}+ (\sum_{r \in \mathcal{L}}(\Psi_r \otimes I_p ))^{-1} \sum_{k \in \mathcal{L}}(\Psi_k \otimes I_p ) \underline{x}_k) \\
   &+ \mu_2 \hat{\Pi}_b\sum_{k \in \mathcal{L}}(\Psi_k^D \otimes I_p )\tilde{z} +B_b \tilde{\chi}
,
\end{aligned}
 \end{equation}
 where $\bar{A}_b={\rm blkdiag}(A_i+B_i K_i)$, $\dot{\hat{\Pi}}_b={\rm blkdiag}(\dot{\hat{\Pi}}_i)$, $\hat{\Pi}_b= {\rm blkdiag}(\hat{\Pi}_i)$ for $i=\textbf{I}[1,n]$ and $\epsilon=[\epsilon_1^{\mathrm{T}} \epsilon_2^{\mathrm{T}} \dots \epsilon_N^{\mathrm{T}}]^{\mathrm{T}}$, $\tilde{\chi}=[\tilde{\chi}_1^{\mathrm{T}} \tilde{\chi}_2^{\mathrm{T}} \dots \tilde{\chi}_N^{\mathrm{T}}]^{\mathrm{T}}$.
%
%


Consider the following Lyapunov function candidate:
\begin{equation} \label{EQ64}
    V= \epsilon ^{\mathrm{T}} P_b \epsilon,
\end{equation}
where $P_b={\rm blkdiag}(P_i)$,

The time derivate of (\ref{EQ64}) is given as follows: 
\begin{equation}\label{EQ80}
\begin{aligned}
    \dot{V}
    =& 2\epsilon^{\mathrm{T}} P_b \dot{\epsilon}_i \\
    =&-\epsilon^{\mathrm{T}} {\rm blkdiag}(Q_i) \epsilon +2\epsilon^{\mathrm{T}} P_b (\dot{\hat{\Pi}}_b+  \mu_2 \hat{\Pi}_b\sum_{k \in \mathcal{L}}(\Psi_k^D \otimes I_p ) )\tilde{z} \\
    & \!+ \!2\epsilon^{\mathrm{T}}\! P_b \dot{\hat{\Pi}}_b(\sum_{r \in \mathcal{L}}\!(\Psi_r \otimes I_p ))^{-1} \! \sum_{k \in \mathcal{L}}\!(\Psi_k \otimes I_p ) \underline{x}_k\! +\! 2\epsilon^{\mathrm{T}} \!P_b B_b\tilde{\chi}\\
    \leq& -\sigma_{min}(Q_b) \left\lVert \epsilon\right\rVert^2   
    + 2\epsilon^{\mathrm{T}} P_b B_b\tilde{\chi}\\
    &+2\left\lVert \epsilon^{\mathrm{T}}\right\rVert \left\lVert P_b\right\rVert  \left\lVert \dot{\hat{\Pi}}_b+  \mu_2 \hat{\Pi}_b\sum_{k \in \mathcal{L}}(\Psi_k^D \otimes I_p )  \right\rVert  \left\lVert \tilde{z}\right\rVert\\ 
 & +2\left\lVert \epsilon^{\mathrm{T}}\right\rVert \left\lVert P_b\right\rVert  ||\dot{\hat{\Pi}}_b 
  (\sum_{r \in \mathcal{L}}(\Psi_r \otimes I_p ))^{-1}\sum_{k \in \mathcal{L}}(\Psi_k \otimes I_p )  \underline{x}_k||, 
\end{aligned}
\end{equation}
where $Q_b={\rm blkdiag}(Q_i)$.
%
%
%
%

By $\dot{\hat{\Pi}}_i $ and $ \dot{\hat{\Pi}}_i x_k $ converge to zero exponentially in Lemma \ref{Lemma 5},  there exist positive constants $ V_{\Pi}$ and $ \alpha_{\Pi}$ such that
the following holds:
\begin{equation}\label{EQ81}
    \left\lVert  \dot{\hat{\Pi}}_b 
  (\sum_{r \in \mathcal{L}}(\Psi_r \otimes I_p ))^{-1}\sum_{k \in \mathcal{L}}(\Psi_k \otimes I_p ) \underline{x}_k\right\rVert
  \leq  V_{\Pi} \exp (-\alpha_{\Pi}),
\end{equation}
use Young's inequality, we have
\begin{equation}\label{EQ61}
\begin{aligned}
    & 2\left\lVert \epsilon^{\mathrm{T}}\right\rVert \left\lVert P_b\right\rVert   \left\lVert  \dot{\hat{\Pi}}_b 
  (\sum_{r \in \mathcal{L}}(\Psi_r \otimes I_p ))^{-1}\sum_{k \in \mathcal{L}}(\Psi_k \otimes I_p ) \underline{x}_k\right\rVert \\
  \leq& \left\lVert \epsilon\right\rVert^2 + \left\lVert P_b\right\rVert  ^2 \left\lVert \dot{\hat{\Pi}}_b 
  (\sum_{r \in \mathcal{L}}(\Psi_r \otimes I_p ))^{-1}\sum_{k \in \mathcal{L}}(\Psi_k \otimes I_p ) \underline{x}_k\right\rVert^2 \\
  \leq& (\frac{1}{4} \sigma_{min}(Q_b) - \frac{1}{2}\beta_{V1} )\left\lVert \epsilon_i\right\rVert ^2 \\
  &+\frac{\left\lVert P\right\rVert  ^2}{ (\frac{1}{4} \sigma_{min}(Q_b) - \frac{1}{2}\beta_{V1} )} V_{\Pi}^2 \exp (-2\alpha_{\Pi})\\
  \leq&
  (\frac{1}{4} \sigma_{min}(Q_b) - \frac{1}{2}\beta_{V1} )\left\lVert \epsilon_i\right\rVert ^2 +\beta_{v21}e^{-2\alpha_{v1}t}.
\end{aligned}
\end{equation}

From Lemma \ref{Lemma 5}, we can obtain that $\dot{\hat{\Pi}}$ converges to 0 exponentially. By $\tilde{z}$ also converges to zero exponentially, we similarly obtain that
\begin{equation}\label{EQ62}
\begin{aligned}
   & 2\left\lVert \epsilon^{\mathrm{T}}\right\rVert \left\lVert P\right\rVert  \left\lVert \dot{\hat{\Pi}}_b+  \mu_2 \hat{\Pi}_b\sum_{k \in \mathcal{L}}(\Psi_k \otimes I_p )  \right\rVert  \left\lVert \tilde{z}\right\rVert \\
  \leq& (\frac{1}{4} \sigma_{min}(Q_b) - \frac{1}{2}\beta_{V1} )\left\lVert \epsilon\right\rVert ^2+\beta_{v22}e^{-2\alpha_{v2}t}.
\end{aligned}
\end{equation}

Next, noting that
\begin{equation}
\begin{aligned}
\epsilon_i^{\mathrm{T}} P_i  B_i\tilde{\chi}_i
        =& \epsilon_i^{\mathrm{T}} P_i  B_i \chi_i -\frac{\left\lVert \epsilon_i^{\mathrm{T}} P_i B_i \right\rVert^2}{ \left\lVert \epsilon_i^{\mathrm{T}} P_i B_i \right\rVert +\omega} \hat{\rho_i} \\
\leq&  \left\lVert \epsilon_i^{\mathrm{T}} P_i  B_i\right\rVert \left\lVert \chi_i \right\rVert - \frac{\left\lVert \tilde{y}_i^{\mathrm{T}} P_i C_i B_i \right\rVert^2  }{\left\lVert \epsilon_i^{\mathrm{T}} P_i B_i \right\rVert + \omega} \hat{\rho_i} \\
=& \frac{\left\lVert \epsilon_i^{\mathrm{T}} P_i B_i \right\rVert ^2 (  \left\lVert \chi_i \right\rVert-  \hat{\rho}_i)+ \left\lVert \epsilon_i^{\mathrm{T}} P_i B_i \right\rVert\left\lVert \chi_i \right\rVert \omega}{\left\lVert \epsilon_i^{\mathrm{T}} P_i B_i \right\rVert +\omega } \\
=&\frac{\left\lVert \epsilon_i^{\mathrm{T}} P_i B_i \right\rVert ^2 (\frac{\left\lVert \epsilon_i^{\mathrm{T}} P_i B_i \right\rVert +\omega}{\left\lVert \epsilon_i^{\mathrm{T}} P_i B_i \right\rVert}||\chi_i||-\hat{\rho}_i)}{\left\lVert \epsilon_i^{\mathrm{T}} P_i B_i \right\rVert +\omega}.
\end{aligned}
\end{equation}

Noting that $d\left\lVert \chi_i \right\rVert/dt$ is bounded,  so, if $\left\lVert \epsilon_i^{\mathrm{T}} P_i B_i \right\rVert\geq \bar{d} \geq \frac{d||\chi_i||}{d t}$, that is, $\frac{\bar{d}+\omega}{\bar{d}} \frac{d||\chi_i||}{dt} -\dot{\hat{\rho}} \leq \bar{d}+\omega-\dot{\hat{\rho}} \leq -\omega < 0$. Then, there exists $t_2 > 0$ such that for all $t \geq t_2$, we have 
\begin{equation}\label{EQ76}
     (\frac{\left\lVert \epsilon_i^{\mathrm{T}} P_i B_i \right\rVert +\omega}{\left\lVert \epsilon_i^{\mathrm{T}} P_i B_i \right\rVert}||\chi_i||-\hat{\rho}_i) \leq (\frac{\bar{d}+\omega}{\bar{d}}||\chi_i||-\hat{\rho}_i) <0.
\end{equation}

Thus, we can obtain that $\epsilon_i^{\mathrm{T}} P_i  B_i\tilde{\chi}_i <0$ and $\epsilon^{\mathrm{T}} P_b  B_b\tilde{\chi}<0$ over $t \in[t_2,\infty)$.

From (\ref{EQ81}) to (\ref{EQ76}), it yields
\begin{equation}\label{EQ70}
  \dot{V} \leq -(\beta_{V1}+\frac{1}{2} \sigma_{min}(Q_b) ) \epsilon^{\mathrm{T}} \epsilon +\beta_{V2}e^{\alpha_{V1}t}.
\end{equation}

Solving (\ref{EQ70}) yields the following:
\begin{equation}
\begin{aligned}
     V(t) 
    \leq& V(0) \! -\! \int_{0}^{t} \! (\beta_{V1} \!+\!\frac{1}{2} \sigma_{min}(Q_b) ) \epsilon^{\mathrm{T}}\epsilon \,d\tau \!+\! \int_{0}^{t} \! \beta_{V2}e^{\alpha_{V1}t} \,d\tau,\\
\end{aligned}
\end{equation}
then
\begin{equation}\label{EQ72}
    \epsilon^{\mathrm{T}} \epsilon \leq -\int_{0}^{t} \frac{1}{\sigma_{min}(P)} \beta_{V3} \epsilon^{\mathrm{T}}\epsilon \,d\tau + \bar{B}.
\end{equation}
where $\beta_{V3}= \beta_{V1}+\frac{1}{2} \sigma_{min}(Q_b) $ and $ \bar{B}=V(0)-\int_{0}^{t} \beta_{V2}e^{\alpha_{V1}t} \,d\tau $ are define as a bounded constant.
Recalling Bellman-Gronwall Lemma, (\ref{EQ72}) is rewritten as follows:

\begin{equation}
  \left\lVert \epsilon \right\rVert \leq \sqrt{\bar{B}} e^{-\frac{\beta_{V3}t}{2\sigma_{min}(P)} },
\end{equation}
conclude (70) and (74), we can that $\epsilon$ is bounded by $\bar{\epsilon}$, where $\bar{\epsilon}=[\bar{\epsilon}_1^{\mathrm{T}} \bar{\epsilon}_2^{\mathrm{T}} \dots \bar{\epsilon}_N^{\mathrm{T}}]^{\mathrm{T}}$ with
$||\bar{\epsilon}_i||=\frac{\bar{d}}{\sigma_{\rm min}(P_i B_i)}$.
%
%


%}
%{\color{red}







Hence, similar to Lemma 5, the global output synchronization error satisfied that

\begin{equation}
\begin{aligned}
e =& y - (\sum_{r\in \mathcal{L} }(\Phi_r \otimes I_p))^{-1} \sum_{k \in \mathcal{L} } (\Psi_k \otimes I_p) \underline{y}_k \\
=&y-(I_N \otimes R)(z-\tilde{z})\\
=& {\rm blkdiag}(C_i)x   -{\rm blkdiag}(C_i \hat{\Pi}_i)z +{\rm blkdiag}(C_i \hat{\Pi}_i)z\\
&-(I_N \otimes R)z +(I_N \otimes R) \tilde{z} \\
=&  {\rm blkdiag}(C_i)\epsilon  -{\rm blkdiag}(\tilde{R}_i)z +(I_N \otimes R) \tilde{z}.
\end{aligned}
\end{equation}

%{\color{black}
Since $\epsilon$, $\tilde{R}_i$, $\tilde{z}$  converge to 0  exponentially which have been proofed, it is obviously that the global output containment error $e$ is bounded by $\bar{e}$ with $\bar{e}=[\bar{e}_1^{\mathrm{T}} \bar{e}_2^{\mathrm{T}} \dots \bar{e}_i^{\mathrm{T}}]^{\mathrm{T}}$ and $\bar{e}_i=\frac{\bar{d} ||C_i||}{\sigma_{\rm min}(P_i B_i)}$ for $i= \textbf{I}[1,n]$. 
%}
The whole proof is completed.
$\hfill \hfill \blacksquare $

\begin{remark}
Compared with \cite{chen2019} which only solved the tracking problem with a single leader against bounded attacks, we consider more challenge work with multiple leaders output containment problem against unbounded attacks. Moreover, the boundary of the output containment error is given, that is, $ \frac{ ||C_i|| \bar{d}}{\sigma_{\rm min}(P_i B_i)}$. 
%It's helpful for designers to pre-estimate of the controller.
$\hfill \hfill  \square $
\end{remark}








\section{Numerical Simulation}\label{SecSm}
\begin{figure}[!]
  %\begin{minipage}[t]{1\linewidth}
  \centering
  \includegraphics[width=0.2\textwidth]{pic/tp1.png}
  \caption{Topology graph.}
  \label{fig:figure1}
\end{figure}

\begin{figure}[!]
  %\begin{minipage}[t]{1\linewidth}
  \centering
  \includegraphics[width=0.45\textwidth]{picnew2/Upsilon.eps}
  \caption{The estimations of leader dynamics in Theorem 1: The shadow areas denote time intervals against DoS attacks.}
  \label{fig:figure1}
\end{figure}

\begin{figure}[htbp]
  %\begin{minipage}[t]{1\linewidth}
  \centering
  \includegraphics[width=0.45\textwidth]{picnew2/zx.eps}
  \caption{Performance of the TL: the shadow areas denote time intervals against DoS attacks.}
  \label{fig:figure2}
\end{figure}




\begin{figure}[!]
  %\begin{minipage}[t]{1\linewidth}
  \centering
  \includegraphics[width=0.45\textwidth]{picnew2/Delta.eps}
  \caption{The estimated solution of the output regulation equation in Theorem 3: The shadow areas denote time intervals against DoS attacks.}
  \label{fig:figure3}
\end{figure}
  

\begin{figure}[!]
  %\begin{minipage}[t]{1\linewidth}
  \centering
  \includegraphics[width=0.45\textwidth]{picnew2/track2.eps}
  \caption{ Output trajectories of the leaders and followers.}
  \label{fig:figure4}
\end{figure}




\begin{figure}[htbp]
  %\begin{minipage}[t]{1\linewidth}
  \centering
  \includegraphics[width=0.45\textwidth]{picnew2/ex.eps}
  \caption{Output containment errors.}
  \label{fig:figure5}
\end{figure}


\begin{figure}[!]
  %\begin{minipage}[t]{1\linewidth}
  \centering
  \includegraphics[width=0.45\textwidth]{picnew2/ebar.eps}
  \caption{The estimation error between TL and CPL: The blue shadow areas denote the  UUB bound by $\bar{d}$.}
  \label{fig:figure6}
\end{figure}

In this section, two examples will be given to llustrate the effectiveness of the above control protocol.


\subsection{Example 1}
We consider  a multi-agent system consisting of seven agents (three leaders indexed by 5$\sim$7 and four followers indexed by 1$\sim$4) with the  corresponding graph show in Fig. 3. Set that $a_{ij}=1$  if there exist a path from node $j$ to node $i$.
Consider the leader dynamics are described by 
$$
S
=
 \left[
  \begin{array}{cc}
 0.5  & -0.4  \\
 0.8 &  0.5   \\ 
  \end{array}
  \right],R
  =
   \left[
    \begin{array}{cc}
   1 &  0    \\
   0  & 1  \\ 
    \end{array}
    \right],
$$
and the dynamics of follows are given by
$$
A_1
\!=\!
 \left[\!
  \begin{array}{cc}
 3  & -2  \\
 1 &  -2   \\ 
  \end{array}
 \! \right],B_1
 \! =\!
   \left[\!
    \begin{array}{cc}
   1.8 &  -1    \\
   2  & 3  \\ 
    \end{array}
   \! \right],C_1
  \!=\!
   \left[\!
    \begin{array}{cc}
   -0.5 &  1    \\
  2  & -1.5  \\ 
    \end{array}
   \! \right],
$$
$$
A_2
\!=\!
 \left[\!
  \begin{array}{cc}
 0.6  & -1  \\
 1 &  -2   \\ 
  \end{array}
  \!\right],B_2
  \!=\!
   \left[\!
    \begin{array}{cc}
   1 &  -2    \\
   1.9  & 4  \\ 
    \end{array}
   \! \right],C_2
  \!=\!
   \left[\!
    \begin{array}{cc}
   -0.5 &  1    \\
  1.5  & 1.4  \\ 
    \end{array}
  \!  \right],
$$
$$
A_3=A_4
=
 \left[
  \begin{array}{ccc}
 0  &  1  &0\\
 0 &  0 & 1  \\ 
  0 &  0 & -2  \\ 
  \end{array}
  \right],B_3
  =B_4=
   \left[
    \begin{array}{ccc}
   6 &  0    \\
   0  & 1  \\ 
   1  & 0 \\ 
    \end{array}
    \right],$$
    $$
    C_3=C_4
  =
   \left[
    \begin{array}{ccc}
   0.5 &  -0.5  & 0.5  \\
  -0.5  & -0.5  &0.5\\ 
    \end{array}
    \right].
$$

The DoS attack periods are given as $[0.5+2k,1.53+2k)s$ for $k\in \mathbb{N}$, which satisfied Assumption 3. The MAS can defence the CAs and FDI attacks since the information transmitted on the CPL is not used in the hierarchal control scheme.
The AAs  are designed as $\chi_1=\chi_2=\chi_3=0.01 \times [2t \quad t]^{\mathrm{T}}$ and $\chi_4=-0.01\times[2t\quad t]^{\mathrm{T}}$. 


 %the   light red zone denotes the UUB bound || εTi PiBi|| ≤ d
 
Chosen the gains in (\ref{EQ15}), (\ref{equation 200}) and (\ref{EQ48}) as $\mu_1=2$, $\mu_2=0.5$ and $\mu_3=6$, respectively. It can be obtain by Theorem 2 that $P2 =\left[ \begin{array}{cc}
    0.6038 & 0.0108 \\
    0.0108 & 0.1455
\end{array}\right]$ and $G=\left[\begin{array}{cc}
     4.5191& 0.5186 \\
     0.5186& 4.0746
\end{array}\right]$. By employing the above parameters, the trajectories of leader dynamics estimations are shown in Fig. \ref{fig:figure1}, the estimate of leader dynamics $||\hat{\Upsilon}_i||$ against DoS attacks  remained stable after $6s$ with the value equal to the value of  leader dynamics. The state estimation errors of TL are given in Fig. \ref{fig:figure2}, which can be seen the performance of TL is stable under the DoS attacks.
The trajectories of  the solution errors of output regulation equations are given in Fig. \ref{fig:figure3},  which validates Theorem  3.
 And to solve the AAs on the CPL, we design the controller gain in (\ref{EQ0968}) as 
$$
K_1
=
 \left[
  \begin{array}{cc}
 -1.71  & 0.07  \\
 2.31  &  -1.12   \\ 
  \end{array}
  \right],K_2
  =
   \left[
    \begin{array}{cc}
   -0.62 &  -0.29   \\
   1.16  & -0.56  \\ 
    \end{array}
    \right],$$
    $$K_3=K_4
  =
   \left[
    \begin{array}{ccc}
   -1.00 &  -0.19  & -0.08  \\
  0.02  & -0.96  &-0.30\\ 
    \end{array}
    \right].
$$






Now, we can solve the ACMCA problem by Theorem 4.  The output trajectories of all the agents are shown in the Fig. \ref{fig:figure4}. It can be seen that followers eventually enter the convex hull formed by the leaders. The output containment errors of followers are shown in Fig. \ref{fig:figure5}, which can be seen that the local output errors $e_i(t)$ is UUB. The tracking error  between the CPL and the TL  is depicted in Fig. \ref{fig:figure6}, which show the tracking error $\epsilon_i$ is UUB by the prescribed error bound.  These results show that the output  containment problem  is solved under the composite attacks.

\subsection{Example 2}

\begin{figure}[!]
  %\begin{minipage}[t]{1\linewidth}
  \centering
  \includegraphics[width=0.25\textwidth]{picnew/txtp0705.png}
  \caption{Communication Topology graph of UAV swarm : blue UAVs  and green UAVs represent leaders and followers, respectively. }
  \label{fig:figure8}
\end{figure}

 We consider a practice example of UAVs with the  corresponding graph shown in Fig. \ref{fig:figure8}. We assume that  the weights of all the edges are 1.
 
 
%  Choose
%  $$\mathcal{A}_f=\left[ \begin{array}{ccccc}
%      0 & 0& 0& 0& 1 \\
%       1 & 0& 0& 0& 1\\
%       1 & 1& 0& 0& 0\\
%       0 & 0& 1& 0& 0\\
%       0 & 0& 0& 1& 0\\
%  \end{array}
%  \right],
%  $$
%  and 
%   $$G_{ik}=\left[ \begin{array}{ccccc}
%      1 & 0& 0\\
%       0 & 0&0 \\
%       0 & 0&0\\
%       0 & 0&1 \\
%       0 & 1& 0\\
%  \end{array}
%  \right],
%  $$
%  where $\mathcal{A}_f$ represent  the associated adjacency matrix between followers and $G_{ik}={\rm diag}(g_{ik})$ with $g_{ik}$  is the weight of the path from $i$th leader to $k$th follower. 
 
 The dynamics of leader UAVs and follower UAVs can be approximately described as the following second-order systems \cite{wang2018optimal,dong2016time}:
 \begin{equation}
  \begin{cases}
 \dot{p}_k(t)=v_k(t), \\
 \dot{v}_k(t)=\alpha_{p_0} p_k (t) + \alpha_{v_0} v_k(t),\\
 y_k(t)=p_k(t),
\end{cases}   
 \end{equation}
and
\begin{equation}
  \begin{cases}
 \dot{p}_i(t)=v_i(t), \\
 \dot{v}_i(t)=\alpha_{p_i} p(t) + \alpha_{v_i} v_i(t)+u_i(t),\\
 y_i(t)=p_i(t),
\end{cases}  
\end{equation}
 where $p_k(t)$  and $v_k(t)$ ($p_i(t)$ and $v_i(t)$) represent the position and velocity of the $k$th leader UAVs ($i$th follower UAVs), respectively. The damping constants of leaders are selected as: $\alpha_{p_0}=-0.6,\alpha_{v_0}=0$. The damping constants of followers are choose as: $\alpha_{p_1}=-1,\alpha_{v_1}=-1$, $\alpha_{p_2}=-0.5,\alpha_{v_2}=-1.2$, $\alpha_{p_3}=-1.2,\alpha_{v_3}=-1$, $\alpha_{p_4}=-0.8,\alpha_{v_4}=-0.5$, and $\alpha_{p_5}=-0.4,\alpha_{v_5}=-1.2$. Select the gains as $\mu_1=5$, $\mu_2=1$, and $\mu_3=7$, respectively. We choose the DoS attack periods as $[0.2+2k,1.86+2k)s$ for $k\in \mathbb{N}$, and the AAs given as $\chi_1=[0.01t\quad0.02t\quad0.02t]^{\mathrm{T}},\chi_2=[0.02t\quad0.01t\quad0.02t]^{\mathrm{T}},\chi_3=[0.02t\quad0.02t\quad0.02t]^{\mathrm{T}},\chi_4=[-0.01t\quad-0.02t\quad-0.02t]^{\mathrm{T}},\chi_5=[-0.02t\quad-0.01t\quad0.02t]^{\mathrm{T}}$.
 
   \begin{figure}[!]
  %\begin{minipage}[t]{1\linewidth}
  \centering
  \includegraphics[width=0.45\textwidth]{picnew/Upsilon.eps}
  \caption{The leader dynamics modeling of TL in Theorem 1: The shadow areas denote time intervals against DoS attacks.}
  \label{fig:figure9}
\end{figure}
 
 
  %\begin{figure*}[htbp]
  \begin{figure}[htbp]
  \centering
    \includegraphics[width=0.45\textwidth]{picnew/zx.eps}
  \caption{Distributed estimation performance of TL: the shadow areas denote time intervals against DoS attacks.}
  \label{fig:figure9}
   \label{fig:figure10}
\end{figure}

 Choose the initial values of updating parameters in (\ref{EQ15}) and (\ref{EQ48}) to be zero. Then select the initial leader's state as $p_6=[1\quad1 \quad 1]^{\mathrm{T}},v_6=[4 \quad 5\quad4]^{\mathrm{T}}$, $p_7=[4 \quad 1 \quad 2]^{\mathrm{T}},v_7=[5 \quad 6 \quad 6]^{\mathrm{T}}$, and $p_8=[6 \quad 5 \quad 8]^{\mathrm{T}},v_6=[4 \quad 6 \quad 5]^{\mathrm{T}}$, and randomly choose other initial values. According to Theorem 2 and Theorem 4, it yields that the state estimator gain of TL $G=\left[\begin{array}{cc}
     2.8483 & 0.2594 \\
     0.2594 & 3.0611
 \end{array}\right]$ and the gain of CPL $K_1=[-0.4142 \quad 	-0.6818]$, $K_2= [-0.6180 \quad 	-0.7173]$, $K_3=[-0.3620 \quad 	-0.6505]$, $K_4=[-0.4806	 \quad -0.9870]$, $K_5= [-0.6770	 \quad -0.7478]$, respectively.

 
 \begin{figure}[!]
  %\begin{minipage}[t]{1\linewidth}
  \centering
  \includegraphics[width=0.45\textwidth]{picnew/Delta.eps}
  \caption{The estimated solution of the output regulation equation in Theorem 3: The shadow areas denote time intervals against DoS attacks.}
  \label{fig:figure11}
\end{figure}
 
 
\begin{figure}[!]
  %\begin{minipage}[t]{1\linewidth}
  \centering
  \includegraphics[width=0.45\textwidth]{picnew/track/track3.eps}
  \caption{ Trajectories of the UAV swarm during $0 \sim 30s$: The hollow and solid blocks represent the start and end points of all UAVs, respectively.}
  \label{fig:figure12}
\end{figure}


 
  
    \begin{figure}[htbp]
  \centering 
  \includegraphics[width=0.45\textwidth]{picnew/ex.eps}
  \caption{The performance of our control scheme during the output containment.}
  \label{fig:figure16}
  \end{figure}
 

  
\begin{figure}[htbp]
  %\begin{minipage}[t]{1\linewidth}
  \centering
  \includegraphics[width=0.45\textwidth]{picnew/ebar.eps}
  \caption{The estimation error between TL and CPL: The blue shadow areas denote  the UUB bound by $\bar{d}$.}
  \label{fig:figure12}
\end{figure}

 From Fig. \ref{fig:figure9}, it depicts that  $\hat{\Upsilon}$  converges to $\Upsilon$ under DoS attacks, which implies that the molding error converges to $0$. The state estimator error  of TL are shown in Fig. \ref{fig:figure10}, which show that the performance of TL is good under the DoS attacks.  Fig. \ref{fig:figure11} verifies the validity of Theorem 3.
 Fig. 14 shows the trajectories of UAVs reference in the simulation within $t=30s$, where the initial positions and the final positions of UAVs are marked by unfilled squares and  filled squares, respectively. It can be seen that,  the trajectory of each follower stays in a small neighborhood around the dynamic convex hull spanned by the leaders after the moment $t=10s$, that is, the output containment is achieved, which can also valid by Fig. \ref{fig:figure16}.
 Fig. \ref{fig:figure12} shows the rationality of the upper bound of output containment errors.


% \begin{figure}[!htbp]
% %\begin{minipage}[t]{1\linewidth}
% \centering
% \includegraphics[width=0.6\textwidth]{4Ag.pdf}
% \caption{Time-varying directed communication topology among all agents}
% \label{fig:figure1}
% \end{figure}



%{\color{blue}
%\begin{figure}[htbp]
%\centering
%\subfigure[Performance of observer w.r.t. the leader]{
%\begin{minipage}[t]{0.475\textwidth}
%\centering
%\includegraphics[width=0.85\textwidth]{pic/pobs.eps}
%%\caption{fig1}
%\end{minipage}\label{fig:figure2:1}
%}
%%\hspace{-0.1in}
%\subfigure[Performance of observer w.r.t. the first leader]{
%\begin{minipage}[t]{0.475\textwidth}
%\centering
%\includegraphics[width=0.85\textwidth]{pic/vobs.eps}
%%\caption{fig2}
%\end{minipage}\label{fig:figure2:2}
%}\\%
%\centering
%\caption{Performance of two observers}
%\label{fig:figure2}
%\end{figure}














\section{Conclusion}
The distributed resilient output containment problem of heterogeneous multi-agent systems against composite attacks has been solved in this work. Inspired by the concept of hierarchical protocol, a TL, which is resilient to most attacks (including the CAs, FDI attacks, and AAs), is provided to decouples the defense strategy into two sub-problems. The first sub-problem is the defense against DoS attacks on the TL and the second sub-problem is the defense against unbounded AAs on the CPL.
For the first problem, we used double-TL to solve the modeling error and defence DoS attacks: the first TL use
the distributed observers to remold the leader dynamics. Then, the distributed  estimators is provided to estimate the followers states under DoS attacks on the second TL.  For the second problem, the double-CPL is provided to achieve the output containment under AAs: the first CPL use output regulation equation solver to deal with the output containment problem, and adaptive distributed control scheme is provided  to solve the unbounded AAs on the second CPL. Finally, we have prove the UUB of output containment error is UUB under the composite attacks  and give the error bound explicitly. The simulations illustrate the effectiveness  of these proposed methods.




 

\

% Proof: Consider the Lyapunov function candidate
% $$
% V_{1}=\frac{1}{2} \sum_{i=1}^{N} \xi_{i}^{T} P \xi_{i}+\sum_{i=1}^{N} \sum_{j=1, j \neq i}^{N} \frac{\left(c_{i j}-\alpha\right)^{2}}{8 \kappa_{i j}}
% $$
% where $\alpha$ is a positive constant that is to be determined later. Evidently, $V_{1}$ is positive definite. The time derivative of $V_{1}$ along the trajectory of (5) is given by
% $$
% \begin{aligned}
% \dot{V}_{1}=& \sum_{i=1}^{N} \xi_{i}^{T} P \dot{\xi}_{i}+\sum_{i=1}^{N} \sum_{j=1, j \neq i}^{N} \frac{c_{i j}-\alpha}{4 \kappa_{i j}} \dot{c}_{i j} \\
% =& \sum_{i=1}^{N} \xi_{i}^{T} P A \xi_{i}+\sum_{i=1}^{N} \xi_{i}^{T} P B K \sum_{j=1}^{N} c_{i j} a_{i j}\left(\tilde{x}_{i}-\tilde{x}_{j}\right) \\
% &+\sum_{i=1}^{N} \sum_{j=1, j \neq i}^{N} \frac{c_{i j}-\alpha}{4 \kappa_{i j}} \dot{c}_{i j}
% \end{aligned}
% $$
% Since $a_{i j}=a_{j i}$ and $c_{i j}(t)=c_{j i}(t)$, it can be easily verified that
% $$
% \begin{array}{rl}
% \sum_{i=1}^{N} \xi_{i}^{T} & P B K \sum_{j=1}^{N} c_{i j} a_{i j}\left(\tilde{x}_{i}-\tilde{x}_{j}\right) \\
% =&-\frac{1}{2} \sum_{i=1}^{N} \sum_{j=1}^{N} c_{i j} a_{i j}\left(\xi_{i}-\xi_{j}\right)^{T} \Gamma\left(\tilde{x}_{i}-\tilde{x}_{j}\right)
% \end{array}
% $$

\bibliography{PIDFR}
\end{document}\grid