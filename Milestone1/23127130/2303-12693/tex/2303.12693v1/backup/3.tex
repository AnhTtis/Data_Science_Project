%\documentclass[12pt,thmsb,a4paper]{article}
\documentclass[12pt,a4paper]{article}
%%%%%%%%%%%%%%%%%%%%%%%%%%%%%%%%%%%%%%%%%%%%%%%%%%
\usepackage{mathrsfs}
\usepackage{graphics}
\usepackage{mathptmx}
\usepackage{times}
\usepackage{epstopdf}
\usepackage{color}
\usepackage{enumerate}
\usepackage{float}
%\usepackage{AMSFonts}
\usepackage{indentfirst}
\usepackage{multirow}
\usepackage[noadjust]{cite}
%\usepackage{natbib}
%\usepackage{doi}
%\usepackage{natbib}
%\usepackage{cases}
%\usepackage{mathbbold}
%\usepackage{amsthm,amsmath,amssymb}
\usepackage{amsthm,amsmath,amssymb,amsfonts}
\usepackage{psfrag}
\usepackage{epsfig}
\usepackage{graphicx,subfigure}
\usepackage{multirow}
\usepackage{setspace}
\usepackage{caption}
\usepackage{booktabs} %需要加载宏包{booktabs}
%\usepackage[linesnumbered]{algorithm}
\usepackage{algorithm}
\usepackage[noend]{algpseudocode}
\usepackage{dsfont}

\usepackage{url}
\usepackage{tikz}
\usepackage{subeqnarray}
\usepackage{cases}
\usepackage{color}

\usepackage[shortlabels]{enumitem}
\usepackage{exscale}
\usepackage{relsize}
\usepackage{bm}
\usepackage{enumerate}
\usepackage{enumitem}




\baselineskip=12pt
\normalbaselineskip=\baselineskip
\renewcommand{\baselinestretch}{1.2}
\hoffset=-2.75cm
\topmargin=-2cm
\textheight=25cm
\textwidth=18cm
\footskip=0.8cm
\oddsidemargin=1.8cm
\evensidemargin=1.8cm
\marginparwidth=1.8cm
\parindent=15pt
\setlength{\baselineskip}{24pt}
\renewcommand{\thefootnote}{\fnsymbol{footnote}}
%\newtheorem{theorem}{Theorem}
%\newtheorem{definition}{Definition}
%\newtheorem{corollary}{Lemma}
%\newtheorem{example}{Example}

%\newtheorem{hypothesis}{Hypothesis}
%\newtheorem{lemma}{Lemma}
%\newtheorem{remark}{Remark}
%\newtheorem{algorithm}{Algorithm}
%\newtheorem{proposition}{Proposition}
%\newtheorem{myDef}{Definition}
\newtheorem{myTheo}{Theorem}


%\newtheorem{thm}{Theorem}[section] %如果不采用章节号做前缀,则不用[section]
\newtheorem{myDef}[myTheo]{Definition} %这句定义使得环境和thm共享编号
\newtheorem{lemma}[myTheo]{Lemma} %这句定义使得lem环境和thm共享编号
\newtheorem{myCollo}[myTheo]{Corollary}
\newtheorem{remark}[myTheo]{Remark}
%\newtheorem{lemma}{Lemma}
\newtheorem{myPro}[myTheo]{Proposition}
\newtheorem{assumption}[myTheo]{Assumption}
\newtheorem{property}[myTheo]{Property}
%\newtheorem{algorithm}{Algorithm}
\def\proof{\par{ \textbf{Proof}}. \ignorespaces}
\def\endproof{\vbox{\hrule height0.6pt\hbox{   \vrule height1.3ex width0.6pt\hskip0.8ex
   \vrule width0.6pt}\hrule height0.6pt
  }}

\begin{document}

\author{\textbf{Xin Gong}\footnotemark[2]\ \ \ \ \ \textbf{Tieniu Wang}\footnotemark[3]\ \ \ \ \ \textbf{Yukang Cui}\footnotemark[3]}
%Double-Integrator \ \ \ \ \ \textbf{James Lam}\footnotemark[2]
\title{{\LARGE \textbf{Distributed Prescribed-time Consensus Observer for High-order Integrator Multi-agent Systems on Directed Graphs}}\vspace{0.5cm}}
\date{\today}       %日期
\maketitle
%
\begin{abstract}
\ \ \ This brief deals with the distributed consensus observer design problem for high-order integrator multi-agent systems on directed graphs, which intends to estimate accurately the leader state in a prescribed time interval. A new kind of distributed prescribed-time observers (DPTO) on directed graphs is first formulated for the followers, which is implemented in a cascading manner. Then, the prescribed-time zero-error estimation performance of the above DPTO is guaranteed for both time-invariant and time-varying directed interaction topologies, based on strictly Lyapunov stability analysis and mathematical induction method. Finally, the practicability and validity of this new distributed observer are illustrated via a numerical simulation example.

\vspace{0.2cm}

\noindent \textbf{Keywords:} Consensus observer, Directed graphs, High-order Multi-agent systems, Prescribed-time stability
% Nano-quadcopters formation,
% Linear matrix inequality, Linear programming, Leader-follower consensus, Multi-agent systems, Positive consensus, Positive linear systems, Robust consensus, Static output-feedback
\end{abstract}

%\bigskip \footnotetext[1]{Corresponding author.}

\footnotetext[2]{Department of Mechanical Engineering, The University of Hong Kong, Pokfulam Road, Hong Kong (Email: gongxin@connect.hku.hk).}
\footnotetext[3]{College of Mechatronics and Control Engineering, Shenzhen University, Shenzhen, 518060, China (Email: cuiyukang@gmail.com).}
%liujinrjason@connect.hku.hk; yaminwang1994@hotmail.com;
%;  james.lam@hku.hk
%\footnotetext[3]{School of Engineering Sciences, University of Southampton, Southampton SO17 1BJ, U.K. (Email: Z.Shu@soton.ac.uk (Z. Shu)).}
%\footnotetext[3]{%
%School of Automation, Nanjing University of Science and Technology, Nanjing
%210094, Jiangsu, China (Email:baoyongzhang@njust.edu.cn).}
%\footnotetext[2]{%
%This work was partially supported by GRF HKU 7140/11E.}

%%%%%%%%%%%%%%%%%%%%%%%%%%%%%%%%%%%%%%%%%%%%%%%%%%%%%%%%%%%%%%%%%%%%%%%%%%%%%%%%%%%%%%%%%%%%%%%%%%%%%%%%%%%%%%%%%%%%%%%%%%%%%%%%%%%%%%%%%%%%%%

Since the state $x_i$ is unknown for other agent, we design a new state observer which design as (68)~(70) to observe the state $x_i$, which used output feedback control. This observer takes into account the unbounded attack that the agents may be suffered.
Next, we show that the observer state error converges exponentially to 0.

Consider the observer takes into account the unbounded attack as follows
\begin{align}
& \dot{\hat{x}}_i=A_i \hat{x}_i +B_i(u_i +\hat{d}_i)+G_i(C_i\hat{x}_i - y_i) \\ 
&\hat{d}_i=\frac{B_i^T C_i^T P_i \tilde{y}_i}{\left\lVert \tilde{y}_i^T P_i C_i B_i \right\rVert +e^{-\beta_i t}} \hat{\rho_i} \\ \label{EQ65}
&\dot{\hat{\rho}}_i=\left\lVert \tilde{y}_i^T P_i C_i B_i \right\rVert
\end{align}
Define the estimator state error $\tilde{x}_i= x_i-\hat{x}_i$, from (\ref{EQ2}) and (\ref{EQ58}),  we obtain that
\begin{equation}
\begin{aligned}
        \dot{\tilde{x}}_i  
        &= A_i x_i +B_i(u_i +d_i) - A_i x_i -B_i(u_i +\hat{d}_i) -G(C_i \hat{x_i} - y_i)\\
        &=(A_i+ G_i C_i) \tilde{x}_i+ B_i \tilde{d}_i.
\end{aligned}
\end{equation}
where $\tilde{d}_i = d_i - \hat{d}_i$.

Define the following Lyapunov function:
\begin{equation}
    V_i^x =\tilde{x}_i^T C_i^T P_i C_i \tilde{x}_i,
\end{equation}
and its times derivative is presented as follows:
\begin{equation}\label{EQ74}
    \dot{V}_i^x = 2 \tilde{x}_i^T C_i^T P_i C_i (A_i+ G_i C_i) \tilde{x}_i + 2 \tilde{y}_i^T P_i C_i B_i \tilde{d}_i,
\end{equation}
%}
%{\color{red}
noting that
\begin{equation}
\begin{aligned}
\tilde{y}_i^T P_i C_i B_i\tilde{d}_i
        &= \tilde{y}_i^T P_i C_i B_i d_i -\frac{\left\lVert \tilde{y}_i^T P_i C_i B_i \right\rVert^2}{ \left\lVert \tilde{y}_i^T P_i C_i B_i \right\rVert +e^{-\beta_i t}} \hat{\rho_i} \\
& \leq  \left\lVert \tilde{y}_i^T P_i C_i B_i\right\rVert \left\lVert d_i \right\rVert - \frac{\left\lVert \tilde{y}_i^T P_i C_i B_i \right\rVert^2  }{\left\lVert \tilde{y}_i^T P_i B_i \right\rVert + e^{-\beta_i t}} \hat{\rho_i} \\
& \leq \frac{\left\lVert \tilde{y}_i^T P_i C_i B_i \right\rVert ^2 (  \left\lVert d_i \right\rVert-  \hat{\rho}_i)+ \left\lVert \tilde{y}_i^T P_i C_i B_i \right\rVert\left\lVert d_i \right\rVert e^{-\beta_i t}}{\left\lVert \tilde{y}_i^T P_i C_i B_i \right\rVert +e^{-\beta_i t} } .
\end{aligned}
\end{equation}
Noting that $d\left\lVert d_i \right\rVert/dt $ is bounded,  so, $\left\lVert d_i \right\rVert e^{-\beta_i t} \rightarrow 0$ . Choose $\left\lVert \tilde{y}_i^T P_i C_i B_i \right\rVert \geq d\left\lVert d_i \right\rVert/dt $, that is, $d\left\lVert d_i \right\rVert/dt - \dot{\hat{\rho}}_i < 0$. Then , $ \exists t_2 > 0$ such that for all $t \geq t_2$ , we have 
\begin{equation}\label{EQ76}
    \tilde{y}_i^T P_i C_i B_i \tilde{d}_i < 0  .
\end{equation}
%
Substituting (\ref{EQ76}) into (\ref{EQ74}) yields
\begin{equation}\label{EQ77}
\begin{aligned}
      \dot{V}_i^x 
      & \leq 2 \tilde{x}_i^T C_i^T P_i C_i (A_i+ G_i C_i) \tilde{x}_i \\
      &\leq 2\sigma_{\rm max}(A_i + G_iC_i) V_i^x,
\end{aligned}
\end{equation}
Solving inequality (\ref{EQ77}), we get
\begin{equation}
    V_i^x(t) \leq V(t_2)e^{2\sigma_{\rm max}(A_i + G_i C_i)(t-t_2)} ,\forall t \geq t_2.
\end{equation}
Since ($C_i,A_i$) is detectable, let $G_i$ be such that $A_i +G_i C_i$ is Hurwitz. It is obvious that $V_i^x$ converges to 0 exponentially.











%In previous works \cite{satunin2014multi, dongITCST2015, MCYY2017, JZSZM2017,lewis2013cooperative,mesbahi2010graph,ren2010distributed}, two types of consensus: 1) leaderless and 2) leader-following consensus w.r.t. a leader agent have been widely studied, where all agents reach an agreement on specific states of interests. 
%Compared with the above two types of consensus, a more complex, practical yet challenging issue named formation-containment (FC) control problem has become an emerging popular topic with multiple interactive and non-autonomous leaders considered. When it comes to the FC problem, it signifies such a role-play scenario that the leaders achieve a predefined formation configuration centered by a predefined trajectory, while the followers converge into the above formation pattern formed by leaders. Image the following escort scenario: For a fleet convoy of ships sailing across the Aden Gulf, the warships (leaders) equipped with advanced weapons and armors should achieve a compact configuration cooperatively to detect and expel the Somali pirates; the merchant ships (followers)  without essential defenses are required to enter the moving protective hull formed by the warships. The above FC mission can also be extended to numerous applications with multiple autonomous leaders, such as combined UAV operations in smart agriculture \cite{maddikunta2021unmanned} and collective transportation of platoon systems \cite{hu2020cooperative}.
% , including the combat aerial vehicles'  hybrid formation of lead aircraft and its wingmen  \cite{humphreys2015optimal}, cooperative penetration of multiple missiles \cite{wang2017composite}, and cooperative transportation of multi-robot systems \cite{alonso2017multi}.


%$1$%See $\url{http://www.baidu.com}$.





\bibliographystyle{IEEEtran}


\bibliography{PIDFR}

\end{document}