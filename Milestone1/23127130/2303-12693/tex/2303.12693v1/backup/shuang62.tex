%Editor: We have provided a PDF that shows the tracked changes in your file as in a Word document. This method makes it easier for you to match the edited file with your original file and make any necessary edits to your file in your LaTeX program. Please let us know if you require further assistance.

\documentclass[letterpaper, journal, twoside, 10pt,twocolumn]{support/IEEEtran}
%\documentclass[letterpaper, 12pt, journal, twoside]{support/IEEEtran}
\usepackage[fleqn]{amsmath}
\usepackage{times}
\usepackage[pdftex]{graphicx}
\usepackage{subfigure}
\usepackage{amsmath,amssymb,amsopn,amstext,amsfonts}
\usepackage{cancel}
\usepackage[noadjust]{cite}
\usepackage{soul}
\usepackage{caption}
\captionsetup{font={small}}

\captionsetup[figure]{labelfont={},textfont={}}


\usepackage{balance}
\usepackage{color}
\usepackage{mathtools}
% \usepackage{algorithm}
% \usepackage{algorithmic}
\usepackage{bm}
%\newtheorem{theorem}{Theorem}
\usepackage{ diagbox}
\usepackage{float}
\usepackage{epstopdf}
\usepackage{url}
\usepackage{multirow}
\usepackage{tikz}
\usepackage{subeqnarray}
\usepackage{cases}
\usepackage{booktabs}
\usepackage[linkcolor=black,citecolor=black,urlcolor=black,colorlinks=true]{hyperref}
\usepackage{algorithm}
\usepackage[noend]{algpseudocode}
\newtheorem{myTheo}{Theorem}
%\newtheorem{thm}{Theorem}[section] %???????????,???[section]
\newtheorem{myDef}{Definition} %??????defn???thm????
\newtheorem{Lemma}{Lemma} %??????lem???thm????
\newtheorem{myCollo}{Corollary}
\newtheorem{remark}{Remark}
%\newtheorem{Lemma}{Lemma}
\newtheorem{myPro}{Proposition}
\newtheorem{assumption}{Assumption}
\newtheorem{example}{Example}
\soulregister\cite7
\soulregister\citep7
\soulregister\citet7
\soulregister\ref7
\soulregister\it7
\soulregister\pageref7

\bibliographystyle{support/IEEEtran}

\newcommand\px{\mathrel{/\mkern-5mu/}}  %??
\newcommand{\ann}[1]{%
    \begin{tikzpicture}[remember picture, baseline=-0.75ex]%
        \node[coordinate] (inText) {};%
    \end{tikzpicture}%
    \marginpar{%
        \renewcommand{\baselinestretch}{1.0}%
        \begin{tikzpicture}[remember picture]%
            \definecolor{orange}{rgb}{1,0.5,0}%
            \draw node[fill=red!20,rounded corners,text width=\marginparwidth] (inNote){\footnotesize#1};%
    \end{tikzpicture}%
    }%
    \begin{tikzpicture}[remember picture, overlay]%
        \draw[draw = orange, thick]
            ([yshift=-0.2cm] inText)
                -| ([xshift=-0.2cm] inNote.west)
                -| (inNote.west);%
    \end{tikzpicture}%
}%

\graphicspath{{figures/}}
\DeclareGraphicsExtensions{.pdf,.png,.jpg,.eps}
\IEEEoverridecommandlockouts
%\overrideIEEEmargins

\title{\LARGE \bf Resilient Output Containment Control of Heterogeneous Multiagent Systems Against Composite Attacks: A Novel Digital Twin Approach}

%\title{Distributed Optimization in Prescribed-Time: Theory and Experiment}%
\author{
  \vskip 1em
  {Yukang Cui, \emph{Member, IEEE},
  Lingbo Cao,
  Xin Gong, \emph{Member, IEEE}
  }

  \thanks{
    This work was partially supported by the National Natural Science Foundation of China under Grant 61903258, Guangdong Basic and Applied Basic Research Foundation 2022A1515010234 and the Project of Department of Education of Guangdong Province 2022KTSCX105. %(\emph{Corresponding author: Yukang Cui.}) %the National Natural Science Foundation of China under Grant 61903258

Y. Cui and L. Cao are with the College of Mechatronics and Control Engineering, Shenzhen University, Shenzhen, 518060, China (e-mail: {\tt\small cuiyukang,lingbcao@gmail.com}).

X. Gong is with the Department of Mechanical Engineering, The University of Hong Kong, Pokfulam Road, Hong Kong (e-mail: {\tt\small gongxin@connect.hku.hk}).





  
%J. He is with the Department of Mechanical Engineering, The University of Hong Kong, Pokfulam Road, Hong Kong (e-mail: {\tt\small esmehe@connect.hku.hk}). 

%X. Gong is with the Department of Mechanical Engineering, The University of Hong Kong, Pokfulam Road, Hong Kong, and also with the College of Mechatronics and Control Engineering, Shenzhen University, Shenzhen 518060, China. (e-mail: {\tt\small gongxin@connect.hku.hk}).
%China, and also
%with the Department of Mechanical Engineering, University of Hong Kong,
%Hong Kong
    
  }
%\thanks{$^{*}$ means the corresponding author.}
}

%\maketitle
%\author{}%\vspace{-0.0cm}
%%\thanks{This work was partially supported by.}% <-this % stops a space
%\thanks{$^{*}$These authors contribute equally and share the first authorship.}
%\thanks{$^{1}$Author is with the Group Robotics with Intelligent Planning (GRIP) Lab, Department of Mechanical Engineering, University of Hong Kong, Hong Kong,
%   {\tt\small gongxin@connect.hku.hk}}
%\thanks{Digital Object Identifier (DOI): see the top of this page.}
%\vspace{-0.5cm}}

% The note headers
%\markboth{Journal of \LaTeX\ Class Files,~Vol.~14, No.~8, August~2015}%
%{Shell \MakeLowercase{\textit{et al.}}: Bare Demo of IEEEtran.cls for IEEE Journals}
\markboth{IEEE Transactions on ...}{GONG \MakeLowercase{\textit{et al.}}: Resilient Output Containment Control of Heterogeneous MAS}%{He \MakeLowercase{\textit{et al.}}: Resilient Path Planning of UAVs against Covert Attacks on UWB Sensors}



\begin{document}
  \maketitle
  \begin{abstract}
 This work studies the distributed resilient output containment control of heterogeneous multiagent systems against composite attacks, including denial-of-services (DoS) attacks, false-data injection (FDI) attacks, camouflage attacks, and actuation attacks. Inspired by digital twins, a twin layer (TL) with higher security and privacy is used to decouple the above problem into two tasks: defense protocols against DoS attacks on TL and defense protocols against actuation attacks on cyber-physical layer (CPL). First, considering modeling errors of leader dynamics, we provide distributed observers to reconstruct the leader dynamics for each follower on TL. Second, distributed estimators are used to estimate follower states by the reconstructed leader dynamics under DoS attacks on TL. {\color{blue}Third, according to the reconstructed leader dynamics, we develop decentralized solvers that calculate the output regulator equations on CPL.} Fourth,
decentralized adaptive attack-resilient control schemes that resist unbounded actuation attacks are provided on CPL. Furthermore, we apply the above control protocols to prove that the followers can achieve uniformly ultimately bounded (UUB) convergence, and the upper bound of the UUB convergence is determined explicitly. Finally, two simulation examples are provided to show the effectiveness of the proposed control protocols.
\end{abstract}
\begin{IEEEkeywords}
Composite attacks, Containment, Directed graphs, High-order multiagent systems, Twin layer
% Periodic positive systems, hyper-pyramid,
% reachable set estimation, S-procedure, state-feedback control.
%Formation-containment control,  high-order multi-agent systems,  observer-type protocols,  time-varying formation configuration
\end{IEEEkeywords}
\section{Introduction}
\IEEEPARstart{D}{ISTRIBUTED} cooperative control of multiagent systems (MASs) has attracted extensive attention over the last decade due to its broad applications in satellite formation \cite{scharnagl2019combining}, mobile multirobots \cite{wang2017adaptive}, and smart grids \cite{ansari2016multi}.
%  multi-robot cooperative control, ,
%  mobile multirobot [1], distributed microgrid [2], and vehicle platoon 
% physics, biology, social activity, and engineering. 
Consensus problems are one of the most common phenomena in the cooperative control of MASs. Consensus problems can be divided into two categories: leaderless consensus \cite{fax2004information,ren2007information,olfati2007consensus,wieland2011internal} and leader-follower consensus \cite{hong2013distributed,zhao2016distributed,su2011cooperative,su2012cooperative}.
{\color{blue}
%\cite{,,kim2010output,wang2010distributed,lunze2012synchronization,yaghmaie2016output}
Their control schemes drive all agents to reach an agreement on some certain variables of interest. In recent years, researchers have introduced multiple leaders into consensus problems, and the control schemes are to drive the certain variables of all followers into the convex hulls formed by the leaders.} This phenomenon is also known as the containment problem \cite{haghshenas2015,zuo2017output,zuo2019,zuo2020,ma2021observer}. Haghshenas et al. \cite{haghshenas2015} and Zuo et al. \cite{zuo2017output} solved the containment control problem of heterogeneous linear MASs with output regulator equations. Then, Zuo et al. \cite{zuo2019} considered the containment control problem of heterogeneous linear MASs against camouflage attacks. Moreover, Zuo et al. \cite{zuo2020} presented an adaptive compensator against unknown actuator attacks. Lu et al. \cite{lu2021} considered containment control problems against false-data injection (FDI) attacks. Ma et al. \cite{ma2021observer} solved the containment control problem against denial-of-service (DoS) attacks by developing event-triggered observers. 
%Editor: Please ensure that the intended meaning has been maintained in the following edit.
However, the above containment control works only considered one type of attack.
Considering the vulnerability of MASs, the situation may be more complex in reality, that is, these systems may face multiple attacks at the same time. Therefore, this work considers the multiagent resilient output containment control problem against composite attacks, including DoS attacks, FDI attacks, camouflage attacks and actuation attacks.


In recent years, digital twins \cite{zhu2020,gehrmann2019digital} have been widely used in various fields, including industry \cite{abburu2020cognitwin}, aerospace
%\cite{salinger2020hardware}
\cite{liu2022intelligent}, and medicine \cite{voigt2021digital}. Digital twin technology involves the real-time mapping of objects in virtual space. Inspired by this technology, in this work, a double-layer control structure with a twin layer (TL) and cyber-physical layer (CPL) is designed. The TL is a virtual mapping of the CPL that has the same communication topology as the CPL and can interact with the CPL in real time. Moreover, the TL has high privacy and security and can effectively address data manipulation attacks in communication networks, such as FDI attacks and camouflage attacks. Therefore, the main work of this paper is to design a corresponding control framework that resists DoS attacks
and actuation attacks. Since the TL is only affected by DoS attacks, this paper can be divided into two parts: one is to resist DoS attacks on the TL and the other is to resist actuation attacks on the CPL.

Resilient control protocols of MASs against DoS attacks were investigated in \cite{fengz2017,zhangd2019,yed2019,yang2020,deng2021}. Feng et al. \cite{fengz2017} considered the leader-follower consensus through its latest state and leader dynamics during  DoS attacks. Yang et al. \cite{yang2020} designed a dual-terminal event-triggered mechanism against DoS attacks for linear leader-following MASs.
However, the methods proposed in the above works need to know the leader dynamics for each follower, which is unrealistic in some cases. To address this issue, Cai et al. \cite{cai2017} and Chen et al. \cite{chen2019} used adaptive distributed observers to estimate the leader dynamics and states for an MAS with a single leader. Deng et al. \cite{deng2021} proposed a distributed resilient learning algorithm for learning unknown dynamic models. However, these works did not consider leader estimation errors and DoS attacks simultaneously. Thus, motivated by the above works, {\color{blue}we consider a TL with modeling errors of the leader dynamics, which is inevitable in online modeling.} This is an interesting and challenging problem because modeling errors of leader dynamics affect follower state estimation errors and output containment errors. {\color{blue}To address this problem, we formulate a distributed TL} that handles modeling errors of leader dynamics and DoS attacks. On this TL, distributed observers are established to reconstruct the leader dynamics of each follower. Then, distributed state estimators are used to reconstruct follower states under DoS attacks.
 
 



Resilient control protocols of MASs against actuation attacks were studied in \cite{chen2019,zuo2020,lu2021,de2014resilient,xie2017decentralized,jin2017adaptive}. One pioneering work \cite{de2014resilient} presented a distributed resilient adaptive control scheme against attacks with exogenous disturbances and interagent uncertainties. Xie et al. \cite{xie2017decentralized} considered distributed adaptive fault-tolerant control problem for uncertain large-scale interconnected systems with external disturbances and actuation attacks. Moreover, Jin et al. \cite{jin2017adaptive} provided an adaptive controller that guaranteed that closed-loop dynamic systems against time-varying adversarial sensor and actuator attacks could achieve uniformly ultimately bounded (UUB) performance. However, the above works  only consider bounded actuation attacks (or disturbances), which cannot maximize their destructive capacity indefinitely. Recently, Zuo et al. \cite{zuo2020} provided an adaptive control scheme for MASs against unbound actuation attacks.
{\color{blue}Based on the above works, we present a decentralized CPL to address the output containment problem against unbounded actuation attacks. On this CPL, we solve the output regulator equation with reconstructed leader dynamics to address the output containment of heterogeneous MASs. Then, decentralized attack-resilient protocols with adaptive compensation signals are used to resist unbounded actuation attacks.}

%Moreover, in contrast to \cite{zuo2020}, our adaptive compensation signal includes a soft-sign function, and the containment error bound can be determined explicitly.


%Editor: Please ensure that the intended meaning has been maintained in the following edit.
The above discussion indicates that the output containment control of heterogeneous MASs with TL modeling error of leader dynamics under the composite attacks has  not yet been solved or studied. 
Thus, in this work, we provide a hierarchical control scheme to address the above problem.
Moreover, our main contributions can be summarized as follows:
\begin{enumerate}
  \item In contrast to previous works, our work considers TL modeling errors of leader dynamics. Thus, the TL needs to consider both modeling errors  and DoS attacks, and the CPL needs to consider both output regulator equation errors and actuation attacks, thereby increasing the complexity and practicality of the analysis. {\color{blue}Moreover, compared with previous work \cite{deng2021}, in which the state estimator requires some time to estimate the leader dynamics, in this work, the leader dynamic observers and follower state estimators on the TL work synchronously.}
  \item A double layer control structure is introduced, and a TL is established in the control framework. The TL has higher security than the CPL, that is, it can effectively defend against various types of attacks, including FDI attacks and camouflage attacks. Moreover, due to the existence of the TL, the resilient control scheme can be decoupled into defense against DoS attacks on the TL and defense against potentially unbounded actuation attacks on the CPL.
  \item Compared with \cite{zuo2020}, our work considers the composite attacks and employs an adaptive compensation signal against unbounded actuation attacks that includes UUB convergence and chattering-free characteristics. Moreover, the containment output error bound of the UUB performance is given explicitly.
\end{enumerate}




\noindent\textbf{Notations:}
In this work, $I_r$ is the identity matrix with compatible dimensions.
$\boldsymbol{1}_m$ (or $\boldsymbol{0}_m$) denotes a column vector of size $m$ with values of $1$ (or $0$, respectively). Denote the index set of sequential integers as $\textbf{I}[m,n]=\{m,m+1,\ldots~,n\}$, where $m<n$ are two natural numbers. Define the set of real numbers and the set of natural numbers as $\mathbb{R}$ and $\mathbb{N}$, respectively. For any matrix $A\in \mathbb{R}^{M\times N}$, ${\rm vec }(A)= {\rm col}(A_1,A_2,\dots,A_N)$, where $A_i\in \mathbb{R}^{M} $ is the $i$th column of $A$.
Define ${\rm {\rm blkdiag}}(A_1,A_2,\dots,A_N)$ as a block diagonal matrix whose principal diagonal elements are equal to the given matrices $A_1, A_2, \dots, A_N$. $\sigma_{\rm min}(A)$, $\sigma_{\rm max}(A)$ and $\sigma(A)$ denote the minimum singular value, maximum singular value and spectrum of matrix $A$, respectively. $\lambda_{\rm min}(A)$ and $\lambda_{\rm max}(A)$ represent the minimum and maximum eigenvalues of $A$, respectively. Herein, $||\cdot||$ denotes the Euclidean norm, and $\otimes$ denotes the Kronecker product.
%; moreover, $A>0$ means that $\lambda_1(A)>0$.
%For a time-varying function $x(t): \mathbb{R}_{\geq 0 }\mapsto \mathbb{R}$, denote that $\sup_{t\in [t_0, t_1]} x(t) $ and $\inf_{t\in [t_0, t_1]} x(t)$ as the upper bound and lower bound of $x(t)$ over the time interval $[t_0, t_1]$, respectively. Moreover, denote that $\|x(t)\|_{[t_0, t_1]} =\sup_{t\in [t_0, t_1]} \|x(t)\| $. Define that $L_{\infty}:=\{x(t)|x(t): \mathbb{R}_{\geq 0 }\mapsto \mathbb{R}^n,\ \|x(t)\|_{[t_0, t_1]}<\infty\}$. In the following sections, $x(t) \in L_{\infty}$, $t\in [t_0, t_1]$, represents that the variable $x$ is uniformly bounded over $[t_0, t_1]$.   %$A>eq 0$ (or $A> 0$) denotes that $A$ is a nonnegative matrix (positive matrix, respectively), which means all elements of $A$ are nonnegative (positive, respectively).
 %${\rm span}(x)$ denotes the span vector of a given vector $x=[p_1, p_2,\ldots~, p_n]^{\mathrm{T}}\in \mathbb{R}^n$.

\label{introduction}


\section{Preliminaries}\label{section2}

%\subsection{Notations}




{\color{black}
\subsection{Graph Theory}
A directed graph is used to represent interactions among agents. For an MAS with $n$ agents, the graph
$\mathcal{G}$ can be expressed by $(\mathcal{V}, \mathcal{E}, \mathcal{A} )$, where $\mathcal{V}=\{1, 2, \ldots~, N \}$ is the node set,
$\mathcal{E} \subset \mathcal{V} \times \mathcal{V}=\{(v_j,\ v_i)\mid v_i,\ v_j \in \mathcal{V}\}$ is the edge set, and $\mathcal{A}=[a_{ij}] \in \mathbb{R}^{N\times N} $ is the associated adjacency matrix.
The weight of edge $(v_j,\ v_i)$ is denoted by $a_{ij}$. if $(v_j, \ v_i) \in \mathcal{E}$, $a_{ij} > 0$,  and otherwise, $a_{ij} = 0$ . The neighbor set of node $v_i$ is represented by $\mathcal{N}_{i}=\{v_{j}\in \mathcal{V}\mid (v_j,\ v_i)\in \mathcal{E} \}$. Define the Laplacian matrix as
$L=\mathcal{D}-\mathcal{A}  \in \mathbb{R}^{N\times N}$,
where $\mathcal{D}={\rm blkdiag}(d_i) \in \mathbb{R}^{N\times N}$ and $d_i=\sum_{j \in \mathcal{F}} a_{ij}$
is the weight in-degree of node $v_i$.
%A directed graph is \emph{strongly connected} if there is a path from $v_i$ to $v_j$ for any pair of nodes $(v_i,v_j)$.
}
\subsection{Some Useful Lemmas and Definitions}
\begin{myDef}\label{def41}
  For the $i$th follower, the system achieves containment if there exists a series of $\alpha_{\cdot i}$ that satisfy $\sum _{k \in \mathcal{L}} \alpha_{k i} =1$, thereby ensuring that the following equation holds:
   \begin{equation}
     {\rm  lim}_{t\rightarrow \infty } \left(y_i(t)-\sum _{k\in \mathcal{L}} \alpha_{k i}y_k(t)\right)=0,
   \end{equation}
   where $i \in \textbf{I}[1, N]$.
\end{myDef}
\begin{myDef}{\cite{khalil2002nonlinear}}
The signal $x(t)$ is UUB with the ultimate bound $b$ if there exist $b>0$ and $c>0$ that are independent of $t_0 \geq 0$ and for every $a \in (0, c)$ if there exists $T = T(a, b) \geq 0$ , independent of $t_0$, such that
 \begin{equation}
     ||x(t)||\leq a \Rightarrow ||x(t)|| \leq b,~\forall t \in [t_0+T,\infty).
 \end{equation}
\end{myDef}
\begin{Lemma}[Bellman-Gronwall Lemma \cite{lewis2003}] \label{Bellman-Gronwall Lemma 2}
    Assume that $\Phi : [T_a,T_b] \rightarrow \mathbb{R}$ is a nonnegative continuous function,  $\alpha: [T_a, T_b] \rightarrow \mathbb{R}$ is a continuous function, $\kappa \geq 0$ is a constant, and
    \begin{equation}
    \Phi(t) \leq \kappa + \int_{0}^{t} \alpha (\tau)\Phi(\tau) \,{\rm d}\tau,~t \in [T_a,T_b];
    \end{equation}
    then, we have 
    $$\Phi(t) \leq \kappa \mathrm{e}^{\int_{0}^{t} \alpha(\tau)\,{\rm d}\tau }$$ for all $t \in [T_a, T_b]$.
\end{Lemma}
\section{System Setup and Problem Formulation}\label{section3}
In this section, a new problem, namely, the resilient containment of
MAS groups against composite attacks, is proposed. First, a model of the MAS group is established and some composite attacks are defined, respectively.
{\color{black}
\subsection{MAS Group Model}
In the framework of containment control, we consider a group of $N+M$ MASs, which can be divided into two groups:

1) The $M$ leaders are the roots of the directed graph $\mathcal{G}$ and have no neighbors. We define the index set of the leaders as $\mathcal{L}= \textbf{I}[N+1,N+M]$.

2) The $N$ followers coordinate with their neighbors to achieve the containment set
of the above leaders. We define the index set of the followers
as $\mathcal{F} = \textbf{I}[1, N]$.

Similar to previous works \cite{zuo2020,zuo2021,haghshenas2015}, we consider the following leader dynamics:
\begin{equation}\label{EQ1}
\begin{cases}
\dot{x}_k=S x_k,\\
y_k=R x_k,
\end{cases}
\end{equation}
where $x_k\in \mathbb{R}^q$ is $k$th leader state, $y_k\in \mathbb{R}^p$ is the $k$th leader reference output and $k \in \mathcal{L}$.
The dynamics of each follower is given by
\begin{equation}\label{EQ2}
  \begin{cases}
  \dot{x}_i=A_i x_i + B_i u_i,\\
  y_i=C_i x_i,
  \end{cases}
\end{equation}
where $x_i\in \mathbb{R}^{ni}$, $u_i\in \mathbb{R}^{mi}$ and $y_i\in \mathbb{R}^p$ are
the system state, the control input and output of the $i$th follower with $i\in \mathcal{F}$, respectively. And the leader dynamics $(S,R)$ and follower dynamics $(A_i,B_i,C_i)$ have different and appropriate dimensions.    For convenience, the notation $``(t)"$ can be omitted in the following discussion. We make the following assumptions about the agents and communication network.

{\color{black}
\begin{assumption}\label{assumption 1}
There exists a directed path from at least one leader to each follower $i\in\mathcal{F}$ in graph $\mathcal{G}$.
\end{assumption}

\begin{assumption}\label{assumption 2}
  The real parts of the eigenvalues of $S$ are nonnegative.
\end{assumption}




\begin{assumption}\label{assumption 3}
the pair $(A_i, B_i)$ is stabilizable and the pair $(A_i, C_i)$ is detectable for $i \in \mathcal{F}$.
\end{assumption}



\begin{assumption}\label{assumption 4}
For all $\lambda \in \sigma(S)$, where $\sigma(S)$ represents the spectrum of $S$, we have
  \begin{equation}
   {\rm rank} \left[
      \begin{array}{c|c}
     A_i-\lambda I_{n_i} &  B_i  \\ \hline
    C_i  & 0   \\
      \end{array}
      \right]=n_i+p,~
      i \in \mathcal{F}.
  \end{equation}
\end{assumption}

}


\begin{remark}
Assumptions 1, 3 and 4 are standard in traditional output regulation problems, and Assumption 2 prevents the trivial case of a stable $S$. The modes associated with eigenvalues of $S$ with negative real parts decay exponentially to zero and therefore do not affect the asymptotic behavior of the closed-loop system. $\hfill \hfill \square $
\end{remark}






}


\subsection{Attack Descriptions}
In this work, we consider an MAS consisting of cooperative agents with potential malicious attackers. As shown in Fig. \ref{fig:figure0}, the attackers use four kinds of attacks to compromise the containment performance of the MAS:

\begin{figure}[!]
  %\begin{minipage}[t]{1\linewidth}
  \centering
  \includegraphics[width=0.45\textwidth]{picnew2/TLCPL1009.png}
  \caption{Resilient MASs against composite attacks: A double layer framework.}
  \label{fig:figure0}
\end{figure}




1) DoS attacks: The communication graphs among the agents (in both the TL and CPL) are denied by the attackers;

2) Actuation attacks: The motor inputs are infiltrated by the attackers to destroy the input signals of the agent;

3) FDI attacks: The information exchanged among the agents is distorted by the attackers;

4) Camouflage attacks: The attackers mislead downstream agents by disguising themselves as leaders.

To resist composite attacks, we introduce a new layer known as the TL. The TL has the same communication topology as the CPL with greater security and fewer physical meanings. Therefore, the TL can effectively resist most of the above attacks. With the introduction of the TL, a decoupled resilient control scheme can be introduced to defend against DoS attacks on the TL and potential unbounded actuation attacks on the CPL. The following subsections present the definitions and essential constraints of the DoS attacks and actuation attacks.

1) DoS attacks: DoS attacks refer to attacks where an adversary attacks some or all components of a control system. DoS attacks can affect the measurement and control channels simultaneously, resulting in the loss of data availability. Define $\{t_l \}_{l \in \mathbb{N}}$ and $\{T_l \}_{l \in \mathbb{N}}$ as the start and terminal time of the $l$th attack sequence of a DoS attack, that is, the $l$th DoS attack time interval is $A_l = [t_l, T_l)$, where $t_{l+1} > T_l$ for all $l \in \mathbb{N}$. Therefore, for all $t\geq t_0 \in \mathbb{R}$, the set of time instants where the communication network is under DoS attacks can be represented by
\begin{equation}
    \Xi_A(t_0,t) = \cup A_l \cap [t_0, t],~l\in \mathbb{N}.
\end{equation}
Moreover, the set of time instants of the denied communication network can be defined as
\begin{equation}
    \Xi_N(t_0,t) = [t_0,t] / \Xi_A(t_0,t).
\end{equation}

\begin{myDef} [{Attack Frequency \cite{fengz2017}}]
For any $\delta_2 > \delta_1 \geq t_0$, let $N_a(t_1, t_2)$ represent the number of DoS attacks during $[t_1, t_2)$. Therefore, $F_a(t_1, t_2)= \frac{N_a(t_1, t_2)}{t_2 - t_1}$ is defined as the attack frequency during $[t_1, t_2)$ for all $t_2 > t_1 \geq t_0$.
\end{myDef}

\begin{myDef} [{Attack Duration \cite{fengz2017}}]\label{Def3}
For any $t_2 > t_1 \geq t_0$, let $T_a(t_1, t_2)$ represent the total time interval of DoS attacks on MASs during $[t_1, t_2)$. The attack duration $[t_1, t_2)$ can be defined with constants $\tau_a > 1$ and $T_0 > 0$, with 
  \begin{equation}
    T_a(t_1,t_2) \leq T_0 + \frac{t_2-t_1}{\tau_a}. 
  \end{equation}
\end{myDef}

2) Unbounded Actuation Attacks:
For each follower, the system input is subjected to unknown actuator faults that can be defined as
\begin{equation}\label{EQ9}
    \bar{u}_i=u_i+\chi_i,~ \forall i  \in \mathcal{F},
\end{equation}
where $\chi_i$ denotes the unknown unbounded attack signal caused by actuator faults. Thus, the true values of $u_i$ and $\chi_i$ are unknown, and we can only measure the damaged control input information $\bar{u}_i$.


\begin{assumption}\label{assumption 5}
The unknown actuator attack signal $\chi_i$ is unbounded, and its derivative $\dot{\chi}_i$ is bounded by $\bar{d}$.
\end{assumption}

{\color{black}
\begin{remark}
In contrast to \cite{deng2021} and \cite{chen2019}, which only considered bounded actuation attacks, this work deal with composite attacks, including unbounded actuation attacks under Assumption \ref{assumption 5}. When the derivative of the attack signal is unbounded, the attack signal increases substantially, and the MAS can reject the signal by removing excessively large derivative values, which can easily be detected.
$\hfill \hfill \square$
\end{remark}

}





\subsection{Problem Formulation}


%{\color{red}
Under the condition that there exists no attack, the output containment error of the $i$th follower can be represented as
\begin{equation}\label{EQ xi}
    \xi_i = \sum_{j\in \mathcal{F}} a_{ij}(y_j -y_i) +\sum_{k \in \mathcal{L}} g_{ik}(y_k - y_i),
\end{equation}
where $a_{ij}$ is the weight of edge $(v_i, v_j)$ in graph $\mathcal{G}_f$
($\mathcal{G}_f$ is the communication graph of all followers),
and $g_{ik}$ is the weight of the path between the $i$th leader and the $k$th follower.

The global form of (\ref{EQ xi}) is written as
\begin{equation}
    \xi = - \sum_{k \in \mathcal{L}}(\Psi_k \otimes I_p)(y -  \underline{y}_k),
\end{equation}
where $\Psi_k = (\frac{1}{m} L_f + G_{ik})$ with $L_f$ is the Laplacian matrix of communication digraph $\mathcal{G}_f$ and $G_{ik}={\rm diag}(g_{ik})$, $\xi = [\xi_1^{\mathrm{ T}},\xi_2^{\mathrm{T}},\dots,\xi_n^{\mathrm{T}}]^{\mathrm{T}}$, $y=[y_1^{\mathrm{T}},y_2^{\mathrm{T}},\dots,y_n^{\mathrm{T}}]^{\mathrm{T}}$, $\underline{y}_k = (l_n \otimes y_k)$.

\begin{Lemma}{\cite{haghshenas2015}}\label{Lemma3}
   Suppose that Assumption 1 holds. Then, matrices $\Psi_k$ and $\bar{\Psi}_L= \sum_{k \in \mathcal{L}} \Psi_k$ are positive-definite and nonsingular. Therefore, $(\Psi_k)^{-1}$ and $(\bar{\Psi}_L)^{-1}= (\sum_{k \in \mathcal{L}} \Psi_k)^{-1}$ are both nonsingular.
\end{Lemma}


The global output containment error can be written as \cite{zuo2019}:
\begin{equation}\label{EQ13}
e= y - \left(\sum_{r\in \mathcal{L} }(\Psi_r \otimes I_p)\right)^{-1} \sum_{k \in \mathcal{L} } (\Psi_k \otimes I_p) \underline{y}_k,
\end{equation}
where $e=[e_i^{\mathrm{T}},e_2^{\mathrm{T}},\dots,e_n^{\mathrm{T}}]^{\mathrm{T}}$ and $\xi = -\sum_{k \in \mathcal{L}}(\Psi_k \otimes I_p )e$.


\begin{Lemma}
    Define $\Theta={\rm diag}(v)$, where $v=(\bar{\Psi}_L)^{-1} \boldsymbol{1}_N$. Under Assumption 1 and Lemma \ref{Lemma3}, $\Omega=\Theta \bar{\Psi}_L +\bar{\Psi}_L^{\mathrm{T}} \Theta >0$ is a positive diagonal matrix.
\end{Lemma}


\begin{Lemma}{\cite[Lemma 1]{zuo2019}}
    According to Assumption 1, the output containment control objective in Definition \ref{def41} is achieved if ${\rm lim}_{t \rightarrow \infty} e = 0$.
\end{Lemma}




Considering the composite attacks discussed in Subsection III-B, the local neighboring relative output containment information can be rewritten as
\begin{equation}\label{EQ xi2}
    \bar{\xi}_i = \sum_{j\in \mathcal{F}} d_{ij}(\bar{y}_{i,j} -\bar{y}_{i,i}) +\sum_{k \in \mathcal{L}} d_{ik}(\bar{y}_{i,k} - \bar{y}_{i,i})+\sum_{l\in {\mathcal{C}}} c_{il}(y_l-\bar{y}_{i,i}),
\end{equation}
where $d_{ij}(t)$ and $d_{ik}(t)$ are edge weights that are influenced by the DoS attacks. For denied communication, $d_{ij}(t)=0$ and $d_{ik}(t)=0$, while for normal communication, $d_{ij}(t)=a_{ij}$ and $d_{ik}(t)=g_{ik}$. In addition, $\bar{y}_{i,j}$ (or $\bar{y}_{i,i}$) represents the information flow between $y_j$ and the $i$th follower compromised by FDI attacks. $\mathcal{C}$ is the index set of all camouflage attackers. $c_{il}$ represents the edge weight of the $l$th camouflage attack on the $i$th follower. According to Equation (\ref{EQ xi2}), attackers can disrupt communication between agents and distort the actuation inputs of all followers.


Based on the above discussion, the attack-resilient containment control problem is introduced as follows.
%Next, we introduce the attack-resilient containment control problem.

\noindent \textbf{Problem ACMCA} (Attack-resilient Containment control of MASs
against Composite Attacks): The resilient containment control problem involves designing the input $u_i$ in (\ref{EQ2}) for each follower such that the output containment error $e$ in (\ref{EQ13}) is UUB under composite attacks, i.e., the trajectories of each follower converge to a point in the dynamic convex hull spanned by the trajectories of multiple leaders.


%}


































\section{Main Results}


In this section, a double layer resilient control scheme is used to solve the \textbf{Problem ACMCA}. First, a distributed virtual resilient TL that can resist FDI attacks, camouflage attacks, and actuation attacks naturally is proposed to resist DoS attacks. Considering TL modeling errors of leader dynamics, we use distributed observers to reconstruct the leader dynamics of all agents under DoS attacks. Next, distributed estimators are proposed to reconstruct the follower states under DoS attacks. Then, we use the reconstructed leader dynamics proposed in Theorem 1 to solve the output regulator equation. Finally, adaptive controllers are proposed to resist unbounded actuation attacks on the CPL. The specific control framework is shown in Fig. \ref{fig:figure0c}.

\begin{figure}[!]
  %\begin{minipage}[t]{1\linewidth}
  \centering
  \includegraphics[width=0.45\textwidth]{picnew2/cxkt1026.png}
  \caption{The ACMCA control framework.}
  \label{fig:figure0c}
\end{figure}

\subsection{Distributed Observers for reconstructing Leader Dynamics}
In this section, we design distributed observers to reconstruct the leader dynamics on the TL.


 To facilitate this analysis, we define the leader dynamics in (2) as follows:
  \begin{equation}
      \Upsilon =\left[\begin{array}{cc}
           S  \\
           R 
      \end{array}\right]\in \mathbb{R} ^{(p+q)\times q}.
  \end{equation}
and the reconstructed leader dynamics can be defined as follows:
\begin{align} \label{EQ14}
     &\hat{\Upsilon } _{i}(t)=\left[\begin{array}{cc}
           \hat{S}_{i}(t)  \\
           \hat{R}_{i}(t) \\
      \end{array}\right] \in \mathbb{R} ^{(p+q)\times q},
  \end{align}
where $\hat{\Upsilon } _{i}(t)$ converges to $\Upsilon$ exponentially  by Theorem 1.





\begin{myTheo}\label{Theorem 1}
    Consider a group of $M$ leaders and $N$ followers composed of (\ref{EQ1}) and (\ref{EQ2}). Suppose that Assumption \ref{assumption 1} holds. Let the reconstructed leader dynamics $\hat{\Upsilon } _{i}(t)$, $i=\textbf{I}[1,N]$, in (\ref{EQ14}) on the TL be updated as follows:
\begin{equation}\label{EQ15}
   \dot{\hat{\Upsilon }} _{i}(t)=  \mu_1 \! \left(\sum_{j \in \mathcal{F}} \! d_{ij} \left(\hat{\Upsilon } _{j}(t) \!-\! \hat{\Upsilon } _{i}(t) \!\right) 
   \! +\! \sum_{k  \in \mathcal L} d_{ik} \! \left(\Upsilon\!-\!\hat{\Upsilon } _{i}(t) \!\right) \! \right),
\end{equation}
where $\mu_1 > \frac{\sigma_{\rm max}(S)}{ \lambda_{\rm min}(\Omega \Theta^{-1} ) (1-\frac{1}{\tau_a})}$ is the estimator gain. Then, the reconstructed leader dynamics $\hat{\Upsilon } _{i}(t)$ coverage to $\Upsilon$ exponentially.
\end{myTheo}

%{\color{blue}
\textbf{Proof.} 
From (\ref{EQ15}), it can see that we only use the relative neighborhood information to reconstruct the leader dynamics. Therefore, the leader dynamics observers are influenced by DoS attacks.

%\textbf{Step 1:}
Define
$\tilde{\Upsilon}_{i}(t)=\hat{\Upsilon}_{i}  (t) -\Upsilon$ as the leader dynamic modeling error. According to Equation (\ref{EQ15}), we have
\begin{equation}\label{EQ17}
   \begin{aligned}
    \dot{\tilde{\Upsilon}} _{i} (t)
=&\dot{\hat{\Upsilon}}_{i} (t) -\dot{\Upsilon} \\
=&\mu_1\! \left( \sum_{j = 1}^{N}\! d_{ij} \! \left(\hat{\Upsilon } _{j}\!(t)\!-\!\hat{\Upsilon } _{i}\!(t) \! \right) \!+\!\sum_{k=N+1}^{N+M} \! d_{ik} \!\left(\Upsilon \!- \!\hat{\Upsilon } _{i}\!(t) \!\right)\! \right).
\end{aligned} 
\end{equation}

Then, the global modeling error of leader dynamics can be defined as
\begin{align}\label{EQ 18}
    \dot{\tilde{\Upsilon}}  (t) =& -  \mu_1  \left(\sum_{k\in \mathcal{L}}\Psi_k^D \otimes I_{p+q}\tilde{\Upsilon} (t)\right)\nonumber\\
    =&-  \mu_1 \left(\bar{\Psi}_{L}^D \otimes I_{p+q}\right) \tilde{\Upsilon} (t), ~t \geq t_0,
\end{align}
where $\bar{\Psi}^D_{L}=\sum_{k\in \mathcal{L}}\Psi_k^D $ with $\Psi_k^D (t) = \begin{cases}
0,~t \in \Xi_A, \\ \Psi_k,~t \in \Xi_N,
\end{cases}$ and $\tilde{\Upsilon} (t)= [\tilde{\Upsilon}_{1}(t)^{\mathrm{T}},\tilde{\Upsilon}_{1}(t)^{\mathrm{T}},\dots,\tilde{\Upsilon}_{N}(t)^{\mathrm{T}}]^{\mathrm{T}}$.
From (\ref{EQ 18}), we have ${\rm vec }(\dot{\tilde{\Upsilon}}) =-\mu_1(I_{q}\otimes \bar{\Psi}^D_{L} \otimes I_{p+q} ){\rm vec }(\tilde{\Upsilon}) $.
Consider the following Lyapunov function:
\begin{equation}
    V_1(t)={\rm vec}(\tilde{\Upsilon})^{\mathrm{T}} (\Theta \otimes \bar{P}_1) {\rm vec}(\tilde{\Upsilon}).
\end{equation}
The derivative of $V_1(t)$ can be written as follows:
\begin{align}
    \dot{V}_1 =& {\rm vec}(\tilde{\Upsilon})^{\mathrm{T}}( -\mu_1 ((\bar{\Psi}^D_{L})^{\mathrm{T}} \Theta + \Theta \bar{\Psi}^D_{L} )\otimes \bar{P}_1) {\rm vec}(\tilde{\Upsilon}) \notag\\
    \leq&  -\mu_1 \lambda_{\rm min}(\Omega^D \Theta^{-1} ) V_1,
    \end{align}
where $ \Omega^D=(\bar{\Psi}^D_{L})^{\mathrm{T}} \Theta + \Theta \bar{\Psi}^D_{L} $.
Then, we have
\begin{flalign}
&V_1(t)\leq {\rm e}^{-\mu_1 \lambda_{\rm min}(\Omega^D \Theta^{-1} )(t-t_0)} V_1(t_0), &\\
&{\rm vec} (\tilde{\Upsilon}) \leq \sqrt{ \frac{V_1(t_0)}{\lambda_{\rm min}(\Theta \otimes \bar{P}_1)} } {\rm e}^{-\frac{\mu_1}{2} \lambda_{\rm min}(\Omega^D \Theta^{-1} ) (t-t_0)}. &
\end{flalign}
From Definition \ref{Def3}, we have
\begin{align}
  \bar{\Psi}_{L}^D(t-t_0) =& \bar{\Psi}_{L}\left\lvert \Xi_N(t_0,t)\right\rvert \notag\\
  =&\bar{\Psi}_{L}|t-t_0-\Xi_A(t_0,t)|\notag\\
  \geq & \bar{\Psi}_{L} \left(t-t_0-\left(T_0+ \frac{t-t_0}{\tau_a} \right) \right)\notag\\
  %\geq & \bar{\Psi}_{L}((1-\frac{1}{\tau_a})(t-t_0)+T_0) \\
  \geq & \left(1-\frac{1}{\tau_a}\right) \bar{\Psi}_{L} (t-t_0)  ,~ t\geq t_0.
\end{align}
Then, it is easy to obtain that $\lambda_{\rm min}(\Omega^D \Theta^{-1} ) (t-t_0) \geq \left(1-\frac{1}{\tau_a}\right) \lambda_{\rm min}(\Omega \Theta^{-1} ) (t-t_0).$

Therefore, we conclude that
\begin{equation}
    {\rm vec} (\tilde{\Upsilon}) \leq c_{\Upsilon} {\rm e}^{-\alpha_{\Psi}(t-t_0)}, 
\end{equation}
where $c_{\Upsilon}= \sqrt{ \frac{V_1(t_0)}{\lambda_{\rm min}(\Theta \otimes \bar{P}_1)}}$ and $\alpha_{\Psi}=\frac{ \mu_1}{2} (1-\frac{1}{\tau_a}) \lambda_{\rm min}(\Omega \Theta^{-1} ) $. It implies that ${\rm vec}(\tilde{\Upsilon})$ and $\tilde{\Upsilon}$ converge to zero exponentially at rates of at least $\alpha_{\Psi}$. Moreover, because $\alpha_{\Psi}>\sigma_{\rm max}(S)$, $\tilde{S}_i \zeta_k$ converges to zero exponentially for $i\in \mathcal{F}, k\in \mathcal{L}$.
$\hfill \hfill \blacksquare $


\begin{remark}
In practice, various noises and attacks exist among agents. Therefore, we reconstruct the leader dynamics of all followers according to the observers (\ref{EQ15}). The observer gain $\mu_1$ is designed to guarantee that the convergence rate of the estimated value $\tilde{S}$ is greater than the divergence rate of the leader state. Moreover, $\mu_1$ guarantees the convergence of the containment error.
Compared with the previous work \cite{chen2019}, which only considered the observers in the consensus of a single leader under bounded actuation attacks, this work considers heterogeneous output containment under composite attacks.
$\hfill \hfill \square $
\end{remark}
\subsection{Design of the Distributed Resilient State Estimators}
Based on the reconstructed leader dynamics, distributed resilient estimators are proposed to estimate follower states under DoS attacks.
Consider the following distributed state estimators on the TL:
\begin{equation}\label{equation 200}
  \dot{\zeta}_i=\hat{S}_i \zeta_i +\mu_2 G \left(\sum _{j \in \mathcal{F} }d_{ij}(\zeta_j-\zeta_i)+\sum_{k \in \mathcal{L}}d_{ik}(\zeta_k-\zeta_i)\right),
\end{equation}
%{\color{blue}
where $\zeta_k=x_k$, $\zeta_i$ is the $i$th follower state estimation on the TL, $\mu_2  > 0$ is the estimator gain, which can be chosen by the user, and $G$ is the  coupling gain, which will be determined later.
%}
The global state of the TL can be written as
\begin{equation}
  \dot{\zeta}= \hat{S}_b \zeta-\mu_2 G_b \left(\sum_{k \in \mathcal{L}}(\Psi_k^D \otimes I_p )(\zeta-\bar{\zeta}_k)\right), 
\end{equation}
where $\hat{S}_b={\rm blkdiag}(\hat{S}_i) $, $G_b=I_N \otimes G$, $\zeta=[\zeta_1^{\mathrm{T}},\zeta_2^{\mathrm{T}},\dots,\zeta_N^{\mathrm{T}}]^{\mathrm{T}}$, and $\bar{\zeta}_k=l_n \otimes \zeta_k$.

Define the global state estimation error of the TL as
\begin{equation}
\begin{aligned}
  \tilde{\zeta}\!
  =&\zeta-\left(\sum_{r \in \mathcal{L}}(\Psi_r \otimes I_p )\right)^{-1} \sum_{k \in \mathcal{L}}(\Psi_k\otimes I_p ) \bar{\zeta}_k.
\end{aligned}
\end{equation}
Then, during  normal communication, we have
\begin{equation}\label{EQ26}
\begin{aligned}
      \dot{\tilde{\zeta}}\!
       =&\hat{S}_b \zeta-\mu_2 (I_N \otimes G) \left(\sum_{k \in \mathcal{L}}(\Psi_k \otimes I_p )(\zeta-\bar{\zeta}_k)\right) \\
       &-
  \left(\sum_{r \in \mathcal{L}}(\Psi_r \otimes I_p )\right)^{-1}\sum_{k \in \mathcal{L}}(\Psi_k \otimes I_p ) (I_n \otimes S) \bar{\zeta}_k \\
      =&\hat{S}_b \zeta- (I_n \otimes S)  \zeta+ (I_n \otimes S)  \zeta \\
  &-(I_n \otimes S)\left(\sum_{r \in \mathcal{L}}(\Psi_r \otimes I_p )\right)^{-1} \sum_{k \in \mathcal{L}}(\Psi_k \otimes I_p )  \bar{\zeta}_k  +M \\
  & -\mu_2 (I_N \otimes G) \sum_{k \in \mathcal{L}}(\Psi_k \otimes I_p )\\
  & \Bigg( \zeta -\left(\sum_{\bar{r} \in \mathcal{L}}(\Psi_{\bar{r}} \otimes I_p )\right)^{-1} \sum_{\bar{k} \in \mathcal{L}}(\Psi_{\bar{k}} \otimes I_p )\bar{\zeta}_{\bar{k}}\\
  &+
  \left(\sum_{\bar{r} \in \mathcal{L}} (\Psi_{\bar{r}} \otimes I_p )\right)^{-1} \sum_{\bar{k} \in \mathcal{L}}(\Psi_{\bar{k}} \otimes I_p )\bar{\zeta}_{\bar{k}}-  \bar{\zeta}_k \Bigg)\\
  =&\tilde{S}_b \zeta+(I_n \otimes S) \tilde{\zeta}-\mu_2 (I_N \otimes G)\sum_{k \in \mathcal{L}}(\Psi_k \otimes I_p ) \tilde{\zeta} \!+\!M \\
  &-
  \mu_2 (I_N \otimes G) \! \left(\! \sum_{\bar{k} \in \mathcal{L}} \! (\Psi_{\bar{k}} \otimes I_p ) \bar{\zeta}_{\bar{k}}-\! \sum_{k \in \mathcal{L}} \! (\Psi_k \otimes I_p )\bar{\zeta}_k \! \right)  \\
  =&(I_n \otimes S) \tilde{\zeta} \!-\!\mu_2 (I_N \otimes G)\sum_{k \in \mathcal{L}}(\Psi_k \otimes I_p ) \tilde{\zeta}\!+\! \tilde{S}_b \tilde{\zeta} \!+\! F_{\zeta}(t),&
    \end{aligned}
\end{equation}
where  $F_{\zeta}(t)=\tilde{S}_b\left(\sum_{r \in \mathcal{L}}(\Psi_r \otimes I_p )\right)^{-1} \sum_{k \in \mathcal{L}}(\Psi_k \otimes I_p ) \bar{\zeta}_k +M$ and
$M = (I_n \otimes S) \left(\sum_{r \in \mathcal{L}}(\Psi_r \otimes I_p )\right)^{-1}\sum_{k \in \mathcal{L}}(\Psi_k \otimes I_p ) \bar{\zeta}_k - \left(\sum_{r \in \mathcal{L}}(\Psi_r \otimes I_p )\right)^{-1}\sum_{k \in \mathcal{L}}(\Psi_k \otimes I_p ) (I_n \otimes S) \bar{\zeta}_k$ and $\tilde{S}_b={\rm blkdiag}(\tilde{S}_i)$ for $i\in \mathcal{F}$.

During denied communication, we have
\begin{flalign}\label{EQ27}
      \dot{\tilde{\zeta}}\!
       =&\hat{S}_b \zeta-
  \left(\sum_{r \in \mathcal{L}}(\Psi_r \otimes I_p )\right)^{-1}\sum_{k \in \mathcal{L}}(\Psi_k \otimes I_p ) (I_n \otimes S) \bar{\zeta}_k &\notag\\
      =&\hat{S}_b \zeta- (I_n \otimes S)  \zeta+ (I_n \otimes S)  \zeta &\notag\\
  &\!-\!(I_n \otimes S)\left(\sum_{r \in \mathcal{L}}(\Psi_r \otimes I_p )\right)^{-1} \sum_{k \in \mathcal{L}}(\Psi_k \otimes I_p )  \bar{\zeta}_k  +M &\notag\\
  %=&\tilde{S}_b \zeta+(I_n \otimes S) \tilde{\zeta} +M\\
  =&(I_n \otimes S) \tilde{\zeta}+ \tilde{S}_b \tilde{\zeta} + F_{\zeta}(t).&
    \end{flalign}
Therefore, according to (\ref{EQ26}) and (\ref{EQ27}), we can conclude that
\begin{equation}\label{EQ32}
    \dot{\tilde{\zeta}} = \begin{cases}
    \begin{aligned}
    \hat{S}_b \tilde{\zeta}-\mu_2 (I_N \otimes G)\sum_{k \in \mathcal{L}}
    (\Psi_k \otimes I_p )  \tilde{\zeta} + F_{\zeta}(t),\\
    ~t \in \Xi_N(t_0,t), 
    \end{aligned} \\
     \hat{S}_b \tilde{\zeta} + F_{\zeta}(t),~ t \in \Xi_A(t_0,t).
    \end{cases}
\end{equation}



\begin{Lemma}\cite{haghshenas2015}\label{Lemma5}
Under Lemma 2 and the Kronecker product property $(P \otimes Q)(Y \otimes Z) =(PY)\otimes(QZ)$, it is easy to show that
$\left(\sum_{r \in \mathcal{L}}(\Psi_r \otimes I_p )\right)^{-1}\sum_{k \in \mathcal{L}}(\Psi_k \otimes I_p ) (I_n \otimes S) \bar{\zeta}_k=
(I_n \otimes S) \left(\sum_{r \in \mathcal{L}}(\Psi_r \otimes I_p )\right)^{-1}\sum_{k \in \mathcal{L}}(\Psi_k \otimes I_p ) \bar{\zeta}_k$. 

\textbf{Proof:}
Let
\begin{flalign}
M= &\sum_{k \in \mathcal{L}} M_k
=\sum_{k \in \mathcal{L}} \Bigg((I_n \otimes S) \left(\sum_{r \in \mathcal{L}}(\Psi_r \otimes I_p )\right)^{-1}(\Psi_k \otimes I_p ) &\notag\\
&-\left(\sum_{r \in \mathcal{L}}(\Psi_r \otimes I_p )\right)^{-1} (\Psi_k \otimes I_p ) (I_n \otimes S)\Bigg) \bar{\zeta}_k.&
\end{flalign}
According to the Kronecker product property $(P \otimes Q)(Y \otimes Z) =(PY)\otimes(QZ)$, we have
\begin{align}
  & (I_N \otimes S)\left(\sum_{r \in \mathcal{L}} \Psi_r \otimes I_p\right)^{-1} (\Psi_k \otimes I_p) \notag\\
   =& (I_N \otimes S)\left(\left(\sum_{r \in \mathcal{L}} \Psi_r\right)^{-1} \Psi_k\right) \otimes I_p) \notag\\
   =&\left(I_N \times \left(\sum_{r \in \mathcal{L}} \Psi_r\right)^{-1} \Psi_k\right)\otimes(S \times I_p)\notag \\
   =& \left(\sum_{r \in \mathcal{L}} \Psi_r \otimes I_p\right)^{-1} (\Psi_k \otimes I_p)  (I_N \otimes S).
\end{align}
Therefore, we can show that $M_k=0$, yielding 
\begin{equation}
     M=\sum_{k \in \mathcal{L}}M_k =0.
 \end{equation}
This completes the proof.
$\hfill \hfill \blacksquare $
\end{Lemma}




%}




%Editor: Please ensure that the intended meaning has been maintained in the following edit.
Then, by Lemma \ref{Lemma5} and Theorem 1, we obtain that $F_{\zeta}(t)=\tilde{S}_b\left(\sum_{r \in \mathcal{L}}(\Psi_r \otimes I_p )\right)^{-1} \sum_{k \in \mathcal{L}}(\Psi_k \otimes I_p ) \bar{\zeta}_k$ exponentially converges to zero.
\begin{myTheo}\label{Theorem 2}
    Consider the MAS (\ref{EQ1})-(\ref{EQ2}) under DoS attacks. Suppose Assumption 1 holds, Definitions 2 and 3 are satisfied by DoS attacks. Then, there exist scalars 
    $\tilde{\alpha}_1>0$, $\tilde{\alpha}_2>0$, and $\mu_2>0$ and positive definite symmetric matrix $\bar{P}_2 >0$ such that
    \begin{align}
    \bar{P}_2 S+S^{\mathrm{T}}\bar{P}_2 - \bar{P}_2^{\mathrm{T}} \mu_2^2 \bar{P}_2 + \tilde{\alpha}_1 \bar{P}_2 =0 \\
    \bar{P}_2 S+S^{\mathrm{T}}\bar{P}_2  - \tilde{\alpha}_2 \bar{P}_2 \leq 0 \\
    \frac{1}{\tau_a} < \frac{\alpha_1}{\alpha_1+\alpha_2} \label{EQ36}
    \end{align}
%where  $\bar{\Psi}_{L}=\sum_{k \in \mathcal{L}}(\Psi_k ) $,
where $\alpha_1=\tilde{\alpha}_1-k_1 ||\Theta \otimes \bar{P}_2||$, $\alpha_2=\tilde{\alpha}_2+k_1 ||\Theta \otimes \bar{P}_2||$ with $k_1>0$, and $G=\mu_2 \lambda_{\rm max}(\Omega^{-1} \Theta) \bar{P}_2$.
Then, it can be guaranteed that the global state estimation error $\tilde{\zeta}$ of TL converges to $0$ exponentially under DoS attacks.
\end{myTheo}


\textbf{Proof.}
We prove that the TL state can achieve containment under DoS attacks.
For clarity, we first redefine the set $\Xi_A[t_0, t)$ as $\Xi_A[t_0, t)= \bigcup_{k=0,1,2,\dots} [t_{2k+1},t_{2k+2})$, where $t_{2k+1}$ and $t_{2k+2}$ indicate the times that the attacks start and end, respectively.
Then, the set $\Xi_N[t_0,t)$ can be redefined as $\Xi_N[t_0, t)= \bigcup_{k=0,1,2,\dots} [t_{2k},t_{2k+1})$.

Consider the following Lyapunov function candidate:
\begin{align}
   V_2(t)=\tilde{\zeta}^{\mathrm{T}} (\Theta \otimes \bar{P}_2) \tilde{\zeta}.
\end{align}
During normal communication, the time derivative of $V_2(t)$ is computed as follows:
\begin{flalign}
\dot{V}_2
=&\tilde{\zeta}^{\mathrm{T}}(\Theta \otimes (\bar{P}_2 S + S^{\mathrm{T}}\bar{P}_2))\tilde{\zeta} &\notag\\
&- \tilde{\zeta}^{\mathrm{T}}
((\bar{\Psi}^{\mathrm{T}}_L \Theta  +\Theta  \bar{\Psi}_L) \otimes \lambda_{\rm max}(\Omega^{-1} \Theta) \mu_2^2 \bar{P}_2^2 )\tilde{\zeta} &\notag\\
&+\tilde{\zeta}^{\mathrm{T}} (\Theta \otimes (\bar{P}_2 \tilde{S}_i + \tilde{S}_i^{\mathrm{T}}\bar{P}_2) ) \tilde{\zeta}  +2\tilde{\zeta}^{\mathrm{T}} (\Theta \otimes \bar{P}_2) F_{\zeta} &\notag\\
\leq& \tilde{\zeta}^{\mathrm{T}} ((\Theta \otimes \bar{P}_2) \tilde{S}_b+ \tilde{S}_b^{\mathrm{T}} (\Theta \otimes \bar{P}_2)) \tilde{\zeta} +2\tilde{\zeta}^{\mathrm{T}} (\Theta \otimes \bar{P}_2) F_{\zeta}&\notag\\
&+\tilde{\zeta}^{\mathrm{T}} (\Theta \otimes (\bar{P}_2 S + S^{\mathrm{T}}\bar{P}_2) - \Theta \otimes \mu_2^2 \bar{P}_2^2)\tilde{\zeta}.&
\end{flalign}
By Young's inequality, there exists a scalar $k_1>0$ that yields
\begin{align} \label{EQ39}
    2\tilde{\zeta}^{\mathrm{T}}(\Theta \otimes \bar{P}_2) F_{\zeta} \leq k_1 \tilde{\zeta}^{\mathrm{T}} (\Theta \otimes \bar{P}_2)^2 \tilde{\zeta} + \frac{1}{k_1} ||F_{\zeta}||^2 .
\end{align}
Then, we have
\begin{flalign}
\dot{V}_2\leq& -\tilde{\alpha}_1 V_2 + 2  ||\tilde{S}_b|| V_2 +k_1 ||\Theta \otimes \bar{P}_2||V_2 +\frac{1}{k_1} ||F_{\zeta}||^2&\notag\\
=&\left(-\tilde{\alpha}_1+k_1 ||\Theta \otimes \bar{P}_2||+2  ||\tilde{S}_b||\right)V_2+\frac{1}{k_1} ||F_{\zeta}||^2.&
\end{flalign}
Because $\tilde{S}_b$ and $F_{\zeta}$ converge to $0$ exponentially, we have
\begin{equation} \label{EQ45}
    \dot{V}_2\leq  (-\alpha_1+ a_s {\rm e}^{-b_s t})V_2 + \varphi(t),
\end{equation}
where $a_s>0$, $b_s>0$, and $\varphi(t)=a_f {\rm e}^{-b_f}$ with $a_f>0$ and $b_f>0$.
When communication is denied, the time derivative of $V_2(t)$ yields that
\begin{flalign}\label{EQ42}
\dot{V}_2
=&\tilde{\zeta}^{\mathrm{T}}(\Theta \otimes (\bar{P}_2 S + S^{\mathrm{T}}\bar{P}_2))\tilde{\zeta} 
+ \tilde{\zeta}^{\mathrm{T}}
((\Theta \otimes \bar{P}_2) \tilde{S}_b &\notag\\
&+ \tilde{S}_b^{\mathrm{T}} (\Theta \otimes \bar{P}_2)) \tilde{\zeta} +2\tilde{\zeta}^{\mathrm{T}} (\Theta \otimes \bar{P}_2) F_{\zeta} &\notag\\
\leq& \tilde{\alpha}_2 V_2 + 2||\tilde{S}_b||  V_2 +k_1 ||\Theta \otimes \bar{P}_2||V_2 +\frac{1}{k_1} ||F_{\zeta}||^2 &\notag\\
 \leq& (\alpha_2+ a_s {\rm e}^{-b_s t}) V_2 + \varphi(t). &
\end{flalign}
By solving inequalities (\ref{EQ45}) and (\ref{EQ42}), we obtain the following inequality:
\begin{equation} \label{EQ47}
   V_2(t)  \! \leq  \! \begin{cases}\begin{aligned}
   & {\rm e}^{\int_{t_{2k}}^{t}-\alpha_1+ a_s {\rm e}^{-b_s \tau}\,{\rm d}\tau}V_2(t_{2k}) \\
   & \!+ \!\int_{t_{2k}}^{t} {\rm e}^{\int_{\tau}^{t}-\alpha_1+ a_s {\rm e}^{-b_s s}\,{\rm d}s} \varphi(\tau)\,{\rm d}\tau ,~ t \! \in  \! [t_{2k},t_{2k+1}),
   \end{aligned}\\
   \begin{aligned}
    & {\rm e}^{\int_{t_{2k+1}}^{t}\alpha_2+ a_s {\rm e}^{-b_s \tau}\,{\rm d}\tau} V_2(t_{2k+1}) \\
    &\!+ \!\int_{t_{2k+1}}^{t}\! \! {\rm e}^{\int_{\tau}^{t}\!\alpha_2+ a_s {\rm e}^{-b_s s}\!\,{\rm d}s} \varphi(\tau)\!\,{\rm d}\tau ,~ t \! \in  \! [t_{2k+1},t_{2k+2}) .
   \end{aligned}
   \end{cases}
\end{equation}
Let $\alpha=\begin{cases}
 -\alpha_1,~ t\in[t_{2k},t_{2k+1}),\\
 \alpha_2,~t \in[t_{2k+1},t_{2k+2}),
\end{cases}$ and we reconstruct (\ref{EQ47}) as
\begin{flalign}\label{EQ46}
    &V_2(t)\! \leq \! {\rm e}^{\int_{t_0}^{t}\!\alpha  + a_s {\rm e}^{-b_s \tau}\! \,{\rm d}\tau}V_2(t_{0}) \!+\! \int_{t_{0}}^{t} \!{\rm e}^{\int_{\tau}^{t}\!\alpha + a_s {\rm e}^{-b_s s}\! \,{\rm d}s}\! \varphi(\tau)\,{\rm d}\tau .&
\end{flalign}
For $t\in \Xi_{N}[t_0,t)$, (\ref{EQ46}) suggests that
\begin{equation}\label{EQ548}
\begin{aligned}
   V_2(t)\! \leq& {\rm e}^{\int_{t_0}^{t} a_s {\rm e}^{-b_s \tau}\, d\tau} {\rm e}^{\int_{t_0}^{t}\alpha \,{\rm d}\tau}V_2(t_{0}) \\
   &+ \int_{t_{0}}^{t}          a_f {\rm e}^{-b_f+ \int_{\tau}^{t} a_s {\rm e}^{-b_s s}\,{\rm d} s }  {\rm e}^{\int_{\tau}^{t}\alpha\,{\rm d}s} \,{\rm d}\tau \\
    \leq& {\rm e}^{-\frac{a_s}{b_s}({\rm e}^{-b_s t} -{\rm e}^{-b_s t_0})}  {\rm e}^{-\alpha_1|\Xi_N(t_0,t)|+\alpha_2|\Xi_A(t_0,t)|}V_2(t_0)\\
    &\!+\!\int_{t_0}^{t}\! a_f \! {\rm e}^ { \!-\! b_f \tau\!-\!\frac{a_s}{b_s}({\rm e}^{\!-\!b_s t}\! -\!{\rm e}^{-b_s \tau}\!)} {\rm e}^{\!-\!\alpha_1|\Xi_N(\tau,t)|\!+\!\alpha_2|\Xi_A(\tau,t)|}\,{\rm d} \tau .
\end{aligned}
\end{equation}

Similarly, it can easily be concluded that (\ref{EQ548}) holds for $t\in \Xi_{A}[t_0,t)$. According to Definition \ref{Def3}, we have
\begin{flalign}\label{EQ49}
   & -\alpha_1|\Xi_N(t_0,t)+\alpha_2|\Xi_A(t_0,t)| &\notag\\
    =& -\alpha_1 \left(t-t_0-|\Xi_A(t_0,t)|\right) +\alpha_2|\Xi_A(t_0,t)| &\notag\\
    \leq& -\alpha_1 (t-t_0)+(\alpha_1+\alpha_2)  \left(T_0+\frac{t-t_0}{\tau_a} \right) &\notag\\
    \leq& -\eta(t-t_0)+(\alpha_1+\alpha_2)T_0,&
\end{flalign}
where $\eta =(\alpha_1  - \frac{\alpha_1+\alpha_2}{\tau} )$.
Substituting (\ref{EQ49}) into (\ref{EQ548}), we obtain
\begin{flalign}
   V_2(t)
    \leq& {\rm e}^{-\frac{a_s}{b_s}({\rm e}^{-b_s t}-{\rm e}^{-b_s t_0})+(\alpha_1+\alpha_2)T_0}V_2(t_0){\rm e}^{-\eta(t-t_0)} &\notag\\
    &+\frac{a_f}{-b_f+\eta}  {\rm e}^{-\frac{a_s}{b_s}({\rm e}^{-b_s t}-{\rm e}^{-b_s t_0})+(\alpha_1+\alpha_2)T_0} &\notag\\
    &({\rm e}^{-b_f t}-{\rm e}^{-\eta t+(-b_f +\eta)t_0}) &\notag\\
   \leq& c_1 {\rm e}^{-\eta(t-t_0)}+c_2 {\rm e}^{-b_f t}&
    \end{flalign}
with $c_1={\rm e}^{\frac{a_s}{b_s}{\rm e}^{-b_s t_0}+(\alpha_1+\alpha_2)T_0}\left(V_2(t_0)-\frac{a_f}{-b_f+\eta}{\rm e}^{-b_f t_0}\right)$ and $c_2= \frac{a_f}{-b_f+\eta} {\rm e}^{\frac{a_s}{b_s}{\rm e}^{-b_s t_0}+(\alpha_1+\alpha_2)T_0}$.
From (\ref{EQ36}), we obtain $\eta>0$. Therefore, it is obvious that $V_2(t)$ converges to $0$ exponentially.





Here, we have proved that $\tilde{\zeta}$ converges to $0$, which is bounded. Thus, inequality (\ref{EQ39}) can be rewritten as $ 2\tilde{\zeta}^{\mathrm{T}}(\Theta \otimes \bar{P}_2) F_{\zeta} \leq k_1^* ||F_{\zeta}|| \leq \varphi^*(t)$, with $ \varphi^*(t)$ converging to 0 exponentially, which shows that $k_1$ is not necessary and that $\alpha_1=\tilde{\alpha}_1$ and $\alpha_2=\tilde{\alpha}_2$ hold.





\begin{remark}
Compared with previous work \cite{deng2021}, our state estimators can work without an accurate leader dynamics, which is more applicable in practical applications. 
Moreover, Deng et al. \cite{deng2021} only considered single leader consensus, while we consider the more general case of output containment with multiple leaders. In addition, in contrast to previous work \cite{yang2020}, which only obtained the bounded state tracking errors under DoS attacks, in our work, the state error of the TL converges to zero exponentially, which is a more conservative result.
$\hfill \hfill \square $
\end{remark}






















%}













\subsection{Decentralized Output Regulator Equation Solvers}
Our control protocol uses the output regulator equations to provide appropriate feedforward control gain and achieve output containment. However, the output regulator equations require that the dynamics of the leader system be known. Since MASs are fragile, various disturbances and attacks in the information transmission channel of the CPL can occur.
Therefore, we cannot use the leader dynamics matrices $S$ and $R$ directly, as discussed in \cite{zuo2020}. Since the leader dynamics are reconstructed on the TL, we use the reconstructed dynamic matrices $\hat{S}_i$ and $\hat{R}_i$ to solve the output regulator equations. 

From Assumption \ref{assumption 4}, we obtain the following output regulator equations:
 \begin{equation}\label{EQ10}
     \begin{cases}
      A_i\Pi_i+B_i\Gamma_i=\Pi_i S \\
      C_i\Pi_i = R
     \end{cases}
 \end{equation}
have a pair unique solution matrices $\Pi_i$ and $\Gamma_i$ for each $i=\textbf{I}[1, N]$. Rewriting the output regulator equation (\ref{EQ10}) yields 
\begin{flalign*}
 &\left[
  \begin{array}{cc}
  A_i &  B_i  \!\\
  C_i &  0 \!  \\
  \end{array}
  \right]\!\!
   \left[
    \begin{array}{cc}
   \Pi_i \\
   \Gamma_i \\
    \end{array}
    \right]\!
    I_q -\!
    \left[
  \begin{array}{cc}
   I_{n_i} & 0  \\
   0 &  0   \\
  \end{array}
  \right]\!\!
   \left[
    \begin{array}{cc}
\Pi_i\\
\Gamma_i \\
    \end{array}
    \right]\!S
    =\!\!
    \left[\begin{array}{cc}
        \! 0\!  \\
       \!  R\!
    \end{array}\right]&
\end{flalign*}
or
\begin{equation}\label{EQ511}
    M_iY I_q -N_iY S=R_i^{\ast},
\end{equation}
where $M_i=\left[
  \begin{array}{cc}
  A_i &  B_i  \\
  C_i &  0   \\
  \end{array}
  \right],
  N_i=\left[
  \begin{array}{cc}
   I_{n_i} & 0  \\
   0 &  0   \\
  \end{array}
  \right],
  Y_i= \left[
    \begin{array}{cc}
\Pi_i\\
\Gamma_i \\
    \end{array}
    \right], $ and $R_i^{\ast}= \left[\begin{array}{cc}
         0  \\
         R
    \end{array}\right] $. According to Theorem 1.9 of Huang et al. \cite{huang2004}, the standard form of the linear equation in Equation (\ref{EQ511}) can be rewritten as
\begin{equation}\label{EQ52}
    \Phi_i \Delta_i=\mathcal{R}_i,
\end{equation}
where $\Phi_i=(I_q \otimes  M_i-S^{\mathrm{T}} \otimes N_i)$, $\Delta_i={\rm vec}( Y_i)$, 
    $\mathcal{R}_i={\rm vec}(\mathcal{R}^{\ast})$, $R^{\ast}=\left[
    \begin{array}{cc}
0\\
R \\
    \end{array}
    \right]$.

Similarly, with reconstructed leader dynamics, it can obtain that output regulator equations:
 \begin{equation} \label{EQ 53}
     \begin{cases}
      A_i\hat{\Pi}_i+B_i\hat{\Gamma}_i=\hat{\Pi}_i \hat{S}_i \\
      C_i\hat{\Pi}_i = \hat{R}_i
     \end{cases}
 \end{equation}
and
\begin{equation}\label{EQ54}
   \hat{\Phi}_i \hat{\Delta}_i=\hat{\mathcal{R}}_i,
\end{equation}
where $\hat{\Delta}_i={\rm vec}(\hat{Y}_i)$, $\hat{\Phi}_i=(I_q \otimes  M_i-\hat{S}_i^{\mathrm{T}} \otimes N_i)$, $\hat{R}_i={\rm vec}(\mathcal{\hat{R}}^{\ast}_i)$,  $\hat{Y_{i}}=[\hat{\Pi}_{i}^{\mathrm{T}}$, $\hat{\Gamma}_{i}^{\mathrm{T}}]^{\mathrm{T}}$ and $\hat{R}^{\ast}_i=[0,\hat{R}_i^{\mathrm{T}}]^{\mathrm{T}}$.
\begin{myTheo}
Suppose that Assumptions \ref{assumption 1} and \ref{assumption 4} hold. The estimated solutions $\hat{\Delta}_{i}$ to the output regulator equations in (\ref{EQ10}) can be adaptively solved as follows:
\begin{align} \label{EQ48}
    &\dot{\hat{\Delta}}_{i} = - \mu_3 \hat{\Phi }^{\mathrm{T}}_i(\hat{\Phi }_i \hat{\Delta}_{i}-\hat{\mathcal{R}}_i),
\end{align}
%\begin{equation}
%   \dot{\hat{\Delta}}_{i} = - u \hat{\Phi }^{\mathrm{T}}_i(\hat{\Phi }_i \hat{\Delta}_{i}-\hat{\mathcal{R}}_i)
%\end{equation}
%with the gain $\mu_3 > \frac{\sigma_{\rm max}(S)}{\lambda_{\rm min}(\Phi_i^{\mathrm{T}} \Phi_i )}$, then, the estimated solution $\hat{\Delta}_i$ exponentially coverage to $\Delta$ .
where $\mu_3 > \frac{\sigma_{\rm max}(S)}{\lambda_{\rm min}(\Phi_i^{\mathrm{T}} \Phi_i )}$, then, $\hat{\Delta}_i$ coverage to $\Delta$ exponentially.
\end{myTheo}

\textbf{Proof.}

In Theorem 1, we note that the estimated leader dynamics are time-varying, thus, the output regulator equations influenced by the estimated leader dynamics are also time-varying. Next, we prove that the estimated solutions of the output regulator equations $\hat{\Delta}_{i}$ converge exponentially to the solutions of the output regulator equations $\Delta_i$.

Note that
\begin{flalign}
\dot{\hat{\Delta}}_{i}
=& - \mu_3 \hat{\Phi }^{\mathrm{T}}_i(\hat{\Phi }_i \hat{\Delta}_{i}-\hat{\mathcal{R}}_i) &\notag\\
=&- \mu_3  \hat{\Phi}^{\mathrm{T}}_i \hat{\Phi}_i \hat{\Delta}_{i}+  \mu_3 \hat{\Phi}_i \hat{\mathcal{R}_i} &\notag\\
=&-\!  \mu_3\Phi_i^{\mathrm{T}}\! \Phi_i \hat{\Delta}_{i} \!+\! \mu_3 \Phi_i^{\mathrm{T}} \!\Phi_i \hat{\Delta}_{i} \!-\! \mu_3 \hat{\Phi}_i^{\mathrm{T}}  \hat{\Phi}_i \hat{\Delta}_{i}\! +\!  \mu_3 \hat{\Phi}_i^{\mathrm{T}} \hat{\mathcal{R}_i} &\notag\\
&- \mu_3 \Phi_i^{\mathrm{T}} \hat{\mathcal{R}_i} +  \mu_3 \Phi_i^{\mathrm{T}} \hat{\mathcal{R}_i} - \mu_3 \Phi_i^{\mathrm{T}} \mathcal{R}_i +  \mu_3 \Phi_i^{\mathrm{T}} \mathcal{R}_i &\notag\\
=&-  \mu_3\Phi_i^{\mathrm{T}} \Phi_i \hat{\Delta}_{i} +  \mu_3 (\Phi_i^{\mathrm{T}} \Phi_i - \hat{\Phi}_i^{\mathrm{T}} \hat{\Phi}_i) \hat{\Delta}_{i} &\notag\\
&+   \mu_3 (\hat{\Phi}_i^{\mathrm{T}} - \Phi_i^{\mathrm{T}}) \hat{\mathcal{R}_i} 
+  \mu_3 \Phi_i^{\mathrm{T}} (\hat{\mathcal{R}_i}-\mathcal{R}_i) +  \mu_3 \Phi_i^{\mathrm{T}} \mathcal{R}_i &\notag\\
=&- \mu_3 \Phi_i^{\mathrm{T}} \Phi_i \hat{\Delta}_{i} + \mu_3  \Phi_i^{\mathrm{T}} \mathcal{R}_i +d_i(t),&
\end{flalign}
where  $d_i(t)= - \mu_3 (\hat{\Phi}_i^{\mathrm{T}}\hat{\Phi}_i - \Phi_i^{\mathrm{T}}\Phi_i) \hat{\Delta}_{i} + \mu_3  \tilde{\Phi}_i^{\mathrm{T}} \hat{\mathcal{R}}_i +         \mu_3 \Phi_i^{\mathrm{T}} \tilde{\mathcal{R}}_i$ with $\tilde{\Phi}_i=\hat{\Phi}_i-\Phi_i=\tilde{S}_i^{\mathrm{T}} \otimes N_i$ and $\tilde{\mathcal{R}}={\rm vec}(\left[\begin{array}{cc}
    0  \\
    \tilde{R}_i
    \end{array}
    \right])={\rm vec}(\left[\begin{array}{cc}
    0  \\
    \hat{R}_i-R
    \end{array}
    \right])$.
It is obvious that $\lim _{t \rightarrow \infty}d_i(t) =0 $ exponentially at a rate of $\alpha_{\Psi}$.

Let $\tilde{\Delta}_{i}= \hat{\Delta}_{i}-\Delta_i $. The time derivative of $\tilde{\Delta}_{i}$
is given as follows:
\begin{equation}\label{EQ53}
\begin{aligned}
    \dot{\tilde{\Delta}}_{i}
    =&- \mu_3 \Phi_i^{\mathrm{T}} \Phi_i \tilde{\Delta}_{i} -  \mu_3 \Phi_i^{\mathrm{T}} \Phi_i \Delta_i+  \mu_3\Phi_i^{\mathrm{T}} \mathcal{R}_i +d_i(t) \\
   =&- \mu_3 \Phi_i^{\mathrm{T}} \Phi_i \tilde{\Delta}_{i} +d_i(t).
\end{aligned}
\end{equation}
Solving Equation (\ref{EQ53}), we obtain
\begin{equation}
\begin{aligned}
 \tilde{\Delta}_i(t)=\tilde{\Delta}_i(t_0){\rm e}^{-\mu_3 \Phi_i^{\mathrm{T}} \Phi_i(t-t_0)}+\int_{t_0}^{t} d_i(\tau) {\rm e}^{- \mu_3 \Phi_i^{\mathrm{T}} \Phi_i(t -\tau)}\,{\rm d}\tau .
\end{aligned}
\end{equation}

Since $ \Phi_i^{\mathrm{T}} \Phi_i$ is positive and $d_i(t)$ converges to $0$ at a rate of $\alpha_{\Psi}$, $\lim _{t \rightarrow \infty }\tilde{\Delta}_{i}=0$ exponentially. Moreover, because $\mu_3 > \frac{\sigma_{\rm max}(S)}{\lambda_{\rm min}(\Phi_i^{\mathrm{T}} \Phi_i )}$, the exponential convergence rate of $\lim _{t \rightarrow \infty }\tilde{\Delta}_{i}=0$ is larger than $\sigma_{\rm max}(S)$.
$\hfill \hfill \blacksquare $
%
\begin{Lemma}\label{Lemma 5}%[\cite{chen2019}]
The distributed leader dynamics observers in (\ref{EQ15}) ensure that $\dot{\hat{\Pi}}_i$ and $\dot{\hat{\Pi}}_i  \zeta_k$ converge to zero exponentially.
\end{Lemma}



\begin{remark}
%Editor: Please ensure that the intended meaning has been maintained in the following edit.
Because $\dot{\hat{\Pi}}_i$ is a component of $\dot{\hat{\Delta}}_i$, the convergence rate of $\dot{\hat{\Pi}}_i$ is at least as large as that of $\dot{\tilde{\Delta}}_i$. Thus, the exponential convergence rate of $\dot{\hat{\Pi}}_i $ is larger than $\sigma_{\rm max}(S)$ according to $\mu_3 > \frac{\sigma_{\rm max}(S)}{\lambda_{\rm min}(\Phi_i^{\mathrm{T}} \Phi_i )}$. Then, it is obvious that $\dot{\hat{\Pi}}_i  \zeta_k$ converges exponentially to zero.
%Compare with the existing work \cite{cai2017} that the gain $\mu_3 > \frac{\alpha_{\Psi}}{\lambda_{\rm min}(\Phi_i^{\mathrm{T}} \Phi_i )}$, our work have more conservative result.
$\hfill \hfill  \square $
\end{remark}
\subsection{Adaptive Decentralized Resilient Controllers Design}
From the above Theorems, we can create a control scheme to achieve the output  containment synchronization of heterogeneous MAS under DoS attacks. Next, we combine the distributed observers, estimators and decentralized solvers and adaptive control to achieve the containment resilience of heterogeneous MAS against DoS attacks and unknown unbounded actuator attacks (\ref{EQ9}).

Define the state tracking errors as follows:
 \begin{align}
      \epsilon_i=x_i - \hat{\Pi}_i \zeta_i. \label{EQ58}
 \end{align}
Then, we present the decentralized adaptive attack-resilient control protocols on the CPL as follows:
\begin{align}
 & u_i=\hat{\Gamma}_i \zeta_i +K_i \epsilon_i -\hat{\chi}_i , \\  
&\hat{\chi}_i=\frac{B_i^{\mathrm{T}}  P_i \epsilon_i}{\left\lVert \epsilon_i^{\mathrm{T}} P_i B_i \right\rVert +\omega} \hat{\rho_i}, \\ \label{EQ65}
&\dot{\hat{\rho}}_i=\begin{cases}
 \left\lVert \epsilon_i^{\mathrm{T}} P_i B_i \right\rVert +2\omega,& \mbox{if}  \left\lVert \epsilon_i^{\mathrm{T}} P_i B_i \right\rVert \geq \bar{d}, \\
 \left\lVert \epsilon_i^{\mathrm{T}} P_i B_i \right\rVert +2\omega \frac{\left\lVert \epsilon_i^{\mathrm{T}} P_i B_i \right\rVert}{\bar{d}},& \mbox{ otherwise},
\end{cases}
\end{align}
where $\hat{\chi}_i$ is an adaptive compensation signal, $\hat{\rho}_i$ is an adaptive updating parameter, and the controller gain $K_i$ is designed as
\begin{equation}\label{EQ0968}
    K_i=-R_i^{-1}B_i^{\mathrm{T}} P_i,
\end{equation}
where $P_i$ is the solution to
\begin{equation}\label{EQ 64}
    A_i^{\mathrm{T}} P_i + P_i A_i + Q_i -P_i B_i R_i^{-1} B_i^{\mathrm{T}} P_i =0.
\end{equation}
\begin{myTheo}
Consider a heterogeneous MAS (\ref{EQ1})-(\ref{EQ2}) with $M$ leaders and $N$ followers in the present of the composite attacks including unbounded actuation attacks. Under Assumptions \ref{assumption 1}-\ref{assumption 5}, \textbf{Problem ACMCA} can be solved by designing the leader dynamic observers (\ref{EQ15}), distributed state estimators (\ref{equation 200}), decentralized output regulator equation solvers (\ref{EQ48}) and 
%Editor: Please ensure that the intended meaning has been maintained in the following edit.
decentralized adaptive controllers consisting of (\ref{EQ58})-(\ref{EQ 64}).
\end{myTheo}

\textbf{Proof.}
In Theorems 1 and 2, we proved that the TL can resist DoS attacks for which the DoS attack frequency satisfies Theorem 2. In the following, we show that the state tracking errors (\ref{EQ58}) are UUB under DoS attacks and unbounded actuation attacks.

The derivative of $\epsilon_i$ is given as follows:
  %\begin{equation}\label{EQ63}
  %\begin{aligned}
  \begin{flalign}\label{EQ63}
   \dot{\epsilon}_i
      =& A_i x_i + B_i u_i +B_i \chi_i -\dot{\hat{\Pi}}_i  \zeta_i &\notag \\
      &\! - \! \hat{\Pi}_i \!  \left( \! \hat{S}_i \zeta_i \!+\! \mu_2 G \!\left( \! \sum _{j \in \mathcal{F} } \!d_{ij}(\zeta_j-\zeta_i) \!+ \!\sum_{k \in \mathcal{L}}  \! d_{ik}(\zeta_k-\zeta_i  \!)\right)\!  \right) \! &\notag \\
      =&(A_i+B_i K_i)\epsilon_i  - \dot{\hat{\Pi}}_i \zeta_i &\notag\\
      &-\mu_2 \hat{\Pi}_i G  \! \left( \!  \sum _{j \in \mathcal{F} } \! d_{ij}(\zeta_j-\zeta_i) \! + \! \sum_{k \in \mathcal{L}} \! d_{ik}(\zeta_k-\zeta_i) \! \right) \! + \! B_i \tilde{\chi}_i,
  \end{flalign}
  %\end{aligned}
  %\end{equation}
where $\tilde{\chi}_i=\chi_i-\hat{\chi}_i$.
 
 The global state tracking error of (\ref{EQ63}) is
 \begin{flalign}
 \dot{\epsilon}
  =&\bar{A}_b\epsilon -\dot{\hat{\Pi}}_b \zeta +\! \mu_2 \hat{\Pi}_b G_b \! \left( \! \sum_{k \in \mathcal{L}}\!(\Psi_k^D \!\otimes\! I_p )(\zeta \!-\!\bar{\zeta}_k) \! \right)\!+\!B_b\tilde{\chi}
 &\notag\\   
 % =&\bar{A}_b\epsilon -\dot{\hat{\Pi}}_b(\tilde{\zeta}+ \left(\sum_{r \in \mathcal{L}}(\Psi_r \otimes I_p )\right)^{-1} \sum_{k \in \mathcal{L}}(\Psi_k \otimes I_p ) \bar{\zeta}_k) \\
  %  & + \mu_2  \hat{\Pi}_b G_b \sum_{k \in \mathcal{L}}(\Psi_k^D \otimes I_p )\tilde{\zeta} 
  %  +      \mu_2  \hat{\Pi}_b G_b \sum_{k \in \mathcal{L}}(\Psi_k^D \otimes I_p )\\
   % &(\left(\sum_{r \in \mathcal{L}}(\Psi_r \otimes I_p )\right)^{-1} \sum_{\bar{k} \in \mathcal{L}}(\Psi_{\bar{k}} \otimes I_p ) \bar{\zeta_{\bar{k}}) -  \bar{\zeta}_k))     +        B_b \tilde{\chi}    \\
   =&\bar{A}_b\epsilon -\dot{\hat{\Pi}}_b \left( \tilde{\zeta}+ \left(\sum_{r \in \mathcal{L}}(\Psi_r \otimes I_p )\right)^{-1} \sum_{k \in \mathcal{L}}(\Psi_k \otimes I_p ) \bar{\zeta}_k \right) &\notag\\
   &+ \mu_2 \hat{\Pi}_b G_b \sum_{k \in \mathcal{L}}(\Psi_k^D \otimes I_p )\tilde{\zeta} +B_b \tilde{\chi}
,
 \end{flalign}

where $\bar{A}_b={\rm blkdiag}(A_i+B_i K_i)$, $B_b={\rm blkdiag}(B_i)$,  $\dot{\hat{\Pi}}_b={\rm blkdiag}(\dot{\hat{\Pi}}_i)$, $\hat{\Pi}_b= {\rm blkdiag}(\hat{\Pi}_i)$  for $i=\textbf{I}[1,N]$ and $\epsilon=[\epsilon_1^{\mathrm{T}}, \epsilon_2^{\mathrm{T}}, \dots ,\epsilon_N^{\mathrm{T}}]^{\mathrm{T}}$, $\tilde{\chi}=[\tilde{\chi}_1^{\mathrm{T}}, \tilde{\chi}_2^{\mathrm{T}}, \dots, \tilde{\chi}_N^{\mathrm{T}}]^{\mathrm{T}}$.
%
%


Consider the following Lyapunov function:
\begin{equation} \label{EQ64}
    V= \epsilon ^{\mathrm{T}} P_b \epsilon,
\end{equation}
where $P_b={\rm blkdiag}(P_i)$.
The time derivate of (\ref{EQ64}) is given as follows:
\begin{flalign}\label{EQ80}
    \dot{V} 
    =& 2\epsilon^{\mathrm{T}} P_b \dot{\epsilon}&\notag \\
    =&-\epsilon^{\mathrm{T}} Q_b \epsilon +2\epsilon^{\mathrm{T}} P_b \left(\dot{\hat{\Pi}}_b+  \mu_2 \hat{\Pi}_b G_b \sum_{k \in \mathcal{L}}(\Psi_k^D \otimes I_p ) \right)\tilde{\zeta}&\notag \\
    & \!+ \!2\epsilon^{\mathrm{T}}  P_b \dot{\hat{\Pi}}_b  \! \left(  \sum_{r \in \mathcal{L}}(\Psi_r \otimes I_p ) \right) \! ^ {-1} \! \sum_{k \in \mathcal{L}} (\Psi_k \otimes I_p ) \bar{\zeta}_k\! &\notag\\
    & +\! 2\epsilon^{\mathrm{T}} \!P_b B_b\tilde{\chi}&\notag\\
    \leq& -\sigma_{\rm min}(Q_b) \left\lVert \epsilon\right\rVert^2   
    + 2\epsilon^{\mathrm{T}} P_b B_b\tilde{\chi}&\notag\\
    &+2\left\lVert \epsilon^{\mathrm{T}}\right\rVert \left\lVert P_b\right\rVert  \left\lVert \dot{\hat{\Pi}}_b+  \mu_2 \hat{\Pi}_b G_b \sum_{k \in \mathcal{L}}(\Psi_k \otimes I_p )  \right\rVert  \left\lVert \tilde{\zeta}\right\rVert&\notag\\ 
 & \! + \! 2  \! \left\lVert  \epsilon^{\mathrm{T}}\right\rVert  \! \left\lVert \!  P_b\right\rVert \!   \left\lVert \!  \dot{\hat{\Pi}}_b  \! 
  \left(\sum_{r \in \mathcal{L}}(\Psi_r \otimes I_p ) \! \right) \! ^{-1} \! \sum_{k \in \mathcal{L}} \! (\Psi_k \otimes I_p )   \bar{\zeta}_k \! \right\lVert, 
\end{flalign}
where $Q_b={\rm blkdiag}(Q_i)$.
%
%
%
%

Because $\dot{\hat{\Pi}}_i $ and $ \dot{\hat{\Pi}}_i \zeta_k $ converge to zero exponentially in Lemma \ref{Lemma 5}, we have
\begin{equation}\label{EQ81}
    \left\lVert  \dot{\hat{\Pi}}_b 
  \left(\sum_{r \in \mathcal{L}}(\Psi_r \otimes I_p )\right)^{-1}\sum_{k \in \mathcal{L}}(\Psi_k \otimes I_p ) \bar{\zeta}_k\right\rVert
  \leq  V_{\Pi} \exp (-\alpha_{\Pi})
\end{equation}
with $ V_{\Pi}>0$ and $\alpha_{\Pi}>0$.
According to Young's inequality, we have
\begin{equation}\label{EQ61}
\begin{aligned}
    & 2\left\lVert \epsilon^{\mathrm{T}}\right\rVert \left\lVert P_b\right\rVert   \left\lVert  \dot{\hat{\Pi}}_b 
  \left(\sum_{r \in \mathcal{L}}(\Psi_r \otimes I_p )\right)^{-1}\sum_{k \in \mathcal{L}}(\Psi_k \otimes I_p ) \bar{\zeta}_k\right\rVert \\
  \leq& \left\lVert \epsilon\right\rVert^2 + \left\lVert P_b\right\rVert  ^2 \left\lVert \dot{\hat{\Pi}}_b 
  \left(\sum_{r \in \mathcal{L}}(\Psi_r \otimes I_p )\right)^{-1}\sum_{k \in \mathcal{L}}(\Psi_k \otimes I_p ) \bar{\zeta}_k\right\rVert^2 \\
  \leq& \left(\frac{1}{4} \sigma_{\rm min}(Q_b)\! - \frac{1}{2}b_{ 1 } \! \right)\! \left\lVert \epsilon\right\rVert ^2 \!+\! \frac{\left\lVert P\right\rVert  ^2}{ \left(\frac{1}{4} \sigma_{\rm min}(Q_b) - \frac{1}{2}b_{ 1 } \! \right)} b_{\Pi}^2 {\rm e}^{-2 \beta_{\Pi}t}\\
  \leq&
  \left(\frac{1}{4} \sigma_{\rm min}(Q_b) - \frac{1}{2}b_{ 1 } \right)\left\lVert \epsilon\right\rVert ^2 +b_{21}{\rm e}^{-2\beta_{21}t}.
\end{aligned}
\end{equation}
From Lemma \ref{Lemma 5}, we obtain that $\dot{\hat{\Pi}}$ converges to zero exponentially. Because $\tilde{\zeta}$ also converges exponentially to zero, we obtain that
\begin{equation}\label{EQ62}
\begin{aligned}
   & 2\left\lVert \epsilon^{\mathrm{T}}\right\rVert \left\lVert P\right\rVert  \left\lVert \dot{\hat{\Pi}}_b+  \mu_2 \hat{\Pi}_b G_b \sum_{k \in \mathcal{L}}(\Psi_k \otimes I_p )  \right\rVert  \left\lVert \tilde{\zeta}\right\rVert \\
  \leq& \left(\frac{1}{4} \sigma_{\rm min}(Q_b) - \frac{1}{2}b_{ 1 } \right)\left\lVert \epsilon\right\rVert ^2+b_{22} {\rm e}^{-2\beta_{22}t}.
\end{aligned}
\end{equation}

Next, we note that
%\begin{equation}
%\begin{aligned}
\begin{flalign}
\epsilon_i^{\mathrm{T}} P_i  B_i\tilde{\chi}_i
        =& \epsilon_i^{\mathrm{T}} P_i  B_i \chi_i -\frac{\left\lVert \epsilon_i^{\mathrm{T}} P_i B_i \right\rVert^2}{ \left\lVert \epsilon_i^{\mathrm{T}} P_i B_i \right\rVert +\omega} \hat{\rho_i} &\notag \\
\leq&  \left\lVert \epsilon_i^{\mathrm{T}} P_i  B_i\right\rVert \left\lVert \chi_i \right\rVert - \frac{\left\lVert \epsilon_i^{\mathrm{T}} P_i B_i \right\rVert^2  }{\left\lVert \epsilon_i^{\mathrm{T}} P_i B_i \right\rVert + \omega} \hat{\rho_i} &\notag \\
=& \frac{\left\lVert \epsilon_i^{\mathrm{T}} P_i B_i \right\rVert ^2 \left(  \left\lVert \chi_i \right\rVert-  \hat{\rho}_i \right)+ \left\lVert \epsilon_i^{\mathrm{T}} P_i B_i \right\rVert\left\lVert \chi_i \right\rVert \omega}{\left\lVert \epsilon_i^{\mathrm{T}} P_i B_i \right\rVert +\omega } &\notag \\
=&\frac{\left\lVert \epsilon_i^{\mathrm{T}} P_i B_i \right\rVert ^2 \left(\frac{\left\lVert \epsilon_i^{\mathrm{T}} P_i B_i \right\rVert +\omega}{\left\lVert \epsilon_i^{\mathrm{T}} P_i B_i \right\rVert}||\chi_i||-\hat{\rho}_i \right)}{\left\lVert \epsilon_i^{\mathrm{T}} P_i B_i \right\rVert +\omega}.
\end{flalign}
%\end{aligned}
%\end{equation}
Note that $d\left\lVert \chi_i \right\rVert/dt$ is bounded. Thus, if $\left\lVert \epsilon_i^{\mathrm{T}} P_i B_i \right\rVert\geq \bar{d} \geq \frac{d||\chi_i||}{d t}$, that is, $\frac{\bar{d}+\omega}{\bar{d}} \frac{d||\chi_i||}{dt} -\dot{\hat{\rho}} \leq \bar{d}+\omega-\dot{\hat{\rho}} \leq -\omega < 0$, there exists $t_{\chi} > 0$ such that for all $t \geq t_{\chi}$, we have
\begin{equation}\label{EQ76}
     \left(\frac{\left\lVert \epsilon_i^{\mathrm{T}} P_i B_i \right\rVert +\omega}{\left\lVert \epsilon_i^{\mathrm{T}} P_i B_i \right\rVert}||\chi_i||-\hat{\rho}_i \right) \leq \left(\frac{\bar{d}+\omega}{\bar{d}}||\chi_i||-\hat{\rho}_i \right) <0.
\end{equation}
Thus, $\epsilon_i^{\mathrm{T}} P_i  B_i\tilde{\chi}_i <0$ and $\epsilon^{\mathrm{T}} P_b  B_b\tilde{\chi}<0$ over $t \in[t_{\chi},\infty)$.

According to Equations (\ref{EQ81}) $\sim$ (\ref{EQ76}), we have
\begin{equation}\label{EQ70}
  \dot{V} \leq -\left(b_{ 1 }+\frac{1}{2} \sigma_{\rm min}(Q_b) \right) \epsilon^{\mathrm{T}} \epsilon +b_{ 2 }{\rm e}^{-\beta_{2}t}, ~t \geq t_{\chi}.
\end{equation}

Solving (\ref{EQ70}) yields the following:
\begin{flalign}
&V(t) \leq  V(0) \! -\! \!\int_{0}^{t} \! \left(b_{ 1 } \!+\!\frac{1}{2} \sigma_{\rm min}(Q_b) \right) \epsilon^{\mathrm{T}}\epsilon  \!\,{\rm d}\tau \!+\!\! \int_{0}^{t} \! b_{ 2 }{\rm e}^{-\beta_{2} \tau}\! \,{\rm d}\tau.&
\end{flalign}
Then, we obtain
\begin{equation}\label{EQ72}
    \epsilon^{\mathrm{T}} \epsilon \leq -\int_{0}^{t} \frac{1}{\sigma_{\rm min}(P)} b_{ 3 }  \epsilon^{\mathrm{T}}\epsilon\,{\rm d}\tau + \bar{B}.
\end{equation}
where $b_{ 3 } = b_{ 1 }+\frac{1}{2} \sigma_{\rm min}(Q_b) $ and $ \bar{B}=V(0)-\int_{0}^{t} b_{ 2 }{\rm e}^{-\beta_{2}\tau}\,{\rm d}\tau $ are bounded constants.
Recalling the Bellman-Gronwall Lemma, (\ref{EQ72}) is rewritten as
\begin{equation}
  \left\lVert \epsilon \right\rVert \leq \sqrt{\bar{B}} {\rm e}^{-\frac{b_{ 3 } t}{2\sigma_{\rm min}(P_b)} }.
\end{equation}
According to (70) and (74), we conclude that $\epsilon$ is bounded by $\bar{\epsilon}$, where $\bar{\epsilon}=[\bar{\epsilon}_1^{\mathrm{T}}, \bar{\epsilon}_2^{\mathrm{T}}, \dots,\bar{\epsilon}_N^{\mathrm{T}}]^{\mathrm{T}}$ with
$||\bar{\epsilon}_i||=\frac{\bar{d}}{\sigma_{\rm min}(P_i B_i)}$ for $i= \textbf{I}[1,N]$.
%

Hence, according to Lemma 5, the global output synchronization error satisfies

\begin{equation}
\begin{aligned}
e =& y - \left(\sum_{r\in \mathcal{L} }(\Phi_r \otimes I_p)\right)^{-1} \sum_{k \in \mathcal{L} } (\Psi_k \otimes I_p) \underline{y}_k \\
=&y-(I_N \otimes R)(\zeta-\tilde{\zeta})\\
=& {\rm blkdiag}(C_i)x   -{\rm blkdiag}(C_i \hat{\Pi}_i)\zeta +{\rm blkdiag}(C_i \hat{\Pi}_i)\zeta\\
&-(I_N \otimes R)\zeta +(I_N \otimes R) \tilde{\zeta} \\
=&  {\rm blkdiag}(C_i)\epsilon  -{\rm blkdiag}(\tilde{R}_i)\zeta +(I_N \otimes R) \tilde{\zeta}.
\end{aligned}
\end{equation}

%{\color{black}
Since we proved that $\epsilon$, $\tilde{R}_i$, and $\tilde{\zeta}$ converge to 0 exponentially, the global output containment error $e$ is bounded by $\bar{e}$, with $\bar{e}=[\bar{e}_1^{\mathrm{T}}, \bar{e}_2^{\mathrm{T}}, \dots ,\bar{e}_i^{\mathrm{T}}]^{\mathrm{T}}$ and $\bar{e}_i=\frac{\bar{d} ||C_i||}{\sigma_{\rm min}(P_i B_i)}$ for $i= \textbf{I}[1,N]$.
%}
Thus, the proof is completed.
$\hfill \hfill \blacksquare $

\begin{remark}
Compared with \cite{zuo2020}, which only solved the containment problem against unbounded attacks, we consider more challenge work with multiple leaders output
containment problem against composite attacks. Moreover, the boundary of the output containment error is given as $\frac{ ||C_i|| \bar{d}}{\sigma_{\rm min}(P_i B_i)}$.
%It's helpful for designers to pre-estimate of the controller.
$\hfill \hfill  \square $
\end{remark}


\section{Numerical Simulation}\label{SecSm}
\begin{figure}[!]
  %\begin{minipage}[t]{1\linewidth}
  \centering
  \includegraphics[width=0.2\textwidth]{pic/tp1.png}
  \caption{Topology graph.}
  \label{fig:figure3}
\end{figure}

\begin{figure}[!]
  %\begin{minipage}[t]{1\linewidth}
  \centering
  \includegraphics[width=0.45\textwidth]{picnew2/Upsilon.eps}
  \caption{The estimations of the leader dynamics according to Theorem 1: The shadowed areas denote the time intervals against DoS attacks.}
  \label{fig:figure4}
\end{figure}

\begin{figure}[htbp]
  %\begin{minipage}[t]{1\linewidth}
  \centering
  \includegraphics[width=0.45\textwidth]{picnew2/zeta21026.eps}
  \caption{Performance of the TL: The shadowed areas denoting the time intervals against DoS attacks.}
  \label{fig:figure5}
\end{figure}




\begin{figure}[!]
  %\begin{minipage}[t]{1\linewidth}
  \centering
  \includegraphics[width=0.45\textwidth]{picnew2/Delta1.eps}
  \caption{The estimated solution of the output regulator equation in Theorem 3: The shadowed areas denote the time intervals against DoS attacks.}
  \label{fig:figure6}
\end{figure}
  

\begin{figure}[!]
  %\begin{minipage}[t]{1\linewidth}
  \centering
  \includegraphics[width=0.45\textwidth]{picnew2/track2.eps}
  \caption{Output trajectories of the leaders and followers.}
  \label{fig:figure7}
\end{figure}



\begin{figure}[htbp]
  %\begin{minipage}[t]{1\linewidth}
  \centering
  \includegraphics[width=0.45\textwidth]{picnew2/ex.eps}
  \caption{Output containment errors.}
  \label{fig:figure8}
\end{figure}


\begin{figure}[!]
  %\begin{minipage}[t]{1\linewidth}
  \centering
  \includegraphics[width=0.45\textwidth]{picnew2/ebar.eps}
  \caption{The estimation error between the TL and CPL: The blue shadowed areas denote the UUB bound of $\bar{d}$.}
  \label{fig:figure9}
\end{figure}

In this section, we present two examples to illustrate the effectiveness of the proposed control protocol.



\subsection{Example 1}
We consider a MAS consisting of seven agents (three leaders indexed by 5$\sim$7 and four followers indexed by 1$\sim$4), with the corresponding graph shown in Fig. \ref{fig:figure3}. Set $a_{ij}=1$ if there exists a path from node $j$ to node $i$.
Consider the leader dynamics are described by
$$
S
=
 \left[
  \begin{array}{cc}
 0.5  & -0.4  \\
 0.8 &  0.5   \\ 
  \end{array}
  \right],R
  =
   \left[
    \begin{array}{cc}
   1 &  0    \\
   0  & 1  \\ 
    \end{array}
    \right],
$$
and the follower dynamics are given by
$$
A_1
\!=\!
 \left[\!
  \begin{array}{cc}
 3  & -2  \\
 1 &  -2   \\ 
  \end{array}
 \! \right],B_1
 \! =\!
   \left[\!
    \begin{array}{cc}
   1.8 &  -1    \\
   2  & 3  \\ 
    \end{array}
   \! \right],C_1
  \!=\!
   \left[\!
    \begin{array}{cc}
   -0.5 &  1    \\
  2  & -1.5  \\ 
    \end{array}
   \! \right],
$$
$$
A_2
\!=\!
 \left[\!
  \begin{array}{cc}
 0.6  & -1  \\
 1 &  -2   \\ 
  \end{array}
  \!\right],B_2
  \!=\!
   \left[\!
    \begin{array}{cc}
   1 &  -2    \\
   1.9  & 4  \\ 
    \end{array}
   \! \right],C_2
  \!=\!
   \left[\!
    \begin{array}{cc}
   -0.5 &  1    \\
  1.5  & 1.4  \\ 
    \end{array}
  \!  \right],
$$
$$
A_3=A_4
=
 \left[
  \begin{array}{ccc}
 0  &  1  &0\\
 0 &  0 & 1  \\ 
  0 &  0 & -2  \\ 
  \end{array}
  \right],B_3
  =B_4=
   \left[
    \begin{array}{ccc}
   6 &  0    \\
   0  & 1  \\ 
   1  & 0 \\ 
    \end{array}
    \right],$$
    $$
    C_3=C_4
  =
   \left[
    \begin{array}{ccc}
   0.5 &  -0.5  & 0.5  \\
  -0.5  & -0.5  &0.5\\ 
    \end{array}
    \right].
$$

The DoS attack periods are given as $[0.5+2k, 1.53+2k)s$ for $k\in \mathbb{N}$, which satisfies Assumption 3. The MAS can defend against FDI attacks and camouflage attacks, since the information transmitted on the CPL is not used in the hierarchical control scheme.
The actuation attacks are designed as $\chi_1=\chi_2=\chi_3=0.01 \times [2t \quad t]^{\mathrm{T}}$ and $\chi_4=-0.01\times[2t\quad t]^{\mathrm{T}}$.


 %the   light red zone denotes the UUB bound || eTi PiBi|| = d
 
The gains in (\ref{EQ15}), (\ref{equation 200}) and (\ref{EQ48}) are set as $\mu_1=2$, $\mu_2=0.5$ and $\mu_3=6$, respectively. According to Theorem 2, $P2 =\left[ \begin{array}{cc}
    0.6038 & 0.0108 \\
    0.0108 & 0.1455
\end{array}\right]$ and $G=\left[\begin{array}{cc}
     4.5191& 0.5186 \\
     0.5186& 4.0746
\end{array}\right]$. By employing the above parameters, the estimated trajectories of the leader dynamics are shown in Fig. \ref{fig:figure4}. The estimated leader dynamics $||\hat{\Upsilon}_i||$ against DoS attacks remained stable after $6 s$, and the value was equal to the estimated leader dynamics. The state estimation errors on the TL are shown in Fig. \ref{fig:figure5}, showing that the performance of the TL remains stable under DoS attacks.
The trajectories of the solution errors of the output regulator equations are shown in Fig. \ref{fig:figure6}, and the results validate Theorem 3.
To analyze the actuation attacks on the CPL, we design the controller gain in (\ref{EQ0968}) as
$$
K_1
=
 \left[
  \begin{array}{cc}
 -1.71  & 0.07  \\
 2.31  &  -1.12   \\ 
  \end{array}
  \right],K_2
  =
   \left[
    \begin{array}{cc}
   -0.62 &  -0.29   \\
   1.16  & -0.56  \\ 
    \end{array}
    \right],$$
    $$K_3=K_4
  =
   \left[
    \begin{array}{ccc}
   -1.00 &  -0.19  & -0.08  \\
  0.02  & -0.96  &-0.30\\ 
    \end{array}
    \right].
$$




Then, we solve \textbf{Problem ACMCA} according to Theorem 4. Fig. \ref{fig:figure7} displays the output trajectories of all agents. The followers eventually enter the convex hull formed by the leaders. The output containment errors of the followers are shown in Fig. \ref{fig:figure8}, demonstrating that the local output error $e_i(t)$ is UUB. The tracking error between the CPL and TL is depicted in Fig. \ref{fig:figure9}, showing that the tracking error $\epsilon_i$ is UUB by the prescribed error bound. These results show that the output containment problem can be solved under composite attacks.

\subsection{Example 2}

\begin{figure}[!]
  %\begin{minipage}[t]{1\linewidth}
  \centering
  \includegraphics[width=0.25\textwidth]{picnew2/txtp1005.png}
  \caption{Communication topology graph of a UAV swarm, with the blue and green UAVs representing the leaders and followers, respectively.}
  \label{fig:figure10}
\end{figure}

Next, we consider a practical example involving unmanned aerial vehicles (UAVs), and the corresponding graph is shown in Fig. \ref{fig:figure10}. We assume that all edges have a weight of 1.
 
 
%  Choose
%  $$\mathcal{A}_f=\left[ \begin{array}{ccccc}
%      0 & 0& 0& 0& 1 \\
%       1 & 0& 0& 0& 1\\
%       1 & 1& 0& 0& 0\\
%       0 & 0& 1& 0& 0\\
%       0 & 0& 0& 1& 0\\
%  \end{array}
%  \right],
%  $$
%  and 
%   $$G_{ik}=\left[ \begin{array}{ccccc}
%      1 & 0& 0\\
%       0 & 0&0 \\
%       0 & 0&0\\
%       0 & 0&1 \\
%       0 & 1& 0\\
%  \end{array}
%  \right],
%  $$
%  where $\mathcal{A}_f$ represent  the associated adjacency matrix between followers and $G_{ik}={\rm diag}(g_{ik})$ with $g_{ik}$  is the weight of the path from $i$th leader to $k$th follower. 
 
 The dynamics of the leader and follower UAVs can be approximated as the following second-order systems \cite{wang2018optimal,dong2016time}:
 \begin{equation}
  \begin{cases}
 \dot{p}_k(t)=v_k(t), \\
 \dot{v}_k(t)=\alpha_{p_0} p_k (t) + \alpha_{v_0} v_k(t),\\
 y_k(t)=p_k(t),
\end{cases}   
 \end{equation}
and
\begin{equation}
  \begin{cases}
 \dot{p}_i(t)=v_i(t), \\
 \dot{v}_i(t)=\alpha_{p_i} p(t) + \alpha_{v_i} v_i(t)+u_i(t),\\
 y_i(t)=p_i(t),
\end{cases}  
\end{equation}
where $p_k(t)$ and $v_k(t)$ ($p_i(t)$ and $v_i(t)$) represent the position and velocity of the $k$th leader UAV ($i$th follower UAV), respectively. The damping constants of the leaders are set as $\alpha_{p_0}=-0.6$ and $\alpha_{v_0}=0$. The damping constants of the followers are set as $\alpha_{p_1}=-1, \alpha_{v_1}=-1$, $\alpha_{p_2}=-0.5, \alpha_{v_2}=-1.2$, $\alpha_{p_3}=-1.2, \alpha_{v_3}=-1$, $\alpha_{p_4}=-0.8, \alpha_{v_4}=-0.5$, and $\alpha_{p_5}=-0.4, \alpha_{v_5}=-1.2$. The gains are set as $\mu_1=5$, $\mu_2=1$, and $\mu_3=7$. We set the DoS attack period as $[0.2+2k, 1.86+2k)s$ for $k\in \mathbb{N}$, and the actuation attacks are defined as $\chi_1=[0.01t\quad0.02t\quad0.02t]^{\mathrm{T}},\chi_2=[0.02t\quad0.01t\quad0.02t]^{\mathrm{T}},\chi_3=[0.02t\quad0.02t\quad0.02t]^{\mathrm{T}},\chi_4=[-0.01t\quad-0.02t\quad-0.02t]^{\mathrm{T}},\chi_5=[-0.02t\quad-0.01t\quad0.02t]^{\mathrm{T}}$.
 
\begin{figure}[!]
  %\begin{minipage}[t]{1\linewidth}
  \centering
  \includegraphics[width=0.45\textwidth]{picnew/Upsilon.eps}
  \caption{The leader dynamics model of the TL in Theorem 1: The shadowed areas denote the time intervals against DoS attacks.}
  \label{fig:figure11}
\end{figure}
 
 
  %\begin{figure*}[htbp]
\begin{figure}[htbp]
  \centering
    \includegraphics[width=0.45\textwidth]{picnew/zeta31026.eps}
  \caption{Distributed estimation performance of the TL, with the shadowed areas denoting the time intervals against DoS attacks.}
   \label{fig:figure12}
\end{figure}

Set the initial values of the updating parameters in (\ref{EQ15}) and (\ref{EQ48}) to zero. Then, we set the initial leader's state as $p_6=[1\quad1 \quad 1]^{\mathrm{T}}, v_6=[4 \quad 5\quad4]^{\mathrm{T}}$, $p_7=[4 \quad 1 \quad 2]^{\mathrm{T}}, v_7=[5 \quad 6 \quad 6]^{\mathrm{T}}$, and $p_8=[6 \quad 5 \quad 8]^{\mathrm{T}}, v_6=[4 \quad 6 \quad 5]^{\mathrm{T}}$ and randomly set the remaining initial values. Then, according to Theorems 2 and 4, the state estimator gain of the TL is  $G=\left[\begin{array}{cc}
     2.8483 & 0.2594 \\
     0.2594 & 3.0611
 \end{array}\right]$, and the gain of the CPL is $K_1=[-0.4142 \quad 	-0.6818]$, $K_2= [-0.6180 \quad 	-0.7173]$, $K_3=[-0.3620 \quad 	-0.6505]$, $K_4=[-0.4806	 \quad -0.9870]$, $K_5= [-0.6770	 \quad -0.7478]$.

 
 \begin{figure}[!]
  %\begin{minipage}[t]{1\linewidth}
  \centering
  \includegraphics[width=0.45\textwidth]{picnew/Delta.eps}
  \caption{The estimated solution of the output regulator equation in Theorem 3: The shadowed areas denote the time intervals against DoS attacks.}
  \label{fig:figure13}
\end{figure}
 
 
\begin{figure}[!]
  %\begin{minipage}[t]{1\linewidth}
  \centering
  \includegraphics[width=0.45\textwidth]{picnew/track/track3.eps}
  \caption{Trajectories of the UAV swarm between $0 \sim 30 s$. The hollow and solid blocks represent the start and end points of the UAVs, respectively.}
  \label{fig:figure14}
\end{figure}


 
  
    \begin{figure}[htbp]
  \centering 
  \includegraphics[width=0.45\textwidth]{picnew/ex.eps}
  \caption{The output containment performance of our control scheme.}
  \label{fig:figure15}
  \end{figure}
 

  
\begin{figure}[htbp]
  %\begin{minipage}[t]{1\linewidth}
  \centering
  \includegraphics[width=0.45\textwidth]{picnew/ebar.eps}
  \caption{The estimation error between the TL and CPL: The blue shadow areas denote the UUB bound of $\bar{d}$.}
  \label{fig:figure16}
\end{figure}

Fig. \ref{fig:figure11} reveals that $\hat{\Upsilon}_i$ converges to $\Upsilon$ under DoS attacks, which implies that the modeling errors of leader dynamics converge to $0$. The state estimator errors of the TL are shown in Fig. \ref{fig:figure12}, demonstrating the good performance of the TL under DoS attacks. Fig. \ref{fig:figure13} verifies the validity of Theorem 3.
We depict the output trajectories of the reference UAVs over time in Fig. \ref{fig:figure14}, where the initial and final positions of the UAVs are marked by hollow and filled squares, respectively. The trajectory of each follower remains in a small neighborhood around the dynamic convex hull spanned by the leaders after $t=10 s$, that is, output containment is achieved, which can also be seen in Fig. \ref{fig:figure15}.
Fig. \ref{fig:figure16} shows the rationality of the upper bound of the output containment errors.


% \begin{figure}[!htbp]
% %\begin{minipage}[t]{1\linewidth}
% \centering
% \includegraphics[width=0.6\textwidth]{4Ag.pdf}
% \caption{Time-varying directed communication topology among all agents}
% \label{fig:figure1}
% \end{figure}



%{\color{blue}
%\begin{figure}[htbp]
%\centering
%\subfigure[Performance of observer w.r.t. the leader]{
%\begin{minipage}[t]{0.475\textwidth}
%\centering
%\includegraphics[width=0.85\textwidth]{pic/pobs.eps}
%%\caption{fig1}
%\end{minipage}\label{fig:figure2:1}
%}
%%\hspace{-0.1in}
%\subfigure[Performance of observer w.r.t. the first leader]{
%\begin{minipage}[t]{0.475\textwidth}
%\centering
%\includegraphics[width=0.85\textwidth]{pic/vobs.eps}
%%\caption{fig2}
%\end{minipage}\label{fig:figure2:2}
%}\\%
%\centering
%\caption{Performance of two observers}
%\label{fig:figure2}
%\end{figure}












\section{Conclusion}
The distributed resilient output containment problem of heterogeneous MASs against composite attacks is investigated in this work. Inspired by hierarchical protocols, a TL that can resist to most attacks (including FDI attacks, camouflage attacks, and actuation attacks) is provided to decouple the defense strategy into two tasks. {\color{blue}The first one considers defense against DoS attacks on the TL, and the other one considers defense against unbounded actuation attacks on the CPL.
To address the first task, we introduce a TL to solve the modeling errors of leader dynamics and defend against DoS attacks. Then,
distributed observers and distributed estimators are used to reconstruct the leader dynamics and the follower states under DoS attacks on the TL. To address the second task, output regulator equation solvers and adaptive decentralized control schemes are introduced to address the output containment problem and resist unbounded actuation attacks on the CPL.} Finally, we prove that the 
%Editor: Please ensure that the intended meaning has been maintained in the following edit.
upper bound 
of the output containment error is UUB under composite attacks and give the error bound explicitly. Simulations are provided to illustrate the effectiveness of the proposed methods.




 

\

% Proof: Consider the Lyapunov function candidate
% $$
% V_{1}=\frac{1}{2} \sum_{i=1}^{N} \xi_{i}^{T} P \xi_{i}+\sum_{i=1}^{N} \sum_{j=1, j \neq i}^{N} \frac{\left(c_{i j}-\alpha\right)^{2}}{8 \kappa_{i j}}
% $$
% where $\alpha$ is a positive constant that is to be determined later. Evidently, $V_{1}$ is positive definite. The time derivative of $V_{1}$ along the trajectory of (5) is given by
% $$
% \begin{aligned}
% \dot{V}_{1}=& \sum_{i=1}^{N} \xi_{i}^{T} P \dot{\xi}_{i}+\sum_{i=1}^{N} \sum_{j=1, j \neq i}^{N} \frac{c_{i j}-\alpha}{4 \kappa_{i j}} \dot{c}_{i j} \\
% =& \sum_{i=1}^{N} \xi_{i}^{T} P A \xi_{i}+\sum_{i=1}^{N} \xi_{i}^{T} P B K \sum_{j=1}^{N} c_{i j} a_{i j}\left(\tilde{x}_{i}-\tilde{x}_{j}\right) \\
% &+\sum_{i=1}^{N} \sum_{j=1, j \neq i}^{N} \frac{c_{i j}-\alpha}{4 \kappa_{i j}} \dot{c}_{i j}
% \end{aligned}
% $$
% Since $a_{i j}=a_{j i}$ and $c_{i j}(t)=c_{j i}(t)$, it can be easily verified that
% $$
% \begin{array}{rl}
% \sum_{i=1}^{N} \xi_{i}^{T} & P B K \sum_{j=1}^{N} c_{i j} a_{i j}\left(\tilde{x}_{i}-\tilde{x}_{j}\right) \\
% =&-\frac{1}{2} \sum_{i=1}^{N} \sum_{j=1}^{N} c_{i j} a_{i j}\left(\xi_{i}-\xi_{j}\right)^{T} \Gamma\left(\tilde{x}_{i}-\tilde{x}_{j}\right)
% \end{array}
% $$

\bibliography{PIDFR}
\end{document}\grid

