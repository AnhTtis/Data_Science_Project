
\documentclass[letterpaper, 12pt, journal, twoside]{support/IEEEtran}
\usepackage[fleqn]{amsmath}
\usepackage{times}
\usepackage[pdftex]{graphicx}
\usepackage{subfigure}
\usepackage{amsmath,amssymb,amsopn,amstext,amsfonts}
\usepackage{cancel}
\usepackage[noadjust]{cite}
\usepackage{soul}
\usepackage{caption}
\captionsetup{font={small}}

\captionsetup[figure]{labelfont={},textfont={}}


\usepackage{balance}
\usepackage{color}
\usepackage{mathtools}
% \usepackage{algorithm}
% \usepackage{algorithmic}
\usepackage{bm}
%\newtheorem{theorem}{Theorem}
\usepackage{ diagbox}
\usepackage{float}
\usepackage{epstopdf}
\usepackage{url}
\usepackage{multirow}
\usepackage{tikz}
\usepackage{subeqnarray}
\usepackage{cases}
\usepackage{booktabs}
\usepackage[linkcolor=black,citecolor=black,urlcolor=black,colorlinks=true]{hyperref}
\usepackage{algorithm}
\usepackage[noend]{algpseudocode}
\newtheorem{myTheo}{Theorem}
%\newtheorem{thm}{Theorem}[section] %如果不采用章节号做前缀,则不用[section]
\newtheorem{myDef}{Definition} %这句定义使得defn环境和thm共享编号
\newtheorem{lemma}{Lemma} %这句定义使得lem环境和thm共享编号
\newtheorem{myCollo}{Corollary}
\newtheorem{remark}{Remark}
%\newtheorem{lemma}{Lemma}
\newtheorem{myPro}{Proposition}
\newtheorem{assumption}{Assumption}
\newtheorem{example}{Example}
\soulregister\cite7
\soulregister\citep7
\soulregister\citet7
\soulregister\ref7
\soulregister\it7
\soulregister\pageref7

\bibliographystyle{support/IEEEtran}

\newcommand\px{\mathrel{/\mkern-5mu/}}  %平行
\newcommand{\ann}[1]{%
    \begin{tikzpicture}[remember picture, baseline=-0.75ex]%
        \node[coordinate] (inText) {};%
    \end{tikzpicture}%
    \marginpar{%
        \renewcommand{\baselinestretch}{1.0}%
        \begin{tikzpicture}[remember picture]%
            \definecolor{orange}{rgb}{1,0.5,0}%
            \draw node[fill=red!20,rounded corners,text width=\marginparwidth] (inNote){\footnotesize#1};%
    \end{tikzpicture}%
    }%
    \begin{tikzpicture}[remember picture, overlay]%
        \draw[draw = orange, thick]
            ([yshift=-0.2cm] inText)
                -| ([xshift=-0.2cm] inNote.west)
                -| (inNote.west);%
    \end{tikzpicture}%
}%

\graphicspath{{figures/}}
\DeclareGraphicsExtensions{.pdf,.png,.jpg,.eps}
\IEEEoverridecommandlockouts
%\overrideIEEEmargins

\title{\LARGE \bf Resilient Output Containment Control of Heterogeneous Multi-agent Systems against Composite Attacks: A Novel Digital Twin Approach}

%\title{Distributed Optimization in Prescribed-Time: Theory and Experiment}%
\author{
  \vskip 1em
  { 
  Xin Gong, \emph{Graduate Student Member, IEEE}, 
	Yukang Cui, \emph{Member, IEEE},
  Lingbo Cao
  }

  \thanks{
    This work was partially supported by the National Natural Science Foundation of China under Grant 61903258, 61973156, 61603180, Qatar National Research Fund NPRP12C-0814-190012. %(\emph{Corresponding author: Yukang Cui.}) %the National Natural Science Foundation of China under Grant 61903258

X. Gong is with the Department of Mechanical Engineering, The University of Hong Kong, Pokfulam Road, Hong Kong (e-mail: {\tt\small gongxin@connect.hku.hk}).


Y. Cui and T. Wang are with the College of Mechatronics and Control Engineering, Shenzhen University, Shenzhen, 518060, China (e-mail: {\tt\small cuiyukang,szuwtn@gmail.com}).


  
%J. He is with the Department of Mechanical Engineering, The University of Hong Kong, Pokfulam Road, Hong Kong (e-mail: {\tt\small esmehe@connect.hku.hk}). 

%X. Gong is with the Department of Mechanical Engineering, The University of Hong Kong, Pokfulam Road, Hong Kong, and also with the College of Mechatronics and Control Engineering, Shenzhen University, Shenzhen 518060, China. (e-mail: {\tt\small gongxin@connect.hku.hk}).
%China, and also
%with the Department of Mechanical Engineering, University of Hong Kong,
%Hong Kong
    
  }
%\thanks{$^{*}$ means the corresponding author.}
}

%\maketitle
%\author{}%\vspace{-0.0cm}
%%\thanks{This work was partially supported by.}% <-this % stops a space
%\thanks{$^{*}$These authors contribute equally and share the first authorship.}
%\thanks{$^{1}$Author is with the Group Robotics with Intelligent Planning (GRIP) Lab, Department of Mechanical Engineering, University of Hong Kong, Hong Kong,
%   {\tt\small gongxin@connect.hku.hk}}
%\thanks{Digital Object Identifier (DOI): see the top of this page.}
%\vspace{-0.5cm}}

% The note headers
%\markboth{Journal of \LaTeX\ Class Files,~Vol.~14, No.~8, August~2015}%
%{Shell \MakeLowercase{\textit{et al.}}: Bare Demo of IEEEtran.cls for IEEE Journals}
\markboth{IEEE Transactions on ...}{GONG \MakeLowercase{\textit{et al.}}: Resilient Output Containment Control of Heterogeneous MAS}%{He \MakeLowercase{\textit{et al.}}: Resilient Path Planning of UAVs against Covert Attacks on UWB Sensors}



\begin{document}
  \maketitle
  \begin{abstract}
    This brief deals with the distributed consensus observer design problem for high-order integrator multi-agent systems on directed graphs, which intends to estimate the leader state accurately in a prescribed time interval. A new kind of distributed prescribed-time observers (DPTO) on directed graphs is first formulated for the followers, which is implemented in a cascading manner. Then, the prescribed-time zero-error estimation performance of the above DPTO is guaranteed for both time-invariant and time-varying directed interaction topologies, based on strictly Lyapunov stability analysis and mathematical induction method. Finally, the practicability and validity of this new distributed observer are illustrated via a numerical simulation example.
\end{abstract}
\begin{IEEEkeywords}
  Consensus observer, Directed graphs, High-order Multi-agent systems, Prescribed-time stability
% Periodic positive systems, hyper-pyramid,
% reachable set estimation, S-procedure, state-feedback control.
%Formation-containment control,  high-order multi-agent systems,  observer-type protocols,  time-varying formation configuration
\end{IEEEkeywords}
\section{Introduction}
\IEEEPARstart{T}{he} last decade has witnessed substantial progresses contributed by various industries (see \cite{ xu2020distributed, liang2016leader, hua2017distributed, de2014controlling} and references therein) on distributed coordination of multi-agent systems (MAS). 








Two inevitable yet challenging difficulties arise when designing distributed finite-time observers for MAS:

\begin{enumerate}
  \item How can all followers obtain finite-time zero-error converge since the actual states of each pinned leader are only available to only a portion of followers, especially on a directed topology?
  \item How to regulate the consensus observation time arbitrarily despite the influences of the initial states of the MAS and network algebraic connectivity, particularly for high-order MAS on large directed networks?
\end{enumerate}





\begin{enumerate}
\item  A new kind of cascaded DPTO, based on a \textbf{hybrid constant and time-varying feedback}, is developed for the agents on directed graphs, which achieves the distributed accurate estimation on each order of leader state in a cascaded manner. 
\item \textbf{Zero-error converge on time-invariant/varying directed graphs}: In contrast to the previous work \cite{zuo2019distributed} which only guarantees finite-time attractiveness of a fixed error bound, we manage to regulate the observation error on directed graphs into zero in a finite-time sense. It is further found that the feasibility of this DPTO can be extended to some time-varying directed graphs.
%The difficulties caused by the asymmetrical Laplacian matrix under the circumstance of single-way directed communication topology are circumvented in the frameworks of distributed prescribed-time fault-tolerant control.
\item \textbf{Prescribed-time converge}: This DPTO could achieve distributed zero-error estimation in a predefined-time manner, whose needed time interval is independent of the initial states of all agents and network algebraic connectivity. Thus, the observer design procedure is much more easily-grasped for the new users than that in \cite[Theorem 2]{zuo2019distributed}.
\end{enumerate}



\noindent\textbf{Notations:}
 In this brief, $\boldsymbol{1}_m$ (or $\boldsymbol{0}_m$) denotes a column vector of size $m$ filled with $1$ (respectively, 0). Denote the index set of sequential integers as $\textbf{I}[m,n]=\{m,m+1,\ldots~,n\}$ where $m<n$ are two natural numbers. Define the set of real numbers as $\mathbb{R}$ and let $\mathbb{N}$ be a set of positive natural numbers. Define ${\rm {\rm blkdiag}}(A_1,A_2,\dots,A_N)$ as a block diagonal matrix whose principal diagonal elements equal to given matrices $A_1,A_2,\dots,A_N$. $\sigma_{\rm min}(X)$, $\sigma_{\rm max}(X)$ and $\sigma(X)$ denote the minimum singular value, maximum singular value and  the spectrum of matrix $X$, respectively. ${\rm Re}(X)$ represents the set of the real parts of all the elements of $X$. $||\cdot||$ denotes the Euclidean norm, $\otimes$  denotes the Kronecker product.
%; moreover, $A>0$ means that $\lambda_1(A)>0$.
%For a time-varying function $x(t): \mathbb{R}_{\geq 0 }\mapsto \mathbb{R}$, denote that $\sup_{t\in [t_0, t_1]} x(t) $ and $\inf_{t\in [t_0, t_1]} x(t)$ as the upper bound and lower bound of $x(t)$ over the time interval $[t_0, t_1]$, respectively. Moreover, denote that $\|x(t)\|_{[t_0, t_1]} =\sup_{t\in [t_0, t_1]} \|x(t)\| $. Define that $L_{\infty}:=\{x(t)|x(t): \mathbb{R}_{\geq 0 }\mapsto \mathbb{R}^n,\ \|x(t)\|_{[t_0, t_1]}<\infty\}$. In the following sections, $x(t) \in L_{\infty}$, $t\in [t_0, t_1]$, represents that the variable $x$ is uniformly bounded over $[t_0, t_1]$.   %$A>eq 0$ (or $A> 0$) denotes that $A$ is a nonnegative matrix (positive matrix, respectively), which means all elements of $A$ are nonnegative (positive, respectively).
 %${\rm span}(x)$ denotes the span vector of a given vector $x=[p_1, p_2,\ldots~, p_n]^{\mathrm{T}}\in \mathbb{R}^n$.

\label{introduction}


\section{Preliminaries}\label{section2}

%\subsection{Notations}




{\color{black}
\subsection{Graph Theory}
In this article, a MAS is consisidered on a directed graph $\mathcal{G}$.
The directed graph of followers  can be represented by the subgraph                              
$\mathcal{G}_f =(\mathcal{V}, \mathcal{E}, \boldsymbol{A} )$  with the node set 
$\mathcal{V}=\{ 1, 2, \ldots~ , N \}$, the edge set
$\mathcal{E} \subset \mathcal{V} \times \mathcal{V}=\{(v_j,\ v_i)\mid\ v_i,\ v_j \in \mathcal{V}\}$ 
, and the associated adjacency matrix $\boldsymbol{A}=[a_{ij}] \in \mathbb{R}^{N\times N} $ .
The weight of the edge $(v_j,\ v_i)$ is denoted by $a_{ij}$ with $a_{ij} > 0$ if $(v_j,\ v_i) \in \mathcal{E}$ 
otherwise $a_{ij} = 0$. The neighbor set of node $v_i$ is represented by $\mathcal{N}_{i}=\{v_{j}\in \mathcal{V}\mid (v_j,\ v_i)\in \mathcal{E} \}$. Define the Laplacian matrix as 
$L=\mathcal{D}-\mathcal{A}  \in \mathbb{R}^{N\times N}$ 
with $\mathcal{D}={\rm blkdiag}(d_i) \in \mathbb{R}^{N\times N}$ where $d_i=\sum_{j \in \mathcal{F}} a_{ij}$ 
is the weight in-degree of node $v_i$.
The leader has no incoming edges and thus exhibits an autonomous behavior, 
while the follower has incoming edges and receives neighbor(including leader and follower) information 
directly. 
The interactions among the leaders and the followers are represented by 
$G_{ik}=${\rm blkdiag}($g_{ik}$) $\in \mathbb{R}^{N\times N}$  while $g_{ik}$ is the weight of the path from 
$i$th leader to kth follower. If there is a direct path from $i$th leader 
to $k$th follower , $g_{ik} > 0$, otherwise $g_{ik} = 0$.
}

\subsection{Some Useful Lemmas and Definition}

\begin{lemma}[Bellman-Gronwall Lemma \cite{lewis2003} ] \label{Bellman-Gronwall Lemma 2}
    Assuming $\Phi : [T_a,T_b] \rightarrow \mathbb{R}$ is a nonnegative continuous function, $\alpha:[T_a,T_b] \rightarrow \mathbb{R}$ is integrable, $\kappa \geq 0$ is a constant, and
    \begin{equation}
    \Phi(t) \leq \kappa + \int_{0}^{t} \alpha (\tau)\Phi(\tau) \,d\tau ,t \in [T_a,T_b] ,
    \end{equation}
    then we obtain that 
    $$\Phi(t) \leq \kappa e^{\int_{0}^{t} \alpha(\tau) \,d\tau }$$ for all $t \in [T_a,T_b]$.
\end{lemma}





\begin{lemma}[{\cite[Lemma 1]{cai2017 }}]\label{Lemma 1}
    Consider the following system
    $$\dot{x}=\epsilon F x +F_1(t)x+F_2(t)$$
    where $ x \in \mathbb{R}^{n\times n}$ , $F \in \mathbb{R}^{n\times n}$
    is Hurwitz, $\epsilon > 0$, $F_1(t) \in \mathbb{R}^{n\times n}$
    and $F_2^x(t) \in \mathbb{R}^{n}$ are bounded and continuous for all $t \geq  t_0$. We have (i) if
    $F_1(t)$, $F_2(t) \rightarrow 0 $ as  $t \rightarrow  \infty$ (exponentially), then for any $x(t_0)$ and
    any $\epsilon > 0$, $x(t) \rightarrow 0$ as $ t \rightarrow \infty$ (exponentially); 
    (ii) if $F_1(t) = 0$, $F_2(t)$ decays to zero exponentially at the rate of $\alpha$,
     and $\epsilon \geq \frac{\alpha }{\alpha_F} $, where $\alpha_F=\min({\rm Re}(\sigma(-F)))$, 
     then, for any $ x(t_0)$, $x(t) \rightarrow 0 $ as $ t \rightarrow \infty$
    exponentially at the rate of $\alpha$.
\end{lemma}


\section{SYSTEM SETUP AND PROBLEM FORMULATION}\label{section3}
In this section, a new problem called resilient containment of
MAS group against composite attacks is proposed. First, the model of the MAS group is formulated and some basic definitions of the composite attacks is given.
{\color{black}
\subsection{MAS group Model}
In the framework of containment control, we consider a group of $N+M$ MAS, which can be divided into two groups:

1) $M$ leaders are the roots of directed graph $\mathcal{G}$, which have no neighbors. Define the index set of leaders as $\mathcal{L}= \textbf{I}[N+1,N+M]$.

2) $N$ followers who coordinate with their neighbors to achieve the containment set
of the above leader. Define the index set of followers
as $\mathcal{F} = \textbf{I}[1, N]$.

Similar to many existing works[]-[], we consider the following dynamics of leader :
\begin{equation}\label{EQ1}
\begin{cases}
\dot{x}_k=S x_k,\\
y_k=R x_k,
\end{cases}
\end{equation}
where $x_k\in \mathbb{R}^q$ and $y_k\in \mathbb{R}^p$ are system states and reference output of the $k$th leader, respectively.

The dynamics of each follower is given by 
\begin{equation}\label{EQ2}
  \begin{cases}
  \dot{x}_i=A_i x_i + B_i \bar{u}_i,\\
  y_i=C_i x_i,
  \end{cases}
\end{equation}
where $x_i\in \mathbb{R}^{ni}$, $u_i\in \mathbb{R}^{mi}$ and $y_i\in \mathbb{R}^p$ are 
system state, control input and output of the $i$th follower, respectively. For convenience, the notation $'(t)'$ can be omitted in the following discussion. We make the following assumptions about the agents and the communication network.

\begin{assumption}\label{assumption 2}
  The directed graph $\mathcal{G}$ contains a spanning tree with
the leaders as its root.
\end{assumption}

\begin{assumption}\label{assumption 4}
  The real parts of the eigenvalues of $S$ are non-negative.
\end{assumption}




\begin{assumption}\label{assumption 3}
 The pair $(A_i, B_i)$ is stabilizable for $i \in \textbf{I}[1,N]$.
\end{assumption}



\begin{assumption}\label{assumption 5}
For all $\lambda \in \sigma(S)$, where $\sigma(S)$ represents the spectrum of $S$,
  \begin{equation}
   {\rm rank} \left[
      \begin{array}{c|c}
     A_i-\lambda I_{n_i} &  B_i  \\ \hline
    C_i  & 0   \\
      \end{array}
      \right]=n_i+p,
      i \in \mathcal{F}.
  \end{equation}
\end{assumption}

\begin{assumption}
  The graph $\mathcal{G}$ is strongly connected.
\end{assumption}


{\color{blue}
\begin{remark}
Assumptions 1,3 and 4 are standard in the classic output regulation problem, Assumption 2 is avoid the trivial case of stable S. Assumption 5 guarantees that $\Psi_k$ defined in (19) is positive definite in \cite{haghshenas2015}, which is useful to prove the exponential convergence of estimators and observers in the following. $\hfill \hfill \square $
\end{remark}
}



\begin{myDef}\label{def41}
  For the $i$th follower, the system accomplishes containment if there exists series of $\alpha_{\cdot i}$,
   which satisfy $\sum _{k \in \mathcal{L}} \alpha_{k i} =1$ to ensure following equation hold:
   \begin{equation}
     {\rm  lim}_{t\rightarrow \infty } (y_i(t)-\sum _{k\in \mathcal{L}} \alpha_{k i}y_k(t))=0,
   \end{equation}
   where $i \in \textbf{I}[1,N]$.
\end{myDef}

}


\subsection{ Attack Descriptions}
In this work, we consider the MASs consisting of cooperative agents with potential malicious attackers. As shown in the Fig.1, the attackers will use four kinds of attacks to compromise the containment performance of the MASs:

1) DoS attacks (Denial-of-Services Attacks): The communication graphs among agents (both in TL and CPL)  denied by attacker;

2) AAs (Actuation Attacks):the motor inputs infiltrated by attacker to destroy the input signal of the agent;

3) FDI attacks (Fault injection Attacks):  the exchanging information among agents distort by attack;

4) CAs (Camoufalge Attacks): mislead its downstream agents by disguising as a leader .

To resist the composite attack, we introduced a new layer named TL with the same communication topology as CPL, which yet have greater security and less physical meanings. Therefore, this TL could effectively against most of the above attacks. With the introduction of TL,the resilient control scheme can be decoupled to defend against DoS attacks on TL and defend against   potential unbounded AAs on CPL. The following two small subsections give the definitions and essential constraints for the DoS attacks and AAs, respectively.

1) Dos attacks: DOS attack refers to a type of attack where an adversary presents some or all components of a control system.It can affect the measurement and control channels simultaneously, resulting in the loss of data availability. Suppose that attackers can attack the communication network in a varing active period. Then it has to stop the attack activity and shift to a sleep period to reserve energy for the next attacks. Assume that there exists a $l \in \mathbb{N}$ , define $\{t_l \}_{l \in \mathbb{N}}$ and $\{\Delta_* \}_{l \in \mathbb{N}}$  as the start time and the duration time of the $l$th attack sequence of DoS attacks, that is , the $l$th DoS attack time-interval is $A_l = [t_l , t_l +\Delta_* )$ with $t_{l+1} > t_l +\Delta_* $ for all $l \in \mathbb{N}$. Therefore, for all $t\geq \tau \in \mathbb{R}$, the sets of time instants where the communication network is under Dos attacks are represent by
\begin{equation}
    S_A(\tau,t) = \cup A_l \cap [\tau , t],l\in \mathbb{N},
\end{equation}
and the sets of time instants where the communication network is allowed are 
\begin{equation}
    S_N(\tau,t) = [\tau,t] S_A / (\tau,t).
\end{equation}

\begin{myDef} [{Attack Frequency \cite{feng2017}   }]
For any $\delta_2 > \delta_1 \geq t_0$, let $N_a(t_1,t_2)$ represent the number of DoS attacks in $[t_1,t_2)$. Therefore, $F_a(t_1,t_2)= \frac{N_a(t_1,t_2)}{t_2 - t_1}$ is defined as the attack frequency at $[t_1,t_2)$ for all $t_2 > t_1 \geq t_0$.
\end{myDef}

\begin{myDef} [{ Attack Duration \cite{feng2017}  }]
For any $t_2 > t_1 \geq t_0$, let $T_a(t_1,t_2)$ represent the total time interval of DoS attack on multi-agent systems during  $[t_1,t_2)$. The attack duration over $[t_1,t_2)$ is defined as: there exist constants $\tau_a > 1$ and $T_0 > 0$ such that
  \begin{equation}
    T_a(t_1,t_2) \leq T_0 + \frac{t_2-t_1}{\tau_a}. 
  \end{equation}
\end{myDef}

2) Unbounded Actuation Attacks:
For each follower, the system input is under unknown actuator fault, which is described as
\begin{equation}
    \bar{u}_i=u_i+\chi, \forall i  \in \mathcal{F},
\end{equation}
where $\chi$ denotes the unknown actuator fault caused in actuator channels. That is, the ture values of $\bar{u}_i$ and $\chi$ are unknown and we can only measure the damaged control input information $\bar{u}_i$.


\begin{assumption}
  The actuator attack $\chi$ is unbounded and its derivative $\dot{d}_i$ is bounded by $\bar{\kappa}$.
\end{assumption}

{\color{blue}
\begin{remark}
In contrast to the works \cite{deng2021} and \cite{chen2019}, which only consider the bounded AAs, this work tackles with unbounded AAs under the Assumption 6. In the case When the derivative of the attack signal is unbounded, that is, the attack signal increases at a extreme speed, the MAS can reject the signal by removing the excessively large value, which can be easily detected.
$\hfill \hfill \square$
\end{remark}

}





\subsection{ Problem Formulation}


%{\color{red}
Define the following local output containment error:
\begin{equation}\label{EQ xi}
    \xi_i = \sum_{j\in \mathcal{F}} a_{ij}(y_j -y_i) +\sum_{k \in \mathcal{L}} g_{ik}(y_k - y_i).
\end{equation}
The global form of (\ref{EQ xi}) is written as 
\begin{equation}
    \xi = - \sum_{k \in \mathcal{L}}(\Psi_k \otimes I_p)(y -  \underline{y}_k),
\end{equation}
where $\Psi_k = (\frac{1}{m} \mathcal{L } + G_{ik})$, $\xi = [\xi_1^{\mathrm{ T}},\xi_2^{\mathrm{T}},\dots,\xi_n^{\mathrm{T}}]^{\mathrm{T}}$, $y=[y_1^{\mathrm{T}},y_2^{\mathrm{T}},\dots,y_n^{\mathrm{T}}]^{\mathrm{T}}$, and $\underline{y}_k = (l_n \otimes y_k)$.
\begin{lemma}
    Under Assumption 1, the matrixs $\Psi_k$ and $\sum_{k \in \mathcal{L}}\Psi_k $ are positive-definite and non-singular. Moreover, both $(\Psi_k)^{-1}$ and $(\sum_{k \in \mathcal{L}} \Psi_k)^{-1}$ are non-negative. 
\end{lemma}




Define the following global output containment error:
\begin{equation}
e= y - (\sum_{r\in \mathcal{L} }(\Phi_r \otimes I_p))^{-1} \sum_{k \in \mathcal{L} } (\Psi_k \otimes I_p) \underline{y}_k,
\end{equation}
where $e=[e_i^{\mathrm{T}},e_2^{\mathrm{T}},\dots,e_n^{\mathrm{T}}]^{\mathrm{T}}$ and $\xi = -\sum_{k \in \mathcal{L}}(\Phi \otimes I_p )e$.

\begin{lemma}[{\cite[Lemma 1]{zuo2019}}]
    Under Assumption 1, the output containment control objective in (\ref{def41}) is achieved if ${\rm lim}_{t \rightarrow \infty} e = 0$.
\end{lemma}



\noindent \textbf{Problem ACMCA} (Attack-resilient Containment control of MASs
against Composite Attacks): The resilient containment control problem is to design the input $u_i$ in (1) for each follower , such that ${\rm lim}_{t \rightarrow \infty}e= 0$ in (10) with the case of unknown leader dynamics and under unknown unbounded cyber-attacks and network DoS attackers, i.e., the trajectories of each follower converges into a point in the dynamic convex hull spanned by trajectories of multiple leaders.


%}


































\section{Main Results}

Since the leader output $y_k(t)$ are only available to its neighbors and $S$, $R$ are unknown for all agents, the full distributed observers are proposed to estimate the unknown matrix $S$ and $R$ for all agents under the effect of Dos attacks. Then, a fully distributed virtual resilient layer is proposed to estimate the state of followers containment. Finally, new adaptive resilient state estimators and controllers are proposed to resist the influence of both DoS attacks and actuator faults.


\subsection{ Fully Distributed Observers to Estimate Leader States and Dynamics}
 In this section, we design distributed leader states and dynamics observers that are independent of the global graph topology and the global leader information.
  
  To facilitate the analysis, define the leader dynamics in (2) as follows:
  \begin{equation}
      \Upsilon =[S;R]\in \mathbb{R} ^{(p+q)\times q}
  \end{equation}
and its estimations be devided in two parts as follows:
 \begin{align}
      &\hat{\Upsilon } _{0i}(t)=[\hat{S}_{0i}(t);\hat{R}_{0i}(t)]\in \mathbb{R} ^{(p+q)\times q}\\
     &\hat{\Upsilon } _{i}(t)=[\hat{S}_{i}(t);\hat{R}_{i}(t)]\in \mathbb{R} ^{(p+q)\times q},
  \end{align} 
where $\hat{\Upsilon } _{0i}(t)$ and $\hat{\Upsilon } _{i}(t)$ will be updated by (\ref{EQ15})and (\ref{EQ16}), and it converge to $\Upsilon$ at different rates.





%{\color{blue}
\begin{myTheo}\label{Theorem 1}
    Consider a group of $M$ leaders and $N$ followers with dynamic in (\ref{EQ1}) and (\ref{EQ2}) . Suppose that Assumptions 2 and 6 hold. The problem of the leader unknown dynamics under the Dos attack is solved by the following dynamic estimates $\hat{\Upsilon } _{0i}(t)$ and 
$\hat{\Upsilon } _{i}(t)$ :
\begin{equation}\label{EQ15}
   \dot{\hat{\Upsilon }} _{0i}(t)=\sum_{j \in \mathcal{F}} d_{ij}(\hat{\Upsilon } _{0j}(t)-\hat{\Upsilon } _{0i}(t)) + \sum_{k  \in \mathcal L}d_{ik}(\Upsilon-\hat{\Upsilon } _{0i}(t)) ,\forall i \in \mathcal{F} ,
\end{equation}
\begin{equation}\label{EQ16}
    \dot{\hat{\Upsilon }} _{i}(t)=\left\lVert \hat{\Upsilon } _{0i}(t) \right\rVert_2(\hat{\Upsilon } _{0i}(t)-\hat{\Upsilon }_{i}(t)) +\sum_{j\in \mathcal{F}} d_{ij}(\hat{\Upsilon } _{j}(t)-\hat{\Upsilon } _{i}(t)) +\sum_{k\in \mathcal{L}}d_{ik}(\Upsilon-\hat{\Upsilon } _{i}(t)), \forall i \in \mathcal{F} .
\end{equation}
\end{myTheo}
%}

%{\color{blue}
\textbf{Proof.} 
Since we only use relative neighborhood information to estimate leader dynamics, the leader estimator will suffer the influence of Dos attacks. $d_{ij}(t)$ and $d_{ik}(t)$ is a designed weight for $i,j \in \mathcal{F}$ and $k\in \mathcal{L}$. For the denied
communication link , $d_{ij}=0$ and $d_{ik}=0$; And for for the  normal communication link, $d_{ij}=a_{ij}$ and $d_{ik}=g_{ik}$.

\textbf{Step 1:}
Let
$\tilde{\Upsilon}_{0i}(t)=\Upsilon-\hat{\Upsilon}_{0i}  (t)$,
then
\begin{equation}\label{EQ17}
    \dot{\tilde{\Upsilon}} _{0i} (t)
=\dot{\Upsilon}-\dot{\hat{\Upsilon}}_{0i} (t) 
=-(\sum_{j = 1}^{N} d_{ij}(\hat{\Upsilon } _{0j}(t)-\hat{\Upsilon } _{0i}(t)) +\sum_{k\in \mathcal{L}}d_{ik}(\Upsilon-\hat{\Upsilon } _{0i}(t))).
\end{equation}
 So, for the normal communication, the global estimation dynamics error  of system (\ref{EQ16}) can be written as
\begin{equation}\label{EQ 18}
    \dot{\tilde{\Upsilon}}_0 (t) = -\sum_{k\in \mathcal{L}}\Psi_k \otimes I_{p+q}\tilde{\Upsilon}_0(t) =-\bar{\Psi}_k \tilde{\Upsilon}_0(t), t \geq t_0.
\end{equation}
where $\bar{\Psi}_k=\sum_{k\in \mathcal{L}}\Psi_k \otimes I_{p+q}$, $\Psi_k= \frac{1}{M}L + G_k$, $\tilde{\Upsilon}_0(t)= [\tilde{\Upsilon}_{01}(t),\tilde{\Upsilon}_{01}(t),\dots,\tilde{\Upsilon}_{0N}(t)]$. 

\begin{lemma}[\cite{haghshenas2015}]
    Suppose Assumption 6 holds. then the matrices $\Phi_k$ and $\sum_{k \in \mathcal{L}} \Phi_k$ are positive-definite and non-singular. Therefore, both $(\Phi_k)^{-1}$ and $(\sum_{k \in \mathcal{L}} \Phi_k)^{-1} $ are non-singular.
\end{lemma}

And for the denied communication, we can see that 
\begin{equation}\label{EQ 19}
    \dot{\tilde{\Upsilon}}_0 (t) = 0, t\geq t_0.
\end{equation}
So, we can conclude (\ref{EQ 18}) and (\ref{EQ 19}) that
\begin{equation}
\begin{aligned}
  \tilde{\Upsilon}_0(t) &\leq \tilde{\Upsilon}_0(t_0)e^{-\sigma_{\rm max}(\bar{\Psi}_k)\left\lvert S_N(t_0,t)\right\rvert } \\
  &\leq \tilde{\Upsilon}_0(t_0)e^{-\sigma_{\rm max}(\bar{\Psi}_k)\left\lvert t - t_0- S_A(t_0,t)\right\rvert }.
\end{aligned}
\end{equation}
Under  the Lemma 5, all the eigenvalues of $\Psi_k$ have positive real parts. 
Therefore, we can conclude  that ${\rm  lim}_{t\rightarrow \infty }  \tilde{\Upsilon}_{0i}(t) = 0 $ for $i=\textbf{I}[1,N]$, exponentially.

\textbf{Step 2:}
Define the dynamics error $\tilde{\Upsilon}_i(t)$ as $\tilde{\Upsilon}_{i}(t)=\Upsilon-\hat{\Upsilon}_{i}(t)$ .
The inverse of $\tilde{\Upsilon}_i$ as the following
\begin{equation}
    \dot{\tilde{\Upsilon}} _{i}(t)
=\dot{\Upsilon}-\dot{\hat{\Upsilon}}_{i}(t)
=-\left\lVert \hat{\Upsilon } _{0i}(t) \right\rVert_2(\hat{\Upsilon } _{0i}(t)-\hat{\Upsilon } _{i}(t)) - (\sum_{j = 1}^{N} d_{ij}(\hat{\Upsilon } _{j}(t)-\hat{\Upsilon } _{i}(t)) +\sum_{k\in \mathcal{L}}d_{ik}(\Upsilon-\hat{\Upsilon } _{i}(t))).
\end{equation}
Then the estimated global dynamics error of $\tilde{\Upsilon}_i$ is
\begin{equation}\label{EQ23}
    \dot{\tilde{\Upsilon}}(t) =-(\sum_{k\in \mathcal{L}}\Psi_k^D \otimes I_{p+q} + {\rm blkdiag}(\left\lVert \hat{\Upsilon } _{0i}(t) \right\rVert_2) \otimes I_{p+q})\tilde{\Upsilon}(t)
+{\rm blkdiag}(\left\lVert \hat{\Upsilon } _{0i}(t) \right\rVert_2)\otimes I_{p+q} \tilde{\Upsilon}_0(t).
\end{equation}
where $\Psi_k^D (t) = \begin{cases}
 0 ,t \in S_A, \\ \Psi_k,t \in S_N,
\end{cases}$ and $\tilde{\Upsilon}_0(t)= [\tilde{\Upsilon}_{01}(t),\tilde{\Upsilon}_{01}(t),\dots,\tilde{\Upsilon}_{0N}(t)]$.
{\color{black}
Solve the equation (\ref{EQ23}), we can obtain that
\begin{equation}
\begin{aligned}
 \tilde{\Upsilon}(t)  
 & = \tilde{\Upsilon}(t_0)e^{-\int_{t_0}^{t} \sum_{k\in \mathcal{L}}\Psi_k^D \otimes I_{p+q} + {\rm blkdiag}(\left\lVert \hat{\Upsilon } _{0i}(\tau) \right\rVert_2) \otimes I_{p+q} \,d\tau} \\
 & + \int_{t_0}^{t} {\rm blkdiag}(\left\lVert \hat{\Upsilon } _{0i}(\tau) \right\rVert_2)\otimes I_{p+q} \tilde{\Upsilon}_0(\tau) e^{-\int_{\tau}^{t} \sum_{k\in \mathcal{L}}\Psi_k^D \otimes I_{p+q} + 
    {\rm blkdiag}(\left\lVert \hat{\Upsilon } _{0i}(s) \right\rVert_2) \otimes I_{p+q} \,d s} \,d\tau \\
    & =\tilde{\Upsilon}(t_0)e^{- \bar{\Psi}_k S_N(t_0,t)- \int_{t_0}^{t}  {\rm blkdiag}(\left\lVert \hat{\Upsilon } _{0i}(\tau) \right\rVert_2) \otimes I_{p+q} \,d\tau} \\
   & + e^{-\bar{\Psi}_k S_N(t_0,t)} \int_{t_0}^{t} {\rm blkdiag}(\left\lVert \hat{\Upsilon } _{0i}(\tau) \right\rVert_2)\otimes I_{p+q} \tilde{\Upsilon}_0(t_0) e^{-\int_{\tau}^{t} 
    {\rm blkdiag}(\left\lVert \hat{\Upsilon } _{0i}(s) \right\rVert_2) \otimes I_{p+q} \,d s} \,d\tau \\
   & = e^{-\bar{\Psi}_k S_N(t_0,t)}(\tilde{\Upsilon}(t_0)e^{- \int_{t_0}^{t}  {\rm blkdiag}(\left\lVert \hat{\Upsilon } _{0i}(\tau) \right\rVert_2) \otimes I_{p+q} \,d\tau} +B_{\Upsilon}).
\end{aligned}
\end{equation}
where $B_{\Upsilon}=\int_{t_0}^{t} {\rm blkdiag}(\left\lVert \hat{\Upsilon } _{0i}(\tau) \right\rVert_2)\otimes I_{p+q} \tilde{\Upsilon}_0(t_0) e^{-\int_{\tau}^{t} 
    {\rm blkdiag}(\left\lVert \hat{\Upsilon } _{0i}(s) \right\rVert_2) \otimes I_{p+q} \,d s} \,d\tau$ is bounded.
}
It is obvious that 
${\rm  lim}_{t\rightarrow \infty }  \tilde{\Upsilon} = 0 $ exponentially as the same rate as $\tilde{\Upsilon}_0$. $\hfill \hfill \blacksquare $

{\color{blue}
\begin{remark}
In practice, the powerful and developed sensor and communication devices which can accept global information are expensive, thus, with the estimators (16) and (17), we can install the poor sensor and communication devices for each follower to  only receive their neighbors’ data and information. And compared with the existing work \cite{chen2019}, which  only used the estimator in the normal communication for consensus, this work consider the heterogeneous output containment with the Dos attack. 
$\hfill \hfill \square $
\end{remark}

}
%}







\subsection{ Distributed Resilient Estimator Design}
 In this section, A fully distributed virtual resilient layer is proposed to resist DoS attacks,
consider the following fully distributed virtual resilient layer: 
\begin{equation}\label{equation 200}
  \dot{z}_i=\hat{S}_i z_i -c_{\Psi} (\sum _{j \in \mathcal{F} }d_{ij}(z_i-z_j)+\sum_{k \in \mathcal{L}}d_{ik}(z_i-x_k)),
\end{equation}
%{\color{blue}
where $z_i$ is the local state of the virtual layer and $c_{\Psi}  >  0$ is the estimator gain designed in Theorem \ref{Theorem 2}. 
%}
The global state of virtual resilient layer   can be written as
\begin{equation}
  \dot{z}= \hat{S}_b z-c_{\Psi} (\sum_{k \in \mathcal{L}}(\Psi_k^D \otimes I_p )(z-\underline{x}_k)), 
\end{equation}
where $\hat{S}_b={\rm blkdiag}(\hat{S}_i) $, $z=[z_1,z_2,\dots,z_N]$, $\underline{x}_k=l_n \otimes x_k$ and $c_{\Psi}  >  0$ is the estimator gain designed in Theorem \ref{Theorem 2}.

Define the global virtual resilient layer state estimation error
\begin{equation}
\begin{aligned}
  \tilde{z} 
  &=z-(\sum_{r \in \mathcal{L}}(\Psi_r \otimes I_p ))^{-1} \sum_{k \in \mathcal{L}}(\Psi_k \otimes I_p ) \underline{x}_k ,\\
\end{aligned}
\end{equation}
then, for the normal communication, we have
%
%{\color{red}
\begin{equation}\label{EQ26}
    \begin{aligned}
      \dot{\tilde{z}}
       &=\hat{S}_bz-c_{\Psi} (\sum_{k \in \mathcal{L}}(\Psi_k \otimes I_p )(z-\underline{x}_k))-
  (\sum_{r \in \mathcal{L}}(\Psi_r \otimes I_p ))^{-1}\sum_{k \in \mathcal{L}}(\Psi_k \otimes I_p ) (I_n \otimes S) \underline{x}_k \\
      &=\hat{S}_bz- (I_n \otimes S)z+ (I_n \otimes S)z 
  -(I_n \otimes S)(\sum_{r \in \mathcal{L}}(\Psi_r \otimes I_p ))^{-1} \sum_{k \in \mathcal{L}}(\Psi_k \otimes I_p )  \underline{x}_k  +M \\
  & -c_{\Psi} \sum_{k \in \mathcal{L}}(\Psi_k \otimes I_p )(z-(\sum_{rr \in \mathcal{L}}(\Psi_{rr} \otimes I_p ))^{-1} (\sum_{kk \in \mathcal{L}}(\Psi_{kk} \otimes I_p )\underline{x}_{kk}+
  (\sum_{rr \in \mathcal{L}}(\Psi_{rr} \otimes I_p )^{-1} (\sum_{kk \in \mathcal{L}}(\Psi_{kk} \otimes I_p )\underline{x}_{kk}-\underline{x}_k)) \\
  &=\tilde{S}_bz+(I_n \otimes S) \tilde{z}-c_{\Psi}\sum_{k \in \mathcal{L}}(\Psi_k \otimes I_p ) \tilde{z} -
  c_{\Psi}(\sum_{kk \in \mathcal{L}}(\Psi_{kk} \otimes I_p ) \underline{x}_{kk}- \sum_{k \in \mathcal{L}}(\Psi_k \otimes I_p )\underline{x}_k) +M\\
  & =(I_n \otimes S) \tilde{z}-c_{\Psi}\sum_{k \in \mathcal{L}}(\Psi_k \otimes I_p ) \tilde{z}+ \tilde{S}_b \tilde{z} + F_2^x(t) ,
    \end{aligned}
\end{equation}
where  $F_2^x(t)=\tilde{S}_b(\sum_{r \in \mathcal{L}}(\Psi_r \otimes I_p ))^{-1} \sum_{k \in \mathcal{L}}(\Psi_k \otimes I_p ) \underline{x}_k +M$ and
$M = (\sum_{r \in \mathcal{L}}(\Psi_r \otimes I_p ))^{-1}\sum_{k \in \mathcal{L}}(\Psi_k \otimes I_p ) (I_n \otimes S) \underline{x}_k-
(I_n \otimes S) (\sum_{r \in \mathcal{L}}(\Psi_r \otimes I_p ))^{-1}\sum_{k \in \mathcal{L}}(\Psi_k \otimes I_p ) \underline{x}_k$ and $\tilde{S}_b={\rm blkdiag}(\tilde{S}_i)$ for $i\in \mathcal{F}$.

For the denied communication, we have 
\begin{equation}\label{EQ27}
    \begin{aligned}
      \dot{\tilde{z}}
       &=\hat{S}_bz-
  (\sum_{r \in \mathcal{L}}(\Psi_r \otimes I_p ))^{-1}\sum_{k \in \mathcal{L}}(\Psi_k \otimes I_p ) (I_n \otimes S) \underline{x}_k \\
      &=\hat{S}_bz- (I_n \otimes S)z+ (I_n \otimes S)z 
  -(I_n \otimes S)(\sum_{r \in \mathcal{L}}(\Psi_r \otimes I_p ))^{-1} \sum_{k \in \mathcal{L}}(\Psi_k \otimes I_p )  \underline{x}_k  +M \\
  &=\tilde{S}_bz+(I_n \otimes S) \tilde{z} +M\\
  & =(I_n \otimes S) \tilde{z}+ \tilde{S}_b \tilde{z} + F_2^x(t) ,
    \end{aligned}
\end{equation}
So, we can conclude (\ref{EQ26}) and (\ref{EQ27}) that
\begin{equation}\label{EQ32}
    \dot{\tilde{z}}_i = \begin{cases}
     (I_n \otimes S) \tilde{z}-c_{\Psi}\sum_{k \in \mathcal{L}}(\Psi_k \otimes I_p ) \tilde{z}+ \tilde{S}_b \tilde{z} + F_2^x(t), t \in S_N(t_0,t) \\
     (I_n \otimes S) \tilde{z}+ \tilde{S}_b \tilde{z} + F_2^x(t) , t \in S_A(t_0,t).
    \end{cases}
\end{equation}

{\color{blue}

\begin{lemma}
Under Lemma 5 and the Kronecker product property $(P \otimes Q)(Y \otimes Z) =(PY)\otimes(QZ) $, it easy to show that 
$(\sum_{r \in \mathcal{L}}(\Psi_r \otimes I_p ))^{-1}\sum_{k \in \mathcal{L}}(\Psi_k \otimes I_p ) (I_n \otimes S) \underline{x}_k=
(I_n \otimes S) (\sum_{r \in \mathcal{L}}(\Psi_r \otimes I_p ))^{-1}\sum_{k \in \mathcal{L}}(\Psi_k \otimes I_p ) \underline{x}_k$. 

\textbf{Proof:}

Let 
\begin{equation}
   M= \sum_{k \in \mathcal{L}} M_k= \sum_{k \in \mathcal{L}} ((\sum_{r \in \mathcal{L}}(\Psi_r \otimes I_p ))^{-1} (\Psi_k \otimes I_p ) (I_n \otimes S)
-(I_n \otimes S) (\sum_{r \in \mathcal{L}}(\Psi_r \otimes I_p ))^{-1}(\Psi_k \otimes I_p )) \underline{x}_k .
\end{equation}


By the Kronecker product property $(P \otimes Q)(Y \otimes Z) =(PY)\otimes(QZ) $, we can obtain that 
\begin{equation}
\begin{aligned}
  & (I_N \otimes S)(\sum_{r \in \mathcal{L}} \Psi_r \otimes I_p)^{-1} (\Psi_k \otimes I_p) \\
   &= (I_N \otimes S)((\sum_{r \in \mathcal{L}} \Psi_r)^{-1} \Psi_k) \otimes I_p) \\
   &=(I_N \times (\sum_{r \in \mathcal{L}} \Psi_r)^{-1} \Psi_k))\otimes(S \times I_p) \\
   &= (\sum_{r \in \mathcal{L}} \Psi_r \otimes I_p)^{-1} (\Psi_k \otimes I_p)  (I_N \otimes S) .\\
\end{aligned}
\end{equation}


 We can show that $M_k=0$ and obtain that 
 \begin{equation}
     M=\sum_{k \in \mathcal{L}}M_k =0.
 \end{equation}
 This Proof is completed.
$\hfill \hfill \blacksquare $
\end{lemma}
}



%}



Then $F_2^x(t)=\tilde{S}_b(\sum_{r \in \mathcal{L}}(\Psi_r \otimes I_p ))^{-1} \sum_{k \in \mathcal{L}}(\Psi_k \otimes I_p ) \underline{x}_k$ and ${\rm  lim}_{t \rightarrow \infty}F_2^x(t)=0$ is exponentially at the rate of $\tilde{S}_b$.


{\color{black}
Consider the following system
\begin{equation}\label{EQF3}
    \dot{\tilde{z}}=F_3(t)\tilde{z}+F_4(t),
\end{equation}
solve the equation (\ref{EQF3}), we have
\begin{equation}\label{EQF4}
   \tilde{z}(t)=(\tilde{z}(t_0)+ \int_{t_0}^{t} F_4(\tau) e^{\int_{t_0}^{\tau} -F_3(s)\,ds} \,d\tau )e^{\int_{t_0}^{t} F_3(\tau)\,d\tau} .
\end{equation}
Based on (\ref{EQF3}) and (\ref{EQF4}), we can obtain the following inequality from (\ref{EQ32})
\begin{equation}
    \tilde{z}(t) = \begin{cases}
     e^{\int_{t_{2k}}^{t} \hat{S}_b -\bar{c}_{\Psi}\, d\tau}\tilde{z}(t_{2k}) +\int_{t_{2k}}^{t} F_2^x(\tau)e^{\int_{\tau}^{t} \hat{S}_b -\bar{c}_{\Psi}\, d\tau^*} \, d\tau , t \in [t_{2k},t_{2k+1}) \\
     e^{\int_{t_{2k+1}}^{t} \hat{S}_b \, d\tau}\tilde{z}(t_{2k+1}) +\int_{t_{2k+1}}^{t} F_2^x(\tau)e^{\int_{\tau}^{t} \hat{S}_b \, d\tau^*} \, d\tau , t \in [t_{2k+1},t_{2k+2}).
    \end{cases}
\end{equation}
where $\bar{c}_{\Psi}= c_{\Psi}\sum_{k \in \mathcal{L}}(\Psi_k \otimes I_p ) $.

{\color{black}

\begin{myTheo}\label{Theorem 2}
    Consider the MASs \ref{EQ1}-\ref{EQ2} suffered from DoS attacks, which satisfy Assumption 1,
    and the DoS attack satisfying Definition 2 and Definition 3. 
    There exists $c_{\Psi} >0$, $\eta^* \in (0,||\bar{c}_{\Psi}||_2)$ such that the duration and frequency of the attack satisfy
    \begin{equation}\label{Dos1}
        \frac{1}{\tau_a} < 1-\frac{\sigma_{\rm max}(S_b)+\eta^*}{\sigma_{\rm min}(\bar{c}_{\Psi})}
    \end{equation}
    and 
    \begin{equation}\label{Dos2}
        \frac{N_a(t_0,t)}{t-t_0} \leq \frac{\eta^*}{||\bar{c}_{\Psi}||_2 \Delta_*}
    \end{equation}
    where $S_b=I_N \otimes S$. Then it can be guaranteed that the estimation error $\tilde{z}$ exponentially converges to zero under DoS attacks.
\end{myTheo}

\textbf{Proof.} 
Nest, we will prove that the virtual layer achieve containment under the Dos attacks.
For clarity, we first redefine the set $S_A[t_0, \infty)$ as $S_A[t_0, \infty)= \bigcup_{k=0,1,2,\dots} [t_{2k+1},t_{2k+2})$, where $t_{2k+1}$ and $t_{2k+2}$ indicate the time instants that attacks start and end, respectively.
Then, the set $S_N[t_0,\infty)$ can be redefined as $S_N[t_0,\infty)$ as $S_N[t_0, \infty)= [t_0,t_1)\bigcup_{k=1,2,\dots} [t_{2k},t_{2k+1})$ .

For convenient, we described the two situation that MAS is with/without attacks as follows:
\begin{equation}
\bar{c}_{\Psi}^D=\begin{cases}
 \bar{c}_{\Psi},t \in S_N(t_0,\infty), \\
 0,t \in S_A(t_0,\infty).
\end{cases}
\end{equation}
and
\begin{equation}
    \tilde{z}(t) = 
     e^{\int_{t_{k}}^{t} \hat{S}_b -\bar{c}_{\Psi}^D\, d\tau}\tilde{z}(t_{k}) +\int_{t_{k}}^{t} F_2^x(\tau)e^{\int_{\tau}^{t} \hat{S}_b -\bar{c}_{\Psi}^D\, d\tau^*} \, d\tau , t \in [t_{k},t_{k+1}).
\end{equation}
When $t\in [t_{2k},t_{2k+1})$, we obtain

\begin{equation}
\begin{aligned}
    \tilde{z} (t)
    &=  e^{\int_{t_{0}}^{t} \hat{S}_b(\tau) \, d\tau- \bar{c}_{\Psi}|S_N|}\tilde{z}(t_0) \\
   & +\int_{t_0}^{t_1} F_2^x(\tau)  e^{ \int_{\tau}^{t} \hat{S}_b-\bar{c}_{\Psi}^D \,d\tau^*  }  \, d\tau   \\ 
   & +\int_{t_1}^{t_2} F_2^x(\tau) e^{ \int_{\tau}^{t} \hat{S}_b -\bar{c}_{\Psi}^D \,d\tau^* }  \, d\tau   \\
   &+ \dots \\
   & +\int_{t_{2k-1}}^{t_{2k}} F_2^x(\tau) e^{  \int_{\tau}^{t} \hat{S}_b -\bar{c}_{\Psi}^D \,d\tau^* }    \, d\tau   \\
   & +\int_{t_{2k}}^{t} F_2^x(\tau) e^{  \int_{\tau}^{t} \hat{S}_b-\bar{c}_{\Psi}^D\,d\tau^* }  \, d\tau   \\
\end{aligned}
\end{equation}
to proof $\tilde{z}$ converges to $0$ exponentially, we use the mean value theorem of integrals to deal with error term $F_2^x$.
\begin{equation}
F_2^x(\tau)=a_f e^{-\lambda \tau}
\end{equation}

Then, we have
\begin{equation}\label{EQD46}
\begin{aligned}
    \tilde{z} (t)
    &=  e^{\int_{t_{0}}^{t} \hat{S}_b(\tau) \, d\tau-\bar{c}_{\Psi}|S_N|}\tilde{z}(t_0) \\
   & +\int_{t_0}^{t_1} a_f e^{-\lambda \tau} e^{ \int_{\tau}^{t}\hat{S}_b-\bar{c}_{\Psi}^D\, d\tau^* }\,d \tau    \\
   & +\int_{t_1}^{t_2} a_f e^{-\lambda \tau} e^{ \int_{\tau}^{t}\hat{S}_b-\bar{c}_{\Psi}^D\, d\tau^*  }\,d \tau  \\
   &+ \dots \\
   & +\int_{t_{2k-1}}^{t_{2k}} a_f e^{-\lambda \tau} e^{ \int_{\tau}^{t}\hat{S}_b-\bar{c}_{\Psi}^D\, d\tau^*   }\,d \tau   \\
   & +\int_{t_{2k}}^{t} a_f e^{-\lambda \tau} e^{\int_{\tau}^{t}\hat{S}_b-\bar{c}_{\Psi}^D\, d\tau^* }\,d \tau  \\
\end{aligned}
\end{equation}

Similar to the case that $t\in [t_{2k},t_{2k+1})$, we have
\begin{equation}\label{EQD47}
\begin{aligned}
    \tilde{z} (t)
    &=  e^{\int_{t_{0}}^{t} \hat{S}_b(\tau) \, d\tau-\bar{c}_{\Psi}|S_N|}\tilde{z}(t_0) \\
   & +\int_{t_0}^{t_1} a_f e^{-\lambda \tau} e^{ \int_{\tau}^{t}\hat{S}_b-\bar{c}_{\Psi}^D\, d\tau^* }\,d \tau    \\
   & +\int_{t_1}^{t_2} a_f e^{-\lambda \tau} e^{ \int_{\tau}^{t}\hat{S}_b-\bar{c}_{\Psi}^D\, d\tau^*  }\,d \tau  \\
   &+ \dots \\
   & +\int_{t_{2k}}^{t_{2k+1}} a_f e^{-\lambda \tau} e^{ \int_{\tau}^{t}\hat{S}_b-\bar{c}_{\Psi}^D\, d\tau^* }\,d \tau   \\
   & +\int_{t_{2k+1}}^{t} a_f e^{-\lambda \tau} e^{\int_{\tau}^{t}\hat{S}_b-\bar{c}_{\Psi}^D\, d\tau^* }\,d \tau  \\
\end{aligned}
\end{equation}
from (\ref{EQD46}) and (\ref{EQD47}) , one has

\begin{equation}
\begin{aligned}
   \tilde{z} & = e^{\int_{t_0}^{t} \hat{S}_b(\tau) \, d\tau-\bar{c}_{\Psi}|S_N|}\tilde{z}(t_0) +    a_f  \int_{t_0}^{t} e^{ \int_{\tau}^{t} \hat{S}_b(\tau^*) \, d\tau^*-\bar{c}_{\Psi}|S_N(\tau,t)| +\lambda I_{N\times p}(-\tau)} \,d\tau \\
   % & \leq e^{\int_{t_0}^{t} \hat{S}_b(\tau) \, d\tau-\bar{c}_{\Psi}|S_N|}\tilde{z}(t_0) +   c_f e^{-\lambda \frac{\sigma_{2i}+t}{2}} \int_{t_0}^{t} e^{ \int_{t_0}^{t} \hat{S}_b(\tau^*) \, d\tau^*-\bar{c}_{\Psi}|S_N(t_0,t)| +\lambda(t-t_0)} \,d\tau .\\
\end{aligned}
\end{equation}
with

\begin{equation}
    \begin{aligned}
    &\int_{t_0}^{t} \hat{S}_b(\tau) \, d\tau-\bar{c}_{\Psi}|S_N|\\
    & \leq \int_{t_0}^{t} I_N \otimes S + \tilde{S}_b(\tau) \, d\tau-\bar{c}_{\Psi}|S_N(t_0,t)| \\
    & \leq  S_b (t-t_0)+||\int_{t_0}^{t}  \tilde{S}_b(\tau) \, d\tau|| -\bar{c}_{\Psi}(t-t_0-|S_A(t_0,t)|) \\
    &\leq S_b (t-t_0)+||\int_{t_0}^{t}  \tilde{S}_b(\tau) \, d\tau|| - \bar{c}_{\Psi}(t-t_0) + \bar{c}_{\Psi}(\frac{t-t_0}{\tau_a} +S_0 +(1+N_a(t_0,t))\Delta_*)\\
    &\leq S_b (t-t_0)+||\int_{t_0}^{t}  \tilde{S}_b(\tau) \, d\tau|| -\bar{c}_{\Psi}(t-t_0)
    +\frac{\bar{c}_{\Psi}}{\tau_a}(t-t_0) +\bar{c}_{\Psi}(S_0+\Delta_*) + \bar{c}_{\Psi} N_a(t_0,t)\Delta_* \\
    & \leq (S_b-\bar{c}_{\Psi}+\frac{\bar{c}_{\Psi}}{\tau_a}+ \eta^*)(t-t_0) +||\int_{t_0}^{t}  \tilde{S}_b(\tau) \, d\tau||+ \bar{c}_{\Psi}(S_0+\Delta_*)
    \end{aligned}
\end{equation}
where $c_f=a_f {\rm max}({c_k})$ , $S_b=I_N\otimes S$.

According to (\ref{Dos1}), (\ref{Dos2}) and (54), we can obtain that
\begin{equation}
\begin{aligned}
   \tilde{z}(t) 
   &\leq c_{ss} e^{-\eta(t-t_0)} \tilde{z}(t_0) +  \int_{t_0}^{t} a_f c_{ss} e^{-\eta(t-\tau)-\lambda \tau} \, d\tau \\
   &\leq c_{ss} e^{-\eta(t-t_0)} \tilde{z}(t_0)+ \frac{a_f c_{ss} }{\eta-\lambda} (e^{-\lambda t} -e^{-\eta t+(\eta - \lambda)t_0})
\end{aligned}
\end{equation}

where $c_{ss}=e^{||\int_{t_0}^{t}  \tilde{S}_b(\tau) \, d\tau||+ \bar{c}_{\Psi}(S_0+\Delta_*)}$, and 
%$\eta^*(t-t_0)=\bar{c}_{\Psi} N_a(t_0,t)\Delta_*$.
 $\eta=(\bar{c}_{\Psi}- \frac{\bar{c}_{\Psi}}{\tau_a}-S_b- \eta^*)>0$,  by $\tilde{S}_b$ converge to $0$ exponentially, we can obtain that $c_{ss}$ is bounded. 
Obviously, it can conclude that $\tilde{z}$ exponentially converges to zero  exponentially. $\hfill \hfill \blacksquare $


{\color{blue}
\begin{remark}
Compare with \cite{deng2021} which the leader' state  estimator start to work after finishing the leader' dynamic estimation, this work does not need to wait for the Leader state estimator to estimate an accurate state and the twin layer begin to work at the first time which is more applicable to the actual situation.  Moreover, deng et al. \cite{deng2021} only dealt with the case of a single leader with constant trajectory , we consider the more general case of output containment with multiple  leaders. And different from the existing work \cite{yang2020} which can only obtain the bounded state tracking error under Dos attack, the state tracking error of twin layer in our work converges to zero exponentially which have more conservative result.
$\hfill \hfill \square $
\end{remark}
}












}
}











%}













\subsection{ Full Distributed Output Regulator Equation Solvers}
{\color{blue}
Since agents only can receive information from their neighbors, that is, leaders' dynamic and information are unknown for some agents. Therefore,  the output regulator equation which need to know the global information as zuo et al. \cite{zuo2020} cannot be used here. With the estimated leader dynamics coverage to the actual leader dynamics in Theorem 1, the output regulator of estimated leader dynamics can be solved by the following theorem.
}

\begin{myTheo}
 suppose the Assumptions 2,3,4 hold, the estimated solutions to the output regulator equations in (51), including 
$\hat{\Delta}_{ji}$ for $j$ = $0, 1$, and $\hat{\Delta}_{i}$ , are adaptively solved as follows:
\begin{align}
    &\dot{\hat{\Delta}}_{0i} = -\mu_i \hat{\Phi }^{\mathrm{T}}_i(\hat{\Phi }_i \hat{\Delta}_{0i}-\hat{\mathcal{R}}_i) \label{equation 310} \\
    &\dot{\hat{\Delta}}_{1i} = \left\lVert \hat{\Upsilon}_i\right\rVert_2(\hat{\Delta}_{0i}- \hat{\Delta}_{1i})-\mu_i\hat{\Phi }^{\mathrm{T}}_i(\hat{\Phi }_i \hat{\Delta}_{1i}-\hat{\mathcal{R}}_i)\\
   &\dot{\hat{\Delta}}_{i} = \left\lVert \hat{\Upsilon}_i\right\rVert_2(\hat{\Delta}_{1i}- \hat{\Delta}_{i})-\mu_i\hat{\Phi }^{\mathrm{T}}_i(\hat{\Phi }_i \hat{\Delta}_{i}-\hat{\mathcal{R} }_i) ,
\end{align}

where $\mu_i > 0$, $\hat{\Delta}_{ji}={\rm vec}(\hat{Y}_{ji})$ for $j=0,1$ , $\hat{\Delta}_i={\rm vec}(\hat{Y}_i), \hat{\Phi}_i=(I_q \otimes  A_{1i}-\hat{S}_i^{\mathrm{T}} \otimes A_{2i})$, $\hat{R}_i={\rm vec}(\mathcal{\hat{R}}^{\ast}_i)$; $\hat{Y}_{ji}=[\hat{\Pi}_{ji}^{\mathrm{T}}$, $\hat{\Gamma}_{ji}^{\mathrm{T}}]^{\mathrm{T}}$ for $j=0,1$ , $\hat{Y_{i}}=[\hat{\Pi}_{i}^{\mathrm{T}}$, $\hat{\Gamma}_{i}^{\mathrm{T}}]^{\mathrm{T}}$, $\hat{R}^{\ast}_i=[0,\hat{R}_i^{\mathrm{T}}]^{\mathrm{T}}$, $
A_{1i}=\left[
  \begin{array}{cc}
  A_i &  B_i  \\
  C_i &  0   \\
  \end{array}
  \right]$, $A_{2i}=
    \left[
  \begin{array}{cc}
   I_{n_i} & 0  \\
   0 &  0   \\
  \end{array}
  \right]$.
\end{myTheo}
\textbf{Proof.}
In Theorem 1, we realize that leader dynamic estimation is time-varying, so the output solution regulator equation estimation influenced by leader dynamic estimation is also time-varying. Next, we begin proof that the output solution regulator equation estimation Converges exponentially to the output solution regulator equation.

\textbf{Step 1:}
Firstly, we proof that $\hat{\Delta}_{0i}$ converges to $\Delta$ at an exponential rate.

From Assumption \ref{assumption 5} , we can get that the following output regulator equation:
 \begin{equation}\label{EQ10}
     \begin{cases}
      A_i\Pi_i+B_i\Gamma_i=\Pi_i S \\
      C_i\Pi_i = R
     \end{cases}
 \end{equation}
have solution matrices  $\Pi_i$ and $\Gamma_i$ for $i=\textbf{I}[1,N]$ .

Rewriting the standard output regulation equation (\ref{EQ10}) yields as follow:
\begin{equation}\label{EQ51}
 \left[
  \begin{array}{cc}
  A_i &  B_i  \\
  C_i &  0   \\
  \end{array}
  \right]
   \left[
    \begin{array}{cc}
   \Pi_i \\
   \Gamma_i \\
    \end{array}
    \right]
    I_q -
    \left[
  \begin{array}{cc}
   I_{n_i} & 0  \\
   0 &  0   \\
  \end{array}
  \right]
   \left[
    \begin{array}{cc}
\Pi_i\\
\Gamma_i \\
    \end{array}
    \right] S
    =
    \left[\begin{array}{cc}
         0  \\
         R
    \end{array}\right].
\end{equation}
%
Reformulating the equation (\ref{EQ51}) as follow:
\begin{equation}
    A_{1i}Y I_q -A_{2i}Y S=R_i^{\ast},
\end{equation}
where $A_{1i}=\left[
  \begin{array}{cc}
  A_i &  B_i  \\
  C_i &  0   \\
  \end{array}
  \right],
  A_{2i}=\left[
  \begin{array}{cc}
   I_{n_i} & 0  \\
   0 &  0   \\
  \end{array}
  \right],
  Y_i= \left[
    \begin{array}{cc}
\Pi_i\\
\Gamma_i \\
    \end{array}
    \right], $ and $R_i^{\ast}= \left[\begin{array}{cc}
         0  \\
         R
    \end{array}\right] $. Then the standard form of the linear equation in Equation (\ref{EQ51}) can be rewritten as:
\begin{equation}
    \Phi_i \Delta=\mathcal{R}_i,
\end{equation}
where $\Phi_i=(I_q \otimes  A_{1i}-S^{\mathrm{T}} \otimes A_{2i})$, $\Delta_i={\rm vec}( \left[
    \begin{array}{cc}
\Pi_i\\
\Gamma_i \\
    \end{array}
    \right])$, 
    $\mathcal{R}_i={\rm vec}(\mathcal{R}^{\ast})$, $R^{\ast}=\left[
    \begin{array}{cc}
0\\
R \\
    \end{array}
    \right]$.
\begin{equation}
\begin{aligned}
\dot{\hat{\Delta}}_{0i}
&= -\mu_i \hat{\Phi }^{\mathrm{T}}_i(\hat{\Phi }_i \hat{\Delta}_{0i}-\hat{\mathcal{R}}_i) \\
&=-\mu_i \hat{\Phi}^{\mathrm{T}}_i \hat{\Phi}_i \hat{\Delta}_{0i}+ \mu_i \hat{\Phi}_i \hat{\mathcal{R}_i} \\
&=-\mu_i \Phi_i^{\mathrm{T}} \Phi_i \hat{\Delta}_{0i} +\mu_i \Phi_i^{\mathrm{T}} \Phi_i \hat{\Delta}_{0i} -\mu_i \hat{\Phi}_i^{\mathrm{T}}  \hat{\Phi}_i \hat{\Delta}_{0i} +\mu_i \hat{\Phi}_i^{\mathrm{T}} \hat{\mathcal{R}_i} -\mu_i \Phi_i^{\mathrm{T}} \hat{\mathcal{R}_i} + \mu_i \Phi_i^{\mathrm{T}} \hat{\mathcal{R}_i} -\mu_i \Phi_i^{\mathrm{T}} \mathcal{R}_i + \mu_i \Phi_i^{\mathrm{T}} \mathcal{R}_i \\
&=-\mu_i \Phi_i^{\mathrm{T}} \Phi_i \hat{\Delta}_{0i} + \mu_i (\Phi_i^{\mathrm{T}} \Phi_i - \hat{\Phi}_i^{\mathrm{T}} \hat{\Phi}_i) \hat{\Delta}_{0i} + \mu_i (\hat{\Phi}_i^{\mathrm{T}} - \Phi_i^{\mathrm{T}}) \hat{\mathcal{R}_i} + \mu_i \Phi_i^{\mathrm{T}} (\hat{\mathcal{R}_i}-\mathcal{R}_i) + \mu_i \Phi_i^{\mathrm{T}} \mathcal{R}_i \\
&=-\mu_i \Phi_i^{\mathrm{T}} \Phi_i \hat{\Delta}_{0i} + \mu_i \Phi_i^{\mathrm{T}} \mathcal{R}_i +d_1(t) ,\\
\end{aligned}
\end{equation}
where  $d_1(t)= -\mu_i (\hat{\Phi}_i^{\mathrm{T}}\hat{\Phi}_i - \Phi_i^{\mathrm{T}}\Phi_i) \hat{\Delta}_{0i} + \mu_i \tilde{\Phi}_i^{\mathrm{T}} \hat{\mathcal{R}}_i + \mu_i \Phi_i^{\mathrm{T}} \tilde{\mathcal{R}}_i$ with $\tilde{\Phi}_i=\hat{\Phi}_i-\Phi_i=\tilde{S}^{\mathrm{T}} \otimes A_{2i}$ and $\tilde{\mathcal{R}}={\rm vec}(\left[\begin{array}{cc}
    0  \\
    \tilde{R}_i
    \end{array}
    \right])$,
so $\lim _{t \rightarrow \infty}d_1(t) =0 $ exponentially at rate of $\alpha_{\Upsilon}$($\alpha_{\Upsilon}$ is the exponential convergence rate of $\tilde{\Upsilon}$ ).


Let $\tilde{\Delta}_{0i}=\Delta - \hat{\Delta}_{0i}$, The time derivative of  $\tilde{\Delta}_{0i}$
can be computed via
\begin{equation}
\begin{aligned}
    \dot{\tilde{\Delta}}_{0i}
    =-\mu_i \Phi_i^{\mathrm{T}} \Phi_i \tilde{\Delta}_{0i} - \mu_i \Phi_i^{\mathrm{T}} \Phi_i \Delta+\mu_i \Phi_i^{\mathrm{T}} \mathcal{R}_i +d_1(t)
    =-\mu_i \Phi_i^{\mathrm{T}} \Phi_i \tilde{\Delta}_{0i} +d_1(t),
\end{aligned}
\end{equation}
 since $ \Phi_i^{\mathrm{T}} \Phi_i$ is  Hurwitz, and $d_i(t)$ coverage to $0$ at rate of $\alpha_{\Upsilon}$, by Lemma \ref{Lemma 1}   $\lim _{t \rightarrow \infty }\tilde{\Delta}_{0i}=0$ exponentially . In particular, if $\mu_i \geq \frac{\alpha_{\Upsilon}}{ \sigma_{\rm min}(\Phi_i^{\mathrm{T}}\Phi_i)}$($\mu_i$ is sufficiently large ), $\lim _{t \rightarrow \infty }\tilde{\Delta}_{0i}=0$ exponentially at the rate of $\alpha_{\Upsilon}$.
 
 \textbf{Step 2:}
To prepare the proof for the next step, we use the solution of Step 1 to calculate the convergence rates of $\mu_i\hat{\Phi}^{\mathrm{T}}_i(\hat{\Phi}_i \Delta_i-\hat{\mathcal{R} }_i)$.

Rewrite equation (56) as follow
 \begin{equation}\label{EQ2057}
 \dot{\tilde{\Delta}}_{0i}
=\dot{\Delta}-\dot{\hat{\Delta}}_{0i}
=-\mu_i\hat{\Phi}^{\mathrm{T}}_i\hat{\Phi}_i\tilde{\Delta}_{0i}+\mu_i\hat{\Phi}^{\mathrm{T}}_i(\hat{\Phi}_i \Delta_i-\hat{\mathcal{R} }_i),
 \end{equation}
 and solved equation (\ref{EQ2057}) yield 
 \begin{equation}
     \tilde{\Delta}_{0i}=\tilde{\Delta}_{0i}(t_0) e^{-\int_{t_0}^{t} \mu_i\hat{\Phi}^{\mathrm{T}}_i(\hat{\Phi}_i \, d\tau } +\int_{t_0}^{t} e^{-\int_{\tau}^{t} \mu_i\hat{\Phi}^{\mathrm{T}}_i\hat{\Phi}_i \, d s} \,d\tau \int_{t_0}^{t}\mu_i\hat{\Phi}^{\mathrm{T}}_i(\hat{\Phi}_i \Delta_i-\hat{\mathcal{R} }_i) \,d\tau.
 \end{equation}
By $\tilde{\Delta}_{0i}$  coverage to $0$ exponentially, it obvious that ${\rm  lim}_{t\rightarrow \infty} \mu_i\hat{\Phi}^{\mathrm{T}}_i(\hat{\Phi}_i \Delta_i-\hat{\mathcal{R} }_i)=0$ exponentially. And if $ \sigma_{\rm min}(\mu_i\hat{\Phi}^{\mathrm{T}}_i\hat{\Phi}_i)) > \alpha_{\Upsilon}$, we can get that
${\rm  lim}_{t\rightarrow \infty} \mu_i\hat{\Phi}^{\mathrm{T}}_i(\hat{\Phi}_i \Delta_i-\hat{\mathcal{R} }_i)=0$ exponentially at the rate of $\alpha_{\Upsilon}$.

\textbf{Step 3:}
Now, we proof that $\hat{\Delta}_{1i}$ and $\hat{\Delta}_i$ converge to $\Delta$, respectively.

Let $\tilde{\Delta}_{1i}= \Delta - \Delta_{1i}$, then, The time derivative of  $\tilde{\Delta}_{1i}$
can be computed via
\begin{equation}
    \dot{\tilde{\Delta}}_{1i}
=\dot{\Delta}-\dot{\hat{\Delta}}_{1i}
=-(\left\lVert \hat{\Upsilon}_i\right\rVert_2I_{p^2+pq}+ \mu_i\hat{\Phi}^{\mathrm{T}}_i\hat{\Phi}_i)\tilde{\Delta}_{1i}+\left\lVert \hat{\Upsilon}_i\right\rVert_2\tilde{\Delta}_{0i} +\mu_i\hat{\Phi}^{\mathrm{T}}_i(\hat{\Phi}_i \Delta_i-\hat{\mathcal{R} }_i),
\end{equation}
$ \sigma_{\rm min}(\left\lVert \hat{\Upsilon}_i\right\rVert_2I_{p^2+pq}+ \mu_i\hat{\Phi}^{\mathrm{T}}_i\hat{\Phi}_i) > \alpha_{\Upsilon}$, $\left\lVert \hat{\Upsilon}_i\right\rVert_2\tilde{\Delta}_{0i} +\mu_i\hat{\Phi}^{\mathrm{T}}_i(\hat{\Phi}_i \Delta_i-\hat{\mathcal{R} }_i)$ coverage to $0$ exponentially at the rate of $\alpha_{\Upsilon}$ under the condition of $ \sigma_{\rm min}(\mu_i\hat{\Phi}^{\mathrm{T}}_i\hat{\Phi}_i)) > \alpha_{\Upsilon}$.
By Lemma1, we obtain ${\rm  lim}_{t\rightarrow \infty }  \tilde{\Delta}_{1i} = 0$ exponentially at the rate of $\alpha_{\Upsilon}$ under the condition of $ \sigma_{\rm min}(\mu_i\hat{\Phi}^{\mathrm{T}}_i\hat{\Phi}_i)) > \alpha_{\Upsilon}$.


Let $\tilde{\Delta}_{i}= \Delta - \Delta_{i}$ , as the same proof of $\tilde{\Delta}_{1i}$, it easy to get that ${\rm lim}_{t\rightarrow \infty }  \tilde{\Delta}_{i} = 0$ exponentially at the rate of $\alpha_{\Upsilon}$ under the condition of $ \sigma_{\rm min}(\mu_i\hat{\Phi}^{\mathrm{T}}_i\hat{\Phi}_i)) > \alpha_{\Upsilon}$ , the proof is completed. $\hfill \hfill \blacksquare $
%
\begin{lemma}[\cite{chen2019},Lemma 4]\label{Lemma 5}
    The adaptive distributed leader dynamics observers in (\ref{EQ16}) ensure $ \left\lVert \tilde{\Delta}_i \right\rVert \left\lVert x_k\right\rVert  $ and $\dot{\hat{\Pi}}_i x_k $  exponentially converge to zero.
\end{lemma}





\subsection{ Distributed Resilient Controller Design}

%{\color{blue}

In this section, we propose a fully distributed observer containment control method to solve the containment problem.

 Define the state tracking error as follow:
 \begin{equation}
      \epsilon_i=x_i - \hat{\Pi}_i z_i\\ \label{EQ58}
 \end{equation}
then considering  the full distributed resilient control protocols as follows:
\begin{align}
 & u_i=\hat{\Gamma}_i z_i +K_i \epsilon_i -\hat{\chi}_i  \\  
&\hat{\chi}_i=\frac{B_i^{\mathrm{T}}  P_i \epsilon_i}{\left\lVert \epsilon_i^{\mathrm{T}} P_i B_i \right\rVert +\omega} \hat{\rho_i} \\ \label{EQ65}
&\dot{\hat{\rho}}_i=\begin{cases}
 \left\lVert \epsilon_i^{\mathrm{T}} P_i B_i \right\rVert +2\omega , \mbox{if}  \left\lVert \epsilon_i^{\mathrm{T}} P_i B_i \right\rVert \geq \bar{\kappa} \\
 \left\lVert \epsilon_i^{\mathrm{T}} P_i B_i \right\rVert +2\omega \frac{\left\lVert \epsilon_i^{\mathrm{T}} P_i B_i \right\rVert}{\bar{\kappa}}, \mbox{ otherwise},
\end{cases}
\end{align}
where  $\hat{\chi}_i$ is an adaptive compensational signal, $\hat{\rho}_i$ is an adaptive updating parameter and the controller gain $K_i$ is designed as 
\begin{equation}
    K_i=-R_i^{-1}B_i^{\mathrm{T}} P_i
\end{equation}
where $P_i$ is the solution to
\begin{equation}
    A_i^{\mathrm{T}} P_i + P_i A_i + Q_i -P_i B_i R_i^{-1} B_i^{\mathrm{T}} P_i =0.
\end{equation}
%


\begin{myTheo}
 Consider  heterogeneous MAS consisting of   $M$ leader (\ref{EQ1}) and $N$ followers (\ref{EQ2}) with unbounded faults. Under Assumptions 1-6, the \textbf{Problem ACMCA} is solved by designing the dynamic estimation (19)-(20), the fully distributed virtual resilient layer(30), the standard output regulation equation (58), the  output  solution regulator  equation  estimation(55)-(57) and the fully distributed controller consisting of (66)-(70).
\end{myTheo}
\textbf{Proof.}
Firstly, we need to prove the state tracking error (67) is unbounded.

The derivative of $\epsilon_i$ is presented as follows:
  \begin{equation}\label{EQ63}
  \begin{aligned}
      \dot{\epsilon}_i
      &= A_i x_i + B_i u_i +B_i \chi_i -\dot{\hat{\Pi}}_i  z_i - \hat{\Pi}_i(\hat{S}_i z_i - c_{\Psi} (\sum _{j \in \mathcal{F} }a_{ij}(z_i-z_j)+\sum_{k \in \mathcal{L}}g_{ik}(z_i-x_k)))\\
      &=(A_i+B_i K_i)\epsilon_i  - \dot{\hat{\Pi}}_i z_i +c_{\Psi} \hat{\Pi}_i  (\sum _{j \in \mathcal{F} }a_{ij}(z_i-z_j)+\sum_{k \in \mathcal{L}}g_{ik}(z_i-x_k))+B_i \tilde{\chi}_i.
  \end{aligned}
  \end{equation}
  where $\tilde{\chi}_i=\chi_i-\hat{\chi}_i$.
%  
 the global state tracking error of (\ref{EQ63}) is
 \begin{equation}
 \begin{aligned}
  \dot{\epsilon}
  &={\rm blkdiag}(A_i+B_iK_i)\epsilon -{\rm blkdiag}(\dot{\hat{\Pi}}_i)z + c_{\Psi} {\rm blkdiag}(\hat{\Pi}_i)(\sum_{k \in \mathcal{L}}(\Psi_k \otimes I_p )(z-\underline{x}_k))+{\rm blkdiag}(B_i)\tilde{\chi}
 \\
  &={\rm blkdiag}(A_i+B_i K_i)\epsilon -\dot{\hat{\Pi}}_b(\tilde{z}+ (\sum_{r \in \mathcal{L}}(\Psi_r \otimes I_p ))^{-1} \sum_{k \in \mathcal{L}}(\Psi_k \otimes I_p ) \underline{x}_k) + c_{\Psi} \hat{\Pi}_b\sum_{k \in \mathcal{L}}(\Psi_k \otimes I_p )\tilde{z} +B_b \tilde{\chi}
  .\\
\end{aligned}
 \end{equation}
 where $\dot{\hat{\Pi}}_b={\rm blkdiag}(\dot{\hat{\Pi}}_i)$, $\hat{\Pi}_b= {\rm blkdiag}(\hat{\Pi}_i)$ for $i=\textbf{I}[1,N]$ and $\epsilon=[\epsilon_1^{\mathrm{T}} \epsilon_2^{\mathrm{T}} \dots \epsilon_N^{\mathrm{T}}]^{\mathrm{T}}$, $\tilde{\chi}=[\tilde{\chi}_1^{\mathrm{T}} \tilde{\chi}_2^{\mathrm{T}} \dots \tilde{\chi}_N^{\mathrm{T}}]^{\mathrm{T}}$.
%
%


Consider the following Lyapunov function candidate :
\begin{equation} 
    V= \epsilon ^{\mathrm{T}} P_b \epsilon,
\end{equation}
where $P_b={\rm blkdiag}(P_i)$
and its time derivate is given as follows: 
\begin{equation}\label{EQ80}
\begin{aligned}
    \dot{V}
    &= 2\epsilon^{\mathrm{T}} P_b \dot{\epsilon}_i \\
    &=-\epsilon^{\mathrm{T}} {\rm blkdiag}(Q_i) \epsilon +2\epsilon^{\mathrm{T}} P_b (\dot{\hat{\Pi}}_b+  c_{\Psi} \hat{\Pi}_b\sum_{k \in \mathcal{L}}(\Psi_k \otimes I_p ) )\tilde{z} \\
    & + 2\epsilon^{\mathrm{T}} P_b \dot{\hat{\Pi}}_b(\sum_{r \in \mathcal{L}}(\Psi_r \otimes I_p ))^{-1} \sum_{k \in \mathcal{L}}(\Psi_k \otimes I_p ) \underline{x}_k + 2\epsilon^{\mathrm{T}} P_b B_b\tilde{\chi}\\
    &\leq -\sigma_{min}(Q_b) \left\lVert \epsilon\right\rVert^2  +2\left\lVert \epsilon^{\mathrm{T}}\right\rVert \left\lVert P\right\rVert_2  \left\lVert \dot{\hat{\Pi}}_b+  c_{\Psi} \hat{\Pi}_b\sum_{k \in \mathcal{L}}(\Psi_k \otimes I_p )  \right\rVert  \left\lVert \tilde{z}\right\rVert\\ 
 & +2\left\lVert \epsilon^{\mathrm{T}}\right\rVert \left\lVert P\right\rVert_2   \left\lVert  
  (\sum_{r \in \mathcal{L}}(\Psi_r \otimes I_p ))^{-1}\sum_{k \in \mathcal{L}}(\Psi_k \otimes I_p ) \dot{\hat{\Pi}}_b \underline{x}_k\right\rVert + 2\epsilon^{\mathrm{T}} P_b B_b\tilde{\chi}, \\
\end{aligned}
\end{equation}
where $Q_b={\rm blkdiag}(Q_i)$.
%
%
%
%

$\dot{\hat{\Pi}}_i x_k$ converge to zero exponentially and $\Psi_k$ is positive, so there exist positove constants $ V_{\Pi}$ and $ \alpha_{\Pi}$ such that
the following holds :
\begin{equation}\label{EQ81}
  \left\lVert  (\sum_{r \in \mathcal{L}}(\Psi_r \otimes I_p ))^{-1}\sum_{k \in \mathcal{L}}(\Psi_k \otimes I_p ) \dot{\hat{\Pi}}_b \underline{x}_k\right\rVert 
  \leq  V_{\Pi} \exp (-\alpha_{\Pi}),
\end{equation}
use Young's inequality , we have
\begin{equation}\label{EQ61}
\begin{aligned}
    & 2\left\lVert \epsilon^{\mathrm{T}}\right\rVert \left\lVert P\right\rVert_2   \left\lVert  \dot{\hat{\Pi}}_b 
  (\sum_{r \in \mathcal{L}}(\Psi_r \otimes I_p ))^{-1}\sum_{k \in \mathcal{L}}(\Psi_k \otimes I_p ) \underline{x}_k\right\rVert \\
  & \leq \left\lVert \epsilon\right\rVert^2 + \left\lVert P\right\rVert_2  ^2 \left\lVert \dot{\hat{\Pi}}_b 
  (\sum_{r \in \mathcal{L}}(\Psi_r \otimes I_p ))^{-1}\sum_{k \in \mathcal{L}}(\Psi_k \otimes I_p ) \underline{x}_k\right\rVert^2 \\
  & \leq (\frac{1}{4} \sigma_{min}(Q_b) - \frac{1}{2}\beta_{V1} )\left\lVert \epsilon_i\right\rVert ^2 +\frac{\left\lVert P\right\rVert_2  ^2}{ (\frac{1}{4} \sigma_{min}(Q_b) - \frac{1}{2}\beta_{V1} )} V_{\Pi}^2 \exp (-2\alpha_{\Pi})\\
  & \leq
  (\frac{1}{4} \sigma_{min}(Q_b) - \frac{1}{2}\beta_{V1} )\left\lVert \epsilon_i\right\rVert ^2 +\beta_{v21}e^{-2\alpha_{v1}t} .
\end{aligned}
\end{equation}
From Lemma \ref{Lemma 5}, we can obtain that $\dot{\hat{\Pi}}$ converge to 0 exponentially. By $\tilde{z}$ also converge to zero exponentially, we similarly obtain that
\begin{equation}\label{EQ62}
\begin{aligned}
   & 2\left\lVert \epsilon^{\mathrm{T}}\right\rVert \left\lVert P\right\rVert_2  \left\lVert \dot{\hat{\Pi}}_b+  c_{\Psi} \hat{\Pi}_b\sum_{k \in \mathcal{L}}(\Psi_k \otimes I_p )  \right\rVert  \left\lVert \tilde{z}\right\rVert \\
  & \leq (\frac{1}{4} \sigma_{min}(Q_b) - \frac{1}{2}\beta_{V1} )\left\lVert \epsilon\right\rVert ^2+\beta_{v22}e^{-2\alpha_{v2}t}.
\end{aligned}
\end{equation}

and noting that
\begin{equation}
\begin{aligned}
\epsilon_i^{\mathrm{T}} P_i  B_i\tilde{\chi}_i
        &= \epsilon_i^{\mathrm{T}} P_i  B_i \chi_i -\frac{\left\lVert \epsilon_i^{\mathrm{T}} P_i B_i \right\rVert^2}{ \left\lVert \epsilon_i^{\mathrm{T}} P_i B_i \right\rVert +\omega} \hat{\rho_i} \\
& \leq  \left\lVert \epsilon_i^{\mathrm{T}} P_i  B_i\right\rVert \left\lVert \chi_i \right\rVert - \frac{\left\lVert \tilde{y}_i^{\mathrm{T}} P_i C_i B_i \right\rVert^2  }{\left\lVert \epsilon_i^{\mathrm{T}} P_i B_i \right\rVert + \omega} \hat{\rho_i} \\
& = \frac{\left\lVert \epsilon_i^{\mathrm{T}} P_i B_i \right\rVert ^2 (  \left\lVert \chi_i \right\rVert-  \hat{\rho}_i)+ \left\lVert \epsilon_i^{\mathrm{T}} P_i B_i \right\rVert\left\lVert \chi_i \right\rVert \omega}{\left\lVert \epsilon_i^{\mathrm{T}} P_i B_i \right\rVert +\omega } \\
&=\frac{\left\lVert \epsilon_i^{\mathrm{T}} P_i B_i \right\rVert ^2 (\frac{\left\lVert \epsilon_i^{\mathrm{T}} P_i B_i \right\rVert +\omega}{\left\lVert \epsilon_i^{\mathrm{T}} P_i B_i \right\rVert}||\chi_i||-\hat{\rho}_i)}{\left\lVert \epsilon_i^{\mathrm{T}} P_i B_i \right\rVert +\omega}.
\end{aligned}
\end{equation}
Noting that $d\left\lVert \chi_i \right\rVert/dt $ is bounded,  so, if $\left\lVert \epsilon_i^{\mathrm{T}} P_i B_i \right\rVert\geq \bar{\kappa} \geq \frac{d||\chi_i||}{d t}$, that is, $\frac{\bar{\kappa}+\omega}{\bar{\kappa}} \frac{d||\chi_i||}{dt} -\dot{\hat{\rho}} \leq \bar{\kappa}+\omega-\dot{\hat{\rho}} \leq -\omega < 0$. Then , there exists $t_2 > 0$ such that for all $t \geq t_2$ , we have 
\begin{equation}\label{EQ76}
     (\frac{\left\lVert \epsilon_i^{\mathrm{T}} P_i B_i \right\rVert +\omega}{\left\lVert \epsilon_i^{\mathrm{T}} P_i B_i \right\rVert}||\chi_i||-\hat{\rho}_i) \leq (\frac{\bar{\kappa}+\omega}{\kappa}||\chi_i||-\hat{\rho}_i) <0  .
\end{equation}
Thus, we can obtain that $\epsilon_i^{\mathrm{T}} P_i  B_i\tilde{\chi}_i <0$ and $\epsilon^{\mathrm{T}} P_b  B_b\tilde{\chi}$ over $t \in[t_2,\infty)$.

From (\ref{EQ81}) ~ (\ref{EQ76}), it yields
\begin{equation}\label{EQ70}
  \dot{V} \leq -(\beta_{V1}+\frac{1}{2} \sigma_{min}(Q_b) ) \epsilon^{\mathrm{T}} \epsilon +\beta_{V2}e^{\alpha_{V1}t}.
\end{equation}
Solving (\ref{EQ70}) yields the following:
\begin{equation}
\begin{aligned}
     V(t) 
    & \leq V(0) - \int_{0}^{t} (\beta_{V1}+\frac{1}{2} \sigma_{min}(Q_b) ) \epsilon^{\mathrm{T}}\epsilon \,d\tau + \int_{0}^{t} \beta_{V2}e^{\alpha_{V1}t} \,d\tau ,\\
\end{aligned}
\end{equation}
then
\begin{equation}\label{EQ72}
    \epsilon^{\mathrm{T}} \epsilon \leq -\int_{0}^{t} \frac{1}{\sigma_{min}(P)} \beta_{V3} \epsilon^{\mathrm{T}}\epsilon \,d\tau + \bar{B}.
\end{equation}
where $\beta_{V3}= \beta_{V1}+\frac{1}{2} \sigma_{min}(Q_b) $ and $ \bar{B}=V(0)-\int_{0}^{t} \beta_{V2}e^{\alpha_{V1}t} \,d\tau $ are define as a bounded constant.
Recalling Bellman-Gronwall Lemma, (\ref{EQ72}) is rewritten as follows:

\begin{equation}
  \left\lVert \epsilon \right\rVert \leq \sqrt{\bar{B}} e^{-\frac{\beta_{V3}t}{2\sigma_{min}(P)} },
\end{equation}
conclude (73) and (77), we can that $\epsilon$ is unbound by $\bar{\epsilon}$, where $\bar{\epsilon}=[\bar{\epsilon}_1^{\mathrm{T}} \bar{\epsilon}_2^{\mathrm{T}} \dots \bar{\epsilon}_N^{\mathrm{T}}]^{\mathrm{T}}$ with $||\bar{\epsilon}_i||=\frac{d||\chi_i|| / dt}{||P_i B_i||}$.
%
%


%}
%{\color{red}







Hence, based on Lemma 6, the global output synchronization error satisfieds that
\begin{equation}
\begin{aligned}
e &= y - (\sum_{r\in \mathcal{L} }(\Phi_r \otimes I_p))^{-1} \sum_{k \in \mathcal{L} } (\Psi_k \otimes I_p) \underline{y}_k \\
&=y-(I_N \otimes R)(z-\tilde{z})\\
&= {\rm blkdiag}(C_i)x   -{\rm blkdiag}(C_i \hat{\Pi}_i)z +{\rm blkdiag}(C_i \hat{\Pi}_i)z -(I_N \otimes R)z +(I_N \otimes R) \tilde{z} \\
&=  {\rm blkdiag}(C_i)\epsilon  -{\rm blkdiag}(\tilde{R}_i)z +(I_N \otimes R) \tilde{z}.
\end{aligned}
\end{equation}
%{\color{blue}
Since $\epsilon$, $\tilde{R}_i$, $\tilde{z}$  converge to 0  exponentially and $\tilde{x}$ is  unbounded which have been proofed before, it is obviously that the global output containment error $e$ is unbounded by $\bar{e}$ with $\bar{e}=[\bar{e}_1^{\mathrm{T}} \bar{e}_2^{\mathrm{T}} \dots \bar{e}_i^{\mathrm{T}}]^{\mathrm{T}}$ and $\bar{e}_i=\frac{||C_i|| d||\chi_i||/dt}{||\bar{P}_i B_i||}$ for $i= \textbf{I}[1,N]$. 
%}
The whole proof is completed.
$\hfill \hfill \blacksquare $

\begin{remark}
Compared with \cite{chen2019} which only solved the tracking problem with a single leader against bounded attacks, we consider more challenge work with multiple leaders output containment problem against unbounded attacks. Moreover, the boundary of the output containment error is given, that is, $ \frac{||C_i|| d||\chi_i||/dt}{||\bar{P}_i B_i||}$. It's helpful for designers to pre-estimate of the controller.
$\hfill \hfill  \square $
\end{remark}

%}








\section{Numerical Simulation}\label{SecSm}

In this section, we consider  a multigroup system consisting of seven agents(three leaders 5~7 and four followers 1~4) with the corresponding graph shown in
Fig.1. Assuming that $a_{i j}=1$ if $(v_j,v_i)\in \mathcal{\epsilon}$ . The dynamics of leaders are described by 
$$
S
=
 \left[
  \begin{array}{cc}
 1  & -2  \\
 2 &  -1   \\ 
  \end{array}
  \right],R
  =
   \left[
    \begin{array}{cc}
   1 &  0    \\
   0  & 1  \\ 
    \end{array}
    \right].
$$

The dynamics of follows are given by
$$
A1
=
 \left[
  \begin{array}{cc}
 3  & -2  \\
 1 &  -2   \\ 
  \end{array}
  \right],B1
  =
   \left[
    \begin{array}{cc}
   1.8 &  -1    \\
   2  & 3  \\ 
    \end{array}
    \right],C1
  =
   \left[
    \begin{array}{cc}
   1 &  -2    \\
  4  & -3  \\ 
    \end{array}
    \right],
$$
$$
A2
=
 \left[
  \begin{array}{cc}
 0  & -1  \\
 1 &  -2   \\ 
  \end{array}
  \right],B2
  =
   \left[
    \begin{array}{cc}
   1 &  -2    \\
   3  & 3  \\ 
    \end{array}
    \right],C2
  =
   \left[
    \begin{array}{cc}
   -2 &  1    \\
  3  & -2  \\ 
    \end{array}
    \right],
$$
$$
A3=A4
=
 \left[
  \begin{array}{ccc}
 0  &  1  &0\\
 0 &  0 & 1  \\ 
  0 &  0 & -2  \\ 
  \end{array}
  \right],B3
  =B4=
   \left[
    \begin{array}{ccc}
   6 &  0    \\
   0  & 1  \\ 
   1  & 0 \\ 
    \end{array}
    \right],C3=C4
  =
   \left[
    \begin{array}{ccc}
   1 &  -1  & 1  \\
  -1  & -1  &1\\ 
    \end{array}
    \right].
$$

The Dos attack periods are given as $[0.15,0.55)s,[1.2,1.6)s,[2.7,3.1)s,[3.5,3.6)s,[4.3,5)s,[6.8,8)s$ and $[8.8,9.5)s$, which satisfied Assumption 3. The AAs  are designed as $\chi_1=\chi_2=\chi_3=0.01 \times [2t, t]^{\mathrm{T}}$ and $\chi_4=-0.01\times[2t, t]^{\mathrm{T}}$. 

To ensure TL can work stability under Dos attack, the gain of twin layer is chosen as  $c_{\Psi}=30$ which satisfied the condition in theorem and the state estimation error of TL can be seen in Fig 2 and Fig 3. The MAS can againt the CAs and FDI attacks since the information transmitted on the CPL is not used in the hierarchal control scheme. And to solve the AAs on the CPL, we design the controller gain in (64) as 
$$
K1
=
 \left[
  \begin{array}{cc}
 -1.71  & 0.07  \\
 2.31  &  -1.12   \\ 
  \end{array}
  \right],K2
  =
   \left[
    \begin{array}{cc}
   -0.40 &  -0.44   \\
   0.90  & -0.48  \\ 
    \end{array}
    \right],K3=K4
  =
   \left[
    \begin{array}{ccc}
   -1.00 &  -0.19  & -0.08  \\
  0.02  & -0.96  &-0.3\\ 
    \end{array}
    \right].
$$
 The trajectories of leaders dynamics errors and the solution errors of output regulating equations are given in Fig 4 and Fig 5 , respectively, which validate Theorems 1 and 3.  Fig 6 illustrates the tracking error $\epsilon_i$ is bounded with the upper boundary proved by Theorem 4. The output trajectories of all the agents are showed in the Fig 7 with $y_i=[y_i(1),y_i(2)]$. It can be seen that followers eventually enter the convex hull formed by the leaders. The output containment errors of follows shown in Fig 8. It can be seen that the local output errors $e_i(t)$ stay within a boundary.


% \begin{figure}[!htbp]
% %\begin{minipage}[t]{1\linewidth}
% \centering
% \includegraphics[width=0.6\textwidth]{4Ag.pdf}
% \caption{Time-varying directed communication topology among all agents}
% \label{fig:figure1}
% \end{figure}



%{\color{blue}
%\begin{figure}[htbp]
%\centering
%\subfigure[Performance of observer w.r.t. the leader]{
%\begin{minipage}[t]{0.475\textwidth}
%\centering
%\includegraphics[width=0.85\textwidth]{pic/pobs.eps}
%%\caption{fig1}
%\end{minipage}\label{fig:figure2:1}
%}
%%\hspace{-0.1in}
%\subfigure[Performance of observer w.r.t. the first leader]{
%\begin{minipage}[t]{0.475\textwidth}
%\centering
%\includegraphics[width=0.85\textwidth]{pic/vobs.eps}
%%\caption{fig2}
%\end{minipage}\label{fig:figure2:2}
%}\\%
%\centering
%\caption{Performance of two observers}
%\label{fig:figure2}
%\end{figure}












\section{Conclusion}
We investigate the distributed prescribed-time consensus observer for multi-agent systems with
high-order integrator dynamics and directed topologies in this brief.  To our best knowledge, the DPTO on time-invariant/varying digraphs with prescribed-time zero-error converge has been formulated for the first time, which could achieve distributed estimation w.r.t. the leader state within an arbitrary time interval predefined by the users. An illustrative simulation example has been conducted, which confirms the prescribed-time performance of the above DPTO. Future works will consider using this prescribed-time observer to deal with distributed fault-tolerant control problems \cite{hua2017distributed, xu2020distributed}. %\cite{3xiao2021distributed} 

\section*{Appendix}

 

\

% Proof: Consider the Lyapunov function candidate
% $$
% V_{1}=\frac{1}{2} \sum_{i=1}^{N} \xi_{i}^{T} P \xi_{i}+\sum_{i=1}^{N} \sum_{j=1, j \neq i}^{N} \frac{\left(c_{i j}-\alpha\right)^{2}}{8 \kappa_{i j}}
% $$
% where $\alpha$ is a positive constant that is to be determined later. Evidently, $V_{1}$ is positive definite. The time derivative of $V_{1}$ along the trajectory of (5) is given by
% $$
% \begin{aligned}
% \dot{V}_{1}=& \sum_{i=1}^{N} \xi_{i}^{T} P \dot{\xi}_{i}+\sum_{i=1}^{N} \sum_{j=1, j \neq i}^{N} \frac{c_{i j}-\alpha}{4 \kappa_{i j}} \dot{c}_{i j} \\
% =& \sum_{i=1}^{N} \xi_{i}^{T} P A \xi_{i}+\sum_{i=1}^{N} \xi_{i}^{T} P B K \sum_{j=1}^{N} c_{i j} a_{i j}\left(\tilde{x}_{i}-\tilde{x}_{j}\right) \\
% &+\sum_{i=1}^{N} \sum_{j=1, j \neq i}^{N} \frac{c_{i j}-\alpha}{4 \kappa_{i j}} \dot{c}_{i j}
% \end{aligned}
% $$
% Since $a_{i j}=a_{j i}$ and $c_{i j}(t)=c_{j i}(t)$, it can be easily verified that
% $$
% \begin{array}{rl}
% \sum_{i=1}^{N} \xi_{i}^{T} & P B K \sum_{j=1}^{N} c_{i j} a_{i j}\left(\tilde{x}_{i}-\tilde{x}_{j}\right) \\
% & =-\frac{1}{2} \sum_{i=1}^{N} \sum_{j=1}^{N} c_{i j} a_{i j}\left(\xi_{i}-\xi_{j}\right)^{T} \Gamma\left(\tilde{x}_{i}-\tilde{x}_{j}\right)
% \end{array}
% $$

 \bibliography{PIDFR}
 
 
 
 
 


 
  \end{document}\grid
