\documentclass[../main.tex]{subfiles}

\begin{document}

\textit{A. K. Shaw, N. N. Patra, A. Chakraborty, R. Mondal, S. Choudhuri, A. Mazumder, M. Jagannath}\newline
\begin{figure}
    \centering
    \includegraphics[trim={0cm 0.5cm 5.cm 0cm},clip,width=0.8\columnwidth]{images/Cubelet_schematic3.pdf}
    \caption{Team Spardha: The 2D projection along one axis of the schematic division of the data into {\it Normal} and {\it Overlapping} cubelets (top row), and the corresponding Acceptance regions (black hashing; bottom row). Normal cubelets are illustrated by black outlined boxes. Overlapping cubelets are centred at the common boundaries of Normal cubelets and are illustrated by orange boxes. Grey regions (top row) represent buffer zones. }
    \label{fig:spardha}
\end{figure}

\noindent The SPARDHA team developed a {\sc python}-based pipeline which starts by dividing the 1\,TB Challenge dataset into several small cubelets. We performed soource finding using an MPI-based implementation to run parallel instances of {\sc SoFiA}-$2$ on each cubelet. We tuned the parameters of {\sc SoFiA}-$2$ to maximize the number of detected sources. A total of $118$ cubelets were analysed, which were categorised into two groups, namely: 1) Normal cubelets and 2) Overlapping cubelets. The whole datacube was first divided into consecutive blocks of equal dimensions to create Normal cubelets (Fig.~\ref{fig:spardha}). Overlapping cubelets were then centred at the common boundaries of Normal cubelets in order to detect sources that fall at their common boundaries.


In order to avoid source duplication, buffer regions were defined around the faces of each cubelet (see (Fig.~\ref{fig:spardha}, top row).  We always accepted any source whose centre was detected within the cubelet but not in the buffer zone (see Fig.~\ref{fig:spardha}, bottom row).  We conservatively set the width of buffer zones based on the physically motivated values of the spatial and frequency extent of typical galaxies scaled at the desired redshifts. We chose the maximum extent of the galaxy on the sky plane to be $\sim$80~kpc \citep{2016MNRAS.460.2143W}, corresponding to $\sim$10 pixels in the nearest frequency channel. The buffer region was set to be twice this extent, \textit{i.e.} $20$ pixels. Overlapping regions were therefore $4\times 20=80$ pixels wide. Along the frequency direction, galaxies can have a line-width extent of $\sim$500~km/s, which corresponds to  $\sim$72 channels. The widths of the buffer regions and Overlapping regions along the frequency axis were therefore $144$ and 288 channels, respectively. The acceptance regions of the cubelets (normal and overlapping) were such that they spanned the whole data cube contiguously when arranged accordingly. Although this approach increased the computation slightly due to analysing some regions of the data twice, it ensured that there was no common source present in the list. Analysing cubelets was the most time consuming part in our pipeline. We analysed $118$ cubelets on $472$ cores in parallel in around $15$ minutes.


We used physical equations to convert the {\sc SoFiA}-$2$ catalogue into the SDC-prescribed units and to discard bad detections such as those sources having \textit{NaN} values in the columns or those with negative flux values. In the final stage we put limits on the line width, discarding detections with unusual values. Motivated by physical models and observations of galaxies, we conservatively accepted the sources having $w_{20}\in [60,\, 500]~{\rm km/s}$ \citep{2000ApJ...533L..99M}. We finally arranged the catalogue in descending order of the flux values. Based on tests using the development datacube, for which the exact source properties are known, we chose the top $35\%$ of total sources to generate the final catalogue for submission.


\end{document}