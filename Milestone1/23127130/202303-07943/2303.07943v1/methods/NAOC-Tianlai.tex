\documentclass[../main.tex]{subfiles}

\begin{document}

\textit{K. Yu, Q. Guo, W. Pei, Y. Liu, Y. Wang, X. Chen, X. Zhang, S. Ni, J. Zhang, L. Gao, M. Zhao, L. Zhang, H. Zhang, X. Wang, J. Ding, S. Zuo, Y. Mao}\newline


\noindent After testing several methods, the NAOC-Tianlai team used the {\sc SoFiA-2} software to process of the SDC2 datasets. We optimised the  {\sc SoFiA-2} input parameters by first performing a grid search in parameter space before refining the result using an MCMC simulation. We are currently developing a dedicated cosmological simulation on which to test our methods. However, during the Challenge time frame we mainly used the development and large development datasets to perform the optimisation. The optimised parameters were then used for the processing of the full Challenge dataset.

Due to the memory constraints and the consideration of avoiding excessive division along the spectral axis, the datasets were split into subcubes of size  $\sim$$330\times330\times3340$ pixels for processing. Adjacent subcubes had an overlap of 10 or 20 pixels along each axis to ensure that H{\sc i} galaxies on the border region were not missed. The full Challenge dataset was therefore divided into $18\times18\times2$ subcubes when processing.

Our main parameter selection procedure is as follows:
\begin{enumerate}
\item\label{step1} We set a list of values to be searched for each parameter of interest, such as: \texttt{replacement}, \texttt{threshold} in the {\it scfind} module; \texttt{minSizeZ}, \texttt{radiusZ} in the {\it linker} module; and \texttt{minSNR}, \texttt{threshold}, \texttt{scaleKernel} in the{\it reliability} module. We then processed in parallel the development dataset with the different combinations of parameters values.

\item Next, we selected the optimal parameter combination by comparing the output catalogues from the previous step with the  development dataset truth catalogue. To choose the optimal parameters, thresholds were applied to  the {\it total detection number}, to the {\it match rate} (true detection/total detection), and to the final {\it score}.

\item To make the found optimal parameter combination more robust, different subcubes were processed following the procedure given above, and the combination that performed well on all subcubes was selected. 

\end{enumerate}


For reference, our trial produced the following optimised parameter settings: {\tt scaleNoise.windowXY/Z = 55} for normalising the noise across the whole datacube; {\tt kernelsXY = [0, 3, 7]}, {\tt kernelsZ = [0, 3, 7, 15, 21, 45]}, {\tt threshold = 4.0}, {\tt replacement = 1.0} in the {\it scfind} module for the S+C finder in {\sc SoFiA-2}; {\tt radiusXY/Z = 2}, {\tt minSizeXY = 5}, {\tt minSizeZ = 20} in the {\it linker} module for merging the masked pixels detected by the finder; and {\tt threshold = 0.5}, {\tt scaleKernel = 0.3}, {\tt minSNR = 2.0} in the {\it reliability} module for reliability calculation and filtering. In our processing, each parameter combination  instance took $\sim$5 minutes with one CPU thread to process one subcube.

Finally, we applied the optimal parameter combination to the processing of all subcubes from the Challenge dataset, and merged the results. 



\end{document}