\section{Cable Fabrication Process and Research Approach}
\label{sec:photolith}
\begin{figure*}[t]
    \centering
    \includegraphics[width=\textwidth]{figures/Fab_Process_2.pdf}
    \caption{Schematic showing the typical fabrication steps for a two-layer flexible cable. The figures associated with each step show an example of what a simplified cable with a single trace and via looks like after the step is completed. Icons (with legend on the far right) indicate possible vectors of contamination at each step.}
    \label{fig:fab_process}
\end{figure*}

All cable production work was spearheaded by Q-Flex Inc. who specialize in the fabrication of custom long flexible cables. The production of flexible cables involves a number of fabrication steps that start from a simple laminate of copper and polyimide and then adds vias, patterns of traces and interconnections, insulating coverlays, and metallized pads for external connections. The typical commercial photolitohographic process for producing a two-layer flexible cable is shown in Figure~\ref{fig:fab_process}. As part of the fabrication process, the starting laminate is subjected to various chemicals, addition of additional copper and polyimide layers, and mechanical manipulation (drilling, cutting, lamination, etc.). Each one of these steps is a potential source of radioactive contamination. 

To investigate the contamination levels at each step of the fabrication process, we produced panels of small detachable “coupons” (Figure \ref{fig:panel}). The coupons act as surrogates for real cables – they contain all the relevant features such as traces, vias, coverlays, and go through the same production steps as regular cables. The advantage of this approach is that individual coupons can be easily detached after each step in the process and measured for radioactive contaminants. By comparing the contamination levels in each successive coupon one can assess the level of contamination imparted from each step towards production of the final cable. Where a spray or chemical bath was used for a particular production step, aliquots of the chemicals were also collected for assay along with the coupon for that step. 

\begin{figure}
\centering
 \includegraphics[height=0.8\columnwidth, angle=90]{figures/Panel.png}
\caption{An example of a panel of detachable coupons with typical cable features, utilized to study the contamination levels at each step of the fabrication process.}
\label{fig:panel}
\end{figure}

Inductively coupled plasma mass spectrometry (ICP-MS) was chosen as the analysis method for measuring the radioactive contamination in each coupon. ICP-MS allows for ultrasensitive assays of materials, processes, and reagents, and the ICP-MS capability at Pacific Northwest National Laboratory (PNNL) for supporting ultralow background physics has developed a variety of methods to reach sub-ppt sensitivities for \ur~,  \th, and $^{nat}$K in a variety of materials \cite{arnquist2017mass,arnquist2020ultra,arnquist2020automated,arnquist2017quick}. ICP-MS also has the distinct advantage of being able to investigate small samples ($<$ 1 gram) with sufficient sensitivity and having quick turnarounds of sample batches, allowing for detailed, systematic investigations of a large number of samples. A detailed description of the ICP-MS capability and method  is given later in this paper. 
Because ICP-MS testing is destructive, the technique cannot be used to measure the actual final cables that are deployed. By adding these small detachable surrogate coupons to the same panel as the cable, we can sample the contamination during the production of the final cables – enabling verification that the cables meet radiopurity requirements. 

ICP-MS directly detects the atoms (ions) within a sample and is therefore most sensitive to those radioisotopes with long half-lives (e.g., \ur~ and \th). In order to measure contamination levels of isotopes further down in the chain, one could use gamma counting of the radioactive decays. However, the sensitivity is typically limited to specific activities at the level of \si{\milli\becquerel\per\kg} (equivalent to 81 ppt \ur, 246 ppt \th), and it requires kilograms of material and counting times on the order of a month. It should be noted that comparison of ICP-MS measurements of the \ur~levels and gamma counting of $^{226}$Ra for cables used in the DAMIC experiment indicated secular disequilibrium, i.e., the activity of $^{226}$Ra in the lower half of the chain was found to be significantly lower than the \ur~activity at the top of the chain \cite{aguilar2022characterization}. We do intend to eventually measure our final cables through gamma counting but the limited sensitivity, large sample mass and long counting times needed, led to us not pursuing it at the R\&D stage.