\section{Reagents and Labware}
\label{sec:labware}
 Electronic grade tetramethylammonium hydroxide (TMAH), 100\% pure ethanol, and Optima grade nitric, hydrochloric, and sulfuric acids (Fisher Scientific, Pittsburg, PA, USA) were used for sample cleaning and preparation. 18.2 M$\Omega\cdot$cm water from a MilliQ system (Merk Millipore GmbH, Burlington, MA) was used for sample rinsing and in the preparation of reagent solutions. Ultralow background perfluoroalkoxy alkane (PFA) screw cap vials from Savillex (Eden Prairie, MN) were used as sample containers, to collect solutions from microwave digestion vessels and as ICP-MS autosampler vials. \\
All labware involved in sample handling and analysis (vials, microwave vessels, tongs, pipette tips) were cleaned with 2\% \vov~Micro-90 detergent\textsuperscript{\textregistered} detergent (Cole-Parmer, Vernon Hills, IL), triply rinsed with MilliQ water, and leached in Optima grade 25\% \vov~HCl and 6M HNO$_3$ solutions. Following leaching, all labware underwent validation to ensure cleanliness. The validation step consisted of pipetting a small volume of 5\% \vov~HNO$_3$ into each container: 1.5 mL in the PFA vials and 5mL in the iPrep$^{TM}$ microwave vessels. Vials were closed, shaken, and kept at $\SI{80}{\celsius}$ for at least 12 hours. Microwave vessels underwent a microwave digestion run at $\SI{220}{\celsius}$. Tongs and pipette tips were soaked into a 5\% \vov~HNO$_3$ leaching solution ($ca.$ 1.5 mL) for 5-10 minutes. The leachate from all labware was then analyzed via ICP-MS. The validation was performed to assure sufficiently low background for Th and U. Only labware for which Th and U signals were at reagent background levels passed validation. Labware failing validation underwent additional cycles of leaching and validation tests until background requirements were met.

%\begin{table*}[b]
%\centering
%\begin{tabular}{c c c c c c c c}
%\hline
%\multicolumn{2}{c}{Fabrication Step} & \multicolumn{2}{c}{Materials} & \multicolumn{2}{c}{Solutions} & \multicolumn{2}{c}{Coupons}\\
%& & \textbf{\th} & \textbf{\ur} & \textbf{\th} & \textbf{\ur} & \textbf{\th} & \textbf{\ur} \\
% & & \textbf{[pg/g]} & \textbf{[pg/g]} & \textbf{[pg/g]} & \textbf{[pg/g]} & \textbf{[pg/g]} & \textbf{[pg/g]}\\
%\hline
%1 & laminate* & \num{8 \pm 6} & \num{9 \pm 4} & & & &\\
%2 & cut and drill & & & & & \num{9.5 \pm 0.7} & \num{8 \pm 0}\\
%3 & seed & & & & & &\\
%3 & Cu plating & & & & & \num{6 \pm 6} & \num{137 \pm 5}\\
%4 & sanding & & & & & \num{240 \pm 110} & \num{190 \pm 20}\\
%5 & photoresist & \num{8 \pm 6} & \num{19 \pm 13} & & & &\\
%6 & develop & & & \num{5.65 \pm 0.09} & \num{2620 \pm 30} & \num{205 \pm 7} & \num{530 \pm 40}\\
%7 & etch & & & \num{83 \pm 2} & \num{2310 \pm 30} & \num{150 \pm 40} & \num{2600 \pm 600}\\
%8 & strip & & & \num{22.4 \pm 0.6} & \num{1510 \pm 50} & \num{120 \pm 50} & \num{4350 \pm 70}\\
%10 & dry & & & & & &\\
%9 & coverlay** & \num{400 \pm 40} & \num{4500 \pm 500} & & & \num{705 \pm 7} & \num{7390 \pm 14}\\
%10 & microetch & & & \num{1090 \pm 60} & \num{1060 \pm 40} & \num{730 \pm 0} & \num{7400 \pm 300}\\
%11 & ENIG & & & & & \num{750 \pm 40} & \num{7300 \pm 200}\\
%\hline 
%\end{tabular}
%\caption{\th~and \ur~concentrations at each cable fabrication step for raw materials, photolithography solutions, and representative coupons. *starting laminate from PNNL testing coupon;
%**coverlay from Taiflex\\ \todo{SWITCH Th and U ORDER}} 

%\label{tab:init_process_contamination}
%\end{table*}

\section{Contamination by Fabrication Step}
\begin{table*}[t]
\centering
\begin{tabular}{c l c c c c c c}
\hline
\multicolumn{2}{c}{Fabrication Step} & \multicolumn{2}{c}{Materials} & \multicolumn{2}{c}{Solutions} & \multicolumn{2}{c}{Coupons}\\
& & \textbf{\ur} & \textbf{\th} & \textbf{\ur} & \textbf{\th} & \textbf{\ur} & \textbf{\th} \\
 & & \textbf{[pg/g]} & \textbf{[pg/g]} & \textbf{[pg/g]} & \textbf{[pg/g]} & \textbf{[pg/g]} & \textbf{[pg/g]}\\
\hline
1 & Laminate* & \num{9 \pm 4} & \num{8 \pm 6} & & & &\\
2 & Cut and Drill & & & & & \num{<16} & \num{<20}\\
%3 & seed & & & & & &\\
3 & Copper Plating & & & & & \num{140 \pm 10} & \num{<20}\\
4 & Sanding & & & & & \num{190 \pm 20} & \num{240 \pm 110}\\
5 & Developing & \num{19 \pm 13}$^{\dagger}$ & \num{8 \pm 6}$^{\dagger}$ & \num{2630 \pm 110} & \num{6 \pm 2} & \num{530 \pm 40} & \num{210 \pm 10}\\
6 & Etching & & & \num{2300 \pm 200} & \num{80 \pm 10} & \num{2600 \pm 600} & \num{150 \pm 40}\\
7 & Stripping & & & \num{1510 \pm 70} & \num{22 \pm 2} & \num{4400 \pm 500} & \num{120 \pm 30}\\
%10 & dry & & & & & &\\
8 & Coverlay* & \num{18000 \pm 2000} & \num{1600 \pm 140} & & & \num{7400 \pm 200} & \num{710 \pm 20}\\
9 & Microetching & & & \num{1060 \pm 60} & \num{1090 \pm 60} & \num{7400 \pm 400} & \num{730 \pm 50}\\
10 & ENIG Processing & & & & & \num{7300 \pm 200} & \num{750 \pm 40}\\
\hline 
\end{tabular}
\caption{\ur~and \th~concentrations at each cable fabrication step for raw materials, photolithography solutions, and representative coupons. Measurements of the materials and solutions are normalized to their individual component masses, while the values listed for the coupons are normalized to the mass of the coupon, including all layers present, at that step. *Laminate and coverlay from Taiflex (2FPDR2005JA and FHK1025, respectively). $^{\dagger}$Photoresist from DuPont (Riston MM500)}

\label{tab:init_process_contamination}
\end{table*}




