\section{Contaminating Steps}
\label{sec:contamination}

\begin{figure*}
    \centering
    \includegraphics[width=0.8\textwidth]{figures/QFlex_Fab_Step_Init.pdf}
    \caption{Results of the systematic ICP-MS analysis of coupons, materials, and solutions involved in each step of the fabrication process. The starting laminate used here was Taiflex 2FPDR2005JA, the photoresist was DuPont Riston MM500, and the coverlay was Taiflex FHK1025. For values and uncertainties refer to Table~\ref{tab:init_process_contamination}.}
    \label{fig:init_process_contamination}
\end{figure*}

The measured contamination of \ur~and \th~in coupons at each step of the fabrication process, along with measurements of any materials and solutions involved in those steps, is shown graphically in Figure~\ref{fig:init_process_contamination} with the detailed results and uncertainties presented in Table~\ref{tab:init_process_contamination}. As can be seen, despite starting the process with clean material (few ppt \ur~and \th), the final cable has contamination levels of $\sim$ 7000 ppt \ur~ and $\sim$ 700 ppt \th. These values are similar to the previous findings of the EXO-200 \cite{pocar_slides} and EDELWEISS \cite{zhang2015novel} collaborations. By studying the evolution of the cumulative contamination in the coupons after each step, we were able to target specific production steps that were the dominant source of contamination. In the following subsections we describe each of the key contaminating steps (not necessarily in order) and the approach used to reduce the contamination introduced during that step.

\subsection{Starting Laminate}
The starting material for the production of flexible cables is a laminate, a foil consisting of an insulating flexible polymer, typically polyimide, layered with a conductive metal, usually copper. Since this material forms the bulk of the mass of the finished cable, it is critical that it be as low in radioactivity as possible.

In previous work \cite{arnquist2020ultra}, we have demonstrated the possibility to obtain low-background laminates with single-digit ppt levels of $^{232}$Th and $^{238}$U impurities ($\sim$\si{\micro\becquerel\per\kilo\gram} range). As shown in Table~\ref{tab:commercial_laminates}, this is significantly cleaner than other laminates we surveyed as part of that study as well as laminates measured by other low background experiments. The specialized Kapton laminate produced by the DuPont R\&D division \cite{arnquist2020ultra} was only available in short individual sheets, limiting the ability to make long cables. Measurements conducted prior to the rest of the work described in this paper, found that a copper-polyimide laminate from Taiflex \cite{taiflex} has similar radiopurity to that of the Dupont R\&D material (see Table~\ref{tab:commercial_laminates}). The Taiflex material is two-sided copper laminated to a central polyimide layer with a thermoplastic polyimide. The central polyimide layer is \SI{50}{\micro \meter} and each copper layer is \SI{18}{\micro \meter} thick. This commercial material is available in rolls, making it suitable for manufacturing long cables, as is often needed in low background experiments. All samples made for testing subsequent steps in the process were made starting from this Taiflex laminate.

\begin{table*}[t]
\centering
\begin{tabular}{l c c c c c c}
\hline
\textbf{Type} & \textbf{Vendor} & \textbf{Polyimide} & \textbf{Copper }  & \textbf{\ur} & \textbf{\th} & \textbf{\knat} \\ %PNNL ID and nEXO #
 & & \textbf{Thickness} & \textbf{Thickness} & & & \\
  & & \textbf{[\si{\micro\meter}]} & \textbf{[\si{\micro\meter}]} & \textbf{[\si{\ppt}]} & \textbf{[\si{\ppt}]} & \textbf{[\si{\ppb}]} \\
\hline
Taiflex 2FPDR2005JA  & Taiflex & 50.8 & 18 (x2) & \num{8 \pm 6} & \num{9 \pm 4} & $<$ 100 \\
300ELJ+Cu laminate \cite{arnquist2020ultra} & \dupont~R\&D & 76.2 & 18 (x2) & \num{8.6 \pm 3.6} & \num{20 \pm 14} & \num{164 \pm 82} \\
Pyralux AP8535R \cite{arnquist2020ultra} & \dupont & 76.2 & 18 (x2) & \num{158.0 \pm 6.1} & \num{24.1 \pm 0.9} & $<$ 210\\ %2019-10  including all 6 reps
Novaclad 146319-009 \cite{arnquist2020ultra} & Sheldahl & 50.8 & 5.0 (x1) & \num{283 \pm 21} & \num{50.1 \pm 3.9} & $<$ 210\\
Cirlex \cite{arnquist2020ultra} & Fralock & 228.6 & 34 (x2) & \num{413 \pm 45} & \num{71.4 \pm 2.1} & $<$ 210\\
Espanex MC15-40-00VEG \cite{leonard2008systematic} & Nippon Steel & 40 & 15 (x1) & \num{121 \pm 32} & $< 250$ & \num{880 \pm 120}\\
Espanex MC18-25-00CEM \cite{leonard2008systematic} & Nippon Steel & 25 & 18 (x1) & $< 46$ & $< 260$ & $< 146$\\
Espanex SB \cite{zhang2015novel} &  & 25 & 18 (x2) & $< 284$ & $< 812$ & $< 1777$ \\
\hline
\hline
\end{tabular}
\caption{Comparison of the radiopurity of copper-polyimide laminates. The (xN) factor  in the fourth row indicates whether the copper was on one or both sides of the laminate.}
\label{tab:commercial_laminates}
\end{table*}

\subsection{Photolithography and Cleaning}\label{sec:cleaning}

Photolithography \cite{lithography} is the patterning of electrical traces and other features on the copper surface of the laminate. A photo-sensitive material (photoresist), in our case DuPont Riston MM500, is applied to the laminate to mask sections of the copper and then hardened (developed) by exposure to ultraviolet light. The exposed copper is then removed (etched) and finally the resist is removed (stripped) and dried. This process involves immersing the coupons in a sequence of different chemical solutions to perform these steps. As can be seen in Figure~\ref{fig:init_process_contamination}, the photolithography steps (``Developing'', ``Etching'', and ``Stripping'') were found to be a major source of contamination with 1000's of ppt \ur~ and 100's of ppt \th~measured after the stripping step. We assayed the chemical solutions used in the developing, etching and stripping steps: anhydrous  Na$_2$CO$_3$ diluted to 1$\%$, 0.5 M HCl, and %$Rstrip35$ 
a KOH based solution diluted to 20$\%$, respectively, and found significant levels of contamination (shown in Figure~\ref{fig:init_process_contamination}) indicating that the solutions were the likely source of contamination. The chemicals used in this process have been carefully chosen and balanced for cable production and changing the chemicals (e.g. purifying or replacing with alternatives) was determined to require significant amount of R\&D and to likely be impractically expensive. Instead, we pursued the development of a post-cleaning method to try and reduce the contamination after lithography.

The lithogragraphic process for flexible cable manufacturing is almost entirely a "subtractive" procedure: the electrical traces and connections are produced through removal of copper from the outer layers of the starting copper-polyimide laminate with no additional material added (the photoresist is removed at the end) - see Figure~\ref{fig:fab_process}. With this in mind, it was believed that contaminant species from the processing should be at or near the surface and be removable through well-tailored post-processing cleaning. 

Several different cleaning methods were tested. A partially manufactured test cable, processed up to and including the stripping step in Figure~\ref{fig:init_process_contamination} but skipping the "Copper plating" step, was used for this dedicated study. This test cable had a contamination level of $\sim$ 4000 ppt \ur~ and $\sim$ 100 ppt \th. The cleaning consisted of submerging subsamples of the test coupon in the chosen cleaning solution for a fixed length of time while agitating, followed by rinsing in deionized water and air drying in a class 10 laminar flowhood.  Subsamples of the same size as those used for ICP-MS assay ($\sim$~50 mg, $\sim$~1 cm$^2$ surface) were cleaned and assayed before and after cleaning to assess the efficacy of the method. 

Initial attempts tested combinations of various diluted acid solutions (e.g., nitric acid, sulfuric acid) and durations (30 sec - 15 min) to determine the optimal procedure. Although some of these cleaning methods provided a very significant reduction of contaminants in the cable ($\sim$~75x and $\sim$~2x in \ur~and ~\th~respectively) the residual contamination was still above our target radiopurity levels.

Surveying the literature, we learnt that treatments of polyimide films with alkaline solutions, e.g., ethylenediamine,  resulted in modifications to the chemical structure of the polyimide surface, increasing adhesion \cite{park2012surface}. Based on this finding, a cleaning procedure involving the use of an alkaline solution was developed, targeting a more efficient removal of surface contaminants, in particular from the polyimide surface. The cleaning consisted of submerging the samples in 5$\%$ v/v tetramethylammonium hydroxide (TMAH) for 15 min, rinsing in deionized water (DIW), submerging the samples in a 2$\%$ nitric acid solution for $\sim$~1 min, followed by rinsing in DIW and air drying in a class 10 laminar flow hood. A nylon brush with soft polyester bristles was used for scrubbing the samples while they were submerged in each cleaning solution. The duration of the submersion in nitric acid was set based on previous test results, while the duration of submersion in TMAH was set based on a study of cleaning efficacy and visual inspection of the cables after treatment. The efficiency of the cleaning improved with longer duration, but long exposures to TMAH eventually resulted in visible modification of the copper surface. It was found that repeating the entire cleaning cycle thrice was more effective at reducing the contamination than a single cycle. This cleaning recipe reduced the $^{238}$U and $^{232}$Th contamination in the test coupon to $\sim$20 and $\sim$ 10 ppt, respectively. 

Optical inspection of the cable (at 100, 200, and 500x magnification) after cleaning did not show any visible structural changes of the copper traces. For the cleaning of the final cables, described later in Section~\ref{sec:final_cables}, electrical tests were performed after cleaning to ensure the cable still functioned electrically.

\subsection{Copper plating}
\begin{figure*}
    \centering
    \includegraphics[width=\textwidth]{figures/Copper_plating.pdf}
    \caption{Comparison of standard electroless copper plating process (left) with the shadow plating process (right). In both processes the original laminate of copper (pink) and polyimide (yellow) has the vias drilled out. Left:  A thin copper seed layer (green) is chemically attached to the entire laminate surface. Right: A thin carbon layer is attached to only the polyimide layer, making it conductive. After these steps, in both processes a thicker copper layer (brown) is electroplated on to all conductive surfaces. The tables alongside each step show the measured \ur~and \th~contamination at that step, before and after the cleaning is applied.}
    \label{fig:electroplating}
\end{figure*}
For cables that require interconnections between the conductive copper layers, vias are used. A hole is drilled and then plated with copper to form the electrical connection between the outer two copper layers, across the polyimide insulation (Copper plating step in Figure \ref{fig:init_process_contamination}). Because copper cannot be directly electroplated onto polyimide, a first “seeding” step is used to chemically attach a thin conductive layer to the entire laminate surface. Once the conductive seed layer is added, copper is deposited through electroplating (see the left side of Figure~\ref{fig:electroplating} for a schematic overview). 

Despite several attempts, it was found that the surface cleaning methods we developed (described in Section~\ref{sec:cleaning}) were not effective in removing the contamination introduced during electroplating - see the tables on the left side of Figure~\ref{fig:electroplating} for the results of the measurements before and after cleaning. The electroless seeding process is not selective and deposits the conductive layer over the entire surface of the cable. The subsequent copper electroplating then also deposits a layer of copper over the entire cable. Thus contamination introduced in these steps can be trapped between the base laminate, seed, and electroplating layers, preventing removal through surface cleaning. 
 
To investigate whether the contamination was being trapped between the seed and electroplated layers, we made coupons using a ``pad plating" process \cite{allflex_plating}. In this method, after the seed layer is applied, photoresist is applied to mask the areas of the cable that do not need an additional copper layer (i.e. everything except the vias). The electroplating then only deposits copper in the via regions and the photoresist is removed after the plating. This is a standard industrial process used to avoid the additional plated copper from covering the entire surface area and stiffening the cable. A systematic analysis of coupons after each step of the process, found similar contamination levels with and without pad plating, indicating that the dominant contamination was not trapped between the seed and electroplated layers, but likely between the base laminate and seed layer. 
 
The standard electroless seeding process involves several chemical treatment steps and typically requires a catalyst \cite{ghosh2019electroless}. A newer alternative process to the electroless Cu step is the “shadow” process \cite{allflex_vias, macdermid_shadow}. In this approach, a thin layer of carbon is laid down on the cable. An etch is then performed so that carbon only adheres to the polyimide and so does not cover the copper surface \cite{macdermid_carbon} (see the right side of Figure~\ref{fig:electroplating} for a schematic overview). Unlike the standard electroless seed process, the shadow process uses fewer chemicals and also deposits less material over much smaller area than the standard approach. The shadow process therefore has the potential to be intrinsically less dirty than the standard electroless seed method. 

We investigated the contamination levels in coupons before and after the shadow process as well as after the subsequent electroplating (see tables on right side of Figure~\ref{fig:electroplating}). We found significant contamination after both seeding and plating steps, however, unlike in the case of the electroless seeding, the contamination could be removed by our cleaning method. In fact, the resulting coupons had contamination levels consistent with the base laminate level, roughly 9 times cleaner in \ur~than the electroless seed process. Given these promising results, we elected to use the shadow plating process for copper plating vias in our final cables.

\subsection{Sanding}
 Prior to the application of photoresist, the copper surface of the cable is prepared for optimal dry film adhesion and subsequent clean release. Surface preparation processes typically aim to remove any surface imperfections after drilling and electroplating and roughen the surface to increase film contact area. At Q-Flex, this is done by mechanical scrubbing of the surface with abrasive pads. The scrubbing process was found to increase $^{232}$Th contamination, presumably due to the implantation of small amounts of the abrasive material into the laminate surface. With this in mind, we worked to remove the abrasive material from the surface by sonication in 18.2 M$\Omega\cdot$cm water. The results before and after sonication were within error of each other, indicating that this method did not remove the contamination. The acid-base cleaning method described in Section~\ref{sec:cleaning} was also applied at this step but did not significantly reduce the contamination.
 
 Since our attempts to remove any embedded material from the surface through sonication and cleaning proved ineffective, we reviewed the available commercial options for scouring pads. We decided to exclusively use commercial Scotch Brite$^{TM}$ pads made from SiC \cite{scotch_brite_sic}, known to generally be a radiopure material, rather than previously used pads that used aluminum oxide, titanium dioxide, and other fillers and pigments \cite{scotch_brite_al}. The change in abrasive material led to roughly a 10x reduction in $^{232}$Th at this step.

\subsection{Coverlay}
\begin{table*}[t]
    \centering
    \begin{tabular}{lcccccc}
        \hline
         Sample & PI Thick. & Adh. Thick. & Notes & \ur & \th & \knat\\
         & \textbf{[mil]} & \textbf{[mil]} & & \textbf{[\si{\ppt}]} & \textbf{[\si{\ppt}]} & \textbf{[\si{\ppb}]}\\
         \hline
         Taiflex FHK1025 & 1 & 1 & \multirow{3}{*}{Use epoxy adhesive} & \num{18000 \pm 2000} & \num{1600 \pm 140} & \\
         ShinEtsu CA 333 \cite{leonard2017trace} & 1 & 1 &  & \num{5179 \pm 424} & $< 242$  & \\
         ShinEtsu CA 335 \cite{leonard2017trace} & 1 & 1.4 &  & \num{12020 \pm 390} & \num{9370 \pm 340 }  & \\
         \hline
         Dupont LF0110 & 1 & 1 & \multirow{5}{*}{Use acrylic adhesive} & \num{314 \pm 13} & \num{49 \pm 8} & \num{4000 \pm 2000}\\
         Upilex C120 & 2 & 1 &  & \num{30 \pm 2} & \num{280 \pm 20} & \num{21300 \pm 300} \\
         Panasonic MCL Plus 110 & 1 & 1 & & \num{78 \pm 4} & \num{45 \pm 7} & \num{5030 \pm 140}\\
         Dupont FR 70001 \cite{leonard2017trace} & 0.5 & 0.5 &  & $< 1065$ & $< 473$  & \\
         Dupont FR 0110 \cite{leonard2017trace} & 1 & 1 &  & $< 818$ & $< 273$  & \\
         \hline
         Dupont LF0100 & 0 & 1 & Adhesive in LF0110 & \num{16 \pm 4} & \num{39 \pm 11} & \\
         Imitex MI-100 & 0 & 1 & Adhesive & \num{9 \pm 5} & $<$ \num{14} & \\
         \hline
         \hline
    \end{tabular}
    \caption{Radiopurity results from survey of commercially available coverlay materials, including results from previous measurements in the literature \cite{leonard2017trace}. The second column indicates the polyimide (PI) thickness and the third column the adhesive thickness. The materials in the last two rows are adhesives only.}
    \label{tab:coverlays}
\end{table*}

A coverlay is an insulating layer that is applied over the outer surfaces of a cable to prevent oxidation and shorting of the exposed copper traces. Commercial coverlays are typically composed of two materials — the polyimide that produces the electrical insulation and an adhesive that bonds the polyimide to the surface of the cable. In addition to reviewing previous measurements of coverlays in the literature \cite{leonard2017trace}, we surveyed several different commercially available coverlays, as shown in Table~\ref{tab:coverlays}. 

As can be seen from the results, the contamination levels of \ur~and \th~can vary by several orders of magnitude amongst commercial coverlays. Although Taiflex copper-polyimide laminates were found to be extremely radiopure, the epoxy-based adhesives they utilize have extremely high levels of radiocontaminants. Our measurements indicate that coverlays based on acrylic adhesives have the lowest \ur~and \th~contamination. We also measured the \knat~contamination levels in these acrylic coverlays and found high contamination levels ($>$ ppm) with significant variations (see final column of Table~\ref{tab:coverlays}). Overall, the Panasonic MCL coverlay was the best choice, with contaminations of \ur~and \th~lower by factors of roughly 200 and 30, respectively, compared to the Taiflex coverlay. Upilex C120 should also be considered for experiments that are significantly more sensitive to backgrounds from \ur~than \th~or~$^{40}$K.  
We also separately measured acrylic adhesives and found them to be cleaner than the combined acrylic + polyimide coverlays (last two rows of Table~\ref{tab:coverlays}). For future work we intend to investigate coverlays made from custom lamination of radiopure acrylic and polyimide as well as all-acrylic coverlays.