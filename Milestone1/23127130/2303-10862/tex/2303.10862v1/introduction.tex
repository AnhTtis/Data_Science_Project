\section {Introduction} 
\label{sec:intro}
Cabling associated with signal sensors and readout electronics are often a significant contributor to the radioactive background budget of physics experiments looking for rare events, such as searches for neutrinoless double beta decay or the direct detection of dark matter \cite{adhikari2021nexo, aguilar2022oscura, leonard2017trace, aguilar2022characterization, adams2021sensitivity, agnese2017projected, armengaud2017performance, abgrall2021legend}. Circuitry and cables are composed of two major components - the conductor (typically copper) and insulator. Polyimides are widely used as an insulating substrate in the electronics industry due to their unique properties of high resistivity, high dielectric strength, and flexibility. Polyimides are also stable across a wide range of temperatures, have good thermal conductivity, a thermal expansion coefficient that is close to copper, and a low outgassing rate, which make them a favorable material for use in the ultra-high vacuum and cryogenic environments that are commonly found in low background experiments. 

However, commercial flexible printed cables composed of laminates of copper and polyimide layers are not very radiopure, with typical contamination levels of the primordial radioactive nuclides $^{238}$U and $^{232}$Th measured at a few thousand and a few hundred parts-per-trillion by mass (ppt or \si{\pico\gram\per\gram}), respectively. These values often exceed the stringent radioactivity requirements of rare event searches and experiments therefore have to either limit the amount of cabling used to the absolute minimum necessary (as in nEXO \cite{kharusi2018nexo}, OSCURA \cite{aguilar2022oscura}), or use other materials \cite{busch2018low, andreotti2009low} that, while more radiopure, do not have all the advantageous properties of polyimide-based flexible cables, such as ease of clean and reliable installation. Additionally, since polyimide-based cables are an industry standard for electronics, the use of alternative material often requires custom-made components, which can increase cost, risk, and production time.

Efforts have previously been made to reduce the contamination in copper-polyimide cables used for rare event searches. The EXO-200 collaboration invested significant effort in selecting amongst different commercial copper-polyimide laminates, measuring contamination levels during the photolithography process, investigating alternative chemicals, as well as working with vendors to reduce contamination during handling and packaging \cite{pocar_slides, auger2012exo}. The EDELWEISS collaboration also surveyed different commercial laminates and adhesives to identify low radioactivity materials and used a custom fabrication process \cite{zhang2015novel, armengaud2017performance}. Even with these efforts, the final cables used in the experiment contained \ur~and \th~contamination at the level of hundreds to thousands of ppt. 

Some of the authors of this paper previously worked to identify clean copper-polyimide laminates, in collaboration with \dupont, leading to the production of custom material that had \ur~contamination levels $< 10$ ppt, roughly $10\times$ cleaner than commercial options \cite{arnquist2020ultra}. However, as described in this paper, even when starting with radiopure laminates the fabrication of the cables introduces large amounts of contamination, leading to roughly the same levels found in commercial cables, i.e., a few thousand ppt \ur~and a few hundred ppt \th.

In this paper we systematically investigate sources of contamination in the existing cable production process and study alternative strategies --- such as modifying processes, developing new cleaning methods, and changing sources of raw materials --- to minimize radioactivity in the end product cable.