\section{Production of an ultra-low background cable}
\label{sec:final_cables}
\begin{figure*}
    \centering
    \includegraphics[width=\textwidth]{figures/Final_cables.pdf}
    \caption{Outline of custom fabrication process and pictures of the panels used for the final SiPM (left) and CCD (right) cables. The color code for the fabrication process flowchart is described in the legend at the bottom left. The top picture shows an entire sample panel with the cables in the center and detachable coupons on the edges. The bottom picture shows a close-up view of the traces and other features on the cables.}
    \label{fig:final_cables}
\end{figure*}

Having identified the key contaminating steps in flexible cable fabrication and developed methods and alternatives to reduce the contamination in each individual step, we  produced two sets of fully-functional flexible cables by combining all the modifications described in the previous section into a single low-background fabrication process. We chose two different flexible cable designs that are being considered for on-going and future nuclear and particle physics experiments. The first design is under development for the readout of silicon photomultipliers (SiPMs) in the nEXO neutrinoless double beta decay experiment \cite{kharusi2018nexo}. Based on findings from previous investigations into radiopure cable conducted for the EXO-200 cable \cite{pocar_slides,leonard2017trace}, the cable design was simplified to keep contamination levels as low as possible. The cable (shown on the left in Figure~\ref{fig:final_cables}) consists of traces on only one side of the cable and a continuous ground plane on the other. As all the traces are on one side of the cable with no interconnections between the two layers, no vias or copper plating is required. Additionally, while it would be advantageous, the design does not include a coverlay or surface metallization of contacts in order to keep contamination levels to a minimum. 

The second cable design was developed for the readout of charged couple devices (CCDs) for the DAMIC-M \cite{arnquist2023first} dark matter experiment. The CCD cable (shown on the right in Figure~\ref{fig:final_cables}) consists of multiple traces and interconnections between the two copper layers that require copper-plated vias. The cable also includes coverlays on both sides of the cable and ENIG metallization on the pads. 

The SiPM and CCD cable designs were chosen because they represent two ends of the spectrum in terms of complexity of two-layer flexible cables. Figure~\ref{fig:final_cables} shows the process steps used to fabricate these two cables, highlighting the modifications made to the standard process. As before, panels consisting of the cables along with several detachable coupons were used and coupons were removed after several key steps so that the contamination levels could be tracked, if needed. The cleaning process described in Section~\ref{sec:cleaning} was applied at key steps in the fabrication process where it would not interfere with the production and would remove surface contamination before material was permanently added to the cable. Since the cleaning recipe required the use of relatively hazardous chemicals and precise cleaning procedures, at intermediate steps (indicated in green in  Figure~\ref{fig:final_cables}) the partially-processed panels were shipped from Q-Flex to PNNL, cleaned, and then shipped back. The shipping was done in sealed plastic bags and the cleaning was done as promptly as possible to avoid oxidation of the copper layers between steps. Once the fabrication steps were complete, the cable was shipped to PNNL for a final cleaning and subsequent measurement. For the measurements of the contamination levels in the final cables, we sub-sampled the cables themselves (not witness coupons) and used 7-8 sub-samples, rather than the standard 3, to increase our confidence that the average of our measurements were representative of the "true" contamination in the cable overall. 

\begin{table}[t]
\centering
\begin{tabular}{c c c c c c }
\hline
Cable & Rep. & \textbf{\ur} & \textbf{\th} & \textbf{\knat} \\
& & \textbf{[pg/g]} & \textbf{[pg/g]} & \textbf{[ng/g]} \\
\hline
\hline
\multirow{9}{*}{\shortstack{SiPM Cable\\(Custom)}} & 1 & \num{20 \pm 2} & \num{< 9.8} & \num{< 38} \\
 & 2 & \num{21 \pm 2} & \num{< 9.4} & \num{< 37} \\
 & 3 & \num{18 \pm 2} & \num{< 8.6} & \num{< 34} \\
 & 4 & \num{20.9 \pm 1.2} & \num{< 10.4} & \num{47 \pm 6} \\
 & 5 & \num{19 \pm 2} & \num{< 10.3} & \num{32 \pm 8} \\
 & 6 & \num{18.8 \pm 1.2} & \num{< 12.3} & \num{< 20} \\
 & 7 & \num{19.6 \pm 1.5} & \num{< 12.0} & \num{52 \pm 7} \\
 & 8 & \num{19 \pm 3} & \num{< 12.0} & \num{28 \pm 7} \\
 \cline{2-5}
 & Avg.* & \num{20 \pm 2} & \num{< 12.3} & \num{40 \pm 12} \\
\hline
\hline
\multirow{8}{*}{\shortstack{CCD Cable\\(Custom)}} & 1 & \num{32 \pm 2} & \num{12 \pm 3} & \num{559 \pm 13}\\
 & 2 & \num{31 \pm 4} & \num{11 \pm 3} & \num{529 \pm 12}\\
 & 3 & \num{29 \pm 2} & \num{< 8.9} & \num{572 \pm 12} \\
 & 4 & \num{32 \pm 3} & \num{16 \pm 4} & \num{569 \pm 13} \\
 & 5 & \num{31 \pm 2} & \num{< 11.7} & \num{558 \pm 12} \\
 & 6 & \num{30 \pm 2} & \num{< 10.9} & \num{546 \pm 9} \\
 & 7 & \num{30 \pm 2} & \num{< 11.1} & \num{515 \pm 9} \\
\cline{2-5}
 & Avg.* & \num{31 \pm 2} & \num{13 \pm 3} & \num{550 \pm 20} \\
\hline
\hline
\end{tabular}
\caption{Measurements of the \ur, \th~and \knat~concentrations in the final SiPM and CCD custom cables. Each row within the cable type lists the values for a given subsample (identified by the repetition number in the second column) with the uncertainties indicating the instrumental uncertainty. 
*Averages are reported as the average and standard deviation of the central values of all replicates for that isotope (upper limits excluded). Where all replicates are upper limits, the largest upper limit is reported.}
\label{tab:cable_final_results}
\end{table}

The results are shown in Table~\ref{tab:cable_final_results}. The levels of contamination measured in the final SiPM cable were extremely low, with roughly 20 ppt of \ur, \th~levels below our sensitivity of $\sim$ 10 ppt, and \knat~at roughly 36 ppt. The CCD cable, which included additional copper plating of the vias, coverlays on both sides, and ENIG metallization of the contacts, was only slightly higher in \ur~and \th, with roughly 31 and 12 ppt respectively. The \knat~levels were significantly higher, consistent with the intrinsic contamination levels measured in the Panasonic MCL Plus 110 coverlay used and the relative contribution of the coverlay to the overall mass of the cable. It is worth remarking that while commercial cables can sometimes have values that vary significantly from subsample to subsample, all subsample replicates measured in the custom final cable were consistent with each other, indicating that the cleaning procedure (applied at key steps) likely removed any non-uniform surface contamination. 

\subsection{ENIG metallization}
While measurements of the center of the cable (where all samples were taken from for the results in Table~\ref{tab:cable_final_results}) represent the contamination levels in the vast majority of the cable material, the copper contacts at the ends of the cable typically have an additional metallization layer to avoid oxidation and improve solderability \cite{macdermid_surface_finish}. This metallization, typically added through electroless deposition in a chemical bath with the addition of a catalyst, can add additional contamination, and though the total mass is often small the contacts at one end of the cable can be located very close to the sensitive volume. The CCD cables were metallized with electroless nickel immersion gold (ENIG) using \SIrange{3.0}{5.6}{\micro\meter} (\SIrange{118}{220}{microinch}) of nickel and \SIrange{0.076}{0.127}{\micro\meter} (\SIrange{3}{5}{microinch}) of gold. In order to measure the contamination levels of the ENIG layer, subsamples were cut from the ends of the cable to include the metallized pads. Since the size of the ENIG region varied by subsample, for each subsample measurement we estimated the ENIG-covered area and subtracted out the average contamination of the rest of the cable layers (estimated from the measurements at the center). The residual contamination was attributed to the application of the ENIG layer. We find contamination levels per unit area of ENIG at roughly \SI{0.1}{\pico\gram\per\milli\meter\squared} for \ur, \SIrange{0.01}{0.1}{\pico\gram\per\milli\meter\squared} for \th, and \SI{10}{\nano\gram\per\milli\meter\squared} for \knat. %Converting these numbers to the contamination per mass of applied ENIG can be done using the thicknesses quoted above, but the large relative uncertainties in the ENIG thickness for our cables leads to larger uncertainties. 
These results can be used to estimate the backgrounds for specific cable layouts with ENIG metallization layers.

%\begin{table*}[t]
%\centering
%\begin{tabular}{c c c c c c c }
%\hline
%ENIG Rep. & \multicolumn{2}{c}{\textbf{\ur}} & \multicolumn{2}{c}{\textbf{\th}} & \multicolumn{2}{c}{\textbf{\knat}} \\
%& \textbf{[pg/mm$^2$]} & \textbf{[pg/g]} &\textbf{[pg/mm$^2$]} & \textbf{[pg/g]} & \textbf{[ng/mm$^2$]} & \textbf{[ng/g]}\\
%\hline
%1 &  & & & &  & \\
%2 &  & & & &  & \\
%3 &  & & & &  & \\
%\hline
%Avg. &  & & & &  & \\
%\hline
%\hline
%\end{tabular}
%\caption{ENIG caption} 
%\label{tab:enig}
%\end{table*}