%auto-ignore
%---
\title{Ultra-low radioactivity flexible printed cables}
%---

%%%%%%%%%%%%%%%%%%%%%%%%%%%%%%%%%%%%%%%%%%%%%%%%%%%%%%%%%%%%%%%%%%%%%%%%%
%--- Version for revtex
\iffalse 
\newcommand{\pnnl}{Pacific Northwest National Laboratory}
\author{Isaac J. Arnquist}\affiliation{\pnnl}
\author{Maria Laura di Vacri}\affiliation{\pnnl}
\author{Nicole Rocco}\affiliation{\pnnl}
\author{Richard Saldanha}\affiliation{\pnnl}
\author{Tyler Schlieder}\affiliation{\pnnl}
\fi
%%%%%%%%%%%%%%%%%%%%%%%%%%%%%%%%%%%%%%%%%%%%%%%%%%%%%%%%%%%%%%%%%%%%%%%%%

%%%%%%%%%%%%%%%%%%%%%%%%%%%%%%%%%%%%%%%%%%%%%%%%%%%%%%%%%%%%%%%%%%%%%%%%%
%--- Version for elsarticle
\author{Isaac J. Arnquist\corref{cor2}}
%\ead{isaac.arnquist@pnnl.gov}
\author{Maria Laura di Vacri\corref{cor2}}
\author{Nicole Rocco\corref{cor2}}
\author{Richard Saldanha\corref{cor2}}
\author{Tyler Schlieder\corref{cor2}}
\address{Pacific Northwest National Laboratory, Richland, Washington, 99352 USA}
\author{Raj Patel, Jay Patil, Mario Perez, Harshad Uka}
\address{Q-Flex Inc., Santa Ana, California, 92705 USA}
%\cortext[cor1]{Corresponding author}
%\ead{richard.saldanha@pnnl.gov}
%---
%\date{\today}
%---
\begin{abstract}
Flexible printed cables and circuitry based on copper-polyimide materials are widely used in experiments looking for rare events due to their unique electrical and mechanical characteristics. However, past studies have found copper-polyimide flexible cables to contain 400-4700 pg $^{238}$U/g, 16-3700 pg $^{232}$Th/g, and 170-2100 ng $^{nat}$K/g, which can be a significant source of radioactive background for many current and next-generation ultralow background detectors. This study presents a comprehensive investigation into the fabrication process of copper-polyimide flexible cables and the development of custom low radioactivity cables for use in rare-event physics applications.  A methodical step-by-step approach was developed and informed by ultrasensitive assay to determine the radiopurity in the starting materials and identify the contaminating production steps in the cable fabrication process. Radiopure material alternatives were identified, and cleaner production processes and treatments were developed to significantly reduce the imparted contamination. Through the newly developed radiopure fabrication process, fully-functioning cables were produced with radiocontaminant concentrations of 20-31 pg $^{238}$U/g, 12-13 pg $^{232}$Th/g, and 40-550 ng $^{nat}$K/g, which is significantly cleaner than cables from previous work and sufficiently radiopure for current and next-generation detectors. This approach, employing witness samples to investigate each step of the fabrication process, can hopefully serve as a template for investigating radiocontaminants in other material production processes.
\end{abstract}
%---
%---
\begin{keyword}
flexible cables, polyimide, radioactivity, ultralow background experiments 
\end{keyword}
%---
%\maketitle