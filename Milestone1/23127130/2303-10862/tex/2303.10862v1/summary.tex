\section{Summary and Discussion}
\definecolor{Gray}{gray}{0.9}
\definecolor{LightCyan}{rgb}{0.88,1,1}
\begin{table*}[]
    \centering
    \begin{tabular}{lccccccc}
    \hline
    Cable & Copper & Polyimide  & Coverlay & Surface & \ur & \th & \knat  \\
     & Layers & Layers &  &  Finish & & &  \\
     & [\si{\micro\meter}] & [\si{\micro\meter}] & & &  [\si{\ppt}] & [\si{\ppt}] & [\si{\ppb}]  \\
    \hline
    \rowcolor{LightCyan}
    nEXO SiPM [This Work] & \num{18} (x2) & \num{50.8} (x1) & No & No & \num{20 \pm 2} & \num{< 12.3} & \num{40 \pm 12}\\
    nEXO SiPM [Comm.] & \num{18} (x2) & \num{50.8} (x1) & No & No & \numrange{1300}{6200} & \numrange{16}{63} & \\
    \hline
    \rowcolor{LightCyan}
    DAMIC-M CCD [This Work] & \num{18} (x2) & \num{50.8} (x1) & x2 & ENIG & \num{31 \pm 2} & \num{13 \pm 3} & \num{550 \pm 20}\\
    DAMIC-M CCD [Comm.] & \num{18} (x2) & \num{50.8} (x1) & x2 & ENIG & \num{2600 \pm 40} & \num{261 \pm 12} & \num{170 \pm 50}\\
    \hline
    EXO-200 \cite{pocar_slides, leonard2017trace} & \num{18} (x1) & \num{25.4} (x1) & No & No & \num{412 \pm 47} & $< 117$ & \\
    EDELWEISS III \cite{armengaud2017performance, zhang2015novel}& \num{18} (x4) & 25/125 (x3/x4) & No & No & \num{650 \pm 490} & \num{3700 \pm 2500} & \num{2100 \pm 840} \\
    DAMIC at SNOLAB \cite{aguilar2022characterization}& \num{18} (x5) & \num{25.4} (x4) & x2 & ENIG & \num{4700 \pm 400} & \num{790 \pm 120} & \num{940 \pm 60} \\
    \hline
    \end{tabular}
    \caption{Comparison of the final radiopurity levels achieved in cables produced as part of this work (highlighted in blue) with other commercial and custom cables used in low background rare event experiments. The commercial (abreviated as Comm.) cables listed near the top of the table are the same design as our custom cables and can be compared directly. The cables listed at the bottom of the table have different construction and fabrication options - see text for details.}
    \label{tab:final_comparisons}
\end{table*}

As a direct point of comparison, we also measured commercial cables with the same design as our SiPM and CCD custom cables. For the SiPM cable we measured a cable that was produced with the same low-background starting laminate but using the original standard Q-Flex fabrication process that did not include any of the modifications described in this work. Cables from different batches showed large variations in the \ur~and \th~contamination levels and so we quote a range of values (see Table~\ref{tab:final_comparisons}). For the CCD cable we obtained a cable from the DAMIC collaboration of the same design, fabricated by a different commercial vendor (PCB Universe \cite{pcbuniverse}) with no special instructions on materials to use or process modifications. It can be seen from comparing the results in Table~\ref{tab:final_comparisons} that, compared to our custom low radioactivity cables, \ur~and\th~ levels in the commercial cables are roughly 100$\times$ and 10$\times$ larger, respectively. Interestingly, the potassium contamination levels in the commercial CCD cable were roughly $2\times$ lower than the levels measured in our custom cable, likely due to the choice of a different coverlay material.

It is also  worthwhile to compare our results to those obtained by other experiments that use flat flexible cables in rare event searches, though the cable designs vary significantly. As noted in the introduction, the EXO-200 collaboration, which searched for the neutrinoless double beta decay of $^{136}$Xe, invested significant effort to reduce the contamination in their detector cables \cite{pocar_slides, auger2012exo}. They used single-layer cables and worked with the vendor to use fresh chemicals and containers and add isopropanol rinses during the production process. After fabrication, all small cables were subjected to a post-production plasma etch followed by a cleaning with acetone and ethanol. To reduce contamination, no metallization of the contacts was applied and no coverlay was used within the detector cryostat. The copper and polyimide layers of the cable were separately assayed for radiocontamination using ICP-MS \cite{leonard2008systematic}. Taking the best obtained values for the copper (Entry 261 in Ref.\cite{leonard2008systematic}) and polyimide (Entry 262 in Ref.\cite{leonard2008systematic}) we have combined them by the mass ratio to calculate the final value shown in Table~\ref{tab:final_comparisons}. The EDELWEISS collaboration, searching for dark matter, surveyed several different materials and selected specific materials for their copper clad polyimide laminates \cite{zhang2015novel}. Though they only used four conducting layers, the cable had additional polyimide layers of varying thickness in between to reduce the capacitance. They also identified the photolithography process as an additional source of contamination and used a custom fabrication procedure at the Oxford Photofabrication Unit \cite{zhang2015novel}. The contamination levels in the final cables used in the EDELWEISS III experiment \cite{armengaud2017performance} are listed in Table~\ref{tab:final_comparisons}. The DAMIC collaboration, searching for dark matter, used a 5-layer cable to read out their CCDs installed at SNOLAB \cite{aguilar2022characterization}. These were commercial cables with no effort made to reduce the contamination. The contamination levels, measured through ICP-MS, are listed in Table~\ref{tab:final_comparisons}. It can be seen that the radiopurity levels achieved in the cables produced as part of this work, compared to the ones used in previous experiments, are at least an order of magnitude better in \ur~and \th, and lower in \knat~ as well.

This work was motivated by the specific need for low radioactivity cables in either planned upgrades of currently running experiments such as DAMIC-M \cite{arnquist2023first} and SuperCDMS \cite{albakry2022strategy} or in future proposed experiments such as nEXO \cite{kharusi2018nexo}, Oscura \cite{aguilar2022oscura}.  We have demonstrated that simple cable designs can be produced with contamination levels at 20 ppt \ur, $< 12$ ppt \th, and 40 ppb \knat, nearly at the levels found in the raw materials. We believe these levels to be sufficiently radiopure for all the experiments listed above. 

We have also shown that cable designs that require additional features such as interconnected layers, coverlays, and metallization can also be made with only a relatively small increase in the \ur~contamination, though there is a significant increase in \knat~due to the coverlay material. Additional investigations into multi-layer cables, higher radiopurity coverlays, and alternatives to ENIG metallization are planned in the near future. It should be noted that low radioactivity cables can increase the sensitivity of these experiments by not only reducing the background budget but also by  allowing for the use of additional sensors and simpler designs of the sensor readout within the same background budget. 

Finally, while we have focused this work on radiopure cables, the same low radioactivity fabrication investigation could be applied to flexible printed circuits and related applications. For example, in future work we will use a similar approach with the goal of developing low radioactivity superconducting cables with possible applications to cryogenic rare-event searches \cite{hollister2017cryogenics, andreotti2009low}, and to develop low radioactivity printed circuit boards, which have been predicted to be the dominant material within dilution refrigerators impacting decoherence from background ionizing radiation \cite{cardani2023disentangling}.
