\section{Analysis Methods}
\label{sec:analysis}
All analyses of the materials reported in this work were performed at PNNL. Analytical work was performed in a Class 10,000 cleanroom, and a laminar flow hood providing a Class 10 environment was used for sample preparations. Details on the chemical reagents used and the preparation of all labware prior to sample handling are given in \ref{sec:labware}.

\subsection{Sample digestion}\label{sec:digestion}

Subsamples of $\sim$\SI{50}{\milli\gram} were cut from original sample with clean stainless steel scissors, collected, and weighed in validated perfluoroalkoxy alkane (PFA) screw cap vials from Savillex (Eden Prairie, MN). Unless explicitly stated, only central portions of the coupons/cables were used for subsampling, excluding any regions with vias, pads, or ENIG processing. Low background clean plastic tongs were used to transfer subsamples to validated polytetrafluoroethylene PTFE iPrep$^{TM}$ vessels for microwave-assisted digestion in a Mars 6 system (CEM Corporation, Matthews, NC). All samples and process blanks were spiked with a known amount ($\sim$ \SIrange{100}{200}{\femto\gram}) of gravimetrically measured $^{229}$Th and $^{233}$U radiotracers (Oak Ridge National Laboratory, Oak Ridge, TN) before digestion. Triplicates of each sample and three process blanks were prepared for each batch of analyses. Digestion was performed in \SI{5}{\milli\liter} of concentrated Optima grade HNO$_3$ at \SI{250}{\celsius}. After complete digestion, sample solutions were transferred to cleaned and validated PFA Savillex vials, and properly diluted for ICP-MS analysis.


\subsection{ICP-MS Analysis} 
Determinations of Th and U were performed using either an Agilent 8800 or 8900 ICP-MS (Agilent Technologies, Santa Clara, CA), each equipped with an integrated autosampler, a microflow PFA nebulizer and a quartz double pass spray chamber. Plasma, ion optics and mass analyzer parameters were tuned based on the instrumental response of a tuning standard solution from Agilent Technologies containing \SI{0.1}{\nano\gram\per\gram} Li, Co, Y, Ce, Tl. In order to maximize the signal-to-noise in the high $m/z$ range for Th and U analysis, the instrumental response for Tl, the element in the standard with $m/z$ and first ionization potential closest to those of Th and U, was used as a reference signal for instrumental parameter optimization. Oxides were monitored and kept below $2\%$ based on the $m/z = 156$ and $m/z = 140$ ratio from Ce (CeO$^{+}$/Ce$^{+}$) in the tuning standard solution. An acquisition method of three replicates and ten sweeps per replicate was used for each reading. Acquisition times for monitoring $m/z$ of interest (\textit{e.g.,} tracers and analytes) were set based on expected signals, in order to maximize instrumental precision by improving counting statistics.

Quantitation of $^{232}$Th and $^{238}$U was performed by isotope dilution methods, using the equation:
\begin{equation}
    \text{Concentration} = \dfrac{A_\text{analyte} \cdot C_\text{tracer}}{A_\text{tracer}} 
\end{equation}
where $A_\text{analyte}$ is the instrument response for the analyte, $A_\text{tracer}$ is the instrument response for the tracer and $C_{tracer}$ is the concentration of the tracer in the sample. Isotope dilution is the most precise and accurate method for quantitation for ICP-MS analysis, allowing the verification of sample preparation efficiency, and accounting for analyte losses and/or matrix effects. Absolute detection limits on the order of a few femtograms were attained for both \ur~and \th. Detection limits normalized to sample mass were on the order of single digit \si{\pico\gram\per\gram} for both Th and U, corresponding to \si{\micro\becquerel\per\kilo\gram}  in terms of radioactivity.

Determinations of natural potassium contamination levels were performed in cool plasma with NH$_3$ reaction mode. Instrumental parameters were optimized based on the instrumental response from a solution containing $\sim$ \SI{1}{\nano\gram\per\gram} $^{nat}$K with natural isotopic composition, diluted in-house from standard solutions. Quantitations of potassium were performed using an external calibration curve, generated using in-house diluted standard solutions with natural isotopic composition. Relatively high potassium backgrounds from the microwave digestion PTFE iPrep$^{TM}$ vessels (see Section~\ref{sec:digestion}) limited detection limits on potassium to the \si{\nano\gram\per\gram} range. Due to the extensive preparation required to minimize these backgrounds, potassium determinations were only performed on a limited number of samples.

All reported central values are the average value of multiple replicates (typically n = 3). Uncertainties for individual replicates were determined from the propagated uncertainties of the instrumental precision. The overall uncertainty on the central value was taken to be the larger of the uncertainty from the instrumental precision and the standard deviation of the replicates. Samples for which the analyte concentration was measured below the detection limit are reported as an upper limit. Values are reported in \si{\pico\gram\per\gram} of sample for $^{238}$U and $^{232}$Th, and in \si{\nano\gram\per\gram} of sample for $^{nat}$K. Corresponding activities in \si{\micro\becquerel\per\kilo\gram} of sample can be obtained based on the specific activity and isotopic abundance of the radionuclides. For reference, 1 \si{\pico\gram~^{238}U\per\gram} of sample corresponds to 12.4 \si{\micro\becquerel\per\kilo\gram} of sample, 1 \si{\pico\gram~^{232}Th\per\gram} corresponds to 4.06 \si{\micro\becquerel\per\kilo\gram}, and 1 \si{\nano\gram~^{nat}K\per\gram} corresponds to 30.5 \si{\micro\becquerel\per\kilo\gram}. 


