\vspace{-0.3cm}\subsection{Effectiveness test}\label{sec_effectiveness}\vspace{-0.2cm}
This section studies how the block degree is affected by varying the budget $k$, the size of the rumor set $R$, and the length threshold $T$ of a random walk.

\noindent
\textbf{Varying the budget $k$.} The block degrees of all algorithms on Gnutella and Email-Enron by varying the $k$ are shown in Figure \ref{fig:exp_ef_Gnutella_B} and Figure \ref{fig:exp_ef_Email_B}, respectively, and we find that when the budget raises from 50 to 250,
BAB outperforms Greedy and TopK by up to 115\% in the Email-Enron.

\noindent
\textbf{Varying the size of $R$.} Figure \ref{fig:exp_ef_Gnutella_R} and Figure \ref{fig:exp_ef_Email_R}  show the result by varying the size of $R$. We find: (1) with the growth of $|R|$, the blocking percentages of all methods are increasing because the increasing influence of $R$ leads to more nodes with higher unit block degrees. (2) ProBAB and BAB are consistently better than that of the rest baselines.

\noindent
\textbf{Varying the random walk length threshold $T$.} Figure \ref{fig:exp_ef_Gnutella_T} and Figure \ref{fig:exp_ef_Email_T}  show the results by varying the threshold $T$, which determines the length of a random walk starting from a node. We observe that: (1) The rumors on Gnutella dataset are much harder to be controlled than Email-Enron dataset. It implies that the network structure is an important variable for \prob. (2) With the increase of $T$, the performance of all algorithms becomes better. The reason is that when the length becomes large, the random walk has more chances to reach the protectors and thus leads to a high unit block degree of the seeds.

%(1) \samgreedy~and \indexgreedy~achieve the same block degree, while \degreeTop~has the worst performance. \samgreedy~and \indexgreedy~are little better than \samgreedyplus~and \indexgreedyplus. (2) With the growth of $B$, the advantage of \samgreedy~and \indexgreedy~over \degreeTop~decreases, from 49\% to 20\% when $B$ varies from 50 to 250 on Email-Enron dataset. This is because that \degreeTop~ignores the blocking overlaps between the selected seeds, which declines its block degree. However, when $B$ is large, the performance of \degreeTop~catches up with the others since the overlaps are inevitable.

% \begin{figure*}
%         \begin{minipage}[b]{0.47\textwidth}
% 	\centering
% 	%\includegraphics[width=1\textwidth]{exp/title5.pdf}
% 	%\vspace{-0.3cm}
% 	%\hspace{-10pt}
% 	\subfloat[Varying $k$]
% 	{\label{fig:exp_ef_Gnutella_B}\includegraphics[width=0.33\textwidth]{exp/k-effect-G.eps}}
% 	%\hspace{-10pt}
% 	\subfloat[Varying $|R|$]
% 	{\label{fig:exp_ef_Gnutella_R}\includegraphics[width=0.33\textwidth]{exp/R-effect-G.eps}}
% 	%\hspace{-10pt}
% 	\subfloat[Varying $T$]
% 	{\label{fig:exp_ef_Gnutella_T}\includegraphics[width=0.33\textwidth]{exp/L-effect-G.eps}}
% 	\caption{Effectiveness test on Gnutella}
% 	\label{fig:para}
% 	\end{minipage}
% 	%\hspace{1in}
% 	\begin{minipage}[b]{0.47\textwidth}
% 	\centering
% 	%\includegraphics[width=1\textwidth]{exp/title5.pdf}
% 	%\vspace{-0.3cm}
% 	%\hspace{-10pt}
% 	\subfloat[Varying $k$]
% 	{\label{fig:exp_ef_Email_B}\includegraphics[width=0.33\textwidth]{exp/k-effect-E.eps}}
% 	%\hspace{-10pt}
% 	\subfloat[Varying $|R|$]
% 	{\label{fig:exp_ef_Email_R}\includegraphics[width=0.33\textwidth]{exp/R-effect-E.eps}}
% 	%\hspace{-10pt}
% 	\subfloat[Varying $T$]
% 	{\label{fig:exp_ef_Email_T}\includegraphics[width=0.33\textwidth]{exp/L-effect-E.eps}}
% 	\caption{Effectiveness test on Email-Enron}
% 	\label{fig:para}
% 	\end{minipage}
% 	\vspace{-0.5cm}
% \end{figure*}

\begin{figure*}[t]
	\centering
	%\includegraphics[width=1\textwidth]{exp/title5.pdf}
	%\vspace{-0.3cm}
	%\hspace{-10pt}
	\subfloat[Varying $k$]
	{\label{fig:exp_ef_Gnutella_B}\includegraphics[width=0.3\textwidth]{exp/k-effect-G.eps}}
	%\hspace{-10pt}
	\subfloat[Varying $|R|$]
	{\label{fig:exp_ef_Gnutella_R}\includegraphics[width=0.3\textwidth]{exp/R-effect-G.eps}}
	%\hspace{-10pt}
	\subfloat[Varying $T$]
	{\label{fig:exp_ef_Gnutella_T}\includegraphics[width=0.3\textwidth]{exp/L-effect-G.eps}}
	\caption{Effectiveness test on Gnutella}
	\label{fig:para}
\end{figure*}
\begin{figure*}[t]
	\centering
	%\includegraphics[width=1\textwidth]{exp/title5.pdf}
	%\vspace{-0.3cm}
	%\hspace{-10pt}
	\subfloat[Varying $k$]
	{\label{fig:exp_ef_Email_B}\includegraphics[width=0.3\textwidth]{exp/k-effect-E.eps}}
	%\hspace{-10pt}
	\subfloat[Varying $|R|$]
	{\label{fig:exp_ef_Email_R}\includegraphics[width=0.3\textwidth]{exp/R-effect-E.eps}}
	%\hspace{-10pt}
	\subfloat[Varying $T$]
	{\label{fig:exp_ef_Email_T}\includegraphics[width=0.3\textwidth]{exp/L-effect-E.eps}}
	\caption{Effectiveness test on Email-Enron}
	\label{fig:para}
\end{figure*}


\iffalse
\begin{figure}[t]
	\centering
	\subfloat[Gnutella]{\includegraphics[clip,width=0.245\textwidth]{exp/k-effect-G.eps}\label{fig:exp_ef_Gnutella_B}}
	\hspace{-5pt}
	\subfloat[Email-Enron]{\includegraphics[clip,width=0.245\textwidth]{exp/k-effect-E.eps}\label{fig:exp_ef_Email_B}}
	\caption{Effectiveness of varying the budget $k$}
	\label{fig:exp_ef_B}
\end{figure}
\fi


\iffalse
\begin{figure}[t]
	\centering
	\subfloat[Gnutella]{\includegraphics[clip,width=0.245\textwidth]{exp/R-effect-G.eps}\label{fig:exp_ef_Gnutella_R}}
	\hspace{-5pt}
	\subfloat[Email-Enron]{\includegraphics[clip,width=0.245\textwidth]{exp/R-effect-E.eps}\label{fig:exp_ef_Email_R}}
	\caption{Effectiveness of varying the size of $R$}
	\label{fig:exp_ef_R}
\end{figure}
\fi

\iffalse
\begin{figure}[t]
	\centering
	\subfloat[Gnutella]{\includegraphics[clip,width=0.245\textwidth]{exp/L-effect-G.eps}\label{fig:exp_ef_Gnutella_T}}
	\hspace{-5pt}
	\subfloat[Email-Enron]{\includegraphics[clip,width=0.245\textwidth]{exp/L-effect-E.eps}\label{fig:exp_ef_Email_T}}
	\caption{Effectiveness of varying the random walk length threshold $T$}
	\label{fig:exp_ef_t}
\end{figure}
\fi