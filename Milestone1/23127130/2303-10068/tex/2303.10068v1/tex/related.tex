\section{Related work}
In the following, we discuss the most relevant literature to our problem.

Two proactive rumor control problems in online networks are close to our work~\cite{DBLP:journals/www/ZhangBNZMGP19,DBLP:conf/nctcs/MoT0P19}, which also study proactive rumor control problem to minimize the spread of the rumor set under the budget. The core difference lies in the influence block model. In particular, the existing work assumes that one anti-rumor node before the rumor node can block the total influence of the rumor set to the user in one browsing process. %Under such an influence block model, when multiple billboards are close to a trajectory, the marginal influence is reduced to capture the property of diminishing returns. Therefore, TIP focuses on identifying and reducing the overlap of the influence among different billboards to the same trajectories while keeping the budget constraint into consideration. That is, TIP can maximize the number of distinct users by impressing as many people as possible for one time. 
It does not consider the relationship between the influence block effect and counts
of impressions on one user because the model assumes one time
impression is enough. On the contrary, \prob~is built upon a logistic influence block model, which has been widely adopted in consumer behavior studies. To minimize the spread of the rumor set, we need to control the overlap to some extent by impressing the same users several times. %Unfortunately, the logistic influence model is non-submodular. Adapting the greedy approach to ICOA, which effectively solves TIP, could lead to arbitrarily bad solutions due to the non-submodular of the influence function.
%, which only studies one specific case of our problem, assuming that the cost of all nodes in $G$ is uniform. Nevertheless, in fact, we cannot ignore the cost information of the nodes. For example, it is impossible that the price of broadcasting information on Baidu's homepage is the same as personal homepage's. Therefore, we study the rumor control within budget constraint problem.

Two other problems close to our problem are influence block and competitive influence maximization. 
Influence block aims to limit the influence of rumors by blocking some nodes or links in a network~\cite{albert2000error,newman2002email,DBLP:conf/kdd/HabibaYBS08}. Their strategies of the seed selection are mainly based on their connectivity, such as degree~\cite{albert2000error,newman2002email}, pagerank~\cite{DBLP:conf/kdd/HabibaYBS08}, and
betweenness~\cite{DBLP:conf/kdd/HabibaYBS08}.
Different from the first problem, competitive influence maximization tries to identify a set of target seed nodes (or protectors) who will spread an `anti-rumor’ to limit the scale of rumor propagation~\cite{DBLP:conf/wine/BharathiKS07,DBLP:conf/wine/BorodinFO10,DBLP:conf/ACMicec/CarnesNWZ07}. Carnes et al.~\cite{DBLP:conf/ACMicec/CarnesNWZ07}, and Bharathi et
al.~\cite{DBLP:conf/wine/BharathiKS07} study competitive influence diffusion under the extension of the Independent Cascade model and show
that the problem of maximizing the influence of one campaign is NP-hard and submodular,
while Borodin et al.~\cite{DBLP:conf/wine/BorodinFO10} studies the similar problem under the Linear Threshold model. 
Our problem is essentially different from the above work for the following reason. Both influence block and competitive influence maximization assume that the information (or rumors) propagations are driven by the effect of word-of-mouth, and they use the Independent Cascade model and Linear Threshold model to simulate the spread of rumors. However, our problem assumes that rumors spread via browsing behaviors and uses a random walk model to describe the influence spread of rumors.