
\subsection{Efficiency test}\label{sec_efficiency}
We evaluate the efficiency of different algorithms on Gnutella and Email-Enron datasets. 

\begin{figure*}[t]
	\centering
	\includegraphics[width=0.5\textwidth]{exp/title.pdf}\\
	%\includegraphics[width=1\textwidth]{exp/title5.pdf}
	%\hspace{-10pt}
	\subfloat[Varying $k$]
	{\includegraphics[clip,width=0.3\textwidth]{exp/k-time-G.eps}\label{fig:exp_time_Gnutella_B}}
	%\hspace{-10pt}
	\subfloat[Varying $|R|$]
	{\includegraphics[clip,width=0.3\textwidth]{exp/R-time-G.eps}\label{fig:exp_time_Gnutella_R}}
	%\hspace{-10pt}
	\subfloat[Varying $T$]
	{\includegraphics[clip,width=0.3\textwidth]{exp/L-time-G.eps}\label{fig:exp_time_Gnutella_T}}
	\caption{Efficiency test on Gnutella}
	\label{fig:para}
\end{figure*}
\begin{figure*}[t]
	\centering
	%\includegraphics[width=0.5\textwidth]{exp/title.pdf}\\
	%\hspace{-10pt}
	\subfloat[Varying $k$]
	{\includegraphics[clip,width=0.3\textwidth]{exp/k-time-E.eps}\label{fig:exp_time_Email_B}}
	%\hspace{-10pt}
	\subfloat[Varying $|R|$]
	{\includegraphics[clip,width=0.3\textwidth]{exp/R-time-E.eps}\label{fig:exp_time_Email_R}}
	%\hspace{-10pt}
	\subfloat[Varying $T$]
	{\includegraphics[clip,width=0.3\textwidth]{exp/L-time-E.eps}\label{fig:exp_time_Email_T}}
	\caption{Efficiency test on Email-Enron}
	\label{fig:para}
\end{figure*}

\iffalse
\begin{figure}[t]
	\centering
	\includegraphics[clip,width=0.3\textwidth]{exp/title.pdf}\hspace{100pt}
	\subfloat[Gnutella]{\includegraphics[clip,width=0.245\textwidth]{exp/k-time-G.eps}\label{fig:exp_time_Gnutella_B}}
	%\hspace{-5pt}
	\subfloat[Email-Enron]{\includegraphics[clip,width=0.245\textwidth]{exp/k-time-E.eps}\label{fig:exp_time_Email_B}}
	\caption{Efficiency of varying the budget $k$}
	\label{fig:exp_time_k}
\end{figure}
\fi

\noindent
\textbf{Varying the budget $k$.} Figure \ref{fig:exp_time_Gnutella_B}  and Figure \ref{fig:exp_time_Email_B} present the efficiency result when $k$ varies from 50 to 250. We have the following observations. (1) The performance of \samgreedy~and ProBAB is about 2 and 1 orders of magnitude faster than BAB, respectively. (2) The runtime of all methods except \degreeTop~is slowly increasing with the growth of $k$. This is because the increase of $k$ directly causes selecting more nodes to $P$, which leads to an increase in the number of updating the influence block of the remaining node.

\noindent
\textbf{Varying the size of $R$.} Figure \ref{fig:exp_time_Gnutella_R}  and Figure \ref{fig:exp_time_Email_R} show the runtime of all algorithms on Gnutella and Email-Enron, respectively. We can see that the runtime of all methods except \degreeTop~is also slowly increasing when $|R|$ varies from 50 to 250 on all datasets. This is because the influence set $\IS(\ru)$ of $\ru$ is increasing with the growth of $|R|$.

\iffalse
\begin{figure}[t]
	\centering
	%\includegraphics[clip,width=0.5\textwidth]{exp/title.pdf}
	\includegraphics[width=0.5\textwidth]{exp/title.pdf}\\
	\subfloat[Gnutella]{\includegraphics[clip,width=0.245\textwidth]{exp/R-time-G.eps}\label{fig:exp_time_Gnutella_R}}
	\hspace{-5pt}
	\subfloat[Email-Enron]{\includegraphics[clip,width=0.245\textwidth]{exp/R-time-E.eps}\label{fig:exp_time_Email_R}}
	\caption{Efficiency of varying the size of $R$}
	\label{fig:exp_time_r}
\end{figure}
\fi
\noindent
\textbf{Varying the random walk length threshold $T$.} We evaluate the efficiencies of algorithms by varying $T$ from 3 to 15. The result is shown in Figure \ref{fig:exp_time_Gnutella_T} and Figure \ref{fig:exp_time_Email_T}. We can see that all the algorithms except for \degreeTop~scale linearly with respect to $T$, which is because they need to scan more nodes to compute the influence block in each random walk .

\iffalse
\begin{figure}[t]
	\centering
	\includegraphics[clip,width=0.3\textwidth]{exp/title.pdf}\hspace{100pt}
	\subfloat[Gnutella]{\includegraphics[clip,width=0.245\textwidth]{exp/L-time-G.eps}\label{fig:exp_time_Gnutella_T}}
	\hspace{-5pt}
	\subfloat[Email-Enron]{\includegraphics[clip,width=0.245\textwidth]{exp/L-time-E.eps}\label{fig:exp_time_Email_T}}
	\caption{Efficiency of varying the random walk length $T$}
	\label{fig:exp_time_t}
\end{figure}
\fi