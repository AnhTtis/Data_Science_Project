\section{Introduction}
World Wide Web and social networks have become the most commonly utilized vehicles for information propagation and changed people's lifestyles greatly due to the increasing popularity of online networks. However, the ease of information propagation is a double-edged sword. But rumors and misinformation could be quickly spread on social networks, which results in undesirable social effects and even leads to economic losses~\cite{DBLP:conf/www/BudakAA11,DBLP:conf/cikm/TripathyBM10}.%\cite{DBLP:conf/websci/NguyenYTE12,DBLP:conf/www/BudakAA11,DBLP:conf/csonet/ZhangZLT15,DBLP:conf/cikm/TripathyBM10}. %For example, the rumor of explosion in the White House in 2013 caused \$130 billion loss on the stock market\footnote{https://www.dailymail.co.uk/sciencetech/article-3090221} and the fake tweet about the earthquake in Ghazni province in August 2012 made thousands of people leave their home in a panic and be afraid of returning home~\cite{DBLP:conf/icdcs/FanLWTMB13}. 
Therefore, minimizing the spread of rumors in online networks is a crucial problem.

To solve this problem, a lot of work studies the problem of rumor control which aims to minimize the spread of rumors on social network~\cite{DBLP:conf/wine/BharathiKS07,DBLP:conf/wine/BorodinFO10,DBLP:conf/ACMicec/CarnesNWZ07,albert2000error,newman2002email,DBLP:conf/kdd/HabibaYBS08}.
%in this work, we consider a strategy that initiates protectors to fight against rumors on social networks. 
%Unfortunately, most previous work 
However, they only assume that users are passive receivers of rumors even if the users can browse the rumors on their own. Therefore, in this study, we assume that users will actively encounter/contact the rumors via their browsing behaviors, i.e., keyword search, social browsing, etc, which can be modeled by random walk model~\cite{DBLP:journals/www/ZhangBNZMGP19,DBLP:conf/nctcs/MoT0P19}. Unfortunately, existing work~\cite{DBLP:journals/www/ZhangBNZMGP19,DBLP:conf/nctcs/MoT0P19} does not consider the relationship between the influence block and counts of impressions on one user because the model assumes one-time impression is enough. But in the real world, studies in consumer behavior report that users are unlikely to take meaningful action when they receive a message only one time~\cite{feder1985adoption,lancaster1990econometric}. Meanwhile, there is evidence showing that the effect of message repetition should be measured as an S-shaped function (logistic function)~\cite{palda1965measurement,taylor2009once}.

To this end, we study the problem of minimizing the spread of rumor
when impression counts and call it Rumor Control when Impression Counts
(\prob). Suppose that an online network is represented by a graph $G(V,E)$. 
Given a rumor set $R \in V$ and a budget $k$, \prob~aims to find a protector set $P \in V \backslash R$ to minimize the spread of the rumor set $R$ as much as possible under the budget $k$. To the best of our knowledge, this is the first problem for rumor control when impression counts are considered. As a result, the following challenges are important to be addressed.

The first challenge is the NP-hardness of \prob~as we analyze in Theorem~\ref{NP-hardness}. Then, we resort to developing approximate algorithms to solve it efficiently. The second challenge is posed by the property of the logistic function. The influence block model based on the logistic function is non-submodular, which means any straightforward greedy-based
the approach is not applicable to address the \prob~problem as shown in Example~\ref{non-sub}. To overcome this challenge, we proposed a sampling-based greedy method to estimate the upper bounds of the logistic function value. Based on this upper bound estimating method, we devise
a branch-and-bound framework for \prob, with a ($1-1/e-\epsilon$) approximation ratio. Furthermore, we speed up our framework with a progressive upper-bound estimation method.
%However, this framework still suffers from a high computational cost due to heavily invoking the upper bound estimation method. Therefore, to further improve the efficiency of our framework, we devise a progressive sampling-based greedy method for the upper bound estimation, which can provide a trade-off between efficiency and effectiveness.
In summary, we make the following contributions.
\begin{itemize}
	\item We propose and study the \prob~problem, and analyze the monotonicity and non-submodularity of the objective function of \prob. We show that \prob~is NP-hard.
	\item To solve the \prob~problem, we present a Monte Carlo based greedy algorithm (\samgreedy) as the baseline solution. Moreover, we devise an upper-bound estimation method by adaptively solving submodular optimization problems. Based on the upper bound function, we propose a branch-and-bound framework for \prob, with a ($1-1/e-\epsilon$) approximation ratio.
	\item To further improve the efficiency, we speed up our framework with a progressive sampling-based greedy method for upper bound estimation, which achieves a ($1-1/e-\epsilon - \rho$) approximation ratio and a significant reduction in running time.
	\item We conduct extensive experiments on three real-world datasets. The results validate the effectiveness, efficiency, and scalability of our solutions.
\end{itemize}
