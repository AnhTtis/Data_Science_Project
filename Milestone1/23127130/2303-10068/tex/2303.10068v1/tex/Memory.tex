
\subsection{Parameter sensitive test }\label{sec_Parameter} 
\noindent
\textbf{Varying $\beta/\alpha$.} Figure~\ref{fig:exp_pra_t} reports the efficiency and effectiveness of each algorithm when $\beta/\alpha$ is varying. As shown in Figure~\ref{fig:exp_time_Gnutella}, the varying of $\beta/\alpha$ has no impact on the running time of all algorithms. But from Figure~\ref{fig:exp_effect_Gnutella}, we find that the effectiveness of all algorithms is decreasing when the $\beta/\alpha$ varies from $3/7$ to $3/11$. This is because the smaller the $\beta/\alpha$ is, the more times of impression are needed to change a user’s adoption. In particular, with the decrease of $\beta/\alpha$, our solutions outperform \samgreedy~by
70\% to 217\%. Therefore, we choose $\alpha$ = 7 and $\beta$ = 3 as the default setting since our solutions have the smallest advantage of effectiveness for the setting.

%we find: (1) \degreeTop, \samgreedy~and \indexgreedy~have similar memory consumption. It is because they all need to scan the sampling set to calculate the block degree of the nodes. (2) When $T$ is small, the memory consumption of \samgreedyplus~and \indexgreedyplus~is much smaller than that of other methods. This is because only a small number of nodes can reach $R$ when $T$ is small. Therefore, we can reduce the sampling set by eliminating more nodes in pre-sampling stage. But with the growth of $T$, most nodes will reach $R$ in pre-sampling stage, which causes the increasing of the memory consumption of \samgreedyplus~and \indexgreedyplus.

\begin{figure*}[t]
	%\hspace{0.8in}
        %\includegraphics[width=0.35\textwidth]{exp/title.pdf}
	\begin{minipage}[b]{0.48\textwidth} %{0.45\textheight}
		\centering
	%\includegraphics[clip,width=0.5\textwidth]{exp/title.pdf}
	%\includegraphics[width=0.35\textwidth]{exp/title.pdf}\\
	\subfloat[Time]{\includegraphics[clip,width=0.5\textwidth]{exp/p-time-G.eps}\label{fig:exp_time_Gnutella}}
	\hspace{-5pt}
	\subfloat[Effect]{\includegraphics[clip,width=0.5\textwidth]{exp/p-effect-G.eps}\label{fig:exp_effect_Gnutella}}
	\caption{Varying $\beta/\alpha$ in Gnutella}
	\label{fig:exp_pra_t}
	\end{minipage}
	%\hspace{1in}
	\begin{minipage}[b]{0.48\textwidth}
		\centering
	%\includegraphics[clip,width=0.5\textwidth]{exp/title.pdf}
	%\includegraphics[width=0.35\textwidth]{exp/title.pdf}\\
	\subfloat[Time]{\includegraphics[clip,width=0.5\textwidth]{exp/x-time-G.eps}\label{fig:x_time_Gnutella}}
	\hspace{-5pt}
	\subfloat[Effect]{\includegraphics[clip,width=0.5\textwidth]{exp/x-effect-G.eps}\label{fig:x_effect_Gnutella}}
	%\caption{Varying the number of samples $X$ in Gnutella}
        \caption{Varying $X$ in Gnutella}
	\label{fig:exp_pra_x}
	\end{minipage}
\end{figure*}

% \begin{figure}[t]
% 	\centering
% 	%\includegraphics[clip,width=0.5\textwidth]{exp/title.pdf}
% 	\includegraphics[width=0.35\textwidth]{exp/title.pdf}\\
% 	\subfloat[Time]{\includegraphics[clip,width=0.245\textwidth]{exp/p-time-G.eps}\label{fig:exp_time_Gnutella}}
% 	\hspace{-5pt}
% 	\subfloat[Effect]{\includegraphics[clip,width=0.245\textwidth]{exp/p-effect-G.eps}\label{fig:exp_effect_Gnutella}}
% 	\caption{Varying $\beta/\alpha$ in Gnutella}
% 	\label{fig:exp_pra_t}
% \end{figure}

\noindent
\textbf{Varying the number of samples $X$.} The efficiency and effectiveness of each algorithm when the number of samples $X$ is varying is shown in Figure~\ref{fig:exp_pra_x}. In Figure~\ref{fig:x_time_Gnutella}, the running time of all algorithms increases almost linearly w.r.t. $X$, because all algorithms need to traverse all sampling random walks to calculate the marginal gains or the block degree. From Figure~\ref{fig:x_effect_Gnutella}, we can see that the effectiveness of all algorithms is increasing when the $X$ varies from $500$ to $2500$. But We find that when $X\ge1000$, the changing of effectiveness tends to be stable. Therefore, we choose $X$ = 1000 as the default setting because it reaches an
ideal balance of efficiency and effectiveness.

% \begin{figure}[t]
% 	\centering
% 	%\includegraphics[clip,width=0.5\textwidth]{exp/title.pdf}
% 	\includegraphics[width=0.35\textwidth]{exp/title.pdf}\\
% 	\subfloat[Time]{\includegraphics[clip,width=0.245\textwidth]{exp/x-time-G.eps}\label{fig:x_time_Gnutella}}
% 	\hspace{-5pt}
% 	\subfloat[Effect]{\includegraphics[clip,width=0.245\textwidth]{exp/x-effect-G.eps}\label{fig:x_effect_Gnutella}}
% 	\caption{Varying the number of samples $X$ in Gnutella}
% 	\label{fig:exp_pra_x}
% \end{figure}

\iffalse
\begin{figure*}[t]
	\centering
	%\includegraphics[clip,width=0.5\textwidth]{exp/title.pdf}
	\hspace{-10pt}
	\subfloat[Time]{\includegraphics[clip,width=0.345\textwidth]{exp/p-time-G.eps}\label{fig:exp_time_Gnutella}}
	\hspace{-10pt}
	\subfloat[Effect]{\includegraphics[clip,width=0.345\textwidth]{exp/p-effect-G.eps}\label{fig:exp_effect_Gnutella}}
	\hspace{-10pt}
	\subfloat[Effect]{\includegraphics[clip,width=0.345\textwidth]{exp/p-effect-G.eps}\label{fig:exp_effect_Gnutella}}
	\caption{Varying $\rho$ in Gnutella}
	\label{fig:exp_pra_t}
\end{figure*}
\fi

\noindent
\textbf{Varying $\rho$.} $\rho$ is used to adjust the step distance of decreasing threshold $h$ in Algorithm~4. Figure~\ref{fig:exp_pra_rho} shows the experimental results of varying $\rho$. When $\rho$ increases from 0.0001 to 1, our solutions decrease by at most 10\% in effectiveness but speed up by 1 order of magnitude.
We find that when $\rho\ge$ 0.1, the changing of effectiveness and efficiency tends to be stable. Therefore, we choose $\rho =$ 0.1 as the default setting.

% \begin{figure}[t]
% 	\centering
% 	%\includegraphics[clip,width=0.5\textwidth]{exp/title.pdf}
% 	\subfloat[Time]{\includegraphics[clip,width=0.245\textwidth]{exp/rho-time-G.eps}\label{fig:rho_time_Gnutella}}
% 	\hspace{-5pt}
% 	\subfloat[Effect]{\includegraphics[clip,width=0.245\textwidth]{exp/rho-effect-G.eps}\label{fig:rho_effect_Gnutella}}
% 	\caption{Varying $\rho$ in Gnutella}
% 	\label{fig:exp_pra_rho}
% \end{figure}

\begin{figure*}[t]
	%\hspace{0.8in}
        %\includegraphics[width=0.35\textwidth]{exp/title.pdf}
	\begin{minipage}[b]{0.48\textwidth} %{0.45\textheight}
		\centering
	%\includegraphics[clip,width=0.5\textwidth]{exp/title.pdf}
	\subfloat[Time]{\includegraphics[clip,width=0.5\textwidth]{exp/rho-time-G.eps}\label{fig:rho_time_Gnutella}}
	\hspace{-5pt}
	\subfloat[Effect]{\includegraphics[clip,width=0.5\textwidth]{exp/rho-effect-G.eps}\label{fig:rho_effect_Gnutella}}
	\caption{Varying $\rho$ in Gnutella}
	\label{fig:exp_pra_rho}
	\end{minipage}
	%\hspace{1in}
	\begin{minipage}[b]{0.48\textwidth}
		\centering
	\subfloat[Time]{\includegraphics[clip,width=0.5\textwidth]{exp/scal-time.eps}\label{fig:exp_scalability_time}}
	\hspace{-5pt}
	\subfloat[Memory]{\includegraphics[clip,width=0.5\textwidth]{exp/scal-mem.eps}\label{fig:exp_scalability_space}}
	\caption{Scalability test on Gowalla}
	\label{fig:exp_scalability}
	\end{minipage}
\end{figure*}