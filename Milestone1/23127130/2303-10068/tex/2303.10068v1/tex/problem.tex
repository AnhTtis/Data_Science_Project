
\section{Problem formulation}\label{formulation}
In this section, we first formally define the influence model and influence block. In the following, we give the formulation definition of \prob.  In the end, we show the non-submodularity of the objective function of \prob~and prove that \prob~is NP-hard. %Important notations used in our paper are presented in Table~\ref{table:notation}.
% \vspace{-0.2cm}
% \begin{table*}[t]
% 	\caption{{Notations for problem formulation and solutions}}
% 	\centering
% 	\begin{tabular}{lp{9.0cm}}
% 		\toprule
% 		{\bfseries Symbol} & {\bf Description}\\
% 		\midrule
% 		$G(V,E)$ & An online graph \\
% 		$\puv$ & the transition probability from neighbor node $u$ to node $v$\\
% 		$\wk_u$ & An instance of a random walk starting from $u$\\
% 		$T$     & A given threshold to bound the length of $\wk_{u}$\\
% 		$k$     & A given budget\\
% 		$R$, $P$    & $R \subset V$ is a rumor set, $P \subset V$ is a protector set, and $R \cap P=\phi$  \\
% 		$\ic_{\Wu}(\pr|\ru)$ &the total impressions of $P$ to block the influence of $R$ to $u$ in $\Wu$\\
% 		$I_{\Wu}(\pr|\ru)$ &The value of probability that $P$ blocks the influence of $R$ to $u$ for $\wk_u$\\
% 		$I_{u}(\pr|\ru)$  & $ I_{u}(\pr|\ru) = E[I_{\Wu}(\pr|\ru)]$ for any $\wk_u$\\ 
% 		$\object({\pr}|{\ru})$ & The object function of our problem (the block degree of $P$)\\
% 		\bottomrule
% 	\end{tabular}
% 	\label{table:notation}
% \end{table*}
\vspace{-0.3cm}
\subsection{Influence model}\vspace{-0.2cm}
Let $G = (V ,E)$ be an online network with $n = |V|$ nodes and $m = |E|$ edges. The random walk process can be used to model the user’s browsing process on $G$ as follows~\cite{spitzer2013principles,DBLP:journals/pvldb/MoBZP20a,DBLP:journals/www/ZhangBNZMGP19,DBLP:conf/nctcs/MoT0P19}. Given a node $u \in V$, a browsing process starting from $u$ can be represented by a random walk $\Wu$. In particular, $w_u$ picks a neighbor $v$ of $u$ by the probability of $\puv=1/|$neighbors of $v|$ and moves to this neighbor and then follows this way recursively. We say that $u$ hits $v$ at step $t$, if $\wk_u$ first visits $v$ after $t$ walk steps. 



%This process can be viewed as a Markov chain, and each element $\puv$ in the transition matrix $\pr$ is given by:
% \begin{equation}
% {p_{uv}} = \left\{ {\begin{array}{*{20}{l}}
% 	{1/d_u}	&	{\textrm{$u$ is a neighbour of $v$}}\\
% 	0			&	\textrm{otherwise}
% 	\end{array}} \right.
% \end{equation}
% where $p_{uv}$ is the transition probability from $u$ to $v$, and $d_u$ is the out-degree of~$u$.

%Specifically, given an arbitrary random walk $\wk_u=(n_0, n_1,...)$, where $(...)$ represents a sequence of nodes, and $u=n_0$ starting from $n_0$ and a node $v \in V$. We say that
% $u$ hits $v$ at step $t$, if $\wk_u$ first visits $v$ after $t$ walk steps. 
% We define that $\wk_{uv}$ is a hitting path of $u$ to $v$, if $\wk_{uv}=(n_0, n_1,...,n_k)$, $u=n_0$, and $v=n_k$, for $v \neq n_0, n_1...,n_{k-1}$. Let $\mathcal{W}_{uv}$ denote the set of all possible $\wk_{uv}$, the possibility of $\wk_{uv}$ being generated is computed by $\mathcal{P}(\wk_{uv})=\prod\nolimits_{i = 0}^{k - 1} 1/{{d_i}} $.

Similarly, we say that $u$ hits (or is influenced by) set $S$ at the time step $t$ if $\Wu$ first visits set $S$ by a $t$-hop jump. It is worth noting that $t$ should not be very large in the real world, as most social media users only browse a small number of pages each day. Therefore, we can use a threshold $T$ to bound the hitting time $t$ for any nodes and sets.

\subsection{Influence block}
Based on the influence model, we introduce the concept of influence block when impression counts as follows. 

Before that, we first introduce the conception of impression. For nodes $n_1$ and $n_2$ in a random walk $\Wu$, we define that $n_1$ have an impression of blocking the influence of $n_2$ to $u$ if $\Wu$ visits $n_1$ before $n_2$.
%w.r.t. $\Wu$ first visits $n_1$ and $n_2$ at the time step $t_1$ and $t_2$ respectively, and $t_1 < t_2$. 
Therefore, we use the Bernoulli random variable $\ic_{\Wu}(n_1|n_2)$ denoting the states whether $n_1$ have a impression of block the influence of $n_2$ to $u$, where $\ic_{\Wu}(n_1|n_2) = 1$ denotes that $n_1$ have a impression of block the influence of $n_2$ to $u$, otherwise $\ic_{\Wu}(n_1|n_2) = 0$. Then the total impressions of $P$ ($P \subset V$ is a protector set) to block the influence of $R$ ($R \subset V$ is a rumor set) to $u$ in $\Wu$ can be computed by $\ic_{\Wu}(P|R) = \sum\nolimits_{v \in \pr} {\ic_{\Wu}(v|n_r)}$. Here $n_r$ is the first node in $\Wu$, which is contained in the set $R$. 

Our influence block when impression counts are based on the logistic function. We use the following equation to compute the influence block of a protector set $P$ to a rumor set $R$ in $\Wu$: 
\begin{equation}
{\I_{\Wu}(P|R)} = \left\{ {\begin{array}{*{20}{l}}
	{\frac{1}{1+exp\{\alpha-\beta\cdot\ic_{\Wu}(P|R)\}}}	&	{\textrm{if $\ic_{\Wu}(P|R) > 0$}}\\
	0			&	\textrm{otherwise}
	\end{array}} \right.
\end{equation}
Here $\alpha$ and $\beta$ are the parameters that control the turning point of the user $u$ for being influenced by the protect information, where $\alpha$ controls the overall effectiveness of the influence block of $P$ to $R$ and $\beta$ controls the incremental effectiveness of influence block of one node in $P$ to $R$ in $\Wu$.
Then, let $ \I_{u}(\pr|\ru) = E[\I_{\Wu}(\pr|\ru)]$ for any $\wk_u$ denote the expected value of possibility that $P$ blocks the influence of $R$ to $u$.


\subsection{Problem definition}\label{definition}
Based on $ \I_{u}(\pr|\ru)$, the problem of Rumor Control when Impression Counts (\prob) can be described as follows.


\begin{defn}[Problem Definition]
	Given a graph $G = (V,E)$, an initial set $\ru \subset V$ and a budget $k$, $\prob$ is dedicated to finding a $k$-size set $\pr \subset V\backslash \ru$, which can maximize the influence block $\object({\pr}|{\ru}) = \sum\nolimits_{u \in V\backslash {\ru}} {{\I_u}} ({\pr}|{\ru})$.
\end{defn}

Next, we analyze the monotonicity and submodularity of $\object({\pr}|{\ru}) $ and the hardness of \prob.
%Given two sets of billboards S1 and S2, the marginal influence of adding S2 into S1 is Δ(S2 |S1) = I (S1 ∪ S2)−I (S1). Then, we define the monotonicity and submodularity of an influence function as follows. I (S) is monotone iff, I (S1) ≤ I (S2) for all S1 ⊆ S2. Furthermore, I (S) is submodular iff, given any set of billboards S∗, it satisfies Δ(S∗ |S1) ≥ Δ(S∗ |S2) for all S1 ⊆ S2. 
%The following presents a counterexample for the objective function to be submodular.

\begin{defn}
	We say that $\object({\pr}|{\ru}) $ is monotone iff, for any two assignment protector sets $P^a$ and $P^b$ such that $P^a\subseteq P^b$, it holds that $\object({\pr}^a|{\ru})\le\object({\pr}^b|{\ru})$. We say that $\object({\pr}|{\ru}) $ is submodular iff, for any two such protector sets and any  $P$, it has $\object({\pr}^a\cup P|{\ru}) - \object({\pr}^a|{\ru})\ge\object({\pr}^b\cup P|{\ru})-\object({\pr}^b|{\ru})$.
\end{defn}

It is trivial to show that $\object({\pr}|{\ru}) $ is monotone. However, as the following counterexample shows, $\object({\pr}|{\ru}) $ is not submodular.


\begin{figure}[t]
		\centering
		{\includegraphics[width=0.4\textwidth]{fig/example.pdf}}
		\caption{An example of RCIC}
		\label{fig:example}
\end{figure}

\begin{figure}[t]
	
		\centering
		{\includegraphics[width=0.4\textwidth]{fig/func.pdf}}
		\caption{The upper-bound influence block function}
		\label{fig:func}
\end{figure}

\begin{example}\label{non-sub}
	As shown in Figure~\ref{fig:example}, the rumor set $R=\{v4\}$. We choose $P^a=\{\}$, $P^b=\{v1\}$, $P=\{v2\}$, $\alpha = 3$, $\beta = 1$ and $T=2$. Then we have $\object(P^a|{\ru})=0$, $\object(P^b|{\ru})=1.372$ and $\object(P|{\ru})=0.358$. Furthermore, we have $\object({\pr}^a\cup P|{\ru}) - \object({\pr}^a|{\ru})=0.358$ and $\object({\pr}^b\cup P|{\ru})-\object({\pr}^b|{\ru}) = 2.433 -  1.372 = 1.061$. Since $P^a\subseteq P^b$ and $\object({\pr}^a\cup P|{\ru}) - \object({\pr}^a|{\ru})\le\object({\pr}^b\cup P|{\ru})-\object({\pr}^b|{\ru})$. We thus conclude $\object({\pr}|{\ru}) $ is not submodular.
\end{example}

\begin{thm}\label{NP-hardness}
	The $\prob$ problem is NP-hard.
\end{thm}
$Proof.$ We prove it by reducing the Set Cover problem to the $\prob$ problem. In the Set Cover problem, given a collection of subsets $S_1, S_2, ..., S_n$ of a universe of elements $U$ = $\{u_1,u_2,...,u_j\}$, we wish to know whether there exist $k$ of the subsets whose union is equal to $U$.
% %Consider an instance of the NP-complete Set Cover problem, defined by a collection of subsets $S_1, S2, ..., S_n$ of a ground set $U$ = $\{u_1,u_2,...,u_j\}$; we wish to know whether there exist $k$ of the subsets whose union is equal to $U$. (We can assume that $k < n$.) We show that this can be viewed as a special case of the weighted random walk domination problem.
We define a corresponding graph $G(V,E)$ with $n$ nodes. Each node in graph $G$ has $d$ edges connected. We set $\alpha$ as 0 and $\beta$ as $\infty$. Then we have ${\I_{\Wu}(P|R)} = 1$, if $\ic_{\Wu}(P|R) > 0$.


Given a rumor set $R$, we map a subset $S_i$ to a node $i$ in $V\backslash R$. Next, we generate all possible random walk instances with a total length equal to $T$ as a universe of elements $|U_0|$. Intuitively, $|U_0|=|V\backslash R|*d^T$. Then we map an element $u_j$ to a random walk instance $j$ in $U_0$ and $S_i$ contains $u_j$ when the random walk instance $j$ visits the node $i$ before visiting rumor set $R$. The Set Cover problem is equivalent to deciding whether there is a set $S$ of $k$ nodes in graph $G$ with $\object({\pr}|{\ru}) = |V\backslash R|$. As the set cover problem is NP-complete, the decision problem of \prob~is NP-complete, and the optimization problem is NP-hard. $\blacksquare$
