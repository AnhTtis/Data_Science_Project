\section{Experiments}\label{sec_5}
In this section, we present our experimental results on the effectiveness, efficiency, memory consumption, and scalability of our proposed methods.
\subsection{Experimental settings}
\noindent
\textbf{DataSets.} We use three real-world datasets in the experiments: Gnutella, Email-Enron, and Gowalla. All the datasets are obtained from an open-source website\footnote{http://snap.stanford.edu/data/}, and
their statistics are shown in Table~\ref{tab_datasets}. The Gnutella dataset is a peer-to-peer file-sharing network, the Email-Enron dataset is an email communication network, and the Gowalla dataset is a location-based social networking website where users share their locations by checking in.

\begin{table}[t]	
		\centering
	\caption{Parameter setting.}
	\begin{small}
		\label{exp_param}
		\begin{tabular}[r]{|p{1.8cm}<{\centering}|p{4cm}<{\centering}|}
			\hline
			\multicolumn{1}{|c|}{Parameters}                                  & \multicolumn{1}{c|}{Values} \\ \hline
			$k$                                                       &	50, 100, \textbf{150}, 200, 250\\ \hline
			$|R|$                                                             &	50, 100, \textbf{150}, 200, 250\\ \hline
			$T$                                                               &	3, 6, \textbf{9}, 12, 15    \\ \hline
			$\beta/\alpha$                                                               &	\textbf{3/7}, 3/8, 3/9, 3/10, 3/11    \\ \hline
			$X$                                                               &	500, \textbf{1000}, 1500, 2000, 2500    \\ \hline
			$\rho$                                                               &	0.0001, 0.001, 0.01, \textbf{0.1}, 1    \\ \hline
		\end{tabular}
	\end{small}
\end{table}

\begin{table}[t]
	\caption{Summary of the datasets.}
	\centering
	\begin{small}
		\label{tab_datasets}
		\begin{tabular}[c]{|c|c|c|c|c|}
			\hline
			&$n$			& $m$		     &\#AvgDegree       &\#MaxDegree 	 \\ \hline
			Gnutella    	&8.8k     	 	&63k      	 &7.2    		    &88   		\\ \hline
			Email-Enron    	&37k     		&184k       	 &5.01   		    &1383  		 	 \\ \hline
			Gowalla     	&197k     	&950k          &4.83      	    &14730   	   \\ \hline    	
		\end{tabular}
	\end{small}
\end{table}

\noindent
\textbf{Algorithms.} To the best of our knowledge, this is the first work to study \prob, and thus there exists no previous work for direct comparison. In particular, we compare the four following methods. (1) \degreeTop: It is to select the top-$k$ high block degree nodes in the sampling random walk set as the targeted nodes. (2) \samgreedy: A basic sampling-based greedy algorithm~(Algorithm \ref{al_samgreedy}). (3) \bab~(BAB): The branch-and-bound framework (Algorithm \ref{al_bab}) with Algorithm~\ref{al_sbound} for bound estimations. (4) Progressive \bab~(ProBAB): The branch-and-bound framework (Algorithm \ref{al_bab}) with Algorithm~\ref{al_psbound} for bound estimations.

% \begin{itemize}
% 	\item \degreeTop: It is to select the top-$k$ high block degree nodes in sampling random walk set as the targeted nodes.
% 	\item \samgreedy: A basic sampling-based greedy algorithm~(Algorithm \ref{al_samgreedy}). In each iteration, it selects the node $u$ which maximizes the marginal gain  $(\object({\pr} \cup \{ u \}|{\ru}) - \object({\pr}|{\ru}))$ to a candidate solution set $\pr$, until the budget $k$ is exhausted.
% 	\item \bab~(BAB): The branch-and-bound framework (Algorithm \ref{al_bab}) with Algorithm~\ref{al_sbound} for bound estimations.
% 	\item Progressive \bab~(ProBAB): The branch-and-bound framework (Algorithm \ref{al_bab}) with Algorithm~\ref{al_psbound} for bound estimations.
% \end{itemize}

%\degreeTop, \samgreedy~(Algorithm \ref{al_samgreedy}), and \bab~(BAB for short, Algorithm \ref{al_bab}). 

\noindent
\textbf{Evaluation metrics.} We evaluate the performance of all methods by the runtime and the blocking percentage of the selected nodes. In particular, the percentage is computed by $\object({\pr}|{\ru})/\IS(\ru)$, where $\IS(\ru)$ denote the random walk set influenced by rumor set $\ru$. %Moreover, to accurately measure $\object({\pr}|{\ru})$ for each algorithm, we compute it by running a sufficient number of MC simulations, i.e., $X = 1000$.


\noindent
\textbf{Parameter.} Table \ref{exp_param} shows the settings of all parameters, such as the budget $k$, the size of the rumor set $R$, the (random walk) length threshold $T$, the number of samples $X$, the parameter $\alpha$, the parameter $\beta$ and parameter $\rho$.
Here the default one is highlighted in bold.
To simulate the rumor set $R$, we select nodes uniformly at random from the nodes whose degrees are in the top 10\% of $G$.

% \begin{table}[t]
% 	\caption{Parameter setting.}
% 	\centering
% 	\begin{small}
% 		\label{exp_param}
% 		\begin{tabular}[r]{|p{1.8cm}<{\centering}|p{4cm}<{\centering}|}
% 			\hline
% 			\multicolumn{1}{|c|}{Parameters}                                  & \multicolumn{1}{c|}{Values} \\ \hline
% 			$k$                                                       &	50, 100, \textbf{150}, 200, 250\\ \hline
% 			$|R|$                                                             &	50, 100, \textbf{150}, 200, 250\\ \hline
% 			$T$                                                               &	3, 6, \textbf{9}, 12, 15    \\ \hline
% 			%$\beta/\alpha$                                                               &	\textbf{3/7}, 3/8, 3/9, 3/10, 3/11    \\ \hline
% 			%$X$                                                               &	500, \textbf{1000}, 1500, 2000, 2500    \\ \hline
% 			%$\rho$                                                               &	0.0001, 0.001, 0.01, \textbf{0.1}, 1    \\ \hline
% 		\end{tabular}
% 	\end{small}
% \end{table}

\noindent
\textbf{Setup.} All codes are implemented in Java, and experiments are conducted on a server with 2.1 GHz Intel Xeon 8 Core CPU and 32GB memory running CentOS/6.8 OS.
\vspace{-0.3cm}\subsection{Effectiveness test}\label{sec_effectiveness}\vspace{-0.2cm}
This section studies how the block degree is affected by varying the budget $k$, the size of the rumor set $R$, and the length threshold $T$ of a random walk.

\noindent
\textbf{Varying the budget $k$.} The block degrees of all algorithms on Gnutella and Email-Enron by varying the $k$ are shown in Figure \ref{fig:exp_ef_Gnutella_B} and Figure \ref{fig:exp_ef_Email_B}, respectively, and we find that when the budget raises from 50 to 250,
BAB outperforms Greedy and TopK by up to 115\% in the Email-Enron.

\noindent
\textbf{Varying the size of $R$.} Figure \ref{fig:exp_ef_Gnutella_R} and Figure \ref{fig:exp_ef_Email_R}  show the result by varying the size of $R$. We find: (1) with the growth of $|R|$, the blocking percentages of all methods are increasing because the increasing influence of $R$ leads to more nodes with higher unit block degrees. (2) ProBAB and BAB are consistently better than that of the rest baselines.

\noindent
\textbf{Varying the random walk length threshold $T$.} Figure \ref{fig:exp_ef_Gnutella_T} and Figure \ref{fig:exp_ef_Email_T}  show the results by varying the threshold $T$, which determines the length of a random walk starting from a node. We observe that: (1) The rumors on Gnutella dataset are much harder to be controlled than Email-Enron dataset. It implies that the network structure is an important variable for \prob. (2) With the increase of $T$, the performance of all algorithms becomes better. The reason is that when the length becomes large, the random walk has more chances to reach the protectors and thus leads to a high unit block degree of the seeds.

%(1) \samgreedy~and \indexgreedy~achieve the same block degree, while \degreeTop~has the worst performance. \samgreedy~and \indexgreedy~are little better than \samgreedyplus~and \indexgreedyplus. (2) With the growth of $B$, the advantage of \samgreedy~and \indexgreedy~over \degreeTop~decreases, from 49\% to 20\% when $B$ varies from 50 to 250 on Email-Enron dataset. This is because that \degreeTop~ignores the blocking overlaps between the selected seeds, which declines its block degree. However, when $B$ is large, the performance of \degreeTop~catches up with the others since the overlaps are inevitable.

% \begin{figure*}
%         \begin{minipage}[b]{0.47\textwidth}
% 	\centering
% 	%\includegraphics[width=1\textwidth]{exp/title5.pdf}
% 	%\vspace{-0.3cm}
% 	%\hspace{-10pt}
% 	\subfloat[Varying $k$]
% 	{\label{fig:exp_ef_Gnutella_B}\includegraphics[width=0.33\textwidth]{exp/k-effect-G.eps}}
% 	%\hspace{-10pt}
% 	\subfloat[Varying $|R|$]
% 	{\label{fig:exp_ef_Gnutella_R}\includegraphics[width=0.33\textwidth]{exp/R-effect-G.eps}}
% 	%\hspace{-10pt}
% 	\subfloat[Varying $T$]
% 	{\label{fig:exp_ef_Gnutella_T}\includegraphics[width=0.33\textwidth]{exp/L-effect-G.eps}}
% 	\caption{Effectiveness test on Gnutella}
% 	\label{fig:para}
% 	\end{minipage}
% 	%\hspace{1in}
% 	\begin{minipage}[b]{0.47\textwidth}
% 	\centering
% 	%\includegraphics[width=1\textwidth]{exp/title5.pdf}
% 	%\vspace{-0.3cm}
% 	%\hspace{-10pt}
% 	\subfloat[Varying $k$]
% 	{\label{fig:exp_ef_Email_B}\includegraphics[width=0.33\textwidth]{exp/k-effect-E.eps}}
% 	%\hspace{-10pt}
% 	\subfloat[Varying $|R|$]
% 	{\label{fig:exp_ef_Email_R}\includegraphics[width=0.33\textwidth]{exp/R-effect-E.eps}}
% 	%\hspace{-10pt}
% 	\subfloat[Varying $T$]
% 	{\label{fig:exp_ef_Email_T}\includegraphics[width=0.33\textwidth]{exp/L-effect-E.eps}}
% 	\caption{Effectiveness test on Email-Enron}
% 	\label{fig:para}
% 	\end{minipage}
% 	\vspace{-0.5cm}
% \end{figure*}

\begin{figure*}[t]
	\centering
	%\includegraphics[width=1\textwidth]{exp/title5.pdf}
	%\vspace{-0.3cm}
	%\hspace{-10pt}
	\subfloat[Varying $k$]
	{\label{fig:exp_ef_Gnutella_B}\includegraphics[width=0.3\textwidth]{exp/k-effect-G.eps}}
	%\hspace{-10pt}
	\subfloat[Varying $|R|$]
	{\label{fig:exp_ef_Gnutella_R}\includegraphics[width=0.3\textwidth]{exp/R-effect-G.eps}}
	%\hspace{-10pt}
	\subfloat[Varying $T$]
	{\label{fig:exp_ef_Gnutella_T}\includegraphics[width=0.3\textwidth]{exp/L-effect-G.eps}}
	\caption{Effectiveness test on Gnutella}
	\label{fig:para}
\end{figure*}
\begin{figure*}[t]
	\centering
	%\includegraphics[width=1\textwidth]{exp/title5.pdf}
	%\vspace{-0.3cm}
	%\hspace{-10pt}
	\subfloat[Varying $k$]
	{\label{fig:exp_ef_Email_B}\includegraphics[width=0.3\textwidth]{exp/k-effect-E.eps}}
	%\hspace{-10pt}
	\subfloat[Varying $|R|$]
	{\label{fig:exp_ef_Email_R}\includegraphics[width=0.3\textwidth]{exp/R-effect-E.eps}}
	%\hspace{-10pt}
	\subfloat[Varying $T$]
	{\label{fig:exp_ef_Email_T}\includegraphics[width=0.3\textwidth]{exp/L-effect-E.eps}}
	\caption{Effectiveness test on Email-Enron}
	\label{fig:para}
\end{figure*}


\iffalse
\begin{figure}[t]
	\centering
	\subfloat[Gnutella]{\includegraphics[clip,width=0.245\textwidth]{exp/k-effect-G.eps}\label{fig:exp_ef_Gnutella_B}}
	\hspace{-5pt}
	\subfloat[Email-Enron]{\includegraphics[clip,width=0.245\textwidth]{exp/k-effect-E.eps}\label{fig:exp_ef_Email_B}}
	\caption{Effectiveness of varying the budget $k$}
	\label{fig:exp_ef_B}
\end{figure}
\fi


\iffalse
\begin{figure}[t]
	\centering
	\subfloat[Gnutella]{\includegraphics[clip,width=0.245\textwidth]{exp/R-effect-G.eps}\label{fig:exp_ef_Gnutella_R}}
	\hspace{-5pt}
	\subfloat[Email-Enron]{\includegraphics[clip,width=0.245\textwidth]{exp/R-effect-E.eps}\label{fig:exp_ef_Email_R}}
	\caption{Effectiveness of varying the size of $R$}
	\label{fig:exp_ef_R}
\end{figure}
\fi

\iffalse
\begin{figure}[t]
	\centering
	\subfloat[Gnutella]{\includegraphics[clip,width=0.245\textwidth]{exp/L-effect-G.eps}\label{fig:exp_ef_Gnutella_T}}
	\hspace{-5pt}
	\subfloat[Email-Enron]{\includegraphics[clip,width=0.245\textwidth]{exp/L-effect-E.eps}\label{fig:exp_ef_Email_T}}
	\caption{Effectiveness of varying the random walk length threshold $T$}
	\label{fig:exp_ef_t}
\end{figure}
\fi

\subsection{Efficiency test}\label{sec_efficiency}
We evaluate the efficiency of different algorithms on Gnutella and Email-Enron datasets. 

\begin{figure*}[t]
	\centering
	\includegraphics[width=0.5\textwidth]{exp/title.pdf}\\
	%\includegraphics[width=1\textwidth]{exp/title5.pdf}
	%\hspace{-10pt}
	\subfloat[Varying $k$]
	{\includegraphics[clip,width=0.3\textwidth]{exp/k-time-G.eps}\label{fig:exp_time_Gnutella_B}}
	%\hspace{-10pt}
	\subfloat[Varying $|R|$]
	{\includegraphics[clip,width=0.3\textwidth]{exp/R-time-G.eps}\label{fig:exp_time_Gnutella_R}}
	%\hspace{-10pt}
	\subfloat[Varying $T$]
	{\includegraphics[clip,width=0.3\textwidth]{exp/L-time-G.eps}\label{fig:exp_time_Gnutella_T}}
	\caption{Efficiency test on Gnutella}
	\label{fig:para}
\end{figure*}
\begin{figure*}[t]
	\centering
	%\includegraphics[width=0.5\textwidth]{exp/title.pdf}\\
	%\hspace{-10pt}
	\subfloat[Varying $k$]
	{\includegraphics[clip,width=0.3\textwidth]{exp/k-time-E.eps}\label{fig:exp_time_Email_B}}
	%\hspace{-10pt}
	\subfloat[Varying $|R|$]
	{\includegraphics[clip,width=0.3\textwidth]{exp/R-time-E.eps}\label{fig:exp_time_Email_R}}
	%\hspace{-10pt}
	\subfloat[Varying $T$]
	{\includegraphics[clip,width=0.3\textwidth]{exp/L-time-E.eps}\label{fig:exp_time_Email_T}}
	\caption{Efficiency test on Email-Enron}
	\label{fig:para}
\end{figure*}

\iffalse
\begin{figure}[t]
	\centering
	\includegraphics[clip,width=0.3\textwidth]{exp/title.pdf}\hspace{100pt}
	\subfloat[Gnutella]{\includegraphics[clip,width=0.245\textwidth]{exp/k-time-G.eps}\label{fig:exp_time_Gnutella_B}}
	%\hspace{-5pt}
	\subfloat[Email-Enron]{\includegraphics[clip,width=0.245\textwidth]{exp/k-time-E.eps}\label{fig:exp_time_Email_B}}
	\caption{Efficiency of varying the budget $k$}
	\label{fig:exp_time_k}
\end{figure}
\fi

\noindent
\textbf{Varying the budget $k$.} Figure \ref{fig:exp_time_Gnutella_B}  and Figure \ref{fig:exp_time_Email_B} present the efficiency result when $k$ varies from 50 to 250. We have the following observations. (1) The performance of \samgreedy~and ProBAB is about 2 and 1 orders of magnitude faster than BAB, respectively. (2) The runtime of all methods except \degreeTop~is slowly increasing with the growth of $k$. This is because the increase of $k$ directly causes selecting more nodes to $P$, which leads to an increase in the number of updating the influence block of the remaining node.

\noindent
\textbf{Varying the size of $R$.} Figure \ref{fig:exp_time_Gnutella_R}  and Figure \ref{fig:exp_time_Email_R} show the runtime of all algorithms on Gnutella and Email-Enron, respectively. We can see that the runtime of all methods except \degreeTop~is also slowly increasing when $|R|$ varies from 50 to 250 on all datasets. This is because the influence set $\IS(\ru)$ of $\ru$ is increasing with the growth of $|R|$.

\iffalse
\begin{figure}[t]
	\centering
	%\includegraphics[clip,width=0.5\textwidth]{exp/title.pdf}
	\includegraphics[width=0.5\textwidth]{exp/title.pdf}\\
	\subfloat[Gnutella]{\includegraphics[clip,width=0.245\textwidth]{exp/R-time-G.eps}\label{fig:exp_time_Gnutella_R}}
	\hspace{-5pt}
	\subfloat[Email-Enron]{\includegraphics[clip,width=0.245\textwidth]{exp/R-time-E.eps}\label{fig:exp_time_Email_R}}
	\caption{Efficiency of varying the size of $R$}
	\label{fig:exp_time_r}
\end{figure}
\fi
\noindent
\textbf{Varying the random walk length threshold $T$.} We evaluate the efficiencies of algorithms by varying $T$ from 3 to 15. The result is shown in Figure \ref{fig:exp_time_Gnutella_T} and Figure \ref{fig:exp_time_Email_T}. We can see that all the algorithms except for \degreeTop~scale linearly with respect to $T$, which is because they need to scan more nodes to compute the influence block in each random walk .

\iffalse
\begin{figure}[t]
	\centering
	\includegraphics[clip,width=0.3\textwidth]{exp/title.pdf}\hspace{100pt}
	\subfloat[Gnutella]{\includegraphics[clip,width=0.245\textwidth]{exp/L-time-G.eps}\label{fig:exp_time_Gnutella_T}}
	\hspace{-5pt}
	\subfloat[Email-Enron]{\includegraphics[clip,width=0.245\textwidth]{exp/L-time-E.eps}\label{fig:exp_time_Email_T}}
	\caption{Efficiency of varying the random walk length $T$}
	\label{fig:exp_time_t}
\end{figure}
\fi

\subsection{Parameter sensitive test }\label{sec_Parameter} 
\noindent
\textbf{Varying $\beta/\alpha$.} Figure~\ref{fig:exp_pra_t} reports the efficiency and effectiveness of each algorithm when $\beta/\alpha$ is varying. As shown in Figure~\ref{fig:exp_time_Gnutella}, the varying of $\beta/\alpha$ has no impact on the running time of all algorithms. But from Figure~\ref{fig:exp_effect_Gnutella}, we find that the effectiveness of all algorithms is decreasing when the $\beta/\alpha$ varies from $3/7$ to $3/11$. This is because the smaller the $\beta/\alpha$ is, the more times of impression are needed to change a user’s adoption. In particular, with the decrease of $\beta/\alpha$, our solutions outperform \samgreedy~by
70\% to 217\%. Therefore, we choose $\alpha$ = 7 and $\beta$ = 3 as the default setting since our solutions have the smallest advantage of effectiveness for the setting.

%we find: (1) \degreeTop, \samgreedy~and \indexgreedy~have similar memory consumption. It is because they all need to scan the sampling set to calculate the block degree of the nodes. (2) When $T$ is small, the memory consumption of \samgreedyplus~and \indexgreedyplus~is much smaller than that of other methods. This is because only a small number of nodes can reach $R$ when $T$ is small. Therefore, we can reduce the sampling set by eliminating more nodes in pre-sampling stage. But with the growth of $T$, most nodes will reach $R$ in pre-sampling stage, which causes the increasing of the memory consumption of \samgreedyplus~and \indexgreedyplus.

\begin{figure*}[t]
	%\hspace{0.8in}
        %\includegraphics[width=0.35\textwidth]{exp/title.pdf}
	\begin{minipage}[b]{0.48\textwidth} %{0.45\textheight}
		\centering
	%\includegraphics[clip,width=0.5\textwidth]{exp/title.pdf}
	%\includegraphics[width=0.35\textwidth]{exp/title.pdf}\\
	\subfloat[Time]{\includegraphics[clip,width=0.5\textwidth]{exp/p-time-G.eps}\label{fig:exp_time_Gnutella}}
	\hspace{-5pt}
	\subfloat[Effect]{\includegraphics[clip,width=0.5\textwidth]{exp/p-effect-G.eps}\label{fig:exp_effect_Gnutella}}
	\caption{Varying $\beta/\alpha$ in Gnutella}
	\label{fig:exp_pra_t}
	\end{minipage}
	%\hspace{1in}
	\begin{minipage}[b]{0.48\textwidth}
		\centering
	%\includegraphics[clip,width=0.5\textwidth]{exp/title.pdf}
	%\includegraphics[width=0.35\textwidth]{exp/title.pdf}\\
	\subfloat[Time]{\includegraphics[clip,width=0.5\textwidth]{exp/x-time-G.eps}\label{fig:x_time_Gnutella}}
	\hspace{-5pt}
	\subfloat[Effect]{\includegraphics[clip,width=0.5\textwidth]{exp/x-effect-G.eps}\label{fig:x_effect_Gnutella}}
	%\caption{Varying the number of samples $X$ in Gnutella}
        \caption{Varying $X$ in Gnutella}
	\label{fig:exp_pra_x}
	\end{minipage}
\end{figure*}

% \begin{figure}[t]
% 	\centering
% 	%\includegraphics[clip,width=0.5\textwidth]{exp/title.pdf}
% 	\includegraphics[width=0.35\textwidth]{exp/title.pdf}\\
% 	\subfloat[Time]{\includegraphics[clip,width=0.245\textwidth]{exp/p-time-G.eps}\label{fig:exp_time_Gnutella}}
% 	\hspace{-5pt}
% 	\subfloat[Effect]{\includegraphics[clip,width=0.245\textwidth]{exp/p-effect-G.eps}\label{fig:exp_effect_Gnutella}}
% 	\caption{Varying $\beta/\alpha$ in Gnutella}
% 	\label{fig:exp_pra_t}
% \end{figure}

\noindent
\textbf{Varying the number of samples $X$.} The efficiency and effectiveness of each algorithm when the number of samples $X$ is varying is shown in Figure~\ref{fig:exp_pra_x}. In Figure~\ref{fig:x_time_Gnutella}, the running time of all algorithms increases almost linearly w.r.t. $X$, because all algorithms need to traverse all sampling random walks to calculate the marginal gains or the block degree. From Figure~\ref{fig:x_effect_Gnutella}, we can see that the effectiveness of all algorithms is increasing when the $X$ varies from $500$ to $2500$. But We find that when $X\ge1000$, the changing of effectiveness tends to be stable. Therefore, we choose $X$ = 1000 as the default setting because it reaches an
ideal balance of efficiency and effectiveness.

% \begin{figure}[t]
% 	\centering
% 	%\includegraphics[clip,width=0.5\textwidth]{exp/title.pdf}
% 	\includegraphics[width=0.35\textwidth]{exp/title.pdf}\\
% 	\subfloat[Time]{\includegraphics[clip,width=0.245\textwidth]{exp/x-time-G.eps}\label{fig:x_time_Gnutella}}
% 	\hspace{-5pt}
% 	\subfloat[Effect]{\includegraphics[clip,width=0.245\textwidth]{exp/x-effect-G.eps}\label{fig:x_effect_Gnutella}}
% 	\caption{Varying the number of samples $X$ in Gnutella}
% 	\label{fig:exp_pra_x}
% \end{figure}

\iffalse
\begin{figure*}[t]
	\centering
	%\includegraphics[clip,width=0.5\textwidth]{exp/title.pdf}
	\hspace{-10pt}
	\subfloat[Time]{\includegraphics[clip,width=0.345\textwidth]{exp/p-time-G.eps}\label{fig:exp_time_Gnutella}}
	\hspace{-10pt}
	\subfloat[Effect]{\includegraphics[clip,width=0.345\textwidth]{exp/p-effect-G.eps}\label{fig:exp_effect_Gnutella}}
	\hspace{-10pt}
	\subfloat[Effect]{\includegraphics[clip,width=0.345\textwidth]{exp/p-effect-G.eps}\label{fig:exp_effect_Gnutella}}
	\caption{Varying $\rho$ in Gnutella}
	\label{fig:exp_pra_t}
\end{figure*}
\fi

\noindent
\textbf{Varying $\rho$.} $\rho$ is used to adjust the step distance of decreasing threshold $h$ in Algorithm~4. Figure~\ref{fig:exp_pra_rho} shows the experimental results of varying $\rho$. When $\rho$ increases from 0.0001 to 1, our solutions decrease by at most 10\% in effectiveness but speed up by 1 order of magnitude.
We find that when $\rho\ge$ 0.1, the changing of effectiveness and efficiency tends to be stable. Therefore, we choose $\rho =$ 0.1 as the default setting.

% \begin{figure}[t]
% 	\centering
% 	%\includegraphics[clip,width=0.5\textwidth]{exp/title.pdf}
% 	\subfloat[Time]{\includegraphics[clip,width=0.245\textwidth]{exp/rho-time-G.eps}\label{fig:rho_time_Gnutella}}
% 	\hspace{-5pt}
% 	\subfloat[Effect]{\includegraphics[clip,width=0.245\textwidth]{exp/rho-effect-G.eps}\label{fig:rho_effect_Gnutella}}
% 	\caption{Varying $\rho$ in Gnutella}
% 	\label{fig:exp_pra_rho}
% \end{figure}

\begin{figure*}[t]
	%\hspace{0.8in}
        %\includegraphics[width=0.35\textwidth]{exp/title.pdf}
	\begin{minipage}[b]{0.48\textwidth} %{0.45\textheight}
		\centering
	%\includegraphics[clip,width=0.5\textwidth]{exp/title.pdf}
	\subfloat[Time]{\includegraphics[clip,width=0.5\textwidth]{exp/rho-time-G.eps}\label{fig:rho_time_Gnutella}}
	\hspace{-5pt}
	\subfloat[Effect]{\includegraphics[clip,width=0.5\textwidth]{exp/rho-effect-G.eps}\label{fig:rho_effect_Gnutella}}
	\caption{Varying $\rho$ in Gnutella}
	\label{fig:exp_pra_rho}
	\end{minipage}
	%\hspace{1in}
	\begin{minipage}[b]{0.48\textwidth}
		\centering
	\subfloat[Time]{\includegraphics[clip,width=0.5\textwidth]{exp/scal-time.eps}\label{fig:exp_scalability_time}}
	\hspace{-5pt}
	\subfloat[Memory]{\includegraphics[clip,width=0.5\textwidth]{exp/scal-mem.eps}\label{fig:exp_scalability_space}}
	\caption{Scalability test on Gowalla}
	\label{fig:exp_scalability}
	\end{minipage}
\end{figure*}
\subsection{Scalability test}\label{sec_scalability}
This experiment is to evaluate the scalability of \samgreedy~and BAB~when we increase the network size. To vary the network size, we partition Gowalla dataset into five subgraphs, and each of them covers 20\% nodes of the dataset. To avoid smashing the network into pieces, each subgraph is generated by a breadth-first traversal process. %It is worth noting that if one method has more than 25G of memory consumption or runs more than 3000s, we will omit it. 
Figure \ref{fig:exp_scalability} shows the result, and we have the following observations. (1) The performance of \samgreedy~and ProBAB is about 2 and 1 orders of magnitude faster than BAB, respectively. (2) When the graph size is increasing, the memory consumption of \samgreedy, BAB and ProBAB is increasing slowly but no more than 25GB.


%The run time of \samgreedy~and \indexgreedy~increases faster than that of \samgreedyplus~and \indexgreedyplus, with the growth of graph size. This is because the runtime of \samgreedy~and \indexgreedy increases with respect to $n^2$, while the runtime of \samgreedyplus~and \indexgreedyplus increases with respect to $n\log n$. (2) When the graph size is increasing, the memory consumption of \degreeTop, \samgreedy~and \indexgreedy~also increases faster than that of \samgreedyplus~and \indexgreedyplus. As we analyzed in Section~\ref{sec_4:pre-sampling}, \presample~can eliminate some node, which can't reach rumor set $\ru$. Therefore, \presample~can slow down the growth of the memory consumption when the graph size is increasing.


% \begin{wrapfigure}{r}{.5\textwidth}
% 	%\centering
% 	\subfloat[Time]{\includegraphics[clip,width=0.245\textwidth]{exp/scal-time.eps}\label{fig:exp_scalability_time}}
% 	\hspace{-5pt}
% 	\subfloat[Memory]{\includegraphics[clip,width=0.245\textwidth]{exp/scal-mem.eps}\label{fig:exp_scalability_space}}
% 	\caption{Scalability test on Gowalla dataset}
% 	\label{fig:exp_scalability}
% \end{wrapfigure}

\begin{figure}[h]
	\centering%\vspace{-0.8cm}
	\subfloat[Time]{\includegraphics[clip,width=0.3\textwidth]{exp/scal-time.eps}\label{fig:exp_scalability_time}}
	\hspace{-5pt}
	\subfloat[Memory]{\includegraphics[clip,width=0.3\textwidth]{exp/scal-mem.eps}\label{fig:exp_scalability_space}}
	\caption{Scalability test on Gowalla dataset}
	\label{fig:exp_scalability}
\end{figure}



