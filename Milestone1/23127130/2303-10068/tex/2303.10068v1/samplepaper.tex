% This is samplepaper.tex, a sample chapter demonstrating the
% LLNCS macro package for Springer Computer Science proceedings;
% Version 2.20 of 2017/10/04
%
\documentclass[runningheads]{llncs}
%
\usepackage{graphicx}

\usepackage{cite}
\usepackage{amsmath,amssymb,amsfonts}
\usepackage{algorithmic}
\usepackage{graphicx}
\usepackage{textcomp}
\usepackage{xcolor}
\usepackage{graphicx}
\usepackage[linesnumbered,ruled,algonl,vlined,noend]{algorithm2e} 
\usepackage{graphicx}
\usepackage{tabularx}
\usepackage{booktabs}
\usepackage{amssymb, amsmath}
\usepackage{float}
\usepackage{subfig}
\usepackage{wrapfig}
%\newtheorem{example}{Example}
%\newtheorem{defn}{Definition}[section]
%\newtheorem{thm}{Theorem}[section]
\newtheorem{defn}{Definition}
\newtheorem{thm}{Theorem}
%\newtheorem{lem}[thm]{Lemma}
\newtheorem{lem}{Lemma}
%\newtheorem{prop}[thm]{Proposition}
\newtheorem{prop}{Proposition}
\newtheorem{cor}{Corollary}
\iffalse
\newtheorem{proof}{Proof}
\fi
%\newtheorem{obser}[thm]{Observation}
\newtheorem{obser}{Observation}
% Used for displaying a sample figure. If possible, figure files should
% be included in EPS format.
%
% If you use the hyperref package, please uncomment the following line
% to display URLs in blue roman font according to Springer's eBook style:
% \renewcommand\UrlFont{\color{blue}\rmfamily}

\begin{document}
%
\title{Proactive Rumor Control: When Impression Counts (Full Version)}
%
%\titlerunning{Abbreviated paper title}
% If the paper title is too long for the running head, you can set
% an abbreviated paper title here
%
\author{Pengfei Xu\and
Zhiyong Peng \and
Liwei Wang}
%
%\authorrunning{Pengfei Xu et al.}
% First names are abbreviated in the running head.
% If there are more than two authors, 'et al.' is used.
%
\institute{ School of Computer Science, Wuhan University, Hubei, China\\
\email{\{xupengfei,peng,liwei.wang\}@whu.edu.cn}}
%
\maketitle              % typeset the header of the contribution
%
\definecolor{purple}{rgb}{1, 0, 1}

\newcommand{\ie}{\emph{i.e.,}\xspace}
\newcommand{\eg}{\emph{e.g.,}\xspace}
\newcommand{\abr}{\emph{abbr.}\xspace}
\newcommand{\ea}{\emph{et al.}\xspace}
\newcommand{\gensync}{\emph{GenSync}\xspace}
\newcommand{\colosseum}{\emph{Colosseum}\xspace}
\newcommand{\srep}{\emph{SREP}\xspace} % Set Reconciliation Enhances
\newcommand{\srepsim}{\emph{SREPSim}\xspace}
% Propagation
\newcommand{\esrep}{\emph{E-SREP}\xspace}
\newcommand{\epsrep}{\emph{EP-SREP}\xspace}
\newcommand{\mesrep}{\emph{ME-SREP}\xspace}
\newcommand{\mempoolsync}{\emph{MempoolSync}}

\newcommand{\fref}[1]{Fig.~\ref{#1}}
\newcommand{\tref}[1]{Table~\ref{#1}}
\newcommand{\aref}[1]{Algorithm~\ref{#1}}
\newcommand{\procref}[1]{Procedure~\ref{#1}}
\newcommand{\sref}[1]{Section~\ref{#1}}
\newcommand{\lineref}[1]{line~\ref{#1}}
\newcommand{\appref}[1]{Appendix~\ref{#1}}

% Change \eqref
\LetLtxMacro{\originaleqref}{\eqref}
\renewcommand{\eqref}{Eq.~\originaleqref}

% Theorems and corollaries
\newcounter{theoremcount}
\setcounter{theoremcount}{0}
\DeclareRobustCommand{\theorem}[1]{%
  \refstepcounter{theoremcount}%
  \noindent\textit{\textbf{Theorem \thetheoremcount\label{theorem:#1}: }}%
}
\DeclareRobustCommand{\theoremref}[1]{Theorem~\ref{theorem:#1}}

\DeclareRobustCommand{\proof}{\emph{Proof:}\xspace}
\DeclareRobustCommand{\qqed}{\hfill$\blacksquare$}

\newcounter{corollcount}
\setcounter{corollcount}{0}
\DeclareRobustCommand{\coroll}[1]{%
  \refstepcounter{corollcount}%
  \noindent\textit{\textbf{Corollary \thecorollcount\label{coroll:#1}: }}%
}
\DeclareRobustCommand{\corollref}[1]{Corollary~\ref{coroll:#1}}

\newcounter{lemmacount}
\setcounter{lemmacount}{0}
\DeclareRobustCommand{\lemma}[1]{%
  \refstepcounter{lemmacount}%
  \noindent\textit{\textbf{Lemma \thelemmacount\label{lemma:#1}: }}%
}
\DeclareRobustCommand{\lemmaref}[1]{Lemma~\ref{lemma:#1}}

\newcounter{definitioncount}
\setcounter{definitioncount}{0}
\DeclareRobustCommand{\definition}[1]{%
  \refstepcounter{definitioncount}%
  \noindent\textit{\textbf{Definition \thedefinitioncount\label{definition:#1}: }}%
}
\DeclareRobustCommand{\defref}[1]{Definition~\ref{definition:#1}}

%notes of different authors
\newif\ifnotes
\notestrue
\notesfalse

\newif\ifdiff
\difftrue
\difffalse

\newcommand{\anote}[1]{\ifnotes $\ll$\textsf{\textcolor{purple}{Ari: {#1}}}$\gg$ \fi}
\newcommand{\nnote}[1]{\ifnotes $\ll$\textsf{\textcolor{orange}{Novak: {#1}}}$\gg$ \fi}
\newcommand{\diff}[1]{\ifdiff\textcolor{orange}{#1}\else#1\fi}

%%% Local Variables:
%%% mode: latex
%%% TeX-master: "main"
%%% End:


\begin{abstract}
The spread of rumors in online networks threatens public safety and results in economic losses. To overcome this problem, a lot of work studies the problem of rumor control which aims at limiting the spread of rumors. However, all previous work ignores the relationship between the influence block effect and counts of impressions on the user.
In this paper, we study the problem of minimizing the spread of rumors when impression counts. Given a graph $G(V,E)$, a rumor set $R \in V$ and a budget $k$, it aims to find a protector set $P \in V \backslash R$ to minimize the spread of the rumor set $R$ under the budget $k$. Due to the impression counts, two following challenges of our problem need to be overcome: (1) our problem is NP-hard; (2) the influence block is non-submodular, which means a straightforward greedy approach is not applicable. Hence, we devise a branch-and-bound framework for this problem with a ($1-1/e-\epsilon$) approximation ratio. To further improve the efficiency, we speed up our framework with a progressive upper bound estimation method, which achieves a ($1-1/e-\epsilon - \rho$) approximation ratio. We conduct experiments on real-world datasets to verify the efficiency, effectiveness, and scalability of our methods.

\keywords{social network\and rumor control\and random walk\and non-submodularity.}
\end{abstract}


\section{Introduction}

The increasing complexity of source code poses a key challenge to the reliability of large-scale software systems. Software bugs in these systems can lead to safety issues~\cite{bug_safety} for users around the world as well as cause non-negligible financial losses~\cite{bug_loss}. As such, developers have to spend a large amount of time and effort on bug fixing. Consequently, \aprfull (\apr), designed to automatically generate patches to fix software bugs, has attracted wide attention from both academia and industry~\cite{long2016prophet, legoues2012genprog, long2015spr, lou2020can, tufano2018empstudy}. 


To achieve \apr, one popular approach is known as Generate-and-Validate (G\&V)~\cite{qi2015gv, ghanbari2019prapr, lou2020can, le2016hdrepair, legoues2012genprog, wen2018capgen, hua2018sketchfix, martinez2016astor, koyuncu2020fixminder, liu2019tbar, liu2019avatar}, which is typically based on the following pipeline: First, fault localization techniques~\cite{wong2016fl, abreu2007ochiai, zhang2013injecting, papadakis2015metallaxis, li2019deepfl, li2017transforming} are applied to determine the suspicious locations in programs where bugs are likely to exist. Then, the buggy locations are used by the \apr tools to generate a list of patches that replace buggy lines with correct lines. Afterward, each patch is validated against the original test suite to identify any \emph{plausible patches} (i.e., passing all tests in the test suite). Finally, to determine the \emph{correct patches}, developers examine the list of plausible patches to see if any of them can correctly fix the bug. 

Traditional \apr tools can mainly be categorized into heuristic-based~\cite{legoues2012genprog, le2016hdrepair, wen2018capgen}, constraint-based~\cite{mechtaev2016angelix, le2017s3, demacro2014nopol, long2015spr} and \template~\cite{ghanbari2019prapr, hua2018sketchfix, martinez2016astor, liu2019tbar, liu2019avatar}. Among these traditional tools, \template \apr tools~\cite{ghanbari2019prapr, liu2019tbar, benton2020effectiveness} have been able to achieve state-of-the-art results. \Template \apr tools typically leverage pre-defined templates (e.g., adding a nullness check) for bug fixing. However, since these fix templates are typically handcrafted, the number and types of bugs they are able to fix can be limited. 



To address the limitations of traditional \apr, researchers have proposed various \learning \apr tools~\cite{li2020dlfix, chen2018sequencer, jiang2021cure, lutellier2020coconut, zhu2021recoder, ye2022rewardrepair} based on the \nmtfull (\nmt) architecture~\cite{sutskever2014mt} where the input is the buggy code snippets and the goal is to translate the buggy code snippets into a fixed version. To accomplish this, \learning \apr tools require supervised training datasets with pairs of both buggy and fixed code snippets in order to learn how to perform this translation step. These training data are usually obtained by mining historical bug fixes using heuristics/keywords~\cite{dallmeier2007benchmark}, which can be imprecise for identifying bug-fixing commits; even the actual bug-fixing commits can include irrelevant code changes, leading to further pollution in the dataset~\cite{xia2022alpharepair}.
% 
Moreover, it can be hard for such \apr tools to generalize and fix bug types unseen during training. 



To better leverage recent advances in \plmfull{s} (\plm{s}), researchers~\cite{xia2022alpharepair, xia2023repairstudy, kolak2022patch, prenner2021codexws} have directly applied \plm{s} to generate patches without bug-fixing datasets. These \llm-based \apr tools work by either directly generating a complete code function~\cite{prenner2021codexws, xia2023repairstudy} or predict/infill the correct code snippet given its surrounding context~\cite{xia2022alpharepair, xia2023repairstudy}. By directly using \llm{s} that are pre-trained on billions of open-source code snippets, \llm-based \apr tools can achieve state-of-the-art performance on many repair datasets~\cite{xia2022alpharepair}. 


% 
%
%

Traditional \apr tools have long used the insight of the \emph{plastic surgery hypothesis}~\cite{barr2014plastic} where it states that the code ingredients to fix a bug already exist within the same project. Traditional \apr tools have manually designed pattern-~\cite{ghanbari2019prapr, saha2017elixir} or heuristic-based~\cite{jiang2018simfix, legoues2012genprog} approaches to finding and using such relevant code ingredients to generate fixes for bugs. However, the plastic surgery hypothesis has been largely ignored in \llm-based \apr. In fact, \llm provides a unique opportunity to fully automate the plastic surgery hypothesis idea via fine-tuning (learning project-specific information via model updates from the buggy project) and prompting (directly providing relevant code ingredients to the model), and make it directly applicable to different languages (since the \llm{s} are typically multi-lingual).%
Moreover, despite the intensive manual efforts involved, traditional \apr tools still cannot fully leverage project-specific information due to large search space for leveraging/composing existing code ingredients. In contrast, the project-specific information can effectively leveraged by \llm{s} due to their power in code understanding/vectorization, e.g., even partial/imprecise information may still guide \llm{s} in correct patch generation!
 To this end, we ask the question: \emph{How useful is the plastic surgery hypothesis in the era of \plm{s}}?








\mypara{Our Work.} To answer the question, we present \ourtech{\xspace} -- a \llm-based approach that automatically utilizes the plastic surgery hypothesis by systematically combining multiple fine-tuning and prompting strategies for \apr. \ourtech fine-tunes \plm{s} using two novel domain-specific training strategies: \textbf{\epfinetune} -- we fine-tune using the original buggy project by aggressively masking out a high percentage of tokens, which allows \plm to learn project-specific code tokens and programming styles; and \textbf{\rofinetune} -- which only masks out a single continuous code sequence per training sample, allowing the model to get used to the final \csapr task of predicting a single continuous code sequence. Furthermore, we directly leverage the ability for \plm{s} to understand natural language instructions and introduce a novel prompting strategy, \textbf{\idprompting}, which uses information retrieval and static analysis to obtain a list of relevant identifiers for the buggy lines. While such relevant identifiers are critical for fixing some difficult bugs, they may not be seen by the \llm during inference due to limited context window size. Through the use of prompting, we directly tell the model to use these extracted identifiers (relevant code ingredients) to generate the correct code. Finally, to perform repair, we combine all four model variants (including the base model, both fine-tuned models and the base model with prompting) for the final repair.





While our insight of leveraging the plastic surgery hypothesis for \llm-based \apr is generalizable across different types of \plm{s}, to implement \ourtech, we choose a recent \plm{\xspace}, \ctfive~\cite{wang2021codet5}, which is pre-trained on millions of open-source code snippets. \ctfive is an encoder-decoder model trained using \mspfull (\msp) objective where a percentage of tokens are masked out and each continuous masked token sequence is referred to as a masked span. Also, although we only extract relevant identifiers from the current buggy project (since this paper focuses on the plastic surgery hypothesis), our work can be easily extended to obtain other code information (such as relevant statements or functions) from other sources, such as  the massive pre-training corpora~\cite{husain2020codesearchnet} or historical bug-fixing datasets~\cite{jiang2019infer}, which can provide more coding knowledge for \llm{s}. Besides, although we mainly focus on using traditional string comparison algorithms for information retrieval in this paper, these techniques can be easily replaced by other frequency-based retrieval~\cite{robertson2009probabilistic} and neural search (or embedding-based search)~\cite{reimers2019sentence}.
  In summary, this paper makes the following contributions:


%


\begin{itemize}[noitemsep, leftmargin=*, topsep=0pt]
    \item \textbf{Dimension.} This paper is the first to revisit the important plastic surgery hypothesis in the era of \llm{s}. It opens up a new dimension for \llm-based \apr to incorporate previously neglected information from the buggy project itself to boost \apr performance. Furthermore, it demonstrates the promising future of retrieval-based prompting for modern \llm-based \apr.
    \item \textbf{Implementation.} We implement \ourtech based on the recent \ctfive model. We augment the model using two novel fine-tuning strategies: \epfinetune and \rofinetune, along with a novel prompting strategy based on information retrieval and static analysis: \idprompting. We combine the patches generated by all four models together and perform patch ranking to speed up \apr.% 
    \item \textbf{Evaluation Study.} We conduct an extensive evaluation against state-of-the-art \apr tools. On the widely studied \dfj 1.2 and 2.0 datasets~\cite{just2014dfj}, \ourtech is able to achieve the new state-of-the-art results of 89 and 44 correct bug fixes (15 and 8 more than best baseline) respectively.  Furthermore, we perform a broad ablation study to justify our design. \ourtech demonstrates for the first time that the plastic surgery hypothesis can substantially boost \llm-based \apr and advance state-of-the-art \apr, while being fully automated and general. Moreover, even partial/imprecise code ingredients may still effectively guide \llm{s} for \apr!
\end{itemize}



\section{Related work}
\noindent \textbf{Video foundation models.}
With sufficient computational power and an abundant source of data, there have been attempts to build a single large-scale foundation model that can be adapted to diverse downstream tasks.
Along with the success of foundations models in the natural language processing domain~\cite{brown2020language,chen2021evaluating,devlin2019bert} and in computer vision~\cite{bertasius2021space,jia2021scaling,radford2021learning}, video data has become another data type of interest, as it has grown in scale due to numerous internet video-sharing platforms.
Accordingly, several methods to train a video foundation model have been proposed.
Due to the innate multi-modality of video data, \textit{i.e.}, a combination of visual $\cdot$ vocal $\cdot$ textual context, most works have centered around the variations of the cross-modal attention mechanism \cite{akbari2021vatt,bertasius2021space,gabeur2020multi,luo2020univl,neimark2021video,tan2021look,wei2020multi,yang2021taco}.
In addition, as most video data lack proper labels or descriptions, contrastive learning methods were studied to learn meaningful feature representations or enhance video-text alignment in a self-supervised manner \cite{akbari2021vatt,kuang2021video,luo2020univl,yang2021taco}.

More specifically, MERLOT \cite{zellers2021merlot} proposed a multi-modal representation learning method for visual commonsense reasoning, which also performed well in twelve video reasoning tasks.
VATT \cite{akbari2021vatt} introduced a multi-modal learning method via contrastive learning. 
The pre-trained model performed well in a variety of vision tasks from image classification to video action recognition and zero-shot video retrieval.
Another representative work, UniVL \cite{luo2020univl} proposed a straightforward pre-training method with auxiliary loss functions. 
After fine-tuning on a specific task, the pre-trained model performed outstandingly in a wide range of tasks of text-to-video retrieval, action segmentation, action step localization, video sentiment analysis, and video captioning.
Other foundation models for multiple video tasks include \cite{li2020hero,sun2019learning,sun2019videobert,zhu2020actbert,fu2021violet,wang2022all}. 

\noindent \textbf{Auxiliary learning.}
In order to enhance the performance of one or a multitude of primary tasks, auxiliary learning methods can be incorporated.
\cite{ruder2017overview} introduced Multi-task learning (MTL) to the deep neural networks by training a single model with multiple task losses to assist learning on the main task.
Such a method is generally adapted to pre-train the foundation models in the self-supervised manner~\cite{li2020hero,sun2019learning,sun2019videobert,zhu2020actbert,fu2021violet,wang2022all}.
However, these various pretext task losses used in the pre-training phase are ignored in the fine-tuning phase, and only the primary task loss is minimized.

Recently, meta-learning methods have been introduced for auxiliary learning.
\cite{liu2019self,navon2020auxiliary,shu2019meta} proposed a meta-learning method in which the model learns auxiliary tasks to generalize well to unseen data. 
In these settings, a separate subset of data is held out as the primary task, while the others are used as auxiliary tasks that aid the primary task's performance.
Similar methods were adopted for computer vision tasks such as semantic segmentation \cite{xu2021leveraging}.
Other domain applications include navigation tasks with reinforcement learning \cite{ye2021auxiliary}, or self-supervised learning methods on graph data \cite{hwang2020self}.

\section{The Semi-Oblivious Chase Procedure}\label{sec:semi}
%

The semi-oblivious chase (or simply chase) takes as input a database $D$ and a set $\dep$ of TGDs, and constructs an instance that contains $D$ and satisfies $\dep$.
%
A central notion in this context is that of trigger.
%are those of trigger, active trigger, and trigger application.

\begin{definition}%[\textbf{Trigger Application}]
	Given a set $\dep$ of TGDs and an instance $I$, a {\em trigger} for $\dep$  on $I$ is a pair $(\sigma,h)$, where $\sigma \in \dep$ and $h$ is a homomorphism from $\body{\sigma}$ to $I$.
	%
	The {\em result} of $(\sigma,h)$, denoted $\result{\sigma}{h}$, is the set $\mu(\head{\sigma})$, where $\mu : \var{\head{\sigma}} \ra \ins{C} \cup \ins{N}$ is defined as follows:
	%
	%$\mu(x) = h(x)$ if $x \in \fr{\sigma}$, and $\mu(x) = \bot_{\sigma,h_{|\fr{\sigma}}}^{x}$ otherwise,
	\[
	\mu(x)\
	=\ \left\{
	\begin{array}{ll}
	h(x) & \quad \text{if } x \in \fr{\sigma}\\
	&\\
	\bot_{\sigma,h_{|\fr{\sigma}}}^{x} & \quad \text{otherwise}
	\end{array} \right.
	\]
	where $\bot_{\sigma,h_{|\fr{\sigma}}}^{x} \in \ins{N}$.  Let $T(\dep,I)$ be the set of triggers for $\dep$ on $I$.	\hfill\markfull
\end{definition}




Observe that in the definition of $\result{\sigma}{h}$, each existentially quantified variable $x$ of $\head{\sigma}$ is mapped by $\mu$ to a null value of $\ins{N}$ whose name is uniquely determined by the trigger $(\sigma,h)$ and the variable $x$ itself. This means that, given a trigger $(\sigma,h)$, we can unambiguously construct the set of atoms $\result{\sigma}{h}$.
%
The central idea of the chase is, starting from a database $D$, to exhaustively apply triggers for the given set $\dep$ of TGDs on the instance constructed so far.
%
More precisely, given a database $D$ and a set $\dep$ of TGDs, let
\[
\mathsf{chase}^{0}(D,\dep)\ =\ D,
\]
and for each $i>0$, let
\[
\mathsf{chase}^{i}(D,\dep)\ =\ \mathsf{chase}^{i-1}(D,\dep)\ \cup\ \bigcup_{(\sigma,h) \in S} \result{\sigma}{h},
\]
where $S = T(\dep,\mathsf{chase}^{i-1}(D,\dep))$. 
%
We finally define {\em the result of the chase of $D$ w.r.t.~$\dep$} as the (possibly infinite) instance
\[
\chase{D}{\dep}\ =\ \bigcup_{i \geq 0} \mathsf{chase}^{i}(D,\dep).
\]


\ignore{
The semi-oblivious chase procedure (or simply chase) takes as input a database $D$ and a set $\dep$ of TGDs, and constructs an instance that contains $D$ and satisfies $\dep$.
%
Central notions in this context are those of trigger, active trigger, and trigger application.

\begin{definition}%[\textbf{Trigger Application}]
	Given a set $\dep$ of TGDs and an instance $I$, a {\em trigger} for $\dep$  on $I$ is a pair $(\sigma,h)$, where $\sigma \in \dep$ and $h$ is a homomorphism from $\body{\sigma}$ to $I$.
	%
	The {\em result} of $(\sigma,h)$, denoted $\result{\sigma}{h}$, is the set $\mu(\head{\sigma})$, where $\mu : \var{\head{\sigma}} \ra \ins{C} \cup \ins{N}$ is defined as follows:
	%
	%$\mu(x) = h(x)$ if $x \in \fr{\sigma}$, and $\mu(x) = \bot_{\sigma,h_{|\fr{\sigma}}}^{x}$ otherwise,
	\[
	\mu(x)\
	=\ \left\{
	\begin{array}{ll}
	h(x) & \quad \text{if } x \in \fr{\sigma}\\
	&\\
	\bot_{\sigma,h_{|\fr{\sigma}}}^{x} & \quad \text{otherwise}
	\end{array} \right.
	\]
	where $\bot_{\sigma,h_{|\fr{\sigma}}}^{x}$ is a null value from $\ins{N}$.
	%
	The trigger $(\sigma,h)$ is {\em active} if $\result{\sigma}{h} \not\subseteq I$.
	%
	The {\em application} of $(\sigma,h)$ to $I$ returns the instance $J = I \cup \result{\sigma}{h}$ and is denoted as $I \app{\sigma}{h} J$.
	\hfill\markfull
\end{definition}


Observe that in the definition of $\result{\sigma}{h}$ above, each existentially quantified variable $x$ of $\head{\sigma}$ is mapped by $\mu$ to a null value of $\ins{N}$ whose name is uniquely determined by the trigger $(\sigma,h)$ and the variable $x$ itself. This means that, given a trigger $(\sigma,h)$, we can unambiguously extract the set of atoms 
$\result{\sigma}{h}$.



%\medskip

%\noindent
%\textbf{Semi-Oblivious Chase.}
The central idea of the chase is, starting from a database $D$, to exhaustively apply active triggers for the given set $\dep$ of TGDs on the instance constructed so far. This is formalized via the notion of (semi-oblivious) chase derivation, which can be finite or infinite.


\begin{definition}
	Consider a database $D$ and a set $\dep$ of TGDs.
	%We consider the two cases where a derivation is finite or infinite:
	\begin{itemize}
		\item A finite sequence $(I_i)_{0 \leq i \leq n}$ of instances, with $D = I_0$ and $n \geq 0$, is a {\em chase derivation} of $D$ w.r.t.~$\dep$ if, for each $i \in \{0,\ldots,n-1\}$, there is an active trigger $(\sigma,h)$ for $\dep$ on $I_i$ with $I_i \app{\sigma}{h} I_{i+1}$, and there is no active trigger for $\dep$ on $I_n$. The {\em result} of such a chase derivation is the instance $I_n$.
		
		
		\item An infinite sequence $(I_i)_{i \geq 0}$ of instances, with $D = I_0$, is a {\em chase derivation} of $D$ w.r.t.~$\dep$ if, for each $i \geq 0$, there is an active trigger $(\sigma,h)$ for $\dep$ on $I_i$ such that $I_i \app{\sigma}{h} I_{i+1}$. Moreover, $(I_i)_{i \geq 0}$ is {\em fair} if, for each $i \geq 0$, and for every active trigger $(\sigma,h)$ for $\dep$ on $I_i$, there exists $j > i$ such that $(\sigma,h)$ is not an active trigger for $\dep$ on $I_j$. 
		%The latter is known as the {\em fairness condition}, and guarantees that all the active triggers will be deactivated. %
		The {\em result} of such a chase derivation is the instance $\bigcup_{i \geq 0} \, I_i$.
	\end{itemize}
	%
	%The {\em result} of a chase derivation is defined as the union of all the instances occurring in it. 
	A chase derivation is {\em valid} if it is finite or infinite and fair.  \hfill\markfull
\end{definition}


Let us stress that infinite but unfair chase derivations are not considered as valid ones since they do not serve the main purpose of the chase, that is, to build an instance that satisfies the given set of TGDs. Indeed, given the set $\dep$ consisting of the TGDs
\[
\sigma\ =\ R(x,y) \ra \exists z \, R(y,z) \qquad \sigma'\ =\ R(x,y) \ra P(x,y),
\]
the result of the unfair chase derivation of $D = \{R(a,b)\}$ w.r.t.~$\dep$ that involves only triggers of the form $(\sigma,\cdot)$, i.e., only the TGD $\sigma$ is used, does not satisfy $\sigma'$, and thus, it does not satisfy $\dep$.
%
Interestingly, for every database $D$ and set $\dep$ of TGDs, any two valid chase derivations of $D$ w.r.t.~$\dep$ have always the same result, which implies that all valid chase derivations are either finite or infinite~\cite{GrOn18}. Therefore, in the rest of the paper, we can safely refer to {\em the} result of the chase of $D$ w.r.t. $\dep$, which we will denote by $\chase{D}{\dep}$. 
}


%\subsection{Non-Uniform Chase Termination}\label{sec:problem}
%

\medskip

\noindent
\textbf{Chase Termination.}
The result of the chase may be infinite even for very simple settings: it is easy to see that for $D = \{R(a,b)\}$ and $\dep = \{R(x,y) \ra \exists z \, R(y,z)\}$, $\chase{D}{\dep}$ is infinite.
%; in particular, $\chase{D}{\dep} = \{R(a,b),R(b,\bot_1),R(\bot_1,\bot_2),R(\bot_2,\bot_3),\ldots\}$, where $\bot_1,\bot_2,\ldots$ are null values.
%
This leads to the following problem, parameterized by a class $\class{C}$ of TGDs such as $\class{SL}$ (the class of simple-linear TGDs) and $\class{L}$ (the class of linear TGDs):


\medskip

\begin{center}
	\fbox{
		\begin{tabular}{ll}
			%{\small PROBLEM} : & %$\mathsf{ChaseTermination}(\class{C})$
			%\\
			{\small INPUT} : & A database $D$ and a set $\dep$ of TGDs from $\class{C}$.
			\\
			{\small QUESTION} : &  Is the instance $\chase{D}{\dep}$ finite?
	\end{tabular}}
\end{center}

\medskip

\noindent This problem has been recently studied in~\cite{CaGP22} for the classes of simple-linear and linear TGDs. Interestingly, for both classes, the finiteness of the result of the chase has been syntactically characterized by exploiting the notion of non-uniform weak-acyclicity. 
%
We proceed to recall this acyclicity notion, and then present the characterizations established in~\cite{CaGP22}, which in turn lead to simple algorithms for checking the finiteness of the result of the chase.
%
Note that, for the sake of clarity, in the rest of the paper we assume TGDs with a non-empty frontier, i.e., we assume that there is at least one variable in a TGD $\sigma$ that occurs both in $\body{\sigma}$ and $\head{\sigma}$. This assumption can be made without loss of generality since, given a database $D$ and a set $\dep$ of TGDs, we can easily construct a set $\dep'$ of TGDs with a non-empty frontier by slightly modifying $\dep$ such that $\chase{D}{\dep}$ is finite iff $\chase{D}{\dep'}$ is finite.


\medskip

\noindent
\textbf{Non-Uniform Weak-Acyclicity.} Weak-acyclicity was introduced in~\cite{FKMP05} as the main formalism for data exchange purposes, which guarantees the finiteness of the result of the chase for {\em every} input database. Non-uniform weak-acyclicity is the database-dependent variant of weak-acyclicity introduced in~\cite{CaGP22}. We proceed to give the formal definitions.
%
We first need to recall the notion of the {\em dependency graph} of a set $\dep$ of TGDs, 
%which symbolically encodes how terms may propagate during the chase.
%The {\em dependency graph} of set $\dep$ of TGDs 
defined as a directed multigraph $\depg{\dep}=(N,E)$, where $N = \pos{\sch{\dep}}$ and $E$ contains {\em only} the following edges.
%
For each TGD $\sigma \in \dep$ with $\head{\sigma} = \{\alpha_1,\ldots,\alpha_k\}$, for each $x \in \frontier{\sigma}$, and for each position $\pi \in \posvar{\body{\sigma}}{x}$:
\begin{itemize}
	\item For each $i \in [k]$ and for each $\pi' \in \posvar{\alpha_i}{x}$, there exists a \emph{normal} edge $(\pi,\pi') \in E$.
	%
	\item For each existentially quantified variable $z$ in $\sigma$, $i \in [k]$, and $\pi' \in \posvar{\alpha_i}{z}$, there is a \emph{special} edge $(\pi,\pi') \in E$.
\end{itemize}
%
We further need to define when a predicate is reachable from another predicate. 
%
Given predicates $R,P \in \sch{\dep}$, {\em $P$ is reachable from $R$ (w.r.t.~$\dep$)} if $R = P$, or there exists a path in $\depg{\dep}$ from a position of the form $(R,i)$ to a position of the form $(P,j)$.
%
%we write $R \ra_\dep P$  if $R = P$, or there exists a TGD $\sigma \in \dep$ such that $R$ occurs in $\body{\sigma}$ and $P$ occurs in $\head{\sigma}$. We say that {\em $P$ is reachable from $R$ (w.r.t.~$\dep$)}, denoted $R \reach{\dep} P$, if (i) $R \ra_\dep P$, or (ii) there exists $T \in \sch{\dep}$ such that $R \reach{\dep} T$ and $T \ra_\dep P$.
%in $\depg{\dep}$, denoted $R \reach{\dep} P$, if there exists a path in $\depg{\dep}$ from a position $(R,i)$ to a position $(P,j)$, for some $i \in [\arity{R}]$ and $j \in [\arity{P}]$.
Given a database $D$, we say that a (not necessarily simple and possibly cyclic) path $C$ in $\depg{\dep}$ is \emph{$D$-supported} if there exists an atom $R(\bar t) \in D$ and a node of the form $(P,i)$ in $C$ such that $P$ is reachable from $R$.
%
We are now ready to recall (non-uniform) weak-acyclicity.



\begin{definition}\label{def:dwa}
	Consider a database $D$ and a set $\dep$ of TGDs. We say that $\dep$ is {\em weakly-acyclic w.r.t.~$D$}, or {\em $D$-weakly-acyclic}, if there is no $D$-supported cycle in $\depg{\dep}$ with a special edge. 
	%
	We say that $\dep$ is {\em weakly-acyclic} if there is no cycle in $\depg{\dep}$ with a special edge. \hfill\markfull
\end{definition}


\smallskip

\noindent
\textbf{Characterizing the Finiteness of the Chase.}
It is not very difficult to show that whenever a set $\dep$ of TGDs (not necessarily linear) is $D$-weakly-acyclic, then the instance $\chase{D}{\dep}$ is finite. In other words, the $D$-weak-acyclicity of $\dep$ is a sufficient condition for the finiteness of $\chase{D}{\dep}$. What is more interesting is that, assuming that $\dep$ is a set of simple-linear TGDs, the $D$-weak-acyclicity of $\dep$ is also a necessary condition for the finiteness of $\chase{D}{\dep}$. This leads to the following characterization established in~\cite{CaGP22}:

\begin{theorem}\label{the:characterization-simple-linear}
	Consider a database $D$ and a set $\dep \in \class{SL}$ of TGDs. It holds that $\chase{D}{\dep}$ is finite iff $\dep$ is $D$-weakly-acyclic.
\end{theorem}

For linear TGDs, it turned out that non-uniform weak-acyclicity is not powerful enough for characterizing the finiteness of the chase instance. Here is an example given in~\cite{CaGP22} that illustrates this fact:
%This is illustrated by the following example.


\begin{example}
	Consider the database $D = \{R(a,b)\}$ and the singleton set $\dep$ consisting of the (non-simple) linear TGD
	\[
	R(x,x)\ \ra\ \exists z \, R(z,x). 
	\]
	It is easy to see that there is no trigger for $\dep$ on $D$. This means that $\chase{D}{\dep} = D$ is finite, whereas $\dep$ is {\em not} $D$-weakly-acyclic. \hfill\markfull
\end{example}


To obtain a characterization analogous to Theorem~\ref{the:characterization-simple-linear}, the authors of~\cite{CaGP22} used the technique of {\em simplification} to convert linear TGDs into simple-linear TGDs, while preserving the finiteness of the chase instance. We proceed to recall this technique.
%
Let $\bar t = (t_1,\ldots,t_n)$ be a tuple of (not necessarily distinct) terms. We write $\unique{\bar t}$ for the tuple obtained from $\bar t$ by keeping only the first occurrence of each term in $\bar t$.
%
For example, if $\bar t = (x,y,x,z,y)$, then $\unique{\bar t} = (x,y,z)$.
%
For each $i \in [n]$, the \emph{identifier of $t_i$ in $\bar t$}, denoted $\id{\bar t}{t_i}$, is the integer that identifies the position of $\unique{\bar t}$ at which $t_i$ appears. 
%
We write $\id{}{\bar t}$ for the tuple $(\id{\bar t}{t_1},\ldots,\id{\bar t}{t_n})$.
%
For example, if $\bar t = (x,y,x,z,y)$, then $\id{}{\bar t} = (1,2,1,3,2)$.
%
For an atom $\alpha = R(\bar t)$, the {\em simplification of $\alpha$}, denoted $\simple{\alpha}$, is the atom $R_{\id{}{\bar t}}(\unique{\bar t})$, whereas the {\em shape of $\alpha$}, denoted $\shape{\alpha}$, is the predicate $R_{\id{}{\bar t}}$. We can naturally refer to the simplification and the shape of a set of atoms.
%
For a tuple of variables $\bar x = (x_1,\ldots,x_n)$, a \emph{specialization of $\bar x$} is a function $f$ from $\bar x$ to $\bar x$ such that $f(x_1) = x_1$, and $f(x_i) \in \{f(x_1),\ldots,f(x_{i-1}),x_i\}$, for each $i \in \{2,\ldots,n\}$.
We write $f(\bar x)$ for $(f(x_1),\ldots,f(x_n))$. We are now ready to recall how a set of linear TGDs is converted into a set of simple-linear TGDs.

\begin{definition}\label{def:simplification}
	Consider a linear TGD $\sigma$ of the form
	\[
	R(\bar x) \ra \exists \bar z\, \psi(\bar y,\bar z), 
	\]
	where $\bar y \subseteq \bar x$, and a specialization $f$ of $\bar x$. The {\em simplification of $\sigma$ induced by $f$} is the simple-linear TGD
	\[
	\simple{R(f(\bar x))} \rightarrow \exists \bar z\, \simple{\psi(f(\bar y),\bar z)}.
	\]
	We write $\simple{\sigma}$ for the set of all simplifications of $\sigma$ induced by some specialization of $\bar x$.
	%
	For a set $\dep \in \class{L}$ of TGDs, the {\em simplification of $\dep$} is defined as the set
	\[
	\simple{\dep}\ =\ \bigcup_{\sigma \in \dep} \simple{\sigma}
	\]
	consisting only of simple-linear TGDs. \hfill\markfull
\end{definition}

We can now recall the characterization for the finiteness of the chase instance for linear TGDs, established in~\cite{CaGP22}, which is similar to the one for simple-linear TGDs, with the key difference that first we need to simplify both the database and the set of linear TGDs:

\begin{theorem}\label{the:characterization-linear}
	Consider a database $D$ and a set $\dep \in \class{L}$ of TGDs. Then, $\chase{D}{\dep}$ is finite iff $\simple{\dep}$ is $\simple{D}$-weakly-acyclic.
\end{theorem}

It is clear that Theorems~\ref{the:characterization-simple-linear} and~\ref{the:characterization-linear} provide simple algorithms for checking whether the chase instance is finite. In particular, given a database $D$ and a set $\dep$ of simple-linear TGDs, we simply need to check whether $\dep$ is $D$-weakly-acyclic, in which case the algorithm returns \true; otherwise, it returns \false. The same holds when $\dep$ is a set of linear TGDs, with the difference that the algorithm first needs to simplify $D$ and $\dep$, and then perform the acyclicity check.
%
Our goal is to experimentally evaluate the above algorithms with the aim of understanding which input parameters affect their performance, clarifying whether they can be applied in a practical context, and revealing their performance limitations. Of course, a naive implementation of the above algorithms, especially for linear TGDs where the expensive simplification must be applied, will lead to poor performance, and thus, will not be very useful towards our goal. Hence, we need to somehow convert the above theoretical algorithms into practical algorithms that are amenable to efficient implementations. This is the subject of the next section.

\vspace{-0.3cm}\section{Our framework}
In this section, we first present a Monte Carlo-based greedy method (\samgreedy) as a baseline. Unfortunately, the effectiveness of this method is poor, and \samgreedy~cannot obtain any theoretical guarantees because the objective function of \prob~is non-submodular. Then we devise a Branch-and-Bound framework to solve this problem effectively. The core of this framework is how to estimate the upper bound of each candidate solution. In particular, we propose sampling-based bound estimation techniques for each branch under exploration by setting a submodular function to a tight upper bound of ${\I_{\Wu}(P|R)}$.

\subsection{A Baseline}
The core idea of \samgreedy~is to select the node $u$ which maximizes the unit marginal gain, i.e., $(\object({\pr} \cup \{ u \}|{\ru}) - \object({\pr}|{\ru}))$ , to a candidate solution set $\pr$, until the budget $k$ is exhausted. The pseudo-code of \samgreedy~is presented in Algorithm~\ref{al_samgreedy}. It first initializes $P$ as an empty set and $V\gets V \backslash R$. Next, it finds a set $P$ according to the greedy heuristic (Lines 1.6 to 1.10). In the end, it outputs set $P$ as a result.
%\vspace{-0.5cm}
%\setlength{\algomargin}{0.5em} 
%\setlength{\belowcaptionskip}{-3cm}
%\setlength{\intextsep}{0.5cm}
\setlength{\textfloatsep}{0pt}
\begin{algorithm}[t]
\caption{\samgreedy$(G, R,, k)$} 
\label{al_samgreedy}
\DontPrintSemicolon
\begin{small}
{\bf Input:} a graph $G$, a rumor set $R$%, sampling time $X$ for each node 
and a budget $k$

{\bf Output:} a protector set $P$

Run $X$ random walks for each node in $V\backslash R$;
$\IS(\ru) \gets$ all the random walks influenced by $R$;
Initialize $P$ as an empty set and $V \gets V \backslash R$.

\Repeat{$k = 0$}
{
	%\tcc{$\object(P|R)$ is computed based on Monte Carlo simulations $\IS(\ru)$.}
	
	Select $u \gets \arg \max _{v \in V}((\object({\pr} \cup \{ v \}|{\ru}) - \object({\pr}|{\ru})))$
	
	$V \gets V \backslash \{u\}$ and $P \gets P \cup \{u\}$;
	$k \gets k - 1$
	
}

\Return {$P$}
\end{small}
\end{algorithm}

% \subsubsection{Time Complexity} 
% Let $X$ be the number of random walk simulations for each node, and the length of random walk is bounded by $T$. Thus the sampling set is generated in $O(nXT)$ time. In each iteration, Algorithm~\ref{al_samgreedy} needs to scan the sampling set once to pick the node with the highest marginal gain. Therefore, the time and space complexity of
% Algorithm~\ref{al_samgreedy} are $O(knXT)$ and $O(nXT)$, respectively.

% \subsubsection{Index for Efficient Marginal Gain Computation}
% It is noticed that the computation of the marginal gain is the main bottleneck of Algorithm~\ref{al_samgreedy}. To address this issue, we propose two mapping indexes, \emph{forward list} and \emph{inverted list}.%, as shown in Fig.~\ref{fig:forward_list} and Fig.~\ref{fig:inverted_list} respectively. 

% The former is for nodes $u_i \in V \backslash R$, maintaining a blocked random walk list that could be blocked by node $u_i$.
% The latter is for random walk $W_i$, maintaining a list of node candidates that could block the influence of rumor set $R$ in random walk $W_i$. Meanwhile, the number of the intersection of the current protector set $P$ and each node candidate set is recorded and will be updated when a new node is added into $P$.

% \begin{figure*}[t]
% 	\hspace{0.8in}
% 	\begin{minipage}[b]{0.15\textheight}%{0.47\textwidth}
% 		\centering
% 		\begin{tabular}{|c||c|}
% 			\hline
% 			Node List & Blocked Random Walks \\ \hline
% 			$u_1$ & $W_1,W_3, W_k$ \\ \hline
% 			$u_2$ & $W_1,W_2$ \\ \hline
% 			$u_3$ & $W_3$ \\ \hline
% 			$\cdots$ & $\cdots$ \\ \hline
% 			$u_{n}$ & $W_2$ \\ \hline
% 		\end{tabular}
% 		\caption{Forward list}
% 		\label{fig:forward_list}
% 	\end{minipage}
% 	\hspace{1in}
% 	\begin{minipage}[b]{0.15\textheight}%{0.47\textwidth}
% 		\centering
% 		\begin{tabular}{|c||c|c|}
% 			\hline
% 			Random Walk List & Node Candidates & $\ic_{W_i}(P|R)$ \\ \hline
% 			$W_1$ & $u_1,u_2$ & 0 \\ \hline
% 			$W_2$ & $u_2,u_n$ & 0 \\ \hline
% 			$W_3$ & $u_1,u_3$ & 0\\ \hline
% 			$\cdots$ & $\cdots$ & $\cdots$\\ \hline
% 			$W_k$ & $u_1$ & 0 \\ \hline
% 		\end{tabular}
% 		\caption{Inverted list}
% 		\label{fig:inverted_list}
% 	\end{minipage}
% \end{figure*}
\subsection{Branch-and-Bound Framework}

As we analyzed above, \samgreedy~cannot obtain any theoretical guarantees because the objective function of \prob~is non-submodular. Then inspired by~\cite{DBLP:conf/kdd/ZhangLBMZ19}, we introduce a branch and bound framework to solve this problem effectively, and this solution can achieve a theoretical guarantee.

Algorithm~\ref{al_bab} shows the pseudo-code of the branch-and-bound framework.
We first initialize the global upper bound $U_G$ and global lower
bound $L_G$, and a max heap $H$ with each entry denoted as $\{P', V, U\}$, where $P'$ is the current node set that has been selected
as a protector set, $V$ is the set of a node that has not been considered yet, and $U$ is the upper bound influence block of the corresponding search space. $H$ is ordered by the upper bound value of each $P'$. While $L_G < U_G$, $H$ will pop the top entry that has the
maximum upper bound influence block. For each entry, if it matches the budget $k$ constraint, it will generate two new candidate sets ($P^a$ and $P^b$) by adding a new node $u \in V$ or not.
Then it computes the upper bound for each candidate set and updates $L_G$, $P$, and $H$ when $L^a>L_G$ and $U^a>L_G$, respectively.

%\setlength{\algomargin}{1.2em} 
\vspace{-0.5cm}
\begin{algorithm}[t]
\caption{\bab$(G, R, k)$} 

\label{al_bab}
\DontPrintSemicolon
\begin{small}
{\bf Input:} a graph $G$, a rumor set $R$ and a budget $k$

{\bf Output:} a protector set $P$


Initialize $P$ and $P'$ as an empty set and $V \gets V \backslash R$;
$L_G \gets 0$ and $U_G \gets \infty$;
Initialize max heap $H \gets \{P', V, U\}$

\Repeat{$L_G\ge U_G$}
{
	$\{P', V, U\} \gets$ top of $H$;
	Select $u \in V$
	
	\If{$|P'| < k$}
	{
		$V \gets V \backslash u$
		
		$P^a \gets P'\cup u$ and  $P^b \gets P'$
		
		$ \{P^c, L^a, U^a\} \gets \sbound(P^a, V)$ \label{bab:start}
		
		\If{$ L^a > L_G$}
		{
			$L_G\gets L^a$ and $P\gets P^c$
		}
		
		\If{$ U^a > L_G$}
		{
			$H\gets H\cup \{P^a, V, U^a\} $
		}\label{bab:end}
		
		Repeat line~\ref{bab:start} to \ref{bab:end} for $P^b$
	}
	
}

\Return {$P$}
\end{small}
\end{algorithm}

\subsection{Computing Upper Bound}

To estimate the upper bound of the current protector set $P^a$, we devise a submodular function ($\overline{\I}_{\Wu}(P|R)$ and $P = P^a \cup P^*$) as shown in Figure~\ref{fig:func} to compute the upper bound of $P^a$:\vspace{-0.2cm}
\begin{equation}
{\overline{\I}_{\Wu}(P|R)} = \left\{ {\begin{array}{*{20}{l}}
	l(\ic(P))	&	{\textrm{if $l(x)$ exists and }} \\
	&{ \textrm{$\ic(P^a) <\ic(P) < Tan(\ic(P^a))$}}\\
	\I_{\Wu}(P|R)			&	\textrm{otherwise}
	\end{array}} \right.\vspace{-0.2cm}
\end{equation}
Here, $\ic(P) = \ic_{\Wu}(P|R)$ for simplicity. $l(x)$ is the tangent through point ($\ic(P^a)$, $\I_{\Wu}(P^a|R)$) to function $\I_{\Wu}(P|R)$ and $Tan(\ic(P^a))$ is the x-coordinate of the tangent point. It is easy to see that $\overline{\I}_{\Wu}(P|R)$ is submodular as it concatenates two submodular functions for different domains. 

Furthermore, we have the following submodular function $\overline{\object}({\pr}|{\ru}) = \sum\nolimits_{u \in V\backslash {\ru}} {{\overline{\I}_u}} ({\pr}|{\ru})$ (here ${{\overline{\I}_u}} ({\pr}|{\ru}=  E[\overline{\I}_{\Wu}(\pr|\ru)]$ for any $\wk_u$) that
upper bounds the influence block function $\overline{\object}({\pr}|{\ru})$. It is also easy to see that $\overline{\object}({\pr}|{\ru}) $ is submodular as it is a sum of submodular functions. 

Due to the submodularity of $\overline{\object}({\pr}|{\ru})$, we turn to devise a greedy-based heuristic algorithm to find the upper bound for a given protector set $P^a$. In particular, we propose a sampling-based upper bound estimation algorithm to compute the upper bound for a given protector set.

\subsubsection{Sampling-based ComputeBound}
%\setlength{\algomargin}{1.2em} 
\begin{algorithm}[t]
\caption{\sbound$(P^a, V)$} 

\label{al_sbound}
\DontPrintSemicolon
\begin{small}
{\bf Input:} Protector set $P^a$ and candidate node set $V$

{\bf Output:} $ \{P, L^a, U^a\}$

Run $X$ $T$-random walks for each node in $V$;
$\IS(\ru) \gets$ all the random walks influenced by $R$;
Initialize $P$ as $P^a$ and $k\gets k-|P^a|$

\Repeat{$k = 0$}
{
	%\tcc{$\overline{\object}(P|R)$ is computed based on Monte Carlo simulations $\IS(\ru)$.}
	
	Select $u \gets \arg \max _{v \in V}((\overline{\object}({\pr} \cup \{ v \}|{\ru}) - \overline{\object}({\pr}|{\ru})))$
	
	$V \gets V \backslash \{u\}$ and $P \gets P \cup \{u\}$;
	$k \gets k - 1$
	
}

\Return {$P$, $L^a\gets {\object}(P|R)$, $U^a\gets \overline{\object}(P|R)$}
\end{small}
\end{algorithm}

As shown in algorithm~\ref{al_sbound}, it selects the node $u$ which maximizes the unit marginal gain%i.e., $(\overline{\object}({\pr} \cup \{ u \}|{\ru}) - \overline{\object}({\pr}|{\ru}))$ , 
to a candidate solution set $\pr$, until the budget $k$ is exhausted. 
%It is worth noting that we compute the unit marginal gain by the sampling random works.
In the end, it outputs set $P$, ${\object}(P|R)$ as $L^a$ and $\overline{\object}(P|R)$ as $U^a$.

%In order to obtain a more accurate approximation ratio, we devise a matrix-based solution for computing the upper bound in the following.

%\subsection{Matrix-based ComputeBound}
%\setlength{\algomargin}{1.2em} 
\begin{algorithm}[h]
\caption{\mbound$(P^a, V)$} 

\label{al_mbound}
\DontPrintSemicolon
\begin{small}
{\bf Input:} Protector set $P^a$ and candidate node set $V$

{\bf Output:} $ \{P, L^a, U^a\}$

Initialize $P$ as $P^a$ and $k\gets k-|P^a|$



\Repeat{$k = 0$}
{
	\tcc{$\overline{\object}(P|R)$ is computed based on Monte Carlo simulations $\IS(\ru)$.}
	
	Select $u \gets \arg \max _{v \in V}((\overline{\object}({\pr} \cup \{ v \}|{\ru}) - \overline{\object}({\pr}|{\ru})))$
	
	$V \gets V \backslash \{u\}$ and $P \gets P \cup \{u\}$
	
	$k \gets k - 1$
	
}

\Return {$P$, $L^a\gets {\object}(P|R)$, $U^a\gets \overline{\object}(P|R)$}
\end{small}
\end{algorithm}


\subsection{Analysis of Solutions}\label{sec:Analysis}

In this section, we first analyze the approximate marginal gain computation in Algorithm~\ref{al_sbound} and show the proposed branch and bound framework with sampling-based computeBound can achieve a ($1-1/e-\epsilon$)-approximation factor through setting an appropriate sampling time $X$.
\vspace{-0.3cm}
\subsubsection{Approximate ratio of  SamComputeBound}\vspace{-0.2cm}

In algorithm~\ref{al_sbound}, it uses ${\overline{\I}'_u} ({\pr}|{\ru})$ which is computed according to the random walk sampling set as an estimator of ${\overline{\I}_u}({\pr}|{\ru})$. To estimate the expectation of ${\overline{\I}_{\Wu}(P|R)}$, we independently run $X$ random walks starting from $u$, and take the average of $\overline{\I}_{\Wu}(P|R) $ as the estimator. The proposed sampling process is equivalent to a simple random sampling with replacement, thus the estimator is unbiased. Then we use $\overline{\object}'({\pr}|{\ru}) = \sum\nolimits_{u \in V\backslash {\ru}} {{\overline{\I}'_u}} ({\pr}|{\ru})$ as a estimator of $\overline{\object}({\pr}|{\ru})$.

Next, we apply Hoeffding’s inequality~\cite{hoeffding1994probability} to bound the sample size $X$. Specifically, we have the following lemma.

\begin{lem}\label{Hoeffding}
	Given a protector set $P$ and a rumor set $R$, for two small constants $\epsilon$ and $\delta$, if $X \ge \frac{1}{2\epsilon^2}log\frac{n -|R|}{\delta}$, then $\mathbf{Pr}[|\overline{\object}'({\pr}|{\ru})  - \overline{\object}({\pr}|{\ru})  | \ge\epsilon(n -|R||)] \le \delta$.
\end{lem}

$Proof.$ First, we have %$\mathbf{Pr}[|\overline{\object}'({\pr}|{\ru})  - \overline{\object}({\pr}|{\ru})  | \ge \epsilon(n -|R|)]\le \mathbf{Pr}[\sum\nolimits_{u \in V\backslash {\ru}} |{{\overline{\I}'_u}} ({\pr}|{\ru}) - {{\overline{\I}_u}} ({\pr}|{\ru})| \ge \epsilon(n -|R|)]$,
\begin{equation*}
\begin{aligned}
\mathbf{Pr}[&|\overline{\object}'({\pr}|{\ru})  - \overline{\object}({\pr}|{\ru})  | \ge \epsilon(n -|R|)]\\&\le \mathbf{Pr}[\sum\nolimits_{u \in V\backslash {\ru}} |{{\overline{\I}'_u}} ({\pr}|{\ru}) - {{\overline{\I}_u}} ({\pr}|{\ru})| \ge \epsilon(n -|R|)],
\end{aligned}
\end{equation*}
as $|\overline{\object}'({\pr}|{\ru})  - \overline{\object}({\pr}|{\ru})  | \ge\epsilon(n -|R|) \ge \epsilon(n -|R|)$ implies $\sum\nolimits_{u \in V\backslash {\ru}} |{{\overline{\I}'_u}} ({\pr}|{\ru}) - {{\overline{\I}_u}} ({\pr}|{\ru})|\ge \epsilon(n -|R|)$. Then, by the union bound, we have %$\mathbf{Pr}[\sum\nolimits_{u \in V\backslash {\ru}} |{{\overline{\I}'_u}} ({\pr}|{\ru}) - {{\overline{\I}_u}} ({\pr}|{\ru})| \ge \epsilon(n -|R|)] \le  \sum\nolimits_{u \in V\backslash {\ru}} \mathbf{Pr}[|({{\overline{\I}'_u}} ({\pr}|{\ru}) - {{\overline{\I}_u}} ({\pr}|{\ru})) \ge \epsilon].$
\begin{equation*}
\begin{aligned}
\mathbf{Pr}[\sum\nolimits_{u \in V\backslash {\ru}} &|{{\overline{\I}'_u}} ({\pr}|{\ru}) - {{\overline{\I}_u}} ({\pr}|{\ru})| \ge \epsilon(n -|R|)] \le \\ &\sum\nolimits_{u \in V\backslash {\ru}} \mathbf{Pr}[|({{\overline{\I}'_u}} ({\pr}|{\ru}) - {{\overline{\I}_u}} ({\pr}|{\ru})) \ge \epsilon].
\end{aligned}
\end{equation*}

Since $0 \le {{\overline{\I}_u}} ({\pr}|{\ru})) \le 1$, we can apply Hoeffding’s
inequality~\cite{hoeffding1994probability} to bound the sample size $X$. Specifically, we
have %$\mathbf{Pr}[|({{\overline{\I}'_u}} ({\pr}|{\ru}) - {{\overline{\I}_u}} ({\pr}|{\ru}))| \ge \epsilon]\le exp(-2\epsilon^2X)$.
\begin{equation*}
\begin{aligned}
\mathbf{Pr}[|({{\overline{\I}'_u}} ({\pr}|{\ru}) - {{\overline{\I}_u}} ({\pr}|{\ru}))| \ge \epsilon]\le exp(-2\epsilon^2X).
\end{aligned}
\end{equation*}
Based on this, the following inequality immediately holds %$\mathbf{Pr}[|\overline{\object}'({\pr}|{\ru})  - \overline{\object}({\pr}|{\ru})  | \ge\epsilon(n -|R|)] \le (n-|R|)exp(-2\epsilon^2X)$
\begin{equation*}
\begin{aligned}
\mathbf{Pr}[|&\overline{\object}'({\pr}|{\ru})  - \overline{\object}({\pr}|{\ru})  | \\&\ge\epsilon(n -|R|)] \le (n-|R|)exp(-2\epsilon^2X).
\end{aligned}
\end{equation*}
Let $(n-|R|)exp(-2\epsilon^2X) \le \delta$, then we can get $X \ge \frac{1}{2\epsilon^2}log\frac{n -|R|}{\delta}$, which completes the proof. $\blacksquare$


According to \cite{nemhauser1978analysis}, the greedy heuristic achieves an approximation factor
of $(1 - 1/e)$ for maximizing monotone and submodular functions. Based on Lemma~\ref{Hoeffding}, by a similar analysis presented in \cite{DBLP:conf/icde/LiYHC14}, the sampling-based greedy algorithm achieves a ($1-1/e-\epsilon$) approximation factor through setting an appropriate parameter $X$ with at least ($1-\delta$) probability.

\subsubsection{Approximate ratio of branch and bound}
The upper bounding techniques lead to a constant approximation ratio for the solution returned by the branch and bound framework. In particular, we have the following theorem.

\begin{thm}\label{ratio}
	The branch and bound framework with sampling-based computeBound achieves an approximation factor of ($1-1/e-\epsilon$) for the $\prob$ through setting an appropriate parameter $X$.
\end{thm}
$Proof.$ Let $P$ denote the solution outputted by Algorithm~3 and $P^*$ denote the optimal solution for sampling-based computeBound.
As we analyzed above, Algorithm~3 achieves a ($1-1/e-\epsilon$) approximation factor through setting an appropriate parameter $X$ with at least ($1-\delta$) probability. Then we have %$\overline{\object}(P|{\ru}) \ge (1-1/e-\epsilon)(\overline{\object}(P^*|{\ru})\ge (1-1/e-\epsilon)({\object}(P^* |{\ru}).$
\begin{equation*}
\begin{aligned}
\overline{\object}(P|{\ru}) &\ge (1-1/e-\epsilon)(\overline{\object}(P^*|{\ru})\\&\ge (1-1/e-\epsilon)({\object}(P^* |{\ru}).
\end{aligned}
\end{equation*}

Let $P_{out}$ denote the returned solution by Algorithm~2. For any branch that has not been searched, under the termination condition $L<U$. Then we have $\object(P_{out}|{\ru})\ge \overline{\object}(P|{\ru})$. Therefore, Algorithm~2 achieves $\object(P_{out}|{\ru})\ge (1-1/e-\epsilon)({\object}(P^* |{\ru})$.
$\blacksquare$





% Table generated by Excel2LaTeX from sheet 'Class-wise'
\begin{tabular}{cccccccccccccc}
\toprule
MVTec-AD (AS) & $K=0$ & \multicolumn{4}{c}{$K=1$}     & \multicolumn{4}{c}{$K=2$}     & \multicolumn{4}{c}{$K=4$} \\
\cmidrule(lr){2-2} \cmidrule(lr){3-6} \cmidrule(lr){7-10} \cmidrule(l){11-14}
PRO   & \textbf{WinCLIP} & SPADE & PaDiM & PatchCore & \textbf{WinCLIP+} & SPADE & PaDiM & PatchCore & \textbf{WinCLIP+} & SPADE & PaDiM & PatchCore & \textbf{WinCLIP+} \\
\cmidrule(r){1-1} \cmidrule(lr){2-2} \cmidrule(lr){3-6} \cmidrule(lr){7-10} \cmidrule(l){11-14}
Bottle & 76.4\dev{0.0} & 91.1\dev{0.4} & 89.8\dev{0.8} & 93.5\dev{0.3} & 91.2\dev{0.4} & 91.8\dev{0.5} & 91.7\dev{0.2} & 93.9\dev{0.3} & 91.8\dev{0.3} & 92.5\dev{0.1} & 92.2\dev{0.2} & 94.0\dev{0.2} & 91.6\dev{0.2} \\
Cable & 42.9\dev{0.0} & 63.5\dev{0.7} & 59.1\dev{3.2} & 84.7\dev{1.0} & 72.5\dev{2.3} & 66.7\dev{0.9} & 66.5\dev{2.8} & 88.5\dev{0.9} & 74.7\dev{2.3} & 69.5\dev{0.4} & 74.2\dev{1.8} & 91.7\dev{0.6} & 77.0\dev{1.1} \\
Capsule & 62.1\dev{0.0} & 92.7\dev{0.4} & 80.0\dev{2.0} & 83.9\dev{0.9} & 85.6\dev{2.7} & 93.4\dev{0.3} & 82.3\dev{2.1} & 86.6\dev{1.0} & 90.6\dev{0.6} & 94.1\dev{0.6} & 85.7\dev{1.3} & 87.8\dev{1.9} & 90.1\dev{1.5} \\
Carpet & 84.1\dev{0.0} & 96.1\dev{0.0} & 92.9\dev{0.3} & 93.3\dev{0.3} & 97.4\dev{0.4} & 96.2\dev{0.0} & 93.9\dev{0.2} & 93.7\dev{0.4} & 97.3\dev{0.3} & 96.3\dev{0.0} & 94.4\dev{0.2} & 93.9\dev{0.4} & 97.0\dev{0.2} \\
Grid  & 57.0\dev{0.0} & 67.7\dev{1.9} & 41.2\dev{4.6} & 21.7\dev{9.5} & 90.5\dev{2.7} & 72.1\dev{1.5} & 45.1\dev{3.6} & 23.7\dev{3.8} & 92.8\dev{2.5} & 78.0\dev{1.5} & 55.5\dev{3.4} & 30.4\dev{4.6} & 93.6\dev{0.6} \\
Hazelnut & 81.6\dev{0.0} & 94.9\dev{0.3} & 85.7\dev{1.9} & 88.3\dev{1.3} & 93.7\dev{0.9} & 95.6\dev{0.2} & 89.4\dev{0.9} & 89.8\dev{1.3} & 94.2\dev{0.3} & 95.6\dev{0.1} & 90.4\dev{0.7} & 92.0\dev{0.3} & 94.2\dev{0.3} \\
Leather & 91.1\dev{0.0} & 98.7\dev{0.0} & 95.6\dev{0.2} & 95.2\dev{1.0} & 98.6\dev{0.0} & 98.8\dev{0.0} & 96.2\dev{0.2} & 95.9\dev{0.3} & 98.3\dev{0.4} & 98.8\dev{0.0} & 96.3\dev{0.1} & 96.4\dev{0.1} & 98.0\dev{0.4} \\
Metal nut & 31.8\dev{0.0} & 73.4\dev{1.1} & 38.1\dev{1.6} & 66.7\dev{2.9} & 84.7\dev{1.1} & 78.1\dev{1.8} & 48.2\dev{5.0} & 79.6\dev{4.2} & 86.7\dev{0.8} & 81.2\dev{1.4} & 54.0\dev{8.8} & 83.8\dev{5.5} & 89.4\dev{0.1} \\
Pill  & 65.0\dev{0.0} & 92.8\dev{0.3} & 78.9\dev{0.6} & 89.5\dev{1.6} & 93.5\dev{0.2} & 93.3\dev{0.2} & 84.3\dev{0.4} & 91.6\dev{0.5} & 94.5\dev{0.2} & 93.9\dev{0.2} & 86.6\dev{0.4} & 92.5\dev{0.4} & 94.6\dev{0.3} \\
Screw & 68.5\dev{0.0} & 85.0\dev{0.8} & 51.6\dev{1.7} & 68.1\dev{1.3} & 82.3\dev{1.1} & 87.2\dev{1.2} & 69.5\dev{2.1} & 69.0\dev{2.1} & 84.1\dev{0.5} & 89.5\dev{1.3} & 72.3\dev{0.8} & 72.4\dev{3.1} & 86.3\dev{1.8} \\
Tile  & 51.2\dev{0.0} & 84.2\dev{0.4} & 66.7\dev{1.5} & 82.5\dev{1.1} & 89.4\dev{0.4} & 84.6\dev{0.2} & 71.9\dev{0.5} & 82.5\dev{0.5} & 89.6\dev{0.4} & 84.9\dev{0.1} & 73.6\dev{0.9} & 83.0\dev{0.1} & 89.9\dev{0.3} \\
Toothbrush & 67.7\dev{0.0} & 83.5\dev{1.3} & 82.1\dev{1.5} & 79.0\dev{2.4} & 85.3\dev{1.0} & 87.4\dev{1.1} & 83.3\dev{2.6} & 81.0\dev{0.7} & 84.7\dev{1.4} & 89.0\dev{1.1} & 87.1\dev{1.7} & 85.5\dev{3.0} & 86.0\dev{3.3} \\
Transistor & 43.4\dev{0.0} & 55.3\dev{2.0} & 70.3\dev{7.0} & 70.9\dev{4.6} & 65.0\dev{1.8} & 57.6\dev{1.4} & 76.5\dev{5.5} & 78.8\dev{1.5} & 68.6\dev{1.1} & 58.5\dev{0.7} & 82.2\dev{7.4} & 79.5\dev{2.8} & 69.0\dev{1.1} \\
Wood  & 74.1\dev{0.0} & 92.9\dev{0.1} & 86.5\dev{0.6} & 87.1\dev{1.0} & 91.0\dev{0.6} & 93.1\dev{0.1} & 88.0\dev{0.2} & 86.8\dev{1.4} & 91.8\dev{0.6} & 93.2\dev{0.1} & 88.4\dev{0.2} & 87.7\dev{0.4} & 91.7\dev{0.3} \\
Zipper & 71.7\dev{0.0} & 86.8\dev{0.6} & 81.7\dev{2.0} & 91.2\dev{1.1} & 86.0\dev{1.7} & 89.0\dev{0.4} & 85.6\dev{0.7} & 92.8\dev{0.4} & 86.4\dev{1.6} & 90.1\dev{0.2} & 87.2\dev{0.8} & 93.4\dev{0.2} & 86.9\dev{0.7} \\
\cmidrule(r){1-1} \cmidrule(lr){2-2} \cmidrule(lr){3-6} \cmidrule(lr){7-10} \cmidrule(l){11-14}
Mean  & \textbf{64.6\dev{0.0}} & 83.9\dev{0.7} & 73.3\dev{2.0} & 79.7\dev{2.0} & \textbf{87.1\dev{1.2}} & 85.7\dev{0.7} & 78.2\dev{1.8} & 82.3\dev{1.3} & \textbf{88.4\dev{0.9}} & 87.0\dev{0.5} & 81.3\dev{1.9} & 84.3\dev{1.6} & \textbf{89.0\dev{0.8}} \\
\bottomrule
\end{tabular}%


\section{Experiments}
\label{sec:exp}

In this section, we demonstrate the wide range of applications and the high capabilities of Uni-Fusion. 
First, we evaluate Uni-Fusion in application 1) Incremental surface and color reconstruction, comparing its performance with SOTAs.
%
For applications 2) and 5), which are new topics, no specific benchmarks are available. 
Therefore, we showcase the performance on existing results.
%
Next, we implement application 3) and compare it with SOTA zero-shot semantic segmentation models.
%
Finally, for application 4), since infrared data is not commonly used, we collect our own dataset containing infrared values and show all applications on this data.

\subsection{Implementation Details}
\label{sec:exp:details}

In the experiments, we use our sample-based GPIS for local geometry encoding.
For each point, two additional points are sampled along normal direction, one positive and one negative, with distance $d_s=0.1$ in the local voxel's normalized space. 
Compared to derivative-based GPIS, our sample-based GPIS is more efficient in both space and time. 
For the encoder, we randomly sample $256$ anchor points from the range $[-0.5,0.5]^3$.
We utilize the first $20$ eigenpairs, resulting in a feature dimension of $20$.
The model selection process is discussed in the ablation study.

Different latent maps use different granularities.
For the surface LIM, we use a voxel size of $5\si{\centi\meter}$. 
For color which requires later comparison to NeRF, we use a voxel size of $2\si{\centi\meter}$.
For other property LIM and feature LIM, we use a voxel size of $10\si{\centi\meter}$.

For smooth reconstruction, the encoded voxel is designed overlapped following~\cite{huang2021di}.
The encoded voxel uses twice the voxel size, resulting in a half-space overlap with each neighboring voxel.
During meshing, SDFs are retrieved and interpolated from the overlapped voxels~\cite{huang2021di}.
While for the remaining properties, we sample only from its own voxel part.

The implementation runs on a PC with AMD Ryzen 9 5950X 16-core CPU and an Nvidia Geforce RTX 3090 GPU (24 GB).

\subsection{Datasets}

We evaluate incremental reconstruction on the ScanNet dataset~\cite{dai2017scannet}, TUM RGB-D dataset~\cite{sturm2012benchmark}, and Replica dataset~\cite{sucar2021imap}.
Using MSG-Net~\cite{zhang2018multi}'s material set, we transfer styles to the 3D canvas.
For open-vocabulary scene understanding, we evaluate on ScanNet segmentation data~\cite{qi2017pointnet++} and S3DIS dataset~\cite{armeni20163d}.

\subsubsection{ScanNet~\cite{dai2017scannet}}

ScanNet is a densely annotated RGB-D video dataset.
It is captured with the structure sensor~\cite{occipital} and contains 1513 scenes for training and validation.
For each scene, both images and a 3D mesh is provided, along with their 2D and 3D semantic annotations. 

ScanNet provides 312 scenes for validation, which contains a wide range of different room structures.
It has now been widely used in the thorough evaluation of the performance of reconstruction and semantic segmentation.

\subsubsection{TUM RGB-D~\cite{sturm2012benchmark}}

TUM RGB-D is a benchmark to mainly evaluate the tracking performance.
It is captured with Microsoft Kinect sensor together with ground-truth trajectory from the sensor.

\subsubsection{Replica~\cite{sucar2021imap}}

The Replica dataset refers to iMAP's pre-processed dataset~\cite{sucar2021imap}.
It is a synthetic rendered RGB-D dataset from given 3D models.
The advantage of including this dataset is that Replica does not have motion blur. 
This is better to evaluate the capability of the algorithms on reconstructing surface color.

\subsubsection{MSG-Net Style~\cite{zhang2018multi}}

MSG-Net provides material images for transfering the styles.
We select 21style fold for demonstration.
These images are given in \cref{fig:style} together with our result.

\subsubsection{ScanNet Point Cloud Segmentation Data~\cite{qi2017pointnet++}}

For point cloud semantic segmentation benchmarking, PointNet++~\cite{qi2017pointnet++} preprocesses the original ScanNet~\cite{dai2017scannet} and generates subsampled point clouds and corresponding annotations for each scene.

\subsubsection{S3DIS~\cite{armeni20163d} and 2D-3D-S~\cite{armeni2017joint}}

S3DIS is a semantic segmentation dataset for 3D point clouds.
Which is also a subset of the 2D-3D-S dataset.
The 2D-3D-S dataset is a multi-modality dataset containing 2D, 2.5D and 3D domains. 
This dataset is densely annotated with semantic labels.

Note that 2D-3D-S's 2D captures is not a RGB-D video as ScanNet.
2D-3D-S's images only have small overlap. 
Therefore, it is only suitable for semantic segmentation and not for incremental reconstruction.

\subsection{Baselines}

For online surface mapping evaluation, we select TSDF-Fusion~\cite{curless1996volumetric}, iMAP~\cite{sucar2021imap}, SOTA DI-Fusion~\cite{huang2021di} and BNV-Fusion~\cite{li2022bnv} as four baseline methods.

For the color field, we choose TSDF-Fusion~\cite{curless1996volumetric}, $\sigma$-Fusion~\cite{rosinol2023probabilistic}, iMAP~\cite{sucar2021imap}, NICE-SLAM~\cite{zhu2022nice} and even the recent hot Neural Radiance Fields model NeRF-SLAM~\cite{rosinol2022nerf} as five baselines.
While including NeRF in the comparison may not be entirely fair, we want to show how Uni-Fusion narrows the performance gap.

For the scene understanding application, we evaluate generalized zero-shot point cloud semantic segmentation with ZSLPC~\cite{cheraghian2019zero}, DeViSe~\cite{frome2013devise} and SOTA 3DGenZ~\cite{michele2021generative} for comparison.

\subsection{Metrics}

For incremental reconstruction, we evaluate the geometric reconstruction using \textbf{Accuracy}, \textbf{Completeness}, and \textbf{F1 score} according to SOTA BNV-Fusion. It firstly uniformly samples $100,000$ points from the reconstruction and ground truth meshes respectively.
Then \textbf{Accuracy} (\textbf{Completeness}) measures the percentage of reconstruction-to-groundtruth (groundtruth-to-reconstruction) distances that are lower than $2.5\si{\centi\meter}$ threshold. \textbf{F1 score} is the harmonic mean of accuracy and completeness.
For tracking performance, we use \textbf{ATE RMSE}.

To evaluate color reconstruction, we follow SOTA on this topic, NeRF to render both depth and RGB images to compare the image level \textbf{Depth L1} and \textbf{RGB PSNR}.

To compare scene understanding, we follow zero-shot point cloud semantic segmentation SOTA 3DGenZ to evaluate the \textbf{Intersection-of-Union (IoU)} and \textbf{Accuracy}.


\subsection{Reconstruction Results}

For evaluation, we first use the ScanNet validation set with 312 sequences to thoroughly test the geometric reconstruction on a large variant of scenes.
%
Then, we use TUM RGB-D to compare our modified tracking model with related works.
Because this part is not the main contribution, we give a rough overview of the tracking results.
%
To fairly evaluate the color reconstruction, we use the high quality rendered Replica dataset to compare with related works, including NeRF.

%\subsubsection{Object}
% on instance-gp
% Objective data usually has more fine detail
% 1. for detail precision
% A: no, object reconstruction is not as good as instance-ngp, so cancelled.
\begin{table*}[!]
	\centering
	\caption{Comparison to ScanNet~\cite{dai2017scannet}.
       Our method generalizes better to various scenes.
       $^*$ indicates the result from our runs of the official BNV-Fusion code.}
	\small
	%\setlength{\tabcolsep}{5mm}
	\setlength{\tabcolsep}{0.9em}
		%\resizebox{\textwidth}{!}{
		\begin{tabular}{l  c c c| c c c }
			\toprule
			Method & \begin{tabular}{@{}c@{}}Pre-Train\\ with extra dataset\end{tabular} & \begin{tabular}{@{}c@{}}Train \\ with sequences\end{tabular} & Real-time & Accuracy (\%) $\uparrow$ & Completeness (\%) $\uparrow$ & F1 score $\uparrow$ \\
			\midrule
			TSDF Fusion~\cite{zhou2018open3d} & None & None & $\checkmark$ &73.83 & 85.85 & 78.84 \\
			iMAP~\cite{sucar2021imap} & None & Online train& &68.96 & 82.12 & 74.96 \\
			DI-Fusion~\cite{huang2021di} &Object Pretrain & None & $\checkmark$&66.34 & 79.65 & 72.97 \\
			BNV-Fusion~\cite{li2022bnv} &Object Pretrain &  Post Optimization& &{74.90} & \textbf{88.12} & {80.56} \\
			BNV-Fusion$^{*}$~\cite{li2022bnv} &Object Pretrain & Post Optimization &&{73.42} & {81.75} & {77.18} \\
			\textbf{Uni-Fusion (Ours)} &None &None &$\checkmark$&\textbf{80.43} & {84.91} & \textbf{82.44} \\
			\bottomrule
		\end{tabular}
	  %}
	\label{tab:scannet}
	\vspace{-.6cm}
\end{table*}
\begin{figure*}[t]
	\subfloat[width=.33\textwidth][Accuracy]{
		\centering
		\includegraphics[width=.22\linewidth]{im/exp/recons/scannet/scannet_acc.png}
		\includegraphics[width=.1\linewidth]{im/exp/recons/scannet/scannet_acc_box.png}
	}
	\subfloat[width=.33\textwidth][Completeness]{
		\centering
		\includegraphics[width=.22\linewidth]{im/exp/recons/scannet/scannet_comp.png}
		\includegraphics[width=.1\linewidth]{im/exp/recons/scannet/scannet_comp_box.png}
	}
	\subfloat[width=.33\textwidth][F1 score]{
		\centering
		\includegraphics[width=.22\linewidth]{im/exp/recons/scannet/scannet_F1.png}
		\includegraphics[width=.1\linewidth]{im/exp/recons/scannet/scannet_F1_box.png}
	}
	\label{fig:recon:scannet:elementwise}
	\caption{Quantitative comparison on 312 scenes of the ScanNet validation set.
       We demonstrate the performance of SOTA BNV-Fusion and our Uni-Fusion.
       We sort our evaluation value and reordered all of the scores.
       The zigzag pink is the BNV-Fusion result;
       we also plot a deep-pink smoothed curve for better visualization.}
\end{figure*}

\subsubsection{Evaluation on ScanNet Dataset~\cite{dai2017scannet}}
\label{sec:exp:scannet}

We use the 312 diversified scenes from the ScanNet validation set to evaluate the performance of surface reconstruction. 
We follow the pure mapping SOTA BNV-Fusion to take every 10th posed frame as input. 
%
Without using any learning (in contrast iMAP, DI-Fusion, and BNV-Fusion do) or any post optimization (as BNV-Fusion does), our Uni-Fusion is capable to achieve precise continuous mapping performance. 

As shown in~\cref{tab:scannet}, our Uni-Fusion achieves \textbf{$+6$ higher accuracy} than the incremental surface reconstruction SOTA BNV-Fusion.
Our model does not exceed on completeness comparing to BNV-Fusion that support completion in post-optimization.
Though, Uni-Fusion's completion is still much higher than one other optimization based model iMAP.
%We consider it because our model does not support hole-completion as the optimization based models iMap and BNV-Fusion.
Overall, our Uni-Fusion model achieves higher F1-scores that quantifies the overall quality.

Please note that, SOTA BNV-Fusion is not real-time capable, since it requires post optimization of all fed frames.
Without the post-optimization, the real-time model Di-Fusion shows much worse results.
However, our \textbf{real-time} model \textbf{Uni-Fusion} is able to achieves \textbf{much better} reconstruction quality than these approaches even without post-optimization. 

\newcommand{\scannetImSize}{.16}
\begin{figure*}[t!]
	\centering
	\setlength{\tabcolsep}{0.1em}
	\renewcommand{\arraystretch}{.1}
	\begin{tabular}{|c | c |c |||c |c | c|}
		\hline
		{\Large{BNV-Fusion}} & {\Large{Uni-Fusion}} &{\Large{Ground Truth}} & {\Large{BNV-Fusion}} &{\Large{Uni-Fusion}} & {\Large{Ground Truth}} \\ \hline \hline
		
\includegraphics[width=\scannetImSize\linewidth]{im/exp/recons/scannet_qualifi/scene0568_00_bnv.png}
		&\includegraphics[width=\scannetImSize\linewidth]{im/exp/recons/scannet_qualifi/scene0568_00_mine.png}
		&\includegraphics[width=\scannetImSize\linewidth]{im/exp/recons/scannet_qualifi/scene0568_00_gt.png}
		&		\includegraphics[width=\scannetImSize\linewidth]{im/exp/recons/scannet_qualifi/scene0164_00_bnv.png}
		&\includegraphics[width=\scannetImSize\linewidth]{im/exp/recons/scannet_qualifi/scene0164_00_mine.png}
		&\includegraphics[width=\scannetImSize\linewidth]{im/exp/recons/scannet_qualifi/scene0164_00_gt.png}\\
		
\includegraphics[width=\scannetImSize\linewidth]{im/exp/recons/scannet_qualifi/scene0249_00_bnv.png}
		&\includegraphics[width=\scannetImSize\linewidth]{im/exp/recons/scannet_qualifi/scene0249_00_mine.png}
		&\includegraphics[width=\scannetImSize\linewidth]{im/exp/recons/scannet_qualifi/scene0249_00_gt.png}
		&		\includegraphics[width=\scannetImSize\linewidth]{im/exp/recons/scannet_qualifi/scene0435_00_bnv.png}
		&\includegraphics[width=\scannetImSize\linewidth]{im/exp/recons/scannet_qualifi/scene0435_00_mine.png}
		&\includegraphics[width=\scannetImSize\linewidth]{im/exp/recons/scannet_qualifi/scene0435_00_gt.png}\\
		
\includegraphics[width=\scannetImSize\linewidth]{im/exp/recons/scannet_qualifi/scene0046_00_bnv.png}
		&\includegraphics[width=\scannetImSize\linewidth]{im/exp/recons/scannet_qualifi/scene0046_00_mine.png}
		&\includegraphics[width=\scannetImSize\linewidth]{im/exp/recons/scannet_qualifi/scene0046_00_gt.png}
		&		\includegraphics[width=\scannetImSize\linewidth]{im/exp/recons/scannet_qualifi/scene0050_00_bnv.png}
		&\includegraphics[width=\scannetImSize\linewidth]{im/exp/recons/scannet_qualifi/scene0050_00_mine.png}
		&\includegraphics[width=\scannetImSize\linewidth]{im/exp/recons/scannet_qualifi/scene0050_00_gt.png}\\
		\hline
	\end{tabular}
	%\captionof{figure}
	\caption{Surface reconstruction on ScanNet dataset.}
	\label{fig:recons:scannet_demo}
	\vspace{-.5cm}
\end{figure*}

We additionally run BNV-Fusion's official implementation (emphasized with $^*$) on the 312 videos of ScanNet and do a post element-wise comparison in \cref{fig:recon:scannet:elementwise}. 
Our result is the {\color{Cyan}light blue} curve, BNV-Fusion's result is colored with {\color{Lavender}pink}.
Scene index is sorted corresponding to the score value of Uni-Fusion.
For better visualization, we smooth BNV-Fusion's curve and plot it with dark pink.
It is obvious that the score of Uni-Fusion is overall higher than BNV-Fusion's. 
Moreover, we use box-plot to conclude the statistics besides the curve plot. Uni-Fusion's scores are distributed in a higher region. For completeness which is less obvious better, Uni-Fusion's box is smaller while in a relative higher position. This means that Uni-Fusion has more stable completeness result while BNV-Fusion is more likely to get low completeness in some cases.

To summarize, our model is almost better on all 312 scenes on all accuracy, completeness and F1-score.
Which is also revealed in \cref{tab:scannet} with BNV-Fusion$^*$, that the BNV-Fusion's official implementation does not exceed Uni-Fusion on all metrics.

We plot reconstruction on selected scenes from ScanNet in~\cref{fig:recons:scannet_demo}. 
Both BNV-Fusion and our Uni-Fusion are able to produce high quality reconstruction.
We see that BNV-Fusion gives lots of small meshes on walls, which are shown as small particles in the reconstruction. 
We consider it is because BNV-Fusion use very small voxel size ($0.02\si{\meter}$) to get a high score.
This is also revealed by their \textbf{\SI{247}{MB}} mesh in average, while ours is only \textbf{\SI{54}{Mb}} in average.
Furthermore, our Uni-Fusion's mesh is more smooth and
%Both BNV-Fusion and Uni-Fusion demonstrate high quality result.
also provides high-precise color to the mesh which is not available for the Surface SOTA.

%In this test, we purely evaluate the surface reconstruction capacity with SOTAs. 
%While reconstruction is not merely surface.  
%Thus in the following, we find benchmarks for both surface and color.


\subsubsection{Tracking Evaluation on TUM RGB-D Dataset~\cite{sturm2012benchmark}}
% follow nice-slam

In the above test, we compare the performance of pure mapping.
While tracking is not the contribution focus in our paper, it is part of the reconstruction model. We follow the novel reconstruction model NICE-SLAM~\cite{zhu2022nice} to evaluate the camera tracking on the small-scale TUM RGB-D dataset.
Our Uni-Fusion uses a coarse-to-fine strategy for 3D reconstruction tracking.
From~\cref{tab:tum_rmse}, it demonstrates overall better ATE RMSE than other implicit representation models.

\begin{table}[]
		\caption{Tracking on TUM RGB-D~\cite{sturm2012benchmark}.
		ATE RMSE [$\si{\centi\meter}$] ($\downarrow$) is used as the evaluation metric.
	}
	\centering
	\footnotesize
	\setlength{\tabcolsep}{0.7em}
	\resizebox{\linewidth}{!}{
		\begin{tabular}{l|ccc}
			\hline
			& \tt{fr1/desk} &  \tt{fr2/xyz} &  \tt{fr3/office} \\
			
			\hline
			{iMAP}~\cite{sucar2021imap}      & 4.9 & 2.0 & 5.8  \\
			{iMAP$^*$}~\cite{sucar2021imap} & 7.2 & 2.1  & 9.0 \\
			{DI-Fusion~\cite{huang2021di}} & 4.4 & 2.3 & 15.6 \\
			NICE-SLAM~\cite{zhu2022nice}           & 2.7 & 1.8 & 3.0 \\
			Ours& 1.8& 0.5& 2.1 \\
			\hline
			{BAD-SLAM}\cite{schops2019bad} & 1.7  & 1.1  & 1.7 \\
			{Kintinuous}\cite{whelan2012kintinuous} & 3.7  &  2.9  & 3.0 \\
			{ORB-SLAM2}\cite{mur2017orb} & \bf 1.6  & \bf 0.4  & \bf 1.0 \\
			\hline
	\end{tabular}}
	\vspace{2pt}

	\label{tab:tum_rmse}
\end{table}

On the other hand, there also exist high accuracy algorithms from SLAM. 
By additional using Bundle Adjustment and Loop-closing techniques, their tracking quality is much better than all of the implicit based models.

%But it is dangerous to directly apply SLAM result on reconstruction. Please find our demonstration in Fig [?]. Which explains the more widely used frame-to-model strategy in 3D reconstruction.
Even though, our coarse-to-fine strategy firstly ensure not easy to lose track. Secondly, it is more suitable for surface fitting.

Which further support our test in Replica dataset.



\begin{table*}[t!]
	\centering
	\caption{Geometric (L1) and Photometric (PSNR) evaluation on the Replica dataset~\cite{sucar2021imap}.}
	\footnotesize
	\setlength{\tabcolsep}{0.36em}
	\renewcommand{\arraystretch}{1.2}
	\begin{tabular}{clcccccccccccccccccc}
		\toprule
		& & \multicolumn{1}{c}{\makecell{\tt{office-0}}} & \multicolumn{1}{c}{\makecell{\tt{office-1}}} & \multicolumn{1}{c}{\makecell{\tt{office-2}}}& \multicolumn{1}{c}{\makecell{\tt{office-3}}} & \multicolumn{1}{c}{\makecell{\tt{office-4}}} & \multicolumn{1}{c}{\makecell{\tt{room-0}}} & \multicolumn{1}{c}{\makecell{\tt{room-1}}} &  \multicolumn{1}{c}{\makecell{\tt{room-2}}} & Avg. \\
		\midrule
		\multicolumn{5}{l}{\textit{Non-continuous mapping method}}\\
		\multirow{2}{*}{\makecell{\textbf{TSDF-Fusion}~\cite{curless1996volumetric}}}
		& {\bf Depth L1} [$\si{\centi\meter}$] $\downarrow$
	 & 14.11 & 10.50 & 30.89 & 28.92 & 22.83	& 23.51 & 20.94 & 23.34 & 21.88 \\
		& {\bf PSNR } [$\si{\dB}$] $\uparrow$
		& 11.16 & 15.92 & 4.86 & 5.68 & 5.46 & 3.43 & 4.51 & 5.57 & 7.07 \\
		
		\midrule
		\multirow{2}{*}{\makecell{\textbf{$\sigma$-Fusion}\cite{rosinol2023probabilistic} }}
		& {\bf Depth L1} [$\si{\centi\meter}$] $\downarrow$
		 & 13.80 & 10.21 & 22.27 & 28.70 & 22.21& 21.92 & 19.28 & 22.40 & 20.10 \\
		& {\bf PSNR } [$\si{\dB}$] $\uparrow$
		 & 11.16 & 15.92 & 4.86 & 5.69 & 5.46& 3.45  & 4.51 & 5.57 & 7.08 \\
		
		
		
		
		
		\midrule
		\midrule
		\multicolumn{5}{l}{\textit{Continuous mapping method}}\\
		\multirow{2}{*}{\makecell{\textbf{iMAP$^*$}~\cite{sucar2021imap}}}
		& {\bf Depth L1} [$\si{\centi\meter}$] $\downarrow$
		 & 6.43 & 7.41 & 14.23 & 8.68 & 6.80& 5.70 & 4.93 & 6.94 & 7.64\\
		& {\bf PSNR } [$\si{\dB}$] $\uparrow$
		& 7.39 & 11.89 & 8.12 & 5.62 & 5.98& 5.66 & 5.31 & 5.64  & 6.95\\
		\midrule
		\multirow{2}{*}{{\makecell{\textbf{Nice-SLAM}~\cite{zhu2022nice} }}}
		& {\bf Depth L1} [$\si{\centi\meter}$] $\downarrow$
		& { 1.51 } & { 0.93 } & { 8.41 } & { 10.48 } & {2.43} & { 2.53 } & { 3.45 } & { 2.93 }  & { 4.08 } \\
		& {\bf PSNR } [$\si{\dB}$] $\uparrow$
		 & { 22.44 } & { 25.22 } & { 22.79 } & { 22.94 } & { 24.72 } & \textbf{ 29.90 } & \textbf{ 29.12 } & { 19.80 }& { 24.61 } \\
		
		
		
		
		\midrule	
		\multirow{2}{*}{{\makecell{\textbf{Uni-Fusion} (Ours) }}}
		% using abs(diff)
		%	& {\bf Depth L1} [$\si{\centi\meter}$] $\downarrow$ &\textbf{1.98}&\textbf{1.18}&\textbf{1.64}&\textbf{1.23}&\textbf{0.84}&\textbf{1.61}&\textbf{3.01}&\textbf{1.60} &\textbf{1.64}
		% follow nerf-slam to remove outlier gt first
		& {\bf Depth L1} [$\si{\centi\meter}$] $\downarrow$
		& \textbf{0.79}&\textbf{0.56}&\textbf{1.59}&\textbf{2.71}&\textbf{1.66}&\textbf{1.94}&\textbf{0.69}&\textbf{1.80}& \textbf{1.47}
		\\
		& {\bf PSNR } [$\si{\dB}$] $\uparrow$ &\textbf{33.88}&\textbf{33.31}&\textbf{25.84}&\textbf{26.01}&\textbf{28.14}&24.02&26.20&\textbf{27.17} &\textbf{28.07}
		\\
		
		\midrule
		\midrule
		\multicolumn{5}{l}{\textit{Neural radiance field method}}\\
		\multirow{2}{*}{{\makecell{\textbf{NeRF-SLAM}~\cite{rosinol2022nerf} }}}
		& {\bf Depth L1} [$\si{\centi\meter}$] $\downarrow$
	 & {2.49}   & {1.98}  & {9.13}  & {10.58} & {3.59}	& {2.97}  & {2.63}  & {2.58}  & {4.49} \\
		& {\bf PSNR } [$\si{\dB}$] $\uparrow$
	 & \textbf{48.07}  & \textbf{53.44} & \textbf{39.30} & \textbf{38.63} & \textbf{39.21} 	& \textbf{34.90} & \textbf{36.95} & \textbf{40.75}& \textbf{41.40} \\
		
		\bottomrule
	\end{tabular}%
	
	\label{tab:replica_per_scene}
\end{table*}


\begin{table*}[t!]
	\centering
	\caption{Differences among different Surface \& Color reconstruction models.}
	\small
	\setlength{\tabcolsep}{.6em}
	%{
		%\resizebox{\textwidth}{!}{
			\begin{tabular}{l | c c c c c c }
				\toprule
				Method & 
				\begin{tabular}{@{}c@{}}Pre-Train\\ with extra dataset\end{tabular}
				& \begin{tabular}{@{}c@{}}Train\\ with sequences\end{tabular}
				& Real-time	
				& Direct Output &  \begin{tabular}{@{}c@{}}Light\\ direction\end{tabular} 
				&Render\\
				\hline 
				TSDF-Fusion & None & None & $\checkmark$& Discrete TSDF &  &Ray Rasterization\\\hline
				$\sigma$-Fusion & None & None &$\checkmark$&Discrete TSDF  && Ray Rasterization\\\hline
				iMAP & None & Online Train && MLPs  & &Volumetric Rendering\\\hline
				NICE-SLAM & \begin{tabular}{@{}c@{}}Pretrain\\ with indoor dataset\end{tabular} & Online Train&& Neural Implicit Grid&  & Volumetric Rendering\\\hline
				
				NeRF-SLAM & None & Train hundred epochs &-&NeRF & $\checkmark$ &Volumetric Rendering \\\hline
				
				\textbf{Uni-Fusion} & None & None&$\checkmark$& Latent Implicit Map && Ray Rasterization\\				
				\hline
			\end{tabular}
		%}
	%}
	\label{tab:replica_diff}
\end{table*}
\subsubsection{Evaluation on Replica RGB-D Dataset~\cite{sucar2021imap}}
In this evaluation, we compare with implicit reconstruction (TSDF-Fusion, $\sigma$-Fusion) and latent implicit reconstruction models (iMAP, NICE-SLAM) that support colors. 
We also add a large-scale NeRF model, NeRF-SLAM in to the table.  
NeRF is SOTA in view-synthesis task, which is unfair to direct compare with the rest. As the rest model does not even model light directions.
We add NeRF in this part to demonstrate that Uni-Fusion strongly reduce the gap.
Note that, NeRF-SLAM embeds external tracking model ~\cite{teed2021droid} to provide poses while using SOTA NeRF implementation Instance-ngp~\cite{muller2022instant} for NeRF construction.
%Therefore it is considered the SOTA to model the colors.

Uni-Fusion track and follow our previous setting in ScanNet test to take every 10 frames for mapping.
NICE-SLAM and NeRF-SLAM produce depth and color by rendering,
To obtain result from Uni-Fusion, we cast rays from virtual camera to our result surface mesh for depth image. 
Then Uni-Fusion infer the cast points in Uni-Fusion's color LIM for color result.

From~\cref{tab:replica_per_scene}, Uni-Fusion demonstrate
best Depth L1 on all scenes with an average of \textbf{$\pmb{1.47}$$\si{\centi\meter}$ depth L1}. Which is \textbf{$\pmb{177\%}$ boost} comparing to the second best.

Moreover, excluding NeRF, our Uni-Fusion also shows the best performance to model the colors with an average of $28.07$$\si{\dB}$ PSNR.

However, it is strange that NICE-SLAM lost details while in two cases, it shows better PSNR than Uni-Fusion. 
To highlight the true result,
we plot the rendered image in \cref{fig:replica_render}.
It is obvious that our Uni-Fusion models the details of painting, carpet and quilt well. 
While NICE-SLAM just roughly models the average color.

Moreover, from the  \cref{fig:replica_render}, our Uni-Fusion's rendering quality is as precise as NeRF. 
Please also find the painting, carpet and quilt, Uni-Fusion recovered the original appearance well.
Please find the {\color{green} green window} for the emphasized region.
Uni-Fusion reproduce the high-quality appearance which is very close to NeRF on qualitative evaluation.
%It can hardly find difference between the results from NeRF-SLAM, Uni-Fusion and Ground Truth.
%
But, Uni-Fusion still has a quantitative score gap to the color rending of NeRF ($41.4$$\si{\dB}$).
Though the Uni-Fusion's rendering result is highly close to NeRF and ground truth.
%
We consider the main reasons are that \textbf{1.} Uni-Fusion does not model the light directions to points, which is essential to NeRF.
\textbf{2.} NeRF optimizes on the rendering image quality by focussing mainly on color while less on depth.
It can be revealed by the higher color rendering score with much worse depth rendering than our Uni-Fusion.
\textbf{3.} our Uni-Fusion does not support hole filling.
This directly leads to black holes in our rendered images.

We summarize the differences in \cref{tab:replica_diff}.
Similar to TSDF-Fusion and $\sigma$-Fusion, our Uni-Fusion is a forward method which, does not need any training, i.e., pre- or online training. 
Uni-Fusion also produces similar to NICE-SLAM and NeRF-SLAM an implicit map with set of latent that outputs results at arbitrary resolution.
However, we differ on the extracting of the signed distance field.
%FIXME: I do not understand the next sentence.
Uni-Fusion's latent feature rule its own region independently.
Each query value is directly inferred with the corresponding ruling latent.
While NICE-SLAM and NeRF-SLAM use a much denser grid to interpolate query features. This requires volumetric rendering for inference.

Similar to TSDF-Fusion, $\sigma$-Fusion, our Uni-Fusion is also a real-time algorithm.
iMAP, NICE-SLAM and NeRF-SLAM run hardly in real-time.
NeRF-SLAM is claiming to be real-time, which is questionable as they still need hundreds of epochs training after feeding the data.

Nevertheless, optimization with backpropagation learns pixel-to-pixel well.
It is theoretically advanced for a regression-and-fusion strategy. 
Though Uni-Fusion demonstrates its high capability to model the color, NeRF-like post-optimization would still be a good direction for further improvements of Uni-Fusion.

\newcommand{\replicaImSize}{.24}
\begin{figure*}[t]
	\centering
	\setlength{\tabcolsep}{0.1em}
	\renewcommand{\arraystretch}{.1}
	\begin{tabular}{|c | c |c |c| }
		 \hline
		{\Large{NICE-SLAM}} &{\Large{NeRF-SLAM}}&\textbf{\Large{Uni-Fusion}}&\Large{Ground Truth}\\
		%		\hline
		%		\includegraphics[width=\replicaImSize\linewidth]{im/exp/recons/replica/nice-slam/of2_1286.png} &
		%		\includegraphics[width=\replicaImSize\linewidth]{im/exp/recons/replica/mine/of2_1286.jpg} &
		%		\includegraphics[width=\replicaImSize\linewidth]{im/exp/recons/replica/mine/of2_1286.jpg} &
		%		\includegraphics[width=\replicaImSize\linewidth]{im/exp/recons/replica/gt/of2_1286.jpg} \\
		
		\hline
		\includegraphics[width=\replicaImSize\linewidth]{im/exp/recons/replica/nice-slam/rm0_769_window.png} &
		\includegraphics[width=\replicaImSize\linewidth]{im/exp/recons/replica/nerf-slam/rm0_769_window.jpg} &
		\includegraphics[width=\replicaImSize\linewidth]{im/exp/recons/replica/mine/rm0_769_window.jpg} &
		\includegraphics[width=\replicaImSize\linewidth]{im/exp/recons/replica/gt/rm0_769_window.jpg} \\
		\hline
		\includegraphics[width=\replicaImSize\linewidth]{im/exp/recons/replica/nice-slam/of3_575_window.png} &
		\includegraphics[width=\replicaImSize\linewidth]{im/exp/recons/replica/nerf-slam/of3_575_window.jpg} &
		\includegraphics[width=\replicaImSize\linewidth]{im/exp/recons/replica/mine/of3_575_window.jpg} &
		\includegraphics[width=\replicaImSize\linewidth]{im/exp/recons/replica/gt/of3_575_window.jpg} \\
		\hline
		\includegraphics[width=\replicaImSize\linewidth]{im/exp/recons/replica/nice-slam/rm1_425_window.png} &
		\includegraphics[width=\replicaImSize\linewidth]{im/exp/recons/replica/nerf-slam/rm1_425_window.jpg} &
		\includegraphics[width=\replicaImSize\linewidth]{im/exp/recons/replica/mine/rm1_425_window.jpg} &
		\includegraphics[width=\replicaImSize\linewidth]{im/exp/recons/replica/gt/rm1_425_window.jpg} \\
		\hline
		\includegraphics[width=\replicaImSize\linewidth]{im/exp/recons/replica/nice-slam/rm2_1085_window.png} &
		\includegraphics[width=\replicaImSize\linewidth]{im/exp/recons/replica/nerf-slam/rm2_1085_window.jpg} &
		\includegraphics[width=\replicaImSize\linewidth]{im/exp/recons/replica/mine/rm2_1085_window.jpg} &
		\includegraphics[width=\replicaImSize\linewidth]{im/exp/recons/replica/gt/rm2_1085_window.jpg} \\		
		
	\end{tabular}
	%\captionof{figure}
	\caption{Demonstration of color rendering on the Replica dataset. Fine appearances are highlighted in {\color{green}green window}. Small flaws are in a {\color{red}red} box.}
	\label{fig:replica_render}
	\vspace{-.5cm}
\end{figure*}

%(2) NeRF model learning radiance field that model the light on different direction on surface. While Uni-Fusion naturally treat different directional light the same color.

%\begin{table*}[t!]
%	\centering
%	\setlength{\tabcolsep}{0.1em}
%	\renewcommand{\arraystretch}{.1}
%	\begin{tabular}{c | c |c |c |c }
%		\hline 
%		\rotatebox{90}{\large{NICE-SLAM}} &
%		\includegraphics[width=\replicaImSize\linewidth]{im/exp/recons/replica/nice-slam/of3_575.png} &
%		\includegraphics[width=\replicaImSize\linewidth]{im/exp/recons/replica/nice-slam/rm0_769.png} &
%		\includegraphics[width=\replicaImSize\linewidth]{im/exp/recons/replica/nice-slam/rm1_425.png} &
%		\includegraphics[width=\replicaImSize\linewidth]{im/exp/recons/replica/nice-slam/rm2_1085.png} \\
%		\hline
%		\rotatebox{90}{\large{NeRF-SLAM}} &
%		\includegraphics[width=\replicaImSize\linewidth]{im/exp/recons/replica/mine/of3_575.jpg} &
%		\includegraphics[width=\replicaImSize\linewidth]{im/exp/recons/replica/mine/rm0_769.jpg} &
%		\includegraphics[width=\replicaImSize\linewidth]{im/exp/recons/replica/mine/rm1_425.jpg} &
%		\includegraphics[width=\replicaImSize\linewidth]{im/exp/recons/replica/mine/rm2_1085.jpg} \\	
%		\hline
%		\rotatebox{90}{\textbf{\Large{Uni-Fusion}}} &
%		\includegraphics[width=\replicaImSize\linewidth]{im/exp/recons/replica/mine/of3_575.jpg} &
%		\includegraphics[width=\replicaImSize\linewidth]{im/exp/recons/replica/mine/rm0_769.jpg} &
%		\includegraphics[width=\replicaImSize\linewidth]{im/exp/recons/replica/mine/rm1_425.jpg} &
%		\includegraphics[width=\replicaImSize\linewidth]{im/exp/recons/replica/mine/rm2_1085.jpg} \\
%		\hline
%		\rotatebox{90}{\large{Ground Truth}} &
%		\includegraphics[width=\replicaImSize\linewidth]{im/exp/recons/replica/gt/of3_575.jpg} &
%		\includegraphics[width=\replicaImSize\linewidth]{im/exp/recons/replica/gt/rm0_769.jpg} &
%		\includegraphics[width=\replicaImSize\linewidth]{im/exp/recons/replica/gt/rm1_425.jpg} &
%		\includegraphics[width=\replicaImSize\linewidth]{im/exp/recons/replica/gt/rm2_1085.jpg} \\
%		\hline		
%		
%	\end{tabular}
%	\captionof{figure}{Demonstration of color rendering on Replica dataset.}
%\end{table*}

\subsection{Ablation study}
\label{exp:surface:ablation}

\begin{figure}[]
	\centering
%		\subfloat[width=\textwidth][Sample based]{
%		\centering
%		\includegraphics[width=.7\linewidth]{im/exp/ablation/GPIS/seq3_sample_color.png}
%	}\\
%	\subfloat[width=\textwidth][Derivative based]{
%		\centering
%		\includegraphics[width=.7\linewidth]{im/exp/ablation/GPIS/seq3_derivative_color.png}
%	}
		\includegraphics[width=.7\linewidth]{im/exp/ablation/GPIS/seq3_sample_color_a.png}
		\includegraphics[width=.7\linewidth]{im/exp/ablation/GPIS/seq3_derivative_color_b.png}
	\caption{Ablation study on surface construction basis. (a) Sample based. (b) Derivative based.}
	\label{fig:ablation:GPIS}
\end{figure}


\begin{table}[]
	\caption{Ablation study on tracking.
	}
	\centering
	\footnotesize
	\setlength{\tabcolsep}{0.7em}
	\resizebox{\linewidth}{!}{
		\begin{tabular}{l|ccc}
			\hline
			& \tt{fr1/desk} &  \tt{fr2/xyz} &  \tt{fr3/office} \\
			\hline
			External& 2.1& 0.5& 2.5 \\
			External+Internal&1.8& 0.5& 2.1 \\
			\hline
	\end{tabular}}
	\vspace{-2pt}
	%\vspace{-1cm}
	\label{tab:tum_rmse2}
\end{table}

\begin{figure}
	\centering
	\includegraphics[width=.7\linewidth]{im/exp/ablation/voxel_size/seq_voxel_size.png}
	%	\subfloat[width=.33\textwidth][0.1]{
		%		\centering
		%		\includegraphics[width=.33\linewidth]{im/exp/ablation/voxel_size/seq3_0_1_color.png}
		%	}
	%	\subfloat[width=.3\textwidth][0.05]{
		%		\centering
		%		\includegraphics[width=.33\linewidth]{im/exp/ablation/GPIS/seq3_sample_color.png}
		%	}
	%	\subfloat[width=.33\textwidth][0.02]{
		%	\centering
		%	\includegraphics[width=.33\linewidth]{im/exp/ablation/voxel_size/seq3_0_02_color.png}
		%	}
	\caption{Ablation study on voxel size.}
	\label{fig:ablation:voxel_size}
\end{figure}




 \newcommand{\styleImSize}{.2}
\begin{figure*}[b!]
	\vspace{-.5cm}
	\centering
	\setlength{\tabcolsep}{0.1em}
	\renewcommand{\arraystretch}{.1}
	\resizebox{\textwidth}{!}{\begin{tabular}{ccccc}
			%		\includegraphics[width=\styleImSize\line]{im/exp/style/style/0} &
			%		\includegraphics[width=\styleImSize\linewidth]{im/exp/style/style/1} &
			%		\includegraphics[width=\styleImSize\linewidth]{im/exp/style/style/2} &
			%		\includegraphics[width=\styleImSize\linewidth]{im/exp/style/style/3} &
			%		\includegraphics[width=\styleImSize\linewidth]{im/exp/style/style/4} &
			%		\includegraphics[width=\styleImSize\linewidth]{im/exp/style/style/5} &
			%		\includegraphics[width=\styleImSize\linewidth]{im/exp/style/style/6} \\
			\hline\hline
			\includegraphics[width=\styleImSize\linewidth]{im/exp/style/processed/office0_0.png} &
			\includegraphics[width=\styleImSize\linewidth]{im/exp/style/processed/office0_1.png} &
			\includegraphics[width=\styleImSize\linewidth]{im/exp/style/processed/office0_2.png} &
			\includegraphics[width=\styleImSize\linewidth]{im/exp/style/processed/office0_3.png} &
			\includegraphics[width=\styleImSize\linewidth]{im/exp/style/processed/office0_4.png} \\
			\includegraphics[width=\styleImSize\linewidth]{im/exp/style/processed/office0_5.png} &
			\includegraphics[width=\styleImSize\linewidth]{im/exp/style/processed/office0_6.png} &
			\includegraphics[width=\styleImSize\linewidth]{im/exp/style/processed/office0_7.png} &
			\includegraphics[width=\styleImSize\linewidth]{im/exp/style/processed/office0_8.png} &
			\includegraphics[width=\styleImSize\linewidth]{im/exp/style/processed/office0_9.png} \\
			\includegraphics[width=\styleImSize\linewidth]{im/exp/style/processed/office0_10.png} &
			\includegraphics[width=\styleImSize\linewidth]{im/exp/style/processed/office0_11.png} &
			\includegraphics[width=\styleImSize\linewidth]{im/exp/style/processed/office0_12.png} &
			\includegraphics[width=\styleImSize\linewidth]{im/exp/style/processed/office0_13.png} &
			\includegraphics[width=\styleImSize\linewidth]{im/exp/style/processed/office0_14.png} \\
			\includegraphics[width=\styleImSize\linewidth]{im/exp/style/processed/office0_15.png} &
			%\includegraphics[width=\styleImSize\linewidth]{im/exp/style/processed/office0_16.png} &
			\includegraphics[width=\styleImSize\linewidth]{im/exp/style/processed/office0_17.png} &
			\includegraphics[width=\styleImSize\linewidth]{im/exp/style/processed/office0_18.png} &
			\includegraphics[width=\styleImSize\linewidth]{im/exp/style/processed/office0_19.png} &
			\includegraphics[width=\styleImSize\linewidth]{im/exp/style/processed/office0_20.png} 
			\\	\hline
		\end{tabular}
	}
	%\captionof{figure}
	\caption{Style transfer on 3D canvas.}
	\label{fig:style}
\end{figure*}


\subsubsection{ Sample-based or Derivative-based}

We select the surface model with our own captured sequences. 
All settings are detailed in \cref{sec:exp:details}.
As shown in~\cref{fig:ablation:GPIS}, reconstruction of Yijun's office is demonstrated. 
Both models are able to construct, but the derivative-based model produces a lot of noise on the surface.
This is because for smoothness purpose, we build voxels that are overlapped to its neighbor, which causes redundant voxels near the surface.
For those redundant voxels, no center sample is provided and thus the derivative based surface construction builds bad SDFs on unknow region of the voxels.


Instead, sample-based surface construction does not have this problem as it adds more points in voxels and is able to construct highly-smooth surfaces.
From which, we find well constructed and colored white board, chair, school bag and even the oranges.

\subsubsection{Tracking}


Our Uni-Fusion use a coarse-to-fine strategy for tracking. 
An external tracking model is running in one thread aside from the mapping thread.
In the mapping thread, it takes pose result from the external tracking and applies the internal tracking for colored point cloud.

The result is demonstrated in~\cref{tab:tum_rmse2}. 
The coarse-to-fine is relatively better on trajectory estimation.

\subsubsection{Voxel size}

Testing the office scene, we vary the voxel size from low to high. 
From~\cref{fig:ablation:voxel_size}, when low voxel size $0.02$m is used, the surface is rough.
Then when voxel size goes larger, the smoothness is improved.
However, when we use $0.1$m voxel size, the surface color is blur. 
Considering Uni-Fusion produces a surface color field, the quality of surface directly affect the coloring.
Thus, continuing enlarging the voxel size also results in worse surface results.

Therefore, in the above experiments, $0.05$m voxel size is utilized for surface construction.
In addition, each voxel for encoding are actually with size $0.1$m, since we use overlapped voxel.

%\subsubsection{Anchor number and feature dimension}


\subsection{Application: 2D-to-3D Transfer}
\label{sec:fabircated_prop}

Applications such as 2) and 4) can be easily integrated with application 1) incremental reconstruction (\cref{sec:incremental_reconstruction}) by incorporating the fabricated result together with the point cloud.
%
For instance, given RGB-D frames, we detect saliency or transfer image styles to generate a fabricated $X$ image. Here, $X$ represents saliency, style, or other properties. 
By combining $X$ with depth information through unprojection,
we assign
the fabricated values to corresponding points, resulting in point pairs ($\V X$, $\V Q_{X}$).

Similar to the reconstruction pipeline in~\cref{fig:recons_and_scene_understanding}, we employ encoding (\cref{sec:encoder}) and fusion (\cref{eq:fuse}) to construct a global LIM for the fabricated properties $X$.
This global LIM represents a surface $X$ fields that is utilized for subsequent inference.

While it is possible to similarly transfer a 2D semantic image to 3D,
it may not be feasible in practice due to the need for multiple passes of different categories of semantic information 
 on the same dataset (such as object, usability, etc.).
Therefore, in the following section, we demonstrate the construction of a surface feature field for scene understanding application that satisfies various 
requirements through a single mapping pass.

\begin{table*}[b!]
	%\renewcommand{\arraystretch}{0.9}
	%\setlength{\tabcolsep}{3pt}
	\caption{GZSL semantic segmentation results. Scores are in \%.
	  $^\dagger$ indicate 3DGenZ's adaption of the method.
       Note that, Uni-Fusion-SU does not even train with the seen classes.}
	\centering
	\begin{tabular}{l|c|c |c ||ccc|ccc}
		\toprule
		\multicolumn{1}{c}{}& \multicolumn{2}{c|}{Training set} & Inference input &\multicolumn{3}{c|}{ScanNet } & \multicolumn{3}{c}{S3DIS}\\
		& Backbone & Classifier & &$Seen$& $Unseen$ & $All$&$Seen$& $Unseen$ & $All$
%		\multicolumn{3}{c|}{mIoU} & 
%		\multicolumn{3}{c|}{mIoU} \\ 
%		&&&& $Seen$& $Unseen$ & $All$&$Seen$& $Unseen$ & $All$
		%\cellcolor{white}{}  
		%\cellcolor{white}{}
		\\
		\midrule
		
		\multicolumn{5}{l}{\textit{Supervised methods with different levels of supervision}}\\
		
		Full supervision & $seen \cup unseen$ & $seen \cup unseen$ & Point Cloud &43.3&51.9 &45.1&74.0&50.0&66.6 \\
		
		ZSL backbone & $seen$ & $seen \cup unseen$  &Point Cloud&41.5&39.2 & 40.3&60.9& 21.5&  48.7 \\
		
		ZSL-trivial & $seen$ & $seen$ &Point Cloud&39.2&0.0&31.3&70.2 &0.0&48.6  \\
		\midrule
		\multicolumn{5}{l}{\textit{Generalized zero-shot-learning methods}}\\
		
		ZSLPC-Seg~\cite{cheraghian2019zero}$^\dagger$ & $seen$ & $unseen$  &Point Cloud&28.2&0.0& 22.6&65.6 &0.0& 45.3\\
		
		DeViSe-3DSeg~\cite{frome2013devise}$^\dagger$ & $seen$ & $unseen$   &Point Cloud &20.0&0.0&16.0&70.2&0.0& 48.6\\ 
		%ZSLPC-Seg~\cite{cheraghian2019zero}$^\dagger$ & $seen$ & $unseen$  &  4.0&13.9\\
		%DeViSe-3DSeg~\cite{frome2013devise}$^\dagger$ & $seen$ & $unseen$   &  3.0&10.9\\
		3DGenZ~\cite{michele2021generative} & $seen$ & $seen \cup \hat{unseen}$  &Point Cloud &32.8&7.7& {27.8}&53.1&7.3&   \textbf{39.0} \\
		\midrule
		\multicolumn{5}{l}{\textit{Zero-shot learning + map fusion}}\\
		Uni-Fusion-SU (Ours) &None&None&Sparse Frames&31.0&\textbf{41.9}&\textbf{32.9} &31.3&\textbf{24.0}&29.0\\
		\bottomrule
		\multicolumn{1}{l}{}\\[-7pt]
	\end{tabular}

	\label{tab:sem_seg_overview}
\end{table*}

\begin{figure*}[t!]
	\centering
	\setlength{\tabcolsep}{0.1em}
	\renewcommand{\arraystretch}{.1}
	\begin{tabular}{|c | c |c |||c |c | c|}
		\toprule
		{\Large{3DGenZ}} & {\Large{Uni-Fusion}} &{\Large{Ground Truth}} & {\Large{3DGenZ}} &{\Large{Uni-Fusion-SU}} & {\Large{Ground Truth}} \\ \midrule
		
		\includegraphics[width=\scannetImSize\linewidth]{im/exp//ss/gen3dz_0568.png}
		&\includegraphics[width=\scannetImSize\linewidth]{im/exp//ss/mine_0568.png}
		&\includegraphics[width=\scannetImSize\linewidth]{im/exp//ss/gt_0568.png}
		&		\includegraphics[width=\scannetImSize\linewidth]{im/exp//ss/gen3dz_0164.png}
		&\includegraphics[width=\scannetImSize\linewidth]{im/exp//ss/mine_0164.png}
		&\includegraphics[width=\scannetImSize\linewidth]{im/exp//ss/gt_0164.png}\\
		
		
		\includegraphics[width=\scannetImSize\linewidth]{im/exp//ss/gen3dz_0249.png}
		&\includegraphics[width=\scannetImSize\linewidth]{im/exp//ss/mine_0249.png}
		&\includegraphics[width=\scannetImSize\linewidth]{im/exp//ss/gt_0249.png}
		&		\includegraphics[width=\scannetImSize\linewidth]{im/exp//ss/gen3dz_0435.png}
		&\includegraphics[width=\scannetImSize\linewidth]{im/exp//ss/mine_0435.png}
		&\includegraphics[width=\scannetImSize\linewidth]{im/exp//ss/gt_0435.png}\\
		
		
		\includegraphics[width=\scannetImSize\linewidth]{im/exp//ss/gen3dz_0046.png}
		&\includegraphics[width=\scannetImSize\linewidth]{im/exp//ss/mine_0046.png}
		&\includegraphics[width=\scannetImSize\linewidth]{im/exp//ss/gt_0046.png}
		&		\includegraphics[width=\scannetImSize\linewidth]{im/exp//ss/gen3dz_0050.png}
		&\includegraphics[width=\scannetImSize\linewidth]{im/exp//ss/mine_0050.png}
		&\includegraphics[width=\scannetImSize\linewidth]{im/exp//ss/gt_0050.png}\\
		\bottomrule
		
	\end{tabular}
	\includegraphics[width=\linewidth]{im/ss_colorbar}
	%\captionof{figure}
	\caption{Demonstration of semantic segmentation on the ScanNet dataset.
       Selected scenes are consistent with~\cref{fig:recons:scannet_demo}}
	\label{fig:segmentation_demo}
	
\end{figure*}

\subsection{Scene Understanding Results}

Saliency detection effectively highlights the objects of interest.
This is also considered part of 3D semantic understanding.
However, as the semantics categories vary, fusing different categories of semantics into multiple LIMs can be inefficient.
%
Therefore, in this section, we utilize Uni-Fusion to fuse and construct a surface field for high-dimensional CLIP embeddings.
With a single LIM, we can generate different semantic results based on corresponding commands.
%
Since now our Uni-Fusion works with OpenSeg for scene understanding purposes, we call it Uni-Fusion-SU.

\subsubsection{Semantic Segmentation}
\label{sec:exp:semantic}

We first evaluate our model on generalized zero-shot point cloud semantic segmentation application.
Generalized Zero-Shot Learning (GZSL) differs from Zero-Shot Learning (ZSL) in that ZSL only predicts classes unseen during training, while GZSL predicts both unseen and seen classes~\cite{michele2021generative}.
Therefore, comparing our results with GZSL SOTAs provides a better understanding of the potential of Uni-Fusion-SU, as it does not train on both seen and unseen. 

This test uses ScanNet and S3DIS datasets for benchmarking. 
It is important to note that the \textbf{compared baselines are trained on the corresponding datasets}.
Our Uni-Fusion-SU uses OpenSeg to provide the 2D image level feature ebmedding.
Although \textbf{Uni-Fusion-SU} is also zero-shot, \textbf{it does not touch any ScanNet or S3DIS annotations}.

We demonstrate the mIoU scores in~\cref{tab:sem_seg_overview}.
In particular, our model achieves best results among the zero-shot learning methods on the ScanNet dataset and remains competitive with fully supervised methods.

Furthermore, we provide results specifically for the unseen classes in~\cref{sup:tab:sn_acc_miou}.
Although not as good as the fully supervised approach, Uni-Fusion-SU performs much better than 3DGenZ.
In addition, our Uni-Fusion-SU demonstrates high precision in classes such as sofa and Toilet, even when compared to the fully supervised model.

\begin{table}[htbp]
		\caption{Classwise GZSL semantic segmentation performance (\%) on the ScanNet unseen split.}
	\centering
	\newcommand*\rotext{\multicolumn{1}{R{45}{1em}}}
	\setlength{\tabcolsep}{1.7pt}
	\begin{tabular}{@{}l@{~}c|rrrr|r@{}}
		\toprule		
		& &
		{\textbf{Bookshelf}} & {\textbf{Desk}} & {\textbf{Sofa}} & {\textbf{Toilet}} & \stackbox{mean} \\
		
		\midrule
		FSL (Fully supervise) & IoU & 	56.9&	30.0&	57.4&	63.4 & 51.9
		\\ 
		3DGenZ (Zero-shot) & IoU & 	6.3&	3.3&	13.1&	8.1 & 7.7
		\\
		Uni-Fusion-SU (Ours) & IoU &38.3&16.8&51.7&60.9&41.9
	\\ \midrule 
	3DGenZ (Zero-shot)& Acc. & 	13.4&	5.9&	49.6&	26.3 &23.8
	\\
	Uni-Fusion-SU (Ours) & Acc. &61.9&29.6&67.4&91.6& 62.6
		\\
		\bottomrule
	\end{tabular}

	\label{sup:tab:sn_acc_miou}
\end{table}

However, in the S3DIS dataset, our model does not outperform 3DGenZ and other methods as shown in~\cref{tab:sem_seg_overview}.

Even in the result of unsceened data, as presented in \cref{sup:tab:s3dis_acc_miou}, we observe that Uni-Fusion-SU hardly finds some classed, e.g. Beam and Column, which are not commonly annotated objects. 
However, for common objects like sofa and window, our model performs much better.

\begin{table}[htbp]
		\caption{Classwise GZSL semantic segmentation performance (\%) on the S3DIS unseen split.}
	\centering
	\newcommand*\rotext{\multicolumn{1}{R{45}{1em}}}
	\setlength{\tabcolsep}{1.7pt}
	\begin{tabular}{@{}l@{~}c|rrrr|r@{}}
		\toprule		
		& &
		{\textbf{Beam}} & {\textbf{Column}} & {\textbf{Sofa}} & {\textbf{Window}} & \stackbox{mean} \\
		
		\midrule
		FSL (Fully supervise) & IoU & 	63.1&	10.2&	54.1&	72.4 & 50.0
		\\ 
		3DGenZ (Zero-shot) & IoU & 	13.9&	2.4&4.9&	8.1 &7.3
		\\
		Uni-Fusion-SU (Ours) & IoU &5.5&0.02&57.4&32.9&	24.0
		\\ \midrule 
		3DGenZ (Zero-shot) & Acc. & 	20.0&	9.1&	62.4&	23.7 &28.8
		\\
		Uni-Fusion-SU (Ours) & Acc. &41.5&0.02&78.3&42.1& 40.5
		\\	
		\bottomrule
	\end{tabular}

	\label{sup:tab:s3dis_acc_miou}
\end{table}

We present the results of the semantic segmentation in~\cref{fig:segmentation_demo}. 
It is evident that, 3DGenZ's result contains more noise, as seen in the spotted sofa, bed and other objects.
Conversely, Uni-Fusion-SU's results are generally smoother and more precise.

%
%\begin{figure*}[htbp]
%	\centering
%	\includegraphics[width=.3\linewidth]{example-image-golden}
%	\includegraphics[width=.3\linewidth]{example-image-golden}
%	\includegraphics[width=.3\linewidth]{example-image-golden}
%	\\
%	\includegraphics[width=.3\linewidth]{example-image-golden}
%	\includegraphics[width=.3\linewidth]{example-image-golden}
%	\includegraphics[width=.3\linewidth]{example-image-golden}
%	
%	\caption{Semantic segmentation result on ScanNet.}
%\end{figure*}
%
%\begin{figure*}[htbp]
%	\centering
%	\includegraphics[width=.3\linewidth]{example-image-golden}
%	\includegraphics[width=.3\linewidth]{example-image-golden}
%	\includegraphics[width=.3\linewidth]{example-image-golden}
%	\\
%	\includegraphics[width=.3\linewidth]{example-image-golden}
%	\includegraphics[width=.3\linewidth]{example-image-golden}
%	\includegraphics[width=.3\linewidth]{example-image-golden}
%	
%	\caption{Semantic segmentation result on S3DIS.}
%\end{figure*}

\subsubsection{Scene Understanding with Different Properties}

\begin{figure*}[t!]
	\centering
	\setlength{\tabcolsep}{0.1em}
	\renewcommand{\arraystretch}{.1}
	\resizebox{\textwidth}{!}{\begin{tabular}{|c | c | c | c | c | c|}
			\toprule 
			& \textbf{scene0568\_00} & \textbf{scene0249\_00} & \textbf{scene0435\_00} & \textbf{office3} & \textbf{room0}\\
			\midrule
			{} &
			\raisebox{-.5\height}{\includegraphics[width=\fabImSize\linewidth]{im/exp/fab/scannet/0568_color.png}} & %\raisebox{-.5\height}{\includegraphics[width=\fabImSize\linewidth]{im/exp/fab/scannet/0164_color.png}} &
			\raisebox{-.5\height}{\includegraphics[width=\fabImSize\linewidth]{im/exp/fab/scannet/0249_color.png}} & \raisebox{-.5\height}{\includegraphics[width=\fabImSize\linewidth]{im/exp/fab/scannet/0435_color.png}}
			&
			\raisebox{-.5\height}{\includegraphics[width=\fabImSize\linewidth]{im/exp/fab/replica/office3_color.png}}
			&
			\raisebox{-.5\height}{\includegraphics[width=\fabImSize\linewidth]{im/exp/fab/replica/room0_color.png}}\\ %\raisebox{-.5\height}{\includegraphics[width=\fabImSize\linewidth]{im/exp/fab/scannet/0050_color.png}} %\includegraphics[width=\fabImSize\linewidth]{im/exp/fab/replica/office3_color.png}
			\\
			\textbf{Desk}  &
			\raisebox{-.5\height}{\includegraphics[width=\fabImSize\linewidth]{im/exp/fab/scannet/0568_lt_desk.png}}&
			%\raisebox{-.5\height}{\includegraphics[width=\fabImSize\linewidth]{im/exp/fab/scannet/0164_lt_desk.png}}&
			\raisebox{-.5\height}{\includegraphics[width=\fabImSize\linewidth]{im/exp/fab/scannet/0249_lt_desk.png}}&
			\raisebox{-.5\height}{\includegraphics[width=\fabImSize\linewidth]{im/exp/fab/scannet/0435_lt_desk.png}}
			&
			\raisebox{-.5\height}{\includegraphics[width=\fabImSize\linewidth]{im/exp/fab/replica/office3_lt_desk.png}}
			&
			\raisebox{-.5\height}{\includegraphics[width=\fabImSize\linewidth]{im/exp/fab/replica/room0_lt_desk.png}}\\
			%\raisebox{-.5\height}{\includegraphics[width=\fabImSize\linewidth]{im/exp/fab/scannet/0050_lt_desk.png}}
			%\includegraphics[width=\fabImSize\linewidth]{im/exp/fab/replica/office3_saliency.png}
			\\
			
			\textbf{Sofa} &
			\raisebox{-.5\height}{\includegraphics[width=\fabImSize\linewidth]{im/exp/fab/scannet/0568_lt_sofa.png}} &
			%\raisebox{-.5\height}{\includegraphics[width=\fabImSize\linewidth]{im/exp/fab/scannet/0164_lt_sofa.png}} &
			\raisebox{-.5\height}{\includegraphics[width=\fabImSize\linewidth]{im/exp/fab/scannet/0249_lt_sofa.png}} &
			\raisebox{-.5\height}{\includegraphics[width=\fabImSize\linewidth]{im/exp/fab/scannet/0435_lt_sofa.png}}&
			\raisebox{-.5\height}{\includegraphics[width=\fabImSize\linewidth]{im/exp/fab/replica/office3_lt_sofa.png}}
			&
			\raisebox{-.5\height}{\includegraphics[width=\fabImSize\linewidth]{im/exp/fab/replica/room0_lt_sofa.png}}\\
			%\raisebox{-.5\height}{\includegraphics[width=\fabImSize\linewidth]{im/exp/fab/scannet/0050_lt_sofa.png}}
			%\includegraphics[width=\fabImSize\linewidth]{im/exp/fab/replica/office3_style.png}
			\\
			\textbf{Work} &
			\raisebox{-.5\height}{\includegraphics[width=\fabImSize\linewidth]{im/exp/fab/scannet/0568_lt_work.png}} &
			%\raisebox{-.5\height}{\includegraphics[width=\fabImSize\linewidth]{im/exp/fab/scannet/0164_lt_work.png}} &
			\raisebox{-.5\height}{\includegraphics[width=\fabImSize\linewidth]{im/exp/fab/scannet/0249_lt_work.png}} &
			\raisebox{-.5\height}{\includegraphics[width=\fabImSize\linewidth]{im/exp/fab/scannet/0435_lt_work.png}}&
			\raisebox{-.5\height}{\includegraphics[width=\fabImSize\linewidth]{im/exp/fab/replica/office3_lt_work.png}}
			&
			\raisebox{-.5\height}{\includegraphics[width=\fabImSize\linewidth]{im/exp/fab/replica/room0_lt_work.png}}\\
			%\raisebox{-.5\height}{\includegraphics[width=\fabImSize\linewidth]{im/exp/fab/scannet/0050_lt_work.png}}
			%\includegraphics[width=\fabImSize\linewidth]{im/exp/fab/replica/office3_style.png}
			\\
			\textbf{Sittable} &
			\raisebox{-.5\height}{\includegraphics[width=\fabImSize\linewidth]{im/exp/fab/scannet/0568_lt_sit.png}} &
			%\raisebox{-.5\height}{\includegraphics[width=\fabImSize\linewidth]{im/exp/fab/scannet/0164_lt_sit.png}} &
			\raisebox{-.5\height}{\includegraphics[width=\fabImSize\linewidth]{im/exp/fab/scannet/0249_lt_sit.png}} &
			\raisebox{-.5\height}{\includegraphics[width=\fabImSize\linewidth]{im/exp/fab/scannet/0435_lt_sit.png}}&
			\raisebox{-.5\height}{\includegraphics[width=\fabImSize\linewidth]{im/exp/fab/replica/office3_lt_sit.png}}
			&
			\raisebox{-.5\height}{\includegraphics[width=\fabImSize\linewidth]{im/exp/fab/replica/room0_lt_sit.png}}\\
			%\raisebox{-.5\height}{\includegraphics[width=\fabImSize\linewidth]{im/exp/fab/scannet/0050_lt_sit.png}}
			%\includegraphics[width=\fabImSize\linewidth]{im/exp/fab/replica/office3_style.png}
			\\
			\textbf{Wood} &
			\raisebox{-.5\height}{\includegraphics[width=\fabImSize\linewidth]{im/exp/fab/scannet/0568_lt_wood.png}} &
			%\raisebox{-.5\height}{\includegraphics[width=\fabImSize\linewidth]{im/exp/fab/scannet/0164_lt_wood.png}} &
			\raisebox{-.5\height}{\includegraphics[width=\fabImSize\linewidth]{im/exp/fab/scannet/0249_lt_wood.png}} &
			\raisebox{-.5\height}{\includegraphics[width=\fabImSize\linewidth]{im/exp/fab/scannet/0435_lt_wood.png}}&
			\raisebox{-.5\height}{\includegraphics[width=\fabImSize\linewidth]{im/exp/fab/replica/office3_lt_wood.png}}
			&
			\raisebox{-.5\height}{\includegraphics[width=\fabImSize\linewidth]{im/exp/fab/replica/room0_lt_wood.png}}\\
			%\raisebox{-.5\height}{\includegraphics[width=\fabImSize\linewidth]{im/exp/fab/scannet/0050_lt_wood.png}}
			%\includegraphics[width=\fabImSize\linewidth]{im/exp/fab/replica/office3_style.png}
			\\
			
			
			\bottomrule
		\end{tabular}
	}
	%\captionof{figure}
	\caption{Demonstration of the original mesh, highlighted semantic mesh given various queries.}
	\label{fig:fab_lt}
	\vspace{-.5cm}
\end{figure*}

The main contribution of this application is that, Uni-Fusion is the first model to construct a continuous mapping of high-dimensional embeddings onto the surface without the need for any training of the map representation.
%
In the previous experiment (\cref{sec:exp:semantic}), we evaluate the performance of generalized zero-shot semantic segmentation.
However, the potential of Uni-Fusion goes beyond semantic segmentation.
%
By constructing a LIM, we obtain a surface CLIP feature field.
This enables us to query various semantic categories such as 
%without the need of multiple LIMs or rerun for other properties, we query 
\textbf{Object, Room Type, Material, Affordance and Activity} without requiring multiple LIMs or re-running the model.

We present the results in \cref{fig:fab_lt}, 
where we query object (desk, sofa), activity (work), affordance (sittable), and material (wood).
Uni-Fusion-SU accurately identifies and highlights the object and material regions.
However, for less specific commands such as work or sittable, the model provides a wider range of results with less confidence (indicated by dull yellow).
Nevertheless, the suggested options are also roughly correct.









\subsection{Time}

We run all of the applications in a single pass using our captured office sequences and evaluate the time cost of construction and fusion of each LIM. 
The average time cost across frames is shown in~\cref{tab:time}.

\begin{table}[htbp]
	\caption{Time required for each frame.
	}
	\centering
	\footnotesize
	\setlength{\tabcolsep}{0.7em}
	\resizebox{\linewidth}{!}{
		\begin{tabular}{l|ccccccc}
			\toprule
			&Surface & Color & Infrared & Style & Saliency & Latent&Internal Track \\ \midrule
			Time ($\si{\second}$)&0.100 & 0.038 & 0.045 & 0.048 & 0.045 &0.011 &0.225 \\ \bottomrule
	\end{tabular}}
	
	\label{tab:time}
\end{table}

\newcommand{\mineImSize}{.32}
%\begin{table*}[t!]
%	\centering
%	\setlength{\tabcolsep}{0.1em}
%	\renewcommand{\arraystretch}{.1}
%	\begin{tabular}{|c | c |c |}
%		\hline 
%		{Color} &{Infrared} & {Saliency} \\
%		\includegraphics[width=\mineImSize\linewidth]{im/exp/fab/mine/office/seq3_color.png} &
%		\includegraphics[width=\mineImSize\linewidth]{im/exp/fab/mine/office/seq3_color.png} &
%		\includegraphics[width=\mineImSize\linewidth]{im/exp/fab/mine/office/seq3_saliency.png} \\
%		{Style 1}&{Style 1}&{Style 1}\\
%		\includegraphics[width=\mineImSize\linewidth]{im/exp/fab/mine/office/seq3_style.png}&
%		\includegraphics[width=\mineImSize\linewidth]{im/exp/fab/mine/office/seq3_style.png}&
%		\includegraphics[width=\mineImSize\linewidth]{im/exp/fab/mine/office/seq3_style.png}\\
%		{Sofa}&{Desk}&{Soft}\\
%		\includegraphics[width=\mineImSize\linewidth]{im/exp/fab/mine/office/seq3_lt_sofa.png}&
%		\includegraphics[width=\mineImSize\linewidth]{im/exp/fab/mine/office/seq3_lt_desk.png}&
%		\includegraphics[width=\mineImSize\linewidth]{im/exp/fab/mine/office/seq3_lt_soft.png}\\		
%		\hline
%	\end{tabular}
%	\captionof{figure}{Demonstration on captured office data.}
%	\label{fig:mine_demo}
%\end{table*}
%\begin{figure*}[t!]
%	\centering
%	\setlength{\tabcolsep}{0.1em}
%	\renewcommand{\arraystretch}{.1}
%	\begin{tabular}{|c | c |c |}
%		\hline \hline
%		\includegraphics[width=\mineImSize\linewidth]{im/exp/fab/mine/office/seq3_w_slam_color.png}&	\includegraphics[width=\mineImSize\linewidth]{im/exp/fab/mine/office/seq3_w_slam_ir.png}&	\includegraphics[width=\mineImSize\linewidth]{im/exp/fab/mine/office/seq3_w_slam_saliency.png}\\
%		{Color} &{Infrared} & {Saliency}\\
%<<<<<<< HEAD


%=======
%		
%		\includegraphics[width=\mineImSize\linewidth]{im/exp/fab/mine/office/seq3_w_slam_style.png}
%		&\includegraphics[width=\mineImSize\linewidth]{im/exp/fab/mine/office/seq3_w_slam_lt_desk.png}
%		&\includegraphics[width=\mineImSize\linewidth]{im/exp/fab/mine/office/seq3_w_slam_lt_wood.png}\\
%		{Style} & {Object-desk} & {Material-wood} \\\hline
%>>>>>>> e014bc950c14dec9ffa1d2d7a6de9b7abfefabdd
%	\end{tabular}
%	%\captionof{figure}
%	\caption{Demonstration on captured Office data.}
%	\label{fig:office}
%\end{figure*}

\begin{figure*}[]
	\centering
	\setlength{\tabcolsep}{0.1em}
	\renewcommand{\arraystretch}{.1}
	\begin{tabular}{|c | c |c |}
	\hline \hline
	\includegraphics[width=\mineImSize\linewidth]{im/exp/fab/mine/appartment2/appartment2_color.png}&	\includegraphics[width=\mineImSize\linewidth]{im/exp/fab/mine/appartment2/appartment2_ir.png}&	\includegraphics[width=\mineImSize\linewidth]{im/exp/fab/mine/appartment2/appartment2_saliency.png}\\
		{Color} &{Infrared} & {Saliency}\\
	\includegraphics[width=\mineImSize\linewidth]{im/exp/fab/mine/appartment2/appartment2_style.png}
&\includegraphics[width=\mineImSize\linewidth]{im/exp/fab/mine/appartment2/appartment2_lt_sofa.png}
&\includegraphics[width=\mineImSize\linewidth]{im/exp/fab/mine/appartment2/appartment2_lt_desk.png}
\\
{Style} & {Object-sofa} & {Object-desk}\\
\includegraphics[width=\mineImSize\linewidth]{im/exp/fab/mine/appartment2/appartment2_lt_coat.png}
&\includegraphics[width=\mineImSize\linewidth]{im/exp/fab/mine/appartment2/appartment2_lt_sit.png}
&\includegraphics[width=\mineImSize\linewidth]{im/exp/fab/mine/appartment2/appartment2_lt_wood.png}\\
{Object-coat} & {Affordance-sit} & {Material-wood} \\\hline
	\end{tabular}
%\captionof{figure}
\caption{Demonstration on the captured apartment data.}
\label{fig:appartment}
%\vspace{-.5cm}
\end{figure*}


Using depth and property images of size $720\times1280$ as input, it is evident from the table, that our model operates at a frequency of $\sim10\si{\hertz}$ for  surface (sample mode) LIM construction and integration. 
It alse achieves a frequency of over $20\si{\hertz}$ for color, infrared, style, and saliency.
These results demonstrate the suitability of Uni-Fusion for real-time applications.

However, our internal tracking process takes around $0.225\si{\second}$ per frame, which is relatively slower compared to the mapping module. 
Nevertheless, Uni-Fusion uses external tracking to prevent tracking loss, enabling our internal tracking and mapping to operate at a lower frequency.
As a result, the entire model can be effectively applied in real-time in various scenarios.

\section{Extensive experiment on our own dataset}

In previous experiments, we evaluate the capabilities of Uni-Fusion in different applications. 
To further demonstrate its effectiveness in robotic environmental understanding, we capture our own dataset to show all applications together.

We capture two scenes: The office and apartment of the first author using a Microsoft Kinect Azure. 
%
RGB-D and infrared video are captured. After calibration, RGB, depth, infrared inputs have resolution of $720\times1280$.
Uni-Fusion tracks and reconstructs all applications in one pass.
%
While office data has been involved in ablation study (\cref{exp:surface:ablation}), we showcase all applications using the apartment dataset, as depicted in~\cref{fig:appartment}.

For better visualization, the ceiling of reconstruction is removed.
The top row of images presents the colored mesh with room details, the infrared mesh revealing the lighting effect, and the saliency reconstruction highlighting objects crucial for navigation.
Additionally, we select the second style from~\cref{fig:style} for style transfer to the apartment canvas.
%
As a result, the wooden floor in the room is colored with dark green.
The whole apartment is in a warm style.

The remaining results are generated from the surface field of the CLIP embeddings. 
We issue commands to locate objects, e.g., where is the sofa, desk and coat.
In addition, it easily identifies affordances such as being sittable.
For material, it successfully detects the wooden floor in each room.


\section{Conclusion}
High Dynamic Range (HDR) Imaging is a desirable function for modern digital cameras. To overcome the hardware limitation of current camera sensors in dynamic range, we have described a new neural model to blend low dynamic range images captured at tri-exposures into HDR images. We have shown that, by integrating the gating mechanism with the shifted-window self-attention mechanism, our model progressively aligns the cross-exposure images and screen outliers caused by large motion from HDR fusion. We have also demonstrated that, in the framework of multi-scale encoder-decoder, the Gated Swin Transformer achieves global reception, long-range matching and adaptive fusion. Extensive ablation study has been presented to validate the effectiveness of our model in addressing large motion in HDR imaging.

\bibliographystyle{splncs04}
\bibliography{cas-refs}
%
\end{document}
