\documentclass[notitlepage,prd,twocolumn,showpacs,preprintnumbers,nofootinbib,superscriptaddress,floatfix,tightenlines]{revtex4-1}
\usepackage{mathrsfs}
\usepackage{amssymb}
\usepackage{amsmath}
\usepackage{graphicx}
\usepackage[utf8]{inputenc}
\usepackage{placeins}
\usepackage{amsfonts,amsbsy}
\usepackage{nicefrac}
\usepackage{xcolor}
\definecolor{lcolor}{rgb}{0.5,0,0}
\definecolor{citcolor}{rgb}{0,0.3,0.0}
\usepackage[breaklinks,colorlinks,urlcolor=blue,citecolor=citcolor,linkcolor=lcolor]{hyperref}
\usepackage{comment}
\usepackage{bbm}
\usepackage[normalem]{ulem} 
\usepackage{enumitem}


\newcommand{\qhat}{\hat{q}}
\newcommand{\mbf}{\mathbf}
\newcommand{\tout}{t_{\text{out}}}
\newcommand{\tpert}{t_{\text{pert}}}
\newcommand{\tcent}{\bar{t}}			
\newcommand{\mrm}{\mathrm}
\newcommand{\ah}{\mrm{ah}}
\newcommand{\Tr}{\mrm{Tr}}
\newcommand{\HTL}{\mrm{HTL}}
\newcommand{\fit}{\mrm{fit}}
\newcommand{\rel}{\mrm{rel}}
\newcommand{\qs}{Q_\mathrm{s}}
\newcommand{\pInit}{Q}
\newcommand{\Q}{Q_s}
\newcommand{\wplas}{\omega_{\mrm{pl}}}
\newcommand{\gplas}{\gamma_{\mrm{pl}}}
\newcommand{\fig}{Fig.~}
\newcommand{\figs}{Figs.~}
\newcommand{\eq}{Eq.~}
\newcommand{\eqs}{Eqs.~}
\newcommand{\se}{Sec.~}
\newcommand{\ses}{Secs.~}
\newcommand{\re}{Ref.~}
\newcommand{\res}{Refs.~}
\newcommand{\app}{App.~}
\newcommand{\nr}[1]{(\ref{#1})}
\newcommand{\bs}[1]{\boldsymbol{#1}}
\newcommand{\pperp}{\bs{p}_\perp}
\newcommand{\qperp}{\bs{q}_\perp}

\newcommand{\ud}{\mathrm{d}}
\newcommand{\ii}{{\boldsymbol{\hat{\i}}}}
\newcommand{\jj}{{\boldsymbol{\hat{\j}}}}
\newcommand{\kk}{{\mathbf{\hat{k}}}}
\newcommand{\cl}{\mrm{cl}}
\newcommand{\uj}{\mrm{j}}
\newcommand{\dtmax}{\Delta t_{\text{max}}}
\newcommand{\md}{m_D}
\newcommand{\pmin}{p_{\mathrm{min}}}
\newcommand{\pmax}{p_\mathrm{max}}
\newcommand{\tstar}{T_*}
\newcommand{\pt}{p_T}



\newcommand{\nn}{\nonumber}
\newcommand{\li}{\mathrm{Li}_2}
\newcommand{\nc}{N_c}


\newcommand{\der}{\mathrm{d}}


\newcommand{\be}{\begin{equation}}
\newcommand{\ee}{\end{equation}}
\newcommand{\bea}{\begin{eqnarray}}
\newcommand{\eea}{\end{eqnarray}}
\newcommand{\OO}{{\cal O}}


\newcommand{\calk}{\mathcal{K}}
\newcommand{\calx}{\mathcal{X}}
\newcommand{\intx}{\int_{\mathcal{X}}}
\newcommand{\intk}{\int_{\calk}}
\newcommand{\phihat}[2]{\hat{\phi}_{#1}\left(#2\right)}
\newcommand{\phihatdagger}[2]{\hat{\phi}^\dagger_{#1}\left(#2\right)}
\newcommand{\qpdof}{Q_{pdof}}
\newcommand{\derthree}{\frac{\der^3 p }{\left(2 \pi \right)^3}}
\newcommand{\epspdof}{\varepsilon_{pdof}}
\newcommand{\eps}{\varepsilon}
\newcommand{\Npdof}{N_{pdof}}
\newcommand{\fbar}{\bar{f}}
\newcommand{\nbar}{\bar{n}}
\newcommand{\pz}{p_z}
\newcommand{\Pmin}{p_{\text{min}}}
\newcommand{\Pmax}{p_{\text{max}}}
\newcommand{\tauT}{\tau_{\mathrm{BMSS}}}
\newcommand{\tauR}{\tau_R}
\newcommand{\distvec}[1]{f_{\bs{#1}}}



\begin{document}

\title{Heavy quark diffusion coefficient in heavy-ion collisions via kinetic theory}

\author{K.~Boguslavski} 
\affiliation{Institute for Theoretical Physics, Technische Universit\"{a}t Wien, 1040 Vienna, Austria }

\author{A.~Kurkela} 
\affiliation{Faculty of Science and Technology, University of Stavanger, 4036 Stavanger, Norway}

\author{T.~Lappi} 
\affiliation{Department of Physics, P.O.~Box 35, 40014 University of Jyv\"{a}skyl\"{a}, Finland}
\affiliation{Helsinki Institute of Physics, P.O.~Box 64, 00014 University of Helsinki, Finland}

\author{F.~Lindenbauer} 
\affiliation{Institute for Theoretical Physics, Technische Universit\"{a}t Wien, 1040 Vienna, Austria }

\author{J.~Peuron} 
\email{jarkko.t.peuron@jyu.fi}
\affiliation{Department of Physics, P.O.~Box 35, 40014 University of Jyv\"{a}skyl\"{a}, Finland}
\affiliation{Helsinki Institute of Physics, P.O.~Box 64, 00014 University of Helsinki, Finland}
\affiliation{Dept. of Physics, Lund University,  S\"{o}lvegatan  14A, Lund,SE-223 62, Sweden}


\begin{abstract}
We compute the heavy quark momentum diffusion coefficient $\kappa$ using QCD kinetic theory for a system going through bottom-up isotropization in the  initial stages of a heavy ion collision. We find that the values of $\kappa$ 
are within 30\% from a thermal system at the same energy density. When matching for other quantities we observe considerably larger deviations. We also observe that the diffusion coefficient in the transverse direction is larger at high occupation numbers, whereas for an underoccupied system the longitudinal diffusion coefficient dominates. 
The behavior of the diffusion coefficient 
can be understood on a qualitative level based on the Debye mass $m_D$ and the effective temperature of soft modes $\tstar$.
Our results for the kinetic evolution of $\kappa$ in different directions can be used in phenomenological descriptions of heavy quark diffusion and quarkonium dynamics to include the impact of pre-equilibrium stages.

\end{abstract}



\maketitle





\section{Introduction}


Several interesting signatures of the quark-gluon plasma, such as quarkonium suppression, are sensitive to the dynamics of heavy quarks in the plasma. 
Due to their large mass $m_Q \gg T$, heavy quarks are solely produced in initial hard scattering processes. Furthermore 
they interact strongly with the medium during the entire evolution. The typical momentum exchange of a heavy quark with the QGP is small ($\mathcal{O}(gT)$) and thus a significant change in momentum requires several scatterings with the medium. This is characteristic of diffusive motion, and hence diffusion properties of heavy quarks within the QGP have been subject to extensive research~\cite{He:2022ywp,Rothkopf:2019ipj,Brambilla:2022fqa}.
The interaction of the heavy quark with the medium is conventionally quantified by 
the momentum diffusion coefficient $\kappa$. 

The diffusion coefficient is not only interesting in its own right as a property of the medium---it also plays an instrumental role in understanding and describing the evolution of quarkonia using various phenomenological frameworks \cite{Rothkopf:2019ipj,Brambilla:2022fqa}, including e.g., the open quantum systems approach
\cite{Young:2010jq, Brambilla:2016wgg, Brambilla:2020qwo, Brambilla:2021wkt, Brambilla:2023hkw}.

The behavior of $\kappa$ in equilibrium is relatively well understood \cite{Moore:2004tg,Caron-Huot:2007rwy,Caron-Huot:2008dyw,Brambilla:2020siz,Casalderrey-Solana:2006fio,vanHees:2005wb,Svetitsky:1987gq,vanHees:2007me,Banerjee:2011ra,Capellino:2022nvf,Yao:2020xzw,Banerjee:2022gen}, but out of equilibrium it remains elusive. The existing literature on transport coefficients out of equilibrium \cite{Ipp:2020nfu, Ipp:2020mjc, Avramescu:2023qvv,Carrington:2022bnv,Carrington:2021dvw,Carrington:2020sww,Khowal:2021zoo,Ruggieri:2022kxv,Sun:2019fud,Boguslavski:2020tqz} indicates that the overoccupied glasma stage can have considerable impact on transport phenomena, such as momentum broadening,  jet quenching and heavy quark transport coefficients.
The effect of anisotropic initial stages has been studied using a simplified description of the underlying anisotropic momentum distribution \cite{Romatschke:2006bb, Romatschke:2004au, Hauksson:2021okc}. 
However, to our knowledge, the time evolution of transport coefficients has not yet been extracted during the kinetic evolution between the glasma stage and the hydrodynamical phase.


The aim of this paper is to bridge 
this gap and to study the heavy quark momentum diffusion coefficient $\kappa$ during the bottom-up isotropization process \cite{Baier:2000sb}. With this aim in mind, we use QCD effective kinetic theory with the same initial conditions and numerical setup as in \cite{Kurkela:2015qoa} for the limit of an infinitely massive stationary quark. Furthermore, our aim is to understand how $\kappa$ out of equilibrium compares to  similar computations in thermal equilibrium. To this end, we compare values of $\kappa$ to thermal systems with 
 the same screening mass, energy density or effective infrared (IR) temperature. 
 We establish that out of these options $\kappa$ in the pre-equilibrium system is closest to the thermal value at the same energy density. We will also study the transverse (to the direction of the expansion), and longitudinal (along the beam direction) diffusion coefficients separately. We observe that there is a dynamical hierarchy between the longitudinal and transverse directions, which has a natural explanation in terms of the different stages of the bottom-up thermalization process. We will also derive a set of parametric estimates that explain the evolution of the matched quantities and $\kappa$ relative to their equilibrium value 
in terms of the dynamical occupation number and anisotropy of the system.

This paper is structured as follows. In \se \ref{sec:theory} we discuss the kinetic theory framework, how we extract the heavy quark momentum diffusion coefficient and introduce our initial conditions and the relevant physical scales $\md$ and $\tstar$. We continue in \se \ref{sec:results} by presenting our results for the diffusion coefficient during the initial stages in heavy-ion collisions and
discuss the diffusion coefficient in the transverse and longitudinal directions separately. Finally we conclude in \se \ref{sec:conc}.


\newpage

\section{Theoretical background}
\label{sec:theory}
Let us start by introducing the
effective kinetic theory, heavy quark diffusion coefficient and other observables relevant for this work. 
Some details of our discretization procedure and discretization effects are discussed in  Appendix \ref{app:numerics}.


\subsection{Effective kinetic theory}
In this paper we use the same numerical setup as in Ref.~\cite{Kurkela:2015qoa}, 
with the aim of reproducing the bottom-up isotropization process \cite{Baier:2000sb} using the numerical framework of QCD effective kinetic theory (EKT) \cite{Arnold:2002zm}. 
Kinetic theory is a quasiparticle description of the quark-gluon plasma in terms of its individual constituents, quarks and gluons. 
Here we consider only gluons for which the central dynamical quantity
is the quasiparticle distribution function 
\begin{equation}
\label{eq:fdef}
f(\bs{p}) = \dfrac{1}{\nu_g}\dfrac{\der N}{\der^3 x\, \der^3 \bs{p}}.
\end{equation}
 This is the phase space density of gluons averaged over their 
 degrees of freedom 
 $\nu_g = 2 d_A,$ including $d_A = N_c^2 -1$ colors and two spins. 
 We assume that the distributions are independent of transverse coordinate, boost invariant  and do not depend on spin.
The time-evolution of the distribution is given by the Boltzmann equation
\begin{equation}
\label{eq:EKTEOMS}
-\dfrac{\partial f(\bs{p})}{\partial \tau } = \mathcal{C}_{1 \leftrightarrow 2  }[f(\bs{p})] + \mathcal{C}_{2 \leftrightarrow 2  }[f(\bs{p})] + \mathcal{C}_{\mathrm{exp} }[f(\bs{p})].
\end{equation}
Here the $\mathcal{C}$ terms describe the number of scatterings/splittings per unit time for momentum state $\bs{p}$. These processes involve the possibility to scatter/split into or out of a state. 
Effective particle 1 to 2 splittings are encoded in $\mathcal{C}_{1 \leftrightarrow 2  }$ and 2 to 2 scatterings are described by $\mathcal{C}_{2 \leftrightarrow 2  }$. 

The boost invariant  expansion is treated as an effective scattering term, taking the simple form \cite{Mueller:1999pi}
\begin{equation}
\mathcal{C}_{\mathrm{exp} }[f(\bs{p})] = - \dfrac{p_z}{\tau} \dfrac{\partial}{\partial p_z} f(\bs{p}).
\end{equation}
These assumptions leave
our distributions $f$ dependent only on the magnitude of the momentum $p$ and on the polar angle $\cos \theta = \hat{\bs{z}} \cdot \hat{\bs{p}}$, i.e $f(\bs{p}) = f(p, \cos \theta_p)$.
Our spherical coordinate system is defined such that 
\begin{align}
\pz &= p \cos \theta \\
\pt &= p \sin \theta.
\end{align}
Momenta in the $x$ and $y$ directions are given in terms of the azimuthal angle $\phi$ 
\begin{align}
p_x &=  \pt \cos \phi  \\
p_y &= \pt \sin \phi.
\end{align}
The 2 to 2 scattering term reads  \cite{Arnold:2002zm}
\begin{align}
\label{eq:C22}
\mathcal{C}_{2 \leftrightarrow 2  }&[f(\tilde{\bs{p}})] = \dfrac{\left( 2 \pi \right)^3}{4 \pi \tilde{p}^2} \dfrac{1}{8 \nu_g} \int \der \Gamma_{\mathrm{PS}} \left|M \right|_{2 \leftrightarrow 2  }^2 \nn \\
& \times  \left( f_{\bs{p}} f_{\bs{k}} (1 + f_{\bs{p^\prime}}) ( 1 + f_{\bs{k^\prime}}) - \distvec{p^\prime} \distvec{k^\prime} (1 + \distvec{p}) (1 + \distvec{k}) \right) \nn \\
& \times \left( \delta(\bs{\tilde{p}} - \bs{p})  + \delta(\bs{\tilde{p}} - \bs{k}) - \delta (\bs{\tilde{p}} - \bs{p^\prime}) - \delta(\bs{\tilde{p}} - \bs{k^\prime})\right),
\end{align}
where $\der \Gamma_{\mathrm{PS}}$ is a phase space integration over the incoming and outgoing momenta $\bs{p},\bs{p^\prime},\bs{k}$ and $\bs{k^\prime}$. The matrix element squared
 $|M|_{2 \leftrightarrow 2  }^2$  at leading order $gg \to gg$ is
 \footnote{Note that there is a typo in \cite{Kurkela:2015qoa} where some factors of $1/4$ were missing.}
\begin{align}
\label{eq:ggMatrixElem}
\dfrac{\left|M\right|_{2 \leftrightarrow 2  }^2}{4 \lambda^2 d_A} = \left( 9 + \dfrac{(s-t)^2}{u^2} + \dfrac{(u-s)^2}{t^2} + \dfrac{(t-u)^2}{s^2} \right).
\end{align}

Here $\lambda = g^2 N_c$ is the 't Hooft coupling. In \eqref{eq:ggMatrixElem} the division by the Mandelstam variables $t$ and $u$ leads to infrared divergences. These can be regulated by including contributions arising from medium modifications. In the employed numerical framework these are approximated by substituting the denominators in \eqref{eq:ggMatrixElem} with 
\begin{equation}  \label{eq:regt}
 t \to t\, \frac{q^2  + 2\xi_0^2 m^2 }{q^2},
\end{equation}
with  $q = |\bs{q}|$ and $\bs{q}=\bs p' - \bs p$, and symmetrically for $u$.
Note that the limit $t\to 0$ with $s$ fixed is reached only when $q^2 \to 0$. 
We follow the prescription 
of \cite{AbraaoYork:2014hbk} and use $\xi_0 = \nicefrac{e^{\nicefrac{5}{6}}}{\sqrt{8}}$. Here $m$ is  the asymptotic mass in the gluon dispersion relation at high momentum that is perturbatively related to the Debye mass $\md$ as
\begin{align}
    \label{eq:mDdefinition}
    \md^2 = 2 m^2 = 4\lambda \int \frac{\ud^3 p}{(2\pi)^3}\,\frac{f(\bs p)}{p}\,.
\end{align}
The final state Bose enhancement in \eqref{eq:C22} is governed by the  factors $(1 + f)$. The first term  in \eqref{eq:C22} is a loss term for particles scattering out of the state and the second term is the term associated with scattering into the state.  The phase space integration measure is given by
\begin{align}
 \int \der \Gamma_{\mathrm{PS}}  &=  \int_{\bs{p k p' k'}}
 \left(2 \pi \right)^4 \delta\left( P + K - P^\prime - K^\prime \right) \nn \\
 & = \dfrac{1}{2^{11} \pi^7} \int_0^\infty \der q \int_{-q}^q \der \omega \int_{\frac{q-\omega}{2}}^\infty \der p  \int_{\frac{q+\omega}{2}}^\infty \der k \nn \\ 
 & \times \int_{-1}^1 \der x_q \int_0^{2 \pi} \der \phi_{pq} \int_0^{2 \pi} \phi_{kq},
 \end{align}
 with $\omega = q^0,$ $x_q = \cos \theta_q$, 
$\int_{\bs{p}} =  \int \frac{\der p^3}{2 p^0\left(2 \pi \right)^3}$ and $p^0 = p$. 
The simplification above is achieved by changing variables from $p^\prime$ to $q$ and using delta functions to eliminate the $k^\prime$ integral. The residual three-dimensional integrals can be written as integrals over four-momenta, introducing delta functions that constrain the energy component. The integration limits are found by inspecting the  energy delta functions. One also makes use of the fact that the system is azimuthally symmetric. 

The collinear 1 to 2 splitting rate  $\mathcal{C}_{1 \leftrightarrow 2}$  is given by 
 \begin{align}
 &\mathcal{C}_{1 \leftrightarrow 2  }\left[f\right] (\bs{\tilde{p}})= \dfrac{\left( 2 \pi \right)^2}{4 \pi \tilde{p}^2} \dfrac{1}{\nu_g}  \int_0^\infty \der p \int_0^{p/2} \der k^\prime \left( 4 \pi \gamma(p ; p^\prime , k^\prime) \right) \nn \\
 &\times \Big[\distvec{p}  \left( 1 + f(x_p , p^\prime) \right) (1+f(x_p , k^\prime)) \nn \\ &  - (1 + \distvec{p}) f(x_p, p^\prime) f(x_p , k^\prime) \Big] \nn \\
 &\times \left[ \delta(\tilde{p} - p) - \delta(\tilde{p} - p^\prime)  - \delta(\tilde{p} - k^\prime) \right].
\end{align}
 Due to collinearity all momenta are pointing in the same direction. We take the particle with momentum $p$ to be the particle that splits, and hence $p^\prime = p - k^\prime$.
Here $\gamma$ parametrizes the differential rate for the splitting processes and is given by 
\begin{equation}
\gamma^g_{gg}(p , p^\prime , k^\prime) = \dfrac{p^4 + p^{\prime 4} + k^{\prime4}}{p^3 p^{\prime^3} k^{\prime^3}} \mathcal{F}_g(p; p^\prime , k^\prime).
\end{equation}
The function $\mathcal{F}$ is computed from 
\begin{equation}
\mathcal{F}(p , p^\prime , k^\prime) = \dfrac{d_A C_A \alpha_s}{2 (2 \pi)^3} \int \dfrac{\der^2 h }{\left( 2 \pi \right)^2} 2 \bs{h} \cdot \mathfrak{Re} \bs{F}_g (\bs{h} , p , p^\prime , k^\prime),
\end{equation}
where $C_A = N_c$ and $\alpha_s = \nicefrac{g^2}{4 \pi}.$

The function $\bs{F}_g$ is a solution of an integral equation 
\begin{align}
\label{eq:integralEqu}
&2 \bs{h} = i \delta E( \bs{h} ; p , p^\prime, k^\prime) \bs{F}_g(\bs{h}; p , p^\prime, k^\prime) +  \nn \\ 
& \times \dfrac{g^2 C_A}{2} T_* \int \dfrac{\der^2 q_\perp}{\left( 2 \pi \right)^2} \left[ \dfrac{1}{q_\perp^2} - \dfrac{1}{q_\perp^2 + m_D^2 } \right] \nn \\ 
& \times \Big[\left(\bs{F}_g(\bs{h}; p , p^\prime, k^\prime) - \bs{F}_g(\bs{h} - k^\prime \bs{q}_\perp; p , p^\prime, k^\prime) \right)  \nn \\
& + \left(\bs{F}_g(\bs{h}; p , p^\prime, k^\prime) - \bs{F}_g(\bs{h} - p^\prime \bs{q}_\perp; p , p^\prime, k^\prime) \right) \nn \\
& + \left(\bs{F}_g(\bs{h}; p , p^\prime, k^\prime) - \bs{F}_g(\bs{h} + p \bs{q}_\perp; p , p^\prime, k^\prime) \right)\Big].
\end{align}
Here $\bs{\hat{n}}$ is a unit vector in the direction of the splitting or merging hard particle(s). The vector $\bs{h}$ is perpendicular to $\bs{\hat{n}}$, and can be related to the transverse momentum \cite{Arnold:2002ja}. In deriving \eqref{eq:integralEqu} the Wightman correlation function
 is evaluated by making use of the 
 sum rule derived in \cite{Aurenche:2002pd} using an isotropic screening approximation.
The energy difference in \eqref{eq:integralEqu} is given by  
\begin{align}
\delta E( \bs{h} ; p , p^\prime, k^\prime) = \dfrac{m^2}{2 k^\prime} + \dfrac{m^2}{2 p^\prime} - \dfrac{m^2}{2 p} + \dfrac{\bs{h}^2}{2 p k^\prime p^\prime }.
\end{align}
The integral equation \eqref{eq:integralEqu} is solved by using the numerical method described in \cite{Ghiglieri:2014kma}.

For a more comprehensive description of EKT we refer the reader to \cite{Arnold:2002zm} and we discuss our discretization procedure in \ref{app:numerics}. 



\subsection{Initial conditions: bottom-up thermalization}

Our initial conditions are chosen to correspond to the bottom-up isotropization scenario, as implemented in Ref.~\cite{Kurkela:2015qoa}. 
The initial distribution function at the time $\Q \tau = 1$ 
is taken to be
\begin{align}
f(p_\perp, p_z) & = \dfrac{2}{\lambda} A \dfrac{\langle p_T \rangle}{\sqrt{p_\perp^2 + (\xi p_z)^2}} \nonumber \\
& \times \exp{\left( \dfrac{-2}{3 \langle p_T \rangle^2 } \left(p_\perp^2 + (\xi p_z)^2  \right) \right)}.
\label{eq:IC}
\end{align}
In this paper we consider the same two sets of initial parameters with different anisotropies $\xi$ as in Ref.
~\cite{Kurkela:2015qoa}, given by
\begin{eqnarray}
    \label{eq:GoodParamsXi10}
    A = 5.24171\,, \quad \langle p_T \rangle = 1.8  \Q\,, \quad \xi = 10 \\
    A = 2.05335\,, \quad \langle p_T \rangle = 1.8 \Q\,, \quad \xi = 4\,.
    \label{eq:GoodParamsXi4}
\end{eqnarray}
In the following we will refer to these only by the value of the anisotropy parameter $\xi$. In our figures the initial condition with $\xi=10$ is always represented by full lines. When $\xi=4$ results are shown for comparison, we use dash-dotted transparent lines. 




\subsection{Heavy quark diffusion coefficient}

The expression for the heavy quark momentum diffusion coefficient $\kappa$ for pure glue QCD has been originally derived in \cite{Moore:2004tg}. Keeping only the leading order contributions in $\nicefrac{1}{M}$, where $M$ is the mass of the heavy quark, we obtain the diffusion coefficient as
\begin{align}
\label{eq:kappa_master_formula}
3 \kappa &= \frac{1}{2M}\int_{\bs{k} \bs{k^\prime} \bs{p^\prime}}\left(2 \pi \right)^3 \delta^3\left( \bs{p} +\bs{k} - \bs{p^\prime} - \bs{k^\prime} \right) \nn \\
& \times  2 \pi \delta \left(k^\prime - k \right) \bs{q}^2 
\left[ \left| \mathcal{M}_\kappa \right|^2 f(\bs{k}) (1+f(\bs{k^\prime})) \right].
\end{align}
The heavy quark mass is taken to be larger than  any other scale in the system (e.g. $M \gg T, \Q, m_D$). The in- and outgoing heavy quark momenta are given by  $\bs{p}, \bs{p^\prime}$, the momenta 
of gluons in the plasma are labeled by $\bs{k} , \bs{k^\prime}$, and $\bs{q}=\bs{p}-\bs{p^\prime}$ is the momentum transfer to the heavy quark. The gluon distribution function $f$ is obtained from solving the Boltzmann equation \eqref{eq:EKTEOMS}.
A final state Bose-enhancement factor appears in \eqref{eq:kappa_master_formula} similarly as in the Boltzmann equation. 
We set $p^0 = p'{}^0 = M$ 
for the heavy quark. For the gluons and $k^0 = \left| \bs{k} \right|$, $k'{}^0 = \left| \bs{k'} \right|$. 

To leading order in the coupling, the dominant contribution to $\kappa$ is given by $t$-channel gluon exchange \cite{Moore:2004tg}, while other diagrams are suppressed by inverse powers of the heavy quark mass. The corresponding matrix element is then given by
\begin{align}
\label{eq:MatrixElement}
\left| \mathcal{M}_\kappa \right|^2 = N_c C_H g^4\,  \frac{16 M^2 k_0^2 (1+ \cos^2 \theta_{\bs{k} \bs{k^\prime} } )}{\left(q^2 + m_D^2 \right)^2},
\end{align}
where $C_H = \nicefrac{(N_c^2-1)}{2 N_c}$.
Note that due to the heavy quark mass the HTL propagator reduces in this case to just Debye screening~\cite{Moore:2004tg}, unlike for massless particles in  Eq.~\nr{eq:regt}.

The resulting expression $\left| \mathcal{M}_\kappa \right|^2/M^2$ entering the Boltzmann equation is independent of the heavy quark mass. In our simulations, $\kappa$ is computed using \eq \eqref{eq:kappa_master_formula}. The discretization of \eqref{eq:kappa_master_formula} and a detailed description of how transverse and longitudinal diffusion coefficients are extracted, are explained in more detail in the \app \ref{se:DiscreteKappa}. 


Perturbatively, in thermal equilibrium the screening mass \eqref{eq:mDdefinition} becomes
\begin{equation}
\label{eq:mDvsT}
\md^2 = \dfrac{T^2 \lambda}{3}.
\end{equation}
In our kinetic simulations we compute $\md$ according to \eqref{eq:mDdefinition} using the nonequilibrium distribution $f(\bs p)$. 
For more details on the diffusion coefficient we refer the reader to \cite{Moore:2004tg} and to our previous work \cite{Boguslavski:2020tqz}. 






For an isotropic distribution in the continuum,  the diffusion coefficient \eqref{eq:kappa_master_formula} becomes 
\begin{align}
\label{eq:kappaLO}
\kappa &= \dfrac{\lambda^2 C_H}{12 N_c \pi^3 } \int_0^\infty \der k\,k^2  f(k) (1 + f(k)) \nn \\
& \quad \times \int_0^{2 	k}\der q\,\dfrac{q^3}{\left(q^2 + m_D^2 \right)^2}\left( 2 - \dfrac{q^2}{k^2} + \dfrac{q^4}{4 k^4} \right).
\end{align}
This expression is very easy to evaluate in thermal equilibrium using a Bose-Einstein distribution for $f(p)$ and the thermal result \eqref{eq:mDvsT} for $m_D$. 



\subsection{Other observables}

Here we describe the observables that we are extracting during the nonequilibrium evolution in addition to the heavy quark diffusion coefficient. 

\subsubsection{Expectation values}
In general we use the following definition for expectation value of observable $X$
\begin{equation}
\label{eq:ExpectationValueDef}
\langle X \rangle = \frac{\int \derthree\,  X\, f(\bs p)}{\int \derthree  f(\bs p)}\,.
\end{equation}

\subsubsection{Energy density}
The energy density corresponds to the first moment of the distribution 
\begin{equation}
\label{eq:epsdef}
\eps= \nu_g \int \dfrac{\der^3 p}{\left( 2 \pi \right)^3}\, p f(\bs p),
\end{equation}
where $\nu_g = 2 \left(\nc^2 -1\right)$ for pure glue QCD.



\subsubsection{Temperature}
There is no unambiguous way to define a temperature out of equilibrium. One option is to formulate it effectively  in terms of the (time-dependent) energy density
\begin{equation}
\label{eq:sensibleTemperature}
 T_\eps = \left(\dfrac{30\,\eps}{\pi^2\nu_g} \right)^{1/4} ,
\end{equation}
which agrees with the temperature in thermal equilibrium.
For ideal Bjorken hydrodynamics, the temperature scales as 
\begin{equation}
\label{eq:Tscaling}
T_{id} \sim \left( \Q \tau \right)^{-1/3}.
\end{equation}
This scaling is also expected to hold for $T_\eps$ at sufficiently late times 
due to the approximate $\tau^{-4/3}$ scaling of the energy density in the expanding system.


The effective infrared temperature $\tstar$ is given by 
\begin{equation}
\label{eq:Tstardef}
\tstar = \dfrac{I}{J},
\end{equation}
where 
\begin{align}
I &= \dfrac{1}{2} \int \dfrac{\der^3 p}{\left(2 \pi \right)^3}\, f(\bs p) (1+f(\bs p)) \\
J &= \int \dfrac{\der^3 p }{\left(2 \pi \right)^3} \dfrac{f(\bs p)}{p} = \dfrac{m_D^2}{4 \lambda }.
\end{align}
Slightly different definitions of $\tstar$, based on other integral moments,  have been also introdued in the literature~\cite{Kurkela:2018oqw}, but we do not consider them here. 




\subsubsection{Energy momentum tensor}

The components of the energy momentum tensor are obtained as moments of the distribution function by
\begin{equation}
T^{\mu \nu} = \nu_g \int \derthree \dfrac{p^\mu p^\nu }{p} f(\bs p).
\end{equation}
The components relevant for this paper are 
\begin{align}
T_{xx} &= \nu_g \int \derthree f(\bs p) \dfrac{\pt^2}{p} \cos^2\phi \\ 
T_{yy} &= \nu_g  \int \derthree f(\bs p) \dfrac{\pt^2}{p} \sin^2\phi \\
T_{zz} &= \nu_g \int \derthree  f(\bs p) \dfrac{\pz^2}{p}.
\end{align}
 They are connected to the longitudinal and transverse pressure by
\begin{align}
P_T &= \dfrac{T_{xx}+T_{yy}}{2} \\
P_z &= T_{zz}.
\end{align}
The temporal component of the energy-momentum tensor corresponds to the energy density $T_{00} \equiv \eps$.



\begin{figure}
  \centering
  \includegraphics[width=0.48\textwidth]{Plots/occupancyVsAnisotropyPRL.pdf}
\caption{Occupation number as a function of anisotropy as in \cite{Kurkela:2015qoa}. The markers indicate different stages of the bottom up thermalization as described in the text. The color coding of the lines indicates the coupling used in the simulation, and all figures use the same color coding. Full lines correspond to the $\xi=10$ initial condition \eqref{eq:GoodParamsXi10}, dash-dotted lines to the $\xi=4$ initial condition \eqref{eq:GoodParamsXi4}. }
\label{fig:occupVsAnisotropy}
\end{figure}


\section{Results }
\label{sec:results}


We start by discussing the bottom-up isotropization process and how it is reproduced in our simulations in \se \ref{sec:results-bottom-up}. We will also discuss how we highlight different stages of the isotropization process. In \se \ref{sec:matching} we will explain how we compare our $\kappa$ results out of equilibrium to thermal values, and especially how we choose the corresponding matching scales. The comparison is then carried out in \se\ref{se:kappaInOutEqu}, first without distinguishing between directions. Then in \se\ref{se:kappaTransLong} we discuss the transverse and longitudinal diffusion coefficients separately. In order to better understand the observed time-evolution of $\kappa$, we will consider how the matching scales evolve during the bottom-up isotropization compared to equilibrium in \se \ref{se:MatchingEvolution}. Finally, we will derive simple parametric estimates that can be used to explain our results in \se \ref{se:paramest}.



\subsection{Different stages of the bottom up thermalization scenario}
\label{sec:results-bottom-up}

Our kinetic theory simulations follow the different phases of the bottom-up thermalization scenario \cite{Baier:2000sb}. It consists of the following stages: during the first stage the overoccupied gluons become more dilute as the system expands, and consequently the occupation number of the hard gluons becomes of the order of unity $f_h \sim 1$ at time scale $\Q \tau \sim \alpha_s^{-3/2}$. At this point the system is no longer describable with classical fields. In the second stage hard gluons radiate softer gluons, creating a soft thermal bath. Remarkably, at the end of this process the hard gluons become underoccupied $f_h \sim \alpha_s$. 
This occurs at the timescales $\Q \tau \sim \alpha_s^{\nicefrac{-5}{2}}$. In the final stage,
the hard particles lose their energy to the soft thermal bath.
For a review on the thermalization processes see e.g.~\cite{Schlichting:2019abc,Berges:2020fwq}.

In this scenario, the system is expected to thermalize parametrically on a timescale of the order of \cite{Baier:2000sb}
\begin{equation}
\label{eq:tauT}
\tauT =  \nicefrac{\alpha_s^{\nicefrac{13}{5}}}{\Q},
\end{equation}
where $\alpha_s = \frac{\lambda}{4 \pi N_c}$. We will therefore use this quantity to rescale the time in our figures. Note that an alternative time scale, the hydrodynamical relaxation time $\tau_R = \frac{4 \pi \eta/s(\lambda)}{ T}$ with shear viscosity $\eta$ and entropy density $s$, is also often used to rescale the time variable. We will investigate the universality of these time scales for different observables and couplings in a separate paper, while employing only $\tauT$ in the present work.

The bottom-up thermalization process is shown in \fig \ref{fig:occupVsAnisotropy} in terms of the anisotropy $\nicefrac{P_T}{P_L}$ and the mean occupation number $\nicefrac{\langle p\lambda f \rangle}{\langle p \rangle}$ for different couplings as in \re \cite{Kurkela:2015qoa}. In order to illustrate how observables behave during different stages of the thermalization process, we have placed three time markers on the curves in \fig \ref{fig:occupVsAnisotropy}. The first marker (star) is placed during the highly occupied regime, when $f \sim \nicefrac{1}{\lambda}$. For smaller values of the coupling this corresponds to maximal anisotropy. However, for large couplings the first stage of the evolution proceeds differently, and the anisotropy does not increase initially. Hence, we have chosen the occupancy as the criterion for the time marker instead of maximum anisotropy. The second marker (circle) is inserted at the minimum occupancy, which in the bottom-up thermalization scenario is expected to be $f \sim \alpha_s$. The third marker (triangle) is placed at $\nicefrac{P_T}{P_L} = 2.$ The purpose of this is to illustrate when the system is approximately close to equilibrium. We will use the same markers in other figures throughout this paper to allow the reader to connect the time-evolution of observables to the stages of the bottom-up scenario.



The curves for different values of the coupling $\lambda$ also use the same color coding in all figures throughout this paper.  The initial conditions with $\xi=10$ from  \eqref{eq:GoodParamsXi10} are shown as full lines. For comparison, we will often add curves for the initial conditions with $\xi=4$ in \eqref{eq:GoodParamsXi4} as more transparent dash-dotted lines.



We see from \fig \ref{fig:occupVsAnisotropy} that for the extremely anisotropic initial condition ($\xi=10$) the bottom-up picture is better realized at smaller values of the coupling. For intermediate couplings $\lambda = 2,5,10$ the system does not experience an initial growth in anisotropy. Instead, the system takes a more straightforward path to thermal equilibrium, without resolving of the different stages of the bottom-up picture in detail.
However, the third stage of the scenario is still visible and emerges after the circle marker.



\begin{figure}[tbhp!]
  \centering
  \includegraphics[width=0.45\textwidth]{Plots/kappaVsKappaEqMatchedEnergyDensitySmallEpsilonTauLambda135.pdf}

  \includegraphics[width=0.45\textwidth]{Plots/kappaVsKappaEqMatchedMDSmallEpsilonTauLambda135.pdf}

  \includegraphics[width=0.45\textwidth]{Plots/kappaVsKappaEqMatchedTstarSmallEpsilonLambdaTau135.pdf}
 \caption{Ratio of $\kappa$ to the thermal value at the  same energy density $\varepsilon$ (top), the same  screening mass $\md$ (middle) and effective soft mode temperature $\tstar$ (bottom).
We have applied Savitzky-Golay filter to smoothen the data. The filter is also applied to all following figures involving $\kappa$.
}
\label{fig:kappa_vs_kappaeq}
\end{figure}



\begin{figure}[tbh!]
  \includegraphics[scale=0.5]{Plots/kappaTkappaLratioThermalization.pdf}
  \caption{Ratio of the transverse and longitudinal diffusion coefficients during the bottom-up thermalization scenario. We observe that the transverse coefficient is enhanced compared to the longitudinal one during the initial evolution. When the system becomes underoccupied, this ordering reverses. 
  Finally when the system reaches approximate equilibrium the ratio approaches unity.}
\label{fig:KappaRatio}
\end{figure}

\begin{figure*}[tbh!]
\includegraphics[scale=0.47]{Plots/kappaTvsKappaEqMatchedEpsilonBMSS.pdf}
\rule{3em}{0pt}
\includegraphics[scale=0.47]{Plots/kappaZvsKappaEqMatchedEpsilonBMSS.pdf}

\includegraphics[scale=0.47]{Plots/kappaTvsKappaEqMatchedMDBMSS.pdf}
\rule{3em}{0pt}
\includegraphics[scale=0.47]{Plots/kappaZvsKappaEqMatchedMDBMSS.pdf}

  \includegraphics[scale=0.47]{Plots/kappaTvsKappaEqMatchedTstarTBMSS.pdf}
\rule{3em}{0pt}
  \includegraphics[scale=0.47]{Plots/kappaZvsKappaEqMatchedTstarBMSS.pdf}
\caption{
\label{fig:kappa_vs_kappaeq_TL}
Transverse (left panel) and longitudinal (right panel) diffusion coefficients compared to thermal values for the same $\varepsilon$ (top row), $\md$ (middle) and $\tstar$  (bottom row). 
}
\label{fig:kappa_vs_kappaeq_translong_tstarmd}
\end{figure*}


\subsection{Comparing nonequilibrium to equilibrium}
\label{sec:matching}
Our aim in this paper is to calculate 
the heavy quark momentum diffusion coefficient $\kappa$ during the hydrodynamization process 
in order to eventually assess the importance and impact of the initial nonequilibrium evolution on heavy quark observables. To facilitate the quantitative interpretation we will mostly present our results as ratios to the thermal equilibrium values. There is no unique method to compare equilibrium and nonequilibrium systems. A reasonable way to construct such a comparison is to match a dimensionful observable to its thermal counterpart. 
We will consider three different quantities - the energy density $\eps$ (which leads to the temperature defined in \eqref{eq:sensibleTemperature}), the temperature of infrared modes $\tstar$, and the screening mass $\md$. We have chosen these observables 
because they are straightforward to compute both in equilibrium and out of equilibrium  and are physical scales that play an important role in transport phenomena, as discussed in Sec.~\ref{se:MatchingEvolution}.
However, we would like to emphasize that the matching could be done with respect to any dimensionful observable that can be defined in and out of equilibrium.
The matching is performed by choosing the temperature of the thermal distribution to reproduce either the energy density  $\varepsilon$, the Debye mass $ \md$ or the effective temperature of the soft modes $\tstar$ calculated from the EKT simulation, and then calculating the diffusion coefficient $\kappa$ with this thermal distribution.

Due to the longitudinal expansion, the discretization accuracy in our EKT simulation becomes gradually worse over time. 
For instance, the dimensionless ratio of the infrared cutoff and the temperature $\pmin/T(\tau)$ grows with time due to the approximate scaling of $T(\tau)$ in \eqref{eq:Tscaling}. 
In practice this means that the thermal equilibrium that the system approaches will have also have 
discretization effects. We will take these into account by matching, instead of continuum thermal quantities ($\varepsilon,\md,\tstar$), ones calculated from a thermal distribution with the same cutoff as the EKT simulation.
This correction enables us to compare our nonequilibrium results to the  thermal state that the system is actually approaching.
We will apply this correction to all observables when comparing with equilibrium quantities unless stated otherwise. This effect is discussed in more detail in Appendix~\ref{app:pminDependence}.





\subsection{Diffusion coefficient in and out of equilibrium}
\label{se:kappaInOutEqu}

Figure \ref{fig:kappa_vs_kappaeq} shows our result for the heavy quark diffusion coefficient during bottom-up thermalization. 
we observe an agreement between the equilibrium and thermal dynamics within $20\%$ even at early times, while at late times the system thermalizes and the ratio thus approaches unity. 
This is one of the main results in this paper.
Matching the  screening mass $\md$ (\ref{fig:kappa_vs_kappaeq} middle) or the effective infrared temperature $\tstar$ (\ref{fig:kappa_vs_kappaeq} bottom), $\kappa$ is much further away from the thermal system. The main observation is that at early times the non-equilibrium diffusion coefficient is considerably larger than the equilibrium coefficient for the same $\md$, and considerably smaller than that for the same $\tstar$.


\subsection{Transverse and longitudinal diffusion coefficient}
\label{se:kappaTransLong}

During most of the bottom up thermalization procedure the system is highly anisotropic.  This could have a significant effect on experimental observables sensitive to the initial stage of the evolution. It is therefore interesting to study also the anisotropy of the diffusion coefficient, which we parametrize in terms of the ratio of the transverse diffusion coefficient  $\kappa_T$ to the longitudinal one $\kappa_z$. Our results for this quantity are shown in \fig \ref{fig:KappaRatio}. In the initial overoccupied phase the transverse diffusion coefficient dominates.  
This can naturally be explained by the Bose enhancement, which at the early overoccupied and highly anisotropic stage benefits scatterings where only transverse momentum is exchanged.
As the system becomes underoccupied, the longitudinal diffusion coefficient becomes dominant. When the system approaches equilibrium the two become equal again, as expected due to the emerging isotropization. Thus the hierarchy of the diffusion coefficients has a natural explanation in terms of the stages of the bottom-up thermalization scenario. The ordering of the coefficients $\kappa_T < \kappa_z$ after the star marker, i.e., after the initially large occupancies,
is also in line with what is observed using squeezed thermal distributions \cite{Romatschke:2006bb} and results from the momentum anisotropy of the system.



We can compare the pre-equilibrium system to the thermal one using the same three matching procedures as in \fig \ref{fig:kappa_vs_kappaeq} to the transverse and longitudinal diffusion coefficients separately. The results are shown in 
\fig \ref{fig:kappa_vs_kappaeq_TL},
and show qualitatively similar results as for the full coefficient $\kappa$. Similarly as in \fig \ref{fig:kappa_vs_kappaeq} we find that the best way to compare our nonequilibrium results to a thermal system is by matching for the same energy density $\varepsilon$. This way the deviations from equilibrium turn out to be not larger than approximately 40 \%.  For the same screening mass $\md$ we observe very large deviations after the initially highly occupied regime, i.e., after the star time marker. 
In particular, we observe 
deviations up to a factor of 4 for the transverse and 6 for the longitudinal coefficient, respectively, for the smallest coupling. 


\begin{figure}[tbh!]
\centering
\includegraphics[height=6cm]{Plots/occupancyVsTLambda135PRL.pdf}
\caption{ Time evolution of the occupation number of the hard modes. }
\label{fig:scalesVsEquil1}
\end{figure}
 
\begin{figure}[tbh!]
\centering
  \includegraphics[height=6cm]{Plots/energyDensityVsTauLambda135.pdf}

  \includegraphics[height=6cm]{Plots/TepsVsTidRatio.pdf}
\caption{ Time evolution of  the energy density scaled by the coupling and the ideal hydrodynamics time dependence (top) and the temperature corresponding to the energy density scaled by the time dependence of the temperature in ideal hydrodynamics $T_{id}\sim \tau^{-1/3}$ (bottom).
}
\label{fig:scalesVsEquil2}
\end{figure}

\begin{figure}[tbh!]
\centering
  \includegraphics[height=6cm]{Plots/mDvsmDEquilTauLambda135.pdf}

  \includegraphics[height=6cm]{Plots/tstarvststarEquilTLambda135.pdf}
\caption{ Time evolution of the Debye mass $\md$ (top) and the effective temperature of the soft scales $\tstar$ (bottom), divided by the thermal value at the same energy density. 
The thick dotted curves  correspond to the parametric estimates derived in \se \ref{se:paramest} corresponding to $\lambda=1, \xi=10$.
}
\label{fig:scalesVsEquil3}
\end{figure}



\subsection{Time-evolution of the relevant scales }
\label{se:MatchingEvolution}
To better understand the time dependence of $\kappa$, let us now discuss the evolution of relevant scales that it depends on. The different time dependences that are relevant for this discussion are summarized in \figs\ref{fig:scalesVsEquil1}, \ref{fig:scalesVsEquil2} and~\ref{fig:scalesVsEquil3}. As a first step, we start with the time dependence of the occupation number of the hard modes, shown in \fig\ref{fig:scalesVsEquil1}. It starts in an overoccupied state, then becomes underoccupied before reaching a thermal value. 

Figure \ref{fig:scalesVsEquil2} shows the time dependence of the energy density. In the upper panel, the initial value $\varepsilon\sim 1/\lambda$ is scaled out by multiplying with the coupling $\lambda$, so that all the curves start at the same  value by construction. The time dependence is additionally divided by the ideal hydrodynamical behavior $\tau^{-4/3}$. We see that during most the bottom up thermalization  the energy density decreases as $\varepsilon\sim 1/\tau$, until approaching the hydrodynamical behavior only at the triangle marker near the end. The bottom panel shows the temperature $T_\eps$ extracted from the energy density and the temperature in ideal hydrodynamics $T_{id}$, extrapolated backwards from the final state.  The latter is computed by first matching it with $T_\eps$ at the triangle time marker  of near isotropy when the system exhibits an almost hydrodynamical behavior, and  propagating backwards in time as in $T_{id} \sim \tau^{\nicefrac{-1}{3}}$, as discussed in  \cite{Kurkela:2018oqw}. We find that the temperature deviates from the ideal hydrodynamics temperature at most by a factor of 2 at early times for the couplings we have considered. However, when the system is underoccupied at the circle time marker, the ideal temperature is in reasonably good agreement (within 20\%) with the temperature extracted from the energy density. 

Let us now return to the heavy-quark diffusion coefficient.
Parametrically, it is given by \cite{Boguslavski:2020tqz} 
\begin{align}
\kappa \sim \md^2\, g^2 \tstar \log \left(\frac{\Lambda}{\md} \right), 
\end{align}
where $\Lambda$ is the largest momentum scale of particles in the plasma; for a thermal system we have $\Lambda\approx T$.
Ignoring the logarithmic contribution, we estimate
\begin{equation}
    \frac{\kappa}{\kappa_{eq}} \approx \frac{\md^2}{(\md^{eq})^2} \frac{\tstar}{\tstar^{eq}}.
    \label{eq:kappaRatioPocketFormula}
\end{equation}



To understand the behavior of the heavy quark diffusion coefficient, we must thus look at the time dependence of $\md$ and $\tstar$  which are shown on the top and bottom, respectively, in \fig\ref{fig:scalesVsEquil3}.
We observe that initially $\md$ is enhanced compared to the thermal system at the same energy density. When the system becomes underoccupied, especially for weak couplings, $\md$ gets relatively suppressed. The suppression is mainly driven by a decreasing occupation number, which is discussed in more detail in \se \ref{se:paramest}.
For the effective temperature $\tstar$ we observe a strong enhancement at the early stage, which is understood from the large occupancies encountered initially as discussed in more detail in \se \ref{se:paramest}.
When the system becomes underoccupied, i.e., in the vicinity of the circle time marker, $\tstar$ is already very close to its equilibrium value for the same energy density. Combining the curves in \fig\ref{fig:scalesVsEquil3} we find a quite clear understanding for the behavior seen in \fig\ref{fig:kappa_vs_kappaeq}; first an enhancement compared to the the equilibrium value due to the enhancement of both $\tstar$ and $\md$ in the overoccupied phase, followed by a suppression due mostly to the suppression of $\md$. Let us now construct a parametric estimate that makes this behavior explicit.





\begin{figure}[tbh!]
\includegraphics[scale=0.5]{Plots/kappaVsKappaEqAndParametricGuesstimates.pdf}
\caption{Ratio of the diffusion coefficient to the equilibrium value  for the same energy density. The wide transparent lines show the parametric estimate given by \eqref{eq:kappaRatioPocketFormula} using the extracted values for $\md$ and $\tstar$, using the data in \fig\ref{fig:scalesVsEquil3}. The gray dotted line corresponds to the parametric estimate \eqref{eq:kappaguesstimate}.}
\label{fig:parametricGuesstimate}
\end{figure}




\subsection{Understanding the evolution of scales and $\kappa$}
\label{se:paramest}

Let us try to qualitatively understand the time-evolution of the matching scales and of the heavy-quark diffusion coefficient in terms of the occupation number and anisotropy. For the purposes of our parametric estimate, we use the following squeezed and scaled distribution
\begin{equation}
\label{eq:f_params}
f(\pt, \pz) = f_0\, \theta \left(\Q^2 - (\pt^2 + \left( \nicefrac{\pz}{\delta}\right)^2 \right).  
\end{equation}
 We determine the typical occupation number $f_0$ during bottom-up thermalization by taking its value be the typical occupation number of the hard modes, calculated as $f_0 = \nicefrac{\langle p f \rangle}{\langle p \rangle}$. The values of this parameter can be read off  \fig\ref{fig:scalesVsEquil1}.
The squeezing parameter $\delta$ describes the momentum anisotropy, $ \delta \sim \frac{\langle p_z \rangle}{\langle \pt \rangle }$. We take its value during bottom up evolution from  $\delta = \sqrt{\nicefrac{P_L}{P_T}}$, where the values taken by the pressure ratio can be seen from \fig\ref{fig:occupVsAnisotropy}.

For the purposes of this parametric estimate, we approximate the thermal distribution with the same simplified form \eqref{eq:f_params}. For this, we obtain the value $f_0$
by calculating the expectation values for a thermal (Bose-Einstein) distribution (corresponding to to the crosses in \fig \ref{fig:occupVsAnisotropy}). The value is independent of $\lambda$ and reads
\begin{align}
    f_0^{\mathrm{BE}} = \frac{\langle p f \rangle}{\langle p \rangle} = 0.1106.
\end{align}
For a thermal system we naturally have $\delta=1$, and we take the scale $\Q$ in \eqref{eq:f_params} to be $\Q=T$. 

The screening mass $\md$ is given by \eqref{eq:mDdefinition}. Performing the integral for the distribution \eqref{eq:f_params}, 
we get 
\begin{align}
   \md^2 \sim \delta  f_0 \lambda  \Q^2.
\end{align}
Thus, the ratio to the thermal case becomes
\begin{equation}
    \frac{\md}{\md^{eq}} = \sqrt{\delta \frac{f_0}{f_0^{\mathrm{BE}}}}\,.
    \label{eq:mdparam}
\end{equation}
This parametric estimate is shown as a gray dotted line in \fig \ref{fig:scalesVsEquil3} (top)  for $\lambda=1$ (as indicated by the color coding) and $\xi=10$. We will use these parameters throughout this section for other quantities as well. The time-evolution in \fig \ref{fig:scalesVsEquil3} can now be understood using \fig \ref{fig:occupVsAnisotropy}. The decrease in the value of $\md$ when the system evolves from overoccupation to underoccupation is mainly driven by the falling occupation number $f_0$. As the system evolves towards thermal equilibrium from the underoccupation regime (circle marker), the value of $\md$ starts to increase. This process is driven by both the growing occupation number and decreasing anisotropy. As can be seen in \fig \ref{fig:scalesVsEquil3}, this estimate does not describe the evolution quantitatively, but offers a simple qualitative description of the evolution. Possibly our parametric estimate undershoots the actual value of $\md$ in the underoccupied phase (the circle marker) because the modes contributing to the Debye mass are softer and not quite as anisotropic and under-occupied as the modes contributing to the pressure, leading to the factor $\sqrt{\delta f_0}$ in \eqref{eq:mdparam} to overestimate the effect. 



Similar estimates can be applied to $\tstar$ given by \eqref{eq:Tstardef}. Using the result for $\md$, we obtain
\begin{equation}
    \tstar \sim \Q \left(f_{0} + 1\right)\,.
\end{equation}
Thus $\tstar$ is less sensitive to the anisotropy.  Comparing with the thermal estimate, we get 
\begin{equation}
    \frac{\tstar}{\tstar^{eq}} =  \frac{\left( f_0 +1\right)}{\left(  f_0^{\mathrm{BE}} + 1 \right)}.
\end{equation}
This result naturally explains the behavior of $\tstar$ observed in \fig \ref{fig:scalesVsEquil3} (bottom). The  initial enhancement of $\tstar$ is driven by the large occupation number, and the observed enhancement is larger than for $\md$ since $\tstar$ has a stronger dependence on the occupation number.
Due to the constant term, there is no suppression from underoccupation, and $\tstar$ is less sensitive to $\delta$ than $\md$. Hence, our parametric estimate is better than for $\md$, and $\tstar$ does not go below the thermal value  in the underoccupied regime, but is  already close to unity.



Using the estimates  for $\md$ and $\tstar$, we obtain
\begin{align}
\label{eq:kappaguesstimate}
    \frac{\kappa}{\kappa_{eq}^\epsilon} =  \frac{\delta f_{0}}{f_0^{\mathrm{BE}}} \frac{\left( f_{0} + 1\right)}{\left( f_0^{\mathrm{BE}} + 1 \right)}\,.
\end{align}
Plugging in the values for $f_0$ and $\delta$ during the evolution  (calculated from $f_0$ and $\delta$ in  \figs \ref{fig:occupVsAnisotropy} and \ref{fig:scalesVsEquil1}) leads to the gray dotted curve in \fig  \ref{fig:parametricGuesstimate}. We observe that the general trend of enhancement followed by suppression and equilibration is even somewhat exaggerated by this parametric estimate. 
The transparent curves in \fig  \ref{fig:parametricGuesstimate} show the estimate obtained using \eq\nr{eq:kappaRatioPocketFormula} with the actual calculated values of $\md$ and $\tstar$ from \fig\ref{fig:scalesVsEquil3}, exhibiting the same behavior in a less exaggerated way.



\FloatBarrier

\section{Conclusions}
\label{sec:conc}
In this paper we have extracted the heavy quark momentum diffusion coefficient using effective kinetic theory simulations during the bottom-up isotropization process. In order to better quantify  the importance of the pre-equilibrium evolution in relation to a thermal equilibrium, we have displayed our result as  the ratio  $\nicefrac{\kappa}{\kappa^{eq}}$ to a thermal distribution with the same energy density $\varepsilon$, Debye mass $\md$, or effective infrared temperature $\tstar$. Our main conclusion is that during the kinetic stages of the pre-equilibrium evolution that can be described using effective kinetic theory, i.e., after the initial glasma evolution, the diffusion coefficient is within 20\% of the equilibrium value in a thermal system with  the same energy density. In the early overoccupied stage $\kappa$ is somewhat larger than the equilibrium value, driven by the enhancement of both $\md$ and $\tstar$. In the later underoccupied phase, $\kappa$ is below the thermal comparison value due to the suppression of $\md$.
We have tested initial conditions with two different anisotropies, and we have found that our conclusions are not strongly affected by the value of the initial anisotropy parameter. We expect this  information to be important for pre-equilibrium transport modelling and modelling of quarkonia. 

We have also studied the diffusion coefficient separately for the transverse and longitudinal (beam) direction. The transverse diffusion coefficient is larger than the longitudinal one at the very early stages when the occupation numbers are large, due to the Bose enhancement available for scaterings with a momentum exchange in the transverse direction. When the system becomes underoccupied, the longitudinal diffusion coefficient dominates. 
When the system approaches thermal equilibrium, the transverse and longitudinal diffusion coefficients become equal as expected. 
In a similar manner, we have qualitatively explained the evolution of $\md$, $\tstar$ and $\kappa$ using the behavior of typical hard occupation numbers and momentum anisotropy. 



It would be interesting to study the phenomenological implications of these results in more detail. One exciting perspective is to include them into the quantum trajectories framework \cite{Brambilla:2020qwo,Brambilla:2022ynh} to study the impact of initial stages on quarkonium dynamics via $\kappa_i(\tau)$. It is also possible to address the  evolution of other transport coefficients during the initial non-equilibrium dynamics: in a separate paper we use the EKT setup to calculate the evolution of the jet quanching parameter $\hat{q}$. These studies  open the possibility of assessing the impact of initial stages on other phenomenological observables.



\begin{acknowledgments}

The authors would like to thank N.~Brambilla, M.~Escobedo, D.I.~Müller, A.~Rothkopf and M.~Strickland for valuable discussions. 
This work is supported  by the European Research Council, ERC-2018-ADG-835105 YoctoLHC.  This work was also supported under the European Union’s
Horizon 2020 research and innovation  by the STRONG-2020 project (grant agreement No. 824093). The content of this article does not reflect the official opinion of the European Union and responsibility for the information and views expressed therein lies entirely with the authors.  This work was funded in part by the Knut and Alice Wallenberg foundation, contract number 2017.0036. TL and JP have been supported by the Academy of Finland, by the Centre of Excellence in Quark Matter (project 346324) and project 321840.
KB and FL would like to thank the Austrian Science Fund (FWF) for support under project P 34455, and FL is additionally supported by the Doctoral Program W1252-N27 Particles and Interactions.
The authors wish to acknowledge CSC – IT Center for Science, Finland, for computational resources.  We acknowledge grants of computer capacity from the Finnish Grid and Cloud Infrastructure (persistent identifier urn:nbn:fi:research-infras-2016072533 ).
 The authors wish to acknowledge the Vienna Scientific
Cluster (VSC) project 71444 for computational resources.

\end{acknowledgments}







\appendix





\section{Numerical framework}
\label{app:numerics}

\subsection{Details of the discretization procedure}

Our numerical framework is the same as in \cite{Kurkela:2015qoa}. Instead of the distribution function, we use the 
quantity 
\begin{equation}
\nbar_{ij} = \int \derthree \lambda f(\bs{p}) w_{ij}(\bs{p}),
\end{equation}
which represents the particle number per degree of freedom and unit (spatial) volume,
as our dynamical degree of freedom.
Here the $w_{ij}$ involve the discretization in momentum $p \equiv |\bs p|$ and polar angle $\cos \theta$ as
\begin{equation}
w_{ij}\left( \bs{p} \right) = w_j(p) w_i(\cos \theta),
\end{equation}
with the wedge functions defined as
\begin{align}
w_i(z) &= \dfrac{z_{i+1}-z}{z_{i+1} - z_i} \theta(z-z_i) \theta(z_{i+1} - |z|) \nn \\
& + \dfrac{z-z_{i-1}}{z_{i} - z_{i-1}} \theta(z_i-z) \theta(z - z_i).
\end{align}
 This discretization conserves particle number, energy density and $\langle \pz \rangle$ exactly. 

Our numerical framework has in total seven discretization parameters. Two of them correspond to the number of bins used in the momentum $\in [\pmin,\pmax]$
and polar angle. There are two parameters associated to the Monte Carlo sampling procedure, one of which determines the number of samples that is needed for the time-evolution, the other one being associated to the sampling of the diffusion coefficient. For the momentum discretization we have parameters which are associated to the minimum and maximum momentum in our momentum grid $\pmin$ and $\pmax$. Finally for the time-evolution we use an adaptive step-size algorithm, which needs an initial value. 

We have verified that our numerical results are independent of the discretization parameter set by reproducing the results of \cite{Kurkela:2015qoa}. The most important dependencies on the discretization parameters are the following: Reproducing the correct behavior during the underoccupied region is very sensitive to the length of the time-step. It turns out that the minimum momentum on the grid is important to understand the approach to equilibrium -- as the system becomes thermal, it will approach a thermal distribution with an infrared cutoff $\pmin$. When we compare our simulations to thermal equilibrium, this has to be taken into account. This effect is described in detail in Appendix~\ref{app:pminDependence}. In principle we receive corrections also from the maximum momentum parameter $\pmax$ (and other discretization parameters). However, in practice the dependence on $\pmin$ appears to be by far the most important effect. We have also checked that our results do not depend on the accuracy of the discretization of the $\theta$ angle. 



\subsection{Discretization of $\kappa$}
\label{se:DiscreteKappa}



Due to  cylindrical symmetry around the beam direction, the distribution function $f(\bs{p})$ out of equilibrium is assumed to only depend on the magnitude of the momentum and the polar angle $\theta$, and not on the azimuthal angle $\phi$. We start by evaluating the $\bs{p}^\prime$ integral in \eqref{eq:kappa_master_formula} by making use of the momentum conserving delta function. Furthermore we integrate over the radial component of $\bs{k}^\prime$ using the energy conserving delta function. This leads to
\begin{align}
3 \kappa &= \frac{\lambda^2 C_H}{N_c}  \int_{0}^{\infty} \dfrac{\der k}{(2 \pi)^3}\,k^2 \int_{-1}^{1}\der x_k \int_{0}^{2\pi}\der \phi_k \nn \\ 
&\times \int_{-1}^{1} \dfrac{\der x_k^\prime}{(2 \pi)^2}\int_{0}^{2\pi} \der \phi_k^\prime\,k^2 q^2\,\dfrac{1+ \cos^2 \theta_{\bs{k} \bs{k^\prime} }}{\left(q^2 + m_D^2 \right)^2} \nn \\ 
& \times f(k,x_k) \left(1+f(k,x_k^\prime)\right),
\end{align}
 where we use the notation $x_k = \cos \theta_k$, $x_k^\prime = \cos \theta_{k^\prime}$ for the polar angles. The azimuthal angles are denoted by $\phi_k$ and $\phi_k^\prime$. The cosine of the angle between $\bs{k}$ and $\bs{k}^\prime$ can be written as 
\begin{align}
 \cos \theta_{\bs{k} \bs{k^\prime} } &= 1 - \frac{q^2}{2k^2} \\
 \label{eq:azimuthdependence}
 &= \sin \theta_k \sin \theta_k^\prime \cos(\phi_k - \phi_k^\prime) + x_k x_k^\prime.
\end{align}
As can be seen from \eqref{eq:azimuthdependence}, the expression depends only on the difference of the azimuthal angles. Thus we can change the variables by $\int_{0}^{2\pi} \der \phi_k \int_{0}^{2\pi} \der \phi_k^\prime = \int_{0}^{2\pi} \der \phi_k^\prime \int_{0}^{2\pi} \der \phi_{\bs{k} \bs{k^\prime}} = 2\pi \int_{0}^{2\pi} \der \phi_{\bs{k} \bs{k^\prime}}$ and trivially carry out one of the integrals over the azimuthal angles. 
Further evaluating the integral and using the results above yields
\begin{align}
\label{eq:kappaDiscretized}
 &3 \kappa = \frac{\lambda^2}{ (2 \pi)^4}\,\,\frac{C_H}{N_c}\int_{\Pmin}^{\Pmax}\der k\,k^4 \int_{-1}^{1}\der x_k \int_{-1}^{1} \der x_k^\prime \nn \\
 &f(k,x_k) \left(1+f(k,x_k^\prime)\right) \int_{0}^{2\pi}\der\phi_{\bs{k} \bs{k^\prime}}\, q^2\, \frac{1 + (1 - q^2/(2k^2))^2}{\left(q^2 + m_D^2 \right)^2}.
\end{align}
 The momentum transfer $q^2$ can be written in terms of the integration variables as 
\begin{align}
 q^2 &= 2k^2 \left( 1 - \sqrt{1-x_k^2} \sqrt{1-(x_k^\prime)^2} \cos(\phi_{\bs{k} \bs{k^\prime}}) - x_k x_k^\prime \right).
\end{align}
We also want to study the transverse and longitudinal diffusion coefficients separately. 
These  are given by 
\begin{align}
\label{eq:kappaDiscretizedTz}
 &\kappa_i = \frac{\lambda^2}{ (2 \pi)^4}\,\,\frac{C_H}{N_c}\int_{\Pmin}^{\Pmax}\der k\,k^4 \int_{-1}^{1}\der x_k \int_{-1}^{1} \der x_k^\prime \nn \\
 &f(k,x_k) \left(1+f(k,x_k^\prime)\right) \int_{0}^{2\pi}\der\phi_{\bs{k} \bs{k^\prime}}\, q_i^2\, \frac{1 + (1 - q^2/(2k^2))^2}{\left(q^2 + m_D^2 \right)^2}.
\end{align}
 Here $q_i = q_T,q_z,$ is the momentum transfer in different directions, given by
\begin{align}
\dfrac{q_z^2}{k^2} &= \left(x_k - x_{k^\prime} \right)^2 \\
\dfrac{q_T^2}{k^2} & =  \left[ 1- \dfrac{x_k^2}{2} - \dfrac{x_{k^\prime}^2}{2} - \sqrt{\left(1- x_k^2\right) \left(1- x_{k^\prime}^2 \right)} \cos \phi_{k k^\prime}\right].
\end{align}
The normalization is such that $q^2 = 2 q_T^2 + q_z^2$, and consequently 
$ 3 \kappa = 2 \kappa_T + \kappa_z$. For a thermalized system this corresponds to $\nicefrac{\kappa_{T}}{\kappa_{z}} \to 1$ and $\nicefrac{\kappa_{T,z}}{\kappa} \to 1$.  In our numerical framework the integrals are computed using Monte-Carlo techniques. In \eqref{eq:kappaDiscretized} and \eqref{eq:kappaDiscretizedTz} we have also discretized the momentum interval with IR and UV cutoffs $[\Pmin,\Pmax]$. The dominating discretization effects are discussed in more detail in  Appendix~\ref{app:pminDependence}.


\begin{figure}
\includegraphics[scale=0.5]{Plots/mDSqrdPminCorrectionsBMSS.pdf}
\caption{Debye mass extracted during the bottom-up evolution, divided by the value in a \emph{continuum} thermal system with the same energy density. The dashed curve has been obtained by computing the screening mass in a thermal system with the IR regulator $\pmin$ given by \eqref{eq:mDpmin}.
The temperature $T_\varepsilon$ is extracted from the energy density using \eq\eqref{eq:sensibleTemperature}. 
The main observation is that the IR cutoff $\pmin$ decreases $\md$ by roughly 20\% from the continuum value, depending on $T/\pmin$.
}
\label{fig:mDdiscretizationEffect}
\end{figure}


\subsection{Observables and $\pmin$ dependence}
\label{app:pminDependence}

Due to the discretization effects, our system does not reach the continuum thermal equilibrium. Instead, it approaches a discretized version of thermal equilibrium. This affects our observables, as illustrated in \fig \ref{fig:mDdiscretizationEffect}. 
We observe that the numerically obtained value for $\md$ divided by its thermal expectation (obtained for a gluonic system with a Bose-Einstein distribution) deviates from unity at late times. However, when we take into account the discretization effects in terms of the momentum cutoff $\pmin$, we see a nice agreement for  $\lambda=10$. A similar agreement is observed for other curves as well, but here we show only one for clarity. In principle we could take into account also other discretization parameters such as $\pmax$ and the angular discretization, but we find that the dominant discretization contribution arises from $\pmin$.

We will now discuss the $\pmin$ dependence in $\md$, $\tstar$ and $\kappa$ in more detail.


\subsubsection{Debye mass $\md$}
\label{sec:mdpmin}

The $\pmin$ dependence in $\md$ given by \eqref{eq:mDdefinition} is obtained by inserting an IR cutoff into the integral  as follows
\begin{align}
\label{eq:mDpmin}
&m_D^2(\pmin) = \dfrac{8 \lambda}{\left(2 \pi\right)^2} \int_{\pmin}^\infty \der p p f(p) \\
& \quad = \frac{2 \lambda  T \left(T \mathrm{Li}_2\left(e^{-\frac{\pmin}{T}}\right)-\pmin \log
   \left(1-e^{-\frac{\pmin}{T}}\right)\right)}{\pi ^2}, \nn
\end{align}
where the distribution is given by the Bose-Einstein distribution. The dilogarithm is defined as 
\begin{align}
\mathrm{Li}_2(z) = \sum_{k=1}^\infty \dfrac{z^k}{k^2} = - \int_0^z \dfrac{\ln(1-u)}{u}  \der u.
\label{eq:dilogdef}
\end{align}

The curve labeled as thermal in \fig \ref{fig:mDdiscretizationEffect} is obtained using \eqref{eq:mDpmin}.

\subsubsection{Effective IR temperature $\tstar$}
\label{sec:tstarpmin}
The same procedure can be carried out for $\tstar$
\begin{align}
\lambda \tstar(t,\pmin) =  \dfrac{\lambda^2}{\pi^2  \md^2(\pmin)} \int_{\pmin}^\infty \der p p^2  f(t,p) \left(1 +f(t,p) \right).
\end{align}
For the thermal distribution the result becomes 
\begin{align}
 \label{eq:tstarpminexpression}
& \lambda \tstar(\pmin)  = \frac{\lambda }{6} \Bigg[ 6 T^2 \mathrm{Li}_2\left(e^{-\frac{\pmin}{T}}\right) \nn \\ 
&+ 3 \pmin \Bigg(\pmin
   \left(\frac{1}{e^{\pmin/T}-1}+2\right)  -2 T \log \left(e^{\pmin/T}-1\right)\Bigg)\Bigg] \nn \\ 
   & \times \Bigg[ \Big(T
   \mathrm{Li}_2\left(e^{-\frac{\pmin}{T}}\right) -\pmin \log \left(1-e^{-\frac{\pmin}{T}}\right)\Big)\Bigg]^{-1}.
 \end{align}
Expanding this for small $x = \dfrac{\pmin}{T}$ yields
 \begin{align}
 \lambda \tstar(\pmin)  =  \lambda T \dfrac{12 \pi -36x + x^3  }{12 \pi - 72 x + 18 x^2 - 2 x^3    } + \mathcal{O}(x^5).
    \end{align}
 Thus in the limit $x \to 0$ we recover the equilibrium relation $\lambda \tstar = \lambda T$.

 In the comparisons to equilibrium distributions in the main text (e.g. in \fig\ref{fig:scalesVsEquil3})
 these corrections are applied to $\md$ and $\tstar$. Consequently, the ratios approach unity when the system thermalizes, contrary to the behavior observed in \fig \ref{fig:mDdiscretizationEffect} without the corrections.

 
\subsubsection{Diffusion coefficient $\kappa$}
\label{sec:kappapmin}

 For $\kappa$ the corrections arising from the infrared regulator $\pmin$ are taken into account as follows. We start with the expression \eqref{eq:kappaLO} and perform the $q$ integral analytically without an infrared regulator as before. Then we replace $\md$ with the infrared regulated expression  $\md(\pmin)$ given by \eqref{eq:mDpmin}. 
 The resulting expression is 
 \begin{widetext}
\begin{align}
\kappa_{LO}^{therm} &= \int_{0}^{\infty}  \frac{\der k}{{48 \pi ^3 \text{Nc} k^2
   \left(e^{k/T}-1\right)^2}}  C_H \lambda ^2 e^{k/T} \Bigg[-\Bigg[4 k^2 \left(\frac{\lambda  T \left(T
   \text{Li}_2\left(e^{-\frac{\pmin}{T}}\right)-\pmin \log \left(1-e^{-\frac{\pmin}{T}}\right)\right)}{\pi ^2}+k^2\right) \nn \\
   & \times
   \left(\frac{6 \lambda  T \left(T \text{Li}_2\left(e^{-\frac{\pmin}{T}}\right)-\pmin \log
   \left(1-e^{-\frac{\pmin}{T}}\right)\right)}{\pi ^2}+8 k^2\right)\Bigg] \nn \\
   &\times \Big[\frac{2 \lambda  T \left(T   \text{Li}_2\left(e^{-\frac{\pmin}{T}}\right)-\pmin \log \left(1-e^{-\frac{\pmin}{T}}\right)\right)}{\pi ^2}+4
   k^2\Big]^{-1} \nn \\
   &-\frac{1}{2} \log \left(\frac{\lambda  T \left(T \text{Li}_2\left(e^{-\frac{\pmin}{T}}\right)-\pmin \log
   \left(1-e^{-\frac{\pmin}{T}}\right)\right)}{\lambda  T^2 \text{Li}_2\left(e^{-\frac{\pmin}{T}}\right)-\lambda  \pmin T
   \log \left(1-e^{-\frac{\pmin}{T}}\right)+2 \pi ^2 k^2}\right)  \nn \\
    & \times \left(\frac{16 \lambda  k^2 T \left(T
   \text{Li}_2\left(e^{-\frac{\pmin}{T}}\right)-\pmin \log \left(1-e^{-\frac{\pmin}{T}}\right)\right)}{\pi ^2}+\frac{12
   \lambda ^2 T^2 \left(\pmin \log \left(1-e^{-\frac{\pmin}{T}}\right)-T
   \text{Li}_2\left(e^{-\frac{\pmin}{T}}\right)\right){}^2}{\pi ^4}+8 k^4\right)\Bigg]
   \label{eq:EpicKappa}
\end{align}
\end{widetext}

 The remaining $k$ integral in \eqref{eq:EpicKappa} is  evaluated numerically for different values of $\lambda$ and $T$ using a small independent IR regulator for numerical stability. The results are interpolated in order to find a $\kappa$ corresponding to an arbitrary temperature within the tabulated range. We have applied these corrections to the results shown in \fig \ref{fig:kappa_vs_kappaeq}. We observe that the ratio to equilibrium is very close to unity when the system is approximately thermal.
 
 We would like to emphasize that this treatment takes only into account the discretization effects arising from $m_D$. We do not regulate the $k$ and $q$ integrals with the regulator $\pmin$. Thus this treatment takes only a subset of the corrections into account.  Hence this treatment is not expected to maintain the ratio at unity for infinitely large times. This can be seen for instance in \figs \ref{fig:kappa_vs_kappaeq} and even more prominently  \ref{fig:parametricGuesstimate}, where the ratio starts to slightly deviate from unity at very late times. 

The ratios $\nicefrac{\kappa}{\kappa_{eq}^{\md, \tstar}}$ comparing to the thermal system  at the same $\md$ and $\tstar$ are obtained using the same data. The temperature $T$ and $\pmin$ uniquely determine $\md$ given by \eqref{eq:mDpmin} and $\tstar$ given by \eqref{eq:tstarpminexpression}. 


\bibliographystyle{JHEP-2modlong}
\bibliography{spires}
\end{document}
