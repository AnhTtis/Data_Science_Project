\section{Experiments}
\begin{figure*}[t]
        \centering
        \begin{minipage}[t!]{0.235\textwidth}
            \centering 
            \includegraphics[width=\textwidth]{supplementary_material/figures/oa_env_final.png}
        \end{minipage}
        \hfill
        \begin{minipage}[t!]{0.235\textwidth}
            \centering 
            \includegraphics[width=\textwidth]{supplementary_material/figures/hbp_env_final.png}
        \end{minipage}
         \hfill
        \centering
        \begin{minipage}[t!]{0.35\textwidth}
            \centering 
            \includegraphics[width=\textwidth]{supplementary_material/figures/Fig_MT_table_tennis_env.pdf}
        \end{minipage}
         \hfill
        \caption[ ]
        {\small \textbf{Behavior learning environments:} Visualization of the obstacle avoidance task (left), the block pushing task (middle), and the table tennis task (right).}
        \label{fig:environments}
    \end{figure*}

%%%%%%%%%%%%%%%%%%%%%%%%%%%%%%%%%%%%%%%%%%%%%%%%%%%%%%%%%%%%%%%%%%%%%%%
\begin{figure*}[t]
        \centering
        \begin{minipage}[t!]{0.105\textwidth}
            \centering
            \includegraphics[width=\textwidth]{results/obs_avoid/obsavoid_gt.pdf}
        \end{minipage}
        \hfill
         \begin{minipage}[t!]{0.105\textwidth}
            \centering
            \includegraphics[width=\textwidth]{results/obs_avoid/obsavoid_mdn.pdf}
        \end{minipage}
        \hfill
            \begin{minipage}[t!]{0.105\textwidth}
            \centering
            \includegraphics[width=\textwidth]{results/obs_avoid/obsavoid_em.pdf}
        \end{minipage}
        \hfill
        \begin{minipage}[t!]{0.105\textwidth}
            \centering 
            \includegraphics[width=\textwidth]{results/obs_avoid/obsavoid_diffusion.pdf}
        \end{minipage}
        \hfill
            \begin{minipage}[t!]{0.105\textwidth}
            \centering
            \includegraphics[width=\textwidth]{results/obs_avoid/obsavoid_nf.pdf}
        \end{minipage}
        \hfill
            \begin{minipage}[t!]{0.105\textwidth}
            \centering
            \includegraphics[width=\textwidth]{results/obs_avoid/obsavoid_vae.pdf}
        \end{minipage}
        \hfill
            \begin{minipage}[t!]{0.105\textwidth}
            \centering
            \includegraphics[width=\textwidth]{results/obs_avoid/obsavoid_ibc.pdf}
        \end{minipage}
        \hfill
            \begin{minipage}[t!]{0.105\textwidth}
            \centering
            \includegraphics[width=\textwidth]{results/obs_avoid/obsavoid_bet.pdf}
        \end{minipage}
        \hfill
        \begin{minipage}[t!]{0.105\textwidth}
            \centering 
            \includegraphics[width=\textwidth]{results/obs_avoid/obsavoid_imc.pdf}
        \end{minipage}
         \hfill
        \caption[ ]
        {\small \textbf{Obstacle Avoidance:} Visualization of $100$ end-effector trajectories for all trained models.}
        \label{fig:planar_reacher_vis}
    \end{figure*}
%%%%%%%%%%%%%%%%%%%%%%%%%%%%%%%%%%%%%%%%%%%%%%%%%%%%%%%%%%%%%%%%%%%%%%%
We evaluate IMC on challenging conditional density estimation tasks. To that end, we focus on behavior learning tasks with human-collected data. The inherent versatility in human behavior introduces high variability and outliers to data, inducing complex multimodal distributions. Hence, our experiments assess the ability of models to \textit{i)} avoid mode averaging and \textit{ii)} cover all modes present in the data distribution.  
% [Mention the intention of the experiments (multimodality, performance)]
% [Maybe explain all four experiments and their intentions]
% \begin{itemize}
%     \item Obstacle avoidance. Highly multimodal
%     \item Block Pushing. Multimodal complex data distribution 
%     \item Franka kitchen. Multimodality and modeling optimizing for long-term prediction?
%     \item Table Tennis. Although not recorded by humans. RL agent learns versatile skills
% \end{itemize}
 For all experiments, we employ conditional Gaussian experts, i.e., $p_{\vtheta_o}(\mathbf{y}|\mathbf{x},o) = \mathcal{N}(\mathbf{y}|\vmu_{\vtheta_o}(\mathbf{x}), \sigma^2\textbf{I})$ due to efficient sample routines and the simplicity of the resulting optimization problem. 
Please note that we parameterize the expert means $\vmu_{\vtheta_o}$ as (shallow) neural networks. 
% for linear $\vmu_{\vtheta_o}$ Equation \ref{eq:expert_update} can be solved in closed form using weighted least squares \cite{murphy2012machine}. For non-linear models such as neural networks, the optimization is straightforward using stochastic gradient updates. 
Moreover, we use a fixed variance of $\sigma^2=1$ for all experiments as full Gaussian likelihood optimization often results in unstable updates \cite{guo2017calibration}.

We compare our method to state-of-the-art generative models including denoising diffusion probabilistic models (DDPM) \cite{ho2020denoising}, normalizing flows (NF) \cite{papamakarios2021normalizing} and conditional variational autoencoders (CVAE). Moreover, we consider energy-based models for behavior learning (IBC) \cite{florence2022implicit} and the recently proposed behavior transformer (BeT) \cite{shafiullah2022behavior}. Lastly, we compare against mixture of experts trained using expectation maximization (EM) \cite{jacobs1991adaptive} and backpropagation (MDN) \cite{bishop1994mixture}. We extensively tune the hyperparameters of the baselines using Bayesian optimization \cite{snoek2012practical} on all experiments. We report the mean and the standard deviation over ten random seeds for all experiments. For a detailed explanation of tasks, data, performance metrics and hyperparameters see Appendix \ref{appendix:experiment_setup}.
% \begin{itemize}
%     \item Briefly mention structure of experiments: We introduce the environment with goals. Then the data. Then the metrics. 
% \end{itemize}
%%%%%%%%%%%%%%%%%%%%%%%%%%%%%%%%%%%%%%%%%%%%%%%%%%%%%%%%%%%%%%%%%%%%%%%
%%%%%%%%%%%%%%%%%%%%%%%%%%%%%%%%%%%%%%%%%%%%%%%%%%%%%%%%%%%%%%%%%%%%%%%
\begin{table*}[t!]
\caption{\textbf{Performance Table}: We compare IMC against strong baselines on three complex robot manipulation tasks. The best results use bold formatting. $\uparrow$ indicates that higher and $\downarrow$ that lower values are better. For further details, please refer to the accompanying text.}
\label{table:result_table}
\vskip 0.15in
\begin{center}
\begin{small}
\begin{sc}
\resizebox{\textwidth}{!}{%
\begin{tabular}{lcc|ccc|cc}
\toprule
\rule{0pt}{2ex} &
\multicolumn{2}{c}{\textbf{Obstacle Avoidance}} &
\multicolumn{3}{c}{\textbf{Block Pushing}} &
\multicolumn{2}{c}{\textbf{Table Tennis}}  \\
 & 
success rate ($\uparrow$) & Entropy ($\uparrow$) & 
success rate ($\uparrow$) & Entropy ($\uparrow$) & Distance Error ($\downarrow$) & 
success rate ($\uparrow$) & Distance Error ($\downarrow$)\\
\midrule
%
MDN & 
% ------------------------------------ %
$0.200 \scriptstyle{\pm 0.421}$ &
$0.000 \scriptstyle{\pm 0.000}$ &
% ------------------------------------ %
$0.000 \scriptstyle{\pm 0.000}$ &
$0.000 \scriptstyle{\pm 0.000}$ &
$0.360 \scriptstyle{\pm 0.005}$ &
% ------------------------------------ %
$0.031 \scriptstyle{\pm 0.013}$ &
$0.549 \scriptstyle{\pm 0.056}$ \\
%
EM & 
% ------------------------------------ %
$0.675 \scriptstyle{\pm 0.033}$ &
$0.902 \scriptstyle{\pm 0.035}$ &
% ------------------------------------ %
$0.458 \scriptstyle{\pm 0.048}$ &
$0.707 \scriptstyle{\pm 0.042}$ &
$0.127 \scriptstyle{\pm 0.011}$ &
% ------------------------------------ %
$0.725 \scriptstyle{\pm 0.042}$ &
$0.220 \scriptstyle{\pm 0.012}$ \\
%
DDPM & 
% ------------------------------------ %
$0.719 \scriptstyle{\pm 0.075}$ &
$0.638 \scriptstyle{\pm 0.079}$ &
% ------------------------------------ %
$0.516 \scriptstyle{\pm 0.036}$ &
$0.713 \scriptstyle{\pm 0.043}$ &
$0.123 \scriptstyle{\pm 0.007}$ &
% ------------------------------------ %
$0.866 \scriptstyle{\pm 0.010}$ &
$0.185 \scriptstyle{\pm 0.007}$ \\
%
NF & 

% ------------------------------------ %
$0.313 \scriptstyle{\pm 0.245}$ &
$0.349 \scriptstyle{\pm 0.208}$ &
% ------------------------------------ %
$0.001 \scriptstyle{\pm 0.001}$ &
$0.000 \scriptstyle{\pm 0.000}$ &
$0.346 \scriptstyle{\pm 0.034}$ &
% ------------------------------------ %
$0.422 \scriptstyle{\pm 0.035}$ &
$0.371 \scriptstyle{\pm 0.013}$ \\
%
CVAE & 
% ------------------------------------ %
$0.853 \scriptstyle{\pm 0.113}$ &
$0.465 \scriptstyle{\pm 0.183}$ &
% ------------------------------------ %
$0.505 \scriptstyle{\pm 0.089 }$ &
$0.162 \scriptstyle{\pm 0.071}$  &
$0.123 \scriptstyle{\pm 0.027 }$ &
% ------------------------------------ %
$0.620 \scriptstyle{\pm 0.050}$ &
$0.320 \scriptstyle{\pm 0.010}$ \\
%
IBC & 
% ------------------------------------ %
$0.379 \scriptstyle{\pm 0.411}$ &
$0.098 \scriptstyle{\pm 0.131}$ &
% ------------------------------------ %
$0.482 \scriptstyle{\pm 0.026 }$ & 
$\mathbf{0.758 \scriptstyle{\pm 0.022 }}$ &
$0.123 \scriptstyle{\pm 0.007 }$ &
% ------------------------------------ %
$0.567 \scriptstyle{\pm 0.030}$ &
$0.310 \scriptstyle{\pm 0.010}$ \\
%
BET & 
% ------------------------------------ %
$0.504 \scriptstyle{\pm 0.076}$ &
$0.837 \scriptstyle{\pm 0.066}$ &
% ------------------------------------ %
$ 0.374\scriptstyle{\pm 0.041}$ &
$ 0.607\scriptstyle{\pm 0.037}$ &
$ 0.151\scriptstyle{\pm 0.008}$ &
% ------------------------------------ %
$0.758 \scriptstyle{\pm 0.025}$ &
$0.235 \scriptstyle{\pm 0.011}$ \\
%
ML-Cur & 
% ------------------------------------ %
$0.454 \scriptstyle{\pm 0.223}$ &
$0.035 \scriptstyle{\pm 0.024}$ &
% ------------------------------------ %
$0.000 \scriptstyle{\pm 0.000}$ &
$0.000 \scriptstyle{\pm 0.000}$ &
$0.408 \scriptstyle{\pm 0.030}$ &
% ------------------------------------ %
$0.836 \scriptstyle{\pm 0.020}$ &
$0.181 \scriptstyle{\pm 0.011}$ \\
%
IMC & 
% ------------------------------------ %
$\mathbf{0.855 \scriptstyle{\pm 0.053}}$ &
$\mathbf{0.930 \scriptstyle{\pm 0.031}}$ &
% ------------------------------------ %
$ \mathbf{0.521\scriptstyle{\pm 0.045}}$ &
$ 0.654\scriptstyle{\pm 0.041}$ &
$ \mathbf{0.120\scriptstyle{\pm 0.014}}$ &
% ------------------------------------ %
$\mathbf{0.870 \scriptstyle{\pm 0.017}}$ &
$\mathbf{0.153 \scriptstyle{\pm 0.007}}$ \\
\bottomrule
\end{tabular}
}
\end{sc}
\end{small}
\end{center}
\vskip -0.1in
\end{table*}
%%%%%%%%%%%%%%%%%%%%%%%%%%%%%%%%%%%%%%%%%%%%%%%%%%%%%%%%%%%%%%%%%%%%%%%


% %%%%%%%%%%%%%%%%%%%%%%%%%%%%%%%%%%%%%%%%%%%%%%%%%%%%%%%%%%%%%%%%%%%%%%%
% \begin{table*}[t!]
% \caption{\textbf{Performance Table}: We compare IMC against strong baselines on three complex robot manipulation tasks. The results are highlighted in bold. For further details, please refer to the accompanying text.}
% \label{table:result_table}
% \vskip 0.15in
% \begin{center}
% \begin{small}
% \begin{sc}
% \resizebox{\textwidth}{!}{%
% \begin{tabular}{lccc|cc}
% \toprule
% \rule{0pt}{2ex} &
% \multicolumn{3}{c}{\textbf{Block Pushing}} &
% \multicolumn{2}{c}{\textbf{Table Tennis}}  \\
%  & 
% success rate ($\uparrow$) & Entropy ($\uparrow$) & Distance Error ($\downarrow$) & 
% success rate ($\uparrow$) & Distance Error ($\downarrow$)\\
% \midrule
% %
% MDN & 
% % ------------------------------------ %
% % ------------------------------------ %
% $0.000 \scriptstyle{\pm 0.000}$ &
% $0.000 \scriptstyle{\pm 0.000}$ &
% $0.360 \scriptstyle{\pm 0.005}$ &
% % ------------------------------------ %
% $0.031 \scriptstyle{\pm 0.013}$ &
% $0.549 \scriptstyle{\pm 0.056}$ \\
% %
% EM & 
% % ------------------------------------ %
% % ------------------------------------ %
% $0.458 \scriptstyle{\pm 0.048}$ &
% $0.707 \scriptstyle{\pm 0.042}$ &
% $0.127 \scriptstyle{\pm 0.011}$ &
% % ------------------------------------ %
% $0.725 \scriptstyle{\pm 0.042}$ &
% $0.220 \scriptstyle{\pm 0.012}$ \\
% %
% DDPM & 
% % ------------------------------------ %
% % ------------------------------------ %
% $0.516 \scriptstyle{\pm 0.036}$ &
% $0.713 \scriptstyle{\pm 0.043}$ &
% $0.123 \scriptstyle{\pm 0.007}$ &
% % ------------------------------------ %
% $0.866 \scriptstyle{\pm 0.010}$ &
% $0.185 \scriptstyle{\pm 0.007}$ \\
% %
% NF & 

% % ------------------------------------ %
% % ------------------------------------ %
% $0.001 \scriptstyle{\pm 0.001}$ &
% $0.000 \scriptstyle{\pm 0.000}$ &
% $0.346 \scriptstyle{\pm 0.034}$ &
% % ------------------------------------ %
% $0.422 \scriptstyle{\pm 0.035}$ &
% $0.371 \scriptstyle{\pm 0.013}$ \\
% %
% CVAE & 
% % ------------------------------------ %
% % ------------------------------------ %
% $0.505 \scriptstyle{\pm 0.089 }$ &
% $0.162 \scriptstyle{\pm 0.071}$  &
% $0.123 \scriptstyle{\pm 0.027 }$ &
% % ------------------------------------ %
% $0.620 \scriptstyle{\pm 0.050}$ &
% $0.320 \scriptstyle{\pm 0.010}$ \\
% %
% IBC & 
% % ------------------------------------ %
% % ------------------------------------ %
% $0.482 \scriptstyle{\pm 0.026 }$ & 
% $\mathbf{0.758 \scriptstyle{\pm 0.022 }}$ &
% $0.123 \scriptstyle{\pm 0.007 }$ &
% % ------------------------------------ %
% $0.567 \scriptstyle{\pm 0.030}$ &
% $0.310 \scriptstyle{\pm 0.010}$ \\
% %
% BET & 
% % ------------------------------------ %
% % ------------------------------------ %
% $ 0.374\scriptstyle{\pm 0.041}$ &
% $ 0.607\scriptstyle{\pm 0.037}$ &
% $ 0.151\scriptstyle{\pm 0.008}$ &
% % ------------------------------------ %
% $0.758 \scriptstyle{\pm 0.025}$ &
% $0.235 \scriptstyle{\pm 0.011}$ \\
% %
% IMC & 
% % ------------------------------------ %
% % ------------------------------------ %
% $ \mathbf{0.521\scriptstyle{\pm 0.045}}$ &
% $ 0.654\scriptstyle{\pm 0.041}$ &
% $ \mathbf{0.120\scriptstyle{\pm 0.014}}$ &
% % ------------------------------------ %
% $\mathbf{0.870 \scriptstyle{\pm 0.017}}$ &
% $\mathbf{0.153 \scriptstyle{\pm 0.007}}$ \\
% \bottomrule
% \end{tabular}
% }
% \end{sc}
% \end{small}
% \end{center}
% \vskip -0.1in
% \end{table*}
% %%%%%%%%%%%%%%%%%%%%%%%%%%%%%%%%%%%%%%%%%%%%%%%%%%%%%%%%%%%%%%%%%%%%%%%


% \mathbf{E}_{p(z)} \mathbf{E}_{p(\mathcal{D}|z)}[\log p_{\theta_{z}}(a|o,z)] + \eta \mathcal{H}(\mathcal{D})


%%%%%%%%%%%%%%%%%%%%%%%%%%%%%%%%%%%%%%%%%%%%%%%%%%%%%%%%%%%%%%%%%%%%%%%
%%%%%%%%%%%%%%%%%%%%%%%%%%%%%%%%%%%%%%%%%%%%%%%%%%%%%%%%%%%%%%%%%%%%%%%
\begin{figure*}[t]
        \centering
        \begin{minipage}[t!]{0.235\textwidth}
            \centering 
            \includegraphics[width=\textwidth]{supplementary_material/figures/gg_rr_env_023_05.pdf}
        \end{minipage}
        \hfill
        \begin{minipage}[t!]{0.235\textwidth}
            \centering 
            \includegraphics[width=\textwidth]{supplementary_material/figures/rr_gg_env_009_05.pdf}
        \end{minipage}
         \hfill
        \centering
        \begin{minipage}[t!]{0.235\textwidth}
            \centering 
            \includegraphics[width=\textwidth]{supplementary_material/figures/gr_rg.pdf}
        \end{minipage}
        \hfill
        \begin{minipage}[t!]{0.235\textwidth}
            \centering 
            \includegraphics[width=\textwidth]{supplementary_material/figures/rg_gr.pdf}
        \end{minipage}
         \hfill
        \caption[ ]
        {\small \textbf{Block pushing:} Top view of four different push sequences for the same initial block configuration $\mathbf{c}$. Following the gray line from the black cross visualizes the end-effector trajectory of the robot manipulator. The small rectangles indicate different box configurations in the push sequence while the big rectangles mark the target zones.}
        \label{fig:box_env_trajs}
    \end{figure*}
%%%%%%%%%%%%%%%%%%%%%%%%%%%%%%%%%%%%%%%%%%%%%%%%%%%%%%%%%%%%%%%%%%%%%%%
\label{section:experiments}
%%%%%%%%%%%%%%%%%%%%%%%%%%%%%%%%%%%%%%%%%%%%%%%%%%%%%%%%%%%%%%%%%%%%%%%
%%%%%%%%%%%%%%%%%%%%%%%%%%%%%%%%%%%%%%%%%%%%%%%%%%%%%%%%%%%%%%%%%%%%%%%
% \subsection{UCI Regression}
% \begin{table}[t]
% \caption{\textbf{UCI regression}: Comparison between various combining methods on UCI regression datasets.}
% \label{table:obs_avoid}
% \vskip 0.15in
% \begin{center}
% \begin{small}
% \begin{sc}
% \begin{tabular}{lcccc}
% \toprule
%     Dataset & {Na\"ive} & {EM} & {MDN} &  {IMC}  \\ 
%     \midrule
%     Boston
%     & $- \scriptstyle{\pm -}$
%     & $- \scriptstyle{\pm -}$
%     & $- \scriptstyle{\pm -}$
%     & $- \scriptstyle{\pm -}$
%     \\
%     %%%%%%%%%%%%%%%%%%%%%%%%%%%%%%
%     Concrete
%     & $- \scriptstyle{\pm -}$
%     & $- \scriptstyle{\pm -}$
%     & $- \scriptstyle{\pm -}$
%     & $- \scriptstyle{\pm -}$
%     \\
%     %%%%%%%%%%%%%%%%%%%%%%%%%%%%%%
%     Energy
%     & $- \scriptstyle{\pm -}$
%     & $- \scriptstyle{\pm -}$
%     & $- \scriptstyle{\pm -}$
%     & $- \scriptstyle{\pm -}$
%     \\
%     %%%%%%%%%%%%%%%%%%%%%%%%%%%%%%
%     Kin8nm
%     & $- \scriptstyle{\pm -}$
%     & $- \scriptstyle{\pm -}$
%     & $- \scriptstyle{\pm -}$
%     & $- \scriptstyle{\pm -}$ 
%     \\
%     %%%%%%%%%%%%%%%%%%%%%%%%%%%%%%
%     Power
%     & $- \scriptstyle{\pm -}$
%     & $- \scriptstyle{\pm -}$
%     & $- \scriptstyle{\pm -}$
%     & $- \scriptstyle{\pm -}$
%     \\
%     %%%%%%%%%%%%%%%%%%%%%%%%%%%%%%
%     Protein
%     & $- \scriptstyle{\pm -}$
%     & $- \scriptstyle{\pm -}$
%     & $- \scriptstyle{\pm -}$
%     & $- \scriptstyle{\pm -}$
%     \\
%     %%%%%%%%%%%%%%%%%%%%%%%%%%%%%%
%     Wine
%     & $- \scriptstyle{\pm -}$
%     & $- \scriptstyle{\pm -}$
%     & $- \scriptstyle{\pm -}$
%     & $- \scriptstyle{\pm -}$
%     \\
%     %%%%%%%%%%%%%%%%%%%%%%%%%%%%%%
%     Yacht
%     & $- \scriptstyle{\pm -}$
%     & $- \scriptstyle{\pm -}$
%     & $- \scriptstyle{\pm -}$
%     & $- \scriptstyle{\pm -}$
%     \\
%     %%%%%%%%%%%%%%%%%%%%%%%%%%%%%%
%     \bottomrule
%   \end{tabular}
% \end{sc}
% \end{small}
% \end{center}
% \vskip -0.1in
% \end{table}
% %%%%%%%%%%%%%%%%%%%%%%%%%%%%%%%%%%%%%%%%%%%%%%%%%%%%%%%%%%%%%%%%%%%%%%%

%%%%%%%%%%%%%%%%%%%%%%%%%%%%%%%%%%%%%%%%%%%%%%%%%%%%%%%%%%%%%%%%%%%%%%%
%%%%%%%%%%%%%%%%%%%%%%%%%%%%%%%%%%%%%%%%%%%%%%%%%%%%%%%%%%%%%%%%%%%%%%%
\subsection{Obstacle Avoidance}
\label{section:obs_avoid}
The obstacle avoidance environment is visualized in Figure \ref{fig:environments} (left) and consists of a seven DoF Franka Emika Panda robot arm equipped with a cylindrical end effector simulated using the MuJoCo physics engine \cite{todorov2012mujoco}.  The task is to reach the green finish line without colliding with one of the six obstacles.
% There are $24$ different ways to avoid the obstacles and solve the task. For each way there are four human demonstrations in the dataset collected using a game-pad controller.
The dataset contains four human demonstrations for all $24$ ways of avoiding obstacles and completing the task which are collected using a game-pad controller and inverse kinematics (IK) in the xy-plane amounting to $7.3$k $(\mathbf{x}, \mathbf{y})$ pairs.
The inputs $\mathbf{x} \in \mathbb{R}^{4}$ contain the end-effector position and velocity of the robot. 
The targets $\mathbf{y} \in \mathbb{R}^{2}$ represent the desired position of the robot.
To evaluate the susceptibility to mode averaging, we use the \textit{success rate}, i.e., the percentage of trajectories that reach the finish line. Moreover, we assess a model's ability to learn multimodal distributions by computing the \textit{entropy} of the categorical distribution that contains the  probabilities of a model completing the different ways of avoiding obstacles.  
% successful trajectories $\vtau$, that is, $\mathcal{H}_{24}(\vtau) = - \sum_\vtau p(\vtau) \log_{24} p(\vtau)$. Note that we use $\log_{24}$ for easier comparison since $\mathcal{H}_{24}(\vtau)\in [0, 1]$. 
For more details see Appendix \ref{appendix:obs_avoidance}. For a visual illustration of end-effector trajectories see Figure \ref{fig:planar_reacher_vis}. The results are shown in Table \ref{table:result_table} and are generated using $1000$ evaluation trajectories for each seed. IMC achieves a superior success rate and entropy compared to the baselines. CVAE closely follows the success rate of IMC but lacks the ability to discover modes in the data distribution, indicated by the low entropy value. In contrast, EM achieves an entropy similar to IMC but is inferior with respect to the success rate.
% CVAE achieves the highest success rate but only discovers few modes as indicated by the low entropy value. In contrast, IMC achieves the highest entropy while closely following the success rate of CVAE. 
% [Put the exact number of samples in the appendix]


% %%%%%%%%%%%%%%%%%%%%%%%%%%%%%%%%%%%%%%%%%%%%%%%%%%%%%%%%%%%%%%%%%%%%%%%
% \begin{figure}[t]
% \centering
% \resizebox{0.23\textwidth}{!}{\includegraphics[width=\textwidth]{results/obs_avoid/Fig:ICML23:obsavoid_success.pdf}}
% \resizebox{0.23\textwidth}{!}{\includegraphics[width=\textwidth]{results/obs_avoid/Fig:ICML23:obsavoid_entropy.pdf}}
%     \caption{\small \textbf{Obstacle Avoidance.} Comparison between the EM and IMC algorithm for training mixtures of experts for an increasing number of components.
%     }
%     \label{fig:obs_avoid_res}
% \end{figure}
% %%%%%%%%%%%%%%%%%%%%%%%%%%%%%%%%%%%%%%%%%%%%%%%%%%%%%%%%%%%%%%%%%%%%%%%
\subsection{Block Pushing}
The block pushing environment is visualized in Figure \ref{fig:environments} (middle) and uses the setup explained in Section \ref{section:obs_avoid} with the 2-D gamepad controller. However, the robot manipulator is tasked to push blocks into target zones. Having two blocks and target zones amounts to four different push sequences. See Figure \ref{fig:box_env_trajs} for an example. We consider $30$ different block configurations $\vc$ (i.e., initial orientation and position) that are uniformly sampled from a configuration space. Using a game-pad controller we recorded four trajectories for all push sequences and block configurations amounting to a total of $30 \times 4 \times 4 = 480$ demonstrations and thus $100$k $(\mathbf{x}, \mathbf{y})$ pairs. The inputs $\mathbf{x} \in \mathbb{R}^{16}$ contain information about the robot's state and the block configurations. The targets $\mathbf{y} \in \mathbb{R}^{2}$ represent the desired position of the robot.
We evaluate the models using three different metrics: First, the \textit{success rate} which is the proportion of trajectories that manage to push both boxes to the target zones. Next, the expected \textit{entropy} over
a categorical distribution containing the probabilities of a model completing different push sequences conditioned on the initial block configuration $\mathbf{c}$. Lastly, to evaluate the performance on non-successful trajectories, we employ the \textit{distance error}, that is, the distance from the blocks to the target zones at the end of a trajectory. The success rate and distance error indicate whether a model is able to avoid averaging over different behavior. Moreover, the entropy assesses the ability to represent multimodal data distributions by completing different push-sequences for the same configuration. See Appendix  \ref{appendix:block_pushing} for more details. 
The results are reported in Table \ref{table:result_table} and generated simulating $16$ evaluation trajectories for all $30$ contexts per seed. IMC achieves superior success rate and distance error while BET has the highest entropy. The difficulty of the task is reflected by the low success rates of most models. Besides being a challenging manipulation task, the high task complexity is caused by having various sources of multimodality in the data distribution: First, the inherent versatility in human behavior. Second, multiple human demonstrators, and lastly different push sequences for the same block configuration.
% %%%%%%%%%%%%%%%%%%%%%%%%%%%%%%%%%%%%%%%%%%%%%%%%%%%%%%%%%%%%%%%%%%%%%%%
% \begin{figure}[t]
%         \centering
%         \begin{minipage}[t!]{0.11\textwidth}
%             \centering 
%             \includegraphics[width=\textwidth]{supplementary_material/figures/gg_rr_env_023_05.pdf}
%         \end{minipage}
%         \hfill
%         \begin{minipage}[t!]{0.11\textwidth}
%             \centering 
%             \includegraphics[width=\textwidth]{supplementary_material/figures/rr_gg_env_009_05.pdf}
%         \end{minipage}
%          \hfill
%         \centering
%         \begin{minipage}[t!]{0.11\textwidth}
%             \centering 
%             \includegraphics[width=\textwidth]{supplementary_material/figures/gr_rg.pdf}
%         \end{minipage}
%         \hfill
%         \begin{minipage}[t!]{0.11\textwidth}
%             \centering 
%             \includegraphics[width=\textwidth]{supplementary_material/figures/rg_gr.pdf}
%         \end{minipage}
%          \hfill
%         \caption[ ]
%         {\small Bird view of the block push data for 4 modes. In this task, robot starts from a fixed position and the gray lines refer to the trajectory of the robot's end effector. Two push boxes' trajectories are represented by red box and green box.}
%         \label{fig:box_env}
%     \end{figure}
% %%%%%%%%%%%%%%%%%%%%%%%%%%%%%%%%%%%%%%%%%%%%%%%%%%%%%%%%%%%%%%%%%%%%%%%


%%%%%%%%%%%%%%%%%%%%%%%%%%%%%%%%%%%%%%%%%%%%%%%%%%%%%%%%%%%%%%%%%%%%%%%
\subsection{Franka Kitchen}
The Franka kitchen environment was introduced by \citet{gupta2019relay} and uses a seven DoF Franka Emika Panda robot with a two DoF gripper to interact with a simulated kitchen environment. The corresponding dataset contains $566$ human-collected trajectories collected using a virtual reality setup amounting to $128$k $(\mathbf{x}, \mathbf{y})$ pairs. Each trajectory executes a sequence completing four out of seven different tasks. The inputs $\mathbf{x} \in \mathbb{R}^{30}$ contain information about position and orientation of the task-relevant objects in the environment. The targets $\mathbf{y} \in \mathbb{R}^{9}$ represent the control signals for the robot and gripper. To asses a model's ability to avoid mode averaging we again use the \textit{success rate} over the number of tasks solved within one trajectory. Completing tasks in different orders introduces multimodality into the data distribution which is assessed by the \textit{entropy} of the distribution over completed task sequences. For more details see Appendix \ref{appendix:fk_kitchen}. The results are shown in Figure \ref{fig:kitchen_res} and are generated using $100$ evaluation trajectories for each seed. There are no results reported for IBC and MDN as we did not manage to obtain reasonable results. All models except for EM manage to complete one and two tasks with a success rate close to $1$. For three and four task completions DDPM has the highest success rate closely followed by IMC and CVAE. EM has the highest entropy for one task. For two, three and four tasks BET achieves superior entropy, being slightly ahead of IMC and DDPM. 
%%%%%%%%%%%%%%%%%%%%%%%%%%%%%%%%%%%%%%%%%%%%%%%%%%%%%%%%%%%%%%%%%%%%%%%
\begin{figure}[t]
\centering
\resizebox{0.4\textwidth}{!}{\includegraphics[width=\textwidth]{results/kitchen/fk_legend.pdf}}
\resizebox{0.23\textwidth}{!}{\includegraphics[width=\textwidth]{results/kitchen/fk_success.pdf}}
\resizebox{0.23\textwidth}{!}{\includegraphics[width=\textwidth]{results/kitchen/fk_entropy.pdf}}
    \caption{\textbf{Franka Kitchen:} Performance comparison between various generative models.
    }
    \label{fig:kitchen_res}
\end{figure}
%%%%%%%%%%%%%%%%%%%%%%%%%%%%%%%%%%%%%%%%%%%%%%%%%%%%%%%%%%%%%%%%%%%%%%%
\subsection{Table Tennis}
%%%%%%%%%%%%%%%%%%%%%%%%%%%%%%%%%%%%%%%%%%%%%%%%%%%%%%%%%%%%%%%%%%%%%%%
\begin{figure*}[t]
        \centering
        \begin{minipage}[t!]{0.49\textwidth}
            \centering
            \includegraphics[width=.49\textwidth]{results/obs_avoid/obsavoid_success.pdf}
            \includegraphics[width=.49\textwidth]{results/obs_avoid/obsavoid_entropy.pdf}
            \subcaption[]{Obstacle avoidance performance comparison.}
            \label{fig:ablation_performance_obstacle_avoidance}
        \end{minipage}
                \hfill
        \begin{minipage}[t!]{0.49\textwidth}
            \centering 
            \includegraphics[width=.49\textwidth]{results/TableTennisStrike.pdf}
            \includegraphics[width=.49\textwidth]{results/TableTennisDistance.pdf}
            \subcaption[]{Table tennis performance comparison.}
            \label{fig:ablation_performance_tabletennis}
        \end{minipage}
         \hfill
         \begin{minipage}[t!]{0.49\textwidth}
            \centering 
            \includegraphics[width=.32\textwidth]{supplementary_material/figures/obstacle_avoid_figures/obsavoid_em_1c.pdf}
            \includegraphics[width=.32\textwidth]{supplementary_material/figures/obstacle_avoid_figures/obsavoid_em_5c.pdf}
            \includegraphics[width=.32\textwidth]{supplementary_material/figures/obstacle_avoid_figures/obsavoid_em_10c.pdf}
            \subcaption[]{Obstacle avoidance end-effector trajectories for EM.}
            \label{fig:ablation_EM_obstacle_avoidance}
        \end{minipage}
         \hfill
         \begin{minipage}[t!]{0.49\textwidth}
            \centering 
            \includegraphics[width=.32\textwidth]{supplementary_material/figures/obstacle_avoid_figures/obsavoid_imc_1c.pdf}
            \includegraphics[width=.32\textwidth]{supplementary_material/figures/obstacle_avoid_figures/obsavoid_imc_5c.pdf}
            \includegraphics[width=.32\textwidth]{supplementary_material/figures/obstacle_avoid_figures/obsavoid_imc_10c.pdf}
            \subcaption[]{Obstacle avoidance end-effector trajectories for IMC.}
            \label{fig:ablation_IMC_obstacle_avoidance}
        \end{minipage}
         \hfill
        \caption[ ]
        {\small \textbf{Ablation study:} Performance comparison between the EM and IMC algorithm for an increasing number of components.}
        \label{fig:ablation}
    \end{figure*}
%%%%%%%%%%%%%%%%%%%%%%%%%%%%%%%%%%%%%%%%%%%%%%%%%%%%%%%%%%%%%%%%%%%%%%%
The table tennis environment is visualized in Figure \ref{fig:environments} (right) and consists of a seven DOF robot arm equipped with a table tennis racket and is simulated using the MuJoCo physics
engine. The goal is to return the ball to varying target positions after it is launched from a randomized initial position. Although not collected by human experts, the $5000$ demonstrations are generated using a reinforcement learning agent that is optimized for highly multimodal behavior such as backhand and forehand strokes \cite{celik2022specializing}. Each demonstration consists of an input $\mathbf{x} \in \mathbb{R}^4$ defining the initial and target ball position. Movement primitives (MPs) \cite{paraschos2013probabilistic} are used to describe the joint space trajectories of the robot manipulator using two basis functions per joint and thus $\mathbf{y} \in \mathbb{R}^{14}$. We evaluate the model performance using the \textit{success rate}, that is, how frequently the ball is returned to the other side. Moreover, we employ the \textit{distance error}, i.e., the euclidean distance from the landing position of the ball to the target position. Both metrics reflect if a model is able to avoid averaging over different movements. For this experiment, there is no metric to assess multimodality as it is difficult to quantify the versatility in the model behavior. The results are shown in Table \ref{table:result_table} and are generated using $500$ different initial and target positions. Please note that the reinforcement learning agent used to generate the data achieves an average success rate of $0.91$ and a distance error of $0.14$. This performance is closely followed by IMC which achieves superior performance compared to the other methods.  
% \begin{figure}[t]
% \centering
% % \resizebox{0.46\textwidth}{!}{\includegraphics[width=\textwidth]{results/Fig:TableTennisLegend.pdf}}
% \resizebox{0.23\textwidth}{!}{\includegraphics[width=\textwidth]{results/Fig:TableTennisStrike.pdf}}
% \resizebox{0.23\textwidth}{!}{\includegraphics[width=\textwidth]{results/Fig:TableTennisDistance.pdf}}
%     \caption{\small \textbf{Table Tennis.} Comparison between the EM and IMC algorithm for training mixtures of experts for an increasing number of components. Additionally, the expert performance is reported. 
%     }
%     \label{fig:tt_algocomp}
% \end{figure}
% %%%%%%%%%%%%%%%%%%%%%%%%%%%%%%%%%%%%%%%%%%%%%%%%%%%%%%%%%%%%%%%%%%%%%%%

\subsection{Ablation Studies}

Additionally, we compare the performance of IMC with EM for a varying number of components on the obstacle avoidance and table tennis task.
Please note that the incremental component adding scheme of IMC allows for evaluating the model performance after additional components are added. In contrast, the model is trained from scratch when employing EM for evaluating the performance using a different number of components. The results are shown in Figure \ref{fig:ablation} and highlight the properties of the moment and information projection: Using limited model complexity, e.g. $1$ or $5$ components, EM suffers from mode averaging, resulting in poor performances (Figure \ref{fig:ablation_performance_obstacle_avoidance} and Figure \ref{fig:ablation_performance_tabletennis}). This is further illustrated in Figure \ref{fig:ablation_EM_obstacle_avoidance}. In contrast, the zero forcing property of the information projection allows IMC to avoid mode averaging (see Figure \ref{fig:ablation_IMC_obstacle_avoidance}) which is reflected in the success rates and distance error for a small number of components. The performance gap between EM and IMC for high model complexities suggests that EM still suffers from averaging problems. Moreover, the results show that IMC needs fewer components to achieve the same performance as EM.
