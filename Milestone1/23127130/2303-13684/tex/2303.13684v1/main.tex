%%
%% This is file `sample-manuscript.tex',
%% generated with the docstrip utility.
%%
%% The original source files were:
%%
%% samples.dtx  (with options: `manuscript')
%% 
%% IMPORTANT NOTICE:
%% 
%% For the copyright see the source file.
%% 
%% Any modified versions of this file must be renamed
%% with new filenames distinct from sample-manuscript.tex.
%% 
%% For distribution of the original source see the terms
%% for copying and modification in the file samples.dtx.
%% 
%% This generated file may be distributed as long as the
%% original source files, as listed above, are part of the
%% same distribution. (The sources need not necessarily be
%% in the same archive or directory.)
%%
%% Commands for TeXCount
%TC:macro \cite [option:text,text]
%TC:macro \citep [option:text,text]
%TC:macro \citet [option:text,text]
%TC:envir table 0 1
%TC:envir table* 0 1
%TC:envir tabular [ignore] word
%TC:envir displaymath 0 word
%TC:envir math 0 word
%TC:envir comment 0 0
%%
%%
%% The first command in your LaTeX source must be the \documentclass command.
\documentclass[manuscript,screen,nonacm]{acmart}
%\usepackage[final]{changes}
\usepackage{soul}
%% \BibTeX command to typeset BibTeX logo in the docs
\AtBeginDocument{%
  \providecommand\BibTeX{{%
    \normalfont B\kern-0.5em{\scshape i\kern-0.25em b}\kern-0.8em\TeX}}}

%% Rights management information.  This information is sent to you
%% when you complete the rights form.  These commands have SAMPLE
%% values in them; it is your responsibility as an author to replace
%% the commands and values with those provided to you when you
%% complete the rights form.
\settopmatter{printacmref=false}
\setcopyright{none}
%\copyrightyear{2018}
%\acmYear{2018}
%\acmDOI{XXXXXXX.XXXXXXX}

%% These commands are for a PROCEEDINGS abstract or paper.
% \acmConference[Conference acronym 'XX]{Make sure to enter the correct
%   conference title from your rights confirmation emai}{June 03--05,
%   2018}{Woodstock, NY}
% \acmPrice{15.00}
% \acmISBN{978-1-4503-XXXX-X/18/06}


%%
%% Submission ID.
%% Use this when submitting an article to a sponsored event. You'll
%% receive a unique submission ID from the organizers
%% of the event, and this ID should be used as the parameter to this command.
%%\acmSubmissionID{123-A56-BU3}

%%
%% For managing citations, it is recommended to use bibliography
%% files in BibTeX format.
%%
%% You can then either use BibTeX with the ACM-Reference-Format style,
%% or BibLaTeX with the acmnumeric or acmauthoryear sytles, that include
%% support for advanced citation of software artefact from the
%% biblatex-software package, also separately available on CTAN.
%%
%% Look at the sample-*-biblatex.tex files for templates showcasing
%% the biblatex styles.
%%

\usepackage{hyperref}
\usepackage[hyphenbreaks]{breakurl}
%%
%% The majority of ACM publications use numbered citations and
%% references.  The command \citestyle{authoryear} switches to the
%% "author year" style.
%%
%% If you are preparing content for an event
%% sponsored by ACM SIGGRAPH, you must use the "author year" style of
%% citations and references.
%% Uncommenting
%% the next command will enable that style.
%%\citestyle{acmauthoryear}

%%
%% end of the preamble, start of the body of the document source.
\begin{document}

%%
%% The "title" command has an optional parameter,
%% allowing the author to define a "short title" to be used in page headers.
\title{Contextual Integrity of A Virtual (Reality) Classroom}



%%
%% The "author" command and its associated commands are used to define
%% the authors and their affiliations.
%% Of note is the shared affiliation of the first two authors, and the
%% "authornote" and "authornotemark" commands
%% used to denote shared contribution to the research.

\author{Karoline Brehm}
  \email{brehm@yorku.ca}
  \orcid{https://orcid.org/0009-0007-2701-8621}
  \affiliation{
      \institution{Bauhaus-Universität Weimar}
      \country{Germany}
  }
  \affiliation{%
      \institution{York University}
      \country{Canada}
  }

  \author{Yan Shvartzshnaider}
  \orcid{0000-0001-5954-916X}
  \email{yansh@yorku.ca}
  \affiliation{%
      \institution{York University}
      \country{Canada}
  }


  \author{David Goedicke}
  \email{dg536@cornell.edu}
  \orcid{0000-0002-4837-893X}
  \affiliation{%
      \institution{Cornell Tech}
      \country{USA}
  }

  \settopmatter{printacmref=false}
%%
%% By default, the full list of authors will be used in the page
%% headers. Often, this list is too long, and will overlap
%% other information printed in the page headers. This command allows
%% the author to define a more concise list
%% of authors' names for this purpose.
%\renewcommand{\shortauthors}{Trovato and Tobin, et al.}

%%
%% The abstract is a short summary of the work to be presented in the
%% article.
\begin{abstract}
The multicontextual nature of immersive VR makes it difficult to ensure contextual integrity of VR-generated information flows using existing privacy design and policy mechanisms. In this position paper, we call on the HCI community to do away with lengthy disclosures and permissions models and move towards embracing privacy mechanisms rooted in Contextual Integrity theory.
\end{abstract}

%%
%% The code below is generated by the tool at http://dl.acm.org/ccs.cfm.
%% Please copy and paste the code instead of the example below.
%%
 \begin{CCSXML}
<ccs2012>
   <concept>
       <concept_id>10002978.10003029.10011150</concept_id>
       <concept_desc>Security and privacy~Privacy protections</concept_desc>
       <concept_significance>500</concept_significance>
       </concept>
   <concept>
       <concept_id>10003120</concept_id>
       <concept_desc>Human-centered computing</concept_desc>
       <concept_significance>500</concept_significance>
       </concept>
   <concept>
       <concept_id>10003120.10003121.10003124.10010866</concept_id>
       <concept_desc>Human-centered computing~Virtual reality</concept_desc>
       <concept_significance>500</concept_significance>
       </concept>
 </ccs2012>
\end{CCSXML}

\ccsdesc[500]{Security and privacy~Privacy protections}
\ccsdesc[500]{Human-centered computing}
\ccsdesc[500]{Human-centered computing~Virtual reality}



%%
%% Keywords. The author(s) should pick words that accurately describe
%% the work being presented. Separate the keywords with commas.

%\keywords{Virtual Reality, Education}

%\received{\today}
%\received[revised]{12 March 2009}
%\received[accepted]{5 June 2009}

%%
%% This command processes the author and affiliation and title
%% information and builds the first part of the formatted document.
\maketitle

\section{Threat Model and Advantages of Our Hardware-based Adversarial Detector} \label{sec: motivation}
\ry{In this part, I want to highlight the comparison between hardware and software attacks}
%Normally, software-based adversarial detectors are easier to implement, cheaper to develop and more well-studied than those based on hardware computational signals.
% We would like to stress that our goal for investigating hardware-based adversarial detectors is not to achieve better performance in detection than the conventional white-box software based methods.  
\subsection{Threat Model} \label{sec: threat model}
\ry{This section is threat model: attack is `white-box', detector is `black-box'}
The victim is a DNN classifier, which is pre-trained with a public dataset. The testing dataset may be kept private.
We assume the strongest `white-box' attack model, where the attacker has full knowledge of the victim model and training dataset in order to generate adversarial samples with minimum perturbations. 
On the contrary, the detection system assumes the most limited scenario, under a `black-box' view of the victim, without access to the victim's inputs, parameters, and intermediate outputs or execution details. 
The only information available to the detector to distinguish adversarial samples is the EM side-channel measurement and the victim model's prediction class.
For training the adversarial detector with EM traces, a public benign dataset is used. 

\if false 
\ry{In this part, we discuss more settings of the detector especially the data used in two phases.}
In general, the detecting process can be summed up into two phases, training phase and detecting phase.
To begin with, we train an Out-of-Distribution(OOD) detector on a public benign dataset of the same classification task, which should be distinct from the victim's training dataset.
For each query, the detector will obtain the classification result and an EM trace along with the model execution to fit its EM classifiers and anomaly detectors.  
During the detection phase, the victim model is in operation and under attack when the pre-trained detector decides whether the current input is adversarial or not, only based on the victim model output and its EM trace.
\fi 

\subsection{Advantages}
Compared to software-based adversarial detection methods, our hardware-based detector, EMShepherd, has three distinct advantages: privacy-preserving, portability, and robustness.

\begin{itemize}[leftmargin=*]
    \item \ry{Add a new motivation here. The motivation is that using \name can help the user protect their privacy.} 
    \name protects the DNN model user's data privacy as it is agnostic to the model's inputs, which instead are always required by prior reconstruction-based detection methods~\cite{meng2017magnet, yang2022you}. 
    %Most model users are benign whose inputs may be sensitive and should not be shared with \textit{third-party detectors}. 
    The sensitive inputs should not be shared with \textit{third-party detectors}. 
    Our design only requires the output class labels and the EM signals, which are passively leaked to common acquisition equipment. 
    %    Our design is suitable for such cases as it only requires the EM signals and the inference outputs during the model execution. Generally speaking, EM signals and labels have less private information leakage.
    \item \ry{The second motivation is still related to privacy. This time we consider model privacy when the model structure or parameters should be kept private.}
   \name also protects the model confidentiality.  No model information, including %Using hardware-based detectors can prevent the third-party defender from accessing some confidential model information such as  
   hyper-parameters, parameters, and logits, is needed, in stark contrast to the previous software-based detection methods~\cite{ma2019nic,feinman2017detecting}.
    %Our \name only acquires the EM traces during model inference in a passive and noninvasive manner, 
    The EM data processing and the adversarial detector training process are both victim model-agnostic. 
    Therefore, our method has more general usage, applicable to closed-source DNN applications, which are pervasive in edge devices where the user only queries the models for the final prediction output. 
    \item \ry{The third motivation is portability.}  
    Owing to the model-agnostic feature, EMShepherd can be easily ported for wide-range hardware devices with different DNN implementations for diverse applications. It can be used as a `plug and play' (PnP) device, aside from the target system, to work automatically without user intervention or contact with the victim system. 
    \item \ry{The last motivation is about adaptive attacks, we should propose that EM signal is hard to imitate, so it is hard for adaptive attacks to generate sample fraud both detector and victim.} 
    Adaptive attack~\cite{adaptive} is a threat to most software defense methods where the attacker adjusts the adversarial perturbations to mislead both the victim models and defense systems.
   %  The hardware-based detection method can provide a double protection on top of most software defense methods such as adversarial training.
   %  Although the adptive adversarial example fools the robust model, its computation patterns during the DNN model execution are still well kept in the EM traces and our EMShepherd framework still works well for detecting the new type of adversarial examples.  
   %  Meanwhile, due to the high complexity of EM signals and non-explicit dependency of the EM signals on computations, it is extremely hard to have an adaptive attack on our detection method, i.e., adversarial examples whose EM signals are deliberately controlled to evade the EM-based detector.
   However, due to the high complexity and non-explicit dependency of the EM signals on computations and data, 
   it is extremely hard to have an adaptive attack on our detection method, 
   i.e., adversarial examples whose EM signals are deliberately controlled to evade the EM-based detector. 
\end{itemize}





\section{The Multicontextual Policy Challenge}

A student attending a lecture in VR from home is acting both as a student in the classroom and as a member of a household. This means that the student operates in multiple social contexts. These contexts are governed by different privacy norms~\cite{nissenbaumPrivacyContextTechnology2010}: data processing in the context of the home may clash with what is considered appropriate in the virtual classroom. For example, optical and depth sensors on the students' VR headsets map the environment. While this is necessary for navigating in the virtual classroom, the same sensors can capture other people in the vicinity~\cite{mcgillExtendedRealityXR2021}, such as family members or roommates.
A similar issue arose with the introduction of video conferences and online educational platforms at universities, which caused privacy violations due to a mismatch of students' expectations and data practices of the platforms~\cite{cohneyVirtualClassroomsReal2021}.

Existing privacy-enhancing approaches in VR are predominately based on the notions of transparency and ``informed consent''~\cite{EthicalConsiderationsExtended2021, XRSIPrivacySafety2022,xrassociationChapterDesigningImmersive2022}. Suggestions include ``contextual privacy communications'' at the time of data processing~\cite{XRSIPrivacySafety2022}, visualisation of privacy risks and customisable privacy settings ~\cite{kimVirtualRealityData2022}.  In doing so, the burden of understanding the risks and preventing harm in VR falls disproportionately on the user. This further contributes to ``[widening] the gap between the users’ understanding of the types of data collected, the implications of their consent and the actual implications''~\cite{kimVirtualRealityData2022}. 
 
The multicontextual nature of VR applications calls for privacy-enhancing mechanisms that incorporate privacy norms in multiple overlapping contexts. In our work, we draw on the theory of Contextual Integrity (CI)~\cite{nissenbaumPrivacyContextTechnology2010} to address these issues. Specifically, how can we ensure that privacy norms are respected in VR-based platforms and what CI-based methodologies could be employed by policymakers and designers?
\section{VR Privacy-Preserving Design Challenges}

One of the main challenges in designing a privacy-preserving VR experience is maintaining contextual integrity of users' data. Specifically, this entails developing mechanisms to ensure that information flows in accordance with users' expectations and established societal norms. A suitable solution would need to differentiate between a multitude of nested contexts. For example, in a VR classroom, a conversation between fellow students is subject to different privacy norms than a meeting with the academic supervisor or a student counsellor. Privacy protections would also need to account for  different modes of interaction and be robust to accommodate the privacy norms associated with future VR adaptations. 

In search of answers to these challenges, the research community is  building tools to help users make \emph{meaningful} privacy-related choices. Examples of such efforts include: designing a VR-native interface for privacy settings, in lieu of a traditional text-based menu~\cite{yaoVirtualEquipmentSystem2021}; providing users with an option to engage in an ``incognito mode" to stop the flow of information to other VR users~\cite{yaoVirtualEquipmentSystem2021}; and informing the user about privacy risks through  a role-playing privacy tutorial that visually reenacts the information flow prescribed by a privacy policy~\cite{limMineYourselfRoleplaying2022}.

These privacy-preserving approaches, however helpful, still build on privacy models that embody  notions of information secrecy, control, personal anonymity and informed consent, despite their well-documented shortcomings~\cite{cateFailureFairInformation2006, schaubDesignSpaceEffective2015}. Here, too, with close coordination between all the relevant stakeholders of the VR ecosystem\footnote{We envision collaboration between VR users, designers, system engineers, developers, policymakers and others.}, the theory of CI can help inform the design of new privacy protection mechanisms based on the notion of appropriate information flows in accordance with enforced privacy norms. The CI-based approach would rely on tools for identifying information flows generated in the system~\cite{shvartzshnaiderVACCINEUsingContextual2019}  and prescribed by the policy~\cite{shvartzshnaiderGoingAppropriateFlow2019}, capturing users' expectations~\cite{shvartzshnaiderLearningPrivacyExpectations2016, apthorpeDiscoveringSmartHome2018}, and devising effective mechanisms of detecting deviation from contextual privacy norms in a VR environment.

\section{Acknowledgements}
This project has been funded by the Office of the Privacy Commissioner of Canada (OPC); the views expressed herein are those of the author(s) and do not necessarily reflect those of the OPC.

% \section{Conclusion}\label{sec:conclusion}
In this work, we focus on addressing the fundamental challenge of OOD detection tasks, which is how to fully understand the semantic discrepancy between the ID/OOD samples. We reveal that the key to success in the realistic SCOOD task is to allocate as many ID samples in the unlabeled set correctly as possible. To this end, we propose a novel uncertainty-aware optimal transport scheme that introduces class-specific energy scores as guidance for effective label assignment. Experimental results show that our method achieves better performance than previous state-of-the-art methods on SCOOD benchmarks.

\textbf{Limitations.} In addition to temperature scaling, other techniques such as feature clipping applied in ReAct~\cite{sun2021react} also enhance the performance of energy score, so how to obtain an OOD score that best fits the SCOOD task can be further explored. Moreover, a setting highly related to SCOOD has been proposed in \cite{katz2022training} and formulated as a constrained optimization problem. We will also theoretically analyze these practical OOD settings in our feature work.

% \section*{Acknowledgments}
\textbf{Acknowledgments.} 
This work is supported by National Key R\&D Program of China under Grant 2020AAA0105701, National Natural Science Foundation of China (NSFC) under Grants 61872327, Major Special Science and Technology Project of Anhui, National Natural Science Foundation of China (62033012) and Ant Group through Ant Research Intern Program.


\bibliographystyle{ACM-Reference-Format}
\bibliography{references.bib}
\end{document}
\endinput
%%
%% End of file `sample-manuscript.tex'.
