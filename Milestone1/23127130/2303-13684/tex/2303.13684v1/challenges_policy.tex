\section{The Multicontextual Policy Challenge}

A student attending a lecture in VR from home is acting both as a student in the classroom and as a member of a household. This means that the student operates in multiple social contexts. These contexts are governed by different privacy norms~\cite{nissenbaumPrivacyContextTechnology2010}: data processing in the context of the home may clash with what is considered appropriate in the virtual classroom. For example, optical and depth sensors on the students' VR headsets map the environment. While this is necessary for navigating in the virtual classroom, the same sensors can capture other people in the vicinity~\cite{mcgillExtendedRealityXR2021}, such as family members or roommates.
A similar issue arose with the introduction of video conferences and online educational platforms at universities, which caused privacy violations due to a mismatch of students' expectations and data practices of the platforms~\cite{cohneyVirtualClassroomsReal2021}.

Existing privacy-enhancing approaches in VR are predominately based on the notions of transparency and ``informed consent''~\cite{EthicalConsiderationsExtended2021, XRSIPrivacySafety2022,xrassociationChapterDesigningImmersive2022}. Suggestions include ``contextual privacy communications'' at the time of data processing~\cite{XRSIPrivacySafety2022}, visualisation of privacy risks and customisable privacy settings ~\cite{kimVirtualRealityData2022}.  In doing so, the burden of understanding the risks and preventing harm in VR falls disproportionately on the user. This further contributes to ``[widening] the gap between the users’ understanding of the types of data collected, the implications of their consent and the actual implications''~\cite{kimVirtualRealityData2022}. 
 
The multicontextual nature of VR applications calls for privacy-enhancing mechanisms that incorporate privacy norms in multiple overlapping contexts. In our work, we draw on the theory of Contextual Integrity (CI)~\cite{nissenbaumPrivacyContextTechnology2010} to address these issues. Specifically, how can we ensure that privacy norms are respected in VR-based platforms and what CI-based methodologies could be employed by policymakers and designers?