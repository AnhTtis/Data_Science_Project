\section{Motivation}

The advent of Virtual Reality (VR) technologies fostered new modes of learning, connecting and socialising in virtual worlds that can defy the laws of physics. Bounded solely by VR creators' imaginations, users can seamlessly teleport themselves to a magical fairyland, a doctor's office, a lecture hall and other virtual spaces. Behind the VR ``magic'', however, lie a multitude of sensors that support core functions of VR devices, all the while generating vast amounts of data about VR users (e.g. body and eye movements, voice, facial expressions, biometrics, behaviour) and their surroundings (e.g. room layouts, objects, bystanders)~\cite{XRSIPrivacySafety2022}. Moreover, recent studies~\cite{nairExploringUnprecedentedPrivacy2022, hellerWatchingAndroidsDream2021} have shown that this data can reveal additional information about physiology, cognition and identity of a VR user.  As the VR technology advances, it is important to evaluate the potential privacy risks and harms \cite{abrahamImplicationsXRPrivacy2022,adamsEthicsEmergingStory2018}, before its mainstream adoption in contexts such as education~\cite{asieduMetaNextBig2022,yoshimuraStudyClassMeetings2021,combsMetaFundsVirtual2022},  workplace~\cite{pidelCollaborationVirtualAugmented2020} and healthcare\cite{pillaiImpactVirtualReality2019}.

In this position paper, we explore privacy challenges posed by VR technology in the education context based on a virtual classroom case study. Hailed as ``the New Era of Learning Experience''~\cite{MetaverseVirtualClassroom2022} by its creators, virtual classrooms take full advantage of the VR platform to augment the traditional learning experience, with a range of interactive activities and tools for collaboration. In these co-created virtual spaces, students and teachers can participate from anywhere in the world, seamlessly moving between different classrooms and activities.

While virtual classrooms mimic the real-world experience, the information collection practices behind VR extend beyond the normative expectation of a traditional classroom. The ostensible enhancement of the learning experience relies on a constant supply of VR users' data. Reminiscent of existing digital platforms, the extensive data collection, however, opens up the new education setting to a range of privacy and security threats~\cite{cohneyVirtualClassroomsReal2021,carterWhatAreRisks2021}. This is complicated by the fact that some data practices are necessary to enable core VR functionality. Furthermore, in VR platforms today informed consent mechanisms prevail that fail to adequately informing users and lack effective mechanisms of outing out. As VR users navigate a multitude of contexts, a different approach is needed to address privacy challenges in both policy and design which we articulate in this paper. 