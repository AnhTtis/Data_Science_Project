\section{VR Privacy-Preserving Design Challenges}

One of the main challenges in designing a privacy-preserving VR experience is maintaining contextual integrity of users' data. Specifically, this entails developing mechanisms to ensure that information flows in accordance with users' expectations and established societal norms. A suitable solution would need to differentiate between a multitude of nested contexts. For example, in a VR classroom, a conversation between fellow students is subject to different privacy norms than a meeting with the academic supervisor or a student counsellor. Privacy protections would also need to account for  different modes of interaction and be robust to accommodate the privacy norms associated with future VR adaptations. 

In search of answers to these challenges, the research community is  building tools to help users make \emph{meaningful} privacy-related choices. Examples of such efforts include: designing a VR-native interface for privacy settings, in lieu of a traditional text-based menu~\cite{yaoVirtualEquipmentSystem2021}; providing users with an option to engage in an ``incognito mode" to stop the flow of information to other VR users~\cite{yaoVirtualEquipmentSystem2021}; and informing the user about privacy risks through  a role-playing privacy tutorial that visually reenacts the information flow prescribed by a privacy policy~\cite{limMineYourselfRoleplaying2022}.

These privacy-preserving approaches, however helpful, still build on privacy models that embody  notions of information secrecy, control, personal anonymity and informed consent, despite their well-documented shortcomings~\cite{cateFailureFairInformation2006, schaubDesignSpaceEffective2015}. Here, too, with close coordination between all the relevant stakeholders of the VR ecosystem\footnote{We envision collaboration between VR users, designers, system engineers, developers, policymakers and others.}, the theory of CI can help inform the design of new privacy protection mechanisms based on the notion of appropriate information flows in accordance with enforced privacy norms. The CI-based approach would rely on tools for identifying information flows generated in the system~\cite{shvartzshnaiderVACCINEUsingContextual2019}  and prescribed by the policy~\cite{shvartzshnaiderGoingAppropriateFlow2019}, capturing users' expectations~\cite{shvartzshnaiderLearningPrivacyExpectations2016, apthorpeDiscoveringSmartHome2018}, and devising effective mechanisms of detecting deviation from contextual privacy norms in a VR environment.

\section{Acknowledgements}
This project has been funded by the Office of the Privacy Commissioner of Canada (OPC); the views expressed herein are those of the author(s) and do not necessarily reflect those of the OPC.
