\documentclass[11pt,aps,prd,superscriptaddress,amsmath,amssymb,nofootinbib]{revtex4-2}
\usepackage[T1]{fontenc}
\usepackage[utf8]{inputenc}
\usepackage[colorlinks=true, allcolors=blue]{hyperref}

\begin{document}

\title{Fully conservative $f(R,T)$ gravity and Solar System constraints}
\author{Nicolas R. Bertini}\email{nicolas.bertini@ufop.edu.br}
\affiliation{Departamento de Ciências Exatas e Aplicadas, Universidade 
Federal de Ouro Preto, Campus João Monlevade, João Monlevade, MG, Brazil}
\author{Hermano Velten}\email{hermano.velten@ufop.edu.br}
\affiliation{Departamento de Física, Universidade 
Federal de Ouro Preto, Campus Morro do Cruzeiro, Ouro Preto, MG, Brazil}



\begin{abstract}
The $f(R,T)$ gravity is a model whose action contains an arbitrary function of the Ricci scalar $R$ and the trace of the energy-momentum tensor $T$. We consider the separable model $f (R, T ) = \chi(R) + \varphi(T )$ and shown that, for perfect fluids, the dynamical equations are sufficient to determine how $\varphi$ depends on $T$, independently of the matter state equation and the geometry of space-time. Imposing the energy-momentum tensor conservation we obtain that $\varphi$ must be linear in $T$. However, the $T$ dependence is severely constrained using the full Will-Nordtvedt version of the Parameterized Post-Newtonian (PPN) formalism. The result of the PPN analysis is discussed and in addition it is shown that the diffeomorphism invariance of the matter action imposes strong constraints on conservative versions of $f(R,T)$ gravity.
\end{abstract}

\maketitle

\section{Introduction}

General Relativity is considered the correct description for the gravitational interaction due to its huge success in explaining available astronomical data at the solar system level and also for first predicting new gravitational phenomena posteriorly observed. On the other hand, the existence of the dark sector of the universe, consisting of dark matter and dark energy, is one of the most intriguing scientific challenges of the last decades and remains unsolved. By trusting that General Relativity is the correct theory for describing the gravitational interaction, specially above the galactic scales, the standard cosmological model demands that about $25\%$ of current cosmic energy budget should be composed by dark matter while other $70\%$ sums up the dark energy sector. So far, though there exists a large number of theoretically well motivated particles/fields candidates to compose the dark sector, there is no direct detection of them. Such lack of detection has motivated the search for a new theory in replacement of GR. Currently, this is a vast research field \cite{Clifton:2011jh, CANTATA:2021ktz}.

Either by modifying GR structures or by just adding new elements that yield to extended GR versions, all existing proposals should be subjected to the fact that there are no reasons to doubt GR validity at local scales i.e., the solar system one. Apart from analytical and numerical results for the behavior of a gravitational theory at the solar systems level, this analysis can also be technically performed via the so called post-Newtonian formalism.  

In this work we shall be concentrated on a widely known modified theory of gravity in which the gravitational action of the total theory depends on the trace of the energy momentum $T\equiv T^{\mu}_{\mu}$ in addition to the Ricci scalar $R$ \cite{Harko:2011kv}. In short notation, starting with $f(R,T)$ gravity, GR is fully recovered with $f(R,T)\equiv R$.
In the literature there are many distinct formulations for the functional dependence of $f(R,T)$. In this task, though not a necessary condition, one of the desired aspects is that the resulting dynamics should be conservative i.e., preserve the vanishing of the four-divergence of the energy momentum tensor, $\nabla^{\mu}T_{\mu\nu}=0$. Though the possibility that conservation can be violated, in the sense that $\nabla^{\mu}T_{\mu\nu} \neq 0$, the entire mechanism behind the post-Newtonian analysis demands conservation and has very constraining power on the functional form of $f(R,T)$.

At the level of cosmological solutions, it is known that energy-density conservation is restored as in the GR case if $f(R,T) = f_1(R)+f_2(T)$ where $f_2 = A+BT^{1/2}$, being $A$ and $B$ constants \cite{Alvarenga:2013syu}. 

The issue of conservative $f(R,T)$ models have also been explored in the context of spherically symmetric stellar configurations. In such analysis, it is demanded that the effective energy momentum tensor should be conserved, establishing a criteria for the allowed polytropic equation of state parameters \cite{dosSantos:2018nmu, Pretel:2021kgl}. 



Our goal in this work is to investigate the general feature of conservation in $f(R,T)$ theories at the level of the field equations, i.e., before setting a specific geometry. We show that the case $f_2 \propto T^{1/2}$ represents a particular ``cosmological'' case of the full conservative $f(R,T)$ theory. In addition, by applying the parameterized post-Newtonian (PPN) formalism we demonstrate that the existence of conservative $f(R,T)$ theories is challenged. The only exception is the trivial cosmological constant like case.
 

This work is organized as follows. In the next section we review the field equations for $f(R,T)$ gravity. In the third section we focus on discussing the role played by a minimally coupled $f(R,T)= f_1(R)+f_2(T)$ choice. We then work out the full conservative model at the field equations level and present one of the main results of this work. In Sec. IV we briefly review some essential PPN features and apply them to the case of minimally coupled $f(R,T)$ gravity. In Sec. V the results from the previous section are discussed and the observational bounds for the case developed in Sec. III are determined. In Sec. VI we consider the action invariance under diffeomorphisms and find the condition that must be satisfied for a general $f(R,T)$ theory to be conservative. Our conclusions are in Sec. VII.




\section{The action and field equations}
 
The $f(R,T)$ gravity was first proposed in \cite{Harko:2011kv}, and characterized  by an arbitrary dependence of the gravitational action on the Ricci scalar and the energy-momentum tensor trace $T = T^{\mu}_{\;\;\mu}$. The full action is 
\begin{align}
 S[g,\Psi] =  \frac{1}{2\kappa^{2}}\int f(R,T) \sqrt{-g}\,d^{4}x + S_\mathrm{m}[g,\Psi],\label{eq:action}
\end{align}
where $\kappa^{2} = 8\pi G$ being $G$ the gravitational coupling constant ($c=1$), $f(R,T)$ is an arbitrary function of the $R$ and $T$,  $S_\mathrm{m}$ is the action for the matter fields denoted collectively by $\Psi$. The energy-momentum tensor is obtained from $S_\mathrm{m}$ by the usual definition
\begin{align}
 T_{\mu\nu}\equiv -\frac{2}{\sqrt{-g}}\frac{\delta S_\mathrm{m}}{\delta g^{\mu\nu}}.\label{eq:TEM}
\end{align}


The variation of action \eqref{eq:action} with respect to the metric results in 
\begin{align}
 f_{_R}R_{\mu\nu} - \frac{1}{2}g_{\mu\nu}f + (g_{\mu\nu}\Box - \nabla_{\mu}\nabla_{\nu})f_{_R} = \kappa^{2} T_{\mu\nu} - f_{_T}(T_{\mu\nu} + \Theta_{\mu\nu}),\label{eq:FE1}
\end{align}
where $f\equiv f(R,T)$ and, for sake of simplicity, we defined
\begin{align}
f_{_R} \equiv \frac{\partial f(R,T)}{\partial R}, \quad  f_{_T} \equiv \frac{\partial f(R,T)}{\partial T} \quad\mathrm{and}\quad  \Theta_{\mu\nu} \equiv g^{\alpha\beta}\frac{\delta T_{\alpha\beta}}{\delta g^{\mu\nu}}.
\end{align}
 The equation for the energy-momentum tensor divergence, obtained directly from \eqref{eq:FE1}, is
\begin{align}
\nabla^{\mu}T_{\mu\nu} = \frac{f_{_T}}{\kappa^{2} - f_{_T}}\left[ -\frac{1}{2}\nabla_{\nu}T + (T_{\mu\nu}+\Theta_{\mu\nu})\nabla^{\mu}\ln f_{_T} + \nabla^{\mu}\Theta_{\mu\nu} \right].\label{eq:09}
\end{align}
Of course, $f_T=0$ restores the conservation. 
The energy momentum tensor for perfect fluids reads
\begin{eqnarray}
T_{\mu\nu} = p (u_{\alpha}u_{\beta}+g_{\alpha\beta})+  \varepsilon u_{\alpha}u_{\beta},\label{eq:TEM1}
\end{eqnarray}
where $p$ is the pressure, $\varepsilon$ is the energy density and $u^{\alpha}$ is the four velocity of the fluid element. In this case
\begin{align}
\Theta_{\mu\nu} = -2T_{\mu\nu} +g_{\mu\nu}p,
\end{align}
and $T$ depends on the energy density and pressure through
\begin{align}
T = -\varepsilon+3p.\label{eq:T}
\end{align}
Furthermore, the matter field dynamics is related to $f$ by the equation
\begin{align}
 \frac{1}{2\kappa^{2}}\int d^{4}x'\sqrt{-g'}\frac{\delta f'}{\delta \Psi} = -\frac{\delta S_\mathrm{m}}{\delta\Psi},\label{eq:matt}
\end{align}
where $x'$ denotes the integration variable. Therefore, it should be clear that $p$ and $\varepsilon$ in above equations depend on $f$, by construction.


\section{The case of a minimal coupling}


In this section we develop the general features of the gravitational field equations when the function $f(R,T)$ is written in the separable form $f(R,T)=f_1(R)+f_2(T)$. In this case the matter sector provided by the trace $T$ is minimally coupled to the geometric quantities in the gravitational Lagrangian.



Imposing both the vanishing of the four-divergence of $T_{\mu\nu}$ in Eq.~\eqref{eq:09} i.e., $\nabla^{\mu}T_{\mu\nu}=0$ and $f_{_{T}}-\kappa^{2}\neq 0$, and also taking into account that $\nabla^{\mu}\Theta_{\mu\nu} = \nabla^{\mu}(-2T_{\mu\nu} + g_{\mu\nu}p)=\nabla_{\nu}p$,  this yields to the constraint
\begin{eqnarray}
f_{_T}\left[ -\frac{1}{2}\nabla_{\nu}T + (g_{\mu\nu}p - T_{\mu\nu})\nabla^{\mu}\ln f_{_T} + \nabla_{\nu}p \right]=0.
\end{eqnarray}
Since $f_{_{T}}\nabla^{\mu}\ln f_{_{T}} = \nabla^{\mu}f_{_{T}}$ one has
\begin{eqnarray}
 -\frac{1}{2}(\nabla_{\nu}T) f_{_T} + (g_{\mu\nu}p - T_{\mu\nu})\nabla^{\mu} f_{_T} + (\nabla_{\nu}p) f_{_T}=0. \label{eq:4}
\end{eqnarray}
From this point one has to make a choice on the $f(R,T)$ function. Let us consider the minimal coupling case
\begin{eqnarray}\label{eqfRT}
f(R,T) = \chi (R)+\varphi (T).
\end{eqnarray}


According to Eq.~\eqref{eq:matt} it is technically possible to assume the dependence of $p$ on $\varepsilon$ and $\varphi(T)$. For any equation of state of the type $p=p(\varepsilon)$ admitting an inverse $\varepsilon=\varepsilon(p)$ one has
$T =  -\varepsilon(p) + 3p$, and therefore it is possible to write $p = p(T)$. This is an important assumption made in our work.

Now, let $\zeta \equiv\zeta(R,T)$ be any function of $R$ and $T$. Then, without loss of generality, the four-divergence of this quantity is directly computed as $\nabla_{\mu}\zeta = \nabla_{\mu}T \zeta_{_T} + \nabla_{\mu}R \zeta_{_R}$. Therefore, it is trivial to write down the quantities $\nabla_{\mu}f_{_T}=(\nabla_{\mu}T)f_{_{TT}}+(\nabla_{\mu}R)f_{_{RT}} = (\nabla_{\mu}T)\varphi_{_{TT}}$ and $\nabla_{\mu}p = (\nabla_{\mu}T)p_{_T}$. This allows one to rewrite \eqref{eq:4} into the form
\begin{eqnarray}
\left[ (g_{\mu\nu}p-T_{\mu\nu})\varphi_{_{TT}} + \frac{1}{2}g_{\mu\nu}(2p_{_T} -1)\varphi_{_T} \right]\nabla^{\mu}T=0.
\end{eqnarray}
Therefore, in general we have 
 \begin{align}
 (p\delta^{\mu}_{\nu}-T^{\mu}_{\nu})\varphi_{_{TT}} + \frac{1}{2}(2 p_{_T} -1)\varphi_{_T}\delta^{\mu}_{\nu}=0.\label{eq:phi}
 \end{align}



 

The $\mu=\nu=0$ and $\mu=i,\;\;\nu=j$ components of above equation are (in the comoving frame, $u^{i}=0$), respectively
\begin{align}
(p+\varepsilon)\varphi_{_{TT}}+ \left(p_{_T} - \frac{1}{2} \right)\varphi_{_T}=0,\label{eq:sol100}
\end{align}
\begin{align}
\left(p_{_T} - \frac{1}{2} \right)\varphi_{_T}\delta^{i}_{j}=0.\label{eq:sol1ij}
\end{align}


In the cosmological context, considering the state equation $p = w\varepsilon$, the solution obtained in the references \cite{Alvarenga:2013syu, Baffou:2013dpa} is consistent with that found considering only the equation \eqref{eq:sol100}: $\varphi(T) = A T^{\frac{1+3w}{2(1+w)}} + B$, for $w\neq 1/3$ and $A$ and $B$ being integration constants. As argued in \cite{Alvarenga:2013syu} this model should represent the only viable $f (R,T)$ theory, since it constitutes the only case in which the standard conservation law is preserved. However, the viability of this model in explaining cosmology background observables is discussed in \cite{Velten:2017hhf} and the findings of such reference represent a challenge to this model as a viable modification of gravity. However the approach in \cite{Alvarenga:2013syu, Baffou:2013dpa} is such that equation \eqref{eq:sol1ij} is not considered.

The solution of eq.~\eqref{eq:sol100} using the condition \eqref{eq:sol1ij}, even if one does not know the equation of state, is
\begin{align}
\varphi(T) = \sigma_1 T + \sigma_0,\label{eq:h(T)} \quad (\,p\neq - \varepsilon\,)
\end{align}
being $\sigma_1$ end $\sigma_0$ constants. The only condition behind the above result is $p\neq - \varepsilon$.

Concerning eqs.~\eqref{eq:sol100} and \eqref{eq:sol1ij} it is worth noting that, excluding the case where $\varphi_{_T}=0$, the choice of the comoving frame implies a relationship between $p$ and $T$ so that, given an equation of state $p=p(\varepsilon)$, the fluid equation of state parameter can be determined. For example, assuming $p=\omega\varepsilon$, eq.~\eqref{eq:sol1ij} implies $\omega = 1$, the stiff matter model.

On the other hand, the solution \eqref{eq:h(T)} is found for any arbitrary reference frame. To verify this one simply takes the trace of eq.~\eqref{eq:phi} and uses the result to eliminate $(2p_{_T}-1)\varphi_{_T}$. This procedure results in
\begin{equation}
    \left(T_{\mu\nu}-\frac{1}{4}g_{\mu\nu}T\right)\varphi_{_{TT}}=0\,.
\end{equation}
The above equation can be satisfied in two cases: $i$) if $\varphi (T)$ is given by \eqref{eq:h(T)}; or $ii$) if the term in parentheses is zero. The second case occurs, in general, if $p=-\varepsilon$. This relation corresponds to the general relativistic solution for a cosmological constant which is equivalent to the case $\varphi = const$. However, the relation \eqref{eq:h(T)} is even more general by including this case.


\section{Post-Newtonian expansion}

In this section we apply the Will-Nordtvedt PPN formalism  \cite{Will:1993ns, Will:2014kxa} to $f(R,T)$ gravity in its minimally coupled form \eqref{eqfRT}. In this formalism, a perfect fluid is the source of the gravitational field. It describes the metric of a gravitational theory in terms of ten observable PPN parameters in a theory-independent way. The main small parameter of this formalism is the matter velocity field $|\vec v| = v < 1$. The metric is expanded about Minkowski spacetime,
\begin{equation} \label{eq:getah}
g_{\alpha\beta}=\eta_{\alpha\beta}+h_{\alpha\beta}\,,
\end{equation}
where  $\eta_{\alpha\beta}$ is the Minkowski metric, which is of zeroth-order on $v$, and $h_{\alpha \beta} \sim O(v^2)$, at least. We use the signature $(-,+,+,+)$. 


Up to the first post-Newtonian order, the metric must be known as follows: $g_{00}$ to order $v^4$, $g_{0i}$ to order $v^3$ and $g_{ij}$ to order $v^2$ (Latin indices run from $1$ to $3$). Thus, up to the required order, the Ricci tensor components can be expressed as 
\begin{align}
R_{00} =& -\frac{1}{2}\nabla^2h_{00} - \frac{1}{2}\left(h^k_{~k,00}- 2\,h^k_{~0,k0} \right) - \frac{1}{4}\,|\vec{\nabla}h_{00}|^2 \ +\nonumber\\[1ex]
&+\frac{1}{2}\,h_{00,l}\left(h^{lk}_{~~,k}- \frac{1}{2}\,h^k_{~k,j}\delta^j_l\right)+ \frac{1}{2}\,h^{kl}h_{00,lk}\,,\label{1}\\[2ex]
R_{0i}=& -\frac{1}{2}\left(\nabla^2h_{0i} - h^k_{~0,ik} + h^k_{~k,0i} - h^k_{~i,k0} \right)\,,\label{2}\\[2ex]
R_{ij}=& -\frac{1}{2}\!\left(\nabla^2h_{ij} -\! h_{00,ij} +\! h^k_{~k,ij}-\! h^k_{~i,kj}- \! h^k_{~j,ki} \right).\label{3}
\end{align}
The comas refer to partial derivatives, $\nabla^2 \equiv \eta^{ij}\partial_i\partial_j$, and it is used that time derivatives effectively yield to a higher order in the expansion. Thus, if a quantity $X$ is of order $v^n$ then $X,_k\sim O(v^n)$ and $X,_0\sim O(v^{n+1})$.

Since the gravitational source is  a perfect fluid, the energy-momentum tensor is the one given in \eqref{eq:TEM1} but the energy density is decomposed into the mass density $\rho$ and the specific energy density $\Pi$ in the form
\begin{equation}\label{emt}
\varepsilon = \rho+\rho\Pi.
\end{equation}
 The four velocity of the fluid element $u^\mu=(u^0,v^i)$, with
\begin{equation}\label{u0}
u^0= \sqrt{\frac{1+v^2}{1-h_{00}}}\,,
\end{equation}
such that $u^\mu u_\mu=-1$. The density $\rho$, $\Pi$ and $p/\rho$ are of order $v^2$ \cite{Will:1993ns}.

Substituting \eqref{eqfRT}, with $\chi(R)=R$, into \eqref{eq:FE1} we have
\begin{align}
 R_{\mu\nu} - \frac{1}{2}g_{\mu\nu}R =  \frac{1}{2}g_{\mu\nu}\varphi  + 8\pi G T_{\mu\nu} - \varphi_{_T}(T_{\mu\nu} + \Theta_{\mu\nu}).\label{eq:16}
\end{align}
Using the trace of the above equation to eliminate $R$, one obtains
\begin{align}
 R_{\mu\nu} = -\frac{1}{2}g_{\mu\nu} \varphi + 8\pi G \left( T_{\mu\nu} - \frac{1}{2}g_{\mu\nu}T \right) - \varphi_T \left( T_{\mu\nu} + \Theta_{\mu\nu} - \frac{1}{2}g_{\mu\nu}(T+\Theta) \right).\label{eq:FE2}
\end{align}

By applying the post-Newtonian expansion in the above expression it has to be considered
\begin{align}
    \varphi_n(T) = \sum^{n}_{i=0}\sigma_{i}T^{i}.
\end{align}

With the expressions above, we expand equation \eqref{eq:FE2} to calculate the metric components order by order on powers of $v$.  As a first step, the zeroth order equation in $v$ trivially leads to 
\begin{equation}
    \sigma_{0}=0.
\end{equation}
This is expected since the constant $\sigma_{0}$ in eq.~\eqref{eq:h(T)} necessarily leads to non-asymptotically flat spacetimes, like the cosmological constant in general relativity, which is not considered in the standard PPN Solar System analysis. This is also physically reasonable since, up to first Post-Newtonian order (1PN) and considering its value as inferred from the cosmological observations, it has negligible impact on the Solar System dynamics \cite{Sereno:2006re}. The next steps are described below.
\begin{itemize}
\item{$h_{00}$ up to order $v^2$ (Newtonian limit):}
Up to the required order one finds
\begin{equation}
R_{00}=-\frac{1}{2}\,\nabla^2h_{00},\;\; \varphi_{1} = \sigma_{1}T, \;\; T_{00}=-T=\rho\, \quad \mbox{and} \quad \Theta_{00} = -\Theta = 2\rho.
\end{equation}
Inserting this into eq.~\eqref{eq:FE2} we find that the terms containing $\sigma_{1}$ cancel out, therefore
\begin{equation}
    \nabla^{2}h_{00}= - 8\pi G\rho\,.
\end{equation}
In order to be in agreement with the local Newtonian gravity it is demanded then,
\begin{equation}
    h_{00}=2U,
\end{equation}
where $U$, the negative of the Newtonian potential\footnote{For conciseness, commonly we will call $U$ the Newtonian potential, without writing ``negative'' in front of it. We use $U$ since we are following the notation of Ref.~\cite{Will:1993ns} on the PPN parameters and the potentials.} \cite{Will:1993ns}, given by
\begin{equation}
    U(t, \vec{x}) = G_\mathrm{N} \int \frac{\rho(t, \vec{x} \, ')}{| \vec{x} - \vec{x} \, '| } d^3x' \, ,
\end{equation}
being $G_\mathrm{N}$ the Newtonian gravitational constant. Thus we must set
\begin{equation}
    G=G_\mathrm{N}=1.
\end{equation}

\item{$h_{ij}$ up to order $v^2$:}
Imposing the three gauge conditions, 
\begin{equation}
    h^{\mu}_{\;i,\mu} - \frac{1}{2} h^{\mu}_{\;\mu,i} = 0,\label{eq:gauge1}
\end{equation}
the spatial part of eq.~\eqref{eq:FE2} reduces to
\begin{equation}
    -\frac{1}{2}\nabla^{2}h_{ij} = 4\pi\rho\delta_{ij} + \sigma_{1}\rho\delta_{ij}.
\end{equation}
This equation is easily integrated, providing
\begin{equation}
    h_{ij} = 2 \left( 1+ \frac{\sigma_1}{4\pi} \right)U\delta_{ij}.
\end{equation}

\item{$h_{0i}$ up to order $v^3$:}
With the fourth gauge condition
\begin{equation}
    h^{\mu}_{\;0,\mu} - \frac{1}{2}h^{\mu}_{\;\mu,0}=\frac{1}{2}h_{00,0}\,,\label{eq:gauge2}
\end{equation}
the eq.~\eqref{eq:FE2} becomes
\begin{eqnarray}
\nabla^{2}h_{0j} + U_{,0j} = 16\pi\left(1+\frac{h_1}{8\pi} \right)\rho v_{j}.
\label{eq:Dh0j}
\end{eqnarray}
The above equation can be integrated using the auxiliary potential $\chi(t,\vec{x})$ \cite{Will:1993ns}, given by
\begin{equation}
    \chi (t,\vec{x})\equiv\int \rho(t,\vec{x}')|x-\vec{x}'|d^{3}x'.
\end{equation}
From this definition one can write
\begin{equation}
    \nabla^{2}\chi = -2U \quad \mathrm{and} \quad \chi_{,0j} = V_j - W_j,
\end{equation}
where
\begin{equation}
    V_j = \int \frac{\rho(x-\Vec{x}')v'_{i}}{|x-\vec{x}'|}d^{3}x'\,, \quad \nabla^{2}V_j = -4\pi\rho v_j\,.
\end{equation}
Therefore, eq.~\eqref{eq:Dh0j} gives 
\begin{eqnarray}
h_{0j} = -\left( \frac{7}{2} +\frac{h_1}{2\pi} \right)V_j - \frac{1}{2}W_j\,.
\end{eqnarray}

\item{$h_{00}$ up to order $v^4$:}
To the required order
\begin{equation}
    \varphi_{2} = \sigma_{1}(-\rho +3p)+\sigma_{2}\rho^{2},
\end{equation}
and for the energy-momentum tensor, one finds
\begin{equation}
    T_{00} - \frac{1}{2}g_{00}T = \rho \left( v^{2} - U +\frac{\Pi}{2} + \frac{3p}{2\rho} \right).
\end{equation}
\end{itemize}
With the above relations the $0-0$ component of eq.~\eqref{eq:FE2} gives
\begin{align}
R_{00} =  - 4\pi\rho v^{2}\left( 2+\frac{\sigma_1}{4\pi} \right)-4\pi\rho U\left( -2+\frac{\sigma_{1}}{\pi}\right) -4\pi\rho\Pi
%\nonumber\\
 - 4\pi p\left(3 + \frac{\sigma_{1}}{2\pi} \right) + 4\pi\rho^{2}\frac{\sigma_{2}}{8\pi}.\label{eq:R00}
\end{align}
Let us now find the $0-0$ component of the Ricci tensor. Considering the post-Newtonian potentials introduced in \cite{Will:1993ns}
\begin{align}
\nabla^{2}\phi_{1}  = -4\pi\rho v^{2}, \quad \nabla^{2}\phi_{2}  = -4\pi \rho U,
\\
\nabla^{2}\phi_{3}  = -4\pi \rho\Pi, \quad \nabla^{2}\phi_{4}  = -4\pi p,
\end{align}
we take into account the gauge conditions \eqref{eq:gauge1} and \eqref{eq:gauge2}, and using the relation $|\vec{\nabla}U| = \nabla^{2}(U^{2}/2 - \phi_2)$, the Ricci tensor component reads
\begin{equation}
    R_{00} = -\frac{1}{2}\nabla^{2}(h_{00}+2U^{2}-8\phi_2).
\end{equation} 
With the above results, eq.~\eqref{eq:R00} becomes
\begin{eqnarray}
h_{00} = -2U^{2} + \left( 4+\frac{\sigma_{1}}{2\pi} \right)\phi_{1} + \left(4+\frac{2\sigma_{1}}{\pi} \right)\phi_{2} + 2\phi_3 + \left( 6+\frac{\sigma_{1}}{\pi} \right)\phi_{4} - \frac{\sigma_{2}}{4\pi}{\cal T},\label{eq:h00}
\end{eqnarray}
where ${\cal T}$ is a new post-Newtonian potential defined as
\begin{eqnarray}
 {\cal T}(t,\vec{x}) \equiv \int \frac{[\rho(t,\vec{x}')]^{2}}{|x-\vec{x}'|}d^{3}x', \quad       \nabla^{2}{\cal T} = -4\pi\rho^{2}\,.
\end{eqnarray}

With the above, we conclude the expansion of the separable $f(R,T)$ gravity as a function of the PPN potentials. In the next section, we infer the values of the PPN
parameters and compare them with corresponding observational values.

\section{The PPN parameters in $f(R,T)$ gravity}

The standard Will-Nordvedt PPN formalism \cite{Will:1993ns, Will:2014kxa} does not include the last term in eq.~\eqref{eq:h00}. However, the realization $\varphi_2 (T) = \sigma_0 + \sigma_1 T + \sigma_2 T^{2}$ does not keep null the four-divergence of the field equations \eqref{eq:FE2} due to the term $\propto T^{2}$. Then, we assume
\begin{equation}
    \sigma_{2}=0.
\end{equation}
With this, and the results obtained in the previous section, the metric up to the first post-Newtonian order can be written as
\begin{eqnarray}
g_{00} = 2U -2U^{2} + \left( 4+\frac{\sigma_{1}}{2\pi} \right)\phi_{1} + \left(4+\frac{2\sigma_{1}}{\pi} \right)\phi_{2} + 2\phi_3 + \left( 6+\frac{\sigma_{1}}{\pi} \right)\phi_{4}\,,\label{eq:g00}
\end{eqnarray}
\begin{eqnarray}
g_{0j} = -\left( \frac{7}{2} +\frac{\sigma_1}{2\pi} \right)V_j - \frac{1}{2}W_j\,,\label{eq:g0j}
\end{eqnarray}
\begin{eqnarray}
g_{ij} = \left[1+ 2\left( 1+ \frac{\sigma_1}{4\pi} \right) U\right]\delta_{ij}\,.\label{eq:gij}
\end{eqnarray}
To extract the PPN parameters from the above metric components we compare it to the Will-Nordvedt general post-Newtonian metric \cite{Will:1993ns}
\begin{align}
g_{00} =& -1 + 2U - 2\beta U^2 + (2 \gamma +2+\alpha_3 +\zeta _1-2 \xi ) \phi_1 
 + 2(3 \gamma -2\beta+1+\zeta _2+ \xi ) \phi_2  \nonumber \\[1ex]
& +2(1+\zeta _3 ) \phi_3 \ + 2(3 \gamma +3\zeta _4-2 \xi ) \phi_4 - (\zeta _1-2 \xi ) {\cal A}  -2\xi \phi_w,  \\[2ex]
g_{0i} =& - \frac{1}{2}(4 \gamma +3+\alpha_1-\alpha_2+ \zeta_1-2\xi) V_i - \frac 1 2(1+\alpha_2- \zeta_1+2\xi) W_i\,, \\[2ex]
g_{ij} =& \ (1+2\gamma \,U)\, \delta_{ij}\,.
\end{align}
From this comparison we obtain the following constraints
\begin{eqnarray}
\beta = 1,\quad \gamma = 1 + \frac{\sigma_1}{4\pi}, \quad \alpha_{1} = \alpha_{2} = \alpha_{3} = 0, \quad \xi = 0, \\
\zeta_{1}=0, \quad \zeta_{2} = \frac{\sigma_{1}}{4\pi}, \quad \zeta_{3} = 0, \quad \zeta_{4} = -\frac{\sigma_{1}}{2\pi}.
\end{eqnarray} 
The observational bounds for all PPN parameters are shown in Table \ref{tab}. The strongest limit for $\sigma_1$ comes from the parameter $\gamma$, which is related to how much curvature is produced per unit rest mass, and states that
\begin{equation}\label{sigma1result}
    |\sigma_1|< 2.9 \times 10^{-4}.
\end{equation}
Since all parameters $\alpha_{i}$ are null, the model does not predict preferred-frame effects. On the other hand, non-vanishing values for the  parameters $\zeta_2$ and $\zeta_4$ indicate a violation of the total momentum conservation. This classifies the model as semi-conservative \cite{Will:2014kxa}.

There are well known examples of theories that come from an action and have $\alpha_{1}$ and $\alpha_{2}$ different from zero, but theories with an action are not expected to yield non-zero values for any of the $\zeta$'s and $\alpha_3$ if $\xi=0$ \cite{Lee1685}. A further example of a theory with this characteristic can be seen in \cite{Toniato:2017wmk}.


\begin{table}[ht]
	\centering 
		\begin{tabular}{ c c || c c }
			\toprule
			Parameter & Limit & Parameter & Limit \\
			\hline
			$\gamma-1$& $2.3 \times 10^{-5}$& $\xi$     & $4. \times 10^{-9}$\\
			$\beta-1$ & $8. \times 10^{-5}$ & $\zeta_1$ & $2. \times 10^{-2}$\\
			$\alpha_1$& $4. \times 10^{-5}$ & $\zeta_2$ & $4. \times 10^{-5}$\\ 
			$\alpha_2$& $2. \times 10^{-9}$ & $\zeta_3$ & $1. \times 10^{-8}$\\
			$\alpha_3$& $4. \times 10^{-20}$& $\zeta_4$ & --- \\
			\botrule
		\end{tabular}
    \caption{\label{tab} Limits on the PPN parameters, considering only the strongest limits for each parameter \cite{Will:2014kxa}. The $\zeta_4$ does not have a direct measurement. These limits apply to the absolute value of each parameter.}
\end{table}


\section{Fully conservative model}


The PPN analysis shows that, despite implying $\nabla^{\mu}T_{\mu\nu}=0$ at the field equations level, the model with a linear dependence on $T$ is severely constrained. One can further analyze this by considering the invariance of the theory under diffeomorphisms.


 First, we can consider that the energy-momentum tensor does not depend on derivatives of the matter fields with respect to space-time, as is the case with perfect fluids. Therefore ${\delta f'}/{\delta \Psi} = (\partial f'/\partial\Psi') \delta^{(4)}(x-x')$, and the eq.~\eqref{eq:matt} becomes
\begin{align}
\frac{\delta S_{\mathrm{m}}}{\delta\Psi}= -\frac{1}{2\kappa^{2}}\frac{\partial T}{\partial \Psi}f_{_T}\sqrt{-g}.\label{eq:matter}
\end{align}
 Imposing that the theory is diffeomorphism-invariant \cite{Wald:1984rg}, one obtains 
\begin{align} 
\int_{\cal M} \left[ (\nabla^{\nu}T_{\mu\nu})\xi^{\nu}\sqrt{-g} + \frac{\delta S_\mathrm{m}}{\delta\Psi}\delta_\xi\Psi \right]d^{4}x=0, \label{eq:diffeo}
\end{align}
where $\delta_\xi \Psi = \xi^{\mu}\partial_{\mu}\Psi$ represents an infinitesimal coordinate change given by the Lie derivative along the vector $\xi^{\mu}$. Plug-in \eqref{eq:matter} and \eqref{eq:diffeo}, we gets
\begin{align}
\nabla^{\nu}T_{\mu\nu} =   \frac{1}{2\kappa^{2}}f_{_T}\nabla_{\mu}T .\label{eq:invdiff}
\end{align}
Therefore, any realization for $f(R,T)$ that leads to $\nabla_{\mu}T^{\mu\nu}=0$ must be such that
\begin{equation}
    f_{_{T}}(R,T)=0.
\end{equation}
Then, $\varphi$ constant is the only solution that simultaneously satisfies \eqref{eq:sol100}, \eqref{eq:sol1ij} and \eqref{eq:invdiff} (with $\nabla_{\mu}T^{\mu\nu}=0$).



\section{Final discussion}

Modified gravity theories can display many distinct new geometrical structures such as e.g., violation of the usual energy-momentum conservation. This is particularly observed in the cosmological context where the non-conservation leads to a clear signature on the evolution of the pressureless (dark) matter component, the main ingredient for structure formation.
One of the most known example of that is the class of $f(R,T)= f_1(R)+f_2(T)$ theories where $T$ is the trace of the energy momentum tensor. For this theory, assuming a power law dependence $f_2(T) \propto T^n$, the standard cosmological conservation law is recovered when $n=1/2$. We have shown however this is a sub case of a more general notion of conservation in $f(R,T)$ gravity. Only in the expanding cosmological background the solution $f_2(T) \propto T^{1/2}$ can be seen as the full conservative one.

We then look for a general form of the $f(R,T)$ function such that conservation is preserved at the  field equations level. We firstly find the function $f_2(T)=\sigma_0+\sigma_1 T $. We then turn our attention to test this function with solar system data via the Post-Newtonian formalism. At the end of the day, the parameter $\sigma_1$ is severely constrained as in  \eqref{sigma1result}. Thus, we conclude that only a trivial constant term added to the gravitational Lagrangian is consistent with the full conservation of the energy momentum tensor and also with solar system tests.   

It is also important to mention the discussion on possible approaches to $f(R,T)$ gravity with minimal coupling that can be followed in \cite{Fisher:2019ekh,Harko:2020ivb,Fisher:2020zwx}. One can modify the Lagrangian of $f(R,T)$ theories by including Lagrange multipliers that lead to $\nabla_{\mu}T^{\mu\nu}=0$ at the field equations level, as in \cite{Fisher:2019ekh}. On the other hand, one can also investigate which particular realization of $f(R,T)$ agrees with $\nabla_{\mu}T^{\mu\nu}=0$ by keeping the standard gravitational Lagrangian as in this reference and also as argued by Harko \& Moraes \cite{Harko:2020ivb} (see also \cite{dosSantos:2018nmu, Pretel:2021kgl}). In any case, the fundamental thermodynamical variables of matter ($\rho$ and $p$) are introduced in $S_m$ and are the same that also appear in the gravitational part of the total Lagrangian, through $T$. Considering the model with minimal coupling, $f(R,T)=f_1(R)+f_2(T)$, including $f_2(T)$ in the matter sector is indeed a technically possible approach, but the new effective thermodynamical variables appearing in the effective matter Lagrangia will have a different meaning comparing to those included in $S_m$, and can be distinguished from the fundamental variables. Imposing $\nabla_{\mu}T^{\mu\nu}=0$ we conclude that the cosmological constant is the favored result, and is part of $f_2(T)$. Including the cosmological constant in the energy density or in the geometric part is a choice, but in any case its value can be unequivocally determined by its physical effects. Therefore, we emphasize that the search for the functional form of $f_2(T)$ and the observational constraints is valid, contrarily to argued in Ref. \cite{Fisher:2019ekh,Fisher:2020zwx} and that the result using the PPN analysis is consistent with the imposition set by Eq. \eqref{eq:invdiff}, for the case of minimal coupling.


\bibliographystyle{apsrev4-1}
\bibliography{MSbib}

\end{document}
