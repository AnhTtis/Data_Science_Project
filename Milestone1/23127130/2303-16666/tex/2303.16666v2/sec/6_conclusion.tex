\section{Conclusion}
%In this paper, we proposed a VAE variant, termed SC-VAE.
%The proposed method leverages an unrolled sparse coding learning algorithm to obtain latent sparse representations for the input image. The SC-VAE enabled the gradient flow through the dictionary and overcame the shortcomings of existing VAE models. In the experiments, we demonstrated that SC-VAE performed favorably compared to popular baseline methods in image reconstruction on two benchmark datasets. Additionally, 
%we demonstrated that similar image patches correspond to similar learned sparse codes. This allowed us to perform unsupervised image segmentation. The demonstrated advantages of SC-VAE compared to standard VAE approaches may enable better performance in downstream tasks, such as object recognition, image segmentation, and image retrieval. In future research, we will explore the potential of combining SC-VAE with a transformer model to investigate its ability to generate high-quality images. Additionally, we intend to expand our research into (weakly) supervised scenarios.
In this paper, we introduced a novel variant of VAE, which we refer to as SC-VAE. Our approach harnesses a learned ISTA algorithm to learn latent sparse representations for input images. The utilization of SC-VAE facilitates the smooth flow of gradients through the network, effectively addressing limitations present in existing VAE models. Through our experiments, we showcased the superior performance of SC-VAE compared to well-established baseline methods in image reconstruction. 
Furthermore, we illustrated that SC-VAE can generate new images through manipulating and interpolating sparse code vectors. Moreover, 
SC-VAE's ability to learn relationship between image patches enabled us to perform unsupervised image segmentation, coupled with its resilience to noise.
%SC-VAE's ability to learn relationship between image patches using sparse codes allows us to perform unsupervised image segmentation associated with its noise robustness property. %We also demonstrate the noise robustness property of SC-VAE for unsupervised image segmentation. 
%The proposed SC-VAE has the potential to enhance performance in downstream tasks such as image classification, image retrieval and object detection.
%For future work, we plan to explore the potential of combining SC-VAE with a Transformer model to assess its capacity for generating high-fidelity images. %Additionally, we aim to broaden the scope of our research by delving into (weakly) supervised scenarios.
