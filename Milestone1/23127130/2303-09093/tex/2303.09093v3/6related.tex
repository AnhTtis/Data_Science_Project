\section{Related Work}
As the major contribution of our work is introducing a new large-scale event detection dataset, we review the existing datasets available for event extraction and the various ways they were created. 
\paragraph{Event Extraction Datasets}
ACE05 set the event extraction paradigm which consists of event detection and argument extraction. It is also the most widely used event extraction dataset to date.
The more recent MAVEN~\cite{wang-etal-2020-maven} dataset has a larger ontology selected from a subset of FrameNet~\cite{baker-etal-1998-berkeley-framenet}.
FewEvent~\cite{Deng_2020FewEvent} is a compilation of ACE, KBP, and Wikipedia data designed for few-shot event detection. 
A number of datasets (DCFEE~\cite{yang-etal-2018-dcfee}, ChFinAnn~\cite{zheng-etal-2019-doc2edag}, RAMS\cite{ebner-etal-2020-multi}, WikiEvents~\cite{li-etal-2021-document}, DocEE~\cite{tong-etal-2022-docee}) have also been proposed for document-level event extraction, with a focus on argument extraction, especially when arguments are scattered across multiple sentences. 
Within this spectrum, \datasetname\ falls into the category of event detection dataset with a heavy focus on wider coverage of event types.

\paragraph{Weak Supervision for Event Extraction}
Due to the small size of existing event extraction datasets and the difficulty of annotation, prior work has attempted to leverage distant supervision from knowledge bases such as Freebase\cite{chen-etal-2017-automatically, Zeng2017ScaleUE} and WordNet~\cite{araki-mitamura-2018-open,tong-etal-2020-improving}.
The former is limited by the number of compound value types (only $\sim$20 event types are used in both works) and the latter does not perform any typing.
Weak supervision has also been used to augment training data for an existing ontology
with the help of a masked language model~\cite{yang-etal-2019-exploring} or adversarial training~\cite{wang-etal-2019-adversarial-training}. 
Our dataset is constructed with the help of the DWD Overlay mapping which is defined between event types (Qnodes) and PropBank rolesets instead of on the level of concrete event instances (as in prior work that utilizes knowledge bases). 



