\section{Data Filtering and Annotation Details}
\label{sec:appendix_annotation}
Table \ref{tab:removed_nodes} shows some examples of removed Qnodes from DWD Overlay in our ontology with different kinds of reasons.

Figure \ref{fig:annotation_interface} displays our annotation interface. The left box features the context, highlighting a single trigger, while the right box enumerates candidate event types, expressed as a combination of name and description, with an extra choice labeled "None of the above options is correct." The annotator's task is to select the option that most accurately represents the trigger word.


\begin{table*}[]
\small
\centering
\begin{tabular}{m{5em} m{9em} m{6em} m{26em}}

\toprule
Qnode & Name & Roleset & Description \\
\midrule
\midrule
\multicolumn{4}{c}{Removed during cleaning the heavily used rolesets}\\
\midrule
Q2536390 & abdominal\_distention & ill.01 & Physical symptom\\
Q192989 & acculturation & change.01 & process of cultural and psychological change\\
Q422268 & actinomycosis & ill.01 & Human disease\\
Q1319035 & adult\_education & educate.01 & form of learning adults engage in beyond traditional schooling\\
Q9363879 & stamping & make.01 & metalworking\\
Q615857 & stapedectomy & surgery.01 & surgical procedure of the middle ear performed to improve hearing\\
Q366774 & adrenalectomy & remove.01 & surgical removal of the adrenal gland\\
Q2035485 & subcutaneous\_injection & inject.01 & Medical procedure\\
\midrule
\midrule
\multicolumn{4}{c}{Removing reason: cognitive events that do not involve any physical state change}\\
\midrule
Q241625 & wish & wish.01 & desire for a specific item or event\\
Q26256512 & want & want.01 & economic term for something that is desired\\
Q26253999 & yearning & yearn.01 & deep and aching desire for someone or something\\
Q706622 & intention & intend.01 & mental state representing commitment to perform an action\\
Q3027692 & differentiation & differentiate.01 & process by which two closely related linguistic varieties 
diverge from one another during their evolution\\
Q104776298 & crosspatch & grouch.01 & a person who is easily annoyed\\
Q516519 & suspicion & suspect.01 & emotion\\
Q659974 & trust & trust.02 & assumption of and reliance on the honesty of another party\\
\midrule
\midrule
\multicolumn{4}{c}{Removing reason: mapped to too general or ambiguous rolesets}\\
\midrule
Q105606485 & intellectual\_activity & think.01 & human activity comprising of mental actions\\
Q2944236 & photosensitivity & see.01 & Light sensitivity in homo sapiens\\
Q9174 & religion & believe.01 & set of beliefs, practices and traditions for a group or community\\
Q16513426 & decision & decide.01 & result of deliberation\\
Q2827815 & international\_aid & give.01 & voluntary transfer of resources from one country to another\\
Q9081 & knowledge & know.01 & experience or education by perceiving, discovering, or learning\\
Q1221208 & employment\_contract & agree.01 & agreement between employer and employee on terms of work and compensation\\
Q56274009 & looking & look.01 & act of intentionally focusing visual perception on someone or something\\
\midrule
\midrule
\multicolumn{4}{c}{Removing reason: low frequency}\\
\midrule
Q379788 & advection & advect.01 & transport of a substance by bulk motion\\
Q381105 & aeration & aerate.01 & process of circulating or mixing air with water\\
Q104541 & aerosol & aerosolize.01 & colloid of fine solid particles or liquid droplets, in air or  another gas\\
Q623179 & state\_terrorism & terrorism.03 & acts of terrorism against individuals conducted by organs of a state\\
Q98394474 & stenciling & stencil.01 & artistic technique for transferring images using stencils\\
Q844613 & sintering & sinter.01 & process of forming material by heat or pressure\\
Q249697 & eulogy & eulogize.01 & speech in praise of a person, usually recently deceased\\
Q901882 & interface & interface.01 & boundary between different phases of matter\\
\bottomrule
\end{tabular}

\caption{
Examples of removed Qnodes in XPO Overlay. Note that one node can be removed due to multiple reasons.
}
\label{tab:removed_nodes}
\end{table*}%












\begin{figure*}[ht]
    \centering
    \includegraphics[width=\linewidth]{figures/GLEN_annotation_interface.png}
    \caption{Annotation interface built with Amazon Mechanical Turk for labeling the development and test set.}
    \label{fig:annotation_interface}
\end{figure*}



\section{Baseline Implementation Details}
\label{sec:baseline_implementation}
\begin{table*}[th]
\centering
    \small 
    \begin{tabular}{l | c c c c  }
    \toprule 
    Model & DMBERT & SpanCls & TokCls & ZED   \\
    \midrule
    Training epochs & 10 & 10 & 10 & 10 \\
    Batch size &  64 & 64  & 16 & 16 \\
    Negative samples & 5 & 5 & - & 5 \\
    Learning rate & {5e-5} & 5e-5 & 2e-5 & 2e-5 \\
    Max sequence length & \multicolumn{4}{c}{64} \\
    Base Model & \multicolumn{4}{c}{Bert-base-uncased}\\
    Weight decay & \multicolumn{4}{c}{0.0}\\
    Scheduler & \multicolumn{4}{c}{Linear} \\
    \bottomrule
\end{tabular}
\caption{Hyperparameter settings for baseline models.}
\label{tab:baseline_param}
\end{table*} 

We list the hyperparameters for our baseline models in Table \ref{tab:baseline_param}. 
For ZED, we follow the original paper and set the margin to 0.2. We set the threshold for predicting an event type to 0.3 (if the cosine similarity between the event type representation and the trigger representation is smaller than this value, we will refrain from predicting any event type). 

For the InstructGPT baseline, we use the \texttt{text-davinci-003} model with a temperature of 0.2 and \texttt{top\_p} set to 0.95 for decoding. We show our detailed prompt in Figure \ref{fig:gpt-prompt}. The first part of the prompt is the task instruction and then we include 32 input-output examples. Due to the current input length limit of InstructGPT, we were unable to feed the ontology into the model as part of the input. 


\begin{figure*}
    \centering
    \includegraphics[width=\linewidth]{figures/GPT-Prompt.pdf}
    \caption{Truncated version of our prompt to InstructGPT. }
    \label{fig:gpt-prompt}
\end{figure*}

