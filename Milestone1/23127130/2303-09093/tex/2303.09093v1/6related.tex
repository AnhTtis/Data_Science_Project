\section{Related Work}
\paragraph{Event Extraction Datasets}
ACE05 set the event extraction paradigm which consists of event detection and argument extraction. It is also the most widely used event extraction dataset to date.
The more recent MAVEN~\cite{wang-etal-2020-maven} dataset has a larger ontology selected from a subset of FrameNet~\cite{baker-etal-1998-berkeley-framenet}.
FewEvent~\cite{Deng_2020FewEvent} is a compilation of ACE, KBP, and Wikipedia data constructed with a focus on few-shot event detection. 
A number of datasets (DCFEE~\cite{yang-etal-2018-dcfee}, ChFinAnn~\cite{zheng-etal-2019-doc2edag}, RAMS\cite{ebner-etal-2020-multi}, WikiEvents~\cite{li-etal-2021-document}, DocEE~\cite{tong-etal-2022-docee}) have also been proposed for document-level event extraction, with a focus on argument extraction, especially when arguments are scattered across multiple sentences. 

\paragraph{Weak Supervision for Event Extraction}
Due to the small size of existing event extraction datasets and the difficulty of annotation, a series of prior work has attempted to leverage weak supervision to increase the amount of labeled data for event extraction. Extending the idea of automatic annotation from KB alignment for relation extraction~\cite{mintz-etal-2009-distant}, \cite{chen-etal-2017-automatically, Zeng2017ScaleUE} utilize compound value types in Freebase knowledge graph as distant supervision. However, the number of compound value types is quite limited (only $\sim$20 event types are used in both works) and the arguments only cover named entities.
WordNet provides a larger source of possible event triggers~\cite{araki-mitamura-2018-open,tong-etal-2020-improving} but no effort was done towards typing. 
As an orthogonal direction, weak supervision can also be used to augment training data for an existing ontology\cite{yang-etal-2019-exploring,wang-etal-2019-adversarial-training}.
Our work is the first to utilize weak supervision in creating a larger ontology.




