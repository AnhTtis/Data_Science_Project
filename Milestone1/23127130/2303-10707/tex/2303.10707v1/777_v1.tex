
%%%%%%%%%%%%%%%%%%%%%%%%%%%%%%%%%%%%%%%%%%%%%%%%%%%%%%%%%%%%%%%%%%%%%%%%%%%%%%
%  ************************** AVISO IMPORTANTE **************************    %
%                                                                            %
% Éste es un documento de ayuda para los autores que deseen enviar           %
% trabajos para su consideración en el Boletín de la Asociación Argentina    %
% de Astronomía.                                                             %
%                                                                            %
% Los comentarios en este archivo contienen instrucciones sobre el formato   %
% obligatorio del mismo, que complementan los instructivos web y PDF.        %
% Por favor léalos.                                                          %
%                                                                            %
%  -No borre los comentarios en este archivo.                                %
%  -No puede usarse \newcommand o definiciones personalizadas.               %
%  -SiGMa no acepta artículos con errores de compilación. Antes de enviarlo  %
%   asegúrese que los cuatro pasos de compilación (pdflatex/bibtex/pdflatex/ %
%   pdflatex) no arrojan errores en su terminal. Esta es la causa más        %
%   frecuente de errores de envío. Los mensajes de "warning" en cambio son   %
%   en principio ignorados por SiGMa.                                        %
%                                                                            %
%%%%%%%%%%%%%%%%%%%%%%%%%%%%%%%%%%%%%%%%%%%%%%%%%%%%%%%%%%%%%%%%%%%%%%%%%%%%%%

%%%%%%%%%%%%%%%%%%%%%%%%%%%%%%%%%%%%%%%%%%%%%%%%%%%%%%%%%%%%%%%%%%%%%%%%%%%%%%
%  ************************** IMPORTANT NOTE ******************************  %
%                                                                            %
%  This is a help file for authors who are preparing manuscripts to be       %
%  considered for publication in the Boletín de la Asociación Argentina      %
%  de Astronomía.                                                            %
%                                                                            %
%  The comments in this file give instructions about the manuscripts'        %
%  mandatory format, complementing the instructions distributed in the BAAA  %
%  web and in PDF. Please read them carefully                                %
%                                                                            %
%  -Do not delete the comments in this file.                                 %
%  -Using \newcommand or custom definitions is not allowed.                  %
%  -SiGMa does not accept articles with compilation errors. Before submission%
%   make sure the four compilation steps (pdflatex/bibtex/pdflatex/pdflatex) %
%   do not produce errors in your terminal. This is the most frequent cause  %
%   of submission failure. "Warning" messsages are in principle bypassed     %
%   by SiGMa.                                                                %
%                                                                            % 
%%%%%%%%%%%%%%%%%%%%%%%%%%%%%%%%%%%%%%%%%%%%%%%%%%%%%%%%%%%%%%%%%%%%%%%%%%%%%%

\documentclass[baaa]{baaa}

%%%%%%%%%%%%%%%%%%%%%%%%%%%%%%%%%%%%%%%%%%%%%%%%%%%%%%%%%%%%%%%%%%%%%%%%%%%%%%
%  ******************** Paquetes Latex / Latex Packages *******************  %
%                                                                            %
%  -Por favor NO MODIFIQUE estos comandos.                                   %
%  -Si su editor de texto no codifica en UTF8, modifique el paquete          %
%  'inputenc'.                                                               %
%                                                                            %
%  -Please DO NOT CHANGE these commands.                                     %
%  -If your text editor does not encodes in UTF8, please change the          %
%  'inputec' package                                                         %
%%%%%%%%%%%%%%%%%%%%%%%%%%%%%%%%%%%%%%%%%%%%%%%%%%%%%%%%%%%%%%%%%%%%%%%%%%%%%%
 
\usepackage[pdftex]{hyperref}
\usepackage{subfigure}
\usepackage{natbib}
\usepackage{helvet,soul}
\usepackage[font=small]{caption}

%%%%%%%%%%%%%%%%%%%%%%%%%%%%%%%%%%%%%%%%%%%%%%%%%%%%%%%%%%%%%%%%%%%%%%%%%%%%%%
%  *************************** Idioma / Language **************************  %
%                                                                            %
%  -Ver en la sección 3 "Idioma" para mas información                        %
%  -Seleccione el idioma de su contribución (opción numérica).               %
%  -Todas las partes del documento (titulo, texto, figuras, tablas, etc.)    %
%   DEBEN estar en el mismo idioma.                                          %
%                                                                            %
%  -Select the language of your contribution (numeric option)                %
%  -All parts of the document (title, text, figures, tables, etc.) MUST  be  %
%   in the same language.                                                    %
%                                                                            %
%  0: Castellano / Spanish                                                   %
%  1: Inglés / English                                                       %
%%%%%%%%%%%%%%%%%%%%%%%%%%%%%%%%%%%%%%%%%%%%%%%%%%%%%%%%%%%%%%%%%%%%%%%%%%%%%%

\contriblanguage{1}

%%%%%%%%%%%%%%%%%%%%%%%%%%%%%%%%%%%%%%%%%%%%%%%%%%%%%%%%%%%%%%%%%%%%%%%%%%%%%%
%  *************** Tipo de contribución / Contribution type ***************  %
%                                                                            %
%  -Seleccione el tipo de contribución solicitada (opción numérica).         %
%                                                                            %
%  -Select the requested contribution type (numeric option)                  %
%                                                                            %
%  1: Presentación mural / Poster                                            %
%  2: Presentación oral / Oral contribution                                  %
%  3: Informe invitado / Invited report                                      %
%  4: Mesa redonda / Round table                                             %
%  5: Presentación Premio Varsavsky / Varsavsky Prize contribution           %
%  6: Presentación Premio Sahade / Sahade Prize contribution                 %
%  7: Presentación Premio Sérsic / Sérsic Prize contribution                 %
%%%%%%%%%%%%%%%%%%%%%%%%%%%%%%%%%%%%%%%%%%%%%%%%%%%%%%%%%%%%%%%%%%%%%%%%%%%%%%

\contribtype{1}

%%%%%%%%%%%%%%%%%%%%%%%%%%%%%%%%%%%%%%%%%%%%%%%%%%%%%%%%%%%%%%%%%%%%%%%%%%%%%%
%  ********************* Área temática / Subject area *********************  %
%                                                                            %
%  -Seleccione el área temática de su contribución (opción numérica).        %
%                                                                            %
%  -Select the subject area of your contribution (numeric option)            %
%                                                                            %
%  1 : SH    - Sol y Heliosfera / Sun and Heliosphere                        %
%  2 : SSE   - Sistema Solar y Extrasolares  / Solar and Extrasolar Systems  %
%  3 : AE    - Astrofísica Estelar / Stellar Astrophysics                    %
%  4 : SE    - Sistemas Estelares / Stellar Systems                          %
%  5 : MI    - Medio Interestelar / Interstellar Medium                      %
%  6 : EG    - Estructura Galáctica / Galactic Structure                     %
%  7 : AEC   - Astrofísica Extragaláctica y Cosmología /                      %
%              Extragalactic Astrophysics and Cosmology                      %
%  8 : OCPAE - Objetos Compactos y Procesos de Altas Energías /              %
%              Compact Objetcs and High-Energy Processes                     %
%  9 : ICSA  - Instrumentación y Caracterización de Sitios Astronómicos
%              Instrumentation and Astronomical Site Characterization        %
% 10 : AGE   - Astrometría y Geodesia Espacial
% 11 : HEDA  - Historia, Enseñanza y Divulgación de la Astronomía
% 12 : O     - Otros
%
%%%%%%%%%%%%%%%%%%%%%%%%%%%%%%%%%%%%%%%%%%%%%%%%%%%%%%%%%%%%%%%%%%%%%%%%%%%%%%

\thematicarea{3}

%%%%%%%%%%%%%%%%%%%%%%%%%%%%%%%%%%%%%%%%%%%%%%%%%%%%%%%%%%%%%%%%%%%%%%%%%%%%%%
%  *************************** Título / Title *****************************  %
%                                                                            %
%  -DEBE estar en minúsculas (salvo la primer letra) y ser conciso.          %
%  -Para dividir un título largo en más líneas, utilizar el corte            %
%   de línea (\\).                                                           %
%                                                                            %
%  -It MUST NOT be capitalized (except for the first letter) and be concise. %
%  -In order to split a long title across two or more lines,                 %
%   please use linebreaks (\\).                                              %
%%%%%%%%%%%%%%%%%%%%%%%%%%%%%%%%%%%%%%%%%%%%%%%%%%%%%%%%%%%%%%%%%%%%%%%%%%%%%%

\title{Origin of magnetism in early-type stars}

%%%%%%%%%%%%%%%%%%%%%%%%%%%%%%%%%%%%%%%%%%%%%%%%%%%%%%%%%%%%%%%%%%%%%%%%%%%%%%
%  ******************* Título encabezado / Running title ******************  %
%                                                                            %
%  -Seleccione un título corto para el encabezado de las páginas pares.      %
%                                                                            %
%  -Select a short title to appear in the header of even pages.              %
%%%%%%%%%%%%%%%%%%%%%%%%%%%%%%%%%%%%%%%%%%%%%%%%%%%%%%%%%%%%%%%%%%%%%%%%%%%%%%

\titlerunning{Origin of magnetism in early-type stars}

%%%%%%%%%%%%%%%%%%%%%%%%%%%%%%%%%%%%%%%%%%%%%%%%%%%%%%%%%%%%%%%%%%%%%%%%%%%%%%
%  ******************* Lista de autores / Authors list ********************  %
%                                                                            %
%  -Ver en la sección 3 "Autores" para mas información                       % 
%  -Los autores DEBEN estar separados por comas, excepto el último que       %
%   se separar con \&.                                                       %
%  -El formato de DEBE ser: S.W. Hawking (iniciales luego apellidos, sin     %
%   comas ni espacios entre las iniciales).                                  %
%                                                                            %
%  -Authors MUST be separated by commas, except the last one that is         %
%   separated using \&.                                                      %
%  -The format MUST be: S.W. Hawking (initials followed by family name,      %
%   avoid commas and blanks between initials).                               %
%%%%%%%%%%%%%%%%%%%%%%%%%%%%%%%%%%%%%%%%%%%%%%%%%%%%%%%%%%%%%%%%%%%%%%%%%%%%%%


\author{
J.P. Hidalgo\inst{1}, 
P.J. Käpylä\inst{2,3}, 
C.A. Ortiz-Rodríguez\inst{1}, 
F.H. Navarrete\inst{4}, 
B. Toro \inst{1} 
\& 
D.R.G. Schleicher\inst{1}
}


\authorrunning{Hidalgo et al.}

%%%%%%%%%%%%%%%%%%%%%%%%%%%%%%%%%%%%%%%%%%%%%%%%%%%%%%%%%%%%%%%%%%%%%%%%%%%%%%
%  **************** E-mail de contacto / Contact e-mail *******************  %
%                                                                            %
%  -Por favor provea UNA ÚNICA dirección de e-mail de contacto.              %
%                                                                            %
%  -Please provide A SINGLE contact e-mail address.                          %
%%%%%%%%%%%%%%%%%%%%%%%%%%%%%%%%%%%%%%%%%%%%%%%%%%%%%%%%%%%%%%%%%%%%%%%%%%%%%%

\contact{jhidalgo2018@udec.cl}

%%%%%%%%%%%%%%%%%%%%%%%%%%%%%%%%%%%%%%%%%%%%%%%%%%%%%%%%%%%%%%%%%%%%%%%%%%%%%%
%  ********************* Afiliaciones / Affiliations **********************  %
%                                                                            %
%  -La lista de afiliaciones debe seguir el formato especificado en la       %
%   sección 3.4 "Afiliaciones".                                              %
%                                                                            %
%  -The list of affiliations must comply with the format specified in        %          
%   section 3.4 "Afiliaciones".                                              %
%%%%%%%%%%%%%%%%%%%%%%%%%%%%%%%%%%%%%%%%%%%%%%%%%%%%%%%%%%%%%%%%%%%%%%%%%%%%%%

\institute{
Departamento de Astronomía, Universidad de Concepción, Chile
\and
Institut für Astrophysik und Geophysik, Georg-August-Universität Göttingen, Alemania 
\and
Nordita, KTH Royal Institute of Technology and Stockholm University, Suecia
\and
Hamburger Sternwarte, Universität Hamburg, Alemania}


%%%%%%%%%%%%%%%%%%%%%%%%%%%%%%%%%%%%%%%%%%%%%%%%%%%%%%%%%%%%%%%%%%%%%%%%%%%%%%
%  *************************** Resumen / Summary **************************  %
%                                                                            %
%  -Ver en la sección 3 "Resumen" para mas información                       %
%  -Debe estar escrito en castellano y en inglés.                            %
%  -Debe consistir de un solo párrafo con un máximo de 1500 (mil quinientos) %
%   caracteres, incluyendo espacios.                                         %
%                                                                            %
%  -Must be written in Spanish and in English.                               %
%  -Must consist of a single paragraph with a maximum  of 1500 (one thousand %
%   five hundred) characters, including spaces.                              %
%%%%%%%%%%%%%%%%%%%%%%%%%%%%%%%%%%%%%%%%%%%%%%%%%%%%%%%%%%%%%%%%%%%%%%%%%%%%%%

\resumen{De acuerdo a nuestro entendimiento de la evolución estelar, las estrellas de tipo temprano poseen envolturas radiativas y núcleos convectivos debido a un fuerte gradiente de temperatura producido por el cíclo CNO. Algunas de estas estrellas (principalmente, las subclases Ap y Bp) tienen fuertes campos magnéticos, lo suficiente para ser observados por el efecto Zeeman. Aquí, presentamos simulaciones magnetohidrodinámicas en 3D de una estrella tipo A con $2~\mathrm{M}_{\odot}$ utilizando el modelo star-in-a-box. Nuestra meta es explorar si la estrella modelada es capaz de mantener un campo magnético tan fuerte como los observados, a través de un dínamo en su núcleo convectivo, o manteniendo una configuración estable de un campo fósil proveniente de una etapa evolucionaria temprana, usando diferentes velocidades de rotación. Creamos dos modelos, uno parcialmente radiativo y otro totalmente radiativo, que están determinados por el valor de la conductividad térmica. Nuestro modelo es capaz de explorar ambos escenarios, con dínamos relevantes impulsados por convección.}

\abstract{According to our understanding of stellar evolution, early-type stars have radiative envelopes and convective cores due to a steep temperature gradient produced by the CNO cycle. Some of these stars (mainly, the subclasses Ap and Bp) have strong magnetic fields, enough to be directly observed using the Zeeman effect. Here, we present 3D magnetohydrodynamic simulations of an $2 ~\mathrm{M}_{\odot}$ A-type star using the star-in-a-box model. Our goal is to explore if the modeled star is able to maintain a magnetic field as strong as the observed ones, via a dynamo driven by its convective core, or via maintaining a stable fossil field configuration coming from its early evolutionary stages, using different rotation rates. We created two models, a partially radiative and a fully radiative one, which are determined by the value of the heat conductivity. Our model is able to explore both scenarios, including convection-driven dynamos.
}

%%%%%%%%%%%%%%%%%%%%%%%%%%%%%%%%%%%%%%%%%%%%%%%%%%%%%%%%%%%%%%%%%%%%%%%%%%%%%%
%                                                                            %
%  Seleccione las palabras clave que describen su contribución. Las mismas   %
%  son obligatorias, y deben tomarse de la lista de la American Astronomical %
%  Society (AAS), que se encuentra en la página web indicada abajo.          %
%                                                                            %
%  Select the keywords that describe your contribution. They are mandatory,  %
%  and must be taken from the list of the American Astronomical Society      %
%  (AAS), which is available at the webpage quoted below.                    %
%                                                                            %
%  https://journals.aas.org/keywords-2013/                                   %
%                                                                            %
%%%%%%%%%%%%%%%%%%%%%%%%%%%%%%%%%%%%%%%%%%%%%%%%%%%%%%%%%%%%%%%%%%%%%%%%%%%%%%

\keywords{ Stars: magnetic fields --- stars: massive --- magnetohydrodynamics (MHD) --- dynamo }

\begin{document}

\maketitle

\section{Introduction}\label{S_intro}

Magnetic fields are ubiquitous in the universe, and there is a general consensus that they are amplified and maintained via astrophysical dynamos. In stars, these processes typically require rotation and convection, and therefore are most likely to occur inside convection zones (see \citealt{Brandenburg-2005}). Main-sequence stars with masses above $\sim 1.5~\mathrm{M}_{\odot}$ have stably stratified radiative envelopes and convective cores due to a steep temperature gradient produced by the CNO cycle. The least massive spectral type that fulfils these characteristics are A-type stars, which in general tend to be fast rotators \citep{Royer-2007} and have very weak magnetic fields of the order of a few Gauss. Interestingly, there is a clear bimodality here (see \citealt{Auriere-2007}), the peculiar-subclass Ap stars have slow rotation rates and magnetic fields between $300 ~\mathrm{G}$ and $30~\mathrm{kG}$, with the highest one so far reaching $\sim 34 ~\mathrm{kG}$ \citep{Babcock-1960}. The origin of these magnetic fields remains uncertain, but there are some theories: one includes a very strong core dynamo. This in principle could create a large scale magnetic field in the surface if it is strong enough, but also requires an efficient transport mechanism \citep{Moss-1989}. \cite{Augustson-2016} performed 3D simulations of a $10 ~\mathrm{M}_{\odot}$ B-type star, modeling the inner 64\% of its radius excluding the innermost values of the core to avoid a coordinate singularity. They found core dynamos able to produce strong magnetic fields, with peak strengths exceeding a megagauss. Another theory is that the magnetic field of these stars is a fossil field, a remnant from an earlier evolutionary stage that has survived in a stable configuration. Simulations made by \cite{Braithwaite-2006} of a $2~\mathrm{M}_\odot$ A-type star, have found stable axisymmetric magnetic field configurations starting with random field initial conditions. Also, non-axisymmetric configurations were found starting from turbulent initial conditions \citep{Braithwaite-2008}. 
%Both theories can be compatible, however, how the fossil field interacts with the host core-dynamo is still not well understood. 


The aim of our project is to explore both scenarios mentioned above. We explain our methods, initial conditions and the model in Section \ref{S_model}, the preliminary results in Section \ref{results}, and finally, a brief conclusion followed by the planned future work in Section \ref{future}.

%https://ui.adsabs.harvard.edu/abs/2009ApJ...705.1000F/abstract
 
\begin{figure}[h!]
\centering
\includegraphics[width=\columnwidth]{Figure1.png}
\caption{Snapshots of star-in-a-box simulations, showing the equatorial plane. The values of the heat conductivity $K$ (where $K_0$ is the value for a fully radiative configuration) and the rotation rate $\Omega$ (in $\Omega_\odot$) are shown in each plot. The colorbar represents the radial component of the flow velocity, where regions with $u_r \neq 0$ are convection zones.}
\label{Figure1}
\end{figure}


\section{The Model}\label{S_model}
We use a star-in-a-box set-up based on the model presented by \cite{kapyla-2021} with a star of radius $R$ inside a Cartesian cube of side $l=2.2R$ where all coordinates $(x,y,z)$ range from $-l/2$ to $l/2$. The set of magnetohydrodynamics (MHD) equations is the following:
\begin{align}
    \frac{\partial \mathbf{A}}{\partial t} &= \mathbf{u} \times \mathbf{B} - \eta \mu_0 \mathbf{J}, \label{mhd_1} \\
    \frac{D \ln \rho}{D t} &= - \boldsymbol{\nabla} \cdot \mathbf{u}, \label{mhd_2} \\
    \frac{D \mathbf{u}}{D t} &= - \boldsymbol{\nabla} \Phi - \frac{1}{\rho} \left( \boldsymbol{\nabla} p - \boldsymbol{\nabla} \cdot 2 \nu \rho \mathbf{S} + \mathbf{J} \times \mathbf{B} \right) \nonumber \\
    &\hspace{4.3cm} - 2 \mathbf{\Omega}\times \mathbf{u} + \mathbf{f}_d, \label{mhd_3} \\
    T \frac{Ds}{Dt} &= - \frac{1}{\rho} \left[ \boldsymbol{\nabla}\cdot (\mathbf{F}_\text{rad} + \mathbf{F}_\text{SGS})  + \mathcal{H} - \mathcal{C} + \mu_0 \eta \mathbf{J}^2 \right] \nonumber \\
    &\hspace{4.3cm} + 2 \nu \mathbf{S}^2, \label{mhd_4}
\end{align}
where $\mathbf{A}$ is the magnetic vector potential, $\mathbf{u}$ is the flow velocity, $\mathbf{B} = \boldsymbol{\nabla} \times \mathbf{A}$ is the magnetic field, $\eta$ is the magnetic diffusivity, $\mu_0$ is the magnetic permeability of vacuum, $\mathbf{J} = \boldsymbol{\nabla} \times \mathbf{B}/\mu_0$ is the current density given by Ampère's law, $D/Dt = \partial/\partial t + \mathbf{u} \cdot \boldsymbol{\nabla}$ is the advective (or material) derivative, $\rho$ is the mass density, $\Phi$ is the gravitational potential corresponding to the isentropic hydrostatic state of an A0 star, $p$ is the pressure, $\mathbf{S}$ is the traceless rate-of-strain tensor, $T$ is the temperature, $\mathbf{\Omega} = (0,0,\Omega_0)$ is the rotation rate along the $z$ axis, $\mathbf{f}_\mathrm{d}$ describes damping of flows exterior to the star. Radiation inside the star is approximated as a diffusion process. Therefore, the radiative flux is given by:
\begin{align}
    \mathbf{F}_\mathrm{rad} = - K \boldsymbol{\nabla} T, \label{radiative-flux}
\end{align}
where $K$ is the radiative heat conductivity, a quantity that is assumed to have a constant profile and establishes the size of the radiative zone (see Figure \ref{Figure1}). In addition, it is convenient to introduce a subgrid-scale (SGS) entropy diffusion that does not contribute to the net energy transport, but damps fluctuations near the grid scale. This is given by the SGS entropy flux $\mathbf{F}_\mathrm{SGS} = -\chi_\mathrm{SGS} \rho \boldsymbol{\nabla}s'$, where $s'$ is the fluctuating entropy.


Finally, $\mathcal{H}$ and $\mathcal{C}$ describe additional heating and cooling (respectively), and we adopted similar expressions as \cite{dobler-2006} and \cite{kapyla-2021}. \\

The simulations were run on a grid of $128^3$ using the {\sc Pencil Code}, a highly modular high-order finite-difference code for compressible non-ideal MHD \citep{pencilcode}. The stellar parameters used for a $2~\mathrm{M}_{\odot}$ A0-type star are $R_* = 2~\mathrm{R}_\odot$, $L_* = 23~\mathrm{L}_\odot$, $\rho_* \approx 5.6 \cdot 10^{4} ~\mathrm{kg}\,\mathrm{m}^{-3}$ for the radius, the luminosity and the central mass density respectively, which were obtained using the open-source stellar evolution code {\sc MESA} (see \citealt{MESA-2011}). For the relation to reality and the treatment of the units, we followed the description in Appendix A of \cite{kapyla-units}.


\begin{table}[t!]
\centering
\caption{Summary of all runs. $\Delta r$ denotes the radial extent of the convective core (where $R$ is the stellar radius), $K_0$ is the value for a fully radiative configuration, $\nu$ and $\eta$ in [$\mathrm{m}^2\,\mathrm{s}^{-1}$], $\Omega$ in [$\Omega_\odot$], $\langle u_\mathrm{rms} \rangle$ in [$\mathrm{m}\,\mathrm{s^{-1}}$], and $B_\mathrm{max}$ (the maximum value of $B_\mathrm{rms}$) in [$\mathrm{kG}$].}
\begin{tabular}{lcccccc}
\hline\hline\noalign{\smallskip}
\!\!Run & \!\!\!\!$K/K_0$ & \!\!\!\!$\nu~[10^{9}]$& \!\!\!\!$\eta~[10^{9}]$ &\!\!\!\!$\Omega$&\!\!\!\!$\langle u_\mathrm{rms}\rangle$ & $B_\mathrm{max}$
\\
\hline\noalign{\smallskip}
\!\!Sim1  &  $0.01$ & $2$ & $1$ & 0.14 & 127 & 65\\
\!\! \!\!$\Delta r \approx 1R$& $0.01$ & $2$ & $1$ & 0.70 & 82 & 50\\
\hline\noalign{\smallskip}
\!\!Sim2 & $0.04$ & $1.2$ & $1$ & 0.10 & 118 & 49\\
\!\!$\Delta r \approx 1R$ & $0.04$ & $1.2$ & $1$ & 0.20 & 95 & 65\\
\hline\noalign{\smallskip}
\!\!Sim3 & $0.07$ & $0.2$ & $0.18$ & 0.80 & 284 & 25\\
\!\!$\Delta r \approx 0.3R$ & $0.07$ & $0.2$ & $0.18$ & 1.58 & 229 & 27\\
\hline\noalign{\smallskip}
\!\!Sim4 & $0.1$ & $0.12$ & $0.18$ & 1.24 & 264 & 24\\
\!\!$\Delta r \approx 0.2R$ & $0.1$ & $0.12$ & $0.18$ & 2.48 & 177 & 24\\
\!\! & $0.1$ & $0.12$ & $0.18$ & 6.20 & 162 & 22\\
\hline
\end{tabular}
\label{tabla1}
\end{table}
\section{Preliminary results}\label{results}
\begin{figure*}[h!]
\centering
\includegraphics[scale=0.44]{Figure2.png}
\caption{Azimuthally averaged magnetic field [kG] vs time [year]. The \emph{upper panels} correspond to Sim1, with $\Omega = 0.14~\Omega_\odot$ (\emph{left}) and $\Omega = 0.70~\Omega_\odot$ (\emph{right}). The \emph{lower panels} are the runs from Sim2, $\Omega = 0.10~\Omega_\odot$ (\emph{left}) and $\Omega = 0.20~\Omega_\odot$ (\emph{right}).}
\label{Figure2}
\end{figure*}

The simulations are listed in Table \ref{tabla1}, divided into 4 main groups with different values for the diffusivities $\nu$, $\eta$, and the radiative heat conductivity $K$ which determines the depth of the convective zone $\Delta r$. The averages for the root-mean-square flow velocity $\langle u_\mathrm{rms} \rangle$ are estimated considering motions inside the convection zone. The rotation rates were chosen in order to have the Coriolis number
\begin{align}
    \mathrm{Co} = \frac{2\Omega_0}{u_\mathrm{rms} k_R}, \label{eq-Co}
\end{align}
equal to $\mathrm{Co} \approx 1$, $\mathrm{Co} \approx 2$, and $\mathrm{Co} \approx 10$ (only in Sim4), where $k_R = 2\pi/\Delta r$ corresponds to the scale of the largest convective eddies.

The simulations Sim1 and Sim2 are fully convective ($\Delta r \approx 1R$), which is not realistic for a main-sequence A-type star; however, these scenarios are useful as a way to test the model and can be representative of pre-main sequence evolution. Figure \ref{Figure2} shows the time evolution of the azimuthaly averaged magnetic field $\Bar{B}_\phi$ on the stellar surface. It is possible to find very strong magnetic fields, even though diffusivity values are quite high. We found quasi-steady solutions like in the upper-left and lower-right panels, and more interestingly, a polarity change in the upper-right panel.

\begin{figure}[h!]
\centering
\includegraphics[width=\columnwidth]{Figure3.png}
\caption{Root-mean-square magnetic field $B_\mathrm{rms}$ [kG] vs time [year], of the runs from Sim4.}
\label{Figure3}
\end{figure}

The partially convective runs also generate very strong dynamos inside their cores. Root-mean-square magnetic fields from Sim4 can be seen in Figure \ref{Figure3}. Here, we also included two runs (non-rotating and very rapid rotation) that were not dynamos, and therefore were not included in Table \ref{tabla1}. The run with $\Omega = 6.20~\mathrm{\Omega}_\odot$ has the highest amplitude $\Bar{B}_\phi$ of the group at $0.2R$, with $(\Bar{B}_\phi^{\mathrm{min}}, \Bar{B}_\phi^{\mathrm{max}}) = (-197.0,216.7)~\mathrm{kG}$. Sim3 behaves similarly to Sim4, where we obtain peak $B_\mathrm{rms}$ values around $20-30~\mathrm{kG}$, and the run with $\Omega = 1.58~\mathrm{\Omega}_\odot$ has $(\Bar{B}_\phi^{\mathrm{min}}, \Bar{B}_\phi^{\mathrm{max}}) = (-146.9,134.8)~\mathrm{kG}$, which corresponds to the highest field amplitude in the group at $0.3R$.
%Something similar happened with Sim3, obtaining peak values around $20-30~\mathrm{kG}$.

\section{Conclusions and future work}\label{future}

We explored different scenarios for an A-type star, with convective cores of $100 \%$, $30 \%$ and $20 \%$ of stellar radius. Our model is able to generate magnetic fields in all of them, and the $20 \%$ case, which is the most realistic, has also the highest value of the azimuthally averaged magnetic field. The current results are promising, but Sim3 and Sim4 need to be analyzed more carefully.

%seems to have values of $B_\mathrm{rms}$ close to what we observe in Ap stars. However, Sim3 and Sim4 need to be analyzed more carefully, because we do not know exactly how the magnetic field from the core-dynamo is transferred to the surface, although there are some possible mechanisms that could do this, such as buoyancy \citep{MacGregor-2003}. %Our model is also able to perform fully radiative simulations, but this is still something that needs to be explored with more simulations.



%%%%%%%%%%%%%%%%%%%%%%%%%%%%%%%%%%%%%%%%%%%%%%%%%%%%%%%%%%%%%%%%%%%%%%%%%%%%%%
% Para figuras de dos columnas use \begin{figure*} ... \end{figure*}         %
%%%%%%%%%%%%%%%%%%%%%%%%%%%%%%%%%%%%%%%%%%%%%%%%%%%%%%%%%%%%%%%%%%%%%%%%%%%%%%




\begin{acknowledgement}
We gratefully acknowledge support by the ANID BASAL projects ACE210002 and FB210003, as well as via Fondecyt Regular (project code 1201280).
\end{acknowledgement}

%%%%%%%%%%%%%%%%%%%%%%%%%%%%%%%%%%%%%%%%%%%%%%%%%%%%%%%%%%%%%%%%%%%%%%%%%%%%%%
%  ******************* Bibliografía / Bibliography ************************  %
%                                                                            %
%  -Ver en la sección 3 "Bibliografía" para mas información.                 %
%  -Debe usarse BIBTEX.                                                      %
%  -NO MODIFIQUE las líneas de la bibliografía, salvo el nombre del archivo  %
%   BIBTEX con la lista de citas (sin la extensión .BIB).                    %
%                                                                            %
%  -BIBTEX must be used.                                                     %
%  -Please DO NOT modify the following lines, except the name of the BIBTEX  %
%  file (whithout the .BIB extension).                                       %
%%%%%%%%%%%%%%%%%%%%%%%%%%%%%%%%%%%%%%%%%%%%%%%%%%%%%%%%%%%%%%%%%%%%%%%%%%%%%% 

\bibliographystyle{baaa}
\small
\bibliography{777_v1.bib}

\end{document}
