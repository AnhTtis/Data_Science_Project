\section{Conclusions}\label{sec:conclusions}
%TODO: Tälle pitää tehdä jotain. Muutama passiivi lause joista johtopäätökset jää hämäräksi. Tarvitaan paperin highlightien lyhyt kertaus (where is the contribution) ja sitten ajatuksia mitä opittiin ja mitä pitäisi jatkossa tutkia tai kouluttaa tai muuttaa politiikassa etc...

When evaluating the credibility of social media posts, readers can pay attention to the quality of content and trustworthiness of the source of the post. As shown in this study, readers' prior beliefs about the topic play a significant role in credibility evaluation. Readers judge belief-consistent information as more credible than belief-inconsistent information. At worst, this tendency, accompanied by algorithm bias, may strengthen some readers' false beliefs. This, in turn, for its part, accelerates the spread of misinformation in social media. Alarmingly, readers tend to ignore, to some extent, the quality of evidence in their credibility evaluation. This may be due to reluctance to revise one's beliefs, inability to differentiate the quality of different types of evidence, or not valuing research or expert evidence. However, content posted by an expert is usually evaluated as more credible than the content posted by a layperson. In our study, this was the case regardless of whether the post was accurate or inaccurate. Thus, readers can be more vulnerable to misinformation that is posted by a person who is considered an expert.      

Thorough credibility evaluation requires substantial processing effort. In everyday life, however, readers usually rely on cost-effective evaluation practices and, thus, may judge the credibility superficially. Therefore, there is a need for further development of features, such as warning labels, that aid readers in their credibility judgments. However, despite these aids, readers' role in the battle against the spread of misinformation is crucial. Consequently, our education system should equip all citizens with adequate critical evaluation skills. 