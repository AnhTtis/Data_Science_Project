\section{Introduction}

 The internet has become a major source of health information for laypersons \citep{sun2019consumer}. Because information online can be published without adhering to verification standards \citep{braasch2014sensitivity}, the internet has also become a platform for spreading misinformation, including issues concerning health \citep{ecker2022psychological}. This requires criticality that challenges laypersons, especially when their prior knowledge on the issue at hand is limited \citep{andreassen2012reading, braasch2014sensitivity}
 
 The challenge of credibility evaluation is even more pronounced when people encounter health-related issues on social networking sites. Social media messages are often very short and informal and may use improper syntax or spelling \citep{addawood2016your}. On social networking sites, such as Twitter, short texts can contain arguments with appropriate, inappropriate, or missing justifications, especially when these texts involve controversial topics. Although modern societies are portrayed by the division of cognitive labor \citep{scharrer2017science}, on social networking sites anyone can share their knowledge, views, or opinions, regardless of their expertise.

Taken together these challenges, we designed a study to investigate how adults evaluate the credibility of accurate and inaccurate short, health-related social media posts. In particular, we were interested how source characteristics and the quality of evidence affect peoples' credibility evaluations. Because peoples' prior beliefs about a topic reflect their evaluations \citep{richter2017comprehension}, we also examined the role of participants' prior beliefs on the topic in their credibility evaluations. 

