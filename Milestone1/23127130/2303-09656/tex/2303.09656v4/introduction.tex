\section{Introduction}

 The internet has become a major source of health information for laypersons \autocite{sun2019consumer}. Because information online can be published without adhering to verification standards \autocite{braasch2014sensitivity}, the internet has also become a platform for spreading misinformation, including issues concerning health \autocite{ecker2022psychological}. This requires a level of criticality that can challenge laypersons, especially when their prior knowledge about the issue at hand is limited \autocite{andreassen2012reading, braasch2014sensitivity}.
 
The challenge of credibility evaluation is even more pronounced when people encounter health-related issues on social networking sites. Social media messages are often very short and informal and may use improper syntax or spelling \autocite{addawood2016your}. On social networking sites, such as Twitter, short texts can contain arguments with accurate, inaccurate, inappropriate, or missing justifications, especially when these texts involve controversial topics. Although modern societies are portrayed by the division of cognitive labor wherein individuals defer to relevant experts to ground their understanding of less familiar topics \autocite{scharrer2017science}, on social networking sites anyone can share their knowledge, views, or opinions, regardless of their expertise.

Recent frameworks for credibility evaluation \autocites{barzilai2020dealing, forzani2022does} have depicted factors that readers can attend to when evaluating online information, including the quality of reasoning, consistency of content with one's prior knowledge or other relevant resources, the author's expertise and intentions, and text genre. Previous studies have mainly focused on a limited number of credibility evaluation aspects or employed experimental designs in which participants are assigned to specific conditions varying in credibility features. For example, \textcite{chinn2021effects} assigned participants to four conditions that differed in expertise (expert vs. layperson) and evidence type (statistical evidence vs. testimony) and then asked them to evaluate statements presented in the news articles. This study revealed an interaction between source expertise and evidence type such that the quality of evidence affected the perceived credibility of the laypersons but not that of the experts. 

However, studies designed to only manipulate features of content and source cannot reveal all the complexities involved in credibility evaluation.  Although there is substantial evidence that indicates variations in readers, texts, activities, and contexts can influence performance when making sense of information on the Internet \autocite{coiro2021toward}, few studies have investigated a more complex combination of factors (i.e., content, source, and context) that may predict how individuals evaluate the quality of social media texts and corresponding sources. Further, drawing research-based conclusions from crowdsourcing platforms whose membership represents mostly Western populations can lead to an imbalance of cultural and social perspectives \autocite{venturebeat2022how}. Consequently, sampling across platforms with more diverse membership can help examine whether the findings gleaned from Western populations hold in other regions around the globe. 

Therefore, the present study approached credibility evaluation more holistically by considering multiple factors, including the content (prior-belief consistency, evidence type), the source (expertise, gender, ethnicity) and the context (two crowdsourcing platforms representing multiple nationalities) across four health topics when adult readers evaluated either accurate or inaccurate social media posts. By choosing this approach, the study provides insights about the most dominating factors to consider when advancing theoretical frameworks and models of credibility evaluation. Moreover, this study advances the understanding of credibility judgments among the adult population as readers interact with information in globally networked social media contexts. 

