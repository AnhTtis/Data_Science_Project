\section{Conclusions}\label{sec:conclusions}


Our study sheds light on how variations of multiple factors of social media posts (i.e., content, evidence type, source expertise, gender, and ethnicity) can influence credibility evaluation performance. When evaluating the credibility of social media posts, readers can pay attention to the quality of content and trustworthiness of the source of the post. This study shows that readers tend to judge belief-consistent information as more credible than belief-inconsistent information. At worst, this tendency, accompanied by algorithm bias, may strengthen some readers' false beliefs. This, in turn, may accelerate the spread of misinformation in social media. Alarmingly, readers tend to ignore, to some extent, the quality of evidence in their credibility evaluation. This may be due to reluctance to revise one's beliefs, inability to differentiate the quality of different types of evidence, or not valuing research or expert evidence. However, regardless of whether the post was accurate or inaccurate, content posted by an expert was evaluated as more credible than content posted by a layperson. Thus, readers can be more vulnerable to misinformation posted by someone considered an expert. Moreover, our study revealed that different demographic backgrounds of the readers may significantly affect credibility evaluation performance.

Overall, our study suggests a need for further developing the credibility evaluation skills of readers across all ages, not only of children and young people who have typically been the target group of critical reading interventions. An inclusive education system should equip all citizens with adequate credibility evaluation skills. Further, the design and findings of our study answers calls to capture more of the complexities of digital reading tasks associated with texts, readers, and contexts \autocite{coiro2021toward} in addition to calls for more credibility evaluation research that considers the demographic attributes of readers (see \cite{shariff2020review}). Last, our study bridges theory and research efforts in psychology, communication, and media studies with those that explore critical online resource evaluation \autocite{forzani2020three, forzani2022does} in educational contexts. 

 
