% CVPR 2023 Paper Template
% based on the CVPR template provided by Ming-Ming Cheng (https://github.com/MCG-NKU/CVPR_Template)
% modified and extended by Stefan Roth (stefan.roth@NOSPAMtu-darmstadt.de)

\documentclass[10pt,twocolumn,letterpaper]{article}

%%%%%%%%% PAPER TYPE  - PLEASE UPDATE FOR FINAL VERSION
%\usepackage[review]{cvpr}      % To produce the REVIEW version
\usepackage{cvpr}              % To produce the CAMERA-READY version
%\usepackage[pagenumbers]{cvpr} % To force page numbers, e.g. for an arXiv version

% Include other packages here, before hyperref.
\usepackage[accsupp]{axessibility}
%\usepackage{hyperref} 
\usepackage{graphicx}
\usepackage{amsmath}
\usepackage{amssymb}
\usepackage{booktabs}
\usepackage{bm}
\usepackage{array}
\newcommand{\jw}[1]{\textcolor[rgb]{0.2,0.9,0.2}{[#1]}}
\newcommand{\xw}[1]{\textcolor[rgb]{1,0,0}{[#1]}}
\usepackage{multirow}
\usepackage{multicol}
\newcommand{\up}[1]{\tiny\textcolor{blue}{#1}}
\newcommand{\down}[1]{\tiny\textcolor{red}{#1}}
% It is strongly recommended to use hyperref, especially for the review version.
% hyperref with option pagebackref eases the reviewers' job.
% Please disable hyperref *only* if you encounter grave issues, e.g. with the
% file validation for the camera-ready version.
%
% If you comment hyperref and then uncomment it, you should delete
% ReviewTempalte.aux before re-running LaTeX.
% (Or just hit 'q' on the first LaTeX run, let it finish, and you
%  should be clear).
\usepackage[pagebackref,breaklinks,colorlinks]{hyperref}
% Support for easy cross-referencing
\usepackage[capitalize]{cleveref}
\crefname{section}{Sec.}{Secs.}
\Crefname{section}{Section}{Sections}
\Crefname{table}{Table}{Tables}
\crefname{table}{Tab.}{Tabs.}


%%%%%%%%% PAPER ID  - PLEASE UPDATE
\def\cvprPaperID{5123} % *** Enter the CVPR Paper ID here
\def\confName{CVPR}
\def\confYear{2023}

\makeatletter
\def\thanks#1{\protected@xdef\@thanks{\@thanks
\protect\footnotetext{#1}}}
\makeatother

\begin{document}

%%%%%%%%% TITLE - PLEASE UPDATE
\title{Partial Network Cloning}
%{Partial Functional Cloning for Neural Networks}

\author{\bf Jingwen Ye \quad %\orcidID{0000-0001-8415-3597} 
Songhua Liu \quad
Xinchao Wang$^{\dagger}$ \thanks{ $^{\dagger}$ Corresponding author.}\\
National University of Singapore\\
{\tt\small jingweny@nus.edu.sg, songhua.liu@u.nus.edu, xinchao@nus.edu.sg}\\
}
%\orcidID{0000-0003-0057-1404}
%\affiliations{ National University of Singapore}
\maketitle

%%%%%%%%% ABSTRACT
\begin{abstract}
In this paper, we study a novel task
that enables partial knowledge transfer 
from pre-trained models,
which we term as Partial Network Cloning~(PNC).
Unlike prior methods that
update  
all or at least part of the
parameters 
in the target network
throughout the knowledge transfer process, 
PNC conducts partial parametric ``cloning''
from a source network
and then injects the cloned 
module to the target, without modifying
its  parameters.
Thanks to the transferred module,
the target network is expected to gain
additional functionality, such as inference
on new classes;
whenever needed,
the cloned module can be readily removed
from the target,
with its original parameters
and competence
kept intact.
Specifically, we introduce 
an innovative
learning scheme that allows
us to identify simultaneously
the component to be cloned
from the source and the 
position to be inserted within
the target network,
so as to ensure the optimal performance.
Experimental results on several datasets
demonstrate that,
our method yields 
a significant improvement of $\>5\%$
in accuracy
and 50\%
in locality 
when compared 
with
{parameter-tuning based methods.}
Our code is available at \href{https://github.com/JngwenYe/PNCloning}{https://github.com/JngwenYe/PNCloning}.



%encouraging results 
%on both the accuracy and locality metrics: 
%it promtes $\>5\%$ accuracy than base continual
%learning setting and the enhanced locality on the cloned functionality.
%\xw{give some detailed numbers here}.
%Specifically, we firstly localize the explicit part of the source network directly related to the target task. Then the extracted transferable module is inserted into the target network by cascading it to a certain part of the target network. 
%In addition, the performance after the transferable module insertion, in reverse, optimize its localization in the source network. The two steps are learned together to ensure the best partial functional cloning performance.


\iffalse
These days deep neural networks are ubiquitously used in a wide range of tasks, from image classification and machine translation to face identification and self-driving cars.
While large pre-trained models have enabled impressive results on a variety of downstream tasks, it is also of vital importance to make small and easy post-hoc modifications on these networks, enabling flexible re-use of these models for upcoming changing function demands. 
In this work, we investigate the problem of neural network partial functional cloning: how one can efficiently extract and clone partial function of the source model on a particular task,
which could be readily added to a target network without influencing the previous task performance.
Specifically, we firstly localize the explicit part of the source network directly related to the target task. Then the extracted transferable module is inserted into the target network by cascading it to a certain part of the target network. 
In addition, the performance after the transferable module insertion, in reverse, optimize its localization in the source network. The two steps are learned together to ensure the best partial functional cloning performance.
We show on several datasets that our method yields encouraging results on both the accuracy and locality metrics.  
\fi

   
\end{abstract}

%%%%%%%%% BODY TEXT
\section{Introduction}
\label{sec:intro}

%\textbf{Large models' application, sometimes we just need to make small modification}
With the recent advances in deep learning, an 
increasingly number of pre-trained models have been
released online, demonstrating favourable performances on
various computer vision applications.
As such, many model-reuse approaches have been proposed
to take advantage of the pre-trained models.
In practical scenarios, users may request
to aggregate partial functionalities from multiple 
pre-trained networks, and customize
a target network whose competence
differs from any network in the model zoo.

%Existing knowledge-transfer methods, however,
%largely focused on training a 



%Thanks to the generosity of our community, the development and production deployment process are sped up when using these free pre-trained models from model zoo instead of training our own models.
%Considering that in a standardized model zoo community, all the models are pre-trained on public datasets, which   conducive standardization management over a huge amount of pre-trained models.
%Thus, the problem comes in practical deployment, we may not be able to find a pre-trained networks that have the exact same functions as the same as we need. 
%One direct solution is to find a pre-trained model with the most similar functions and add the rest functions to this target network.  

\begin{figure}[t]
\centering
\includegraphics[scale = 0.55]{figure/cloning.pdf}
\caption{Illustration of partial network cloning.
Given a set of pre-trained source models, 
we ``clone'' the transferable modules from the source, 
and insert them into the target model (left)
while preserving the functionality (right).
%This is achieved through the $Local(\cdot)$ and $Insert(\cdot)$ operations.
%{Cloning with the transferable module directly clones the model functionality on the target task (right), which is controlled by $Local(\cdot)$ and $Insert(\cdot)$ operations}.
%\xw{Don't understand this sentence.}
%\xw{What is direct cloning?}
%\xw{We show cloning from two networks, do we have experiments on this?}
}
\label{fig:clone}
\end{figure}

A straightforward solution to the functionality dynamic changing is to re-train 
the target network using 
the original training dataset,
or to conduct finetuning together with 
{regularization strategies to alleviate catastrophic forgetting}~\cite{Li2016LearningWF,Chaudhry2020ContinualLI,Titsias2020FunctionalRF},
%\xw{the new task data embedded some regularization?} 
which is known as continual learning.
However, direct re-training is extremely inefficient,
let alone the fact that  original training dataset is often  unavailable.
Continual learning, on the other hand,
is prone to 
catastrophic forgetting especially
when the amount of data for finetuning is small,
which, unfortunately, often occurs in practice.
Moreover, both strategies 
inevitably overwrite the original parameters
of the target network, indicating that,
without explicitly storing original parameters of the target network,
there is no way to recover 
its original performance or competence when
this becomes necessary.



%Continually finetuning still exists bias or catastrophic forgetting particularly when the data for finetuning is much fewer than the original set.
%More notably, both works on adjusting the network's weight to satisfy the functional addition demand, which, we thought is not a benign learning scheme, as new memory space are allocated for the newly-tuned networks otherwise function-nonstandard networks are introduced to update model zoo.


\iffalse
As a simple approach to making such functional modification, one could re-train the target network using the original training dataset, or finetuning with the new task data embedded some regularization items~\cite{Li2016LearningWF,Chaudhry2020ContinualLI,Titsias2020FunctionalRF} (which is known as continual learning). 
However, directly re-training is extremely inefficient let alone the original training dataset is often not available; Continually finetuning still exists bias or catastrophic forgetting particularly when the data for finetuning is much fewer than the original set.
More notably, both works on adjusting the network's weight to satisfy the functional addition demand, which, we thought is not a benign learning scheme, as new memory space are allocated for the newly-tuned networks otherwise function-nonstandard networks are introduced to update model zoo.
\fi

In this paper, we investigate
a novel task, termed as \textbf{P}artial \textbf{N}etwork \textbf{C}loning~(PNC),
to migrate knowledge from the source network,
in the form of a transferable module,
to the target one.
Unlike  prior methods that rely on updating 
parameters of the target network,
% in the knowledge transfer process,
PNC attempts to \emph{clone}
partial parameters from the source network
and then directly inject the cloned module into the target,
as shown in Fig.~\ref{fig:clone}.
In other words, the cloned module
is transferred to the target
in a copy-and-paste manner.
Meanwhile, the original parameters of the
target network remain intact, indicating that
whenever necessary, the newly added 
module can be readily removed
to fully recover its original functionality.
Notably, the cloned module 
\emph{per se}
is a fraction of the source network, 
and therefore requirements 
no additional storage expect for the lightweight adapters.
%hich, would not largely increase the burden of storage when the cases are extended to cloning from multiple sources. 
Such flexibility to expand the 
network functionality
and to detach the cloned module
without altering the base of the target or 
{allocating extra storage},
in turn, greatly enhances the utility
of pre-trained model zoo
and largely enables plug-and-play
model reassembly.



%shown in Fig.~\ref{fig:clone}, where new functions are added to the target network in the form of transferable modules.
%As for the transferable modules, we build them by directly extracting from the source pre-trained networks that own the to-be-cloned functions. 
%The `cloning' in PNC means that \textbf{\textit{we don't tune the network itself }}(including both the source and the target networks), on the contrary, we focus on learning the process of cloning part of the source network and the process of applying the cloned part to the target network. 
%It could be actually thought as the process to link part of the source networks to the target network to customize the any functions included in model zoo.

\iffalse
{In this paper, we propose to solve the functional addition problem by \textbf{P}artial \textbf{N}etwork \textbf{C}loning (PNC) shown in Fig.~\ref{fig:clone}, where new functions are added to the target network in the form of transferable modules.
As for the transferable modules, we build them by directly extracting from the source pre-trained networks that own the to-be-cloned functions. 
The `cloning' in PNC means that \textbf{\textit{we don't tune the network itself }}(including both the source and the target networks), on the contrary, we focus on learning the process of cloning part of the source network and the process of applying the cloned part to the target network. 
It could be actually thought as the process to link part of the source networks to the target network to customize the any functions included in model zoo.}
\fi
%\xw{Add a couple of sentence here to highlight the advantages}
%\xw{This paragraph is about the value of the task ``cloning'', why this task is important? What is the uniqueness? For example, remaining parameters remain the same?}
%\xw{Why not learning?}

Admittedly, the ambitious goal of PNC
comes with significant challenges,
mainly attributed to the 
black-box nature of the neural networks,
alongside our intention to preserve the 
performances on
both the previous and newly-added tasks of the target.
The first challenge concerns the localization
of the to-be-cloned module
within the source network, since
we seek discriminant representations
and good transferability 
to the downstream target task.
The second challenge, on the other hand,
lies in how to inject the cloned module
to ensure the performance.

\iffalse
However, the black-box nature of large neural networks makes such direct cloning between networks extremely difficult, for there is no direct connections between the network functions and the parameters. 
Specifically, the two main challenges are:
Firstly, it is difficult to localize explicit parts of the source networks as transferable modules which should ensure the rich representative of the to-be-cloned functions and good transferability to the downstream target task;
Secondly, it is non-trivial to decide where to apply these transferable modules regarding both the the maximum performance on the to-be-cloned functions and minimum performance drop on the previous pre-trained functions.
\fi
%\xw{This paragraph is about challenges}


To solve these challenges, 
we introduce an innovative strategy for PNC,
through learning the localization
and insertion in an intertwined manner
between the source and target network.
Specifically, 
to localize the transferable module
in the source network, we adopt a local-performance-based 
pruning scheme for parameter selection.
To adaptively insert the module into the target network, 
we utilize a positional search method 
in the aim to achieve the optimal performance, 
which, in turn, optimizes the localization operation.
The proposed PNC scheme
achieves performances
significantly superior to those of 
the continual learning setting ($5\%\sim 10\%$), 
while reducing  data dependency to $30\%$.

\iffalse
To solve the above challenges, 
we propose a new network partial cloning framework with the learnable localization and insertion strategies. 
To explicitly localize the transferable module in the source network, we use the local-performance based pruning for parameter selection.  
To adaptively insert the transferable module in the target network, we utilize a simple position searching method for the best insertion performance, which, in reverse, optimize the localization operation.
The proposed network partial cloning achieves better performance than continual learning setting ($5\%\sim 10\%$), while reducing  data dependency to $30\%$.
\fi
%\xw{How to solve these challenges}
%\xw{Better add some results here. Without results, it does not look serious}

Our contributions are therefore summarized as follows.
\begin{itemize}
    \item We introduce a novel yet practical model re-use setup, termed as partial network cloning~(PNC). 
    In contrast to conventional settings
    the rely on updating all or part of
    the parameters in the target network, 
    PNC migrates parameters from the source  
    in a copy-and-paste manner to the target,
    while preserving original parameters of the target
    unchanged.
    
    %that incorporates one or multiple pre-trained models, network partial cloning extracts part of the network and inserts it into another, which enables flexible customization on model zoo. 
    
    %We argue that this problem is extremely important in practical but, to the best of our knowledge, receives little attention from the academic community.
    
    %\item We propose a novel and effective learnable cloning framework to localize and insert the transferable module. The two learning processes are jointly optimized, which are adaptively refined with the best performance on customized functions. 
    
    \item We propose an effective scheme towards solving PNC, 
    which conducts learnable localization and insertion of
    the transferable module jointly between the source and target network.
    The two operations reinforce each other  
    and together ensure the performance of the target network.
    
    
    
    %learnable cloning framework to localize and insert the transferable module. The two learning processes are jointly optimized, which are adaptively refined with the best performance on customized functions. 
    
    \item We conduct experiments on {four widely-used} datasets and 
    showcase that the proposed method consistently  achieves 
    results superior to the conventional knowledge-transfer settings,
    including continual learning and {model ensemble}.
    
    
    %and comparing with the most relative  settings-- class-incremental learning, confirming its advantage over existing baselines.
\end{itemize}




\section{Related Work}
%As we propose the first work on network partial functional cloning applied on the pre-trained models, here we review three related topics that focus on designing algorithm on the pre-trained models.
\subsection{Life-long Learning}
Life-long/online/incremental learning, which is capable of learning, retaining and transferring knowledge over a lifetime, has been a long-standing research area in many fields~\cite{Wu2018MemoryRG,Shmelkov2017IncrementalLO,ye2022learning,Ye2020DataFreeKA}. 
The key of continual learning is to solve catastrophic forgetting, and there are three main solutions, which are the regularization-based methods~\cite{Li2016LearningWF,Chaudhry2020ContinualLI,Titsias2020FunctionalRF,Huihui21AAAI}, the rehearsal-based methods~\cite{Choi2019AutoencoderBasedIC,Shin2017ContinualLW,Venkatesan2017ASF} and architecture-based methods~\cite{mallya2018piggyback,li2019learn,Yan_2021_CVPR,Kang2022ForgetfreeCL}. 

Among these three streams of methods, the most related one to PNC is the architecture-based pruning, which aims at minimizing the inter-task interference via newly designed architectural components. Li \textit{et al.}~\cite{li2019learn} 
propose to separate the explicit neural structure learning and the parameter estimation, and apply evolving neural structures to alleviate catastrophic forgetting. At each incremental step, DER~\cite{Yan_2021_CVPR} freezes the previously learned representation and augment it with additional feature dimensions from a new learnable feature extractor.
Singh \textit{et al.}~\cite{singh2020calibrating} choose to calibrate the activation maps produced by each network layer using spatial and channel-wise calibration modules and train only these calibration parameters for each new task.


The above incremental methods are fine-tuning all or part of the current network to solve functionality changes.
Differently, we propose a more practical life-long solution, which learns to transfer the functionality from pre-trained networks instead of learning from the new coming data.
%is capable of dealing with both the data adding and deleting cases.

\begin{figure*}[t]
\centering
\includegraphics[scale = 0.56]{figure/framework.pdf}
\caption{The proposed partial network cloning framework.
The localized samples are fed into the source network for the original transferable module localization. To refine the transferable module, we learn how to locate and insert it, with the network weights fixed.}
\label{fig:framework}
\end{figure*}

\subsection{Network Editing}

Model editing is proposed to fix the bugs in networks, which aims to enable fast, data-efficient updates to a pre-trained base model’s behavior for only a small region of the domain, without damaging model performance on other inputs of interest~\cite{sinitsin2020editable, mitchell2022memory, sotoudeh2021provable}.

A popular approach to model editing is to establish learnable
model editors, which are trained to predict updates to the
weights of the base model to produce the desired change in behavior~\cite{sinitsin2020editable}. MEND~\cite{mitchell2021fast} utilizes a collection of small auxiliary editing networks as a model editor.
Eric \textit{et al.}~\cite{mitchell2022memory} propose to store edits in an explicit memory and learn to reason over them to modulate the base model’s predictions as needed. 
Provable point repair algorithm \cite{sotoudeh2021provable} finds a provably minimal repair satisfying the safety specification over a finite set of points. 
Cao \textit{et al.}~\cite{de2021editing} propose to train a hyper-network with constrained optimization to modify without affecting the rest of the knowledge, which is then used to predict the weight update at test time. 

Different from network edition that directly modifies a certain of weights to fix several bugs, our work do the functionality-wise modification by directly inserting the transferable modules.

\subsection{Model Reuse}
With a bulk of pre-trained models online, model reuse becomes a hot topic, which attempts to construct the model by utilizing existing available models, rather than building a model from scratch. 
Model reuse is applied for the purpose of reducing the time complexity, data dependency or and expertise requirement, which is studied by knowledge transfer~\cite{Han2020NeuralCM,FangPrunging2023CVPR,gao2017knowledge,Yuan2020CKDCK,Ye2021SafeDB,Ren_2021_CVPR} and model ensemble~\cite{peng2022model,shen2019meal,cao2020ensemble,walawalkar2020online}.

Knowledge transfer~\cite{Ren_2022_CVPR,liu2022dataset,yang2022KF,YangKD2023CVPR,yang2020CVPR,yang2020NeurIPS,LiuStyle2022ECCV} utilizes the pre-trained models by transferring knowledge from these networks to improve the current network, which has promoted the performance of domain adaptation~\cite{jing2020adaptively}, multi-task learning~\cite{wang2021coarse},  Few-Shot Learning~\cite{Li_2019_CVPR} and so on~\cite{jing2022learning}.
For example, KTN~\cite{Peng_2019_ICCV} is proposed to jointly incorporate visual feature learning, 
knowledge inferring and classifier learning into one unified framework for their optimal compatibility.
To enable transferring knowledge from multiple models, 
Liu \textit{et al.} ~\cite{liu2020adaptive} 
propose an adaptive multi-teacher multi-level knowledge distillation learning framework which associates each teacher with a latent representation to adaptively learn the importance weights.

Model ensemble~\cite{jing2021amalgamating,jing2021meta,yang2022deep} integrates multiple pre-trained models to obtain a low-variance and generalizable model. Peng \textit{et al.}~\cite{peng2022model} apply sample-specific ensemble of source models by adjusting the contribution of each source model for each target sample. MEAL~\cite{shen2019meal} proposes an adversarial-based learning strategy in block-wise training to distill diverse knowledge from different trained models.



The above model reuse methods transfer knowledge from networks to networks, with the base functionality unchanged. 
We make the first step work to directly transfer part of the knowledge into a transferable module by cloning part of the parameters from the source network, which enables network functionality addition.


\section{Proposed Method}

The goal of the proposed partial network cloning framework is to clone part of the source networks to the target network so as to enable the corresponding functionality transfer in the target network.
%In what follows, we first give preliminaries of the partial network cloning task,
%and then discuss the detailed procedure of the proposed framework.

The illustration of the proposed PNC framework is shown in Fig.~\ref{fig:framework}, where we extract a transferable module that could be directly inserted into the target network.


\subsection{Preliminaries}

Given a total number of $P$ pre-trained source models $\mathcal{M}_s = \{\mathcal{M}_{s}^0, \mathcal{M}_{s}^1,...,\mathcal{M}_{s}^{P-1}\}$, 
each $\mathcal{M}_{s}^{\rho}$ ($0\le \rho< P$) serves for cloning the functionality $t_s^{\rho}$, where $t_s^{\rho}$ is a subset of the whole functionality set of $\mathcal{M}_{s}^{\rho}$ and the to-be-cloned target set is denoted as $T_s=\{t_s^{0}, t_s^{1}, ..., t_s^{P-1}\}$.
%is capable of the target task trained on the target data ${D}_t=\{(x_i,y_i)\}_{i=1}^n$ and ${D}_t$ is much smaller than original dataset to train $\mathcal{M}_s$.
The partial network cloning is applied on the target model $\mathcal{M}_t$ for new functionalities addition, which is the pre-trained model on the original set $T_t$ ($T_t\cap T_s = \emptyset$). 

Partial network cloning aims at expending the functionality set of target network on the new $T_s$ by directly cloning.
In the proposed framework, it is achieved by firstly extracting part of $\mathcal{M}_s$ to form a transferable module $\mathcal{M}_f$, and then inserting it into target model $\mathcal{M}_t$ to build a after-cloned target network $\mathcal{M}_c$. 
The whole process won't change any weights of the source and target models, and also each transferable module is directly extracted from the source model free of any tuning on its weights. 
Thus, the process can be formulated as:
\begin{equation}
\begin{split}
    \mathcal{M}_c \leftarrow Clone(\mathcal{M}_t,M,\mathcal{M}_s,R ),
    \label{eq:clone}
\end{split}
\end{equation}
which is directly controlled by $M$ and $R$, where $M$ is a set of selection functions  for deciding how to extract the explicit transferable module on source networks $\mathcal{M}_s$, and $R$ is the position parameters for deciding where to insert the transferable modules to the target network. 
Thus, partial network cloning $Clone(\cdot)$ consists of two steps:
\begin{equation}
    \begin{split}
            \mathcal{M}_f^{\rho} &\leftarrow Local(\mathcal{M}_s^{\rho}, M^{\rho}),\\
    \mathcal{M}_c &\leftarrow Insert_{\rho=0}^P(\mathcal{M}_t,\mathcal{M}_f^{\rho},R^{\rho}),    \end{split}
\end{equation}
where both $M$ and $R$ are learnable and optimized jointly. Once $M$ and $R$ are learned, $\mathcal{M}_c$ can be determined with some lightweight adapters. 
%This maintains a good network environment for model zoo, as we enable online functional cloning without modifying the pre-trained models' weights. 

Notably, 
we assume that only the samples related to the to-be-cloned task set $T_s$ are available in the whole process, keeping the same setting of continual learning.
And to be practically feasible, partial network cloning must meet three natural requirements:
\begin{itemize}
    \item \textbf{Transferability:} The extracted transferable module should contain the explicit knowledge of the to-be-cloned task $T_s$, which could be transferred effectively to the downstream networks;
    \item \textbf{Locality:} The influence on the cloned model $\mathcal{M}_c$ out of the target data $D_t$ should be minimized;
    \item \textbf{Efficiency:} Functional cloning should be efficient in terms of runtime and memory;
    \item \textbf{Sustainability:} The process of cloning wouldn't do harm to the model zoo, meaning that no modification the pre-trained models are allowed and the cloned model could be fully recovered.
\end{itemize}

In what follows, we consider the partial network cloning from one pre-trained network to another, which could certainly be extended to the multi-source cloning cases, thus we omit $\rho$ in the rest of the paper.

%The key of functional cloning is to obtain a transferable meet the above three demands. Thus, in the proposed framework, we propose a three-step strategy, which is localizing with pruning, refining with linearization and inserting with adaptation.

%Recall that we attempt to propose an effective functional transfer method with the least modification and influence on the target work,
%we further define the goal of partial cloning as follows.

\if
In the framework of network partial cloning, there are three main steps:
\begin{itemize}
    \item \textbf{Step 1}: We need to localize the part of the network that are mostly related to the target function or task, which is achieved by locally disturbing the input.
    \item \textbf{Step 2}: Insert and optimize.
    \item \textbf{Step 3}: Adversarial adjustment with step 1 and step 2.
\end{itemize}
\fi

\subsection{Localize with pruning}
Localizing the transferable module from the source network is actually to learn the selection function $M$. 

In order to get an initial transferable module $\mathcal{M}_f$, we locate the explicit part in the source network $\mathcal{M}_s$ that contributes most to the final prediction. Thus, the selection function $M$ is optimized by the transferable module's performance locally on the to-be-cloned task $T_s$.

Here, we choose the selection function as a kind of mask-based pruning method mainly for two purposes: the first one is that it applies the binary masks on the filters for pruning without modifying the weights of $\mathcal{M}_s$, thus, ensuring sustainability; 
the other is for transferability that pruning would be better described as `selective knowledge damage'~\cite{Hooker2020WhatDC}, which helps for partial knowledge extraction.

Note that unlike the previous pruning method with the objective function to minimize the error function on the whole task set of $\mathcal{M}_s$,
here, the objective function is designed to minimize the \textit{locality performance} on the to-be-cloned task set $T_s$.
Specifically, for the source network $\mathcal{M}_s$ with $L$ layers $W_s = \{w_s^0, w_s^1, ..., w_s^{L-1}\}$, the localization can be denoted as:
\begin{equation}
\begin{split}
    \mathcal{M}_f =& M\cdot\mathcal{M}_s \Leftrightarrow \{m^l\cdot w_s^l|0\le l<L\},\\
    M =& \arg\max_M \ Sim(\mathcal{M}_f,\mathcal{M}_s|D_t)\\
    &- Sim(\mathcal{M}_f, \mathcal{M}_s|\overline{D}_{t}),
\label{eq:local}
\end{split}
\end{equation}
where $M=\{m^0, m^1, ..., m^{L-1}\}$ is a set of learnable masking parameters, which are also the selection function as mentioned in Eq.~\ref{eq:clone}. $Sim(\cdot|\cdot)$ represents the conditional similarity among networks, $\overline{D}_{t}$ is the rest data set of the source network. The localization to extract the explicit part on the target $D_t$ is learned by maximizing the similarity between $\mathcal{M}_s$ and $\mathcal{M}_f$ on $D_t$ while minimizing it on $\overline{D}_{t}$.


Considering the black-box nature of deep networks that all the knowledge (both from $D_t$ and $\overline{D}_t$) is deeply and jointly embedded in $\mathcal{M}_s$, it is non-trivial to calculate the similarity on the ${D_t}$-neighbor source network $\mathcal{M}_s|_{D_t}$. Motivated by LIME~\cite{Ribeiro2016WhySI} 
that utilizes interpretable
representations locally faithful to the classifier,
we train a model set containing $N$ small local models $\mathcal{G}=\{g_i\}^{(N)}$ to model the source $\mathcal{M}_s$ in the $D_t$ neighborhood, and then use the local model set as the surrogate: $\mathcal{G}\approx \mathcal{M}_s|_{D_t}$.
To obtain $\mathcal{G}$, for each $x_i \in D_t$, we get its augmented neighborhood by separating it into patches (i.e. $8\times 8$) and applying the patch-wise perturbations with a set of binary masks $B$. 
Thus, $\mathcal{G}$ is obtained by:
\begin{equation}
\min_{g_i} \frac{1}{|B|}\sum_{b\in B}\Pi_b\cdot\big\|\mathcal{M}_s(b\cdot x_i)-g_i(b))\big\|^2+\Omega(g_i),    
\end{equation}
where $\Pi_{b}$ is the weight measuring sample locality according to ${x}_i$,
$\Omega(g_i)$ is the complexity of $g_i$ and $|B|$ donates the total number of masks. 
$\mathcal{G}$ is optimized by the least square method and more details are given in the supplementary.
For each $x_i$, we calculate a corresponding $g_i$. And actually, we set $N<|D_t|$ (about $30\%$), which is clarified in the experiments.


The new $\mathcal{G}$, %\xw{comparing with ?} 
{calculated from the original source network $\mathcal{M}_s$ in the $D_t$ neighborhood}, models the locality of the target task $T_s$ on $\mathcal{M}_s$. Note that $\mathcal{G}$ can be calculated in advance for each pre-trained model, as it could also be a useful tool for the model distance measurement and others~\cite{Ilyas2022DatamodelsPP}.
In this paper, $\mathcal{G}$ perfectly matches our demand for the transferable module localization.
So the localization process in Eq.~\ref{eq:local} could be optimized as:
\begin{equation}
\begin{split}
 \min_{M}\sum_{g_i\in \mathcal{G}}\sum_{b\in B} \Big\|& f_t[\mathcal{M}_s\big(M\cdot W_s ; b\cdot x\big)] - f_t[g_i(b)]\Big\|^2\\
&s.t.\quad |m^l|\le c^l
\label{eq:locloss}
\end{split}
\end{equation}
where $f_t$ is for selecting the $T_s$ related output and $c^l$ is the parameter controlling the number of non-zero values of $M$ ($c^l<|W^l|$). 
And for inference, the learned soft masks $M$ are binarized by selecting $c^l$ filters with the top-$c^l$ masking values in each layer.




\iffalse
\subsection{Refine with linearization}
After the first step, an initialized transferable module is obtained, which could make reliable predictions on the target task.
Recall that one important requirement for functional cloning is the transferability, we refine the transferable module $\mathcal{M}_f$ and make it a ready-to-insert module, which would not influence much on the target model.



Thus, to refine the transferable module, we do the partial linearization on the non-linearzation by replacing the activation layer (i.e. ReLu) with a learnable partial-activation layer (PA).
For example, for the network with a conv layer with activation $\sigma(\cdot)$, the partial activation is $PA(\cdot)$ is added as:
\begin{equation}
\begin{split}
z = c^l\cdot conv (u) + (\mathbf{1}-c^l) \cdot \sigma(conv(u))
\end{split}
\end{equation}
where $0\le l< L$ and $u$ is the input $z$ is the output of the PA layer.

Our goal is to refine the transferable model with high accuracy but with
as few non-activations as possible. This motivates the following
$\ell_0$-constrained optimization problem.
Thus, to optimize this PA layer, the objective function is:
\begin{equation}
    \min_c \|PA\cdot\mathcal{M}_f(x)-\mathcal{M}_f(x)\|^2+\lambda\sum_l\|c^l\|_1;
\end{equation}
where $c$ can be updated via standard gradient iterations over this
objective while the rest part of the transferable module $\mathcal{M}_f$ keeps fixed. That is, we only update the new replacing partial activation layer.

Then when inference with this partial-activated transferable module, the activation is calculated as:
\begin{equation}
z_i=
\begin{cases}
      \sigma(conv(u_i) & c_i<\epsilon\\
      u_i & \text{otherwise} \\
      \end{cases},
\end{equation}
where $z_i$ and $u_i$ is the $i$-th layer and $\epsilon$ is a very
small non-negative hyperparameter.

This Partial linear layer can be thought as a kind of masking operation. But compared with the soft masks obtained from the step `localize with pruning', it won't decrease or increase the total parameter number of the transferable module $\mathcal{M}_f$. It aims at, to some extent, linearizing the module to make it transfer better.  
As can be observed, as we set the $\epsilon$ to be small, plenty action functions are removed and the transferable module refined with partial activation layer has less non-activation neuros, which transfers better to the down-stream target network.
\fi

\subsection{Insert with adaptation }
After the transferable module $\mathcal{M}_f$ being located at the source network, it could be directly extracted from $\mathcal{M}_s$ with $M$, without any modifications on its weights. 
Then the following step is to decide where to insert $\mathcal{M}_f$ into $\mathcal{M}_t$, as to get best insertion performance.

The insertion is controlled by the position parameter $R$ mentioned in Eq.~\ref{eq:local}.
Following most of the model reuse settings that keep the first few layers of the pre-trained model as a general feature extractor, the learning-to-insert process with $R$ is simplified as finding the best position ($R$-th layer to insert $\mathcal{M}_f$).
The insertion could be denoted as:
\begin{equation}
\begin{split}
 \mathcal{M}_c^R =& \mathcal{M}_t\big(W_t^{[0:R]}\big)\circ \big\{\mathcal{M}_t\big(W_t^{[R:L]}\big),\mathcal{M}_f\big\},\\
R^* =&\arg\max_R \ Sim(\mathcal{M}_f,\mathcal{M}_c^R|D_t)\\
&+ Sim(\mathcal{M}_t,\mathcal{M}_c^R|D_o),
\label{eq:insert}
\end{split}
\end{equation}
where $D_o$ is the original set for pre-training the target network, and $D_o\cup D_t = \emptyset$. 
The cloned $\mathcal{M}_c$ is obtained by the parallel connection of the transferable module into the target network $\mathcal{M}_t$.
Thus the insertion learned by Eq.~\ref{eq:insert} is to find the best insertion position by maximizing the similarity between $\mathcal{M}_f$ and $\mathcal{M}_c$ on $D_t$ (for the best insertion performance on $D_t$) and the similarity between $\mathcal{M}_t$ and $\mathcal{M}_c$ on $D_o$ (for the least accuracy drop on the previously learned $D_o$).

In order to learn the best position R, we need maximize the network similarities $Sim(|)$. Different from the solution used to optimize the objective function while localizing, insertion focuses on the prediction accuracies on the original and the to-be-cloned task set. So we use the network outputs to calculate $Sim(\cdot)$, which is the KL-divergence loss $\mathcal{L}_{kd}$. we write:
\begin{equation}
\begin{split}
       \min_{\mathcal{F}_c, \mathcal{A}}&\mathcal{L}_{kd}\circ f_t\big[\mathcal{F}_c(\mathcal{A};\mathcal{M}_c^R(B\cdot x)), \mathcal{G}(B)\big]\\
       +&\mathcal{L}_{kd}\circ \overline{f}_t\big[\mathcal{F}_c(\mathcal{A};\mathcal{M}_c^R(B\cdot x)), \mathcal{M}_t(B\cdot x)\big],\\
       &s.t.\quad R \in \{0,1,...,L-1\}
\label{eq:rtrain}
\end{split}
\end{equation}
where $f_t$ is for selecting the $T_s$ related output while $\overline{f}_t$ is for selecting the rest. 
$\mathcal{F}_c$ is the extended fully connection layers from the original FC layers of $\mathcal{M}_t$. And we add an extra adapter module $\mathcal{A}$ to do the feature alignment for the transferable module, which further enables cloning between heterogeneous models.
The adapter is consisted of one $1\times 1$ conv layer following with ReLu, which, comparing with $\mathcal{M}_s$ and $\mathcal{M}_t$, is much smaller in scale.
$\mathcal{G}$ and $B$ are defined in Eq.~\ref{eq:locloss}.

While training, $R$ is firstly set to be $L-1$ and then moving layer by layer to $R=0$.
In each moving step, we fine-tune the adapter $\mathcal{A}$ and the corresponding fully connected layers $\mathcal{F}_c$. It is a light searching process, since only a few of weights ($\mathcal{A}$ and $\mathcal{F}_c$) need to be fine-tuned for only a couple epochs (5$\sim$ 20). Extra details for heterogeneous model pair are in the supplementary.
Please note that although applying partial network cloning from the source to the target needs two steps ($Clone(\cdot)$ and $Insert(\cdot)$), the learning process is not separable and are interacted on each other.
As a result, the whole process can be jointly formulated as:
\begin{equation}
\begin{split}
 \!\min_{M,\mathcal{F}_c,\mathcal{A}}\mathcal{L}_{loc}\Big(\mathcal{M}_s^{[0:R]}&, M\!\cdot\! W_s^{[R:L]}, \mathcal{G}\Big)\!+\! \mathcal{L}_{ins} \Big( \mathcal{M}_t^{[0:R]}, \\
 \mathcal{A}\circ(M&\!\cdot\! W_s^{[R:L]}), \mathcal{M}_t^{[R:L]}, \mathcal{F}_c, \mathcal{G}\Big),\\
where &\quad R: (L-1) \rightarrow 0 
\end{split}
\label{eq:lossall}
\end{equation}
where $\mathcal{L}_{loc}(\cdot)$ is the objective function in Eq.~\ref{eq:locloss} and $\mathcal{L}_{ins}(\cdot)$ is the objective function in Eq.~\ref{eq:rtrain}. And in this objective function, $\mathcal{M}_s$ and $\mathcal{M}_t$ are using the same $R$ for simplification, while in practice a certain ratio exists for the heterogeneous model pair. 

Once the above training process is completed, we could roughly estimate the performance by the loss convergence value, which follows the previous work~\cite{Ye_Amalgamating_2019}.
Finally the layer with least convergence value is marked as the final $R$. The insertion is completed by this determined $R$ and the corresponding $\mathcal{A}$ and $\mathcal{F}_c$.

\subsection{Cloning in various usages}
The proposed partial network cloning by directly inserting a fraction of the source network enables flexible reuse of the pre-trained models in various practical scenarios.\\
\textit{\textbf{Scenario I:} Partial network cloning is a better form for information transmission.} When there is a request for transferring the networks, it is better to transfer the cloned network obtained by PNC as to reduce latency and transmission loss. 

In the transmission process, we only need to transfer the set $\{M,R,\mathcal{A},\mathcal{F}_c\}$, which together with the public model zoo, could be recovered by the receiver. 
$\{M,R,\mathcal{A},\mathcal{F}_c\}$ is extremely small in scale comparing with a complete network, thus could reduce the transmission latency.
And if there is still some transmission loss on $\mathcal{A}$ and $\mathcal{F}_c$, 
it could be easily revised by the receiver by fine-tuning on $D_t$. 
As a result, PNC provides a new form of networks for high-efficiency transmission.\\
\textit{\textbf{Scenario II:} Partial network cloning enables model zoo online usage.} In some resource limited situation, the users could flexibly utilize model zoo online without downloading it on local.

Note that the cloned model is determined by $Clone(\mathcal{M}_t,M,\mathcal{M}_s,R )$, $\mathcal{M}_t$ and $\mathcal{M}_s$ are fixed and unchanged in the whole process. There is not any modifications on the pre-trained models ($\mathcal{M}_t$ and $\mathcal{M}_s$) nor introducing any new models. PNC enables any functional combinations in the model zoo,
which also helps maintain a good ecological environment for the model zoo, 
since PNC with $M$ and $R$ is a simple masking and positioning operation, which is easy of revocation. Thus, the proposed PNC supports to establish a sustainable model zoo online inference platform.

\section{Experiments}

We provide the experimental results on four publicly available benchmark datasets, and evaluate the cloning performance in the commonly used metrics as well as the locality metrics.
And we compare the proposed method with the most related field -- continual learning, to show concrete difference between these two streams of researches.  
More details and experimental results including partially cloning from multiple source networks, can be found in the supplementary.

\subsection{Experimental settings}
\textbf{Datasets.}
Following the setting of previous continual methods, we report experiments on MNIST, CIFAR-10, CIFAR-100 and TinyImageNet datasets.
For MNIST, CIFAR-10 and CIFAR-100 datasets, we are using input size of $32\times 32$.
For TinyImageNet dataset, we are using input size of $256\times 256$.
In the normal network partial cloning setting, 
the first 50\% of classes are selected to pre-train the target network $\mathcal{M}_t$, and the last 50\% of classes classes are selected to pre-train the source network $\mathcal{M}_s$.

In the partial network cloning process, $30\%$ of the training data are used for each sub dataset, which reduces the data dependency to 30\%. And for training the local model set $\mathcal{G}$, we set $|B|=100$ and segment the input into $4\times 4$ patches for the MNIST, CIFAR-10 and CIFAR-100 datasets, set $|B|=1000$ and segment the input into $8\times 8$ patches for the Tiny-ImageNet dataset.
%For CUB200-2011 dataset that has 200 classes with 5,994 training images and 5,794 test images, we are using input size of $256\times 256$.

\textbf{Training Details.}
We used PyTorch framework for the implementation.
We apply the experiments on the several network backbones, including plain CNN, LeNet, ResNet, MobileNetV2 and ShuffleNetV2.
In the pre-training process, we employ a standard data augmentation strategy: random crop, horizontal flip, and rotation.
In the process of partial cloning, 10 epochs fine-tuning are operated for each step on MNIST and CIFAR-10 datasets, 20 epochs for CIFAR-100 and Tiny-ImageNet datasets.

For simplifying and accelerating the searching process in Eq.~\ref{eq:lossall}, we split LeNet into 3 blocks, the ResNet-based network into 5 blocks, MobileNetV2 into 8 blocks and ShuffleNetV2 into 5 blocks (excluding the final FC layers). Thus the block-wise adjustment for $R$ is applied for acceleration. 

\textbf{Evaluation Metrics.}
For the cloning performance evaluation, we evaluate the task performance by average accuracy:`Ori. Acc' (accuracy on the original set), `Tar. Acc' (accuracy on the to-be-cloned set) and `Avg. Acc' (accuracy on the original and to-be-cloned set), which is evaluated on the after-cloning target network $\mathcal{M}_c$. 

For evaluating the transferable module quality evaluation on local-functional representative ability, we use the conditional similarity $Sim(|)$ with $\mathcal{G}$~\cite{Jia2022AZO}, which can be calculated as:
\begin{equation}
    Sim(\mathcal{M}_a | D_a, \mathcal{M}_b | D_b) = Sim_{cos}\theta(\mathcal{G}_a,\mathcal{G}_b)
\end{equation}
where $Sim_{cos}(\cdot)$ is the cosine similarity, $\mathcal{G}_a$ and $\mathcal{G}_b$ are the corresponding local model sets of $\mathcal{M}_a(D_a)$ and $\mathcal{M}_b(D_b)$.

For evaluating the transferable module quality on transferability to other networks other than the target network, it is in the supplementary. 

\subsection{Experimental Results}
\begin{table*}[!htb]
\centering
\scriptsize
\caption{Overall performance on partial network cloning on MNIST, CIFAR10, CIFAR100 and Tiny-ImageNet datasets. We report the accuracies to evaluate the performance, including the comparison with the other functional addition methods and the ablation study. We choose \textit{`Continual+KD' as baseline} and mark the accuracy promotion in blue, accuracy drop in red.}
\label{tab:overall}
\begin{tabular}{p{19mm}|p{8mm}<{\centering}p{8mm}<{\centering}p{8mm}<{\centering}|p{8mm}<{\centering}p{8mm}<{\centering}p{8mm}<{\centering}||p{8mm}<{\centering}p{8mm}<{\centering}p{8mm}<{\centering}|p{8mm}<{\centering}p{8mm}<{\centering}p{8mm}<{\centering}}
\toprule
& \multicolumn{6}{c||}{\textbf{Acc on MNIST (LeNet5, \#3 Steps)}}& \multicolumn{6}{c}{\textbf{Acc on CIFAR-10 (ResNet-18, \#5 Steps)}} \\
\textbf{Method}  &\textbf{Ori.-S}&  \textbf{Tar.-S}&\textbf{Avg.-S}& \textbf{Ori.-M}&  \textbf{Tar.-M}&\textbf{Avg.-M} &\textbf{Ori.-S}&  \textbf{Tar.-S}&\textbf{Avg.-S}& \textbf{Ori.-M}&  \textbf{Tar.-M}&\textbf{Avg.-M}\\\midrule
Pre-trained & 99.7 & 99.5 &99.7&99.7&99.5&99.6 &95.9&97.2&96.1 & 95.9 & 97.6& 96.5\\
Joint+Full Set & 99.8  &  98.3 & 99.6 &99.7 & 99.3 & 99.5&95.2 & 96.8& 95.5&94.4&95.1& 94.7\\
Continual&83.4\down{-10.1}&100.0\up{+17.3}& 86.2\down{-5.5} &65.1\down{-27.9}&98.8\up{+16.8}& 77.7\down{-11.2}& 67.7\up{+2.8}&97.2\up{+2.6}& 75.3\down{-14.8} & 92.8\up{+18.7}&78.2\up{+16.6}&87.3\down{-2.1} \\
Direct Ensemble &94.6\up{+1.1} & 56.1\down{-26.4} &88.2\down{-3.5} &94.6\up{+1.6}& 81.9\down{-0.1} & 89.8\up{+0.9} &90.5\up{+25.6}& 39.3\down{-55.3}& 82.0\up{+12.1} &90.5\up{+16.4}& 43.8\down{-17.8}&73.0\up{+3.6}\\
\textit{Continual+KD} & \textit{93.5} &\textit{82.7}& \textit{91.7}& \textit{93.0}&\textit{82.0}& \textit{88.9}& \textit{64.9} &\textit{94.6} & \textit{69.9} &\textit{74.1}&\textit{61.6}& \textit{69.4}\\
\midrule
PNC-F (w/o Local) & 87.7\down{-5.8} & 100.0+\up{17.3} &90.0\down{-1.7} & 90.9\down{-2.1} &98.2\up{+16.2} & 93.6\up{+4.7}& 88.6\up{+23.7} & \textbf{97.3}\up{+2.7} & 90.1\up{+20.2} &85.5\up{+11.4} & 95.8\up{+34.2} & 89.4\up{+20.0}\\
PNC-F (w/o Insert)   & 86.9\down{-6.6} &  100.0+\up{17.3} & 89.1\down{-2.6}& 90.4\down{-2.6} &97.7\up{+15.7} &93.1\up{+4.2} & 86.1\up{+21.2}& 96.8\up{+2.2} &87.9\up{+18.0} & 86.0\up{+11.9} & \textbf{96.2}\up{+34.6} & 89.8\up{+30.4} \\
PNC-F (full)   & 88.5\down{-5.0}&  \textbf{99.7}\up{+17.0}& 90.4\down{-2.6} & 91.1\down{-1.9} &98.8\up{+16.8} & 94.0\up{+5.1}& 83.0\up{+18.1} & 96.5\up{+1.9}&85.3\up{+15.4} & 85.4\up{+11.3}&95.5\up{+33.9}& 89.2\up{+19.8}\\
\midrule
PNC (w/o Local) & 93.6\up{+0.1} & 96.2\up{+13.5} &94.0\up{+2.3} &92.9\down{-0.1}&94.0\up{+12.0}& 93.3\up{+4.4}& 90.5\up{+25.6} & 93.9\down{-0.7} & 91.7\up{+21.8}&87.1\up{+13.0}& 94.6\up{+33.1}&89.9\up{+29.8}\\
PNC (w/o Insert) & 92.8\down{-0.7} & 99.5\up{+16.8} & 93.9\up{+2.2}&91.9\down{-1.1} & 97.3\up{+15.3}& 93.9\up{+5.0}& 89.5\up{+24.6}&94.4\down{-0.2}& 90.3\up{+20.4}& 89.2\up{+15.1}&94.7\up{+33.2}& 91.3\up{+21.9}\\
\textbf{PNC (Ours, full)} & \textbf{96.4}\up{+2.9} & \textbf{99.7}\up{+17.0} &\textbf{97.0}\up{+5.3} & \textbf{96.2}\up{+3.2} &\textbf{97.8}\up{15.8} & \textbf{96.8}\up{+7.9}& \textbf{94.9}\up{+30.0} & 95.5\up{+0.9}&\textbf{95.0}\up{+25.1} & \textbf{93.7}\up{+19.6}& 94.5\up{+32.9}& \textbf{94.0}\up{+24.6} \\\midrule\midrule
& \multicolumn{6}{c||}{\textbf{Acc on CIFAR-100 (ResNet-50, \#5 Steps)}}& \multicolumn{6}{c}{\textbf{Acc on Tiny-ImageNet ( ResNet-18, \#5 Steps)}} \\
\textbf{Method}  &\textbf{Ori.-S}&  \textbf{Tar.-S}&\textbf{Avg.-S}& \textbf{Ori.-M}&  \textbf{Tar.-M}&\textbf{Avg.-M} &\textbf{Ori.-S}&  \textbf{Tar.-S}&\textbf{Avg.-S}& \textbf{Ori.-M}&  \textbf{Tar.-M}&\textbf{Avg.-M}\\\midrule
Pre-trained &80.0&	80.3& 80.1&80.0 &77.2 & 79.0 & 71.3 & 67.6  & 70.7&  71.3 & 68.9 & 70.4\\
Joint+Full Set  & 78.0& 74.9&77.5 &	76.3&	77.9 &76.9 &  63.1 & 60.8 &62.7 &63.7&61.6 & 62.9 \\
Direct Ensemble & 59.3\down{-6.2}  &46.4\down{-26.3} &57.2\down{-9.6} & 56.0\down{-18.4} & 46.4\down{-26.6} &52.4\down{-21.5} &58.0\up{+0.8} & 35.9\down{-20.5} & 54.3\down{-2.8}& 50.6\down{-9.3} & 30.2\down{-27.9} &43.0\down{-16.3}\\
Continual & 52.3\down{-13.2} & \textbf{79.4}\up{+6.7} & 56.8\down{-9.9} & 58.8\down{-15.6} & \textbf{78.0}\up{+5.0} &66.0\down{-7.9} &54.6\down{-2.6} & \textbf{70.1}\up{+13.7} & 57.2\up{+0.1} & 55.9\down{-4.0} & \textbf{64.9}\up{+6.8}& 59.3\up{+0.1} \\
\textit{Continual + KD} & \textit{65.5} & \textit{72.7} &66.7 & \textit{74.4} & \textit{73.0} &73.9 &\textit{57.2} &\textit{56.4} & 57.1& \textit{59.9} & \textit{58.1} & 59.2\\\midrule
PNC (w/o Local) &72.2\up{+6.7} & 70.4\down{-2.3} &71.9\up{+5.2} &75.7\up{+1.3} & 68.3\down{-4.7} &72.9\down{-1.0} & \textbf{65.6}\up{+8.4} &52.5\down{-3.9} & \textbf{63.4}\up{+6.4} &56.4\down{-3.5} & 55.9\down{-2.2} & 56.2\down{-3.0}\\
PNC (w/o Insert)  &63.2\down{-2.3} & 76.1\up{+3.4} &65.4\down{-1.3} & 66.1\down{-8.3} &76.0\up{+3.0} &69.8\down{-4.1} & 60.7\up{+3.5} & 63.5\up{+7.1} &61.2\up{+4.1} &58.8\down{-1.1} & 60.9\up{+2.8} & 59.6\up{+0.4} \\
\textbf{PNC (Ours, full)} &\textbf{76.7}\up{+11.2} & 74.9\up{+2.2}  & \textbf{76.4}\up{+9.7}& \textbf{76.9}\up{+2.5}  &76.5\up{+3.5} & \textbf{76.8}\up{+2.9}& 63.2\up{+6.0} & 60.7\up{+4.3} &62.8\up{+5.7} & \textbf{63.5}\up{+3.6} &60.4\up{+2.3} & \textbf{62.3}\up{+3.1} \\
\bottomrule
\end{tabular}
\end{table*}
\subsubsection{Overall Performance}
Table~\ref{tab:overall} shows overall performance of partial network cloning on MNIST, CIFAR-10, CIFAR-100 and Tiny-ImageNet datasets, where the target network and the source network are set to be the same architecture and the number of search steps $R$ is also listed.
We compare the proposed partial network cloning (`PNC') with the baseline `Pre-trained' original networks ( Acc on $\mathcal{M}_s$ and $\mathcal{M}_t$), 
the ensemble network of the source and the target (`Direct Ensemble'),
the networks scratch trained on the set including the original and the target (`Joint + Full set'), 
the continual-learned network with some regularization item (`Continual') and the continual-learned network with KD loss from the source network (`Continual+KD'). Specially, we compare the proposed `PNC' with `PNC-F', where `PNC-F' is the afterward all-parameter-tuned version of `PNC' on the to-be-cloned dataset.
And we also give the comparisons on the small-scale functionality addition (`-S', 20\% of the source functionalities are transferred), and middle-scale functionality addition (`-S', 60\% of the source functionalities are transferred).

From Table~\ref{tab:overall}, several observations are obtained.
Firstly, {the proposed PNC is capable of dealing with various datasets and network architectures} and its effectiveness has been proved on four datasets and on different network architectures.
Secondly, the full setting PNC gives the best solution to the new functionality addition task, our full setting (`PNC(full)') outperforms almost all of the other methods.
Thirdly, PNC shows better performance when cloning smaller functionality (`Avg.-S' vs `Avg.-M'), and it is practical in use when the most related network is chosen as the target and minor functionality is added with the proposed PNC. 
Finally, fully fine-tuning all the parameters of $\mathcal{M}_c$ after PNC doesn't bring any benefits (`PNC' vs `PNC-F'), since fine-tuning with the to-be-cloned dataset would bring bias on the new functionality.

\begin{figure}[t]
\centering
\includegraphics[scale = 0.44]{figure/scale.pdf}
\caption{The performance on partial network cloning with different scales of the transferable module.}
\label{fig:size}
\end{figure}

\subsubsection{More Analysis of the Transferable Module}
\textbf{How does the scale of the transferable module influence the cloning performance?}
The transferable module can be denoted as $\mathcal{M}_f\leftarrow M \cdot W_s^{[R:L]}$.
And the scale of the transferable module is influenced by two factors, which are the selection function $M$ and the position parameters $R$. 
We explore the influence of the scale on the CIFAR-10, with the same setting from Table~\ref{tab:overall} of cloning small functionality.
The selection function $M$ is directly controlled by the masking rate ${c}/{|W|}$ ($0\le l<L$, defined in Eq.~\ref{eq:locloss}), where larger $c^l$ makes larger transferable modules, shown in Fig.~\ref{fig:size} (left).
As can be observed from the figure, the accuracy of the original functionality (`Ori. Acc') slightly decreases with larger $\mathcal{M}_f$. While larger $\mathcal{M}_f$ doesn't ensure higher accuracy of the to-be-cloned function (`Tar. Acc', first increase and then drop), indicating that the appropriate localization strategy on the source instead of inserting the whole source network benefits a lot. 


The position parameter $R$ ($0\le l<L$) is learned in the insertion process, here we show the performance for $R = 1\sim 4$, which further shows the validation of our selection strategy. Bigger $R$ makes smaller transferable modules, the accuracy based on which is shown in Fig.~\ref{fig:size} (right). The accuracy on the to-be-cloned set (`Tar. Acc') doesn't largely influenced by it, while it does influence the accuracy on the original set (`Ori. Acc') a lot.
Notably, $R=2$ is the position learned in the insertion process, which shows to the best according to the average accuracy (`Avg. Acc').  

\begin{figure}[t]
\centering
\includegraphics[scale = 0.5]{figure/sim.pdf}
\caption{The similarity matrix maps computed on the source network and on the transferable modules on MNIST dataset. Deeper color indicates higher similarity.}
\label{fig:sim}
\end{figure}

\textbf{Has the transferable module been explicitly localized?} 
For evaluating the quality of the transferable module on whether the learned localization strategy $Local(\cdot)$ has successfully selected the explicit part for the to-be-cloned functionality or not, we compute the similarity matrix for the source network and the transferable module, which is displayed in Fig.~\ref{fig:sim}.
The comparison is conducted on the MNIST dataset, which is split into 10 sub-datasets ($D_0\sim D_9$, according to the label) and each time we localize one-label functionality from the source, thus obtaining 10 transferable modules ($\mathcal{M}_f^0\sim \mathcal{M}_f^9$). 
For the source network, we compute $Sim(\mathcal{M}_s|D_i,\mathcal{M}_s|D_j)$ for each sub-dataset pair. It could be observed from Fig.~\ref{fig:sim} (left) that each functionality learned from each sub-dataset are more or less related, showing the importance of the localization process to extract local functionality from the rest.
For the transferable module, we compute $Sim(\mathcal{M}_s|D_j,\mathcal{M}^i_f|D_i)$ . Fig.~\ref{fig:sim} (right) shows that the transferable module has high similarity with the source network on the to-be-cloned sub-dataset, and its relation with the rest sub-dataset is weakened (Non-diagonal regions are marked in lighter color than the matrix map of the source network).
Thus, the conclusion can be drawn that \textit{the transferable module successfully models the locality performance on the to-be-cloned task set}, proving the correctness of the learned $Local(\cdot)$.

\begin{table}[t]
\centering
\caption{The cloning performance evaluation on heterogeneous model pairs on CIFAR-10 and CIFAR-100 datasets.}
\scriptsize
\begin{tabular}{p{14mm}<{\centering}p{9mm}<{\centering}p{9mm}<{\centering}ccc}
\toprule
\textbf{Source}& & &  \multicolumn{3}{c}{\textbf{Acc. on $\mathcal{M}_c$}} \\ \cmidrule(r){4-6}
\textbf{(Target)}&\textbf{Dataset}& \textbf{Method} & \textbf{Ori. Acc} & \textbf{Tar. Acc}& \textbf{Avg. Acc} \\\midrule \midrule
ResNet-18&CIFAR10& Pre-train & 88.1 &95.9 &- \\ 
/&CIFAR10& Continual &75.3 & 92.8 & 78.2 \\
(CNN)&CIFAR10& PNC & 86.9 & 90.3& \textbf{87.5}\\ \midrule
ResNet-18&CIFAR10&Pre-train& 92.6 &95.9&-  \\ 
/&CIFAR10& Continual& 85.0 &96.7&87.0\\
(ResNetC-20)&CIFAR10& PNC & 92.1  & 93.5&\textbf{92.3}\\ \midrule
ResNet-18 &CIFAR100&Pre-train &72.9& 78.1&-\\
/&CIFAR100& Continual& 66.7&79.9&68.9\\
(MobileNetV2)&CIFAR100&PNC&70.8 &76.1& \textbf{71.7}\\\midrule 
ResNet-18&CIFAR100&Pre-train &74.8& 78.1&-\\
/&CIFAR100& Continual & 63.8& 82.3& 66.9\\
(ShuffleNetV2)&CIFAR100&PNC&72.9&77.1& \textbf{73.6} \\
\bottomrule
\end{tabular}
\vspace{-1.5em}
\label{tab:Heterogeneous Models}
\end{table}


\subsubsection{Cloning between Heterogeneous Models}
Here we evaluate the effectiveness of the proposed PNC between heterogeneous models.
The experiments are conducted on the CIFAR-10 and CIFAR-100 datasets, where 20\% of the functionalities are cloned from the source network to the target network. The results are depicted in Table~\ref{tab:Heterogeneous Models}. In the figure, we compare our PNC with the original pre-trained models and the network trained in continual learning setting.
As can be seen in the figure, cloning shows superiority between similar architectures of the source and target pair (`ResNet-18 (ResNetC-20)' has higher accuracies than `ResNet-18 (CNN)').




\subsubsection{Comparing with Incremental Learning}
The proposed partial network cloning framework can be also conducted to tackle incremental learning task.
The comparisons are made on the CIFAR-100 dataset when using ResNet-18 as the base network. We pre-train the target network with the first 50 classes and continually add the rest from the source network with different class-incremental step $s$.
The comparative results with some classic incremental learning methods are displayed in Table~\ref{tab:il}, where we compare PNC with the regularization- rehearsal- and the architecture- based continual learning methods, and show its superior in classification performance. More insight analysis and comparison with incremental learning are included in the supplementary.

\begin{table}[t]
\centering
\scriptsize
\caption{Comparative experimental results on incremental learning CIFAR-100 dataset.}
\label{tab:recoverorder}
\begin{tabular}{lccccc}
\toprule
\textbf{Method}& \textbf{Description} &$s=5$& $s=10$ &$s=20$&$s=50$\\ \hline\hline
LwF~\cite{Li2016LearningWF} & Regularization&29.5 &  40.4 & 47.6& 52.9  \\
iCaRL~\cite{Rebuffi2017iCaRLIC}& Rehearsal &57.8 & 60.5  & 62.0  & 61.8  \\
EEIL~\cite{Castro2018EndtoEndIL}& Rehearsal &63.4  & 63.6  & 63.7  & 60.8 \\
BiC~\cite{Wu2019LargeSI}& Architecture &60.1 & 60.4&68.9 & 70.2\\
PNC(ours)& Architecture &\textbf{71.5}&\textbf{73.6}&\textbf{75.2}&\textbf{74.2} \\
\bottomrule
\end{tabular}
\vspace{-1.5em}
\label{tab:il}
\end{table}

\section{Conclusion}
In this work, we study a new knowledge-transfer task,
termed as \emph{Partial Network Cloning}~(PNC),
which clones a module of parameters
from the source network
and inserts it to the target
in a copy-and-paste manner.
Unlike prior knowledge-transfer settings
the rely on updating parameters 
of the target network,
our approach preserves the parameters
extracted from the source
and those of the target
unchanged.
%thereby largely simplifies the recovery whenever needed.
Towards solving PNC, 
we introduce an effective learning
scheme that jointly conducts
localizing  
and  insertion,
where the two steps reinforce each other.
We show on several datasets that our method yields encouraging results on both the accuracy and locality metrics,
which consistently outperform the results
from other settings.

\section*{Acknowledgements}
This work is supported by the Advanced Research and Technology Innovation Centre (ARTIC), the National University of Singapore (project number:~A-0005947-21-00, project reference:~ECT-RP2),
and National Research Foundation, Singapore under its Medium Sized Center for Advanced Robotics Technology Innovation.

%AI Singapore (AISG2-100E-2021-077), and
%NUS Faculty Research Committee (FRC) Grant (R-263-000-E95-133).

%, and discuss its practical importance in various situations.
%These two steps are together optimize the selection and the insertion position of the transferable module, which is also a plug and play module acting as the partial function transferring media.
%We show on several datasets that our method yields encouraging results on both the accuracy and locality metrics, and discuss its practical importance in various situations.


%investigate to efficiently transfer partial function by partial network cloning from the source pre-trained network to the target pre-trained network.
%The whole process doesn't modify the weights of the pre-trained models nor introducing extra weights to be learned (the adapter is too small to be omitted), which is achieved by inserting part of the source network to the target one. 
%Specifically, we first localize the explicit part of the source network directly related to the target task. Then the extracted transferable module is inserted into the target network by cascading it to a certain part of the target network.
%The performance after the transferable module insertion, in reverse, optimize its localization in the source network.
%These two steps together optimize the selection and the insertion position of the transferable module, which is also a plug and play module acting as the partial function transferring media.
%We show on several datasets that our method yields encouraging results on both the accuracy and locality metrics, and discuss its practical importance in various situations.

%%%%%%%%% REFERENCES
{\small


% Created 3/2/2011
% Last modified: 3/7/09
%
\documentclass[journal, twocolumn]{IEEEtran}
%\pdfoutput=1
\IEEEoverridecommandlockouts



%\pdfoutput=1
\IEEEoverridecommandlockouts
\usepackage[dvips]{graphicx}
\usepackage[cmex10]{amsmath}
\usepackage{amsfonts,amssymb,bm}
%\usepackage[caption=false]{caption}
\usepackage[tight,footnotesize]{subfigure}
%\usepackage[caption=false,font=footnotesize]{subfig}
\usepackage{fixltx2e}
\usepackage{dblfloatfix}
%\usepackage{stfloats}
%\usepackage{enumerate}
\usepackage{rotating}
\usepackage{multirow}
\usepackage{url}
\usepackage{cite}
%\usepackage{enumitem}
%\usepackage{algorithmic,algorithm}
\usepackage[linesnumbered,lined, ruled]{algorithm2e}

%\usepackage{slashbox}
\usepackage{cite}
\usepackage{setspace}
\usepackage{footnote}
\usepackage[T1]{fontenc}
\usepackage{ae,aecompl}
\usepackage{epsfig}
\usepackage{multicol}
\usepackage{multirow}
%-------------
\usepackage{subfigure}
\usepackage{times}
%======================colored fonts
\usepackage{color}
\usepackage{comment}
\usepackage{amsthm}
\usepackage{caption}

\usepackage{colortbl}
\usepackage[official]{eurosym}

\newcommand\pro[1]{{\texttt{#1}}}
\newcommand\proalg[1]{{{#1}}}
\newcommand\mycommfont[1]{\footnotesize
\textcolor{blue}{#1}}
\SetCommentSty{mycommfont}
\newlength{\commentWidth}
\setlength{\commentWidth}{3.75cm}
\newcommand{\atcp}[1]{\tcp*[r]{\makebox[\commentWidth]{#1\hfill}}}

%\newcommand{\myitem}{\noindent$\bullet$}

\newcommand{\closebracket}{)}
\SetKwRepeat{Repeat}{repeat}{until}
\SetKwRepeat{Forever}{repeat}{forever}
\SetKwRepeat{On}{on}{end}
\SetNlSty{bfseries}{\color{black}}{}


%\usepackage{subfig}
%------------
%\usepackage{flushend}
%
%\newcommand\T{\rule{0pt}{2.6ex}}
%\newcommand\B{\rule[-1.2ex]{0pt}{0pt}}
%\DeclareMathOperator*{\argmax}{arg\,max}
%\DeclareMathOperator*{\arglexmax}{arg\,lex\,max}
%\long\def\symbolfootnote[#1]#2{\begingroup%2
%\def\thefootnote{\fnsymbol{footnote}}\footnote[#1]{#2}\endgroup}
%
\pagestyle{empty}
%\algsetup{linenodelimiter=.}
%\newtheorem{theorem}{Theorem}[section]
%\newtheorem{claim}{Claim}[section]
%\newtheorem{corollary}{Corollary}[section]
%\newtheorem{proposition}{Proposition}
%\newtheorem{proof}{Proof}
\newtheorem{remark}{Remark}
%\newtheorem{lemma}{Lemma}[section]
\newtheorem{definition}{Definition}

\newcommand{\ie}[0]{\textit{i.e.},~}
\newcommand{\eg}[0]{\textit{e.g.},~}
\newcommand{\etal}[0]{\textit{et al.}}
\newcommand{\etc}[0]{\textit{etc.}}
\newcommand{\vs}[0]{\textit{vs.}~}

\begin{document}

\vspace{-10pt}
\title{Fair Energy Allocation in Risk-aware Energy Communities}

%\vspace{-5pt}
%\author{\IEEEauthorblockN{authors}\\ \vspace{-12pt}
%\IEEEauthorblockA{Department of Informatics and Telecommunications,\\
%National and Kapodistrian University of Athens\\
%%Ilissia, 157 84 Athens, Greece\\
%Email: \{authors\}@di.uoa.gr}
%%\thanks{This work has been supported by EINS, the Network of Excellence in Internet Science (FP7-ICT-288021) and the Univ. of Athens (ELKE-10812).}
%}
%
%`
\author{
\IEEEauthorblockN{Eleni Stai,~\IEEEmembership{Member,~IEEE},  Lesia Mitridati, \IEEEmembership{Member,~IEEE}, Ioannis Stavrakakis,~\IEEEmembership{Fellow,~IEEE}, Evangelia Kokolaki, Petros Tatoulis, Gabriela Hug,~\IEEEmembership{Senior Member,~IEEE}} 


\thanks{E. Stai, P. Tatoulis and G. Hug are with the EEH - Power Systems Laboratory, ETH Z\"urich, Physikstrasse 3, 8092 Z\"urich, Switzerland.  L. Mitridati is with the Department of Wind \& Energy Systems, DTU. I. Stavrakakis is with the Department of Informatics and Telecommunications, National \& Kapodistrian University of Athens, 15784 Athens, Greece. E. Kokolaki is with the Hellenic Ministry of Environment and Energy, Mesogeion 119, 11526 Athens, Greece. E. Kokolaki’s work was carried out while she was with the National \& Kapodistrian University of Athens. Emails: elstai@ethz.ch, lemitri@dtu.dk, ioannis@di.uoa.gr, e.kokolaki@prv.ypeka.gr, petrost@student.ethz.ch, ghug@ethz.ch.}
%\thanks{Acknowledgement:}
}


\maketitle

\begin{abstract} 
This work introduces a decentralized mechanism for the fair and efficient allocation of limited renewable energy sources among consumers in an energy community. In the proposed non-cooperative game, the self-interested community members independently decide whether to compete or not for access to RESs during peak hours and shift their loads analogously. In the peak hours, a proportional allocation (PA) policy is used to allocate the limited RESs among the competitors. The existence of a Nash equilibrium (NE) or dominant strategies in this non-cooperative game is shown, and closed-form expressions of the renewable energy demand and social cost are derived. Moreover, a decentralized algorithm for choosing consumers' strategies that lie on NE states is designed. The work shows that the risk attitude of the consumers can have a significant impact on the deviation of the induced social cost from the optimal. Besides, the proposed decentralized mechanism with the PA policy is shown to attain a much lower social cost than one using the naive equal sharing policy.
\end{abstract}

\begin{IEEEkeywords}
energy communities; renewable energy sources; game theory; risk; demand side management;
\end{IEEEkeywords}
\section{Introduction}
\label{sec:introduction}
% \begin{itemize}
%     % Diffusion of FL
%     \item {\st{Diffusion of FL}}
%     % Security threats to FL
%     \item {\st{Security threats to FL with particular focus on model poisoning}}
%     % Limitations of existing countermeasures
%     \item {\st{Current countermeasures (e.g., KRUM) and their limitations}}
%     % Proposed method and its advantages
%     \item {\st{Intuitive description of the proposed method and its difference (i.e., advantages) w.r.t. state of the art}}
%     % Main contributions
%     \item {\st{Summary of the main contributions of this work}}
%     % Paper's structure and organization
%     \item {\st{Paper's structure and organization}}
% \end{itemize}

% Diffusion of FL
Recently, {\em federated learning} (FL) has emerged as the leading paradigm for training distributed, large-scale, and privacy-preserving machine learning (ML) systems~\cite{mcmahan2017googleai,mcmahan2017aistats}. 
The core idea of FL is to allow multiple edge clients to collaboratively train a shared, global model without disclosing their local private training data.
%Specifically, an FL system consists of a central server and many edge clients; 
A typical FL round involves the following steps: {\em(i)} the server randomly picks some clients and sends them the current, global model; {\em(ii)} each selected client locally trains its model with its own private data; then, it sends the resulting local model to the server;\footnote{Whenever we refer to global/local model, we mean global/local model {\em parameters}.} {\em(iii)} the server updates the global model by computing an \emph{aggregation function}, usually the average (FedAvg), on the local models received from clients.
% \begin{enumerate}
%     \item[{\em(i)}] the server sends the current, global model to the clients and appoints some of them for training;
%     \item[{\em(ii)}] each selected client locally trains its copy of the global model with its own private data; then, it sends the resulting local model back to the server;\footnote{Whenever we refer to global/local model, we mean global/local model {\em parameters}.}
%     \item[{\em(iii)}] the server updates the global model by computing an \emph{aggregation function} on the local models received from clients (by default, the average, also referred to as FedAvg~\cite{mcmahan2017aistats}).
% \end{enumerate}
This process goes on until the global model converges. %(e.g., after a certain number of rounds or other similar stopping criteria).
%\\
% The advantages of FL over the traditional, centralized learning paradigm are undoubtedly clear in terms of flexibility/scalability (clients can join/disconnect from the FL network dynamically), network communications (only model weights\footnote{We will use \textit{parameters} and \textit{weights} interchangeably.} are exchanged between clients and server), and privacy (each client's private training data is kept local at the client's end and not uploaded to the server).
\\
% Security threats to FL
%However, the growing adoption of FL also raises security concerns~\cite{costa2022covert}, particularly about its confidentiality, integrity, and availability.
Although its advantages over standard ML, FL also raises security concerns~\cite{costa2022covert}. %, particularly about its confidentiality, integrity, and availability~\cite{costa2022covert}.
% OLD, LONG VERSION
% Indeed, some work deals with privacy leakage that may expose the local data of some clients~\cite{melis2019sp}. 
% A large body of work, instead, investigates attacks that usually aim to detriment the predictive accuracy of the learned global model. For instance, \emph{data poisoning} attacks achieve this goal by letting an adversary pollute the training set of some corrupt FL clients with maliciously crafted examples~\cite{jagielski2018sp}.
% Similarly, in \emph{model poisoning} the attacker attempts to tweak the global model weights~\cite{bhagoji2019pmlr} by directly perturbing the local model's weights of some infected FL clients before these are sent to the central server for aggregation, usually via so-called Byzantine attacks. 
% It turns out that Byzantine model poisoning attacks severely impact standard FedAvg; therefore, more robust aggregation functions must be designed to make FL systems secure.
Here, we focus on \emph{untargeted model poisoning} attacks~\cite{bhagoji2019pmlr}, where an adversary attempts to tweak the global model weights %\footnote{We will use the terms \textit{parameters} and \textit{weights} interchangeably.} 
by directly perturbing the local model's parameters of some infected clients before these are sent to the central server for aggregation.
In doing so, the adversary aims to jeopardize the global model \textit{indiscriminately} at inference time.
Such model poisoning attacks severely impact standard FedAvg; therefore, more robust aggregation functions must be designed to secure FL systems.
\\
% In this paper, we focus on designing a novel robust aggregation scheme at the server's end to contrast the effect of Byzantine model poisoning attacks.
%
% Current countermeasures and their limitations
%Several countermeasures have been proposed in the literature to combat model poisoning attacks on FL systems.
% Some methods use simple statistics more robust than plain average to smooth the impact of malicious updates (e.g., Trimmed Mean and FedMedian~\cite{yin2018icml}). 
% Other defenses implement outlier detection techniques to discard malicious updates from the aggregation performed at the server's end. Those are either based on heuristics (e.g., Krum/Multi-Krum~\cite{blanchard2017nips} and Bulyan~\cite{mhamdi2018pmlr}) or data-driven approaches (e.g., K-means clustering~\cite{shen2016acm} or DnC via spectral analysis~\cite{shejwalkar2021ndss}). 
% Finally, some strategies rely on a centralized ``source of trust'' to spot potential malicious updates (e.g., FLTrust~\cite{cao2020fltrust}).
% Several countermeasures have been proposed in the literature to combat model poisoning attacks on FL systems, i.e., to discard possible malicious local updates from the aggregation performed at the server's end. 
% These techniques range from simple statistics more robust than plain average (e.g., Trimmed Mean and FedMedian~\cite{yin2018icml}) to outlier detection heuristics (e.g., Krum/Multi-Krum~\cite{blanchard2017nips} and Bulyan~\cite{mhamdi2018pmlr}) or data-driven approaches (e.g., spectral analysis via K-means clustering~\cite{shen2016acm} or spectral analysis), or methods based on ``source of trust'' (e.g., FLTrust~\cite{cao2020fltrust}).
% OLD, LONG VERSION
%Several countermeasures have been proposed in the literature to combat Byzantine model poisoning attacks on FL systems.
% Descriptive statistics
% For example, Trimmed Mean and FedMedian aggregate local model updates using more robust statistics than standard average~\cite{yin2018icml}.
%
% % Heuristics for outlier detection
% Many existing Byzantine-resilient strategies implement some outlier detection heuristics to discard the model updates sent by potentially malicious clients from the input of the aggregation function.
% One of the most popular heuristics is Krum~\cite{blanchard2017nips}.
% This strategy tries to mitigate the impact of Byzantine attacks by selecting as a global model the local model with the smallest sum of Euclidean distances to {\em all} the other local models.
% Although powerful, Krum requires the server to know (or, at least, estimate) the number of malicious FL clients upfront, which is generally impossible in a realistic attack scenario. %
% Moreover, Krum may become ineffective for complex, high-dimensional model parameter spaces due to the curse of dimensionality.
% Bulyan~\cite{mhamdi2018pmlr} tries to overcome this issue by combining Krum with a variant of Trimmed Mean.
% % Data-driven outlier detection
% Other strategies use data-driven outlier detection techniques -- e.g., via K-means clustering~\cite{shen2016acm} -- to spot potential malicious local model updates. 
% %For instance, Shen et al. propose to cluster local model updates with K-means and thus identify outliers.
%
% % Other techniques
% As far as the server is concerned, any local model received can be from a potential malicious client. 
% FLTrust~\cite{cao2020fltrust} assumes the server acts as a client, i.e., trains a local model on an additional {\em trustworthy} dataset at the server's end and compares it against all the local models from other clients. 
% This way, the server can rely on some ``source of trust'' when discarding potentially malicious clients.
%\\
% Limitations of existing Byzantine-resilient strategies
Unfortunately, existing defense mechanisms either rely on simple heuristics (e.g., Trimmed Mean and FedMedian by~\cite{yin2018icml}) or need strong and unrealistic assumptions to work effectively (e.g., foreknowledge or estimation of the number of malicious clients in the FL system, as for Krum/Multi-Krum~\cite{blanchard2017nips} and Bulyan~\cite{mhamdi2018pmlr}, which, however, cannot exceed a fixed threshold).
Furthermore, outlier detection methods using K-means clustering~\cite{shen2016acm} or spectral analysis like DnC~\cite{shejwalkar2021ndss} do not directly consider the temporal evolution of local model updates received.
Finally, strategies like FLTrust~\cite{cao2020fltrust} require the server to collect its own dataset and act as a proper client, thereby altering the standard FL protocol.
\\
% OLD, LONG VERSION
% Overall, existing Byzantine-resilient strategies are either simple heuristics (e.g., FedMedian) or, if they are more complex, they rely on strong and unrealistic assumptions to work effectively (e.g., knowing the number of malicious clients in the FL system in advance, as for Krum and alike).
% Furthermore, data-driven outlier detection methods do not consider the temporary evolution of local model updates received (e.g., K-means clustering). 
% Finally, strategies like FLTrust requires the server to collect its own dataset and act as a proper client, thereby altering the standard FL protocol.
%
% Description of the proposed method
This work introduces a novel pre-aggregation \textit{filter} robust to untargeted model poisoning attacks. Notably, this filter $(i)$ operates without requiring prior knowledge or constraints on the number of malicious clients and $(ii)$ inherently integrates temporal dependencies. 
The FL server can employ this filter as a preprocessing step before applying \textit{any} aggregation function, be it standard like FedAvg or robust like Krum or Bulyan.
Specifically, we formulate the problem of identifying corrupted updates as a multidimensional (i.e., matrix-valued) time series anomaly detection task. 
The key idea is that legitimate local updates, resulting from well-calibrated iterative procedures like stochastic gradient descent (SGD) with an appropriate learning rate, show \textit{higher predictability} compared to malicious updates. This hypothesis stems from the fact that the sequence of gradients (thus, model parameters) observed during legitimate training exhibit regular patterns, as validated in Section~\ref{subsec:intuition}. %until convergence. 
%This regularity may be more pronounced for smooth convex loss functions, but it can still be captured within an appropriate time window, even for more complex and convoluted loss surfaces. 
%We provide evidence of this claim in Appendix~B, where we show that the average mutual information (i.e., ``predictability''), calculated over pairs of legitimate model updates sent at different FL rounds, is significantly higher than the corresponding computation for a malicious client.
\\
Inspired by the matrix autoregressive (MAR) framework for multidimensional time series forecasting~\cite{chen2021je}, we propose the FLANDERS ({\em \textbf{F}ederated \textbf{L}earning meets \textbf{AN}omaly \textbf{DE}tection for a \textbf{R}obust and \textbf{S}ecure}) filter.
The main advantages of FLANDERS over existing strategies like FLDetector~\cite{zhao2020multivariate} are its resilience to large-scale attacks, where $50\%$ or more FL participants are hostile, and the capability of working under realistic non-iid scenarios.
We attribute such a capability to two key factors: $(i)$ FLANDERS works without knowing a priori the ratio of corrupted clients, and $(ii)$ it embodies temporal dependencies between intra- and inter-client updates, quickly recognizing local model drifts caused by evil players. Below, we summarize our main contributions:

\begin{itemize}
\item[{\em(i)}]
We provide empirical evidence that the sequence of models sent by legitimate clients is more predictable than those of malicious participants performing untargeted model poisoning attacks.
\\
\item[{\em(ii)}] 
We introduce FLANDERS, the first pre-aggregation filter for FL robust to untargeted model poisoning based on multidimensional time series anomaly detection.
\\
\item[{\em(iii)}] 
We integrate FLANDERS into Flower,\footnote{\scriptsize{\url{https://flower.dev/}}} a popular FL simulation framework for reproducibility.
\\
\item[{\em(iv)}] 
We show that FLANDERS improves the robustness of the existing aggregation methods under multiple settings: different datasets, client's data distribution (non-iid), models, and attack scenarios.
\\
\item[{\em(v)}] 
We publicly release all the implementation code of FLANDERS along with our experiments.\footnote{\scriptsize{\url{https://anonymous.4open.science/r/flanders_exp-7EEB}}}
\end{itemize}

% Paper's structure and organization
The remainder of the paper is structured as follows. %some related work and the current state-of-the-art solutions to security issues that FL entails. 
Section~\ref{sec:background} covers background and preliminaries. 
In Section~\ref{sec:related}, we discuss related work.
Section~\ref{sec:problem} and Section~\ref{sec:method} describe the problem formulation and the method proposed. % to tackle it. 
Section~\ref{sec:experiments} gathers experimental results. %, and Section~\ref{sec:limitations} discusses some limitations of this work.
Finally, we conclude in Section~\ref{sec:conclusion}.
 %discusses the limitations of this work and draws future research directions.
%reports conclusions and draws perspectives for future research directions.

%%%%%%% OLD %%%%%%%
%to overcome the resilience of Byzantine failures in distributed Stochastic Gradient Descent computations. 
% The strength of Krum is its time complexity, which is linear in the gradient dimension. 
% However, the robustness of the approach is guaranteed for gradient-based learning applications only when the majority of the clients are not compromised. 
% Besides, the aggregation mechanism of Krum, as well as that of similar methods, is robust from a coarse-grained perspective and does not provide solutions to errors and perturbations that may occur at inference time.
%A related approach to~\cite{blanchard2017nips} is the work of Su et al.~\cite{su2016dc}. Here, the authors propose an iterated approximate agreement to tackle a multi-layer scenario attacked by Byzantine agents. 
%However, the method works efficiently on the sole discrete context and it is inapplicable to continuous state environments.
%\gabri{Maybe, we should just talk about the main limitations of existing countermeasures without digging into their details (or, we can just mention Krum as this is the most popular one). I will move the description of all these methods to the Related Work section.}
\vspace{-0.1in}
\section{Decentralized Energy Sharing Mechanism} \label{sec:esc}


\subsection{Energy Sharing Community} \label{sec:community}

The energy community consists of $N$ consumers, indexed by $i \in \mathcal{N}= \{1,...,N\}$, who have access to multiple energy sources in order to cover their flexible loads.

\subsubsection{Energy Sources}\label{sec:energy_sources}

We consider that the energy community has access to two distinct types of energy sources, namely \textit{local production} from community-owned RESs, and \textit{imports} from the distribution grid. We consider that the local RESs production is available only during daytime (e.g., PV panels), with a limited capacity $\mathcal{RE}>0$, whereas the community's imports from the grid are unlimited. Therefore, during nighttime the community's aggregate load is fully covered by imports from the grid, and, during daytime if the community's aggregate load exceeds the available RES capacity, the remainder is covered by imports from the grid.

%%
Production from the community-owned RESs is priced by the community manager at a constant low tariff $c^{RES}$ (in units per energy), whereas imports from the grid are priced by an energy retailer using TOU tariffs, typically for daytime and nighttime consumption. We define the daytime and nighttime tariffs with respect to the RESs tariff, as $c^{grid,d}= \gamma c^{RES}$ and $c^{grid,n} = \beta c^{RES}$, respectively, with $\gamma>\beta>1$.
%%
These TOU tariffs reflect the sum of energy prices and grid tariffs and are designed to incentivize consumers to shift their flexible loads from daytime to nighttime to reduce energy production costs and congestion during peak hours. In addition, the low cost of the local RESs production promotes self-consumption within the community and reduction of grid imports. We assume that the energy source-related parameters $\Omega = \{\mathcal{RE}, c^{RES}, \beta, \gamma\}$ are perfectly known by all consumers in the community at the beginning of the day.

\subsubsection{Consumers Preferences}

The consumers have a broad range of flexible loads, namely, (i) shiftable appliances (e.g. washing machines) that do not need to be scheduled every day, (ii) batteries or electric vehicles (EVs) with flexible state-of-charge requirements at the end of the day, and (iii) thermostatically controlled loads (e.g., water heater, heat pumps) with flexible set-points. The level of consumption and the time-schedule of these loads are flexible. For instance, an EV owner has a daily inflexible load required to cover her daytime transportation needs, and a daily flexible load, representing the additional energy to achieve a desired state-of-charge by the end of the day.
%At the beginning of the day, the daily flexible loads of all consumers in the community can be scheduled during daytime or nighttime. The level and time interval at which they schedule their daily flexible loads depend on their consumption preferences, as well as the available energy sources. 
However, once scheduled, these loads cannot be interrupted or shifted to another time interval.
%The available energy sources available at a given TOU interval will then be shared among the consumers whose daily flexible loads are scheduled during the same TOU interval. 
As a result, consumers whose daily flexible loads are scheduled during daytime incur the risk of paying for high-priced imports from the grid if the community's aggregate daytime energy demand exceeds the available local RES production. When scheduling their daily flexible loads across different time intervals, consumers wish to achieve a trade-off between their desired daily energy consumption, and the financial risks incurred. And, risk-averse consumers may choose to reduce their daily energy consumption if they are scheduled during daytime, to mitigate the financial risks incurred. For instance, if scheduled during nighttime, a risk-averse EV owner may prefer to consume enough energy to fully charge her EV by the end of the day, whereas, if scheduled during daytime, she may prefer to consume a smaller amount of energy in order to charge her EV at e.g., $75\%$ by the end of the day.

The risk attitude and daily energy consumption preferences of each consumer $i \in \mathcal{N}$ in the community can be represented by her type $\vartheta_i \in \Theta = \{1,...M\}$. The type accounts for consumer's (i) daily flexible load $U_{\vartheta_i}>0$ (in energy unit); and (ii) risk-aversion degree $\mu_{\vartheta_i} \in [0,1]$, representing the share of her daily flexible load that she is willing to consume if scheduled during daytime. 

%%
With this parametric representation of the consumers' flexibility preferences, if the daily flexible load of a consumer $i$ of type $\vartheta_i$ is scheduled during daytime, her daytime energy demand is $E_{\vartheta_i} = \mu_{\vartheta_i} U_{\vartheta_i}$ (and the remainder of her daily flexible load $(1-\mu_{\vartheta_i})U_{\vartheta_i}$ is deferred to the following day), whereas, if her daily flexible load is scheduled during nighttime, her nighttime demand is $U_{\vartheta_i}$. Therefore, $\mu_{\vartheta_i}=1$ represents a risk-seeking consumer, and $\mu_{\vartheta_i} < 1$ a risk-conservative consumer. 

At the beginning of each day, each consumer knows her own flexibility preferences and type, but this information is considered private. We assume that the community manager and consumers in the community only know the probability distribution $\bm{r}=[r_1,...,r_{M}]^T$ over the consumers types $\Theta$, where $0 \leq r_{\vartheta} \leq 1$ is the probability that a consumer in the community is of type $\vartheta \in \Theta$. Furthermore, the consumers' preferences, and therefore their types, can vary from day to day. Since this paper studies a single scheduling day, the daily time indexes are omitted. 

Following the law of large numbers, the number of consumers of type $\vartheta \in \Theta$ can be approximated as $r_\vartheta \cdot N$. Thus, based on the above, the maximum daytime energy demand of the community, i.e., if the daily flexible loads of all consumers are scheduled during daytime is 
\vspace{-0.08in}
\begin{align}
 D^{Total} = N  \sum_{\vartheta \in \Theta} r_{\vartheta}~E_{\vartheta}.  
\end{align}
\vspace{-0.02in}
For notational simplicity, in the remainder of the paper, we introduce $\varepsilon_{\vartheta_i}=\frac{1}{\mu_{\vartheta_i}}$, such that $U_{\vartheta_i}=\varepsilon_{\vartheta_i} \cdot E_{\vartheta_i}$. Thus, $\varepsilon_{\vartheta_i}=1$ represents a risk-seeking consumer $i$, and $\varepsilon_{\vartheta_i}>1$ a risk-conservative consumer. Finally, we assume without loss of generality that $E_1\leq E_2 \leq ...\leq E_{M}$.

%Therefore, the expected maximum daytime schedule of the community, i.e. if all consumers' flexible loads are scheduled during daytime can be expressed as $D^{d,Total} = N \sum_{\vartheta \in \Theta} r_{\vartheta} E_{\vartheta}$. 
\vspace{-0.15in}
\subsection{Decentralized Energy Sharing Mechanism (D-ESM)}

The problem faced by the energy sharing community is to schedule the daily flexible loads of all consumers across the different TOU intervals and to allocate the different energy sources among them within each TOU interval. The role of the community manager is to design a mechanism that optimally coordinates the interactions among consumers in the community towards desirable outcomes, namely: (i) minimizing the social cost for the community as a whole, and (ii) sharing the community-owned assets among the consumers fairly. We introduce below the proposed decentralized energy sharing mechanism (D-ESM) for this energy sharing community.

\subsubsection{Load Scheduling}

In the proposed D-ESM, each consumer independently schedules her own daily flexible loads across the different TOU intervals, at the beginning of the day, in order to maximize her own utility under the set energy source allocation and payment policies. In contrast, in a Centralized ESM (C-ESM), the community manager would schedule the daily flexible loads of all consumers across the different TOU intervals in order to minimize the social cost of the community as a whole under the set energy source allocation and payment policies (see Section \ref{sec:coordinated}). As implementing this centralized approach would require for the community manager to have information on each consumer's preferences, it can only be considered as an ideal benchmark against which to compare the efficiency of the proposed D-ESM.

In this paper, we study \textit{mixed strategies} of consumer types. A \textit{mixed strategy} is a probability distribution $\mathbf{p}_{\vartheta}=[p_{\vartheta}^d, p_{\vartheta}^n]^T$, with $p_{\vartheta}^d \in [0,1]$ denoting the probability that a consumer of type $\vartheta \in \Theta$ schedules her daily flexible load during daytime, and $p_{\vartheta}^n \in [0,1]$ during nighttime. At the beginning of the day a consumer $i$ determines her mixed strategy based on her type $\vartheta_i \in \Theta$, $\mathbf{p}_{\vartheta_i}$. Then she schedules her daily flexible loads either in daytime or in nighttime with probabilities $p_{\vartheta}^d$, $p_{\vartheta}^n$, correspondingly. Let also $\bm{p}$ be the collection of mixed strategies of all consumers, i.e., $\bm{p}= \{\bm{p}_{\vartheta_i}\}_{i \in \mathcal{N}}$. 

%As the self-interested consumers internalize the energy source allocation and payment policies set by the community manager to schedule they daily flexible loads, 
%It is essential to design adequate energy source allocation policies that will incentivize consumers to act in a beneficial way for the community as a whole and in the following we introduce the proposed PA policy that satisfies desirable notions of fairness.
 
\subsubsection{Energy Source Allocation and Payment Policies} \label{sec:policies}
 
Once the daily flexible loads of all consumers have been scheduled, the community manager must allocate the available energy sources at each TOU interval (daytime or nighttime) among them. During nighttime, all scheduled loads are covered by grid imports since this is the sole available energy source for this TOU interval. During daytime, the community manager allocates in priority the local RESs production to cover the scheduled daytime loads, in order to maximize local consumption from the community and reduce energy costs. However, if the expected aggregate daytime energy demand exceeds the available local RESs production, the community manager must share this limited resource among those consumers with loads scheduled during daytime. This raises the challenging issue of allocating fairly a limited resource among users with equal claims to it.

%%
In order to ensure a notion of fairness among community members, the community manager allocates to each consumer $i$ of type $\vartheta_i$ a share of the local RESs production proportional to her daytime load schedule. As a result, under this PA policy, the local RESs production allocated to a consumer whose daily flexible load is scheduled during daytime is

\vspace{-0.1in}
\begin{small}
\begin{eqnarray}
res^{PA}_{\vartheta_i}(\mathbf{p}) &=& \frac{E_{\vartheta_i}}{\max(\mathcal{RE} , D^d(\mathbf{p}))}\mathcal{RE},
\label{eq:prop_alloc_energy}
\end{eqnarray}
\end{small}
\vspace{-0.1in}

\noindent where $D^d(\mathbf{p})$ denotes the expected aggregate daytime demand of the community.

Each consumer $i$ of type $\vartheta_i$ must then pay for the different energy sources covering her scheduled load at each TOU interval, ensuring budget balance of the proposed mechanism.

%%% assumptions
\vspace{-0.1in}
\subsection{Non-cooperative Game Formulation} \label{sec:game_def}
%\vspace{-0.05in}

Based on the proposed D-ESM framework, if a consumer schedules her daily flexible load during daytime, she competes with other consumers to use the limited local RESs production and incurs a financial risk. This competition among the consumers participating in the proposed D-ESM (for one single day) can be modeled as an Energy Sharing Game (ESG), as defined bellow.

%%When scheduling their daily flexible loads, each consumer $i$ has perfect knowledge about the energy sources parameters $\Omega = \{\mathcal{RE}, c^{RES}, c^{grid,d}, c^{grid,n}\}$ and its own flexibility preferences ($E_{\vartheta_i}$ and $\mu_{\vartheta_i}$), represented by its type $\vartheta_i$, and imperfect knowledge about other players' preferences, represented by the distribution $\mathbf{r}$ over consumers' types $\Theta = \{\vartheta_i\}_{i \in \mathcal{N}}$.

\vspace{-0.05in}
\begin{definition}\label{def:energy_source_game}
An \emph{Energy Sharing Game (ESG)} is a single-shot noncooperative game, defined by the tuple
\\$\Gamma=(\mathcal{N}, \{\mathcal{P}_{\vartheta_i}\}_{i\in\mathcal{N}}, \{\upsilon_{\vartheta_i}\}_{i\in \mathcal{N}})$, where:
\begin{itemize}
    \item $\mathcal{N}=\{1,...,N\}$ is the set of players, i.e., the consumers in the energy sharing community.
    \item $\mathcal{P}_{\vartheta_i} = \{ \mathbf{p}_{\vartheta_i} | \mathbf{p}_{\vartheta_i} : A_i \in \mathcal{A} \rightarrow p^{A_i}_{\vartheta_i} \in \mathbb{R}^+ , \text{ with } \sum_{A_i \in \mathcal{A}} p^{A_i}_{\vartheta_i} = 1 \}$ is the set of mixed strategies of player $i$ of type $\vartheta_i$ over the set of pure strategies $\mathcal{A}=\{d,n\}$, consisting of the choices to schedule her daily flexible load during daytime ($A_i = d$) or during nighttime ($A_i = n$). Therefore, each consumer $i$ of type $\vartheta_i$ with a mixed strategy $\mathbf{p}_{\vartheta_i}$, plays this game by randomly selecting an action $A_i \in \mathcal{A}$ with probability $p^{A_i}_{\vartheta_i}$\footnote{Note that a \textit{pure strategy} is a special case of a mixed strategy where one action has a probability equal to 1 (and the remaining have 0).}.
    \item $\upsilon_{\vartheta_i} : A_i \in \mathcal{A} \rightarrow  \upsilon^{A_i}_{\vartheta_i}$ is the payoff function of a consumer $i$ of type $\vartheta_i$ over the set of pure strategies $\mathcal{A}$. The cost of a consumer $i$ of type $\vartheta_i$ who plays the pure strategy $A_i = d$, is 
    \begin{small}
    \begin{align}
    \upsilon^{d}_{\vartheta_i} = c^{RES} res_{\vartheta_i}^{PA}(\mathbf{p}) + c^{grid,d} (E_{\vartheta_i}-res_{\vartheta_i}^{PA}(\mathbf{p})),
    \label{eq:RES_cost}
    \end{align}
    \end{small}
    \hspace{-0.1in} and depends on the strategy profile $\bm{p}$ of all consumers via the community's expected aggregate daytime energy demand $D^d(\mathbf{p})$. The cost of a consumer who plays the pure strategy $A_i = n$ is
    \begin{small}
    \begin{align}\upsilon^{n}_{\vartheta_i} = U_{\vartheta_i}  c^{grid,n},
    \label{eq:nonRES_cost}
    \end{align}
    \end{small}
    \hspace{-0.1in} and is independent on other consumers' mixed strategies.
    Before making their decisions all players have perfect knowledge of the energy sources parameters in the set $\Omega$ and their own preferences and type, and have prior knowledge on the probability distribution $\bm{r}$ over the other consumers' types.
\end{itemize}
\end{definition}


A consumer of type $\vartheta \in \Theta$ repeatedly playing the mixed strategy $\bm{p}_{\vartheta}$ over multiple instances of the ESG %(or equivalently, a large number of consumers of type $\vartheta \in \Theta$ playing the mixed strategy $\bm{p}_{\vartheta}$ over a single instance of the ESG), 
would have an \textit{expected} daytime and nighttime energy demand equal to $D_{\vartheta}^{d} = p_{\vartheta}^{d} E_{\vartheta}$ and $D_{\vartheta}^{n} = p_{\vartheta}^{n} U_{\vartheta}$, respectively. Therefore, the mixed strategy of a consumer $i$ of type $\vartheta_i$ can alternatively be interpreted as splitting her daily flexible loads between daytime and nighttime, such that her daytime load schedule is equal to $D_{\vartheta_i}^{d}$, and their nighttime load schedule is equal to $D_{\vartheta_i}^{n}$. %Then, each consumer $i$ tries to maximize its expected profit by scheduling a load equal to $E_{\vartheta_i}$ during the day with probability $p^{d}_{\vartheta_i}$, and a load equal to $U_{\vartheta_i}$ during the night with probability $p^{d}_{\vartheta_i}$.
With these notations, the \textit{expected} aggregate daytime and nighttime energy demands of the community are respectively expressed as

\vspace{-0.1in}
\begin{small}
\begin{subequations}
\begin{align}
    & D^d (\mathbf{p}) = N\sum_{\vartheta\in \Theta} r_{\vartheta }~p^d_{\vartheta}~E_{\vartheta,}\label{eq:demand} \\
     & D^n (\mathbf{p}) = N\sum_{\vartheta\in \Theta} r_{\vartheta }~p^n_{\vartheta}~U_{\vartheta}\label{eq:demand_n}.
 \end{align}
\end{subequations}
\end{small}
\vspace{-0.15in}



\vspace{-0.1in}\section{Analysis of the Decentralized Energy Sharing Mechanism}\label{sec:gameanalysis}

%This behavior represents consumers with a broad range of flexible loads, including shiftable appliances, such as washing machines that do not need to run every day, as well as EVs, water heaters, and batteries that do not need to be fully charged at the end of a given day. For instance EV owners would compute their minimum (inflexible) daily energy load, representing the energy needed to cover their transportation needs for the day, as well as their flexible energy load, representing the additional energy needed to fully charge their EV. Then, if they decide to compete for RESs, they may be willing to engage only part of their daily flexible load to mitigate the risk of paying for the high-priced peak-load production. At the beginning of the following day that they play this game, they would update their daily inflexible and flexible loads and their risk attitude based on their new state-of-charge and transportation needs.

In this section, we study analytically the uncoordinated decisions of the self-interested consumers participating in the proposed D-ESM. In the following, we study the conditions on the parameter values for the existence of dominant strategies and mixed-strategy NE under the proposed PA and payment policies, and provide closed-form formulations of these equilibrium states, i.e., ranges on the values of the vector of mixed strategies at NE, denoted as $\mathbf{p^{NE}}$ and an analytical expression on the value of the expected aggregate daytime energy demand. The proofs of the theoretical results presented below are available in the {Appendix \ref{sec:proofsESSG}} of \cite{arxiv_version}.

First, we recall that, for a mixed-strategy NE to exist, the expected costs of each consumer for all pure strategies in the support of the mixed-strategy NE must be equal. Using the expressions of the costs in \eqref{eq:RES_cost} and \eqref{eq:nonRES_cost}, we obtain that the amount of RESs allocated to a consumer type $\vartheta \in \Theta$ at a NE must satisfy:

 \vspace{-0.1in} 
\small
\begin{equation}\label{eq:conditionEQ_extra_demand}
res^{NE}_{\vartheta}(\mathbf{p^{NE}}) = \frac{\gamma-\varepsilon_{\vartheta}\beta}{\gamma-1}E_{\vartheta},~ \forall \vartheta \in \Theta.
\end{equation}
\normalsize 
\vspace{-0.1in}

\noindent Thus, in the ESG, a mixed-strategy NE exists under the condition:

 \vspace{-0.1in} 
 \small
\begin{equation}\label{eq:condition_PA_NE}
res_{\vartheta}^{PA}(\mathbf{p}^{NE}) = res_{\vartheta}^{NE}(\mathbf{p}^{NE}), ~\forall \vartheta \in \Theta, \end{equation}
\normalsize 
\vspace{-0.15in}

\noindent where $res_{\vartheta}^{PA}(\mathbf{p}^{NE})$ is defined in \eqref{eq:prop_alloc_energy}. In the following analysis, we obtain the mixed-strategy NE competing probabilities $\mathbf{p}^{NE}$ by solving Equation \eqref{eq:condition_PA_NE}. We further distinguish cases with respect to the available RESs production, TOU tariffs, and consumers' types.


\subsubsection*{\textbf{Case $1$: $\bm{\mathcal{RE}}$ exceeds $\bm{D^{Total}}$}}

As consumers have knowledge of $\mathcal{RE}$ and $D^{Total}$, it is straightforward to show that the dominant-strategy for all consumers is to schedule their daily flexible loads during daytime. As a result, the competing probabilities that lead to equilibrium states are equal to $p_{\vartheta}^{d,NE} = 1$ for all consumer types $\vartheta \in \Theta$.


\subsubsection*{\textbf{Case $2$: $\bm{\mathcal{RE}}$ is lower than $\bm{D^{Total}}$}} 

In this case, the strategies of the consumers depend on their respective risk aversion degrees and the TOU tariffs. We define two complementary subsets of consumer types, depending on their risk aversion degrees: $\Sigma_1 = \Bigl\{\vartheta \in \Theta : \varepsilon_{\vartheta} \geq \gamma/\beta \Bigr\} \subset \Theta$, and $\Sigma_2 = \Bigl\{ \vartheta \in \Theta :1\leq \varepsilon_{\vartheta} < \gamma/\beta \Bigr\} \subset \Theta$.


Firstly, the dominant strategy for all consumers $i$ whose type $\vartheta_i$ is in the set $\Sigma_1$ is to schedule their daily flexible loads during daytime, i.e., to play the pure strategy $A_i = d$ with probability $p_{\vartheta_i}^{d,NE} = 1$. 
%Their expected aggregate daytime load schedule at NE is $D^{Total}_{\Sigma_1£\mathcal{S}$=N\sum_{\theta \in \Sigma_1}r_{\theta} E_{\theta}$.

Secondly, the strategies of the consumers $i$ whose type $\vartheta_i$ is in the set $\Sigma_2$ depend on their daily flexible loads and risk-aversion degrees. We define two distinct subsets of consumer types in $\Sigma_2$: $\Sigma_{2,1} = \left\{ \vartheta \in \Sigma_2 : E_{\vartheta} > \mathcal{RE}\frac{(\gamma-1)}{(\gamma-\varepsilon_{\vartheta}\beta)} \right\}$ and $\Sigma_{2,2} = \left\{ \vartheta \in \Sigma_2 : E_{\vartheta} \leq \mathcal{RE}\frac{(\gamma-1)}{(\gamma-\varepsilon_{\vartheta}\beta)}\right\}$.

%$\Sigma_{2,1} = \left\{ \vartheta \in \Sigma_2 : E_{\vartheta} > \left(\mathcal{RE}-D^{Total}_{\Sigma_1}\right)\frac{(\gamma-1)}{(\gamma-\varepsilon_{\vartheta}\beta)} \right\}$ and $\Sigma_{2,2} = \left\{ \vartheta \in \Sigma_2 : E_{\vartheta} \leq \left(\mathcal{RE}-D^{Total}_{\Sigma_1}\right)\frac{(\gamma-1)}{(\gamma-\varepsilon_{\vartheta}\beta)}\right\}$

For consumers $i$ whose type $\vartheta_i$ is in the set $\Sigma_{2,1}$, the dominant strategy is to schedule their daily flexible loads during nighttime, i.e., to play the pure strategy $A_i=n$ with probability $p^{n,NE}_{\vartheta_i}=1$ and $A_i=d$ with probability $p^{d,NE}_{\vartheta_i}=0$.

For consumers $i$ whose type $\vartheta_i$ is in the set $\Sigma_{2,2}$, a mixed-strategy NE under the PA policy exists if and only if the following condition holds:

 \vspace{-0.1in} 
 \small
\begin{equation}\label{eq:relation_E_0_E_1_pa_ne_extra_demand}
\begin{split}
& \mathcal{RE}\frac{(\gamma-1)}{(\gamma-\varepsilon_{\vartheta}\beta)}-E_{\vartheta}  = \mathcal{RE}\frac{(\gamma-1)}{(\gamma-\varepsilon_{\tilde{\vartheta} }\beta)}-E_{\tilde{\vartheta}} , \ \forall \vartheta , \tilde{\vartheta} \in \Sigma_{2,2}.
\end{split}
 \end{equation}
\normalsize 
\vspace{-0.1in}

\noindent Assuming that all consumers of the same type play the same mixed strategy, the competing probabilities that lead to NE states lie in the range $p^{min}_{\vartheta} \leq p^{d,NE}_{\vartheta} \leq p^{max}_{\vartheta}$ for all consumer types $\vartheta \in \Sigma_{2,2}$ with:

 \vspace{-0.1in} 
 \footnotesize
   \begin{align}
& p^{max}_{\vartheta} = \min \left\{1,\frac{\frac{\mathcal{RE}(\gamma-1)}{(\gamma-\varepsilon_{\vartheta}\beta)}-D^{Total}_{\Sigma_1}}{N~ r_{\vartheta}~ E_{\vartheta}}\right\},
    \label{eq:plmaxbound} \\
&  p^{min}_{\vartheta}= \max \left\{0,\frac{\frac{\mathcal{RE}(\gamma-1)}{(\gamma-\varepsilon_{\vartheta}\beta)}-D^{Total}_{\Sigma_1 \bigcup \Sigma_{2,2}\setminus \{\vartheta\}}}{N~ r_{\vartheta}~E_{\vartheta}} \right\},\label{eq:plminbound}
  \end{align}
\normalsize  
where for any subset of consumer types $\mathcal{S} \subset \Theta$, $D^{Total}_{\mathcal{S}}$ represents the maximum aggregate daytime demand of consumers whose type is in $\mathcal{S}$, e.g., $D^{Total}_{\Sigma_1}=N\sum_{\theta \in \Sigma_1}r_{\theta} E_{\theta}$.


 As a result, the expected aggregate daytime demand, $D^{d,NE}$ at NE is

\vspace{-0.1in} 
\footnotesize
\begin{align}
& D^{d,NE} = D^{Total}_{\Sigma_1} \nonumber \\
& + \min \left\{ D^{Total}_{\Sigma_{2,2}} , \max \left\{\frac{N\left(\frac{\mathcal{RE}(\gamma-1)}{(\gamma-\varepsilon_{\vartheta}\beta)}-E_{\vartheta}-D^{Total}_{\Sigma_1}\right)}{(N-1)},0 \right\} \right\}. \label{eq:demand1_2c}
\end{align}
\normalsize 
%\vspace{-0.1in}

\begin{remark} \label{rem:risk_degrees_relation}
Note that condition \eqref{eq:relation_E_0_E_1_pa_ne_extra_demand} can hold, and therefore a NE can exist, only if for any pair $\vartheta , \tilde{\vartheta} \in \Sigma_{2,2}$ such that $\vartheta \leq \tilde{\vartheta}$, it holds that $\varepsilon_{\vartheta} \leq \varepsilon_{\tilde{\vartheta}}$. Since by assumption, $E_{\vartheta} \leq E_{\tilde{\vartheta}}$, this means that consumers with lower daytime energy demand levels should be more risk-seeking than those with higher ones.
\end{remark}
%\vspace{-0.15in}
\begin{remark}\label{rem:risk_seeking}In particular, if all consumers whose type is in $\Sigma_{2,2}$ are risk-seeking (i.e., $\varepsilon_{\vartheta}= 1, \forall \vartheta \in \Sigma_{2,2}$), a NE can only exist if  $E_{\vartheta}=E_{\tilde{\vartheta}} , \ \forall \vartheta , \tilde{\vartheta} \in \Sigma_{2,2}$.
\end{remark}

%Finally, the social cost is given by:
%\vspace{-0.15in} 

%\begin{small}
	%\begin{align}
	%C(\mathbf{p}^{NE}) &=  \min\{\mathcal{RE}, D^d(\mathbf{p^{NE}})\} c^{RES} \nonumber \\&+  \max\{0,D^d(\mathbf{p^{NE}})-\mathcal{RE}\}c^{grid,d} \nonumber \\
	%&+N \left[ \sum_{\vartheta \in \Theta} r_{\vartheta }~ p^{n,NE}_{\vartheta}~ \varepsilon_{\vartheta }~E_{\vartheta}\right] c^{grid,n}.
	%\label{eq:social_cost_pa_extra_demand}
	%\end{align}
%\end{small}



\section{Centralized Energy Sharing Mechanism}\label{sec:coordinated}

In this section we study an ideal centralized scheduling problem, in which an energy community manager with perfect knowledge of the available energy sources and types of the consumers in the community, centrally schedules their daily flexible loads.

\subsection{Problem Formulation}

%Therefore, the aggregate amount of daytime loads is equal to $D^{Total}$. 

Based on the available information, the community manager aims at finding the optimal load schedule of each consumer type, which minimize the social cost of the community under the chosen PA and payment policy. The community's social cost $C^{PA}(\bm{p})$ can be expressed as a function of the \textit{expected} aggregate daytime energy demand ($D^d(\bm{p})$) and nighttime energy demand ($D^n(\bm{p})$) of the community (as defined in Section \ref{sec:game_def}), such that:

\vspace{-0.15in} 
\begin{small}
	\begin{align}
	C^{PA}(\bm{p}) &=  \min\{\mathcal{RE}, D^d(\bm{p})\} \cdot c^{RES} \nonumber \\
    &+ \max\{0,D^d(\bm{p})-\mathcal{RE}\} \cdot c^{grid,d} + D^n(\bm{p}) \cdot c^{grid,n},
\label{eq:social_cost_pa_extra_demand}
	\end{align}
\end{small}

\noindent where the probabilities $p^d_{\vartheta}$ and $p^n_{\vartheta}$ (as defined in Section \ref{sec:game_def}) can be interpreted as the proportion of consumers of type $\vartheta$ that the community manager schedules during daytime and nighttime, respectively.
Although this objective cost is non-convex, we observe that during daytime, for any expected aggregate load schedule, the community manager minimizes the cost from grid imports. Therefore, by introducing the optimization variable $D^{grid}$ representing the expected aggregate grid imports during daytime, we can write the community manager's optimal load scheduling problem under the PA policy as a linear optimization problem, as follows:

 \vspace{-0.1in} 
 \begin{small}
 \begin{subequations} \label{eq:social_cost_x_opt}
\begin{alignat}{2}
& \min_{\mathbf{p},D^{grid}} \ && c^{grid,d}  D^{grid} + c^{RES}  \left(N\sum_{\vartheta \in \Theta} r_{\vartheta} p_{\vartheta}^d E_{\vartheta} - D^{grid}\right) \nonumber \\
& \quad && + c^{grid,n}  N \sum_{\vartheta \in \Theta} r_{\vartheta} {p}_{\vartheta}^{n}U_{\vartheta} \label{eq:opt_1} \\
 & \text{s.t. } &&  p^{d}_{\vartheta} + p^{n}_{\vartheta} = 1 , \ \forall \vartheta \in \Theta, \label{eq:opt_2.1} \\
  & \quad && 0 \leq p^{d}_{\vartheta},  p^{n}_{\vartheta},  \ \forall \vartheta \in \Theta, \label{eq:opt_2.2} \\
 & \quad && D^{grid} \geq \mathcal{ER} - N \sum_{\vartheta \in \Theta} r_{\vartheta} {p}_{\vartheta}^{d} E_{\vartheta}, \label{eq:opt_3.1} \\
 & \quad && D^{grid} \geq 0. \label{eq:opt_3.2}
 \end{alignat}
 \end{subequations}
\end{small} \vspace{-0.1in}  

\noindent This optimization problem minimizes the social cost of the community \eqref{eq:opt_1}, subject to constraints on the daytime and nighttime probabilities \eqref{eq:opt_2.1}-\eqref{eq:opt_2.2}  as well as to lower bounds on the expected aggregate grid imports during daytime \eqref{eq:opt_3.1}-\eqref{eq:opt_3.2}.

\subsection{Solution Analysis} \label{sec:centralsol}

In the following, we provide insights and analytical formulations of the optimal solutions $\mathbf{p^{*}}$ of this centralized mechanism in different cases. The proofs are available in the {Appendix \ref{appendix:dual}} of \cite{arxiv_version}.
 
\subsubsection*{\textbf{Case $1$: $\bm{\mathcal{RE}}$ exceeds $\bm{D^{Total}}$}}

In this trivial case, the optimal solutions to the C-ESM is to schedule all consumers' daily flexible loads during daytime, such that $p^{d,*}_{\vartheta}=1$, $\forall \vartheta \in \Theta$, and the expected grid imports $D^{grid,*} =0$.

\subsubsection*{\textbf{Case $2$: $\bm{\mathcal{RE}}$ is lower than $\bm{D^{Total}}$}}

In this case, it is optimal for the centralized ESM to schedule loads during the day so that the total RES capacity is fully utilized. To perform the analysis, we use the two complementary subsets of consumer types, $\Sigma_1$ and $\Sigma_2$, as those are defined in Section \ref{sec:gameanalysis}. 

For all consumers whose type $\vartheta \in \Sigma_1$, it is optimal for the community to schedule them during daytime, such that $p^{d,*}_{\vartheta}=1 $. For the optimal load schedule of the remaining consumers whose type $\vartheta \in \Sigma_2$, we observe that the consumer types are scheduled during daytime in order of increasing risk aversion (i.e., decreasing $\varepsilon_\vartheta$), until the local RESs production is fully utilized. Therefore, the optimal competing probabilities for the consumers whose types are in $\Sigma_2=\{\tilde{\vartheta}^1, \tilde{\vartheta}^2, \dots, \tilde{\vartheta}^K \}$, can be expressed as:

 \vspace{-0.1in} 
 \footnotesize
\begin{align}
 & p^{d,*}_{\tilde{\vartheta}^k} = \max \Bigg\{ \min \Bigg\{ 1, \dfrac{\left( \mathcal{RE} - D^{Total}_{\Sigma_1} - N \sum_{i=1}^{k-1}r_{{\tilde{\vartheta}^i}} E_{{\tilde{\vartheta}^i}} p^{d,*}_{\tilde{\vartheta}^i}  \right)}{N r_{{\tilde{\vartheta}^k}} E_{{\tilde{\vartheta}^k} }}\Bigg\} , 0 \Bigg\} , \nonumber \\
&  \forall k \in \{1,...,K\},
\end{align}
\normalsize 
\vspace{-0.1in}  

\noindent where the consumer types in $\Sigma_2$ are ordered such that $\varepsilon_{\tilde{\vartheta}^1} \geq \varepsilon_{\tilde{\vartheta}^2} \geq ... \geq \varepsilon_{\tilde{\vartheta}^K}$.

%If $N\sum_{{\vartheta} \in \Sigma_1}r_{{\vartheta}} E_{{\vartheta}} \geq \mathcal{RE}$, i.e., the daytime consumption of the consumers whose type $\vartheta \in \Sigma_1$ fully utilizes the RES capacity, then the solution to this optimization problem is trivial, and for all consumers whose type $\vartheta \in \Sigma_2$, $p^{}_{RES,\vartheta}=0$.


\section{Efficiency Loss of D-ESM vs. C-ESM}\label{sec:efficiency}
The (in)efficiency of equilibrium strategies in the D-ESM compared to the optimal C-ESM solution is quantified by the Price of Anarchy (PoA) metric \cite{Koutsoupias09}, representing the ratio of the worst case social cost among all mixed strategy NE, denoted as $C^{PA,NE}_{WC}$, over the optimal minimum social cost of the C-ESM, such that:

%\vspace{-5pt}
 \vspace{-0.1in} 
 \small
\begin{align}
 \hspace{-5pt} \textit{PoA}  = \frac{C^{PA,NE}_{WC}}{C^{PA}(\mathbf{p^{^*}})}.
\label{eq:poa_pa}
\end{align}
\normalsize 
\vspace{-0.1in}  

First observe that $C^{PA}(\mathbf{p^{^*}})$ is uniquely determined for each particular case (Section \ref{sec:coordinated}). Now, in order to obtain $C^{PA,NE}_{WC}$ when there exist multiple possible NE, we can maximize the social cost $C^{PA}(\mathbf{p^{NE}})$ (Eq. \eqref{eq:social_cost_pa_extra_demand}) with respect to $\mathbf{p^{NE}}$. 


%Note that $C^{PA,NE}(\mathbf{p^{NE}})$ takes its optimal value (i.e, minimum value) when the night-time cost is minimized. This solution coincides with the optimal centralized solution and thus in this case PoA takes the optimal (unity) value. 


 %The last observation is the fact that the demand $D^{PA,NE}$ is constant with respect to $\mathbf{p^{PA,NE}}$. Also, in the special case of the risk-seeking consumers, from Eq. \eqref{eq:social_cost_pa_sc}, the social cost is constant with respect to the probabilities $\mathbf{p^{PA,NE}}$ for each case of energy profile values. Thus, $C^{PA,NE}_w$ is given by Eq. \eqref{eq:social_cost_pa_sc}.

%$2$. If having risk-conservative consumers, there exist multiple possible equilibria in the sub-cases 2(a) and 2(c) as well as in case 4. Of course, the multiple combinations of probabilities can lead to NE with possibly different social cost values. Based on the Remark \ref{rem:risk_degrees_relation}, the existence of NE is possible if consumers with lower energy demands have lower risk aversion degrees. Thus, the night cost is minimized if the optimal probabilities for RES take lower values for consumers with lower energy demands. 

\begin{algorithm}
\footnotesize
    \caption{$k$-SALSA}
    \label{alg:overall}
    \textbf{Input:} Private dataset $X=(x_1,\dots,x_n)$, auxiliary dataset $X_0$ for GAN model training, integer $k>1$ (assume $n = mk$ for integer $m$ without loss of generality), number of iterations $T$, loss ratio parameter $\lambda$ \\
    \textbf{Output:} Synthetic dataset $\tilde{X}$ of size $m$ with $k$-anonymity

    \begin{algorithmic}[1]
        \State Train a GAN generator $G$ and a GAN inversion encoder $E$ on $X_0$
        \State Obtain latent code $w_i = E(x_i)$ for each $i\in [n]$ and let $W=\{w_i\}_{i=1}^n$
        \State $(C_1,\dots,C_m)= \textsf{SameSizeClustering}(W, k)$  \Comment{$C_j\subset W$, $|C_j|=k$, $|C_j\cap C_{j'\neq j}|=0$, $\forall j$}
      
        \State Initialize $\tilde{X} = \emptyset$
        \For {each cluster $j\in [m]$}
        \State Let $C_j = (w_1',\dots,w_k')$, and $x_i'$ the original image of $w_i'$ for each $i$
        \State Compute $w_0 = \frac{1}{k} \sum_{i=1}^{k} w'_i$ and generate $x_0 = G(w_0)$
        \State Initialize $w_\text{avg}^{(0)} = w_0$
        \For {each iteration $t \in [T]$}
            \State Generate $x_\text{avg}^{(t-1)} = G(w_\text{avg}^{(t-1)})$
            \State Compute content loss $\mathcal{L}_\text{content}(x_0, x_\text{avg}^{(t-1)})$ using Eq.~\ref{eq:loss-content} 
            \State Compute local style alignment loss $\mathcal{L}_\text{style}((x'_1,\dots,x'_k), x_\text{avg}^{(t-1)})$ using Eq.~\ref{eq:loss-style}
            \State Compute total loss $\mathcal{L}_\text{total}=\lambda \mathcal{L}_\text{content}+(1-\lambda)\mathcal{L}_\text{style}$
            \State Update $w_{\text{avg}}^{(t)}$ using $w_{\text{avg}}^{(t-1)}$ and the gradient $\nabla_{w_\text{avg}^{(t-1)}} \mathcal{L}_\text{total}$
        \EndFor
        \State Add $G(w_{\text{avg}}^{(T)})$ to $\tilde{X}$
        \EndFor
    \State    \Return $\tilde{X}$

    \end{algorithmic}
\end{algorithm}

 \section{Benchmarks and Evaluation}
\label{sec:eval}

We evaluate \krakenSpace to answer the following set of questions:
\begin{itemize}
\item How much improvement does partial evaluation and our implemented compiler optimizations give \kraken? %(\S \ref{sec:eval2})
\item How much faster is our purely functional f-expr language, \krakenSpace, compared to other implementations of fexprs? %(\S \ref{sec:eval1} - \ref{sec:eval2})
\item How does \kraken's performance, with its fexprs, compare to macros? %(\S \ref{sec:eval1}, \S \ref{sec:eval3})
\item How do the different partial evaluation mechanisms/optimizations in \krakenSpace contribute towards reduction in overall runtime?
%\item What does \krakenSpace do internally when we create a data structure and evaluate it for some function? (\S \ref{sec:casestudy})
\end{itemize}

\textbf{Experimental Setup}: 
We ran these experiments in a reproducible Nix environment on a NixOS install \cite{10.1145/1411203.1411255} (Kernel 6.0.0) on a laptop with 8 cores / 16 threads and 64 GB of RAM.
Our code contains the scripts and Nix Flakes needed to reproduce the exact set of dependencies to run our tests.
%The code can be found at \url{https://github.com/limvot/kraken}.

The Kraken benchmarks were run using both the Wasmtime and WAVM WebAssembly engines for most benchmarks.
The Wasmtime WebAssembly engine is one of the most popular, developed by the Bytecode Alliance itself, and uses the CraneLift code generation backend.
The WAVM WebAssembly engine is interesting for its use of LLVM, and it often produces the fastest code on benchmarks but has a higher startup time.
We eliminated the Cfold Wasmtime benchmark due to problems running out of stack space (a known property of the Cfold benchmark).

\textbf{Benchmarks}: 
To showcase the capability of Kraken, we created benchmarks that are commonly implemented in functional languages and have been used as benchmarks in other papers \cite{reinking2021perceus, 10.1145/3547646}.
The benchmarks are
\begin{itemize}
\item Fib - Calculating the nth Fibonacci number
\item RB-Tree - Inserting n items into a red-black tree, then traversing the tree to sum its values
\item Deriv - Computing a symbolic derivative of a large expression
\item Cfold - Constant-folding a large expression
\item NQueens - Placing n number of queens on the board such that no two queens are diagonal, vertical, or horizontal from each other
\end{itemize}
All benchmarks besides Fibonacci use the fexpr version of match for pattern matching in \kraken, which is equivalent to the macro version in NewLisp. We also RB-Tree using NewLisp's~\cite{mueller2018newlisp} version of fexpr match. We modified the sizes of the problems presented to the benchmark to account for the longer running times of some of the less-optimized implementations.
The code for Kraken and NewLisp is very similar, and we should note that it is very unidiomatic NewLisp.
Our goal was not to compare Kraken and NewLisp as implementation languages for Red-Black Trees, but to stress test a single reasonably complex fexpr/macro, namely pattern matching.
% \textbf{Comparison with other languages}: We evaluated \krakenSpace against a language that contains f-exprs, as well as against itself with various optimizations disabled. The only other language we could find which contains a real f-expr mechanism is NewLisp~\cite{mueller2018newlisp} and so we ported \kraken's benchmark implementation to NewLisp.

%The six state-of-the-art languages are Java 17.0.1, Swift 5.4.2, Koka 2.3.2, C++, Haskell 8.10.7, and OCaml 4.12.
%The language choices were taken directly from Perceus reference-counting paper \cite{reinking2021perceus}.
%The Fibonacci benchmark additionally tests Python 3.9.11 and Chez Scheme 9.5.4.
%Koka, Ocaml and Haskell are good comparison points as statically-typed, compiled, functional programming languages, while Chez Scheme is a good comparison point as a mature and industrial strength dynamically-typed Scheme implementation known for its performance. 
%\subsection{Basic Level Comparison}
\subsection{The Effect of Partial Evaluation on Eval Calls}

\begin{table}[h]
\caption{Number of eval calls with no partial evaluation for Fexprs}
	\begin{tabular}{||c | c c c c c ||} 
		\hline
		&Evals & Eval w1 Calls & Eval w0 Calls & Comp Dyn & Comp Dyn\\ 
        & & & & w1 Calls & w0 Calls\\ [0.5ex] 
		\hline\hline
		Cfold 5 & 10897376 & 2784275 & 879066  & 1 & 0 \\ 
		\hline
		  Deriv 2  & 11708558 & 2990090 & 946500 & 1 & 0 \\ 
        \hline
		  NQueens 7 & 13530241 & 3429161 & 1108393 & 1 & 0 \\ 
    \hline
		  Fib 30 & 119107888 & 30450112 & 10770217 & 1 & 0 \\ 
    \hline
		  RB-Tree 10 & 5032297 & 1291489 & 398104 & 1 & 0 \\ 
		\hline
	\end{tabular}
    \label{npe:calls}
 \end{table}

As mentioned before, using fexprs without partial evaluation will prelude optimization and cause a massive amount of repeated work. Table \ref{npe:calls} and Table \ref{pe:calls} show the number of calls to the \krakenSpace runtime's eval function, the number of times the runtime's eval function executed a call to an applicative with wrap\_level=1, the number of times the runtime's eval function executed a call to an operative with wrap\_level=0, the number of compiled dynamic calls to applicatives with wrap\_level=1, and the number of compiled dynamic calls to operatives with wrap\_level=0.
These are shown for \krakenSpace test cases with partial evaluation turned off and turned on. 
\begin{table}[h]
\caption{Number of eval calls in Partially Evaluated Fexprs}
	\begin{tabular}{||c | c c c c c ||} 
		\hline
		&Evals & Eval w1 Calls & Eval w0 Calls & Comp Dyn & Comp Dyn\\ 
        & & & & w1 Calls & w0 Calls\\ [0.5ex] 
		\hline\hline
		Cfold 5 & 0 & 0 & 0  & 0 & 0 \\ 
		\hline
		  Deriv 2  & 0 & 0 & 0 & 2 & 0 \\ 
        \hline
		  NQueens 7 & 0 & 0 & 0 & 0 & 0 \\ 
    \hline
		  Fib 30 & 0 & 0 & 0 & 0 & 0 \\ 
    \hline
		  RB-Tree 10 & 0 & 0 & 0 & 10 & 0 \\ 
		\hline
	\end{tabular}
    \label{pe:calls}
 \end{table}

\begin{table}[h]
\caption{Number of calls to the runtime's eval function for RB-Tree. The table shows the non-partial evaluation numbers -> partial evaluation numbers.}
	\begin{tabular}{||c | c c c c c ||} 
		\hline
		&Evals & Eval w1 Calls & Eval w0 Calls & Comp Dyn & Comp Dyn\\ 
        & & & & w1 Calls & w0 Calls\\ [0.5ex] 
		\hline\hline
		  RB-Tree 7 & 2952848 -> 0 & 757932 -> 0 & 233513 -> 0 & 1 -> 7 & 0 -> 0\\ 
        \hline
		  RB-Tree 8 & 3532131 -> 0 & 906548 -> 0 & 279379 -> 0 & 1 -> 8 & 0 -> 0\\ 
        \hline
		  RB-Tree 9 & 4278001 -> 0 & 1097965 -> 0 & 3383831 -> 0 & 1 -> 9 & 0 -> 0\\ 
		\hline
	\end{tabular}
    \label{pe:rb}
    \vspace{-4mm}
 \end{table}

Without partial evaluation, no compilation can be done because it is impossible to tell if arguments to calls will be evaluated. In all benchmarks, partial evaluation removed all calls to the runtime's eval function, resulting in a completely compiled program. Looking at RB-Tree, there are over a million calls to combiners with wrap level 1 (normal functions), and 398,000 calls to combiners with wrap level 0 (operatives replacing macros). This massive blowup in the number of calls is due to the repeated and exponential re-execution of macro-like-combiners in the definition of other macro-like-combiners, as discussed in the Introduction.

The non-partially-evaluated benchmarks show 1 compiled dynamic call to an applicative (its the first call into eval) and 0 compiled dynamic calls to operatives, because there is no compilation at all. For the partially evaluated benchmarks, there are a few compiled dynamic calls to applicatives due to higher-order function use in the benchmarks, and there are no compiled dynamic calls to operatives, as all operative use has been eliminated.
We also varied the inputs for RB-Tree shown in Table \ref{pe:rb} to give a sense for how the number scale with respect to input size.

The incredible slowdown implied by these tables comes to full fruition in our RB-Tree test in Fig.~\ref{fig:kraken_nqueens_rbtree}.
We kept this run shorter because Kraken's non-partial-evaluating interpreter takes an incredibly long time even for 100 insertions (40 minutes).
The compounding layers of repeated macro-like operative calls in the non-partially-evaluated Kraken version cause a ~70,000x slowdown relative to the partial evaluated, optimized, and compiled version.
For the remaining benchmarks, we remove the naive interpreted \krakenSpace version, as in each case its performance is so bad as to blow out the graph and make it impossible to do any comparison.
In our optimized Kraken, our partial evaluation algorithm is able to fully collapse these levels of inefficiency, evaluate and inline the results, and give the backend more specialized code to optimize, emitting a compiled version that handily beats not only the NewLisp-fexpr implementation but even the NewLisp-macro implementation, as can be seen in Fig.~\ref{fig:kraken_vs_world_fib}.
We kept the benchmark sizes small in this test because the stack limits of NewLisp prevent sizes larger then ~880, while the Tail Call Elimination performed by the \krakenSpace compiler allows us to run much larger benchmarks, including the run of 4,800,000 inserts to the RB-Tree.
This result shows the dramatic effect of partial evaluation and compiler optimizations on runtime for \kraken. Our technique takes the performance of a fully fexpr based language from being completely infeasible to being faster than a macro-based dynamic scripting language currently in use.
% \begin{center}
% \begin{table}[ht]
% \caption{Number of call to the runtime's eval function for Fib. The table shows the non-partial evaluation numbers -> partial evaluation numbers}
% 	\begin{tabular}{||c | c c c c c ||} 
% 		\hline
% 		&Evals & Eval w1 Calls & Eval w0 Calls & Comp Dyn w1 Calls & Comp Dyn w0 Calls\\ [0.5ex] 
% 		\hline\hline
% 		Fib 10 & 8468 -> 0 & 2167 -> 0  & 777 -> 0 & 1 -> 0 & 0 -> 0 \\ 
% 		\hline
% 		  Fib 15  & 87916 -> 0 & 22478 -> 0 & 7961 -> 0 & 1 -> 0 & 0 -> 0 \\ 
%         \hline
% 		  Fib 20 & 969010 -> 0 & 247731 -> 0 & 87633 -> 0 & 1 -> 0 & 0 -> 0 \\ 
%     \hline
% 		  Fib 25 & 10740492 -> 0 & 2745825 -> 0  & 971209 -> 0 & 1 -> 0 & 0 -> 0 \\ 
% 		\hline
% 	\end{tabular}
%     \label{pe:fib}
%  \end{table}
% \end{center}

\begin{figure}[h]
\caption{Constant Fold and Deriv}
\includegraphics[width=0.45\textwidth]{cfold_table.csv_}
\includegraphics[width=0.45\textwidth]{deriv_table.csv_}
\label{fig:kraken_const_deriv}
\vspace{-6mm}
\end{figure}
\subsection{Comparison between Kraken Versions}
Beyond the massive speedup from partial-evaluation, Fig. \ref{fig:kraken_const_deriv} and \ref{fig:kraken_nqueens_rbtree} show the effect of the various compiler optimizations we described by disabling them one by one.
 Our main four optimizations have a strong positive effect on runtime, with the exception of lazy environment instantiation. Lazy environment instantiation helps massively on fib, and some on Deriv, but generally hurts the rest slightly.


\begin{figure}[h]
\caption{N-Queens}
\includegraphics[width=0.45\textwidth]{nqueens_table.csv_}
\includegraphics[width=0.45\textwidth]{slow_rbtree_table.csv_}
\label{fig:kraken_nqueens_rbtree}
\vspace{-4mm}
\end{figure}


\subsection{Comparison against Others}


To give a general idea of our current performance, we also show a Fibonacci benchmark that mostly exercises pure function-call speed and inlining as seen in Fig. ~\ref{fig:kraken_vs_world_fib}.
We include Python and Chez Scheme to give a general idea for where an exemplar slow and an exemplar fast dynamic language would fall.
With the benefit of our partial evaluation, compilation, and leaning upon mature WebAssembly implementations, we beat both, but this should be taken with a grain of salt, as this is a very limited micro-benchmark only meant to give a general sense of the order of magnitude of our performance.



\label{sec:eval1}
\begin{figure}[h]
\caption{Kraken vs. Others. Ordered by fastest to slowest}
\includegraphics[width=0.45\textwidth]{fib_table.csv_}
\includegraphics[width=0.45\textwidth]{rbtree_table.csv_}
\label{fig:kraken_vs_world_fib}
\end{figure}

%\label{sec:eval_nqueens}
%\begin{figure}[h]
%\caption{N-Queens}
%\includegraphics[width=0.45\textwidth]{nqueens_table.csv_}
%\includegraphics[width=0.45\textwidth]{slow_nqueens_table.csv_}
%\label{fig:kraken_nqueens}
%\end{figure}

%\label{sec:eval_nqueens}
%\begin{figure}[h]
%\caption{Kraken, N-Queens, absolute value and log-scale}
%\includegraphics[width=0.45\textwidth]{nqueens_table.csv_}
%\includegraphics[width=0.45\textwidth]{nqueens_table.csv_log}
%\label{fig:kraken_nqueens}
%\end{figure}
%\label{sec:eval_nqueensp}
%\begin{figure}[h]
%\caption{Kraken, N-Queens, absolute value and log-scale}
%\includegraphics[width=0.45\textwidth]{slow_nqueens_table.csv_}
%\includegraphics[width=0.45\textwidth]{slow_nqueens_table.csv_log}
%\label{fig:kraken_nqueensp}
%\end{figure}

%\label{sec:eval_cfold}
%\begin{figure}[h]
%\caption{C-Fold}
%\includegraphics[width=0.45\textwidth]{cfold_table.csv_}
%\includegraphics[width=0.45\textwidth]{slow_cfold_table.csv_}
%\label{fig:kraken_cfold}
%\end{figure}
%\label{sec:eval_cfold}
%\begin{figure}[h]
%\caption{Kraken, C-Fold, absolute value and log-scale}
%\includegraphics[width=0.45\textwidth]{cfold_table.csv_}
%\includegraphics[width=0.45\textwidth]{cfold_table.csv_log}
%\label{fig:kraken_cfold}
%\end{figure}
%\label{sec:eval_cfoldp}
%\begin{figure}[h]
%\caption{Kraken, C-Fold, absolute value and log-scale}
%\includegraphics[width=0.45\textwidth]{slow_cfold_table.csv_}
%\includegraphics[width=0.45\textwidth]{slow_cfold_table.csv_log}
%\label{fig:kraken_cfoldp}
%\end{figure}

%\label{sec:eval_deriv}
%\begin{figure}[h]
%\caption{Deriv}
%\includegraphics[width=0.45\textwidth]{deriv_table.csv_}
%\includegraphics[width=0.45\textwidth]{slow_deriv_table.csv_}
%\label{fig:kraken_deriv}
%\end{figure}
%\label{sec:eval_deriv}
%\begin{figure}[h]
%\caption{Kraken, Deriv, absolute value and log-scale}
%\includegraphics[width=0.45\textwidth]{deriv_table.csv_}
%\includegraphics[width=0.45\textwidth]{deriv_table.csv_log}
%\label{fig:kraken_deriv}
%\end{figure}
%\label{sec:eval_derivp}
%\begin{figure}[h]
%\caption{Kraken, Deriv, absolute value and log-scale}
%\includegraphics[width=0.45\textwidth]{slow_deriv_table.csv_}
%\includegraphics[width=0.45\textwidth]{slow_deriv_table.csv_log}
%\label{fig:kraken_derivp}
%\end{figure}

%\subsection{Comparison against state-of-the-art languages}
%\label{sec:eval3}

%\begin{figure}[h]
%\caption{Kraken vs. S.o.t.A.}
%\includegraphics[width=0.45\textwidth]{cfold_table.csv_}
%\includegraphics[width=0.45\textwidth]{rbtree_table.csv_}
%\label{fig:kraken_vs_world1}
%\end{figure}

%\begin{figure}[h]
%\caption{Kraken vs. S.o.t.A.}
%\includegraphics[width=0.45\textwidth]{deriv_table.csv_}
%\includegraphics[width=0.45\textwidth]{nqueens_table.csv_}
%\label{fig:kraken_vs_world2}
%\end{figure}

% \begin{figure}[h]
% \caption{Kraken vs. S.o.t.A. (Log)}
% \includegraphics[width=0.45\textwidth]{cfold_table.csv_log}
% \includegraphics[width=0.45\textwidth]{rbtree_table.csv_log}
% \label{fig:kraken_vs_world_log_1}
% \end{figure}
% \begin{figure}[h]
% \caption{Kraken vs. S.o.t.A. (Log)}
% \includegraphics[width=0.45\textwidth]{deriv_table.csv_log}
% \includegraphics[width=0.45\textwidth]{nqueens_table.csv_log}
% \label{fig:kraken_vs_world_log_2}
% \end{figure}

%As we noted before with the Fib(30) microbenchmark in Section \ref{sec:eval1}, we remain significantly slower than state-of-the-art compiled languages.
%This is particularly true for memory-intensive benchmarks due to our naive reference-counting and malloc/free implementations.
%However, our results are of a similar order of magnitude to the difference between the state-of-the-art compiled languages and dynamic scripting languages, like Python's results in the Fib(30) microbenchmark.
%We assert that is not a fundamental limitation because the classic f-expr slowness is being eliminated, as shown by Fig. \ref{fig:kraken_vs_newlisp1} and Fig. \ref{fig:kraken_vs_newlisp2}.
%In future work, we plan to expand our compile-time analysis and optimization to implement a modified, dynamic-language version of Perceus reference counting.
%With this change, we belive \krakenSpace can be competitive with these state-of-the-art languages.

%\subsection{Case Study: Red-Black Tree}
%\label{sec:casestudy}

%\begin{figure}[h]
%\caption{Kraken vs. S.o.t.A. - RB-Tree Focus}
%\includegraphics[width=0.4\textwidth]{rbtree_table.csv_}
%\includegraphics[width=0.4\textwidth]{rbtree_table.csv_log}
%\label{fig:kraken_vs_world_rbtree}
%\end{figure}


%To evaluate our partial evaluation algorithm and compiler, we extracted the benchmarks used by the Koka language project from their code repository and added Kraken versions, as well as implementing a naive Fibonacci microbenchmark ourselves to evaluate pure function call speed.\\
%With partial evaluation and the compiler optimizations listed above, we get fairly strong performance on purely numerical computations, such as the naive Fibonacci microbenchmark.
%Unfortunately, the overhead of our unsophisticated reference counting, dynamic type checking, and bounds checking causes poor performance on benchmarks involving data structures relative to mainstream programming language implementations.
%This is not a fundamental limitation, and will be addressed in future work, as recounted in the next section.
%It should be noted, however, that while the performance relative to established language implementations is very poor for the memory-intensive benchmarks (600-900x slower), we still realize a massive speedup compared to an unoptimized and non-partial-evaluated f-expr implementation (100,000x faster)!

\section{Conclusion}\label{sec:conclusion}
In this work, we focus on addressing the fundamental challenge of OOD detection tasks, which is how to fully understand the semantic discrepancy between the ID/OOD samples. We reveal that the key to success in the realistic SCOOD task is to allocate as many ID samples in the unlabeled set correctly as possible. To this end, we propose a novel uncertainty-aware optimal transport scheme that introduces class-specific energy scores as guidance for effective label assignment. Experimental results show that our method achieves better performance than previous state-of-the-art methods on SCOOD benchmarks.

\textbf{Limitations.} In addition to temperature scaling, other techniques such as feature clipping applied in ReAct~\cite{sun2021react} also enhance the performance of energy score, so how to obtain an OOD score that best fits the SCOOD task can be further explored. Moreover, a setting highly related to SCOOD has been proposed in \cite{katz2022training} and formulated as a constrained optimization problem. We will also theoretically analyze these practical OOD settings in our feature work.

% \section*{Acknowledgments}
\textbf{Acknowledgments.} 
This work is supported by National Key R\&D Program of China under Grant 2020AAA0105701, National Natural Science Foundation of China (NSFC) under Grants 61872327, Major Special Science and Technology Project of Anhui, National Natural Science Foundation of China (62033012) and Ant Group through Ant Research Intern Program.


%% !TeX spellcheck = en_US
%\section*{Code Availability Statement}
%A MATLAB implementation of the methods and simulations presented in this paper are openly available in an open-source repository available at {\small\texttt{\url{https://fish-tue.github.io/single-origin-destination-routing}}}.


\section*{Acknowledgment}
We thank Dr.\ I.\ New and F.\ Paparella for proofreading the~paper.



%\section*{Acknowledgements}
%
\bibliographystyle{IEEEtran}
\vspace{-10pt}
\bibliography{output.bib}
\appendices
\newpage

\section{Proofs for Case $2$ of the D-ESM}
\label{sec:proofsESSG}
\vspace{-0.05in}
For the consumers in $\Sigma_1$, we need to show that  $\upsilon^d_{\vartheta}(\mathbf{p})<\upsilon^n_{ \vartheta}(\mathbf{p})$, $\forall \mathbf{p}$ and $\forall \vartheta \in \Sigma_1$. Assume a consumer type $\vartheta\in \Sigma_{1}$ and that her allocated RES energy is $E'$. Then, we have that $\upsilon^d_{ \vartheta}(\mathbf{p})= E'  \cdot c^{RES}+(E_{\vartheta}-E')\cdot \gamma \cdot  c^{RES}$ and 
$\upsilon^n_{ \vartheta}(\mathbf{p})= \varepsilon_\vartheta \cdot E_{\vartheta} \cdot \beta \cdot c^{RES}$. The inequality $\upsilon^d_{\vartheta}(\mathbf{p})<\upsilon^n_{ \vartheta}(\mathbf{p})$ is then equivalent to $ E'  (1-\gamma) \cdot c^{RES} <  E_\vartheta \cdot (\varepsilon_\vartheta \cdot \beta -\gamma) \cdot c^{RES}$, which is true by assumption, since $(1-\gamma)<0$ and $(\varepsilon_\vartheta \cdot \beta -\gamma)>0$.

                                     
  Next, for the consumers in $\Sigma_{2,1}$, we need to show that $\upsilon^d_{ \vartheta}(\mathbf{p})>\upsilon^n_{ \vartheta}(\mathbf{p})$, $\forall \mathbf{p}$ and $\forall \vartheta\in \Sigma_{2,1}$. Assume a consumer type $\vartheta \in \Sigma_{2,1}$ and that her allocated RES energy is $E'$. Then, the inequality $\upsilon^d_{\vartheta}(\mathbf{p})>\upsilon^n_{ \vartheta}(\mathbf{p})$ is equivalent to the inequality $E_\vartheta >E' \frac{(\gamma-1)}{(\gamma-\varepsilon_\vartheta\beta)}$, which is true by assumption, since $E'<\mathcal{RE}$.

Now, we prove the condition of existence of a mixed strategies NE for the consumers in $\Sigma_{2,2}$. Recall that in the ESG under the PA policy, a mixed strategy NE, $\mathbf{p^{NE}}$, among consumers in $\Sigma_{2,2}$ exists under the condition
\vspace{-0.05in}

\small
\begin{equation}\label{eq:condition_PA_NE_2}
res_{\vartheta}^{PA}(\mathbf{p}^{NE}) =res_{\vartheta}^{NE}(\mathbf{p}^{NE}), \forall \vartheta \in \Sigma_{2,2}. \end{equation}
\normalsize
%Next we give the conditions such that either \eqref{eq:condition_PA_NE} holds and mixed NE exist or there exist dominant strategies. For this study, we distinguish cases with respect to the RES capacity, the risk aversion degree values and the daytime energy demand levels. 

To derive condition \eqref{eq:relation_E_0_E_1_pa_ne_extra_demand} we re-write \eqref{eq:condition_PA_NE} first with assuming that a consumer $i$ of type $\vartheta_i \in \Sigma_{2,2}$ plays the pure strategy $A_i=d$ (in \eqref{eq:probrelation1}) and second with assuming that a consumer $j$ with type $\vartheta_j \in \Sigma_{2,2} \setminus \{\vartheta_i\}$ plays the pure strategy $A_j=d$ (in \eqref{eq:probrelation2}):

\vspace{-0.1in}
\begin{small}
\begin{align}
&  \mathcal{RE}\frac{(\gamma-1)}{(\gamma-\varepsilon_{\vartheta_i}\beta)}-E_{\vartheta_i}= D^{Total}_{\Sigma_1}+\sum_{ {\vartheta'}\in \Sigma_{2,2}} r_{\vartheta'}~ (N-1)~E_{\vartheta'}~p^{d,NE}_{\vartheta'},
    \label{eq:probrelation1}\\
  &  \mathcal{RE}\frac{(\gamma-1)}{(\gamma-\varepsilon_{\vartheta_j}\beta)}-E_{\vartheta_j}=D^{Total}_{\Sigma_1}+  \sum_{ {\vartheta'}\in \Sigma_{2,2}} r_{\vartheta'} ~(N-1)~E_{\vartheta'}~ p^{d,NE}_{\vartheta'}.
    \label{eq:probrelation2}  
\end{align}
\end{small}

%Eq. \eqref{eq:condition_PA_NE} can be re-written in a similar way for any other type $\vartheta_j \in \Theta$. 
Note that to derive (\ref{eq:probrelation1}) we consider that if a consumer $i$ in $\Sigma_{2,2}$ of type $\vartheta_i$ plays  the pure strategy $A_i=d$, then, the aggregate expected daytime energy of the consumers in $\Sigma_{2,2}$, $D_{\Sigma_{2,2}}(\mathbf{p^{NE}})$ can be expressed as $E_{\vartheta_i}+ \sum_{ {\vartheta'}\in \Sigma_{2,2}} r_{\vartheta'}~ (N-1)~E_{\vartheta'}~p^{d,NE}_{\vartheta'}$ for a large number of consumers and similarly also for (\ref{eq:probrelation2}). Then, since the right-hand sides of \eqref{eq:probrelation1}-\eqref{eq:probrelation2} are equal, the left-hand sides will be also equal and \eqref{eq:relation_E_0_E_1_pa_ne_extra_demand} derives. %\eqref{eq:probrelation1}-\eqref{eq:probrelation2} 

To derive the probability bounds, we re-write \eqref{eq:condition_PA_NE} assuming that all consumers of the same type play the same mixed strategy, i.e., 

\vspace{-0.1in}
\begin{small}
\begin{align}
&  \mathcal{RE}\frac{(\gamma-1)}{(\gamma-\varepsilon_{\vartheta_i}\beta)}= D^{Total}_{\Sigma_1}+N\sum_{ {\vartheta'}\in \Sigma_{2,2}} r_{\vartheta'}~ E_{\vartheta'}~p^{d,NE}_{\vartheta'}.
    \label{eq:probrelation3}  
\end{align}
\end{small}

The minimum bound on the probability for competing for RESs, $p_{\vartheta}^{\min}$, derives by setting in (\ref{eq:probrelation3}) $p^{d,NE}_{\tilde{\vartheta}}=1$, $\forall \tilde{\vartheta}\in \Sigma_{2,2}$ with $\tilde{\vartheta}\neq \vartheta=\vartheta_i$. Similarly, the maximum bound on the probability for competing for RESs, $p_{\vartheta}^{\max}$, derives by setting in (\ref{eq:probrelation3}) $p^{d,NE}_{\tilde{\vartheta}}=0$, $\forall \tilde{\vartheta}\in \Sigma_{2,2}$ with $\tilde{\vartheta}\neq \vartheta=\vartheta_i$. 

Finally, the expression for the aggregate expected daytime energy demand given in \eqref{eq:demand1_2c} is constructed as follows. First we can write that 
\begin{align}
 D^{d,NE} = D^{Total}_{\Sigma_1} +N\sum_{ {\vartheta'}\in \Sigma_{2,2}} r_{\vartheta'}~E_{\vartheta'}~p^{d,NE}_{\vartheta'}. \label{eq:totdemand}
 \end{align}

Second, by multiplying \eqref{eq:probrelation1} with $\frac{N}{N-1}$, we obtain:


\begin{small}
\begin{align}
&  N\sum_{ {\vartheta'}\in \Sigma_{2,2}} r_{\vartheta'}~E_{\vartheta'}~p^{d,NE}_{\vartheta'}=\frac{N}{N-1}\left[\frac{\mathcal{RE}(\gamma-1)}{(\gamma-\varepsilon_{\vartheta_i}\beta)}-E_{\vartheta_i}-D^{Total}_{\Sigma_1}\right].
    \label{eq:probrelation3}
\end{align}
\end{small}

Third, by replacing \eqref{eq:probrelation3} in \eqref{eq:totdemand} we obtain \eqref{eq:demand1_2c}, where the $\min\{.\}, ~\max\{.\}$ operators account for the case that the initially obtained probability values by  \eqref{eq:probrelation1} do not lie in the range $[0,1]$ and should be set to the values $1$ or $0$, correspondingly. 

\section{Proofs for Case $2$ of the C-ESM}
\label{appendix:dual}

In this case, it is optimal for the C-ESM to schedule loads during the day so that the total RES capacity is fully utilized, i.e., the expected aggregate daytime energy demand is greater than or equal to the RES capacity:

 \vspace{-0.1in} 
 \small
\begin{align}
N \sum_{{\vartheta} \in \Theta}r_{{\vartheta}} ~E_{{\vartheta}}~p^{d}_{{\vartheta}} \geq \mathcal{RE}.
\label{eq:optimal}
\end{align}
\normalsize 
\vspace{-0.1in}


\noindent Therefore, the social cost reduces to:

 \vspace{-0.1in} 
 \begin{small}
\begin{align}
C(\mathbf{p}) &=  \mathcal{RE} \cdot c^{RES}   + \left[N \sum_{{\vartheta} \in \Theta} r_{\vartheta} ~p_{{\vartheta}}^{d}~  E_{\vartheta} - \mathcal{RE}\right]  \gamma \cdot c^{RES} \nonumber \\
& + N \left[ \sum_{{\vartheta} \in \Theta} r_{\vartheta } \left(1-p^{d}_{{\vartheta}}\right)\varepsilon_{{\vartheta} }~E_{{\vartheta}}\right] \beta \cdot c^{RES},
 \label{eq:social_cost_pa_extra_demand_2}
 \end{align}
\end{small} \vspace{-0.1in}

\noindent and the C-ESM optimization problem \eqref{eq:social_cost_x_opt} is equivalent to minimizing $ N \sum_{\vartheta \in \Theta} \left[ r_{\vartheta} E_{\vartheta} \left(\gamma - \varepsilon_{\vartheta} \beta \right) p^{d}_{\vartheta} \right] c^{RES}$, subject to constraints \eqref{eq:opt_2.1}-\eqref{eq:opt_3.2} and \eqref{eq:optimal}. Below, we derive closed-form expressions of the solutions of this linear optimization problem.

We define two complementary subsets of consumer types, depending on their risk aversion degrees: $\Sigma_1 = \Bigl\{\vartheta \in \Theta : \varepsilon_{\vartheta} \geq \gamma/\beta \Bigr\} \subset \Theta$, and $\Sigma_2 = \Bigl\{ \vartheta \in \Theta :1\leq \varepsilon_{\vartheta} < \gamma/\beta \Bigr\} \subset \Theta$.

For all consumers whose type $\vartheta \in \Sigma_1$, it is optimal for the C-ESM to schedule them during daytime, such that $p^{d,*}_{\vartheta}=1 $. Therefore, the optimal schedule for the remaining consumers whose type $\vartheta \in \Sigma_2$ can be found by solving the following linear optimization problem: 


%\vspace{-0.1in} 
 \begin{small}
 \begin{subequations} \label{eq:social_cost_x_opt_2}
\begin{alignat}{2}
& \min_{\mathbf{p}} \ &&  N \sum_{\vartheta \in \Sigma_2} \left[ r_{\vartheta} ~E_{\vartheta} \left(\gamma - \varepsilon_{\vartheta} \beta \right) p^{d}_{\vartheta} \right] c^{RES} \label{eq:opt_S2_1} \\
 & \text{s.t. } && \eqref{eq:opt_2.1}-\eqref{eq:opt_3.2} \label{eq:opt_S2_2}\\
 & \quad && N \sum_{{\vartheta} \in \Sigma_2}r_{{\vartheta}} ~E_{{\vartheta}}~p^{d}_{{\vartheta}} \geq \left( \mathcal{RE} - N \sum_{{\vartheta} \in \Sigma_1}r_{{\vartheta}} E_{{\vartheta}}\right). \label{eq:opt_S2_3} 
 \end{alignat}
 \end{subequations}
\end{small} %\vspace{-0.1in}

\noindent And the dual function of this optimization problem is 

\begin{footnotesize}
 \begin{align} \label{eq:social_cost_x_opt_2_dual}
 \max_{\lambda \geq 0}\min_{\mathbf{p}} \quad & N \sum_{\vartheta \in \Sigma_2} \left[ r_{\vartheta} E_{\vartheta} \left(\gamma - \varepsilon_{\vartheta} \beta \right) p^{d}_{\vartheta} \right] c^{RES} \nonumber \\&-\lambda\left( N \sum_{{\vartheta} \in \Sigma_2}r_{{\vartheta}} E_{{\vartheta}}p^{d}_{{\vartheta}} - \left( \mathcal{RE} - N \sum_{{\vartheta} \in \Sigma_1}r_{{\vartheta}} E_{{\vartheta}}\right)\right),
 \end{align}
\end{footnotesize} \vspace{-0.1in}

\hspace{-0.2in} subject to \eqref{eq:opt_S2_2}, where $\lambda$ represents the dual variable associated with \eqref{eq:opt_S2_3} and let $\lambda^*$ represent its optimal value.

It results that:\\
$\bullet$ for all $\vartheta \in \Sigma_{2}$ where $1 \leq \varepsilon_\vartheta < \dfrac{\gamma c^{RES} - \lambda^*}{\beta c^{RES}}$, $p^{d,*}_{\vartheta}=0$,\\
$\bullet$ for all $\vartheta \in \Sigma_2$ where $ \varepsilon_\vartheta = \dfrac{\gamma c^{RES} - \lambda^*}{\beta c^{RES}}$, $0<p^{d,*}_{\vartheta}<1$,\\
$\bullet$  for all $\vartheta \in \Sigma_2$ where $ \dfrac{\gamma c^{RES} - \lambda^*}{\beta c^{RES}} < \varepsilon_\vartheta <\dfrac{\gamma}{\beta}$, $p^{d,*}_{\vartheta}=1$.

This means that the consumer types are fully dispatched during the day in the order of increasing risk aversion degree (or decreasing $\varepsilon_\vartheta$), until constraint \eqref{eq:opt_S2_3} is satisfied. 


\section{Analysis For the ES Allocation Policy}
\subsection{Decentralized Energy Sharing Mechanism Under ES}
The analysis and proofs of this section follow similar lines as the analysis and proofs for the PA policy. Most proofs are however omitted for brevity.

In the ESG with the ES policy, a mixed-strategy NE exists under the condition:


 %\vspace{-0.1in} 
 \small
\begin{equation}\label{eq:condition_ES_NE}
rse^{ES}_{\vartheta_i}(\mathbf{p^{NE}}) =res_{\vartheta}^{NE}(\mathbf{p}^{NE}), ~\forall \vartheta \in \Theta. \end{equation}
\normalsize 
\vspace{-0.1in}  

\noindent 
%Therefore, for all cases, any existing mixed-strategy NE competing probabilities, $\mathbf{p}^{NE}$, are obtained by resolving condition \eqref{eq:condition_ES_NE}.
Let us distinguish the following cases:

\subsubsection*{\textbf{Case $1$: $\bm{\mathcal{RE}}$ exceeds $\bm{D^{Total}}$}}

As consumers have knowledge of $\mathcal{RE}$ and $D^{Total}$, it is straightforward to show that the dominant-strategy for all consumers is to schedule their daily flexible loads during daytime. As a result, the competing probabilities that lead to equilibrium states are equal to $p_{\vartheta}^{d,NE} = 1$ for all consumer types $\vartheta \in \Theta$.


\subsubsection*{\textbf{Case $2$: $\bm{\mathcal{RE}}$ is lower than $\bm{D^{Total}}$}} 

In this case, the strategies of the consumers depend on their respective risk aversion degrees and the TOU tariffs. We define two complementary subsets of consumer types, depending on their risk aversion degrees: $\Sigma_1 = \Bigl\{\vartheta \in \Theta : \varepsilon_{\vartheta} \geq \gamma/\beta \Bigr\} \subset \Theta$, and $\Sigma_2 = \Bigl\{ \vartheta \in \Theta :1\leq \varepsilon_{\vartheta} < \gamma/\beta \Bigr\} \subset \Theta$.


Firstly, the dominant strategy for all consumers $i$ whose type $\vartheta_i$ is in the set $\Sigma_1$ is to schedule their daily flexible loads during daytime, i.e., to play the pure strategy $A_i = d$ with probability $p_{\vartheta_i}^{d,NE} = 1$. Their expected aggregate daytime energy demand is then $D^{Total}_{\Sigma_1}=N\sum_{\theta \in \Sigma_1}r_{\theta} E_{\theta}$.

Secondly, the strategies of the consumers $i$ whose type $\vartheta_i$ is in the set $\Sigma_2$ depends on their daily flexible loads and risk-aversion degrees. Therefore, we define two distinct subsets of consumer types in $\Sigma_2$: $\Sigma_{2,1} = \left\{ \vartheta \in \Sigma_2 : E_{\vartheta} > \mathcal{RE}\frac{(\gamma-1)}{(\gamma-\varepsilon_{\vartheta}\beta)} \right\}$ and $\Sigma_{2,2} = \left\{ \vartheta \in \Sigma_2 : E_{\vartheta} \leq \mathcal{RE}\frac{(\gamma-1)}{(\gamma-\varepsilon_{\vartheta}\beta)}\right\}$.

%$\Sigma_{2,1} = \left\{ \vartheta \in \Sigma_2 : E_{\vartheta} > \left(\mathcal{RE}-D^{Total}_{\Sigma_1}\right)\frac{(\gamma-1)}{(\gamma-\varepsilon_{\vartheta}\beta)} \right\}$ and $\Sigma_{2,2} = \left\{ \vartheta \in \Sigma_2 : E_{\vartheta} \leq \left(\mathcal{RE}-D^{Total}_{\Sigma_1}\right)\frac{(\gamma-1)}{(\gamma-\varepsilon_{\vartheta}\beta)}\right\}$

For consumers $i$ whose type $\vartheta_i$ is in the set $\Sigma_{2,1}$, the dominant strategy is to schedule their daily flexible loads during nighttime, i.e., to play the pure strategy $A_i=n$ with probability $p^{n,NE}_{\vartheta_i}=1$, and $A_i=d$ with probability $p^{d,NE}_{\vartheta_i}=0$.

For consumers whose types are in the set $\Sigma_{2,2}$, a mixed-strategy NE with the ES policy exists if and only if the following condition holds:

 \vspace{-0.1in} 
 \small
\begin{equation}\label{eq:relation_E_0_E_1_es_ne_extra_demand}
(\gamma-\varepsilon_{\vartheta}\beta)\cdot E_{\vartheta} = (\gamma-\varepsilon_{\tilde{\vartheta} }\beta)\cdot E_{\tilde{\vartheta}} , \ \forall \vartheta , \tilde{\vartheta} \in \Sigma_{2,2}.
\end{equation}
\normalsize 


To derive condition \eqref{eq:relation_E_0_E_1_es_ne_extra_demand} we re-write \eqref{eq:condition_ES_NE} first with assuming that a consumer $i$ of type $\vartheta_i \in \Sigma_{2,2}$ plays the strategy $A_i=d$ with probability $p^{d,NE}_{\vartheta_i}=1$ (in \eqref{eq:probrelation1es}) and second with assuming that a consumer $j$ with type $\vartheta_j \in \Sigma_{2,2} \setminus \{\vartheta_i\}$ plays the strategy $A_j=d$ with probability $p^{d,NE}_{\vartheta_j}=1$ (in \eqref{eq:probrelation2es}).

\vspace{-0.1in}
\begin{small}
\begin{align} \label{eq:probrelation1es}
  D^{Total}_{\Sigma_1}+1+\sum_{ {\vartheta'}\in \Sigma_{2,2}} r_{\vartheta'}~ (N-1)~p^{d,NE}_{\vartheta'}=\frac{\mathcal{RE}(\gamma-1)}{E_{\vartheta_i}(\gamma-\varepsilon_{\vartheta_i}\beta)},
\end{align}
\end{small}
\vspace{-0.1in}

\begin{small}
\begin{align} \label{eq:probrelation2es}
  D^{Total}_{\Sigma_1}+1+\sum_{ {\vartheta'}\in \Sigma_{2,2}} r_{\vartheta'}~ (N-1)~p^{d,NE}_{\vartheta'}=\frac{\mathcal{RE}(\gamma-1)}{E_{\vartheta_j}(\gamma-\varepsilon_{\vartheta_j}\beta)}.
\end{align}
\end{small}
Then, since the right-hand sides of \eqref{eq:probrelation1es}-\eqref{eq:probrelation2es} are equal, the left-hand sides will be also equal and \eqref{eq:relation_E_0_E_1_es_ne_extra_demand} derives.


Additionally, for the consumers of type $\vartheta \in \Sigma_{2,2}$, the competing probabilities that lead to NE states lie in the range $p^{min}_{\vartheta} \leq p^{d,NE}_{{\vartheta}} \leq p^{max}_{\vartheta}$, where:

 \vspace{-0.1in} 
 \footnotesize
\begin{align}
&  p^{min}_{\vartheta}=\nonumber\\&\max \left\{0,\frac{\frac{\mathcal{RE}(\gamma-1)}{E_{\vartheta}(\gamma-\varepsilon_{\vartheta}\beta)}-
  \sum\limits_{\tilde{\vartheta} \in \Sigma_{2,2} \cup \Sigma_1 \setminus \{\vartheta\}}N r_{\tilde{\vartheta}} }{N r_{\vartheta}} \right\}, \label{eq:plminbound_appendix}\\
& p^{max}_{\vartheta} = \min \left\{1,\frac{\frac{\mathcal{RE}(\gamma-1)}{E_{\vartheta}(\gamma-\varepsilon_{\vartheta}\beta)}-\sum\limits_{\tilde{\vartheta} \in  \Sigma_1 }N r_{\tilde{\vartheta}} }{N r_{\vartheta}}\right\}.
    \label{eq:plmaxbound_appendix}
\end{align}
\normalsize  
\vspace{-0.1in}

To derive the probability bounds, we re-write \eqref{eq:condition_ES_NE} assuming that all consumers of the same type play the same mixed strategy, i.e., 

\vspace{-0.1in}
\begin{small}
\begin{align} \label{eq:probrelation3es}
  D^{Total}_{\Sigma_1}+\sum_{ {\vartheta'}\in \Sigma_{2,2}} N~r_{\vartheta'}~p^{d,NE}_{\vartheta'}=\frac{\mathcal{RE}(\gamma-1)}{E_{\vartheta_i}(\gamma-\varepsilon_{\vartheta_i}\beta)}.
\end{align}
\end{small}

The minimum bound on the probability for playing RES, $p_{\vartheta}^{\min}$, derives by setting in (\ref{eq:probrelation3es}) $p^{d,NE}_{\tilde{\vartheta}}=1$, $\forall \tilde{\vartheta}\in \Sigma_{2,2}$ with $\tilde{\vartheta}\neq \vartheta=\vartheta_i$. Similarly, the maximum bound on the probability for playing RES, $p_{\vartheta}^{\max}$, derives by setting in (\ref{eq:probrelation3es}) $p^{d,NE}_{\tilde{\vartheta}}=0$, $\forall \tilde{\vartheta}\in \Sigma_{2,2}$ with $\tilde{\vartheta}\neq \vartheta=\vartheta_i$. 

The Remarks 3 and 4, which are stated for the PA allocation policy in Section \ref{sec:gameanalysis}, also hold in case of the ES allocation policy. 

The social cost under the ES policy can be expressed as 

\footnotesize
\begin{align}
&C^{ES}(\mathbf{p^{NE}}) =  N \sum_{\vartheta \in \Theta} r_{\vartheta}~ \min\{sh(\mathbf{p^{NE}}), E_{\vartheta}\} ~p_{\vartheta}^{d,NE}~ c^{RES} \nonumber\\ + &\left[D(\mathbf{p^{NE}}) -N \sum_{\vartheta \in \Theta} r_{\vartheta} ~\min\{sh(\mathbf{p^{NE}}), E_{\vartheta}\} ~p_{ \vartheta}^{d,NE}\right]
 ~c^{grid,d}\nonumber\\ + &
 N \left[ \sum_{\vartheta \in \Theta} r_{\vartheta }~ p^{n,NE}_{\vartheta}~ \varepsilon_{\vartheta }~E_{\vartheta}\right] c^{grid,n}.
 \label{eq:social_cost_es_sc}
 \end{align}
\normalsize

\subsection{Centralized Energy Sharing Mechanism Under ES Policy}
Similar to C-ESM under the PA policy (Section \ref{sec:coordinated}), the C-ESM under the ES policy is modeled as an optimization problem, defined as:

 \vspace{-0.1in} 
 \begin{small}
 \begin{subequations} \label{eq:social_cost_x_opt_es}
\begin{alignat}{2}
& \min_{\mathbf{p}} \ && C^{ES}(\mathbf{p}) \label{eq:opt_1_es} \\
 & \text{s.t. } &&p^{d}_{\vartheta},~ p^{n}_{\vartheta}\geq 0  ,  \ \forall \vartheta \in \Theta \label{eq:opt_2_es} \\
 & \quad && p^{d}_{\vartheta} + p^{n}_{\vartheta} = 1,  \ \forall \vartheta \in \Theta. \label{eq:opt_4_es}
 \end{alignat}
 \end{subequations}
\end{small} 

The problem \eqref{eq:social_cost_x_opt_es} is non-convex due to its objective function and the form of the equal share $sh(\mathbf{p^{NE}})$ (Eq. \eqref{eq:fairshare}). In our simulations in Section \ref{sec:comptoES}, we solve it with genetic algorithms using the Global Optimization Toolbox of MATLAB. 

\clearpage


%APPENDICES are optional
%\balancecolumns
%\appendices
%Appendix A
%% \section{Analogies and disanalogies to nuclear}
% \copied{1. main thing to highlight earlier when linking is that it’s debated by scholars why nuclear verification succeeded, and the similarities vs. differences with AI training will determine whether AI will succeed for a similar reason. We give arguments why it may be easier, and why it may be harder.
% 2. Analogies
%     1. Both involve using a flow (centrifuges/chips) to  aggregate a stock (total training time)
%     2. Small amounts are fine, large amounts are bad
%     3. Positive economic use-cases that everyone should benefit from, negative misuse use-cases that we should limit to the extent possible.
% 3. Disanalogies
%     1. With uranium, after the enrichment occurs, you can still track the physically-produced uranium. (I.e. violations are reversible.) With compute, once a model has been trained, it can be copied at will.
%     2. With nuclear, most countries that have pursued their own nuclear program have eventually been able to discretely build their own centrifuges (e.g. with design-support from AQ Khan). With advanced compute, this seems very unlikely - fabs cost many billions, and are rarely spun up for a single purpose. (Even the US military failed at this.) Thus, compute supply chain is much more concentrated - less possible to “build an AI project in a bunker somewhere” detached from existing supply chains.
%     3. With HEU, inspecting the end-product is sufficient to know how enrichment was done. With compute, it’s hard to determine “how much an NN was trained” without being provided additional information on the process by which we arrived at those weights.
%     4. With HEU, we know ahead of time how much enrichment is sufficient to be dangerous (as there are physical requirements to causing a supercritical fission  reaction). However with AI, algorithmic progress means that the threshold for any particular dangerous use-case decreases over time. That said, “more compute” will always be riskier than “less compute”, and thus a useful heuristic for regulation. Even if relevant thresholds need to change over time, so long as the requirements do not shrink to the point where detection would be impossible, it is still just as important to have a governing framework for compute.
%     5. With HEU, enrichment must happen (mostly) at the same location. With AI training, enrichment can be parallelized over the internet, although there is some critical concentration required per location.
%     6. AI is much more civilian-economically valuable than uranium, and implemented with more parts of the supply chain. Many analysts believe that part of the success of the NPT is that states do not have a strong incentive to cheat as they’re part of security alliances; it is not clear whether this will be the same for AI. It may be that similar alliances are required.
% 4. Inspection mechanisms from nuclear we’d like to copy
%     1. Tracking centrifuge production
%     2. Centrifuge flow monitoring
%     3. Accounting for total usage of centrifuges (i.e. being able to show results)
% }

% \section{Case study in a way this could break: student-teacher}
% \ot{One way to break this scheme: rather than one single long training run, to evade detection, the model can be repeatedly self-distilled into a new model, and then that new model trained for an additional period. This increases a compute-overhead, but so long as the distilling time does not scale linearly with the training time, this can make it possible to hide longer runs as a series of shorter runs. However, the total compute required (and the total time that the in-RAM model is of sufficiently-low-loss to trigger an audit) is still large, meaning that a randomly-sampled chip would still need to attest to being part of a dangerous training run.
% Note also that this behavior would look kind of weird, because each self-distillation run would be using a very large number of chips *in parallel* for a short time. This is because such self-distillation-chains must occur sequentially across time, and at each timestep all the resources go into a particular snapshot.}

\newpage

\section{Discussion on future training requirements}\label{app.howmuch}

\subsection{Will the most capable ML models require large-scale training?}
This paper's proposed framework is premised on the assumption that large-scale training is and continues to be a necessary requirement for the most advanced (and thus most dangerous) ML models.
There is intense disagreement within the field about how important large-scale training is, and how long that will remain the case.

Many of the recent breakthroughs in machine learning model capabilities, across every domain, have come from increasing the model size or quantity of training data, each of which corresponds to a greater usage of compute \cite{kaplan2020scaling, hoffman2022training, zhai2021scaling}.
Indeed, some capabilities, such as chain-of-thought reasoning, appear to only emerge at the largest training scales \cite{wei2022chain}.
At the same time, any one narrow capability can often be achieved with a much smaller compute budget \cite{magister2022teaching, madani2023large}.
Nonetheless, Sutton's ``Bitter Lesson'' \cite{sutton2019bitter} that ``general methods that leverage computation are ultimately the most effective'' is a frequent diagnosis of the likely future of deep learning.
Though algorithmic progress \cite{erdil2022algorithmic} and the continued progress of Moore's Law will continue to reduce the number of chips required for any specific capability, we may compensate by gradually increasing enforcement parameters to work for smaller quantities of specialized compute.
At the same time, the increasing investment in compute by frontier AI firms \cite{lardinois_2022, wiggers_2022} suggests that industry insiders continue to believe that the most capable frontier models --- likeliest to yield new capabilities and surface new risks to public safety --- are expected to require ever more compute.

\subsection{Will large-scale training continue to require specialized datacenter chips?}

Nearly all large-scale training runs are executed on high-end datacenter accelerators \cite{chowdhery2022palm, kaplan2020scaling, zeng2022glm}.
The main difference between these chips and their consumer-oriented counterparts is their much higher inter-chip communication bandwidth (e.g., 900GB/s for the NVIDIA H100 SXM vs. 64GB/s for the NVIDIA GeForce RTX 4090 \cite{nvidiah100, nvidia4090}).
This extra bandwidth is today crucial for parallelizing NN training, especially tensor parallelism and data parallelism, which require frequent transfers of large matrices between many chips \cite{smith2022computation}.
Organizations doing large-scale training also favor these datacenter chips for other reasons: they are generally more energy efficient, and license requirements often prevent organizations from placing consumer-oriented chips in datacenters\cite{moss_2023}.

Still, recent work has suggested it may be \emph{possible} to do large-scale training on consumer chips with low interconnect, though with substantial cost and speed penalties\cite{binhang2022distributed, ryabinin2023distributed}.
If such methods become feasible for bad actors, then we may need to adjust to a different regulatory model for detecting training activity.
Possibilities include focusing on spotting and monitoring datacenters (similar to the IAEA's work to detect undeclared nuclear facilities \cite{harry1996iaea}), or regulating the high-capacity switches that could be necessary to enable fast networking between low-interconnect chips.
So long as they can be detected, it may be possible to retrofit consumer chips (e.g. with a permanently-mated host CPU, see Section \ref{s.onchip}) to enable similar monitoring capabilities.

It is important to note that the current framework \emph{does} apply in the setting where clusters of chips are split across several datacenters (e.g. multiple cloud providers), so long as these high-end chips are used at each datacenter.

%\section{Headings in Appendices}
%\balancecolumns
% That's all folks!

\end{document}



}
\end{document}
