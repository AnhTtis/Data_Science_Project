\section{Conclusion}
\label{sec:conclusion}
We presented a novel method to reconstruct multiple articulated objects in a scene in a category- and joint-agnostic manner from a single stereo image. For reconstruction we learn a SDF-based decoder and show the necessity of regularization to achieve good performance. Our full single-shot pipeline improves over current two-stage approaches in terms of 3DIoU and inference speed.

{\parskip=5pt
\noindent\textbf{Limitations}:
While \ourName{} is able to generalize to unseen instances, it still relies on a learned shape prior. Using test time adaption techniques such as done by \cite{mu_a-sdf_2021} helps mitigating this issues, but is not sufficient to deal with categorically different objects. Additionally, while the single-forward pass is fast, jointly optimizing for pose, scale, codes like done in \cite{liu_catre_2022, irshad_2022_shapo} could further improve results with the cost of added execution time. Currently \ourName{} is only trained on objects with a single joint. To extend \ourName{} to objects with an arbitrary number of joints, we must be able to calculate pairwise similarity between two object states. While not explored in this paper, \ourName{} introduces a framework for future research to pursue this research question. A potential solution could leverage \cref{eqn:joint_sim_real} and Hungarian matching of the cross-product of articulation states to obtain similarities measurements between arbitrary kinematic structures.
}
